\book{The Speech of Web-strengthener. (Vafþrúðnismǫ́l)}

\begin{astanza}%L
(Óðinn kvað:) &
\bv{}Ráð mér nú \alst{F}rigg \hld alls mik \alst{f}ara tíðir &
at \alst{v}itja \alst{V}afþrúðnis; &
\alst{f}orvitni mikla \hld kveð'k mér á \alst{f}ornum stǫfum &
við þann hinn \alst{a}lsvinna \alst{jǫ}tun.\\ \&\end{astanza}%L

\begin{astanza}%L
(Frigg kvað:) &
\bv \alst{H}ęima lętja \hld mynda'k \alst{H}ęrjafǫðr &
í \alst{g}ǫrðum \alst{g}oða; &
\alst{ę}ngi \alst{jǫ}tun \hld hugða'k \alst{ja}fnramman &
sęm \alst{V}afþrúðni \alst{v}esa.\\ \&\end{astanza}%L

\begin{astanza}%L
(Óðinn kvað:) &
\bv Fjǫlð ek fór, \hld fjǫlð fręistaða'k, &
fjǫlð ek ręynda ręgin; &
hitt vil'k vita, \hld hvé Vafþrúðnis &
salakynni séi.\\ \&\end{astanza}%L

\begin{astanza}%L
(Frigg kvað:) &
\bv Hęill þú farir, \hld hęill þú aptr komir, &
hęill á sinnum séir; &
ǿði þér dugi \hld hvar's skalt, Aldafǫðr, &
orðum mæla jǫtun.\\ \&\end{astanza}%L

\begin{astanza}%L
\bv Fór þá Óðinn \hld at fręista orðspęki &
þess hins alsvinna jǫtuns; &
at hǫllu hann kom, \hld es\footnotemark[1] átti Íms faðir; &
inn gekk Yggr þegar.\\ \&\end{astanza}%L
\footnotetext[1]{Ms. \emph{ok} corrected to \emph{es}. Alliteration is lacking in this line, for which reason FJ emends \emph{Íms} to \emph{Hymis}.}

\begin{astanza}%L
(Óðinn kvað:) &
\bv Hęill þú nú, Vafþrúðnir, \hld nú em'k í hǫll kominn &
á þik sjalfan séa; &
hitt vilk fyrst vita, \hld ef fróðr séir &
eða alsviðr, jǫtunn.\\ \&\end{astanza}%L

\begin{astanza}%L
(Vafþrúðnir kvað:) &
\bv Hvat's þat manna, \hld es í mínum sal &
verpumk orði á? &
út þú né kømr \hld órum hǫllum frá. &
nema þú inn snotrari séir.\\ \&\end{astanza}%L

\begin{astanza}%L
(Óðinn kvað:) &
\bv Gagnráðr\footnotemark[5] hęiti'k, \hld nú em'k af gǫngu kominn, &
þyrstr til þinna sala; &
laðar þurfi \hld hęf'k lęngi farit &
ok þinna andfanga, jǫtunn.\\ \&\end{astanza}%L
\footnotetext[5]{R's \emph{Gagnráðr} 'Gain-adviser', is attested as Gangráðr 'Journey-adviser' in \emph{Gylf}.}

\begin{astanza}%L
(Vafþrúðnir kvað:) &
\bv Hví þú þá, Gagnráðr, \hld mælisk af golfi fyrir? &
far þú í sess í sal; &
þá skal fręista, \hld hvárr flęira viti, &
gęstr eða hinn gamli þulr.\\ \&\end{astanza}%L

\begin{astanza}%L
(Gagnráðr kvað:) &
\bv Óauðigr maðr, \hld es til auðigs kømr, &
mæli þarft eða þęgi; &
ofrmælgi mikil \hld hygg at illa geti &
hvęim's við kaldrifjaðan kømr.\\ \&\end{astanza}%L

\begin{astanza}%L
(Vafþrúðnir kvað:) &
\bv Sęg mér, Gagnráðr, \hld alls á golfi vill &
þíns of fręista frama, &
hvé hęstr hęitir, \hld sá's hvęrjan dręgr &
dag of dróttmǫgu.\\ \&\end{astanza}%L

\begin{astanza}%L
(Gagnráðr kvað:) &
\bv Skinfaxi hęitir, \hld es hinn skíra dręgr &
dag of dróttmǫgu; &
hęsta baztr \hld þykkir með Hręiðgotum; &
ęy lýsir mǫn af mari.\\ \&\end{astanza}%L

\begin{astanza}%L
(Vafþrúðnir kvað:) &
\bv Sęg þat, Gagnráðr, \hld alls á golfi vill &
þíns of fręista frama, &
hvé jór hęitir, \hld sá's austan dręgr &
nótt of nýt ręgin.\\ \&\end{astanza}%L

\begin{astanza}%L
(Gagnráðr kvað:) &
\bv Hrímfaxi hęitir, \hld es hvęrja dręgr &
nótt of nýt ręgin; &
méldropa fęllir \hld morgin hvęrjan; &
þaðan kømr dǫgg of dala.\\ \&\end{astanza}%L

\begin{astanza}%L
(Vafþrúðnir kvað:) &
\bv Sęg þat, Gagnráðr, \hld alls á golfi vill &
þíns of fręista frama, &
hvé ǫ́ hęitir, \hld sú's dęilir með jǫtna sonum &
grund ok með goðum.\\ \&\end{astanza}%L

\begin{astanza}%L
(Gagnráðr kvað:) &
\bv Ífing hęitir ǫ́, \hld es dęilir með jǫtna sonum &
grund ok með goðum; &
opin rinna \hld hón skal um aldrdaga; &
verðr-at íss á ǫ́.\\ \&\end{astanza}%L

\begin{astanza}%L
(Vafþrúðnir kvað:) &
\bv Sęg þat, Gagnráðr, \hld alls á golfi vill &
þíns of fręista frama, &
hvé vǫllr hęitir, \hld es finnask vigi at &
Surtr ok hin svǫ́su goð.\\ \&\end{astanza}%L

\begin{astanza}%L
(Gagnráðr kvað:) &
\bv Vígríðr hęitir vǫllr, \hld es finnask vígi at &
Surtr ok hin svǫ́su goð; &
hundrað rasta \hld hann's á hvęrjan veg; &
sá's þęim vǫllr vitaðr.\\ \&\end{astanza}%L

\begin{astanza}%L
(Vafþrúðnir kvað:) &
\bv Fróðr est nú gęstr, \hld far á bękk jǫtuns, &
mælumk í sessi saman; &
hǫfði vęðja \hld vit skulum hǫllu í &
gęstr, of gęðspęki.\\ \&\end{astanza}%L

\begin{astanza}%L
(Gagnráðr kvað:) &
\bv Sęg þat hit ęina, \hld ef þitt ǿði\footnotemark[10] dugir &
ok þú Vafþrúðnir vitir, &
hvaðan jǫrð of kom \hld eða upphiminn &
fyrst, hinn fróði jǫtunn.\\ \&\end{astanza}%L
\footnotetext[10]{Starting with \emph{ǿði}, the poem is also preserved in 748.}

\begin{astanza}%L
(Vafþrúðnir kvað:) &
\bv Ór Ymis holdi \hld vas jǫrð of skǫpuð, &
ęn ór bęinum bjǫrg, &
himinn ór hausi \hld hins hrimkalda jǫtuns, &
ęn ór svęita sær.\\ \&\end{astanza}%L

\begin{astanza}%L
(Gagnráðr kvað:) &
\bv Sęg þat annat, \hld ef þitt ǿði dugir &
ok þú Vafþrúðnir vitir, &
hvaðan máni of kom, \hld svát fęrr menn yfir, &
eða sól hit sama.\\ \&\end{astanza}%L

\begin{astanza}%L
(Vafþrúðnir kvað:) &
\bv Mundilfari hęitir, \hld hann's Mána faðir &
ok svá Solar hit sama; &
himin hverfa \hld þau skulu hvęrjan dag &
ǫldum at ártali.\\ \&\end{astanza}%L

\begin{astanza}%L
(Gagnráðr kvað:) &
\bv Sęg þat þriðja, \hld alls þik svinnan kveða &
ok þú Vafþrúðnir vitir, &
hvaðan dagr of kom, \hld sá's fęrr drótt yfir, &
eða nótt með niðum.\\ \&\end{astanza}%L

\begin{astanza}%L
(Vafþrúðnir kvað:) &
\bv Dęllingr hęitir, \hld hann's Dags faðir, &
ęn Nótt vas Nǫrvi borin; &
ný ok nið \hld skópu nýt ręgin &
ǫldum at ártali.\\ \&\end{astanza}%L

\begin{astanza}%L
(Gagnráðr kvað:) &
\bv Sęg þat fjórða, \hld alls þik fróðan kveða, &
ok þú Vafþrúðnir vitir, &
hvaðan vetr of kom \hld eða varmt sumar &
fyrst með fróð ręgin.\\ \&\end{astanza}%L

\begin{astanza}%L
(Vafþrúðnir kvað:) &
\bv Vindsvalr hęitir, \hld hann's Vetrar faðir, &
ęn Svǫ́suðr Sumars.\footnotemark[15]\\ \&\end{astanza}%L
\footnotetext[15]{Second half of the v. seems missing.}

\begin{astanza}%L
(Gagnráðr kvað:) &
\bv Sęg þat fimta, \hld alls þik fróðan kveða, &
ok þú Vafþrúðnir vitir, &
hvęrr ása ęlztr \hld eða Ymis niðja &
yrði í árdaga.\\ \&\end{astanza}%L

\begin{astanza}%L
(Vafþrúðnir kvað:) &
\bv Ørófi vetra \hld áðr væri jǫrð skǫpuð, &
þá vas Bergęlmir borinn, &
Þrúðgęlmir \hld vas þess faðir, &
ęn Aurgęlmir afi.\\ \&\end{astanza}%L

\begin{astanza}%L
(Gagnráðr kvað:) &
\bv Sęg þat sétta, \hld alls þik svinnan kveða, &
ok þú Vafþrúðnir vitir, &
hvaðan Aurgęlmir kom \hld með jǫtna sonum &
fyrst, hinn fróði jǫtunn.\\ \&\end{astanza}%L

\begin{astanza}%L
(Vafþrúðnir kvað:) &
\bv Ór Élivǫ́gum \hld stukku ęitrdropar, &
svá óx unz ór varð jǫtunn; &
órar ættir \hld kómu þar allar saman; &
því's þat æ alt til atalt.\footnotemark[20]\\ \&\end{astanza}%L
\footnotetext[20]{Lines 3–4 missing in R and 748, but quoted in \emph{Gylf}.}

\begin{astanza}%L
(Gagnráðr kvað:) &
\bv Sęg þat sjaunda, \hld alls þik svinnan kveða, &
ok þú Vafþrúðnir vitir, &
hvé sá bǫrn gat \hld hinn baldni\footnotemark[25] jǫtunn, &
es hann hafði-t gýgjar gaman.\\ \&\end{astanza}%L
\footnotetext[25]{R has \emph{aldni}, 'aged, old'. This breaks alliteration; \emph{baldni} 'bold, defiant' has been substituted from 748.}

\begin{astanza}%L
(Vafþrúðnir kvað:) &
\bv Und hęndi vaxa \hld kvǫ́ðu hrímþursi &
męy ok mǫg saman; &
fótr við fǿti \hld gat hins fróða jǫtuns &
sexhǫfðaðan son.\\ \&\end{astanza}%L

\begin{astanza}%L
(Gagnráðr kvað:) &
\bv Sęg þat áttunda, \hld alls þik fróðan kveða, &
ok þú Vafþrúðnir vitir, &
hvat fyrst of mant \hld eða fręmst of vęizt, &
þú est alsviðr jǫtunn.\\ \&\end{astanza}%L

\begin{astanza}%L
(Vafþrúðnir kvað:) &
\bv Ørófi vetra \hld áðr væri jǫrð of skǫpuð, &
þá vas Bergęlmir borinn; &
þat fyrst um man'k, \hld es hinn fróði jǫtunn &
á vas lúðr of lagiðr.\footnotemark[30]\\ \&\end{astanza}%L
\footnotetext[30]{This verse is quoted in \emph{Gylf}.}

\begin{astanza}%L
(Gagnráðr kvað:) &
\bv Sęg þat níunda, \hld alls þik svinnan kveða, &
ok þú Vafþrúðnir vitir, &
hvaðan vindr of kømr \hld svát fęrr vág yfir, &
æ męnn hann sjalfan of séa.\\ \&\end{astanza}%L

\begin{astanza}%L
(Vafþrúðnir kvað:) &
\bv Hræsvęlgr hęitir, \hld es sitr á himins ęnda, &
jǫtunn í arnar ham; &
af hans vængjum \hld kveða vind koma &
alla męnn yfir.\\ \&\end{astanza}%L

\begin{astanza}%L
(Gagnráðr kvað:) &
\bv Sęg þat tíunda, \hld alls þú tíva rǫk &
ǫll Vafþrúðnir vitir, &
hvaðan Njǫrðr of kom \hld með niðjum ása. &
Hófum ok hǫrgum \hld hann ræðr hundmǫrgum &
ok varð-at hann ǫ́sum alinn.\\ \&\end{astanza}%L

\begin{astanza}%L
(Vafþrúðnir kvað:) &
\bv Í Vanahęimi \hld skópu hann vís ręgin &
ok sęldu at gíslingu goðum, &
í aldar rǫk \hld hann mun aptr koma &
hęim með vísum vǫnum.\\ \&\end{astanza}%L

\begin{astanza}%L
(Gagnráðr kvað:) &
\bv Sęg þat ęllipta, \hld hvar ýtar túnum í &
hǫggvask hvęrjan dag; &
val þęir kjósa \hld ok ríða vígi frá, &
sitja męir of sáttir saman.\footnotemark[35]\\ \&\end{astanza}%L
\footnotetext[35]{This and the next v. are damaged in both R and 748; R has only this verse, but splits it in two (the 2nd starting with \emph{val}), while 748 has 40:1 (Ms.: \emph{S. þ. e. XI}) and then jumps to the answer v. 41. They have here been reconstructed, but it is possible some lines are still missing.}

\begin{astanza}%L
(Vafþrúðnir kvað:) &
\bv Allir ęinhęrjar \hld Óðins túnum í &
hǫggvask hvęrjan dag, &
val þeir kjósa \hld ok ríða vígi frá, &
sitja męir of sáttir saman.\\ \&\end{astanza}%L

\begin{astanza}%L
(Gagnráðr kvað:) &
\bv Sęg þat tolpta, \hld hví þú tíva rǫk &
ǫll Vafþrúðnir vitir, &
frá jǫtna rúnum \hld ok allra goða &
þú hit sannasta sęgir, &
hinn alsvinni jǫtunn.\\ \&\end{astanza}%L

\begin{astanza}%L
(Vafþrúðnir kvað:) &
\bv Frá jǫtna rúnum \hld ok allra goða &
ek kann sęgja satt, &
þvíat hvęrn hęf'k \hld heim of komit, &
níu kom'k hęima \hld fyr niflhęl neðan; &
hinig dęyja ór hęlju halir.\\ \&\end{astanza}%L

\begin{astanza}%L
(Gagnráðr kvað:) &
\bv Fjǫlð ek fór, \hld fjǫlð fręistaða'k, &
fjǫlð ek ręynda ręgin; &
hvat lifir manna, \hld þá's hinn mæra líðr &
fimbulvetr með firum?\\ \&\end{astanza}%L

\begin{astanza}%L
(Vafþrúðnir kvað:) &
\bv Líf ok Lífþrasir, \hld ęn þau lęynask munu &
í holti Hoddmímis; &
morgindǫggvar \hld þau sér at mat hafa; &
þaðan af aldir alask.\\ \&\end{astanza}%L

\begin{astanza}%L
(Gagnráðr kvað:) &
\bv Fjǫlð ek fór, \hld fjǫlð fręistaða'k, &
fjǫlð ek ręynda ręgin; &
hvaðan kømr sól \hld á hinn slétta himin, &
es þessa hęfr Fęnrir farit?\\ \&\end{astanza}%L

\begin{astanza}%L
(Vafþrúðnir kvað:) &
\bv Ęina dóttur \hld berr alfrǫðull, &
áðr hana Fęnrir fari; &
sú skal ríða, \hld þá's ręgin dęyja, &
móður brautir mær.\\ \&\end{astanza}%L

\begin{astanza}%L
(Gagnráðr kvað:) &
\bv Fjǫlð ek fór, \hld fjǫlð fręistaða'k, &
fjǫlð ek ręynda ręgin; &
hvęrjar 'ro męyjar, \hld es líða mar yfir, &
fróðgęðjaðar fara.\\ \&\end{astanza}%L

\begin{astanza}%L
(Vafþrúðnir kvað:) &
\bv Þríar þjóðár \hld falla þorp yfir &
męyja Mǫgþrasis; &
hamingjur ęinar \hld þær's í hęimi eru, &
þó þær með jǫtnum alask.\\ \&\end{astanza}%L

\begin{astanza}%L
(Gagnráðr kvað:) &
\bv Fjǫlð ek fór, \hld fjǫlð fręistaða'k, &
fjǫlð ek ręynda ręgin; &
hvęrir ráða æsir \hld ęignum goða, &
þá's sloknar Surtalogi?\\ \&\end{astanza}%L

\begin{astanza}%L
(Vafþrúðnir kvað:) &
\bv Víðarr ok Váli \hld byggva vé goða, &
þá's sloknar Surtalogi; &
Móði ok Magni \hld skulu Mjǫlni hafa &
Vingnis at vígþroti.\\ \&\end{astanza}%L

\begin{astanza}%L
(Gagnráðr kvað:) &
\bv Fjǫlð ek fór, \hld fjǫlð fręistaða'k, &
fjǫlð ek ręynda ręgin; &
hvat verðr Óðni \hld at aldrlagi, &
þá's rjúfask ręgin?\\ \&\end{astanza}%L

\begin{astanza}%L
(Vafþrúðnir kvað:) &
\bv Ulfr glęypa \hld mun Aldafǫðr, &
þess mun Víðarr vreka; &
kalda kjapta \hld hann klyfja mun &
vitnis vígi at.\\ \&\end{astanza}%L

\begin{astanza}%L
(Gagnráðr kvað:) &
\bv Fjǫlð ek fór, \hld fjǫlð fręistaða'k, &
fjǫlð ek ręynda ręgin; &
hvat mælti Óðinn, \hld áðr á bál stigi, &
sjalfr í ęyra syni?\\ \&\end{astanza}%L

\begin{astanza}%L
(Vafþrúðnir kvað:) &
\bv Ęy mann-gi\footnotemark[40] vęit, \hld hvat þú í árdaga &
sagðir í ęyra syni; &
fęigum munni \hld mælta'k mína forna stafi &
ok of ragna rǫk. &
Nú við Óðin \hld dęilda'k mína orðspęki; &
þú est æ vísastr vera.\\ \&\end{astanza}%L
\footnotetext[40]{Emended from R, 748 \emph{manni}; the word must be in the nominative, but \emph{manni} is dative.}

\end{Leftside}
\endnumbering

\chapterStart
\begin{Rightside}
\beginnumbering\numberlinefalse

\begin{astanza}%R
\bv{}\textbf{Woden} quoth: \\ “Counsel me now, \textbf{Frie}, as I desire to travel to visit \textbf{Web-strengthener}; I have great curiosity to say ancient staves\footnotemark[1] against that all-wise \textbf{ettin}." \\
\&\end{astanza}%R
\footnotetext[1]{Ancient (pieces of) lore; cf. v. 55.} 

\begin{astanza}%R
\bv Frie quoth: \\ “I would encourage the \textbf{Leader of Armies} to [stay at] home in the yards of the gods, for I've judged no ettin be as strong as\footnotemark[3] Web-strengthener." \\
\&\end{astanza}%R
\footnotetext[3]{Lit. 'equal-strong'.}

\begin{astanza}%R
\bv Woden quoth: \\ “Much I travelled, much I tried, much I tested the \textbf{Powers}\footnotemark[4]. \emph{This} I want to know, how the condition of the halls of Web-strengthener might be?" \\
\&\end{astanza}%R
\footnotetext[4]{The gods.}

\begin{astanza}%R
\bv Frie quoth: \\ “Whole may thou travel, whole may thou return, whole may thou be on thy paths! May, thy wisdom suffice, \textbf{Leader of Men}, when thou go to exchange words with the ettin." \\
\&\end{astanza}%R

\begin{astanza}%R
\bv Then went Woden, to try the word-wisdom of that all-wise ettin. To the hall he came, which the father of \textbf{Ime}\footnotemark[5] owned; shortly the \textbf{Frightener}\footnotemark[6] walked in. \\
\&\end{astanza}%R
\footnotetext[5]{Web-strengthener.}
\footnotetext[6]{Woden.}

\begin{astanza}%R
\bv Woden quoth: \\ “Hail thee now, Web-strengthener; now I have come into the hall, to see thee thyself. \emph{This} I want to know first, if knowing thou might be, or all-wise, ettin!" \\
\&\end{astanza}%R

\begin{astanza}%R
\bv Web-strengthener quoth: \\ “What is that of men\footnotemark[10], that in \emph{my} hall throws words at me? Thou will not come \emph{out}, from \emph{our}\footnotemark[11] halls, unless thou be the wiser [of us two]." \\
\&\end{astanza}%R
\footnotetext[10]{That is, 'what man is that'. The use of the neuter pronoun \emph{hvat} by Web-str. may be seen as an insult or a way of belittling the guest.}
\footnotetext[11]{Prob. again meaning 'my', unless Web-str. has allies present in the hall, but no such indication is given.}

\begin{astanza}%R
\bv Woden quoth: \\ “\textbf{Gain-adviser} I am called, I am come from the journey, thirsty to thy halls. I have travelled for a long time in need of hospitality, and of thy reception, ettin!" \\
\&\end{astanza}%R

\begin{astanza}%R
\bv Web-strengthener quoth: \\ “Why then, Gain-adviser, art thou speaking from the floor before [me]? Take a seat in the hall! Then it shall be proven, which of the two might know more; the guest, or the old \textbf{thyle}." \\
\&\end{astanza}%R

\begin{astanza}%R
\bv Gain-adviser quoth: \\ “An unwealthy man, who comes to a wealthy [one], ought to speak what is needed, or be silent.\footnotemark[14] Much over-speaking\footnotemark[15], I judge, will be bad for the one who comes to a cold-ribbed\footnotemark[16] [man]." \\
\&\end{astanza}%R
\footnotetext[14]{Line identical to \emph{High} 18/2. The whole verse strongly reminds of verses from the \emph{Guest-thread} portion of said poem.}
\footnotetext[15]{'Speaking too much'.}
\footnotetext[16]{That is, 'cold-hearted', 'cunning'.}

\begin{astanza}%R
\bv Web-strengthener quoth: \\ “Say to me, Gain-adviser, since on the floor I will to try thy fame: What is the horse called, which drags each \emph{day} above the sons of the retinue\footnotemark[20]?" \\
\&\end{astanza}%R
\footnotetext[20]{Kenning for 'men', 'mankind'.}

\begin{astanza}%R
\bv Gain-adviser quoth: \\ “\textbf{Shining-fax} [that one] is called, who drags the bright day above the sons of the retinue. The best of horses he seems among the \textbf{Rode-goths}; the mane of that stallion ever shines." \\
\&\end{astanza}%R

\begin{astanza}%R
\bv Web-strengthener quoth: \\ “Say this, Gain-adviser, since on the floor I will to try thy fame: What is the horse called, which from the east drags night above the useful \textbf{Powers}?" \\
\&\end{astanza}%R

\begin{astanza}%R
\bv Gain-adviser quoth: \\ “\textbf{Frost-fax} [that one] is called, who drags each night above the useful Powers. Every morning he lets foam fall from his bit\footnotemark[26]; thence comes dew in the valleys." \\
\&\end{astanza}%R
\footnotetext[26]{Lit. 'he fells bit-drops'.}

\begin{astanza}%R
\bv Web-strengthener quoth: \\ “Say this, Gain-adviser, since on the floor I will to try thy fame; How the river is called, which divides the ground between the sons of ettins and the gods?" \\
\&\end{astanza}%R

\begin{astanza}%R
\bv Gain-adviser quoth: \\ “\textbf{Iving} the river is called, which divides the ground between the sons of ettins and the gods. Throughout [her] life-days she shall flow open; ice forms not on the river." \\
\&\end{astanza}%R

\begin{astanza}%R
\bv Web-strengthener quoth: \\ “Say this, Gain-adviser, since on the floor I will to try thy fame: How that valley is called, where \textbf{Surt} and the excellent gods find each other at war?" \\
\&\end{astanza}%R

\begin{astanza}%R
\bv Gain-adviser quoth: \\ “\textbf{Battle-rider} is the valley called, where Surt and the cheerful gods find each other at war. A hundred rests\footnotemark[30], he stretches in each direction; that valley is known for them.\footnotemark[31]" \\
\&\end{astanza}%R
\footnotetext[30]{An old unit of length, from its name prob. the length a horse could travel before resting.}
\footnotetext[31]{That is, known for its great size.}

\begin{astanza}%R
\bv Web-strengthener quoth: \\ “Knowing art thou now, guest, sit down on the ettin's bench; let us speak while sitting together. In the hall we shall wager a head, guest, over [our] mind-wisdom." \\
\&\end{astanza}%R

\begin{astanza}%R
\bv Gain-adviser quoth: \\ “Say the first [lit. one], if thy wisdom suffices, and thou, Web-strengthener, might know: Whence, O knowing ettin, the earth first came, or \textbf{up-heaven}?" \\
\&\end{astanza}%R

\begin{astanza}%R
\bv Web-strengthener quoth: \\ “Out of \textbf{Yime's} hull\footnotemark[35], the earth was created, but the mountains out of his bones. Heaven out of the skull of the frost-cold ettin, but the sea out of his sweat.\footnotemark[36]" \\
\&\end{astanza}%R
\footnotetext[35]{His body.}
\footnotetext[36]{\emph{svęiti} 'sweat' is a common kenning for blood. — This v. closely resembles \emph{Grím} 40.}

\begin{astanza}%R
\bv Gain-adviser quoth: \\ “Say the second, if thy wisdom suffices, and thou, Web-strengthener, might know: Whence the moon came, so that it travels over men, or likewise the sun?" \\
\&\end{astanza}%R

\begin{astanza}%R
\bv Web-strengthener quoth: \\ “\textbf{Moundelfare} [that one] is called, he is the father of the Moon, and likewise of the Sun. They shall circle in the heavens every day, for men to reckon time\footnotemark[40]." \\
\&\end{astanza}%R
\footnotetext[40]{Lit. 'for men to year-reckoning'.}

\begin{astanza}%R
\bv Gain-adviser quoth: \\ “Say the third, since [they] call thee wise, and thou, Web-strengthener, might know: Whence the day came, the one that travels over the rettinue, or night with the moon-phases?" \\
\&\end{astanza}%R

\begin{astanza}%R
\bv Web-strengthener quoth: \\ “\textbf{Delling} [that one] is called, he is the father of \textbf{Day}, but \textbf{Night} was born to \textbf{Nare}. The waxing and waning [of the moon], the useful Powers created, for men to reckon time." \\
\&\end{astanza}%R

\begin{astanza}%R
\bv Gain-adviser quoth: \\ “Say the fourth, since [they] call thee knowing, and thou, Web-strengthener, might know: Whence winter first came, or the warm summer, among the knowing Powers?" \\
\&\end{astanza}%R

\begin{astanza}%R
\bv Web-strengthener quoth: \\ “\textbf{Wind-cool} [that one] is called, he is the father of \textbf{Winter}, but \textbf{Delightful} of \textbf{Summer}." \\
\&\end{astanza}%R

\begin{astanza}%R
\bv Gain-adviser quoth: \\ “Say the fifth, since [they] call thee knowing, and thou, Web-strengthener, might know: Who, in days of yore became the eldest of the \textbf{Anses}, or of the descendants of Yime?" \\
\&\end{astanza}%R

\begin{astanza}%R
\bv Web-strengthener quoth: \\ “Uncountable winters before the earth would be created, then \textbf{Bear-yeller} was born. \textbf{Strength-yeller} was \emph{that one's} father, and \textbf{Mud-yeller} the grandfather." \\
\&\end{astanza}%R

\begin{astanza}%R
\bv Gain-adviser quoth: \\ “Say the sixth, since [they] call thee wise, and thou, Web-strengthener, might know: Whence, O knowing ettin, Mud-yeller first came among the sons of ettins?" \\
\&\end{astanza}%R

\begin{astanza}%R
\bv Web-strengthener quoth: \\ “From the \textbf{Ell-waves}, poison-drops splashed; thus [it] grew until an ettin emerged. \emph{Our} family lines all together originated there, therefore our race\footnotemark[45] is forever fierce against all.\footnotemark[46]" \\
\&\end{astanza}%R
\footnotetext[45]{Lit. 'it' or 'that'.}
\footnotetext[46]{Somewhat strange phrasing, but the line does not appear damaged. It is clearly an explanation of the fierce and maleficent nature of the ettins, as their first ancestors were created from poison.}

\begin{astanza}%R
\bv Gain-adviser quoth: \\ “Say the seventh, since [they] call thee wise, and thou, Web-strengthener, might know: How did that one, the defiant ettin, beget children, when he did not enjoy the [carnal] pleasure of a troll-woman?" \\
\&\end{astanza}%R

\begin{astanza}%R
\bv Web-strengthener quoth: \\ “Neath the hand\footnotemark[50] on the \textbf{frost-thurse}, [they] said that a girl and boy together grew. A foot against a foot begot, for the knowing ettin, a six-headed son." \\
\&\end{astanza}%R
\footnotetext[50]{The word \emph{hǫnd} means 'hand', but might here be a poetic circumlocution for 'arm'.}

\begin{astanza}%R
\bv Gain-adviser quoth: \\ “Say the eigth, since [they] call thee knowing, and thou, Web-strengthener, might know: What dost thou first remember, or earliest know?\footnotemark[55] Thou art all-wise, ettin." \\
\&\end{astanza}%R
\footnotetext[55]{Cf. Vsp 1.}

\begin{astanza}%R
\bv Web-strengthener quoth: \\ “Uncountable winters before the earth would be created, then Bear-yeller was born. \emph{That} I first remember, when the knowing ettin\footnotemark[60] was laid down on the funeral-bed\footnotemark[61]." \\
\&\end{astanza}%R
\footnotetext[60]{That is, Bear-yeller. Cf. v. 29.}
\footnotetext[61]{\emph{lúðr}, a tricky word.}

\begin{astanza}%R
\bv Gain-adviser quoth: \\ “Say the ninth, since [they] call thee wise, and thou, Web-strengthener, might know: Whence the wind comes, so that he travels over the wave; forever men see him himself.\footnotemark[65]" \\
\&\end{astanza}%R
\footnotetext[65]{Perhaps a negation has been lost here; the wind is never seen by men.}

\begin{astanza}%R
\bv Web-strengthener quoth: \\ “\textbf{Corpse-swallower} [that one] is called, which sits at the end of the heavens, an ettin in the shape of an eagle. From his wings, they say [that] the wind comes over all men." \\
\&\end{astanza}%R

\begin{astanza}%R
\bv Gain-adviser quoth: \\ “Say the \emph{tenth}, since thou, Web-strengthener, of all the fates of the \textbf{Tues} might know: Whence \textbf{Nearth} came into the company of the kinsmen of the \textbf{Anses}? He rules an immense number\footnotemark[68] of \textbf{hoves} and \textbf{heargs}, and he was not begotten among the Anses." \\
\&\end{astanza}%R
\footnotetext[68]{Lit. 'he rules hound-many'.}

\begin{astanza}%R
\bv Web-strengthener quoth: \\ “In \textbf{Wane-Home}, the wise \textbf{Powers}\footnotemark[69] created him, and gave\footnotemark[70] him as a hostage to the gods. In the fate of the age, he will come back, home among the wise \textbf{Wanes}." \\
\&\end{astanza}%R
\footnotetext[69]{The gods, or specifically the Wanes.}
\footnotetext[70]{Lit. 'sold'.}

\begin{astanza}%R
\bv Gain-adviser quoth: \\ “Say the eleventh, Where men in yards, hew away at each other every day? They choose those destined to die in war, and ride [away] from battle; [then] they sit more content together." \\
\&\end{astanza}%R

\begin{astanza}%R
\bv Web-strengthener quoth: \\ “In Woden's yards, all the \textbf{Lone Warriors} hew away at each other every day. They choose those destined to die in war, and ride [away] from battle; [then] they sit more content together." \\
\&\end{astanza}%R

\begin{astanza}%R
\bv Gain-adviser quoth: \\ “Say the twelfth, Why thou, Web-strengthener, should know all the fates of the \textbf{Tues}\footnotemark[73]? Regarding the \textbf{runes} of the ettins and of all the gods, thou, the all-wise ettin, speaks most truly." \\
\&\end{astanza}%R
\footnotetext[73]{The gods. Formation identical to \emph{ragna rǫk} 'the fates of the Powers'.}

\begin{astanza}%R
\bv Web-strengthener quoth: \\ “Regarding the runes of the ettins and of all the gods, I can speak truly, for I have come into each \textbf{Home}. I came [into] \textbf{nine Homes} beneath \textbf{Fog-hell}; thereto men die out of \textbf{Hell}\footnotemark[75]." \\
\&\end{astanza}%R
\footnotetext[75]{A difficult verse. Does it imply that if a man dies when already dead, he goes to Fog-hell?}

\begin{astanza}%R
\bv Gain-adviser quoth: \\ “Much I travelled, much I tried, much I tested the \textbf{Powers}.\footnotemark[80] What remains\footnotemark[79] of men, when the famous \textbf{fimbol-winter} passes among firs\footnotemark[81]?" \\
\&\end{astanza}%R
\footnotetext[79]{Lit. 'lives'.}
\footnotetext[80]{Here begins the repetition of the same “mantra" used in v. 3, which continues until the final question (v. 54).}
\footnotetext[81]{Among men.}

\begin{astanza}%R
\bv Web-strengthener quoth: \\ “\textbf{Life} and \textbf{Life-striver}, and they will hide themselves in the woods of \textbf{Hoard-Mime}\footnotemark[85]. Morning-dew they [will] have as food; therefrom generations will be bred." \\
\&\end{astanza}%R
\footnotetext[85]{Prob. the same as \emph{Yggdrasill}, the \textbf{Terrifier-Steed.}}

\begin{astanza}%R
\bv Gain-adviser quoth: \\ “Much I travelled, much I tried, much I tested the Powers. Whence comes sun onto the smooth heaven, when \textbf{Fenner} has killed this one\footnotemark[90]?" \\
\&\end{astanza}%R
\footnotetext[90]{That is, the current incarnation of the sun, as explained in the next v.}

\begin{astanza}%R
\bv Web-strengthener quoth: \\ “One daughter the elf-wheel\footnotemark[95] bears, before \textbf{Fenner} might kill her. When the Powers die, that one, the girl, shall ride the paths of the mother." \\
\&\end{astanza}%R
\footnotetext[95]{The sun.}

\begin{astanza}%R
\bv Woden quoth: \\ “Much I travelled, much I tried, much I tested the Powers. Which are the girls that pass over the ocean; wise-minded they go?" \\
\&\end{astanza}%R

\begin{astanza}%R
\bv Web-strengthener quoth: \\ “Three great rivers fall over the settlement of the girls of \textbf{Boy-striver}; they are the only \textbf{Hamings} in the Home,\footnotemark[99] though they are raised among the ettins\footnotemark[100]." \\
\&\end{astanza}%R
\footnotetext[99]{In ettin-Home, or in the entire world?}
\footnotetext[100]{An opaque verse. These three girls may be the same as the three 'thurse-maidens' in \emph{Vsp} 9, see note there. Their father, Boy-striver, is not mentioned anywhere else.}

\begin{astanza}%R
\bv Gain-adviser quoth: \\ “Much I travelled, much I tried, much I tested the Powers. Which properties of the gods will the \textbf{Anses} [still] rule\footnotemark[105], when the flame of \textbf{Surt} burns out?" \\
\&\end{astanza}%R
\footnotetext[105]{Or 'control'.}

\begin{astanza}%R
\bv Web-strengthener quoth: \\ “\textbf{Wider} and \textbf{Weel} [will] inhabit the sanctuaries of the gods, when the \textbf{flame of Surt} burns out. \textbf{Mood} and \textbf{Main} will own \textbf{Mealen}, when \textbf{Wingen} can no longer fight\footnotemark[110]." \\
\&\end{astanza}%R
\footnotetext[110]{Lit. 'at Wingen's fight-exhaustion', referring to his death.}

\begin{astanza}%R
\bv Gain-adviser quoth: \\ “Much I travelled, much I tried, much I tested the Powers. What happens to Woden at the end of the age, when the Powers are broken?" \\
\&\end{astanza}%R

\begin{astanza}%R
\bv Web-strengthener quoth: \\ “The wolf will swallow the \textbf{Leader of Men}; Wider will avenge that. He will cleave the cold jaws of the wolf at the battle." \\
\&\end{astanza}%R

\begin{astanza}%R
\bv Gain-adviser quoth: \\ “Much I travelled, much I tried, much I tested the Powers. What spoke Woden himself, before [he]\footnotemark[115] would step onto the funeral pyre, into the ear of the son?" \\
\&\end{astanza}%R
\footnotetext[115]{Prob. Woden's son, that is \textbf{Balder}.}

\begin{astanza}%R
\bv Web-strengthener quoth: \\ “No man may ever know\footnotemark[119], what thou in days of yore said into the ear of the son. With a death-doomed\footnotemark[120] mouth, I spoke my ancient staves, and about the \textbf{fates of the Powers}\footnotemark[121]. Now against Woden, I have shared my word-wisdom\footnotemark[122]; thou art forever the wisest of men\footnotemark[123]." \\
\&\end{astanza}%R
\footnotetext[119]{Lit. 'eternally no man knows'.}
\footnotetext[120]{Or '\emph{fey} mouth'. Web-strengthener here realizes that he was bound to die from the moment (v. 19) he proposed the wager; no man, nor god, nor ettin can outwit Woden.}
\footnotetext[121]{'fates of the Powers', that is, \emph{ragna rǫk}.}
\footnotetext[122]{The same word-wisdom Woden in v. 5 set out to try.}
\footnotetext[123]{Word used is \emph{ver} 'husband, man'. Perhaps in the broader sense of 'male being'.}

\end{Rightside}
\endnumbering
\end{pairs}
\Columns

\bookEnd{}
