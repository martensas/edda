\bookStart{Introduction}

\title{%
  A New Critical Edition and Translation of the Poetic Edda \\
  \large Along with Commentary, Fragments and a Few Other Old Germanic Poems}

\author{Mårten SUS}

\maketitle

\tableofcontents

\chapter{Introduction to Eddic poetry}
  Don't go too indepth on individual poems! Each one will have its own introduction.
  \section{Metrics and conventions}
    Alliteration
    Kennings
  \section{How can we know the age of the Eddic poems?}
    Linguistic criteria
    Archeological evidence
    Comparison with known Christian texts (Sólarljóð, Hugsvinnsmál)
    Snorri thought they were old
    Saxo had access to them
    Many of them clearly describe non-Icelandic surroundings
      Especially Hávamál is clearly Norwegian

\chapter{Ancient Germanic cult(ure)}
  \section{Economy (fee)}
  \section{Morals}
    Honour, personal integrity
    Notes on the terms \emph{argr} and \emph{ergi}
  \section{Religious conceptions}
    Cosmic cycles
    Reincarnation
    Analogies with other Indo-European traditions

\chapter{Notes to translation}
  Point about literal translation for use by scholars of comparative mythology
    The “guiding star” of this translation effort has been literality and concistency. All previous translations (to my knowledge) have such issues as: rendering identically repeated phrases differently at various places; covering up or obscuring technical and cultural terminology; simplifying kennings and other expressions—and this often without notes, to a point where the original meaning is, at times, unrecognizable.
    While I wholly encourage all readers of sufficient interest to study Old Norse (and other ancient Germanic languages!), perhaps even using this edition as a tool, I also realize that this is a demanding ask which not all interested students and scholars of comparative mythology, anthropology, literature, religion and other fields will be able to fulfill. I therefore want these groups to be able to have a text that is as close to the original as possible, at the very least when it regards sense and expression.
  Why Anglish names?
    One of the most idiosyncratic parts of this edition will be its handling of proper names. I have opted to render all cultural terms, names of places, gods, men and other entities in their natural English (\emph{Anglish}) forms. I suppose the primary reason for this is ideological. I believe that these myths and poems are a common Germanic or Northern European heritage, and should be treated as such. The English once knew gods such as Weden and Thunder, and called them by names naturally evolved in their language. So too did the Germans and Scandinavians, of course, and I would hope that any translators into those languages would follow this spirit and render the names in their natural forms there as well.\footnote{For instance in German perhaps Wuten, Donner, Froh.}

\chapter{Notes to critical edition}
  My goal with the critical editing of the texts has been to produce something as close to the original manuscripts as possible, without excessive emendation to the preserved recension(s). There are texts in three (TODO) languages in this edition, these being Old Norse, Old English and Old High German. Old Norse texts have been normalized according to roughly the same orthography as Finnur 1932. On the other hand the Old High German and Old English texts have only been lightly normalized, correcting obvious errors and marking vowel length with acute accents. For further information see below.

  \section{Normalization of Old Norse}
    The orthography only differs from Finnur 1932 in its use of \emph{ǿ} rather than \emph{œ} to represent the result of i-umlaut on the ancient Germanic \emph{ō}.
    Superfluous and hypermetrical pronouns (usually \emph{hann}, \emph{hón}) have in many places been removed. \emph{ek} ‘I’, and \emph{es} (particle) ‘which, that, where, when’, \emph{es} (3rd sg. pres. ind. of \emph{vesa} ‘to be’) have been contracted to \emph{’k} and \emph{’s} when metrically beneficient.

  \section{Manuscripts}
    There are two surviving ancient manuscripts which contain full Eddic poems.

    The \emph{first} and most important is GKS 2365 4to, here \Regius. It dates to the 1270s and has 45 surviving leaves, containing TODO poems. Of these 10 are mythological, and the rest heroic, dealing with legends mostly of the Migration Period. Notably, following fol. 32, there is a large gap of missing pages. This occurs in the heroic section, specifically cutting off \Sigrdrifumal. It is unclear how many leaves and poems went missing.
    \Regius\ is not just a compilation of poems, it shows editorial input as well. Several of the mythological poems are separated by short prose sections, which tie them together into a loose frame narrative, though it is clear from their style and composition that they are originally separate works. When it comes to the heroic poems long prose sections occur both within and between them, creating a \inx{saw}-like narrative where the prose in many cases holds up the poetry, rather than the reverse. For further literature see TODO.

    The \emph{second} manuscript is AM 748 I a 4to, here \AM. It dates to the 1300s and has just 6 leaves. \AM, as we have it, contains only mythological poems, and in a different order from \Regius; it has no frame narrative. On the first two leaves are contained \Harbardsljod\ (which lacks its beginning), \Baldrsdraumar\ and \Skirnismal\ (lacking its ending). After this some number of leaves have gone missing, but the other four leaves follow eachother. On them we find \Vafthrudnismal\ (lacking its beginning), \Grimnismal\, \Hymiskvida\ and the prose introduction of \Volundarkvida. \AM\ is the only medieval manuscript attesting \Baldrsdraumar, and further its other poems are not copied from \Regius, but rather derive from a shared ancestor. This fact makes it very valuable for textual criticism. For further literature see TODO.

    Some Eddic poems survive only in younger paper manuscripts. These being: TODO. While I have not consulted the paper mss. for poems attested in medieval mss., I have had to rely on them for these poems. Their exclusive survival there does not \emph{necessarily} prove them works of late antiquarians; had we not been fortunate enough to have \Baldrsdraumar\ in \AM, it would have been counted among them, yet we now know that it is truly ancient. It is not an impossibility that other poems now only found in paper mss. would have survived in now lost medieval mss., perhaps even in the lost pages of \Regius\ or \AM.

    Finally several Eddic poems are quoted in \Gylfaginning, these being (TODO): \Voluspa, \Vafthrudnismal, \Grimnismal. It also contains a few fragments, which have also been edited. For \Gylfaginning\ I have relied on the following four main mss.:\begin{enumerate}
	   \item The Codex Regius of the Prose Edda \RegiusProse\ (GKS 2367 4to; 1300-1350)
     \item The Codex Trajectinus \Trajectinus\ (Traj 1374; a c. 1595 paper copy of a ms. closely related to \RegiusProse.)
     \item The Codex Wormianus \Wormianus\ (AM 242 fol.; 1340–70)
     \item The Codex Upsaliensis \Upsaliensis\ (DG 11; 1300–25)\end{enumerate}

     For sake of brevity I refer to these four collectively as \GylfMS, which is thus equivalent to \RegiusProse\Trajectinus\Wormianus\Upsaliensis. I refer to Haukur Þorgeirsson 2017 for discussion on their internal stemmatics and origins.

     West Germanic poetry

     As all West Germanic poems edited here (TODO: Will we be editing other poems than Hildebrandslied?) survive only in one copy, the specific details are discussed in their accompanying introductions.

\chapter{Bibliography and sigla}
  % TODO: Generate from bibliography.bib

\section{Abbreviations}
  \begin{itemize}
    \item cert. = certainly
    \item cf. = confer
    \item fol. = folio
    \item i.e. = \emph{id est}; that is
    \item l. = line
    \item ll. = lines
    \item lit. = literally
    \item ms. = manuscript
    \item mss. = manucsripts
    \item om. = omits, omitted
    \item p. = page
    \item v. = verse
    \item wo. = without\end{itemize}
