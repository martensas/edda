\title{The Northern Epics:}
\author{Konrad O. L. Rosenberg}

\begin{titlingpage}
  \makeatletter
  \centering
  \HUGE \textsc{\@title} \\
  \Huge The Poetic Edda \\
  \huge and other Old Germanic alliterative poetry \\
  \vspace{1cm}
  \Large\emph{edited and translated by} \\
  \vspace{1cm}
  \huge \@author \\
  \vspace{4cm}
  \Large Compiled \today. \\
  \vspace{1cm}
  THE BOOK IS A WORK IN PROGRESS AND THIS FILE MAY BE OUTDATED. \\
  The reader is kindly asked to periodically download the newest version from https://github.com/martensas/edda.
  \makeatother
\end{titlingpage}

\newpage\thispagestyle{empty}

\vspace*{\fill}

{\leftskip=0pt plus .5fil \rightskip=0pt plus -.5fil \parfillskip=0pt plus 1fil
\noindent {\Large \emph{Lifir hann of allar aldir ok stjórnar ǫllu ríki sínu ok rę́ðr ǫllum hlutum, stórum ok smǫ́m. [...] Hann smíðaði himin ok jǫrð ok lopt’in ok alla eign þeira. [...] Hitt er þó mest, er hann gerði mann’inn ok gaf hónum ǫnd þá, er lifa skal ok aldri týna⸗sk, þó’tt lík-amr fúni at moldu eða brenni at ǫsku; ok skulu allir menn lifa, þeir er rétt eru siðaðir, ok vera með hónum sjǫlfum þar sem heitir Gim·lé eða Vin·gólf.}}
\par}
\begin{flushright}%
\emph{Gylfa ginning} 3:4–7%
\end{flushright}

\vspace{5mm}

\begin{center}{\Large \emph{Vęl kęypts hlutar \hld\ hęf’k vęl notit; \\
fás es fróðum vant; \\
því’t Óð-rǿrir \hld\ es nú upp kominn \\
ȧ alda vés jaðar.}}\end{center}

\begin{flushright}%
\emph{Háva mǫ́l} 106%
\end{flushright}

\vspace{5mm}

\begin{center}{\Large \emph{Dęyr fé, \hld\ dęyja frę̇ndr, \\
dęyr sjalfr hit sama; \\
ek vęit ęinn \hld\ at aldri⸗gi dęyr \\
dómr of dauðan hvęrn.}}\end{center}

\begin{flushright}%
\emph{Háva mǫ́l} 77%
\end{flushright}

\vspace{5mm}

\begin{center}{\Large \emph{Ullar hylli \hld\ hęfr ok allra goða \\
hvęrr’s tękr fyrstr ȧ funa \\
því’t opnir hęimar \hld\ verða umb ȧsa sonum, \\
þȧ’s hęfja af hvera.}}\end{center}

\begin{flushright}%
\emph{Grímnis mǫ́l} 43%
\end{flushright}

\vfill

\newpage\thispagestyle{empty}

\bookStart{Acknowledgments}

{\leftskip=0pt plus .5fil \rightskip=0pt plus -.5fil \parfillskip=0pt plus 1fil
\noindent The following people have been especially helpful in giving suggestions, corrections and providing stimulating discussion: Antonín Ryšavý, C. A., Esma, John Newman, J. Hertling, Joseph S. Hopkins, Nikhil Surya Dwibhashyam, Trevor L. Payne.\par}

\newpage\thispagestyle{empty}

\tableofcontents

\newpage\thispagestyle{empty}

\newpage

\bookStart{Abbreviations}
  \subsection{Languages}
\begin{itemize}%
	\item Eng. = Modern English
	\item Ger. = Modern German
	\item Got. = Gotnish (or Gothic)
	\item Lomb. = Lombardic
	\item MHG = Middle High German
	\item OE = Old English
	\item OF = Old Frisian
	\item OHG = Old High German
	\item ON = Old Norse
	\item OS = Old Saxon
	\item OSwe. = Old Swedish
	\item PGmc. = Proto-Germanic
	\item PN = Proto-Norse
	\item PNWGmc. = Proto-North-West Germanic
\end{itemize}

\subsection{Grammar}
\begin{itemize}%
	\item 1st = first-person
	\item 2nd = second-person
	\item 3rd = third-person
	\item acc. = accusative case
	\item cpd = compound
	\item dat. = dative case
	\item gen. = genitive case
	\item imper. = imperative mood
	\item ind. = indicative mood
	\item instr. = instrumental case
	\item nom. = nominative case
	\item pl. = plural number
	\item sg. = singular number
	\item subj. = subjunctive mood
\end{itemize}

\subsection{Other abbreviations}
\begin{itemize}%
	\item cert. = certainly
	\item c. = circa
	\item cf. = \emph{confere}; compare
	\item corr. = corrected in the ms.
	\item e. = excerpt (not the whole stanza)
	\item ed. = edition, edited (by)
	\item e.g. = \emph{exemplio gratia}; for instance
	\item emend. = emendation, emended (by)
	\item fol., foll. = folio, folios
	\item i.e. = \emph{id est}; that is
	\item l., ll. = line, lines
	\item lit. = literally
	\item metr. emend. = emended based on (secure) metrical criteria
	\item ms., mss. = manuscript, manuscripts
	\item norm. = normalised from the ms. spelling
	\item om. = omitted by
	\item p., pp. = page, pages
	\item tr. = translation, translated (by)
	\item sens. emend. = emended based on sense
	\item st., sts. = stanza, stanzas
	\item viz. = \emph{vidēlicet}; namely, to wit
	\item wo. = without
	\item wrt. = with regard to
\end{itemize}

% Old texts, primary sources
% The command codes must be as close to the original language titles as possible.
\subsection{Primary sources}

\newcommand{\AitareyaBrahmana}{%
	\emph{AB}%
}
\newcommand{\Alvissmal}{% Speeches of Allwise
	\emph{Alv}%
}
\newcommand{\Atlakvida}{% Lay of Attle
	\emph{Akv}%
}
\newcommand{\Atlamal}{% Speeches of Attle
	\emph{Am}%
}
\newcommand{\Baldrsdraumar}{% The Dreams of Balder
	\emph{Bdr}%
}
\newcommand{\Beowulf}{% Beewolf
	\emph{Beow}%
}
\newcommand{\Brot}{% Fragment of a Lay of Siward
	\emph{Brot}%
}
\newcommand{\Deor}{% Deer
	\emph{Deer}%
}
\newcommand{\EyrbyggjaSaga}{% Saw of Harware and Heathric
	\emph{Eb}%
}
\newcommand{\EgilsSaga}{% Saw of Harware and Heathric
	\emph{Eg}%
}
\newcommand{\Fafnismal}{% Speeches of Fathomer
	\emph{Fáfn}%
}
\newcommand{\FostrbroedhraSaga}{% Saw of the Foster-brothers
	\emph{FbrS}%
}

\newcommand{\FraLoka}{% From Lock
	\emph{From Lock}%
}%TODO: remove this

\newcommand{\Grettissaga}{% Saw of Gretter
	\emph{GrettS}%
}
\newcommand{\Grimnismal}{% Speeches of Grimner
	\emph{Grm}%
}
\newcommand{\Gripisspa}{% Spae of Griper
	\emph{Gríp}%
}
\newcommand{\Grottasongr}{% Song of Grotte
	\emph{Grotta}%
}
\newcommand{\Grougaldr}{% Galder of Growe
	\emph{Grg}%
}
\newcommand{\Gudrunarhvot}{% Guthrun’s Instigation
	\emph{Ghv}%
}
\newcommand{\GudrunOne}{% Guthrun’s First Lay
	\emph{Guðr I}%
}
\newcommand{\GudrunTwo}{% Guthrun’s Second Lay
	\emph{Guðr II}%
}
\newcommand{\GudrunThree}{% Guthrun’s Third Lay
	\emph{Guðr III}%
}
\newcommand{\Gulatingslog}{% Law of the Gole-thing
	\emph{Gula}%
}
\newcommand{\Gylfaginning}{% The Guiling of Yilver; for referring to Gylfaginning as a text
	\emph{Gylf}%
}
\newcommand{\Hakonarmal}{% Speeches of Hathkin
	\emph{Hákm}%
}

\newcommand{\HakonarSaga}{% Saw of Hathkin the good
	\emph{HákGóð}%
}
\newcommand{\Haleygjatal}{% Tally of the Hallowlendings
	\emph{HalT}%
}%TODO: Add

\newcommand{\Hamdismal}{% Speeches of Hamthew
	\emph{Hamð}%
}
\newcommand{\Harbardsljod}{% Leed of Hoarbeard
	\emph{Hárb}%
}
\newcommand{\Haustlong}{% Harvest-long
	\emph{Haustl}
}
\newcommand{\Havamal}{% Speeches of the High One
	\emph{Háv}%
}
\newcommand{\HelgakvidaHjorvardssonar}{% Lay of Hallow Harwardson
	\emph{HHj}%
}
\newcommand{\HelgakvidaOne}{% First Lay of Hallow Hundingsbane
	\emph{HHund I}%
}
\newcommand{\HelgakvidaTwo}{% Second Lay of Hallow Hundingsbane
	\emph{HHund II}%
}
\newcommand{\Heliand}{%
  \emph{Heli}%
}
\newcommand{\Helreid}{% Byrnhild’s Hell-ride
	\emph{Helr}%
}
\newcommand{\HervararSaga}{% Saw of Harware and Heathric
	\emph{HarS}%
}
\newcommand{\Hildebrandslied}{% Speeches of Hildbrand
	\emph{Hildebrand}%
}
\newcommand{\Hymiskvida}{% Lay of Hymer
	\emph{Hym}%
}
\newcommand{\Hyndluljod}{% Leed of Hindle
	\emph{Hdl}%
}

\newcommand{\Lacnunga}{% Leekning
	\emph{Lacning}%
}%TODO: add this

\newcommand{\Lokasenna}{% Flyting of Lock
	\emph{Lok}%
}

\newcommand{\Malshattakvadi}{
	\emph{Mhkv}
}%TODO: add this

\newcommand{\Mahabharata}{
	\emph{MBʰ}
}
\newcommand{\MerseburgOne}{% First Merseburg charm
	\emph{Mers I}%
}
\newcommand{\MerseburgTwo}{% Second Merseburg charm
	\emph{Mers II}%
}
\newcommand{\Muspilli}{% Muspell
	\emph{Muspilli}%
}
\newcommand{\Oddrunargratr}{% Weeping of Ordrun
	\emph{Oddrgr}%
}
\newcommand{\Reginsmal}{% The Speeches of Rein
	\emph{Reg}%
}
\newcommand{\Rigsthula}{% Thule of Righ
	\emph{Rþ}%
}
\newcommand{\Rigveda}{%
	\emph{R̥V}%
}
\newcommand{\SaxonGenesis}{% Old Saxon Genesis
	\emph{OSGen}%
}
\newcommand{\Sigurdskamma}{% Short Lay of Siward
	\emph{Sigsk}%
}
\newcommand{\Sigrdrifumal}{% Speeches of Syedrive
	\emph{Sigrdr}%
}
\newcommand{\Skaldskaparmal}{% The Matter of Scoldship
	\emph{Skm}%
}
\newcommand{\Skirnismal}{% Speeches of Shirner
	\emph{Skm}%
}

\newcommand{\Sogubrot}{
	\emph{AncKings}
}%TODO: add
\newcommand{\Solarljod}{
	\emph{Sun}
}%TODO: add
\newcommand{\Sonatorrek}{
	\emph{Sont}
}%TODO: add
\newcommand{\Sorlathattr}{% Strand of Sarle
	\emph{Sarle}%
}%TODO: add
\newcommand{\ThidreksSaga}{% Saw of Thedrich
	\emph{ThidS}%
}%TODO: add

\newcommand{\Thorsdrapa}{% Drape of Thunder
	\emph{Þdr}%
}
\newcommand{\Thrymskvida}{% Lay of Thrim
	\emph{Þrk}%
}
\newcommand{\Vafthrudnismal}{% Speeches of Webthrithner
	\emph{Vafþ}%
}
\newcommand{\Volsathattr}{% Strand of Walse
	\emph{Vǫlsþ}%
}
\newcommand{\VolsungaSaga}{% Saw of the Walsings
	\emph{VǫlsS}%
}
\newcommand{\Volundarkvida}{% Lay of Wayland
	\emph{Vkv}%
}
\newcommand{\Voluspa}{% Spae of the Wallow
	\emph{Vsp}%
}

\newcommand{\Waldere}{% Walder
	\emph{Walder}%
}%TODO: add
\newcommand{\YnglingaSaga}{% Saw of the Inglings
	\emph{IngS}%
}%TODO: add
\newcommand{\Ynglingatal}{% Tally of the Inglings
	\emph{IngT}%
}%TODO: add

\begin{itemize}%
	\item \AitareyaBrahmana\ = \emph{Aitareyá Brā́hmaṇa}
	\item \Alvissmal\ = \emph{Alvíssmǫ́l} (Speeches of Allwise)
	\item \Atlakvida\ = \emph{Atlakviða} (Lay of Attle)
	\item \Atlamal\ = \emph{Atlamǫ́l} (Speeches of Attle)
	\item \Baldrsdraumar\ = \emph{Baldrs draumar} (Dreams of Balder)
	\item \Beowulf\ = \emph{Beowulf}
	\item \Brot\ = \emph{Brot af Sigurðarkviða} (Fragment of a Lay of Siward)
	\item \Deor\ = \emph{Déor} (Deer)
	\item \EyrbyggjaSaga\ = \emph{Eyrbyggja saga} (Saw of the Ere-dwellers)
	\item \Fafnismal\ = \emph{Fáfnismǫ́l} (Speeches of Fathomer)
	\item \FostrbroedhraSaga\ = \emph{Fóstrbrǿðra saga} (Saw of the Fosterbrothers)
	\item \Grettissaga\ = \emph{Grettis saga} (Saw of Gretter)
	\item \Grimnismal\ = \emph{Grímnis mǫ́l} (Speeches of Grimner)
	\item \Gripisspa\ = \emph{Grípisspǫ́} (Spae of Griper)
	\item \Grottasongr\ = \emph{Grottasǫngr} (Song of Grotte)
	\item \Grougaldr\ = \emph{Gróugaldr} (Galder of Growe)
	\item \Gudrunarhvot\ = \emph{Guðrúnarhvǫt} (Goading of Guthrun)
	\item \GudrunOne\ = \emph{Guðrúnarkviða I} (First Lay of Guthrun)
	\item \GudrunTwo\ = \emph{Guðrúnarkviða II} (Second Lay of Guthrun)
	\item \GudrunThree\ = \emph{Guðrúnarkviða III} (Third Lay of Guthrun)
	\item \Gulatingslog\ = \emph{Gulaþingslǫg} (Law of the Gole‑Thing)
	\item \Gylfaginning\ = \emph{Gylfaginning} (Beguiling of Yilver)
	\item \Hakonarmal\ = \emph{Hǫ́konarmǫ́l} (Speeches of Hathkin)
	\item \HakonarSaga\ = \emph{Hǫ́konar saga góða} (Saw of Hathkin the good)
	\item \Hamdismal\ = \emph{Hamðismǫ́l} (Speeches of Hamthew)
	\item \Harbardsljod\ = \emph{Hárbarðljóð} (Leeds of Hoarbeard)
	\item \Haustlong\ = \emph{Haustlǫng} (Harvest‑long)
	\item \Havamal\ = \emph{Hávamǫ́l} (Speeches of the High One)
	\item \HelgakvidaHjorvardssonar\ = \emph{Helgakviða Hjǫrvarðssonar} (Lay of Hallow Harwardson)
	\item \HelgakvidaOne\ = \emph{Helgakviða Hundingsbana I} (First Lay of Hallow Hundingsbane)
	\item \HelgakvidaTwo\ = \emph{Helgakviða Hundingsbana II} (Second Lay of Hallow Hundingsbane)
	\item \Heliand\ = \emph{Heliand}
	\item \Helreid\ = \emph{Helreið Brynhildar} (Hell‑ride of Byrnhild)
	\item \HervararSaga\ = \emph{Hervarar saga} (Saw of Harware and Heathric)
	\item \Hildebrandslied\ = \emph{Hildebrandslied}
	\item \Hymiskvida\ = \emph{Hymiskviða} (Lay of Hymer)
	\item \Hyndluljod\ = \emph{Hyndluljóð} (Leeds of Hindle)
	\item \Lokasenna\ = \emph{Lokasenna} (Flyting of Lock)
	\item \Mahabharata\ = \emph{Mahā́bʰārata}
	\item \MerseburgOne\ = Merseburg galder I
	\item \MerseburgTwo\ = Merseburg galder II
	\item \Oddrunargratr\ = \emph{Oddrúnargrátr} (Weeping of Ordrun)
	\item \Reginsmal\ = \emph{Ręginsmǫ́l} (Speeches of Rein)
	\item \Rigsthula\ = \emph{Rigsþula} (Thule of Righ)
	\item \Rigveda\ = \emph{R̥g-vedá}, with translations from Jamison‑Brereton unless otherwise specified.
	\item \SaxonGenesis\ = \emph{Old Saxon Genesis}
	\item \Sigurdskamma\ = \emph{Sigurðarkviða skamma} (Short Lay of Siward)
	\item \Sigrdrifumal\ = \emph{Sigrdrífumǫ́l} (Speeches of Syedrive)
	\item \Skaldskaparmal\ = \emph{Skaldskaparmǫ́l} (Matter of Scoldship)
	\item \Skirnismal\ = \emph{Skírnismǫ́l} (Speeches of Shirner)
	\item \Thorsdrapa\ = \emph{Þórsdrápa} (Drape of Thunder)
	\item \Thrymskvida\ = \emph{Þrymskviða} (Lay of Thrim)
	\item \Vafthrudnismal\ = \emph{Vafþrúðnismǫ́l} (Speeches of Webthrithner)
	\item \Volsathattr\ = \emph{Vǫlsaþáttr} (Strand of Walse)
	\item \VolsungaSaga\ = \emph{Vǫlsunga saga} (Saw of the Walsings)
	\item \Volundarkvida\ = \emph{Vǫlundarkviða} (Lay of Wayland)
	\item \Voluspa\ = \emph{Vǫluspǫ́} (Spae of the Wallow)
\end{itemize}%


% Manuscripts
\newcommand{\AM}{% AM 748 I a 4to (https://handrit.is/manuscript/view/da/AM04-0748-I-a, https://books.google.se/books?id=L-MOAAAAQAAJ)
	\textbf{A}%
}
\newcommand{\AMb}{% AM 748 I b 4to (https://handrit.is/manuscript/view/is/AM04-0748-Ib)
	\textbf{A\textsubscript{b}}%
}
\newcommand{\EddaBms}{% AM 757 a 4° (https://handrit.is/manuscript/view/is/AM04-0757a)
	\textbf{B}%
}
\newcommand{\FlatMS}{% Flateyjarbok
	\textbf{F}%
}
\newcommand{\GylfMS}{% For referring to Gylfaginning manuscripts when stanzas are attested there.
	\textbf{G}%
}
\newcommand{\Hauksbok}{% Hauksbok
	\textbf{H}%
}
\newcommand{\VolsungaMS}{% NKS 1824 b 4° (https://skaldic.ku.dk/q?p=skp/mss/ms/512 and https://onp.ku.dk/onp/onp.php?b2195-53)
	\textbf{N}%
}
\newcommand{\Regius}{% Codex Regius (of the poetic edda)
	\textbf{R}%
}
\newcommand{\RegiusProse}{% Codex Regius of the Prose Edda
	\textbf{S}%
}
\newcommand{\Trajectinus}{% Codex Trajectinus
	\textbf{T}%
}
\newcommand{\Wormianus}{% Codex Wormianus (https://clarino.uib.no/menota/text/menota/AM-242-fol)
	\textbf{W}%
}
\newcommand{\Upsaliensis}{% Codex Upsaliensis
	\textbf{U}%
}
\newcommand{\HildMS}{% For referring to the Hildebrandslied manuscript.
	ms.%
}

\subsection{Manuscripts}
\begin{itemize}%
	\item \AM\ = AM 748 I a 4° (https://handrit.is/manuscript/view/da/AM04-0748-I-a)
	\item \AMb\ = AM 748 I b 4° (https://handrit.is/manuscript/view/is/AM04-0748-Ib)
	\item \EddaBms\ = AM 757 a 4° (https://handrit.is/manuscript/view/is/AM04-0757a)
	\item \FlatMS\ = Flatsęyjarbók, GKS 1005 fol. (https://handrit.is/manuscript/view/is/GKS02-1005)
	\item \GylfMS\ = all manuscripts of \Gylfaginning; equivalent to \RegiusProse\Trajectinus\Upsaliensis\Wormianus
	\item \Hauksbok\ = Hauksbók, AM 544 4° (https://handrit.is/manuscript/view/en/AM04-0544)
	\item \VolsungaMS\ = NKS 1824 b 4° (https://onp.ku.dk/onp/onp.php?m9641)
	\item \Regius\ = Codex Regius of the Poetic Edda, GKS 2365 4° (https://eae.ku.dk/q?p=eae/vols/text/1)
	\item \RegiusProse\ = Codex Regius of the Prose Edda, GKS 2367 4° (https://handrit.is/manuscript/view/is/GKS04-2367)
	\item \Trajectinus\ = Codex Trajectinus, Traj 1374ˣ
	\item \Upsaliensis\ = Codex Upsaliensis, DG 11
	\item \Wormianus\ = Codex Wormianus, AM 242 fol. (https://clarino.uib.no/menota/text/menota/AM-242-fol)
\end{itemize}

% Meters
\newcommand{\Drottkvett}{% Court-recited
	\emph{Court-recited meter}%
}
\newcommand{\Fornyrdislag}{% Law of Ancient Speeches
	\emph{Ancient-words-law}%
}
\newcommand{\Galdralag}{% Meter of Speeches
	\emph{Galders-law}%
}
\newcommand{\Ljodahattr}{% Meter of Leeds
	\emph{Leeds-meter}%
}
\newcommand{\Kviduhattr}{%
	\emph{Lay-meter}%
}
\newcommand{\Malahattr}{% Meter of Speeches
	\emph{Speeches-meter}%
}

%Modern books and editions (TODO: move these to bibliography)
\newcommand{\CV}{% Cleasby-Vigfússon dictionary of Old Norse
	\textciteshorttitle{CleasbyVigfusson}% \emph{C-V}%
}
\newcommand{\FGT}{% First Grammatical Treatise
	\textciteshorttitle{FGTHaugen}%
}
\newcommand{\ONP}{% Dictionary of Old Norse Prose
	\emph{ONP}%
}
\newcommand{\Skp}{% Skaldic Poetry of the Scandinavian Middle Ages
	\textciteshorttitle{SkP}%
}
%

\printbibliography% Does it work?

\listoffigures

\bookStart{General Introduction (incomplete!)}

The introduction is currently very incomplete and many parts are just outlines.

\section{The Old Germanic world}

  \subsection{Lifestyle and economy}
    Cattle-based; small farmsteads.
  \subsection{Morals and Virtues}
    Honour, personal integrity
    Notes on the terms \emph{argr} and \emph{ęrgi}
  \subsection{Religion}
    Keeping the Powers happy
    Cosmic cycles
    Reincarnation
    Analogies with other Indo-European traditions


\section{Germanic alliterative poetry}

  The historical-literary significance of the poetry in this edition may be thought of in two ways; in relation to later Germanic art forms and in relation to its related, older, Indo-European literary forms.

  On the one hand the oldest extensive monuments in most Germanic languages, and indeed the earliest indigenous Germanic literature (the Gothic being wholly derivative and translational), are in the form of alliterative verse.  As an indigenous literary genre it is our main early source of knowledge about the Iron Age folk-life of Northern Europe and our most important source for the pre-Christian Germanic religion; it represents these peoples in their earliest historical stages.  As the earliest known medium for the transmission of song and story amongst the ancestors of numerous great nations it has to some extent influenced all later Germanic languages and literatures.  The alliterative poetry, albeit together with Latin, Greek, and Hebrew models, laid the ground for the German \emph{Minnessänger}, the \emph{Nibelungenlied}, and Chaucer, which in turn precede such great writers as Goethe, Wagner, and Shakespeare.

  On the other hand, alliterative poetry is hardly a new art-form by the time we first encounter it in manuscript form in the 7th century \hyperref[Cadman]{Cadman’s Hymn}.  In the 1st century CE Tacitus (\emph{Germania} ch. 2) mentions the ‘ancient songs’ in which the Germani documented their mythology and history, and the classic alliterative long-line appears as early as the 3rd century on the Thorsberg chape from Schleswig.  In the early centuries CE we also find examples of typical stylistic traits like kennings, litotes, and poetic formulae: the Tjurkö bracteate (DR IK184) calls gold \emph{walha-kurné} (dat. sg.) ‘Roman grain’, a kenning-like reference to the melting of Roman solidi; the Thorsberg chape (DR 7) expresses ‘famous’ with the litotes \emph{ni wajé-máriʀ} ‘not ill-famed’ (cf. \textlink{Atlakvida}[13]/1); and the Noleby stone (Vg 63) mirrors the formula \emph{rúnó ragina-kundó} (acc. sg.) ‘rune born of the Gods’ later found in the 9th century \hyperref[Havamal:80]{\emph{Háva mǫ́l}, st. 80}.

  Of course, a far greater trove of rich expressions and images, many of great antiquity, are found in the later poems written on medieval manuscripts.  In \Beowulf\ and the Edda ‘sea-stallions’ sail across the ocean; in \textlink{Grimnismal}[38] the sun is drawn across Heaven in her chariot; in numerous poems (e.g. \Beowulf, \textlink{Heliand}, \textlink{Atlakvida}) feasts are held in great chiefly ‘mead-halls’.  These are not uniquely medieval motifs, nor are they inventions of the first centuries CE—they reflect the archeology of the Nordic Bronze Age and have parallels in the lines of \Rigveda\ and Homer.  To be clear, this does not mean that the poems themselves date back to such ancient times but only that they retain very ancient traditions which must have been passed down through an unbroken chain of poetry going back far into prehistory, when the Indo-Europeans were still nomadic steppe-pastoralists, long before any scribe had as of yet set foot in Northern Europe.

  So too, the alliterative poetic language likewise overflows with archaic synonyms, and Indo-European words otherwise extinct in all Germanic languages find their last refuge in it.  Such are the Old English \emph{eoh}, Old Norse \emph{jór}, corresponding to the Sanskrit \emph{áşva}, Latin \emph{equus}, all meaning ‘horse’; Old Norse \emph{týr}, corresponding to Sanskrit \emph{devá}, Latin \emph{deus}, all meaning ‘god’; Old English and Old Norse \emph{fold} 'earth, land', corresponding to Sanskrit \emph{pṛthivī́} 'id.'  The fact that many of these relate to the cult perhaps suggests that the Germanic religion was not as innovative as is commonly supposed.


  \subsection{Structure}
    To describe the structure of Germanic alliterative poetry it is first necessary to describe the general characteristics of the common Germanic alliterative meter as found in the earliest runic inscriptions, West Germanic poetry, and Norse Eddic poetry in the meter \Fornyrdislag.  The following description is thus not exhaustive, but does describe the bulk of the corpus.

    Alliterative poetry has two primary structural elements: \emph{stress} and \emph{alliteration}.  The exact count of syllables is less important, and end-rhyme is only used as a sporadic flourish.

    Based on alliteration, words are grouped into \emph{lines}.

    \subsubsection{Stress}
    Each syllable is classed as having primary stress (p), secondary stress (s), or no stress (x).  Syllables with p and s are stressed syllables, those with x are unstressed syllables.

    Primary stress is reserved for the root syllable in a word, which is not always the same as the first syllable.  Compare the English word \emph{beginning}, where the stress pattern is xPx; the primary stress falls on the syllable \emph{ginn-}.  Note that not all primary stresses are equal; the general rule is that nouns and adjectives have stronger stress than verbs.

    Secondary stress falls on the second element in a compound word, e.g. \emph{hęimr} in \emph{jǫtun-hęimr} ‘Ettin-ham’.

    No stress falls on inflectional syllables, pronouns and prepositions.  The latter categories of words can however take primary stress if it coincides with semantic stress, e.g. in a statement like “It was \emph{you}!”, or when a preposition is placed after the noun it modifies (e.g. \Beowulf\ 19b: \emph{Scęde-landum in} ‘in the Northern lands’)

    \subsubsection{Alliteration}
    Alliteration is the resonance between two stressed syllables beginning with the same phoneme, e.g. \emph{\alst{s}and} with \emph{re\alst{c}eive}, or \emph{\alst{g}reat} with \emph{be\alst{g}in}.

    Any vowel or diphthong can alliterate with any other vowel or dipththong.

    \emph{s} and the clusters \emph{sk, sp} and \emph{st} are counted as four distinct “sounds”.

    Further, in West Germanic poetry, \emph{g} and \emph{j} can alliterate with each other.

    In Norse poetry \emph{j} and \emph{v} can alliterate with vowels, the former always, the latter sporadically.

    In the present edition alliterating sounds are marked with red font.

    \subsubsection{Positions}

    The smallest metrical division is the \emph{position}, a concept related but not identical to the syllable.  The position must be understood as broader than the syllable as several syllables can fill one position, whereas a single syllable can never fill multiple positions.

    A short syllable followed by another syllable can resolve into a single position.  Likewise several unstressed syllables can count as a single position.

    \subsubsection{Verses}

    A \emph{verse} or \emph{half-line} is made up of 4 positions—typically 2 stressed, 2 unstressed—arranged according to set patterns.  These patterns form the core of Sievers’ description of alliterative poetry, and is the description followed here.

    TODO: Sieversian types.

    \subsubsection{Lines}

    2 verses are linked together by shared alliteration to form a couplet or \emph{long-line} (often simply \emph{line}).  They are separated by a short pause or \emph{cæsura}.

    It is not uncommon for editions of Norse poetry to print each verse on its own line, e.g. \textlink{Voluspa}[2]/1–2

    \begin{quote}\begin{small}
      \alst{E}k man \alst{jǫ}tna \\
      \alst{á}r of borna, \\
      þȧ’s \alst{f}orðum mik \\
      \alst{f}ǿdda hǫfðu;
    \end{small}\end{quote}

    whereas editions of West Germanic poetry typically print each long-line on its own line separated by a space to designate the cæsura

    \begin{quote}\begin{small}
      \alst{E}k man \alst{jǫ}tna \hspace*{1em} \alst{á}r of borna, \\
      þȧ’s \alst{f}orðum mik \hspace*{1em} \alst{f}ǿdda hǫfðu;
    \end{small}\end{quote}

    The present edition follows the latter convention, but replaces the space with a bullet point.

    The first verse (or \emph{a-verse}) may have 1 or 2 alliterations, with preference for alliteration on its first position over the second.

    The second verse (or \emph{b-verse}) must have exactly 1 alliteration, almost always on its first stressed position.

    \subsubsection{Fits and stanzas}
    The long-line is the highest metrical division, but long-lines can be combined into complete poems according to either a \emph{stichic} or \emph{stanzaic} manner.

    West Germanic poetry is typically \emph{stichic}.  This means that long-lines simply follow one other, often with enjambment, so that a new sentence begins in the b-verse.

    In longer epic poems lines may be grouped into \emph{fits} which are sections or cantos.  The fit does not have a fixed length, but is generally around 70-85 lines long.  (The 3182-line \Beowulf\ is divided into 44 fits for an average of 72 lines per fit; the surviving 5983 lines of \textlink{Heliand} are divided into 71 for an average of 84 lines per fit.
    It is probably not a coincidence that the length of the fit is similar to the length of shorter legendary poems like \textlink{GudrunOne} or \textlink{Hildebrandslied}.

    In \textlink{Heliand} a new fit can begin on a b-verse.  This does not happen in \Beowulf.

    North Germanic poetry is always more or less \emph{stanzaic} and does not have fits.  Each \emph{stanza} consists of several long-lines, and enjambment is rare.  The regularity of the stanzas varies by poem and meter, but there is a general trend towards stanzas of exactly 4 lines.

    In 4-line stanzas there is generally a semantic division between lines 2 and 3; each group of 2 lines (4 verses) form a \emph{helming} or \emph{half-stanza}.


\section{The corpus}
  The scope of the present corpus is large, and encompasses most of the alliterative poetry extant in Old Germanic languages.  The poetry is divided into the following categories:
  \begin{enumerate}
    \item \textbf{Norse Mythic poetry}, i.e., that which directly treats the Germanic mythology.  This category is exclusively Norse for the simple reason that no West Germanic or Gothic mythic narrative poetry survive.
    \item \textbf{Norse Heroic poetry}, specifically the whole second half of the Codex Regius and then a few other works.  With a few exceptions, subject matter outside of the Walsing cycle is not included.
    \item \textbf{West Germanic Heroic Poetry} in Old English, Old Saxon, and Old High German.
    \item \textbf{Poetry on Christian subjects}.  This category includes explicitly Christian poems where the new religion or its mythology is at the core of the work.  Christian heroic poems depicting native legends, like \Beowulf\ and \textlink{Hildebrandslied}, are not included.
    \item \textbf{Galders}, i.e., alliterative spells and charms, both from runic inscriptions and mediæval manuscripts.
    \item \textbf{Miscellaneous runic poetry}, apart from that already edited under Galders above.
  \end{enumerate}

  \subsection{Exclusions}
    All Norse Scaldic poetry is excluded, as is the Eddic poetry found in the Saws of Icelanders and Saws of Antiquity (\emph{forn-aldar sǫgur}).  Three reasons justify this exclusion.  First, these two categories of poetry have already been edited by the \Skp\ series, in the case of the Scaldic poetry admirably so.  Second, the underlying poetry is often impossible to take out of its prose context.  Third, it is doubtful whether most of the Eddic poetry embedded in prose contexts ever had a life of its own or whether it was composed along with the prose.  For these reasons I would only edit the poetry in its original context, that is, edit the whole Saws, which is an undertaking that falls far outside of the scope of the present edition as an edition of Eddic poetry.

    Most non-legendary Old English poems are excluded.  This includes poems on historical subjects embedded in the Anglo-Saxon Chronicle, the Riddles of the Exeter Book, and the Biblical epics.

  \subsection{Manuscripts}

    See the introduction to each category.

    \subsubsection{Old English poetry}

    The edited Old English poetry primarily derives from a few manuscripts.  Particularly important are the Exeter Book and \Lacnunga.

    \subsubsection{Old Saxon and High German poetry}

    There are no collections of alliterative poetry in these languages; instead the manuscript situation will be disussed in the Introduction to each individual text.

  \subsection{Dating}

  Ways of dating poetry, which naturally differ greatly for each text and tradition, will be brought up in the respective introductions to sections or poems as seen fit.

  The general methodology for dating adopted by the present edition is however one that prioritises linguistic criteria rather than literary analysis or vague historical speulation.  It has been shown for both the Eddic (TODO: cite Males) and Old English poetic traditions (TODO: cite Neidorf Dating of Beowulf) that medieval scribes did not possess the necessary linguistic knowledge to produce archaizing poetry, or even to reproduce metrically required archaic features (e.g. the consonant cluster \emph{vr-} in ON, for which see \textlink{Havamal}[26]/2 n.), and a text that displays linguistic features belonging to a certain century is thus taken to have been composed during that century.  In this way the present methodology aligns closer with that upheld by scholars like Finnur Jónsson, R. D. Fulk, Leonard Neidorf, and Mikael Males and differs from that of Klaus von See or the “Toronto school” of Beowulf.

\section{The presentation of the text}

  The present edition is divided into two equally large parts, presented side by side.  Each stanza or group of verse lines is presented first in the original language, and immediately below in English translation.  In the right margin is a list of manuscript witnesses for the particular passage.  Below the English translation are two layers of notes.  The first contains critical notes, and the second contains all other notes and commentary.  Broader commentary or particularly important critical notes are linked to ALL, while more specific notes are linked to separate lines and lemmas.

  \subsection{The Old Germanic text}

    The present edition encompasses texts in four Old Germanic languages: Old Norse, Old English, Old Saxon, and Old High German.  All texts have been normalized according to the author’s standardised orthography for the respective language.  The orthographies are designed to follow three core principles:

    \begin{enumerate}
      \item A faithfulness to the spoken language at the time when the texts were written, and the distinctions demonstrably found therein.
      \item A respect for the etymological origin of words, and their distinctions.
      \item A striving for a uniform orthography across the various languages, so that the same sound, where possible, should be written with the same character.
    \end{enumerate}

    These choices often stand in conflict with the orthography of the original manuscripts and with most earlier philological tradition, whence there is some reason to justify them.  My goal is to render the texts themselves in a manner that gives as much philological information to the reader as possible—not to present a facsimile edition for students of paleography.  This follows the philological methods used for printing e.g. the \Rigveda, which is generally printed in an entirely scholarly latinized orthography, not the original \emph{Devanāgarī}.  Regardless, such important traits of the original manuscript tradition as the long \emph{ſ}, arbitrary punctuation, arbitrary spelling, and lack of line breaks, are seldom reproduced in modern editions of Old Germanic poetry.

    \subsubsection{General orthographic conventions}

    The following orthographic conventions are applied for all Old Germanic languages:

    \begin{enumerate}
    \item The voiceless dental fricative is always written with the letter \emph{þ}, never \emph{th}.
    \item Long vowels are marked with the acute accent, never the macron or circumflex;
    \item Excepting those long vowels which have their origin in earlier dipththongs, which are written with the circumflex.
    \item In compounds where the first element has primary stress the elements are separated with a dash (-);
    \item But where the first element is a preposition or unstressed prefix they are separated with an interpunct (·).
    \item Non-inflectional suffixes and derivations are marked with a double oblique hyphen (⸗).
    \end{enumerate}

    Below follow specifications for each specific language.

    \subsubsection{Normalization of Old Norse}

    My Old Norse orthography is inspired by \textcite{FinnurEdda} in that it strives for a more archaic form than that of the surviving mss.; a form that instead represents the poetry as it may (in many cases, must) originally have looked. For this reason, it often has more in common with the proposed orthography of the First Grammatical Treatise than with the standard Old Icelandic orthography seen in most editions. The following list describes the differences from the standard Old Icelandic orthography:

    \begin{enumerate}
    \item I distinguish between short \emph{e} (from etymological short \emph{e}) and short \emph{ę} (from etymological short \emph{a} + \emph{i}-umlaut).
    \item I distinguish between long \emph{á} and \emph{ǫ́}, as done by the First Grammatical Treatise.
    \item I use \emph{ǿ} and \emph{ę́} rather than the usual \emph{œ} and \emph{æ}, to represent the vowels descended from Proto-Norse \emph{ó} and \emph{á} after \emph{i}-umlaut (cf. the short \emph{ø, ę} < \emph{o, a} + \emph{i}-umlaut).
    \item I distinguish long nasal \emph{ȧ, ė, ï, ȯ, u̇} from long oral \emph{á, é, í, ó, ú}, as done in the First Grammatical Treatise.
    \item I restore the old \emph{s} (which in modern Scandinavian and even in most Old Norse manuscripts has become \emph{r}, but which is found in old manuscripts such as AM 237 a fol (c. 1150) and fossilized in forms like \emph{þaz} (i.e. \emph{þat’s}) in \Regius) in the words \emph{es} ‘which, that, where, when’, and in inflections of \emph{vesa} (later \emph{vera}) such as \emph{es} ‘is’ (3rd sg. pres. ind.) and \emph{vas} (3rd sg. pret. ind.).  The following forms of \emph{vesa} retain the \emph{r}, as it is there the result of Verner’s law, and not of this (much younger) sound change: the pl. pres. ind. (\emph{erum} etc.), the pl. pret. ind. (\emph{vǫ́rum} etc.), and the pl. pret. subj. (\emph{vę́rim} etc.)
    \item When metrically benefactory, I contract \emph{ek} ‘I’, \emph{eru} ‘are’, and \emph{es} ‘which; is’ to \emph{’k}, \emph{’ro} and \emph{’s}, respectively.
    \item I resolve \emph{z} to \emph{ts, ds, þs} or \emph{ðs} depending on its underlying elements;
    \item \emph{x} is however still used instead of \emph{ks};
    \item I follow \textcite{FinnurEdda} in distinguishing between the relative particle \emph{es} and the verb \emph{es}: the particle is appended to the previous word with only an apostrophe (e.g. \emph{hann’s} ‘he who’), while the verb is separated by a space (e.g. \emph{hann ’s} ‘he is’).
    \item Where metrically beneficient personal pronouns are removed, often silently (cf. Scaldic poetry, \textlink{Volundarkvida}[21]/1–2 vs. 23/3–4, \textlink{HelgakvidaOne}[32] vs. \textlink{HelgakvidaTwo}[18])
    \end{enumerate}

    \subsubsection{Normalization of Old Swedish and Danish}
    I employ the same conventions as those described for Old Norse above, including the marking of \emph{u}-mutated \emph{a} > \emph{ǫ}.  That this indeed occurred in the Eastern Nordic dialects is most proven by the third-person personal pronoun, which shows \emph{u}-mutation in such forms as Swedish \emph{honom} ‘him’ < \emph{hǫ̇num}, \emph{hon} ‘she’ < \emph{hǫ̇n}).

    According to rule 3 in the general orthographic conventions above, I distinguish between \emph{ǿ} (< \emph{ǿ}) and \emph{ø̂} (< \emph{au, ęy}); \emph{é} (< \emph{é}) and \emph{ê} (< \emph{ęi}).

    Where unstressed vowels have been reduced into an schwa-like sound spelled \emph{e}, this is written with \emph{ę}.

    \subsubsection{Normalization of Old English}
    I spell fronted or brightened etymological \emph{a} and \emph{á} with \emph{æ} and \emph{ǽ}, for instance in \emph{dæg} ‘day’ (< \emph{*dagaʀ}) and \emph{rǽd} ‘advice, counsel’ (< \emph{rádaʀ}).  These are contrasted with \emph{ę} and \emph{ę́}, which represent \emph{i}-mutated \emph{a} and \emph{á}, e.g. in \emph{ęllen} ‘zeal, courage’ (< \emph{*aljaną}).

    An assimilated \emph{n} is marked with an overpoint, like in rule 3 of Old Norse above.

    \subsubsection{Normalization of Old Saxon}

    \subsubsection{Normalization of Old High German}

  \subsection{The English translation}

    There is now a very large number of translations of the most popular alliterative poetic texts, namely \Beowulf\ and the \emph{Poetic Edda}.  These generally fall into two camps:
    \begin{enumerate}
      \item \emph{poetic} translations, which distort the precise meaning of the text for the sake of meter, often quite radically; and
      \item \emph{prose} translations, which nowise preserve the style or feeling of the original.
    \end{enumerate}

    Almost all translations, of both types, also tend toward the following inadequacies: obscuring or glossing over difficult technical and cultural terminology; rendering identically repeated phrases and words (formulae) differently at various places; and simplifying or rewriting kennings and other poetic expressions.  Even worse this is often done with little in the way of notes or commentary, to a point where the reader is sometimes left entirely oblivious to the sense of the original text.

    What sets my translation apart from previous English translations is that it aims to follow the style and register of the original text, without sacrificing the literal sense of the words.  This unfortunately means that literality and consistency at times must sometimes come at the cost of fluid idiomatic English, but it has the advantage of giving the reader an image of not just \emph{what} the original text actually says, but \emph{how} it says it.  The reader should keep in mind that he is in a very foreign land, that he is reading words ancient and long forgotten—not the \emph{New York Times}.

    Maybe this is a pointless effort? One could argue that a translation always is a betrayal, and that those truly interested in the exact meaning of every word in the original text should study just the original (in the original language).  While I do agree that the sufficiently interested reader should study the original texts in the languages in which they were written (something made much easier by the present edition with its notes and parallel edition), it is still a “hard ask” for those readers who are not philologically inclined, but instead students and scholars of history, comparative mythology and religion, anthropology, or literature; those who, for whatever reason, are interested in exploring the oldest poetic heritage of the Germanic peoples of northern Europe.

    \subsection{Anglish proper nouns}
      Perhaps the single most idiosyncratic part of the present translation will be its handling of proper nouns. I have opted to render all cultural and religious terms, names of places, heroes, gods, and other entities by their English cognates (thus \emph{Thunder} for Old Norse \emph{Þórr}) and where such do not exist, their philologically expected English (\emph{Anglish}) forms (e.g. \emph{wallow} for Old Norse \emph{vǫlva}).

      There are two reasons for this.  The first is ideological.  I believe that the Old Germanic myths and poems, their gods and heroes, are a shared heritage of Northern Europe.  When you translate texts from across Germany, England and Scandinavia you quickly come to notice how similar the diction is, how many names reappear. The Scandinavian \emph{Vǫlundr} is the same character as the English \emph{Wélund}; likewise Norse \emph{Óðinn} is the same as English \emph{Wóden}.  These are ultimately mere distinctions in pronunciation.

      The second is aesthetic.  Commonly accepted forms like \emph{Odin} and \emph{Thor} are debased.  They do not even represent the Old Norse pronunciation as accurately as possible within the constraints of English ortography (for instance, \emph{Odin} would be better anglicized as \emph{Othin}).  Many are also difficult for English speakers to pronounce, or lead to absurd confusions.  I shudder at hearing the word \emph{ę́sir} pronounced /aɪˈsɪ:ɹ/; even worse is when \emph{Ǫ́s-garðr} becomes “ass-guard”.
