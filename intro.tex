\title{%
  \Huge The \textsc{Old Germanic Scoldship}, \\
  \huge\emph{or, \\
  \textsc{Scandinavian, English} and \textsc{German Mythic} and \textsc{Heroic Alliterative Poetry, Newly Translated, Edited} and \textsc{Commented upon}} \\
  \emph{by} \\
  \Huge \textsc{Konrad Olof Lennart Rosenberg}; \\ \emph{also \textsc{Including} a \textsc{List} of \textsc{Poetic Formulæ}, and \textsc{Several Essays} on the \textsc{Ancient \\ Common Germanic Culture} \\ and \textsc{Worldview}.}}

\maketitle

\newpage\thispagestyle{empty}

\begin{center} The following people have been especially helpful in giving corrections and general feedback: Ęinarr, Nikhilasurya Dwibhashyam, Joseph S. Hopkins, John Newman, Trevor L. Payne, Thibault.\end{center}

\begin{center} \emph{\alst{V}ęl kęypts hlutar \hld\ hęf’k \alst{v}ęl notit; \\
\alst{f}ás es \alst{f}róðum vant; \\
því-at \alst{Ó}ð-rǿrir \hld\ es nú \alst{u}pp kominn \\
á \alst{a}lda vés \alst{ja}ðar} \\
(\emph{Háva mǫ́l} 106)\end{center}

\newpage\thispagestyle{empty}

\tableofcontents

\newpage

\thispagestyle{empty}\section{Abbreviations}
  \begin{itemize}% Manuscript sigla
    \item \AM\ = AM 748 I a 4° (https://handrit.is/manuscript/view/da/AM04-0748-I-a)
    \item \AMb\ = AM 748 I b 4° (https://handrit.is/manuscript/view/is/AM04-0748-Ib)
    \item \EddaBms\ = AM 757 a 4° (https://handrit.is/manuscript/view/is/AM04-0757a)
    \item \FlatMS\ = Flatsęyjarbók, GKS 1005 fol. (https://handrit.is/manuscript/view/is/GKS02-1005)
    \item \Hauksbok\ = Hauksbók, AM 544 4° (https://handrit.is/manuscript/view/en/AM04-0544)
    \item \VolsungaMS\ = NKS 1824 b 4° (https://onp.ku.dk/onp/onp.php?m9641)
    \item \Regius\ = Codex Regius of the Poetic Edda, GKS 2365 4° (https://eae.ku.dk/q.php?p=cr/poems)
    \item \RegiusProse\ = Codex Regius of the Prose Edda, GKS 2367 4° (https://handrit.is/manuscript/view/is/GKS04-2367)
    \item \Trajectinus\ = Codex Trajectinus, Traj 1374ˣ
    \item \Upsaliensis\ = Codex Upsaliensis, DG 11
    \item \Wormianus\ = Codex Wormianus, AM 242 fol. (https://clarino.uib.no/menota/text/menota/AM-242-fol)
  \end{itemize}

  \begin{itemize}% Languages
    \item Eng. = Modern English
    \item Ger. = Modern German
    \item Got. = Gotnish (or Gothic)
    \item Lomb. = Lombardic
    \item MHG = Middle High German
    \item OE = Old English
    \item OF = Old Frisian
    \item OHG = Old High German
    \item ON = Old Norse
    \item OS = Old Saxon
    \item OSwe. = Old Swedish
    \item PGmc. = Proto-Germanic
    \item PN = Proto-Norse
    \item PNWGmc. = Proto-North-West Germanic
  \end{itemize}

  \begin{itemize}% Grammar
    \item 1st = first-person
    \item 2nd = second-person
    \item 3rd = third-person
    \item acc. = accusative case
    \item cpd = compound
    \item dat. = dative case
    \item gen. = genitive case
    \item imper. = imperative mood
    \item ind. = indicative mood
    \item instr. = instrumental case
    \item nom. = nominative case
    \item pl. = plural number
    \item sg. = singular number
    \item subj. = subjunctive mood
  \end{itemize}

  \begin{itemize}% Other abbreviations
    \item cert. = certainly
    \item c. = circa
    \item cf. = \emph{confere}; compare
    \item corr. = corrected in the ms.
    \item e. = excerpt (not the whole stanza)
    \item ed. = edition, edited (by)
    \item e.g. = \emph{exemplio gratia}; for instance
    \item emend. = emendation, emended (by)
    \item fol., foll. = folio, folios
    \item i.e. = \emph{id est}; that is
    \item l., ll. = line, lines
    \item lit. = literally
    \item metr. emend. = emended based on (secure) metrical criteria
    \item ms., mss. = manuscript, manuscripts
    \item norm. = normalised from the ms. spelling
    \item om. = omitted by
    \item p., pp. = page, pages
    \item tr. = translation, translated (by)
    \item sens. emend. = emended based on sense
    \item st., sts. = stanza, stanzas
    \item viz. = \emph{vidēlicet}; namely, to wit
    \item wo. = without
    \item wrt. = with regard to
  \end{itemize}

\newpage

\bookStart{Introduction (INCOMPLETE!)}

\section{Introduction to poetry}
  Don't go too indepth on individual poems! Each one will have its own introduction.
  \subsection{Metrics and conventions}
    Alliteration
    Kennings
  \subsection{How can we know the age of the Eddic poems?}
    Linguistic criteria
    Archeological evidence
    Comparison with known Christian texts (Sólarljóð, Hugsvinnsmál)
    Snorri thought they were old
    Saxo had access to them
    Many of them clearly describe non-Icelandic surroundings
      Especially Hávamál is clearly Norwegian

  \subsubsection{The presentation of poetry}
    \begin{enumerate}
      \item Lines are broken at each long-line rather than each half-line.  This follows traditional practice for the publication of West Germanic poetry, while departing from that of Old Norse poetry.
      \item Cæsuræ are represented with the interpunct (·).
      \item Alliterating sounds are marked with red colour.
    \end{enumerate}

\section{Old Germanic culture}
  \subsection{Economy (fee)}
  \subsection{Morals}
    Honour, personal integrity
    Notes on the terms \emph{argr} and \emph{ęrgi}
  \subsection{Religion}
    Cosmic cycles
    Reincarnation
    Analogies with other Indo-European traditions

\section{About the present corpus}
  The scope of the present corpus is large, containing most alliterative poetry extant in Old Germanic languages.  It may be divided into the following categories:
  \begin{enumerate}
    \item \textbf{Mythic poetry}, i.e., that which directly treats the Germanic mythology; for historical reasons, the poetry in this category is exclusively written in Old Norse.  See also Galders, below.
    \item \textbf{Heroic poetry of the Codex Regius}.  Since the heroic portion of the Codex Regius forms a coherent text, it is edited in full in the format of the manuscript.
    \item \textbf{Other Heroic poetry}, i.e., heroic poetry from sources other than the Codex Regius.  This category includes heroic poetry in Old English and Old High German.
    \item \textbf{Galders}, i.e., alliterative spells and charms, both from runic inscriptions and from manuscripts.
    \item \textbf{Christian poetry}.  This category includes a few explicitly Christian poems, where the new religion is at the core of the work (thus Christianised heroic poems like \Beowulf\ and \Hildebrandslied\ are not included here).  This poetry has been included for its value in the study of poetic expression, and because it may still provide valuable cultural evidence, for instance in the form of glosses.
    \item \textbf{Runic poetry}, apart from that already edited under Galders above.
  \end{enumerate}

  \subsection{Exclusions}
    The corpora formed by the (non-mythological) Norse Scoldic corpus and the Norse poetry found in old legendary saws (the \emph{forn-aldar-sǫgur}) are explicitly excluded.  They have been excellently edited in the \Skp\ series, such that I, a single editor, could scarcely produce something as thorough.  The latter is problematic in another way.  Being entirely embedded in saws, the underlying poetry is often impossible to take out of its prose context, and in some cases one may ask whether it ever had a life of its own, or whether it were simply composed on occasion by the author.  For these reasons I think it would be more conscientious to simply edit the whole saws, rather than artificially extract the poetry found scattered therein.

  \subsection{Manuscripts}

    \subsubsection{Norse poetry}
    The so-called Eddic poetry is foremost found in two medieval Icelandic manuscripts.

    The first and most important is GKS 2365 4to, here \Regius. It dates to the 1270s and has 45 surviving leaves, containing TODO poems. Of these 10 are mythological; the rest heroic, dealing with legends mostly of the Migration Period. Notably, following fol. 32, there is a gap of missing pages in the heroic section, specifically cutting off \Sigrdrifumal. It is unclear how many leaves and poems are missing.
    \Regius\ is not just a compilation of poems, it shows editorial input as well. Several of the mythological poems are separated by short prose sections, which tie them together into a loose frame narrative, though it is clear from their style and composition that they are originally separate works. When it comes to the heroic poems long prose sections occur both within and between them, creating a \inx[C]{saw}-like prosimetrical form, where the prose in many cases holds up the poetry, rather than the reverse. The heroic half of \Regius\ clearly forms the basis for the later \VolsungaSaga. For further literature see TODO.

    The second ms. is AM 748 I a 4to, here \AM. It dates to the 1300s and is but a fragment, consisting of just 6 leaves. It contains only mythological poems, and in a different order from \Regius; unlike it there is no trace of a frame narrative. On the first two leaves are contained the final stanzas of \Harbardsljod\ (1r–v), the complete \Baldrsdraumar\ (1v–2r), and the first verses of \Skirnismal, after which a single leaf has been lost. The next four leaves follow eachother and contain the second half of \Vafthrudnismal, the complete \Grimnismal\ and \Hymiskvida, and the beginning of the prose introduction to \Volundarkvida. \AM\ is the only medieval manuscript attesting \Baldrsdraumar, and its variants of the poems attested in \Regius\ are clearly not copied from it, but rather derive from a common ancestor. This makes it very valuable for textual criticism. For further literature see TODO.

    Several Eddic poems are quoted in \Gylfaginning, namely (TODO): \Voluspa, \Vafthrudnismal, \Grimnismal.  The text also cites a few fragmentary Eddic stanzas, which are edited under “Eddic fragments from Snorre’s Edda”.  For \Gylfaginning\ I give variants from the following four main mss.:
    \begin{enumerate}
      \item The Codex Regius of the Prose Edda \RegiusProse\ (GKS 2367 4to; 1300-1350)
      \item The Codex Trajectinus \Trajectinus\ (Traj 1374; a c. 1595 paper copy of a ms. closely related to \RegiusProse.)
      \item The Codex Wormianus \Wormianus\ (AM 242 fol.; 1340–70)
      \item The Codex Upsaliensis \Upsaliensis\ (DG 11; 1300–25)
    \end{enumerate}

    For discussion on their internal stemmatics and origins I refer to \textcite{Haukur2017}.  When all employed witness mss. of \Gylfaginning\ agree on a reading, I use in the critical apparatus the siglum \GylfMS, which is thus equivalent to \RegiusProse\Trajectinus\Wormianus\Upsaliensis.

    A few other Eddic-style poems are also included.  One of them, \Rigsthula, partially survives in \Wormianus, though it is sadly incomplete (see its Introduction).  \Grottasongr\ is quoted in full in \Skaldskaparmal.  Other Eddic poems survive only in younger Icelandic paper mss., namely TODO.  While I have not consulted such paper mss. for poems attested in medieval mss., I have had to rely on them for these poems.  Of these poems it must be said, that their late attestation not necessarily proves them to be late \emph{compositions}.  This is most clearly shown by \Baldrsdraumar, which is first attested in the fragmentary \AM, and in longer form in later paper mss.  It thus cannot be excluded that some of these poems would have existed in other lost medieval mss., perhaps even on the now-lost pages of \Regius\ or \AM.

    \subsubsection{Old English poetry}

    The edited Old English poetry primarily derives from a few manuscripts.  Particularly important are the Exeter Book and \Lacnunga.


\section{About the original language edition}
  My goal with the edition of the texts has been to hold close to the original mss., without excessive emendation.  Still, emendation is inevitable, and where it has done it is (apart from any oversight on my part) always marked.

  \subsection{Normalization}
    In the present edition are found texts in four languages, namely Old Norse, Old English, Old Saxon, and Old High German.  All texts have been normalized according to my own orthography, which is based on two principles:
    \begin{enumerate}
      \item Faithfulness to the language at the time when the texts were written, and the distinctions found therein, without neglecting etymology.
      \item Striving for a uniform orthography across the various treated idioms, where the same etymological sound is generally written with the same character.
    \end{enumerate}
    Both of these choices entail disregarding local manuscript traditions and philological tradition, something I see as justified.  My goal is to render the texts themselves in a manner that gives as much information to the reader as possible—not to present a facsimile edition for students of paleography.  Anyway, such important traits of the original manuscript tradition as the long \emph{ſ}, arbitrary punctuation, arbitrary spelling, and lack of line breaks, are seldom reproduced in modern editions of Old Germanic poetry.

    \subsubsection{Normalization of Old Norse}
    The orthography is inspired by \textcite{FinnurEdda} in that it strives for a more archaic form than that of the surviving mss., one that instead represents the poetry as it may (in many cases, must) originally have looked. For this reason, it often has more in common with the proposed orthography of the First Grammatical Treatise than with the standard Old Icelandic orthography seen in most editions. The following list describes the differences from the standard Old Icelandic orthography:

    \begin{enumerate}
    \item I distinguish short \emph{e} (from etymological short \emph{e}) and short \emph{ę} (from etymological short \emph{a} + \emph{i}-umlaut).
    \item I distinguish long \emph{á} and \emph{ǫ́}, as done by the First Grammatical Treatise.
    \item I use \emph{ǿ} and \emph{ę́} rather than the traditional \emph{œ} and \emph{æ}, to represent the vowels descended from Proto-Norse \emph{ō} and \emph{ā} after \emph{i}-umlaut (cf. the short \emph{ø, ę} < \emph{o, a} + \emph{i}-umlaut).
    \item I distinguish long nasal \emph{ȧ, ė, ï, ȯ, u̇} from long oral \emph{á, é, í, ó, ú}, as done in the First Grammatical Treatise.
    \item I restore the old \emph{s}—which in modern Scandinavian and even in most Old Norse manuscripts has become \emph{r}, but which is found consistently in old manuscripts such as AM 237 a fol (c. 1150), and fossilized in forms like \emph{þaz} (i.e. \emph{þat’s}) in \Regius—in the words \emph{es} ‘which, that, where, when’, and in inflections of \emph{vesa} (later \emph{vera}) such as \emph{es} ‘is’ (3rd sg. pres. ind.) and \emph{vas} (3rd sg. pret. ind.). The following forms retain the \emph{r}, as it is there the result of Verner’s law, and not of this (much younger) sound change: the pl. pres. ind. (\emph{erum} \&c.), the pl. pret. ind. (\emph{vǫ́rum} \&c.), and the pl. pret. subj. (\emph{vę́rim} \&c.)
    \item When metrically benefactory, I contract \emph{ek} ‘I’, \emph{eru} ‘are’, and \emph{es} ‘which; is’ to \emph{’k}, \emph{’ru} and \emph{’s}, respectively.
    \item I use \textcite{FinnurEdda}’s way of distinguishing between the relative particle \emph{es} and the verb \emph{es}: the first is appended to the previous word with only an apostrophe (e.g. \emph{hann’s} ‘he who’), while the second is separated by a space (e.g. \emph{hann ’s} ‘he is’).
    \end{enumerate}

    \subsubsection{Normalization of Old Swedish and Danish}
    I employ the same conventions as those described for Old Norse above, including the marking of \emph{u}-mutated \emph{a} > \emph{ǫ} (that this was indeed found in the Eastern Nordic dialects is most clearly seen by the third-person personal pronoun, which shows \emph{u}-mutation in such forms as Swedish \emph{honom} ‘him’ < \emph{hǫ́num}, \emph{hon} ‘she’ < \emph{hǫ́n}).  Where diphthongs have been contracted into monophthongs, these are marked with a circumflex accent, giving \emph{ø̂} < \emph{au, ęy} and \emph{ê} < \emph{ęi}.  Where unstressed vowels have been reduced into an e-like sound, this is written with \emph{ę}.

    \subsubsection{Normalization of Old English}
    I write fronted or brightened etymological \emph{a} and \emph{á} with \emph{æ} and \emph{ǽ}, for instance in \emph{dæg} ‘day’ and \emph{rǽd} ‘advice, counsel’.  These are contrasted with \emph{ę} and \emph{ę́}, which represent i-mutated \emph{a} and \emph{á}.

    An assimilated \emph{n} is marked with an overpoint, as in rule 4 of the Old Norse orthography described above.

    \subsubsection{Normalization of Old Saxon}

    \subsubsection{Normalization of Old High German}

  \section{About the English translation}
    Point about literal translation for use by scholars of comparative mythology
      The “guiding star” of this translation effort has been literality and consistency. All previous translations (to my knowledge) have such issues as: rendering identically repeated phrases differently at various places; covering up or obscuring technical and cultural terminology; simplifying kennings and other expressions—and this often without notes, to a point where the original meaning is, at times, unrecognizable.
      While I wholly encourage all readers of sufficient interest to study Old Norse (and other ancient Germanic languages!), perhaps even using the present edition as a tool, I also realize that this is a demanding ask which not all interested students and scholars of comparative mythology, anthropology, literature, religion and other fields will be able to fulfill. I therefore want these groups to be able to have a text that is as close to the original as possible, at the very least when it regards sense and expression.
    \subsection{Anglish proper nouns}
      One of the most idiosyncratic parts of the present edition will be its handling of proper nouns. I have opted to render all cultural and religious terms, names of places, heroes, gods, and other entities by their English cognates (thus \emph{Thunder} for Old Norse \emph{Þórr}) and where such do not exist, their philologically expected English (\emph{Anglish}) forms (e.g. \emph{wallow} for Old Norse \emph{vǫlva}).
      One reason for this is ideological. I believe that these myths and poems are a common Germanic or Northern European heritage, and should be treated as such. The English once knew gods such as Weden and Thunder, and called them by names naturally evolved in their language. So too did the Germans and Scandinavians, of course, and I would hope that any translators into those languages would follow this spirit and render the names in their natural forms there as well.\footnote{For instance in German perhaps Wuten, Donner, Froh, in Swedish Oden, Tor, Frö.}
      Another is philological. Forms like Odin and Thor are, while now commonly accepted, debased. They do not even represent the Old Norse pronunciation as accurate as would be possible (for instance, Odin would be better anglicized as Othin; the dental fricative still survives in English!), and many are difficult for English speakers to pronounce. I shudder when hearing a word like \emph{ę́sir} pronounced /aɪˈsɪ:ɹ/

  \printbibliography% Does it work?
