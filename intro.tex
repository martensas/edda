\title{%
  A New Critical Edition and Translation of the Poetic Edda \\
  \large Along with Commentary, Fragments, Spells and a Few Other Old Germanic Poems}

\author{Translated and Edited by Konrad O.L. Rosenberg}

\maketitle

\newpage

\begin{center} The following people have been especially helpful in giving corrections and general feedback: Nikhil S. Dwibhashyam, Joseph S. Hopkins, John Newman, Trevor L. Payne, Thibault.\end{center}

\newpage

\thispagestyle{empty}\tableofcontents

\newpage

\thispagestyle{empty}\section{Abbreviations}
  \begin{itemize}
    \item 1st = first-person
    \item 2nd = second-person
    \item 3rd = third-person
    \item acc. = accusative case
    \item cert. = certainly
    \item c. = circa
    \item cf. = confer
    \item corr. = corrected in the ms.
    \item cpd = compound
    \item dat. = dative case
    \item e. = excerpt (not the whole stanza)
    \item ed. = edition, edited (by)
    \item e.g. = \emph{exemplio gratia}; for instance
    \item emend. = emended by
    \item fol. = folio
    \item gen. = genitive case
    \item imper. = imperative
    \item i.e. = \emph{id est}; that is
    \item l. = line
    \item ll. = lines
    \item lit. = literally
    \item Lomb. = Lombardic
    \item metr. emend. = metrical emendation
    \item MHG. = Middle High German
    \item ms. = manuscript
    \item mss. = manucsripts
    \item nom. = nominative case
    \item norm. = normalized from the ms. spelling
    \item OE = Old English
    \item OF = Old Frisian
    \item OHG = Old High German
    \item om. = omitted in
    \item ON = Old Norse
    \item OS = Old Saxon
    \item p. = page
    \item PGmc. = Proto-Germanic
    \item pl. = plural number
    \item PN. = Proto-Norse
    \item PNWGmc. = Proto-North-West Germanic
    \item sg. = singular number
    \item tr. = translation, translated (by)
    \item st. = stanza
    \item sts. = stanzas
    \item viz. = namely
    \item wo. = without
    \item wrt. = with regard to
  \end{itemize}

\newpage

\bookStart{Introduction (INCOMPLETE!)}

\section{Introduction to Eddic poetry}
  Don't go too indepth on individual poems! Each one will have its own introduction.
  \subsection{Metrics and conventions}
    Alliteration
    Kennings
  \subsection{How can we know the age of the Eddic poems?}
    Linguistic criteria
    Archeological evidence
    Comparison with known Christian texts (Sólarljóð, Hugsvinnsmál)
    Snorri thought they were old
    Saxo had access to them
    Many of them clearly describe non-Icelandic surroundings
      Especially Hávamál is clearly Norwegian

\section{Ancient Germanic cult(ure)}
  \subsection{Economy (fee)}
  \subsection{Morals}
    Honour, personal integrity
    Notes on the terms \emph{argr} and \emph{ergi}
  \subsection{Religious conceptions}
    Cosmic cycles
    Reincarnation
    Analogies with other Indo-European traditions

\section{Notes to English translation}
  Point about literal translation for use by scholars of comparative mythology
    The “guiding star” of this translation effort has been literality and consistency. All previous translations (to my knowledge) have such issues as: rendering identically repeated phrases differently at various places; covering up or obscuring technical and cultural terminology; simplifying kennings and other expressions—and this often without notes, to a point where the original meaning is, at times, unrecognizable.
    While I wholly encourage all readers of sufficient interest to study Old Norse (and other ancient Germanic languages!), perhaps even using the present edition as a tool, I also realize that this is a demanding ask which not all interested students and scholars of comparative mythology, anthropology, literature, religion and other fields will be able to fulfill. I therefore want these groups to be able to have a text that is as close to the original as possible, at the very least when it regards sense and expression.
  \subsection{Anglish proper nouns}
    One of the most idiosyncratic parts of the present edition will be its handling of proper nouns. I have opted to render all cultural and religious terms, names of places, heroes, gods, and other entities by their English cognates (thus \emph{Thunder} for Old Norse \emph{Þórr}) and where such do not exist, their philologically expected English (\emph{Anglish}) forms (e.g. \emph{wallow} for Old Norse \emph{vǫlva}).
    One reason for this is ideological. I believe that these myths and poems are a common Germanic or Northern European heritage, and should be treated as such. The English once knew gods such as Weden and Thunder, and called them by names naturally evolved in their language. So too did the Germans and Scandinavians, of course, and I would hope that any translators into those languages would follow this spirit and render the names in their natural forms there as well.\footnote{For instance in German perhaps Wuten, Donner, Froh, in Swedish Oden, Tor, Frö.}
    Another is philological. Forms like Odin and Thor are, while now commonly accepted, debased. They do not even represent the Old Norse pronunciation as accurate as would be possible (for instance, Odin would be better anglicized as Othin; the dental fricative still survives in English!), and many are difficult for English speakers to pronounce. I shudder when hearing a word like \emph{ę́sir} pronounced /aɪˈsɪ:ɹ/

\section{Notes to critical edition}
  My goal with the critical editing of the texts has been to produce something as close to the original mss. as possible, without excessive emendation to the preserved recension(s). There are texts in three languages in the present edition, namely Old Norse, Old English and Old High German. Old Norse texts have been normalized according to roughly the same orthography as \textcite{FinnurEdda}. On the other hand the Old High German and Old English texts have only been lightly normalized, correcting obvious errors and marking vowel length with acute accents.

  \subsection{Normalization}
    The general principle in normalizing texts has been to strive for a uniform orthography across languages, where the same sound is written with the same character. This of course means disregarding local manuscript traditions and philological tradition, but I see this as justified. My goal is to render the texts themselves in a manner that gives as much information to the reader as possible—not to present a facsimile edition for students of paleography. Anyway, such obvious aspects of the original manuscripts as the long \emph{ſ}, arbitrary punctuation, arbitrary spelling, and lack of line breaks are almost never reproduced in modern editions of Old Germanic poetry.

    \subsubsection{Normalization of poetry}
    \begin{enumerate}
      \item Lines are broken at each long-line, not each half-line. This follows traditional practice for the publication of West Germanic poetry, while departing from that of Old Norse poetry.
      \item Cæsuræ are represented with the interpunct (·).
      \item Alliterations are marked with red colour.
    \end{enumerate}

    \subsubsection{Normalization of Old West Norse}
    The orthography is inspired by \textcite{FinnurEdda} in that it strives for a more archaic form than that of the surviving mss., one that instead represents the poetry as it may (in many cases, must) originally have looked. For this reason, it often has more in common with the proposed orthography of the First Grammatical Treatise than with the standard Old Icelandic orthography seen in most editions. The following list describes the differences from the standard orthography.

    \begin{enumerate}
    \item I distinguish short \emph{e} (from etymological short \emph{e}) and short \emph{ę} (from etymological short \emph{a} + \emph{i}-umlaut).
    \item I distinguish long \emph{á} and \emph{ǫ́}, as done by the First Grammatical Treatise.
    \item I use \emph{ǿ} and \emph{ę́} rather than the traditional \emph{œ} and \emph{æ}, to represent the vowels descended from Proto-Norse \emph{ō} and \emph{ā} after \emph{i}-umlaut (cf. the short \emph{ø, ę} < \emph{o, a} + \emph{i}-umlaut).
    \item I distinguish long nasal \emph{ȧ, ė, ï, ȯ, u̇} from long oral \emph{á, é, í, ó, ú}, as done by the First Grammatical Treatise.
    \item I restore the old \emph{s}—which in modern Scandinavian and even in most Old Norse manuscripts has become \emph{r}, but which is found consistently in old manuscripts such as AM 237 a fol (c. 1150), and fossilized in forms like \emph{þaz} (i.e. \emph{þat’s}) in \Regius—in the words \emph{es} ‘which, that, where, when’, and in inflections of \emph{vesa} (later \emph{vera}) such as \emph{es} ‘is’ (3rd sg. pres. ind.) and \emph{vas} (3rd sg. pret. ind.). The following forms retain the \emph{r}, as it is there the result of Verner’s law, and not of this (much younger) sound change: the pl. pres. ind. (\emph{erum} \&c.), the pl. pret. ind. (\emph{vǫ́rum} \&c.), and the pl. pret. subj. (\emph{vę́rim} \&c.)
    \item When metrically benefactory, I contract \emph{ek} ‘I’, \emph{eru} ‘are’, and \emph{es} ‘which; is’ to \emph{’k}, \emph{’ru} and \emph{’s}, respectively.
    \item I use \textcite{FinnurEdda}’s way of distinguishing between the relative particle \emph{es} and the verb \emph{es}: the first is appended to the previous word with only an apostrophe (e.g. \emph{hann’s} ‘he who’), while the second is separated by a space (e.g. \emph{hann ’s} ‘he is’).
    \end{enumerate}

    \subsubsection{Normalization of Old English}

    \subsubsection{Normalization of Old High German}

  \subsection{Manuscripts}

    \subsubsection{Eddic poetry}
    There are two surviving ancient mss. which contain full Eddic poems.

    The first and most important is GKS 2365 4to, here \Regius. It dates to the 1270s and has 45 surviving leaves, containing TODO poems. Of these 10 are mythological, and the rest heroic, dealing with legends mostly of the Migration Period. Notably, following fol. 32, there is a large gap of missing pages. This occurs in the heroic section, specifically cutting off \Sigrdrifumal. It is unclear how many leaves and poems went missing.
    \Regius\ is not just a compilation of poems, it shows editorial input as well. Several of the mythological poems are separated by short prose sections, which tie them together into a loose frame narrative, though it is clear from their style and composition that they are originally separate works. When it comes to the heroic poems long prose sections occur both within and between them, creating a \inx[C]{saw}-like narrative where the prose in many cases holds up the poetry, rather than the reverse. For further literature see TODO.

    The second ms. is AM 748 I a 4to, here \AM. It dates to the 1300s and is but a fragment, consisting of just 6 leaves. It contains only mythological poems, and in a different order from \Regius; unlike it there is no trace of a frame narrative. On the first two leaves are contained the final stanzas of \Harbardsljod\ (1r–v), the complete \Baldrsdraumar\ (1v–2r), and the first verses of \Skirnismal, after which a single leaf has been lost. The next four leaves follow eachother and contain the second half of \Vafthrudnismal, the complete \Grimnismal\ and \Hymiskvida, and the beginning of the prose introduction to \Volundarkvida. \AM\ is the only medieval manuscript attesting \Baldrsdraumar, and its variants of the poems attested in \Regius\ are clearly not copied from it, but rather derive from a common ancestor. This makes it very valuable for textual criticism. For further literature see TODO.

    Several Eddic poems are quoted in \Gylfaginning, namely (TODO): \Voluspa, \Vafthrudnismal, \Grimnismal. The text also quotes a few fragmentary verses of Eddic character (possibly from lost Eddic poems), which have here been edited together with their surrounding prose passages. For \Gylfaginning\ I have relied on the following four main mss.:\begin{enumerate}
	   \item The Codex Regius of the Prose Edda \RegiusProse\ (GKS 2367 4to; 1300-1350)
     \item The Codex Trajectinus \Trajectinus\ (Traj 1374; a c. 1595 paper copy of a ms. closely related to \RegiusProse.)
     \item The Codex Wormianus \Wormianus\ (AM 242 fol.; 1340–70)
     \item The Codex Upsaliensis \Upsaliensis\ (DG 11; 1300–25)\end{enumerate}

    For discussion on their internal stemmatics and origins I refer to \textcite{Haukur2017}. When all employed witness mss. of \Gylfaginning\ agree on a reading the siglum \GylfMS\ is used in the critical apparatus, which is thus equivalent to \RegiusProse\Trajectinus\Wormianus\Upsaliensis.

    A few other Eddic poems have also been edited. One of them, \Rigsthula, only survives in \Wormianus, though it is sadly incomplete (see its Introduction). Other Eddic poems survive only in younger paper mss., namely: TODO. While I have not consulted these paper mss. for poems attested in medieval mss., I have had to rely on them for these poems. Their exclusive survival there does not necessarily prove them to be late antiquarian works, as is clearly shown by \Baldrsdraumar, which among medieval mss. is only attested in the fragmentary \AM. It thus cannot be excluded that some of these poems would have existed in other lost medieval mss., perhaps even in the lost pages of \Regius\ or \AM.

    \subsubsection{West Germanic poetry}

    As none of the West Germanic poems edited here (TODO: Will we be editing other poems than Hildebrandslied?) survive in more than one copy, the specific details of their transmission is discussed in their individual Introductions.

  \printbibliography% Does it work?
