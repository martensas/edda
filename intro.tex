Introduction to Eddic poetry
  Don't go too indepth on individual poems! Each one will have its own introduction.
  Metrics and conventions
    Alliteration
    Kennings
  How can we know the age of the Eddic poems?
    Linguistic criteria
    Archeological evidence
    Comparison with known Christian texts (Sólarljóð, Hugsvinnsmál)
    Snorri thought they were old
    Saxo had access to them
    Many of them clearly describe non-Icelandic surroundings
      Especially Hávamál is clearly Norwegian

Ancient Germanic cult(ure)
  Honour, personal integrity
  Notes on the terms \emph{argr} and \emph{ergi}
  Religious conceptions
    Cosmic cycles
    Reincarnation
    Analogies with other Indo-European traditions

Notes to translation
  Why Anglish names?
  Point about literal translation for use by scholars of comparative mythology

Notes to critical edition (TODO: move from introduction to \Voluspa)
  Relevant manuscripts and which poems in each
    R = GKS 2365 4to
    A = AM 748 I a 4to
    Prose Edda group:
    S
    T
    W
    U
    Paper manuscripts? (Fjǫlsvinnsmál, Baldrs draumar)
  
Bibliography and sigla
  TODO: Probably move this from main.tex

Abbreviations
  wo. = without
