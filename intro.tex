Introduction to Eddic poetry
  Don't go too indepth on individual poems! Each one will have its own introduction.
  Metrics and conventions
    Alliteration
    Kennings
  How can we know the age of the Eddic poems?
    Linguistic criteria
    Archeological evidence
    Comparison with known Christian texts (Sólarljóð, Hugsvinnsmál)
    Snorri thought they were old
    Saxo had access to them
    Many of them clearly describe non-Icelandic surroundings
      Especially Hávamál is clearly Norwegian

Ancient Germanic cult(ure)
  Honour, personal integrity
  Notes on the terms \emph{argr} and \emph{ergi}
  Religious conceptions
    Cosmic cycles
    Reincarnation
    Analogies with other Indo-European traditions

Notes to translation
  Why Anglish names?
  Point about literal translation for use by scholars of comparative mythology

Notes to critical edition (TODO: move from introduction to \Voluspa)
  Relevant manuscripts and which poems in each
    R = GKS 2365 4to
    A = AM 748 I a 4to
    
    The noted mss. of \Gylfaginning\ are:\begin{enumerate} %TODO: move this list to introduction since plenty more poems are quoted in it.
	\item The Codex Regius of the Prose Edda \RegiusProse\ (GKS 2367 4to; 1300-1350)
	\item The Codex Trajectinus \Trajectinus\ (Traj 1374; a c. 1595 paper copy of a ms. closely related to \RegiusProse.)
	\item The Codex Wormianus \Wormianus\ (AM 242 fol.; 1340–70)
	\item The Codex Upsaliensis \Upsaliensis\ (DG 11; 1300–25)
\end{enumerate}

Thus, \GylfMS\ is equivalent to \RegiusProse\Trajectinus\Wormianus\Upsaliensis. — I further refer to Haukur 2017 for a very useful overview.
    Paper manuscripts? (Fjǫlsvinnsmál, Baldrs draumar)
  
Bibliography and sigla
  % TODO: Generate from bibliography.bib
  
Abbreviations
  wo. = without
