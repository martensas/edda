Introduction to Eddic poetry
  Don't go too indepth on individual poems! Each one will have its own introduction.
  Metrics and conventions
    Alliteration
    Kennings
  How can we know the age of the Eddic poems?
    Linguistic criteria
    Archeological evidence
    Comparison with known Christian texts (Sólarljóð, Hugsvinnsmál)
    Snorri thought they were old
    Saxo had access to them
    Many of them clearly describe non-Icelandic surroundings
      Especially Hávamál is clearly Norwegian

Ancient Germanic cult(ure)
  Honour, personal integrity
  Notes on the terms \emph{argr} and \emph{ergi}
  Religious conceptions
    Cosmic cycles
    Reincarnation
    Analogies with other Indo-European traditions

Notes to translation
  Why Anglish names?
  Point about literal translation for use by scholars of comparative mythology

Notes to critical edition
  There are two surviving ancient manuscripts containing full mythological Eddic poems.
  The \emph{first} and most important is GKS 2365 4to, here \Regius. It dates to the 1270s and has 45 surviving leaves, containing TODO poems. Of these 10 are mythological, and the rest heroic, dealing with legends mostly of the Migration Period. Notably, following fol. 32, there is a large gap of missing pages. This occurs in the heroic section, specifically cutting off \Sigrdrifumal. It is unclear how many leaves and poems went missing.
  \Regius\ is not just a compilation of poems, it shows editorial input as well. Several of the mythological poems are separated by short prose sections, which tie them together into a loose frame narrative, though it is clear from their style and composition that they are originally separate works. When it comes to the heroic poems long prose sections occur both within and between them, creating a \inx{saw}-like narrative where the prose in many cases holds up the poetry, rather than the reverse. For further literature see TODO.
  The \emph{second} manuscript is AM 748 I a 4to, here \AM. It dates to the 1300s and has just 6 leaves. \AM, as we have it, contains only mythological poems, and in a different order from \Regius; it has no frame narrative. On the first two leaves are contained \Harbardsljod\ (which lacks its beginning), \Baldrsdraumar\ and \Skirnismal\ (lacking its ending). After this some number of leaves have gone missing, but the other four leaves follow eachother. On them we find \Vafthrudnismal\ (lacking its beginning), \Grimnismal\, \Hymiskvida\ and the prose introduction of \Volundarkvida. \AM\ is the only medieval manuscript attesting \Baldrsdraumar, and further its other poems are not copied from \Regius, but rather derive from a shared ancestor. This fact makes it very valuable for textual criticism. For further literature see TODO.
  Some Eddic poems survive only in younger paper manuscripts. These being: TODO. While I have not consulted the paper mss. for poems attested in medieval mss., I have had to rely on them for these poems. Their exclusive survival here does not \emph{necessarily} prove them works of late antiquarians, had we not been fortunate enough to have \Baldrsdraumar\ in \AM, it would have been counted among them, yet we now know that it is not the case. It is not an impossibility that other poems only found in paper mss. would have survived in now lost medieval mss., perhaps even in the lost pages of \AM.
  
  Several Eddic poems are quoted in \Gylfaginning, in particular (TODO: list them all) \Voluspa, \Vafthrudnismal, \Grimnismal. \Gylfaginning\ has four main mss.:\begin{enumerate}
	\item The Codex Regius of the Prose Edda \RegiusProse\ (GKS 2367 4to; 1300-1350)
	\item The Codex Trajectinus \Trajectinus\ (Traj 1374; a c. 1595 paper copy of a ms. closely related to \RegiusProse.)
	\item The Codex Wormianus \Wormianus\ (AM 242 fol.; 1340–70)
	\item The Codex Upsaliensis \Upsaliensis\ (DG 11; 1300–25)
\end{enumerate}

  For sake of brevity I refer to these four collectively as \GylfMS, thus being equivalent \RegiusProse\Trajectinus\Wormianus\Upsaliensis. I refer to Haukur Þorgeirsson 2017 for discussion on their internal stemmatics and origins.
  
Bibliography and sigla
  % TODO: Generate from bibliography.bib
  
Abbreviations
  wo. = without
