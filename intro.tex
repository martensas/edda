\title{The Northern Epics:}
\author{Konrad O. L. Rosenberg}

\begin{titlingpage}
  \makeatletter
  \centering
  \HUGE \@title \\
  \Huge The Poetic Edda \\
  \Large and other Old Germanic alliterative poetry \\
  \large\emph{edited and translated by} \\
  \huge \@author \\
  \vspace{5cm}
  \Large Compiled \today. \\
  THIS IS A WORK IN PROGRESS AND MAY BE OUTDATED. The reader is strongly encourage to periodically download the newest edition from https://github.com/martensas/edda.
  \makeatother
\end{titlingpage}

\newpage\thispagestyle{empty}

\begin{center}{\Large \emph{Dęyr fé, \hld\ dęyja frę́ndr, \\
dęyr sjalfr hit sama; \\
ek vęit ęinn \hld\ at aldri-gi dęyr \\
dómr of dauðan hvęrn.}} \\

(\emph{High} 77)\end{center}

\vspace{5mm}

\begin{center}{\Large \emph{Vęl kęypts hlutar \hld\ hęf’k vęl notit; \\
fás es fróðum vant; \\
því-at Óð-rǿrir \hld\ es nú upp kominn \\
á alda vés jaðar.}} \\

(\emph{High} 106)\end{center}

\vspace{2cm}

The following people have been especially helpful in giving suggestions and corrections: Ęinarr, Nikhilasurya Dwibhashyam, Joseph S. Hopkins, John Newman, Trevor L. Payne, Thibault.

\newpage\thispagestyle{empty}

\tableofcontents

\newpage

\thispagestyle{empty}

\newpage

\bookStart{Abbreviations}
  \subsection{Languages}
\begin{itemize}%
	\item Eng. = Modern English
	\item Ger. = Modern German
	\item Got. = Gotnish (or Gothic)
	\item Lomb. = Lombardic
	\item MHG = Middle High German
	\item OE = Old English
	\item OF = Old Frisian
	\item OHG = Old High German
	\item ON = Old Norse
	\item OS = Old Saxon
	\item OSwe. = Old Swedish
	\item PGmc. = Proto-Germanic
	\item PN = Proto-Norse
	\item PNWGmc. = Proto-North-West Germanic
\end{itemize}

\subsection{Grammar}
\begin{itemize}%
	\item 1st = first-person
	\item 2nd = second-person
	\item 3rd = third-person
	\item acc. = accusative case
	\item cpd = compound
	\item dat. = dative case
	\item gen. = genitive case
	\item imper. = imperative mood
	\item ind. = indicative mood
	\item instr. = instrumental case
	\item nom. = nominative case
	\item pl. = plural number
	\item sg. = singular number
	\item subj. = subjunctive mood
\end{itemize}

\subsection{Other abbreviations}
\begin{itemize}%
	\item cert. = certainly
	\item c. = circa
	\item cf. = \emph{confere}; compare
	\item corr. = corrected in the ms.
	\item e. = excerpt (not the whole stanza)
	\item ed. = edition, edited (by)
	\item e.g. = \emph{exemplio gratia}; for instance
	\item emend. = emendation, emended (by)
	\item fol., foll. = folio, folios
	\item i.e. = \emph{id est}; that is
	\item l., ll. = line, lines
	\item lit. = literally
	\item metr. emend. = emended based on (secure) metrical criteria
	\item ms., mss. = manuscript, manuscripts
	\item norm. = normalised from the ms. spelling
	\item om. = omitted by
	\item p., pp. = page, pages
	\item tr. = translation, translated (by)
	\item sens. emend. = emended based on sense
	\item st., sts. = stanza, stanzas
	\item viz. = \emph{vidēlicet}; namely, to wit
	\item wo. = without
	\item wrt. = with regard to
\end{itemize}

% Old texts, primary sources
% The command codes must be as close to the original language titles as possible.
\subsection{Primary sources}

\newcommand{\Allvismal}{% Speeches of Allwise
	\emph{Alv}%
}
\newcommand{\Atlakvida}{% Lay of Attle
	\emph{Akv}%
}
\newcommand{\Atlamal}{% Speeches of Attle
	\emph{Am}%
}
\newcommand{\Baldrsdraumar}{% The Dreams of Balder
	\emph{Bdr}%
}
\newcommand{\Beowulf}{% Beewolf
	\emph{Beow}%
}
\newcommand{\Brot}{% Fragment of a Lay of Siward
	\emph{Brot}%
}
\newcommand{\Deor}{% Deer
	\emph{Deer}%
}
\newcommand{\EyrbyggjaSaga}{% Saw of Harware and Heathric
	\emph{Eb}%
}
\newcommand{\EgilsSaga}{% Saw of Harware and Heathric
	\emph{Eg}%
}
\newcommand{\Fafnismal}{% Speeches of Fathomer
	\emph{Fáfn}%
}
\newcommand{\FostrbroedhraSaga}{% Saw of the Foster-brothers
	\emph{FbrS}%
}

\newcommand{\FraLoka}{% From Lock
	\emph{From Lock}%
}%TODO: remove this

\newcommand{\Grettissaga}{% Saw of Gretter
	\emph{GrettS}%
}
\newcommand{\Grimnismal}{% Speeches of Grimner
	\emph{Grm}%
}
\newcommand{\Gripisspa}{% Spae of Griper
	\emph{Gríp}%
}
\newcommand{\Grottasongr}{% Song of Grotte
	\emph{Grotta}%
}
\newcommand{\Grougaldr}{% Galder of Growe
	\emph{Grg}%
}
\newcommand{\Gudrunarhvot}{% Guthrun’s Instigation
	\emph{Ghv}%
}
\newcommand{\GudrunOne}{% Guthrun’s First Lay
	\emph{Guðr I}%
}
\newcommand{\GudrunTwo}{% Guthrun’s Second Lay
	\emph{Guðr II}%
}
\newcommand{\GudrunThree}{% Guthrun’s Third Lay
	\emph{Guðr III}%
}
\newcommand{\Gulatingslog}{% Law of the Gole-thing
	\emph{Gula}%
}
\newcommand{\Gylfaginning}{% The Guiling of Yilver; for referring to Gylfaginning as a text
	\emph{Gylf}%
}
\newcommand{\Hakonarmal}{% Speeches of Hathkin
	\emph{Hákm}%
}

\newcommand{\HakonarSaga}{% Saw of Hathkin the good
	\emph{HákGóð}%
}
\newcommand{\Haleygjatal}{% Tally of the Hallowlendings
	\emph{HalT}%
}%TODO: Add

\newcommand{\Hamdismal}{% Speeches of Hamthew
	\emph{Hamð}%
}
\newcommand{\Harbardsljod}{% Leed of Hoarbeard
	\emph{Hárb}%
}
\newcommand{\Haustlong}{% Harvest-long
	\emph{Haustl}
}
\newcommand{\Havamal}{% Speeches of the High One
	\emph{Háv}%
}
\newcommand{\HelgakvidaHjorvardssonar}{% Lay of Hallow Harwardson
	\emph{HHj}%
}
\newcommand{\HelgakvidaOne}{% First Lay of Hallow Hundingsbane
	\emph{HHund I}%
}
\newcommand{\HelgakvidaTwo}{% Second Lay of Hallow Hundingsbane
	\emph{HHund II}%
}
\newcommand{\Heliand}{%
  \emph{Heli}%
}
\newcommand{\Helreid}{% Byrnhild’s Hell-ride
	\emph{Helr}%
}
\newcommand{\HervararSaga}{% Saw of Harware and Heathric
	\emph{HarS}%
}
\newcommand{\Hildebrandslied}{% Speeches of Hildbrand
	\emph{Hildebrand}%
}
\newcommand{\Hymiskvida}{% Lay of Hymer
	\emph{Hym}%
}
\newcommand{\Hyndluljod}{% Leed of Hindle
	\emph{Hdl}%
}

\newcommand{\Lacnunga}{% Leekning
	\emph{Lacning}%
}%TODO: add this

\newcommand{\Lokasenna}{% Flyting of Lock
	\emph{Lok}%
}

\newcommand{\Malshattakvadi}{
	\emph{Mhkv}
}%TODO: add this

\newcommand{\Mahabharata}{
	\emph{Mahā́bʰārata}
}
\newcommand{\MerseburgOne}{% First Merseburg charm
	\emph{Mers I}%
}
\newcommand{\MerseburgTwo}{% Second Merseburg charm
	\emph{Mers II}%
}
\newcommand{\Muspilli}{% Muspell
	\emph{Muspilli}%
}
\newcommand{\Oddrunargratr}{% Weeping of Ordrun
	\emph{Oddrgr}%
}
\newcommand{\Reginsmal}{% The Speeches of Rein
	\emph{Reg}%
}
\newcommand{\Rigsthula}{% Thule of Righ
	\emph{Rþ}%
}
\newcommand{\Rigveda}{%
	\emph{R̥V}%
}
\newcommand{\SaxonGenesis}{% Old Saxon Genesis
	\emph{OSGen}%
}
\newcommand{\Sigurdskamma}{% Short Lay of Siward
	\emph{Sigsk}%
}
\newcommand{\Sigrdrifumal}{% Speeches of Syedrive
	\emph{Sigrdr}%
}
\newcommand{\Skaldskaparmal}{% The Matter of Scoldship
	\emph{Skm}%
}
\newcommand{\Skirnismal}{% Speeches of Shirner
	\emph{Skm}%
}

\newcommand{\Sogubrot}{
	\emph{AncKings}
}%TODO: add
\newcommand{\Solarljod}{
	\emph{Sun}
}%TODO: add
\newcommand{\Sonatorrek}{
	\emph{Sont}
}%TODO: add
\newcommand{\Sorlathattr}{% Strand of Sarle
	\emph{Sarle}%
}%TODO: add
\newcommand{\ThidreksSaga}{% Saw of Thedrich
	\emph{ThidS}%
}%TODO: add

\newcommand{\Thorsdrapa}{% Drape of Thunder
	\emph{Þdr}%
}
\newcommand{\Thrymskvida}{% Lay of Thrim
	\emph{Þrk}%
}
\newcommand{\Vafthrudnismal}{% Speeches of Webthrithner
	\emph{Vafþ}%
}
\newcommand{\Volsathattr}{% Strand of Walse
	\emph{Vǫlsþ}%
}
\newcommand{\VolsungaSaga}{% Saw of the Walsings
	\emph{VǫlsS}%
}
\newcommand{\Volundarkvida}{% Lay of Wayland
	\emph{Vkv}%
}
\newcommand{\Voluspa}{% Spae of the Wallow
	\emph{Vsp}%
}

\newcommand{\Waldere}{% Walder
	\emph{Walder}%
}%TODO: add
\newcommand{\YnglingaSaga}{% Saw of the Inglings
	\emph{IngS}%
}%TODO: add
\newcommand{\Ynglingatal}{% Tally of the Inglings
	\emph{IngT}%
}%TODO: add

\begin{itemize}%
	\item \Allvismal\ = \emph{Allvíssmǫ́l} (Speeches of Allwise)
	\item \Atlakvida\ = \emph{Atlakviða} (Lay of Attle)
	\item \Atlamal\ = \emph{Atlamǫ́l} (Speeches of Attle)
	\item \Baldrsdraumar\ = \emph{Baldrs draumar} (Dreams of Balder)
	\item \Beowulf\ = \emph{Beowulf}
	\item \Brot\ = \emph{Brot af Sigurðarkviða} (Fragment of a Lay of Siward)
	\item \Deor\ = \emph{Déor} (Deer)
	\item \EyrbyggjaSaga\ = \emph{Eyrbyggja saga} (Saw of the Ere-dwellers)
	\item \Fafnismal\ = \emph{Fáfnismǫ́l} (Speeches of Fathomer)
	\item \FostrbroedhraSaga\ = \emph{Fóstrbrǿðra saga} (Saw of the Fosterbrothers)
	\item \Grettissaga\ = \emph{Grettis saga} (Saw of Gretter)
	\item \Grimnismal\ = \emph{Grímnis mǫ́l} (Speeches of Grimner)
	\item \Gripisspa\ = \emph{Grípisspǫ́} (Spae of Griper)
	\item \Grottasongr\ = \emph{Grottasǫngr} (Song of Grotte)
	\item \Grougaldr\ = \emph{Gróugaldr} (Galder of Growe)
	\item \Gudrunarhvot\ = \emph{Guðrúnarhvǫt} (Goading of Guthrun)
	\item \GudrunOne\ = \emph{Guðrúnarkviða I} (First Lay of Guthrun)
	\item \GudrunTwo\ = \emph{Guðrúnarkviða II} (Second Lay of Guthrun)
	\item \GudrunThree\ = \emph{Guðrúnarkviða III} (Third Lay of Guthrun)
	\item \Gulatingslog\ = \emph{Gulaþingslǫg} (Law of the Gole‑Thing)
	\item \Gylfaginning\ = \emph{Gylfaginning} (Beguiling of Yilver)
	\item \Hakonarmal\ = \emph{Hǫ́konarmǫ́l} (Speeches of Hathkin)
	\item \HakonarSaga\ = \emph{Hǫ́konar saga góða} (Saw of Hathkin the good)
	\item \Hamdismal\ = \emph{Hamðismǫ́l} (Speeches of Hamthew)
	\item \Harbardsljod\ = \emph{Hárbarðljóð} (Leeds of Hoarbeard)
	\item \Haustlong\ = \emph{Haustlǫng} (Harvest‑long)
	\item \Havamal\ = \emph{Hávamǫ́l} (Speeches of the High One)
	\item \HelgakvidaHjorvardssonar\ = \emph{Helgakviða Hjǫrvarðssonar} (Lay of Hallow Harwardson)
	\item \HelgakvidaOne\ = \emph{Helgakviða Hundingsbana I} (First Lay of Hallow Hundingsbane)
	\item \HelgakvidaTwo\ = \emph{Helgakviða Hundingsbana II} (Second Lay of Hallow Hundingsbane)
	\item \Heliand\ = \emph{Heliand}
	\item \Helreid\ = \emph{Helreið Brynhildar} (Hell‑ride of Byrnhild)
	\item \HervararSaga\ = \emph{Hervarar saga} (Saw of Harware and Heathric)
	\item \Hildebrandslied\ = \emph{Hildebrandslied}
	\item \Hymiskvida\ = \emph{Hymiskviða} (Lay of Hymer)
	\item \Hyndluljod\ = \emph{Hyndluljóð} (Leeds of Hindle)
	\item \Lokasenna\ = \emph{Lokasenna} (Flyting of Lock)
%	\item \Mahabharata\ = \emph{Mahā́bʰārata}
	\item \MerseburgOne\ = Merseburg galder I
	\item \MerseburgTwo\ = Merseburg galder II
	\item \Oddrunargratr\ = \emph{Oddrúnargrátr} (Weeping of Ordrun)
	\item \Reginsmal\ = \emph{Ręginsmǫ́l} (Speeches of Rein)
	\item \Rigsthula\ = \emph{Rigsþula} (Thule of Righ)
	\item \Rigveda\ = \emph{R̥g-vedá}, with translations from Jamison‑Brereton unless otherwise specified.
	\item \SaxonGenesis\ = \emph{Old Saxon Genesis}
	\item \Sigurdskamma\ = \emph{Sigurðarkviða skamma} (Short Lay of Siward)
	\item \Sigrdrifumal\ = \emph{Sigrdrífumǫ́l} (Speeches of Syedrive)
	\item \Skaldskaparmal\ = \emph{Skaldskaparmǫ́l} (Matter of Scoldship)
	\item \Skirnismal\ = \emph{Skírnismǫ́l} (Speeches of Shirner)
	\item \Thorsdrapa\ = \emph{Þórsdrápa} (Drape of Thunder)
	\item \Thrymskvida\ = \emph{Þrymskviða} (Lay of Thrim)
	\item \Vafthrudnismal\ = \emph{Vafþrúðnismǫ́l} (Speeches of Webthrithner)
	\item \Volsathattr\ = \emph{Vǫlsaþáttr} (Strand of Walse)
	\item \VolsungaSaga\ = \emph{Vǫlsunga saga} (Saw of the Walsings)
	\item \Volundarkvida\ = \emph{Vǫlundarkviða} (Lay of Wayland)
	\item \Voluspa\ = \emph{Vǫluspǫ́} (Spae of the Wallow)
\end{itemize}%


% Manuscripts
\newcommand{\AM}{% AM 748 I a 4to (https://handrit.is/manuscript/view/da/AM04-0748-I-a, https://books.google.se/books?id=L-MOAAAAQAAJ)
	\textbf{A}%
}
\newcommand{\AMb}{% AM 748 I b 4to (https://handrit.is/manuscript/view/is/AM04-0748-Ib)
	\textbf{A\textsubscript{b}}%
}
\newcommand{\EddaBms}{% AM 757 a 4° (https://handrit.is/manuscript/view/is/AM04-0757a)
	\textbf{B}%
}
\newcommand{\FlatMS}{% Flateyjarbok
	\textbf{F}%
}
\newcommand{\GylfMS}{% For referring to Gylfaginning manuscripts when stanzas are attested there.
	\textbf{G}%
}
\newcommand{\Hauksbok}{% Hauksbok
	\textbf{H}%
}
\newcommand{\VolsungaMS}{% NKS 1824 b 4° (https://skaldic.ku.dk/q?p=skp/mss/ms/512 and https://onp.ku.dk/onp/onp.php?b2195-53)
	\textbf{N}%
}
\newcommand{\Regius}{% Codex Regius (of the poetic edda)
	\textbf{R}%
}
\newcommand{\RegiusProse}{% Codex Regius of the Prose Edda
	\textbf{S}%
}
\newcommand{\Trajectinus}{% Codex Trajectinus
	\textbf{T}%
}
\newcommand{\Wormianus}{% Codex Wormianus (https://clarino.uib.no/menota/text/menota/AM-242-fol)
	\textbf{W}%
}
\newcommand{\Upsaliensis}{% Codex Upsaliensis
	\textbf{U}%
}
\newcommand{\HildMS}{% For referring to the Hildebrandslied manuscript.
	ms.%
}

\subsection{Manuscripts}
\begin{itemize}%
	\item \AM\ = AM 748 I a 4° (https://handrit.is/manuscript/view/da/AM04-0748-I-a)
	\item \AMb\ = AM 748 I b 4° (https://handrit.is/manuscript/view/is/AM04-0748-Ib)
	\item \EddaBms\ = AM 757 a 4° (https://handrit.is/manuscript/view/is/AM04-0757a)
	\item \FlatMS\ = Flatsęyjarbók, GKS 1005 fol. (https://handrit.is/manuscript/view/is/GKS02-1005)
	\item \GylfMS\ = all manuscripts of \Gylfaginning; equivalent to \RegiusProse\Trajectinus\Upsaliensis\Wormianus
	\item \Hauksbok\ = Hauksbók, AM 544 4° (https://handrit.is/manuscript/view/en/AM04-0544)
	\item \VolsungaMS\ = NKS 1824 b 4° (https://onp.ku.dk/onp/onp.php?m9641)
	\item \Regius\ = Codex Regius of the Poetic Edda, GKS 2365 4° (https://eae.ku.dk/q?p=eae/vols/text/1)
	\item \RegiusProse\ = Codex Regius of the Prose Edda, GKS 2367 4° (https://handrit.is/manuscript/view/is/GKS04-2367)
	\item \Trajectinus\ = Codex Trajectinus, Traj 1374ˣ
	\item \Upsaliensis\ = Codex Upsaliensis, DG 11
	\item \Wormianus\ = Codex Wormianus, AM 242 fol. (https://clarino.uib.no/menota/text/menota/AM-242-fol)
\end{itemize}

% Meters
\newcommand{\Drottkvett}{% Court-recited
	\emph{Court-recited meter}%
}
\newcommand{\Fornyrdislag}{% Law of Ancient Speeches
	\emph{Ancient-words-law}%
}
\newcommand{\Galdralag}{% Meter of Speeches
	\emph{Galders-law}%
}
\newcommand{\Ljodahattr}{% Meter of Leeds
	\emph{Leeds-meter}%
}
\newcommand{\Kviduhattr}{%
	\emph{Lay-meter}%
}
\newcommand{\Malahattr}{% Meter of Speeches
	\emph{Speeches-meter}%
}

%Modern books and editions (TODO: move these to bibliography)
\newcommand{\CV}{% Cleasby-Vigfússon dictionary of Old Norse
	\textciteshorttitle{CleasbyVigfusson}% \emph{C-V}%
}
\newcommand{\FGT}{% First Grammatical Treatise
	\textciteshorttitle{FGTHaugen}%
}
\newcommand{\ONP}{% Dictionary of Old Norse Prose
	\emph{ONP}%
}
\newcommand{\Skp}{% Skaldic Poetry of the Scandinavian Middle Ages
	\textciteshorttitle{SkP}%
}
%

\bookStart{Introduction (incomplete!)}

The introduction is currently very incomplete and many parts are just outlines.

\section{The Old Germanic world}

  \subsection{Lifestyle and economy}
    Cattle-based; small farmsteads.
  \subsection{Morals and Virtues}
    Honour, personal integrity
    Notes on the terms \emph{argr} and \emph{ęrgi}
  \subsection{Religion}
    Keeping the Powers happy
    Cosmic cycles
    Reincarnation
    Analogies with other Indo-European traditions


\section{Germanic alliterative poetry}

  \subsection{Historical significance}

  The historical-literary significance of the Old Germanic poetry is twofold. On the one hand it forms the oldest extensive monuments in its respective languages, and indeed the earliest indigenous Germanic literature (the Gothic being wholly derivative and translational).  It lays the ground for the \emph{Nibelungenlied} and Chaucer, who in turn precede such famous writers as Shakespeare and Wagner.  It forms the first and most important source of our knowledge about the ancient folk-life of Northern Europe.

  Oon the other hand it is by no means an innovative or newly created genre. Already, and perhaps especially, in our oldest sources the language is rich with expressions and images, many of great antiquity: "sea-stallions" sail across the ocean; the sun is drawn across Heaven in her chariot; feasts are held in great chiefly halls. These motifs are mirrored by Homer and the Rigveda, and must go back as far as the Bronze Age.

  The language likewise overflows with archaic poetic synonyms.  Indo-European words otherwise extinct in all Germanic languages find their last refuge in the alliterative poetry.  Such are the Old English \emph{eoh}, Old Norse \emph{jór}, corresponding to the Sanskrit \emph{áşva}, Latin \emph{equus}, all meaning ‘god’; Old Norse \emph{týr} ‘god’, corresponding to Sanskrit \emph{dēvá}, Latin \emph{deus}, all meaning ‘god’; Old English and Old Norse \emph{fold} 'earth, land', corresponding to Sanskrit \emph{pṛthivī́} 'id.'  The fact that many of these relate to the cult shows that the Germanic religion was not as innovative as is commonly supposed.

  The organizing poetic principle of alliteration must also have been in effect for some time. Even the earliest \emph{scalds} and \emph{scops} have dozens of synonyms for words like man, sword, horse, and hall. Needless to say, many of them—like \emph{jór} above—are very old, and only found in poetry.

  \subsection{Meter(s)}
    The Old Germanic poetry has two primary structural elements: \emph{stress} and \emph{alliteration}.  The exact count of syllables is less important, and end-rhyme is only used as a sporadic flourish.

    \subsubsection{Stress}
    When scanning alliterative meter each syllable is generally classed as having either primary stress (p), secondary stress (s), or no stress (x).

    Primary stress is reserved for the root syllable in a word, which is not always the same as the first syllable.  Compare the English word \emph{beginning}, where the stress pattern is xPx; the primary stress falls on the syllable \emph{ginn-}.

    Secondary stress falls on the second element in a compound word.

    Not all words have the same stress; the general rule is that nouns and adjectives have stronger stress than verbs, which in turn have stronger stress than prepositions and pronouns.  Where exceptions occur this coincides with semantic stress, e.g. in a statement like “It was \emph{you}!”

    \subsubsection{Alliteration}
    The following rules describe Germanic alliteration:

    \begin{enumerate}
      \item Alliteration is the resonance between two stressed syllables beginning with the same “sound”, e.g. \emph{\alst{s}and} with \emph{re\alst{c}eive}, or \emph{\alst{g}reat} with \emph{be\alst{g}in}.
      \item Any vowel or diphthong can alliterate with any other vowel or dipththong.
      \item \emph{s} and the clusters \emph{sk, sp} and \emph{st} are counted as four distinct “sounds”.
    \end{enumerate}

    Further, in West Germanic poetry,

    \begin{enumerate}[4.]
      \item \emph{g} and \emph{j} are treated as the same sound.
    \end{enumerate}

    In the present edition alliterating sounds are marked with red font.

    \subsubsection{Lines}
    Most alliterative poetry is written in the same common meter, which in Old Icelandic poetics gets the name \emph{fornyrðislag} ‘measure of ancient words’.  The smallest metrical division is the \emph{position}, a concept related but not identical to the syllable.  For instance, two short syllables (that is, one where a short vowel is followed by a single consonant) can \emph{resolve} into a single position.

    Four positions—two stressed, two unstressed—make up the normal \emph{half-line} or \emph{verse}.  Two half-lines separated by a short break or \emph{cæsura} (here represented by the interpunct “·”) form a couplet or \emph{long-line}.  The first half-line (or \emph{a-verse}) may have either one or two alliterations on the stressed positions, with preference for the first position over the second.  The second half-line (or \emph{b-verse}) must always have an alliteration on its first stressed position; never on its second.

    In the present edition each long-line is printed on a new line.  This is already standard for the publication of West Germanic poetry, whereas many editions of Scandinavian poetry print each half-line.

    \subsubsection{Fits and stanzas}
    Bigger structures are \emph{fits} and \emph{stanzas}.  The former are found in the Old Saxon and English traditions, the latter only in the Scandinavian.

    A \emph{fit} is a section or canto in a longer epic poem.  It does not have a fixed length, but is generally around 70-85 lines long.  Thus the 3182-line \Beowulf\ is divided into 44 fits (for an average of 72 lines per fit); the surviving 5983 lines of \Heliand\ are divided into 71 (for an average of 84 lines per fit).  It is probably not a coincidence that the length of the fit is similar to the length of shorter legendary poems like \GudrunOne\ or \Hildebrandslied.  In \Heliand\ a new fit can begin in the cæsura; this does not happen in \Beowulf.

    In Scandinavian poetry a \emph{stanza} is a group of long-lines, typically (but far from always) four.  The regularity of stanza-length varies from poem to poem.


  \subsection{The age of the Eddic poems}%TODO: move to specific introduction of Eddic
    Linguistic criteria
    Archeological evidence
    Comparison with known Christian texts (Sólarljóð, Hugsvinnsmál)
    Snorri thought they were old
    Saxo had access to them
    Many of them clearly describe non-Icelandic surroundings
      Especially Hávamál is clearly Norwegian


\section{The present corpus}
  The scope of the present corpus is large; when complete it will contain most alliterative poetry extant in Old Germanic languages.  The poetry is grouped into the following categories:
  \begin{enumerate}
    \item \textbf{Norse Mythic poetry}, i.e., that which directly treats the Germanic mythology.  This category is exclusively Norse for the simple reason that no West Germanic or Gothic mythic narrative poetry survive.
    \item \textbf{Heroic poetry of the Codex Regius}.  Since the heroic portion of the Codex Regius forms a coherent text, it is edited in full.
    \item \textbf{Other Norse Heroic poetry} from sources other than the Codex Regius.
    \item \textbf{West Germanic Heroic Poetry} in Old English and Old High German.
    \item \textbf{Galders}, i.e., alliterative spells and charms, both from runic inscriptions and latinate manuscripts.
    \item \textbf{Poetry on Christian subjects}.  This category includes explicitly Christian poems where the new religion or its stories are at the core of the work (Christian heroic poems depicting native legends, like \Beowulf\ and \Hildebrandslied, are not included here).
    \item \textbf{Runic poetry}, apart from that already edited under Galders above.
  \end{enumerate}

  \subsection{Exclusions}
    The (non-mythological) Norse alliterative poetry found in the saws of Icelanders and of ancient ages (\emph{forn-aldar-sǫgur}) is excluded.  It has already been admirably rendered in the \Skp\ series.  It would also require a somewhat different structure in terms of how it is rendered; the underlying poetry is often impossible to take out of its prose context, and in some cases it is questionable whether it ever existed on its own, or whether it was simply composed on by the prose author.  I think it would be more conscientious to edit the whole saws as \emph{prosimetra}; this falls outside of the scope of the present edition, but I am not adverse to such an undertaking in the future.

  \subsection{Manuscripts}

    \subsubsection{Norse Eddic poetry}

    The by far most important manuscript is GKS 2365 4to, here \Regius. It dates to the 1270s and has 45 surviving foll., containing TODO poems.  The poems can be split into two groups; the first (on foll. 1–20) dealing mostly with mythology, the second (on foll. 20–45) with heroic legend.  Scribal characteristics show that these two parts have been copied from separate source manuscripts.

    \Regius\ is not a mere anthology of poems, but shows substantial editorial input as well.  Short prose sections tie a group of the mythological poems together into a loose narrative, though it is clear from their style and language that they have originally been separate works.  When it comes to the heroic poems long prose segments occur both within and between them, creating a \inx[C]{saw}-like prosimetrical form where the prose sometimes comes to dominate the poetry.  A manuscript closely related to the heroic half of \Regius\ has clearly served as the main source for large swathes of the younger \VolsungaSaga.

    A large gap famously occurs in the heroic half; between foll. 32 and 33 one quire has gone missing.  Its contents are mostly unknown, but it would have included the end of \Sigrdrifumal\ and the beginning of the Fragmentary Lay of Siward (TODO).  Some of the stanzas probably contained in it may be restored from the \VolsungaSaga, and these are edited in \emph{Fragments from the Saw of the Walsings} below.  For further literature on \Regius\ see TODO.


    Second in importance stands is AM 748 I a 4to, here \AM.  It dates to the C14th and is but a fragment, consisting of just 6 foll.  It contains only poems found in the mythological part of \Regius, but in a different order from that ms., nor is there any trace of a frame narrative.  \Regius\ and \AM\ do share a fair bit of prose, a fact which suggests that both stem from a common manuscript archetype, rather than being independent witnesses of oral tradition.

    On the first two foll. are contained the final stanzas of \Harbardsljod\ (1r–v), the complete \Baldrsdraumar\ (1v–2r), and the first stanzas of \Skirnismal\ (2r–v).  After this there is a gap; the next four foll. contain the second half of \Vafthrudnismal\ (3r–v), the complete \Grimnismal\ (3v–5v) and \Hymiskvida\ (5v–6v), and the beginning of the prose introduction to \Volundarkvida\ (6v).  \AM\ is the only medieval attestation of \Baldrsdraumar, and the poems shared with \Regius\ are clearly not directly copied thence.  This makes it very valuable for textual criticism.  For further literature on \AM\ see TODO.


    We find quotations from several Eddic poems in \Gylfaginning\ and \Skaldskaparmal, the first two sections of Snorre’s Edda.  Snorre reproduces stanzas from (TODO) \Voluspa, \Vafthrudnismal, and \Grimnismal\ in \Gylfaginning; \Grottasongr\ is attested in full in \Skaldskaparmal.  Apart from these, Snorre also reprodues a few otherwise unknown stanzas in Eddic meters, which are edited below under \emph{Eddic fragments from Snorre’s Edda}.  The four main mss. for the Prose Edda are:%TODO: use table like in the Heliand introduction

    \begin{enumerate}
      \item Codex Regius of the Prose Edda \RegiusProse\ (GKS 2367 4to; 1300-1350)
      \item Codex Trajectinus \Trajectinus\ (Traj 1374; a c. 1595 paper copy of a ms. closely related to \RegiusProse.)
      \item Codex Wormianus \Wormianus\ (AM 242 fol.; 1340–70)
      \item Codex Upsaliensis \Upsaliensis\ (DG 11; 1300–25)
    \end{enumerate}

    When all four mss. agree on a reading the abbreviation \GylfMS\ is used synonymously with \RegiusProse\Trajectinus\Wormianus\Upsaliensis.  For discussion on their internal stemmatics and origins I refer to \textcite{Haukur2017}.


    A few other Eddic-style poems from various sources are also included in the present edition.  The fragmentary \Rigsthula\ is found at the end of \Wormianus.  TODO (Svipdagsmál and \Grougaldr) are found only in post-reformation Icelandic paper mss., namely TODO.  While I have not consulted such paper mss. for poems attested in medieval mss., I have had to rely on them for these poems.  About these poems it must be said that their late \emph{attestation} does not necessarily prove them to be late \emph{compositions}.  A good proof of this is \Baldrsdraumar, which is first attested in the fragmentary \AM, and then (with some interpolated stanzas) in much later paper mss.  We cannot exclude that some of these poems would have existed in other lost medieval mss., perhaps even on the now-lost pages of \Regius\ or \AM.

    \subsubsection{Old English poetry}

    The edited Old English poetry primarily derives from a few manuscripts.  Particularly important are the Exeter Book and \Lacnunga.

    \subsubsection{Old Saxon and High German poetry}

    There are no collections of alliterative poetry in these languages; instead the manuscript situation will be disussed in the Introduction to each individual text.


\section{The present edition}

  The present edition is divided into two equally large parts, presented side by side.  Each stanza or group of verse lines is presented first in the original Old Germanic language, and then in English translation.

  \subsection{The Old Germanic text}

    In the present edition are found texts in four Old Germanic languages: Old Norse, Old English, Old Saxon, and Old High German.  All texts have been normalized according to my own standardised orthography for the respective languages. The orthographies are all designed to follow three core principles:

    \begin{enumerate}
      \item A faithfulness to the spoken language at the time when the texts were written, and the distinctions demonstrably found therein.
      \item A respect for the etymological origin of words, and their distinctions.
      \item A striving for a uniform orthography across the various languages, so that the same etymological sound should be written with the same character.
    \end{enumerate}

    These choices often stand in conflict with the orthography of the original manuscripts and with most earlier philological tradition, whence there is some reason to justify them.  My goal is to render the texts themselves in a manner that gives as much philological information to the reader as possible—not to present a facsimile edition for students of paleography.  This follows the philological methods used for printing e.g. the \Rigveda, which is generally printed in an entirely scholarly latinized orthography, not the original \emph{devanagari}.  Regardless, such important traits of the original manuscript tradition as the long \emph{ſ}, arbitrary punctuation, arbitrary spelling, and lack of line breaks, are seldom reproduced in modern editions of Old Germanic poetry.

    \subsubsection{General orthographic conventions}

    The following orthographic conventions are followed for all Old Germanic languages:

    \begin{enumerate}
    \item The voiceless dental fricative is always written with the letter \emph{þ}, never \emph{th}.
    \item Long vowels are marked with the acute accent, never the macron or circumflex, excepting
    \item those which have their origin in earlier dipththongs, which are written with the circumflex.
    \item In compounds where the first element has primary stress the elements are separated with a dash,
    \item but where the first element is a preposition they are separated with an interpunct.
    \end{enumerate}

    Below follow specifications for each specific language.

    \subsubsection{Normalization of Old Norse}

    My Old Norse orthography is inspired by \textcite{FinnurEdda} in that it strives for a more archaic form than that of the surviving mss.; a form that instead represents the poetry as it may (in many cases, must) originally have looked. For this reason, it often has more in common with the proposed orthography of the First Grammatical Treatise than with the standard Old Icelandic orthography seen in most editions. The following list describes the differences from the standard Old Icelandic orthography:

    \begin{enumerate}
    \item I distinguish short \emph{e} (from etymological short \emph{e}) and short \emph{ę} (from etymological short \emph{a} + \emph{i}-umlaut).
    \item I distinguish long \emph{á} and \emph{ǫ́}, as done by the First Grammatical Treatise.
    \item I use \emph{ǿ} and \emph{ę́} rather than the traditional \emph{œ} and \emph{æ}, to represent the vowels descended from Proto-Norse \emph{ō} and \emph{ā} after \emph{i}-umlaut (cf. the short \emph{ø, ę} < \emph{o, a} + \emph{i}-umlaut).
    \item I distinguish long nasal vowels \emph{ȧ, ė, ï, ȯ, u̇} from long oral \emph{á, é, í, ó, ú}, as done in the First Grammatical Treatise.
    \item I restore the old \emph{s}—which in modern Scandinavian and even in most Old Norse manuscripts has become \emph{r}, but which is found consistently in old manuscripts such as AM 237 a fol (c. 1150), and fossilized in forms like \emph{þaz} (i.e. \emph{þat’s}) in \Regius—in the words \emph{es} ‘which, that, where, when’, and in inflections of \emph{vesa} (later \emph{vera}) such as \emph{es} ‘is’ (3rd sg. pres. ind.) and \emph{vas} (3rd sg. pret. ind.). The following forms retain the \emph{r}, as it is there the result of Verner’s law, and not of this (much younger) sound change: the pl. pres. ind. (\emph{erum} etc.), the pl. pret. ind. (\emph{vǫ́rum} etc.), and the pl. pret. subj. (\emph{vę́rim} etc.)
    \item When metrically benefactory, I contract \emph{ek} ‘I’, \emph{eru} ‘are’, and \emph{es} ‘which; is’ to \emph{’k}, \emph{’ru} and \emph{’s}, respectively.
    \item I use \textcite{FinnurEdda}’s way of distinguishing between the relative particle \emph{es} and the verb \emph{es}: the first is appended to the previous word with only an apostrophe (e.g. \emph{hann’s} ‘he who’), while the second is separated by a space (e.g. \emph{hann ’s} ‘he is’).
    \end{enumerate}

    \subsubsection{Normalization of Old Swedish and Danish}
    I employ the same conventions as those described for Old Norse above, including the marking of \emph{u}-mutated \emph{a} > \emph{ǫ} (that this was indeed found in the Eastern Nordic dialects is most clearly seen by the third-person personal pronoun, which shows \emph{u}-mutation in such forms as Swedish \emph{honom} ‘him’ < \emph{hǫ́num}, \emph{hon} ‘she’ < \emph{hǫ́n}).

    According to rule 3 in the general orthographic conventions above, I distinguish between \emph{ǿ} (< \emph{ǿ}) and \emph{ø̂} (< \emph{au, ęy}); \emph{é} (< \emph{é}) and \emph{ê} (< \emph{ęi}).

    Where unstressed vowels have been reduced into an schwa-like sound spelled \emph{e}, this is written with \emph{ę}.

    \subsubsection{Normalization of Old English}
    I spell fronted or brightened etymological \emph{a} and \emph{á} with \emph{æ} and \emph{ǽ}, for instance in \emph{dæg} ‘day’ (< \emph{*dagaʀ}) and \emph{rǽd} ‘advice, counsel’ (< \emph{rádaʀ}).  These are contrasted with \emph{ę} and \emph{ę́}, which represent \emph{i}-mutated \emph{a} and \emph{á}, for intance in \emph{ęllen} ‘zeal, courage’ (< \emph{*aljaną}).

    An assimilated \emph{n} is marked with an overpoint, like in rule 3 of Old Norse above.

    \subsubsection{Normalization of Old Saxon}

    \subsubsection{Normalization of Old High German}

  \subsection{The English translation}

    There is now a very large number of translations of the most popular alliterative poetic texts, namely \Beowulf\ and the \emph{Poetic Edda}.  These generally fall into two camps:
    \begin{enumerate}
      \item \emph{poetic} translations, which distort the precise meaning of the text for the sake of meter, often quite radically; and
      \item \emph{prose} translations, which nowise preserve the style or feeling of the original.
    \end{enumerate}

    Almost all translations, of both types, also tend toward the following inadequacies: obscuring or glossing over difficult technical and cultural terminology; rendering identically repeated phrases and words (formulae) differently at various places; and simplifying or rewriting kennings and other poetic expressions.  Even worse this is often done with little in the way of notes or commentary, to a point where the reader is sometimes left entirely oblivious to the sense of the original text.

    What sets my translation apart from previous English translations is that it aims to follow the style and register of the original text, without sacrificing the literal sense of the words.  This unfortunately means that literality and consistency at times must sometimes come at the cost of fluid idiomatic English, but it has the advantage of giving the reader an image of not just \emph{what} the original text actually says, but \emph{how} it says it.  The reader should keep in mind that he is in a very foreign land, that he is reading words ancient and long forgotten—not the \emph{New York Times}.

    Maybe this is a pointless effort? One could argue that a translation always is a betrayal, and that those truly interested in the exact meaning of every word in the original text should study just the original (in the original language).  While I do agree that the sufficiently interested reader should study the original texts in the languages in which they were written (something made much easier by the present edition with its notes and parallel edition), it is still a “hard ask” for those readers who are not philologically inclined, but instead students and scholars of history, comparative mythology and religion, anthropology, or literature; those who, for whatever reason, are interested in exploring the oldest poetic heritage of the Germanic peoples of northern Europe.

    \subsection{Anglish proper nouns}
      Perhaps the single most idiosyncratic part of the present translation will be its handling of proper nouns. I have opted to render all cultural and religious terms, names of places, heroes, gods, and other entities by their English cognates (thus \emph{Thunder} for Old Norse \emph{Þórr}) and where such do not exist, their philologically expected English (\emph{Anglish}) forms (e.g. \emph{wallow} for Old Norse \emph{vǫlva}).

      There are two reasons for this.  The first is ideological.  I believe that the Old Germanic myths and poems, their gods and heroes, are a shared heritage of Northern Europe.  When you translate texts from across Germany, England and Scandinavia you quickly come to notice how similar the diction is, how many names reappear. The Scandinavian \emph{Vǫlundr} is the same character as the English \emph{Wélund}; likewise Norse \emph{Óðinn} is the same as English \emph{Wóden}.  These are ultimately mere distinctions in pronunciation.

      The second is aesthetic.  Commonly accepted forms like \emph{Odin} and \emph{Thor} are debased.  They do not even represent the Old Norse pronunciation as accurately as possible within the constraints of English ortography (for instance, \emph{Odin} would be better anglicized as \emph{Othin}).  Many are also difficult for English speakers to pronounce, or lead to absurd confusions.  I shudder at hearing the word \emph{ę́sir} pronounced /aɪˈsɪ:ɹ/; even worse is when \emph{Ǫ́s-garðr} becomes “ass-guard”.

  \printbibliography% Does it work?
