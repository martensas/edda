\bookStart{Spae of the Wallow}[Vǫluspǫ́]

\begin{flushright}%
\textbf{Dating} \parencite{Sapp2022}: C10th (0.865)–early C11th (0.121)

\textbf{Meter:} \Fornyrdislag%
\end{flushright}

\section{Introduction}

The \textbf{Spae of the Wallow} (\Voluspa) is the most comprehensive mythological text surviving from Heathen times.  The poem is a \inx[C]{spae} (\emph{spǫ́} ‘prophecy’) in the form of a monologue spoken by a \inx[C]{wallow} (\emph{vǫlva} ‘seeress, sibyl, prophetess’) summoned by the god Weden in order to relate mythological knowledge.  Weden’s frequent journeys to question various beings about mythological lore should be seen in the light of his incessant lust for knowledge and wisdom.  The most similar instance is \Baldrsdraumar, wherein Weden summons another wallow out of her grave in \inx[L]{Hell} in order to find out why the god \inx[P]{Balder} is having ominous nightmares.  There is also \Vafthrudnismal, wherein Weden challenges the wise ettin \inx[P]{Webthrithner} to a wisdom contest and defeats him.  These journeys are further alluded to in \Harbardsljod\ TODO.

In its being a mythic catalogue \Voluspa\ also resembles (parts of) poems like \Havamal, \Grimnismal, \Sigrdrifumal, and \Allvismal, but it differs from them all in a key way: instead of being a motley collection of scattered mythological lore, \Voluspa\ offers a chronological overview of the whole Norse mythic timeline, from the creation of the world to its demise and rebirth.  That is not to say that the events in it clearly described; they are related in a highly allusive fashion that presupposes that the audience is already familiar with them.  There may also be some later omissions and inserts that make the poem more difficult to read.

\Voluspa\ is attested in full in two independent recensions.  The first and most important is \Regius, where it is the first poem and found on foll. 1r–3r; the other is \Hauksbok, where it is found in the middle of a large collection of saws and Catholics works at 20r–21r.

Many stanzas from the poem are also cited or paraphrased in \Gylfaginning, for which \Voluspa\ was clearly one of the main sources.  These paraphrases are still of critical value, e.g. in st. 19, where \emph{sal} ‘hall’ in the paraphrase agrees with \Hauksbok\ against \Regius\ \emph{sę́} ‘lake’.  For the four mss. of \Gylfaginning—\RegiusProse, \Trajectinus, \Wormianus, and \Upsaliensis—see the General Introduction.

For the differences between the mss. the reader may consult the following table prepared by the editor.  The several stanzas in \Gylfaginning, which are quoted independently and with little relation to the order of the original poem, are marked with plus signs.  The sequences containg uninterrupted quotations of several stanzas are marked with an incrementing alphabetic symbol, so that \emph{B1} is the first stanza in the second sequence, and so on.  When a stanza found in a ms. is strongly divergent (e.g. st. 10, where \Gylfaginning\ omits the first two half-lines), its number is followed by a star.  The stanzas beginning with \emph{Þȧ gingu ręgin ǫll} ‘Then went the Reins all’ are represented by the half-line immediately following.

\begin{longtabu} to \textwidth {|c c c c c c|}
	\hline
	\multicolumn{2}{|c}{\emph{pres. ed.}} & \Regius & \Hauksbok & \RegiusProse\Trajectinus\Wormianus & \Upsaliensis \\ [0.5ex]
	\hline\hline\endhead
	\hline\endfoot
	1 & Hljóðs bið’k allar & 1 & 1 & − & − \\
	2 & Ek man jǫtna & 2 & 2 & − & − \\
	3 & Ár vas alda & 3 & 3 & + & + \\
	4 & áðr Burs synir & 4 & 4 & − & − \\
	5 & Sól varp sunnan & 5 & 5 & +* & +* \\
	6 & \dots\ nǫ́tt ok niðjum & 6 & 6 & − & − \\
	7 & Hittusk ę̇sir & 7 & 7 & − & − \\
	8 & Tęflðu ï tu̇ni & 8 & 8 & − & − \\
	9 & \dots\ hvęrr skyldi dverga & 9 & 9 & B1 & B1 \\
	10 & Þar vas Móðsognir & 10 & 10 & B2* & B2* \\
	11–15 & \emph{Dwarf-tallies} & 11–15 & 11–16 & + & + \\
	16 & Unds þrír kvǫ̇mu & 16 & 17 & − & − \\
	17 & Ǫnd þau né ǫ́ttu & 17 & 18 & − & − \\
	18 & Ask vęit’k standa & 18 & 19 & + & + \\
	19 & Þaðan koma męyjar & 19–20 & 20–21 & − & − \\
	20 & Þat man hǫ̇n folk-víg & 21–22 & 27 & − & − \\
	21 & Hęiði hétu & 23 & 28 & − & − \\
	22 & \dots\ hvárt skyldu ę̇sir & 24 & 29 & − & − \\
	23 & Flęygði Óðinn & 25 & 30 & − & − \\
	24 & \dots\ hvęrr hęfði lopt alt & 26 & 22 & C1 & C1 \\
	25 & Þȯrr ęinn þar vá & 27 & 23 & C2* & C2* \\
	26 & Vęit hǫ̇n Hęimdallar & 28 & 24 & − & − \\
	27 & Ęin sat hǫ̇n úti & 29 & − & − & − \\
	28 & Alt vęit’k, Óðinn & 29 & − & + & + \\
	29 & Valði hęnni Hęr-fǫðr & 30 & − & − & − \\
	30 & Sá hǫ̇n val-kyrjur & 31 & − & − & − \\
	31 & Ek sá Baldri & 32 & − & − & − \\
	32 & Varð af męiði & 33 & − & − & − \\
	33 & Þó hann ę́va hęndr & 34 & − & − & − \\
	H1 & Þȧ kná Váli & − & 31 & − & − \\
	34a & Hapt sá hǫ̇n liggja & 35a & − & − & − \\
	34b & þar sitr Sigyn & 35b & 32 & − & − \\
	35 & Ǫ́ fęllr austan & 36 & − & − & − \\
	36 & Stóð fyr norðan & 36 & − & − & − \\
	37 & Sal sá hǫ̇n standa & 37 & 36 & E1 & E1 \\
	38 & Sér hǫ̇n þar vaða & 38 & 37 & E2* & E2* \\
	39 & Austr býr hin aldna & 39 & 25 & A1 & A1 \\
	40 & Fyllisk fjǫrvi & 40 & 26 & A2 & A2 \\
	41 & Sat þar ȧ haugi & 41 & 34 & − & − \\
	42 & Gól of ǫ̇sum & 42 & 35 & − & − \\
	43, 48, 56 & Gęyr (nú) Garmr mjǫk & 43, 46, 55 & 33, 38, 43, 48, 51 & − & − \\
	44 & Brǿðr munu bęrjask & 44 & 39 & − & − \\
	45 & Lęika Mïms synir & 45 & 40 & D1* & D1* \\
	46 & Skęlfr Ygg-drasils & 45* & 41 & D1* & D1* \\
	47 & Hvat ’s með ǫ̇sum? & 49 & 42 & D2 & D2* \\
	49 & Hrymr ękr austan & 47 & 44 & D3 & − \\
	50 & Kjóll fęrr austan & 48 & 45 & D4 & − \\
	51 & Surtr fęrr sunnan & 50 & 46 & +, D5 & + \\
	52 & Þȧ kømr Hlïnar & 51 & 47 & D6 & − \\
	53 & Þȧ kømr hinn mikli & 52 & − & D7 & − \\
	H2 & Gïnn lopt yfir & − & 48 & — & − \\
	54 & Þȧ kømr hinn mę́ri & 53* & 49* & C8 & − \\
	55 & Sól tér sortna & 54 & 50 & C9 & − \\
	57 & Sér hǫ̇n upp koma & 56 & 52 & − & − \\
	58 & Finnask ę̇sir & 57* & 53 & − & − \\
	59 & Þar munu ęptir & 58 & 54 & − & − \\
	60 & Munu ȯ·sánir & 59 & 55 & − & − \\
	61 & Þȧ kná Hø̇nir & 60 & 56 & − & − \\
	62 & Sal sér hǫ̇n standa & 61 & 57 & + & + \\
	H3 & Þȧ kømr hinn ríki & − & 58 & − & − \\
	63 & Þar kømr hinn dimmi & 62 & 59 & − & − \\ [1ex]
	\hline
\end{longtabu}


\sectionline

The poem begins with a bid for silence (1), and the wallow recalling her earliest memories (2). She then recounts the ordering of the world by the gods (3–6) and the golden age of peace and plenty (7–8), which is, however, interrupted by the intrusion of three unidentified ettin-maidens (8, and see note there). After this follow two verses about the shaping of the dwarfs (9–10), and then several originally separate \emph{dwarf-tallies} (11–15), which are without doubt later inserts. Returning to the main narrative thread is described the creation and endowment of the first man and woman (16–17), Ugdrassle’s Ash (18), and the three \inx[G]{norns} living under it (19).

This is where the two full recensions of the poem diverge. Because of its older age and larger count of verses I have here followed the order of \Regius: the wallow recalls how a woman named Goldwey was sacrificed and reborn three times (20), and how she, under the name Heath, practiced sorcery and witchcraft (21). She then recalls the first war in the world, between the Eese and Wanes (22–23), and alludes to the slaying of the smith, who according to \Gylfaginning\ 42 was promised \inx[P]{Frow} and the sun and moon in exchange for building the wall of Osyard (24-25). This is followed by a cryptic verse describing Homedal’s hidden silence or hearing (26).

In \Hauksbok\ the structure is quite different. After the description of the norns (19), the Eese go to decide what action to take regarding the promising of Frow to the ettin (my 24-25), and Homedal’s hearing is described (26). Then follows the two verses about the old hag in Ironwood who raises the wolves that will swallow the sun and moon (40-41). After this come verses 20–23 in the same order as \Regius\ (see above).

\sectionline

\section{The Spae of the Wallow}

\bvg\bva\mssnote{\Regius~1r/2, \Hauksbok~20r/1}%
„\alst{H}ljóðs bið’k allar \hld\ \edtext{\alst{h}ęlgar}{\lemma{hęlgar}\Afootnote{om. \Regius}} kindir, &
\edtrans{\alst{m}ęiri ok \alst{m}inni}{greater and lesser}{\Bfootnote{The noun being modified is ambiguous.  It may either be (a) ‘greater and lesser holy kindreds’, in which case it may be equivalent to the phrase \inx[F]{Eese and Elves} (both earthly and heavenly supernatural beings; see Encyclopedia for occurrences) or (b) ‘greater and lesser lads of Homedal \ken{men}’.  (b) is probably to be preferred for reasons of syntax, but should not most likely be seen as referring to varying social classes; it seems unlikely that there would be slaves present in the audience of a poem like this.
In any case, the wallow seems to be asking all intelligent beings present for silence, with the expression being a merism of the type ‘gods and men’; see \textcite{West2007}[99-100].}} \hld\ \edtrans{\alst{m}ǫgu Hęim-dalar;}{lads of Homedal \ken{men}}{\Bfootnote{Cf. \Rigsthula, wherein Righ, identified by the prose as Homedal, sires the ancestors of the three castes of men.}} &
\alst{v}ilt at, \alst{V}al-fǫðr, \hld\ \alst{v}ęl fram tęlja’k &
\alst{f}orn spjǫll \alst{f}ira, \hld\ þau’s \alst{f}ręmst of man?\eva

\bvb “For hearing I ask all holy races, \\
the greater and lesser lads of Homedal \ken{men}! \\
Wilt thou, Walfather \name{= Weden}, that I well tell forth \\
the ancient sayings of men which I foremost recall?\footnoteB{Cf. \Vafthrudnismal\ 34–35 with similar phrasing. The whole introductory formula is positively Indo-European, see \textcite{West2007}[63,92-93,312].}\evb\evg


\bvg\bva\mssnote{\Regius~1r/4, \Hauksbok~20r/2}%
\alst{E}k man \alst{jǫ}tna \hld\ \alst{á}r of borna, &
þȧ’s \alst{f}orðum mik \hld\ \alst{f}ǿdda hǫfðu; &
\alst{n}íu man’k hęima, \hld\ \alst{n}íu \edtext{ïviðjur}{\Afootnote{so \Regius\Hauksbok.  \Regius\ has previously been as read \emph{iviði}, but this was disproven by an x-ray scan undertaken by \textcite{StefanKarlsson1979}.}}, &
\edtrans{\alst{m}jǫt-við \alst{m}ę́ran \hld\ fyr \alst{m}old neðan.}{the renowned Metwood beneath the soil.}{\Bfootnote{Probably \inx[L]{Ugdrassle’s Ash}, being still a seed.}}\eva

\bvb I recall \inx[G]{Ettins} born of yore, \\
they who formerly had nourished me. \\
Nine \inx[C]{Home}[Homes] I recall, nine \inx[G]{Inwithies}; \\
the renowned \inx[P]{Metwood} beneath the soil.\evb\evg


\bvg\bva\mssnote{\Regius~1r/6, \Hauksbok~20r/4, \GylfMS}%
\alst{Á}r vas \alst{a}lda \hld\ \edtrans{þar’s \alst{Y}mir byggði}{where Yimer dwelled}{\Afootnote{\emph{þat’s ękki vas} ‘that when nothing was’ \GylfMS}}, &
vas-a \alst{s}andr né \alst{s}ę́r, \hld\ né \alst{s}valar unnir; &
\edtext{\alst{jǫ}rð fannsk \alst{ę́}va \hld\ né \alst{u}pp-himinn}{\lemma{jǫrð \dots\ né upp-himinn ‘Earth \dots\ nor Up-heaven’}\Bfootnote{A well-attested formulaic cosmological word-pair found in all four Old Germanic languages with poetic traditions (ON, OE, OS, OHG), especially in concern the creation and destruction of the world. See \inx[F]{Earth and Upheaven}.}}; &
\edtrans{\alst{g}ap vas \alst{g}innunga}{there was the Gap of Ginnings}{\Bfootnote{See Index for suggested etymology.}}, \hld\ en \alst{g}ras \edtrans{hvęrgi}{nowhere}{\Afootnote{\emph{ękki} ‘not’ \Hauksbok}};\eva

\bvb It was early of ages where \inx[P]{Yimer} dwelled; \\
there was not sand nor sea nor cool waves. \\
\inx[L]{Earth} was never found, nor \inx[L]{Up-heaven}; \\
there was the \inx[L]{Gap of Ginnings}, but grass nowhere,\footnoteB{A more extensive creation narrative is found in \Gylfaginning\ 4–5, according to which the world first consisted of two extremities: the frozen Nivelham in the north and scorching Muspellsham in the south. From Nivelham the freezing venom-rivers called the \inx[L]{Ilewaves} ran until they froze to ice, while burning lava flowed from Muspellsham. The ice and lava met in the Gap of Ginnings, “which was as calm as windless air”, and there combined to form the first being, \inx[P]{Yimer}, who was the ancestor of the ettins.}\evb\evg


\bvg\bva\mssnote{\Regius~1r/8, \Hauksbok~20r/5}%
áðr \edtrans{\alst{B}urs synir}{the Sons of Byre}{\Bfootnote{In \Gylfaginning\ 6 identified as Weden, Will and Wigh, who sacrificed Yimer and shaped the cosmos out of his body. For this see also \Vafthrudnismal\ 20–21 and \Grimnismal\ 41–42.}} \hld\ \alst{b}jǫðum of ypðu, &
þęir es \alst{M}ið-garð \hld\ \alst{m}ę́ran skópu; &
\alst{s}ól skęin \alst{s}unnan \hld\ ȧ \alst{s}alar stęina; &
þȧ vas \alst{g}rund \alst{g}róin \hld\ \edtrans{\alst{g}rø̇num lauki}{green leek}{\Bfootnote{A sign of the golden age, since the leek was believed to be the noblest plant and had important cultural significance.  This is seen from \GudrunTwo\ 2, where \inx[P]{Siward}’s superiority to the \inx[P]{Yivickings} is compared to a stag among wild beasts, gold among silver, and a green leek in grass. The leek was valued in folk magic, as seen already on gold bracteates from the C5th and C6th, where it appears as a charm word in the form {ᛚᚨᚢᚲᚨᛉ} \emph{laukaʀ}, in one inscription paired with {ᛚᛁᚾᚨ} \emph{lína} ‘linen’.  Classical Norse attestations of magic use include \Sigrdrifumal\ 8, where the leek is thrown into mead against poison; and the \Volsathattr, where a horse penis is said to be \emph{\alst{l}íni gǿddr \hld\ en \alst{l}aukum studdr} ‘endowed with linen and supported by leeks’ in a poetic line.  The leek was particularly associated with women and domestic life, as seen by its pairing with “linen”.  Kennings for women frequently have the leek as a determinant (TODO: Meissner reference?), and Anon \emph{Sveinfl} 1 (\Skp\ I TODO.) sarcastically states that a battle was not \emph{sem manni \hld\ mę́r lauk eða ǫl bę́ri} ‘as if a maiden brought a man leek or ale’.}}.\eva

\bvb before the \inx[P]{Sons of Byre} uplifted the flatlands, \\
they who shaped renowned \inx[L]{Middenyard}. \\
Sun shone from the south on the stones of the hall; \\
then was the ground grown with green leek.\evb\evg


\bvg\bva\mssnote{\Regius~1r/11, \Hauksbok~20r/7, \GylfMS}%
\edtext{\alst{S}ól varp \alst{s}unnan, \hld\ \edtrans{\alst{s}inni Mȧna}{Moon’s companion}{\Bfootnote{At times translated as ‘its moon’. This cannot be correct, as \emph{mȧni} ‘moon’ is masculine, while \emph{sinni}, dat. sg. of \emph{sïnn} ‘its (reflexive)’ is feminine.}}, &
\alst{h}ęndi hinni \alst{h}ǿgri \hld\ of \edtrans{\alst{h}imin-jǫður}{heaven’s rim}{\Afootnote{composite; \emph{himin †iodyr†} \Regius; \emph{ioður} \Hauksbok.}\Bfootnote{Some recent editors have taken it upon themselves to normalize the reading of \Regius\ as \emph{himin-jó-dýr} ‘heaven-horse-beast’, which is not just nonsensical but also unmetrical due the stress pattern.  On the other hand the reading of \Hauksbok, normalized to \emph{jǫður} ‘rim, edge’, is clearly deficient since it lacks the neccessary alliteration on \emph{h}.  If we see \emph{iodyr} \Regius\ as corrupted from \emph{*iodur} we can restore \emph{himin-jǫður}, as done here.}}}{\lemma{Sól \dots\ himin-jǫður ‘Sun \dots\ heaven’s rim’}\Afootnote{om. \GylfMS.}\Bfootnote{Probably a poetic description of the dawn; the Sun lifted herself up over the horizon and rose for the first time.}}; &
\alst{S}ól þat né vissi, \hld\ hvar hǫ̇n \alst{s}ali átti; &
\edtext{\alst{st}jǫrnur þat né vissu, \hld\ hvar þę́r \alst{st}aði ǫ́ttu}{\lemma{stjǫrnur \dots\ ǫ́ttu}\Afootnote{In \GylfMS\ this line comes last, so that the order is sun, moon, stars.}}; &
\edtext{\alst{M}ȧni þat né vissi, \hld\ hvat hann \alst{m}ęgins átti.}{\lemma{Mȧni \dots\ átti ‘Moon \dots\ had’}\Bfootnote{The moon was believed to have supernatural powers and could be invoked in conflict (cf. \Havamal\ 137/7.)}}
\eva

\bvb The Sun cast from the south—the Moon’s companion— \\
her right hand over heaven’s rim. \\
The Sun knew not where halls she had; \\
the stars knew not where seats they had; \\
the Moon knew not what sort of might he had.\evb\evg


\bvg\bva\mssnote{\Regius~1r/13, \Hauksbok~20r/9}%
\edtext{Þȧ gingu \alst{r}ęgin ǫll \hld\ ȧ \edtrans{\alst{r}ǫk-stóla}{rake-seats}{\Bfootnote{Their seats of judgment at the \inx[L]{Thing of the Gods}.}}, &
\alst{g}inn-hęilǫg \alst{g}oð, \hld\ ok umb þat \alst{g}ę́ttusk}{\lemma{Þȧ \dots\ gę́ttusk ‘Then \dots\ of this.’}\Bfootnote{A formulaic expression for the convening of the \inx[L]{Thing of the Gods}, identically repeated below in sts. 9/1–2, 22/1–2, and 24/1–2.  Cf. also the formula shared between \Baldrsdraumar\ 1/1–3 and \Thrymskvida\ 14/1–3, which follows the structure of the present formula very closely: \emph{Sęnn vǫ́ru ę̇sir \hld\ allir ȧ þingi // ok ǫ̇synjur \hld\ allar ȧ máli, // ok umb þat réðu \hld\ ríkir tívar.} ‘Soon were the \inx[G]{Eese} all at the \inx[C]{Thing}, // and the \inx[G]{Ossens} all at speech, // and of this counseled the mighty \inx[G]{Tews}.’

In the five occurrences of these two formulae outside of the present stanza, the demonstrative pronoun \emph{þat} ‘this’ clearly refers to an immediately following question introduced by a \emph{hv}-word (e.g. \Thrymskvida\ 14/4: \emph{hvé þęir Hlórriða \hld\ hamar of sǿtti?} ‘how they Loride’s \name{= Thunder’s} hammer would find?’)  Following this pattern we would expect to find such a question after \emph{umb þat gę́ttusk} ‘took counsel of this’ in the present stanza, and it seems most likely to presume that they have been lost in transmission.}}. &
\edtext{\alst{N}ǫ́tt ok \alst{n}iðjum \hld\ \alst{n}ǫfn of gǫ́fu, &
\alst{m}orgin hétu \hld\ ok \alst{m}iðjan dag, &
\alst{u}ndurn ok \alst{a}ptan, \hld\ \alst{ǫ́}rum at tęlja.}{\lemma{Nǫ́tt \dots\ tęlja ‘To night \dots\ tally’}\Bfootnote{Cf. \Vafthrudnismal\ 23, where it is said that the sun and moon turn round in heaven \emph{ǫldum at ár-tali} ‘for mankind’s tally of years’, and 25, where it is said that the Reins created the moon-phases for the same purpose.}}\eva

\bvb Then went the Reins all onto the rake-seats: \\
the Yin-holy Gods, and from each other took counsel of this. \\
To night and the moon-phases names they gave; \\
morning they named, and middle day, \\
afternoon and evening, the years for to tally.\evb\evg


\bvg\bva\mssnote{\Regius~1r/16, \Hauksbok~20r/10}%
Hittusk \alst{ę̇}sir \hld\ ȧ \alst{I}ða-vęlli, &
\edtext{þęir’s \alst{h}ǫrg ok \alst{h}of \hld\ \alst{h}ǫ́-timbruðu}{\lemma{þęir’s \dots\ hǫ́-timbruðu ‘they who \dots\ timbered on high’}\Afootnote{\emph{afls kostuðu \hld\ alls freistuðu} ‘[their] strength they tried; everything they tempted’ \Hauksbok}\Bfootnote{Two formulæ. \emph{hǫrgr ok hof} ‘harrow and hove’ is a merism, i.e. ritual structures made of stone and wood; cf. \Vafthrudnismal\ 38 and \HelgakvidaHjorvardssonar\ TODO, as well as the Norwegian Christian laws that impose ‘the burning of hoves and the breaking of harrows’ (\emph{brenna hof ok brjóta hǫrga}).  \emph{hǫ́-timbra} ‘high timber, timber on high’ is a rare compound and only occurs at one other place in the ON corpus, viz. in \Grimnismal\ 16, where it describes a harrow ruled by Nearth.

This line has often been wondered at; why would the Gods themselves make cultic buildings?  Yet they partake in ritual slaughter of beasts, divination, and feasting (e.g. \Voluspa\ 61, \Hymiskvida\ 1, 39, \Lokasenna, \Haustlong\ 2), and their deeds form the precedent for upright human behaviour.}}; &
\alst{a}fla lǫgðu, \hld\ \alst{au}ð smíðuðu, &
\alst{t}angir skópu \hld\ ok \alst{t}ól gęrðu.\eva

\bvb The Eese found each other on the \inx[L]{Idewolds}, \\
they who \inx[C]{harrow} and \inx[C]{hove} timbered on high; \\
hearths they laid, wealth they smithed, \\
tongs they shaped and tools they made.\evb\evg


\bvg\bva\mssnote{\Regius~1r/18, \Hauksbok~20r/12}%
\edtext{\edtrans{\alst{T}ęflðu}{played Tables}{\Bfootnote{A verb derived from \emph{tafl} ‘board game’, an old borrowing from Latin \emph{tabula}.  “Tables” is used as a cognate translation; the exact type of board game referred to is unimportant.}} ï \alst{t}u̇ni, \hld\ \alst{t}ęitir vǫ́ru, &
\edtrans{\alst{v}as þęim \alst{v}étter-gis \hld\ \alst{v}ant ór gulli}{for them was nothing golden wanting}{\Bfootnote{Indeed even the bricks they played with were of gold. See st. 59.}}, &
unds \edtext{\alst{þ}ríar kvǫ̇mu \hld\ \alst{þ}ursa męyjar}{\lemma{þríar \dots\ þursa męyjar ‘three maidens of Thurses’}\Bfootnote{These three maidens are never mentioned again (unless they are taken to be the three norns in st. 19, but they would then be introduced twice). It is possible that an additional stanza giving further information about them has been lost. If it originally existed, it was already absent from the version employed by the author of \Gylfaginning, who gives no new information.}}, &
\edtrans{\alst{ȧ}m-átkar}{uncanny}{\Bfootnote{The word \emph{ám-áttigr} has a clear association with supernatural beings; trolls and ettins. It occurs in four other places in \Regius. In \Grimnismal\ 11, \Skirnismal\ 10 and \HelgakvidaHjorvardssonar\ 17 it modifies \emph{jǫtunn} ‘ettin’ in a \Ljodahattr\ c-line. In \HelgakvidaHjorvardssonar\ 14 it is used by the daughter of an ettin to refer to a human hero.}} mjǫk, \hld\ ór \alst{Jǫ}tun-hęimum.}{\lemma{ALL}\Bfootnote{The whole stanza is paraphrased in \Gylfaginning\ ch. 14:
\begin{quote}\emph{Ok því nę́st smíðuðu þeir málm ok stein ok tré ok svá gnóg-liga þann málm, er gull heitir, at ǫll bús-gǫgn ok ǫll reiði-gǫgn hǫfðu þeir af gulli, ok er sú ǫld kǫlluð gull-aldr, áðr en spilltist af til-kvámu kvinnanna; þę́r kómu ór Jǫtun-heimum.}

‘And after this they smithed ore and stone and wood, and so abundantly [did they smith] that ore which is called gold, that all their house tools and riding tools were golden. And that age is called the golden age, before it was spoiled by the arrival of the women; they came from Ettinham.’\end{quote}
after which he describes the creation of the dwarfs (see next stanza)}}\eva

\bvb They played \inx[C]{Tables} in the yard; merry were they; \\
for them was nothing golden wanting— \\
until three maidens of \inx[G]{Thurses} came, \\
most uncanny, from \inx[L]{Ettinham}.\evb\evg

\sectionline

\bvg\bva\mssnote{\Regius~1r/20, \Hauksbok~20r/14, \GylfMS}%
Þȧ gingu \alst{r}ęgin ǫll \hld\ ȧ \alst{r}ǫk-stóla, &
\alst{g}inn-hęilǫg \alst{g}oð, \hld\ ok umb þat \alst{g}ę́ttusk: &
\edtrans{Hvęrr skyldi \alst{d}verga}{Who would \dots\ of dwarfs’}{\Afootnote{so \Regius\Wormianus\Upsaliensis; \emph{at skyldi dverga} ‘That they would \dots\ of dwarfs’ \RegiusProse\Trajectinus; \emph{hverir skyldu dvergar} ‘Which dwarfs would [shape the retinues]’ \Hauksbok}} \hld\ \edtrans{\alst{d}rótt}{the retinue}{\Afootnote{so \GylfMS; \emph{drotin} ‘the lord’ \Regius; \emph{dróttir} ‘the retinues’ \Hauksbok}} \edtrans{of skępja}{shape}{\Afootnote{\emph{spekia} ‘soothe’ \Upsaliensis}} &
\edtext{ór \edtrans{\alst{b}rimi \alst{b}lóðgu}{bloody surf}{\Afootnote{so \Hauksbok\RegiusProse\Wormianus\Upsaliensis; \emph{Brimis blóði} ‘the blood of Brimmer’ \Regius\Trajectinus}} \hld\ ok ór \edtrans{\alst{b}lǫ́um}{blue-black}{\Afootnote{metr. emend. from \emph{blám} \Regius; \emph{Bláins} ‘Blown’s’ \Hauksbok\Wormianus; \emph{Bláms} \RegiusProse\Trajectinus\Upsaliensis\ is prob. a corrupt form of \emph{Bláins}}} lęggjum?}{\lemma{ór brimi \dots\ lęggjum ‘out of the bloody \dots\ legs’}\Bfootnote{I think that the poem simply telling of “the bloody surf” and “the blue-black legs” fits better with its general allusive style, but the resulting composite reading may be somewhat controversial.

According to \Gylfaginning\ 14 the dwarfs first originated as maggots in the corpse of Yimer, out of whose bones the rocks were made (\Grimnismal\ 41, \Vafthrudnismal\ 21).  Dwarfs dwell in the rocks and earth; cf. for instance \Ynglingatal\ 2, where the Swedish king Swayther (\emph{Svęigðir} disappears into a rock in pursuit of a dwarf.  More difficult to explain is the creation of dwarfs out of Yimer’s blood (from which was made the sea, \Grimnismal\ 41, \Vafthrudnismal\ 21), since dwarfs are never said to dwell in water. — If one chooses the reading \emph{Bláinn} ‘Blown’ (named in the \inx[C]{thule}[thules] as a dwarf) instead of \emph{blǫ́um} ‘blue-black’, then following Gurevich (\emph{Skp} 2017, p. 693) one may see a kenning “the legs of Blown \name{dwarf} \ken{stone}”. Blown has otherwise been read as a poetic name for Yimer, but that is never attested elsewhere.}}\eva

\bvb Then went the Reins all onto the rake-seats: \\
the Yin-holy Gods, and from each other took counsel of this: \\
Who would shape the retinue of \inx[G]{Dwarfs}, \\
from the bloody surf and from the blue-black legs?\evb\evg


\bvg\bva\mssnote{\Regius~1r/21, \Hauksbok~20r/15, \GylfMS}%
\edtext{\edtext{Þar vas \alst{M}óðsognir}{\Afootnote{so \Hauksbok; \emph{Þar †mótſognir vitnir†} ‘there Mootsowner wolf(?)’ \Regius. The prose of \Gylfaginning\ 14 agrees with \Hauksbok\ that the correct form of the name is \emph{Móðsognir}, not \emph{Mótsognir}.}} \hld\ \alst{m}ę́tstr of orðinn &
\alst{d}verga allra, \hld\ en \alst{D}urinn annarr;}{\lemma{Þar \dots\ annarr ‘There \dots\ second’}\Bfootnote{om. \GylfMS, but the author must have had the full verse, since he paraphrases these lines in the following way: \emph{Móðsognir var ę́ðstr ok annarr Durinn.} ‘Moodsowner was the highest in rank, and Dorn the second.’ before citing}} &
\edtext{\edtext{þęir \alst{m}an-líkun \hld\ \alst{m}ǫrg of gęrðu,}{\lemma{þęir \dots\ gęrðu ‘They \dots\ did make’}\Afootnote{so \Regius\Hauksbok\Upsaliensis; \emph{þar man-líkun \hld\ mǫrg of gęrðusk} ‘There man-likenesses many were made’ \RegiusProse\Trajectinus\Wormianus}} &
\alst{d}vergar \edtrans{ï}{in}{\Afootnote{so \GylfMS\Hauksbok; \emph{ór} ‘out of’ \Regius}} jǫrðu, \hld\ \edtrans{sęm \alst{D}urinn sagði}{as Dorn said}{\Afootnote{so \Regius\Hauksbok\RegiusProse\Wormianus; \emph{sem †dur menn† sagði} ‘as door-men(?) said’ \Trajectinus; \emph{sem †þeim dyrinn kendi†} ‘as the beasts(?) taught them’ \Upsaliensis}}.}{\lemma{þęir \dots\ sagði ‘They \dots\ said.’}\Bfootnote{There are two conflicting interpretations of the creation of the dwarfs. Either they arose on their own; this is supported by the prose of \Gylfaginning\ (see note to previous st.) and by the form of the stanza quoted there (but it may have been changed to correspond to the author’s vision). On the other hand, both \Regius\ and \Hauksbok\ have the dwarfs Moodsowner and Dorn shaping “man-likenesses” out of soil. The present edition follows the second version.}}\eva

\bvb There was Moodsowner made the worthiest \\
of all dwarfs, but Dorn [was] second. \\
They man-likenesses many did make: \\
dwarfs in the earth, as Dorn said.\evb\evg

\sectionline

{\small Sts. 11–15 contain two originally distinct lists of dwarf-names; part of them are almost certainly later inserts.  There is a repetition of names (Oakenshield, Great-grandfather), and more than one formulaic conclusion.

Sts. 11–13, having no repeated names, seem to belong together. If they do, st. 12, which contains the formulaic conclusion to the list, should probably switch places with 13.

Sts. 14–15 form the second group, having an introduction and a conclusion which both mention the dwarf Loffer.}

%TODO: move these stanzas to appendix?
\bvg\bva\mssnote{\Regius~1r/23, \Hauksbok~20r/17, \GylfMS}%
\alst{N}ýi ok \alst{N}iði, \hld\ \alst{N}orðri, Suðri, &
\alst{Au}stri, Vestri, \hld\ \alst{A}l-þjófr, Dvalinn, &
\alst{B}ívurr, \alst{B}ávurr, \hld\ \alst{B}ǫmburr, Nóri, &
\alst{Ȧ}nn ok \alst{Ȧ}narr, \hld\ \alst{Á}i, Mjǫð-vitnir.\eva

\bvb New and Nithe, Norther and Souther, \\
Easter and Wester, Allthief, Dwollen, \\
Bewer, Bower, Bamber, Noor, \\
Own and Owner, Great-grandfather, Meadwitner.\evb\evg


\bvg\bva\mssnote{\Regius~1r/25, \Hauksbok~20r/18, \GylfMS}%
\alst{V}ęigr ok Gand-alfr, \hld\ \alst{V}ind-alfr, Þráinn, &
\alst{Þ}ękkr ok \alst{Þ}orinn, \hld\ \alst{Þ}rór, Vitr ok Litr, &
\alst{N}ár ok \alst{N}ý-ráðr— \hld\ \alst{n}ú hęf’k dverga &
—\alst{R}ęginn ok \alst{R}áð-sviðr— \hld\ \alst{r}étt of talða.\eva

\bvb Wey and Gandelf, Windelf, Thrown, \\
Thetch and Thorn, Threw, Wit and Lit, \\
Nee and Newred—now have I the dwarfs— \\
Rain and Redswith—rightly tallied.\evb\evg


\bvg\bva\mssnote{\Regius~1r/28, \Hauksbok~20r/20, \GylfMS}%
\alst{F}íli, Kíli, \hld\ \alst{F}undinn, Náli, &
\alst{H}ępti, Víli, \hld\ \alst{H}annarr, Svíurr, &
\alst{F}rár, Horn-bori, \hld\ \alst{F}rę́gr ok Lȯni, &
\alst{Au}r-vangr, \alst{Ja}ri, \hld\ \alst{Ęi}kin-skjaldi.\eva

\bvb Filer, Chiler, Found and Needler, \\
Hefter, Wiler, Hanner, Swigher, \\
Fraw, Hornborer, Fray and Looner, \\
Earwong, Earer, Oakenshield.\evb\evg


\bvg\bva\mssnote{\Regius~1r/30, \Hauksbok~20r/22, \GylfMS}%
Mál es \alst{d}verga \hld\ ï \alst{D}valins liði &
\alst{l}jȯna kindum \hld\ til \alst{L}ofars tęlja, &
\edtext{þęir}{\Afootnote{\emph{þeim} \Hauksbok}} es \alst{s}óttu \hld\ frȧ \alst{s}alar stęini &
\alst{Au}r-vanga sjǫt \hld\ til \alst{Jǫ}ru-valla.\eva

\bvb ’Tis time to tally the dwarfs in Dwollen’s troops \\
{[back]} to Loffer for the races of men;\footnoteB{A standard genealogical introduction (cf. \Haleygjatal\ 1: \emph{meðan hans ę́tt \dots\ til goða tęljum} ‘while we tally his line \dots\ [back] to the gods’).  The (patrilineal) line of dwarfs is to be counted back to their progenitor, Loffer.  This possibly disagrees with st. 10, where Moodsowner is said to be the foremost (and presumably the oldest) of the dwarfs, and Loffer is not mentioned, but such details were probably not very important.} \\
they who sought, from the stone of the hall, \\
the abode of the \inx[L]{Earwongs} to the \inx[L]{Erwolds}.\footnoteB{Cf. \Gylfaginning\ 14: “But these came from Swornshigh (\emph{Svarinshaugr}) to the Earwongs on the Erwolds, and thence Lofer is come; these are their names: Sherper (\emph{Skirpir}), Werper (\emph{Virpir}), Showfind, Great-grandfather, Elf and Ing (\emph{Ingi}), Oakenshield, Fale (\emph{Falr}), Frost, Finn, Ginner.”}\evb\evg


\bvg\bva\mssnote{\Regius~1r/32, \Hauksbok~20r/24, \GylfMS}%
Þar vas \alst{D}raupnir \hld\ ok \alst{D}olg-þrasir, &
\alst{H}ár, \alst{H}aug-spori, \hld\ \alst{H}lé-vangr, Glói, &
\alst{Sk}irfir, Virfir, \hld\ \alst{Sk}áfiðr, Ái, &
\alst{A}lfr ok \alst{Y}ngvi, \hld\ \alst{Ęi}kin-skjaldi, &
\alst{F}jalarr ok \alst{F}rosti, \hld\ \alst{F}innr ok Ginnarr; &
Þat mun \edtext{\alst{ę́}}{\Afootnote{om. \Regius}} \alst{u}ppi, \hld\ meðan \alst{ǫ}ld lifir, &
\alst{l}ang-niðja-tal \hld\ \edtext{til}{\Afootnote{om. \Hauksbok}} \alst{L}ofars hafat.\eva

\bvb There was Dreepner and Dollowthrasher, \\
High, Highspurer, Leewong, Glower, \\
Sherver, Werver, Showfind, Great-grandfather, \\
Elf and Ing, Oakenshield, \\
Feller and Frost, Finn and Ginner.— \\
It will ever be remembered while the age lives,\footnoteB{Two archaic formulæ. The first literally ‘that will ever [be] up above’, cf. \HervararSaga\ TODO: “We two are cursed, brother, thy bane am I become! That will ever be remembered (\emph{þat mun ę́ uppi}, but both mss. \emph{þat mun enn uppi}), evil is the doom of the norns!” The second is found in a runic inscription, U 323 (980–1015): “Ever will lie—while the age lives (\textbf{meþ + altr + lifiʀ} \emph{með aldr lifir})—the hard-hammered bridge, broad, after a good man.” An especially close parallel is found in Þstf \emph{Stuttdr} (st. 5, Kari Ellen Gade ed. in \Skp\ II): \emph{Ęy mun uppi \hld\ Ęndils, meðan stęndr // sól-borgar salr, \hld\ svǫr-gǿðis fǫr.} ‘Always will be remembered—while the hall of the sun’s stronghold \ken{sky/heaven > earth} stands—the journey of the fattener of Andle’s bird \ken{raven/eagle > warrior}.’} \\
the tally of kinsmen lifted to Lofer.\evb\evg

\sectionline

\bvg\bva\mssnote{\Regius~1v/1, \Hauksbok~20r/26}%
\edtrans{Unds}{Until}{\Bfootnote{We seem to be missing a preceding sentence here, probably being contained in a now-lost stanza.  What this st. would have contained is of course impossible to know, but it may have given a reason for the creation of men.}} \edtext{\alst{þ}rír}{\Afootnote{gramm. emend.; \emph{þrjár} \Regius\Hauksbok}} kvǫ̇mu \hld\ \edtext{ór \alst{þ}ví liði}{\Afootnote{\emph{þussa brúðir} ‘brides of thurses’ \Hauksbok\ is probably corrupt due to the influence of st. 8; the adjectives in l. 2 are in the masculine.}} &
\edtrans{\alst{ǫ}flgir ok \alst{ȧ}stkir}{strong and lovely}{\Afootnote{\emph{ȧstkir ok ǫflgir} (norm.) ‘lovely and strong’ \Hauksbok}} \hld\ \alst{ę̇}sir \edtrans{at húsi}{along the settlement}{\Bfootnote{An adverbial, lit. ‘along the house’; the gods were not walking in the wilderness.}}; &
fundu ȧ \alst{l}andi \hld\ \alst{l}ítt męgandi &
\alst{A}sk ok \alst{E}mblu \hld\ \alst{ø}r-lǫg-lausa.\eva

\bvb Until three came out of that host: \\
strong and lovely Eese along the settlement; \\
they found on land the little availing \\
Ash and Emble, \inx[C]{orlay}-less.\footnoteB{This verse is paraphrased in \Gylfaginning\ 9: \emph{Þá er þeir gengu með sę́var-strǫndu Bors synir, fundu þeir tré tvau ok tóku upp trén ok skǫpuðu af menn. Gaf inn fyrsti ǫnd ok líf, annarr vit ok hrę́ring, þriði á-sjónu, mál ok heyrn ok sjón, gáfu þeim klę́ði ok nǫfn. Hét karl-maðrinn Askr, en konan Embla, ok ólst þaðan af mann-kindin, sú er byggðin var gefinn undir Mið-garði.} ‘When the sons of Byre (cf. st. 4) walked along the sea-shore they found two trees and they took up the trees and shaped men from them. The first one gave breath (\emph{ǫnd}) and life, the second wit and movement, the third sight, speech, appearance and sight; they gave them clothes and names. The male was called Ash, and the woman Emble, and from them mankind was begotten, to whom were given the dwelling within Middenyard.’

The ON cognate of tree, \emph{tré}, can also mean ‘pieces of wood’, and it is traditionally seen as referring to pieces of driftwood. Yet as pointed out by \textcite{Hultgård2006} the comparative evidence suggests that the two were in fact living, growing trees (they would thus be part of the foliage described in st. 4) and there is nothing in the sources that speaks against this.

While Ash is easily identified with the same-named wood species (\emph{Fraxinus excelsior}), the etymology of Emble is much more difficult. The shaping of men from trees is used by poets in various kennings for men and women, especially in Scaldic poetry (for a short discussion see \textciteshorttitle{SkP} I, p. lxxv ff.). While this is rarer in the Eddic corpus it does occur, e.g. in \Sigrdrifumal\ 5: \emph{bryn-þings apaldr} ‘apple-tree of the byrnie-\inx[C]{Thing} \ken{battle > warrior}’.}\evb\evg


\bvg\bva\mssnote{\Regius~1v/3, \Hauksbok~20r/27}%
\alst{Ǫ}nd þau né \alst{ǫ́}ttu, \hld\ \alst{ó}ð þau né hǫfðu, &
\alst{l}ǫ́ né \alst{l}ę́ti \hld\ né \alst{l}itu góða; &
\alst{ǫ}nd gaf \alst{Ó}ðinn, \hld\ \alst{ó}ð gaf Hø̇nir, &
\alst{l}ǫ́ gaf \alst{L}óðurr \hld\ ok \alst{l}itu góða.\eva

\bvb Breath they owned not, \inx[C]{wode} they had not, \\
not craft nor sound nor good countenance. \\
Breath gave Weden, wode gave Heener, \\
craft gave Lother, and good countenance.\evb\evg

\sectionline

\bvg\bva\mssnote{\Regius~1v/5, \Hauksbok~20r/29, \GylfMS}%
\alst{A}sk vęit’k \edtext{standa}{\lemma{standa ‘standing’}\Afootnote{so \Regius\Hauksbok\Upsaliensis; \emph{ausinn} ‘poured, sprinkled’ \RegiusProse\Trajectinus\Wormianus}}, \hld\ hęitir \edtext{\alst{Y}gg-drasill}{\Afootnote{\emph{Ygg-drasils} \RegiusProse}}, &
\alst{h}ǫ́r \edtrans{baðmr}{beam}{\Afootnote{\emph{borinn} ‘born’ \Upsaliensis\ is wo. doubt corrupt.}}, \edtrans{ausinn}{poured}{\Afootnote{\emph{hęilagr} ‘holy’ \GylfMS}} \hld\ \alst{h}víta auri; &
þaðan koma \alst{d}ǫggvar \hld\ \edtext{þę́r’s}{\Afootnote{\emph{es} \RegiusProse\Trajectinus}} ï \alst{d}ala falla; &
stęndr \edtext{\alst{ę́}}{\Afootnote{\emph{om.} \Upsaliensis}} \alst{y}fir \edtext{grø̇nn}{\Afootnote{\emph{†grvnn†} \RegiusProse; \emph{†grein†} \Upsaliensis}} \hld\ \alst{U}rðar brunni.\eva

\bvb An ash I know standing, ’tis called \inx[L]{Ugdrassle}; \\
a high beam \ken{tree}, poured with white mud.\footnoteB{i.e. ‘white mud is (or has been) poured upon it.’ Possibly relevant is the Indian ritual pouring of beverages onto the phallic \emph{lingam} (though the good Nikhil S. Dwibhashyam denies that this goes back to the Vedic period, and so it may be unrelated). For the whole passage cf. st. 26.} \\
Thence come the dew-drops which fall in the dales; \\
it stands ever green over \inx[L]{Weird’s Well}.\evb\evg


\bvg\bva\mssnote{\Regius~1v/8, \Hauksbok~20r/31}%
Þaðan koma \alst{m}ęyjar \hld\ \alst{m}args vitandi &
\alst{þ}ríar ór þęim \edtrans{sal}{hall}{\Afootnote{so \Hauksbok, \GylfMS\ (paraphrase); \emph{sę́} ‘lake’ \Regius}} \hld\ es \edtrans{und}{under}{\Afootnote{\emph{ȧ} ‘on’ \Hauksbok}} \edtrans{\alst{þ}olli}{tree}{\Bfootnote{Literally ‘fir’, but the word is only used for the alliteration.  The same may perhaps apply to \emph{askr} ‘ash’ above, the species being indeterminate.}} stęndr; &
\alst{U}rð hétu \alst{ęi}na, \hld\ \alst{a}ðra Verðandi, &
—\edtrans{\alst{sk}ǫ́ru ȧ \alst{sk}íði}{they scored billets}{\Bfootnote{Unclear; perhaps they carve markings for the number of years each man has to live.}}— \hld\ \alst{Sk}uld hina þriðju &
þę́r \alst{l}ǫg \alst{l}ǫgðu, \hld\ þę́r \alst{l}íf køru, &
\alst{a}lda bǫrnum, \hld\ \alst{ø}r-lǫg \edtrans{sęggja}{of youths}{\Afootnote{\emph{at sęgja} ‘to say’ \Hauksbok}}.\eva

\bvb Thence come maidens, much knowing: \\
three from the hall which stands under the tree. \\
Weird they called one, the other Werthing \\
—they scored billets—Shild the third. \\
Laws they laid, lives they chose \\
for the children of mankind, the \inx[C]{orlay} of youths.\footnoteB{i.e. ‘they have carved on boards, they have laid laws, they have chosen lives’. It is well known that in Old Norse as in other old Germanic languages the simple past can have both perfective and imperfective sense. — This st. is paraphrased in \Gylfaginning\ 15: \emph{Þar stendr salr einn fagr undir askinum við brunninn, ok ór þeim sal koma þrjár meyjar, þę́r er svá heita: Urðr, Verðandi, Skuld. Þessar meyjar skapa mǫnnum aldr; þę́r kǫllum vér nornir.} ‘There is a single fair hall beneath the ash-tree by the well, and from that hall come three maidens, who are called thus: Weird, Werthing, Shild. These maidens shape the ages of men (formulaic! TODO.); we call them norns.’}\evb\evg

\sectionline

\bvg\bva\mssnote{\Regius~1v/11, \Hauksbok~20v/5}%
Þat man hǫ̇n \edtrans{\alst{f}olk-víg}{troop-conflict}{\Bfootnote{\emph{folk} here carries its older meaning ‘troop, band’, as seen in the Slavic borrowing exemplified by Russian \textrussian{полк} ‘regiment, host, army’.}} \hld\ \alst{f}yrst ï hęimi, &
es \alst{G}ull-vęigu \hld\ \alst{g}ęirum studdu &
ok ï \alst{h}ǫll \alst{H}áars \hld\ \alst{h}ȧna bręnndu, &
\edtext{\alst{þ}rysvar bręnndu}{\Afootnote{\emph{†þrysvar brendv þrysvar brendv†} \Hauksbok}} \hld\ \alst{þ}rysvar borna, &
\alst{o}pt, \alst{ȯ}-sjaldan, \hld\ þó hǫ̇n \alst{ę}nn lifir.\eva

\bvb That troop-conflict she recalls, the first in the \inx[C]{Home}, \\
when Goldwey with spears they goaded, \\
and in the hall of \inx[P]{Higher} \name{= Weden} \ken*{= Walhall} burned her; \\
thrice they burned the thrice born, \\
often, unseldom, though she still lives.\footnoteB{Very cryptic. TODO: double check Snorri. Goldwey was apparently slain, cremated and reborn three times (in short succession?) by the Eese.}\evb\evg


\bvg\bva\mssnote{\Regius~1v/13, \Hauksbok~20v/7}%
\alst{H}ęiði \alst{h}étu, \hld\ hvar’s til \alst{h}úsa kom, &
\edtext{\alst{v}ǫlu}{\Afootnote{\emph{ok vǫlu} \Hauksbok}} \alst{v}ęl-spáa, \hld\ \alst{v}itti ganda; &
\alst{s}ęið hǫ́n \edtrans{hvar’s hǫ́n kunni}{where she could}{\Afootnote{so \Hauksbok; \emph{hǫ́n kunni} ‘she knew’ \Regius}}, \hld\ \alst{s}ęið hǫ́n \edtrans{hug lęikinn}{deluded minds}{\Afootnote{so \Hauksbok; \emph{leikinn} \Regius}}; &
\alst{ę́} vas hǫ̇n \alst{a}ngan \hld\ \alst{i}llrar brúðar.\eva

\bvb Heath they called—where to houses she came— \\
a well-spaeing \inx[C]{wallow}; she bewitched \inx[C]{gand}[gands]. \\
She sorcered where she could; she sorcered deluded minds; \\
she was always the love of any evil bride.\evb\evg

\sectionline

\bvg\bva\mssnote{\Regius~1v/16, \Hauksbok~20v/9}%
Þȧ gingu \alst{r}ęgin ǫll \hld\ ȧ \alst{r}ǫk-stóla, &
\alst{g}inn-hęilǫg goð, \hld\ ok umb þat \alst{g}ę́ttusk: &
Hvárt skyldu \alst{ę̇}sir \hld\ \alst{a}f-ráð gjalda, &
eða skyldu \edtrans{\alst{g}oð’in}{the Gods}{\Bfootnote{The clitic definite is very rare in older Norse poetry; this is its only occurence in \Voluspa.}} ǫll \hld\ \alst{g}ildi ęiga?\eva

\bvb Then went the Reins all onto the rake-seats: \\
the Yin-holy Gods, and from each other took counsel of this: \\
Whether the Eese should yield tribute, \\
or should the Gods all hold a banquet?\evb\evg


\bvg\bva\mssnote{\Regius~1v/17, \Hauksbok~20v/11}%
\alst{F}lęygði Óðinn \hld\ ok ï \alst{f}olk of skaut; &
þat vas ęnn \alst{f}olk-víg \hld\ \edtrans{\alst{f}yrr}{earlier}{\Afootnote{so \Hauksbok; \emph{fyrst} ‘first’ \Regius. The \Regius\ reading cannot be correct as this st. is describing a different war, and thus not the first. It has probably arisen due to the similarity with st. 20/1.}} ï hęimi; &
\alst{b}rotinn vas \alst{b}orð-vęggr \hld\ \alst{b}orgar ȧsa, &
knǫ́ttu \alst{v}anir \alst{v}íg-spǫ́ \hld\ \alst{v}ǫllu sporna.\eva

\bvb Weden hurled and shot into the troop;\footnoteB{The object, a spear, is understood. This seems to reference a ritual, well-attested in the literature, wherein a war-chief would dedicate an opposing army as a human sacrifice to Weden by throwing a spear over them, typically with the incantation \emph{Óðinn á yðr alla} ‘Weden owns you all!’; he would then own the battle-slain in that they joined him as \inx[G]{Oneharriers} in \inx[L]{Walhall}. Weden is also described as “owning” dead men in \Harbardsljod\ 24 (namely slain nobles, contrasted with \inx[P]{Thunder} who is insultingly said to “own the kin of thralls”) and in runic inscription \emph{N B380} (edited below under Galders), a sort of greeting wherein the receiver is wished to be owned by Weden (and “received” by Thunder). For further literature see \textciteshorttitle{PCRN-HS} II:24, p. 560, II:25, p. 617, and especially III:42, p. 1166ff.} \\
that was yet a troop-conflict earlier in the \inx[L]{Home}. \\
Broken was the plank-wall of the stronghold of the Eese; \\
the Wanes did by a conflict-\inx[C]{spae} tread the fields.\footnoteB{The Wanes used magic spells to win the battle.}\evb\evg

\sectionline

\bvg\bva\mssnote{\Regius~1v/19, \Hauksbok~20r/34, \GylfMS}%
Þȧ gingu \alst{r}ęgin ǫll \hld\ ȧ \alst{r}ǫk-stóla, &
\alst{g}inn-hęilǫg \alst{g}oð, \hld\ ok umb þat \alst{g}ę́ttusk: &
Hvęrr hęfði \alst{l}opt alt \hld\ \alst{l}ę́vi blandit &
eða \alst{ę́}tt \alst{jǫ}tuns \hld\ \alst{Ó}ðs męy gefna?\eva

\bvb Then went the Reins all onto the rake-seats: \\
the Yin-holy Gods, and from each other took counsel of this: \\
Who might have blended all the air with deceit, \\
or to the ettin’s lineage given \inx[P]{Wode}’s maiden \ken*{= Frow}?\footnoteB{That is, promised Frow to the wall-builder.  Cf. \Gylfaginning\ 42.  TODO: elaborate.}\evb\evg


\bvg\bva\mssnote{\Regius~1v/20, \Hauksbok~20r/36, \GylfMS}%
\edtext{\alst{Þ}ȯrr ęinn \edtrans{\alst{þ}ar vá}{fought there}{\Afootnote{so \Hauksbok\Trajectinus\Upsaliensis; \emph{þar var} ‘was there’ \Regius; \emph{þat vann} ‘accomplished it’ \RegiusProse; \emph{þat vá} ‘fought it’ \Wormianus}} \hld\ \alst{þ}runginn móði, &
\edtrans{hann \alst{s}jaldan \alst{s}itr \hld\ es \alst{s}líkt of fregn;}{he seldom sits when of such he learns}{\Bfootnote{Namely ettins encroaching on the gods.  Thunder is the defender of the gods (\Thrymskvida\ 18) and is willing to break certain laws of frith for this purpose (\Lokasenna\ 57–64).}} &
\edtext{\alst{ȧ} gingusk \alst{ęi}ðar, \hld\ \alst{o}rð ok sǿri, &
\alst{m}ǫ́l ǫll \alst{m}ęgin-lig, \hld\ es ȧ \alst{m}eðal \edtext{fóru}{\lemma{fóru ‘had gone’}\Afootnote{\emph{vǫ́ru} ‘had been’ \Hauksbok\Trajectinus}}.}{\lemma{ȧ \dots\ fóru.}\Afootnote{om. \Wormianus}}}{\lemma{ALL}\Afootnote{The order of the lines is that of \Regius\Hauksbok; in \GylfMS\ the two helmings (\emph{Þȯrr \dots\ fregn;} and \emph{ȧ \dots\ fóru.}) are reversed.}}\eva

\bvb Thunder alone fought there, pressed by wrath; \\
he seldom sits when of such he learns. \\
Trampled were oaths, speeches and vows, \\
the mighty treaties all which had gone between them.\evb\evg

\sectionline

\bvg\bva\mssnote{\Regius~1v/23, \Hauksbok~20v/1}%
Vęit hǫ̇n \alst{H}ęim-dalar \hld\ \alst{h}ljóð of folgit &
und \edtrans{\alst{h}ęið-vǫnum}{shady}{\Bfootnote{Literally ‘light-less’, \emph{hęiðr} referring especially to the light of a clear sky.}} \hld\ \alst{h}ęlgum baðmi; &
\alst{ǫ́} sér hǫ̇n \alst{au}sask \hld\ \edtrans{\alst{au}rgum}{muddy}{\Bfootnote{Which should be the same mud (\emph{aurr}) as in st. 19, there said of Weird’s Well.}} forsi &
af \edtrans{\alst{v}eði \alst{V}al-fǫðrs}{Walfather’s pledge}{\Bfootnote{Mimer’s well gives wisdom to any man who drinks from it, so \Gylfaginning\ 15: \emph{Þar kom Alfǫðr ok beiddisk eins drykkjar af brunninum, en hann fekk eigi, fyrr en hann lagði auga sitt at veði.} ‘There came Allfather and asked for a single drink from the well, but he did not get it before he laid down his eye as a pledge.’}}. \hld\ \alst{V}ituð ér ęnn eða hvat?\eva

\bvb She knows Homedal’s sound \ken*{= Horn of Yell?} hidden \\
beneath the shady, hallowed beam \ken*{= Ugdrassle’s Ash?}. \\
A river she sees being fed by a muddy torrent \\
from Walfather’s pledge \ken*{= Mimer’s well}.—Know ye yet, or what?\footnoteB{“Do you (Weden) know enough now, or what?”—repeated in 28, 33, 34, 38, 40, 47, 60, 61.}”\evb\evg

\sectionline

\bvg\bva\mssnote{\Regius~1v/25}%
\alst{Ęi}n sat hǫ̇n \alst{ú}ti, \hld\ þȧ’s hinn \alst{a}ldni kom &
\alst{y}ggjungr \alst{ȧ}sa \hld\ ok ï \alst{au}gu lęit: &
‚hvęrs \alst{f}regnið mik? \hld\ hví \alst{f}ręistið mïn?\eva

\bvb Alone sat she outside when the old one came, \\
the Terrifier of the Eese \ken*{= Weden}, and looked into her eyes. \\
\speakernoteb{[The Wallow:]}%
‘Of what askest thou me? Why tempest thou me?\footnoteB{\emph{fręista} has a sense of testing someone, especially intellectually. Cf. \Havamal\ 2, 26, \Vafthrudnismal\ 3, 5.}\evb\evg


\bvg\bva\mssnote{\Regius~1v/26, \GylfMS}%
\alst{A}lt vęit’k, \alst{Ó}ðinn, \hld\ hvar \alst{au}ga falt &
\edtrans{ï hinum \alst{m}ę́ra}{in the renowned}{\Afootnote{so \Wormianus; \emph{þitt} (corr.) \emph{i enom męra} ‘id.’ \Regius; \emph{j þeim enom meira} ‘in the greater’ \Trajectinus; \emph{i þeim envm mæra} ‘in the renowned’ \Upsaliensis; \emph{vr þeim envm mę́ra} ‘out of the renowned’ \RegiusProse}} \hld\ \alst{M}ímis brunni; &
drekkr \alst{m}jǫð \alst{M}ímir \hld\ \alst{m}orgin hvęrjan &
af \edtrans{\alst{v}eði}{pledge}{\Afootnote{\emph{†veiði†} \RegiusProse}} \alst{V}al-fǫðrs.‘ \hld\ \alst{V}ituð ér ęnn eða hvat?\eva

\bvb I know it all, Weden, where thine eye thou hidst: \\
in the renowned \inx[L]{Mimer’s Well} \\
drinks Mimer mead every morning \\
from Walfather’s pledge.’—Know ye yet, or what?\evb\evg


\bvg\bva\mssnote{\Regius~1v/29}%
Valði hęnni \alst{H}ęr-fǫðr \hld\ \alst{h}ringa ok męn, &
\edtrans{fekk \alst{sp}jǫll \alst{sp}ak-lig}{got foresighted tidings}{\Afootnote{emend.; \emph{fe spioll spaclig} \Regius}\Bfootnote{The reading of \Regius\ may be interpreted either as (1): \emph{fé-spjǫll spak-lig} ‘foresighted wealth-spells’ or (2) \emph{fé, spjǫll spak-lig} ‘wealth, foresighted tidings’; both are metrically deficient.  In (1) a second element in a cpd. like \emph{fé-spjǫll} cannot carry alliteration, and (2) has three strongly stressed nominals; in both cases \emph{fé} which stands first would be expected to carry the alliteration.  The word \emph{fé} ‘wealth, cattle’ also makes little sense in context, since Weden is the one giving her expensive jewellery.

The emendation places the verb \emph{fekk} ‘got, received’ for \emph{fé}.  Verbs carry less stress than verbs, and the line is thus metrically equivalent to 28/3b \emph{drekkr mjǫð Mímir}.  The line parallels st. 1, where the wallow likewise says that she will relate \emph{spjǫll} ‘tidings, sayings’ (cf. English \emph{gospel} lit. ‘good news’ which originally translates the Greek \textgreek{εὐαγγέλιον}).  For discussion on this reading see \textcite[51--53]{Haukur2020}, \textcite[16]{Males2023}.}} \hld\ ok \edtrans{\alst{sp}á-ganda}{spae-gands}{\Bfootnote{Spirits sent out in order to gather hidden wisdom and spaes.  See relevant Encyclopedia entries.}}; &
sá \alst{v}ítt ok umb \alst{v}ítt \hld\ of \alst{v}er-ǫld hvęrja.\eva

\bvb Host-father \name{= Weden} chose for her rings and a necklace, \\
he got foresighted tidings and \inx[C]{spae}-\inx[C]{gands}— \\
she saw widely and more widely, o’er every world.\evb\evg


\bvg\bva\mssnote{\Regius~1v/30}%
Sá hǫ̇n \alst{v}al-kyrjur \hld\ \alst{v}ítt of komnar, &
\alst{g}ǫrvar at ríða \hld\ til \edtrans{\alst{g}oð-þjóðar}{land of the Gots}{\Bfootnote{Ambiguous; ON \emph{goð-þjóð} may mean either (1) ‘land of the Gots’ or (2) ‘land of the Gods’, for the difficult cluster \emph{tþ} in \emph{Got-þjóð} ‘land of the Gots’ was at some point changed to \emph{ðþ}.  Sense (1) is preferred since it is attested in three other places in \Regius, viz. \Helreid\ TODO and \Gudrunarhvot\ TODO and TODO; (2) is entirely unattested.  One may note that ON \emph{Got-þjóð} reflects the attested Gotnish self-name, \emph{Gut-þiuda}, found in the October 29 entry of the Gotnish calender (TODO: reference).

The Walkirries have a particular association with the Gots, who fought the greatest battles of the Migration Period; cf. note to \Volundarkvida\ 1/1b.}}: &
\edtext{\alst{Sk}uld hélt \alst{sk}ildi, \hld\ en \alst{Sk}ǫgul ǫnnur, &
\alst{G}unnr, Hildr, \alst{G}ǫndul \hld\ ok \alst{G}ęir-skǫgul; &
\alst{n}ú eru talðar \hld\ \edtrans{\alst{N}ǫnnur Hęrjans}{Nans of Harn \name{= Weden}}{\Bfootnote{\emph{Nanna} ‘\inx[P]{Nan}’ (the name itself is a nursing word) was the wife of \inx[P]{Balder}, but the word is here certainly being used to refer generically to ‘maidens, women’.  Cf. Þul \emph{Ásynja} (\Skp\ 3), where the walkirries are kenned \emph{Óðins męyjar} ‘Weden’s maidens’.}}, &
\alst{g}ǫrvar at ríða \hld\ \alst{g}rund, val-kyrjur.}{\lemma{Skuld \dots\ val-kyrjur Shild}\Bfootnote{Judging especially by the out-of-place phrase \emph{nú eru talðar} ‘now are tallied’, these four lines seem to be a later insert from a \inx[C]{thule} counting the walkirries.}}\eva

\bvb She saw \inx[G]{Walkirries} come from afar, \\
ready to ride to the land of the \inx[C]{Gots}. \\
Shild held a shield and Shagle another, \\
Guth, Hild, Gandle and Goreshagle— \\
now are tallied the Nans of Harn \name{= Weden}, \\
ready to ride the ground, the walkirries.\evb\evg

\sectionline

{\small Told allusively in \Voluspa\ 31–33 is the myth about Balder’s death.  Balder, the son of Weden and Frie, was slain with an arrow shot by his blind half-brother Hath, whose hand was guided by Lock.  Weden could not slay Hath, who was his son, and so he seduced the woman Rind, apparently through love-magic (Cormac Awmundson’s TODO: \emph{sęið Yggr til rindar} ‘Ug won Rind through sorcery’).  Rind gave birth to Wonnel, who grew very fast; after just one day he was big enough to kill Hath, which he also did, avenging Balder’s death.  The other important sources for this myth are \Baldrsdraumar\ 8–11, \Gylfaginning\ 49, and \textcite[3.4.1–8]{Saxo}.

The language of \Baldrsdraumar\ is so similar to the present sts. that they must be of common origin; \Baldrsdraumar\ 11/2–4 is near-identical to \Voluspa\ 32/4–33/2.  The biggest narrative difference is that \Baldrsdraumar\ mentions Rind, who is not found in \Voluspa.

The most elaborate narrative is found in \Gylfaginning\ 49, which may be shortly summarised as follows: Balder has terrible nightmares about his own death, and so his mother Frie makes all sorts of things (fire, water, venom, metals, stones, trees, diseases, beasts, et. c.) swear oaths not to harm him.  After this the Eese make sport of shooting and striking at him, since he cannot be harmed.  Lock is annoyed by this and approaches Frie while disguised as a woman.  He finds out from her that there is one thing that did not swear the oath—the mistletoe, which was thought too young.  Lock takes a mistletoe and a bow and gives it to the blind god Hath, showing him where to shoot.  Hath does so, and kills Balder.  After this \Gylfaginning\ describes Balder’s funeral (treated poetically in Wolf Ugson’s fragmentary \emph{House-drape}, ÚlfrU \emph{Húsdrp} in \Skp\ III) and how the gods attempted to “weep Balder out of hell”, which failed (see Eddic Fragments in the present ed.)  \Gylfaginning\ 50 goes on to describe how the Eese punished Lock (see st. 34 below.)

It is notable that \Gylfaginning\ 49–50 fails to mention Wonnel.  This part of the myth may have been left out for moral reasons, but was certainly known to the author of the Prose Edda; cf. \Gylfaginning\ 30: \emph{Áli eða Váli heitir einn, sonr Óðins ok Rindar. Hann er djarfr í orrostum ok mjǫk happ-skęytr} ‘Onnel or Wonnel one is called, the son of Weden and Rind. He is brave in battles and a very lucky shot’ and \Skaldskaparmal\ 19: \emph{Hvernig skal kenna Vála? Svá, at kalla hann son Óðins ok Rindar, [\dots] hefni-ás Baldrs, dólg Haðar ok bana hans, [\dots]} ‘How shall one ken Wonnel? Namely by calling him the son of Weden and Rind, [\dots] avenging \inx[C]{os} of Balder, the foe of Hath and his bane, [\dots].’

The last source is \textcite{Saxo}[3.4.1--8], who retells the revenge narrative in typical euhemerized form; his versions of Hath and Balder are distinctly human generals and rulers. It may be summarized as follows: Weden takes counsel from a group of seers; one of them, Horsethief the Finn, foretells that Rind, daughter of the Russian king, will bear him another son to avenge Balder.  Weden soon enlists in the king’s army and leads it to great victories, but is continually spurned by the daughter.  He tries various other disguises but is still refused.  At last he disguises himself as an old woman and becomes her physician.  When she turns sick, he binds her, supposedly in order to give her a certain foul potion—he instead rapes her, apparently with her father’s consent.  Their son, Bo, grows up to become a fierce raider.  One day Weden summons him and reminds him of his duty to avenge his brother, Balder.  Bo slays Hath in a duel, but soon perishes from his wounds.}%TODO: add Saxo’s Latin names in parenthesis

\sectionline

\bvg\bva\mssnote{\Regius~2r/2}%
Ek sá \alst{B}aldri, \hld\ \alst{b}lóðgum \edtrans{tífur}{victim’s}{\Bfootnote{This word is rather difficult and possibly corrupt.  It may be connected with \emph{týr} ‘tew, god’, but the dat. sg. of \emph{týr} is \emph{tívi} and the intrusive \emph{r} is unexplained.  A better explanation is given by \CV, who connect it with OE \emph{tiber, tifer} ‘victim, hostage’, but this also has some problems.  \emph{blóðgum} ‘bloody’ is masc. dat. sg., but OE \emph{tiber} is neuter.  If we are dealing with a masc. noun \emph{*tífurr} with the same declension as \emph{jǫfurr}, we would expect dat. sg. \emph{*tífri}, not \emph{tífur} (which would however be the expected acc. sg.).}}, &
\alst{Ó}ðins barni, \hld\ \alst{ø}r-lǫg \edtrans{folgin}{sealed}{\Bfootnote{Or “hidden”.  The verb \emph{fela} ‘hide, conceal’ is used in poetry to describe burial in mounds, as in \Ynglingatal\ 24 (“[...] And afterwards the victory-havers hid (\emph{fǫ́lu}) the ruler on Borrey.”) or the C10th Karlevi stone (“Hidden (\textbf{fulkin} \emph{folginn}) in this mound lies he whom the greatest deeds followed; [...]”)}}; &
stóð of \alst{v}axinn \hld\ \alst{v}ǫllum hę́ri &
\alst{m}jór ok \alst{m}jǫk fagr \hld\ \alst{m}istil-tęinn.\eva

\bvb I saw Balder’s—the bloody victim’s, \\
Weden’s child’s—\inx[C]{orlay} sealed: \\
there stood grown—higher than the plains, \\
slender and most fair—the mistletoe.\evb\evg


\bvg\bva\mssnote{\Regius~2r/4}%
Varð af \alst{m}ęiði, \hld\ þęim’s \alst{m}ę́r sýndisk, &
\alst{h}arm-flaug \alst{h}ę́ttlig, \hld\ \alst{H}ǫðr nam skjóta. &
\alst{B}aldrs \alst{b}róðir vas \hld\ of \alst{b}orinn snimma, &
sá nam, \alst{Ó}ðins sonr, \hld\ \alst{ęi}n-nę́ttr vega.\eva

\bvb Of the tree which slender seemed \\
became a baneful harm-flier—Hath took to shoot. \\
Balder’s brother \ken*{= Wonnel} was born early; \\
he took, Weden’s son, one night old, to fight.\evb\evg


\bvg\bva\mssnote{\Regius~2r/6}%
\edtext{Þó}{\lemma{Þó \dots\ kęmbði ‘washed \dots\ combed’}\Bfootnote{A collocation, see note to \Havamal\ 61 for discussion and other examples. Wonnel, being oathbound and on the mission to avenge his brother, could not engage in such acts of personal vanity.}} ę́va \alst{h}ęndr \hld\ né \alst{h}ǫfuð kęmbði, &
áðr ȧ \alst{b}ál of \alst{b}ar \hld\ \alst{B}aldrs and-skota; &
en \alst{F}rigg of grét \hld\ í \alst{F}ęn-sǫlum &
\edtrans{\alst{v}ǫ́ \alst{V}al-hallar}{the woe of Walhall}{\Bfootnote{The deaths of two sons; Balder and Hath.}}. \hld\ \alst{V}ituð ér ęnn eða hvat?\eva

\bvb He washed ne’er his hands nor combed his head, \\
before onto the pyre he bore Balder’s opponent \ken*{= Hath}, \\
and Frie lamented in the Fenhalls \\
the woe of Walhall.—Know ye yet, or what?\evb\evg

\sectionline

{\small After Balder was avenged the Eese went to catch lock.  They bound him up with his son’s intestines.  A snake was then placed over his face to drip venom onto it.  His wife, Syein, sat over him and caught the venom in a small basin; when she had to empty it he writhed so greatly that the earth shook.  This myth is found in \FraLoka\ (the prose at the end of \Lokasenna) and \Gylfaginning\ 50.}

\sectionline

\bvg\bva[H1]\mssnote{\Hauksbok~20v/12}%
\edtext{Þȧ kná \edtrans{\alst{V}áli}{Wonnel}{\Afootnote{emend.; \emph{Vála} \Hauksbok}} \hld\ \alst{v}íg-bǫnd snúa &
\alst{h}ęldr vǫ́ru \alst{h}arð-gǫr \hld\ \alst{h}ǫpt ór þǫrmum.}{\lemma{Þȧ \dots\ þǫrmum.}\Bfootnote{Only attested in \Hauksbok, where it replaces ll. 1–2 of 34.}}\eva

\bvb Then did \inx[C]{Wonnel} the war-bonds twist: \\
the most sturdy fetters were made from intestines.\evb\evg


\bvg\bva\mssnote{\Regius~2r/8, \Hauksbok~20v/13}%
\edtext{\alst{H}apt sá hǫ̇n liggja \hld\ und \alst{H}vera-lundi &
\edtrans{\alst{l}ę́-gjarns}{guile-eager}{\Bfootnote{A formulaic epithet of Lock. See note to TODO for other examples and discussion.}} líki \hld\ \alst{L}oka ȧ-þękkjan;}{\lemma{Hapt \dots ȧ-þękkjan ‘A captive \dots\ to Lock,’}\Afootnote{Replaced with H1 \Hauksbok.}} &
\alst{þ}ar sitr Sigyn \hld\ \alst{þ}ęygi of sínum &
\alst{v}eri \alst{v}ęl-glýjuð. \hld\ \alst{V}ituð ér ęnn eða hvat?\eva

\bvb A captive \ken{= Lock} she saw lying beneath Wharlund: \\
a guile-eager man’s form, alike to Lock,
There sits Syein not at all cheerful, \\
o’er her husband.—Know ye yet, or what?\evb\evg

\sectionline

{\small The following sts. are paraphrased in \Gylfaginning\ 52:

\begin{quote}
	\emph{Þá mę́lti Gangleri: „Hvat verðr þá eptir, er brenndr er himinn ok jǫrð ok heimr allr, ok dauð goðin ǫll ok allir Einherjar ok alt mann-folk, ok hafið ér áðr sagt, at hverr maðr skal lifa í nǫkkvǫrum heimi um allar aldir?“}

	\emph{Þá svarar Þriði: „Margar eru þá vistir góðar ok margar illar; batst er þá at vera á Gimléi á himni, ok all-gótt er til góðs drykkjar þeim, er þat þykkir gaman, í þeim sal, er Brimir heitir; hann stendr ok á himni. Sá er ok góðr salr, er stendr á Niða-fjǫllum, gørr af rauðu gulli; sá heitir Sindri. Í þessum sǫlum skulu byggja góðir menn ok sið-látir.}

	\emph{Á Ná-strǫndum er mikill salr ok illr ok horfa norðr dyrr; hann er ok ofinn allr orma-hryggjum sem vanda-hús, en orma hǫfuð ǫll vitu inn í húsit ok blása eitri, svá at eptir salnum renna eitr-ár, ok vaða þę́r ár eið-rofar ok morð-vargar, svá sem hér segir:“}
\end{quote}

\begin{quote}
	‘Then spoke Gangler: “What will then remain, when heaven and earth and the whole world is burned, and gods are dead and all the Oneharriers and all man-kind—and [still] ye have said earlier, that each man will live in some world for all ages?”

	Then answers Third: “Many good dwellings are there then, and many ill: it is then best to be in Gimlee in the heaven, and it is very good of good drink for those who find joy in that, in the hall which is called Brimmer; it also stands in heaven. Another good hall is the one which stands on the Nithfells, made from red gold; it is called Sinder. In these halls good and well-mannered men will dwell.

	On Neestrand is a great and bad hall, and its doors face north. It is all woven with the spines of serpents like a wicker-house, but the heads of the serpents all look into the house and blow venom, so that through the hall rivers of venom run, and in those rivers wade oath-breakers and murder-wargs, as is said here:”’
\end{quote}

after which are quoted sts. 37 and 38/1–2, followed by the prose: \emph{En í Hver-gelmi er verst} ‘But in Wharyelmer is is worst’ and 38/4.}

\sectionline

\bvg\bva\mssnote{\Regius~2r/10}%
\alst{Ǫ́} fęllr \alst{au}stan \hld\ of \alst{ęi}tr-dala &
\alst{s}ǫxum ok \alst{s}verðum, \hld\ \edtrans{\alst{S}líðr}{Slide}{\Bfootnote{i.e. ‘very sharp’. Cf. \Atlakvida\ 23: \emph{sax slíðr-bęitt} ‘slide-biting sax’.}} hęitir sú.\eva

\bvb A river falls from the east, above the venom-dales; \\
{[a river]} of saxes and swords, Slide is that one called.\footnoteB{TODO. There are other examples of such a river.}\evb\evg


\bvg\bva\mssnote{\Regius~2r/11}%
Stóð fyr \alst{n}orðan \hld\ ȧ \edtrans{\alst{N}iða-vǫllum}{Nithwolds}{\Afootnote{\emph{Niða-fjǫllum} ‘Nithfells’ \Regius\Wormianus\ (paraphrase); \emph{fjǫllom nǫkkurum} ‘some certain fells’ \Trajectinus}} &
\alst{s}alr ór gulli \hld\ \alst{S}indra ę́ttar; &
en \alst{a}nnarr stóð \hld\ ȧ \alst{Ȯ}kólni, &
\alst{b}jór-salr jǫtuns, \hld\ \edtrans{en sá \alst{B}rimir hęitir}{and it is called Brimmer}{\Bfootnote{It is not clear if this is the name of the ettin or the hall itself. The author of \Gylfaginning\ considered it the name of the hall.}}.\eva

\bvb Stood to the north on the Nithwolds, \\
a hall of gold, of Sinder’s lineage \ken{dwarfs}. \\
But another one stood on Uncolner, \\
an ettin’s beer-hall, and it is called Brimmer.\evb\evg


\bvg\bva\mssnote{\Regius~2r/13, \Hauksbok~20v/19, \GylfMS}%
\alst{S}al \edtrans{sá hǫ̇n}{she saw}{\Afootnote{\emph{vęit’k} ‘I know’ \GylfMS. Cf. st. 62.}} standa \hld\ \alst{s}ólu fjarri &
\alst{N}á-strǫndu ȧ, \hld\ \alst{n}orðr horfa dyrr; &
falla \alst{ęi}tr-dropar \hld\ \alst{i}nn umb ljóra, &
sá ’s \alst{u}ndinn salr \hld\ \alst{o}rma hryggjum.\eva

\bvb A hall she saw standing, far from the sun, \\
on Neestrand; north face its doors. \\
Venom-drops fall in through the smoke-vent; \\
that hall is wound with the spines of snakes.\evb\evg


\bvg\bva\mssnote{\Regius~2r/15, \Hauksbok~20v/21, \GylfMS}%
\edtrans{Sá hǫ̇n}{she saw}{\Afootnote{so \Regius; \emph{ser hon} ‘she sees’ \Hauksbok; \emph{skulu} ‘shall [be]’ \GylfMS}} \alst{þ}ar vaða \hld\ \alst{þ}unga strauma &
\alst{m}ęnn \alst{m}ęin-svara \hld\ ok \edtrans{\alst{m}orð-varga}{murder-wargs}{\Bfootnote{Murderous outlaws.}} &
ok þann’s \alst{a}nnars glępr \hld\ \alst{ęy}ra-ru̇nu. &
Þar \edtrans{saug}{sucked}{\Afootnote{so \Hauksbok; \emph{†súg†} \Regius; \emph{kvęlr} ‘torments’ \GylfMS}} \alst{N}íð-hǫggr \hld\ \alst{n}ái fram-gingna; &
slęit \alst{v}argr \alst{v}era. \hld\ \alst{V}ituð ér ęnn eða hvat?\eva

\bvb She saw there wading through heavy streams \\
false-swearing men and murder-wargs, \\
and the one who beguiles another’s ear-whisperer \ken{wife}. \\
There sucked \inx[P]{Nithehewer} from corpses passed-on; \\
the warg tore at men.—Know ye yet, or what?\footnoteB{In this st. is clearly described watery punishment in the Heathen afterlife, also seen in \Reginsmal\ 3–4 and possibly in \Grimnismal\ 21. The crimes are what one might expect from the Germanic worldview: perjury, shameful murder, and adultery with a married woman. In Anglo-Saxon and Nordic laws the committer of such crimes gained the title of \inx[C]{nithing}, that is, one afflicted with \inx[C]{nithe} (severe shame). It is not surprising then that such nithings would be tortured by a creature named Nithehewer ‘Nithe-striker’. The practice of burying in bogs and flood-marks (or generally outside of settlements) is well attested in sources about Germanic culture from Tacitī Germania onwards—I consider it likely that the heavy streams in this stanza and others represent such graves. This is further elaborated on in \textcite{GermanicGems2}.}\evb\evg

\sectionline

\bvg\bva\mssnote{\Regius~2r/17, \Hauksbok~20v/2, \GylfMS}%
\edtrans{\alst{Au}str}{In the east}{\Bfootnote{The cardinal direction associated with ettins and other monsters.}} \edtrans{býr}{dwells}{\Afootnote{so \Hauksbok\GylfMS; \emph{sat} ’sat/stayed’ \Regius}} hin \edtrans{\alst{a}ldna}{old}{\Afootnote{\emph{arma} ‘wretched’ \Upsaliensis}} \hld\ í \edtrans{\alst{Éa}rn-viði}{Ironwood}{\Afootnote{metr. emend.; \emph{Járnviði} \Regius\Hauksbok\RegiusProse\Wormianus\Upsaliensis; \emph{Járn-viðjum} ‘Ironwoods’ \Trajectinus}} &
ok \edtrans{\alst{f}ǿðir}{nourishes}{\Afootnote{so \Hauksbok\GylfMS; \emph{fǿddi} ‘nourished’ \Regius}} þar \hld\ \alst{F}ęnris kindir; &
verðr \edtext{af}{\Afootnote{\emph{ór} \Trajectinus\RegiusProse}} þęim \alst{ǫ}llum \hld\ \alst{ęi}nna nøkkurr &
\alst{t}ungls \edtrans{\alst{t}júgari}{seizer}{\Afootnote{\emph{†tuigan†} \Trajectinus; \emph{tregari} ‘griever’ \Upsaliensis. As the young agentive suffix \emph{-ari} is found nowhere else in the poem it is possible that this word is corrupt. If it is, it must have occurred early in the transmission, as reflexes of \emph{tjúgari} are found in all surviving mss.}} \hld\ í \alst{t}rolls hami.\eva

\bvb In the east dwells the old woman, in \inx[L]{Ironwood}, \\
and nourishes there the kindreds of \inx[P]{Fenrer} \ken{wolves}; \\
from them all comes one most certain: \\
a seizer of the Moon in a troll’s \inx[C]{hame}.\footnoteB{The old hag raises the cubs of the wolf Fenrer, of which a particularly fierce one will swallow the moon. According to \Grimnismal\ 40 the sun is chased by a wolf called Skoll, while another wolf, Hate Rothswitner’s son, runs in front of her. This is elaborated upon in \Gylfaginning\ 12, where it is said that Skoll swallows the moon, while Hate swallows the sun. High then explains that “A lone troll-woman (\emph{gýgr}) lives to the east of Middenyard in that forest called Ironwood”, and “feeds the sons of many ettins, all in the likenesses of wolves, and thereof these wolves (i.e. Skoll and Hate) come. And it is also said that from that lineage a single one becomes the mightiest, and he is called \inx[P]{Moongarm}. He fills himself with the life of all those men who die and he swallows the moon and stains heaven and all the air with blood. Thereof the sun loses its rays and the winds are violent and moan hither and thither, and thus it says in the Spae of the Wallow: [...]” after which this and the following st. are quoted. This seems very much like a composite from several sources—probably \Voluspa\ 40–41 and \Grimnismal\ 40—but becomes contradictory when it states that two wolves swallow the moon.
Assuming that this is only a confusion on the part of the author of \Gylfaginning, this st. and the next must be describing Skoll, but it is of course not impossible that there was confusion about the exact details of these events among the Heathen poets. In favour of that seems to speak \Vafthrudnismal\ 46–47, where the sun is said to be swallowed by Fenrer (but see note there).}\evb\evg


\bvg\bva\mssnote{\Regius~2r/19, \Hauksbok~20v/4, \GylfMS}%
\alst{F}yllisk \alst{f}jǫrvi \hld\ \alst{f}ęigra manna, &
\alst{r}ýðr \alst{r}agna sjǫt \hld\ \alst{r}auðum dręyra, &
\alst{s}vǫrt verða \alst{s}ól-skin \hld\ of \alst{s}umur ęptir, &
\alst{v}eðr ǫll \alst{v}á-lynd. \hld\ \alst{V}ituð ér ęnn eða hvat?\eva

\bvb He fills himself with the lifeblood of \inx[C]{fey} men; \\
he reddens the abode of the \inx[G]{Reins} with red gore. \\
Black turn the sun’s rays in summers thereafter; \\
the winds all woeful.—Know ye yet, or what?\evb\evg


\bvg\bva\mssnote{\Regius~2r/21, \Hauksbok~20v/16}%
\edtrans{\alst{S}at þar ȧ haugi}{There sat on the mound}{\Bfootnote{The motif of ettins sitting on burial mounds is also found in \Thrymskvida\ 6 and \Skirnismal\ P2.  The significance of this is uncertain,.}} \hld\ ok \alst{s}ló hǫrpu &
\alst{g}ýgjar hirðir, \hld\ \alst{g}laðr Ęggþér; &
\alst{g}ól of hǫ̇num \hld\ í \alst{G}agl-viði &
\alst{f}agr-rauðr hani, \hld\ sá’s \alst{F}jalarr hęitir.\eva

\bvb There sat on the mound and struck the harp \\
the gow’s herdsman, glad \inx[P]{Edgethew}.\footnoteB{Edgethew “herds” the flock of monstrous wolves for the old woman in st. 39.} \\
Over him crowed in Galewood\footnoteB{\emph{gagl} ‘wild goose’, maybe here referring to carrion-eating ravens? Galewood is probably the same location as Ironwood.} \\
a fair-red cock, he who is called Feller.\evb\evg


\bvg\bva\mssnote{\Regius~2r/23, \Hauksbok~20v/18}%
\alst{G}ól of ǫ̇sum \hld\ \alst{G}ullin-kambi, &
sá vękr \alst{h}ǫlða \hld\ at \alst{H}ęrja-fǫðrs, &
en \alst{a}nnarr gęlr \hld\ fyr \alst{jǫ}rð neðan &
\alst{s}ót-rauðr hani \hld\ at \alst{s}ǫlum Hęljar.\eva

\bvb Over the Eese crowed Goldencomb; \\
he wakes men at the Father of Hosts’s \name{= Weden’s} [hall]— \\
but another one crows beneath the earth: \\
a soot-red cock at the halls of Hell.\evb\evg

{\small With the crowing of these three cocks (the first in Ettinham, the second in Walhall, the third in Hell) the destruction of the world begins, and immediately afterwards we get the first occurrence of the refrain stanza (ON \emph{stęf}).}

\bvg\bva\mssnote{\Regius~2r/25}%
\alst{G}ęyr \alst{G}armr mjǫk \hld\ fyr \alst{G}nipa-hęlli, &
\alst{f}ęstr mun slitna, \hld\ en \alst{F}reki rinna; &
\alst{f}jǫlð vęit hǫ̇n \alst{f}rǿða, \hld\ \alst{f}ramm sé’k lęngra &
of \alst{r}agna \alst{r}ǫk, \hld\ \alst{r}ǫmm sig-tíva.\eva

\bvb Garm barks much before the Gnip-halls; \\
the rope will tear and the Wolf run. \\
She knows much wisdom; I foresee further \\
about the mighty \inx[L]{Rakes of the Reins}, of the victory-Tews \ken{gods}.\evb\evg


\bvg\bva\mssnote{\Regius~2r/28, \Hauksbok~20v/24, \GylfMS}%
\alst{B}rǿðr munu \alst{b}ęrjask \hld\ ok at \alst{b}ǫnum verðask, &
munu \edtrans{\alst{s}ystrungar}{the children of sisters}{\Afootnote{\emph{†stystrungar†} \Trajectinus}} \hld\ \edtrans{\alst{s}ifjum spilla}{defile the kinship}{\Bfootnote{i.e. ‘commit incest’, probably referring to marriages between first cousins.  Compare related words found in laws, e.g. \emph{frę́nd-semis spell} ‘incest’ and especially \emph{sifja spell} ‘id.’
The idea of incest as a sign of the end times is also found in \Rigveda\ 10.10.10a–b (norm. and tr., Nikhil S. Dwibhashyam. (2023, oct. 28). \emph{Véda quote 6}. https://nikhilsd.com/dvq/6/): \emph{Ā́ ghā tā́ gachān \hld\ úttarā yugā́ni, // yátra jāmáyaḥ \hld\ kr̥ṇávann ájāmi} ‘There shall come indeed those later ages when relatives shall do (acts) not (fit for) relatives.’}}; &
\alst{h}art ’s \edtrans{í \alst{h}ęimi}{in the Home}{\Afootnote{so \Regius\Hauksbok\Upsaliensis; \emph{með hǫlðum} ‘among men’ \RegiusProse\Trajectinus\Wormianus}}, \hld\ \alst{h}ór-dȯmr mikill, &
\alst{sk}ęggj-ǫld, \alst{sk}alm-ǫld, \hld\ \edtrans{\alst{sk}ildir}{shields}{\Afootnote{\emph{’ru} ‘are’ add. \Regius}} \edtrans{klofnir}{split}{\Afootnote{\emph{klofna} ‘become split’ \Upsaliensis}}, &
\edtrans{\alst{v}ind-ǫld}{wind-age}{\Bfootnote{In \Hauksbok\ the \emph{v} is capitalized, marking the beginning of a new stanza.}}, \alst{v}arg-ǫld, \hld\ \edtrans{áðr}{before}{\Afootnote{\emph{unz} (norm.) ‘until’ \Upsaliensis}} \edtrans{\alst{v}er-ǫld}{man-age}{\Bfootnote{Translated as such since it stands next to various other compounds ending in \emph{ǫld} ‘age’.  ON \emph{ver-ǫld} is cognate with English “world”, but in ON that sense is usually expressed with \emph{hęimr} (e.g. l. 3 of the present stanza).}} \edtrans{stęypisk}{tumbles down}{\Bfootnote{\emph{grundir gjalla \hld\ gífr fljúgandi} (norm.) ‘foundations shrill, fiends flying’ add. after this l. \Hauksbok}} &
\edtext{mun \edtext{\alst{ę}ngi}{\Afootnote{\emph{†enn†} \Upsaliensis}} maðr \hld\ \alst{ǫ}ðrum þyrma.}{\lemma{mun \dots\ þyrma ‘before \dots\ spare’}\Bfootnote{om. \RegiusProse\Trajectinus\Wormianus}}\eva

\bvb Brothers will fight and become each other’s slayers; \\
the children of sisters will defile the kinship. \\
’Tis hard in the Home; whoredom is great: \\
axe-age, sword-age—shields are split— \\
wind-age, warg-age! Before the man-age tumbles down, \\
no man will another spare.\evb\evg


\bvg\bva\mssnote{\Regius~2r/32, \Hauksbok~20v/27, \GylfMS}%
\edtext{Lęika \alst{M}íms synir, \hld\ en \alst{m}jǫtuðr kyndisk &
at hinu \alst{g}alla \hld\ \alst{G}jallar-horni; &
\alst{h}ǫ́tt blę́ss \alst{H}ęim-dallr, \hld\ \alst{h}orn ’s ȧ lopti; &
\edtrans{\alst{m}ę́lir}{speaks}{\Afootnote{\emph{†mey†} \RegiusProse; \emph{†nie†} \Trajectinus}} Óðinn \hld\ við \alst{M}íms hǫfuð.}{\lemma{Lęika \dots\ hǫfuð.}\Bfootnote{In \GylfMS\ ll. 1–2 (\emph{Lęika \dots\ Gjallarhorni;} ‘Play \dots\ Horn of Yell.’) are missing, and ll. 3–4 (\emph{hǫ́tt \dots\ hǫfuð.} ‘High \dots\ head [of Mime.]’) are instead paired with the first two lines of the next st. (\emph{Skęlfr \dots\ losnar;})}}\eva

\bvb Mime’s sons play and the Metted is kindled \\
at [the sound of] the shrill Horn of Yell. \\
High blows Homedal; the horn is aloft; \\
Weden speaks with the head of Mime.\evb\evg


\bvg\bva\mssnote{\Regius~2v/3, \Hauksbok~20v/28, \GylfMS}%
\edtext{Skęlfr \alst{Y}ggdrasils \hld\ \alst{a}skr standandi, &
\alst{y}mr it \alst{a}ldna tré, \hld\ en \alst{jǫ}tunn losnar;}{\lemma{Skęlfr \dots\ losnar ‘Ugdrassle’s \dots\ loosens’}\Bfootnote{so \Hauksbok\GylfMS; in \Regius\ the two lines are reversed.}} &
\edtext{\alst{h}rę́ðask allir \hld\ ȧ \alst{h}ęl-vegum &
áðr \alst{S}urtar þann \hld\ \alst{s}efi of glęypir.}{\lemma{hrę́ðask \dots\ glęypir ‘All \dots\ devour it.’}\Bfootnote{Only in \Hauksbok.}} \eva

\bvb Ugdrassle’s Ash trembles, standing: \\
the old tree creaks and the ettin loosens. \\
All are frightened on the Hell-ways, \\
before Surt’s kinsman does devour it.\evb\evg


\bvg\bva\mssnote{\Regius~2v/8, \Hauksbok~20v/30, \GylfMS}%
Hvat ’s með \alst{ǫ̇}sum? \hld\ hvat ’s með \edtrans{\alst{ǫ}lfum}{Elves}{\Afootnote{\emph{ǫ́synjum} ‘Ossens’ \Upsaliensis}}? &
\edtext{gnýr \alst{a}llr \alst{Jǫ}tun-hęimr, \hld\ \alst{ę̇}sir ’ru ȧ þingi,}{\lemma{gnýr \dots\ þingi}\Afootnote{om. \Upsaliensis}} &
\alst{st}ynja dvergar \hld\ fyr \edtext{\alst{st}ęin-durum}{\Afootnote{\emph{stęins} \Upsaliensis; \emph{stęin-dyrum} \Hauksbok\Wormianus\Upsaliensis}} &
\edtext{\edtext{\alst{v}ęgg-bergs}{\Afootnote{\emph{veg-bergs} \Hauksbok\Trajectinus\Wormianus}} \alst{v}ísir}{\Afootnote{om. \Upsaliensis}}. \hld\ \alst{V}ituð ér ęnn eða hvat?\eva

\bvb What is with the Eese? What is with the Elves? \\
All Ettinham roars; the Eese are at the Thing. \\
Dwarfs groan before gates of stone, \\
the hillside’s princes.—Know ye yet, or what?\evb\evg


\bvg\bva\mssnote{\Regius~2v/4, \Hauksbok~20v/32}%
\alst{G}ęyr nú \alst{G}armr mjǫk \hld\ fyr \alst{G}nipa-hęlli, &
\alst{f}ęstr mun slitna, \hld\ en \alst{f}reki rinna; &
\alst{f}jǫlð vęit hǫ̇n \alst{f}rǿða, \hld\ \alst{f}ramm sé’k lęngra &
of \alst{r}agna \alst{r}ǫk \hld\ \alst{r}ǫmm sig-tíva.\eva

\bvb Now Garm barks much before the Gnip-halls; \\
the rope will tear and the Wolf run. \\
She knows much wisdom; I foresee further \\
about the mighty Rakes of the Reins, of the victory-Tews \ken{gods}.\evb\evg


\bvg\bva\mssnote{\Regius~2v/4, \Hauksbok~20v/32, \RegiusProse\Trajectinus\Wormianus}%
\alst{H}rymr ękr austan, \hld\ \alst{h}ęfsk lind fyrir, &
snýsk \alst{Jǫ}rmun-gandr \hld\ í \alst{jǫ}tun-móði, &
\alst{o}rmr knýr \alst{u}nnir, \hld\ \edtrans{en \alst{a}ri hlakkar}{and the eagle screams}{\Afootnote{\emph{ǫrn mun hlakka} ‘the eagle will scream’ \RegiusProse\Trajectinus}}, &
slítr \alst{n}ái \alst{n}ef-fǫlr; \hld\ \alst{N}agl-far losnar.\eva

\bvb Rim drives from the east, holding his shield before him; \\
Ermingand writhes about in ettin-wrath. \\
The Wyrm propels the waves and the eagle screams: \\
the pale-beak tears at corpses; Nailfare loosens.\evb\evg


\bvg\bva\mssnote{\Regius~2v/6, \Hauksbok~20v/34, \RegiusProse\Trajectinus\Wormianus}%
\alst{K}jóll fęrr austan \hld\ \alst{k}oma munu Múspells &
of \alst{l}ǫg \alst{l}ýðir, \hld\ en \alst{L}oki stýrir; &
\alst{f}ara \alst{f}ífl-męgir \hld\ með \alst{f}reka allir, &
þęim es \alst{b}róðir \hld\ \alst{B}ýlęists í fǫr.\eva

\bvb The ship fares from the east—come will Muspell’s \\
subjects o’er the sea—and Lock steers it. \\
The devil-lads journey all with the Wolf; \\
with them comes the brother of Bylest \ken*{= Lock} along.\evb\evg


\bvg\bva\mssnote{\Regius~2v/10, \Hauksbok~20v/36, \GylfMS}%
\edtext{\alst{S}urtr}{\Afootnote{\emph{Svartr} \Upsaliensis}} fęrr \alst{s}unnan \hld\ með \alst{s}viga lę́vi, &
skínn af \alst{s}verði \hld\ \edtrans{\alst{s}ól val-tíva}{sun of the slain-Tew}{\Bfootnote{\emph{val-tíva} is here taken as gen. sg. of \emph{val-tívar} ‘slain-Tews’, for which cf. st. 60 below, but the sense of this is obscure.  Perhaps it means that Surt’s sword shines as bright as the heavenly Gods?  The word may also (so \CV) be read as gen. sg. of unattested \emph{*val-tívi} ‘tew of the slain’, referring to Surt, but this is tautological: “Surt comes from the south with fire; from his sword shines the sun of Surt”.}}; &
\alst{g}rjót-bjǫrg \alst{g}nata, \hld\ en \edtrans{\alst{g}ífr rata}{fiends reel}{\Afootnote{\emph{guðar hrata} ‘[but] the gods stagger’ \Upsaliensis}\Bfootnote{The reading of \Upsaliensis is wo. doubt corrupt; the anachronistic masc. pl. ending \emph{-ar} is proof enough, for \emph{goð} \char`~\ \emph{guð} ‘gods’ was always neuter in heathen times.}}, &
troða \alst{h}alir \edtrans{\alst{h}ęl-veg}{Hellway}{\Bfootnote{The road on which one has to travel after death to reach his final resting place.  Cf. \Helreid.}}, \hld\ en \alst{h}iminn klofnar.\eva

\bvb Surt comes from the south with the twig’s betrayer \ken{fire}; \\
from the sword shines the sun of the slain-Tews. \\
Boulders clash and the fiends reel; \\
men tread the \inx[L]{Hellway} and heaven is split.\evb\evg

{\small For the following two sts. cf. the account of \Vafthrudnismal\ 53.}

\bvg\bva\mssnote{\Regius~2v/13, \Hauksbok~20v/37, \RegiusProse\Trajectinus\Wormianus}%
Þȧ kømr \edtrans{\alst{H}línar \hld\ \alst{h}armr annarr}{Line’s second sorrow}{\Bfootnote{The first sorrow being the death of Balder.  Line is described in \Gylfaginning\ 35 as a minor goddess \emph{sett til gę́zlu yfir þeim mǫnnum, er Frigg vill forða við háska nǫkkurum} ‘placed to watch over those men which Frie wishes to protect against any particular danger’. In spite of this almost all translators and editors have understood Line as synonymous with Frie, or even asked whether her existence as a distinct goddess is not something invented by the author of \Gylfaginning.  \textcite{Hopkins2017} argues that this need not be the case; as a maidservant of Frie, Line’s two sorrows would consist in her failure to protect both the son and husband of her mistress.}} framm, &
es \alst{Ó}ðinn fęrr \hld\ við \alst{u}lf vega, &
—en \alst{b}ani \alst{B}ęlja \hld\ \alst{b}jartr at Surti— &
þȧ mun \alst{F}riggjar \hld\ \alst{f}alla \edtext{angan}{\Afootnote{\emph{angantyr} \Regius}}.\eva

\bvb Then comes \inx[P]{Line}’s second sorrow to pass, \\
when Weden goes to fight the Wolf \\
—and \inx[P]{Bellow}’s bane \ken*{= Free}, bright, [goes] against Surt— \\
then will Frie’s beloved \ken*{= Weden} fall.\evb\evg


\bvg\bva\mssnote{\Regius~2v/15, \RegiusProse\Trajectinus\Wormianus}%
\edtext{Þȧ kømr hinn \alst{m}ikli \hld\ \alst{m}ǫgr Sig-fǫður}{\lemma{Þȧ kømr \dots\ Sig-fǫður ‘Then comes \dots\ Syefather’}\Afootnote{\emph{Gęngr Óðins sonr \hld\ við ulf vega} ‘Goes Weden’s son against the wolf to fight’ \GylfMS}}, &
\alst{V}íðarr \edtext{\alst{v}ega}{\Afootnote{\emph{of veg} \GylfMS}} \hld\ at \alst{v}al-dýri; &
lę́tr \alst{m}ęgi \edtrans{Hveðrungs}{Whethring}{\Bfootnote{An obscure name for \inx[P]{Lock}, whose son is the Wolf.}} \hld\ \alst{m}und of standa &
\alst{h}jǫr til \alst{h}jarta; \hld\ þȧ ’s \alst{h}efnt fǫður.\eva

\bvb Then comes the great lad of \inx[P]{Syefather} \name{= Weden}, \\
Wider, to fight that slaughter-beast. \\
He lets his hand through \inx[P]{Whethring}’s lad \ken*{= the Wolf} \\
drive the sword to the heart—then the father \ken*{= Weden} is avenged!\evb\evg


\bvg\bva[H2]\mssnote{\Hauksbok~20v/39}%
\edtext{Gïnn \alst{l}opt yfir \hld\ \alst{l}indi jarðar, &
gapa \alst{ý}gs kjaptar \hld\ \alst{o}rms í hę́ðum; &
mun \alst{Ó}ðins son \hld\ \edtrans{\alst{ęi}tri}{venom}{\Afootnote{emend.; \emph{ormi} ‘Wyrm’ \Hauksbok.}\Bfootnote{Cf. \Gylfaginning\ 51: “Thunder bears the bane-word from the Middenyardswyrm and strides nine paces away from it. Then he falls dead to the earth for the venom (\emph{ęitri}) which the Wyrm blows on him.”}} mǿta &
\alst{v}args at \edtext{dauða}{\Afootnote{da... \Hauksbok}} \hld\ \alst{V}íðars niðja.}{\lemma{Gïnn \dots\ niðja.}\Bfootnote{The final part of this stanza is almost completely illegible.  I have relied on the reading of \textcite[13,44\psqq]{JonHelgason1971}.}}\eva

\bvb Over the air yawns the Girdle of the Earth \ken*{= Middenyardswyrm}; \\
the jaws of the fierce Wyrm gape in the heights. \\
Weden’s son \ken*{= Thunder} will meet the venom \\
of the Warg, after the deaths of Wider’s kinsmen \ken*{= the Eese}.\evb\evg


\bvg\bva\mssnote{\Regius~2v/17, \Hauksbok~20v/41, \RegiusProse\Trajectinus\Wormianus}%
\edtrans{Þȧ kømr}{Then comes}{\Afootnote{\emph{Gęngr} ‘Goes’ \GylfMS}} hinn \alst{m}ę́ri \hld\ \alst{m}ǫgr Hlǫðynjar &
\edtext{gęngr \alst{Ó}ðins sonr \hld\ við \alst{o}rm vega.}{\lemma{gęngr \dots\ vega. ‘Weden’s \dots\ to meet.’}\Afootnote{Only in \Regius.}} &
\edtext{Drepr af \alst{m}óði \hld\ \edtrans{\alst{M}ið-garðs véurr}{Middenyard’s Wigh-ward}{\Bfootnote{“The guardian of the sanctuaries of Middenyard”; a fitting kenning.}}; &
\edtrans{munu \alst{h}alir allir \hld\ \alst{h}ęim-stǫð ryðja;}{all men will clear their homesteads}{\Bfootnote{After Thunder, the protector of men, is slain the earth is no longer inhabitable.  Cf. \Thrymskvida\ 18.}} &
gęngr \alst{f}et níu \hld\ \alst{F}jǫrgynjar burr &
\alst{n}ęppr frȧ \alst{n}aðri, \hld\ \alst{n}íðs ȯ-kvíðnum.}{\lemma{Drepr \dots\ ȯ-kviðnum ‘Middenyard’s \dots\ adder’}\Afootnote{\emph{neppr af naðri \hld\ niðs ȯkvíðnum // munu halir allir \hld\ hęim-stǫð ryðja, // es af móði drepr \hld\ Mið-garðs véurr} ‘pained, away from the loathsome adder. All men will clear their homesteads when out of wrath Middenyard’s Wigh-ward strikes.’ \GylfMS}\Bfootnote{The line-order found in \Regius\ and \Hauksbok\ is rather clumsy, but has been kept due to the rule of the majority.}}\eva

\bvb Then comes the renowned lad of Lathyn \name{= Earth} \ken*{= Thunder}: \\
Weden’s son goes the Wyrm to meet. \\
Middenyard’s Wigh-ward strikes out of wrath; \\
all men will clear their homesteads. \\
The son of Firgyn goes nine paces, \\
pained, away from the loathsome adder \ken*{= Middenyardswyrm}.\evb\evg


\bvg\bva\mssnote{\Regius~2v/20, \Hauksbok~21r/1, \GylfMS}%
\alst{S}ól tér \alst{s}ortna, \hld\ \edtext{\edtrans{\alst{s}økkr}{sinks}{\Afootnote{so \RegiusProse\Trajectinus\Wormianus; \emph{sígr} ‘descends’ \Regius\Hauksbok\Upsaliensis}} fold í mar}{\lemma{søkkr \dots\ mar ‘sinks \dots\ the sea’}\Bfootnote{The reading \emph{søkkr} ‘sinks’ is supported by Arn \emph{Þorfdr} 24 (\Skp\ II), which is probably based on the present line: \emph{Bjǫrt verðr sól at svartri; \hld\ søkkr fold í mar døkkvan;} ‘The bright sun turns to black; the fold sinks into the dark sea’.}}, &
\alst{h}verfa af \alst{h}imni \hld\ \alst{h}ęiðar stjǫrnur; &
gęisar \alst{ęi}mi \hld\ við \alst{a}ldr-nara; &
lęikr \alst{h}ǫ́r \alst{h}iti \hld\ við \alst{h}imin sjalfan.\eva

\bvb Sun starts to blacken; the fold \ken{earth} sinks into the sea; \\
from heaven fade the shining stars. \\
Smoke rages from the life-nourisher \ken{fire}; \\
the high heat licks the very heaven.\evb\evg


\bvg\bva\mssnote{\Regius~2v/22, \Hauksbok~21r/2}%
\alst{G}ęyr nú \alst{G}armr mjǫk \hld\ fyr \alst{G}nipa-hęlli, &
\alst{f}ęstr mun slitna, \hld\ en \alst{f}reki rinna; &
\alst{f}jǫlð vęit hǫ̇n \alst{f}rǿða, \hld\ \alst{f}ramm sé’k lęngra &
of \alst{r}agna \alst{r}ǫk, \hld\ \alst{r}ǫmm sig-tíva.\eva

\bvb Now Garm barks much before the Gnip-halls; \\
the rope will tear and the Wolf run. \\
She knows much wisdom; I foresee further \\
about the mighty Rakes of the Reins, of the Victory-Tews \ken{gods}.\evb\evg

\sectionline

{\small With the last repetition of the refrain stanza the destruction reaches its apex.  Sts. 57–60 are paraphrased in \Gylfaginning\ ch. 53:

\begin{quote}
	\emph{Þá mę́lti Gangleri: „Hvárt lifa nǫkkur goðin þá, eða er þá nǫkkur jǫrð eða himinn?“ Hárr segir: „Upp skýtr jǫrðunni þá ór sę́num, ok er þá grǿn ok fǫgr. Vaxa þá akrar ó·sánir. Víðarr ok Váli lifa, svá at eigi hefir sę́rinn ok Surta-logi grandat þeim, ok byggja þeir á Iða-velli, þar sem fyrr var Ás-garðr, ok þar koma þá synir Þórs, Móði ok Magni, ok hafa þar Mjǫllni. Því nę́st koma þar Baldr ok Hǫðr frá Heljar, setjast þá allir samt, ok talast við, ok minnast á rúnar sínar, ok rǿða of tíðendi þau, er fyrrum hǫfðu verit, of Mið-garðs-orm ok um Fenris-úlf. Þá finna þeir í grasinu gull-tǫflur þę́r, er ę́sirnir hǫfðu átt. Svá er sagt:“}
\end{quote}

\begin{quote}
	‘Then spoke Gangler: “Do any of the gods survive then, or is there then any earth or heaven?” High says: “Then the earth shoots up from the seas, and it is then green and fair. Then grow acres unsown. Wider and Wonnel live, for the sea and Surt’s flame have not harmed them, and they settle on the Idewolds where there earlier was Osyard; and then the sons of Thunder, Mood and Main, come there, and there they have Millner.  Next come Balder and Hath from Hell; then they all make peace with each other and discuss and think back on their runes, and speak about the tidings which had been in antiquity, about the Middenyardswyrm and about the Fenrerswolf.  Then they find in the grass those golden game-bricks which the Eese had owned. So it is said:”’
\end{quote}

after which is quoted \Vafthrudnismal\ 51.}

\sectionline

\bvg\bva\mssnote{\Regius~2v/23, \Hauksbok~21r/4}%
Sér hǫ̇n \alst{u}pp koma \hld\ \edtrans{\alst{ǫ}ðru sinni}{a second time}{\Bfootnote{The first time probably being the lifting of the Earth in st. 4.}} &
\alst{jǫ}rð ór \alst{ę́}gi \hld\ \alst{i}ðja-grø̇na; &
\alst{f}alla \alst{f}orsar, \hld\ \alst{f}lýgr ǫrn yfir, &
sá’s ȧ \alst{f}jalli \hld\ \alst{f}iska vęiðir.\eva

\bvb She sees coming up a second time \\
Earth from the ocean, ever green anew. \\
Torrents fall, flies the eagle above, \\
which on the fells catches fish.\evb\evg


\bvg\bva\mssnote{\Regius~2v/24, \Hauksbok~21r/5}%
\edtrans{Finnask}{find each other}{\Afootnote{\emph{hittask} \Hauksbok\ provides closer parallelism with st. 7, but for the same reason it may also have replaced earlier \emph{finnask}.}} \alst{ę̇}sir \hld\ ȧ \alst{I}ða-vęlli &
ok umb \alst{m}old-þinur \hld\ \alst{m}ǫ́tkan dø̇ma, &
\edtrans{ok \alst{m}innask þar \hld\ ȧ \alst{m}ęgin-dȯma}{and there think back on mighty verdicts}{\Afootnote{om. \Regius}} &
ok ȧ \alst{F}imbul-týs \hld\ \alst{f}ornar ru̇nar.\eva

\bvb The Eese find each other on the Idewolds, \\
and of the mighty Earth-strip \ken*{= the Middenyardswyrm} judge, \\
and there think back on mighty verdicts, \\
and on Fimble-Tew’s \name{= Weden’s} ancient runes.\evb\evg


\bvg\bva\mssnote{\Regius~2v/26, \Hauksbok~21r/7}%
Þar munu \alst{ę}ptir \hld\ \edtext{\alst{u}ndr-samligar &
\alst{g}ullnar tǫflur}{\lemma{undr-samligar gullnar tǫflur ‘wondersome golden game-bricks’}\Bfootnote{A fine literary device.  In st. 8 the golden age of the Eese, exemplified by their playing board games, was spoiled by the three ettin-women.  The rediscovering of the golden board game then betokens a new golden age.}} \hld\ í \alst{g}rasi finnask, &
þę́r’s í \alst{á}r-daga \hld\ \alst{á}ttar hǫfðu.\eva

\bvb There will afterwards wondersome \\
golden game-bricks in the grass be found, \\
those which in days of yore they had owned.\evb\evg


\bvg\bva\mssnote{\Regius~2v/28, \Hauksbok~21r/9}%
Munu \alst{ȯ}-sánir \hld\ \alst{a}krar vaxa, &
\alst{b}ǫls mun alls \alst{b}atna, \hld\ mun \alst{B}aldr koma; &
búa \alst{H}ǫðr ok Baldr \hld\ \alst{H}ropts sig-toptir, &
\alst{v}ęl \alst{v}al-tívar. \hld\ \alst{V}ituð ér ęnn eða hvat?\eva

\bvb Unsown will acres grow; \\
the bale will all be bettered; Balder will come. \\
Hath and Balder bedwell Roft’s \name{= Weden’s} victory-plots \\
well, the slain-Tews.—Know ye yet, or what?\footnoteB{The evil of Hath’s slaying Balder will be forgotten as the two live together in peace.}\evb\evg


\bvg\bva\mssnote{\Regius~2v/30, \Hauksbok~21r/11}%
Þȧ kná \alst{H}ø̇nir \hld\ \edtrans{\alst{h}laut-við kjósa}{choose the leat-wood}{\Bfootnote{Foresee the future by the means of twigs drenched in the blood of slaughtered beasts.  See \Hymiskvida\ 1 and the encyclopedia entry for “leat”.}} &
ok \alst{b}urir \alst{b}yggva \hld\ \edtrans{\alst{b}rǿðra tvęggja}{the two brothers}{\Bfootnote{The present translation understands \emph{tvęggja} as the gen. pl. of \emph{tvęir} ‘two’; the two brothers are presumably Hath and Balder, mentioned in the previous stanza.
Since the original ms. does not capitalize proper nouns one could also read \emph{brǿðra Tvęggja} ‘the brothers of Tway \name{= Weden}’.  Weden’s brothers are attested in \Gylfaginning\ 6 as \inx[P]{Will} and \inx[P]{Wigh}; they are never said to have children.}} &
\alst{v}ind-hęim \alst{v}íðan. \hld\ \alst{V}ituð ér ęnn eða hvat?\eva

\bvb Then does Heener choose the \inx[C]{leat}-wood, \\
and the sons of the two brothers settle \\
the wide wind-home \ken{sky/heaven}.—Know ye yet, or what?\evb\evg


\bvg\bva\mssnote{\Regius~2v/31, \Hauksbok~21r/12, \GylfMS}%
\alst{S}al \edtrans{sér hǫ̇n}{she sees}{\Afootnote{\emph{vęit’k} ‘I know’ \GylfMS}} standa \hld\ \alst{s}ólu fęgra, &
\edtrans{\alst{g}ulli þakðan}{thatched with gold}{\Afootnote{\emph{gulli bętra} ‘better than gold’ \RegiusProse\Trajectinus}}, \hld\ ȧ \edtext{\alst{G}imléi}{\Afootnote{metr. emend.; \emph{Gimlé} \Regius\Hauksbok\GylfMS}}; &
\edtrans{þar}{there}{\Afootnote{\emph{þann} ‘[in] that [hall]’ \Trajectinus\Wormianus}} skulu \alst{d}yggvar \hld\ \alst{d}róttir byggva &
ok umb \alst{a}ldr-daga \hld\ \alst{y}nðis njóta.\eva

\bvb A hall she sees standing, fairer than the sun, \\
thatched with gold, on Gemlee; \\
there shall faithful folk settle, \\
and in their days of life enjoy delight.\evb\evg


\bvg\bva[H3]\mssnote{\Hauksbok~21r/14}%
\edtext{Þȧ kømr hinn \alst{r}íki \hld\ at \alst{r}ęgin-dȯmi &
\alst{ǫ}flugr \alst{o}fan \hld\ sá’s \alst{ǫ}llu rę́ðr.}{\lemma{Þȧ \dots\ rę́ðr.}\Bfootnote{This stanza is found only in \Hauksbok\ and is likely to be a late Christian insert.}}\eva

\bvb Then comes the mighty one to the great judgement, \\
strong from above, he who rules everything.\evb\evg


\bvg\bva\mssnote{\Regius~3r/2, \Hauksbok~21r/15}%
Þar kømr hinn \alst{d}immi \hld\ \alst{d}ręki fljúgandi, &
\alst{n}aðr frȧnn \alst{n}eðan \hld\ frȧ \alst{N}iða-fjǫllum; &
berr sér í \alst{f}jǫðrum \hld\ —\alst{f}lýgr vǫll yfir— &
\alst{N}íð-hǫggr \alst{n}ái; \hld\ \edtrans{\alst{n}ú mun hǫ̇n søkkvask}{Now she will sink!}{\Bfootnote{The wallow, referring to herself in third person, descends back down into her grave, whence Weden woke her.  Cf. the very last half-line of \Helreid: \emph{søkkst-u, gýgjar-kyn} ‘sink, thou gow’s kin!’}}.\eva

\bvb Then comes the gloomy dragon flying, \\
the gleaming adder down below from the \inx[L]{Nithfells}. \\
He carries in his feathers—he flies over the field— \\
Nithehewer, corpses.—Now she will sink!”\evb\evg

\sectionline
