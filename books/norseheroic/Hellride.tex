\bookStart{Hell-ride of Byrnhild}[Hęl-ręið Bryn·hildar]
\setBookCode{Helreid}

\begin{flushright}%
\textbf{Dating} \parencite{Sapp2022}: late C11th (0.650)

\textbf{Meter:} \Fornyrdislag
\end{flushright}%

\section{Introduction}

Byrnhild is burned on her pyre in a beautiful chariot or wagon.  In the afterlife she rides on the \inx[L]{Hellway} to reach her resting place in \inx[L]{Hell} when her road is blocked by a certain \inx[C]{gow} or troll-woman.  The poem consists of their conversation.

\sectionline

\bpg\bpa Eptir dauða Bryn·hildar vóru gǫr bǫ́l tvau: annat Sig·urði, ok brann þat fyrr, en Bryn·hildr var á ǫðru brennd ok var hon \edtrans{í reið þeiri er guð-vefjum var tjǫlduð}{in that chariot which was covered with god-weave}{\Bfootnote{The canopy/tent on the chariot was made of silk (poetically known as \emph{god-weave}, the fabric of the gods).  For the burial of women in wagons and chariots cf. TODO (Oseberg ship?).}}.  Svá er sagt at \edtrans{Bryn·hildr ók með reið’inni á hel-veg}{Byrnhild drove with the chariot on the Hellway}{\Bfootnote{This gives us some interesting insight into old afterlife beliefs.  After Byrnhild is burned she ends up between the worlds of the dead and the living, the so-called “Hell-way” or road to Hell, which she must travel to arrive at her final resting place in the Underworld; she is burned inside a chariot so that she will be able to travel comfortably.  We may presume that the animals driving the chariot were slaughtered and burnt with her on the pyre.}} ok fór um tún þar er gýgr nǫkkur bjó.  Gýgr’in kvað:\epa

\bpb After Byrnhild’s death two pyres were made: one for Siward, and it burned earlier; but Byrnhild was burned on the other, and she was in that chariot which was covered with \inx[C]{god-weave}.  So it is said that Byrnhild drove with the chariot onto the Hellway and passed through a yard where a certain \inx[C]{gow} lived.  The gow quoth:\epb\epg

\section{Byrnhild rode the Hellway (\emph{Bryn·hildr ręið hęl-veg})}

\bvg\bva%
„Skalt ï \alst{g}ǫgnum \hld\ \alst{g}anga ęigi &
\alst{g}rjóti studda \hld\ \alst{g}arða mïna; &
\alst{b}ętr sǿmði þér \hld\ \alst{b}orða at rękja &
\edtrans{hęldr an \alst{v}itja \hld\ \alst{v}ers annarar}{rather than visit the husband of another}{\Bfootnote{The gow insults Guthrun and blames her early death on her immoral and masculine actions; if she had lived like a chaste woman and tended to simple domestic duties she would still have been alive.}}.\eva

\bvb “Thou shalt nowise pass through these \\
rock-supported yards of mine! \\
It befit thee better to weave tapestries \\
rather than visit the husband of another.\evb\evg


\bvg\bva%
Hvat skalt \alst{v}itja \hld\ af \alst{V}al-landi, &
\alst{h}var-fu̇st \alst{h}ǫfuð, \hld\ \alst{h}úsa mïnna? &
Þú hęfir, \alst{V}ǫ́r gulls, \hld\ ef þik \alst{v}ita lystir, &
\alst{m}ild, af hǫndum \hld\ \alst{m}anns blóð þvegit.“\eva

\bvb Why shalt thou visit, come from Walland, \\
O straying head, these houses of mine? \\
Thou hast, mild \inx[P]{Ware} of gold \ken{lady}, if thou hast lust to know, \\
washed man-blood from thy hands.”\evb\evg


\bvg\bva%
\speakernote{[Bryn·hildr svaraði:]}%
„\alst{B}regð ęigi mér, \hld\ \alst{b}rúðr ór stęini, &
þó’tt ek \alst{v}ę́ra’k \hld\ ï \alst{v}íkingu; &
\alst{e}k mun \alst{o}kkur \hld\ \alst{ǿ}ðri þikkja &
hvar’s męnn \alst{ę}ðli \hld\ \alst{o}kkart kunna.“\eva

\bvb “Upbraid me not, O bride from the stone \ken{gow}, \\
although I may have been in the wiking host; \\
of us two will I seem the nobler \\
wherever men know our pedigrees.”\evb\evg


\bvg\bva%
\speakernote{[Gýgr kvað:]}%
„Þú vast, \alst{B}ryn-hildr, \hld\ \alst{B}uðla dóttir, &
\alst{h}ęilli verstu \hld\ ï \alst{h}ęim borin; &
þú hęfir \alst{G}júka \hld\ of \alst{g}latat bǫrnum &
ok \alst{b}úi þęira \hld\ \alst{b}rugðit góðu.“\eva

\bvb “Thou wast, O Byrnhild, Budle’s daughter, \\
under the worst omen born into the Home. \\
Thou hast destroyed Yivick’s children \\
and deprived their house of good.”\evb\evg


\bvg\bva%
\speakernote{[Bryn·hildr kvað:]}%
„Ek mun \alst{s}ęgja þér \hld\ \alst{s}vinn ór ręiðu, &
\alst{v}it-laussi mjǫk, \hld\ ef þik \alst{v}ita lystir: &
hvé \alst{g}ørðu mik \hld\ \alst{G}júka arfar &
\alst{ȧ}sta-lausa \hld\ ok \alst{ęi}ð-rofa.\eva

\bvb “I will tell thee, wise from my chariot, \\
O very witless one, if thou hast lust to know, \\
how the heirs of Yivick made me \\
loveless and an oath-breaker.\evb\evg


\bvg\bva%
Lét \alst{h}ami vȧra \hld\ \alst{h}ug-fullr konungr, &
\alst{á}tta systra, \hld\ undir \alst{ęi}k borit; &
\alst{v}as’k \alst{v}etra tólf, \hld\ ef þik \alst{v}ita lystir, &
es \alst{u}ngum gram \hld\ \alst{ęi}ða sęlda’k.\eva

\bvb A king full of heart let the shapes of us \\
eight sisters be born beneath an oak. \\
I was twelve winters old, if thou hast lust to know, \\
when to the young ruler oaths I swore.\evb\evg


\bvg\bva%
\alst{H}étu mik allir \hld\ ï \alst{H}lym-dǫlum &
\alst{H}ildi und \alst{h}jalmi, \hld\ \alst{h}vęrr es kunni.\eva

\bvb In the Limdales all did call me \\
a Hild ’neath the helmet, whoever knew me.\evb\evg


\bvg\bva%
\Ballnote{As told in \textlink{Sigrdrifumal}[P3].}%
Þȧ lét’k \alst{g}amlan \hld\ ȧ \alst{G}oð-þjóðu &
\alst{H}jalm-Gunnar nę́st \hld\ \alst{h}ęljar ganga; &
gaf’k \alst{u}ngum sigr \hld\ \alst{Au}ðu bróður; &
þar varð mér \alst{Ó}ðinn \hld\ \alst{o}f-ręiðr um þat.\eva

\bvb Then I next among the Gots \\
let old Helm-Guther go the way of Hell. \\
I gave victory to Ead’s young brother; \\
then was Weden furious with me over that.\evb\evg


\bvg\bva%
Lauk hann mik \alst{sk}jǫldum \hld\ ï \alst{Sk}ata-lundi, &
\alst{r}auðum ok hvítum, \hld\ \alst{r}andir snurtu; &
þann bað hann \alst{s}líta \hld\ \alst{s}vefni mínum &
es \alst{h}vęr-gi lands \hld\ \alst{h}rę́ðask kynni.\eva

\bvb He locked me inside shields in Shatelund, \\
red ones and white; their rims clasped. \\
He bade that one end my sleep \\
who in no land could be frightened.\evb\evg


\bvg\bva%
Lét umb \alst{s}al minn \hld\ \alst{s}unnan-verðan &
\alst{h}ávan brinna \hld\ \alst{h}ęr alls viðar; &
þar bað hann \alst{ęi}nn þegn \hld\ \alst{y}fir at ríða, &
þann’s mér \alst{f}ǿrði gull \hld\ þat’s und \alst{F}áfni lá.\eva

\bvb He made around my hall a south-facing \\
high host of all wood \ken{fire} burn. \\
He bade only one thane there ride over it: \\
him who brought me the gold which neath Fathomer lay.\evb\evg


\bvg\bva%
Ręið \alst{g}óðr \alst{G}rana \hld\ \alst{g}ull-miðlandi &
þar’s \alst{f}óstri mïnn \hld\ \alst{f}lętjum stýrði; &
\alst{ęi}nn þȯtti hann þar \hld\ \alst{ǫ}llum bętri, &
\alst{v}íkingr Dana, \hld\ ï \alst{v}erðungu.\eva

\bvb On Grane rode the good gold-dealer \ken*{\textsc{ruler} = Siward} \\
where my foster-son ruled the benches. \\
Alone he seemed there better than all, \\
the wiking of Danes, in the warband.\evb\evg


\bvg\bva%
\alst{S}vǫ́fu vit ok unðum \hld\ ï \alst{s}ę́ing ęinni &
sem hann mïnn \alst{b}róðir \hld\ of \alst{b}orinn vę́ri; &
\alst{h}várt-ki knátti \hld\ \alst{h}ǫnd yfir annat &
\alst{á}tta nǫ́ttum \hld\ \alst{o}kkart lęggja.\eva

\bvb We slept and were content in a single bed \\
as if he had been born my brother; \\
neither laid a hand o’er the other \\
for eight nights, of us two.\evb\evg


\bvg\bva%
Því brá mér \alst{G}uð·ru̇n, \hld\ \alst{G}júka dóttir, &
at ek \alst{S}ig·urði \hld\ \alst{s}vę́fa’k ȧ armi; &
þar varð’k þęss \alst{v}ís \hld\ es \alst{v}ildi’g-a’k &
at þau \alst{v}éltu mik \hld\ ï \alst{v}er-fangi.\eva

\bvb Thus upbraided me Guthrun Yivick’s daughter, \\
because I slept on Siward’s arm. \\
There I became wise of what I did not want [to know]: \\
that those two had tricked me in the catch of man.\evb\evg


\bvg\bva%
Munu við \alst{o}f·stríð \hld\ \alst{a}lls til lęngi &
\alst{k}onur ok \alst{k}arlar \hld\ \alst{k}vikkvir fǿðask; &
vit skulum \alst{o}kkrum \hld\ \alst{a}ldri slíta, &
\alst{S}ig·urðr, \alst{s}aman. \hld\ \alst{S}økks-tu, gýgjar-kyn!“\eva

\bvb In great strife for far too long \\
will men and women alive be born. \\
We two shall end our age, \\
I and Siward, together.—Sink, thou gow-kind!”\evb\evg

\sectionline
