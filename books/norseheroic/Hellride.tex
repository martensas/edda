\bookStart{Hell-ride of Byrnhild}[Hęl-ręið Bryn·hildar]
\setBookCode{Helreid}

\begin{flushright}%
\textbf{Dating} \parencite{Sapp2022}: late C11th (0.650)

\textbf{Meter:} \Fornyrdislag
\end{flushright}%

\section{Introduction}

\textbf{The Hell-ride of Byrnhild} (\Helreid) describes how Byrnhild after her death is burned on her pyre in a beautiful chariot or wagon.  In the afterlife she rides on the \inx[L]{Hellway} to reach her resting place in \inx[L]{Hell} when her road is blocked by a certain \inx[C]{gow} or troll-woman.  The poem consists of their conversation.

\Helreid\ is attested in full in \Regius\ where it follows \Sigurdskamma\ and is followed by \GudrunTwo.  It is also attested with the exception of st. 6 in \Nornagest\ 9.  \Nornagest\ is attested in several mss., including \FlatMS.  The \Nornagest\ variants are divergent to the point that \Nornagest’s \Helreid\ cannot be directly copied from \Regius, but it is unclear whether it ultimately derives from \Regius\ through a chain of copies or whether both derive from a lost archetype predating \Regius.  At the very least we can tell from the similarities between the prose of \emph{Norna-Gests Þáttr} and \Regius\ that the two are unlikely to be related through oral tradition alone.

\sectionline

\section{Text}

\bpg\bpa%
\Ballnote{The equivalent passage in \FlatMS: \emph{En hirð-maðr einn spyrr: „hversu fór Bryn·hildr þá með?“ Gestr svarar: „þá drap Bryn·hildr sjau þrę́la sína ok fimm ambáttir en lagði sik sverði í gegnum ok bað sik aka með þessa menn til báls ok brenna sik dauða.  Ok svá var gert, at henni var gert annat bál en Sigurði annat ok var hann fyrri brenndr en Bryn·hildr.  Henni var ekit í reið einni ok var tjaldat um guð-vef ok purpúra ok glóaði allt við gull.  Ok svá var hón brend.“  Þá spurðu menn Gest hvárt Bryn·hildr hefði nokkut kveðit dauð.  Hann kvað þat satt vera.  Þeir báðu hann kveða ef hann kynni.  Þá mę́lti Gestr: „þá er Bryn·hildi var ekit til brennu’nnar á hel-veg ok var farit med hana nę́rr hǫmrum nokkurum,  þar bjó ein gýgr.  Hón [var] úti firir hellis-dyrum ok var í skinn-kyrtli ok svǫrt yfir·lits.  Hón hefir í hendi sér skógar-vǫnd langan ok mę́lti.  ‚Þessu vil ek beina til brennu þinnar, Bryn·hildr, ok vę́ri betr at þú vę́rir lifandi brennd firir ó·dáðir þínar þę́r at þu létst drepa Sigurð Fáfnis-bana, svá á·gę́tan mann; ok opt var ek hónum †sinunt†.  Ok firir þat skal ek hljóða á þik med hefndar-orðum þeim at ǫllum sér þú at leiðari er slíkt heyra frá þér sagt.‘  Eptir þat hljóðast þę́r á Bryn·hildr ok gýgr.  Gýgr kvað:} ‘But a hirdman asked: “How did Byrnhild come with him [= Siward]?”  Guest answers: “Then Byrnhild killed her seven thralls and five handmaids and ran herself through with a sword and bade herself be driven to a pyre with these men and be burned after her death.  And so it was done, that one pyre was made for her and one for Siward, and he was burned earlier than Byrnhild.  She was driven in a chariot and it was canopied with godweave and purple and it all glowed from gold.  And so was she burned.”  Then men asked Guest whether Byrnhild had sung anything after death  He said it was true.  They bade him sing, if he knew it.  Then spoke Guest: “When Byrnhild had been driven to her burning on the Hellway and had passed near some cliffs, a gow lived there.  She was outside before the doors of the cliffside and was in a skin skirt and swarthy of countenance.  She has in her hand a long tree-branch, and spoke: ‘With this [wood] I wish to assist thy burning, Byrnhild, and it were better hadst thou been burned alive for thy misdeeds that thou lettest Siward Fathomer’s bane be slain, such an excellent man; and I was often ... with him.  And therefore I shall bespeak thee with words of revenge such that all men will hate thee all the more as they hear such things told about thee.’  After that”’}%
Eptir dauða Bryn·hildar vóru gǫr bǫ́l tvau: annat Sig·urði, ok brann þat fyrr, en Bryn·hildr var á ǫðru brennd ok var hón \edtrans{í reið þeiri er guð-vefjum var tjǫlduð}{in that chariot which was canopied with god-weave}{\Bfootnote{The canopy on the chariot was woven from silk (known as \emph{god-weave}, the fabric of the gods).  For the burial of women in wagons and chariots cf. TODO (Oseberg ship?).}}.  Svá er sagt at \edtrans{Bryn·hildr ók með reið’inni á hel-veg}{Byrnhild drove with the chariot on the Hellway}{\Bfootnote{This gives us some interesting insight into old afterlife beliefs.  After Byrnhild is burned she ends up between the worlds of the dead and the living, the so-called “Hell-way” or road to Hell, which she must travel to arrive at her final resting place in the Underworld; she is burned inside a chariot so that she will be able to travel comfortably.  We may presume that the animals driving the chariot were slaughtered and burnt with her on the pyre.}} ok fór um tún þar er gýgr nǫkkur bjó.  Gýgr’in kvað:\epa

\bpb After Byrnhild’s death two pyres were made: one for Siward, and it burned earlier; but Byrnhild was burned on the other, and she was in that chariot which was canopied with \inx[C]{god-weave}.  So it is said that Byrnhild drove with the chariot onto the Hellway and passed near a yard where a certain \inx[C]{gow} lived.  The gow quoth:\epb\epg

\section{Byrnhild rode the Hellway (\emph{Bryn·hildr ręið hęl-veg})}

\bvg\bva%
\Ballnote{The gow begins by insulting Byrnhild and blames her early death on her immoral and unwomanly actions; if she had lived like a chaste woman and tended to simple domestic duties she would still have been alive.  She also points out that Siward is not even her husband, but Guthrun’s.}%
„Skalt ï \alst{g}ǫgnum \hld\ \alst{g}anga ęigi &
\alst{g}rjóti studda \hld\ \alst{g}arða mïna; &
\alst{b}ętr sǿmði þér \hld\ \alst{b}orða at rękja &
hęldr \edtext{an}{\Afootnote{add. \emph{at} \FlatMS}} \alst{v}itja \hld\ \edtrans{\alst{v}ers annarar}{another’s husband}{\Afootnote{\emph{várra ranna} (norm.) ‘our halls’ \FlatMS}}.\eva

\bvb “Thou shalt nowise pass through these \\
rock-supported yards of mine! \\
It befit thee better to weave tapestries \\
rather than visit another’s husband.\evb\evg


\bvg\bva%
Hvat skalt \alst{v}itja \hld\ \edtext{af \alst{V}al-landi}{\Afootnote{\emph{†ua a landi†} \FlatMS}}, &
\edtrans{\alst{h}var-fu̇st}{straying}{\Afootnote{\emph{hverf-lynt} ‘fickle-minded’ \FlatMS}} \alst{h}ǫfuð, \hld\ \alst{h}úsa mïnna? &
\edtext{Þú hęfir, \alst{V}ǫ́r golls, \hld\ ef þik \alst{v}ita lystir, &
\alst{m}ild, af hǫndum \hld\ \alst{m}anns blóð þvegit.}{\lemma{Þú \dots\ þvegit. ‘Thou \dots\ hands’}\Afootnote{\emph{\alst{Þ}ú hęfir vǫrgum \hld\ ef \alst{þ}ïn vitja / \alst{m}ǫrgum til \alst{m}atar \hld\ \alst{m}anns blóð gefit} ‘Thou hast to many—if thou art visited(?)—wolves given man-blood for food.’ \FlatMS}}“\eva

\bvb Why shalt thou visit, come from Walland, \\
O straying head, these houses of mine? \\
Thou hast, mild \inx[P]{Ware} of gold \ken{lady}, if thou hast lust to know, \\
washed man-blood from thy hands.”\evb\evg


\bvg\bva%
\speakernote{[Bryn·hildr svaraði:]}%
„\alst{B}regð \edtext{ęigi mér}{\Afootnote{\emph{mér ęigi} \FlatMS}}, \hld\ \alst{b}rúðr ór stęini, &
þó’tt ek \edtrans{\alst{v}ę́ra’k}{I may have been}{\Afootnote{add. \emph{fyrr} ‘formerly’ \FlatMS}} \hld\ ï \alst{v}íkingu; &
\alst{e}k mun \alst{o}kkur \hld\ \alst{ǿ}ðri þikkja &
\edtrans{hvar’s męnn \alst{ø}ðli \hld\ \alst{o}kkart kunna}{wherever men know our pedigrees}{\Afootnote{\emph{þeim’s øðli mitt um kunna} ‘to those who know my pedigree’ (unmetrical) \FlatMS}}.“\eva

\bvb “Upbraid me not, O bride from the stone \ken{gow}, \\
although I may have been in the wiking host; \\
of us two will I seem the nobler \\
wherever men know our pedigrees.”\evb\evg


\bvg\bva%
\speakernote{[Gýgr kvað:]}%
„Þú \edtrans{vast}{wast}{\Afootnote{\emph{est} ‘art’ \FlatMS}}, \alst{B}ryn-hildr, \hld\ \alst{B}uðla dóttir, &
\alst{h}ęilli verstu \hld\ ï \alst{h}ęim borin; &
þú hęfir \alst{G}júka \hld\ of \alst{g}latat bǫrnum &
ok \alst{b}úi þęira \hld\ \alst{b}rugðit góðu.“\eva

\bvb “Thou wast, O Byrnhild, Budle’s daughter, \\
under the worst omen born into the Home. \\
Thou hast destroyed Yivick’s children \\
and deprived their house of good.”\evb\evg


\bvg\bva%
\speakernote{[Bryn·hildr kvað:]}%
„Ek mun \alst{s}ęgja þér \hld\ \alst{s}vinn ór ręiðu, &
\alst{v}it-laussi mjǫk, \hld\ ef þik \alst{v}ita lystir: &
hvé \alst{g}ørðu mik \hld\ \alst{G}júka arfar &
\alst{ȧ}sta-lausa \hld\ ok \alst{ęi}ð-rofa.\eva

\bvb “I will tell thee, wise from my chariot, \\
O very witless one, if thou hast lust to know, \\
how the heirs of Yivick made me \\
loveless and an oath-breaker.\evb\evg


\bvg\bva%
Lét \alst{h}ami vȧra \hld\ \alst{h}ug-fullr konungr, &
\alst{á}tta systra, \hld\ \edtrans{undir \alst{ęi}k borit}{under an oak}{\Bfootnote{King Budle let Byrnhild and her seven sisters be born under a (holy) oak due to the protection believed to be afforded by that species.}}; &
\alst{v}as’k \alst{v}etra tólf, \hld\ ef þik \alst{v}ita lystir, &
es \alst{u}ngum gram \hld\ \alst{ęi}ða sęlda’k.\eva

\bvb The king full of heart \ken*{= Budle} let the forms of us \\
eight sisters be born beneath an oak. \\
I was twelve winters old, if thou hast lust to know, \\
when to the young lord oaths I gave.\evb\evg


\bvg\bva%
\alst{H}étu mik allir \hld\ ï \alst{H}lym-dǫlum &
\alst{H}ildi und \alst{h}jalmi, \hld\ \alst{h}vęrr es kunni.\eva

\bvb In the Limdales all did call me \\
a Hild ’neath the helmet, whoever knew me.\evb\evg


\bvg\bva%
\Ballnote{As told in \textlink{Sigrdrifumal}[P3].}%
Þȧ lét’k \alst{g}amlan \hld\ ȧ \alst{G}oð-þjóðu &
\alst{H}jalm-Gunn·ar nę́st \hld\ \alst{h}ęljar ganga; &
gaf’k \alst{u}ngum sigr \hld\ \alst{Au}ðu bróður; &
þar varð mér \alst{Ó}ðinn \hld\ \alst{o}f-ręiðr um þat.\eva

\bvb Then I next in the land of the Gots \\
let old Helm-Guther go the way of Hell. \\
I gave victory to Ead’s young brother; \\
then was Weden furious with me over that.\evb\evg


\bvg\bva%
Lauk hann mik \alst{sk}jǫldum \hld\ ï \alst{Sk}ata-lundi, &
\alst{r}auðum ok hvítum, \hld\ \alst{r}andir snurtu; &
þann bað hann \alst{s}líta \hld\ \alst{s}vefni mïnum &
es \alst{h}vęr-gi lands \hld\ \alst{h}rę́ðask kynni.\eva

\bvb He locked me inside shields in Shatelund, \\
red ones and white; their rims clasped. \\
He bade that one upend my sleeping \\
who in no land could be frightened.\evb\evg


\bvg\bva%
Lét umb \alst{s}al minn \hld\ \alst{s}unnan-verðan &
\alst{h}ávan brinna \hld\ \alst{h}ęr alls viðar; &
þar bað hann \alst{ęi}nn þegn \hld\ \alst{y}fir at ríða, &
þann’s mér \alst{f}ǿrði goll \hld\ þat’s und \alst{F}áfni lá.\eva

\bvb He made around my hall a south-facing \\
high host of all wood \ken{fire} burn. \\
He bade only one thane there ride over it: \\
him who brought me the gold which neath Fathomer lay.\evb\evg


\bvg\bva%
Ręið \alst{g}óðr \alst{G}rana \hld\ \alst{g}oll-miðlandi &
þar’s \alst{f}óstri mïnn \hld\ \alst{f}lętjum stýrði; &
\alst{ęi}nn þȯtti hann þar \hld\ \alst{ǫ}llum bętri, &
\alst{v}íkingr Dana, \hld\ ï \alst{v}erðungu.\eva

\bvb On Grane rode the good gold-dealer \ken*{\textsc{ruler} = Siward} \\
where my foster-son ruled the benches. \\
Alone he seemed there better than all, \\
the wiking of Danes, in the warband.\evb\evg


\bvg\bva%
\alst{S}vǫ́fu vit ok unðum \hld\ ï \alst{s}ę́ing ęinni &
sem hann mïnn \alst{b}róðir \hld\ of \alst{b}orinn vę́ri; &
\alst{h}várt-ki knátti \hld\ \alst{h}ǫnd yfir annat &
\alst{á}tta nǫ́ttum \hld\ \alst{o}kkart lęggja.\eva

\bvb We slept and were content in a single bed \\
as if he had been born my brother; \\
neither laid a hand o’er the other \\
for eight nights, of us two.\evb\evg


\bvg\bva%
Því brá mér \alst{G}uð·ru̇n, \hld\ \alst{G}júka dóttir, &
at ek \alst{S}ig·urði \hld\ \alst{s}vę́fa’k ȧ armi; &
þar varð’k þęss \alst{v}ís \hld\ es \alst{v}ildi’g⸗a’k &
at þau \alst{v}éltu mik \hld\ ï \alst{v}er-fangi.\eva

\bvb Thus upbraided me Guthrun Yivick’s daughter, \\
because I slept on Siward’s arm. \\
There I became wise of what I did not want to be: \\
that those two had tricked me in the catch of man.\evb\evg


\bvg\bva%
Munu við \alst{o}f·stríð \hld\ \alst{a}lls til lęngi &
\alst{k}onur ok \alst{k}arlar \hld\ \alst{k}vikkvir fǿðask; &
vit skulum \alst{o}kkrum \hld\ \edtrans{\alst{a}ldri slíta}{spend our ages}{\Bfootnote{In the afterlife.  An interesting expression considering how both their \emph{aldrar} ‘ages, lifetimes’ have concluded.}}, &
\alst{S}ig·urðr, \alst{s}aman. \hld\ \alst{S}økks-tu, gýgjar-kyn!“\eva

\bvb In great strife for far too long \\
will men and women alive be born. \\
We two shall spend our ages, \\
I and Siward, together.—Sink, thou gow-kind!”\evb\evg

\sectionline
