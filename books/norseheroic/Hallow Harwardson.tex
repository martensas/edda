\bookStart{Lay of Hallow Harwardson}[Hęlga kviða Hjǫr·varðs-sonar]
\setBookCode{HelgakvidaHjorvardssonar}

\begin{flushright}%
\textbf{Dating} \parencite{Sapp2022}: early C11th (0.385)–late C11th (0.550)

\textbf{Meter:} \Fornyrdislag\ (1/1–12/2, 31–43), \Ljodahattr\ (12/3–30/4)%
\end{flushright}%

Heroic prosimetrum.  According to Lindblad forms a group with \textlink{HelgakvidaTwo}.

\sectionline

\section{From Harward and Syelind (\emph{Frá Hjǫr·varði ok Sigr·linn})}

\bpg\bpa Hjǫr·varðr hét konungr; hann átti fjórar konur.  Ein hét Alf·hildr; sonr þeira hét Heðinn.  Ǫnnur hét Sę́r·eiðr; þeira sonr hét Humlungr.  In þriðja hét Sin·rjóð; þeira sonr hét Hymlingr.  Hjǫr·varðr konungr hafði þess heit strengt at eiga þá konu er hann vissi vę́nsta.  Hann spurði at Sváfnir konungr átti dóttur \edtext{allra}{\Afootnote{corr. from \emph{‘vęnallra’} \Regius}} fegrsta; sú hét Sigr·linn.  Ið·mundr hét jarl hans; Atli var hans sonr er fór at biðja Sigr·linnar til handa konungi.  Hann dvalðisk vetr-langt með Sváfni konungi.  Frán·marr hét þar jarl, fóstri Sigr·linnar; dóttir hans hét Álǫf.  Jarl’inn réð, at meyjar var synjat, ok fór jarl’inn heim.  Atli jarls sonr stóð einn dag við lund nǫkkurn, en fugl sat í limu’num uppi yfir hánum ok hafði heyrt til, at hans menn kǫlluðu vę́nstar konur þę́r, er Hjǫr·varðr konungr átti.  Fugl’inn kvakaði, en Atli hlýddi, hvat hann sagði. Hann kvað:\epa

\bpb Hearward was the name of a king; he had four wives.  One was called Elfhild; their son was called Headen.  Another was called Searath; their son was called Humbling.  The third was called Sindread; their son was called Himbling.  King Hearward had undertaken a vow to have those women whom he knew the most handsome.  He learned that king Swebner had a daughter fairest of all; she was called Syelind.  Ithmund was the name of his earl; Attle was his son, who journeyed to ask for Syelind’s hand on behalf of the king.  He stayed over the winter with king Swebner.  Frenmar was the name of an earl there, the foster-father of Syelin; his daughter was called Anlave.  The earl decided that the maiden was denied him and the earl journeyed home.  Attle the earl’s son one day stood by a certain tree, but a bird sat in the branches above him and had overheard that his men said that the women which king Hearward had were the most handsome.
The bird twittered, and Attle listened to what it said.  It quoth:\epb\epg


\bvg\bva\speakernote{Fugl kvað:}%
„\alst{S}átt-u \alst{S}igr·linn, \hld\ \alst{S}váfnis dóttur, &
\alst{m}ęyna fęgrstu \hld\ ï \alst{m}unar-hęimi? &
Þó \alst{h}ag-ligar \hld\ \alst{H}jǫr·varðs konur &
\alst{g}umnum þykkja \hld\ at \alst{G}lasis-lundi.“\eva

\bvb “Hast thou seen Syelind Swebner’s daughter, \\
the fairest of maidens in the realm of love \ken{world}? \\
Although Hearward’s wives seem handsome \\
to the men in Glazerslund.”\evb\evg


\bvg\bva\speakernote{Atli kvað:}%
„Munt við \alst{A}tla \hld\ \alst{I}ð·mundar son &
\alst{f}ugl \alst{f}róð-hugaðr \hld\ \alst{f}lęira mę́la?“ &
\speakernote{Fugl kvað:}%
„Mun’k ef mik \alst{b}uðlungr \hld\ \alst{b}lóta vildi &
ok \alst{k}ýs’k þat’s ek vil \hld\ ór \alst{k}onungs garði.“\eva

\bvb “Wilt thou with Attle Idmund’s son, \\
O wise-minded fowl, speak yet more?” \\
“I will, if the prince will make me a bloot, \\
and I may choose what I wish from the house of the king.”\evb\evg


\bvg\bva\speakernote{Atli kvað:}%
„Kjós-at-tu \alst{H}jǫr·varð \hld\ né \alst{h}ans sonu &
né inar \alst{f}ǫgru \hld\ \alst{f}ylkis brúðir, &
ęigi \alst{b}rúðir \hld\ þę́r’s \alst{b}uðlungr á; &
kaupum \alst{v}el saman, \hld\ þat ’s \alst{v}ina kynni.“\eva

\bvb TODO 3.\evb\evg


\bvg\bva\speakernote{Fugl kvað:}%
„\alst{H}of mun’k kjósa, \hld\ \alst{h}ǫrga marga, &
\edtrans{\alst{g}oll-hyrnðar kýr}{golden-horned kine}{\Bfootnote{Formulaic, also found in \textlink{Thrymskvida}[23]/1b.}} \hld\ frȧ \alst{g}rams búi, &
ef hǫ̇num \alst{S}igr·linn \hld\ \alst{s}øfr ȧ armi &
ok \alst{ȯ}·nauðig \hld\ \alst{jǫ}fri fylgir.“\eva

\bvb “Hoves will I choose and many harrows, \\
golden-horned kine from the ruler’s estate \\
if on his arm Syelin will sleep \\
and without force follow the lord.”\evb\evg


\bpg\bpa%
Þetta var áðr Atli fǿri. En er hann kom heim ok konungr spurði hann tíðinda, hann kvað:\epa

\bpb TODO.\epb\epg

\bvg\bva%
„Hǫfum \alst{ę}rfiði \hld\ ok ękki \alst{ø}rendi; &
\alst{m}ara þraut ȯra \hld\ ȧ \alst{m}ęgin-fjalli, &
urðum \alst{s}íðan \hld\ \alst{S}ę́·morn vaða; &
þȧ vas oss \alst{s}ynjat \hld\ \alst{S}váfnis dóttur, &
\alst{h}ringum gǿddrar, \hld\ es vér \alst{h}afa vildum.“\eva

\bvb TODO 5\evb\evg


\bpg\bpa%
Konungr bað at þeir skyldu fara annat sinn; fór hann sjálfr. En er þeir kómu upp á fjall ok sá á Sváfa-land lands-bruna ok jó-reyki stóra. Reið konungr af fjalli’nu fram í land’it ok tók nátt-ból við á eina. Atli helt vǫrð ok fór yfir á’na. Hann fann eitt hús. Fugl mikill sat á húsi’nu ok gę́tti ok var sofnaðr. Atli skaut spjóti fugl’inn til bana en í húsi’nu fann hann Sigr·linn konungs dóttur ok Á·lǫfu jarls dóttur ok hafði þę́r báðar braut með sér. Frán·marr jarl hafði hamazt í arnar líki ok varit þę́r fyr her’num með fjǫl-kynngi.
Hróð·marr hét konungr, biðill Sigr·linnar. Hann drap Sváfa-konung ok hafði rę́nt ok brennt land’it. Hjǫr·varðr konungr fekk Sigr·linnar en Atli Á·lǫfar.\epa

\bpb TODO.\epb\epg


\bpg\bpa%
Hjǫr·varðr ok Sigr·linn áttu son mikinn ok vę́nan. Hann var þǫgull; ekki nafn festist við hann. Hann sat á haugi. Hann sá ríða val-kyrjur níu ok var ein gǫfug-ligust. Hón kvað:\epa

\bpb TODO.  She quoth:\epb\epg


\bvg\bva%
„Síð munt, \alst{H}elgi, \hld\ \alst{h}ringum ráða, &
\alst{r}íkr \alst{r}óg-apaldr, \hld\ né \alst{R}ǫðuls-vǫllum, &
\alst{ǫ}rn gól \alst{á}r-la, \hld\ ef þú \alst{ę́} þęgir, &
þó’tt-u \alst{h}arðan \alst{h}ug, \hld\ \alst{h}ilmir, gjaldir.“\eva

\bvb TODO 6\evb\evg


\bvg\bva\speakernote{[Hęlgi] kvað:}%
„\alst{H}vat lę́tr fylgja \hld\ \alst{H}ęlga nafni, &
\alst{b}rúðr \alst{b}jart-lituð, \hld\ alls \alst{b}jóða rę́ðr? &
Hygg fyr \alst{ǫ}llum \hld\ \alst{a}t·kvę́ðum vel; &
\alst{þ}igg ęigi \alst{þ}at \hld\ nema \alst{þ}ik hafa!“\eva

\bvb TODO 7\evb\evg


\bvg\bva\speakernote{[Val-kyrja] kvað:}%
„\alst{S}verð vęit’k liggja \hld\ ï \alst{S}igars-holmi, &
\alst{f}jórum \alst{f}ę́ra \hld\ an \alst{f}imm tǫgu; &
\alst{ęi}tt es þęira \hld\ \alst{ǫ}llum bętra &
\alst{v}íg-nesta bǫl \hld\ ok \alst{v}arit golli.\eva

\bvb “Swords I know lying in Sigarsholm: \\
four less than fifty. \\
One of them is better than all—\\
a \inx[C]{bale} of war-covers(?) \ken{shields}—and covered with gold.\evb\evg

\begin{figure}[b]
\centering
\includegraphics[width=\textwidth]{Snartemo-hilt}
\caption{Hilt of the Snartemo sword, front and reverse.  Migration period, ca. 500 CE.  © Eirik Irgens Johnsen, \href{https://creativecommons.org/licenses/by-sa/4.0/deed.en}{CC BY-SA 4.0}.  \url{https://www.unimus.no/portal/\#/photos/d8932af5-1082-4938-9b4b-ca6b86f2bdfb}}
\label{fig:snartemo}
\end{figure}

\bvg\bva%
\edtrans{\alst{H}ringr ’s ï \alst{h}jalti}{A ring is on its hilt}{\Bfootnote{The sword is a so-called \emph{ring-sword}.  It was popular among Germanic warriors of the Migration Period to have oath-ring on their sword-hilts as a symbol of fidelity to their lords, but this custom was entirely extinct by ca. 700 \parencite[40–44]{Nerman1931} and the detail thus serves to emphasize the exceptional age of the sword.  It also helps us date at least this part of the poem, for it is highly unlikely that an Icelandic antiquarian of the C12th would have been aware of this circumstance, whereas a Norwegian of the C10th conceivably might.  Cf. \textlink{Sigurdskamma}[68] which probably also refers to a ring-sword.  A well preserved Norwegian ring-sword survives from Snartemo in Vest-Agder,  dating to around 500 CE (object ID C26001); see Fig. \ref{fig:snartemo}.}}, \hld\ \alst{h}ugr ’s ï miðju, &
\alst{ó}gn ’s ï \alst{o}ddi, \hld\ þęim’s \alst{ęi}ga getr; &
liggr með \alst{ę}ggju \hld\ \alst{o}rmr dręyr-fáiðr &
en ȧ \edtrans{\alst{v}al-bǫstu}{walbast}{\Bfootnote{An unclear part of the sword-hilt; see \textlink{Sigrdrifumal} 6.}} \hld\ \alst{v}erpr naðr hala.“\eva

\bvb A ring is on its hilt; heart is in the middle; \\
terror is in the point for him who gets to own it. \\
Along the edge lies a serpent painted in blood \\
and on the walbast an adder eats its tail.”\evb\evg


\bpg\bpa%
Ey·limi hét konungr; dóttir hans var Sváfa.  Hón var val-kyrja ok reið lopt ok lǫg.  Hón gaf Helga nafn þetta ok hlífði hǫ́num opt síðan í orrustum.  Helgi kvað:\epa

\bpb TODO.  Hallow quoth:\epb\epg


\bvg\bva%
„Est-at, \alst{H}jǫr·varðr, \hld\ \alst{h}ęil-ráðr konungr, &
\alst{f}olks odd-viti, \hld\ þó’tt \alst{f}rę́gr séir; &
létst-u \alst{ę}ld \alst{e}ta \hld\ \alst{jǫ}fra byggðir &
en þęir \alst{a}ngr við þik \hld\ \alst{ę}kki gørðu.\eva

\bvb TODO 10\evb\evg


\bvg\bva%
Ęn \alst{H}róð·marr skal \hld\ \alst{h}ringum ráða, &
þeim es \alst{ǫ́}ttu \hld\ \alst{ȯ}rir niðjar; &
sá sés’k \alst{f}ylkir \hld\ \alst{f}ę́st at lífi, &
hyggs’k \alst{a}l-dauðra \hld\ \alst{a}rfi at ráða.“\eva

\bvb TODO 11\evb\evg


\bpg\bpa%
Hjǫr·varðr svaraði at hann myndi fá lið Helga ef hann vill hefna móður-fǫður síns. Þá sótti Helgi sverð’it er Sváfa vísaði hánum til.  Þá fór hón ok Atli ok felldu Hróð·mar ok unnu mǫrg þrek-virki.  Hann drap Hata jǫtun er hann sat á bergi nokkuru.  Helgi ok Atli lǫ́gu skipum í Hata-firði.  Atli helt vǫrð inn fyrra hlut nę́tr’innar.  Hrím·gerðr Hata dóttir kvað:\epa

\bpb TODO.  Attle kept watch the first part of the night.  Rimegird Hate’s daughter quoth:\epb\epg


\bvg\bva%
„\alst{H}vęrir ’ru \alst{h}ǫlðar \hld\ ï \alst{H}ata-firði? &
\alst{Sk}jǫldum es tjaldat \hld\ ȧ \alst{sk}ipum yðrum, &
\alst{f}rǿkn-liga látið, \hld\ \alst{f}átt hygg yðr s\emph{éa}sk; &
\ind \alst{k}ęnnið mér nafn \alst{k}onungs!“\eva

\bvb TODO 12\evb\evg


\bvg\bva\speakernote{Atli kvað:}%
„\alst{H}ęlgi hann \alst{h}ęitir \hld\ ęn þú \alst{h}vęr-gi mátt &
\ind vinna \alst{g}rand \alst{g}rami, &
\alst{éa}rn-borgir \hld\ ’ru umb \alst{ǫ}ðlings flota &
\ind knegu-t oss \alst{f}ǫ́lur \alst{f}ara.“\eva

\bvb TODO 13\evb\evg


\bvg\bva%
\alst{H}vé þik \alst{h}eitir, \hld[kvað Hrím·gęrðr] \alst{h}alr inn ȧ·máttki? &
\ind Hvé þik \alst{k}alla \alst{k}onir? &
\alst{F}ylkir þér trúir \hld\ es þik ï \alst{f}ǫgrum lę́tr &
\ind \alst{b}ęits stafni \alst{b}úa.\eva

\bvb TODO 14\evb\evg


\bvg\bva\speakernote{Atli kvað:}%
„\alst{A}tli ek hęiti, \hld\ \alst{a}tall skal’k þér vera, &
\ind mjǫk em’k \alst{g}ífrum \alst{g}ramastr; &
\alst{ú}rgan stafn \hld\ ek hęfi \alst{o}pt búit &
\ind ok \alst{k}valðar \alst{k}veld-riður.\eva

\bvb TODO 15\evb\evg


\bvg\bva%
\alst{H}vé þú \alst{h}ęitir, \hld\ \alst{h}ála ná-grǫ́ðug? &
\ind Nęfn þinn, \alst{f}ála, \alst{f}ǫður; &
\alst{n}íu rǫstum \hld\ es þú skyldir \alst{n}eðar vesa &
\ind ok vaxi þér ȧ \alst{b}aðmi \alst{b}arr!“\eva

\bvb TODO 16\evb\evg


\bvg\bva\speakernote{Hrím·gerðr kvað:}%
„\alst{H}rím·gerðr ek \alst{h}ęiti, \hld\ \alst{H}ati hét minn faðir, &
\ind þann vissa’k \alst{ȧ}·máttkastan \alst{jǫ}tun; &
margar \alst{b}rúðir \hld\ hann lét frá \alst{b}úi tęknar &
\ind und’s hann \alst{H}ęlgi \alst{h}jó.“\eva

\bvb “Rimegird I am called—Hater was my father called, \\
\ind I knew him to be the uncanniest ettin; \\
many brides he had taken from their homes \\
\ind before Hallow cut him down.”\evb\evg


\bvg\bva%
„Þú vast, \alst{h}ála, \hld\ fyr \alst{h}ildings skipum &
\ind ok látt ï \alst{f}jarðar-mynni \alst{f}yrir; &
\alst{r}ę́sis \alst{r}ekka \hld\ es vildir \alst{R}ǫ́n gefa &
\ind ef þér kǿmi’t ï \alst{þ}verst \alst{þ}vari.“\eva

\bvb TODO 18\evb\evg


\bvg\bva\speakernote{Hrím·gerðr kvað:}%
„\alst{D}uliðr est nú, Atli, \hld\ \alst{d}raums kveð’k þér vesa, &
\ind síga lę́tr þú \alst{b}rẏnn fyr \alst{b}rá\emph{a}r; &
\alst{m}óðir \alst{m}ïn \hld\ lá fyr \alst{m}ildings skipum, &
\ind ek drękkða \alst{H}lǫð·varðs sonum ï \alst{h}afi.\eva

\bvb TODO 19\evb\evg


\bvg\bva%
\alst{G}nęggja myndir þú, Atli, \hld\ ef \alst{g}ęldr né vę́rir: &
\ind bręttir sïnn \alst{H}rím·gerðr \alst{h}ala! &
\alst{A}ptar-la hjarta \hld\ hygg at þitt, \alst{A}tli, sé &
\ind þó’tt þú hafir \edtext{*\alst{r}ęina}{\Afootnote{\emph{hreina} \Regius.}} \alst{r}ǫdd.“\eva

\bvb Thou wouldst neigh, Attle, if thou wert not a gelding; \\
\ind Rimegird wags her tail! \\
A cowering heart I think thine, Attle, is, \\
\ind although thou hast a stallion’s voice.”\evb\evg


\bvg\bva\speakernote{Atli kvað:}%
„\alst{R}ęini mun þér ek þikkja \hld\ ef þú \alst{r}ęyna knátt &
\ind ok stíga’k ȧ \alst{l}and af \alst{l}ęgi; &
\alst{ǫ}ll munt lęmjas’k \hld\ ef mér ’s \alst{a}l-hugat &
\ind ok svęigja þïnn \alst{h}ala, \alst{H}rím·gerðr!“\eva

\bvb “A stallion wilt thou think me if thou mightst experience it, \\
\ind and I step aland from the lake; \\
TODO.”\evb\evg


\bvg\bva\speakernote{Hrím·gerðr kvað:}%
„\alst{A}tli, gakk ȧ land \hld\ ef \alst{a}fli tręystis’k &
\ind ok hittumk ï \alst{v}ík \alst{V}arins, &
\alst{r}ifja \alst{r}étti \hld\ es þú munt, \alst{r}ekkr, fȧ &
\ind ef þú mér ï \alst{k}rymmur \alst{k}ømr.“\eva

\bvb TODO 22\evb\evg


\bvg\bva\speakernote{Atli kvað:}%
„Mun’k-a ek \alst{g}anga \hld\ áðr \alst{g}umnar vakna &
\ind ok halda of \alst{v}ísa \alst{v}ǫrð; &
es-a mér \alst{ø}r·vę̇nt \hld\ nę́r \alst{ȯ}ru kømr &
\ind \alst{sk}ass upp undir \alst{sk}ipi.“\eva

\bvb TODO 23\evb\evg


\bvg\bva\speakernote{Hrím·gerðr kvað:}%
„Vaki þú, \alst{H}ęlgi, \hld\ ok bǿt við \alst{H}rím·gęrði &
\ind es þú létst \alst{h}ǫggvinn \alst{H}ata; &
\alst{ęi}na nǫ́tt \hld\ kná hǫ̇n hjá \alst{jǫ}fri sofa, &
\ind þȧ hęfir hǫ̇n \alst{b}ǫlva \alst{b}ǿtr.“\eva

\bvb “Wake thou, Hallow, and restore Rimegird \\
\ind since thou didst let Hate be cut down. \\
A single night may she sleep by the ruler, \\
\ind then she has restitution for her bales.”\evb\evg


\bvg\bva\speakernote{Hęlgi kvað:}%
„\alst{L}oðinn hęitir es þik skal ęiga, \hld\ —\alst{l}ęið est mann-kyni—, &
\ind sá býr ï \alst{Þ}oll-ęyju \alst{þ}urs, &
\alst{h}und-víss jǫtunn, \hld\ \alst{h}raun-búa verstr, &
\ind sá ’s þér \alst{m}akligr \alst{m}aðr.“\eva

\bvb TODO 25\evb\evg


\bvg\bva\speakernote{Hrím·gerðr kvað:}%
„\alst{H}ina vilt \alst{h}ęldr, \alst{H}ęlgi, \hld\ es réð \alst{h}afnir skoða &
\ind \alst{f}yrri nǫ́tt með \alst{f}irum; &
\alst{m}ar-gollin \alst{m}ę́r \hld\ \alst{m}ér þȯtti afli bera; &
\ind hér sté hǫ̇n \alst{l}and af \alst{l}ęgi &
\ind ok \alst{f}ęsti svá yðarn \alst{f}lota; &
hǫ̇n \alst{ęi}n því veldr \hld\ es ek \alst{ęi}gi má’k &
\ind \alst{b}uðlungs mǫnnum \alst{b}ana.“\eva

\bvb TODO 26\evb\evg


\bvg\bva\speakernote{Hęlgi kvað:}%
„\alst{H}ęyr þú nú, \alst{H}rím·gerðr, \hld\ ef ek bǿti \alst{h}arma þér, &
\ind sęg þú \alst{g}ǫrr \alst{g}rami: &
Vas sú \alst{ęi}n vę́ttr \hld\ es barg \alst{ǫ}ðlings skipum &
\ind eða \alst{f}óru þę́r \alst{f}lęiri saman?“\eva

\bvb TODO 27\evb\evg


\bvg\bva\speakernote{Hrím·gerðr kvað:}%
„\alst{Þ}rennar níundir męyja, \hld\ \alst{þ}ó ręið ęin fyrir, &
\ind \alst{h}vít und \alst{h}jalmi mę́r; &
\alst{m}arir hristusk, \hld\ stóð af \alst{m}ǫnum þęira &
\ind \alst{d}ǫgg ï \alst{d}júpa \alst{d}ali, &
\ind \alst{h}agl ï \alst{h}áva viðu, &
\ind þaðan kømr með \alst{ǫ}ldum \alst{á}r; &
\ind allt vas mér þat \alst{l}ęitt es \alst{l}ęit’k.“\eva

\bvb TODO 28\evb\evg


\bvg\bva\speakernote{Hęlgi kvað:}%
„Austr \alst{l}ít-tu nú, Hrím·gęrðr, \hld\ ef þik \alst{l}ostna hęfr &
\ind \alst{H}ęlgi \alst{h}ęl-stǫfum; &
ȧ landi ok ȧ \alst{v}atni \hld\ borgit ’s \alst{ǫ}ðlings flota &
\ind ok \alst{s}iklings mǫnnum it \alst{s}ama!“\eva

\bvb TODO 29\evb\evg


\bvg\bva\speakernote{Atli kvað:}%
„\alst{D}agr ’s nú, Hrím·gęrðr, \hld\ en þik \alst{d}valða hęfir &
\ind \alst{A}tli til \alst{a}ldr-laga; &
\alst{h}afnar mark \hld\ þykkir \alst{h}lǿg-ligt vesa, &
\ind þar’s þú ï \alst{st}ęins líki \alst{st}ęndr!“\eva

\bvb TODO 30\evb\evg


\bpg\bpa%
Helgi konungr var all-mikill her-maðr.  Hann kom til Ey·lima konungs ok bað Svǫ́fu, dóttur hans.  Þau Helgi ok Sváfa veittus’k várar ok unnusk furðu mikit.  \\
Sváfa var heima með feðr sínum en Helgi í hernaði; var Sváfa val-kyrja enn sem fyrr.  \\
Heðinn var heima með fǫður sínum, Hjǫr·varði konungi, í Noregi.  Heðinn fór einn saman heim ór skógi jóla-aptan ok fann troll-konu; sú reið vargi ok hafði orma at taumum ok bauð fylgð sína Heðni.  „Nei,“ sagði hann.  Hón sagði: „Þess skaltu gjalda at bragar-fulli!“ \\
Um kveld’it óru heit-strengingar; var framm leiddr sonar-gǫltr; lǫgðu menn þar á hendr sínar ok strengdu menn þá heit at bragar-fulli.  Heðinn strengði heit til Svǫ́fu Ey·lima dóttur, unnustu Helga, bróður síns, ok iðraðisk svá mjǫk at hann gekk á braut villi-stígu suðr á lǫnd ok fann Helga bróður sinn.  Helgi kvað:\epa

\bpb TODO.\epb\epg


\bvg\bva%
„Kom \alst{h}ęill, \alst{H}ęðinn, \hld\ \alst{h}vat kannt sęgja &
\alst{n}ýra spjalla \hld\ ór \alst{N}or·egi? &
Hví ’s þér, \alst{st}illir, \hld\ \alst{st}økkt ór landi &
ok est \alst{ęi}nn kominn \hld\ \alst{o}\emph{ss} at finna?“\eva

\bvb TODO 31\evb\evg


\bvg\bva%
„\alst{M}ik hęfir \alst{m}yklu glǿpr \hld\ \alst{m}ęiri sóttan: &
Ek hęfi \alst{k}ørna \hld\ ina \alst{k}onung-bornu &
\alst{b}rúði þïna \hld\ at \alst{b}ragar-fulli.“\eva

\bvb TODO 32\evb\evg


\bvg\bva%
„\alst{S}akask ęigi þú! \hld\ \alst{S}ǫnn munu verða &
\alst{ǫ}l-mǫ́l, Hęðinn, \hld\ \alst{o}kkur bęggja; &
mér hęfir \alst{st}illir \hld\ \alst{st}økkt til ęyrar, &
\alst{þ}riggja nátta, \hld\ skyla’k \alst{þ}ar koma; &
\alst{i}f ’s mér ȧ því \hld\ at \alst{a}ptr koma; &
þȧ má at \alst{g}óðu \hld\ \alst{g}ørask slíkt ef skal.“\eva

\bvb TODO 33\evb\evg


\bvg\bva%
„Sagðir þú, \alst{H}ęlgi, \hld\ at \alst{H}ęðinn vę́ri &
\alst{g}óðs verðr frȧ þér \hld\ ok \alst{g}jafa stórra; &
þér es \alst{s}ǿmra \hld\ \alst{s}verð at rjóða &
an \alst{f}rið gefa \hld\ \alst{f}jǫ́ndum þïnum.“\eva

\bvb TODO 34\evb\evg


\bpg\bpa%
Þat kvað Helgi því at hann grunaði um feigð sína ok þat at fylgjur hans hǫfðu vitjat Heðins þá er hann sá konu’na ríða vargi’num.  Álfr hét konungr, sonr Hróð·mars, er Helga hafði vǫll haslaðan á Sigars-velli á þriggja nátta fresti.  Þá kvað Helgi:\epa

\bpb TODO.\epb\epg


\bvg\bva%
„\alst{R}ęið ȧ vargi \hld\ es \alst{r}økvit vas, &
\alst{f}ljóð ęitt es hann \hld\ \alst{f}ylgju bęiddi; &
hǫ̇n \alst{v}issi þat \hld\ at \alst{v}eginn myndi &
\alst{S}igr·linnar \alst{s}onr \hld\ ȧ \alst{S}igars-vǫllum.“\eva

\bvb TODO 35\evb\evg


\bpg\bpa%
Þar var orrusta mikil ok fekk þar Helgi bana-sár.\epa

\bpb That was a great battle, and there Hallow got his bane-wound.\epb\epg


\bvg\bva%
\alst{S}ęndi Hęlgi \hld\ \alst{S}igar at ríða &
\alst{ę}ptir \alst{Ęy}·lima \hld\ \alst{ęi}nga-dóttur; &
\alst{b}iðr \alst{b}rá-lliga \hld\ \alst{b}u̇na verða &
ef hǫ̇n vill \alst{f}inna \hld\ \alst{f}ylki kvikvan.\eva

\bvb TODO 36\evb\evg


\bvg\bva%
„Mik \alst{h}ęfir \alst{H}ęlgi \hld\ \alst{h}ingat sęndan &
við þik, \alst{S}váfa, \hld\ \alst{s}jalfa at mę́la, &
þik kvaðs’k \alst{h}ilmir \hld\ \alst{h}itta vilja &
\alst{á}ðr \alst{í}tr-borinn \hld\ \alst{ǫ}ndu tẏndi.“\eva

\bvb TODO 37\evb\evg


\bvg\bva%
„\edtext{\alst{H}\emph{vat} varð}{\Afootnote{\emph{‘H varþ’} \Regius}} \alst{H}ęlga \hld\ \alst{H}jǫr·varðs syni? &
Mér ’s \alst{h}arð-liga \hld\ \alst{h}arma lęitat; &
ef hann \alst{s}ę́r of lék \hld\ eða \alst{s}verð of bęit, &
þęim skal ek \alst{g}umna \hld\ \alst{g}rand of vinna!“\eva

\bvb TODO 38\evb\evg


\bvg\bva%
„\alst{F}ell hér ï morgun \hld\ at \alst{F}reka-stęini &
\alst{b}uðlungr sá’s vas \hld\ \alst{b}atstr und sólu; &
\alst{Ǫ́}lfr mun sigri \hld\ \alst{ǫ}llum ráða &
þó’tt \alst{þ}ętta sinn \hld\ \alst{þ}ǫrf-gi vę́ri.“\eva

\bvb TODO 39\evb\evg


\bvg\bva%
„\alst{H}ęil ves, Sváfa! \hld\ \alst{H}ug skalt dęila, &
sjá mun ï \alst{h}ęimi \hld\ \alst{h}indstr fundr vera; &
tjá \alst{b}uðlungi \hld\ \alst{b}lǿða undir; &
mér hęfir \alst{h}jǫrr \alst{k}omit \hld\ \alst{h}jarta it nę́sta.\eva

\bvb TODO 40\evb\evg


\bvg\bva%
\alst{B}ið’k þik, Sváfa, \hld\ \alst{b}rúðr, grátt-at-tu, &
ef þú vill \alst{m}ïnu \hld\ \alst{m}áli hlýða &
at þú \alst{H}ęðni \hld\ \alst{h}vílu gørvir &
ok \alst{jǫ}fur \alst{u}ngan \hld\ \alst{ǫ̇}stum lęiðir.“\eva

\bvb TODO 41\evb\evg


\bvg\bva%
„\alst{M}ę́lt hafða’k þat \hld\ ï \alst{m}unar-hęimi &
þȧ’s mér \alst{H}ęlgi \hld\ \alst{h}ringa valði; &
myndi’g-a ek \alst{l}ostig \hld\ at \alst{l}iðinn fylki &
\alst{jǫ}fur \alst{ȯ}·kunnan \hld\ \alst{a}rmi vęrja.“\eva

\bvb TODO 42\evb\evg


\bvg\bva%
„\alst{K}yss mik, Sváfa, \hld\ \alst{k}øm’k ęigi áðr &
\alst{R}og-hęims ȧ vit \hld\ né \alst{R}ǫðuls-fjalla &
áðr \alst{h}ęfnt \alst{h}ęfi’k \hld\ \alst{H}jǫr·varðs sonar, &
þęss es \alst{b}uðlungr vas \hld\ \alst{b}ętstr und sólu!“\eva

\bvb TODO 43\evb\evg


\bpg\bpa%
Helgi ok Sváfa er sagt at vę́ri endr-borin.\epa

\bpb Hallow and Sweve, it is said, were reborn.\epb\epg

\sectionline
