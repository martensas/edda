\bookStart{Lay of Attle}[Atla kviða]
\setBookCode{Atlakvida}

\begin{flushright}%
\textbf{Dating} \parencite{Sapp2022}: C10th (0.719)–early C11th (0.212)

\textbf{Meter:} \Malahattr, \Fornyrdislag
\end{flushright}%

\section{Introduction}

The \textbf{Lay of Attle} (\textlink{Atlakvida}) is only preserved in \Regius.

\Atlakvida\ has long been held to be a particularly archaic poem, although that may have more to do with its style than its actual age, cleaving as it does to the old Germanic epic style of telling the whole story in poetry rather than delegating the verse exclusively to dialogue and otherwise relying on prose to progress the narrative.

In \Regius\ it has the title \emph{Atlakviða in grǿnlęndska} ‘the Greenlandish Lay of Attle’, but the descriptor is probably due to influence \textlink{Atlamal}, which, unlike \Atlakvida, does in fact show signs of a Greenlandish origin.  Greenland was not settled by Icelanders until ca. 985 CE (TODO: cite), by which point \Atlakvida—judging by the highly archaic use of \emph{v-} before \emph{r}—had certainly already been composed.

Together with \textlink{Atlamal}, \Atlakvida\ clearly serves as a source for \VolsungaSaga\ 36–38.  This is particularly evident for sts. 22–35, which are paraphrased closely in \VolsungaSaga\ 37.

\section{The Death of Attle (\emph{Dauði Atla})}

\bpg\bpa Guð·ru̇n Gjúka dóttir hefndi brǿðra sinna, svá sem frę́gt er orðit.  Hón drap fyrst sonu Atla, en eptir drap hón Atla ok brendi hǫll’ina ok hirð’ina alla; um þetta er sjá kviða ort.\epa

\bpb {\huge G}\textsc{uthrun Yivick’s daughter} avenged her brothers as has become renowned. She first slew the sons of Attle, but afterwards she slew Attle and burned the hall and the whole hird.  About that this lay is wrought.\epb\epg

\section{The Lay of Attle}

\bvg\bva%
\alst{A}tli sęndi \hld\ \alst{ǫ́}r til Gunnars &
\alst{k}unnan sęgg at ríða, \hld\ \edtrans{\alst{K}né·frøðr}{Kneefrith}{\Bfootnote{In \textlink{Atlamal} the messenger is called Winge and plays a greater role in the narrative.}} vas sá hęitinn; &
at \alst{g}ǫrðum kom \alst{G}júka \hld\ ok at \alst{G}unnars hǫllu, &
\alst{b}ękkjum arin-gręypum \hld\ ok at \alst{b}jóri svǫ́sum.\eva

\bvb {\huge A}\textsc{ttle sent} a messenger to Guther \\
a well-known man to ride; \inx[P]{Kneefrith} he was called. \\
He came to the yards of Yivick and the hall of Guther, \\
to the hearth-surrounding benches and the splendid beer.\evb\evg


\bvg\bva%
\alst{D}rukku þar \alst{d}rótt-męgir \hld\ —ęn \edtrans{\alst{d}yljęndr}{concealed ones}{\Bfootnote{\textcite{FinnurEdda} reasonably interprets this as referring to Attle’s spies at Guther’s court.}} þǫgðu— &
\alst{v}ïn ï \edtrans{\alst{v}al-hǫllu}{the walhall}{\Bfootnote{The interpretation of this cpd. is difficult in the current context.  The first element \emph{val-} may be (1) \emph{valr} ‘falcon’, referring to aristocratic hunting practices;
(2) \emph{valr} ‘\inx[G]{Wales}[Wale]’, cognate with ‘Welsh’ but in ON referring to the French or Romans, stressing the southern-ness of the hall (cf. st. 15/3 below);
or (3) \emph{valr} ‘(collective) the battle-slain’, foreshadowing the inevitable death (\inx[C]{feyness}) of the \inx[G]{Yivickings}.  If (3) is correct the word is linguistically identical to \inx[L]{Walhall}, Weden’s hall, whither the battle-slain go.}}, \hld\ \emph{\alst{v}}ręiði sǫ́usk þęir Húna; &
\alst{k}allaði þȧ \alst{K}néfrøðr \hld\ \alst{k}aldri rǫddu, &
\alst{s}ęggr inn \alst{s}uð-rǿni \hld\ \alst{s}at ȧ bękk hǫ́um:\eva

\bvb There the dright-lads \ken{warriors} drank—but the concealed ones shut up— \\
wine in the walhall; they feared the wrath of the Huns. \\
Then Kneefrith called out with a cold voice, \\
the southern man, sitting on a high bench:\evb\evg


\bvg\bva%
„\alst{A}tli mik hingat sęndi \hld\ ríða \alst{ø}ręndi, &
\alst{m}ar inum \alst{m}él-gręypa, \hld\ \alst{M}yrk-við inn ȯ·kunna &
at \alst{b}iðja yðr, Gunnarr, \hld\ at it ȧ \alst{b}ękk kǿmið &
með \alst{h}jǫlmum arin-gręypum \hld\ at sǿkja \alst{h}ęim Atla.\eva

\bvb “Attle sent me hither to ride an errand \\
on the bit-champing steed through Mirkwood uncharted— \\
to ask you, O Guther, that ye two \ken*{= Guther and Hain} on the bench come \\
with hearth-surrounding helmets to seek the home of Attle.\evb\evg


\bvg\bva%
\alst{Sk}jǫldu kneguð þar vęlja \hld\ ok \alst{sk}afna aska, &
\alst{h}jalma goll-roðna \hld\ ok \alst{H}u̇na męngi, &
\alst{s}ilfr-gyllt \alst{s}ǫðul-klę́ði, \hld\ \alst{s}ęrki val-rauða, &
\alst{d}afar, \alst{d}arraða, \hld\ \alst{d}rǫsla mél-gręypa.\eva

\bvb There ye might choose shields and shaven ash-spears, \\
helmets gold-reddened and a multitude of Huns, \\
silver-gilt saddle-cloths, blood-red serks, \\
daves, spears, bit-champing steeds.\evb\evg


\bvg\bva%
\alst{V}ǫll léts’k ykkr ok myndu gefa \hld\ \alst{v}íðrar Gnita-hęiðar &
af \alst{g}ęiri \alst{g}jallanda \hld\ ok af \alst{g}ylltum stǫfnum, &
\alst{st}órar męiðmar \hld\ ok \alst{st}aði Danpar, &
hrís þat it \alst{m}ę́ra \hld\ es męðr \alst{M}yrk-við kalla.“\eva

\bvb He too declared he would give you the plain of the wide Gnit-heath, \\
{[and]} yelling spears and gilded prows, \\
great treasures and the court of Danp; \\
the renowned forest which men call Mirkwood.\evb\evg


\bvg\bva%
\alst{H}ǫfði vatt þȧ Gunnarr \hld\ ok \alst{H}ǫgna til sagði: &
„Hvat rę́ðr okkr, \alst{s}ęggr hinn ø̇ri, \hld\ alls vit \alst{s}líkt hęyrum? &
\alst{G}oll vissa’k ękki \hld\ ȧ \alst{G}nita-hęiði, &
þat’s vit \alst{ę́}ttim-a \hld\ \alst{a}nnat slíkt.\eva

\bvb His head turned then Guther and said to Hain: \\
“What dost thou counsel for us two, O younger man, as such things we hear? \\
I knew of no gold on the Gnit-heath \\
of which we two had not owned even as much.\evb\evg


\bvg\bva%
\alst{S}jau ęigu vit \alst{s}al-hús \hld\ \alst{s}verða full, &
\alst{h}vęrju ’ru þęira \hld\ \alst{h}jǫlt ór golli; &
\alst{m}ïnn vęit’k \alst{m}ar bętstan \hld\ en \alst{m}ę́ki hvassastan, &
\alst{b}oga \alst{b}ękk-sǿma \hld\ en \alst{b}rynjur ór golli;\eva

\bvb We own seven hall-houses filled with swords— \\
on each of them is a golden hilt. \\
I know my horse is best and {[my]} sword the sharpest, \\
{[my]} bow is bench-fit and {[my]} byrnies golden,\evb\evg


\bvg\bva%
\alst{h}jalm ok skjǫld \alst{h}vítastan, \hld\ kominn ór \alst{h}ǫll Kíars; &
\alst{ęi}nn ’s mïnn bętri \hld\ en sé \alst{a}llra Húna.“\eva

\bvb {[my]} helmet, and my whitest shield come from Choser’s hall; \\
mine alone is better than those of all of the Huns might be!”\evb\evg


\bvg\bva%
\Ballnote{That it is the more cautious Hain who speaks here is clear from Guther’s response in the following stanzas.  Whereas Hain judges the wolf-hair to be a warning of Hunnish treachery, Guther thinks that it is a warning that wolves will steal his treasure if he does not show up.}%
„Hvat hyggr \alst{b}rúði \alst{b}ęndu \hld\ þȧ’s hǫ̇n okkr \alst{b}aug sęndi, &
\alst{v}arinn \alst{v}ǫ́ðum hęiðingja? \hld\ Hykk at hǫ̇n \alst{v}ǫrnuð byði! &
\alst{H}ár fann’k \alst{h}ęiðingja \hld\ \emph{\alst{v}}riðit ï \alst{h}ring rauðum; &
\alst{y}lfskr es vegr \alst{o}kkarr \hld\ at ríða \alst{ø}ręndi.“\eva

\bvb\speakernoteb{[Hain quoth:]}%
“What dost thou think the bride meant when she sent us a bigh \\
wrapped in a heath-dweller’s \ken{wolf’s} cloth? I think she offered a warning! \\
A heath-dweller’s hair I found tied round the red ring: \\
wolven is our road if we ride that errand!”\evb\evg


\bvg\bva%
\alst{N}iðjar-gi hvǫttu Gunnar \hld\ né \alst{n}ǫ́ungr annarr, &
\alst{r}ýnęndr né \alst{r}áðęndr, \hld\ né þęir’s \alst{r}íkir vǫ́ru; &
\alst{k}vaddi þȧ Gunnarr \hld\ sęm \alst{k}onungr skyldi, &
\alst{m}ę́rr ï \alst{m}jǫð-ranni \hld\ af \alst{m}óði stórum:\eva

\bvb No kinsmen goaded Guther nor any other relation, \\
not counselors nor advisors nor those who were powerful. \\
Then Guther announced—as a king should— \\
renowned in the mead-hall out of great spirit:\evb\evg


\bvg\bva%
„Rís-tu nú, \edtrans{\alst{F}jǫrnir}{Ferner}{\Bfootnote{An otherwise unknown servant.}}, \hld\ lát-tu ȧ \alst{f}lęt vaða &
\alst{g}reppa \alst{g}oll-skálir \hld\ með \alst{g}umna hǫndum!\eva

\bvb “Rise now, Ferner! Let on the benches wade forth \\
the golden bowls of soldiers along the hands of men!\evb\evg


\bvg\bva%
\alst{U}lfr mun ráða \hld\ \alst{a}rfi Niflunga, &
\alst{g}amlir \alst{g}rán-varðir, \hld\ ef \alst{G}unnars missir; &
\alst{b}irnir \alst{b}lakk-fjallir \hld\ \alst{b}íta þref-tǫnnum, &
\alst{g}amna \alst{g}ręy-stóði, \hld\ ef \alst{G}unnarr \edtrans{né kømr-at}{comes not}{\Bfootnote{Note the archaic double negation \emph{né \dots\ -at}; cf. \textlink{Havamal}[135]/4 n.}}.“\eva

\bvb The wolves will rule the patrimony of the Nivlings— \\
old, grey-pelted—if Guther is absent! \\
Black-furred bears will bite with wrangling teeth— \\
amusing the bitch-pack—if Guther comes not!”\evb\evg


\bvg\bva%
\alst{L}ęiddu land-rǫgni \hld\ \edtrans{\alst{l}ýðar ȯ·nęisir}{unshamed \ken{famous} troops}{\Bfootnote{Compare the long-line on the Thorsberg chape (\char`~\ 160–240 AD): \emph{wlþuþewaʀ \hld\ ni wajē-māriʀ} ‘Wolthew, the not ill-famed \ken{famous}’.}}, &
\alst{g}rátęndr, \alst{g}unn-hvatan, \hld\ ór \alst{g}arði Húna; &
þȧ kvað þat inn \alst{ø̇}ri \hld\ \alst{ę}rfi-vǫrðr Hǫgna: &
„\alst{H}ęilir farið nú ok \alst{h}orskir \hld\ hvar’s ykkr \alst{h}ugr tęygir!“\eva

\bvb Unshamed \ken{famous} troops led the lord of the land— \\
weepers—the battle-bold man out of the yards of the Huns. \\
Then quoth this the younger heritance-guardian \ken{son} of Hain: \\
“Fare ye two now whole and wise, wherever your heart may draw you!”\evb\evg


\bvg\bva%
\alst{F}etum létu \alst{f}rǿknir \hld\ of \alst{f}jǫll at þyrja &
\alst{m}ar ina \alst{m}él-gręypu, \hld\ \alst{M}yrk-við inn ȯ·kunna; &
\alst{h}ristisk ǫll \alst{H}ún-mǫrk \hld\ þar’s \alst{h}arð-móðgir fóru, &
\emph{\alst{v}}rǫ́ku þęir \alst{v}and-styggva \hld\ \alst{v}ǫllu al-grǿna.\eva

\bvb With strides the braves made the bit-champing steed \\
rush o’er the fells through Mirkwood uncharted. \\
All Hunmark shook where the hard-minded went forth; \\
they drove the whip-shy horse across all-green fields.\evb\evg


\bvg\bva%
\alst{L}and sǫ́u þęir Atla \hld\ ok \alst{l}ið-skjalfar djúpar; &
\alst{B}ikka greppar standa \hld\ ȧ \alst{b}org inni hǫ́u, &
\alst{s}al of \alst{s}uðr-þjóðum, \hld\ \alst{s}lęginn sess-męiðum, &
\alst{b}undnum rǫndum, \hld\ \alst{b}lęikum skjǫldum,\eva

\bvb The land of Attle they saw and its ravines deep, \\
\inx[P]{Bicke}’s soldiers standing upon the high stronghold, \\
the hall of the southfolk enclosed with seat-beams, \\
with bound rims, with pale shields,\evb\evg


\bvg\bva%
\alst{d}afar, \alst{d}arraða; \hld\ en þar \alst{d}rakk Atli &
\alst{v}ïn ï \alst{v}al-hǫllu; \hld\ \alst{v}ęrðir sǫ́tu úti &
at \alst{v}arða þęim Gunnari \hld\ ef þęir hér \alst{v}itja kǿmi &
með \edtrans{\alst{g}ęiri \alst{g}jallanda}{a yelling spear}{\Bfootnote{Formulaic.  Also found in \textlink{Widsith} TODO.}} \hld\ at vękja \alst{g}ram hildi.\eva

\bvb [they saw] javelins and spears.  But there Attle drank \\
wine in the wal-hall; watchmen stayed outside \\
to watch for Guther’s men, if they came hither to visit \\
with a yelling spear to wake the ruler with war.\evb\evg


\bvg\bva%
\alst{S}ystir fann þęira \alst{s}nemmst \hld\ at þęir ï \alst{s}al kvǫ́mu, &
\alst{b}rǿðr hęnnar \alst{b}áðir, \hld\ \alst{b}jóri vas hǫ̇n lítt drukkin: &
„\alst{R}áðinn est nú, Gunnarr, \hld\ hvat munt, \alst{r}íkr, vinna &
við \alst{H}úna \alst{h}arm-brǫgðum? \hld\ \alst{H}ǫll gakk ór snemma!\eva

\bvb Their sister found soonest that they had come into the hall, \\
her brothers both—on beer was she lightly drunk: \\
“Betrayed art thou now, Guther! What wilt thou, powerful man, win \\
from the Hunnish harm-tricks? Go out of the hall fast!\evb\evg


\bvg\bva%
\alst{B}ętr hęfðir, \alst{b}róðir, \hld\ at ï \alst{b}rynju fǿrir, &
sęm \alst{h}jǫlmum arin-gręypum \hld\ at séa \alst{h}ęim Atla; &
\alst{s}ę́tir ï \alst{s}ǫðlum \hld\ \alst{s}ól-hęiða daga, &
\alst{n}ái \alst{n}auð-fǫlva \hld\ létir \alst{n}ornir gráta,\eva

\bvb Better hadst thou done, brother, if thou hadst marshalled in thy byrnie \\
with hearth-surrounding helmets to see the realm of Attle; \\
if thou hadst loaded in the saddle during sun-bright days \\
need-pale corpses; if thou hadst let the norns cry,\evb\evg


\bvg\bva%
\edtrans{\alst{H}úna skjald-męyjar \hld\ \alst{h}ęrfi kanna}{the Hunnish shield-maidens know the harrow}{\Bfootnote{I.e. if he enslaved the Hunnish shield-maidens as farmhands.}}, &
en \alst{A}tla sjalfan \hld\ létir ï \alst{o}rm-garð koma; &
nú ’s sá \alst{o}rm-garðr \hld\ \alst{y}kkr of folginn.“\eva

\bvb {[and let]} the Hunnish shield-maidens know the harrow— \\
but Attle himself oughtst thou to have put in the snake-pit; \\
now that snake-pit has enveloped you two!”\evb\evg


\bvg\bva%
„\alst{S}ęinat ’s nú, systir, \hld\ at \alst{s}amna Niflungum, &
\alst{l}angt ’s at \alst{l}ęita \hld\ \alst{l}ýða sinnis til, &
of \alst{r}osmu-fjǫll \alst{R}ïnar, \hld\ \alst{r}ekka ȯ·nęissa.“\eva

\bvb\speakernoteb{Guther answers:}%
“It is too late now, sister, to gather the Nivlings. \\
It is long to look for the support of men \\
o’er the red fells of the Rhine, for unshamed \ken{famous} warriors.”\evb\evg


\bvg\bva%
\alst{F}engu þęir Gunnar \hld\ ok ï \alst{f}jǫtur sęttu &
\edtrans{\edtext{vin}{\Afootnote{\emph{vinir} (‘\emph{vın͛}’) \Regius}} \alst{B}orgunda}{the friend of the Burgends}{\Bfootnote{The historic Guther (Latin \emph{Gundaharius} < \emph{*Gunþi·hari}) was a king of Burgundy; cf. \textlink{Waldere} 46, where Walder addresses Guther, whom he is just about to fight, with the identical \emph{wine Burgenda}.
Addressing a king as a \emph{*winiz} ‘friend’ of his people or dynasty is common in early Germanic heroic poetry; cf. \Beowulf\ 30, 170, 1183 et c.: \emph{wine Scyldinga} ‘friend of the Shieldings \ken{danish king}’, 350 \emph{wine Dęniga} ‘friend of the Danes \ken{danish king}’.
The present instance is the only mention of the Burgends in all of Norse literature and as such is an impressive archaism that further adds support to a very early dating for \Atlakvida. — \Regius\ has a small stroke above the \emph{n} that abbreviates the syllable \emph{ir}, indicating the nom. pl. \emph{vinir} ‘friends’, who would then be the people binding Guther.  This is probably due to scribal corruption, since the ones who bind Guther are Huns, not Burgends.}} \hld\ ok \alst{b}undu fast-la; &
\alst{s}jau hjó Hǫgni \hld\ \alst{s}verði hvǫssu &
en inum \alst{á}tta hratt hann \hld\ ï \alst{ę}ld hęitan.\eva

\bvb They captured Guther and in fetters placed \\
the friend of the Burgends \ken*{= Guther} and bound him firmly. \\
Hain smote seven with his sharp sword, \\
but the eighth one he threw in the hot fire.\evb\evg


\bvg\bva%
\Ballnote{The Huns offer Guther (the “ruler of the Gots”, cf. sts. 1, 3, 10) to pay a ransom for Hain’s life.  Guther instead responds with the following.}%
\edtext{Svá skal \alst{f}rǿkn \hld\ \alst{f}jǫ́ndum vęrjas’k;}{\lemma{Svá \dots\ vęrjask}\Bfootnote{Line moved from the last st. to this one since it seems to connect semantically with the immediately following line, and results in two typical four-line stanzas.}} &
\alst{H}ǫgni varði \hld\ \alst{h}ęndr Gunnars. &
\alst{f}rǫ́gu \alst{f}rǿknan \hld\ ef \alst{f}jǫr vildi &
\alst{G}otna þjóðann \hld\ \alst{g}olli kaupa.\eva

\bvb So shall a brave guard himself against foes; \\
Hain guarded the hands of Guther. \\
They asked the brave \ken*{Guther} if his \ken*{Hain’s} life he wished— \\
the ruler of the Gots \ken*{Guther}—to buy with gold.\evb\evg


\bvg\bva%
„\alst{H}jarta skal mér \alst{H}ǫgna \hld\ ï \alst{h}ęndi liggja &
\alst{b}lóðugt, ór \alst{b}rjósti \hld\ skorit \alst{b}ald-riða, &
\edtrans{\alst{s}axi \alst{s}líðr-bęitu}{slide-biting sax}{\Bfootnote{A short-sword with a blade so sharp that it draws blood when one slides the finger across it.}}, \hld\ \alst{s}yni þjóðans.“\eva

\bvb “The heart of Hain shall lie in my hands: \\
bloody, cut from the breast of the bold rider \ken*{= Hain}, \\
with a slide-biting sax from the son of the sovereign \ken*{= Hain}.”\evb\evg


\bvg\bva%
Skǫ́ru þęir \alst{h}jarta \hld\ \alst{H}jalla ór brjósti, &
\alst{b}lóðugt, ok ȧ \alst{b}jóð lǫgðu \hld\ ok \alst{b}ǫ́ru þat fyr Gunnar.\eva

\bvb They cut the heart of Helle from the breast, \\
bloody, and on a plate laid it, and brought it before Guther.\evb\evg


\bvg\bva%
Þȧ kvað þat \alst{G}unnarr, \hld\ \alst{g}umna dróttinn: &
„\alst{H}ér hęfi’k \alst{h}jarta \hld\ \alst{H}jalla ins blauða, &
ȯ·líkt \alst{h}jarta \hld\ \alst{H}ǫgna ins frǿkna, &
es mjǫk \alst{b}ifask \hld\ es ȧ \alst{b}jóði liggr; &
\alst{b}ifðisk hǫlfu męirr \hld\ es ï \alst{b}rjósti lá!“\eva

\bvb Then quoth this Guther, the lord of men: \\
“Here have I the heart of Helle the soft—unlike the heart of Hain the bold!— \\
which quivers greatly when on the plate it lies; \\
it quivered twice as much when in the breast it lay.”\evb\evg


\bvg\bva%
\alst{H}ló þȧ \alst{H}ǫgni \hld\ es til \alst{h}jarta skǫ́ru &
\alst{k}vikvan \alst{k}umbla-smið \hld\ —\alst{k}løkkva síðst hugði. &
\alst{B}lóðugt þat ȧ \alst{b}jóð lǫgðu \hld\ ok \alst{b}ǫ́ru fyr Gunnar.\eva

\bvb Hain then laughed as to the heart they cut \\
the living wound-smith \ken*{\textsc{warrior} = Hain}; he thought least of sobbing. \\
Bloody on a plate they laid it, and brought it before Guther.\evb\evg


\bvg\bva%
Mę́rr kvað þat \alst{G}unnarr, \hld\ \alst{g}ęir-Niflungr: &
„\alst{H}ér hęfi’k \alst{h}jarta \hld\ \alst{H}ǫgna ins frǿkna, &
ȯ·líkt \alst{h}jarta \hld\ \alst{H}jalla ins blauða, &
es lítt \alst{b}ifask \hld\ es ȧ \alst{b}jóði liggr; &
\alst{b}ifðisk svá-gi mjǫk \hld\ þȧ’s ï \alst{b}rjósti lá!\eva

\bvb Renowned Guther quoth this, the Spear-Nivling: \\
“Here have I the heart of Hain the bold \\
—unlike the heart of Helle the soft!— \\
which quivers lightly when on the plate it lies; \\
it quivered not so much when in the breast it lay.\evb\evg


\bvg\bva%
Svá skalt, \alst{A}tli, \hld\ \alst{au}gum fjarri &
sęm \alst{m}unt \hld\ \alst{m}ęnjum verða; &
es und \alst{ęi}num mér \hld\ \alst{ǫ}ll of folgin &
\alst{h}odd Niflunga: \hld\ lifir-a nú \alst{H}ǫgni!\eva

\bvb Thus shalt thou, Attle, be as far from eyes \\
as thou wilt from the neck-rings be. \\
Under me alone is hidden all \\
the hoard of the Nivlings—now Hain lives not!\evb\evg


\bvg\bva%
Ęy vas mér \alst{t}ýja \hld\ meðan vit \alst{t}vęir lifðum, &
nú ’s mér \alst{ę}ngi \hld\ es \alst{ęi}nn lifi’k; &
\alst{R}ïn skal \alst{r}áða \hld\ \alst{r}óg-malmi skatna, &
svinn, \alst{ǫ̇}s-kunna \hld\ \alst{a}rfi Niflunga.\eva

\bvb I always had doubt while we two lived; \\
now I have none when I alone live. \\
The Rhine shall rule the strife-ore of princes \ken{gold}, \\
the swift [river], the os-born inheritance of the Nivlings!\evb\evg


\bvg\bva%
İ \alst{v}eltanda \alst{v}atni \hld\ lýsask \alst{v}al-baugar &
\alst{h}ęldr an ȧ \alst{h}ǫndum goll \hld\ skïni \alst{H}úna bǫrnum.“\eva

\bvb In tumbling water shall the Welsh bighs gleam, \\
rather than gold shine on the hands of the children of Huns!”\evb\evg


\bvg\bva%
“Ýkvið ér \alst{h}vél-vǫgnum, \hld\ \alst{h}aptr ’s nú ï bǫndum!”\eva

\bvb “Turn ye the wheel-wagons, the captive is now in bonds!”\evb\evg


\bvg\bva%
Atli inn \alst{r}íki \hld\ \alst{r}ęið Glaum mǫnum, &
\alst{s}lęginn róg-þornum, \hld\ \alst{s}ifjungr þęira; &
{[...]} \hld\ Guð·ru̇n sig-tíva &
\alst{v}arnaði við tǫ́rum, \hld\ \alst{v}aðin ï þys-hǫllu:\eva

\bvb TODO\evb\evg


\bvg\bva%
„Svá \alst{g}angi þér, Atli, \hld\ sęm þú við \alst{G}unnar áttir &
\alst{ęi}ða opt of svarða \hld\ ok \alst{á}r of nęfnda &
at \alst{s}ól inni \alst{s}uðr-hǫllu \hld\ ok at \alst{S}ig-týs bergi, &
\alst{h}ulkvi \alst{h}víl-bęðjar \hld\ ok at \alst{h}ringi Ullar,\eva

\bvb “So may it go for thee, Attle, like thou with Guther hadst \\
oaths oft sworn and always mentioned \\
by the south-facing sun and by Victory-Tew’s mountain, \\
in any pleasant bed and by the ring of Woulder,\evb\evg


\bvg\bva%
ok \alst{m}ęirr þaðan \hld\ \alst{m}ęn-vǫrð bituls, &
\alst{d}olg-rǫgni, \alst{d}ró \hld\ til \alst{d}auðs skókr.\eva

\bvb TODO\evb\evg


\bvg\bva%
\Ballnote{Guther was laid in the snake-pit and played a harp as his life expired.  This image was very famous and is depicted pictorally in archeological finds and stave church carvings.}%
\alst{L}ifanda gram \hld\ \alst{l}agði ï garð, &
þann’s \alst{sk}riðinn vas, \hld\ \alst{sk}atna męngi, &
\alst{i}nnan \alst{o}rmum. \hld\ En \alst{ęi}nn Gunnarr &
\alst{h}ęipt-móðr \alst{h}ǫrpu \hld\ \alst{h}ęndi kníði; &
\alst{g}lumðu stręngir. \hld\ Svá skal \alst{g}olli &
\alst{f}rǿkn hring-drifi \hld\ við \alst{f}ira halda!\eva

\bvb Living, the lord \ken*{= Guther} was laid in the enclosure \\
(which was crawling) by a troop of warriors \\
(with snakes inside).  But Guther alone \\
spitefully struck a harp with his hand; \\
its strings rang out.  \emph{So} shall a brave \\
ring-strewer \ken{king} keep his gold from men!\evb\evg


\bvg\bva%
Atli \alst{l}ét \hld\ \alst{l}ands sïns ȧ vit &
\alst{jó} \edtrans{\alst{ø}r·skáan}{watchful}{\Bfootnote{A hapax, best explained as a cognate with Gothic \emph{us-skaws} ‘vigilant, watchful’.}} \hld\ \alst{a}ptr frȧ morði; &
\alst{d}ynr vas ï garði, \hld\ \alst{d}rǫslum of þrungit, &
\alst{v}ápn-sǫngr \alst{v}irða— \hld\ \alst{v}ǫ́ru af hęiði komnir.\eva

\bvb Attle turned towards his land \\
on his watchful steed back from the murder. \\
There was a din in the yard from the trampling horses, \\
the weapon-song of warriors—they were come from the heath.\evb\evg


\bvg\bva%
\alst{Ú}t gekk þȧ Guðru̇n, \hld\ \alst{A}tla ï gǫgn, &
með \alst{g}ylltum kalki \hld\ at ręifa \alst{g}jǫld rǫgnis: &
„\alst{Þ}iggja knátt, \alst{þ}ęngill, \hld\ ï \alst{þ}ïnni hǫllu &
\alst{g}laðr at \alst{G}uð-ru̇nu \hld\ \alst{g}nadda nifl-farna.“\eva

\bvb Out went Guthrun Attle to face \\
with a gilt chalice to cheer the lord. \\
“Thou mightst accept, ruler, in thy hall, \\
glad, from Guthrun, young beasts gone to the shades!”\evb\evg


\bvg\bva%
\alst{U}mðu \alst{ǫ}l-skálir \hld\ \alst{A}tla vïn-hǫfgar &
þȧ’s ï \alst{h}ǫll saman \hld\ \alst{H}únar tǫlðusk, &
\alst{g}umar \alst{g}ran-síðir \hld\ \alst{g}engu inn hvárir.\eva

\bvb The ale-bowls of Attle clanged heavy with wine \\
when in the hall together the Huns were counted, \\
the long-bearded men walked in two by two.\evb\evg


\bvg\bva%
\alst{Sk}ę́vaði þȧ hin \alst{sk}ír-lęita, \hld\ \edtrans{vęigar}{draughts}{\Bfootnote{The alliteration requires a word starting with \emph{sk-}.  For this reason \emph{vęigar} probably ought to be replaced with the synonym \emph{skálir} ‘bowls’.}} þęim at bera, &
\alst{a}f-kǫ́r dís, \alst{jǫ}frum, \hld\ ok \alst{ǫ}l-krásir valði, &
\alst{n}auðug, \alst{n}ęf-fǫlum, \hld\ en \edtrans{\alst{n}íð}{nithe}{\Bfootnote{An evil, cursing word.}} sagði Atla:\eva

\bvb Forth she \ken*{= Guthrun} strode, pure-faced, bearing them draughts, \\
the violent dise to the princes, and chose the ale-dainties, \\
forced, for the pale-nosed men—but she told her nithe to Attle.\evb\evg


\bvg\bva%
„\alst{S}ona hęfir þïnna, \hld\ \alst{s}verða deilir, &
\alst{h}jǫrtu \alst{h}rę́-dręyrug \hld\ við \alst{h}unang of tuggin, &
\alst{m}ęlta knátt, \alst{m}óðugr, \hld\ \alst{m}anna val-bráðir &
\alst{e}ta at \alst{ǫ}l-krǫ́sum \hld\ ok ï \alst{ǫ}nd-ugi at sęnda.\eva

\bvb “Dealer of swords! Thou hast thy own sons’ \\
corpse-bloody hearts with honey chewed. \\
Thou art stomaching, fierce man, the death-flesh of men, \\
eating it by ale-dainties, passing it on from the high seat.\evb\evg


\bvg\bva%
\alst{K}allar-a þú síðan \hld\ til \alst{k}néa þïnna &
\alst{E}rp né \alst{Ęi}til, \hld\ \edtrans{\alst{ǫ}l-ręifa tvȧ}{the ale-rowdy two}{\Bfootnote{At this time there was nothing unusual about young boys drinking.}}; &
\alst{s}ér-a þú \alst{s}íðan \hld\ ï \alst{s}eti miðju &
\alst{g}olls miðlęndr \hld\ \alst{g}ęira skępta, &
\alst{m}anar \alst{m}ęita \hld\ né \alst{m}ara kęyra.“\eva

\bvb Thou wilt not henceforth call up to thy knees \\
Earp nor Oatle, the ale-rowdy two! \\
Thou wilt not henceforth see in the middle of the seat \\
the dealers of gold shafting spears \\
brushing horse-manes or driving steeds.”\evb\evg


\bvg\bva%
\alst{Y}mr varð ȧ bękkjum, \hld\ \alst{a}f-kárr sǫngr virða, &
\alst{g}nýr \edtrans{und \alst{g}uð-vęfjum}{beneath the god-weave}{\Bfootnote{Beneath the silken fabric, presumably of the tents in which the nomadic Huns dwelled.}}, \hld\ \alst{g}rétu \edtrans{bǫrn Hu̇na}{the children of the Huns}{\Bfootnote{Here just meaning “the Huns”; cf. “the children of men”.}}, &
nema \alst{ęi}n Guðru̇n \hld\ es hǫ̇n \alst{ę́}va grét &
\alst{b}rǿðr sïna \edtrans{\alst{b}er-harða}{bear-hard}{\Bfootnote{Before the lion was adopted for this sake on the basis of Classical and Biblical models, the bear was the animal associated with strength and bravery in the North. — \emph{ber-} is a compounding form of \emph{*beri} ‘bear’, an otherwise unattested masc. \emph{n}-stem noun inherited from PGmc. \emph{*berô}, whence also OHG \emph{bero}, OE \emph{bera} ‘id.’  The normal ON word for “bear” is \emph{bjǫrn}, an \emph{u}-stem derived from the oblique cases of \emph{*berô}; there also survive the fem. \emph{bera} ‘she-bear’ and diminutive \emph{bersi} ‘(playful) bear’.  \emph{ber-} appears to be an archaism, since it is only otherwise attested in \textlink{Volundarkvida} 11.}} \hld\ ok \alst{b}uri svása, &
\alst{u}nga, \alst{ȯ}·fróða, \hld\ þȧ’s hǫ̇n við \alst{A}tla gat.\eva

\bvb There was clangour on the benches, violent song of warriors, \\
noise beneath the \inx[C]{god-weave}—the children of the Huns wept, \\
except Guthrun alone, for she never wept \\
for her bear-hard brothers and beloved sons, \\
the young, unlearned, which she with Attle begot.\evb\evg


\bvg\bva%
\alst{G}olli søri \hld\ hin \alst{g}agl-bjarta, &
\alst{h}ringum rauðum \hld\ ręifði hǫ̇n \alst{h}ús-karla; &
\alst{sk}ǫp lét hǫ̇n vaxa \hld\ en \alst{sk}íran malm vaða, &
ę́va \alst{f}ljóð ękki \hld\ gáði \alst{f}jarg-húsa.\eva

\bvb With gold the goose-bright lady sowed [the floor]; \\
with red rings she cheered the housecarls. \\
She let the large vats grow and the pure metal wade;
never did that woman heed the godhouses.\evb\evg


\bvg\bva%
\alst{Ȯ}·varr \alst{A}tli \hld\ \edtrans{*\alst{ó}ðan}{mad}{\Afootnote{emend.; \emph{móðan} ‘tired’ \Regius}\Bfootnote{A word alliterating with a vowel is required by the meter and \emph{*óðan} ‘mad’ lies closest at hand, differing only by a letter from \emph{móðan} ‘tired’.}} hafði sik drukkit; &
\alst{v}ǫ́pn hafði hann ękki, \hld\ \alst{v}arnaði-t við Guð·ru̇nu; &
opt vas sá \alst{l}ęikr bętri \hld\ þȧ’s þau \alst{l}int skyldu &
\alst{o}ptarr of faðmas’k \hld\ fyr \alst{ǫ}ðlingum.\eva

\bvb Unwary, Attle had drunk himself mad; \\
he had no weapon, did not beware Guthrun. \\
Oft their play was better when they would gently \\
oftener embrace each other before the athlings.\evb\evg


\bvg\bva%
Hǫ̇n \alst{b}ęð \alst{b}roddi \hld\ gaf \alst{b}lóð at drekka, &
\alst{h}ęndi \alst{h}ęl-fu̇ssi, \hld\ ok \alst{h}velpa lęysti; &
\alst{h}ratt fyr \alst{h}allar dyrr \hld\ ok \alst{h}ús-karla vakði, &
\alst{b}randi, \alst{b}rúðr, hęitum; \hld\ þau lét hǫ̇n gjǫld \alst{b}rǿðra.\eva

\bvb With a blade she gave the bed blood to drink— \\
with a hell-eager hand—and set loose the whelps, \\
blocked the doors of the hall and awoke the housecarls— \\
the bride— with hot flame; such were her repayments for her brothers!\evb\evg


\bvg\bva%
\alst{Ę}ldi gaf hǫ̇n \alst{a}lla \hld\ es \alst{i}nni vǫ́ru &
ok frȧ \alst{m}orði þęira Gunnars \hld\ komnir vǫ́ru ór \alst{M}yrk-hęimi; &
\alst{f}orn timbr \alst{f}ellu, \hld\ \alst{f}jarg-hús ruku, &
\alst{b}ǿr \alst{B}uðlunga, \hld\ \alst{b}runnu ok skjald-męyjar, &
\alst{i}nni; \alst{a}ldr-stamar \hld\ hnigu ï \alst{ę}ld hęitan.\eva

\bvb To the fire she gave all who were within \\
and from their murder of Guther had come out of Mirkham. \\
Ancient timbers fell, the god-houses smoked— \\
the settlement of the Budlungs.  The shield–maidens too burned \\
inside; short-lived, they sank into hot fire.\evb\evg


\bvg\bva%
\alst{F}ull-rǿtt’s umb þetta; \hld\ \alst{f}ęrr ęngi svá síðan &
\alst{b}rúðr ï \alst{b}rynju \hld\ \alst{b}rǿðra at hęfna; &
hǫ̇n hęfir \alst{þ}riggja \hld\ \alst{þ}jóð-konunga &
\edtrans{\alst{b}an-orð \alst{b}orit}{borne the bane-word}{\Bfootnote{I.e. “she has caused the deaths of three great kings.” This expression is discussed along with its Germanic and Indo-European relatives in detail in \textcite{Watkins1995}[417--422].}}, \hld\ \alst{b}jǫrt, áðr sylti.\eva

\bvb It is told fully about this: henceforth no one will go so, \\
a bride in byrnie her brothers to avenge. \\
She has, bright, of three great kings \\
borne the bane-word before she must die.\evb\evg


\bvg\bva%
Enn segir gløggra í Atla-mǫ́lum inum grǿn-lenskum.\eva

\bvb But this tale is told even more clearly in the Greenlandish Speeches of Attle.\evb\evg

\sectionline
