\bookStart{Fragments from the Saw of the Walsings}
\setBookCode{Lacuna}

\section{Introduction}

In \Regius, \textlink{Sigrdrifumal} ends abruptly at stanza 27 due to the loss of a number of pages; the so-called \textbf{Great Lacuna}.  The poetry contained on these pages undoubtedly belonged to the Walsing cycle and would have dealt at length with the life of Siward.

The author of \VolsungaSaga\ drew heavily from a collection of Walsing-cycle poetry closely related to \Regius.  He quotes many stanzas known from \Regius, but also some which do not survive anywhere else—it is these which are edited here.  They largely correspond to the story which would have been found in the Great Lacuna and it is probable that most of them derive from the now-lost poems.

The stanzas are numbered based on their occurrence in \VolsungaSaga.

\sectionline

\bvg\bva%
Ristu af \alst{m}agni \hld\ \alst{m}ikla hellu, &
\alst{S}ig·mundr hjǫrvi \hld\ ok \alst{S}in·fjǫtli.\eva

\bvb They carved with strength the great stone, \\
\inx[P]{Syemund} with sword, and \inx[P]{Sinfittle}.\evb\evg

\sectionline

\bvg\bva%
\alst{Ę}ldr nam at \alst{ǿ}sask \hld\ en \alst{jǫ}rð at skjalfa &
ok \alst{h}ár logi \hld\ við \alst{h}imni gnę́fa; &
\alst{f}ár tręystisk þar \hld\ \alst{f}ylkis rekka &
\alst{ę}ld at ríða \hld\ né \alst{y}fir stíga.\eva

\bvb The fire took to rage and the earth to shake \\
and high flame to rise to heaven. \\
Few there dared of the marshall’s champions \\
the fire to ride or to step over it.\evb\evg


\bvg\bva%
\alst{S}ig·urðr Grana \hld\ \alst{s}verði kęyrði; &
\alst{ę}ldr sloknaði \hld\ fyr \alst{ǫ}ðlingi; &
\alst{l}ogi allr \alst{l}ę́gðisk \hld\ fyr \alst{l}of-gjǫrnum; &
bliku \alst{r}ęiði, \hld\ es \alst{R}eginn átti.\eva

\bvb Siward drove Grane on by his sword; \\
the fire went out before the athling; \\
the flame all lowered before the praise-eager man; \\
the harness flashed which Rein had owned.\evb\evg

\sectionline

\bvg\bva%
\alst{S}ig·urðr vá at ormi, \hld\ en þat \alst{s}íðan mun &
\alst{ø}ngum fyrnask, \hld\ meðan \alst{ǫ}ld lifir. &
En \alst{h}lýri þïnn \hld\ \alst{h}várki þorði &
\alst{ę}ld at ríða \hld\ né \alst{y}fir stíga.\eva

\bvb Siward smote the Wyrm and that will afterwards \\
by none be forgotten while mankind lives— \\
but \emph{thy brother} dared neither \\
the fire to ride nor to step over it.\evb\evg

\sectionline

\bvg\bva%
\alst{Ú}t gekk Sig·urðr \hld\ \alst{a}nn·spjalli frȧ, &
\alst{h}oll-vinr lofða, \hld\ ok \alst{h}nípaði, &
svá at \alst{g}anga nam \hld\ \alst{g}unnar-fúsum &
\alst{s}undr of \alst{s}íður \hld\ \alst{s}erkr járn-ofinn.\eva

\bvb TODO: translation.\evb\evg


\bvg\bva%
\Ballnote{This st. is apparently related to \textlink{Brot}[4], with which it shares (in divergent form) its first two lines.  It is cited in \VolsungaSaga\ 30, where it is introduced by the words \emph{sem skáldit kvað:} ‘as the scald quoth:’}%
Sumir \alst{v}ið-fiska tóku, \hld\ sumir \alst{v}itnis-hrę́ skífðu, &
sumir \alst{G}utt·ormi \alst{g}ǫ́fu \hld\ \alst{g}ęra hold &
við \alst{m}un-gáti \hld\ ok \alst{m}arga hluti &
aðra ï tyfrum.\eva

\bvb Some took wood-fishes \ken{serpents}, some sliced wolf-carrion, \\
some gave Godthorm hound-flesh \\
with strong drink and many other \\
things in magic potions.\evb\evg


\sectionline
