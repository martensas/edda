\bookStart{Second Lay of Hallow Hundingsbane}[II Helga kviða Hundings-bana]
\setBookCode{HelgakvidaTwo}

\begin{flushright}%
\textbf{Dating} \parencite{Sapp2022}: late C11th (0.587)

\textbf{Meter:} \Fornyrdislag\ (TODO)%
\end{flushright}

\section{Introduction}

TODO: Introduction.

The latter part of the poem contains a touching description of Syreun’s visit to Hallow’s grave mound, where the two lovers embrace one last time.  It reflects a folkloric motif found in many traditional British ballads (e.g. Roud 50 “Sweet William’s Ghost˝, Roud 179 “The Lover’s Ghost” or “The Grey Cock”, Roud 22568 “The Night Visiting Song”), where two lovers separated by death reunite one last time in the grave before being forced to part at cock-crow, although in some variants of 179 and 22568 the supernatural element is not explicit.  The following text is a version of Roud 22568 as recorded by \emph{The Dubliners} in 1972:

\begin{quote}\begin{small}\itshape I must away now; I can no longer tarry \\
This morning’s tempest I have to cross \\
I must be guided without a stumble \\
Into the arms I love the most. \\

And when he came to his true love’s dwelling \\
He knelt down gently upon a stone \\
And through her window he’s whispered lowly: \\
“Is my true lover within at home?” \\

“Wake up, wake up, love, it is thine own true lover \\
Wake up, wake up, love, and let me in \\
For I am tired, love, and oh so weary \\
And more than near drenched to the skin.” \\

She’s raised her off her down soft pillow \\
She’s raised her up and she’s let him in \\
And they were locked in each other’s arms \\
Until that long night was past and gone. \\

And when that long night was past and over \\
And when the small clouds began to grow \\
He’s taken her hand and they’ve kissed and parted \\
Then he saddled and mounted and away did go. \\

I must away now \emph{et c.}\end{small}\end{quote}

\sectionline

\section{From the Walsings (\emph{Frá Vǫlsungum})}

\bpg\bpa Sig·mundr konungr Vǫlsungs sonr átti Borg·hildi af Brá-lundi.  Þau hétu son sinn Helga ok eptir Helga Hjǫr·varðs syni.  Helga fóstraði Hagall.  Hundingr hét ríkr konungr; við hann er Hund-land kennt.  Hann var her-maðr mikill ok átti marga sonu þá er í hernaði vǫ́ru.  Ó·friðr ok dylgjur vóru á milli þeira Hundings konungs ok Sig·mundar konungs; drǫ́pu hvárir annarra frę́ndr.  Sig·mundr konungr ok hans ę́tt-menn hétu Vǫlsungar ok ylfingar.  Helgi fór ok njósnaði til hirðar Hundings konungs á laun.  Hemingr, sonr Hundings konungs, var heima.  En er Helgi fór í brott þá hitti hann hjarðar \emph{svein} ok kvað:\epa

\bpb TODO.\epb\epg


\bvg\bva „Sęg \alst{H}ęmingi \hld\ at \alst{H}ęlgi man &
hvęrn ï \alst{b}rynju \hld\ \alst{b}ragnar fęlldu, &
\alst{é}r \alst{u}lf grá\emph{a}n \hld\ \alst{i}nni hǫfðuð &
þar’s \alst{H}amal \alst{h}ugði \hld\ \alst{H}undingr konungr.“\eva

\bvb TODO.\evb\evg


\bpg\bpa Hamall hét sonr Hagals.  Hundingr konungr sendi menn til Hagals at leita Helga.  En Helgi mátti eigi forðask annan veg en tók klę́ði ambáttar ok gekk at mala.  Þeir leituðu ok fundu eigi Helga.  Þá kvað Blindr inn bǫl-vísi:\epa

\bpb TODO.\epb\epg


\bvg\bva
„\alst{H}vǫss eru augu \hld\ ï \alst{H}agals þýju, &
es⸗a þat \alst{k}arls ę́tt \hld\ es ȧ \alst{k}vęrnum stęndr, &
\alst{st}ęinar rifna, \hld\ \alst{st}økkr lúðr fyrir.\eva

\bvb TODO.\evb\evg


\bvg\bva Nú hęfir \alst{h}ǫrð dǿmi \hld\ \alst{h}ildingr þegit &
es \alst{v}ísi skal \hld\ \alst{v}al-bygg mala; &
\alst{h}ęldr es sǿmri \hld\ \alst{h}ęndi þęiri &
\alst{m}eðal-kafli \hld\ an \alst{m}ǫndul-tré.“\eva

\bvb TODO.\evb\evg


\bpg\bpa Hagall svaraði ok kvað:\epa

\bpb TODO.\epb\epg


\bvg\bva
„Þat ’s \alst{l}ítil vǫ́ \hld\ þó’tt \alst{l}úðr þrumi &
es \alst{m}ę́r konungs \hld\ \alst{m}ǫndul hrǿrir; &
hǫ̇n \alst{sk}ę́vaði \hld\ \alst{sk}ýjum øfri &
ok \alst{v}ega þorði \hld\ sęm \alst{v}íkingar &
áðr \alst{h}ana \alst{H}elgi \hld\ \alst{h}ǫptu gørði; &
\alst{s}ystir ’s hǫ̇n þęira \hld\ \alst{S}igars ok Hǫgna, &
því hęfir \alst{ǫ}tul \alst{au}gu \hld\ \alst{y}lfinga man.“\eva

\bvb TODO.\evb\evg


\bpg\bpa Undan komsk Helgi ok fór á her-skip.  Hann felldi Hunding konung ok var síðan kallaðr Helgi Hundings bani.  Hann lá með her sinn í Bruna-vágum ok hafði þar strand-hǫgg ok ǫ́tu þar hrátt.  Hǫgni hét konungr; hans dóttir var Sig·rún; hón var val-kyrja ok reið lopt ok lǫg; hón var Svá\emph{va} endr-borin.  Sig·rún reið at skipum Helga ok kvað:\epa

\bpb TODO.  Syerun rode up to Hallow’s ships and quoth:\epb\epg


\bvg\bva%
„Hvęrir láta \alst{f}ljóta \hld\ \alst{f}lęy við bakka? &
\alst{H}var, \alst{h}ęr-męgir, \hld\ \alst{h}ęima ęiguð? &
Hvęrs \alst{b}íðið ér \hld\ ï \alst{B}runa-vǫ́gum? &
Hvęrt \alst{l}ystir yðr \hld\ \alst{l}ęið at kanna?“\eva

\bvb TODO.\evb\evg


\bvg\bva%
„Hamall lę́tr \alst{f}ljóta \hld\ \alst{f}lęy við bakka; &
ęigum \alst{h}ęima \hld\ ï \alst{H}lés-eyju; &
\alst{b}íðum \alst{b}yrjar \hld\ ï \alst{B}runa-vǫ́gum; &
austr \alst{l}ystir oss \hld\ \alst{l}ęið at kanna!“\eva

\bvb TODO.\evb\evg


\bvg\bva%
„\alst{H}var hęfir, \alst{h}ilmir, \hld\ \alst{h}ildi vakða &
eða \alst{g}ǫgl alin \hld\ \alst{G}unna\emph{r} systra? &
Hví ’s \alst{b}rynja þín \hld\ \alst{b}lóði stokkin? &
\alst{H}ví skal und \alst{h}jǫlmum \hld\ \alst{h}rátt kjǫt eta?“\eva

\bvb TODO.\evb\evg


\bvg\bva%
„Þat vann \alst{n}ę́st \alst{n}ýs \hld\ \alst{n}iðr Ylfinga &
fyr \alst{v}estan \alst{v}er, \hld\ \edtrans{ef þik \alst{v}ita lystir}{if thou hast lust to know}{\Bfootnote{Formulaic b-verse, also occurring in \textlink{Helreid}[2]/3b, 5/2b, 6/3b.}}, &
es ek \alst{b}jǫrnu tók \hld\ ï \alst{B}raga-lundi &
ok \alst{ę́}tt \alst{a}ra \hld\ \alst{o}ddum sadda’k.\eva

\bvb TODO.\evb\evg


\bvg\bva%
Nú ’s \alst{s}agt, mę́r, \hld\ hvaðan \alst{s}akar gørðusk, &
því vas ȧ \alst{l}ęgi mér \hld\ \alst{l}ítt stęikt etit.“\eva

\bvb TODO.\evb\evg


\bvg\bva%
„\alst{V}íg lýsir þú; \hld\ \alst{v}arð fyr Hęlga &
\alst{H}undingr konungr \hld\ \alst{h}níga at vęlli; &
bar \alst{s}ókn \alst{s}aman \hld\ es \alst{s}efa hęfnduð &
ok \alst{b}usti \alst{b}lóð \hld\ ȧ \alst{b}rimis ęggjar.“\eva

\bvb TODO.\evb\evg


\bvg\bva%
„Hvat vissir \alst{þ}ú \hld\ at \alst{þ}ęir séi, &
\alst{s}nót \alst{s}vinn-huguð, \hld\ es \alst{s}efa hęfnduð? &
Margir ’ro \alst{h}vassir \hld\ \alst{h}ildings synir &
ok \alst{ȧ}·munir \hld\ \alst{o}ssum niðjum.“\eva

\bvb TODO.\evb\evg


\bvg\bva%
„Vas’k⸗a ek \alst{f}jarri, \hld\ \alst{f}olks odd-viti, &
\alst{g}ę́r ȧ morgun \hld\ \alst{g}rams aldr-lokum; &
þó tęl’k \alst{s}lǿgjan \hld\ \alst{S}ig·mundar bur &
es ï \alst{v}al-ru̇num \hld\ \alst{v}íg-spjǫll sęgir.\eva

\bvb TODO.\evb\evg


\bvg\bva%
\alst{L}ęit’k þik umb sinn fyrr \hld\ ȧ \alst{l}ang-skipum &
þȧ’s þú \alst{b}yggðir \hld\ \alst{b}lóðga stafna &
ok \alst{ú}r-svalar \hld\ \alst{u}nnir léku; &
nú vill \alst{d}ylja⸗sk \hld\ \alst{d}ǫglingr fyr mér &
en \alst{H}ǫgna mę́r \hld\ \emph{\alst{H}ęlga} kęnnir.“\eva

\bvb TODO.\evb\evg


\bpg\bpa%
Gran·marr hét ríkr konungr er bjó at Svarins-haugi.  Hann átti marga sonu: Hǫð·broddr, annarr Guð·mundr, þriði Starkaðr.  Hǫð·broddr var í konunga-stefnu.  Hann fastnaði sér Sig·rúnu Hǫgna dóttur en er hón spyrr þat þá reið hón með valkyrjur um lopt ok um lǫg at leita Helga. \\
Helgi var þá at Loga-fjǫllum ok hafði barit⸗sk við Hundings sonu.  Þar felldi hann þá Álf ok Eyj·\emph{ó}lf, Hjǫr·varð ok Her·varð ok var hann all-víg-móðr ok sat undir Ara-steini.  Þar hitti Sigrún hann ok rann á hals hǫ́num ok kyssti hann ok sagði hǫ́num erendi sitt, svá sem segir í Vǫlsunga kviðu inni fornu:\epa

\bpb TODO.\epb\epg


\bvg\bva%
\alst{S}ótti \alst{S}ig·ru̇n \hld\ \alst{s}ikling glaðan, &
\alst{h}ęim nam hǫ̇n \alst{H}ęlga \hld\ \alst{h}ǫnd at sǿkja, &
\alst{k}yssti ok \alst{k}vaddi \hld\ \alst{k}onung und hjalmi, &
þȧ varð \alst{h}ilmi \hld\ \alst{h}ugr ȧ vífi; &
fyrr lét⸗sk hǫ̇n \alst{u}nna \hld\ af \alst{ǫ}llum hug &
\alst{s}yni \alst{S}ig·mundar \hld\ an hǫ̇n \alst{s}ét hafði.\eva

\bvb TODO.\evb\evg


\bvg\bva
„Vas’k \alst{H}ǫð·broddi \hld\ ï \alst{h}ęr fǫstnuð &
en \alst{jǫ}fur \alst{a}nnan \hld\ \alst{ęi}ga vilda’k; &
þó séumk, \alst{f}ylkir, \hld\ \alst{f}rę́nda ręiði; &
hęfi’k \alst{m}ïns fǫður \hld\ \alst{m}un-ráð brotit.“\eva

\bvb TODO.\evb\evg


\bvg\bva%
Nam⸗a \alst{H}ǫgna mę́r \hld\ of \alst{h}ug mę́la; &
\alst{h}afa kvað⸗sk hǫ̇n \alst{H}ęlga \hld\ \alst{h}ylli skyldu.\eva

\bvb TODO.\evb\evg


\bvg\bva%
„\alst{H}irð ęigi þú \hld\ \alst{H}ǫgna ręiði &
né \alst{i}llan hug \hld\ \alst{ę́}ttar þïnnar; &
þú skalt, \alst{m}ę́r ung, \hld\ at \alst{m}ér lifa; &
\alst{ę́}tt \alst{á}tt, in góða, \hld\ es \alst{e}k s\emph{éu}mk.“\eva

\bvb TODO.\evb\evg


\bpg\bpa%TODO: This passage and the next should maybe be P?a and P?b. All later Ps should be numbered one less in references.
Helgi samnaði þá miklum skipa-her ok fór til Freka-steins ok fengu í hafi of·viðri mann-hę́tt. Þá kvǫ́mu leiptr yfir þá ok stóðu geislar í skipin.  Þeir sá í lopti’nu at val-kyrjur níu riðu ok kenndu þeir Sig·ru̇nu.  Þá lę́gði storm’inn ok kvǫ́mu þeir heilir til lands.  Gran·mars synir sǫ́tu á bjargi nǫkkuru er skipin sigldu at landi.  Guð·mundr hljóp á hest ok reið á njósn á berg’it við hǫfn’ina; þá hlóðu Vǫlsungar seglum; þá kvað Guð·mundr, \edtrans{svá sem fyrr er ritat í Hęlga-kviðu}{TODO}{\Bfootnote{Viz. in \textlink{HelgakvidaOne}[32].}}:\epa

\bpb TODO.\epb\epg


„\bvg\bva Hvęrr es \alst{f}ylkir \hld\ sá’s \alst{f}lota stýrir &
ok \alst{f}ęikna-lið \hld\ \alst{f}ǿrir at landi?“\eva

\bvb TODO.\evb\evg


\bpg\bpa%
Sin·fjǫtli Sig·mundar sonr svaraði ok \emph{er} þat enn ritat. Guð·mundr reið heim með her-sǫgu.  Þá sǫmnuðu Gran·mars synir her; kómu þar margir konungar.  Þar var Hǫgni, faðir Sig·rúnar, ok synir hans, Bragi ok Dagr.  Þar var orrusta mikil ok fellu allir Gran·mars synir ok allir þeira hǫfðingjar nema Dagr Hǫgna sonr fekk grið ok vann eiða Vǫlsungum.  Sig·rún gekk í val’inn ok hitti Hǫð·brodd at kominn dauða.  Hón kvað:\epa

\bpb TODO.\epb\epg


\bvg\bva%
Mun⸗a þér, \alst{S}ig·ru̇n, \hld\ frá \alst{S}efa-fjǫllum, &
\alst{H}ǫð-broddr konungr, \hld\ \alst{h}níga at armi; &
liðin es \alst{ę́}vi, \hld\ \alst{o}pt náir hr\emph{ę́}vi &
\alst{g}rán-stóð \alst{g}ríðar, \hld\ \alst{G}ran·mars sona.\eva

\bvb TODO.\evb\evg


\bpg\bpa Þá hitti hón Helga ok varð all-fegin. Hann kvað:\epa

\bpb TODO.\epb\epg


\bvg\bva%
Es⸗at þér at \alst{ǫ}llu, \hld\ \alst{a}l-vitr, gefit, &
þó kveð’k \alst{n}ǫkkvi \hld\ \alst{n}ornir valda; &
\alst{f}ellu ï morgun \hld\ at \alst{F}reka-stęini &
\alst{B}ragi ok Hǫgni \hld\ —varð’k \alst{b}ani þęira—\eva

\bvb TODO.\evb\evg


\bvg\bva%
en at \alst{St}yr·kleifum \hld\ \alst{St}arkaðr konungr &
en at \alst{H}lé·bjǫrgum \hld\ \alst{H}ro·llaugs synir; &
þann sá’k \alst{g}ylfa \hld\ \alst{g}rimm-úðgastan &
es \alst{b}arði⸗sk \alst{b}olr, \hld\ vas ȧ \alst{b}rot hǫfuð.\eva

\bvb TODO.\evb\evg


\bvg\bva%
Liggja at \alst{jǫ}rð\emph{u} \hld\ \alst{a}llra flęstir &
\alst{n}iðjar þïnir \hld\ at \alst{n}ǫ́m orðnir; &
\alst{v}annt⸗at-tu \alst{v}ígi, \hld\ \alst{v}as þér þat skapat &
at þú at \alst{r}ógi \hld\ \alst{r}ík-męnni vart.\eva

\bvb TODO.\evb\evg


\bpg\bpa Þá grét Sig·rún.  Hann kvað:\epa

\bpb Then Syerun wept.  He quoth:\epb\epg


\bvg\bva
„\alst{H}uggask-tu, Sig·rún! \hld\ \alst{H}ildr hęfir þú oss verit; &
\ind vinna⸗t \alst{sk}jǫldungar \alst{sk}ǫpum.“ &
„\alst{L}ifna mynda’k nú kjósa \hld\ es \alst{l}iðnir eru &
\ind ok knę́tta’k þér þó ï \alst{f}aðmi \alst{f}ela⸗sk.“\eva

\bvb TODO.\evb\evg


\bpg\bpa Þetta kvað Guð·mundr, Gran·mars sonr:\epa

\bpb TODO.\epb\epg


\bvg\bva%
Hvęrr es \alst{sk}jǫldungr \hld\ sá’s \alst{sk}ipum stýrir, &
lę́tr \alst{g}unn-fana \hld\ \alst{g}ollinn fyr stafni? &
Þikkja mér \alst{f}riðr \hld\ ï \alst{f}arar broddi; &
\alst{v}erpr \alst{v}íg-roða \hld\ umb \alst{v}íkinga.\eva

\bvb TODO.\evb\evg


\bvg\bva%
\speakernote{Sin·fjǫtli kvað:}%
\alst{H}ér má \alst{H}ǫð·broddr \hld\ \alst{H}ęlga kęnna &
\alst{f}lótta trauðan \hld\ ï \alst{f}lota miðjum; &
hann hęfi\emph{r} \alst{ø}ðli \hld\ \alst{ę́}ttar þïnnar, &
\alst{a}rf fjǫrsunga, \hld\ \alst{u}nd sik þrungit.\eva

\bvb TODO.\evb\evg


\bvg\bva%
Því \alst{f}yrr skulu \hld\ at \alst{F}reka-steini &
\alst{s}áttir \alst{s}aman \hld\ umb \alst{s}akar dǿma; &
mál es, \alst{H}ǫð·broddr, \hld\ \alst{h}ęfnd at vinna &
ef vér \alst{l}ę́gra hlut \hld\ \alst{l}ęngi bǫ́rum.\eva

\bvb TODO.\evb\evg


\bvg\bva%
Fyrr munt, \alst{G}uð·mundr, \hld\ \alst{g}ęitr of halda &
ok \alst{b}erg-skorar \hld\ \alst{b}rattar klífa, &
\alst{h}afa þér ï \alst{h}ęndi \hld\ \alst{h}ęsli-kylfu; &
þat ’s þér \alst{b}líðara \hld\ an \alst{b}rimis dómar.\eva

\bvb TODO.\evb\evg


\bvg\bva%
Þér es, \alst{S}in·fjǫtli, \hld\ \alst{s}ǿmra myklu &
\alst{g}unni at hęyja \hld\ ok \alst{g}laða ǫrnu &
an \alst{ȯ}·nýtum \hld\ \alst{o}rðum at dęila &
þó’tt \alst{h}ildingar \hld\ \alst{h}ęiptir dęili.\eva

\bvb TODO.\evb\evg


\bvg\bva%
Þikki⸗t mér \alst{g}óðir \hld\ \alst{G}ran·mars synir, &
þó dugir \alst{s}iklingum \hld\ \alst{s}att at mę́la, &
þęir \alst{m}ęrkt hafa \hld\ ȧ \alst{M}óins-hęimum &
at \alst{h}ug \alst{h}afa \hld\ \alst{h}jǫrum at bręgða; &
eru \alst{h}ildingar \hld\ \alst{h}øldsti snjallir.\eva

\bvb TODO.\evb\evg


\bpg\bpa Helgi fekk Sigrúnar ok ǫ́ttu þau sonu; var Helgi eigi gamall.  Dagr Hǫgna sonr blótaði Óðin til fǫður-hefnda.  Óðinn léði Dag g\emph{eir}s síns.  Dagr fann Helga, mág sinn, þar sem heitir at Fjǫturlundi.  Hann lagði í gǫgnum Helga með geir’num.  Þar fell Helgi, en Dagr reið til fjalla ok sagði Sigrúnu tíðindi:\epa

\bpb Hallow got Syerun for a wife and they had sons; Hallow was not old.  Day, Hain’s son, made a \inx[C]{bloot} to Weden for the sake of avenging his father.  Weden lent Day his spear.  Day found Hallow, his brother-in-law, at the place called Fetterlund; he ran Hallow through with the spear.  There fell Hallow, but Day rode to the fells and told Syerun the news:\epb\epg


\bvg\bva%
„\alst{T}rauðr em’k, systir, \hld\ \alst{t}rega þér at sęgja &
því’t ek hęfi \alst{n}auðigr \hld\ \alst{n}ipti grǿtta: &
\alst{F}ell ï morgun \hld\ und \alst{F}jǫturlundi &
\alst{b}uðlungr sá’s vas \hld\ \alst{b}ętstr ï hęimi &
ok \alst{h}ildingum \hld\ ȧ \alst{h}alsi stóð.“\eva

\bvb “Regretful am I, sister, to grieve thee by saying it— \\
for, forced, must I make my kinswoman weep: \\
this morning fell in Fetterlund \\
that noble who was the best in the world \\
and on the throats of princes stood.”\evb\evg


\bvg\bva%
\speakernote{[Sigrún kvað:]}%
„Þik skyli \alst{a}llir \hld\ \alst{ęi}ðar bíta, &
þęir es \alst{H}ęlga \hld\ \alst{h}afðir unna, &
at inu \alst{l}jósa \hld\ \alst{L}ęiptrar vatni &
ok at \alst{ú}r-svǫlum \hld\ \alst{U}nnar steini!\eva

\bvb “\emph{Thee} should all oaths bite, \\
which thou to Hallow hast sworn, \\
by the shining water of Lafter, \\
and by the spray-cold stone of Ithe.\evb\evg


\bvg\bva%
\alst{Sk}ríði⸗at þat \alst{sk}ip, \hld\ es und þér \alst{sk}ríði, &
þó’tt \alst{ȯ}ska-byrr \hld\ \alst{ę}ptir lęggi⸗sk! &
\alst{R}enni⸗a sá marr, \hld\ es und þér \alst{r}enni, &
þó’tt \alst{f}íęndr þïna \hld\ \alst{f}orða⸗sk ęigir!\eva

\bvb May the ship not glide which glides beneath thee, \\
although it has a wished-for gust behind it! \\
May the horse not run which runs beneath thee, \\
although from thy foes thou must escape!\evb\evg


\bvg\bva%
\alst{B}íti⸗a þér þat sverð, \hld\ es þú \alst{b}ręgðir, &
nema \alst{s}jǫlfum þér \hld\ \alst{s}yngvi of hǫfði! &
Þá vę́ri þér \alst{h}ęfnt \hld\ \alst{H}ęlga dauða, &
ef þú \alst{v}ę́rir \alst{v}argr \hld\ ȧ \alst{v}iðum úti, &
\alst{a}uðs \alst{a}nd-vani \hld\ ok \alst{a}lls gamans, &
\alst{h}ęfðir ęigi mat, \hld\ nema á \alst{h}rę́um spryngir!“\eva

\bvb May the sword not bite for thee which thou brandishest, \\
unless it sing around thy very own head! \\
\emph{Then} were on thee Hallow’s death avenged, \\
if thou wert a wolf in the woods outside, \\
bereft of wealth and all pleasure; \\
hadst no food, save thou plundered carrion!“\evb\evg


\bvg\bva%
\speakernote{Dagr kvað:}%
„\edtext{\alst{Ǿ}r est, systir, \hld\ ok \alst{ø}r·vita}{\lemma{Ǿr \dots\ ok ør·viti ‘Mad \dots\ and out of thy wits’}\Bfootnote{Formulaic, also occurring in \textlink{Lokasenna}[21]/1 and \textlink{Oddrunargratr}[15]/1.}}, &
es \alst{b}rǿðr þïnum \hld\ \alst{b}iðr for·skapa! &
\edtrans{\alst{Ęi}nn vęldr \alst{Ó}ðinn \hld\ \alst{ǫ}llu bǫlvi}{Alone is Weden at fault for all evil}{\Bfootnote{Formulaic, also occurring (with other names than \emph{Óðinn} ‘Weden’) at \textlink{GudrunOne}[25]/2 and \textlink{Sigurdskamma}[27]/4.}}, &
því’t með \alst{s}ifjungum \hld\ \alst{s}ak-ru̇nar bar!\eva

\bvb\speakernoteb{Day quoth:}%
“Mad art thou, sister, and out of thy wits, \\
when onto thy brother thou dost bid a cruel \inx[C]{shape}! \\
Alone is Weden at fault for all evil \\
for he carried strife-runes amidst kin!\evb\evg


\bvg\bva%
Þér \alst{b}ýðr \alst{b}róðir \hld\ \alst{b}auga rauða, &
ǫll \alst{V}andils-\alst{v}é \hld\ ok \alst{V}íg-dali; &
\alst{h}af \alst{h}alfan \alst{h}ęim \hld\ \alst{h}arms at gjǫldum &
\alst{b}rúðr \alst{b}aug-varið \hld\ ok \alst{b}urir þïnir.\eva

\bvb Thy brother offers thee red bighs, \\
all Wendelswigh and the Wighdales. \\
Have half the realm as restitution for the harm— \\
O bigh-adorned bride, and thy sons also.\evb\evg


\bvg\bva%
„\alst{S}it’k⸗a svá \alst{s}ę́l \hld\ at \alst{S}efa-fjǫllum, &
\alst{á}r né of nę́tr, \hld\ at ek \alst{u}na lífi, &
nema at \alst{l}iði \alst{l}ofðungs \hld\ \alst{l}jóma bręgði, &
renni und \alst{v}ísa \hld\ \alst{V}íg-blę́r þinig, &
\alst{g}oll-bitli vanr, \hld\ knega’k \alst{g}rami fagna!\eva

\bvb “I will not sit so happy in the Sevefells, \\
at dawn nor night, that I should be content with living, \\
unless the retinue of the man of praise splendidly shone, \\
{[and]} beneath the ruler Wighblaw ran hither, \\
wont to the golden bit—{[and]} I might greet the prince!\evb\evg


\bvg\bva%
Svá \alst{h}afði \alst{H}ęlgi \hld\ \alst{h}rę́dda gǫrva &
\alst{f}jȧndr sïna alla \hld\ ok \alst{f}rę̇ndr þęira, &
sem fyr \alst{u}lfi \hld\ \alst{ó}ðar rynni &
\alst{g}ęitr af fjalli, \hld\ \alst{g}ęiska fullar!\eva

\bvb So would Hallow have terrified \\
his enemies all and the kinsmen of theirs, \\
like from a wolf did madly rush \\
goats down a fell, full of fright.\evb\evg


\bvg\bva%
\Ballnote{Cf. the very similar description of Siward in \textlink{GudrunTwo} 2.}%
Svá bar \alst{H}ęlgi \hld\ af \alst{h}ildingum &
sem \alst{í}tr-skapaðr \hld\ \alst{a}skr af þyrni &
eða sá \alst{d}ýr-kalfr \hld\ \alst{d}ǫggu slunginn &
es \alst{ø}fri fęrr \hld\ \alst{ǫ}llum dýrum, &
ok \alst{h}orn glóa \hld\ við \alst{h}imin sjalfan.“\eva

\bvb So did Hallow surpass the princes \\
like the nobly shaped ash the thorn, \\
or the deer-calf, dew-besprinkled, \\
which fares higher than all beasts, \\
and its horns gleam against heaven itself.”\evb\evg


\bpg\bpa Haugr var gǫrr eptir Helga.  En er hann kom til Valhallar, þá bauð Óðinn hǫ́num ǫllu at ráða með sér.  Helgi kvað:\epa

\bpb A barrow was made for Hallow.  But when he came to Walhall Weden offered him to rule everything together with him.  Hallow quoth:\epb\epg


\bvg\bva%
„Þú skalt, \alst{H}undingr, \hld\ \alst{h}vęrjum manni &
\alst{f}ót-laug geta \hld\ ok \alst{f}una kynda; &
\alst{h}unda binda, \hld\ \alst{h}esta gę̇ta, &
gefa \alst{s}vínum \alst{s}oð, \hld\ áðr \alst{s}ofa gangir!“\eva

\bvb “Thou shalt, Hunding, for every man \\
make a foot-bath and kindle the fire, \\
bind the hounds, feed the horses, \\
give wastewater to the swine—before thou mightst go to sleep!”\evb\evg


\bpg\bpa Ambótt Sigrúnar gekk um aptan hjá haugi Helga ok sá at Helgi reið til haugs’ins með marga menn. Ambótt kvað:\epa

\bpb Syerun’s maid-servant walked in the evening by Hallow’s barrow and saw that Hallow rode to the barrow with many men.  The maid-servant quoth:\epb\epg


\bvg\bva%
„Hvárt ’ro þat \alst{s}vik ęin \hld\ es \alst{s}éa þikkjumk &
eða \alst{r}agna \alst{r}ǫk \hld\ \alst{r}íða męnn dauðir, &
es \alst{jó}a \alst{y}ðra \hld\ \alst{o}ddum kęyrið, &
eða es \alst{h}ildingum \hld\ \alst{h}ęim-fǫr gefin?“\eva

\bvb “Either these are only tricks, as I seem to see \\
—or the \inx[L]{Rakes of the Reins}?—dead men riding; \\
as ye drive your steeds by spear-points on— \\
or are the princes granted leave to go home?”\evb\evg


\bvg\bva%
\speakernote{[Ęinn þęira kvað:]}%
„Es⸗a þat \alst{s}vik ęin \hld\ es \alst{s}éa þikki⸗sk &
né \edtrans{\alst{a}ldar rof}{Ripping of the Age}{\Bfootnote{Formulaic.  Cf. TODO \emph{rjúfa⸗sk ręgin}. This is the same root, only zero-grade.}} \hld\ þó’tt-u \alst{o}ss lítir, &
þó’tt vér \alst{jó}a \alst{ó}ra \hld\ \alst{o}ddum keyrim, &
né es \alst{h}ildingum \hld\ \alst{h}ęim-fǫr gefin.“\eva

\bvb\speakernoteb{[One of them quoth:]}%
“It is not only tricks, as thou seemest to see— \\
nor the Ripping of the Age, although thou behold us; \\
although we drive our steeds by spear-points on \\
the princes are not granted leave to go home.”\evb\evg


\bpg\bpa Heim gekk ambótt ok sagði Sigrúnu:\epa

\bpb The maid-servant went home and said to Syerun:\epb\epg


\bvg\bva%
„Út gakk \alst{S}igrún, \hld\ frá \alst{S}ęfa-fjǫllum &
ef þik \alst{f}olks jaðarr \hld\ \alst{f}inna lystir; &
upp ’s \alst{h}augr lokinn, \hld\ kominn es \alst{H}ęlgi! &
\alst{D}ólg-spor \alst{d}ręyra \hld\ \alst{d}ǫglingr bað þik &
at þú \alst{s}ár-dropa \hld\ \alst{s}vęfja skyldir.“\eva

\bvb “Go outside, Syerun from the Sevefells, \\
if thou hast lust to find the leader of the troop— \\
the barrow is unlocked; Hallow is come! \\
The ruler of bloody wounds bade thee \\
that thou his wound-drops \ken{blood} shouldst calm.”\evb\evg


\bpg\bpa Sigrún gekk í haug’inn til Helga ok kvað:\epa

\bpb Syerun walked into Hallow’s barrow, and quoth:\epb\epg


\bvg\bva%
„Nú em’k svá \alst{f}ęgin \hld\ \alst{f}undi okkrum &
sem \alst{á}t-frękir \hld\ \alst{Ó}ðins haukar &
es \alst{v}al \alst{v}itu, \hld\ \alst{v}armar bráðir, &
eða \alst{d}ǫgg-litir \hld\ \alst{d}ags-brún séa.“\eva

\bvb “Now do I so rejoice at our meeting \\
like the ravenous hawks of Weden \ken{ravens} \\
when they know corpses, warm carrion, \\
or, gleaming with dew, they see the day’s brow \ken{dawn}!\evb\evg


\bvg\bva%
Fyrr vil’k \alst{k}yssa \hld\ \alst{k}onung ó·lifðan &
an þú \alst{b}lóðugri \hld\ \alst{b}rynju kastir; &
\alst{h}ár ’s þitt, \alst{H}elgi, \hld\ \alst{h}élu þrungit, &
\edtrans{allr es \alst{v}ísi \hld\ \alst{v}al-dǫgg slęginn}{the prince is all with corpse-dew whipped}{\Bfootnote{Cf. \textlink{Baldrsdraumar} 5, where the dead wallow says something similar.}}, &
\alst{h}ęndr úr-svalar \hld\ \alst{H}ǫgna mági; &
hvé skal’k þér, \alst{b}uðlungr, \hld\ þess \alst{b}ót of vinna?“\eva

\bvb Sooner would I kiss the unliving king, \\
than thou the bloody byrnie mightst cast away! \\
Thy hair is, Hallow, with hoarfrost thick; \\
the prince is all with corpse-dew \ken{blood} whipped; \\
the hands spray-cold on Hain’s in-law \ken*{= Hallow}— \\
how shall I for thee, noble, remedy that?”\evb\evg


\bvg\bva%
\speakernote{[Hęlgi kvað:]}%
„Ęin vęldr þú, \alst{S}igrún \hld\ frá \alst{S}efafjǫllum, &
es \alst{H}ęlgi es \hld\ \alst{h}arm-dǫgg slęginn: &
\alst{G}rę́tr þú, \alst{g}oll-varið, \hld\ \alst{g}rimmum tǫ́rum, &
\alst{s}ól-bjǫrt \alst{s}uð-rǿn, \hld\ áðr þú \alst{s}ofa gangir, &
hvęrt fęllr \alst{b}lóðugt \hld\ á \alst{b}rjóst grami, &
\alst{ú}r-svalt, \alst{i}nn-fjalgt \hld\ \alst{ę}kka þrungit.\eva

\bvb “Thou alone causest, Syerun from the Sevefells, \\
that Hallow be with harm-dew whipped. \\
Thou weepest—gold-covered—bitter tears— \\
sun-bright southern lady—before thou goest to sleep. \\
Each one falls bloody on the prince’s chest, \\
spray-cold, stifled, pressed forth by grief.\evb\evg


\bvg\bva%
Vęl skulum \alst{d}rekka \hld\ \alst{d}ýrar vęigar &
þó’tt \alst{m}isst hafim \hld\ \alst{m}unar ok landa! &
Skal \alst{ę}ngi maðr \hld\ \alst{a}ngr-ljóð kveða &
þó’tt mér á \alst{b}rjósti \hld\ \alst{b}ęnjar líti. &
Nú eru \edtext{\alst{b}rúðir \hld\ \alst{b}yrgðar í haugi, &
\alst{l}ofða dísir, \hld\ hjá oss}{\lemma{brúðir, dísir, oss ‘brides, dises, us’}\Bfootnote{Hallow speaks in the plural.  “Now has my bride, my goddess, come into the barrow, next to me, who am dead.”}} \alst{l}iðnum!“\eva

\bvb Well shall we drink costly draughts \\
although we may have lost both love and land! \\
No one shall sing songs of sorrow, \\
although he behold the wounds on my chest. \\
Now are the brides shut within the barrow, \\
the praised one’s \inx[C]{dise}[dises], next to us, passed-on.”\evb\evg


\bpg\bpa Sigrún bjó sę́ing í haug’inum.\epa

\bpb Syerun made a bed in the barrow:\epb\epg


\bvg\bva%
„\alst{H}ér hęfi’k þér, \alst{H}ęlgi, \hld\ \alst{h}vílu gørva, &
\alst{a}ngr-lausa mjǫk, \hld\ \alst{Y}lfinga niðr; &
vil’k þér í \alst{f}aðmi, \hld\ \alst{f}ylkir, sofna &
\edtrans{sem’k \alst{l}ofðungi \hld\ \alst{l}ifnum mynda’k!}{like I would with the living man of praise}{\Bfootnote{I.e. “just as I would if you were still alive.”}}“\eva

\bvb “Here I’ve for thee, Hallow, made a place of rest \\
almost sorrowless, kinsman of the Wolvings! \\
I will in thy arms, marshal, fall asleep, \\
like I would with the living man of praise.”\evb\evg


\bvg\bva%
\speakernote{[Hęlgi kvað:]}%
„Nú kveð’k \alst{ę}ns-kis \hld\ \alst{ø}r·vę̇nt vesa, &
\alst{s}íð né \alst{s}nimma, \hld\ at \alst{S}efa-fjǫllum &
es þú ȧ \alst{a}rmi \hld\ \alst{ȯ}·lifðum søfr, &
\alst{h}vít, ï \alst{h}augi, \hld\ \alst{H}ǫgna dóttir, &
ok est-u \alst{k}vik, \hld\ in \alst{k}onung-borna!“\eva

\bvb\speakernoteb{[Hallow quoth:]}%
“Now, I say, there is naught more missing \\
neither late nor soon from the Sevefells, \\
when thou sleepest on the unliving arm \\
(O white daughter of Hain) in the barrow— \\
and thou art alive! (borne of the king).”\evb\evg


\bvg\bva%
\speakernote{[Hęlgi kvað:]}%
\Ballnote{The night has passed; dawn is breaking, and Hallow speaks.  The manuscript does not indicate the change of scene.}%
„Mál ’s mér at \alst{r}íða \hld\ \edtrans{\alst{r}oðnar}{reddening}{\Bfootnote{From the rising dawn.}} brautir, &
láta \alst{f}ǫlvan jó \hld\ \alst{f}lug-stíg troða; &
skal’k fyr \alst{v}estan \hld\ \alst{v}ind-hjalms brúar &
áðr \alst{S}al-gofnir \hld\ \alst{s}igr-þjóð vęki.“\eva

\bvb “It is time for me to ride the reddening roads, \\
to let my pale steed tread the path of flight \ken{sky/heaven}. \\
I must be west of the wind-helm’s bridges \ken{sky/heaven > clouds?} \\
before Salgovner awakens the victorious folk.”\evb\evg


\bpg\bpa Þęir Hęlgi riðu lęið sína, en þę́r fóru hęim til bǿjar. Annan aptan lét Sigrún ambótt halda vǫrð á haugi’num.  En at dag-setri, es Sigrún kom til haugs’ins, hón kvað:\epa

\bpb Hallow and his men rode on their way, but the women journeyed home to the farm. The next evening Syerun made her maid-servant keep watch on the barrow.  And at sunset as Syerun came to the barrow, she \ken*{= the maid-servant} quoth:\epb\epg


\bvg\bva%
„\alst{K}ominn vę́ri nú, \hld\ ef \alst{k}oma hygði, &
\alst{S}ig·mundar burr \hld\ frá \alst{s}ǫlum Óðins; &
kveð’k \alst{g}rams þinig \hld\ \alst{g}rę̇na⸗sk vȧnir &
\edtrans{es á \alst{a}sk-limum \hld\ \alst{ę}rnir sitja}{when on ashen branches eagles sit}{\Bfootnote{i.e. “when the eagles roost on yonder trees”.  This is a sign of Hallow and his men not coming; if they were, the eagles would be following them and picking at their bodies.}} &
ok \edtext{\alst{d}rífr \alst{d}rótt ǫll \hld\ \alst{d}raum-þinga til}{\lemma{drífr \dots\ draum-þinga til ‘drifts off to dream-Things’}\Bfootnote{Drifts off to the courts of dreams, i.e. falls asleep.  A fine metaphor.}}.“\eva

\bvb “He would be come by now if he had thought to come, \\
Syemund’s son \ken*{= Hallow} from Weden’s halls. \\
I say, hopes are fading of the prince’s coming \\
when on ashen branches eagles sit, \\
and all mankind drifts off to dream-\inx[C]{Thing}[Things].\evb\evg


\bvg\bva%
Ves \alst{ęi}gi svá \alst{ǿ}r \hld\ at \alst{ęi}n farir, &
\alst{d}ís skjǫldunga, \hld\ \alst{d}raug-húsa til! &
Verða \alst{ǫ}flgari \hld\ \alst{a}llir á nǫ́ttum &
\alst{d}auðir \alst{d}ólgar, mę́r, \hld\ an of \alst{d}aga ljósa.“\eva

\bvb Be not so mad that thou journey alone, \\
O dise of the Shieldings, to the ghost-houses! \\
Mightier at night do all become \\
dead fiends, maiden, than during the bright days!”\evb\evg


\bpg\bpa Sigrún varð skamm-líf af harmi ok trega.  \edtrans{Þat var trúa í forneskju}{It was the belief in olden times}{\Bfootnote{For another instance of beliefs in \emph{forneskju} ‘olden times, antiquity’, see \textlink{Fafnismal}[P1].}}, at menn vę́ri endr·bornir, en þat er nú kǫlluð kerlinga-villa.  Helgi ok Sigrún er kallat at vę́ri endr·borin.  Hét hann þá Helgi Haddingja-skati en hón Kára Hálf·danar dóttir, svá sem kveðit er í \edtrans{Kǫ́ru ljóðum}{Leeds of Cheer}{\Bfootnote{A now-lost heroic poem.}}, ok var hón val-kyrja.\epa

\bpb Syerun became short-lived for pain and grief.  It was the belief in olden times that men were reborn, but that is now called an old wives’ tale.  Of Hallow and Syerun it is claimed that they were reborn.  He was then called Hallow Hardingskate and she Cheer Halfdanesdaughter, as is told in the Leeds of Cheer, and she was a walkirrie.\epb\epg

\sectionline
