\bookStart{Speeches of Fathomer}[Fáfnis mǫ́l]
\setBookCode{Fafnismal}

\begin{flushright}%
\textbf{Dating} \parencite{Sapp2022}: C10th (0.442)–early C11th (0.402)

\textbf{Meter:} \Ljodahattr, \Fornyrdislag\ (TODO)%
\end{flushright}

\section{Introduction}

The \textbf{Speeches of Fathomer} (\textlink{Fafnismal}) is only preserved in \Regius, where it has the title \emph{Frá dauða Fáfnis} ‘From the death of Fathomer’.  It forms a continuous text with the preceding \textlink{Reginsmal} and the following \textlink{Sigrdrifumal} and should not be analyzed in isolation.

The poetry of \textlink{Fafnismal} is closely paraphrased by \VolsungaSaga\ 18–19, and it is clear that the now-lost source underlying that text was near-identical to \Regius.

Siward’s slaying of the wyrm Fathomer was an exceptionally famous story in the Wiking Age and Scandinavian Middle Ages.  Outside of \textlink{Reginsmal}–\textlink{Fafnismal} and \VolsungaSaga\ the narrative is referenced in Scaldic poetry (TODO) and depicted pictorally on numerous objects.  The most important of these is the Swedish runic inscription Sö 101 (ca. 1030 CE) from Ramsund, Södermanland.  The carving consists of a long serpent or wyrm inscribed with a generic memorial inscription.  At the bottom right a figure thrusts a sword through the wyrm’s body and in the space enclosed by it several important events following Siward’s slaying of Fathomer are depicted with remarkably close correspondence to the version preserved in the Norse-Icelandic sources; see Fig. \ref{fig:ramsund}.

\begin{figure}[b]
\centering
\includegraphics[width=\textwidth]{Ramsund}
\caption{The Ramsund carving.  Wiking Age, ca. 1030 CE.  Details depicted: \emph{1. Siward slays Fathomer.  2. Siward roasts Fathomer’s heart by a fire; he burns his finger and puts it in his mouth to cool it, inadvertently tasting the blood.  3. Two birds sit in the tree, presumably talking to Siward.  4. Rein lies decapitated surrounded by his bellows, tongs, and anvil.  5. Grane stands loaded with a chest on his back.  6. A wolf or dog; the only detail not found in the Norse version.}  © Bengt A. Lundgren/RAÄ, \href{https://creativecommons.org/licenses/by/4.0/deed.en}{CC BY 4.0}.  \url{https://pub.raa.se/visa/dokumentation/7dd6614e-950b-42bf-b31f-4f910c74e936}}
\label{fig:ramsund}
\end{figure}

\newpage

\section{Text}

\subsection{From Fathomer’s death (\emph{Frá dauða Fáfnis})}

\bvg\bva%
„\alst{S}vęinn ok \alst{s}vęinn! \hld\ Hvęrjum est \alst{s}vęini of borinn? &
\ind Hvęrra est \alst{m}anna \alst{m}ǫgr? &
es þú ȧ \alst{F}áfni rautt \hld\ þïnn inn \alst{f}rȧna mę́ki; &
\ind stǫndumk til \alst{h}jarta \alst{h}jǫrr!“\eva

\bvb\speakernoteb{[Fathomer quoth:]}%
“{\huge O} \textsc{swain} and swain! To which swain art thou born; \\
\ind of which men art thou the son? \\
When thou on Fathomer hast reddened this thy gleaming blade; \\
\ind the sword stands unto my heart!”\evb\evg


\bpg\bpa Sig·urðr dulði nafns síns fyr því at \edtrans{þat var trúa þeira í forneskju}{it was their belief in olden times}{\Bfootnote{For another instance of beliefs in \emph{forneskju} ‘olden times, antiquity’, see \textlink{HelgakvidaTwo}[P18].}} at orð feigs manns mę́tti mikit ef hann bǫlvaði ó·vin sínum með nafni. Hann kvað:\epa

\bpb Siward belied his name, for it was their belief in olden times that a \inx[C]{fey} man’s word could do much if he cursed his foe by name. He quoth:\epb\epg


\bvg\bva%
„\alst{G}ǫfugt dýr ek hęiti \hld\ en ek \alst{g}ęngit hef’k &
\ind inn \alst{m}óður-lausi \alst{m}ǫgr, &
\alst{f}ǫður ek á’kk⸗a \hld\ sem \alst{f}ira synir, &
\ind gęng \alst{e}k \alst{ęi}nn saman.“\eva

\bvb “Noble Beast am I called, but I have gone \\
\ind as the motherless lad. \\
A father I have not like the sons of men; \\
\ind I go alone.”\evb\evg


\bvg\bva%
„Vęitst, ef \alst{f}ǫður né átt⸗at \hld\ sem \alst{f}ira synir, &
\ind af hvęrju vastu \alst{u}ndri \alst{a}linn? &
\edtext{[...]}{\Bfootnote{Two lines appear to be missing here, but may have survived in the ms. underlying \VolsungaSaga.  \VolsungaSaga\ 18 paraphrases: \emph{Ef þú átt engan fǫður né móður, af hverju undri ertu þá alinn? Ok þó’tt þú segir mér eigi þitt nafn á bana-dǿgri mínu, þá veiztu, at þú lýgr nú.} ‘If thou hast no father or mother, by which wonder art thou begotten? And although thou wilt not tell me thy name in my hour of death, thou knowest that thou art lying.’  It is apparently this now-missing appeal to Siward’s conscience—it would be shameful to lie to a dying man—that makes him reveal his true name.}}“\eva

\bvb\speakernoteb{[Fathomer quoth:]}%
“Knowest thou, if thou hast no father like the sons of men, \\
\ind by which wonder thou wast begotten?”\evb\evg


\bvg\bva%
„\alst{Ę́}tterni mitt \hld\ kveð’k þér \alst{ȯ}·kunnigt vesa &
\ind ok mik \alst{s}jalfan hit \alst{s}ama: &
\alst{S}ig·urðr ek hęiti \hld\ \alst{S}ig·mundr hét mïnn faðir &
\ind es hęf’k þik \alst{v}ǫ́pnum \alst{v}egit.“\eva

\bvb\speakernoteb{[Siward quoth:]}%
“My lineage, I say, is unknown to thee, \\
\ind and my self the same.\footnoteB{The sense is that Fathomer would not recognize Siward’s lineage (i.e. his father) or name, since he is an orphan who up until this point has not accomplished much. He is not saying that he is lineage is unknown even to himself, since \emph{sjalfan mik} ‘my self’ is accusative, not dative.} \\
Siward I am called—Syemund was called my father— \\
\ind who with weapons have smitten thee.”\evb\evg


\bvg\bva%
„\alst{H}vęrr þik \alst{h}vatti, \hld\ hví \alst{h}vętjask lést, &
\ind mínu \alst{f}jǫrvi at \alst{f}ara? &
Inn \alst{f}rán-ęygi svęinn, \hld\ þú áttir \alst{f}ǫður bitran, &
\ind \edtrans{á-bornu \alst{sk}jór á \alst{sk}ęið.}{inborn traits show quickly}{\Bfootnote{The original is cryptic.  \emph{á skęið} means roughly ‘rapidly, quickly’, whence the expression \emph{ríða á skęið} ‘\CV: to ride at full speed’, but the other words are uncertain.  \textcite{LaFargeGlossary} read ‘your innate qualities show quickly’, suggesting two unattested words: an adjective \emph{*áborinn} ‘innate, inborn’ and a verb \emph{*skjóa} ‘to show’. Yet the lack of i-umlaut in the supposed 3rd sg. pres. ind. \emph{skjór} is difficult. We would expect \emph{**skýr}, as in \emph{skjóta} ‘to shoot,’ with 2nd/3rd sg. pres. ind \emph{skýtr}. A solution here would be reading a 2nd sg. pres. subj. \emph{skjóir}, with a vowel TODO}}“\eva

\bvb\speakernoteb{[Fathomer quoth:]}%
“Who goaded thee; why didst thou let thee be goaded \\
\ind my life for to destroy? \\
O gleaming-eyed swain, thou hadst a sharp father; \\
\ind inborn traits show quickly.”\evb\evg


\bvg\bva%
„\alst{H}ugr mik \alst{h}vatti, \hld\ \alst{h}ęndr mér full-týðu &
\ind ok mïnn inn \alst{h}vassi \alst{h}jǫrr; &
fár es \alst{h}vatr \hld\ es \alst{h}røðask tękr &
\ind ef ï \alst{b}arn-esku es \alst{b}lauðr.“\eva

\bvb\speakernoteb{[Siward quoth:]}%
“My heart goaded me; my hands availed me \\
\ind and this my sharp sword. \\
Few a man is bold when he takes to grow, \\
\ind if in his youth he be soft.”\evb\evg


\bvg\bva%
„\alst{V}ęit’k, ef þú \alst{v}axa nę́ðir \hld\ fyr þinna \alst{v}ina brjósti, &
\ind séi maðr þik \alst{v}ręiðan \alst{v}ega; &
nú est \alst{h}aptr \hld\ ok \alst{h}ęr-numinn, &
\ind ę́ kveða \alst{b}andingja \alst{b}ifask.“\eva

\bvb\speakernoteb{[Fathomer quoth:]}%
“I know that if thou hadst grown up upon thy kinsmen’s breast \\
\ind man would see thee wrathfully fight; \\
now art thou a captive and war-taken; \\
\ind they say the boundling always trembles.”\evb\evg


\bvg\bva%
„Því bregðr þú nú mér, \alst{F}áfnir, \hld\ at til \alst{f}jarri sjá’k &
\ind \alst{m}ïnum fęðr-\alst{m}unum, &
ęigi em’k \alst{h}aptr \hld\ þó’tt vę́ra \alst{h}ęr-numi; &
\ind þú fannt, at ek \alst{l}auss \alst{l}ifi!“\eva

\bvb\speakernoteb{[Siward quoth:]}%
“For that dost thou now upbraid me, Fathomer, that I be too far \\
\ind from the love of my fathers. \\
I am not at all a captive, although I be war-taken; \\
\ind thou hast found that I live loose!”\evb\evg


\bvg\bva%
„\alst{H}ęipt-yrði ęin \hld\ tęlr þú þér í \alst{h}ví-vętna &
\ind en ek þér \alst{s}att ęitt \alst{s}ęgi’k: &
It \alst{g}jalla \alst{g}oll \hld\ ok it \alst{g}lóð-rauða fé, &
\ind þér verða þęir \alst{b}augar at \alst{b}ana!“\eva

\bvb\speakernoteb{[Fathomer quoth:]}%
“With hateful words alone dost thou answer anything, \\
\ind but I tell thee truth alone: \\
The clanging gold and the glowing red wealth— \\
\ind those bighs will be thy bane!”\evb\evg


\bvg\bva%
„\alst{F}éi ráða \hld\ skal \alst{f}yrða hvęrr &
\ind \alst{ę́} til \edtrans{ins \alst{ęi}na dags}{the one day}{\Bfootnote{His predetermined day of death.  Siward dismisses the curse; he must die regardless of whether he takes the gold or not, and it is better to die wealthy and renowned than wretched and unknown.}} &
því’t \alst{ęi}nu sinni \hld\ skal \alst{a}lda hvęrr &
\ind fara til \alst{h}ęljar \alst{h}eðan.“\eva

\bvb\speakernoteb{[Siward quoth:]}%
“Rule his wealth shall every man, \\
\ind always, until the one day; \\
for at one time shall every man \\
\ind journey hence to Hell.”\evb\evg


\bvg\bva%
„\alst{N}orna dóm \hld\ munt \edtrans{fyr \alst{n}ęsjum}{before the headlands}{\Bfootnote{I.e. ‘close at hand, imminent’.  An established metaphor for imminent death, cf. the last st. of \Sonatorrek\ (TODO).}} hafa &
\ind ok \alst{ȯ}·svinns \alst{a}pa; &
ï \alst{v}atni þú drukknar \hld\ ef ï \alst{v}indi rę́r; &
\ind allt es \alst{f}ęigs \alst{f}orað.“%
\Ballnote{Fathomer points out the danger of the curse: death will find Siward in any circumstance.  The redactor of \VolsungaSaga\ clearly misunderstood the import of the stanza when he thought it was a warning specifically against sailing on the windy sea, when it was  only an illustration of one of the myriad potential ways Siward might die.  \VolsungaSaga\ 18 paraphrases: \emph{Fátt vill þú at mínum dǿmum gera, en drukkna muntu, ef þú ferr um sjá ó·varliga, ok bíð heldr á landi, und’s logn er.} ‘Thou hast little wish to act according to my examples, but thou wilt drown if thou goest to journey carelessly at sea; and rather abide on land until it is calm.’}\eva

\bvb\speakernoteb{[Fathomer quoth:]}%
“The doom of the Norns shalt thou have before the headlands, \\
\ind and that of an unwise ape. \\
Thou wilt drown in water if thou rowest in wind; \\
\ind everything is the pit of the \inx[C]{fey}.”\evb\evg


\bvg\bva%
„Sęg mér, \alst{F}áfnir, \hld\ alls þik \alst{f}róðan kveða &
\ind ok \alst{v}ęl mart \alst{v}ita: &
Hvęrjar ’ro þę́r \alst{n}ornir \hld\ \edtrans{es \alst{n}auð-gǫnglar ’ro}{attend in need}{\Bfootnote{Lit. ‘are attendant in need’; they help ailing mothers during childbirth.  Cf. \textlink{Sigrdrifumal} 9.}} &
\ind ok kjósa \alst{m}ǿðr frá \alst{m}ǫgum?“%
\Ballnote{Siward asks a series of general mythological questions in a style closely resembling \textlink{Vafthrudnismal}.  These questions do not at all contribute to the narrative and it is not impossible that yet more of them have been removed for that reason; cf. note to st. 16.}\eva

\bvb\speakernoteb{[Siward quoth:]}%
“Tell me, Fathomer, as they call thee wise \\
\ind and knowing well enough: \\
Who are the Norns which attend in need \\
\ind and choose mothers from their lads?”\evb\evg


\bvg\bva%
„\alst{S}undr-bornar mjǫk \hld\ hygg at nornir \alst{s}é, &
\ind \alst{ęi}gu-t þę́r \alst{ę́}tt saman; &
sumar ’ro \alst{ǫ́}s-kunngar, \hld\ sumar \alst{a}lf-kunngar, &
\ind sumar \alst{d}ǿtr \alst{D}valins.“\eva

\bvb\speakernoteb{[Fathomer quoth:]}%
“Of most sundry birth I judge the norns to be, \\
\ind they come not from a common lineage: \\
some are Os-born, some Elf-born, \\
\ind some the daughters of Dwollen \ken{dwarfesses}.”\evb\evg


\bvg\bva%
„Sęg mér þat, \alst{F}áfnir, \hld\ alls þik \alst{f}róðan kveða &
\ind ok \alst{v}ęl margt \alst{v}ita, &
hvé sá \alst{h}olmr hęitir \hld\ es blanda \alst{h}jǫr-lęgi &
\ind \alst{S}urtr ok ę́sir \alst{s}aman.“\eva

\bvb\speakernoteb{[Siward quoth:]}%
“Tell me this, Fathomer, as they call thee wise \\
\ind and knowing well enough: \\
What is the islet called, where Surt and the Eese \\
\ind blend sword-water \ken{blood} together?”\evb\evg


\bvg\bva%
„\alst{Ó}-skópnir hęitir \hld\ en þar \alst{ǫ}ll skulu &
\ind \alst{g}ęirum lęika \alst{g}oð; &
\alst{B}il-rǫst \alst{b}rotnar \hld\ es á \alst{b}rott fara &
\ind ok svima í \alst{m}óðu \alst{m}arir.“\eva

\bvb\speakernoteb{[Fathomer quoth:]}%
“Unshopner it is called, and there shall all \\
\ind the Gods play with spears \ken{make war}; \\
Bilrest shatters when they go away, \\
\ind and the steeds swim in the sea.”\evb\evg


\bvg\bva%
„\alst{Ǿ}gis hjalm \hld\ bar’k of \alst{a}lda sonum &
\ind \alst{m}eðan of \alst{m}ęnjum lá’k; &
\alst{ęi}nn rammari \hld\ hugðumk \alst{ǫ}llum vesa, &
\ind fann’k⸗a’k \alst{m}arga \alst{m}ǫgu.“%
\Ballnote{Fathomer continues speaking, but there is probably something missing here, since the transition is abrupt. Between its paraphrases of st. 15 and of st. 16, \VolsungaMS\ has \emph{Ok enn mę́lti Fáfnir: „Reginn bróðir minn veldr mínum dauða, ok þat hlę́gir mik, er hann veldr ok þínum dauða, ok ferr þá, sem hann vildi.“} ‘And further spoke Fathomer: “My brother Rein causes my death, and it gladdens me that he also causes thy death, and then it will go like he has willed.”’, which may perhaps be a paraphrase of a lost st.}\eva

\bvb “The helmet of awe I carried over the sons of men \\
\ind while on the neckrings I lay; \\
stronger than all I thought me alone to be; \\
\ind I did not find many lads.”\evb\evg


\bvg\bva%NOTE: Heavily formulaic.
„\alst{Ǿ}gis hjalmr \hld\ bergr \alst{ęi}nu-gi &
\ind hvar’s skulu \alst{v}ręiðir \alst{v}ega; &
\edtext{þȧ þat \alst{f}innr \hld\ es með \alst{f}lęirum kømr &
\ind at \alst{ę}ngi es \alst{ęi}nna hvatastr}{\lemma{þȧ þat finnr \hld\ es með flęirum kømr / at ęngi es ęinna hvatastr ‘this he then finds when among the many he comes— / that none is the boldest of all’}\Bfootnote{Near-identical to \textlink{Havamal}[64]/3–4; see there.}}.“\eva

\bvb\speakernoteb{[Siward quoth:]}%
“The helmet of awe rescues no man \\
\ind wherever wroth ones should fight; \\
this he then finds when among the many he comes— \\
\ind that none is the boldest of all.”\evb\evg


\bvg\bva%
„\alst{Ęi}tri ek fnę́sta \hld\ es á \alst{a}rfi lá’k &
\ind \alst{m}iklum \alst{m}íns fǫður.“\eva

\bvb\speakernoteb{[Fathomer quoth:]}%
“Venom I snorted while I lay on the great \\
\ind inheritance of my father.”\evb\evg


\bvg\bva%
„Inn rammi ormr, \hld\ þú gørðir frę́s mikla &
\ind ok gatst \alst{h}arðan \alst{h}ug; &
\alst{h}ęipt at męiri \hld\ verðr \alst{h}ǫlða sonum &
\ind at þann \alst{h}jalm \alst{h}afi.“\eva

\bvb\speakernoteb{[Siward quoth:]}%
“O mighty wyrm, thou madest a great snort, \\
\ind and didst get a hard heart. \\
Greater hatred arises for the sons of men \\
\ind who might have that helm.”\evb\evg


\bvg\bva%
„\alst{R}ę́ð’k þér nú, Sig·urðr, \hld\ en þú \alst{r}áð nemir &
\ind ok ríð \alst{h}ęim \alst{h}eðan; &
it \alst{g}jalla \alst{g}oll \hld\ ok it \alst{g}lóð-rauða fé, &
\ind þér verða þęir \alst{b}augar at \alst{b}ana!“\eva

\bvb\speakernoteb{[Fathomer quoth:]}%
“I counsel thee now, Siward—and thou oughtst to take the counsel, \\
\ind and ride home hence: \\
The clanging gold and the glowing red wealth— \\
\ind those bighs will be thy bane!”\evb\evg


\bvg\bva%
„\alst{R}áð ’s þér \alst{r}áðit \hld\ en ek \alst{r}íða mun &
\ind til þęss golls es ï \alst{l}yngvi \alst{l}iggr, &
en þú, \alst{F}áfnir, \hld\ ligg ï \alst{f}jǫr-brotum &
\ind \edtrans{þar’s þik \alst{H}ęl \alst{h}afi}{where Hell may have thee}{\Bfootnote{Formulaic. TODO.}}!“\eva

\bvb\speakernoteb{[Siward quoth:]}%
“Thy counsel has been counseled, but I will ride \\
\ind to the gold which in the heather lies, \\
but thou, Fathomer, lie in the lifeblood tracks, \\
\ind where Hell may have thee!”\evb\evg


\bvg% NOTE: Pun.
\bva „\alst{R}ęginn mik \alst{r}éð, \hld\ hann þik \alst{r}áða mun, &
\ind hann mun okkr verða \alst{b}ǫ́ðum at \alst{b}ana; &
\alst{f}jǫr sitt láta \hld\ hygg at \alst{F}áfnir myni; &
\ind þitt varð nú \alst{m}ęira \alst{m}ęgin.“\eva

\bvb\speakernoteb{[Fathomer quoth:]}%
“Rein betrayed me; he will betray thee; \\
\ind he will become the bane of us both! \\
Give up his life I think that Fathomer will— \\
\ind thy strength was now the greater.”\evb\evg


\bpg\bpa Reginn var á brott horfinn meðan Sig·urðr vá Fáfni ok kom þá aptr er Sig·urðr strauk blóð af sverði’nu. Reginn kvað:\epa

\bpb Rein had disappeared while Siward fought Fathomer, and then came back as Siward wiped the blood off the sword. Rein quoth:\epb\epg


\bvg\bva%
„Hęill þú nú, \alst{S}ig·urðr, \hld\ nú hęfir \alst{s}igr vegit &
\ind ok \alst{F}áfni of \alst{f}arit; &
\alst{m}anna þęira \hld\ es \alst{m}old troða &
\ind þik kveð’k \alst{ȯ}·blauðastan \alst{a}linn.“\eva

\bvb “Hail thee now, Siward—now hast thou won victory \\
\ind and Fathomer destroyed! \\
Of those men who tread on the earth \\
\ind I declare thee unsoftest begotten.”\evb\evg


\bvg\bva%
„Þat ’s \alst{ȯ}·víst at vita \hld\ þá’s komum \alst{a}llir saman, &
\ind \alst{s}ig-tíva \alst{s}ynir, &
\ind hvęrr \alst{ȯ}·blauðastr es \alst{a}linn; &
margr es sá \alst{h}vatr \hld\ es \alst{h}jǫr né rýðr &
\ind \alst{a}nnars brjóstum \alst{ï}.“\eva

\bvb\speakernoteb{[Siward quoth:]}%
“It is unsure to know, when we all come together, \\
\ind sons of the victory-Tews \ken{men}, \\
\ind who is unsoftest begotten. \\
Many a man is bold who reddens no sword \\
\ind in another’s chest.”\evb\evg


\bvg\bva%
„\alst{G}laðr est nú, Sig·urðr, \hld\ ok \alst{g}agni fęginn &
\ind es þú þęrrir \alst{G}ram á \alst{g}rasi; &
\alst{b}róður mïnn \hld\ hęfir þú \alst{b}ęnjaðan &
\ind ok vęld ek þó \alst{s}jalfr \alst{s}umu.“\eva

\bvb\speakernoteb{[Rein quoth:]}%
“Glad art thou now, Siward, and in gain rejoicing \\
\ind when thou driest Gram on the grass. \\
My brother hast thou deathly wounded, \\
\ind and yet I myself bear some fault.”\evb\evg


\bvg\bva%
„Þú því \alst{r}étt \hld\ es ek \alst{r}íða skyldak &
\ind \alst{h}eilǫg fjǫll \alst{h}inig; &
\alst{f}éi ok \alst{f}jǫrvi \hld\ réði sá inn \alst{f}ráni ormr &
\ind nema þú frýðir mér \alst{h}vats \alst{h}ugar.“\eva

\bvb\speakernoteb{[Siward quoth:]}%
“Thou didst counsel that I should ride \\
\ind o’er the holy fells hither. \\
Wealth and life would the gleaming Wyrm rule \\
\ind if thou didst not brave my bold heart.”\evb\evg


\bpg\bpa%
Þá gekk Reginn at Fáfni ok skar hjarta ór hánum með sverði er Riðill heitir ok þá drakk hann blóð ór undinni eptir.\epa

\bpb Then Siward walked up to Fathomer and cut the heart out of him with the sword called Riddle, and then he drank blood from the wound afterwards.\epb\epg


\bvg\bva%
„\alst{S}it-tu nú, \alst{S}ig·urðr, \hld\ en ek mun \alst{s}ofa ganga &
\ind ok halt \alst{F}áfnis hjarta við \alst{f}una! &
\edtrans{\alst{Ęi}skǫld}{heart-strings}{\Bfootnote{An obscure poetic synonym for heart in the neuter plural.  The translation “heart-strings” is guesswork.}} ek vil \hld\ \alst{e}tin láta &
\ind eptir þęnna \alst{d}ręyra \alst{d}rykk.“\eva

\bvb\speakernoteb{[Rein quoth:]}%
“Sit thou now, Siward—but I will go sleep— \\
\ind and hold Fathomer’s heart by the fire! \\
The heart-strings I wish to eat \\
\ind after this drink of blood.”\evb\evg


\bvg\bva%
„\alst{F}jarri þú gekkt \hld\ meðan ek ȧ \alst{F}áfni rauð’k &
\ind mïnn inn \alst{h}vassa \alst{h}jǫr; &
\alst{a}fli mïnu \hld\ átta’k við \alst{o}rms męgin &
\ind meðan þú ï \alst{l}yngvi \alst{l}átt.“\eva

\bvb\speakernoteb{[Siward quoth:]}%
“Far didst thou go while I on Fathomer reddened \\
\ind this my sharp sword. \\
My strength I held against the might of the Wyrm, \\
\ind while thou in the heather layst.”\evb\evg


\bvg\bva%
„\alst{L}ęngi \alst{l}iggja \hld\ létir þú þann \alst{l}yngvi ï, &
\ind inn \alst{a}ldna \alst{jǫ}tun, &
ef þú \alst{s}verðs né nytir, \hld\ þęss es ek \alst{s}jalfr gørða, &
\ind ok þïns ins \alst{h}vassa \alst{h}jǫrs.“\eva

\bvb\speakernoteb{[Rein quoth:]}%
“Long in the heather wouldst thou have let lie \\
\ind this ancient ettin [me], \\
if thou hadst not used the blade which I myself made \\
\ind and this thy sharp sword.”\evb\evg


\bvg\bva%
„\alst{H}ugr es bętri \hld\ en sé \alst{h}jǫrs męgin &
\ind hvar’s \alst{v}ręiðir skulu \alst{v}ega, &
því at \alst{h}vatan mann \hld\ ek sé \alst{h}arð⸗liga vega &
\ind með \alst{s}lę́vu \alst{s}verði \alst{s}igr.\eva

\bvb\speakernoteb{[Siward quoth:]}%
“Heart is better than might of sword may be \\
\ind wherever wroth men should fight, \\
for a bold man I see furiously fighting \\
\ind with a sluggish sword to victory.\evb\evg


\bvg\bva%
\edtrans{\alst{H}vǫtum ’s bętra \hld\ an séi ȯ·\alst{h}vǫtum}{For the bold it is better than it may be for the unbold}{\Bfootnote{The line type “it is better for the X than it may be for the un-X” is apparently formulaic, also appearing in \textlink{Havamal}[71]/1 (with emendation).}} &
\ind ï \alst{h}ildi-lęik \alst{h}afask &
\alst{g}lǫðum ’s bętra \hld\ an séi \alst{g}lúpnanda &
\ind \alst{h}vat sęm at \alst{h}ęndi kømr.“\eva

\bvb For the bold it is better than it may be for the unbold \\
\ind to hold themselves in battle-play \ken{war}; \\
for the glad it is better than it may be for the gloomy \\
\ind no matter what comes to their hands.”\evb\evg

\sectionline

\bpg\bpa \edtrans{Sig·urðr}{Siward}{\Afootnote{The \emph{S} is a bold capital in \Regius.}} tók Fáfnis hjarta ok steikði á teini. Er hann hugði at full-steikt vę́ri ok freyddi sveit’inn ór hjarta’nu þá tók hann á fingri sínum ok skynjaði hvárt full-steikt vę́ri. Hann brann ok brá fingri’num í munn sér. En er hjart-blóð Fáfnis kom á tungu hánum ok skildi hann fugls rǫdd. Hann heyrði at igður klǫkuðu á hrísi’num. Igða’n kvað:\epa

\bpb {\huge S}\textsc{iward took} Fathomer’s heart and roasted it on a stick. When he thought that it was fully roasted and the blood frothed out of the heart then he touched it with his finger to see whether it was fully roasted. He burned himself and jerked his finger in his mouth. But when the heart’s blood of Fathomer came upon his tongue and he understood the speech of birds—he heard that tits were chirping in the brushes. The tit quoth:\epb\epg


\bvg\bva%
\Ballnote{Sts. 32–33 are cited in \Skaldskaparmal\ 47.}%
„Þar \alst{s}itr \alst{S}ig·urðr \hld\ \alst{s}vęita stokkinn, &
\alst{F}áfnis hjarta \hld\ við \alst{f}una stęikir; &
\alst{sp}akr þø̇tti mér \hld\ \alst{sp}illir bauga &
ef hann \alst{f}jǫr-sega \hld\ \alst{f}ránan ę́ti.“\eva

\bvb “There sits Siward spattered with blood; \\
Fathomer’s heart by the fire he roasts. \\
Wise would I think the spiller of rings \\
if he the gleaming life-muscle ate.”\evb\evg


\bvg\bva\speakernote{Ǫnnur:}%
„Þar liggr \alst{R}ęginn, \edtext{\hld}{\Afootnote{add. \emph{kvað ǫnnur} ‘quoth the other one’ \Skaldskaparmal}} \alst{r}ę́ðr umb við sik, &
vill \alst{t}ę́la mǫg \hld\ þann’s \alst{t}rúir hǫ̇num; &
berr af \alst{r}ęiði \hld\ \alst{r}ǫng orð saman, &
vill \alst{b}ǫlva smiðr \hld\ \alst{b}róður hefna.“\eva

\bvb\speakernoteb{The other one:} \\
“There lies Rein, takes counsel with himself, \\
wants to betray the lad who trusts in him. \\
Out of wrath he carries twisted words together; \\
the smith of bales wants to avenge his brother.”\evb\evg


\bvg\bva\speakernote{Þriðja:}%
„\alst{H}ǫfði skęmmra \hld\ láti hann inn \alst{h}ára þul &
\ind fara til \alst{h}ęljar \alst{h}eðan! &
\alst{Ǫ}llu golli \hld\ þá kná hann \alst{ęi}nn ráða, &
\ind \alst{f}jǫlð, því’s und \alst{F}áfni lá.“\eva

\bvb\speakernoteb{The third one:} \\
“A head shorter ought he to let the hoary thyle \\
\ind journey hence to Hell! \\
All the gold he can then rule alone: \\
\ind the trove which ’neath Fathomer lay.”\evb\evg


\bvg\bva\speakernote{Fjórða:}%
„\alst{H}orskr þø̇tti mér \hld\ ef \alst{h}afa kynni &
\alst{ǫ́}st-ráð mikit \hld\ \alst{y}ðvar systra; &
\alst{h}ygði umb sik \hld\ ok \alst{H}ugin ględdi; &
þar’s mér \alst{u}lfs vǫ́n \hld\ es \alst{ęy}ru sé’k.“\eva

\bvb\speakernoteb{The fourth one:}%
“Sharp would he seem to me if he could take \\
the great loving counsel from you, my sisters; \\
he would think for himself and gladden Highen— \\
I expect a wolf where I see an ear!”\evb\evg


\bvg\bva%
„Es⸗at svá \alst{h}orskr \hld\ \alst{h}ildi-meiðr &
sęm ek \alst{h}ęrs jaðar \hld\ \alst{h}yggja mynda’k &
ef hann \alst{b}róður lę́tr \hld\ ȧ \alst{b}rott komask &
ęn hann \alst{ǫ}ðrum hęfr \hld\ \alst{a}ldrs of synjat.“\eva

\bvb\speakernoteb{[Siward quoth:]}%
“The battle-tree \ken{warrior} is not as sharp \\
as I would have thought the peak of the host \ken{ruler} \\
if he lets one brother get away \\
and he has denied the other old age.”\evb\evg


\bvg\bva%
„Mjǫk es \alst{ȯ}·sviðr \hld\ ef hann \alst{ę}nn sparir &
\ind \alst{f}jánda inn \alst{f}olk-skáa, &
þar’s \alst{R}ęginn liggr \hld\ es hann \alst{r}áðinn hęfr; &
\ind kann⸗at hann við \alst{s}líku at \alst{s}éa.“\eva

\bvb\speakernoteb{[Siward quoth:]}%
“Very foolish is he if still he spares \\
\ind that fight-shy fiend, \\
where Rein lies who has planned against him, \\
\ind he cannot look out against such.”\evb\evg


\bvg\bva%
„\alst{H}ǫfði skęmmra \hld\ láti hann þann inn \alst{h}rím-kalda jǫtun &
\ind ok af \alst{b}augum \alst{b}úa; &
þá mund-u \alst{f}éar \hld\ þęss es \alst{F}áfnir réð &
\ind ęin-\alst{v}aldi \alst{v}esa.“\eva

\bvb\speakernoteb{[Siward quoth:]}%
“A head shorter ought he to make that rime-cold ettin, \\
\ind and take from him the bighs, \\
then of the wealth which Fathomer ruled \\
\ind wilt thou be the lone ruler.”\evb\evg


\bvg\bva\speakernote{[Sig·urðr:]}%
„Verða\emph{-t} svá \alst{r}ík skǫp \hld\ at \alst{R}ęginn skyli &
\ind mitt \alst{b}an-orð \alst{b}era &
því at þęir \alst{b}áðir \alst{b}rǿðr \hld\ skulu \alst{b}rá⸗liga &
\ind fara til \alst{H}ęljar \alst{h}eðan.“\eva

\bvb\speakernoteb{[Siward quoth:]}%
“The Shapes will not be so powerful that Rein should \\
\ind bear my bane-word, \\
for those brothers both shall hurriedly \\
\ind journey hence to Hell!”\evb\evg


\bpg\bpa Sig·urðr hjó hǫfuð af Regin ok þá át hann Fáfnis hjarta ok drakk blóð þeira beggja, Regins ok Fáfnis. Þá heyrði Sig·urðr hvað igður mę́ltu:\epa

\bpb Siward struck the head off of Rein and then he ate Fathomer’s heart and drank the blood of them both, Rein’s and Fathomer’s. Then Siward heard what the tits spoke:\epb\epg


\bvg\bva%
„\alst{B}itt þú, Sig·urðr, \hld\ \alst{b}auga rauða; &
es⸗a \alst{k}onung⸗ligt \hld\ \alst{k}víða mǫrgu. &
\alst{M}ęy vęit’k ęina, \hld\ \alst{m}yklu fęgrsta, &
\alst{g}olli \alst{g}ǿdda, \hld\ ef þú \alst{g}eta mę́ttir.\eva

\bvb “Bind, O Siward, the red bighs; \\
it is not kinglike to tarry much. \\
I know one maiden fairest of all, \\
endowed with gold—if thou mighst get her!”\evb\evg


\bvg\bva%
„Liggja til \alst{G}júka \hld\ \alst{g}rø̇nar brautir, &
\alst{f}ramm vísa skǫp \hld\ \alst{f}olk-líðǫndum; &
þar hęfir \alst{d}ýrr konungr \hld\ \alst{d}óttur alna, &
þá \alst{m}unt, Sig·urðr, \hld\ \alst{m}undi kaupa.“\eva

\bvb “Towards Yivick’s home lie green highways: \\
the Shapes show the way forth for wandering exiles. \\
There the wealthy king has reared a daughter; \\
her wilt thou, Siward, for a bride-fee buy.”\evb\evg


\bvg\bva%
„Salr ’s ȧ \alst{h}ǫ́u \hld\ \alst{H}indar-fjalli, &
\alst{a}llr ’s hann \alst{ú}tan \hld\ \alst{ę}ldi svęipinn; &
þann \alst{h}afa \alst{h}orskir \hld\ \alst{h}alir of gǫrvan &
\alst{ó}r \alst{ȯ}·dǫkkum \hld\ \alst{ó}gnar ljóma.“\eva

\bvb “A hall stands on the high Hinderfell; \\
it is all outside by fire enwrapped, \\
that one have wise men made \\
from an un-dark radiance of terror.”\evb\evg


\bvg\bva%
„Vęit’k ȧ \alst{f}jalli \hld\ \alst{f}olk-vitr sofa &
ok \alst{l}ęikr yfir \hld\ \alst{l}indar váði; &
\alst{Y}ggr stakk þorni— \hld\ \alst{a}ðra fęlldi &
\alst{h}ǫr-Gefn \alst{h}ali \hld\ es \alst{h}afa vildi.“\eva

\bvb “I know on the fell a war-wight sleeps \\
and over her licks the linden’s harm \ken{fire}. \\
Ug stung her with a thorn; the flax-Yevn \ken{lady} slew \\
the other heroes who wished to have her.”\evb\evg


\bvg\bva%
„Knátt, \alst{m}ǫgr, séa \hld\ \alst{m}ęy und hjalmi &
þá’s frȧ \alst{v}ígi \hld\ \alst{V}ing·skorni ręið; &
má⸗at \alst{S}igr·drífar \hld\ \alst{s}vefni bręgða, &
\alst{sk}jǫldunga niðr, \hld\ fyr \alst{sk}ǫpum norna.“\eva

\bvb “Thou wilt, lad, see the maiden beneath a helmet, \\
her who from the fray on Wingshorner rode. \\
No man may break Syedrive’s sleep, \\
O heir of the Shieldings, against the Shapes of the Norns.”\evb\evg


\bpg\bpa Sig·urðr reið eptir slóð Fáfnis til bǿlis hans ok fann þat opit ok hurðir af járni ok gę́tti; af járni vóru ok allir timbr-stokkar í húsi’nu en grafit í jǫrð niðr.
Þar fann Sig·urðr stór-mikit gull ok fylldi þar tvę́r kistur.
Þar tók hann ǿgis-hjálm ok gull-brynju ok sverð’it Hrotta ok marga dýr-gripi ok klyfjaði þar með Grana.
En hestr’inn vildi eigi fram ganga fyrr en Sig·urðr steig á bak hǫ́num.\epa

\bpb%
Siward rode along Fathomer’s trail to his dwelling and found it open and doors and rabbets of iron. Of iron were also all the timber trunks in the house, and dug down into the earth. There Siward found very much gold and he filled there two chests. Then he took the helmet of awe and a golden byrnie and the sword Rotte and many precious things and loaded Grane with them. But the horse did not want to go forth until Siward mounted his back.\epb\epg

\sectionline
