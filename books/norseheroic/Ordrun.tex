\bookStart{Weeping of Ordrun}[Odd·rúnar grátr]
\setBookCode{Oddrunargratr}

\begin{flushright}%
\textbf{Dating} \parencite{Sapp2022}: C10th (0.954)

\textbf{Meter:} \Fornyrdislag%
\end{flushright}%

\section{Introduction}

The \textbf{Weeping of Ordrun} (\Oddrunargratr) is another heroic poem.  The following edition and translation is by no means complete.

\newpage

\section{Text}

\subsection{From Burgny and Ordrun (\emph{Frá Borgnýju ok Odd·rúnu})}

\bpg\bpa \edtext{Heið·rekr}{\Afootnote{The \emph{H} (‘\emph{}’) is an ornamented initial 2 lines in height with the left line rising 4 lines high in \Regius.}} hét konungr; dóttir hans hét Borg·ný. Vil·mundr hét sá er var friðill hennar. Hón mátti eigi fǿða bǫrn áðr til kom Odd·rún, Atla systir; hón hafði verit unnusta Gunn·ars, Gjúka sonar. Um þessa sǫgu er hér kveðit:\epa

\bpb {\huge H}\textsc{eathric was the name of a king}.  His daughter’s name was Burgny.  Wilmund was the name of her lover.  She could not bear children before Ordrun, Attle’s sister, arrived.  She had been the lover of Guther, Yivick’s son.  About this saw it is here sung:\epb\epg


\bvg\bva%
\edtext{Hęyrða’k}{\Afootnote{The \emph{H} is an ornamented initial 3 lines in height in \Regius.}} \alst{s}ęgja \hld\ ï \alst{s}ǫgum fornum &
hvé \alst{m}ę́r of kom \hld\ til \alst{M}orna-lands; &
\alst{ę}ngi mátti \hld\ fyr \alst{jǫ}rð ofan &
\alst{H}ęiðreks dóttur \hld\ \alst{h}jalpir vinna.\eva

\bvb {\huge I} \textsc{heard it said} in ancient saws\footnoteB{Probably formulaic; cf. \textlink{Hildebrandslied} 1: \emph{ik gi-hórta dat seggen} ‘I heard it said’ which likewise uses the 1sg pret. of ‘hear’ and the infinitive of ‘say’. Both would go back to a Proto-Northwest Germanic phrase \emph{*ek (ga-)hauʀidō (þat) sagjaną}.} \\
how a maiden came to Mornland; \\
noone could—above the earth— \\
find help for Heathric’s daughter \ken*{= Burgny}.\evb\evg


\bvg\bva%
Þat frá \alst{O}dd·ru̇n, \hld\ \alst{A}tla systir, &
at sú \alst{m}ę́r hafði \hld\ \alst{m}iklar sóttir; &
brá hǫ̇n af \alst{st}alli \hld\ \alst{st}jórn-bitluðum &
ok ȧ \alst{s}vartan \hld\ \alst{s}ǫðul of lagði.\eva

\bvb This learned Ordrun, Attle’s sister, \\
that the maiden \ken*{= Burgny} had great ailments; \\
she grabbed from the stable a rudder-bitted steed, \\
and a black saddle on [it] did lay.\evb\evg


\bvg\bva%
Lét hǫ̇n \alst{m}ar fara \hld\ \alst{m}old-veg sléttan &
und’s at \alst{h}ári kom \hld\ \alst{h}ǫll standandi; &
\edtrans{ok hǫ̇n \alst{i}nn of gekk \hld\ \alst{ę}nd-langan sal}{and she went inside the endlong hall.}{\Bfootnote{The whole line is formulaic; cf. \textlink{Volundarkvida}[8] n.}}; &
\alst{s}vipti hǫ̇n \alst{s}ǫðli \hld\ af \alst{s}vǫngum jó &
\edtext{ok hǫ̇n þat \alst{o}rða \hld\ \alst{a}lls fyrst of kvað:}{\lemma{ok \dots\ of kvað ‘and ... did say’}\Bfootnote{The whole line is formulaic; cf. \textlink{Thrymskvida}[2] n.}}\eva

\bvb She let the steed travel the smooth soil-way \ken{earth} \\
until she came to the high standing house \\
and she went inside the endlong hall. \\
She cast the saddle off the slender horse \\
and she this word first of all did say:\evb\evg


\bvg\bva%
„Hvat es \alst{f}rę́gst \hld\ ȧ \alst{f}oldu &
eða \alst{h}vat er \alst{h}léðst \hld\ \alst{H}u̇na-lands?“ &
„Hér liggr \alst{B}org·ný \hld\ of \alst{b}orin vęrkjum, &
\alst{v}ina þïn, Odd·ru̇n, \hld\ \alst{v}it ef hjalpir!“\eva

\bvb TODO.\evb\evg


\bvg\bva%
„Hvęrr hęfir \alst{v}ísir \hld\ \alst{v}amms of lęitat? &
Hví eru \alst{B}org·nýjar \hld\ \alst{b}ráðar sóttir?“\eva

\bvb TODO.\evb\evg


\bvg\bva%
„\alst{V}il·mundr hét \hld\ \alst{v}inr hauk-stalda, &
hann \alst{v}arði męy \hld\ \alst{v}armri blę́ju, &
\alst{f}imm vetr alla, \hld\ svá hǫ̇n sïnn \alst{f}ǫður lęyndi.“\eva

\bvb TODO.\evb\evg


\bvg\bva%
\alst{Þ}ę́r hykk mę́ltu \hld\ \alst{þ}vígit flęira, &
gekk \alst{m}ild fyr kné \hld\ \alst{m}ęyju at sitja; &
\alst{r}íkt gól Odd·ru̇n, \hld\ \alst{r}ammt gól Odd·ru̇n, &
\alst{b}itra galdra \hld\ at \alst{B}org·nýju.\eva

\bvb TODO.\evb\evg


\bvg\bva%
Knátti \alst{m}ę́r ok \alst{m}ǫgr \hld\ \alst{m}old-veg sporna, &
\alst{b}ǫrn þau in \alst{b}líðu \hld\ við \alst{b}ana Hǫgna; &
þat nam at \alst{m}ę́la \hld\ \alst{m}ę́r fjǫr-sjúka &
svá’t hǫ̇n \alst{ę}kki kvað \hld\ \alst{o}rð it fyrra:\eva

\bvb TODO.\evb\evg


\bvg\bva%
„Svá \alst{h}jalpi þér \hld\ \alst{h}ollar vę́ttir, &
\alst{F}rigg ok \alst{F}ręyja \hld\ ok \alst{f}lęiri goð &
sem þú \alst{f}ęlldir mér \hld\ \alst{f}ár af hǫndum.“\eva

\bvb TODO.\evb\evg


\bvg\bva%
„\alst{H}né’k⸗at ek af því \hld\ til \alst{h}jalpar þér &
at þú \alst{v}ę́rir þess \hld\ \alst{v}erð aldri⸗gi; &
\alst{h}ét’k ok ęfnda’k \hld\ es ek \alst{h}inig mę́lta &
at ek \alst{h}ví-vetna \hld\ \alst{h}jalpa skylda’k &
þȧ’s \alst{ǫ}ðlingar \hld\ \alst{a}rfi skiptu.“\eva

\bvb TODO.\evb\evg


\bvg\bva%
Þȧ nam at \alst{s}ętjask \hld\ \alst{s}org-móð kona &
at \alst{t}ęlja bǫl \hld\ af \alst{t}rega stórum:\eva

\bvb TODO.\evb\evg


\bvg\bva%
„Vas’k \alst{u}pp \alst{a}lin \hld\ ï \alst{jǫ}fra sal, &
\alst{f}lęstr \alst{f}agnaði, \hld\ at \alst{f}ira ráði; &
\alst{u}nða ek \alst{a}ldri \hld\ ok \alst{ęi}gn fǫður &
\alst{f}imm vetr ęina \hld\ svá’t minn \alst{f}aðir lifði.\eva

\bvb I was reared up in the princely hall, \\
TODO. \\
I was content with life and my father’s estate \\
for but five winters while my father lived.\evb\evg


\bvg\bva%
Þat nam at \alst{m}ę́la, \hld\ \alst{m}ál it øfsta &
\alst{s}já móðr konungr \hld\ áðr hann \alst{s}ylti.\eva

\bvb TODO.\evb\evg


\bvg\bva%
Mik bað hann \alst{g}ǿða \hld\ \alst{g}olli rauðu &
ok \alst{s}uðr gefa \hld\ \alst{s}yni Grím·hildar; &
kvað-a hann ina \alst{ǿ}ðri \hld\ \alst{a}lna myndu &
\alst{m}ęy ï hęimi \hld\ nema \alst{m}jǫtuðr spillti.“\eva

\bvb TODO.\evb\evg


\bvg\bva%
„\alst{Ǿ}r est, \alst{O}dd·ru̇n, \hld\ ok \alst{ø}r·vita &
es þú mér af \alst{f}ári \hld\ \alst{f}lęst orð of kvað; &
ęn ek \alst{f}ylgða’k þér \hld\ ȧ \alst{f}jǫrgynju &
sęm vit \alst{b}rǿðrum tvęim \hld\ of \edtext{\alst{b}or\emph{nar}}{\Afootnote{\emph{borin} \Regius}} vę́rim.\eva

\bvb TODO.\evb\evg


\bvg\bva%
\edtext{Man’k hvat mę́ltir \hld\ ęnn umb aptan}{\Bfootnote{This line is clearly corrupt as seen by the lack of alliteration connecting the two halves.  The simplest solution is to switch the places of \emph{mę́ltir} and \emph{ęnn}, but this produces a three-syllabic a-verse and a five-syllabic b-verse.  Another possibility would be \emph{Man’k hvat *\alst{ę}nn \hld\ umb \alst{a}ptan mę́ltir}, but the metrics are still questionable.}} &
þȧ’s ek \alst{G}unn·ari \hld\ \alst{g}ørða’g drekku; &
\alst{s}líks dǿmi kvað⸗at-tu \hld\ \alst{s}íðan mundu &
\alst{m}ęyju verða \hld\ nema \alst{m}ér ęinni.\eva

\bvb TODO.\evb\evg


\bvg\bva%
\alst{B}ryn·hildr ï \alst{b}úri \hld\ \alst{b}orða rakði, &
hafði hǫ̇n \alst{l}ýði \hld\ ok \alst{l}ǫnd umb sik; &
\alst{j}ǫrð dúsaði \hld\ ok \alst{u}pp-himinn &
þȧ’s \alst{b}ani Fáfnis \hld\ \alst{b}org of þȧtti.\eva

\bvb TODO.\evb\evg


\bvg\bva%
Þȧ vas \alst{v}íg \alst{v}egit \hld\ \alst{v}ǫlsku sverði &
ok \alst{b}org \alst{b}rotin, \hld\ sú’s \alst{B}ryn·hildr átti; &
\alst{v}as⸗a langt af því \hld\ hęldr \alst{v}ǫ́-lítit &
und’s þę́r \alst{v}élar \hld\ \alst{v}issi allar.\eva

\bvb TODO.\evb\evg


\bvg\bva%
Þęss lét hǫ̇n \alst{h}arðar \hld\ \alst{h}ęfndir verða &
svá’t vér \alst{ǫ}ll hǫfum \hld\ \alst{ǿ}rnar raunir; &
þat mun ȧ \alst{h}ǫlða \hld\ \alst{h}vęrt land fara &
es hǫ̇n lét \alst{s}veltask \hld\ at \alst{S}ig·urði.\eva

\bvb TODO.\evb\evg


\bvg\bva%
Ęn ek \alst{G}unn·ari \hld\ \alst{g}at’k at unna, &
\alst{b}auga dęili, \hld\ sęm \alst{B}ryn·hildr skyldi; &
ęn hann \alst{B}ryn·hildi \hld\ \alst{b}að hjalm geta, &
hana kvað hann \alst{ȯ}sk-męy \hld\ \alst{v}erða skyldu.\eva

\bvb TODO.\evb\evg


\bvg\bva%
\alst{B}uðu þęir ár⸗la \hld\ \alst{b}auga rauða &
ok \alst{b}rǿðr mïnum \hld\ \alst{b}ǿtr ȯ·smá\emph{a}r; &
\alst{b}auð hann ęnn við mér \hld\ \alst{b}ú \edtext{fimm-t\emph{ía}n}{\Afootnote{\emph{.xv.} \Regius}\lemma{fimm-t\emph{ía}n ‘fifteen’}\Bfootnote{For metrical reasons the abbreviation \emph{xv} is resolved to the archaic hiatus form \emph{fimm-tían} rather than the standard \emph{fimm-tán}.
The meter of \Oddrunargratr\ does rarely permit 3-syllable verses (\textlink{Oddrunargratr}[4]/1a–b, 4/2b, 6/1a), but restoring hiatus is the better option since it agrees with the typical metrical patterns of the poem; \emph{-tían} is also metrically secured from the early C11th by Þorm \emph{Þorgdr} (\Skp\ 5), 1/4b: \emph{fimm-tían} ‘fifteen’, 15/2b: \emph{þré-tían} ‘thirteen’.

The suffix \emph{-t(j)án} ‘-teen’ in the numerals 13–19 behaves curiously in classical ON and modern Icelandic.  In 13–16 (\emph{þré-tán, fjór-tan, fimm-tán, sex-tán}) the suffix is \emph{-tán}, but in 17–19 (\emph{sjau-tján, át-tján, ní-tjan}) it is \emph{-tján} (< \emph{-tían}).  The latter is undoubtedly the older form, deriving from PN \emph{*-tihan} corresponding to OHG \emph{-zehan}, Gothic \emph{-taihun}; see further \textcite{KonradGislason1879}, \textcite[137--142]{Kock1893}.}}, &
\alst{h}lið-farm Grana \hld\ ef hann \alst{h}afa vildi.\eva

\bvb TODO.\evb\evg


\bvg\bva%
En \alst{A}tli kvað⸗sk \hld\ \alst{ęi}gi vilja &
\alst{m}und aldri⸗gi \hld\ at \alst{m}ęgi Gjúka; &
þęy⸗gi vit \alst{m}ǫ́ttum \hld\ við \alst{m}unum vinna &
nema ek \alst{h}élt \alst{h}ǫfði \hld\ við \alst{h}ring-brota.\eva

\bvb TODO.\evb\evg


\bvg\bva%
\alst{M}ę́ltu \alst{m}argir \hld\ \alst{m}ïnir niðjar, &
kvǫ́ðu⸗sk \alst{o}kkr hafa \hld\ \alst{o}rðit bę́ði &
en mik \alst{A}tli kvað \hld\ \alst{ęi}gi myndu &
\alst{l}ýti ráða \hld\ né \alst{l}ǫst gøra.\eva

\bvb TODO.\evb\evg


\bvg\bva%
En \alst{s}líks skyli \hld\ \alst{s}ynja aldri &
\alst{m}aðr fyr annan \hld\ þar’s \alst{m}un-úð dę̇ilir.\eva

\bvb TODO.\evb\evg


\bvg\bva%
\edtrans{Sęndi \alst{A}tli \hld\ \alst{ǫ́}ru sïna}{Attle sent his messengers}{\Bfootnote{Cf. \textlink{Atlakvida}[1]/1.}} &
of \alst{m}yrkvan við \hld\ \alst{m}ïn at fręista; &
ok þęir \alst{k}vǫ́mu \hld\ þar’s \alst{k}oma né skyldu⸗t &
þȧ’s \alst{b}ręiddu vit \hld\ \alst{b}lę́ju ęina.\eva

\bvb Attle sent his messengers \\
TODO.\evb\evg


\bvg\bva%
\alst{B}uðu vit þegnum \hld\ \alst{b}auga rauða &
at þęir \alst{ęi}gi til \hld\ \alst{A}tla sęgði; &
en þęir \alst{ȯ}⸗liga \hld\ \alst{A}tla sǫgðu &
ok \alst{h}vat⸗liga \hld\ \alst{h}ęim skunduðu.\eva

\bvb TODO.\evb\evg


\bvg\bva%
Ęn þęir \alst{G}uð·ru̇nu \hld\ \alst{g}ǫr⸗la lęyndu &
því at hǫ̇n \alst{h}ęldr vita \hld\ \alst{h}ǫlfu skyldi.\eva

\bvb TODO.\evb\evg


\bvg\bva%
\alst{H}lymr vas at \alst{h}ęyra \hld\ \alst{h}óf-gollinna &
þȧ’s ï \alst{g}arð riðu \hld\ \alst{G}júka arfar; &
þęir ór \alst{H}ǫgna \hld\ \alst{h}jarta skǫ́ru &
ęn ï \alst{o}rm-garð \hld\ \alst{a}nnan lǫgðu.\eva

\bvb There was a din to be heard from the golden-hooved ones \\
when into the court rode Yivick’s heirs; \\
out of Hain they cut the heart, \\
but in the snake-pit they laid the other.\evb\evg


\bvg\bva%
Vas’k \alst{ę}nn farin \hld\ \alst{ęi}nu sinni &
til \alst{G}ęir·mundar \hld\ \alst{g}ørva drykkju; &
nam \alst{h}orskr konungr \hld\ \alst{h}ǫrpu svęigja &
því’t hann \alst{h}ugði mik \hld\ til \alst{h}jalpar sér, &
\alst{k}yn-ríkr \alst{k}onungr, \hld\ of \alst{k}oma myndu.\eva

\bvb TODO.\evb\evg


\bvg\bva%
Nam’k at \alst{h}ęyra \hld\ ór \alst{H}lés-ęyju &
hvé þar af \alst{st}ríðum \hld\ \alst{st}ręngir mę́ltu, &
\alst{b}að’k ambáttir \hld\ \alst{b}u̇nar verða, &
vilda’k \alst{f}ylkis \hld\ \alst{f}jǫrvi bjarga; &
létum \alst{f}ljóta \hld\ \alst{f}ar lund yfir &
und’s \alst{a}lla sá’k \hld\ \alst{A}tla garða.\eva

\bvb TODO.\evb\evg


\bvg\bva%
Þȧ kom in \alst{a}rma \hld\ \alst{ú}t skę́vandi, &
\alst{m}óðir Atla, \hld\ hǫ̇n skyli \alst{m}orna!\eva

\bvb TODO.\evb\evg


\bvg\bva%
Ok \alst{G}unn·ari \hld\ \alst{g}róf til hjarta &
svá’t \alst{m}átti’g⸗a’k \hld\ \alst{m}ę́rum bjarga.\eva

\bvb TODO.\evb\evg


\bvg\bva%
\alst{O}pt \alst{u}ndr⸗umk þat \hld\ hví \alst{ę}ptir má’k, &
\alst{l}inn-vęngis Bil, \hld\ \alst{l}ífi halda &
es ek \alst{ó}gn-hvǫtum \hld\ \alst{u}nna þȯtt⸗umk, &
\alst{s}verða dęili, \hld\ sęm \alst{s}jalfri mér.\eva

\bvb TODO.\evb\evg


\bvg\bva%
\alst{S}atst ok hlýddir \hld\ meðan \alst{s}agða’k þér &
\alst{m}ǫrg ill \edtext{of skǫp}{\Bfootnote{An instance of the archaic particle \emph{of} in a prenominal position, corresponding to a lost Germanic prefix \emph{*ga-}; see \textlink{Sigurdskamma}[23]/1 n.  \emph{of skǫp} (< PGmc. \emph{*ga·skapō} > OS \emph{gi·skapu}, OE \emph{ge·sceap}) also occurs in \textlink{Sigurdskamma}[58]/5 and the present occurrence may be due to borrowing.}} \hld\ \alst{m}ïn ok þęira; &
\alst{m}aðr hvęrr lifir \hld\ at \alst{m}unum sïnum; &
nú ’s of \alst{g}ęnginn \hld\ \alst{g}rátr Odd·ru̇nar.\eva

\bvb TODO.\evb\evg

\sectionline
