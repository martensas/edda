\bookStart{Second Lay of Hallow Hundingsbane}[Helgakviða Hundingsbana aðra]

\begin{flushright}%
\textbf{Dating} \parencite{Sapp2022}: late C11th (0.587)

\textbf{Meter:} \Fornyrdislag\ (TODO)%
\end{flushright}

\section{Introduction}

TODO: Introduction.

The latter part of the poem features a touching description of Syreun’s visit to Hallow’s grave.  It reflects a folkloric motif found in many traditional British ballads, e.g. Roud 50 (Sweet William’s Ghost), Roud 179 (the Lover’s Ghost or the Grey Cock), and Roud 22568 (the Night Visiting Song), where two lovers must part at cock-crow, although in some variants of 179 and 22568 the supernatural element is not explicit.  Compare the version recorded by \emph{The Dubliners} in 1972:

\begin{quote}\itshape I must away now; I can no longer tarry \\
This morning’s tempest I have to cross \\
I must be guided without a stumble \\
Into the arms I love the most. \\

And when he came to his true love’s dwelling \\
He knelt down gently upon a stone \\
And through her window he’s whispered lowly: \\
“Is my true lover within at home?” \\

“Wake up, wake up, love, it is thine own true lover \\
Wake up, wake up, love, and let me in \\
For I am tired, love, and oh so weary \\
And more than near drenched to the skin.” \\

She’s raised her off her down soft pillow \\
She’s raised her up and she’s let him in \\
And they were locked in each other’s arms \\
Until that long night was past and gone. \\

And when that long night was past and over \\
And when the small clouds began to grow \\
He’s taken her hand and they’ve kissed and parted \\
Then he saddled and mounted and away did go. \\

I must away now \emph{et c.}\end{quote}

\sectionline

\section{The Second Lay of Hallow Hundingsbane}

... TODO ...

\bpg\bpa Hęlgi fekk Sigrúnar ok ǫ́ttu þau sonu; vas Hęlgi ęigi gamall.  Dagr Hǫgna sonr blótaði Óðin til fǫður-hefnda. Óðinn léði Dag gęirs síns.  Dagr fann Helga, mág sinn, þar sem hęitir at Fjǫturlundi.  Hann lagði í gǫgnum Hęlga með gęir’num.  Þar fell Hęlgi, en Dagr ręið til fjalla ok sagði Sigrúnu tíðindi:\epa

\bpb Hallow got Syerun and they had sons; Hallow was not old.  Day, son of Hain, made a \inx[C]{bloot} to Weden for the sake of avenging his father.  Weden lent Day his spear. Day found Hallow, his brother-in-law, where it is called Fetterlund; he ran through Hallow with the spear.  There Hallow fell, but Day rode to the fells and told Syerun the tidings:\epb\epg


\bvg\bva „\alst{T}rauðr em ek, systir, \hld\ \alst{t}rega þér at sęgja &
því-at ek hęfi \alst{n}auðigr \hld\ \alst{n}ipti grǿtta: &
\alst{F}ell í morgun \hld\ und \alst{F}jǫturlundi &
\alst{b}uðlungr sá’s vas \hld\ \alst{b}ętstr í hęimi &
ok \alst{h}ildingum \hld\ á \alst{h}alsi stóð.“\eva

\bvb “Regretful am I, O sister, to grieve thee by saying it— \\
for, forced, must I make my kinswoman weep: \\
this morning fell in Fetterlund \\
that noble who was the best in the world, \\
and on the throats of princes stood.”\evb\evg


\bvg\bva\speakernote{[Sigrún kvað:]}%
„Þik skyli \alst{a}llir \hld\ \alst{ęi}ðar bíta, &
þęir es \alst{H}ęlga \hld\ \alst{h}afðir unna, &
at inu \alst{l}jósa \hld\ \alst{L}ęiptrar vatni &
ok at \alst{ú}r-svǫlum \hld\ \alst{U}nnar steini!\eva

\bvb “\emph{Thee} should all oaths bite, \\
which thou to Hallow hast sworn, \\
by the shining water of Lafter, \\
and by the spray-cold stone of Ithe.\evb\evg


\bvg\bva \alst{Sk}ríði-at þat \alst{sk}ip, \hld\ es und þér \alst{sk}ríði, &
þótt \alst{ó}ska-byrr \hld\ \alst{e}ptir lęggisk! &
\alst{R}enni-a sá marr, \hld\ es und þér \alst{r}enni, &
þótt \alst{f}íęndr þína \hld\ \alst{f}orðask ęigir!\eva

\bvb May the ship not glide, which glides beneath thee, \\
though it has a wished-for gust behind it! \\
May the sea not run, which runs beneath thee, \\
though from thy foes thou must escape!\evb\evg


\bvg\bva \alst{B}íti-a þér þat sverð, \hld\ es þú \alst{b}ręgðir, &
nema \alst{s}jǫlfum þér \hld\ \alst{s}yngvi of hǫfði! &
Þá vę́ri þér \alst{h}ęfnt \hld\ \alst{H}ęlga dauða, &
ef þú \alst{v}ę́rir \alst{v}argr \hld\ á \alst{v}iðum úti, &
\alst{a}uðs \alst{a}nd-vani \hld\ ok \alst{a}lls gamans, &
\alst{h}ęfðir ęigi mat, \hld\ nema á \alst{h}rę́um spryngir!“\eva

\bvb May the sword not bite for thee, which thou brandishest, \\
save it sing over thy very own head! \\
\emph{Then} were on thee Hallow’s death avenged, \\
if thou wert a wolf in the woods outside, \\
deprived of wealth and all pleasure; \\
hadst no food, save thou plundered carrion!“\evb\evg


\bvg\bva\speakernote{Dagr kvað:}%
„\edtext{\alst{Ǿ}r ert, systir, \hld\ ok \alst{ø}r-vita}{\lemma{Ǿr \dots\ ok ør-viti ‘Mad \dots\ and out of wits’}\Bfootnote{Formulaic, also occurring in \Lokasenna\ and others TODO.}}, &
es \alst{b}rǿðr þínum \hld\ \alst{b}iðr for-skapa! &
\alst{Ęi}nn vęldr \alst{Ó}ðinn \hld\ \alst{ǫ}llu bǫlvi, &
því-at með \alst{s}ifjungum \hld\ \alst{s}ak-rúnar bar!\eva

\bvb\speakernoteb{Day quoth:}“Mad art thou, sister, and out of wits, \\
when onto thy brother thou dost bid a cruel \inx[C]{shape}. \\
Weden alone causes all the bale, \\
for he bore strife-runes among relatives!\evb\evg


\bvg\bva Þér \alst{b}ýðr \alst{b}róðir \hld\ \alst{b}auga rauða, &
ǫll \alst{V}andils-\alst{v}é \hld\ ok \alst{V}íg-dali; &
\alst{h}af \alst{h}alfan \alst{h}ęim \hld\ \alst{h}arms at gjǫldum &
\alst{b}rúðr \alst{b}aug-varið \hld\ ok \alst{b}úrir þínir.\eva

\bvb \emph{Thee} thy brother offers red bighs, \\
all Wendelswigh and the Wighdales. \\
Have half the realm as recompense for the injury, \\
O bigh-adorned bride—and thy sons, too.\evb\evg


\bvg\bva „\alst{S}it’k-a svá \alst{s}ę́l \hld\ at \alst{S}efa-fjǫllum, &
\alst{á}r né of nę́tr, \hld\ at ek \alst{u}na lífi, &
nema at \alst{l}iði \alst{l}ofðungs \hld\ \alst{l}jóma bręgði, &
renni und \alst{v}ísa \hld\ \alst{V}íg-blę́r þinig, &
\alst{g}ull-bitli vanr, \hld\ knega’k \alst{g}rami fagna!\eva

\bvb “I will not sit so happy in the Sevefells, \\
at dawn nor night, that I should be content with life, \\
unless the retinue of the man of praise were struck with light: \\
{[and]} beneath the ruler ran Wighblaw hither, \\
wont to the golden bit—{[and]} I might greet the prince!\evb\evg


\bvg\bva Svá \alst{h}afði \alst{H}ęlgi \hld\ \alst{h}rę́dda gǫrva &
\alst{f}jándr sína alla \hld\ ok \alst{f}rę́ndr þęira, &
sem fyr \alst{u}lfi \hld\ \alst{ó}ðar rynni &
\alst{g}ęitr af fjalli, \hld\ \alst{g}ęiska fullar!\eva

\bvb So would Hallow have terrified \\
his enemies all and their kinsmen, \\
like from a wolf did madly run \\
goats down a fell, full of fright.\evb\evg


\bvg\bva \edtext{Svá bar \alst{H}ęlgi \hld\ af \alst{h}ildingum &
sem \alst{í}tr-skapaðr \hld\ \alst{a}skr af þyrni &
eða sá \alst{d}ýr-kalfr \hld\ \alst{d}ǫggu slunginn &
es \alst{ø}fri fęrr \hld\ \alst{ǫ}llum dýrum, &
ok \alst{h}orn glóa \hld\ við \alst{h}imin sjalfan.“}{\lemma{ALL}\Bfootnote{Cf. the very similar description of Siward in \GudrunTwo\ 2.}}\eva

\bvb So did Hallow surpass the princes \\
like the nobly shaped ash the thorn, \\
or the deer-calf, dew-besprinkled, \\
who fares higher than all beasts, \\
and its horns gleam against heaven itself.”\evb\evg


\bpg\bpa Haugr var gǫrr eptir Helga.  En er hann kom til Valhallar, þá bauð Óðinn hánum ǫllu at ráða með sér.  Helgi kvað:\epa

\bpb A barrow was made for Hallow.  But when he came to Walhall Weden offered him to rule everything together with him.  Hallow quoth:\epb\epg


\bvg\bva „Þú skalt, \alst{H}undingr, \hld\ \alst{h}vęrjum manni &
\alst{f}ót-laug geta \hld\ ok \alst{f}una kynda; &
\alst{h}unda binda, \hld\ \alst{h}esta gę́ta, &
gefa \alst{s}vínum \alst{s}oð, \hld\ áðr \alst{s}ofa gangir!“\eva

\bvb “Thou shalt, Hunding, for every man \\
make a foot-bath and kindle the fire, \\
bind the hounds, feed the horses, \\
give broth to the swine—before thou mightst go to sleep!”\evb\evg


\bpg\bpa Ambótt Sigrúnar gekk um aptan hjá haugi Helga ok sá at Helgi reið til haugs’ins með marga menn. Ambótt kvað:\epa

\bpb Syerun’s maid-servant walked by Hallow’s barrow at evening, and saw that Hallow rode to the barrow with many men.  The maid-servant quoth:\epb\epg


\bvg\bva „Hvárt ’ru þat \alst{s}vik ęin \hld\ es \alst{s}éa þikkjumk &
eða \alst{r}agna \alst{r}ǫk \hld\ \alst{r}íða męnn dauðir, &
es \alst{jó}a \alst{y}ðra \hld\ \alst{o}ddum kęyrið, &
eða es \alst{h}ildingum \hld\ \alst{h}ęim-fǫr gefin?“\eva

\bvb “Either these are only tricks, as I seem to see \\
—or the \inx[L]{Rakes of the Reins}?—dead men riding; \\
as ye drive your steeds on by spear-points— \\
or are the princes granted leave to go home?”\evb\evg


\bvg\bva\speakernote{[Ęinn þęira kvað:]}%
„Es-a þat \alst{s}vik ęin \hld\ es \alst{s}éa þikkisk &
né \edtrans{\alst{a}ldar rof}{Ripping of the Age}{\Bfootnote{Formulaic.  Cf. TODO \emph{rjúfask ręgin}. This is the same root, only zero-grade.}} \hld\ þótt-u \alst{o}ss lítir, &
þótt vér \alst{jó}a \alst{ó}ra \hld\ \alst{o}ddum keyrim, &
né es \alst{h}ildingum \hld\ \alst{h}ęim-fǫr gefin.“\eva

\bvb\speakernoteb{[One of them quoth:]}%
“It is not only tricks, as thou seemest to see— \\
nor the Ripping of the Age, although thou behold us; \\
although we drive our steeds on by spear-points \\
the princes are not granted leave to go home.”\evb\evg


\bpg\bpa Heim gekk ambótt ok sagði Sigrúnu:\epa

\bpb The maid-servant walked home and said to Syerun:\epb\epg


\bvg\bva „Út gakk \alst{S}igrún, \hld\ frá \alst{S}ęfa-fjǫllum &
ef þik \alst{f}olks jaðarr \hld\ \alst{f}inna lystir; &
upp ’s \alst{h}augr lokinn, \hld\ kominn es \alst{H}ęlgi! &
\alst{D}ólg-spor \alst{d}ręyra \hld\ \alst{d}ǫglingr bað þik &
at þú \alst{s}ár-dropa \hld\ \alst{s}vęfja skyldir.“\eva

\bvb “Go out, O Syerun from the Sevefells, \\
if thou hast lust to find the leader of the troop! \\
The barrow is unlocked; Hallow is come! \\
The ruler of bloody wounds bade thee \\
that thou his wound-drops shouldst soothe.”\evb\evg


\bpg\bpa Sigrún gekk í haug’inn til Helga ok kvað:\epa

\bpb Syerun walked into Hallow’s barrow, and quoth:\epb\epg


\bvg\bva „Nú em’k svá \alst{f}ęgin \hld\ \alst{f}undi okkrum &
sem \alst{á}t-frękir \hld\ \alst{Ó}ðins haukar &
es \alst{v}al \alst{v}itu, \hld\ \alst{v}armar bráðir, &
eða \alst{d}ǫgg-litir \hld\ \alst{d}ags-brún séa.“\eva

\bvb “Now do I so rejoice at our meeting, \\
like do the ravenous hawks of Weden \ken{ravens} \\
when they know corpses, warm venison, \\
or, gleaming with dew, they see the day’s brow \ken{dawn}.\evb\evg


\bvg\bva Fyrr vil’k \alst{k}yssa \hld\ \alst{k}onung ó·lifðan &
an þú \alst{b}lóðugri \hld\ \alst{b}rynju kastir; &
\alst{h}ár ’s þitt, \alst{H}elgi, \hld\ \alst{h}élu þrungit, &
\edtrans{allr es \alst{v}ísi \hld\ \alst{v}al-dǫgg slęginn}{the prince is all with corpse-dew whipped}{\Bfootnote{Cf. \Baldrsdraumar\ 5, where the dead wallow says something similar.}}, &
\alst{h}ęndr úr-svalar \hld\ \alst{H}ǫgna mági; &
hvé skal’k þér, \alst{b}uðlungr, \hld\ þess \alst{b}ót of vinna?“\eva

\bvb Sooner would I kiss the unliving king, \\
than thou the bloody byrnie mightst cast away! \\
Thy hair is, O Hallow, with hoarfrost swollen; \\
the prince is all with corpse-dew \ken{blood} whipped; \\
the hands spray-cold on Hain’s in-law \ken*{= Hallow}.— \\
How shall I for thee, O noble, remedy that?”\evb\evg


\bvg\bva\speakernote{[Hęlgi kvað:]}„Ęin vęldr þú, \alst{S}igrún \hld\ frá \alst{S}efafjǫllum, &
es \alst{H}ęlgi es \hld\ \alst{h}arm-dǫgg slęginn: &
\alst{G}rę́tr þú, \alst{g}ull-varið, \hld\ \alst{g}rimmum tǫ́rum, &
\alst{s}ól-bjǫrt \alst{s}uð-rǿn, \hld\ áðr þú \alst{s}ofa gangir, &
hvęrt fęllr \alst{b}lóðugt \hld\ á \alst{b}rjóst grami, &
\alst{ú}r-svalt, \alst{i}nn-fjalgt \hld\ \alst{ę}kka þrungit.\eva

\bvb “Thou alone causest, O Syerun from the Sevefells, \\
that Hallow be with harm-dew whipped. \\
Thou weepest—O gold-covered—bitter tears— \\
O sun-bright southern lady—before thou go to sleep. \\
Each one falls bloody on the prince’s chest, \\
spray-cold, stifled, pressed forth by grief.\evb\evg


\bvg\bva Vęl skulum \alst{d}rekka \hld\ \alst{d}ýrar vęigar &
þótt \alst{m}isst hafim \hld\ \alst{m}unar ok landa! &
Skal \alst{ę}ngi maðr \hld\ \alst{a}ngr-ljóð kveða &
þótt mér á \alst{b}rjósti \hld\ \alst{b}ęnjar líti. &
Nú eru \edtext{\alst{b}rúðir \hld\ \alst{b}yrgðar í haugi, &
\alst{l}ofða dísir, \hld\ hjá oss}{\lemma{brúðir, dísir, oss ‘brides, dises, us’}\Bfootnote{Hallow speaks in the plural.  “Now has my bride, my goddess, come into the barrow, next to me, who am dead.”}} \alst{l}iðnum!“\eva

\bvb Well shall we drink dear draughts, \\
although we have lost both love and land! \\
Let no one sing songs of sorrow, \\
although he behold the wounds on my chest. \\
Now are the brides shut within the barrow, \\
the praised one’s \inx[C]{dise}[dises], next to us, passed-on.”\evb\evg


\bpg\bpa Sigrún bjó sę́ing í haug’inum.\epa

\bpb Syerun made a bed in the barrow:\epb\epg


\bvg\bva „\alst{H}ér hęfi’k þér, \alst{H}ęlgi, \hld\ \alst{h}vílu gørva, &
\alst{a}ngr-lausa mjǫk, \hld\ \alst{Y}lfinga niðr; &
vil’k þér í \alst{f}aðmi, \hld\ \alst{f}ylkir, sofna &
\edtrans{sem’k \alst{l}ofðungi \hld\ \alst{l}ifnum mynda’k!}{like I would with the living man of praise}{\Bfootnote{i.e. “just as I would if you were still alive.”}}“\eva

\bvb “Here I’ve for thee, Hallow, made a place of rest, \\
all without sorrow, O kinsman of the Wolvings! \\
I will in thy arms, O marshal, fall asleep, \\
like I would with the living man of praise.”\evb\evg


\bvg\bva\speakernote{[Hęlgi kvað:]}„Nú kveð’k \alst{ę}nskis \hld\ \alst{ø}r-vę́nt vesa, &
\alst{s}íð né \alst{s}nimma, \hld\ at \alst{S}efa-fjǫllum &
es þú á \alst{a}rmi \hld\ \alst{ó}·lifðum søfr, &
\alst{h}vít, í \alst{h}augi, \hld\ \alst{H}ǫgna dóttir, &
ok est-u \alst{k}vik, \hld\ in \alst{k}onung-borna!“\eva

\bvb\speakernoteb{[Hallow quoth:]}%
“Now, I say, there is naught more missing \\
neither late nor soon from the Sevefells, \\
when thou dost sleep on the unliving arm, \\
O white daughter of Hain—in the barrow, \\
and thou art alive!—of kingly birth.”\evb\evg

\sectionline

{\small (The night has passed; dawn is breaking, and Hallow speaks.  The manuscript does not indicate the change of scene.)}

\sectionline

\bvg\bva\speakernote{[Hęlgi kvað:]}„Mál ’s mér at \alst{r}íða \hld\ \edtrans{\alst{r}oðnar}{reddening}{\Bfootnote{From the rising dawn.}} brautir, &
láta \alst{f}ǫlvan jó \hld\ \alst{f}lug-stíg troða; &
skal’k fyr \alst{v}estan \hld\ \alst{v}ind-hjalms brúar &
áðr \alst{S}al-gofnir \hld\ \alst{s}igr-þjóð vęki.“\eva

\bvb “’Tis time for me to ride the reddening roads, \\
to let my pale steed tread the path of flight \ken{sky/heaven}. \\
I shall go west of the wind-helm’s bridges \ken{sky/heaven > clouds?}, \\
before Salgovner may awaken the victorious folk.”\evb\evg


\bpg\bpa Þęir Hęlgi riðu lęið sína, en þę́r fóru hęim til bǿjar. Annan aptan lét Sigrún ambótt halda vǫrð á haugi’num.  En at dag-setri, es Sigrún kom til haugs’ins, hón kvað:\epa

\bpb Hallow and his men rode on their way, but the women journeyed home to the farm. The next evening Syerun made her maid-servant keep watch on the barrow.  And at sunset as Syerun came to the barrow, she \ken*{= the maid-servant} quoth:\epb\epg


\bvg\bva „\alst{K}ominn vę́ri nú, \hld\ ef \alst{k}oma hygði, &
\alst{S}igmundar burr \hld\ frá \alst{s}ǫlum Óðins; &
kveð’k \alst{g}rams þinig \hld\ \alst{g}rę́nask vánir &
\edtrans{es á \alst{a}sk-limum \hld\ \alst{ę}rnir sitja}{when on ashen branches eagles sit}{\Bfootnote{i.e. “when the eagles roost on yonder trees”.  This is a sign of Hallow and his men not coming; if they were, the eagles would be following them and picking at their bodies.}} &
ok \edtext{\alst{d}rífr \alst{d}rótt ǫll \hld\ \alst{d}raum-þinga til}{\lemma{drífr \dots\ draum-þinga til ‘drifts off to dream-Things’}\Bfootnote{i.e. “falls asleep”.  A fine metaphor.}}.“\eva

\bvb “Come were now, if to come he had thought, \\
Syemund’s son \ken*{= Hallow} from Weden’s halls; \\
hopes fade, I say, of the prince’s coming, \\
when on ashen branches eagles sit, \\
and all mankind drifts off to dream-\inx[C]{Thing}[Things].\evb\evg


\bvg\bva Ves \alst{ęi}gi svá \alst{ǿ}r \hld\ at \alst{ęi}n farir, &
\alst{d}ís skjǫldunga, \hld\ \alst{d}raug-húsa til! &
Verða \alst{ǫ}flgari \hld\ \alst{a}llir á nǫ́ttum &
\alst{d}auðir \alst{d}ólgar, mę́r, \hld\ an of \alst{d}aga ljósa.“\eva

\bvb Be not so mad that thou journey alone, \\
O dise of the Shieldings, to the ghost-houses! \\
Mightier at night do all become \\
dead fiends, O maiden, than during the bright days!”\evb\evg


\bpg\bpa Sigrún varð skamm-líf af harmi ok trega. Þat var trúa í forneskju, at menn vę́ri endr-bornir, en þat er nú kǫlluð kerlinga-villa.  Helgi ok Sigrún er kallat at vę́ri endr-borin.  Hét hann þá Helgi Haddingjaskati en hon Kára Hálfdanar dóttir, svá sem kveðit er í \edtrans{Káruljóðum}{Leeds of Cheer}{\Bfootnote{A now-lost heroic poem.}}, ok var hon val-kyrja.\epa

\bpb Syerun became short-lived for pain and grief.  It was the belief in olden times that men were born again, but that is now called an old wives’ tale.  Of Hallow and Syerun it is said that they were born again.  He was then called Hallow Hardingskate and she Cheer Halfdanesdaughter, as is told in the Leeds of Cheer, and she was a walkirrie.\epb\epg

\sectionline
