\bookStart{Second Lay of Hallow Hundingsbane}[Helgakviða Hundingsbana aðra]

\begin{flushright}%
Dating \parencite{Sapp2022}: early C11th (0.346)–late C11th (0.587)

Meter: \Fornyrdislag\ (TODO)%
\end{flushright}

TODO: Introduction.

\sectionline

\bpg
\bpa Helgi fekk Sigrúnar ok áttu þau sonu; var Helgi eigi gamall. Dagr Hǫgna sonr blótaði Óðin til fǫðurhefnda. Óðinn léði Dag geirs síns. Dagr fann Helga, mág sinn, þar sem heitir at Fjǫturlundi. Hann lagði í gǫgnum Helga með geirnum. Þar fell Helgi en Dagr reið til fjalla ok sagði Sigrúnu tíðindi:\epa

\bpb Hallow got Sighrun, and they owned sons; Hallow was not old. Day, son of Hain, \inx[C]{bloot}[blooted] to Weden to take revenge for his father. Weden lent Day his spear. Day found Hallow, his brother-in-law, at a place called Fetterlund; he laid the spear through Hallow. There fell Hallow, but Day rode to the fells and told Sighrun the news:\epb
\epg


\bvg
\bva „Trauðr em ek, systir, \hld\ trega þér at sęgja &
því-at ek hęfi nauðigr \hld\ nipti grę́tta: &
Fell í morgun \hld\ und Fjǫturlundi &
buðlungr sá’s vas \hld\ bęztr í hęimi &
ok hildingum \hld\ á halsi stóð.“\eva

\bvb “Regretful am I, sister, to grieve thee by saying—for, forced must I cause my kinswoman to cry: This morning fell, ’neath Fetterlund, that prince who was in the world the best, and on the throats of rulers stood.”\evb
\evg

...

\bpg
\bpa Ambótt Sigrúnar gekk um aftan hjá haugi Helga ok sá at Helgi reið til haugsins með marga menn. Ambótt kvað:\epa

\bpb Sighrun’s maid-servant walked in the evening near Hallow’s mound, and saw that Hallow rode to the mound along with many men. The maid-servant quoth:\epb
\epg


\bvg
\bva „Hvárt eru þat svik ęin \hld\ es séa þikkjumk &
eða ragna rǫk \hld\ ríða męnn dauðir, &
es jóa yðra \hld\ oddum kęyrið, &
eða es hildingum \hld\ hęim-fǫr gefin?“\eva

\bvb “Either these are deceits only, as I think myself seeing—or the Rakes of the Reins?—dead men riding, as ye drive forth your steeds with spear-points—or are the princes granted leave to go home?”\evb
\evg


\bvg
\bva „Es-a þat svik ęin \hld\ es séa þikkisk &
né \edtrans{aldar rof}{rending of the age}{\Bfootnote{Cf. TODO \emph{rjúfask ręgin}. This is the same root, only zero-grade.}} \hld\ þótt-u oss lítir, &
þótt vér jóa óra \hld\ oddum keyrim, &
né es hildingum \hld\ hęim-fǫr gefin.“\eva

\bvb “’Tis not deceits only, as thou thinkest thyself seeing—nor the rending of the age, although thou behold us; although we drive forth our steeds with spear-points the princes are not granted leave to go home.”\evb
\evg


\bpg
\bpa Heim gekk ambótt ok sagði Sigrúnu:\epa

\bpb The maid-servant walked home and said to Sighrun:\epb
\epg


\bvg
\bva „Út gakk Sigrún, \hld\ frá Sęfafjǫllum &
ef þik folks jaðarr \hld\ finna lystir; &
upp ’s haugr lokinn, \hld\ kominn es Hęlgi! &
Dólg-spor dręyra \hld\ dǫglingr bað þik &
at þú sár-dropa \hld\ svęfja skyldir.“\eva

\bvb “TODO.”\evb
\evg


\bpg
\bpa Sigrún gekk í hauginn til Helga ok kvað:\epa

\bpb Sighrun walked into the mound, to Hallow, and quoth:\epb
\epg

\bvg
\bva „Nú em’k svá fęgin \hld\ fundi okkrum &
sem át-frękir \hld\ Óðins haukar &
es val vitu, \hld\ varmar bráðir, &
eða dǫgg-litir \hld\ dags-brún séa.“\eva

\bvb “Now do I so rejoice at our meeting, as the food-greedy hawks of Weden \ken{ravens} when they see corpses, warm venison, or when the dew-gleaming ones do see the day’s brow \ken{dawn}.\evb
\evg


\bvg
\bva Fyrr vil’k kyssa \hld\ konung ó-lifðan &
an þú blóðugri \hld\ brynju kastir; &
hár es þitt, Helgi, \hld\ hélu þrungit, &
allr es vísi \hld\ val-dǫgg slęginn, &
hęndr úr-svalar \hld\ Hǫgna mági; &
hvé skal’k þér, buðlungr, \hld\ þess bót of vinna?“\eva

\bvb Sooner will I kiss the unliving king, than thou the bloody byrnie mightst cast away! Thy hair is, O Hallow, with hoarfrost thick; the prince is all with corpse-dew \ken{blood} whipped;\footnoteB{For the formulation cf. \Baldrsdraumar\ 5.} the hands wet-cold on the kinsman of Hain \ken*{= Hallow}.—How shall I for thee, O nobleman, remedy that?”\evb
\evg


\bvg
\bva „Ęin vęldr þú, Sigrún \hld\ frá Sefafjǫllum, &
es Hęlgi es \hld\ harm-dǫgg slęginn: &
Grę́tr þú, gull-varit, \hld\ grimmum tǫ́rum, &
sól-bjǫrt suðrǿn, \hld\ áðr þú sofa gangir, &
hvęrt fęllr blóðugt \hld\ á brjóst grami, &
úr-svalt, inn-fjalgt \hld\ ękka þrungit.\eva

\bvb “Thou alone causest, Sighrun from the Sevefells, that Hallow be by harm-dew whipped; thou weepest, gold-covered, bitter tears, O sun-bright southern lady, before thou to sleep mightst go. Each one falls bloody on the breast of the ruler, wet-cold and stifled, pressed forth by sorrow.\evb
\evg


\bvg
\bva Vęl skulum drekka \hld\ dýrar vęigar &
þótt misst hafim \hld\ munar ok landa. &
Skal ęngi maðr \hld\ angrljóð kveða &
þótt mér á brjósti \hld\ bęnjar líti! &
Nú eru brúðir \hld\ byrgðar í haugi, &
lofða dísir, \hld\ hjá oss liðnum!“\eva

\bvb Translation.”\evb
\evg


\bpg
\bpa Sigrún bjó sę́ing í hauginum.\epa

\bpb Sighrun made the bed in the mound:\epb
\epg


\bvg
\bva „Hér hęfi’k þér, Hęlgi, \hld\ hvílu gørva, &
angrlausa mjǫk, \hld\ Ylfinga niðr; &
vil’k þér í faðmi, \hld\ fylkir, sofna &
sem’k lofðungi \hld\ lifnum mynda’k!“\eva

\bvb “Translation.”\evb
\evg


\bvg
\bva „Nú kveð’k ęnskis \hld\ ørvę́nt vesa, &
síð né snimma, \hld\ at Sefafjǫllum &
es þú á armi \hld\ ólifðum sefr, &
hvít, í haugi, \hld\ Hǫgna dóttir, &
ok est-u kvik, \hld\ in konungborna!“\eva

\bvb “Translation.”\evb
\evg

\sectionline

\bvg
\bva „Mál ’s mér at ríða \hld\ roðnar brautir, &
láta fǫlvan jó \hld\ flugstíg troða; &
skal’k fyr vestan \hld\ vindhjalms brúar &
áðr Salgofnir \hld\ sigrþjóð vęki.“\eva

\bvb “’Tis time for me to ride the reddening roads, [to] let my pale steed tread the flight-path \ken{sky}; I shall go west of the wind-helm’s \ken{sky’s} bridges, before Salgovner might wake the victorious people.”\evb
\evg

\bpg\bpa Þeir Helgi riðu leið sína, en þę́r fóru heim til bǿjar. Annan aftan lét Sigrún ambótt halda vǫrð á hauginum. En at dagsetri, er Sigrún kom til haugsins, hon kvað:\epa

\bpb Hallow and his men rode on their way, but the women journeyed home to the farmstead. The next evening Sighrun made the maid-servant keep watch on the mound. But at sunset, when Sighrun came to the mound, she \ken*{= the maid-servant} quoth:\epb\epg


\bvg
\bva „Kominn vę́ri nú, \hld\ ef koma hygði, &
Sigmundar burr \hld\ frá sǫlum Óðins; &
kveð’k grams þinig \hld\ grę́nask vánir &
es á asklimum \hld\ ęrnir sitja &
ok drífr drótt ǫll \hld\ draumþinga til.“\eva

\bvb “He were now come—if to come he intended—Sighmund’s son \ken*{= Hallow} from Weden’s halls; TODO.”\evb
\evg


\bvg
\bva „Ves þú eigi svá ǿr \hld\ at ęin farir, &
dís skjǫldunga, \hld\ draughúsa til! &
Verða ǫflgari \hld\ allir á nǫ́ttum &
dauðir dólgar, mę́r, \hld\ en of daga ljósa.“\eva

\bvb “Be not so mad that thou journey alone, oh lady of the Shieldings \ken*{= Sighrun}, to the ghost-houses \ken{mounds}! They become mightier at night, all the dead fiends, oh maiden, than during the bright days.”\evb
\evg


\bpg\bpa Sigrún varð skammlíf af harmi ok trega. Þat var trúa í forneskju at menn vę́ri endrbornir en þat er nú kǫlluð kerlingavilla. Helgi ok Sigrún er kallat at vę́ri endrborin. Hét hann þá Helgi Haddingjaskati en hon Kára Hálfdanardóttir, svá sem kveðit er í Káruljóðum, ok var hon valkyrja.\epa

\bpb Sighrun became short-lived due to harm and pain. It was the belief in ancient times, that men were reborn; but that is now called an old wives’ tale. Hallow and Sighrun, it is said, were reborn. He was then called Hallow Haddingskate, but she Cheer Halfdanesdaughter, as is sung in the Leed of Cheer; and she was a walkirrie.\epb\epg
