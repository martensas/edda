\bookStart{Muspilli}

\begin{flushright}%
\textbf{Dating:} 800s

\textbf{Meter:} \Fornyrdislag%para
\end{flushright}%

Found in the margins of a single theological manuscript from the 820s, \emph{CLM 14098}.

The second sound shift is applied consistently.  That this was the case at composition is seen by the alliteration between Latin words starting with \emph{p-} and Germanic words which originally began with \emph{b-}:

\begin{itemize}
  \item l. 16: Germanic \emph{pú} (= OE, ON \emph{bú}) with borrowed \emph{pardísu} (< Latin \emph{paradīsum}),
  \item l. 21: Germanic \emph{piutit} (= OE \emph{bíett}, ON \emph{býðr}) with borrowed \emph{pehhes} (< Latin \emph{pix}) and \emph{pína} (< Latin \emph{poena}),
  \item l. 25: Germanic \emph{prinnan} (= OE \emph{biernan}, ON \emph{brinna}), \emph{palw-} (= OE \emph{bealu}, ON \emph{bǫlv-}) with borrowed \emph{pehhe} (see above).
\end{itemize}

\sectionline

\bvg\bva Sín \alst{t}ak pi·kweme, \hld\ daz er \alst{t}ouwan skal. &
Wanta \alst{s}ár só sih diu \alst{s}êla \hld\ in den \alst{s}ind ar·hęvit, &
ęnti si den \alst{l}íh-hamun \hld\ \alst{l}ikkan lázzit, &
só kwimit ęin \alst{h}ęri \hld\ fona \alst{h}imil-zungalon; &
daz andar fona \alst{p}ehhe: \hld\ dár \alst{p}ágant siu umpi. &
\alst{S}orgén mak diu \alst{s}êla, \hld\ unzi diu \alst{s}uona ar·gét, &
za wederemo \alst{h}ęrje \hld\ si gi·\alst{h}alót werde. &
Wanta ipu sia daz \alst{S}atanazses \hld\ ki·\alst{s}indi ki·winnit, &
daz \alst{l}ęitit sia sár \hld\ dár iru \alst{l}ęid wirdit, &
in \alst{f}uir ęnti in \alst{f}instrí: \hld\ daz ist rehto \alst{v}irin-líh ding. &
Upi sia avar ki·\alst{h}alónt die \hld\ die dár fona \alst{h}imile kwemant, &
ęnti si dero \alst{ę}ngilo \hld\ \alst{ęi}gan wirdit, &
die pringent sia sár úf in himilo ríhi: &
dár ist \alst{l}íp áno tôd, \hld\ \alst{l}ioht áno finstrí, &
\alst{s}ęlida áno \alst{s}orgun: \hld\ dár n·ist neo-man \alst{s}iuh. &
Denne der man in \alst{p}ardísu \hld\ \alst{p}ú ki·winnit, &
\alst{h}ús in \alst{h}imile, \hld\ dár kwimit imo \alst{h}ilfa ki·nuok. &
Pi·diu ist durft mihhil allero \alst{m}anno we-líhemo, \hld\ daz in es sín \alst{m}uot ki·spane, &%NOTE: daz] 119v
daz er \alst{k}otes willun \hld\ \alst{k}erno tuoo &
ęnti \alst{h}ęlla fuir \hld\ \alst{h}arto wíse, &
\alst{p}ehhes \alst{p}ína: \hld\ dár \alst{p}iutit der Satanasz altist &
\alst{h}ęizzan lauk. \hld\ Só mak \alst{h}ukkan za diu, &%NOTE: za] 120r
\alst{s}orgén dráto, \hld\ der sih \alst{s}untigen węiz. &
Wê demo in \alst{v}instrí skal \hld\ síno \alst{v}iriná stúén, &
\alst{p}rinnan in \alst{p}ehhe: \hld\ daz ist rehto \alst{p}alwík dink, &
daz der man \alst{h}arét ze gote \hld\ ęnti imo \alst{h}ilfa ni kwimit. &
\alst{W}ánit sih ki·náda \hld\ diu \alst{w}ênaga sêla: &%NOTE: wánit] 120v
ni ist in ki·\alst{h}uktin \hld\ \alst{h}imiliskin gote, &
wanta hiar in \alst{w}er-olti \hld\ after ni \alst{w}erkóta. &
Só denne der \alst{m}ahtigo khunink \hld\ daz \alst{m}ahal ki·pannit, &
dara skal \alst{k}weman \hld\ \alst{kh}unno ki·líhaz: &
denne ni ki·tar \alst{p}arno nohhęin \hld\ den \alst{p}an furi·sizzan, &
ni allero \alst{m}anno we-líh \hld\ ze demo \alst{m}ahale skuli. &
Dár skal er vora demo \alst{r}íhhe \hld\ az \alst{r}ahhu stantan, &
pí daz er in \alst{w}er-olti eo \hld\ ki·\alst{w}erkót hapéta. &
Daz hôrt’ ih \alst{r}ahhón \hld\ dia wer-olt-\alst{r}eht-wíson, &
daz skuli der \alst{a}nti-khristo \hld\ mit \alst{E}líase págan. &
Der \alst{w}arkh ist ki·\alst{w}áfanit, \hld\ denne wirdit untar in \alst{w}ík ar·hapan. &
\alst{Kh}enfun sint só \alst{k}reftík; \hld\ diu \alst{k}ósa ist só mihhil. &
\alst{E}lías strítit \hld\ pí den \alst{ê}wigon líp, &
wili dén \alst{r}eht-kernón \hld\ daz \alst{r}íhhi ki·starkan: &
pi·diu skal imo \alst{h}elfan \hld\ der \alst{h}imiles ki·waltit. &
Der \alst{A}nti-khristo \hld\ stét pí demo \alst{a}lt-fíante, &
stét pí demo \alst{S}atanase, \hld\ der inan var·\alst{s}enkan skal: &
pi·diu skal er in deru \alst{w}ík-stęti \hld\ \alst{w}unt pi·vallan &
ęnti in demo \alst{s}inde \hld\ \alst{s}iga-lôs werdan. &
Doh wánit des vilo got-manno, &
daz Elías in demo \alst{w}íge \hld\ ar·\alst{w}artit werde. &
Só daz \alst{E}líases pluot \hld\ in \alst{e}rda ki·triufit, &
só in·\alst{p}rinnant die \edtext{\alst{p}erga, \hld\ \alst{p}oum}{\lemma{perga \dots\ poum ‘mountains \dots woods’}\Bfootnote{Formulaic word-pair; see note to \Muspilli\ 3.}} ni ki·stęntit &
\alst{é}nihk in \alst{e}rdu, \hld\ \alst{a}há ar·truknént, &
muor var·\alst{s}wilhit sih, \hld\ \alst{s}wilizót lougiu der himil, &
\alst{m}áno vallit, \hld\ prinnit \alst{m}ittila-gart, &
\alst{st}ên ni ki·\alst{st}ęntit, \hld\ vęrit denne \alst{st}úatago in lant, &
\alst{v}ęrit mit diu \alst{v}uiru \hld\ \alst{v}iriho wísón: &
dár ni mak denae \alst{m}ák andremo \hld\ helfan vora demo \alst{M}úspille. &
Denne daz \alst{p}ręita wasal \hld\ allaz var·\alst{p}rinnit, &
ęnti vuir ęnti luft \hld\ iz allaz ar·furpit. &
Wár ist denne diu \alst{m}arha, \hld\ dár man dár eo mit sínén \alst{m}ágon piehk? &
Diu marha ist far·prunnan, \hld\ diu sêla stét pi·dungan, &
ni węiz mit wiu puaze: \hld\ só vęrit si za wíze. &
Pi·diu ist demo \alst{m}anne só guot, \hld\ denner ze demo \alst{m}ahale kwimit, &
daz er \alst{r}ahóno we-líha \hld\ \alst{r}ehto ar·tęile. &
Denne ni darf er \alst{s}orgén, \hld\ denne er ze deru \alst{s}uonu kwimit. &
Ni \alst{w}ęiz der \alst{w}ênago man, \hld\ wie-líhan \alst{w}artil er habét, &
denner mit den \alst{m}iatón \hld\ \alst{m}arrit daz rehta, &
daz der \alst{t}iuval dár pí \hld\ ki·\alst{t}arnit stęntit. &
Der hapét in \alst{r}uovu \hld\ \alst{r}ahóno we-líha, &
daz der man \alst{ê}r ęnti síd \hld\ \alst{u}piles ki·frumita, &
daz er iz allaz ki·\alst{s}agét, \hld\ denne er ze deru \alst{s}uonu kwimit; &
ni skolta síd \alst{m}anno nohhęin \hld\ \alst{m}iatun int·fáhan. &
Só daz \alst{h}imiliska \alst{h}orn \hld\ \edtrans{ki·\alst{h}lútit}{sounds}{\Afootnote{\emph{kilutit} ms.}\Bfootnote{Restoration of the cluster \emph{hl-} is required by the alliteration.}} wirdit, &
ęnti sih der \alst{s}uanari \hld\ ana den \alst{s}ind ar·hęvit &
der dár suannan skal \hld\ tôten ęnti lepentén, &
denne \alst{h}ęvit sih mit imo \hld\ \alst{h}ęrjo męista, &
daz ist allaz só \alst{p}ald, \hld\ daz imo nio-man ki·\alst{p}ágan ni mak. &
Denne vęrit er ze deru \alst{m}ahal-stęti, \hld\ deru dár ki·\alst{m}arkhót ist: &
dár wirdit diu \alst{s}uona, \hld\ dia man dár io \alst{s}agéta. &
Denne varant \alst{ę}ngila \hld\ \alst{u}per dio marha, &
\alst{w}ękhant deota, \hld\ \alst{w}íssant ze dinge. &
Denne skal \alst{m}anno gi·líh \hld\ fona deru \alst{m}oltu ar·stén, &
\alst{l}ôssan sih ar dero \alst{l}éwo vazzón: \hld\ skal imo avar sín \alst{l}íp pi·kweman, &
daz er sín \alst{r}eht allaz \hld\ ki·\alst{r}ahhón muozzi, &
ęnti imo after sínén \alst{t}átin \hld\ ar·\alst{t}ęilit werde. &
Denne der gi·\alst{s}izzit, \hld\ der dár \alst{s}uonnan skal &
ęnti ar·\alst{t}ęillan skal \hld\ \alst{t}ôtén ęnti kwekkhén, &
denne stét dár \alst{u}mpi \hld\ \alst{ę}ngilo męnigí, &
\alst{g}uotero \alst{g}omóno: \hld\ \alst{g}art ist só mihhil: &
dara kwimit ze deru \alst{r}ihtungu só vilo \hld\ dia dár ar \alst{r}ęstí ar·stént. &
Só dár \alst{m}anno nohhęin \hld\ wiht pi·\alst{m}ídan ni mak, &
dár skal denne \alst{h}ant sprehhan, \hld\ \alst{h}oupit sagén, &
allero \alst{l}ido we-líhk \hld\ unzi in den \alst{l}uzígun vinger, &
waz er untar desen \alst{m}annun \hld\ \alst{m}ordes ki·frumita. &
Dár ni ist eo só \alst{l}istík man \hld\ der dár io·wiht ar·\alst{l}iugan męgi, &
daz er ki·\alst{t}arnan męgi \hld\ \alst{t}áto dehhęina, &
niz al fora demo \alst{kh}uninge \hld\ ki·\alst{kh}undit werde, &
\alst{ú}zzan er iz \hld\ mit \alst{a}lamusanu furi·męgi &
ęnti mit \alst{f}astún \hld\ dio \alst{v}iriná ki·puazti. &
Denne der \alst{p}aldét \hld\ der gi·\alst{p}uazzit hapét, &
denner ze deru suonu kwimit. &
Wirdit denne \alst{f}uri ki·tragan \hld\ daz \alst{f}rôno khrúki, &
dár der \alst{h}êligo Khrist \hld\ ana ar·\alst{h}angan ward. &
Denne augit er dio \alst{m}ásún, \hld\ dio er in deru \alst{m}ęnniskí an·fénk, &
dio er duruh desse \alst{m}an-kunnes \hld\ \alst{m}inna far·doléta.\eva

\bvb TODO: Split into multiple parts. Translate.\evb\evg

\sectionline
