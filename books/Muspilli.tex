\bookStart{Muspell}[Muspilli]
\def\thisBookCode{Muspilli}

\begin{flushright}%
\textbf{Dating:} C9th

\textbf{Meter:} \Fornyrdislag%para
\end{flushright}%

\section{Introduction}

The \textbf{Muspell} (\Muspilli) is an Old High German Christian poem dealing with the Day of Judgment.

\Muspilli\ survives in a single copy, found scribbled in a Latin-language theological manuscript from the 820s CE with signum \emph{CLM 14098}; since the poem is marginalia, the dating of the manuscript can unfortunately only serve as a \emph{terminus post quem}.  The use of occasional end rhyme (see note to ll. 60–61) suggests a relation to Otfrid’s \emph{Evangelienbuch} (written 863–871 CE), as does the exact correspondence between \Muspilli\ 14 and \emph{Evangelienbuch} 1.18.9.  Whatever the direction of influence, the author of \Muspilli\ surely belonged to the same monastic C9th milieu as Otfrid.

Its dialect is that of the southern High German area, as seen by the consistent application of the most extensive form of the second sound shift, where \emph{g, b, k} change to \emph{k, p, ch}.  That this was the case at the time of composition is seen by the fact that Germanic roots originally beginning with \emph{b} consistently alliterate with Latin borrowings beginning with \emph{p}, namely in:

\begin{itemize}
  \item l. 16: Germanic \emph{pú} (= OS \emph{bú}) : borrowed \emph{pardísu} (< Latin \emph{paradīsum}),
  \item l. 21: Germanic \emph{piutit} (= OS \emph{biudid}) : borrowed \emph{pehhes} (< Latin \emph{pix}) and \emph{pína} (< Latin \emph{poena}),
  \item l. 25: Germanic \emph{prinnan} (= OS \emph{brinnan}) and \emph{palw-} (= OS \emph{balu}) : borrowed \emph{pehhe} (see above).
\end{itemize}

Interestingly, the alliteration also shows that the poet retained old \emph{h-} before \emph{l} (l. 72), by extension almost certainly also before \emph{r} and \emph{n}, and probably also before \emph{w} (l. 7).  This sound is, however, consistently ommitted by the scribe.

\sectionline

\section{The “Muspell”}

\bvg\bva Sín \alst{t}ak pi·kweme, \hld\ daz er \alst{t}ouwan skal. &
Wanta \alst{s}ár só sih diu \alst{s}êla \hld\ in den \alst{s}ind ar·hęvit, &
ęnti sí den \alst{l}íh-hamun \hld\ \edtext{\alst{l}ikkan \alst{l}ázzit}{\Bfootnote{The double alliteration in the second half-line is defective, but probably not due to any scribal corruption.}}, &
só kwimit ęin \alst{h}ęri \hld\ fona \alst{h}imil-zungalon; &
daz andar fona \alst{p}ehhe: \hld\ dár \alst{p}ágant siu umpi. &
\alst{S}orgén mak diu \alst{s}êla, \hld\ unzi diu \alst{s}uona ar·gét, &
za \edtext{\emph{\alst{h}}wederemo}{\Afootnote{\emph{wederemo} ms.}\Bfootnote{Restoration of the initial \emph{h-} is not strictly required for the line to alliterate properly, but is done on the basis of l. 72.}} \alst{h}ęrje \hld\ si gi·\alst{h}alót werde. &
Wanta ipu sia daz \alst{S}atanazses \hld\ ki·\alst{s}indi ki·winnit, &
daz \alst{l}ęitit sia sár \hld\ dár iru \alst{l}ęid wirdit, &
in \alst{f}uir ęnti in \alst{f}instrí: \hld\ daz ist rehto \alst{v}irin-líh ding. &
Upi sia avar ki·\alst{h}alónt die \hld\ die dár fona \alst{h}imile kwemant, &
ęnti si dero \alst{ę}ngilo \hld\ \alst{ęi}gan wirdit, &
die pringent sia sár úf \hld\ in himilo ríhi: &
\edtext{dár ist \alst{l}íp áno tôd, \hld\ \alst{l}ioht áno finstrí}{\Bfootnote{This line also appears in Otfrid’s \emph{Evangelienbuch} 1.18.9, in the form: \emph{Thár ist líb ána tôd, \hld\ lioht ána finstri.}  It is one of Otfrid’s rhymeless lines where alliteration compensates for the expected end-rhyme.  For the relevance of this shared line to the relation between \Muspilli\ and \emph{Evangelienbuch} see Introduction above.}}, &
\alst{s}ęlida áno \alst{s}orgun: \hld\ dár n·ist neo-man \alst{s}iuh. &
Denne der man in \alst{p}ardísu \hld\ \alst{p}ú ki·winnit, &
\alst{h}ús in \alst{h}imile, \hld\ dár kwimit imo \alst{h}ilfa ki·nuok. &
Pi·diu ist durft \alst{m}ihhil allero \alst{m}anno \emph{h}we-líhemo, \hld\ daz in es sín \alst{m}uot ki·spane, &%NOTE: daz] 119v
daz er \alst{k}otes willun \hld\ \alst{k}erno tuoo &
ęnti \alst{h}ęlla fuir \hld\ \alst{h}arto wíse, &
\alst{p}ehhes \alst{p}ína: \hld\ dár \alst{p}iutit der Satanasz altist &
\alst{h}ęizzan lauk. \hld\ Só mak \alst{h}ukkan za diu, &%NOTE: za] 120r
\alst{s}orgén dráto, \hld\ der sih \alst{s}untigen węiz. &
Wê demo in \alst{v}instrí skal \hld\ síno \alst{v}iriná stúén, &
\alst{p}rinnan in \alst{p}ehhe: \hld\ daz ist rehto \alst{p}alwík dink, &
daz der man \alst{h}arét ze gote \hld\ ęnti imo \alst{h}ilfa ni kwimit. &
\alst{W}ánit sih ki·náda \hld\ diu \alst{w}ênaga sêla: &%NOTE: wánit] 120v
ni ist in ki·\alst{h}uktin \hld\ \alst{h}imiliskin gote, &
wanta hiar in \alst{w}er-olti \hld\ after ni \alst{w}erkóta. &
Só denne der \alst{m}ahtigo khunink \hld\ daz \alst{m}ahal ki·pannit, &
dara skal \alst{k}weman \hld\ \alst{kh}unno ki·líhaz: &
denne ni ki·tar \alst{p}arno nohhęin \hld\ den \alst{p}an furi·sizzan, &
ni allero \alst{m}anno \emph{h}we-líh \hld\ ze demo \alst{m}ahale skuli. &
Dár skal er vora demo \alst{r}íhhe \hld\ az \alst{r}ahhu stantan, &
pí daz er in \alst{w}er-olti eo \hld\ ki·\alst{w}erkót hapéta. &
Daz hôrt’ ih \alst{r}ahhón \hld\ dia wer-olt-\alst{r}eht-wíson, &
daz skuli der \alst{a}nti-khristo \hld\ mit \alst{E}líase págan. &
Der \alst{w}arkh ist ki·\alst{w}áfanit, \hld\ denne wirdit untar in \alst{w}ík ar·hapan. &
\alst{Kh}ęnfun sint só \alst{k}ręftík; \hld\ diu \alst{k}ósa ist só mihhil. &
\alst{E}lías strítit \hld\ pí den \alst{ê}wigon líp, &
wili dén \alst{r}eht-kernón \hld\ daz \alst{r}íhhi ki·starkan: &
pi·diu skal imo \alst{h}elfan \hld\ der \alst{h}imiles ki·waltit. &
Der \alst{A}nti-khristo \hld\ stét pí demo \alst{a}lt-fíante, &
stét pí demo \alst{S}atanase, \hld\ der inan var·\alst{s}enkan skal: &
pi·diu skal er in deru \alst{w}ík-stęti \hld\ \alst{w}unt pi·vallan &
ęnti in demo \alst{s}inde \hld\ \alst{s}iga-lôs werdan. &
Doh wánit des vilo got-manno, &
daz Elías in demo \alst{w}íge \hld\ ar·\alst{w}artit werde. &
Só daz \alst{E}líases pluot \hld\ in \alst{e}rda ki·triufit, &
só in·\alst{p}rinnant die \edtext{\alst{p}erga, \hld\ \alst{p}oum}{\lemma{perga \dots\ poum ‘mountains \dots\ woods’}\Bfootnote{Formulaic word-pair; see note to \Wessobrunn\ 3.}} ni ki·stęntit &
\alst{ê}nihk in \alst{e}rdu, \hld\ \alst{a}há ar·truknént, &
muor var·\alst{s}wilhit sih, \hld\ \alst{s}wilizót lougiu der himil, &
\alst{m}áno vallit, \hld\ prinnit \alst{m}ittila-gart, &
\alst{st}ên ni ki·\alst{st}ęntit, \hld\ vęrit denne \alst{st}úa-tago in lant, &
\alst{v}ęrit mit diu \alst{v}uiru \hld\ \alst{v}iriho wísón: &
dár ni mak denae \alst{m}ák andremo \hld\ helfan vora demo \alst{M}úspille. &
Denne daz \alst{p}ręita wasal \hld\ allaz var·\alst{p}rinnit, &
ęnti \alst{v}uir ęnti luft \hld\ iz allaz ar·\alst{f}urpit. &
\emph{H}wár ist denne diu \alst{m}arha, \hld\ dár man dár eo mit sínén \alst{m}ágon piehk? &
\edtext{Diu marha ist far·prunnan, \hld\ diu sêla stét pi·d\emph{w}ungan, &
ni węiz mit \emph{h}wiu puaze: \hld\ só vęrit sí za wíze.}{\lemma{Diu \dots\ wíze}\Bfootnote{In these two lines the poet replaces the usual alliteration with end-rhyme within each half-lines pair (\emph{prunnan} : \emph{dungan — puaze} : \emph{wíze}).  The very same meter, including the quite loose rhyme scheme, is used by Otfrid throughout the whole of his \emph{Evangelienbuch}, written some time between 863 and 871 CE.  The direction of influence between that work and \Muspilli\ is unclear owing to the difficulties of dating the latter, for which see introduction above.}} &
Pi·diu ist demo \alst{m}anne só guot, \hld\ denner ze demo \alst{m}ahale kwimit, &
daz er \alst{r}ahóno \emph{h}we-líha \hld\ \alst{r}ehto ar·tęile. &
Denne ni darf er \alst{s}orgén, \hld\ denne er ze deru \alst{s}uonu kwimit. &
Ni \alst{w}ęiz der \alst{w}ênago man, \hld\ \emph{h}wie-líhan \alst{w}artil er habét, &
denner mit den \alst{m}iatón \hld\ \alst{m}arrit daz rehta, &
daz der \alst{t}iuval dár pí \hld\ ki·\alst{t}arnit stęntit. &
Der hapét in \alst{r}uovu \hld\ \alst{r}ahóno \emph{h}we-líha, &
daz der man \alst{ê}r ęnti síd \hld\ \alst{u}piles ki·frumita, &
daz er iz allaz ki·\alst{s}agét, \hld\ denne er ze deru \alst{s}uonu kwimit; &
ni skolta síd \alst{m}anno nohhęin \hld\ \alst{m}iatun int·fáhan. &
Só daz \alst{h}imiliska \alst{h}orn \hld\ \edtrans{ki·\emph{\alst{h}}lútit}{sounds}{\Afootnote{\emph{kilutit} ms.}\Bfootnote{Restoration of the cluster \emph{hl-} is required by the alliteration; cf. l. 7.}} wirdit, &
ęnti sih der \alst{s}uanari \hld\ ana den \alst{s}ind ar·hęvit &
der dár suannan skal \hld\ tôten ęnti lepentén, &
denne \alst{h}ęvit sih mit imo \hld\ \alst{h}ęrjo męista, &
daz ist allaz só \alst{p}ald, \hld\ daz imo nio-man ki·\alst{p}ágan ni mak. &
Denne vęrit er ze deru \alst{m}ahal-stęti, \hld\ deru dár ki·\alst{m}arkhót ist: &
dár wirdit diu \alst{s}uona, \hld\ dia man dár io \alst{s}agéta. &
Denne varant \alst{ę}ngila \hld\ \alst{u}per dio marha, &
\alst{w}ękhant deota, \hld\ \alst{w}íssant ze dinge. &
Denne skal \alst{m}anno gi·líh \hld\ fona deru \alst{m}oltu ar·stén, &
\alst{l}ôssan sih ar dero \alst{l}éwo vazzón: \hld\ skal imo avar sín \alst{l}íp pi·kweman, &
daz er sín \alst{r}eht allaz \hld\ ki·\alst{r}ahhón muozzi, &
ęnti imo after sínén \alst{t}átin \hld\ ar·\alst{t}ęilit werde. &
Denne der gi·\alst{s}izzit, \hld\ der dár \alst{s}uonnan skal &
ęnti ar·\alst{t}ęillan skal \hld\ \alst{t}ôtén ęnti kwekkhén, &
denne stét dár \alst{u}mpi \hld\ \alst{ę}ngilo męnigí, &
\alst{g}uotero \alst{g}omóno: \hld\ \alst{g}art ist só mihhil: &
dara kwimit ze deru \alst{r}ihtungu só vilo \hld\ dia dár ar \alst{r}ęstí ar·stént. &
Só dár \alst{m}anno nohhęin \hld\ wiht pi·\alst{m}ídan ni mak, &
dár skal denne \alst{h}ant sprehhan, \hld\ \alst{h}oupit sagén, &
allero \alst{l}ido \emph{h}we-líhk \hld\ unzi in den \alst{l}uzígun vinger, &
\emph{h}waz er untar desen \alst{m}annun \hld\ \alst{m}ordes ki·frumita. &
Dár ni ist eo só \alst{l}istík man \hld\ der dár io·wiht ar·\alst{l}iugan męgi, &
daz er ki·\alst{t}arnan męgi \hld\ \alst{t}áto dehhęina, &
niz al fora demo \alst{kh}uninge \hld\ ki·\alst{kh}undit werde, &
\alst{ú}zzan er iz \hld\ mit \alst{a}lamusanu furi·męgi &
ęnti mit \alst{f}astún \hld\ dio \alst{v}iriná ki·puazti. &
Denne der \alst{p}aldét \hld\ der gi·\alst{p}uazzit hapét, &
denner ze deru suonu kwimit. &
Wirdit denne \alst{f}uri ki·tragan \hld\ daz \alst{f}rôno khrúki, &
dár der \alst{h}êligo Khrist \hld\ ana ar·\alst{h}angan ward. &
Denne augit er dio \alst{m}ásún, \hld\ dio er in deru \alst{m}ęnniskí an·fénk, &
dio er duruh desse \alst{m}an-kunnes \hld\ \alst{m}inna far·doléta.\eva

\bvb TODO: Split into multiple parts. Translate.\evb\evg

\sectionline
