\bookStart{Lay of Thrim}[Þrymskviða]
\def\thisBookCode{Thrymskvida}

\begin{flushright}%
\textbf{Dating} \parencite{Sapp2022}: C9th (0.741)

\textbf{Meter:} \Fornyrdislag%
\end{flushright}

\section{Introduction}

The \textbf{Lay of Thrim} (\Thrymskvida) is only found in \Regius, where it follows \Lokasenna\ and precedes \Volundarkvida.  It has oft been considered the oldest poem in the \Regius\ collection, and Sapp’s model agrees with that judgment.

Comedic stories involving Thunder and his ettin-bashing seem to have been very popular in Wiking age Norway and Iceland, and the god himself is not infrequently the butt of the joke in them.  Apart from \Thrymskvida\ there are also the Eddic poems \Hymiskvida\ and \Harbardsljod, and the Scaldic poems \Haustlong and \Thorsdrapa.  Fragments of a lost Eddic poem about Thunder’s fight with the ettin Garfrith and his daughters also survive in \Gylfaginning; see Eddic fragments below.

\sectionline

\section{Lay of Thrim}

\bvg\bva\mssnote{\Regius~17r/13}%
\edtrans{\alst{\emph{V}}ręiðr}{Wroth}{\Afootnote{\emph{Reiðr} \Regius}\Bfootnote{Initial \emph{v-} is restored for the sake of alliteration but is not strictly metrically neccessary; cf. st 13.  In any case \Thrymskvida—generally considered to be the oldest Eddic poem—most almost certainly predates the West Norse sound change \emph{vr-} > \emph{r-}.}} vas þȧ \edtrans{\alst{V}ing-Þȯrr}{Wing-Thunder}{\Bfootnote{A rare poetic synonym for Thunder; it only elsewhere occurs in \Alvissmal\ 6.  See Index for etymology.}} \hld\ es hann \alst{v}aknaði &
ok \alst{s}ïns hamars \hld\ of \alst{s}aknaði, &
\edtext{\alst{sk}ęgg nam at hrista, \hld\ \alst{sk}ǫr nam at dýja}{\lemma{skęgg \dots\ dýja ‘beard \dots\ pull’}\Bfootnote{Apparently formulaic. Cf. \Brot\ TODO.}}, &
réð \alst{Ja}rðar burr \hld\ \alst{u}mb at þręifask.\eva

\bvb Wroth was then \inx[P]{Wing-Thunder} when he woke, \\
and of his hammer was bereaved. \\
His beard he took to rustle, his locks he took to rip; \\
the son of Earth resolved to grope about.\evb\evg


\bvg\bva\mssnote{\Regius~17r/15}%
\edtrans{\alst{O}k hann þat \alst{o}rða \hld\ \alst{a}lls fyrst of kvað}{And he this word first of all did say}{\Bfootnote{The whole line is formulaic, occuring in five other places: sts. 3, 9 and 12 of the present poem; st. 3 of \Oddrunargratr; st. 5 of \Brot.}}: &
„\alst{H}ęyr-ðu nú, Loki, \hld\ \alst{h}vat ek nú mę́li &
es \alst{ęi}gi vęit \hld\ \edtext{\alst{ja}rðar hvęr-gi &
né \alst{u}pp-himins}{\lemma{jarðar \dots\ upp-himins ‘earth \dots\ up-heaven’}\Bfootnote{The whole cosmos.  Formulaic, see Index: \inx[F]{Earth and Up-heaven}.}}: \hld\ \alst{ǫ̇}ss es stolinn hamri!“\eva

\bvb And he this word first of all did say: \\
“Hear thou now, Lock, what I now speak, \\
which no man knows anywhere on earth \\
nor in up-heaven: the \inx[G]{Eese}[os] \ken*{= Thunder = I} is robbed of His hammer!”\evb\evg


\bvg\bva\mssnote{\Regius~17r/17}%
Gingu þęir \alst{f}agra \hld\ \alst{F}ręyju tu̇na &
\alst{o}k \edtrans{hann}{he}{\Bfootnote{The speaker is Thunder, since he speaks about “my hammer”.}} þat \alst{o}rða \hld\ \alst{a}lls fyrst of kvað: &
„Munt-u mér, \alst{F}ręyja, \hld\ \edtrans{\alst{f}jaðr-hams}{feather-hame}{\Bfootnote{A “feather-skin” by which the wearer can transform or fly like a bird.}} léa &
ef ek \alst{m}ïnn hamar \hld\ \alst{m}ę́tta’k hitta?“\eva

\bvb Went they to the fair yards of \inx[P]{Frow}, \\
and he this word first of all did say: \\
“Wilt thou me, O Frow, the \inx[C]{feather-hame} lend, \\
if I my hammer might find?”\evb\evg


\bvg\bva\mssnote{\Regius~17r/19}\speakernote{Fręyja kvað:}%
„Þó mynda’k \alst{g}efa þér \hld\ þótt ór \alst{g}ulli vę́ri &
ok þó \edtrans{\alst{s}ęlja}{hand}{\Bfootnote{\emph{sęlja}, cognate of English \emph{sell}, here has its older sense of ‘hand over’, cf. Gotish \emph{saljan} ‘\emph{opfern}; \textgreek{θύειν}’ \parencite[116]{Streitberg}.}} \hld\ at vę́ri ór \alst{s}ilfri.“\eva

\bvb\speakernoteb{[Frow quoth:]}%
“Yet would I give it to thee though it were golden, \\
and yet hand it to thee if it were silvern.”\evb\evg


\bvg\bva\mssnote{\Regius~17r/20}%
\alst{F}ló þȧ \edtrans{Loki}{Lock}{\Bfootnote{Though Thunder is the one asking for the feather-hame (“if I \emph{my} hammer might find”), Lock is the one that takes off flying with it.}}, \hld\ \alst{f}jaðr-hamr dunði, &
unds fyr \alst{ú}tan kom \hld\ \alst{ȧ}sa garða &
ok fyr \alst{i}nnan kom \hld\ \alst{jǫ}tna hęima.\eva

\bvb Then flew Lock—the feather-hame rustled— \\
until he came outside the \inx[L]{Osyard}[Yards of the Eese], \\
and he came inside the \inx[L]{Ettinham}[Homes of the Ettins].\evb\evg


\bvg\bva\mssnote{\Regius~17r/22}%
\alst{Þ}rymr \edtrans{sat ȧ haugi}{sat on the mound}{\Bfootnote{Meditating on mounds was a common pastime for the ancients.  See \Voluspa\ 41 for other attestations.}}, \hld\ \edtrans{\alst{þ}ursa dróttinn}{lord of Thurses}{\Bfootnote{This formulaic expression also occurs in several Runic charms against such thursen lords (see below under Galders); an example of the close connection between mythology and ritual.}}, &
\edtext{\alst{g}ręyjum sïnum \hld\ \alst{g}ull-bǫnd snøri &
ok \alst{m}ǫrum sïnum}{\lemma{gręyjum sïnum \dots\ mǫrum sïnum ‘his greyhounds \dots\ his steeds’}\Bfootnote{Thrim sits surrounded by dogs and horses.  The scene is reminiscent of the ancient “master of animals” motif, especially as attested on panel A of the Gundestrup cauldron.}} \hld\ \alst{m}ǫn jafnaði.\eva

\bvb Thrim sat on the mound, the lord of \inx[G]{Thurses}: \\
on his greyhounds the golden leashes he twirled, \\
and on his steeds the manes he cut even.\evb\evg


\bvg\bva\mssnote{\Regius~17r/23}\speakernote{[Þrymr kvað:]}%
„\edtrans{Hvat ’s með \alst{ǫ̇}sum? \hld\ Hvat ’s með \alst{ǫ}lfum?}{What is with the Eese? What is with the Elves?}{\Bfootnote{Formulaic, the same line occurs in \Voluspa\ 46/1.}} &
Hví est \alst{ęi}nn kominn \hld\ ï \alst{jǫ}tun-hęima?“ &
\speakernote{[Loki kvað:]}%
„\alst{I}llt ’s með \alst{ǫ̇}sum, \hld\ \edtext{\alst{i}llt ’s með \alst{ǫ}lfum}{\Afootnote{Required by the meter; om. \Regius}}! &
\alst{H}ęfir þú \alst{H}lórriða \hld\ \alst{h}amar of folginn?“\eva

\bvb\speakernoteb{[Thrim quoth:]}%
“What’s with the Eese? What’s with the Elves? \\
Why art thou alone come into the \inx[L]{Ettinham}[Ettin-homes]?”— \\
\speakernoteb{[Lock quoth:]}%
“’Tis ill with the Eese! ’Tis ill with the Elves! \\
Hast thou the hammer of Loride \name{= Thunder} hid?”\evb\evg


\bvg\bva\mssnote{\Regius~17r/25}\speakernote{[Þrymr kvað:]}%
„Ek \alst{h}ęfi \alst{H}lórriða \hld\ \alst{h}amar of folginn &
\edtrans{\alst{á}tta rǫstum}{eight rests}{\Bfootnote{Eight leagues; a “rest” being an old distance measurement. See Index.}} \hld\ fyr \alst{jǫ}rð neðan; &
hann \alst{ę}ngi maðr \hld\ \alst{a}ptr of hęimtir &
nema \alst{f}ǿri mér \hld\ \alst{F}ręyju at kvę̇n.“\eva

\bvb\speakernoteb{[Thrim quoth:]}
“I have the hammer of Loride hid \\
eight \inx[C]{rest}[rests] beneath the earth. \\
It no man might fetch back, \\
unless he bring me Frow for a wife.”\evb\evg


\bvg\bva\mssnote{\Regius~17r/27}%
\alst{F}ló þȧ Loki, \hld\ \alst{f}jaðr-hamr dunði, &
unds fyr \alst{ú}tan kom \hld\ \alst{jǫ}tna hęima &
ok fyr \alst{i}nnan kom \hld\ \alst{ȧ}sa garða; &
\edtrans{\alst{m}ǿtti hann Þȯr}{He met Thunder}{\Bfootnote{This line is compatible with the reconstructed disyllabic form \emph{*Þȯar} if the pronoun \emph{hann} is excised.  For that form see note to \Hymiskvida\ 23/1.}} \hld\ \alst{m}iðra garða &
\alst{o}k \edtext{hann þat}{\Afootnote{emend.; \emph{þat hann} \Regius, with elsewhere unprecedented word order. Cf. note to st. 2.}} \alst{o}rða \hld\ \alst{a}lls fyrst of kvað:\eva

\bvb Then flew Lock—the feather-hame rustled— \\
until he came outside the Homes of the Ettins \\
and he came inside the Yards of the Eese. \\
He met Thunder in the middle yards, \\
and he \ken*{= Thunder} this word first of all did say:\evb\evg


\bvg\bva\mssnote{\Regius~17r/29}%
„\edtrans{Hęfir þú \alst{ø}rendi \hld\ sem \alst{ę}rfiði?}{Hast thou an errand of hardship?}{\Bfootnote{Thunder asks Lock whether he is the bearer of ill tidings.  The rhyming pair \emph{ørendi} ‘errand’ \dots\ \emph{ęrfiði} ‘trouble, hardship’ is formulaic and occurs in X other (TODO!!) places, including \HelgakvidaHjorvardssonar\ 5.}} &
Seg-ðu ȧ \alst{l}opti \hld\ \alst{l}ǫng tíðendi! &
\edtext{Opt \alst{s}itjanda \hld\ \alst{s}ǫgur of fallask, &
ok \alst{l}iggjandi \hld\ \alst{l}ygi of bęllir.}{\lemma{Opt sitjanda \hld\ sǫgur of fallask, // ok liggjandi · lygi of bęllir. ‘Oft the sitting man’s stories fail each other // and the lying down blows up his lie.’}\Bfootnote{Proverbial. If one waits and mulls over bad news after receiving them, details will be left out and excuses thought up.  It is therefore best that Lock immediately tell Thunder what he has learned.  ON \emph{liggja} ‘recline’ and \emph{ljúga} ‘speak untruth’ are entirely different verbs; it is very unfortunate that they sound the same in English.}}“\eva

\bvb “Hast thou an errand of hardship? \\
Tell thou the long tidings aloft! \\
Oft the sitting man’s stories fail each other \\
and the lying down blows up his lie.”\evb\evg


\bvg\bva\mssnote{\Regius~17r/31}\speakernote{[Loki kvað:]}%
„Hefi’k \alst{ø}rendi, \hld\ \alst{ę}rfiði ok: &
\alst{Þ}rymr hęfir þïnn hamar, \hld\ \alst{þ}ursa dróttinn; &
hann \alst{ę}ngi maðr \hld\ \alst{a}ptr of hęimtir &
nema hǫ́num \alst{f}ǿri \hld\ \alst{F}ręyju at kvę̇n.“\eva

\bvb\speakernoteb{[Lock quoth:]}%
“I have an errand, hardship also: \\
Thrim has thy hammer, the lord of Thurses. \\
It no man will fetch back, \\
unless he bring him Frow for a wife.”\evb\evg


\bvg\bva\mssnote{\Regius~17r/33}%
Ganga þęir \alst{f}agra \hld\ \alst{F}ręyju at hitta &
\alst{o}k \edtrans{hann}{he}{\Bfootnote{The speaker is either Thunder or Lock.}} þat \alst{o}rða \hld\ \alst{a}lls fyrst of kvað: &
„\alst{B}itt-u þik, Fręyja, \hld\ \edtrans{\alst{b}rúðar lïni!}{bridal linen}{\Bfootnote{The dress of the bride.}} &
Vit skulum \alst{a}ka tvau \hld\ ï \alst{jǫ}tun-hęima.“\eva

\bvb Go they the fair Frow to find, \\
and he this word first of all did say: \\
“Bind thyself, Frow, with bridal linen! \\
We two shall drive into the Ettin-homes.”\evb\evg


\bvg\bva\mssnote{\Regius~17v/1}%
\emph{V}ręið varð þȧ \alst{F}ręyja \hld\ ok \alst{f}nasaði, &
\alst{a}llr \alst{ȧ}sa salr \hld\ \alst{u}ndir bifðisk, &
stǫkk þat it \alst{m}ikla \hld\ \edtrans{\alst{m}ęn Brísinga}{Torc of the Brisings}{\Bfootnote{A legendary jewel owned by Frow.}}: &
„Mik \alst{v}ęitst \edtrans{\alst{v}erða \hld\ \alst{v}er-gjarnasta}{become the most man-eager}{\Bfootnote{Presumably Frow is speaking out of self-awareness of her own lustful inclinations, i.e., she will be gripped by uncontrollable lust.  It is also possible that she complains about being accused of promiscuity by the other gods, but that is not the literal sense.  For Frow’s promiscuity cf. \Lokasenna\ 30, and also st. 26 of that poem where Frie is likewise called \emph{ver-gjǫrn} ‘man-eager’.}} &
ef ek \alst{ę}k með þér \hld\ ï \alst{jǫ}tun-hęima.“\eva

\bvb Wroth became Frow then, and snorted; \\
the whole hall of the Eese shook beneath; \\
down crashed the great \inx[P]{Torc of the Brisings}— \\
“Thou knowest that I will become the most man-eager, \\
if I drive with thee into the Ettin-homes.”\evb\evg


\bvg\bva\mssnote{\Regius~17v/3}%
\edtext{Sęnn vǫ́ru \alst{ę̇}sir \hld\ \alst{a}llir ȧ þingi &
ok \alst{ǫ̇}synjur \hld\ \alst{a}llar ȧ máli, &
ok umb þat \alst{r}éðu \hld\ \alst{r}íkir tívar:}{\lemma{Sęnn \dots\ tívar ‘Soon \dots\ Tews’}\Bfootnote{The exact same three lines also occur \Baldrsdraumar\ 1/1–3; see Note there.}} &
\alst{h}vé þęir \alst{H}lórriða \hld\ \alst{h}amar of sǿtti?\eva

\bvb Soon were the \inx[G]{Eese} all at the \inx[C]{Thing}, \\
and the \inx[G]{Ossens} all at speech, \\
and of this counseled the mighty \inx[G]{Tews}: \\
How they Loride’s \name{= Thunder’s} hammer would get?\evb\evg


\bvg\bva\mssnote{\Regius~17v/5}%
Þȧ kvað þat \alst{H}ęimdallr, \hld\ \alst{h}vítastr ȧsa, &
\edtrans{\alst{v}issi \alst{v}ęl framm}{he foreknew well}{\Bfootnote{i.e. saw the future.  Compare the derived adjective \emph{fram-víss} ’forth-wise, prescient.’}} \hld\ sęm \alst{v}anir aðrir: &
„\alst{B}indu vér Þȯr þȧ \hld\ \alst{b}rúðar lïni; &
hafi hann it \alst{m}ikla \hld\ \alst{m}ęn Brísinga!\eva

\bvb Then quoth this \inx[P]{Homedal}, whitest of the Eese; \\
he foreknew well like the other \inx[G]{Wanes}: \\
“Let us bind Thunder, then, with bridal linen; \\
let him have the great torc of the Brisings!\evb\evg


\bvg\bva\mssnote{\Regius~17v/6}%
\edtext{Lǫ́tum und \alst{h}ǫ́num \hld\ \alst{h}rynja lukla &
ok \alst{k}ven-váðir \hld\ umb \alst{k}né falla &
en ȧ \alst{b}rjósti \hld\ \alst{b}ręiða stęina &
ok \alst{h}ag-liga \hld\ umb \alst{h}ǫfuð typpum!“}{\lemma{ALL}\Bfootnote{A unique description of Wiking Age bridal dress.  Cf. the description’s of dress in \Rigsthula, which is, however, a much younger poem than \Thrymskvida.  Being the mistress of the household, keys were the mark of a respectable married woman.  The “broad stones” on the breast may be tortoise brooches (also mentioned in \Volundarkvida\ 25, 36.) or beads in a large necklace.  The “tipping” of the head refers to some sort of bridal hat which would have included a veil (cf. st. 27 below).}}\eva

\bvb Let us by his side hang jingling keys, \\
and women’s garments to fall about his knees, \\
but on the breast broad stones, \\
and skillfully let us tip his head.”\evb\evg


\bvg\bva\mssnote{\Regius~17v/8}%
Þȧ kvað þat \alst{Þ}ȯrr, \hld\ \alst{þ}rúðugr ǫ̇ss: &
„Mik munu \alst{ę̇}sir \hld\ \alst{a}rgan kalla &
ef ek \alst{b}indask lę́t \hld\ \alst{b}rúðar lïni!“\eva

\bvb Then quoth this Thunder, the mighty Os: \\
“Me will the Eese call \inx[C]{queer} \\
if I let me be bound with bridal linen!”\evb\evg


\bvg\bva\mssnote{\Regius~17v/9}%
Þȧ kvað þat \alst{L}oki \hld\ \alst{L}aufęyjar sonr: &
„\alst{Þ}ęgi þú, \alst{Þ}ȯrr, \hld\ \alst{þ}ęira orða! &
\edtext{Þegar munu \alst{jǫ}tnar \hld\ \alst{Ǫ̇}s-garð búa &
nema \alst{þ}ú \alst{þ}ïnn hamar \hld\ \alst{þ}ér of hęimtir.}{\lemma{Þegar \dots\ hęimtir. ‘Shortly \dots\ dost fetch!’}\Bfootnote{Guarding Osyard from transgressive and destructive forces was Thunder’s task, and the hammer his most important tool.  Cf. \Harbardsljod\ TODO, and a couplet by the obscure poet Thurbern Disescold, cited in \Skaldskaparmal\ 11: \emph{Þȯrr hęfr Yggs með ǫ́rum \hld\ Ǫ̇sgarð af þrek varðan.} ‘Thunder has with the messengers of Ug \ken{gods} mightily guarded Osyard.’}}“\eva

\bvb Then quoth this Lock, Leafie’s son: \\
“Shut up thou, Thunder, with those words! \\
Shortly the Ettins will settle Osyard, \\
unless thou thy hammer for thyself dost fetch!”\evb\evg


\bvg\bva\mssnote{\Regius~17v/11}%
\alst{B}undu þęir Þȯr þȧ \hld\ \alst{b}rúðar lïni &
ok hinu \alst{m}ikla \hld\ \alst{m}ęni Brísinga, &
létu und \alst{h}ǫ́num \hld\ \alst{h}rynja lukla &
ok \alst{k}ven-váðir \hld\ umb \alst{k}né falla &
en ȧ \alst{b}rjósti \hld\ \alst{b}ręiða stęina &
ok \alst{h}ag-liga \hld\ of \alst{h}ǫfuð typpðu.\eva

\bvb They bound Thunder then with bridal linen, \\
and with the great Torc of the Brisings. \\
They by his side set keys to jingle, \\
and women’s garments to fall about the knees, \\
but on the breast broad stones, \\
and skillfully they tipped his head.\evb\evg


\bvg\bva\mssnote{\Regius~17v/13}%
Þȧ kvað þat \alst{L}oki \hld\ \alst{L}aufęyjar sonr: &
„Mun’k \alst{au}k með þér \hld\ \alst{a}mbǫ́tt vesa, &
\edtext{vit skulum \alst{a}ka tvau}{\lemma{vit \dots\ tvau ‘we two’}\Bfootnote{\emph{tvau} ‘two’ is here in the neuter, which is used for mixed-sex groups.  This is either an error due to mindless copying of st. 11, or a backhanded insult against Thunder by Lock.}} \hld\ ï \alst{jǫ}tun-hęima.“\eva

\bvb Then quoth this Lock, Leafie’s son: \\
“I will also with thee be a handmaid; \\
we two shall drive into the Ettin-homes.”\evb\evg


\bvg\bva\mssnote{\Regius~17v/14}%
Sęnn vǫ́ru \edtrans{\alst{h}afrar}{he-goats}{\Bfootnote{Thunder’s chariot was driven by his two goats; cf. the kenning Lord of He-goats (\emph{hafra dróttinn}, \Hymiskvida\ 20, 31).}} \hld\ \alst{h}ęim of vreknir, &
\alst{sk}yndir at \alst{sk}ǫklum, \hld\ \alst{sk}yldu vęl renna; &
\edtrans{\alst{b}jǫrg \alst{b}rotnuðu, \hld\ \alst{b}rann jǫrð loga}{Crags burst, earth burned with flame}{\Bfootnote{Thunder’s driving is often heralded by cosmic disturbance.  So, his arrival in \Lokasenna\ 55 is signalled by the mountains quaking.  The description most similar to the present stanza is found in Thedwolf’s \Haustlong\ 14–16, where crags (\emph{bjǫrg}) burst asunder and fires rage before him as he rides to fight \inx[P]{Rungner}.  A possibly Indo-European parallel is the Vedic myth of Índra breaking the mountains and releasing the rivers (as described most famously in \Rigveda\ 1.32).  Cf. also \Baldrsdraumar\ 3 where the ground rumbles beneath the riding Weden.}}; &
\alst{ó}k \alst{Ó}ðins sonr \hld\ ï \alst{jǫ}tun-hęima.\eva

\bvb Soon were the \inx[C]{he-goats} driven home, \\
hastened onto the cart-poles—they were to run well. \\
Crags burst, earth burned with flame; \\
Weden’s son \ken*{= Thunder} drove to the Ettin-homes.\evb\evg


\bvg\bva\mssnote{\Regius~17v/16}%
Þȧ kvað þat \alst{Þ}rymr, \hld\ \alst{þ}ursa dróttinn: &
„\alst{St}andið upp, jǫtnar, \hld\ ok \alst{st}ráið bękki! &
Nú \alst{f}ǿrið mér \hld\ \alst{F}ręyju at kvȧn, &
\alst{N}jarðar dóttur \hld\ ór \alst{N}óa-tu̇num.\eva

\bvb Then quoth this Thrim, the lord of Thurses: \\
“Stand up, ye ettins, and strew the benches! \\
Now bring me Frow for a wife, \\
\inx[P]{Nearth}’s daughter from the \inx[L]{Nowetowns}!\evb\evg


\bvg\bva\mssnote{\Regius~17v/18}%
\alst{G}anga hér at \alst{g}arði \hld\ \alst{g}ull-hyrnðar kýr, &
\edtrans{\alst{ø}xn \alst{a}l-svartir}{all-black oxen}{\Bfootnote{Formulaic, also occurring in \Hymiskvida\ 18.  That all-black (i.e. spotlessly black) oxen were most valued is seen by the pairing with “golden-horned”.  One may also compare \textcite{Saxo} 1.8.12, where the hero Hadding has to atone for his slaying of a heavenly being by a sacrifice of dark-coloured victims (\emph{furvae hostiae}): \emph{Siquidem propiciandorum numinum gratia Frø deo rem diuinam furuis hostiis fecit. Quem litationis morem annuo feriarum circuitu repetitum posteris imitandum reliquit. Frøblod Sueones uocant.} ‘In order to mollify the divinities he [= Hadding] did indeed make a holy sacrifice of dark-coloured victims to the god Frø.  He repeated this mode of propitiation at an annual festival and left it to be imitated by his descendants.  The Swedes call it Frøblot.’  This ancient ritual taboo finds parallel even in the Tanakh, where animals dedicated to YHWH were to be without blemish (\textgreek{תָּמִ֖ים}, Leviticus 1:3)}}, \hld\ \alst{jǫ}tni at gamni, & %TODO: Hebrew.
fjǫlð á’k \alst{m}ęiðma, \hld\ fjǫlð á’k \alst{m}ęnja; &
\alst{ęi}nnar mér Fręyju \hld\ \alst{ȧ}-vant þykkir.“\eva

\bvb Here march to the farm golden-horned kine, \\
all-black oxen to the ettin’s [my] pleasure. \\
A multitude I own of treasures, a multitude I own of torcs— \\
only Frow I think me missing.”\evb\evg


\bvg\bva\mssnote{\Regius~17v/20}%
Vas þar at \alst{k}veldi \hld\ of \alst{k}omit snimma &
\alst{o}k fyr \alst{jǫ}tna \hld\ \alst{ǫ}l framm borit. &
\edtext{\alst{Ęi}nn át \alst{o}xa, \hld\ \alst{á}tta laxa, &
\alst{k}rásir allar, \hld\ þę́r’s \alst{k}onur skyldu, &
drakk \alst{S}ifjar verr \hld\ \alst{s}áld þrjú mjaðar.}{\lemma{Ęinn \dots\ mjaðar. ‘He alone \dots\ of mead.’}\Bfootnote{Thunder is renowned for his great appetite; cf. \Hymiskvida\ 15, where he eats two of Hymer’s oxen.  It is curious that the same kenning (\emph{Sifjar verr} ‘Sib’s husband’) is used in that stanza.}}\eva

\bvb There was the evening come early, \\
and for the ettins ale brought forth. \\
He \ken*{= Thunder} alone ate an ox, eight salmons, \\
all the dainties meant for the women; \\
drank Sib’s husband three sieves of mead.\evb\evg


\bvg\bva\mssnote{\Regius~17v/23}%
Þȧ kvað þat \alst{Þ}rymr, \hld\ \alst{þ}ursa dróttinn: &
„Hvar sátt-u \alst{b}rúðir \hld\ \alst{b}íta hvassara? &
Sá’k-a \alst{b}rúðir \hld\ \alst{b}íta ęnn \alst{b}ręiðara &
né ęnn \alst{m}ęira \alst{m}jǫð \hld\ \alst{m}ęy of drekka!“\eva

\bvb Then quoth this Thrim, the lord of Thurses: \\
“Where sawest thou brides bite sharper? \\
I never saw brides bite yet broader; \\
nor yet more mead a maiden drink!”\evb\evg


\bvg\bva\mssnote{\Regius~17v/25}%
Sat hin \alst{a}l-snotra \hld\ \alst{a}mbǫ́tt fyrir &
es \alst{o}rð of fann \hld\ við \alst{jǫ}tuns máli: &
„\alst{Á}t vę́tr Fręyja \hld\ \alst{á}tta nǫ́ttum, &
svá vas hón \alst{ó}ð-fu̇s \hld\ ï \alst{jǫ}tun-hęima.“\eva

\bvb Sat the all-clever handmaid \ken*{= Lock} in front, \\
who a word did find against the ettin’s speech: \\
“Frow ate naught for eight nights; \\
so madly she longed for the Ettin-homes.”\evb\evg


\bvg\bva\mssnote{\Regius~17v/27}%
\alst{L}aut und \edtrans{\alst{l}ïnu}{linen}{\Bfootnote{The bridal veil.}}, \hld\ \alst{l}ysti at kyssa, &
en hann \alst{ú}tan stǫkk \hld\ \alst{ę}nd-langan sal: &
„Hví eru \alst{ǫ}ndótt \hld\ \alst{au}gu Fręyju? &
\edtrans{Þykki mér \alst{ó}r \hld\ \alst{au}gum brenna!}{Methinks it burning from the eyes!}{\Bfootnote{The meter of this line is very poor: the first half-line is only three syllables long, and the alliteration falls on \emph{ór} ‘from’, which has no reason to be stressed.  It would be much improved by inserting \emph{ęldar} ‘fires’ between \emph{augum} ‘eyes’ and \emph{brenna} ‘burns’, and this expression is actually attested in \Gylfaginning\ 51: \emph{Eldar brenna ór augum hans ok nǫsum} ‘Fires burn from his eyes and nostrils’.}}“\eva

\bvb He \ken*{= Thrim} looked ’neath the linen, lusted to kiss— \\
but flung back out across the length of the hall— \\
“Why are the eyes of Frow blazing? \\
Methinks it burning from the eyes!”\evb\evg


\bvg\bva\mssnote{\Regius~17v/29}%
Sat hin \alst{a}l-snotra \hld\ \alst{a}mbǫ́tt \edtext{fyrir}{\Afootnote{add. \emph{†ſ.†} \Regius.}} &
es \alst{o}rð of fann \hld\ við \alst{jǫ}tuns máli: &
„Svaf vę́tr Fręyja \hld\ átta nǫ́ttum, &
svá vas hón \alst{ó}ð-fu̇s \hld\ ï \alst{jǫ}tun-hęima.“\eva

\bvb Sat the all-clever handmaid in front, \\
who a word did find against the ettin’s speech: \\
“Frow slept naught for eight nights; \\
so madly she longed for the Ettin-homes.”\evb\evg


\bvg\bva\mssnote{\Regius~17v/30}%
\alst{I}nn kom hin \alst{a}rma \hld\ \alst{jǫ}tna systir, &
hin’s \alst{b}rúð-féar \hld\ \alst{b}iðja þorði: &
„Lát þér af \alst{h}ǫndum \hld\ \alst{h}ringa rauða &
ef þú \alst{ǫ}ðlask vill \hld\ \alst{ȧ}stir mïnar, &
\edtrans{\alst{ȧ}stir mïnar, \hld\ \alst{a}lla hylli}{my love; all [my] holdness’}{\Bfootnote{Probably formulaic.  There are no preserved parallels in poetry, but there may be one in \Gylfaginning\ 49 (excerpt, following the death of Balder):
\emph{En er goðin vitkuðust, þá mę́lti Frigg ok spurði, hverr sá vę́ri með ásum, er \emph{eignast vildi „allar ástir mínar}} (so \Trajectinus\Wormianus; \emph{ástir hennar} ‘her loves’ \RegiusProse\Upsaliensis) \emph{\emph{ok hylli}, ok vili hann ríða á hel-veg ok freista, ef hann fái fundit Baldr, ok bjóða Helju út-lausn, ef hon vill láta fara Baldr heim í Ás-garð.“} ‘But when the gods came back to their wits, then Frie spoke and asked which one among the Eese \emph{would own “all my loves and holdness}, and will ride on the \inx[L]{Hellway} and see if he may find Balder and offer Hell a ransom if she will let Balder come home to Osyard.”’
We can tell from the citation of a \Ljodahattr\ stanza at the end of ch. 49 (see Eddic Fragments below) that Snorre knew one or more now-lost Eddic poems about Balder’s death, and it may be that one of these poems contained the same two long-lines as the present stanza.  For such a sharing of whole lines cf. e.g. st. 14/1–3 above, which are identical to \Baldrsdraumar\ 1/1–3.}}!“\eva

\bvb In came the wretched sister of the ettins, \\
she who for the bride-fee \ken*{= Millner} dared ask: \\
“Slide off from thy hands the red rings, \\
if thou wilt win my affections, \\
my affections, all [my] \inx[C]{holdness}.”\footnoteB{The sister, who was apparently the one who asked for the Hammer, now has the audacity to ask Thunder (disguised as Frow) to give her the very rings on his hands.}\evb\evg


\bvg\bva\mssnote{\Regius~17v/32}%
Þȧ kvað þat \alst{Þ}rymr, \hld\ \alst{þ}ursa dróttinn: &
„\alst{B}erið inn hamar \hld\ \alst{b}rúði at vígja, &
lęggið \alst{M}jǫllni \hld\ ï \alst{m}ęyjar kné, &
\alst{v}ígið okkr saman \hld\ \edtrans{\alst{V}árar}{Ware}{\Bfootnote{A goddess who, according to \Gylfaginning\ 35, rules vows between men and women.  See Index.}} hęndi!“\eva

\bvb Then quoth this Thrim, the lord of Thurses: \\
“Bear ye in the hammer the bride for to bless; \\
lay ye Millner in the maiden’s knee; \\
bless us two together by \inx[P]{Ware}’s hand!”\evb\evg


\bvg\bva\mssnote{\Regius~17v/34}%
\alst{H}ló \alst{H}lórriða \hld\ \alst{h}ugr ï brjósti &
es \alst{h}arð-\alst{h}ugaðr \hld\ \alst{h}amar of þękkði; &
\alst{Þ}rym drap hann fyrstan, \hld\ \alst{þ}ursa dróttin, &
ok \alst{ę́}tt \alst{jǫ}tuns \hld\ \alst{a}lla lamði.\eva

\bvb Laughed Loride’s \name{= Thunder’s} heart in his chest, \\
when, hard-hearted, he recognised the hammer. \\
Thrim he smote first, the lord of Thurses, \\
and all the ettin’s lineage he beat lame.\evb\evg


\bvg\bva\mssnote{\Regius~18r/1}%
Drap hann ina \alst{ǫ}ldnu \hld\ \alst{jǫ}tna systur, &
hin’s \alst{b}rúð-féar \hld\ of \alst{b}eðit hafði; &
hǫ́n \alst{sk}ell of hlaut \hld\ fyr \alst{sk}illinga, &
en \alst{h}ǫgg \alst{h}amars \hld\ fyr \alst{h}ringa fjǫlð. &
Svá kom \alst{Ó}ðins sonr \hld\ \alst{ę}ndr at hamri.\eva

\bvb He smote the aged sister of the ettins, \\
she who for the bride-fee had asked; \\
she got a smiting for shillings, \\
and a blow of the hammer for a multitude of rings. \\
So came Weden’s son back to his hammer.\evb\evg

\sectionline
