\bookStart{The Lay of Thrim}[Þrymskviða]

\begin{flushright}%
Dating \parencite{Sapp2022}: C9th (0.741)–C10th (0.259)

Meter: \Fornyrdislag%
\end{flushright}

Compare \Haustlong, \Hymiskvida, other poems and refer to the SkP intro to one of the big Thunder poems. TODO.

\sectionline

\bvg\bva \edtext{\alst{V}ręiðr}{\lemma{Vręiðr}\Afootnote{TODO: Note about ambiguity of alliteration.}} vas þá \alst{V}ing-Þórr \hld\ es hann \alst{v}aknaði &
ok \alst{s}íns hamars \hld\ of \alst{s}aknaði, &
\edtext{\alst{sk}ęgg nam at hrista, \hld\ \alst{sk}ǫr nam at dýja}{\lemma{skęgg \dots\ dýja ‘beard \dots\ pull’}\Bfootnote{Apparently formulaic. Cf. a certain heroic poem (TODO).}}, &
réð \alst{Ja}rðar burr \hld\ \alst{u}mb at þręifask.\eva

\bvb Wroth was then Wing-Thunder when he woke, \\
and of his hammer was bereaved. \\
His beard he took to rustle, his locks he took to rip; \\
the son of Earth resolved to grope about.\evb
\evg


\bvg\bva \edtext{\alst{O}k hann þat \alst{o}rða \hld\ \alst{a}lls fyrst of kvað:}{\lemma{Ok \dots\ of kvað ‘And ... did say’}\Bfootnote{The whole line is formulaic, occuring in five other places: sts. 3, 9 and 12 of the present poem; st 3 of \Oddrunargratr; st. 5 of \Brot.}} &
„\alst{H}ęyr-ðu nú, Loki, \hld\ \alst{h}vat ek nú mę́li &
es \alst{ęi}gi vęit \hld\ \alst{ja}rðar hvęr-gi &
né \alst{u}pp-himins: \hld\ \alst{á}ss es stolinn hamri!“\eva

\bvb And he this word first of all did say: \\
“Hear thou now, Lock, what I now speak, \\
which man knows not anywhere on earth \\
nor in up-heaven:\footnoteB{Formulaic, see Encyclopedia: \inx[F]{Earth and Up-heaven}.} \\
the \inx[G]{Eese}[os] \ken*{= Thunder = I} is robbed of his hammer!”\evb
\evg


\bvg\bva Gingu þęir \alst{f}agra \hld\ \alst{F}ręyju túna &
\alst{o}k hann þat \alst{o}rða \hld\ \alst{a}lls fyrst of kvað: &
„Munt-u mér, \alst{F}ręyja, \hld\ \alst{f}jaðr-hams léa &
ef ek \alst{m}ínn hamar \hld\ \alst{m}ę́tta’k hitta?“\eva

\bvb Went they to the fair yards of \inx[P]{Frow}, \\
and he this word first of all did say: \\
“Wilt thou me, O Frow, the \inx[P]{feather-hame} lend, \\
if I my hammer might find?”\evb
\evg


\bvg\bva „Þó mynda’k \alst{g}efa þér \hld\ þótt ór \alst{g}ulli vę́ri &
ok þó \alst{s}ęlja \hld\ at vę́ri ór \alst{s}ilfri.“\eva

\bvb {[Frow quoth:]} “I would yet give it to thee though it were golden, \\
and yet hand\footnoteB{\emph{sęlja}, cognate of English \emph{sell} here has its older sense of ‘hand over’, cf. Gotish \emph{saljan} \textcite[116]{Streitberg}: ‘\emph{opfern}; \textgreek{θύειν}’.} it to thee as it were silvern.”\footnoteB{Regaining the hammer is of such importance to the gods (cf. st. 17; without it the Eese stand powerless against the \inx[G]{Ettins}), that Frow would lend the feather-hame to the greedy and untrusty Lock, even if it were made out of gold or silver.}\evb
\evg

\bvg\bva \alst{F}ló þá Loki, \hld\ \alst{f}jaðr-hamr dunði, &
unds fyr \alst{ú}tan kom \hld\ \alst{á}sa garða &
ok fyr \alst{i}nnan kom \hld\ \alst{jǫ}tna hęima.\eva

\bvb Flew then Lock\footnoteB{Though Thunder is the one asking for the hame (“if I \emph{my} hammer might find”), Lock is the one that takes off flying.}—the feather-hame rustled— \\
until outside he came of the \inx[L]{Osyard}[yards of the Eese], \\
and inside he came of the \inx[L]{Ettinham}[homes of the Ettins].\evb
\evg


\bvg\bva \alst{Þ}rymr sat á haugi, \hld\ \edtrans{\alst{þ}ursa dróttinn}{lord of Thurses}{\Bfootnote{This formulaic expression also occurs in several Runic charms against such thursen lords; an example of the close connection between narrative and ritual poetic language.}}, &
\alst{g}ręyjum sínum \hld\ \alst{g}ull-bǫnd snøri &
ok \alst{m}ǫrum sínum \hld\ \alst{m}ǫn jafnaði.\eva

\bvb Thrim sat on the mound,\footnoteB{Apparently a typical seat for ettins. See \Voluspa\ 42 for other attestations.} the lord of \inx[G]{Thurses}: \\
on his greyhounds the golden leashes he twirled, \\
and on his mares the manes he cut even.\footnoteB{The image suggested here reminds one of the ancient “master of animals” motif, especially as attested on panel A of the Gundestrup cauldron.}\evb
\evg


\bvg\bva „\edtrans{Hvat ’s með \alst{ǫ́}sum? \hld\ Hvat ’s með \alst{ǫ}lfum?}{What is with the Eese? What is with the elves?}{\Bfootnote{Formulaic, identical line occurs in \Voluspa\ .}} &
Hví est \alst{ęi}nn kominn \hld\ í \alst{jǫ}tun-hęima?“ &
„\alst{I}llt ’s með \alst{ǫ́}sum, \hld\ \edtext{\alst{i}llt ’s með \alst{ǫ}lfum}{\Afootnote{Required by the meter; om. \Regius}}! &
\alst{H}ęfir þú \alst{H}lórriða \hld\ \alst{h}amar of folginn?“\eva

\bvb {[Thrim quoth:]} “What is with the Eese? What is with the elves? \\
Why art thou alone come into the \inx[L]{Ettinham}[Ettin-homes]?”— \\
{[Lock quoth:]} “’Tis ill with the Eese, ’tis ill with the elves! \\
Hast thou the hammer of Loride \name{= Thunder} hidden?”\evb
\evg


\bvg\bva „Ek \alst{h}ęfi \alst{H}lórriða \hld\ \alst{h}amar of folginn &
\alst{á}tta rǫstum \hld\ fyr \alst{jǫ}rð neðan; &
hann \alst{ę}ngi maðr \hld\ \alst{a}ptr of hęimtir &
nęma \alst{f}ǿri mér \hld\ \alst{F}ręyju at kvę́n.“\eva

\bvb {[Thrim quoth:]} “I have the hammer of Loride hidden, \\
eight \inx[C]{rest}[rests] beneath the earth; \\
it no man will fetch again, \\
unless he bring me Frow as wife.”\evb
\evg


\bvg\bva \alst{F}ló þá Loki, \hld\ \alst{f}jaðr-hamr dunði, &
unds fyr \alst{ú}tan kom \hld\ \alst{jǫ}tna hęima &
ok fyr \alst{i}nnan kom \hld\ \alst{á}sa garða; &
\alst{m}ǿtti hann Þór \hld\ \alst{m}iðra garða &
\alst{o}k \edtext{hann þat}{\Afootnote{emend.; \emph{þat hann} \Regius, with elsewhere unprecedented word order. Cf. note to st. 2.}} \alst{o}rða \hld\ \alst{a}lls fyrst of kvað:\eva

\bvb Flew then Lock—the feather-hame rustled— \\
until outside he came the homes of the Ettins, \\
and inside he came the yards of the Eese. \\
He met Thunder in the middle of the yards, \\
and he \ken*{= Thunder} that word first of all did say:\evb
\evg


\bvg\bva „Hęfir þú \alst{ø}rendi \hld\ sem \alst{ę}rfiði? &
Seg-ðu á \alst{l}opti \hld\ \alst{l}ǫng tíðendi! &
Opt \alst{s}itjanda \hld\ \alst{s}ǫgur of fallask, &
ok \alst{l}iggjandi \hld\ \alst{l}ygi of bęllir.“\eva

\bvb {[Thunder quoth:]} “Hast thou an errand of trouble?\footnoteB{Thunder asks Lock if he has bad news. The collocation \emph{ørendi} ‘errand’ \dots\ \emph{ęrfiði} ‘trouble, hardship’ is formulaic and occurs in X other (TODO!!) places, including in st. 5 of \HelgakvidaHjorvardssonar.} \\
Say thou aloft, the long tidings! \\
Often the sitter’s tales fail each other \\
and the lier blows up his lie.”\footnoteB{Proverbial. If one sits or lies (\emph{liggjandi} means to ‘lie down’; it is rather unfoprtunate that the two sound the same in English) down and thinks too much over bad news, details will be left out, excuses thought up. Thus it is best that Lock immediately tell Thunder what he has learned.}\evb
\evg


\bvg\bva „Hefi’k \alst{ø}rendi \hld\ \alst{ę}rfiði ok: &
\alst{Þ}rymr hęfir þinn hamar, \hld\ \alst{þ}ursa dróttinn; &
hann \alst{ę}ngi maðr \hld\ \alst{a}ptr of hęimtir &
nęma hǫ́num \alst{f}ǿri \hld\ \alst{F}ręyju at kvę́n.“\eva

\bvb {[Lock quoth:]} “I have an errand, trouble also: \\
Thrim has thy hammer, the lord of Thurses; \\
it no man will fetch again, \\
unless he bring him Frow as wife.”\evb
\evg


\bvg\bva Ganga þęir \alst{f}agra \hld\ \alst{F}ręyju at hitta &
\alst{o}k hann þat \alst{o}rða \hld\ \alst{a}lls fyrst of kvað: &
„\alst{B}itt-u þik, Fręyja, \hld\ \alst{b}rúðar líni! &
Vit skulum \alst{a}ka tvau \hld\ í \alst{jǫ}tun-hęima.“\eva

\bvb Go they the fair Frow to find, \\
and he\footnoteB{Unclear. Possibly Lock, since he was the speaker of the last verse.} this word first of all did say: \\
“Bind thyself, Frow, with bride’s linen!\footnoteB{A linen band tied around the bride’s head. TODO: Reference this note.} \\
We two shall drive into the Ettin-homes.”\evb
\evg


\bvg\bva Ręið varð þá \alst{F}ręyja \hld\ ok \alst{f}nasaði, &
\alst{a}llr \alst{á}sa salr \hld\ \alst{u}ndir bifðisk, &
stǫkk þat it \alst{m}ikla \hld\ \alst{m}ęn Brísinga: &
„Mik \alst{v}ęitst \alst{v}erða \hld\ \alst{v}er-gjarnasta &
ef ek \alst{ę}k með þér \hld\ í \alst{jǫ}tun-hęima.“\eva

\bvb Wroth became then Frow, and snorted; \\
the whole hall of the Eese trembled below; \\
down crashed the great necklace of the Brisings— \\
“Thou knowest that I will become the most man-eager,\footnoteB{Either Frow is speaking out of self-awareness of her own lustful inclinations, or the sense is that she will be accused of being lustful by the other gods, but there is no verb here corresponding to ‘accuse’. For Frow’s promiscuity see \Lokasenna\ 30 and Note.} \\
if I drive with thee into the Ettin-homes.”\evb
\evg


\bvg\bva \edtext{Sęnn vǫ́ru \alst{ę́}sir \hld\ \alst{a}llir á þingi &
ok \alst{ǫ́}synjur \hld\ \alst{a}llar á máli, &
ok umb þat \alst{r}éðu \hld\ \alst{r}íkir tívar:}{\lemma{Sęnn \dots\ tívar ‘Soon \dots\ Tews’}\Bfootnote{Formulaic, identically shared with \Baldrsdraumar\ 1/1–3 (see Note there).}} &
\alst{h}vé þęir \alst{H}lórriða \hld\ \alst{h}amar of sǿtti?\eva

\bvb Soon were the \inx[G]{Eese} all at the \inx[C]{Thing}, \\
and the \inx[G]{Ossens} all at speech, \\
and of this counseled the mighty \inx[G]{Tews}: \\
How they Loride’s \name{= Thunder’s} hammer would find?\evb
\evg


\bvg\bva Þá kvað þat \alst{H}ęimdallr, \hld\ \alst{h}vítastr ása, &
\alst{v}issi \alst{v}ęl framm \hld\ sęm \alst{v}anir aðrir: &
„\alst{B}indu vér Þór þá \hld\ \alst{b}rúðar líni; &
hafi hann it \alst{m}ikla \hld\ \alst{m}ęn Brísinga!\eva

\bvb Then quoth that \inx[P]{Homedall}, whitest of the Eese; \\
he knew well forth,\footnoteB{\emph{vita framm} ‘to know forth’, i.e. to know the future. Compare \emph{fram-víss} ’forth-wise; prescient.’} like the other \inx[G]{Wanes}: \\
“Let us bind Thunder then, with bride’s linen; \\
he may have the great \inx[P]{necklace of the Brisings}.\evb
\evg


\bvg\bva Lǫ́tum und \alst{h}ǫ́num \hld\ \alst{h}rynja lukla &
ok \alst{k}ven-váðir \hld\ umb \alst{k}né falla &
en á \alst{b}rjósti \hld\ \alst{b}ręiða stęina &
ok \alst{h}ag-liga \hld\ umb \alst{h}ǫfuð typpum!“\eva

\bvb Let us place by his side keys to jingle, \\
and women’s garments to fall down about his knees, \\
and on the breast broad stones, \\
and skillfully let us tip his head!\footnoteB{This verse contains an interesting description of Viking age bridal dress: As the everyday manager of the household, keys were the mark of a respectable married woman. The “broad stones” on the breast are probably tortoise brooches, while the tipping of the head refers to some sort of bridal hat (TODO: Literature). Breast-brooches are also mentioned in \Volundarkvida\ 25, 36.}”\evb
\evg


\bvg\bva Þá kvað þat \alst{Þ}órr, \hld\ \alst{þ}rúðugr áss: &
„Mik munu \alst{ę́}sir \hld\ \alst{a}rgan kalla &
ef ek \alst{b}indask lę́t \hld\ \alst{b}rúðar líni!“\eva

\bvb Then quoth that Thunder, the mighty Os: \\
“Me will the Eese call \inx[C]{degenerate}, \\
if I let myself be bound with bride’s linen!”\evb
\evg


\bvg\bva Þá kvað þat \alst{L}oki \hld\ \alst{L}aufęyjar sonr: &
„\alst{Þ}ęgi þú, \alst{Þ}órr, \hld\ \alst{þ}ęira orða! &
Þegar munu \alst{jǫ}tnar \hld\ \alst{Á}s-garð búa &
nęma \alst{þ}ú \alst{þ}inn hamar \hld\ \alst{þ}ér of hęimtir.“\eva

\bvb Then quoth that Lock, Leafie’s son: \\
“Shut up thou, Thunder, with those words! \\
Shortly the Ettins will settle Osyard, \\
unless thou thy hammer for thyself dost fetch!”\evb
\evg


\bvg\bva \alst{B}undu þęir Þór þá \hld\ \alst{b}rúðar líni &
ok hinu \alst{m}ikla \hld\ \alst{m}ęni Brísinga, &
létu und \alst{h}ǫ́num \hld\ \alst{h}rynja lukla &
ok \alst{k}ven-váðir \hld\ umb \alst{k}né falla &
en á \alst{b}rjósti \hld\ \alst{b}ręiða stęina &
ok \alst{h}ag-liga \hld\ of \alst{h}ǫfuð typpðu.\eva

\bvb Bound they Thunder then, with bride’s linen, \\
and with the great necklace of the Brisings. \\
They placed by his side keys to jingle, and women’s garments to fall down about his knees, and on the breast broad stones, and skillfully they tipped his head.\evb
\evg


\bvg\bva Þá kvað þat \alst{L}oki \hld\ \alst{L}aufęyjar sonr: &
„Mun’k \alst{au}k með þér \hld\ \alst{a}mbǫ́tt vesa, &
vit skulum \alst{a}ka tvau \hld\ í \alst{jǫ}tun-hęima.“\eva

\bvb Then quoth that Lock, Leafie’s son: \\
“I will also with thee be a handmaid; \\
we two\footnoteB{The form used, \emph{tvau}, is the neuter plural, i.e. one of the pair is female and the other male. This is either an error due to mindless copying of v. 11, or a backhanded insult against Thunder.} shall drive into the Ettin-homes.”\evb
\evg


\bvg\bva Sęnn vǫ́ru \alst{h}afrar \hld\ \alst{h}ęim of vreknir, &
\alst{sk}yndir at \alst{sk}ǫklum, \hld\ \alst{sk}yldu vęl renna; &
\alst{b}jǫrg \alst{b}rotnuðu, \hld\ \alst{b}rann jǫrð loga; &
\alst{ó}k \alst{Ó}ðins sonr \hld\ í \alst{jǫ}tun-hęima.\eva

\bvb Soon \inx[C]{he-goats}\footnoteB{Thunder’s cart was driven by he-goats, for which he is called (for instance) “the lord of he-goats” in \Hymiskvida\ 20, 31. See Encyclopedia.} were driven home, \\
hastened onto the cart-poles; they were to run well. \\
Crags burst, burned the earth with flame; \\
drove Weden’s son \ken*{= Thunder} into the Ettin-homes.\footnoteB{Thunder’s driving of his chariot is often connected with cosmic disturbance. So, his arrival in \Lokasenna\ (st. 55) is signalled by the mountains quaking. The most similar description to the present stanza is found in Thedwolf’s \Haustlong\ 14–16, where crags (there likewise \emph{bjǫrg}) burst asunder and fires rage before him. A possibly Indo-European parallel to this is the Vedic myth of Indra breaking the mountains and releasing the mountains (as described most famously in \Rigveda\ hymn 1.32). See also \Baldrsdraumar\ 3 for a related description of the god Weden’s riding.}\evb
\evg


\bvg\bva Þá kvað þat \alst{Þ}rymr, \hld\ \alst{þ}ursa dróttinn: &
„\alst{St}andið upp, jǫtnar, \hld\ ok \alst{st}ráið bękki! &
Nú \alst{f}ǿrið mér \hld\ \alst{F}ręyju at kván, &
\alst{N}jarðar dóttur \hld\ ór \alst{N}óa-túnum.\eva

\bvb Then quoth that Thrim, the lord of Thurses: \\
“Stand ye up, ettins, and strew the benches! \\
Now bring ye me Frow as wife, \\
\inx[P]{Nearth}’s daughter from the \inx[L]{Nowetowns}.\evb
\evg


\bvg\bva \alst{G}anga hér at \alst{g}arði \hld\ \alst{g}ull-hyrnðar kýr, &
\edtrans{\alst{ø}xn \alst{a}l-svartir}{all-black oxen}{\Bfootnote{Formulaic, also occurring in \Hymiskvida\ 18. That all-black (i.e. spotlessly black) oxen were most valued is seen by the pairing with “golden-horned”. One may also compare Saxo (I.8.12), where the hero Hadding has to atone for his slaying of a heavenly being by the blooting of dark-coloured victims (\emph{furvae hostiae}): \emph{Siquidem propiciandorum numinum gratia Frø deo rem diuinam furuis hostiis fecit. Quem litationis morem annuo feriarum circuitu repetitum posteris imitandum reliquit. Frøblod Sueones uocant.} ‘In order to mollify the divinities he [= Hadding] did indeed make a holy sacrifice of dark-coloured victims to the god Frø. He repeated this mode of propitiation at an annual festival and left it to be imitated by his descendants. The Swedes call it Frøblot.’ This ancient ritual taboo is further paralleled e.g. by the Tanakh, where animals dedicated to Yhwh were to be without blemish (\textgreek{תָּמִ֖ים}, Leviticus 1:3)}}, \hld\ \alst{jǫ}tni at gamni, & %TODO: Hebrew.
fjǫlð á’k \alst{m}ęiðma, \hld\ fjǫlð á’k \alst{m}ęnja; &
\alst{ęi}nnar mér Fręyju \hld\ \alst{á}-vant þykkir.“\eva

\bvb Here march to the estate golden-horned cows, \\
all-black oxen, for the ettin’s \ken*{= my} pleasure. \\
A multitude I own of treasures, a multitude I own of necklaces; \\
only Frow I think myself missing.”\evb
\evg


\bvg\bva Vas þar at \alst{k}veldi \hld\ of \alst{k}omit snimma &
\alst{o}k fyr \alst{jǫ}tna \hld\ \alst{ǫ}l framm borit. &
\alst{Ęi}nn át \alst{o}xa, \hld\ \alst{á}tta laxa, &
\alst{k}rásir allar, \hld\ þę́r’s \alst{k}onur skyldu, &
drakk \alst{S}ifjar verr \hld\ \alst{s}áld þrjú mjaðar.\eva

\bvb There was the evening early come, \\
and for the ettins ale brought forth. \\
Alone ate he \ken*{= Thunder} an ox, eight salmons, \\
all the dainties which were meant for the women; \\
drank the husband of Sib \ken*{= Thunder} three sieves of mead.\footnoteB{Cf. \Hymiskvida\ 15, where Thunder eats two of Hymer’s oxen. It is rather interesting that the same kenning is used in both stanzas when both concern the god’s great eating; perhaps one poet was playing on the other’s expression, or they were both referencing another, now-lost work.}\evb
\evg


\bvg\bva Þá kvað þat \alst{Þ}rymr, \hld\ \alst{þ}ursa dróttinn: &
„Hvar sátt-u \alst{b}rúðir \hld\ \alst{b}íta hvassara? &
Sá’k-a \alst{b}rúðir \hld\ \alst{b}íta ęnn \alst{b}ręiðara &
né ęnn \alst{m}ęira \alst{m}jǫð \hld\ \alst{m}ęy of drekka!“\eva

\bvb Then quoth that Thrim, the lord of Thurses: \\
“Where sawest thou brides bite sharper? \\
Saw I never brides bite yet broader, \\
nor yet more mead a maiden drink.”\evb
\evg


\bvg\bva Sat hin \alst{a}l-snotra \hld\ \alst{a}mbǫ́tt fyrir &
es \alst{o}rð of fann \hld\ við \alst{jǫ}tuns máli: &
„\alst{Á}t vę́tr Fręyja \hld\ \alst{á}tta nǫ́ttum, &
svá vas hón \alst{ó}ð-fús \hld\ í \alst{jǫ}tun-hęima.“\eva

\bvb Sat the all-clever maid-servant \ken*{= Lock} in front, \\
who a word did find against the ettin’s speech: \\
“Ate Frow naught, for eight nights; \\
so madly did she long for the Ettin-homes.”\evb
\evg


\bvg\bva \alst{L}aut und \alst{l}ínu, \hld\ \alst{l}ysti at kyssa, &
en hann \alst{ú}tan stǫkk \hld\ \alst{ę}nd-langan sal: &
„Hví eru \alst{ǫ}ndótt \hld\ \alst{au}gu Fręyju? &
Þykki mér \alst{ó}r \hld\ \alst{au}gum brenna!“\eva

\bvb He looked ’neath the linen, lusted for a kiss,—
but he from the outside leapt back, across the length of the hall:—
“Why are the eyes of Frow fiery?—
Methinks there be flame coming out of the eyes!\footnoteB{Lit. “Methinks out of the eyes burn.”}”\evb
\evg


\bvg\bva Sat hin \alst{a}l-snotra \hld\ \alst{a}mbǫ́tt \edtext{fyrir}{\Afootnote{add. \emph{†ſ.†} \Regius.}} &
es \alst{o}rð of fann \hld\ við \alst{jǫ}tuns máli: &
„Svaf vę́tr Fręyja \hld\ átta nǫ́ttum, &
svá vas hón \alst{ó}ð-fús \hld\ í \alst{jǫ}tun-hęima.“\eva

\bvb Sat the all-clever maid-servant \ken*{= Lock} in front, \\
who a word did find against the ettin’s speech: \\
“Slept Frow naught, for eight nights; \\
so madly did she long for the Ettin-homes.”\evb
\evg


\bvg\bva \alst{I}nn kom hin \alst{a}rma \hld\ jǫtna systir, &
hin’s \alst{b}rúð-féar \hld\ \alst{b}iðja þorði: &
„Lát þér af \alst{h}ǫndum \hld\ \alst{h}ringa rauða &
ef þú \alst{ǫ}ðlask vill \hld\ \alst{á}stir mínar, &
\edtext{\alst{á}stir mínar, \hld\ \alst{a}lla hylli}{\lemma{ástir mínar, alla hylli ‘my love, [and] all [my] holdness’}\Bfootnote{Probably formulaic. There are no preserved parallels in poetry, but there seems to be one in \Gylfaginning\ 49 (excerpt):
\begin{quote}
\emph{En er goðin vitkuðust, þá mę́lti Frigg ok spurði, hverr sá vę́ri með ásum, er \textbf{eignast vildi „allar ástir mínar}} (so \Trajectinus\Wormianus; \emph{ástir hennar} ‘her loves’ \RegiusProse\Upsaliensis) \emph{\textbf{ok hylli}, ok vili hann ríða á hel-veg ok freista, ef hann fái fundit Baldr, ok bjóða Helju út-lausn, ef hon vill láta fara Baldr heim í Ás-garð.“}

‘But when the gods came to their wits [after Balder’s death], then Frie spoke and asked which one among the Eese \textbf{would own “all my loves and holdness}, and will ride onto the Hellway and see if he can find Balder, and offer Hell a ransom if she will let Balder come home to Osyard.”’
\end{quote}
We can tell from the citation of a \Ljodahattr\ stanza at the end of ch. 49 that Snorre knew one or more now-lost Eddic poems about Balder’s death (cf. \Gylfaginning\ 37, where \Skirnismal\ is retold in prose, and then the final st. is cited), and it seems that one of these contained the same two long-lines as the present stanza. For such a sharing of lines cf. e.g. st. 14 above, the first three long-lines of which are identically shared with \Baldrsdraumar\ 1.}}!“\eva

\bvb In came the wretched sister of the ettins, \\
the one who for the bride-fee \ken*{= Millner} had dared ask: \\
“Slide off from thy hands the red rings, \\
if thou wilt win my love, \\
my love, [and] all [my] \inx[C]{holdness}.”\footnoteB{The sister, who was apparently the one who asked for the Hammer, now has the audacity to ask Thunder (disguised as Frow) to give her the very rings on his hands.}\evb
\evg


\bvg\bva Þá kvað þat \alst{Þ}rymr, \hld\ \alst{þ}ursa dróttinn: &
„\alst{B}erið inn hamar \hld\ \alst{b}rúði at vígja, &
lęggið \alst{M}jǫllni \hld\ í \alst{m}ęyjar kné, &
\alst{v}ígið okkr saman \hld\ \edtrans{\alst{V}árar}{Ware}{\Bfootnote{A minor goddess presiding over romantic relationships and weddings. See Encyclopedia.}} hęndi!“\eva

\bvb Then quoth that Thrim, the lord of Thurses: \\
“Bear ye in the hammer, the bride for to bless; \\
lay ye Millner in the maiden’s knee; \\
bless ye us two together by \inx[P]{Ware}’s hand!”\evb
\evg


\bvg\bva \alst{H}ló \alst{H}lórriða \hld\ \alst{h}ugr í brjósti &
es \alst{h}arð-\alst{h}ugaðr \hld\ \alst{h}amar of þękkði; &
\alst{Þ}rym drap hann fyrstan, \hld\ \alst{þ}ursa dróttin, &
ok \alst{ę́}tt \alst{jǫ}tuns \hld\ \alst{a}lla lamði.\eva

\bvb Laughed the heart in Loride’s \name{= Thunder’s} chest, \\
when, hard-hearted, he recognized the hammer. \\
Thrim he slew first, the lord of Thurses, \\
and all the ettin’s lineage he beat lame.\evb
\evg


\bvg\bva Drap hann ina \alst{ǫ}ldnu \hld\ \alst{jǫ}tna systur, &
hin’s \alst{b}rúð-féar \hld\ of \alst{b}eðit hafði; &
hón \alst{sk}ell of hlaut \hld\ fyr \alst{sk}illinga, &
en \alst{h}ǫgg \alst{h}amars \hld\ fyr \alst{h}ringa fjǫlð. &
Svá kom \alst{Ó}ðins sonr \hld\ \alst{ę}ndr at hamri.\eva

\bvb He slew the aged sister of the ettins, \\
the one who for the bride-fee had asked; \\
a smiting she received for shillings, \\
and a strike of the hammer for a multitude of rings. \\
So got Weden’s son \ken*{= Thunder} back his hammer.\evb
\evg
