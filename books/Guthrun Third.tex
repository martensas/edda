\bookStart{The Third Lay of Guthrun}[Guðrúnarkviða þriðja]

\begin{flushright}%
\textbf{Dating} \parencite{Sapp2022}: 900s (0.731), early 1000s (0.178)

\textbf{Meter:} \Fornyrdislag%
\end{flushright}

A very short narrative poem of ballad-type, depicting a single event from the legendary cycle.  It is especially notable for its depiction of a trial by ordeal and the mention of a woman being drowned in a bog.

Herch, one of Attle’s concubines tells Attle that she has seen his wife Guthrun sleeping with Thedric. Attle becomes distressed upon hearing this (P1). Guthrun asks him what is wrong (1), and he responds that Herch has accused her of sleeping with Thedric (2). Guthrun promises to to prove her innocence through a trial by ordeal involving picking up a white stone from boiling water (3). She further says that while she and Thedric did sit down together, they did so in mutual grief over the deaths of her brothers (4–5). She tells Attle to summon a German lord named Saxe, who knows how to carry out the trial. Seven hundred men arrive to witness the event (6). Before picking up the stone, Guthrun laments over her brothers’ deaths, saying that they would have disputed the accusation through violence, but that she must now prove her innocence by herself (7). She then puts her hand in the boiling water, and unscathed takes out the stones. She holds it up and shows it to the witnesses (8). Attle laughs, knowing that his wife has been faithful, and orders Herch to pick up the stone (9). She does so, but her hands are horribly scorched, and men lead her to a “foul bog”, presumably to be drowned. The poet ends by laconically stating that Guthrun in such a way was “reconstituted for her affronts”.

\sectionline

\bpg\bpa Herkja hét ambǫ́tt Atla; hón hafði verit frilla hans. Hón sagði Atla at hón hefði sét Þjóðrek ok Guðrúnu bę́ði saman. Atli var þá allókátr. Þá kvað Guðrún:\epa

\bpb Herch was named the female thrall of Attle; she had been his concubine. She told Attle that she had seen Thedric and Guthrun both together. Attle was then wholly displeased. Then Guthrun quoth:\epb\epg


\bvg\bva „Hvat ’s þér, \alst{A}tli? \hld\ \alst{ę́}, Buðla sonr, &
es þér \alst{h}ryggt í \alst{h}ug; \hld\ hví \alst{h}lę́r þú ę́va? &
Hitt myndi \alst{ǿ}ðra \hld\ \alst{jǫ}rlum þykkja &
at við \alst{m}ęnn \alst{m}ę́ltir \hld\ ok \alst{m}ik sę́ir.“\eva

\bvb “What is with thee, Attle? Always, O son of Bodle, \\
art thou sad at heart—why laughest thou never? \\
TODO.”\evb\evg

%TODO: Add speaker notes

\bvg\bva „Tregr mik þat, \alst{G}uðrún, \hld\ \alst{G}júka dóttir, &
mér í \alst{h}ǫllu \hld\ \alst{H}ęrkja sagði &
at \alst{þ}it \alst{Þ}jóðrekr \hld\ undir \alst{þ}aki svę́fið &
ok \alst{l}éttliga \hld\ \alst{l}íni vęrðið.“\eva

\bvb “This troubles me, Guthrun, Yivick’s daughter: \\
in the hall has Herch told me \\
that thou and Thedric beneath thatched roof slept, \\
and ye lightly warded the linen.\footnoteB{i.e., they threw off their clothes and slept together.}”\evb\evg


\bvg\bva „Þér mun’k \alst{a}lls þęss \hld\ \alst{ęi}ða vinna &
at inum \alst{h}víta \hld\ \alst{h}ęlga stęini, &
at ek við \alst{Þ}jóðmar \hld\ \alst{þ}at-ki átta’k, &
es \alst{v}ǫrðr né \alst{v}err \hld\ v\alst{i}nna knátti,—\eva

\bvb “To thee I will swear oaths of all of that— \\
by the white, holy stone— \\
that I did not do such a thing with Thedmar,\footnoteB{Historically, Thedmar was the father of Thedric, who took over the kingdom after his father’s death (see Encyclopedia). Thedmar may here be a scribal error for Thedric, a scribal error for “Thedmar’s son”, or a nickname due to conflation of the father and son.} \\
which neither wife nor husband has been able to swear upon,—\footnoteB{Guthrun says that she will prove her innocence through a trial by ordeal (that is, by lifting “the white holy stone” out of boiling water; see st. 8). She further strengthens her position by pointing out that no reliable person has sworn an oath attesting to her guilt.}\evb\evg


\bvg\bva nema ek \alst{h}alsaða \hld\ \alst{h}ęrja stilli, &
\alst{jǫ}fur \alst{ó}·nęisinn, \hld\ \alst{ęi}nu sinni; &
\alst{a}ðrar vǫ́ru \hld\ \alst{o}kkrar spękjur &
es vit \alst{h}ǫrmug tvau \hld\ \alst{h}nigum at rúnum.\eva

\bvb unless I embraced the stiller of hosts \ken*{\textsc{ruler} = Thedmar}: \\
the unshamed prince, a single time. \\
Different were the dealings of us two, \\
when distressed [Guthrun and Thedric] we reclined in whispers.\evb\evg


\bvg\bva Hér kom \alst{Þ}jóðrekr \hld\ með \alst{þ}ría tøgu, &
lifa \alst{þ}ęir né ęinir, \hld\ \alst{þ}riggja tega manna; &
\edtrans{hrink-tu}{surround}{\Bfootnote{Consisting of \emph{hring}, 2nd sg. imper. of \emph{hringja} ‘surround, encircle’ + \emph{þú} ‘thou’.  The clitic form \emph{-tu} has caused devoicing.}} mik at \alst{b}rǿðrum \hld\ ok at \alst{b}rynjuðum, &
\alst{h}rink-tu mik at ǫllum \hld\ á \alst{h}ǫfuð-niðjum.\eva

\bvb Here came Thedric with thirty men; \\
of those thirty none still lives.— \\
Surround me with brothers and with byrnied men; \\
surround me with all close kinsmen!\evb\evg


\bvg\bva \alst{S}ęnd at \alst{S}axa, \hld\ \alst{s}unn-manna gram; &
\alst{h}ann kann \alst{h}ęlga \hld\ \alst{h}ver vellanda;“ &
\alst{s}jau hundruð manna \hld\ í \alst{s}al gingu &
áðr \alst{k}vę́n \alst{k}onungs \hld\ í \alst{k}ętil tǿki.\eva

\bvb Send for Saxe, the lord of the Southmen, \\
he can hallow a boiling cauldron!” \\
Seven hundred men went into the hall, \\
before the king’s wife should reach into the kettle.\evb\evg


\bvg\bva „\alst{K}ømr-a nú Gunnarr, \hld\ \alst{k}alli’k-a Hǫgna, &
\alst{s}é’k-a \alst{s}íðan \hld\ \alst{s}vása brǿðr; &
\alst{s}verði myndi Hǫgni \hld\ \alst{s}líks harms reka, &
nú verð’k \alst{s}jǫlf fyr mik \hld\ \alst{s}ynja lýta.“\eva

\bvb “Now Guther comes not; I cannot call on Hain; \\
I see not henceforth [my] beloved brothers. \\
by his sword would Hain avenge such an affront; \\
now must I for myself disprove the slanders!”\evb\evg


\bvg\bva \alst{B}rá hón til \alst{b}otns \hld\ \alst{b}jǫrtum lófa &
ok hón \alst{u}pp of tók \hld\ \edtrans{jarkna-stęina}{arkenstones}{\Bfootnote{Gems, crystals; probably a borrowing from the Old English \emph{eorcnan-stânas} ‘id.’  The modern English form \emph{arkenstone} was coined by Tolkien.}}: &
„\alst{S}é nú \alst{s}ęggir \hld\ —\alst{s}ykn em ek orðin &
\alst{h}ęilag-liga— \hld\ hvé sjá \alst{h}verr velli.“\eva

\bvb She thrust to the bottom her bright palms, \\
and she up did take the arkenstones: \\
“Let men now see—I am proven innocent, \\
through holy means!—how this cauldron boils!”\evb\evg


\bvg\bva \alst{H}ló þá Atla \hld\ \alst{h}ugr í brjósti &
es hann \alst{h}ęilar sá \hld\ \alst{h}ęndr Guðrúnar: &
„Nú skal \alst{H}ęrkja \hld\ til \alst{h}vers ganga, &
sú’s \alst{G}uðrúnu \hld\ \alst{g}randi vę́nti.“\eva

\bvb Then laughed the heart in Attle’s chest, \\
when he saw unscathed the hands of Guthrun: \\
“Now shall Herch to the cauldron go, \\
she who hoped for Guthrun’s harm.”\evb\evg


\bvg\bva \alst{S}á-at maðr armligt, \hld\ hvęrr es þat \alst{s}á-at, &
\alst{h}vé þar á \alst{H}ęrkju \hld\ \alst{h}ęndr sviðnuðu; &
\edtrans{lęiddu þá \alst{m}ęy \hld\ í \alst{m}ýri fúla}{Led they that maiden into a foul bog}{\Bfootnote{I.e. to be drowned. Drowning in bogs was a common Germanic punishment for perjurers; see note to \Voluspa\ 38.}}, &
\alst{s}vá þá Guðrún \hld\ \alst{s}inna harma.\eva

\bvb Man saw nothing pitiful, who did not see that: \\
how there on Herch the hands were scorched. \\
Led they that maiden into a foul bog; \\
so was Guthrun reconstituted for her affronts.\evb\evg

\sectionline
