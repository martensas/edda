\bookStart{Third Lay of Guthrun}[Guðrúnarkviða þriðja]
\def\thisBookCode{GudrunThree}

\begin{flushright}%
\textbf{Dating} \parencite{Sapp2022}: C10th (0.731)–early C11th (0.178)

\textbf{Meter:} \Fornyrdislag%
\end{flushright}

\section{Introduction}

The \textbf{Third Lay of Guthrun} (\GudrunThree) is a short narrative poem, depicting just a single scene.  At 10 stanzas it is the shortest poem in \Regius, and arguably one of the most forgettable.  Its only notable moments are its depiction of an ordeal by hot water and its allusion to the drowning of a slave-woman in a bog.

\subsection{Dating}

The most important factor towards dating \GudrunThree\ is its conception of the ordeal by hot water.  This type of ordeal first appears in the early C6th Frankish \emph{Salic Law}, and is always closely associated with the Catholic clergy.  TODO: We ought to investigate when it went out of fashion.  https://www.degruyterbrill.com/document/doi/10.9783/9781512817492-003/html

\subsection{Summary}

Herch, one of Attle’s slave-women and concubines tells him that she has seen his wife Guthrun sleep with Thedric. Attle becomes distressed upon hearing this (P1). Guthrun asks him what is wrong (1), and he responds that Herch has accused her of sleeping with Thedric (2). Guthrun promises to prove her innocence through a trial by ordeal involving taking up a white stone out of boiling water (3); while she and Thedric did sit down together, they did so only in mutual grief over the deaths of her brothers (4–5).  Guthrun tells Attle to summon the German lord Saxe to carry out the trial, and seven hundred men arrive as witnesses (6). Before picking up the stone, Guthrun laments over her brothers’ deaths, saying that they would have disputed the accusation through violence, but that she must prove her innocence alone (7). She then puts her hand in the boiling water, and takes out the stone unscathed. She holds it up and shows it to the witnesses (8). Attle laughs, knowing that his wife has been faithful, and orders Herch to pick up the stone (9). She does so—and her hands are horribly scorched. She is dragged to a “foul bog”, presumably to be drowned. The poet ends by laconically stating that this was how Guthrun in such a way was “restored for her affronts”.

\section{The Third Lay of Guthrun}

\bpg\bpa Herkja hét ambǫ́tt Atla; hón hafði verit frilla hans. Hón sagði Atla at hón hefði sét Þjóðrek ok Guðrúnu bę́ði saman. Atli var þá all-ó-kátr. Þá kvað Guðrún:\epa

\bpb Herch was named the female thrall of Attle; she had been his concubine. She told Attle that she had seen Thedric and Guthrun together. Thereafter Attle was very unhappy. Then Guthrun quoth:\epb\epg


\bvg\bva%
„Hvat ’s þér, \alst{A}tli? \hld\ \alst{ę́}, Buðla sonr, &
es þér \alst{h}ryggt í \alst{h}ug; \hld\ hví \alst{h}lę́r þú ę́va? &
Hitt myndi \alst{ǿ}ðra \hld\ \alst{jǫ}rlum þykkja &
at við \alst{m}ęnn \alst{m}ę́ltir \hld\ ok \alst{m}ik sę́ir.“\eva

\bvb “What is with thee, Attle? Always, son of Bodle, \\
art thou sad at heart—why laughest thou never? \\
It would seem more proper to earls \\
that thou spoke with men and looked at me.”\evb\evg

%TODO: Add speaker notes

\bvg\bva%
„Tregr mik þat, \alst{G}uðrún, \hld\ \alst{G}júka dóttir, &
mér ï \alst{h}ǫllu \hld\ \alst{H}ęrkja sagði &
at \alst{þ}it \alst{Þ}jóðrekr \hld\ undir \alst{þ}aki svę́fið &
ok \edtrans{\alst{l}éttliga \hld\ \alst{l}íni vęrðið}{lightly did ye mind your linen}{\Bfootnote{Euphemistic; they threw off their clothes and slept with each other.}}.“\eva

\bvb “It troubles me, O Guthrun, Yivick’s daughter, \\
that in the hall Herch told me \\
that thou and Thedric slept beneath one roof, \\
and lightly did ye mind your linen.”\evb\evg


\bvg\bva%
„Þér mun’k \alst{a}lls þęss \hld\ \alst{ęi}ða vinna &
\edtrans{at inum \alst{h}víta \hld\ \alst{h}ęlga stęini}{by the white, holy stone}{\Bfootnote{The stone lifted out of a pot of boiling water in the trial by ordeal, as described further in st. 8.}}, &
at ek við \edtrans{\alst{Þ}jóðmar}{Thedmar}{\Bfootnote{Historically, Thedmar (\emph{Theodemir}) was the father of Thedric (\emph{Theoderic}) the Great, who took over the kingdom after his father’s death (see Index).  The use of the name here may either be a scribal error (whether for “Thedric” or for “Thedmar’s son”), or a nickname caused by the conflation of the two persons in the late Norse tradition.}} \hld\ \alst{þ}at-ki átta’k, &
\edtrans{es \alst{v}ǫrðr né \alst{v}err \hld\ \alst{v}inna knátti}{those which no married woman nor man has sworn}{\Bfootnote{I.e., “those oaths which et c.” — Guthrun’s use of \emph{vǫrðr} ‘wife, married woman’ and \emph{verr} ‘husband, married man’ serve to question the reliability of Herch’s testimony by pointing out that she, as an unmarried slave-woman, is not in a position to make legally binding accusations.}},—\eva

\bvb “For thee I will swear the oaths to all of it \\
—by the white, holy stone, \\
that I did no such thing with Thedmar— \\
those which no married woman nor man has sworn,\evb\evg


\bvg\bva%
nema ek \alst{h}alsaða \hld\ \alst{h}ęrja stilli, &
\alst{jǫ}fur \alst{ó}·nęisinn, \hld\ \alst{ęi}nu sinni; &
\alst{a}ðrar vǫ́ru \hld\ \alst{o}kkrar spękjur &
es vit \alst{h}ǫrmug tvau \hld\ \alst{h}nigum at rúnum.\eva

\bvb unless I embraced the stiller of hosts \ken*{\textsc{ruler} = Thedmar}, \\
the unshamed prince, a single time. \\
Different were the dealings of us two, \\
when we, distressed, reclined in whispers.\evb\evg


\bvg\bva%
Hér kom \alst{Þ}jóðrekr \hld\ með \alst{þ}rjá tøgu, &
lifa \alst{þ}ęir né ęinir, \hld\ \alst{þ}riggja tega manna; &
\edtrans{hrink-tu}{surround}{\Bfootnote{Consisting of \emph{hring}, 2nd sg. imper. of \emph{hringja} ‘surround, encircle’ + \emph{þú} ‘thou’.  The clitic form \emph{-tu} has caused devoicing.}} mik at \alst{b}rǿðrum \hld\ ok at \alst{b}rynjuðum, &
\alst{h}rink-tu mik at ǫllum \hld\ ȧ \alst{h}ǫfuð-niðjum.\eva

\bvb Hither Thedric came with thirty men; \\
of those thirty none still lives.— \\
Surround me with brothers and with byrnied men; \\
surround me with all close kinsmen!\evb\evg


\bvg\bva%
\alst{S}ęnd at \edtrans{\alst{S}axa, \hld\ \alst{s}unn-manna gram}{Saxe, the lord of Southmen}{\Bfootnote{The Southmen being the Germans. — This line shows that the trial by cauldron was considered a foreign, specifically German custom by the poet, who naturally imagined Attle and Guthrun to have belonged to his own, Norse culture.  For its bearing on dating the poem see Introduction.}}; &
\alst{h}ann kann \alst{h}ęlga \hld\ \alst{h}ver vellanda;“ &
\alst{s}jau hundruð manna \hld\ ï \alst{s}al gingu &
áðr \alst{k}vę́n \alst{k}onungs \hld\ ï \alst{k}ętil tǿki.\eva

\bvb Send for Saxe, the lord of Southmen; \\
he can hallow the boiling cauldron.” \\
Seven hundred men went into the hall, \\
before the king’s wife should reach into the kettle.\evb\evg


\bvg\bva%
„\alst{K}ømr-a nú Gunnarr, \hld\ \alst{k}alli’k-a Hǫgna, &
\alst{s}é’k-a \alst{s}íðan \hld\ \alst{s}vása brǿðr; &
\alst{s}verði myndi Hǫgni \hld\ \alst{s}líks harms reka, &
nú verð’k \alst{s}jǫlf fyr mik \hld\ \alst{s}ynja lýta.“\eva

\bvb “Now Guther will not come; I will not call on Hain; \\
I will not henceforth see my beloved brothers. \\
By his sword would Hain avenge such an affront; \\
now I for myself must disprove the slanders!”\evb\evg


\bvg\bva%
\alst{B}rá hȯn til \alst{b}otns \hld\ \alst{b}jǫrtum lófa &
ok hȯn \alst{u}pp of tók \hld\ \edtrans{jarkna-stęina}{arkenstones}{\Bfootnote{Gems, crystals; probably a borrowing from the Old English \emph{eorcnan-stânas} ‘id.’  The modern English form \emph{arkenstone} was coined by Tolkien.}}: &
„\alst{S}é nú \alst{s}ęggir \hld\ —\alst{s}ykn em’k orðin &
\alst{h}ęilag-liga— \hld\ hvé sjá \alst{h}verr velli.“\eva

\bvb She thrust to the bottom her bright palms, \\
and she up did take the arkenstones: \\
“Let men now see—I am proven innocent \\
through holy means!—how this cauldron boils!”\evb\evg


\bvg\bva%
\alst{H}ló þȧ Atla \hld\ \alst{h}ugr ï brjósti &
es hann \alst{h}ęilar sá \hld\ \alst{h}ęndr Guðrúnar: &
„Nú skal \alst{H}ęrkja \hld\ til \alst{h}vers ganga, &
sú’s \alst{G}uðru̇nu \hld\ \alst{g}randi vę̇nti.“\eva

\bvb Then laughed the heart in Attle’s chest, \\
when he saw unscathed the hands of Guthrun: \\
“Now shall Herch to the cauldron go, \\
she who hoped for Guthrun’s injury!”\evb\evg


\bvg\bva%
\alst{S}á-at maðr armligt, \hld\ hvęrr es þat \alst{s}á-at, &
\alst{h}vé þar ȧ \alst{H}ęrkju \hld\ \alst{h}ęndr sviðnuðu; &
\edtrans{lęiddu þȧ \alst{m}ęy \hld\ ï \alst{m}ýri fúla}{Then they led the maiden into the foul bog}{\Bfootnote{To be drowned, as was the customary Germanic punishment for perjurers; see note to \Voluspa\ 38.}}, &
\alst{s}vá þá Guðrún \hld\ \alst{s}ïnna harma.\eva

\bvb Man saw nothing pitiful if he did not see that, \\
how there on Herch the hands were scorched. \\
Then they led the maiden into the foul bog; \\
so was Guthrun restored for her affronts.\evb\evg

\sectionline
