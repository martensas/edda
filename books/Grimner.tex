\bookStart{Speeches of Grimner}[Grímnismǫ́l]

\begin{flushright}%
\textbf{Dating} \parencite{Sapp2022}: C10th (0.976)

\textbf{Meter:} \Ljodahattr, \Fornyrdislag\ (2/3–4, 28/3–5, 45/3–5, 48/4, 49/1–2, 53), \Galdralag\ (46)%
\end{flushright}

\section{Introduction}

The \textbf{Speeches of Grimner} (\Grimnismal) are preserved whole in both \Regius\ and \AM.

The poem itself is enclosed by prose passages.  It is hard to say for how long these have accompanied the poem, but since they are found in both \Regius\ and \AM\ they must go back to a now-lost archetypal manuscript.  Together with sts. 1–3 and 53–55 of the poem they form a narrative frame for the gnomic stanzas.  The gnomic sts. themselves, the bulk of the poem, are mythological and sometimes obscure.  They align closely with other Eddic gnomic poems like \Havamal, \Vafthrudnismal, \Sigrdrifumal, and \Allvismal.

Weden begins by listing the individual dwellings of the gods (4–17).  The locations are numbered, but a few facts speak to these numbers being a later insert:

\begin{enumerate}
  \item The alliteration is never reliant on the numbers; if one compares the numbered questions in \Vafthrudnismal\ 20–42 the difference is striking.
  \item The numbering is inconsistent; Thunder’s realm (st. 4) is not counted, and Wider’s land (st. 17) has no numeral (perhaps since the form of the stanza would not allow it.)
  \item In sts. 11–15 cited in \Gylfaginning, the numbers are missing.
\end{enumerate}

%This section has been discussed in detail by \textcite{deVries1952} TODO! who considers it corrupt. Specifically, he sees the second half of v. 4 as a later insert, since it does not elaborate on the “holy land” mentioned in the first half. \textcite{Jackson1995} argues convincingly against this, showing how the first half serves as a generalized introduction to the list; the holy land is the dwelling-places of the gods.)

After this list come several sts relating to Weden and his hall, Walhall (18–23). Mentioned are the preparation of food in Walhall (18), Weden’s wolves (19) and ravens (20), the river through which the dead have to wade (21) and the gate through which they have to pass (22), the count of doors in Walhall (23), the count of doors in Thunder’s hall Bilshirner (24), and two animals which stand on the hall and gnaw on the branches of the tree Leered (25–26). From the latter animal’s—the stag Oakthirner’s—horns droplets fall into Wharyelmer, which is the origin of all rivers (26).

This introduces a list of mythic rivers (27–28), ending with the waters through which Thunder must wade on his way to Ugdrassle (29). This leads to a list of the horses ridden by the other gods on their way to Ugdrassle (31) which is followed by a description of the roots of Ugdrassle (31), then its animals (32–36) the Walkirries (37), and beings associated with the sun and moon (38–40), the things created from Yimer’s body (41–42) with a digression on the significance of the \inx[P]{bloot} for men in the present (43, see note there!), the creation of the ship Shidebladner (44) and finally a list of the noblest of several categories of things and groups (45).

After these lists Weden utters an unclear st. invoking the gods (46), before listing many of his names and the circumstances in which they were used (47–50). He then turns to Garfrith, disappointed by the inhospitality and poor conduct of his former protégé, and predicts his imminent death (51–53). He finally reveals himself by his true name, daring Garfrith to face him (53). After this he repeats several of his names (54), and the poem ends.

In the final prose section we are told that Garfrith, after learning that he was torturing Weden, hurried up to take the god away from the fires, but tripped and fell on his sword and died. After this his son Ayner ruled for a long time.

\sectionline

\section{From the sons of king Reading (\emph{Frá sonum Hrauðungs konungs})}

\bpg\bpa\mssnote{\Regius~8v/31, \AM~3v/23}%
Hrauðungr konungr átti tvá sonu. Hét annarr Agnarr, enn annarr Geirrøðr.
Agnarr var tíu vetra enn Geirrøðr átta vetra. Þeir reru tveir á báti með dorgar sínar at smá-fiski.
Vindr rak þá í haf út. Í nátt-myrkri brutu þeir við land ok gingu upp; fundu kot-bónda einn.
Þar vǫ́ru þeir um vetrinn. Kerling fostraði Agnar, enn karl Geirrøð.
At vári fekk karl þeim skip. Enn er þau kerling leiddu þá til strandar, þá mę́lti karl ein-mę́li við Geirrøð.
Þeir fengu byr ok kvǫ́mu til stǫðva fǫður síns. Geirrøðr var fram í skipi.
Hann hljóp upp á land enn hratt út skipinu, ok mę́lti: „Far þú þar er smyl hafi þik.“
Skipit rak út. Enn Geirrøðr gekk út til bǿjar; hánum var vel fagnat; þá var faðir hans andaðr.
Var þá Geirrøðr til konungs tekinn, ok varð maðr ágę́tr.\epa

\bpb King Reading had two sons. One was called Ayner, and the other Garfrith.
Ayner was ten winters old, but Garfrith eight winters. The two were rowing in a boat with their trolling-lines for small fishing.
The wind drove them out into the sea. In the dark of night they crashed onto land and walked ashore; they found a lone cottage farmer.
There they stayed over the winter. The wife fostered Ayner, but the husband Garfrith.\footnoteB{The wife was Frie, and the husband Weden; this is clarified by the following prose. The motif of Weden preferring the youngest brother is also found in \Rigsthula.}
In the spring the husband gave them ships, but when he and his wife led them to the shore, the husband spoke privately with Garfrith.\footnoteB{Surely instructing him to push his brother out to sea.}
They caught good wind, and came to their father’s harbour. Garfrith was in the front of the ship.
He leapt onto land and pushed out the ship, and spoke: ”Go thou whither the fiends may have thee!”
The ship drove out. But Garfrith walked towards the farm; he was welcomed well; by then was his father ended.
Garfrith was then taken as king, and became an excellent man.\epb\epg


\bpg\bpa\mssnote{\Regius~9r/10, \AM~4r/3}%
Óðinn ok Frigg sátu í Hliðskjǫlfu ok sá um heima alla.
Óðinn mę́lti: „Sér þú Agnar fóstra þinn, hvar hann elr bǫrn við gýgi í hellinum?
En Geirrøðr, fóstri minn, er konungr ok sitr nú at landi.“
Frigg segir: „Hann er mat-níðingr sá at hann kvelr gesti sína ef hánum þykkja of-margir koma.“
Óðinn segir at þat er in mesta lygi. Þau veðja um þetta mál.
Frigg sendi eskis-mey sína, Fullu, til Geirrøðar. Hon bað konung varask at eigi fyr-gerði hánum fjǫl-kunnigr maðr sá er þar var kominn í land, ok sagði þat mark á at engi hundr var svá ólmr at á hann myndi hlaupa.
En þat var inn mesti hé-gómi at Geirrøðr vę́ri eigi mat-góðr ok þó lę́tr hann hand-taka þann mann er eigi vildu hundar á ráða.
Sá var í feldi blám ok nefndisk Grímnir ok sagði ekki fleira frá sér þótt hann vę́ri at spurðr.
Konungr lét hann pína til sagna ok setja milli elda tveggja ok sat hann þar átta nę́tr.
Geirrøðr konungr átti son tíu vetra gamlan ok hét Agnarr eptir bróður hans.
Agnarr gekk at Grímni ok gaf hánum horn fullt at drekka, sagði at konungr gerði illa er hann lét pína hann sak-lausan.
Grímnir drakk af. Þá var eldrinn svá kominn at feldrinn brann af Grímni. Hann kvað:\epa

\bpb Weden and Frie sat in the \inx[L]{Lithshelf} and looked over all the Homes.\footnoteB{Very similar to the Longbeard Origin Myth (TODO: reference and elaborate).}
Weden spoke: “Dost thou see Ayner, thy foster-son, where he begets children with a troll-woman in her cave?\footnoteB{This may relate to Frie’s role as love-goddess. Ayner is in any case to be understood as a weak, effeminate man.}
But Garfrith, \emph{my} foster-son, is king and now rules his land.”
Frie says: “He is such a meat-nithing that he torments his guests if he thinks too many are coming!”
Weden says that this is the greatest lie; they make a wager over this matter.
Frie sent her handmaid, Full, to Garfrith’s hall. She bade the king be wary, lest he be destroyed by the \inx[C]{many-cunning} man who had come to his land; and said that his mark was that no hound was so fierce that it would rush at him.
But it was the greatest falsehood that Garfrith was not \inx[C]{good of meat}; and yet he has that man bound whom the hounds would not touch.
He was in a blue cloak and called himself Grimner, and did not tell anything more about himself, even though he was asked.
The king had him tortured that he would speak, and set him between two fires; and he sat there for eight nights.
King Garfrith had a son ten winters old, and he was called Ayner after his brother.
Ayner went up to Grimner and gave him a full horn to drink, saying that the king did badly as he had him tortured without cause.
Grimner drank it up. Then the fire had grown so much that the cloak burned on Grimner. He quoth:\epb\epg\stepcounter{prosea}

\sectionline

\section{The Speeches of Grimner}

\bvg\bva\mssnote{\Regius~9r/27, \AM~4r/17}„\alst{H}ęitr est \alst{h}ripuðr \hld\ ok \alst{h}ęldr til mikill, &
\ind gǫngumk \alst{f}irr \alst{f}uni! &
\alst{L}oði sviðnar, \hld\ þótt á \alst{l}opt bera’k; &
\ind brinnumk \alst{f}eldr \alst{f}yrir.\eva

\bvb “Hot art thou, flame, and rather too great; \\
\ind go far from me, fire! \\
The wool-cape is singed though I hold it aloft; \\
\ind the cloak burns before me!\evb\evg


\bvg\bva\mssnote{\Regius~9r/29, \AM~4r/18}%
\alst{Á}tta nę́tr \hld\ sat’k milli \alst{ę}lda hér, &
\ind svá’t mér \alst{m}ann-gi \alst{m}at né bauð &
nema \alst{ęi}nn Agnarr, \hld\ es \alst{ęi}nn skal ráða, &
\alst{G}ęirrøðar sonr, \hld\ \alst{G}otna landi.\eva

\bvb For eight nights I sat between the fires here, \\
\ind while no man offered me food, \\
save for Ayner alone, who alone shall rule— \\
Garfrith’s son—the land of the Gots!\evb\evg


\bvg\bva\mssnote{\Regius~9r/31, \AM~4r/20}%
\alst{H}ęill skalt, Agnarr, \hld\ alls \alst{h}ęilan biðr &
\ind þik \alst{V}era-týr \alst{v}esa; &
\alst{ęi}ns drykkjar \hld\ skalt \alst{a}ldri-gi &
\ind \edtrans{bętri \alst{g}jǫld}{better recompense}{\Bfootnote{Namely the mythic lore which takes up sts. 4–53.}} \alst{g}eta:\eva

\bvb Hale shalt thou be, Ayner, for hale \\
\ind does Were-Tew \name{= Weden} bid thee be! \\
For a single drink shalt thou never get \\
\ind better recompense.\evb\evg

\sectionline

\bvg\bva\mssnote{\Regius~9r/33, \AM~4r/22}%
\alst{L}and es hęilagt, \hld\ es \alst{l}iggja sé’k &
\ind \alst{ǫ́}sum ok \alst{ǫ}lfum nę́r; &
en í \alst{Þ}rúð-hęimi \hld\ skal \alst{Þ}órr vesa &
\ind \edtrans{unds of \alst{r}júfask \alst{r}ęgin}{until the Reins are ripped}{\Bfootnote{i.e. until the \inx[L]{Rakes of the Reins}.  A formulaic expression; see note to \Baldrsdraumar\ 14 for further occurrences.}}.\eva

\bvb The land is holy which lying I see \\
\ind near the \inx[F]{Eese and Elves}, \\
but in Thrithham shall Thunder dwell \\
\ind until the Reins are ripped.\evb\evg


\bvg\bva\mssnote{\Regius~9v/2, \AM~4r/23}%
\alst{Ý}-dalir hęita, \hld\ þar’s \alst{U}llr hęfir &
\ind \alst{s}ér of gǫrva \alst{s}ali; &
\alst{A}lf-hęim Fręy \hld\ gǫ́fu í \alst{á}r-daga &
\ind \alst{t}ívar at \edtrans{\alst{t}ann-féi}{tooth-gift}{\Bfootnote{The gift the child receives when he sheds his first tooth.}}.\eva

\bvb Yewdales they are called where Woulder has \\
\ind made for himself a hall. \\
Elfham to Free in days of yore \\
\ind the Tews as a tooth-gift gave.\evb\evg


\bvg\bva\mssnote{\Regius~9v/3, \AM~4r/25}%
\alst{B}ǿr es sá (hinn þriði), \hld\ es \alst{b}líð ręgin &
\ind \alst{s}ilfri þǫkðu \alst{s}ali; &
\alst{V}ala-skjǫlf hęitir, \hld\ \edtrans{es \alst{v}élti sér}{won through wiles}{\Bfootnote{Several previous editors and translators (e.g. \textcite{FinnurEdda}, \textcite{PettitEdda}, \textcite{LarringtonEdda}) have rendered this phrase with variants of “craftily made for himself”, where the verb \emph{véla} would mean ‘craftily make’.  To my knowledge this sense is never otherwise attested, and its common meaning is ‘defraud, trick, betray’.  A simpler reading would be to see this as a reference to the myth of the Ettin-smith who built the wall of Osyard.  The Gods had promised him Sun, Moon, and Frow, if he could build it in a year, but employed various tricks to hinder him.  When it at last looked like he would make it in time, Thunder slew him.  This myth is told in \Gylfaginning\ 42 and alluded to in \Voluspa\ 24–25.}} &
\ind \alst{ǫ́}ss í \alst{á}r-daga.\eva

\bvb Bower is (the third) one, where the blithe Reins \\
\ind with silver thatched a hall. \\
Waleshelf is it called which he won through wiles, \\
\ind the Os in days of yore.\evb\evg


\bvg\bva\mssnote{\Regius~9v/5, \AM~4r/26}%
\alst{S}økkva-bękkr hęitir (hinn fjórði), \hld\ en þar \alst{s}valar knegu &
\ind \alst{u}nnir glymja \alst{y}fir; &
þar þau \alst{Ó}ðinn ok Sága \hld\ drekka umb \alst{a}lla daga &
\ind \alst{g}lǫð ór \alst{g}ullnum kęrum.\eva

\bvb Sinkbench is (the fourth) one called, and there do cool \\
\ind waves clash over above; \\
there Weden and Sey drink all days, \\
\ind glad, out of golden casks.\evb\evg


\bvg\bva\mssnote{\Regius~9v/7, \AM~4r/28}%
\alst{G}laðs-hęimr hęitir (hinn fimti) \hld\ þar’s hin \alst{g}ull-bjarta &
\ind \alst{V}al-hǫll \alst{v}íð of þrumir; &
en þar \alst{H}roptr \hld\ kýss \alst{h}vęrjan dag &
\ind \alst{v}ápn-dauða \alst{v}era.\eva

\bvb Gladsham is (the fifth) one called, where the gold-bright \\
\ind Walhall wide stands fast; \\
and there Roft \name{= Weden} chooses every day \\
\ind weapon-dead warriors.\footnoteB{Cf. st. 14.}\evb\evg

\sectionline

In \AM\ the order of the following two sts. is reversed.

\sectionline

\bvg\bva\mssnote{\Regius~9v/9, \AM~4r/31}%
Mjǫk ’s \alst{au}ð-kęnnt \hld\ þęim’s til \alst{Ó}ðins koma &
\ind \edtext{\alst{s}al-kynni at \alst{s}éa}{\Afootnote{\emph{‘sia at sia’} \AM}}, &
\alst{v}argr hangir \hld\ fyr \alst{v}estan dyrr &
\ind ok drúpir \alst{ǫ}rn \alst{y}fir.\eva

\bvb Very easily recognized, for those who come to Weden, \\
\ind is the hall to see: \\
A wolf hangs before the western door, \\
\ind and an eagle droops over.\footnoteB{Something very similar is found in Widukind’s History of the Saxons, book 1:12.  The Saxons have just conquered a fortress, and \emph{mane [...] facto ad orientalem portam ponunt aquilam, aramque victoriae construentes secundum errorem paternum sacra sua propria veneratione venerati sunt} ‘at the coming of morning they set an eagle at the eastern gate, and, building an altar of victory, they worshipped it with their own holy worship in accordance with their ancestral error.’  The altar was pledged to \inx[P]{Ermin}, whom the author identifies with Mars or Hermes, but who is surely Weden.

According to \textcite{HyltenCavallius1863}[156] it was custom in Wärend, southern Sweden to hang the bodies of killed wolves high up in old oaks, and killed birds of prey above the stable-door.}\evb\evg%TODO: bibliography for both works


\bvg\bva\mssnote{\Regius~9v/10, \AM~4r/30}%
Mjǫk ’s \alst{au}ð-kęnnt \hld\ þęim’s til \alst{Ó}ðins koma &
\ind \alst{s}al-kynni at \alst{s}éa, &
\edtrans{\alst{sk}ǫptum}{shafts}{\Bfootnote{Spear-shafts.}} ’s rann rępt, \hld\ \alst{sk}jǫldum ’s salr þakiðr, &
\ind \alst{b}rynjum of \alst{b}ękki stráat.\eva

\bvb Very easily recognized, for those who come to Weden, \\
\ind is the hall to see: \\
With shafts is the house roofed, with shields is the hall thatched; \\
\ind with byrnies the benches strewn.\evb\evg


\bvg%TODO: All variants are not yet noted.
\bva\mssnote{\Regius~9v/12, \AM~4v/2, \GylfMS}%
\alst{Þ}rym-hęimr hęitir \edtrans{(hinn sétti)}{the sixth}{\Afootnote{om. \GylfMS}}, \hld\ \edtrans{es}{where}{\Afootnote{\emph{þar nú} ‘where now’}} \alst{Þ}jatsi \edtrans{bjó}{dwelled}{\Afootnote{om. \Wormianus; \emph{býr} ‘dwells’ \Upsaliensis}}, &
\ind sá hinn \edtrans{\edtext{\alst{á}m-átki}{\Afootnote{\emph{mátki} \Upsaliensis}} \alst{jǫ}tunn}{uncanny ettin}{\Bfootnote{Formulaic. See note to \Voluspa\ 8.}}; &
en nú \alst{Sk}aði byggvir, \hld\ \alst{sk}ír brúðr \edtrans{goða}{of the Gods}{\Afootnote{\emph{guma} ‘of men’ \Upsaliensis}}, &
\ind \alst{f}ornar toptir \alst{f}ǫður.\eva

\bvb Thrimham is (the sixth) one called, where Thedse dwelled, \\
\ind that uncanny ettin; \\
but now Shede bedwells—the pure bride of the Gods— \\
\ind the ancient plots of her father.\evb\evg


\bvg\bva\mssnote{\Regius~9v/14, \AM~4v/3, \GylfMS}%
\alst{B}ręiða-\alst{b}lik \edtrans{eru (hin sjaundu)}{are (the seventh)}{\Bfootnote{\emph{hęita} ‘[they] are called’ \GylfMS.}}, \hld\ en þar \alst{B}aldr hęfir &
\ind \alst{s}ér of gǫrva \alst{s}ali, &
á því \alst{l}andi \hld\ es \alst{l}iggja vęit’k &
\ind \alst{f}ę́sta \edtrans{\alst{f}ęikn-stafi}{wicked deeds}{\Bfootnote{Lit. ‘staves of wickedness’, where ‘stave’ originally means something like ‘word, speech’.  Cf. \Beowulf\ 1018b: \emph{fâcen-stafas}, referring to treacherous intrigues among the \inx[G]{Shieldings}.}}.\eva

\bvb Broadblicks are (the seventh), and there Balder has \\
\ind made for himself a hall, \\
on that land where I know lying \\
\ind the fewest wicked deeds.\evb\evg


\bvg\bva\mssnote{\Regius~9v/16, \AM~4v/5, \GylfMS}%
\alst{H}imin-bjǫrg \edtrans{eru (hin ǫ́ttu)}{are (the eighth)}{\Bfootnote{\emph{hęita} ‘[they] are called’ \GylfMS.}}, \hld\ en þar \alst{H}ęim-dall &
\ind kveða \alst{v}alda \alst{v}éum; &
þar \edtrans{\alst{v}ǫrðr goða}{Watchman of the Gods}{\Bfootnote{Formulaic epithet of Homedal, also occurring in \Lokasenna\ 49 and possibly in \Skirnismal\ 28: \emph{vǫrðr með goðum} ‘the Watchman among the Gods’.  \Gylfaginning\ 27, where the present stanza is cited, gives some further details: \emph{Hann býr þar er heitir Himinbjǫrg við Bifrǫst. Hann er vǫrðr goða ok sitr þar við himins enda at gę́ta brúarinnar fyrir berg-risum. Hann þarf minna svefn en fugl. Hann sér jafnt nótt sem dag hundrað rasta frá sér; hann heyrir ok þat, er gras vex á jǫrðu eða ull á sauðum, ok allt þat er hę́ra lę́tr.} ‘He lives at the place called the Heavenbarrows near Bivrest. He \ken*{= Homedal} is the Watchman of the Gods and sits there at Heaven’s end to guard the bridge against barrow-risers.  He needs less sleep than a bird.  Both night and day he sees a hundred rests away from him; he also hear when grass grows on the ground or wool on sheep, and everything which sounds louder.’}} \hld\ drekkr í \alst{v}ę́ru ranni &
\ind \alst{g}laðr \edtext{hinn}{\Afootnote{so \AM\GylfMS; om. \Regius}} \alst{g}óða mjǫð.\eva

\bvb Heavenbarrows are (the eighth), and there Homedal, \\
\ind they say, wields over wighs. \\
There the Watchman of the Gods \ken*{= Homedal} drinks in the tranquil house, \\
\ind glad, the good mead.\evb\evg


\bvg\bva\mssnote{\Regius~9v/17, \AM~4v/6, \GylfMS}%
\alst{F}olk-vangr \edtrans{es (hinn níundi)}{is (the ninth)}{\Bfootnote{\emph{hęitir} ‘[one] is called’ \GylfMS}}, \hld\ en þar \alst{F}ręyja rę́ðr &
\ind \alst{s}essa kostum í \alst{s}al; &
\alst{h}alfan val \hld\ hon kýss \alst{h}vęrjan dag, &
\ind en halfan \alst{Ó}ðinn \alst{á}.\eva

\bvb Folkwong is (the ninth), and there Frow decides \\
\ind the choice of seats in the hall; \\
half the slain she chooses each day, \\
\ind but half does Weden own.\footnoteB{This st. is cited and closely paraphrased in \Gylfaginning\ 24. — The roots of \emph{kjósa val} ‘choose the slain’ are the same as those in \inx[C]{walkirrie} (\emph{val-kyrja} ‘chooser of the slain’), and as Frow is a prominent goddess this would surely make her the chief walkirrie.
This is paralleled by \Sorlathattr, where Frow assumes the name \inx[C]{Gandle} (\emph{Gǫndul}, a name attested in several lists of walkirries; see \Voluspa\ 30 and Notes) and incites the legendary never-ending Conflict of the Headnings (\emph{Hjaðningavíg}).
In spite of this parallel, there are good reasons to believe that the chief walkirrie was \inx[C]{Frie}, Weden’s wife.
First, one of the functions of the walkirries is to bear ale to the Oneharriers (\Grimnismal\ 37). This mirrors royal Germanic banquets attested in heroic poetry, where the host’s wife or daughter would pour ale to his retainers and guests (the so-called ‘lady with a mead cup’ ritual; see \textcite{Enright1996} and \textcite{Riseley2014}). As Weden’s wife, we would expect Frie to have this role.
Second, at Balder’s funeral as attested in \Gylfaginning\ (TODO. chapter number), Weden rides with Frie and the Walkirries, while Frow rides alone with her cats. If she were chief walkirrie, it is rather strange that she should not ride with them.
Third, there are two separate myths where Frie and Weden contend over the fates of armies and men. These are the prose introduction to the present poem and the Longbeard origin myth (for which see Introduction to the present poem).}\evb\evg


\bvg\bva\mssnote{\Regius~9v/19, \AM~4v/8, \GylfMS}%
\alst{G}litnir \edtrans{es (hinn tíundi)}{is (the tenth)}{\Bfootnote{\emph{hęitir salr} ‘a hall is called’ \GylfMS}}, \hld\ hann ’s \alst{g}ulli studdr &
\ind ok \alst{s}ilfri þakðr it \alst{s}ama; &
en þar \alst{F}or-seti \hld\ byggir \alst{f}lęstan dag &
\ind ok \alst{s}vę́fir allar \alst{s}akir.\eva

\bvb Glitner is (the tenth): it is supported by gold, \\
\ind and thatched with silver likewise. \\
And there Foresitter dwells for most of the day, \\
\ind and puts all disputes to sleep.\evb\evg


\bvg\bva\mssnote{\Regius~9v/21, \AM~4v/9}%
\alst{N}óa-tún eru (hin ęlliptu), \hld\ en þar \alst{N}jǫrðr hęfir &
\ind \alst{s}ér of gǫrva \alst{s}ali; &
\edtrans{\alst{m}anna þęngill \hld\ hinn \alst{m}ęins-vani}{The lord of men, the guileless one}{\Bfootnote{Interesting epithets probably relating to Nearth’s roles in upholding the bounty of the land and the law.  Cf. my article on pre-Christian oaths (TODO).}} &
\ind \edtrans{\alst{h}ǫ́-timbruðum \alst{h}ǫrgi rę́ðr}{rules the harrow timbered on high}{\Bfootnote{The rare verb \emph{hǫ́-timbra} ‘timber on high’ otherwise only occurs in \Voluspa\ 7, likewise in connection with the \emph{hǫrgr} ‘harrow’.  The harrow is an outdoors holy place; see Index.  Cf. also \Vafthrudnismal\ 38 where Nearth is said to rule a great many hoves and harrows.}}.\eva

\bvb Nowetowns are (the eleventh), and there Nearth has \\
\ind made for himself a hall. \\
The lord of men, the guileless one, \\
\ind rules the \inx[C]{harrow} timbered on high.\evb\evg


\bvg\bva\mssnote{\Regius~9v/23, \AM~4v/11}%
\edtrans{\alst{H}rísi vęx \hld\ ok \alst{h}ǫ́u grasi}{with brushwood grows, and with tall grass,}{\Bfootnote{Identical to \Havamal\ 119/6.}} &
\ind \alst{V}íðars land, \alst{v}iði, &
en þar \alst{m}ǫgr of lę́tsk \hld\ af \alst{m}ars baki &
\ind \alst{f}rǿkn at hęfna \alst{f}ǫður.\eva

\bvb With brushwood grows, and with tall grass, \\
\ind \inx[P]{Wider}’s land, with wood, \\
and there the lad vows from the back of his steed, \\
\ind brave, to avenge his father.\footnoteB{At the Rakes of the Reins Wider avenges His father, Weden.  See \Voluspa\ 54–55, \Vafthrudnismal\ 53.}\evb\evg


\bvg\bva\mssnote{\Regius~9v/24, \AM~4v/12, \GylfMS}%
\alst{A}nd-hrímnir \hld\ lę́tr í \alst{Ę}ld-hrímni &
\ind \alst{S}ę́-hrímni \alst{s}oðinn, &
\alst{f}lęska bętst, \hld\ en þat \alst{f}áir vitu, &
\ind við hvat \alst{ę}in-hęrjar \alst{a}lask.\eva

\bvb Andrimner lets Sowrimner \\
\ind in Eldrimner be boiled. \\
The best of meats, but few know this: \\
\ind by what the \inx[G]{Oneharriers} are nourished.\footnoteB{The cook Andrimner ‘face-sooty’ cooks the boar Sowrimner ‘sow-sooty’ in the cauldron Eldrimner ‘fire-sooty’; by this meat are the Oneharriers nouished.}\evb\evg


\bvg\bva\mssnote{\Regius~9v/26, \AM~4v/14, \GylfMS}%
\edtext{\alst{G}era ok Freka \hld\ sęðr \alst{g}unn-tamiðr, &
\ind \alst{h}róðigr \alst{H}ęrjafǫðr, &
en við \alst{v}ín ęitt \hld\ \alst{v}ápn-gǫfugr &
\ind \alst{Ó}ðinn \alst{ę́} lifir.}{\lemma{Gera \dots\ lifir ‘Gar \dots\ live’}\Bfootnote{With what Weden feeds his two hounds it is not said, but it is most likely with the corpses of dead warriors.  The wine on which he subsists may perhaps be identified with drink offerings.  Cf. the 7th century \emph{vita} of Saint Columban (TODO: cite source), describing a rite of the Swabians: \emph{Quo cum moraretur, et inter habitatores loci illius progrederetur, reperit eos sacrificium profanum litare velle, vasque magnum, quod vulgo cupam vocant, quod viginti et sex modios amplius minusve capiebat, cervisia plenum in medio habebant positum. Ad quod vir Dei accessit, et sciscitatur quid de illo fieri vellent. Illi aiunt Deo suo Vodano, quem Mercurium vocant alii, se velle litare.} ‘While he was satying there and going about the dwellers of that place, he found out that they were going to offer a profane sacrifice, and a large cask called a \emph{cupa}, which held about twenty-six measures, was filled with beer and set in their midst.  When the man of God asked what they wanted to do with it, they answered that they were wanted to offer to their God Wodan, whom others call Mercury.’}}\eva

\bvb \inx[P]{Gar and Freak} does the battle-accustomed \\
\ind glorious Father of Hosts \name{= Weden} feed; \\
but on wine alone, esteemed of weapons, \\
\ind Weden ever lives.\evb\evg


\bvg\bva\mssnote{\Regius~9v/28, \AM~4v/15, \GylfMS}%
\alst{H}uginn ok Muninn \hld\ fljúga \alst{h}vęrjan dag &
\ind \edtrans{\alst{jǫ}rmun-grund}{ermin-ground}{\Bfootnote{i.e. ‘the immense ground’ (for the rare prefix \inx[C]{ermin-} see Index), denoting the earth as a vast flat expanse of land. This compound also occurs in a kenning in the st. on the late C10th Karlevi stone (Öl 1) referring to the unbounded sea as \emph{Ęndils jǫrmungrund} ‘Andle’s ermin-ground’ (Andle being a known “sea-king”), and in \Beowulf\ 859 as \emph{eormen-grund} carrying the same sense.}} \alst{y}fir; &
\alst{ó}umk of Hugin, \hld\ at \alst{a}ptr né komi-t; &
\ind þó séumk \alst{m}ęir of \alst{M}unin.\eva

\bvb Highen and Minden fly every day \\
\ind over the ermin-ground \ken{earth}. \\
I worry for Highen, that he might not come back, \\
\ind yet I fear more for Minden.\evb\evg


\bvg\bva\mssnote{\Regius~9v/30, \AM~4v/17}%
\alst{Þ}ýtr \alst{Þ}und, \hld\ unir \edtext{\alst{Þ}jóð-vitnis &
\ind fiskr}{\lemma{Þjóðvitnis fiskr ‘Thedwitner’s fish’}\Bfootnote{\emph{Þjóðvitnir} is easily analyzed as \emph{þjóð-} ‘great, main’ + \emph{vitnir} ‘wolf’.  The great wolf is naturally the \inx[P]{Fenrerswolf}, the brother of the Middenyardswyrm.  That the Wyrm can be called a fish is shown by \Hymiskvida\ 24.}} flóði í; &
\alst{á}ar-straumr \hld\ þykkir \alst{o}f-mikill &
\ind \alst{v}al-glaumi at \alst{v}aða.\eva

\bvb \inx[P]{Thound} roars; Thedwitner’s fish \\
\ind thrives in the flood. \\
The river-stream seems far too great \\
\ind for the noisy slain host to wade.\footnoteB{A difficult stanza.  Thound may be the river surrounding Walhall, which the dead have to pass over to reach it.  The stanza may also be referring to the punishment of criminals in waters; see note to \Voluspa\ 38 for discussion on that.}\evb\evg


\bvg\bva\mssnote{\Regius~9v/32, \AM~4v/18}\edtrans{\alst{V}al-grind}{Walgrind}{\Bfootnote{‘Slain-gate;’ the gate standing before Walhall.}} hęitir \hld\ es stęndr \alst{v}ęlli á &
\ind \alst{h}ęilǫg fyr \alst{h}ęlgum durum; &
\alst{f}orn ’s sú grind, \hld\ en þat \alst{f}áir vitu, &
\ind hvé hǫ́n ’s í \alst{l}ás of \alst{l}okin.\eva

\bvb \inx[L]{Walgrind} ’tis called, which stands on the plain, \\
\ind holy, before the holy doors. \\
Old is that gate, but few know this: \\
\ind how its lock is locked.\evb\evg


\bvg\bva\mssnote{\Regius~9v/34, \AM~4v/22}%
\alst{F}imm hundruð golfa \hld\ ok umb \alst{f}jórum tøgum &
\ind svá hygg’k \alst{B}il-skirni með \alst{b}ugum; &
\alst{r}anna þęira, \hld\ es \alst{r}ępt vita’k, &
\ind \alst{m}íns vęit’k męst \alst{m}agar.\eva

\bvb With five hundred floors, and around fourty, \\
\ind so I judge \inx[L]{Bilshirner} altogether. \\
Of those houses which I might know rafted \\
\ind I know my lad’s \ken*{= Thunder} to be the greatest.\evb\evg


\bvg\bva\mssnote{\Regius~10r/2, \AM~4v/20}%
\alst{F}imm hundruð dura \hld\ ok umb \alst{f}jórum tøgum, &
\ind svá hygg at \alst{V}alhǫllu \alst{v}esa; &
\edtrans{\alst{á}tta hundruð}{eight hundred}{\Bfootnote{The hundred is probably here the long hundred (120, rather than 100), which gives a sum of \(640 * 960 = 614~400\) Oneharriers.}} \alst{Ę}in-hęrja \hld\ ganga ór \alst{ęi}num durum, &
\ind þá’s fara við \alst{v}itni at \alst{v}ega.\eva

\bvb Five hundred doors, and around fourty, \\
\ind so I judge there to be on Walhall. \\
Eight hundred \inx[G]{Oneharriers} go out of one door, \\
\ind when to fight with the wolf they go.\evb\evg


\bvg\bva\mssnote{\Regius~10r/4, \AM~4v/24}%
\alst{H}ęið-rún hęitir gęit, \hld\ es stęndr \edtrans{\alst{h}ǫllu á Hęrja-fǫðrs}{on the hall of the Father of Hosts}{\Bfootnote{The hall of Weden, i.e. Walhall.  \emph{Hęrja-fǫðrs} looks like an unmetrical addition.}} &
\ind ok bítr af \alst{L}ę́-raðs \alst{l}imum; &
\edtrans{\alst{sk}ap-kęr}{shape-vats}{\Bfootnote{According to \CV\ the central beer-vat, from which drinks were poured into smaller vessels.}} fylla \hld\ skal \edtrans{hins \alst{sk}íra mjaðar}{the pure mead}{\Bfootnote{The mead is the goat’s milk.}}, &
\ind kná-at sú \alst{v}ęig \alst{v}anask.\eva

\bvb Heathrune is the goat called which stands on the hall of the Father of Hosts, \\
\ind and bites off Leered’s branches. \\
The shape-vats shall she fill with the pure mead; \\
\ind those draughts cannot wane.\evb\evg


\bvg\bva\mssnote{\Regius~10r/6, \AM~4v/26}Ęik-þyrnir hęitir \alst{h}jǫrtr \hld\ es stęndr \alst{h}ǫllu á Hęrja-fǫðrs &
\ind ok bítr af \alst{L}ę́-raðs \alst{l}imum; &
en af hans \alst{h}ornum \hld\ drýpr í \alst{H}ver-gęlmi &
\ind þaðan ęiga \alst{v}ǫtn ǫll \alst{v}ega:\eva

\bvb Oakthirner is called the stag who stands on the hall of the Father of Hosts, \\
\ind and bites off Leered’s branches. \\
And from his horns [drops] drip into Wharyelmer; \\
\ind thence have all waters their ways:\evb\evg


\bvg\bva\mssnote{\Regius~10r/9, \AM~4v/28}%
\alst{S}íð ok Víð, \alst{S}ę́kin ok Ęikin, \hld\ \alst{S}vǫl ok Gunn-þró, &
\ind \alst{F}jǫrm ok \alst{F}imbul-þul, &
\ind \alst{R}ín ok \alst{R}innandi, &
\alst{G}ipul ok \alst{G}ǫpul, \hld\ \alst{G}ǫmul ok \alst{G}ęir-vimul, &
\ind þę́r \alst{h}verfa umb \alst{h}odd goða, &
\alst{Þ}yn ok Vin, \hld\ \alst{Þ}ǫll ok Hǫll, &
\ind \alst{G}rǫ́ð ok \alst{G}unn-þorin.\eva

\bvb Side and Wide, Seeken and Oaken, Swale and Guththrew, \\
\ind Ferm and Fimblethule, \\
\ind Rine and Rinnend, \\
Gipple, Gapple, Gamble and Garwimble— \\
\ind they run around the hoard of the Gods \ken*{= Osyard}— \\
Thin and Win, Thall and Hall, \\
\ind Gread and Guththorn.\evb\evg


\bvg\bva\mssnote{\Regius~10r/12, \AM~5r/1}%
\alst{V}ína hęitir enn, \hld\ ǫnnur \alst{V}eg-svinn, &
\ind \alst{þ}riðja \alst{Þ}jóð-numa; &
\alst{N}yt ok \alst{N}ǫt, \hld\ \alst{N}ǫnn ok \alst{H}rǫnn, &
\alst{S}líð ok \alst{H}ríð, \hld\ \alst{S}ylgr ok Ylgr, &
\alst{V}íð ok \alst{V}ǫ́n, \hld\ \alst{V}ǫnd ok Strǫnd, &
\alst{G}jǫll ok Lęiptr; \hld\ þę́r falla \alst{g}umnum nę́r &
\ind es falla til \alst{h}ęljar \alst{h}eðan. \eva

\bvb Wine is one further called, another Wayswith, \\
\ind a third Thedenumb; \\
Nit and Nat, Nan and Ran, \\
Slithe and Rithe, Sellow and Wellow, \\
Wide and Ween, Wand and Strand, \\
Yell and Laft—they fall near to men \\
\ind as they fall hence to Hell.\evb\evg


\bvg\bva\mssnote{\Regius~10r/15, \AM~5r/4, \GylfMS}%
\alst{K}ǫrmt ok Ǫrmt \hld\ ok \alst{k}ęr-laugar tvę́r &
\ind \edtrans{\alst{þ}ę́r skal \alst{Þ}órr vaða}{these shall Thunder wade}{\Bfootnote{For Thunder’s association with wading see TODO.}} &
\alst{d}ag hvęrn \hld\ es \alst{d}ǿma fęrr &
\ind at \alst{a}ski \alst{Y}gg-drasils; &
því-at \alst{ǫ́}s-brú \hld\ bręnn \alst{ǫ}ll loga &
\ind \alst{h}ęilǫg vǫtn \edtrans{\alst{h}lóa}{bellow}{\Bfootnote{A hapax. TODO.}}.\eva

\bvb Carmt and Armt, and the two Carlays, \\
\ind these shall Thunder wade \\
every day, when to judge he goes, \\
\ind at \inx[L]{Ugdrassle’s Ash}; \\
for the \inx[G]{eese}[os]-bridge \ken{rainbow} burns all with flame; \\
\ind the holy waters bellow.\evb\evg


\bvg\bva\mssnote{\Regius~10r/17, \AM~5r/6}%
\alst{G}laðr ok \alst{G}yllir, \hld\ \alst{G}lęr ok Skęið-brimir, &
\ind \alst{S}ilfrin-toppr ok \alst{S}inir, &
\alst{G}ísl ok Fal-hófnir, \hld\ \alst{G}ull-toppr ok Létt-feti, &
\ind þęim ríða \alst{ę́}sir \alst{jó}um &
\alst{d}ag hvęrn \hld\ es \alst{d}ǿma fara &
\ind at \alst{a}ski \alst{Y}gg-drasils.\eva

\bvb Glad and Gilder, Glare and Sheathbrimmer, \\
\ind Silvrentop and Sinewer; \\
Yissel and Fallowhofner, Goldtop and Lightfeet; \\
\ind on these horses ride the Eese, \\
every day, when to judge they go, \\
\ind at \inx[L]{Ugdrassle’s Ash}.\evb\evg


\bvg\bva\mssnote{\Regius~10r/20, \AM~5r/8}%
\alst{Þ}ríar rǿtr \hld\ standa á \alst{þ}ría vega &
\ind undan \alst{a}ski \alst{Y}gg-drasils; &
\alst{H}ęl býr und \alst{ęi}nni, \hld\ \alst{a}nnarri \alst{h}rím-þursar, &
\ind þriðju \alst{m}ęnnskir \alst{m}ęnn.\eva

\bvb Three roots grow on three ways, \\
\ind from beneath Ugdrassle’s Ash. \\
Hell lives enclosed by one, [by] the other the \inx[G]{Rime-Thurses}, \\
\ind {[by]} the third manly men.\evb\evg


\bvg\bva\mssnote{\Regius~10r/22, \AM~5r/9}%
\alst{R}ata-toskr hęitir íkorni \hld\ es \alst{r}inna skal &
\ind at \alst{a}ski \alst{Y}gg-drasils; &
\alst{a}rnar \alst{o}rð \hld\ hann skal \alst{o}fan bera &
\ind ok sęgja \alst{N}íð-hǫggvi \alst{n}iðr.\eva

\bvb Wratetusk is the squirrel called who shall run \\
\ind at Ugdrassle’s Ash. \\
The eagle’s words he shall carry from above, \\
\ind and say to Nithehewer below.\footnoteB{This st. and the following is paraphrased in \Gylfaginning\ 16 (excerpt):
\begin{quote}
  \emph{Þá mę́lti Gangleri: „Hvat er fleira at segja stór-merkja frá askinum?“ Hár segir: „Mart er þar af at segia. Ǫrn einn sitr í limum asksins, ok er hann margs vitandi, en í milli augna honum sitr haukr sá, er heitir Veðrfǫlnir. Íkorni sá, er heitir Rata-toskr, rennr upp ok niðr eptir askinum ok berr ǫfundar orð millum arnarins ok Níðhǫggs.} ‘Gangler spoke: “What more great marks are there to be said about the ash?” High says: “There is much to say about it. An eagle sits in the limbs of the ash, and he is much knowing, but between his eyes sits the hawk called Weatherfalner. The squirrel, which is called Wratetush, runs up and down along the ash and carries words of spite between the eagle and Nithehewer.”’
\end{quote}}\evb\evg


\bvg\bva\mssnote{\Regius~10r/23, \AM~5r/11}%
\alst{H}irtir ’ru ok fjórir \hld\ þęir’s af \alst{h}ę́fingar &
\ind á \alst{g}ag-halsir \alst{g}naga: &
\alst{D}áinn ok \alst{D}valinn, \hld\ \alst{D}ún-ęyrr ok \alst{D}ura-þrór.\eva

\bvb Harts are there also, four, those who TODO \\
\ind TODO gnaw: \\
Dowen and Dwollen, Downeer and Doorthrew.\footnoteB{Paraphrased in \Gylfaginning\ 16 immediately following a paraphrase of the last st.: \emph{En fjórir hirtir renna í limum asksins ok bíta barr; þeir heita svá: Dáinn, Dvalinn, Dún-eyrr, Dura-þrór.} ‘But four harts run in the limbs of the ash and bite its leaves; they are called thus: Dowen, Dwollen, Downeer, Doorthrew.’}\evb\evg


\bvg\bva\mssnote{\Regius~10r/25, \AM~5r/12, \GylfMS}%
\alst{O}rmar flęiri \hld\ liggja und \alst{a}ski \alst{Y}gg-drasils &
\ind an þat of \alst{h}yggi \alst{h}vęrr &
\ind \alst{ó}-sviðra \alst{a}pa:\eva

\bvb More worms lie under Ugdrassle’s Ash \\
\ind than any one would think \\
\ind among unwise \inx[C]{ape}[apes]:\footnoteB{Paraphrased in \Gylfaginning\ 16: \emph{En svá margir ormar eru í Hvergelmi með Níðhǫgg, at engi tunga má telja; svá segir hér:} ‘But so many worms are in Wharyelmer with Nithehewer that no tongue may count them. So it says here:’ after which st. 36 is quoted.}\evb\evg


\bvg\bva\mssnote{\Regius~10r/26, \AM~5r/13, \GylfMS}%
\alst{G}óinn ok Móinn, \hld\ þęir ’ru \alst{G}raf-vitnis synir, &
\ind \alst{G}rá-bakr ok \alst{G}raf-vǫlluðr, &
\alst{O}fnir ok Sváfnir, \hld\ hygg’k at \alst{ę́} skyli &
\ind \alst{m}ęiðs kvistu \alst{m}áa.\eva

\bvb Gowen and Mowen—they are Gravewitner’s sons— \\
\ind Greyback and Gravewalled; \\
Ovner and Sweefner, I ween, shall always \\
\ind injure the beam’s branches.\evb\evg


\bvg\bva\mssnote{\Regius~10r/28, \AM~5r/14}%
\alst{A}skr \alst{Y}gg-drasils \hld\ drýgir \alst{ę}rfiði &
\ind \alst{m}ęira an \alst{m}ęnn viti: &
\alst{h}jǫrtr bítr ofan \hld\ en á \alst{h}liðu fúnar, &
\ind skęrðir \alst{N}íð-hǫggr \alst{n}eðan.\eva

\bvb Ugdrassle’s Ash suffers hardship \\
\ind greater than men might know: \\
a hart bites it above and it rots on the side; \\
\ind Nithehewer harms it below.\evb\evg


\bvg\bva\mssnote{\Regius~10r/30, \AM~5r/16}%
\alst{H}rist ok Mist \hld\ vil’k at mér \alst{h}orn beri, &
\ind \alst{Sk}eggj-ǫld ok \alst{Sk}ǫgul, &
\edtrans{\alst{H}ildr ok Þrúðr}{Hild and Thrith}{\Afootnote{so \AM; \emph{Hildi ok Þrúði} \Regius\ stems from \emph{ꝺꝛ, ðꝛ} with r rotunda being interpreted and copied as \emph{ꝺı, ðr}, this becomes clear upon viewing the facsimile images.}}, \hld\ \alst{H}lǫkk ok \alst{H}ęr-fjǫtur, &
\ind \alst{G}ǫll ok \alst{G}ęir-ǫlul, &
\alst{R}and-gríð ok \alst{R}áð-gríð, \hld\ \alst{R}ęgin-lęif; &
\ind þę́r bera \alst{ęi}n-hęrjum \alst{ǫ}l.\eva

\bvb Rist and Mist I would have bearing to me a horn— \\
\ind Shageld and Shagle; \\
Hild and Thrith, Lank and Harfetter, \\
\ind Gall and Garannel, \\
Randgrith and Redegrith, Rainlaf— \\
\ind they bear the Oneharriers ale.\footnoteB{The women listed in this st. are Walkirries. Their names are known from other lists of Walkirries, but differ somewhat in form. TODO: Note these differences}\evb\evg


\bvg\bva\mssnote{\Regius~10r/32, \AM~5r/18}%
\edtrans{\alst{Á}r-vakr ok \alst{A}l-sviðr}{Yorewaker and Allswith}{\Bfootnote{These horses also appear in \Sigrdrifumal\ 15a/2; see note to the next st.}}, \hld\ skulu \alst{u}pp heðan &
\ind \alst{s}vangir \alst{s}ól draga; &
en und þęira \alst{b}ógum \hld\ fǫ́lu \alst{b}líð ręgin, &
\ind \alst{ę́}sir, \alst{í}sarn-kol.\eva

\bvb Yorewaker and Allswith shall from hence— \\
\ind slender [steeds]—pull up the sun, \\
and under their shoulders the blithe Reins hid \\
\ind —the Eese—iron-cooling.\footnoteB{According to \Gylfaginning\ 11 the gods took two horses to pull the sun’s chariot—Yorewaker and Allswith—and “under the shoulders of the horses the gods placed two wind-bellows to cool them, but in some sources (\emph{í sumum frǿðum}, presumably this st.) they are called iron-cooling (\emph{ísarn-kol}).”}\evb\evg


\bvg\bva\mssnote{\Regius~10v/2, \AM~5r/20}%
\alst{S}valinn hęitir, \hld\ hann stęndr \alst{s}ólu fyrir, &
\ind \alst{sk}jǫldr \alst{sk}ínanda goði; &
\alst{b}jǫrg ok \alst{b}rim \hld\ vęit’k at \alst{b}rinna skulu, &
\ind ef hann \alst{f}ęllr í \alst{f}rá.\eva

\bvb Swalen one is called, it stands before the sun: \\
\ind a shield [before] the shining god \ken{sun}. \\
Crags and surf I know shall burn, \\
\ind if it falls away.\footnoteB{The sun-disc was apparently thought to be a translucent shield, which protected the earth from the full power of the Sun behind it. Without it the whole world (“crags and surf”, \textsc{land} and \textsc{sea}; the totality of the earth) would burn up.  Cf. \Sigrdrifumal\ 15a/1, which mentions the “shield that stands before the shining god \ken{sun}”.}\evb\evg


\bvg\bva\mssnote{\Regius~10v/4, \AM~5r/21}%
\alst{Sk}oll hęitir ulfr, \hld\ es fylgir hinu \alst{sk}ír-lęita &
\ind goði til \alst{v}arna \alst{v}iðar, &
en annarr \alst{H}ati, \hld\ hann ’s \alst{H}róð-vitnis sonr, &
\ind sá skal fyr \alst{h}ęiða brúði \alst{h}imins.\eva

\bvb \inx[P]{Scoll} is called the wolf who follows the pure-faced \\
\ind god \ken*{= Sun} to the shelter of the woods. \\
But another is \inx[P]{Hate}, he is \inx[P]{Rothwitner}’s son— \\
\ind who shall [run] in front of the bright bride of heaven \ken*{= Sun}.\footnoteB{According to \Gylfaginning\ 12 Scoll chases the Sun and Hate chases the Moon (which is why he runs in front of the sun).  See note to \Voluspa\ 40 for discussion on these wolves.}\evb\evg


\bvg\bva\mssnote{\Regius~10v/6, \AM~5r/23, \\ \AMb~9v/14, \EddaBms~3v/11}%TODO: Critical notes for these next two stanzas based on the mss. Sigla for ms. B = AM 757 a 4°.
\edtext{Ór \alst{Y}mis holdi \hld\ vas \alst{jǫ}rð of skǫpuð, &
\ind en ór \edtrans{\alst{s}vęita}{blood}{\Afootnote{\emph{hans sára svęita} ‘blood of his wounds’ \AMb\EddaBms}\Bfootnote{For the sense, see note to this word in \Vafthrudnismal\ 21.}} \edtext{\alst{s}jór}{\Afootnote{so \AM\AMb\EddaBms; \emph{sę́r} \Regius}}, &
\alst{b}jǫrg ór \alst{b}ęinum, \hld\ \alst{b}aðmr ór hári, &
\ind en \edtrans{ór \alst{h}ausi \alst{h}iminn}{from his skull the heaven}{\Afootnote{\emph{himinn ór hausi hans} ‘the heaven from his skull’ \AMb\EddaBms}\Bfootnote{This suggests that the heavens were understood as a dome, something common among many ancients. This also fits well with the floating clouds being Yimer’s brains, as said in the following st.}}.}{\lemma{Ór \dots\ himinn ‘Out of \dots\ heaven’}\Bfootnote{This stanza is clearly related to \Vafthrudnismal\ 21, see note there.}}\eva

\bvb From \inx[P]{Yimer}’s flesh was the earth shaped, \\
\ind and from his blood the sea; \\
mountains from his bones, woods from his hair, \\
\ind and from his skull the heaven.\evb\evg


\bvg\bva\mssnote{\Regius~10v/8, \AM~5r/25, \\ \AMb~9v/16, \EddaBms~3v/12}%
\edtext{En ór hans \alst{b}rǫ́um \hld\ gørðu \alst{b}líð ręgin &
\ind \alst{M}ið-garð \alst{m}anna sonum,}{\lemma{En ór hans brǫ́um \dots\ manna sonum ‘But from his eyebrows \dots\ sons of men’}\Bfootnote{The gods fenced in Middenyard (‘the middle enclosure’) by using the hair of Yimer’s eyebrows as poles.}} &
en ór hans \alst{h}ęila \hld\ vǫ́ru þau hin \edtrans{\alst{h}arð-móðgu}{hard-minded}{\Afootnote{\emph{hríð-fęldu} ‘stormy’ \AMb\EddaBms}} &
\ind \alst{sk}ý ǫll of \alst{sk}ǫpuð.\eva

\bvb And from his eyebrows the blithe \inx[G]{Reins} made \\
\ind \inx[L]{Middenyard} for the sons of men, \\
and from his brains were the hard-minded \\
\ind clouds all shaped.\evb\evg


\bvg\bva\mssnote{\Regius~10v/9, \AM~5r/26}%
\edtext{\edtrans{\alst{U}llar}{Woulder’s}{\Bfootnote{It is uncertain why the rather obscure god Woulder is invoked here; it cannot be simply for the sake of alliteration, since \emph{Óðins} ‘Weden’s’ would work just as well.  It may be that Woulder had a particular role in the setting of the ritual fire, something supported by the large number of firesteel-shaped amulets at the archeological site of \emph{Lilla Ullevi} (‘Woulder’s little \inx[C]{wigh}’) in Sweden.  For this site see Index: \inx[P]{Woulder} and \textcite{afEdholm2009}.}} \edtrans{hylli}{holdness}{\Bfootnote{‘Favour, loyalty, grace’.  This root (from which also the adjective \emph{hollr} ‘hold; favourable, loyal, gracious’ and verb \emph{hylla} ‘to make hold’) is often to refer to godly grace in both a Heathen and Christian context.  See Index: \inx[C]{hold} and \inx[C]{holdness}.}} \hld\ hęfr ok \edtrans{\alst{a}llra goða}{All Gods}{\Bfootnote{Cf. \Sigrdrifumal\ 3–4, \Lokasenna\ 11, which both hail the Gods as a collective (the former as part of a genuine prayer, the latter subversively).  For the oneness of the Gods, see Index: \inx[C]{All Gods}.}} &
\ind hvęrr’s tękr \alst{f}yrstr á \alst{f}una, &
því-at \alst{o}pnir hęimar \hld\ verða umb \alst{á}sa sonum, &
\ind þá’s \alst{h}ęfja af \edtrans{\alst{h}vera}{kettles}{\Bfootnote{acc. pl. of \emph{hverr}, from PGmc. \emph{*hweraz}, from PIE \emph{*kʷer-} ‘pot, vessel’.  Interestingly the Sanskrit cognate \emph{carú} is occasionally used in reference to the vat wherein the ritual drink \emph{soma} is prepared (e.g. \Rigveda\ 10.167.4).}}.}{\lemma{ALL}\Bfootnote{This st. is one of the most difficult in the poem and many interpretations have been made.

The traditional explanation (e.g. \textcite{FinnurEdda}, Bellows, Sijmons and Gering (p. 208)) relates it to the poem’s frame narrative.  In this view, Weden, bound between the two fires, cryptically asks for a cauldron hanging above him to be moved so that the Gods will be able to see him through the smoke-vent and rescue him.  This explanation is strange given the stanza’s placement in the gnomic wisdom section of the poem, unless the whole section is taken to be a later insert (so Finnur), something for which there is no textual support.  The invocation of the obscure god Woulder is also left unexplained, and there is no mention of a cauldron elsewhere in the poem.

A better explanation is given by \textcite{Nordberg2005}, who argues that the stanza is another piece of gnomic wisdom, referring to the cooking of the sacrificial meal in large cauldrons during the \inx[C]{bloot}.  The st. describes the divine grace (\emph{hylli} ‘holdness’, see Note to l. 1) won by the ritualist who sets the fire onto which the cauldron is placed, since this act enables the Gods to become guests at the ritual meal.  Cf. \HakonarSaga\ 14, describing the traditional bloot in the Throndlaw (\emph{Þrǿnda-lǫg}), Norway: \emph{At veizlu þeiri skyldu allir menn ǫl eiga; þar var ok drepinn alls konar smali ok svá hross, [...] en slátr skyldi sjóða til mann-fagnaðar; eldar skyldu vera á miðju gólfi í hofinu ok þar katlar yfir.} ‘At that gathering all men should have ale; thereat was also slain every kind of small cattle and likewise horses, [...] and the fresh meat should be cooked for men to enjoy.  There should be fires in the middle of the floor in the hove and kettles above them.’

This interpretation is especially interesting when one considers the immediately preceding stanzas 41–42, which deal with the ordering of the world through the dismembering of Yimer, the primordial victim sacrificed by the Gods.  It is known from other Indo-European branches that the ritual sacrifice in the present was seen as a reenactment of the primeval sacrifice in the mythic past, which was necessary for the continued existence of the world and the social order (cf. e.g. \Rigveda\ 10.90); for discussion see \textcite{Lincoln1986}, especially the first two chapters.  If this is correct, \Grimnismal\ 41–43 would then attest this conception also in the Germanic tradition.}}\eva

\bvb \inx[P]{Woulder}’s \inx[C]{holdness} and that of \inx[C]{All Gods} \\
\ind has whoever first touches the fire, \\
for the \inx[C]{Home}[Homes] open up for the Sons of the Eese, \\
\ind when men lift off the kettles.\evb\evg


\bvg\bva\mssnote{\Regius~10v/11, \AM~5r/28}%
\alst{Í}valda synir \hld\ gingu í \alst{á}r-daga &
\ind \alst{Sk}íð-blaðni at \alst{sk}apa, &
\alst{sk}ipa batst \hld\ \alst{sk}írum Fręy, &
\ind \alst{n}ýtum \alst{N}jarðar bur.\eva

\bvb Iwald’s sons went in days of yore \\
\ind Shidebladner for to shape: \\
the best of ships for the pure Free, \\
\ind for the useful Son of Nearth.\evb\evg


\bvg\bva\mssnote{\Regius~10v/13, \AM~5r/29}%
\alst{A}skr \alst{Y}gg-drasils, \hld\ hann ’s \alst{ǿ}ðstr viða &
\ind en \alst{Sk}íð-blaðnir \alst{sk}ipa, &
\alst{Ó}ðinn \alst{á}sa \hld\ en \alst{jó}a Slęipnir, &
\alst{B}il-rǫst \alst{b}rúa \hld\ en \alst{B}ragi skalda, &
\alst{H}á-brók \alst{h}auka \hld\ en \alst{h}unda Garmr.\eva

\bvb Ugdrassle’s Ash—it is the noblest of trees, \\
\ind and Shidebladner of ships; \\
Weden of the Eese and Slapner of steeds; \\
Bilrest of bridges and Bray of scolds; \\
Highbrook of hawks and Garm of hounds.\evb\evg

\sectionline

\bvg\bva\mssnote{\Regius~10v/15, \AM~5v/2}%
\alst{S}vipum hęf’k nú ypt \hld\ fyr \alst{s}ig-tíva sonum, &
\ind við þat skal \alst{v}il-bjǫrg \alst{v}aka, &
\alst{ǫ}llum \alst{ǫ́}sum \hld\ þat skal \alst{i}nn koma &
\ind \alst{Ę́}gis bękki \alst{á} &
\ind \alst{Ę́}gis drekku \alst{a}t.\eva

\bvb My gaze have I now lifted up before the sons of the victory-Tews \ken*{= Eese}— \\
\ind by that shall the willed rescue awake! \\
All the Eese shall it bring into here, \\
\ind upon Eagre’s bench, \\
\ind at Eagre’s drinking!\footnoteB{Weden suddenly announces that he has made the other gods aware of his situation; they will leave their feasting at Eagre’s hall (see \Hymiskvida\ and \Lokasenna) and instead come to his rescue.  He then begins to recount his names.}\evb\evg


\bvg\bva\mssnote{\Regius~10v/17, \AM~5v/4}Hétumk \alst{G}rímr, \hld\ hétumk \alst{G}anglęri, &
\ind \alst{H}ęrjann ok \alst{H}jalm-beri, &
\alst{Þ}ękkr ok \alst{Þ}riði, \hld\ \alst{Þ}undr ok Uðr, &
\ind \alst{H}ęl-blindi ok \alst{H}ár.\eva

\bvb I called myself Grim, I called myself Gangler, \\
\ind Harn and Helmbearer. \\
Theck and Third, Thound and Ith, \\
\ind Hellblinder and High.\evb\evg


\bvg\bva\mssnote{\Regius~10v/19, \AM~5v/5}%
\alst{S}aðr ok \alst{S}vipall \hld\ ok \alst{S}ann-getall, &
\ind \alst{H}ęr-tęitr ok \alst{H}nikarr, &
\alst{B}il-ęygr, \alst{B}ál-ęygr, \hld\ \alst{B}ǫl-verkr, Fjǫlnir, &
\alst{G}rímr ok \alst{G}rímnir, \hld\ \alst{G}lap-sviðr ok Fjǫl-sviðr.\eva

\bvb Sooth and Swiple and Soothgettle, \\
\ind Hartote and Nicker, \\
Bileye, Baleeye, Baleworker, Fillner, \\
Grim and Grimner, Glapswith and Fellswith.\evb\evg


\bvg\bva\mssnote{\Regius~10v/21, \AM~5v/7}%
\alst{S}íð-hǫttr, \alst{S}íð-skęggr, \hld\ \alst{S}ig-fǫðr, Hnikuðr, &
\alst{A}l-fǫðr, \alst{V}al-fǫðr, \hld\ \alst{A}t-ríðr ok Farma-týr; &
\alst{ęi}nu nafni \hld\ hétumk \alst{a}ldri-gi &
\ind síðst ek með \alst{f}olkum \alst{f}ór.\eva

\bvb Sidehat, Sideshag, Syefather, Nicked, \\
Allfather, Walfather, Atrider, and Farm-Tew— \\
by just one name have I never called myself, \\
\ind since among manfolk I fared.\evb\evg


\bvg\bva\mssnote{\Regius~10v/23, \AM~5v/9}%
\alst{G}rímni mik hétu \hld\ at \alst{G}ęir-raðar, &
\ind en \alst{Ja}lk at \alst{Ǫ́}s-mundar; &
en þá \alst{K}jalar \hld\ es ek \alst{k}jalka dró, &
\ind \alst{Þ}rór \alst{þ}ingum at.\eva

\bvb Grimner they called me at Garfrith’s [home], \\
\ind but Yelk at Osmund’s, \\
but Keller whenas I drew the sled; \\
\ind Throo at \inx[C]{Thing}[Things].\footnoteB{Presumably referencing other now-lost myths involving Weden travelling in disguise. The last is possibly a reference to the name under which Weden would be invoked at the start of Things (legal assemblies, see Index).}\evb\evg


\bvg\bva\mssnote{\Regius~10v/24, \AM~5v/10}%
\alst{Ó}ski ok \alst{Ó}mi, \hld\ \alst{Ja}fn-hár ok Biflindi, &
\ind \alst{G}ǫndlir ok Hár-barðr með \alst{g}oðum.\eva

\bvb Wish and Ome, Evenhigh and Bivlend; \\
\ind Gandler and Hoarbeard among Gods.\evb\evg


\bvg\bva\mssnote{\Regius~10v/25, \AM~5v/11}%
\alst{S}viðurr ok \alst{S}viðrir \hld\ es ek hét at \alst{S}økk-mímis &
\ind ok dulða’k þann hinn \alst{a}ldna \alst{jǫ}tun &
þá’s \alst{M}ið-vitnis vas’k \hld\ ins \alst{m}ę́ra burar &
\ind \alst{o}rðinn \alst{ęi}n-bani.\eva

\bvb Swither and Swithrer, as I was called at Sink-Mimer’s, \\
\ind and I deceived that aged ettin, \\
when of Midwitner’s famous son \\
\ind I had become the lone slayer.\evb\evg


\bvg\bva\mssnote{\Regius~10v/28, \AM~5v/13}%
\alst{Ǫ}lr est Gęir-røðr, \hld\ hęfr þú \alst{o}f-drukkit; &
\alst{m}iklu est hnugginn, \hld\ es þú est \alst{m}ínu gęngi, &
\edtrans{\alst{ǫ}llum \alst{ęi}n-hęrjum}{of all the Oneharriers}{\Bfootnote{Linguistically, Garfrith is not bereft of the support of the Oneharriers but rather of the Oneharriers themselves, but the sense is the same.  By breaking the Odinic code of conduct he has lost Weden’s favour, and thus been excluded from the community of oath-bound warriors, the Oneharriers.

On the other hand a king who behaved well could expect to have the truce of the Oneharriers; this was the case for Hathkin the Good according to the poem composed about him (Eyv \emph{Hák} in \Skp\ 1).  In that poem (st. 16/1–2) \inx[P]{Bray} greets him in the hall of the Gods, saying: \emph{Ęin-hęrja grið · skalt allra hafa; / þigg þú at Ǫ́sum ǫl.} ‘All the Oneharriers’ truce shalt thou have; take ale from the \inx[G]{Eese}!’}} \hld\ ok \alst{Ó}ðins hylli.\eva

\bvb Worse for ale art thou, Garfrith; thou hast over-drunk. \\
Of much art thou bereft when thou art [bereft] of my support, \\
of all the \inx[G]{Oneharriers}, and of Weden’s \inx[C]{holdness}.\evb\evg


\bvg\bva\mssnote{\Regius~10v/30, \AM~5v/15}%
\alst{F}jǫlð þér sagða’k, \hld\ en þú \alst{f}átt of mant, &
\ind of þik \alst{v}éla \alst{v}inir; &
\edtext{\alst{m}ę́ki liggja \hld\ sé’k \edtext{\alst{m}íns vinar}{\linenum{|2--3}\lemma{vinir, míns vinar ‘friends, my friend’}\Bfootnote{Weden stresses his friendship with Garfrith by using the word \emph{vinr} ‘friend’ twice.  The followers of a god were his friends; see note to \Havamal\ 157.}} &
\ind allan í \alst{d}ręyra \alst{d}rifinn.}{\lemma{mę́ki \dots\ drifinn. ‘The sword \dots\ gore.’}\Bfootnote{Weden foretells Garfrith’s coming death.}}\eva

\bvb Much I told thee, but thou recallest little; \\
\ind ’tis friends that deal with thee! \\
The sword of my friend I see lying \\
\ind all drenched in gore.\evb\evg


\bvg\bva\mssnote{\Regius~10v/31, \AM~5v/16}%
\alst{Ę}gg-móðan val \hld\ nú mun \alst{Y}ggr hafa, &
\ind þitt vęit’k \alst{l}íf of \alst{l}iðit; &
\alst{v}arar ’ru \edtrans{dísir}{Dises}{\Bfootnote{i.e. the Norns, fates, who have determined his hour of death.  Cf. \Fafnismal\ TODO, \Hamdismal\ TODO. }}, \hld\ nú knátt \alst{Ó}ðin séa; &
\ind nálgask \alst{m}ik ef þú \alst{m}ęgir!\eva

\bvb An edge-tired corpse will Ug now have: \\
\ind I know thy life to be past. \\
Wary are the \inx[G]{Dises}, now dost thou see Weden— \\
\ind come near me, if thou mayst!\evb\evg


\bvg\bva\mssnote{\Regius~11r/2, \AM~5v/18}%
\alst{Ó}ðinn nú hęiti’k, \hld\ \alst{Y}ggr áðan hét’k, &
\ind hétumk \alst{Þ}undr fyr \alst{þ}at, &
\alst{V}akr ok Skilfingr, \hld\ \alst{V}ǫ́fuðr ok Hropta-týr &
\ind \alst{G}autr ok Jalkr með \alst{g}oðum.\eva

\bvb Weden am I called now, Ug was I called earlier, \\
\ind I called myself Thound before that; \\
Wacker and Shilving, Waved and Roft-Tew, \\
\ind Geat and Gelding among the Gods.\evb\evg


\bvg\bva\mssnote{\Regius~11r/4, \AM~5v/20}%
\alst{O}fnir ok Sváfnir \hld\ hygg’k at \alst{o}rðnir sé &
\ind \alst{a}llir at \alst{ęi}num mér.\eva

\bvb Ovner and Swebner, I ween, have arisen \\
\ind all from me alone.\evb\evg


\bpg\bpa\mssnote{\Regius~11r/5, \AM~5v/21}Geir-røðr konungr sat, ok hafði sverð um kné sér ok brugðit til miðs. En er hann heyrði, at Óðinn var þar kominn, stóð hann upp, ok vildi taka Óðin frá eldinum. Sverðit slapp ór hendi hánum; vissu hjǫltin niðr. Konungr drap fę́ti, ok steyptist á-fram, en sverðit stóð í gǫgnum hann, ok fekk \edtext{hann}{\Afootnote{þar af \AM}} bana. \edtext{Óðinn hvarf þá.}{\Afootnote{om. \AM}} En Agnarr \edtext{var þar}{\Afootnote{varð \AM}} konungr \edtext{lengi síðan.}{\Afootnote{om. \AM}}\epa

\bpb King Garfrith sat and had the sword about his knee, and it was brandished half-way up. But when he heard that Weden were come there, he stood up and would take Weden from the fire. The sword slipped out of his hand; the hilt pointed downwards. The king tripped and stooped forth, but the sword went through him, and he received his bane. Weden then disappeared, but Ayner was there king for a long while afterwards.\epb\epg
