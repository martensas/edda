\bookStart{The Speeches of Grimner}[Grímnismǫ́l]

\begin{flushright}%
Dating \parencite{Sapp2022}: C10th (0.976)

Meter: \Ljodahattr, \Fornyrdislag\ (2/3–4, 28/3–5, 45/3–5, 48/4, 49/1–2, 53), \Galdralag\ (46)%
\end{flushright}

% Introduction

The \textbf{Speeches of Grimner} are preserved whole in both \Regius\ and \AM.

The poem itself is surrounded by two long introductory prose narratives containing some very old motifs, which are here brought up in the notes. It’s hard to say for how long these texts have accompanied the poem (TODO: I may write about this in the Introduction, since this question is important for several other poems), but since they are found in both \Regius\ and \AM\ and contain these motifs it would seem that they are fairly old. Together with sts. 1–3 they form a frame narrative that gives additional meaning to the gnomic sts. enclosed within.

The gnomic sts. themselves, the meat of the poem, are mythological and often quite obscure. In this they align closely with other Eddic gnomic poems such as \Havamal, \Vafthrudnismal, \Sigrdrifumal, and \Allvismal.

Weden begins by listing the halls of the gods (4–17). This section has been discussed in detail by \textcite{deVries1952} TODO! who considers it corrupt. Specifically, he sees the second half of v. 4 as a later insert, since it does not elaborate on the “holy land” mentioned in the first half. \textcite{Jackson1995} has argued convincingly against this, showing how the first half serves as a generalized introduction to the list; the holy land is the dwelling-places of the gods.

After this list come several sts relating to Weden and his hall, Walhall (18–23). Mentioned are the preparation of food in Walhall (18), Weden’s wolves (19) and ravens (20), the river through which the dead have to wade (21) and the gate through which they have to pass (22), the count of doors in Walhall (23), the count of doors in Thunder’s hall Bilshirner (24), and two animals which stand on the hall and gnaw on the branches of the tree Leered (25–26). From the latter animal’s—the stag Oakthirner’s—horns droplets fall into Wharyelmer, which is the origin of all rivers (26).

This introduces a list of mythic rivers (27–28), ending with the waters through which Thunder must wade on his way to Ugdrassle (29). This leads to a list of the horses ridden by the other gods on their way to Ugdrassle (31) which is followed by a description of the roots of Ugdrassle (31), then its animals (32–36) the Walkirries (37), and beings associated with the sun and moon (38–40), the things created from Yimer’s body (41–42) with a digression on the significance of the \inx[P]{bloot} for men in the present (43, see note there!), the creation of the ship Shidebladner (44) and finally a list of the noblest of several categories of things and groups (45).

After these lists Weden utters an unclear st. invoking the gods (46), before listing many of his names and the circumstances in which they were used (47–50). He then turns to Garfrith, disappointed by the inhospitality and poor conduct of his former protégé, and predicts his imminent death (51–53). He finally reveals himself by his true name, daring Garfrith to face him (53). After this he repeats several of his names (54), and the poem ends.

In the final prose section we are told that Garfrith, after learning that he was torturing Weden, hurried up to take the god away from the fires, but tripped and fell on his sword and died. After this his son Eyner ruled for a long time.

\sectionline

\section{From the sons of king Reeding (\emph{Frá sonum Hrauðungs konungs})}

\bpg
\bpa[1a]\mssnote{\Regius~8v/31, \AM~3v/23}Hrauðungr konungr átti tvá sonu. Hét annarr Agnarr, enn annarr Geirrøðr.
Agnarr var tíu vetra enn Geirrøðr átta vetra. Þeir reru tveir á báti með dorgar sínar at smáfiski.
Vindr rak þá í haf út. Í náttmyrkri brutu þeir við land ok gingu upp; fundu kotbónda einn.
Þar vǫ́ru þeir um vetrinn. Kerling fostraði Agnar enn karl Geirrøð.
At vári fekk karl þeim skip. Enn er þau kerling leiddu þá til strandar, þá mę́lti karl einmę́li við Geirrøð.
Þeir fengu byr ok kvǫ́mu til stǫðva fǫður síns. Geirrøðr var fram í skipi.
Hann hljóp upp á land enn hratt út skipinu, ok mę́lti: ”Far þú þar er smyl hafi þik.”
Skipit rak út. Enn Geirrøðr gekk út til bǿjar; hánum var vel fagnat; þá var faðir hans andaðr.
Var þá Geirrøðr til konungs tekinn, ok varð maðr ágę́tr.\epa

\bpb King Reeding owned two sons. One was called Eyner, and the other Garfrith.
Eyner was ten winters old, and Garfrith eight winters. The two were rowing in a boat with their trolling-lines for small fishing.
The wind then drove them out into the sea. In the dark of night they crashed into land and walked up; they found a lone cottage-farmer.
There they were over the winter. The wife fostered Eyner, but the husband Garfrith.\footnoteB{The wife was Frie, and the husband Weden; this is clarified by the following prose. The motif of Weden preferring the youngest brother is also found in \Rigsthula.}
In the spring the husband gave them ships, but when they followed the farmer’s wife in leading them to the shore, the husband spoke privately with Garfrith.\footnoteB{Surely instructing him to push his brother out to sea.}
They got a good gust, and came to their father’s harbour. Garfrith was in the front of the ship.
He leapt up onto land and pushed out the ship, and spoke: ”Go thou whither the fiends may have thee!”
The ship drove out. But Garfrith walked towards the farm; he was welcomed well; by then his father was passed-on.
Then Garfrith was taken as king, and became an excellent man.\epb
\epg


\bpg
\bpa[1b]\mssnote{\Regius~9r/10, \AM~4r/3}Óðinn ok Frigg sátu í Hliðskjǫlfu ok sá um heima alla.
Óðinn mę́lti: Sér þú Agnar fóstra þinn, hvar hann elr bǫrn við gýgi í hellinum?
En Geirrøðr, fóstri minn, er konungr ok sitr nú at landi.
Frigg segir: Hann er matníðingr sá at hann kvelr gesti sína ef hánum þykkja ofmargir koma.
Óðinn segir at þat er in mesta lygi. Þau veðja um þetta mál.
Frigg sendi eskismey sína, Fullu, til Geirrøðar. Hon bað konung varask at eigi fyrgerði hánum fjǫlkunnigr maðr sá er þar var kominn í land ok sagði þat mark á at engi hundr var svá ólmr at á hann myndi hlaupa.
En þat var inn mesti hégómi at Geirrøðr vę́ri eigi matgóðr ok þó lę́tr hann handtaka þann mann er eigi vildu hundar á ráða.
Sá var í feldi blám ok nefndisk Grímnir ok sagði ekki fleira frá sér þótt hann vę́ri atspurðr.
Konungr lét hann pína til sagna ok setja milli elda tveggja ok sat hann þar átta nę́tr.
Geirrøðr konungr átti son tíu vetra gamlan ok hét Agnarr eftir bróður hans.
Agnarr gekk at Grímni ok gaf hánum horn fullt at drekka, sagði að konungr gerði illa er hann lét pína hann saklausan.
Grímnir drakk af. Þá var eldrinn svá kominn at feldrinn brann af Grímni. Hann kvað:\epa

\bpb Weden and Frie sat in \inx[L]{Lithshelf} and looked over all the Homes.\footnoteB{Very similar to the Longbeard Origin Myth (TODO: reference and elaborate).}
Weden spoke: “Seest thou Eyner, thy foster-son, where he begets children with the troll-woman in the cave?\footnoteB{This may relate to Frie’s role as love-goddess. Eyner is in any case a \inx[C]{degenerate} man, what one would call a ‘coomer’.}
But Garfrith, my foster-son, is king and now sits at land.”
Frie says: “He is such a meat-nithing that he tortures his guests if he judges too many are coming.”
Weden says that this is the greatest lie; they make a wager about this matter.
Frie sent her handmaid Full to Garfrith’s. She bade the king be wary, that he not be ended by that \inx[C]{many-cunning} man who was come in the land, and said that his sign was that no hound was so fierce that he would leap at him.
But that was the greatest vainglory that Garfrith were not meat-good, and yet he has that man seized, whom the hounds would not touch.
He was clad in a blue cloak, and called himself Grimner, and did not tell any more about himself, even though he was interrogated.
The king had him tortured that he would speak, and set him between two fires, and he sat there for eight nights.
King Garfrith had a son ten winters old, and he was named Eyner after his brother.
Eyner walked up to Grimner, and gave him a full horn to drink, saying that the king did ill as he had him tortured without cause.
Grimner drank from it. Then the fire had come such that the cloak burned on Grimner. He quoth:\epb
\epg\stepcounter{prosea}

\sectionline

\bvg
\bva\mssnote{\Regius~9r/27, \AM~4r/17}Hęitr est hripuðr \hld\ ok hęldr til mikill, &
\ind gǫngumk firr funi! &
Loði sviðnar, \hld\ þótt á lopt bera’k; &
\ind brinnumk feldr fyrir.\eva

\bvb Hot art thou, flame, and rather too large; go far from me, fire! The woolen cape is singed though I hold it aloft; the cloak burns before me.\evb
\evg


\bvg
\bva\mssnote{\Regius~9r/29, \AM~4r/18}Átta nę́tr \hld\ sat’k milli ęlda hér, &
\ind svá’t mér mann-gi mat né bauð &
nema ęinn Agnarr, \hld\ es ęinn skal ráða, &
Gęirrøðar sonr, \hld\ Gotna landi.\eva

\bvb For eight nights sat I in the middle of the fires here, while no man offered me food; save for lone Eyner, who lone shall rule—the son of Garfrith—the land of the Gots!\evb
\evg


\bvg
\bva\mssnote{\Regius~9r/31, \AM~4r/20}Hęill skalt, Agnarr, \hld\ alls hęilan biðr &
\ind þik Veratýr vesa; &
ęins drykkjar \hld\ skalt aldrigi &
\ind bętri gjǫld geta:\eva

\bvb Hale shalt thou be, Eyner, as hale Weretue \name{= Weden} bids thee be; for one drink shalt thou never get a better recompense:\footnoteB{The recompense being the esoteric lore which is told from the following st. onwards.}\evb
\evg

\sectionline

\bvg
\bva\mssnote{\Regius~9r/33, \AM~4r/22}Land es hęilagt, \hld\ es liggja sé’k &
\ind ǫ́sum ok ǫlfum nę́r; &
en í Þrúðhęimi \hld\ skal Þórr vesa &
\ind unz of rjúfask ręgin.\eva

\bvb The land is holy, which I see lying close to the \inx[F]{Ease and Elves}; but in Thrithham shall Thunder be, until the Reins are rent.\evb
\evg


\bvg
\bva\mssnote{\Regius~9v/2, \AM~4r/23}Ýdalir hęita, \hld\ þar’s Ullr of hęfr &
\ind sér of gǫrva sali; &
Alfhęim Fręy \hld\ gǫ́fu í árdaga &
\ind tívar at tannféi.\eva

\bvb Yewdales are called where Woulder has made himself a hall. Elfham to Free in days of yore did the Tews as a tooth-gift\footnoteB{The gift that a child receives when he gets his first tooth.} give.
\evg


\bvg
\bva\mssnote{\Regius~9v/3, \AM~4r/25}Bǿr ’s hinn þriði, \hld\ es blíð ręgin &
\ind silfri þǫkðu sali; &
Valaskjǫlf hęitir, \hld\ es vélti sér &
\ind ǫ́ss í árdaga.\eva

\bvb Bower is the third, where the blithe Reins with silver thatched a hall. Waleshelf is called [the hall] which the os in days of yore won through wiles.\footnoteB{Several previous editors and translators (e.g. \textcite{FinnurEdda}, \textcite{PettitEdda}, \textcite{LarringtonEdda}) has rendered this phrase with variants of ‘craftily made for himself’ but I disagree.}\evb
\evg


\bvg
\bva\mssnote{\Regius~9v/5, \AM~4r/26}Søkkvabękkr hęitir hinn fjórði, \hld\ en þar svalar knegu &
\ind unnir glymja yfir; &
þar þau Óðinn ok Sága \hld\ drekka umb alla daga &
\ind glǫð ór gollnum kęrum.\eva

\bvb Sinkbench is called the fourth, but there cool waves do clash above; there Weden and Sey drink all days, glad, out of golden casks.\evb
\evg


\bvg
\bva\mssnote{\Regius~9v/7, \AM~4r/28}Glaðshęimr hęitir hinn fimti \hld\ þar’s hin gollbjarta &
\ind Valhǫll víð of þrumir; &
en þar Hroptr \hld\ kýss hvęrjan dag &
\ind vápndauða vera.\eva

\bvb Gladsham is called the fifth, where the gold-bright Walhall—wide—stands fast; but there Roft \name{= Weden} chooses every day weapon-dead men.\footnoteB{Cf. v. 14.}\evb
\evg


The order of the following two sts is that of \AM. \Regius\ has them st..


\bvg
\bva\mssnote{\Regius~9v/10, \AM~4r/30}Mjǫk ’s auðkęnt \hld\ þęim’s til Óðins koma &
\ind salkynni at séa, &
skǫptum ’s rann rępt, \hld\ skjǫldum ’s salr þakiðr, &
\ind brynjum of bękki stráat.\eva

\bvb Very easily recognized, for those who to Weden’s come, is the hall to see: With spear-shafts is the house roofed; with shields is the hall thatched; with byrnies the benches strewn.\evb
\evg


\bvg
\bva\mssnote{\Regius~9v/9, \AM~4r/31}Mjǫk ’s auðkęnt \hld\ þęim’s til Óðins koma &
\ind \edtext{salkynni at séa}{\lemma{salkynni at séa ‘the hall to see’}\Afootnote{‘sia at sia’ \AM}}, &
vargr hangir \hld\ fyr vestan dyrr &
\ind ok drúpir ǫrn yfir.\eva

\bvb Very easily recognized, for those who to Weden’s come, is the hall to see: A wolf hangs before the western door, and an eagle droops over.\footnoteB{According to Hyltén-Cavallius (1863:156) it was custom to hang the bodies of dead wolves high up in old oaks, and dead birds of prey above the stable-door.}\evb
\evg


\bvg
\bva\mssnote{\Regius~9v/12, \AM~4v/2}Þrymhęimr hęitir hinn sétti, \hld\ es Þjazi bjó, &
\ind sá hinn ámátki jǫtunn; &
en nú Skaði byggvir, \hld\ skír brúðr goða, &
\ind fornar toptir fǫður.\eva

\bvb Thrimham is called the sixth, where Thedse dwelled, that terrifying ettin; but now Shede bedwells—pure bride of the gods—the ancient plots of her father.\evb
\evg


\bvg
\bva\mssnote{\Regius~9v/14, \AM~4v/3}Bręiðablik eru hin sjaundu, \hld\ en þar Baldr hęfir &
\ind sér of gǫrva sali, &
á því landi \hld\ es liggja vęit’k &
\ind fę́sta fęiknstafi.\eva

\bvb Broadblicks are the seventh, and there Balder has made for himself a hall; on that land, where I know lie the fewest staves of treachery.\footnoteB{Evil deeds.}\evb
\evg


\bvg
\bva\mssnote{\Regius~9v/16, \AM~4v/5}Himinbjǫrg eru hin ǫ́ttu \hld\ en þar Hęimdall &
\ind kveða valda véum. &
þar vǫrðr goða \hld\ drekkr í vę́ru ranni &
\ind glaðr góða mjǫð.\eva

\bvb Heavenbarrows are the eighth, and there Homedall, they say, wields over wighs. There the ward of the gods \ken*{= Homedall} drinks in the tranquil house, glad, the good mead.\evb
\evg


\bvg
\bva\mssnote{\Regius~9v/17, \AM~4v/6}Folkvangr es hinn níundi \hld\ en þar Fręyja rę́ðr &
\ind sessa kostum í sal; &
halfan val \hld\ hon kýss hvęrjan dag &
\ind en halfan Óðinn á.\eva

\bvb Folkwong is the ninth, and there Frow rules the choice of seats in the hall; half the slain she chooses each day, but half Weden owns.\footnoteB{This st. is cited and closely paraphrased in \Gylfaginning\ 24. — The roots of \emph{kjósa val} ‘choose the slain’ are the same as those in \inx[C]{walkirrie} (\emph{val-kyrja} ‘chooser of the slain’), and as Frow is a prominent goddess this would surely make her the chief walkirrie.
This is paralleled by \Sorlathattr, where Frow assumes the name \inx[C]{Gandle} (\emph{Gǫndul}, a name attested in several lists of walkirries; see \Voluspa\ 30 and Notes) and incites the legendary never-ending Conflict of the Headnings (\emph{Hjaðningavíg}).
In spite of this parallel, there are good arguments for believing that the chief walkirrie should be \inx[C]{Frie}, Weden’s wife.
First, one of the functions of the walkirries is to bear ale to the Ownharriers (\Grimnismal\ 37). This mirrors royal Germanic banquets attested in heroic poetry, where the host’s wife or daughter would pour ale to his retainers and guests (the so-called ‘lady with a mead cup’ ritual; see \textcite{Enright1996} and \textcite{Riseley2014}). As Weden’s wife, we would expect Frie to have this role.
Second, TODO
Third, TODO.}\evb
\evg


\bvg
\bva\mssnote{\Regius~9v/19, \AM~4v/8}Glitnir es hinn tíundi; \hld\ hann es gulli studdr &
\ind ok silfri þakðr it sama; &
en þar Forseti \hld\ byggir flęstan dag &
\ind ok svę́fir allar sakir.\eva

\bvb Glitner is the tenth, it is studded by gold, and thatched by silver the same; but there Forset dwells most of the day, and resolves\footnoteB{Puts to sleep,} all [legal] matters.\evb
\evg


\bvg
\bva\mssnote{\Regius~9v/21, \AM~4v/9}Nóatún eru hin ęlliptu \hld\ en þar Njǫrðr hęfir &
\ind sér um gǫrva sali, &
manna þęngill \hld\ inn męinsvani &
\ind hǫ́timbruðum hǫrgi rę́ðr.\eva

\bvb Nowetowns are the tenth, and there Nearth has made himself a hall. The prince of men, the guileless one, rules the high-timbered \inx[C]{harrow}.\footnoteB{Cf. \Vafthrudnismal\ 38.}\evb
\evg


\bvg
\bva\mssnote{\Regius~9v/23, \AM~4v/11}\edtext{Hrísi vęx \hld\ ok hǫ́u grasi}{\lemma{hrísi vęx ok hǫ́u grasi ‘with brushwood and with tall grass grows’}\Bfootnote{Identical with \Havamal\ 117/6.}} &
\ind Víðars land, viði, &
en þar mǫgr of lę́zk \hld\ af mars baki &
\ind frǿkn at hęfna fǫður.\eva

\bvb With brushwood and with tall grass grows \inx[P]{Wider}’s land, with forest; but there the lad \ken*{= Wider} declares—on the back of his steed—valiant, to avenge his father \ken*{= Weden}.\footnoteB{Wider will avenge his father, Weden. See \Vafthrudnismal\ 53.}\evb
\evg


\bvg
\bva\mssnote{\Regius~9v/24, \AM~4v/12}Andhrímnir \hld\ lę́tr í Ęldhrímni &
\ind Sę́hrímni soðinn, &
flęska bęzt, \hld\ en þat fáir vitu, &
\ind við hvat ęinhęrjar alask.\eva

\bvb Andrimner lets in Eldrimner Sowrimner be boiled. The best of meats, but few know that, by what the Ownharriers are nourished.\footnoteB{The cook Andrimner ‘face-sooty’ has the boar Sowrimner ‘sow-sooty’ boiled in the cauldron Eldrimner ‘fire-sooty’; by this meat are the Ownharriers nouished.}\evb
\evg


\bvg
\bva\mssnote{\Regius~9v/26, \AM~4v/14}Gera ok Freka \hld\ sęðr gunntamiðr, &
\ind hróðigr Hęrjafǫðr, &
en við vín ęitt \hld\ vápngǫfugr &
\ind Óðinn ę́ lifir.\eva

\bvb The battle-accustomed, glorious Father of Hosts \ken*{= Weden} feeds Gerr and Freck; but by wine alone, the weapon-worshipful Weden ever lives.\evb
\evg


\bvg
\bva\mssnote{\Regius~9v/28, \AM~4v/15}Huginn ok Muninn \hld\ fljúga hvęrjan dag &
\ind \edtext{jǫrmungrund}{\lemma{jǫrmungrund ‘ermin-ground’}\Bfootnote{‘the immense ground’ (for the rare prefix \inx[C]{ermin-} see Encyclopedia.); the earth as a vast expanse of land. This compound also occurs in a kenning in the st. on the late C10th Karlevi stone (Öl 1) referring to the unbounded sea as the “ermin-ground of Andle” (\emph{Ęndils jǫrmungrund}, Andle being a sea-king), and in \Beowulf\ 859 as \emph{eormengrund} with the same sense.}} yfir; &
óumk of Hugin, \hld\ at aptr né komi-t; &
\ind þó séumk męir of Munin.\eva

\bvb Highen and Minden fly every day over the ermin-ground \ken{earth}. I fear for Highen, that he may not come back; yet I worry more for Minden.\evb
\evg


\bvg
\bva\mssnote{\Regius~9v/30, \AM~4v/17}Þýtr Þund, \hld\ unir \edtext{Þjóðvitnis &
\ind fiskr}{\lemma{Þjóðvitnis fiskr ‘Thedwitner’s fish’}\Bfootnote{Thedwitner is easily analyzed as \emph{þjóð} ‘great, main’ + \emph{vitnir} ‘wolf’. Thus the main, great wolf, i.e. the \inx[P]{Fenrerswolf}. Its ‘fish’ would then be the Middenyardsworm; cf. \Hymiskvida\ 24.}} flóði í; &
áarstraumr \hld\ þykkir ofmikill &
\ind valglaumi at vaða.\eva

\bvb \inx[P]{Thound} roars; thrives Thedwitner’s fish \ken*{= Middenyardsworm?} in the flood; the river-stream seems far too great for the noisy slain host \ken*{= Ownharriers} to wade through.\footnoteB{Thound is presumably the river surrounding Walhall, which the dead have to pass over to reach the hall.}\evb
\evg


\bvg
\bva\mssnote{\Regius~9v/32, \AM~4v/18}Valgrind hęitir \hld\ es stęndr vęlli á &
\ind hęilǫg fyr hęlgum durum; &
forn ’s sú grind, \hld\ en þat fáir vitu, &
\ind hvé hon ’s í lás of lokin.\eva

\bvb \inx[L]{Walgrind}\footnoteB{‘Corpse-gate;’ the gate guarding Walhall.} ’tis called, which stands on the plain; holy in front of the holy doors. Ancient is that gate, but few know that, how its lock is locked.\evb
\evg


\bvg
\bva\mssnote{\Regius~9v/34, \AM~4v/22}Fimm hundruð golfa \hld\ ok umb fjórum tøgum &
\ind svá hygg’k Bilskirni með bugum; &
ranna þęira, \hld\ es rępt vita’k, &
\ind míns vęit’k męst magar.\eva

\bvb With five hundred floors, and around fourty, so I judge \inx[L]{Bilshirner} altogether. Of those houses, which I might know rafted, I know my lad’s \ken*{= Thunder} to be the greatest.\evb
\evg


\bvg
\bva\mssnote{\Regius~10r/2, \AM~4v/20}Fimm hundruð dura \hld\ ok umb fjórum tøgum, &
\ind svá hygg at Valhǫllu vesa; &
átta hundruð Ęinhęrja \hld\ ganga ór ęinum durum, &
\ind þá’s fara við vitni at vega.\eva

\bvb Five hundred doors, and around fourty, so I judge there to be on Walhall. Eight hundred \inx[G]{Ownharriers} go out of one door,\footnoteB{The hundred is probably here the long hundred (120, rather than 100), which gives a sum of \(640 * 960 = 614,400\) Ownharriers.} when to fight with the wolf they journey.\evb
\evg


\bvg
\bva\mssnote{\Regius~10r/4, \AM~4v/24}Hęiðrún hęitir gęit, \hld\ es stęndr \edtext{hǫllu á}{\lemma{hǫllu á ‘on the hall’}\Afootnote{TODO.}} &
\ind ok bítr af Lę́raðs limum; &
skapkęr fylla \hld\ skal hins skíra mjaðar, &
\ind kná-at sú vęig vanask.\eva

\bvb Heathrune is called the goat, which stands on the hall \ken*{= Walhall}, and bites off the branches of Leered. The shape-vats\footnoteB{According to \CV\ the central beer-vat, from which drinks were poured into smaller vessels.} shall she fill with the pure mead; those draughts cannot wane.\footnoteB{The mead is the goat’s milk.}\evb
\evg


\bvg
\bva\mssnote{\Regius~10r/6, \AM~4v/26}Ęikþyrnir hęitir hjǫrtr \hld\ es stęndr \edtext{hǫllu á}{\lemma{hǫllu á ‘on hall’}\Afootnote{TODO. See previous v.}}&
\ind ok bítr af Lę́raðs limum; &
en af hans hornum \hld\ drýpr í Hvergęlmi &
\ind þaðan ęiga vǫtn ǫll vega:\eva

\bvb Oakthirner is called the stag, which stands on the hall \ken*{= Walhall}, and bites off the branches of Leered. But from his horns does drip into Wharyelmer; thence all waters have their ways:\footnoteB{After which several vv. of mythic river-names are listed.}\evb
\evg


\bvg
\bva\mssnote{\Regius~10r/9, \AM~4v/28}Síð ok Víð, \hld\ Sę́kin ok Ęikin, \hld\ Svǫl ok Gunnþró, &
\ind Fjǫrm ok Fimbulþul, &
\ind Rín ok Rinnandi, &
Gipul ok Gǫpul, \hld\ Gǫmul ok Gęirvimul, &
\ind þę́r hverfa umb hodd goða, &
Þyn ok Vin, \hld\ Þǫll ok Hǫll, &
\ind Gráð ok Gunnþorin.\eva

\bvb Side and Wide, Seeken and Oaken, Swale and Guththrew, Ferm and Fimblethule, Rine and Rinnend, Gipple, Gapple, Gamble and Garwimble—they circle around the hoard of the gods \ken*{osyard}—Thin and Win, Thall and Hall, Grade and Guththorn.\evb
\evg


\bvg
\bva\mssnote{\Regius~10r/12, \AM~5r/1}Vína hęitir enn, \hld\ ǫnnur Vegsvinn, &
\ind þriðja Þjóðnuma, &
Nyt ok Nǫt, \hld\ Nǫnn ok Hrǫnn, &
Slíð ok Hríð, \hld\ Sylgr ok Ylgr, &
Víð ok Vǫ́n, \hld\ Vǫnd ok Strǫnd, &
Gjǫll ok Lęiptr, \hld\ þę́r falla gumnum nę́r &
\ind es falla til hęljar heðan. \eva

\bvb TODO\evb
\evg


\bvg
\bva\mssnote{\Regius~10r/15, \AM~5r/4, \GylfMS}Kǫrmt ok Ǫrmt \hld\ ok kęrlaugar tvę́r &
\ind þę́r skal Þórr vaða &
dag hvęrn \hld\ es dǿma fęrr &
\ind at aski Yggdrasils; &
því’t ǫ́sbrú \hld\ bręnn ǫll loga &
\ind hęilǫg vǫtn \edtext{hlóa}{\Bfootnote{A hapax. TODO.}}.\eva

\bvb Carmt and Armt, and the two Carlays, those shall Thunder wade\footnoteB{For Thunder’s association with wading cf. TODO.} every day when to judge he fares, at the ash of \inx[L]{Ugdrassle}; for the \inx[G]{ease}[os]-bridge \ken{rainbow} burns all with flame; the holy waters bellow.\evb
\evg


\bvg
\bva\mssnote{\Regius~10r/17, \AM~5r/6}Glaðr ok Gyllir, \hld\ Glęr ok Skęiðbrimir, &
\ind Silfrintoppr ok Sinir, &
Gísl ok Falhófnir, \hld\ Gulltoppr ok Léttfeti, &
\ind þęim ríða ę́sir jóum &
dag hvęrn \hld\ es dǿma fara &
\ind at aski Yggdrasils.\eva

\bvb Glad and Yiller, Glare and Sheathbrimmer, Silvrentop and Sinewer, Yissel and Fallowhofner, Goldtop and Lightfeet; on those horses ride the Ease, every day when to judge they fare, at the ash of \inx[L]{Ugdrassle}.\evb
\evg


\bvg
\bva\mssnote{\Regius~10r/20, \AM~5r/8}Þríar rǿtr \hld\ standa á þría vega &
\ind undan aski Yggdrasils; &
Hęl býr und ęinni, \hld\ annarri hrímþursar, &
\ind þriðju męnnskir męnn. \eva

\bvb Three roots stand on three ways, from beneath Ugdrassle’s Ash. Hell lives under one, [under] another the \inx[G]{Rime-Thurses}, [under] the third manly men.\evb
\evg


\bvg
\bva\mssnote{\Regius~10r/22, \AM~5r/9}Ratatoskr hęitir íkorni \hld\ es rinna skal &
\ind at aski Yggdrasils; &
arnar orð \hld\ hann skal ofan bera &
\ind ok sęgja Níðhǫggvi niðr.\eva

\bvb Wratetusk is called the squirrel, who shall run at Ugdrassle’s Ash. The eagle’s words he shall carry from above, and say to Nithehew below.\evb
\evg


\bvg
\bva\mssnote{\Regius~10r/23, \AM~5r/11}Hirtir ’ru ok fjórir \hld\ þęir’s af hę́fingar &
\ind á gaghálsir gnaga: &
Dáinn ok Dvalinn, \hld\ Dúnęyrr ok Duraþrór.\eva

\bvb TODO\evb
\evg


\bvg
\bva\mssnote{\Regius~10r/25, \AM~5r/12, \GylfMS}Ormar flęiri \hld\ liggja und aski Yggdrasils &
\ind an þat of hyggi hvęrr ósviðra apa:\eva

\bvb More worms lie under Ugdrassle’s Ash than each unwise \inx[C]{ape} might ween:\evb
\evg


\bvg
\bva\mssnote{\Regius~10r/26, \AM~5r/13, \GylfMS}Góinn ok Móinn, \hld\ þęir ’ru Grafvitnis synir, &
\ind Grábakr ok Grafvǫlluðr, &
Ofnir ok Sváfnir, \hld\ hygg’k at ę́ skyli &
męiðs kvistu máa.\eva

\bvb Gowen and Mowen—they are Gravewitner’s sons—Greyback and Gravewalled; Ovner and Sweefner, I ween, shall always injure the branches of the beam \ken*{\textsc{tree} = Ugdrassle’s Ash}.\evb
\evg


\bvg
\bva\mssnote{\Regius~10r/28, \AM~5r/14}Askr Yggdrasils \hld\ drýgir ęrfiði &
\ind męira an męnn viti: &
Hjǫrtr bítr ofan \hld\ en á hliðu fúnar, &
\ind skęrðir Níðhǫggr neðan.\eva

\bvb Ugdrassle’s Ash suffers hardship greater than men might know: a hart bites it from above, but it rots on the side; Nithehew gnaws at it from below.\evb
\evg


\bvg
\bva\mssnote{\Regius~10r/30, \AM~5r/16}Hrist ok Mist \hld\ vil’k at mér horn beri, &
\ind Skeggjǫld ok Skǫgul, &
\edtrans{Hildr ok Þrúðr}{Hild and Thrith}{\Afootnote{so \AM; \emph{Hildi ok Þrúði} \Regius\ stems from \emph{ꝺꝛ, ðꝛ} with r rotunda being interpreted and copied as \emph{ꝺı, ðr}, this becomes clear upon viewing the facsimile images.}}, \hld\ Hlǫkk ok Hęrfjǫtur, &
\ind Gǫll ok Gęirǫlul, &
Randgríð ok Ráðgríð, \hld\ Ręginlęif; &
\ind þę́r bera ęinhęrjum ǫl.\eva

\bvb Rist and Mist I wish might bear to me a horn\footnoteB{i.e. for to drink out of.}—Shageld and Shagle; Hild and Thrith, Lank and Harfetter, Gall and Garalel; Randgrith, Redegrith and Rainlaf; they bear to the Ownharriers ale.\footnoteB{The women listed in this st. are Walkirries. TODO: Their names are known from other lists of Walkirries, but differ somewhat in form.}\evb
\evg


\bvg
\bva\mssnote{\Regius~10r/32, \AM~5r/18}Árvakr ok Alsviðr, \hld\ skulu upp heðan &
\ind svangir sól draga; &
en und þęira bógum \hld\ fǫ́lu blíð ręgin, &
\ind ę́sir, ísarnkol.\eva

\bvb Yorewaker and Allswith\footnoteB{These figures both appear in \Sigrdrifumal\ TODO. Along with the close formulation of the next st., it is clear that they are closely related.} shall above hence—slender [horses]—pull the sun; but under their shoulders hid the blithe Reins—the Ease—iron-coals.\footnoteB{According to \Gylfaginning\ 11 the gods took two horses to pull the sun’s chariot—Yorewaker and Allswith—and “under the shoulders of the horses the gods placed two wind-bags to cool them, but in some sources (\emph{í sumum frǿðum}, i.e. this st.) this is called iron-coals (\emph{ísarnkol}).”}\evb
\evg


\bvg
\bva\mssnote{\Regius~10v/2, \AM~5r/20}Svalinn hęitir, \hld\ hann stęndr sólu fyrir, &
\ind skjǫldr skínanda goði; &
bjǫrg ok brim \hld\ vęit’k at brinna skulu, &
\ind ef hann fęllr í frá.\eva

\bvb Swollen is [one] called, he stands before the sun; a shield [before] the shining god \ken{sun}. Crags and surf I know shall burn, if he falls away.\footnoteB{The sun-disc was apparently thought to be a translucent shield, which protected the earth from the full power of the Sun. Without it the whole world (“crags and surf”, \textsc{land} and \textsc{sea}; the totality of the earth) would burn up. The “shield that stands before the shining god \ken{sun}” is also mentioned in \Sigrdrifumal\ TODO.}\evb
\evg


\bvg
\bva\mssnote{\Regius~10v/4, \AM~5r/21}Skoll hęitir ulfr, \hld\ es fylgir hinu skírlęita &
\ind goði til varna viðar, &
en annarr Hati, \hld\ hann ’s Hróðvitnis sonr, &
\ind sá skal fyr hęiða brúði himins.\eva

\bvb \inx[P]{Skoll} is called the wolf, which follows the pure-skinned god \ken*{= Sun} to the protection of the woods; but another one [is called] \inx[P]{Hate}—he is \inx[P]{Rothwitner}’s son—that one shall [run] in front of the bright bride of heaven \ken*{= Sun}.\footnoteB{According to \Gylfaginning\ 12, which is probably based on this st., Skoll chases the sun but Hate chases the moon. See note to \Voluspa\ 40 for discussion on this.}\evb
\evg


\bvg
\bva\mssnote{\Regius~10v/6, \AM~5r/23, Lítla skálda (TODO)}Ór Ymis holdi \hld\ vas jǫrð of skǫpuð, &
\ind en ór svęita sę́r, &
bjǫrg ór bęinum, \hld\ baðmr ór hári, &
\ind en ór hausi himinn.\eva

\bvb Out of Yimer’s hull was the earth shaped, but out of his blood\footnoteB{\emph{svęiti}, while cognate with ModEngl. ‘sweat’, almost always carries the meaning of ‘blood’ in poetry. This is also the case with the OE cognate \emph{swát} (e.g. \Beowulf\ 1286a: \emph{sweord} swáte \emph{fáh} ‘sword stained with \emph{sweat}’, 2689b–2690: \emph{hé ge-blódegod wearð // sáwul-dríore;} \hld\ swát \emph{ýðum wéoll.} ‘he was bloodied in soul-gore; the \emph{sweat} gushed in waves’).} the seas; crags out of his bones, trees out of his hair, but out of his skull, heaven.\footnoteB{The understanding is of the heavens as a dome, something that fits well with the clouds being Yimer’s brains as mentioned in the following st.}\evb
\evg


\bvg
\bva\mssnote{\Regius~10v/8, \AM~5r/25, Lítla skálda (TODO)}En ór hans brǫ́um \hld\ gęrðu blíð ręgin &
\ind Miðgarð manna sonum, &
en ór hans hęila \hld\ vǫ́ru þau hin harðmóðgu &
\ind ský ǫll of skǫpuð.\eva

\bvb But out of his eyebrows the blithe \inx[G]{Reins} made \inx[L]{Middenyard} for the sons of men;\footnoteB{I agree with \textcite{FinnurEdda} in that this describes the gods fencing in Middenyard (‘the middle enclosure’) by using the hair of Yimer’s eyebrows as poles.} but out of his brains were the hard-stirred clouds all shaped.\evb
\evg


\bvg
\bva\mssnote{\Regius~10v/9, \AM~5r/26}Ullar \edtext{hylli}{\lemma{hylli ‘holdness’}\Bfootnote{i.e. ‘favour, loyalty, grace’. This word and its adjectival equivalent \emph{hollr} ‘hold; favourable, loyal, gracious’ are often used when speaking about divine grace, not just in Christian texts, but likewise as here wrt. to the Heathen gods. See Encyclopedia for other examples.}} \hld\ hęfr ok allra goða &
\ind hvęrr’s tękr fyrstr á funa, &
því’t opnir hęimar \hld\ verða of ása sonum, &
\ind þá’s hęfja af hvera.\eva

\bvb The \inx[C]{holdness} of \inx[P]{Woulder}—and of all the gods—has each who first touches the fire; for the \inx[C]{Home}[Homes] become open o’er the sons of the Ease, when the cauldrons are heaved off.\footnoteB{This st. is one of the most difficult in the poem, and many interpretations have been made (for a summary see \textcite{Nordberg2005}). \textcite{FinnurEdda} and Sijmons and Gering (p. 208, TODO) interpret this st. as relating to the frame narrative, so that Weden, still bound between the two fires, wishes for the gods to rescue him. This, however, scarcely makes sense given its placement in the gnomic wisdom section of the poem, unless the surrounding section is taken to be later “inserts”—this is Finnur’s solution, but there is no textual or internal support for it.
I believe instead (and here I agree with Nordberg) that the st. refers to the cooking and eating of sacred stew in large cauldrons during the \inx[C]{bloot}, and Woulder’s role in the setting of the ritual fire (see Encyclopedia and \parencite{afEdholm2009}). This interpretation is especially interesting in that this st. immediately follows 41–42, which deal with the ordering of the world through the dismembering of the primordial sacrificial victim Yimer. It is well attested comparatively (see \parencite{Lincoln1986}—especially the first two chapters—for its Indo-European analogues) that the ritual sacrifice in the present was seen as a reenactment and continuation of the gods’ creation of the world in the mythic past through the previously mentioned primordial sacrifice—these three sts. would seem to attest this view also in the Germanic tradition.}\evb
\evg


\bvg
\bva\mssnote{\Regius~10v/11, \AM~5r/28}Ívalda synir \hld\ gingu í árdaga &
\ind Skíðblaðni at skapa, &
skipa bazt \hld\ skírum Fręy, &
\ind nýtum Njarðar bur.\eva

\bvb The sons of Iwald went—in days of yore—Shidebladner to shape: the best of ships for the pure Free; for the useful son of Nearth \ken*{= Free}.\evb
\evg


\bvg
\bva\mssnote{\Regius~10v/13, \AM~5r/29}Askr Yggdrasils, \hld\ hann ’s ǿztr viða &
\ind en Skíðblaðnir skipa, &
Óðinn ása \hld\ en jóa Slęipnir, &
Bilrǫst brúa \hld\ en Bragi skalda, &
Hábrók hauka \hld\ en hunda Garmr.\eva

\bvb Ugdrassle’s Ash, that is the noblest of trees, but Shidebladner of ships; Weden of the Ease, but of horses Slopner; Bilrest of bridges, but Bray of scolds; Highbrook of hawks, but of hounds Garm.\evb
\evg


\bvg
\bva\mssnote{\Regius~10v/15, \AM~5v/2}Svipum hęf’k nú ypt \hld\ fyr sigtíva sonum, &
\ind við þat skal vilbjǫrg vaka, &
ǫllum ǫ́sum \hld\ þat skal inn koma &
\ind Ę́gis bękki á &
\ind Ę́gis drekku at.\eva

\bvb My gaze have I now lifted up before the sons of the victory-Tews \ken*{= Ease}; by that shall the willed rescue awake.\footnoteB{Weden has made the Ease aware of his identity, and thus they will come to help him.} All the Ease shall it bring in, on Eagre’s bench, at Eagre’s drinking.\evb
\evg


\bvg
\bva\mssnote{\Regius~10v/17, \AM~5v/4}Hétumk Grímr, \hld\ hétumk Gangleri, &
\ind Herjann ok Hjalmberi, &
Þękkr ok Þriði, \hld\ Þundr ok Uðr, &
\ind Hęlblindi ok Hár.\eva

\bvb I called myself Grim, I called myself Gangler; Harn and Helmbearer. Theck and Third, Thound and Ith, Hellblind and High.\evb
\evg


\bvg
\bva\mssnote{\Regius~10v/19, \AM~5v/5}Saðr ok Svipall \hld\ ok Sanngetall, &
\ind Hęrtęitr ok Hnikarr, &
Bilęygr, Bálęygr, \hld\ Bǫlverkr, Fjǫlnir, &
Grímr ok Grímnir, \hld\ Glapsviðr ok Fjǫlsviðr.\eva

\bvb Sooth and Swiple, and Soothgettle; Hartat and Nicker. Bileye, Baleeye, Baleworker, Fillner, Grim and Grimner, Glapswith and Fellswith.\evb
\evg


\bvg
\bva\mssnote{\Regius~10v/21, \AM~5v/7}Síðhǫttr, Síðskęggr, \hld\ Sigfǫðr, Hnikuðr, &
Alfǫðr, Valfǫðr, \hld\ Atríðr ok Farmatýr; &
ęinu nafni \hld\ hétumk aldrigi &
\ind síz ek með folkum fór.\eva

\bvb Sidehat, Sideshag, Sighfather, Nicked, Allfather, Walfather, Atrider and Farm-Tew; by one name I never called myself, since among men I fared.\evb
\evg


\bvg
\bva\mssnote{\Regius~10v/23, \AM~5v/9}Grímni mik hétu \hld\ at Geirraðar &
\ind en Jálk at Ǫ́smundar &
en þá Kjalar \hld\ es ek kjalka dró, &
\ind Þrór þingum at.\eva

\bvb Grimner they called me at Garred’s [estate], but Yelk at Osmunds. But Keller then, as I drew the sled; Throo at \inx[C]{Thing}[Things].\footnoteB{Presumably referencing other now-lost myths involving Weden travelling in disguise. The last is possibly a reference to the name under which Weden would be invoked at the start of Things (legal assemblies, see Encyclopedia).}\evb
\evg


\bvg
\bva\mssnote{\Regius~10v/24, \AM~5v/10}Óski ok Ómi, \hld\ Jafnhár ok Biflindi, &
\ind Gǫndlir ok Hárbarðr með goðum.\eva

\bvb TODO\evb
\evg


\bvg
\bva\mssnote{\Regius~10v/25, \AM~5v/11}Sviðurr ok Sviðrir \hld\ es ek hét at Søkkmímis &
\ind ok dulða’k þann hinn aldna jǫtun &
þá’s ek Miðviðnis vas’k \hld\ ins mę́ra burar &
\ind orðinn ęin-bani.\eva

\bvb TODO\evb
\evg


\bvg
\bva\mssnote{\Regius~10v/28, \AM~5v/13}Ǫlr est Gęirrøðr, \hld\ hęfr þú of drukkit; &
miklu est hnugginn, \hld\ es þú est mínu gęngi, &
ǫllum ęinhęrjum \hld\ ok Óðins hylli.\eva

\bvb Worse for ale art thou, Garfrith; thou hast drunk too much. Of much art thou bereft when thou art [bereft] of my support; of all the Ownharriers, and of Weden’s \inx[C]{holdness}.\footnoteB{Linguistically, Garfrith is not bereft of the support of the Ownharriers but rather of the Ownharriers themselves, but presumably the sense is the same. By breaking the code of conduct to which he owns his success he lost Weden’s favour, and thus been excluded from the community of oath-bound Odinic warriors (the Ownharriers). Cf. here}\evb
\evg


\bvg
\bva\mssnote{\Regius~10v/30, \AM~5v/15}Fjǫlð þér sagða’k, \hld\ en þú fátt of mant, &
\ind of þik véla vinir; &
mę́ki liggja \hld\ sé’k míns vinar &
\ind allan í dręyra drifinn.\eva

\bvb Much I said to thee, but thou recallest little; ’tis friends that deal with thee! The sword I see, of my friend, lying all drenched in gore.\footnoteB{Weden expresses his disappointment in Garfrith’s conduct and predicts his imminent death.}\evb
\evg


\bvg
\bva\mssnote{\Regius~10v/31, \AM~5v/16}Ęggmóðan val \hld\ nú mun Yggr hafa, &
\ind þitt vęitk líf of liðit; &
varar ro dísir, \hld\ nú knátt Óðin séa; &
\ind nálgask mik ef þú męgir.\eva

\bvb An edge-tired corpse will Ug now have; I know thy life to be passed. Wary are the dises; now thou dost see Weden—approach me, if thou mayst!\evb
\evg


\bvg
\bva\mssnote{\Regius~11r/2, \AM~5v/18}Óðinn nú hęiti’k, \hld\ Yggr áðan hét’k, &
\ind hétumk Þundr fyr þat, &
Vakr ok Skilfingr, \hld\ Vǫ́fuðr ok Hroptatýr &
\ind Gautr ok Jalkr með goðum.\eva

\bvb Weden I am now called, Ug was I earlier called; I called myself Thound before that. Wacker and Shelfing, Waved and Roft-Tew, Geat and Gelding among the gods.\evb
\evg


\bvg
\bva\mssnote{\Regius~11r/4, \AM~5v/20}Ofnir ok Sváfnir \hld\ hygg’k at orðnir sé &
\ind allir at ęinum mér.\eva

\bvb Ovner and Sweefner, I ween, are become all for me alone.\evb
\evg


\bpg
\bpa\mssnote{\Regius~11r/5, \AM~5v/21}Geirröðr konungr sat ok hafði sverð um kné sér ok brugðit til miðs. En er hann heyrði at Óðinn var þar kominn stóð hann upp ok vildi taka Óðin frá eldinum. Sverðit slapp ór hendi hánum; vissu hjöltin niðr. Konungr drap fę́ti ok steyptiz áfram en sverðit stóð í gögnum hann ok fekk \edtext{hann}{\Afootnote{þar af \AM}} bana. \edtext{Óðinn hvarf þá.}{\Afootnote{\emph{om.} \AM}} En Agnarr \edtext{var þar}{\Afootnote{varð \AM}} konungr \edtext{lengi síðan.}{\Afootnote{\emph{om.} \AM}}\epa

\bpb King Garfrith sat and had a sword about his knee, and it was brandished half-way up. But when he heard that Weden were come there, he stood up and wanted to take Weden from the fire. The sword slipped out of his hand; the hilt pointed downwards. The king tripped and threw himself forth, but the sword went through him, and he received his bane. Weden then disappeared, but Eyner was there king for a long while afterwards.\epb
\epg
