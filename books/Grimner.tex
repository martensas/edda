\bookStart{The Speeches of Grimner}[Grímnismǫ́l]

\begin{flushright}%
Dating \parencite{Sapp2022}: C10th (0.976)

Meter: \Ljodahattr, \Fornyrdislag\ (2/3–4, 28/3–5, 45/3–5, 48/4, 49/1–2, 53), \Galdralag\ (46)%
\end{flushright}

% Introduction

The \textbf{Speeches of Grimner} are preserved whole in both \Regius\ and \AM.

The poem itself is surrounded by two long introductory prose narratives containing some very old motifs, which are here brought up in the notes. It’s hard to say for how long these texts have accompanied the poem (TODO: I may write about this in the Introduction, since this question is important for several other poems), but since they are found in both \Regius\ and \AM\ and contain these motifs it would seem that they are fairly old. Together with sts. 1–3 they form a frame narrative that gives additional meaning to the gnomic sts. enclosed within.

The gnomic sts. themselves, the meat of the poem, are mythological and often quite obscure. In this they align closely with other Eddic gnomic poems such as \Havamal, \Vafthrudnismal, \Sigrdrifumal, and \Allvismal.

Weden begins by listing the halls of the gods (4–17). This section has been discussed in detail by \textcite{deVries1952} TODO! who considers it corrupt. Specifically, he sees the second half of v. 4 as a later insert, since it does not elaborate on the “holy land” mentioned in the first half. \textcite{Jackson1995} has argued convincingly against this, showing how the first half serves as a generalized introduction to the list; the holy land is the dwelling-places of the gods.

After this list come several sts relating to Weden and his hall, Walhall (18–23). Mentioned are the preparation of food in Walhall (18), Weden’s wolves (19) and ravens (20), the river through which the dead have to wade (21) and the gate through which they have to pass (22), the count of doors in Walhall (23), the count of doors in Thunder’s hall Bilshirner (24), and two animals which stand on the hall and gnaw on the branches of the tree Leered (25–26). From the latter animal’s—the stag Oakthirner’s—horns droplets fall into Wharyelmer, which is the origin of all rivers (26).

This introduces a list of mythic rivers (27–28), ending with the waters through which Thunder must wade on his way to Ugdrassle (29). This leads to a list of the horses ridden by the other gods on their way to Ugdrassle (31) which is followed by a description of the roots of Ugdrassle (31), then its animals (32–36) the Walkirries (37), and beings associated with the sun and moon (38–40), the things created from Yimer’s body (41–42) with a digression on the significance of the \inx[P]{bloot} for men in the present (43, see note there!), the creation of the ship Shidebladner (44) and finally a list of the noblest of several categories of things and groups (45).

After these lists Weden utters an unclear st. invoking the gods (46), before listing many of his names and the circumstances in which they were used (47–50). He then turns to Garfrith, disappointed by the inhospitality and poor conduct of his former protégé, and predicts his imminent death (51–53). He finally reveals himself by his true name, daring Garfrith to face him (53). After this he repeats several of his names (54), and the poem ends.

In the final prose section we are told that Garfrith, after learning that he was torturing Weden, hurried up to take the god away from the fires, but tripped and fell on his sword and died. After this his son Eyner ruled for a long time.

\sectionline

\section{From the sons of king Reeding (\emph{Frá sonum Hrauðungs konungs})}

\bpg
\bpa[1a]\mssnote{\Regius~8v/31, \AM~3v/23}Hrauðungr konungr átti tvá sonu. Hét annarr Agnarr, enn annarr Geirrøðr.
Agnarr var tíu vetra enn Geirrøðr átta vetra. Þeir reru tveir á báti með dorgar sínar at smáfiski.
Vindr rak þá í haf út. Í náttmyrkri brutu þeir við land ok gingu upp; fundu kotbónda einn.
Þar vǫ́ru þeir um vetrinn. Kerling fostraði Agnar enn karl Geirrøð.
At vári fekk karl þeim skip. Enn er þau kerling leiddu þá til strandar, þá mę́lti karl einmę́li við Geirrøð.
Þeir fengu byr ok kvǫ́mu til stǫðva fǫður síns. Geirrøðr var fram í skipi.
Hann hljóp upp á land enn hratt út skipinu, ok mę́lti: ”Far þú þar er smyl hafi þik.”
Skipit rak út. Enn Geirrøðr gekk út til bǿjar; hánum var vel fagnat; þá var faðir hans andaðr.
Var þá Geirrøðr til konungs tekinn, ok varð maðr ágę́tr.\epa

\bpb King Reeding owned two sons. One was called Eyner, and the other Garfrith.
Eyner was ten winters old, and Garfrith eight winters. The two were rowing in a boat with their trolling-lines for small fishing.
The wind then drove them out into the sea. In the dark of night they crashed into land and walked up; they found a lone cottage-farmer.
There they were over the winter. The wife fostered Eyner, but the husband Garfrith.\footnoteB{The wife was Frie, and the husband Weden; this is clarified by the following prose. The motif of Weden preferring the youngest brother is also found in \Rigsthula.}
In the spring the husband gave them ships, but when they followed the farmer’s wife in leading them to the shore, the husband spoke privately with Garfrith.\footnoteB{Surely instructing him to push his brother out to sea.}
They got a good gust, and came to their father’s harbour. Garfrith was in the front of the ship.
He leapt up onto land and pushed out the ship, and spoke: ”Go thou whither the fiends may have thee!”
The ship drove out. But Garfrith walked towards the farm; he was welcomed well; by then his father was passed-on.
Then Garfrith was taken as king, and became an excellent man.\epb
\epg


\bpg
\bpa[1b]\mssnote{\Regius~9r/10, \AM~4r/3}Óðinn ok Frigg sátu í Hliðskjǫlfu ok sá um heima alla.
Óðinn mę́lti: Sér þú Agnar fóstra þinn, hvar hann elr bǫrn við gýgi í hellinum?
En Geirrøðr, fóstri minn, er konungr ok sitr nú at landi.
Frigg segir: Hann er matníðingr sá at hann kvelr gesti sína ef hánum þykkja ofmargir koma.
Óðinn segir at þat er in mesta lygi. Þau veðja um þetta mál.
Frigg sendi eskismey sína, Fullu, til Geirrøðar. Hon bað konung varask at eigi fyrgerði hánum fjǫlkunnigr maðr sá er þar var kominn í land ok sagði þat mark á at engi hundr var svá ólmr at á hann myndi hlaupa.
En þat var inn mesti hégómi at Geirrøðr vę́ri eigi matgóðr ok þó lę́tr hann handtaka þann mann er eigi vildu hundar á ráða.
Sá var í feldi blám ok nefndisk Grímnir ok sagði ekki fleira frá sér þótt hann vę́ri atspurðr.
Konungr lét hann pína til sagna ok setja milli elda tveggja ok sat hann þar átta nę́tr.
Geirrøðr konungr átti son tíu vetra gamlan ok hét Agnarr eftir bróður hans.
Agnarr gekk at Grímni ok gaf hánum horn fullt at drekka, sagði að konungr gerði illa er hann lét pína hann saklausan.
Grímnir drakk af. Þá var eldrinn svá kominn at feldrinn brann af Grímni. Hann kvað:\epa

\bpb Weden and Frie sat in \inx[L]{Lithshelf} and looked over all the Homes.\footnoteB{Very similar to the Longbeard Origin Myth (TODO: reference and elaborate).}
Weden spoke: “Seest thou Eyner, thy foster-son, where he begets children with the troll-woman in the cave?\footnoteB{This may relate to Frie’s role as love-goddess. Eyner is in any case a \inx[C]{degenerate} man, what one would call a ‘coomer’.}
But Garfrith, my foster-son, is king and now sits at land.”
Frie says: “He is such a meat-nithing that he tortures his guests if he judges too many are coming.”
Weden says that this is the greatest lie; they make a wager about this matter.
Frie sent her handmaid Full to Garfrith’s. She bade the king be wary, that he not be ended by that \inx[C]{many-cunning} man who was come in the land, and said that his sign was that no hound was so fierce that he would leap at him.
But that was the greatest vainglory that Garfrith were not meat-good, and yet he has that man seized, whom the hounds would not touch.
He was clad in a blue cloak, and called himself Grimner, and did not tell any more about himself, even though he was interrogated.
The king had him tortured that he would speak, and set him between two fires, and he sat there for eight nights.
King Garfrith had a son ten winters old, and he was named Eyner after his brother.
Eyner walked up to Grimner, and gave him a full horn to drink, saying that the king did ill as he had him tortured without cause.
Grimner drank from it. Then the fire had come such that the cloak burned on Grimner. He quoth:\epb
\epg\stepcounter{prosea}

\sectionline

\bvg
\bva\mssnote{\Regius~9r/27, \AM~4r/17}\alst{H}ęitr est \alst{h}ripuðr \hld\ ok \alst{h}ęldr til mikill, &
\ind gǫngumk \alst{f}irr \alst{f}uni! &
\alst{L}oði sviðnar, \hld\ þótt á \alst{l}opt bera’k; &
\ind brinnumk \alst{f}eldr \alst{f}yrir.\eva

\bvb Hot art thou, flame, and rather too large; \\
go far from me, fire! \\
The woolen cape is singed though I hold it aloft; \\
the cloak burns before me!\evb
\evg


\bvg
\bva\mssnote{\Regius~9r/29, \AM~4r/18}\alst{Á}tta nę́tr \hld\ sat’k milli \alst{ę}lda hér, &
\ind svá’t mér \alst{m}ann-gi \alst{m}at né bauð &
nema \alst{ęi}nn Agnarr, \hld\ es \alst{ęi}nn skal ráða, &
\alst{G}ęirrøðar sonr, \hld\ \alst{G}otna landi.\eva

\bvb For eight nights sat I in the middle of the fires here, \\
while no man offered me food; \\
save for Eyner alone, who alone shall rule— \\
Garfrith’s son—the land of the Gots!\evb
\evg


\bvg
\bva\mssnote{\Regius~9r/31, \AM~4r/20}\alst{H}ęill skalt, Agnarr, \hld\ alls \alst{h}ęilan biðr &
\ind þik \alst{V}era-týr \alst{v}esa; &
\alst{ęi}ns drykkjar \hld\ skalt aldri-gi &
\ind bętri \alst{g}jǫld \alst{g}eta:\eva

\bvb Hale shalt thou [be], O Eyner, as hale \\
Were-Tew \name{= Weden} bids thee be; \\
for a single drink shalt thou never get \\
a better recompense:\footnoteB{The recompense being the esoteric lore which is told from the following st. onwards.}\evb
\evg

\sectionline

\bvg
\bva\mssnote{\Regius~9r/33, \AM~4r/22}\alst{L}and es hęilagt, \hld\ es \alst{l}iggja sé’k &
\ind \alst{ǫ́}sum ok \alst{ǫ}lfum nę́r; &
en í \alst{Þ}rúð-hęimi \hld\ skal \alst{Þ}órr vesa &
\ind unds of \alst{r}júfask \alst{r}ęgin.\eva

\bvb The land is holy, which I see lying \\
close to the \inx[F]{Ease and Elves}; \\
but in Thrithham shall Thunder be, \\
until the Reins are ripped.\evb
\evg


\bvg
\bva\mssnote{\Regius~9v/2, \AM~4r/23}\alst{Ý}-dalir hęita, \hld\ þar’s \alst{U}llr hęfir &
\ind \alst{s}ér of gǫrva \alst{s}ali; &
\alst{A}lf-hęim Fręy \hld\ gǫ́fu í \alst{á}r-daga &
\ind \alst{t}ívar at \alst{t}ann-féi.\eva

\bvb Yewdales are called where Woulder has \\
made for himself a hall. \\
Elfham to Free in days of yore \\
did the Tews as a tooth-gift\footnoteB{The gift that a child receives when he gets his first tooth.} give.
\evg


\bvg
\bva\mssnote{\Regius~9v/3, \AM~4r/25}\alst{B}ǿr ’s hinn þriði, \hld\ es \alst{b}líð ręgin &
\ind \alst{s}ilfri þǫkðu \alst{s}ali; &
\alst{V}ala-skjǫlf hęitir, \hld\ es \alst{v}élti sér &
\ind \alst{ǫ́}ss í \alst{á}r-daga.\eva

\bvb Bower is the third, where the blithe Reins \\
with silver thatched a hall. \\
Waleshelf is called [the hall] which the os in days of yore \\
won through wiles.\footnoteB{Several previous editors and translators (e.g. \textcite{FinnurEdda}, \textcite{PettitEdda}, \textcite{LarringtonEdda}) has rendered this phrase with variants of ‘craftily made for himself’ but I disagree.}\evb
\evg


\bvg
\bva\mssnote{\Regius~9v/5, \AM~4r/26}\alst{S}økkva-bękkr hęitir hinn fjórði, \hld\ en þar \alst{s}valar knegu &
\ind \alst{u}nnir glymja \alst{y}fir; &
þar þau \alst{Ó}ðinn ok Sága \hld\ drekka umb \alst{a}lla daga &
\ind \alst{g}lǫð ór \alst{g}ullnum kęrum.\eva

\bvb Sinkbench is called the fourth, but there do cool \\
waves clash over [it]; \\
there Weden and Sey drink all days, \\
glad, out of golden casks.\evb
\evg


\bvg
\bva\mssnote{\Regius~9v/7, \AM~4r/28}\alst{G}laðs-hęimr hęitir hinn fimti \hld\ þar’s hin \alst{g}ull-bjarta &
\ind \alst{V}al-hǫll \alst{v}íð of þrumir; &
en þar \alst{H}roptr \hld\ kýss \alst{h}vęrjan dag &
\ind \alst{v}ápn-dauða \alst{v}era.\eva

\bvb Gladsham is called the fifth, where the gold-bright \\
Walhall, wide, stands fast; \\
but there Roft \name{= Weden} chooses every day \\
weapon-dead men.\footnoteB{Cf. st. 14.}\evb
\evg


The order of the following two sts is that of \Regius. In \AM\ they come in the opposite order.


\bvg
\bva\mssnote{\Regius~9v/9, \AM~4r/31}Mjǫk ’s \alst{au}ð-kęnt \hld\ þęim’s til \alst{Ó}ðins koma &
\ind \edtrans{\alst{s}al-kynni at \alst{s}éa}{the hall to see}{\Afootnote{\emph{‘sia at sia’} \AM}}, &
\alst{v}argr hangir \hld\ fyr \alst{v}estan dyrr &
\ind ok drúpir \alst{ǫ}rn \alst{y}fir.\eva

\bvb Very easily recognized, for those who come to Weden, \\
is the hall to see: \\
A wolf hangs before the western door, \\
and an eagle droops over.\footnoteB{According to Hyltén-Cavallius (1863:156) it was custom to hang the bodies of dead wolves high up in old oaks, and dead birds of prey above the stable-door.}\evb
\evg


\bvg
\bva\mssnote{\Regius~9v/10, \AM~4r/30}Mjǫk ’s \alst{au}ð-kęnt \hld\ þęim’s til \alst{Ó}ðins koma &
\ind \alst{s}al-kynni at \alst{s}éa, &
\alst{sk}ǫptum ’s rann rępt, \hld\ \alst{sk}jǫldum ’s salr þakiðr, &
\ind \alst{b}rynjum of \alst{b}ękki stráat.\eva

\bvb Very easily recognized, for those who come to Weden, \\
is the hall to see: \\
With spear-shafts is the house roofed; with shields is the hall thatched; \\
with byrnies the benches strewn.\evb
\evg



\bvg
\bva\mssnote{\Regius~9v/12, \AM~4v/2}\alst{Þ}rym-hęimr hęitir hinn sétti, \hld\ es \alst{Þ}jatsi bjó, &
\ind sá hinn \edtrans{\alst{á}m-átki \alst{jǫ}tunn}{terrifying ettin}{\Bfootnote{Formulaic. See note to \Voluspa\ 8.}}; &
en nú \alst{Sk}aði byggvir, \hld\ \alst{sk}ír brúðr goða, &
\ind \alst{f}ornar toptir \alst{f}ǫður.\eva

\bvb Thrimham is called the sixth, where Thedse dwelled, \\
that terrifying ettin; \\
but now Shede bedwells—pure bride of the gods— \\
her father’s ancient plots.\evb
\evg


\bvg
\bva\mssnote{\Regius~9v/14, \AM~4v/3}\alst{B}ręiða-\alst{b}lik eru hin sjaundu, \hld\ en þar \alst{B}aldr hęfir &
\ind \alst{s}ér of gǫrva \alst{s}ali, &
á því \alst{l}andi \hld\ es \alst{l}iggja vęit’k &
\ind \alst{f}ę́sta \alst{f}ęikn-stafi.\eva

\bvb Broadblicks are the seventh, and there Balder has \\
made for himself a hall; \\
on that land, where I know lie \\
the fewest staves of treachery.\footnoteB{Evil, false words.}\evb
\evg


\bvg
\bva\mssnote{\Regius~9v/16, \AM~4v/5}\alst{H}imin-bjǫrg eru hin ǫ́ttu \hld\ en þar \alst{H}ęim-dall &
\ind kveða \alst{v}alda \alst{v}éum. &
þar \alst{v}ǫrðr goða \hld\ drekkr í \alst{v}ę́ru ranni &
\ind \alst{g}laðr \alst{g}óða mjǫð.\eva

\bvb Heavenbarrows are the eighth, and there Homedall, \\
they say, wields over wighs. \\
There the ward of the gods \ken*{= Homedall} drinks in the tranquil house, \\
glad, the good mead.\evb
\evg


\bvg
\bva\mssnote{\Regius~9v/17, \AM~4v/6}\alst{F}olk-vangr es hinn níundi \hld\ en þar \alst{F}ręyja rę́ðr &
\ind \alst{s}essa kostum í \alst{s}al; &
\alst{h}alfan val \hld\ hon kýss \alst{h}vęrjan dag &
\ind en halfan \alst{Ó}ðinn \alst{á}.\eva

\bvb Folkwong is the ninth, and there Frow decides \\
the choice of seats in the hall; \\
half the slain she chooses each day, \\
but half does Weden own.\footnoteB{This st. is cited and closely paraphrased in \Gylfaginning\ 24. — The roots of \emph{kjósa val} ‘choose the slain’ are the same as those in \inx[C]{walkirrie} (\emph{val-kyrja} ‘chooser of the slain’), and as Frow is a prominent goddess this would surely make her the chief walkirrie.
This is paralleled by \Sorlathattr, where Frow assumes the name \inx[C]{Gandle} (\emph{Gǫndul}, a name attested in several lists of walkirries; see \Voluspa\ 30 and Notes) and incites the legendary never-ending Conflict of the Headnings (\emph{Hjaðningavíg}).
In spite of this parallel, there are good reasons to believe that the chief walkirrie was \inx[C]{Frie}, Weden’s wife.
First, one of the functions of the walkirries is to bear ale to the Ownharriers (\Grimnismal\ 37). This mirrors royal Germanic banquets attested in heroic poetry, where the host’s wife or daughter would pour ale to his retainers and guests (the so-called ‘lady with a mead cup’ ritual; see \textcite{Enright1996} and \textcite{Riseley2014}). As Weden’s wife, we would expect Frie to have this role.
Second, at Balder’s funeral as attested in \Gylfaginning\ (TODO. chapter number), Weden rides with Frie and the Walkirries, while Frow rides alone with her cats. If she were chief walkirrie, it is rather strange that she should not ride with them.
Third, there are two separate myths where Frie and Weden contend over the fates of armies and men. These are the prose introduction to the present poem and the Longbeard origin myth (for which see Introduction to the present poem).}\evb
\evg


\bvg
\bva\mssnote{\Regius~9v/19, \AM~4v/8}\alst{G}litnir ’s hinn tíundi; \hld\ hann ’s \alst{g}ulli studdr &
\ind ok \alst{s}ilfri þakðr it \alst{s}ama; &
en þar \alst{F}or-seti \hld\ byggir \alst{f}lęstan dag &
\ind ok \alst{s}vę́fir allar \alst{s}akir.\eva

\bvb Glitner is the tenth, it is supported by gold, and thatched with silver likewise; but there Forset dwells most of the day, and resolves\footnoteB{lit. ‘puts to sleep’.} all [legal] matters.\evb
\evg


\bvg
\bva\mssnote{\Regius~9v/21, \AM~4v/9}\alst{N}óa-tún eru hin ęlliptu \hld\ en þar \alst{N}jǫrðr hęfir &
\ind \alst{s}ér of gǫrva \alst{s}ali; &
\alst{m}anna þęngill \hld\ hinn \alst{m}ęins-vani &
\ind \alst{h}ǫ́-timbruðum \alst{h}ǫrgi rę́ðr.\eva

\bvb Nowetowns are the tenth, and there Nearth has \\
made for himself a hall. \\
The prince of men, the guileless one, \\
rules the high-timbered \inx[C]{harrow}.\footnoteB{Cf. \Vafthrudnismal\ 38.}\evb
\evg


\bvg
\bva\mssnote{\Regius~9v/23, \AM~4v/11}\edtrans{\alst{H}rísi vęx \hld\ ok \alst{h}ǫ́u grasi}{with brushwood and with tall grass grows}{\Bfootnote{Identical with \Havamal\ 117/6.}} &
\ind \alst{V}íðars land, \alst{v}iði, &
en þar \alst{m}ǫgr of lę́tsk \hld\ af \alst{m}ars baki &
\ind \alst{f}rǿkn at hęfna \alst{f}ǫður.\eva

\bvb With brushwood and with tall grass grows \\
\inx[P]{Wider}’s land, with forest; \\
but there the lad does vow from the back of his steed, \\
valiant, to avenge his father.\footnoteB{Wider declares that he will avenge his father, Weden, which he later does at the Rakes of the Reins. See \Voluspa\ 54–55 and \Vafthrudnismal\ 53.}\evb
\evg


\bvg
\bva\mssnote{\Regius~9v/24, \AM~4v/12}\alst{A}nd-hrímnir \hld\ lę́tr í \alst{Ę}ld-hrímni &
\ind \alst{S}ę́-hrímni \alst{s}oðinn, &
\alst{f}lęska bętst, \hld\ en þat \alst{f}áir vitu, &
\ind við hvat \alst{ę}in-hęrjar \alst{a}lask.\eva

\bvb Andrimner lets in Eldrimner \\
Sowrimner be boiled. \\
The best of meats [is it], but few know that, \\
by what the Ownharriers are nourished.\footnoteB{The cook Andrimner ‘face-sooty’ has the boar Sowrimner ‘sow-sooty’ boiled in the cauldron Eldrimner ‘fire-sooty’; by this meat are the Ownharriers nouished.}\evb
\evg


\bvg
\bva\mssnote{\Regius~9v/26, \AM~4v/14}\alst{G}era ok Freka \hld\ sęðr \alst{g}unn-tamiðr, &
\ind \alst{h}róðigr \alst{H}ęrjafǫðr, &
en við \alst{v}ín ęitt \hld\ \alst{v}ápn-gǫfugr &
\ind \alst{Ó}ðinn \alst{ę́} lifir.\eva

\bvb Gar and Freck does the battle-accustomed, \\
famous Father of Hosts \name{= Weden} feed; \\
but on wine alone does the weapon-worshipful \\
Weden ever live.\evb
\evg


\bvg
\bva\mssnote{\Regius~9v/28, \AM~4v/15}\alst{H}uginn ok Muninn \hld\ fljúga \alst{h}vęrjan dag &
\ind \edtrans{\alst{jǫ}rmun-grund}{ermin-ground}{\Bfootnote{‘the immense ground’ (for the rare prefix \inx[C]{ermin-} see Encyclopedia), denoting the earth as a vast flat expanse of land. This compound also occurs in a kenning in the st. on the late C10th Karlevi stone (Öl 1) referring to the unbounded sea as \emph{Ęndils jǫrmungrund} ‘Andle’s ermin-ground’ (Andle being a known “sea-king”), and in \Beowulf\ 859 as \emph{eormen-grund} carrying the same sense.}} \alst{y}fir; &
\alst{ó}umk of Hugin, \hld\ at \alst{a}ptr né komi-t; &
\ind þó séumk \alst{m}ęir of \alst{M}unin.\eva

\bvb Highen and Minden fly every day \\
over the ermin-ground \ken{earth}. \\
I worry for Highen, that he should not come back; \\
yet I fear more for Minden.\evb
\evg


\bvg
\bva\mssnote{\Regius~9v/30, \AM~4v/17}\alst{Þ}ýtr \alst{Þ}und, \hld\ unir \edtext{\alst{Þ}jóð-vitnis &
\ind fiskr}{\lemma{Þjóðvitnis fiskr ‘Thedwitner’s fish’}\Bfootnote{\emph{Þjóðvitnir} is easily analyzed as \emph{þjóð-} ‘great, main’ + \emph{vitnir} ‘wolf’. The great wolf is naturally the \inx[P]{Fenrerswolf}, and its “fish” should then be the Middenyardsworm. That it could indeed be called a fish is proven by \Hymiskvida\ 24, where the word does not even carry alliteration.}} flóði í; &
\alst{á}ar-straumr \hld\ þykkir \alst{o}f-mikill &
\ind \alst{v}al-glaumi at \alst{v}aða.\eva

\bvb \inx[P]{Thound} roars; thrives Thedwitner’s \\
fish \ken*{= Middenyardsworm?} in the flood; \\
the river-stream seems far too great \\
for the noisy slain host to wade.\footnoteB{Thound may be the river surrounding Walhall, which the dead have to pass over to reach the hall. This stanza may also be referring to the punishment of men in waters; see note to \Voluspa\ TODO for discussion on that.}\evb
\evg


\bvg
\bva\mssnote{\Regius~9v/32, \AM~4v/18}\alst{V}al-grind hęitir \hld\ es stęndr \alst{v}ęlli á &
\ind \alst{h}ęilǫg fyr \alst{h}ęlgum durum; &
\alst{f}orn ’s sú grind, \hld\ en þat \alst{f}áir vitu, &
\ind hvé hǫ́n ’s í \alst{l}ás of \alst{l}okin.\eva

\bvb \inx[L]{Walgrind}\footnoteB{‘Corpse-gate;’ the gate guarding Walhall.} ’tis called, which stands on the plain, \\
holy, before holy doors. \\
Ancient is that gate, but few know that, \\
how its lock is locked.\evb
\evg


\bvg
\bva\mssnote{\Regius~9v/34, \AM~4v/22}\alst{F}imm hundruð golfa \hld\ ok umb \alst{f}jórum tøgum &
\ind svá hygg’k \alst{B}il-skirni með \alst{b}ugum; &
\alst{r}anna þęira, \hld\ es \alst{r}ępt vita’k, &
\ind \alst{m}íns vęit’k męst \alst{m}agar.\eva

\bvb Having five hundred floors, and around fourty, \\
so I judge \inx[L]{Bilshirner} altogether. \\
Of those houses, which I might know rafted, \\
I know my lad’s \ken*{= Thunder} to be the greatest.\evb
\evg


\bvg
\bva\mssnote{\Regius~10r/2, \AM~4v/20}\alst{F}imm hundruð dura \hld\ ok umb \alst{f}jórum tøgum, &
\ind svá hygg at \alst{V}alhǫllu \alst{v}esa; &
\alst{á}tta hundruð \alst{Ę}in-hęrja \hld\ ganga ór \alst{ęi}num durum, &
\ind þá’s fara við \alst{v}itni at \alst{v}ega.\eva

\bvb Five hundred doors, and around fourty, \\
so I judge there to be on Walhall. \\
Eight hundred \inx[G]{Ownharriers} go out of one door,\footnoteB{The hundred is probably here the long hundred (120, rather than 100), which gives a sum of \(640 * 960 = 614,400\) Ownharriers.} \\
when to fight with the wolf they go.\evb
\evg


\bvg
\bva\mssnote{\Regius~10r/4, \AM~4v/24}\alst{H}ęið-rún hęitir gęit, \hld\ es stęndr \edtrans{\alst{h}ǫllu á}{on the hall}{\Afootnote{\emph{hǫllu á Hęrja-fǫðrs} ‘on the Father of Host’s hall’ \Regius\AM\ is unmetrical, and likely added by a later redactor as clarification.}} &
\ind ok bítr af \alst{L}ę́-raðs \alst{l}imum; &
\alst{sk}ap-kęr fylla \hld\ skal hins \alst{sk}íra mjaðar, &
\ind kná-at sú \alst{v}ęig \alst{v}anask.\eva

\bvb Heathrune is called the goat who stands on the hall \ken*{= Walhall}, \\
and bites off Leered’s branches. \\
The shape-vats\footnoteB{According to \CV\ the central beer-vat, from which drinks were poured into smaller vessels.} shall she fill with the pure mead; \\
those draughts cannot wane.\footnoteB{The mead is the goat’s milk.}\evb
\evg


\bvg
\bva\mssnote{\Regius~10r/6, \AM~4v/26}Ęik-þyrnir hęitir \alst{h}jǫrtr \hld\ es stęndr \edtrans{\alst{h}ǫllu á}{on the hall}{\Afootnote{\emph{á hǫllu Hęrja-fǫðrs} ‘on the Father of Host’s hall’ \Regius\AM. See note to previous st.}}&
\ind ok bítr af \alst{L}ę́-raðs \alst{l}imum; &
en af hans \alst{h}ornum \hld\ drýpr í \alst{H}ver-gęlmi &
\ind þaðan ęiga \alst{v}ǫtn ǫll \alst{v}ega:\eva

\bvb Oakthirner is called the stag who stands on the hall \ken*{= Walhall}, \\
and bites off Leered’s branches. \\
But from his horns does drip into Wharyelmer; \\
thence have all waters their ways:\footnoteB{After which several vv. of mythic river-names are listed.}\evb
\evg


\bvg
\bva\mssnote{\Regius~10r/9, \AM~4v/28}\alst{S}íð ok Víð, \alst{S}ę́kin ok Ęikin, \hld\ \alst{S}vǫl ok Gunn-þró, &
\ind \alst{F}jǫrm ok \alst{F}imbul-þul, &
\ind \alst{R}ín ok \alst{R}innandi, &
\alst{G}ipul ok \alst{G}ǫpul, \hld\ \alst{G}ǫmul ok \alst{G}ęir-vimul, &
\ind þę́r \alst{h}verfa umb \alst{h}odd goða, &
\alst{Þ}yn ok Vin, \hld\ \alst{Þ}ǫll ok Hǫll, &
\ind \alst{G}rǫ́ð ok \alst{G}unn-þorin.\eva

\bvb Side and Wide, Seeken and Oaken, Swale and Guththrew, \\
Ferm and Fimblethule, \\
Rine and Rinnend, \\
Gipple, Gapple, Gamble and Garwimble \\
—they circle around the hoard of the gods \ken*{osyard}— \\
Thin and Win, Thall and Hall, \\
Grode and Guththorn.\evb
\evg


\bvg
\bva\mssnote{\Regius~10r/12, \AM~5r/1}\alst{V}ína hęitir enn, \hld\ ǫnnur \alst{V}eg-svinn, &
\ind \alst{þ}riðja \alst{Þ}jóð-numa; &
\alst{N}yt ok \alst{N}ǫt, \hld\ \alst{N}ǫnn ok \alst{H}rǫnn, &
\alst{S}líð ok \alst{H}ríð, \hld\ \alst{S}ylgr ok Ylgr, &
\alst{V}íð ok \alst{V}ǫ́n, \hld\ \alst{V}ǫnd ok Strǫnd, &
\alst{G}jǫll ok Lęiptr; \hld\ þę́r falla \alst{g}umnum nę́r &
\ind es falla til \alst{h}ęljar \alst{h}eðan. \eva

\bvb Wine is further called, another Wayswith, \\
a third Thednum; \\
Nit and Nat, Nan and Ran, \\
Slithe and Rithe, Sellow and Wellow, \\
Wide and Wane, Wand and Strand, \\
Yell and Laft; they fall near to men \\
as they fall hence to Hell.\evb
\evg


\bvg
\bva\mssnote{\Regius~10r/15, \AM~5r/4, \GylfMS}\alst{K}ǫrmt ok Ǫrmt \hld\ ok \alst{k}ęr-laugar tvę́r &
\ind \alst{þ}ę́r skal \alst{Þ}órr vaða &
\alst{d}ag hvęrn \hld\ es \alst{d}ǿma fęrr &
\ind at \alst{a}ski \alst{Y}gg-drasils; &
því-at \alst{ǫ́}s-brú \hld\ bręnn \alst{ǫ}ll loga &
\ind \alst{h}ęilǫg vǫtn \edtext{\alst{h}lóa}{\Bfootnote{A hapax. TODO.}}.\eva

\bvb Carmt and Armt, and the two Carlays, \\
those shall Thunder wade\footnoteB{For Thunder’s association with wading see TODO.} \\
every day when to judge he fares, \\
at \inx[L]{Ugdrassle’s ash}; \\
for the \inx[G]{ease}[os]-bridge \ken{rainbow} burns all with flame; \\
the holy waters bellow.\evb
\evg


\bvg
\bva\mssnote{\Regius~10r/17, \AM~5r/6}\alst{G}laðr ok \alst{G}yllir, \hld\ \alst{G}lęr ok Skęið-brimir, &
\ind \alst{S}ilfrin-toppr ok \alst{S}inir, &
\alst{G}ísl ok Fal-hófnir, \hld\ \alst{G}ull-toppr ok Létt-feti, &
\ind þęim ríða \alst{ę́}sir \alst{jó}um &
\alst{d}ag hvęrn \hld\ es \alst{d}ǿma fara &
\ind at \alst{a}ski \alst{Y}gg-drasils.\eva

\bvb Glad and Yiller, Glare and Sheathbrimmer, \\
Silvrentop and Sinewer, \\
Yissel and Fallowhofner, Goldtop and Lightfeet; \\
on those horses ride the Ease, \\
every day when to judge they fare, \\
at \inx[L]{Ugdrassle’s ash}.\evb
\evg


\bvg
\bva\mssnote{\Regius~10r/20, \AM~5r/8}\alst{Þ}ríar rǿtr \hld\ standa á \alst{þ}ría vega &
\ind undan \alst{a}ski \alst{Y}gg-drasils; &
\alst{H}ęl býr und \alst{ęi}nni, \hld\ \alst{a}nnarri \alst{h}rím-þursar, &
\ind þriðju \alst{m}ęnnskir \alst{m}ęnn.\eva

\bvb Three roots stand on three ways, \\
from beneath Ugdrassle’s Ash. \\
Hell lives under one, [under] the other the \inx[G]{Rime-Thurses}, \\
{[under]} the third manly men.\evb
\evg


\bvg
\bva\mssnote{\Regius~10r/22, \AM~5r/9}\alst{R}ata-toskr hęitir íkorni \hld\ es \alst{r}inna skal &
\ind at \alst{a}ski \alst{Y}gg-drasils; &
\alst{a}rnar \alst{o}rð \hld\ hann skal \alst{o}fan bera &
\ind ok sęgja \alst{N}íð-hǫggvi \alst{n}iðr.\eva

\bvb Wratetusk is called the squirrel who shall run \\
at Ugdrassle’s Ash. \\
The eagle’s words he shall carry from above, \\
and say to Nithehewer below.\footnoteB{This st. and the following is paraphrased in \Gylfaginning\ 16 (excerpt):
\begin{quote}
  \emph{Þa mę́lti Gangleri: „Hvat er fleira at segja stór-merkja frá askinum?“ Hár segir: „Mart er þar af at segia. Ǫrn einn sitr í limum asksins, ok er hann margs vitandi, en í milli augna honum sitr haukr sá, er heitir Veðrfǫlnir. Íkorni sá, er heitir Rata-toskr, rennr upp ok niðr eptir askinum ok berr ǫfundar orð millum arnarins ok Níðhǫggs.} ‘Gangler spoke: “What more great marks are there to be said about the ash?” High says: “There is much to say about it. An eagle sits in the limbs of the ash, and he is much knowing, but between his eyes sits the hawk called Weatherfalner. The squirrel, which is called Wratetush, runs up and down along the ash and carries words of spite between the eagle and Nithehewer.”’
\end{quote}}\evb
\evg


\bvg
\bva\mssnote{\Regius~10r/23, \AM~5r/11}\alst{H}irtir ’ru ok fjórir \hld\ þęir’s af \alst{h}ę́fingar &
\ind á \alst{g}ag-halsir \alst{g}naga: &
\alst{D}áinn ok \alst{D}valinn, \hld\ \alst{D}ún-ęyrr ok \alst{D}ura-þrór.\eva

\bvb Harts are there also, four, those who TODO gnaw: \\
Dowen and Dwollen, Downeer and Doorthrew.\footnoteB{Paraphrased in \Gylfaginning\ 16 immediately following a paraphrase of the last st.: \emph{En fjórir hirtir renna í limum asksins ok bíta barr; þeir heita svá: Dáinn, Dvalinn, Dún-eyrr, Dura-þrór.} ‘But four harts run in the limbs of the ash and bite its leaves; they are called thus: Dowen, Dwollen, Downeer, Doorthrew.’}\evb
\evg


\bvg
\bva\mssnote{\Regius~10r/25, \AM~5r/12, \GylfMS}\alst{O}rmar flęiri \hld\ liggja und \alst{a}ski \alst{Y}gg-drasils &
\ind an þat of \alst{h}yggi \alst{h}vęrr &
\ind \alst{ó}-sviðra \alst{a}pa:\eva

\bvb More worms lie under Ugdrassle’s Ash \\
than anyone would think \\
among unwise \inx[C]{ape}[apes]:\footnoteB{Paraphrased in \Gylfaginning\ 16: \emph{En svá margir ormar eru í Hvergelmi með Níðhǫgg, at engi tunga má telja; svá segir hér:} ‘But so many worms are in Wharyelmer with Nithehewer that no tongue may count them. So it says here:’ after which st. 36 is quoted.}\evb
\evg


\bvg
\bva\mssnote{\Regius~10r/26, \AM~5r/13, \GylfMS}\alst{G}óinn ok Móinn, \hld\ þęir ’ru \alst{G}raf-vitnis synir, &
\ind \alst{G}rá-bakr ok \alst{G}raf-vǫlluðr, &
\alst{O}fnir ok Sváfnir, \hld\ hygg’k at \alst{ę́} skyli &
\ind \alst{m}ęiðs kvistu \alst{m}áa.\eva

\bvb Gowen and Mowen—they are Gravewitner’s sons— \\
Greyback and Gravewalled; \\
Ovner and Sweefner, I ween, shall always \\
injure the beam’s branches.\evb
\evg


\bvg
\bva\mssnote{\Regius~10r/28, \AM~5r/14}\alst{A}skr \alst{Y}gg-drasils \hld\ drýgir \alst{ę}rfiði &
\ind \alst{m}ęira an \alst{m}ęnn viti: &
\alst{h}jǫrtr bítr ofan \hld\ en á \alst{h}liðu fúnar, &
\ind skęrðir \alst{N}íð-hǫggr \alst{n}eðan.\eva

\bvb Ugdrassle’s Ash suffers hardship \\
greater than men might know: \\
a hart bites it from above, but it rots on the side; \\
Nithehewer gnaws at it from below.\evb
\evg


\bvg
\bva\mssnote{\Regius~10r/30, \AM~5r/16}\alst{H}rist ok Mist \hld\ vil’k at mér \alst{h}orn beri, &
\ind \alst{Sk}eggj-ǫld ok \alst{Sk}ǫgul, &
\edtrans{\alst{H}ildr ok Þrúðr}{Hild and Thrith}{\Afootnote{so \AM; \emph{Hildi ok Þrúði} \Regius\ stems from \emph{ꝺꝛ, ðꝛ} with r rotunda being interpreted and copied as \emph{ꝺı, ðr}, this becomes clear upon viewing the facsimile images.}}, \hld\ \alst{H}lǫkk ok \alst{H}ęr-fjǫtur, &
\ind \alst{G}ǫll ok \alst{G}ęir-ǫlul, &
\alst{R}and-gríð ok \alst{R}áð-gríð, \hld\ \alst{R}ęgin-lęif; &
\ind þę́r bera \alst{ęi}n-hęrjum \alst{ǫ}l.\eva

\bvb Rist and Mist I would have bearing to me a horn\footnoteB{i.e. for to drink out of.}— \\
Shageld and Shagle, \\
Hild and Thrith, Lank and Harfetter, \\
Gall and Garalel, \\
Randgrith and Redegrith, Rainlaf— \\
they bear to the Ownharriers ale.\footnoteB{The women listed in this st. are Walkirries. Their names are known from other lists of Walkirries, but differ somewhat in form. TODO: Note these differences}\evb
\evg


\bvg
\bva\mssnote{\Regius~10r/32, \AM~5r/18}\alst{Á}r-vakr ok \alst{A}l-sviðr, \hld\ skulu \alst{u}pp heðan &
\ind \alst{s}vangir \alst{s}ól draga; &
en und þęira \alst{b}ógum \hld\ fǫ́lu \alst{b}líð ręgin, &
\ind \alst{ę́}sir, \alst{í}sarn-kol.\eva

\bvb Yorewaker and Allswith\footnoteB{These horses also appear in \Sigrdrifumal\ 14a/2, immediately after the sun itself. See note to the next st.} shall above hence— \\
slender [steeds]—pull the sun; \\
but under their shoulders hid the blithe Reins \\
—the Ease—iron-cooling.\footnoteB{According to \Gylfaginning\ 11 the gods took two horses to pull the sun’s chariot—Yorewaker and Allswith—and “under the shoulders of the horses the gods placed two wind-bellows to cool them, but in some sources (\emph{í sumum frǿðum}, presumably this st.) they are called iron-cooling (\emph{ísarn-kol}).”}\evb
\evg


\bvg
\bva\mssnote{\Regius~10v/2, \AM~5r/20}\alst{S}valinn hęitir, \hld\ hann stęndr \alst{s}ólu fyrir, &
\ind \alst{sk}jǫldr \alst{sk}ínanda goði; &
\alst{b}jǫrg ok \alst{b}rim \hld\ vęit’k at \alst{b}rinna skulu, &
\ind ef hann \alst{f}ęllr í \alst{f}rá.\eva

\bvb Swollen is [one] called, he stands before the sun, \\
{[as]} a shield [before] the shining god \ken{sun}. \\
Crags and surf I know shall burn, \\
if he falls away.\footnoteB{The sun-disc was apparently thought to be a translucent shield, which protected the earth from the full power of the Sun behind it. Without it the whole world (“crags and surf”, \textsc{land} and \textsc{sea}; the totality of the earth) would burn up. In \Sigrdrifumal\ 14a/1 there is mention of the “shield that stands before the shining god \ken{sun}”, which may or may not derive from the present stanza.}\evb
\evg


\bvg
\bva\mssnote{\Regius~10v/4, \AM~5r/21}\alst{Sk}oll hęitir ulfr, \hld\ es fylgir hinu \alst{sk}ír-lęita &
\ind goði til \alst{v}arna \alst{v}iðar, &
en annarr \alst{H}ati, \hld\ hann ’s \alst{H}róð-vitnis sonr, &
\ind sá skal fyr \alst{h}ęiða brúði \alst{h}imins.\eva

\bvb \inx[P]{Skoll} is called the wolf, which follows the pure-faced \\
god \ken*{= Sun} to the protection of the woods; \\
but second is \inx[P]{Hate}—he is \inx[P]{Rothwitner}’s son— \\
that one shall [run] in front of the bright bride of heaven \ken*{= Sun}.\footnoteB{According to \Gylfaginning\ 12, which is probably based on this st., Skoll chases the sun, but Hate chases the moon (which is why he runs in front of the sun). See note to \Voluspa\ 40 for discussion on these wolves.}\evb
\evg


\bvg
\bva\mssnote{\Regius~10v/6, \AM~5r/23, Lítla skálda (TODO)}Ór \alst{Y}mis holdi \hld\ vas \alst{jǫ}rð of skǫpuð, &
\ind en ór \alst{s}vęita \alst{s}ę́r, &
\alst{b}jǫrg ór \alst{b}ęinum, \hld\ \alst{b}aðmr ór hári, &
\ind en ór \alst{h}ausi \alst{h}iminn.\eva

\bvb Out of Yimer’s hull was the earth shaped, \\
but out of his blood\footnoteB{\emph{svęiti}, while cognate with ModEngl. ‘sweat’, almost always carries the meaning of ‘blood’ in poetry. This is also the case with the OE cognate \emph{swát} (e.g. \Beowulf\ 1286a: \emph{sweord} swáte \emph{fáh} ‘sword stained with \emph{sweat}’, 2689b–2690: \emph{hé ge-blódegod wearð // sáwul-dríore;} \hld\ swát \emph{ýðum wéoll.} ‘he was bloodied in soul-gore; the \emph{sweat} gushed in waves’).} the seas; \\
crags out of his bones, trees out of his hair, \\
but out of his skull, heaven.\footnoteB{The understanding is of the heavens as a dome, something that fits well with the clouds being Yimer’s brains as mentioned in the following st.}\evb
\evg


\bvg
\bva\mssnote{\Regius~10v/8, \AM~5r/25, Lítla skálda (TODO)}En ór hans \alst{b}rǫ́um \hld\ gęrðu \alst{b}líð ręgin &
\ind \alst{M}ið-garð \alst{m}anna sonum, &
en ór hans \alst{h}ęila \hld\ vǫ́ru þau hin \alst{h}arð-móðgu &
\ind \alst{sk}ý ǫll of \alst{sk}ǫpuð.\eva

\bvb But out of his eyebrows the blithe \inx[G]{Reins} made \\
\inx[L]{Middenyard} for the sons of men;\footnoteB{I agree with \textcite{FinnurEdda} in that this describes the gods fencing in Middenyard (‘the middle enclosure’) by using the hair of Yimer’s eyebrows as poles.} \\
but out of his brains were the hard-stirred \\
clouds all shaped.\evb
\evg


\bvg
\bva\mssnote{\Regius~10v/9, \AM~5r/26}\alst{U}llar \edtext{hylli}{\lemma{hylli ‘holdness’}\Bfootnote{i.e. ‘favour, loyalty, grace’. This word and its adjectival equivalent \emph{hollr} ‘hold; favourable, loyal, gracious’ are often used when speaking about divine grace, not just in Christian texts, but likewise as here wrt. to the Heathen gods. See Encyclopedia for other examples.}} \hld\ hęfr ok \alst{a}llra goða &
\ind hvęrr’s tękr \alst{f}yrstr á \alst{f}una, &
því-at \alst{o}pnir hęimar \hld\ verða of \alst{á}sa sonum, &
\ind þá’s \alst{h}ęfja af \alst{h}vera.\eva

\bvb \inx[P]{Woulder}’s \inx[C]{holdness}—and that of all the gods— \\
has each who first touches the fire; \\
for the \inx[C]{Home}[Homes] become open o’er the sons of the Ease, \\
when the cauldrons are heaved off.\footnoteB{This st. is one of the most difficult in the poem, and many interpretations have been made (for a summary see \textcite{Nordberg2005}). Many commentors (e.g. \textcite{FinnurEdda} and Sijmons and Gering (p. 208, TODO)) interpret this st. as relating to the frame narrative, so that Weden, still bound between the two fires, is wishing for the gods to rescue him. This, however, scarcely makes sense given its placement in the gnomic wisdom section of the poem, unless the whole surrounding section is taken to be a later “insert” (as supposed by Finnur) but there is no textual or internal support for that.
I believe instead, agreeing with Nordberg, that the st. refers to the cooking and eating of sacred stew in large cauldrons during the \inx[C]{bloot}, and Woulder’s role in the setting of the ritual fire (see Encyclopedia: Woulder and \parencite{afEdholm2009}). This interpretation is especially interesting when one considers the preceding sts. 41–42, which deal with the ordering of the world through the dismembering of the primordial sacrificial victim Yimer. It is well attested comparatively (see \parencite{Lincoln1986}—especially the first two chapters—for its Indo-European analogues) that the ritual sacrifice in the present was seen as a reenactment and continuation of the gods’ creation of the world in the mythic past through the previously mentioned primordial sacrifice, and these three sts. would seem to attest this view also in the Germanic tradition.}\evb
\evg


\bvg
\bva\mssnote{\Regius~10v/11, \AM~5r/28}\alst{Í}valda synir \hld\ gingu í \alst{á}r-daga &
\ind \alst{Sk}íð-blaðni at \alst{sk}apa, &
\alst{sk}ipa batst \hld\ \alst{sk}írum Fręy, &
\ind \alst{n}ýtum \alst{N}jarðar bur.\eva

\bvb The sons of Iwald went in days of yore \\
Shidebladner for to shape: \\
the best of ships for the pure Free, \\
for the useful son of Nearth \ken*{= Free}.\evb
\evg


\bvg
\bva\mssnote{\Regius~10v/13, \AM~5r/29}\alst{A}skr \alst{Y}gg-drasils, \hld\ hann ’s \alst{ǿ}ðstr viða &
\ind en \alst{Sk}íð-blaðnir \alst{sk}ipa, &
\alst{Ó}ðinn \alst{á}sa \hld\ en \alst{jó}a Slęipnir, &
\alst{B}il-rǫst \alst{b}rúa \hld\ en \alst{B}ragi skalda, &
\alst{H}á-brók \alst{h}auka \hld\ en \alst{h}unda Garmr.\eva

\bvb Ugdrassle’s Ash, that is the noblest of trees, \\
but Shidebladner of ships; \\
Weden of the Ease, but of horses Slopner; \\
Bilrest of bridges, but Bray of scolds; \\
Highbrook of hawks, but of hounds Garm.\evb
\evg


\bvg
\bva\mssnote{\Regius~10v/15, \AM~5v/2}\alst{S}vipum hęf’k nú ypt \hld\ fyr \alst{s}ig-tíva sonum, &
\ind við þat skal \alst{v}il-bjǫrg v\alst{a}ka, &
\alst{ǫ}llum \alst{ǫ́}sum \hld\ þat skal \alst{i}nn koma &
\ind \alst{Ę́}gis bękki \alst{á} &
\ind \alst{Ę́}gis drekku \alst{a}t.\eva

\bvb My gaze have I now lifted up before the sons of the victory-Tews \ken*{= Ease}; \\
by that shall the willed rescue awake. \\
All the Ease shall it bring in, \\
on Eagre’s bench, \\
at Eagre’s drinking.\footnoteB{Weden suddenly announces that he has made the other gods aware of his identity. They will so leave their feasting at Eagre’s and instead come to help him.}\evb
\evg


\bvg
\bva\mssnote{\Regius~10v/17, \AM~5v/4}Hétumk \alst{G}rímr, \hld\ hétumk \alst{G}anglęri, &
\ind \alst{H}erjann ok \alst{H}jalm-beri, &
\alst{Þ}ękkr ok \alst{Þ}riði, \hld\ \alst{Þ}undr ok Uðr, &
\ind \alst{H}ęl-blindi ok \alst{H}ár.\eva

\bvb I called myself Grim, I called myself Gangler, \\
Harn and Helmbearer. \\
Theck and Third, Thound and Ith, \\
Hellblind and High.\evb
\evg


\bvg
\bva\mssnote{\Regius~10v/19, \AM~5v/5}\alst{S}aðr ok \alst{S}vipall \hld\ ok \alst{S}ann-getall, &
\ind \alst{H}ęr-tęitr ok \alst{H}nikarr, &
\alst{B}il-ęygr, \alst{B}ál-ęygr, \hld\ \alst{B}ǫl-verkr, Fjǫlnir, &
\alst{G}rímr ok \alst{G}rímnir, \hld\ \alst{G}lap-sviðr ok Fjǫl-sviðr.\eva

\bvb Sooth and Swiple and Soothgettle, \\
Hartote and Nicker, \\
Bileye, Baleeye, Baleworker, Fillner, \\
Grim and Grimner, Glapswith and Fellswith.\evb
\evg


\bvg
\bva\mssnote{\Regius~10v/21, \AM~5v/7}\alst{S}íð-hǫttr, \alst{S}íð-skęggr, \hld\ \alst{S}ig-fǫðr, Hnikuðr, &
\alst{A}l-fǫðr, \alst{V}al-fǫðr, \hld\ \alst{A}t-ríðr ok Farma-týr; &
\alst{ęi}nu nafni \hld\ hétumk \alst{a}ldri-gi &
\ind síðst ek með \alst{f}olkum \alst{f}ór.\eva

\bvb Sidehat, Sideshag, Syefather, Nicked, \\
Allfather, Walfather, Atrider and Farm-Tew; \\
by a single name [have] I never called myself, \\
since among man-folk I fared.\evb
\evg


\bvg
\bva\mssnote{\Regius~10v/23, \AM~5v/9}\alst{G}rímni mik hétu \hld\ at \alst{G}ęir-raðar, &
\ind en \alst{Ja}lk at \alst{Ǫ́}s-mundar; &
en þá \alst{K}jalar \hld\ es ek \alst{k}jalka dró, &
\ind \alst{Þ}rór \alst{þ}ingum at.\eva

\bvb Grimner they called me at Garfrith’s [estate], \\
but Yelk at Osmunds; \\
but Keller then, as I drew the sled; \\
Throo at \inx[C]{Thing}[Things].\footnoteB{Presumably referencing other now-lost myths involving Weden travelling in disguise. The last is possibly a reference to the name under which Weden would be invoked at the start of Things (legal assemblies, see Encyclopedia).}\evb
\evg


\bvg
\bva\mssnote{\Regius~10v/24, \AM~5v/10}\alst{Ó}ski ok \alst{Ó}mi, \hld\ \alst{Ja}fn-hár ok Biflindi, &
\ind \alst{G}ǫndlir ok Hár-barðr með \alst{g}oðum.\eva

\bvb Wish and Ome, Evenhigh and Bivlend; Gandler and Hoarbeard among gods.\evb
\evg


\bvg
\bva\mssnote{\Regius~10v/25, \AM~5v/11}\alst{S}viðurr ok \alst{S}viðrir \hld\ es ek hét at \alst{S}økk-mímis &
\ind ok dulða’k þann hinn \alst{a}ldna \alst{jǫ}tun &
þá’s \alst{M}ið-vitnis vas’k \hld\ ins \alst{m}ę́ra burar &
\ind \alst{o}rðinn \alst{ęi}n-bani.\eva

\bvb Swither and Swithrer, as I was called at Sink-Mimer’s, \\
and I deceived that aged ettin, \\
when I of Midwitner’s renowned son \\
was become the lone slayer.\evb
\evg


\bvg
\bva\mssnote{\Regius~10v/28, \AM~5v/13}\alst{Ǫ}lr est Gęir-røðr, \hld\ hęfr þú \alst{o}f-drukkit; &
\alst{m}iklu est hnugginn, \hld\ es þú est \alst{m}ínu gęngi, &
\alst{ǫ}llum \alst{ęi}n-hęrjum \hld\ ok \alst{Ó}ðins hylli.\eva

\bvb Worse for ale art thou, Garfrith; thou hast over-drunk. \\
Of much art thou bereft when thou art [bereft] of my support, \\
of all the Ownharriers, and of Weden’s \inx[C]{holdness}.\footnoteB{Linguistically, Garfrith is not bereft of the support of the Ownharriers but rather of the Ownharriers themselves, but presumably the sense is the same. By breaking the code of conduct to which he owns his success he lost Weden’s favour, and thus been excluded from the community of oath-bound Odinic warriors (the Ownharriers). Cf. here}\evb
\evg


\bvg
\bva\mssnote{\Regius~10v/30, \AM~5v/15}\alst{F}jǫlð þér sagða’k, \hld\ en þú \alst{f}átt of mant, &
\ind of þik \alst{v}éla \alst{v}inir; &
\alst{m}ę́ki liggja \hld\ sé’k \alst{m}íns vinar &
\ind allan í \alst{d}ręyra \alst{d}rifinn.\eva

\bvb Much [have] I said to thee, but thou recallest little; \\
’tis friends that deal with thee! \\
The sword of my friend I see lying \\
all drenched in gore.\footnoteB{Weden expresses his disappointment in Garfrith’s conduct and foresees his imminent death.}\evb
\evg


\bvg
\bva\mssnote{\Regius~10v/31, \AM~5v/16}\alst{Ę}gg-móðan val \hld\ nú mun \alst{Y}ggr hafa, &
\ind þitt vęit’k \alst{l}íf of \alst{l}iðit; &
\alst{v}arar ’ru dísir, \hld\ nú knátt \alst{Ó}ðin séa; &
\ind nálgask \alst{m}ik ef þú \alst{m}ęgir!\eva

\bvb An edge-tired corpse will Ug now have: \\
I know thy life to be passed! \\
Wary are the dises, now dost thou see Weden— \\
come near \emph{me}, if thou mayst!\evb
\evg


\bvg
\bva\mssnote{\Regius~11r/2, \AM~5v/18}\alst{Ó}ðinn nú hęiti’k, \hld\ \alst{Y}ggr áðan hét’k, &
\ind hétumk \alst{Þ}undr fyr \alst{þ}at, &
\alst{V}akr ok Skilfingr, \hld\ \alst{V}ǫ́fuðr ok Hropta-týr &
\ind \alst{G}autr ok Jalkr með \alst{g}oðum.\eva

\bvb Weden I am now called, Ug was I earlier called, \\
I called myself Thound before that. \\
Wacker and Shilving, Waved and Roft-Tew, \\
Geat and Gelding among the gods.\evb
\evg


\bvg
\bva\mssnote{\Regius~11r/4, \AM~5v/20}\alst{O}fnir ok Sváfnir \hld\ hygg’k at \alst{o}rðnir sé &
\ind \alst{a}llir at \alst{ęi}num mér.\eva

\bvb Ovner and Sweefner, I ween, have arisen \\
all from me alone.\evb
\evg


\bpg
\bpa\mssnote{\Regius~11r/5, \AM~5v/21}Geir-røðr konungr sat, ok hafði sverð um kné sér ok brugðit til miðs. En er hann heyrði, at Óðinn var þar kominn, stóð hann upp, ok vildi taka Óðin frá eldinum. Sverðit slapp ór hendi hánum; vissu hjǫltin niðr. Konungr drap fę́ti, ok steyptist á-fram, en sverðit stóð í gǫgnum hann, ok fekk \edtext{hann}{\Afootnote{þar af \AM}} bana. \edtext{Óðinn hvarf þá.}{\Afootnote{om. \AM}} En Agnarr \edtext{var þar}{\Afootnote{varð \AM}} konungr \edtext{lengi síðan.}{\Afootnote{om. \AM}}\epa

\bpb King Garfrith sat and had the sword about his knee, and it was brandished half-way up. But when he heard that Weden were come there, he stood up and would take Weden from the fire. The sword slipped out of his hand; the hilt pointed downwards. The king tripped and stooped forth, but the sword went through him, and he received his bane. Weden then disappeared, but Eyner was there king for a long while afterwards.\epb
\epg
