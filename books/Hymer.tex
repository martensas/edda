\bookStart{The Lay of Hymer}[Hymiskviða]

\begin{flushright}%
Dating \parencite{Sapp2022}: C10th (0.694)–early C11th (0.268)

Meter: \Fornyrdislag%
\end{flushright}%

% Introduction.
Attested in two manuscripts, \Regius\ and \AM. The two are surprisingly consistent; all stanzas are shared, and come in the same order. The title \emph{Hymis-kviða} ‘the Lay of Hymer’ comes from \AM.  \Regius\ instead has the title \emph{Þórr dró Mið-garðs-orm} ‘Thunder pulled the Middenyardswyrm’ in typical red ink.

The poem is a comedy about Thunder’s adventures among the Ettins.  This was likely a popular genre, and is also represented by \Thrymskvida.  In spite of these similarities of contents the two poems are far apart stylistically.  Whereas \Thrymskvida\ is written in a simple and sparse style with free \Fornyrdislag\ meter and few kennings, the form of \Fornyrdislag\ used in \Hymiskvida\ is unusually strict, almost syllable-counting, and the stanzas are filled with rare kennings and difficult grammatical constructions, often in forced word order.  In this way \Hymiskvida\ is more akin to Scoldic poetry in intricate measures like \Drottkvett\ than to typical Eddic poetry in \Fornyrdislag.  Because of this it seems likely that the anonymous poet was highly trained in the Scoldic arts, and familiar with composition in more advanced meters.  (See TODO: Difference between Scoldic and Eddic).

Apart from meter and style, the Scoldic composition context of \Hymiskvida\ is also supported by both its dating and subject.  Thunder’s fishing expedition was a very popular myth in the Wiking age, and there are five extant Scoldic poetic fragments (TODO: list them) that deal with it.  The story is also retold in \Gylfaginning, and attested pictorally on the Swedish Altuna runestone and others (TODO).

The Scoldic fragments are very incomplete, and (in their presently reduced form) mostly focus on the subject of Thunder facing off against the hooked Wyrm pressed to the gunwale.  In some of the fragments the encounter ends with the cowardly Hymer cutting off the fishing line and the Wyrm sinking back into the sea (the version preferred by Snorre)—in others Thunder strikes the head off the Wyrm.  There are some interesting verbal correspondences between these fragments and \Hymiskvida—most strikingly the kenning for the Middenyardswyrm in st. 22/4 below—that may also support a common composition context.

\Gylfaginning\ 48 tells a more complete narrative, here paraphrased for shortness’ sake:

{\small Thunder goes out into Middenyard in the shape of a young man (\emph{ungr dręngr}), without his chariot, his goats, or his typical travelling gear.  In the evening he comes to the ettin Hymer and begs for lodgings.  At dawn Hymer plans to go fishing, and so Thunder asks to join in.  Hymer insults Thunder's small stature and youth, and questions his ability to go on such a long and arduous trip as he usually takes.  Thunder, angered, says that he will row very far, and then asks Hymer what bait they will use.  Hymer tells him to get his own bait, and so he turns to Hymer’s flock of oxen and tears off the head from his greatest ox, one named Heavenrid.  The two go out to sea, and Thunder rows far past Hymer's usual fishing spot.  Hymer, unhappy, warns him that if they row any further out they'll be in danger of the Middenyardswyrm, but Thunder goes on.  Eventually Thunder puts away the oars, readies a fishing line, hooks the ox-head and lowers it.  The Wyrm soon bites, and struggles so hard that Thunder is pressed against the gunwale.  This angers the god, and he brings himself into his Os-might.  Strengthened, he pulls back with such force that his feet go through the bottom of the ship and press into the sea-floor; the Wyrm's head goes up against the gunwale.  The two archenemies furiously stare at each other, Thunder “sharpening his eyes” and the Wyrm spitting venom.  Hymer is frightened, reaches for his bait-cutting knife, and cuts off the line—the Wyrm then sinks back into the sea.  Thunder throws the hammer after it, “and men say that he struck off the monster’s head, but I think it true to tell thee, that the Middenyardswyrm still lives and lies in the outer sea.”  Thunder then punches Hymer’s ear with his fist so that he is thrown overboard head-first; the god then wades back to land.}

This account is clearly based on several sources, possibly including the present poem.  The most notable correspondence is when it is said that \emph{Miðgarðs-ormr gein yfir uxa-hǫfuðit, en ǫngullinn vá í góminn orminum} ‘The Middenyardswyrm yawned over the ox-head, and the hook went into the roof of the wyrm’s mouth’, which is decently close to st. 22 below.  The name Heavenrid (\emph{Himinhrjóðr}) is otherwise only found in thules listing names of oxen, and the interesting detail of Thunder’s feet going through the boat is only paralleled by the Swedish Altuna stone (though see note to st. 34/2 below).

While \Gylfaginning\ 48, the Scoldic fragments, and \Hymiskvida\ all share the central narrative of the fishing expedition, \Hymiskvida\ has several additional narratives woven into it.  (I mean not to say that \Hymiskvida\ consists of multiple originally separate poems—unlike, say, \Havamal, which has noticable differences of style and language between its constiuent strands, \Hymiskvida\ comes off as a strong stylistic and narrative whole, composed by a single poet and thereafter transmitted faithfully.)

One may roughly identify the following narrative divisions in \Hymiskvida, of which only numbers 2–4 are found in the other sources for the myth of Thunder’s fishing:

\begin{enumerate}
  \item 1–6 Thunder attempts to force the ettin Eagre to host a banquet for the Gods; Eagre in turn asks for a cauldron big enough to brew enough ale for them all.
  \item 7–16 Thunder and Tew go to visit the stingy ettin Hymer, who owns such a cauldron; horrified at Thunder’s great appetite during the evening, Hymer tells them that they must eat fish the next.
  \item 17–19 Thunder says that he will go fishing if he is given bait; Hymer challenges him with killing one of his oxen for bait, after which Thunder tears off the head of one.
  \item 20–25 Hymer, Thunder and Tew go fishing; Hymer pulls up some whales; with the ox-head as bait Thunder manages to hook the Middenyardswyrm itself; he loses it.
  \item 26–27 Hymer challenges Thunder to carry the boat and whales back to his farm; he does.
  \item 28–32 Hymer challenges Thunder to break a supposedly indestructible chalice; he succeeds by smashing it against the ettin’s forehead.
  \item 33–36 Thunder and Tew depart with the cauldron; they find themselves followed by a troop led by Hymer; Thunder kills them all.
  \item 37–38 Lock makes the leg of one of Thunder’s goats halt.
  \item 39 Thunder returns to the Gods with Hymer’s cauldron; they host a banquet.
\end{enumerate}

The fishing expedition, found at the very center of the poem, is thus framed by the unique narrative of Thunder and Tew obtaining a huge cauldron from Hymer for the sake of brewing ale, and several other superfluous narratives scattered throughout.  The poet has not been entirely successful in his endeavour, and there are several loose strands; most notably Tew, who has nothing to do with the fishing expedition, probably because he was not originally in it, and who has no reaction at all to the murder of his father.  The function of Lock making one of Thunder’s goats halt is also unclear, and he does not appear anywhere else in the poem.

\sectionline

The poem has some interesting reoccurring themes.  The “otherness” of the Ettins, specifically Hymer, is constantly emphasized in several ways:
\begin{itemize}
  \item they live far to the East (st. 5) in an inhospitable, frozen climate (st. 10), associated with mountains (sts. 2, 17) and lava-fields (st. 36)
  \item they are physically deviant, being misshapen (st. 10), grey-haired (st. 16), many-headed (sts. 8, 35), and very hard-boned (sts. 30–31); they are even likened to apes (st. 20), whales (st. 36) and Danes (st. 17; see note!),
  \item they are stingy and inhospitable (sts. 9, 16),
  \item and sarcastic and cowardly (st. 19–20, 25–26, 28–32).
\end{itemize}

In these ways the Ettins oppose the Old Germanic social norms as represented by the Gods, who live in a lush green climate and are young, beautiful and generous.  The one exception is of course Tew’s mother in st. 8, who is light-haired (in contrast to the swarthy grandmother, presumably) and generous.  Perhaps the poet is implying that it is from her that Tew has inherited his good traits?

The last point, viz. sarcasm and cowardice, is seen throughout the poem in the way Thunder comically humiliates the Ettins, especially by completing challenges issued to him.  These follow a similar format: Thunder is given a near-impossible test of strength, which he shortly completes through a mix of physical strength and cleverness, humiliating the challenger.  These tests are finding a huge kettle (st. 3, explicitly called Eagre’s “revenge” (\emph{hęfnd}), taking one of Hymer’s oxen for bait (st. 17–18), carrying home Hymer’s whales and boat (st. 26), breaking Hymer’s finest chalice (st. 28), and perhaps also taking away the kettle (st. 33)—though that may just be Hymer’s wishing to finally be rid of the pestering gods.

Much like in \Thrymskvida\ the conflict is finally resolved with righteous hammer-slaughter.  After the Gods leave, Hymer tries to get his revenge by ambushing them, but Thunder takes his trusty hammer and kills them all.  The poem is clearly humorous and meant to be performed before an audience (see st. 38 where the poet directly addresses the listeners).  The original performance context may perhaps be gleaned from the difficult final stanza. TODO: It hints at a performance at a harvest bloot.

\sectionline

\bvg\bva\mssnote{\Regius~13v/26, \AM~5v/25}Ár \alst{v}al-tívar \hld\ \alst{v}ęiðar nǫ́mu &
ok \alst{s}umbl-\alst{s}amir \hld\ \edtrans{áðr \alst{s}aðir yrði,}{before they might eat}{\Bfootnote{Lit. ‘might become sated’}}, &
\alst{h}ristu tęina \hld\ ok á \alst{h}laut sǫ́u, &
fundu at \alst{Ę́}gis \hld\ \alst{ø}r-kost hvera.\eva

\bvb Of yore the slain-Tews \ken{gods} had caught game, \\
and together at the \inx[C]{simble} before they might eat \\
they shook the twigs and looked at the \inx[C]{leat}; \\
they found at Eagre’s a great choice of cauldrons.\footnoteB{The gods sprinkled the leat (\emph{hlaut} ‘sacrificial blood’) of the beasts and interpreted the pattern; they found it most auspicious to feast at Eagre’s. TODO: reference to leat-twigs.}\evb\evg


\bvg\bva\mssnote{\Regius~13v/28, \AM~5v/27}Sat \alst{b}erg-\alst{b}úi \hld\ \alst{b}arn-tęitr fyrir, &
\alst{m}jǫk glíkr \edtext{\alst{m}ęgi \hld\ \alst{M}iskur-blinda}{\lemma{męgi Miskur-blinda ‘lad of Misherblind’}\Bfootnote{An unexplained reference.  Misherblind might be another name for Firneet, Eagre’s father.}}, &
lęit í \alst{au}gu \hld\ \alst{Y}ggs barn í þrá: &
„þú skalt \alst{ǫ́}sum \hld\ \alst{o}pt sumbl \edtext{gęra}{\lemma{gęra ‘host’}\Afootnote{\emph{gefa} ‘give’ \AM}}!“\eva

\bvb Sat the mountain-dweller \ken*{\textsc{ettin} = Eagre} there, merry like a child, \\
much alike to the lad of Misherblind; \\
into his eyes looked the child of Ug \name{= Weden} \ken*{= Thunder} stubbornly: \\
“Thou shalt for the Eese oft host simbles!”\footnoteB{Having seen that Eagre has a great store of cauldrons, Thunder orders him to host future banquets for the Eese.}\evb\evg


\bvg\bva\mssnote{\Regius~13v/31, \AM~5v/29}\alst{Ǫ}nn fekk \alst{jǫ}tni \hld\ \alst{o}rð-bę́ginn halr, &
\alst{h}ugði at \alst{h}efndum \hld\ \alst{h}ann nę́st við goð, &
bað \alst{S}ifjar ver \hld\ \alst{s}ér fǿra hver, &
„þann’s ek \alst{ǫ}llum \alst{ǫ}l \hld\ \alst{y}ðr of hęita.“\eva

\bvb Great toil for the ettin the word-peevish man \ken*{= Thunder} caused; \\
he \ken*{= Eagre} thought of revenge, soon, against the god; \\
he bade Sib’s husband \ken*{= Thunder} bring him a cauldron, \\
“that one with which I for you all ale might heat.\footnoteB{Eagre gets back at Thunder by telling him that he needs a single cauldron which can hold enough ale to supply all the Eese.}”\evb\evg


\bvg\bva\mssnote{\Regius~14r/1, \AM~5v/30}Né þat \alst{m}ǫ́ttu \hld\ \alst{m}ę́rir tívar &
ok \alst{g}inn-ręgin \hld\ of \alst{g}eta hvęr-gi, &
unds af \alst{t}ryggðum \hld\ \alst{T}ýr Hlórriða &
\alst{á}st-ráð mikit \hld\ \alst{ęi}num sagði:\eva

\bvb But that one might the renowned \inx[G]{Tews} \\
and the \inx[G]{yin-Reins} nowhere get ahold of— \\
until, out of loyalty, a great loving counsel \\
Tew to Loride \name{= Thunder} alone did say:\evb\evg


\bvg\bva\mssnote{\Regius~14r/3, \AM~6r/2}„Býr fyr \alst{au}stan \hld\ \alst{É}li-vága &
\alst{h}und-víss \alst{H}ymir \hld\ at \alst{h}imins ęnda, &
á \alst{m}inn faðir \hld\ \alst{m}óðugr kętil, &
\edtext{\alst{r}úm-brugðinn}{\Afootnote{\emph{†rumbrygðan†} \AM}} hver \hld\ \alst{r}astar djúpan.“\eva

\bvb “Dwells to the east of the \inx[L]{Ilewaves} \\
the hound-wise Hymer, at heaven’s end.\footnoteB{According to \Vafthrudnismal\ 31 the Ilewaves were the poisonous wild rushes out of which the ettins emerged, and so it only makes sense that they would be found in the east, where the ettins dwell. Hymer’s dwelling even further east than them illustrates his fierce nature.} \\
Owns my father \ken*{= Hymer}, fierce, a kettle: \\
a size-famed cauldron one \inx[C]{rest} deep.”\evb\evg


\bvg\bva\speakernote{[Þórr kvað:]}\mssnote{\Regius~14r/4, \AM~6r/4}„Vęitst, ef \alst{þ}iggjum \hld\ \alst{þ}ann lǫg-velli?“ &
\speakernote{[Týr kvað:]}„Ef, \alst{v}inr, \alst{v}élar \hld\ \alst{v}it gørvum til!“\eva

\bvb\speakernoteb{[Thunder quoth:]}
“Knowest thou if we will receive that liquid-boiler \ken{cauldron}?” — \\
\speakernoteb{[Tew quoth:]}
“If, friend, we two make use of wiles!”\footnoteB{Like elsewhere in this poem the speakers are not indicated, but it is most sensible that Thunder asks and Tew answers.}\evb\evg

\bvg\bva\mssnote{\Regius~14r/5, \AM~6r/4}Fóru \alst{d}rjúgum \hld\ \edtrans{\alst{d}ag þann framan}{from the beginning of the day}{\Afootnote{emend. after \textcite{FinnurEdda}; \emph{dag þann fram} ‘on that day forth’ \Regius; \emph{dag fráliga} ‘swiftly at day’ \AM}} &
\alst{Á}sgarði frá \hld\ unds til \edtrans{\alst{Ę}gils}{Agle}{\Afootnote{so \Regius; \emph{Ę́gis} ‘Eagre’ \AM\ is probably from confusion with Eagre (the ettin) described earlier in the poem, though the shepherd may have shared his name.}} kvǫ́mu; &
\edtrans{\alst{h}irði \alst{h}afra \hld\ \alst{h}orn-gǫfgasta}{he herded the he-goats noblest of horns}{\Bfootnote{i.e., he took care of Thunder’s goats.}}; &
\alst{h}urfu at \alst{h}ǫllu \hld\ es \alst{H}ymir átti.\eva

\bvb They journeyed long from the beginning of the day, \\
away from Osyard, until to Agle they came— \\
he herded the he-goats noblest of horns— \\
they turned to the hall which Hymer owned.\evb\evg


\bvg\bva\mssnote{\Regius~14r/7, \AM~6r/6}\alst{M}ǫgr fann ǫmmu, \hld\ \alst{m}jǫk lęiða sér, &
\alst{h}afði \alst{h}ǫfða \hld\ \alst{h}undruð níu. &
en \edtrans{\alst{ǫ}nnur}{another woman}{\Bfootnote{The use of the word “son” in the following line reveals this as Tew’s mother.  The poet stresses her beautiful dress and countenance, in contrast to the grandmother.}} gekk \hld\ \alst{a}l-gullin framm &
\alst{b}rún-hvít \alst{b}era \hld\ \alst{b}jór-vęig syni:\eva

\bvb The lad \ken*{= Tew} found his grandmother very loathsome; \\
of heads she had nine hundred. \\
But another woman, all-golden, walked forth, \\
white-browed, bringing a beer-draught for [her] son \ken*{= Tew}:\evb\evg


\bvg\bva\speakernote{[Týs móðir:]}\mssnote{\Regius~14r/9, \AM~6r/8}„\alst{Á}tt-niðr \alst{jǫ}tna \hld\ \alst{e}k vilja’k ykkr &
\alst{h}ug-fulla tvá \hld\ und \alst{h}vera sętja; &
es \alst{m}ínn \edtrans{fríi}{lover}{\Afootnote{so \Regius; \emph{faðir} ‘father’ \AM}} \hld\ \alst{m}ǫrgu sinni &
\edtext{\alst{g}løggr við \alst{g}ęsti \hld\ \alst{g}ǫrr ills hugar}{\lemma{gløggr \dots\ hugar ‘stingy \dots\ mood’}\Bfootnote{Ettins are characteristically inhospitable, in stark opposition to the Old Germanic social norms; see Introduction to the poem above.  This statement foreshadows the later hunting expedition starting at st. 16 below.}}.“\eva

\bvb\speakernoteb{[Tew’s mother:]}“O descendant of ettins \ken*{= Tew}, \emph{I} would wish to hide \\
you two, full of heart, under the cauldrons; \\
many a time has my lover \ken*{= Hymer} been \\
stingy with guests, quick to bad mood.”\evb\evg


\bvg\bva\mssnote{\Regius~14r/11, \AM~6r/9}En \alst{v}á-skapaðr \hld\ \alst{v}arð \edtrans{síð-búinn}{come late}{\Afootnote{om. \AM}}, &
\alst{h}arð-ráðr \alst{H}ymir, \hld\ \alst{h}ęim af vęiðum; &
\alst{g}ekk inn í sal, \hld\ \alst{g}lumðu \edtrans{jǫklar}{icicles}{\Bfootnote{viz. in Hymer’s frozen beard.  In modern Icelandic the word \emph{jökull} has come to mean ‘glacier’, but its original meaning (as found in the present stanza) is that of its English cognate ‘icicle’.}}, &
vas \alst{k}arls, es \alst{k}om, \hld\ \alst{k}inn-skógr frørinn.\eva

\bvb But the misshapen one was come late, \\
hard-minded Hymer, home from the hunt. \\
He entered the hall—the icicles clattered— \\
on the churl who came \ken*{= Hymer} was the cheek-shaw \ken{beard} frozen.\evb\evg


\bvg\bva\speakernote{[Týs móðir:]}\mssnote{\Regius~14r/13, \AM~6r/11}„\edtext{Ves þú \alst{h}ęill, \alst{H}ymir, \hld\ í \alst{h}ugum góðum!}{\lemma{Ves þú hęill, \dots\ í hugum góðum! ‘Be thou hale \dots\ in good spirits!’}\Bfootnote{A formulaic greeting; cf. the almost identical greeting in \emph{N B380} (edited below under Galders).  Further afield cf. the type exemplified by \Beowulf\ 407a: \emph{Wæs þú, Hróðgâr, hâl} ‘Be thou, Rothgar, hale!’}} &
Nú ’s \alst{s}onr kominn \hld\ til \alst{s}ala þinna, &
sá’s \alst{v}it \alst{v}ę́ttum \hld\ af \alst{v}egi lǫngum; &
fylgir \alst{h}ǫ́num \hld\ \alst{H}róðrs and-skoti, &
\alst{v}inr \alst{v}er-liða; \hld\ \alst{V}éurr hęitir sá.\eva

\bvb\speakernoteb{[Tew’s mother:]}“Be thou hale, Hymer, in good spirits! \\
Now the son \ken*{= Tew} is come to thy halls, \\
the one whom we have been awaiting from a long way off. \\
Follows him the opponent of Rooder \name{ettin}, \\
the friend of manly retinues; \inx[P]{Wighward} \name{= Thunder} is that one called.\evb\evg


\bvg\bva\mssnote{\Regius~14r/15, \AM~6r/13}\alst{S}é þú hvar \alst{s}itja \hld\ und \alst{s}alar gafli, &
\alst{s}vá \edtext{forða \alst{s}ér}{\Afootnote{\emph{forðask} \AM}}, \hld\ stęndr \edtrans{\alst{s}úl}{column}{\Afootnote{\emph{†sol†} \AM}} fyrir.“ &
\alst{S}undr stǫkk \alst{s}úla \hld\ fyr \alst{s}jón jǫtuns, &
en \edtext{\alst{a}llr}{\Afootnote{emend.; \emph{áðr} ‘earlier, before that’ \Regius\AM. TODO: elaborate, mention Finnur}} í tvau \hld\ \alst{á}ss brotnaði.\eva

\bvb See where they sit under the hall’s gable: \\
so they save themselves—a column stands before them!\footnoteB{Tew’s mother reveals the hiding place of the gods.}” \\
The columns sprang asunder before the gaze of the ettin \ken*{= Hymer}, \\
but all in two the roof-beam broke.\evb\evg


\bvg\bva\mssnote{\Regius~14r/17, \AM~6r/15}Stukku \alst{á}tta, \hld\ en \alst{ęi}nn af þęim &
\alst{h}verr \alst{h}arð-slęginn \hld\ \alst{h}ęill af þolli; &
\alst{f}ramm gingu þęir, \hld\ en \alst{f}orn jǫtunn &
\alst{s}jónum lęiddi \hld\ \alst{s}inn and-skota.\eva

\bvb Eight [cauldrons] crashed down, but one of them— \\
a hard-forged cauldron—[came] whole off its peg.\footnoteB{The cauldrons were presumably hanging on the roof-beam. Eight of them broke, but a single one remained whole.} \\
Forth they went, and the ancient ettin \ken*{= Hymer} \\
with his gaze tracked his very opponent \ken*{= Thunder}.\evb\evg


\bvg\bva\mssnote{\Regius~14r/19, \AM~6r/16}\edtrans{Sagði-t \alst{h}ǫ́num \hld\ \alst{h}ugr vęl}{His heart did not please him}{\Bfootnote{Lit. ‘his heart did not speak well to him’.}} þá’s sá &
\alst{g}ýgjar \edtrans{\alst{g}rǿti}{distresser}{\Afootnote{\emph{gę́ti} ‘keeper, warder’ \AM}} \hld\ á \alst{g}olf kominn, &
\alst{þ}ar vǫ́ru \alst{þ}jórar \hld\ \alst{þ}rír of tęknir, &
bað \edtrans{\alst{s}ęnn}{at once}{\Afootnote{\emph{sun} ‘[his] son \ken*{= Tew}?’ \AM}} jǫtunn \hld\ \alst{s}jóða ganga.\eva

\bvb His heart did not please him when as he saw \\
the \inx[C]{gow}’s distresser \ken*{= Thunder} come onto the floor. \\
There three bulls were a-taken: \\
the ettin bade them at once be cooked.\evb\evg


\bvg\bva\mssnote{\Regius~14r/21, \AM~6r/18}\alst{H}vęrn létu þęir \hld\ \alst{h}ǫfði skęmra &
auk á \alst{s}ęyði \hld\ \alst{s}íðan bǫ́ru, &
át \alst{S}ifjar verr \hld\ áðr \alst{s}ofa gingi, &
\alst{ęi}nn með \alst{ǫ}llu \hld\ \alst{ø}xn tvá Hymis.\eva

\bvb Each one they let shorten by a head, \\
and onto the cooking-pit then did carry: \\
Sib’s husband \ken*{= Thunder} ate—before he might go sleep— \\
alone by himself two of Hymer’s oxen.\footnoteB{Cf. \Thrymskvida\ 24 for another instance of Thunder’s great eating, which curiously also uses the kenning \emph{Sifjar verr} ‘Sib’s husband \ken*{= Thunder}’.}\evb\evg


\bvg\bva\mssnote{\Regius~14r/23, \AM~6r/19}Þótti \alst{h}ǫ́rum \hld\ \alst{H}rungnis spjalla &
\alst{v}erðr Hlórriða \hld\ \alst{v}ęl full-mikill, &
„\edtext{munum at \alst{a}ptni \hld\ \alst{ǫ}ðrum verða &
\alst{v}ið \alst{v}ęiði-mat \hld\ \alst{v}ér þrír lifa.}{\lemma{munum \dots\ lifa ‘the next \dots\ live’}\Bfootnote{The poet is pushing at the limits of Old Norse syntax with this word order.  In prose word order it should be construed as: \emph{at ǫðrum aptni munum vér þrír verða lifa við vęiði-mat}, where \emph{verða} ‘have to, must’ is used like its modern German cognate \emph{werden}.

Hymer’s stinginess—he refuses to share more of his own food but instead forces his guests to go hunt—breaks all Indo-European rules of hospitality and illustrates the otherness of the Ettins.  See Introduction to the poem.}}“\eva

\bvb To Rungner’s hoary friend \ken*{= Hymer} did seem \\
Loride’s \name{Thunder’s} eating far too great; \\
“the following evening we three will \\
on game-meat have to live.”\evb\evg


\bvg\bva\mssnote{\Regius~14r/24, \AM~6r/21}\alst{V}éurr kvaðsk \alst{v}ilja \hld\ á \alst{v}ág róa, &
ef \alst{b}allr jǫtunn \hld\ \alst{b}ęitur gę́fi. &
„\alst{H}verf þú til \edtext{\alst{h}jarðar}{\Afootnote{\emph{hallar} corr. \AM}}, \hld\ ef \alst{h}ug trúir, &
\alst{b}rjótr \edtrans{\alst{b}erg-Dana}{boulder-Danes \ken{ettins}}{\Bfootnote{Kennings of this type emphasize the otherness of the Ettins (see Introduction to the poem above) by equating them with ethnic foreigners, and are well known from Anlif Gothrunson’s Drape for Thunder (\emph{Þórsdrápa}), where Ettins are called Scots, Swedes, Danes, Ruges and Hareds; all ethnic enemies of the Norwegian Earl Hathkin, at whose court that poem may have been composed.}}, \hld\ \alst{b}ęitur sǿkja.\eva

\bvb Wighward \name{= Thunder} called himself willing to row on the wave, \\
if the baleful ettin might give pieces of bait. \\
“Turn to the herd—if thou trust in thy heart, \\
O breaker of boulder-Danes \ken*{\textsc{ettins} > = Thunder}—to seek pieces of bait.\evb\evg


\bvg\bva\mssnote{\Regius~14r/26, \AM~6r/23}\alst{Þ}ess \edtext{vę́ntir mik}{\Afootnote{so \AM; \emph{vę́nti ek} \Regius}}, \hld\ at \alst{þ}ér \edtrans{myni-t}{will not}{\Afootnote{so \AM; \emph{myni} ‘will’ \Regius.  The \AM\ reading is preferable since it makes this the first of Hymer’s several challenges of strength to Thunder, which the god, to the ettin’s humiliation, easily accomplishes.}} &
\alst{ǫ}gn at \alst{o}xa \hld\ \alst{au}ð-feng vesa.“ &
\edtrans{\alst{S}vęinn}{The swain}{\Bfootnote{Thunder was apparently in the shape of a youth.  This detail is also found in \Gylfaginning\ 48, where Snorre writes: \emph{Gekk hann út of Miðgarð svá sem ungr drengr \dots} ‘He went out about Middenyard in the shape of a young warrior’.}} \alst{s}ýsliga \hld\ \alst{s}vęif til skógar, &
þar’s \edtext{\alst{o}xi stóð \hld\ \alst{a}l-svartr}{\lemma{oxi \dots\ alsvartr ‘all-black \dots\ ox’}\Bfootnote{Formulaic, also occuring in \Thrymskvida\ 23; see note there for further parallels to the custom of sacrificing animals of certain colours.  It seems that all-black oxen were thought the noblest, and so Thunder’s slaying one instead of an inferior beast is probably intended to humiliate the stingy Hymer.

In \Gylfaginning\ 48 we read that: \emph{Hann tók inn mesta uxann, er Himin-hrjóðr hét, ok sleit af hǫfuðit ok fór með til sjávar.} ‘He took the greatest ox, which was called Heavenrid, and tore of its head and went with it to the sea’.}} fyrir.\eva

\bvb I expect that the bait from the ox \\
will not be an easy catch for thee!”— \\
The swain \ken*{= Thunder} swiftly turned to the wood, \\
where an ox stood, all-black, before [him].\evb\evg


\bvg\bva\mssnote{\Regius~14r/28, \AM~6r/24}Braut af \alst{þ}jóri \hld\ \alst{þ}urs ráð-bani &
\alst{h}ǫ́-tún ofan \hld\ \alst{h}orna tveggja. &
„\alst{V}erk þikkja þín \hld\ \alst{v}erri myklu &
\alst{k}jóla valdi \hld\ an \alst{k}yrr sitir.“\eva

\bvb Off the bull broke the counsel-slayer of the thurse \ken*{= Thunder} \\
the high meadow of the two horns \ken{head} from above.— \\
“Worse by far thy works do seem \\
to the wielder of ships \ken*{= Hymer = me} than if thou mightst sat calm.\footnoteB{I had originally taken this as Hymer snidely belittling Thunder’s feat of pulling the head off the ox (presumably by the horns); he would have earned greater glory had he simply sat and done nothing. However, it may also be read as a factual statement; Thunder just killed one of his finest oxen, and Hymer would certainly have preferred that he had not.}”\evb\evg

\sectionline

{\small The scene now shifts, and the party is out at sea.  It is possible that a stanza has here been lost, or that it would be indicated in some other way in the original performance.}

\sectionline

\bvg\bva\mssnote{\Regius~14r/30, \AM~6r/26}Bað \alst{h}lunn-gota \hld\ \alst{h}afra dróttinn &
\edtext{\alst{á}tt-runn}{\Afootnote{\emph{†atrænn†} \AM}} \edtrans{\alst{a}pa}{ape}{\Bfootnote{The specific sense of \emph{api} ‘ape’ is uncertain.  It seems to generally refer to a fool, but see Encyclopedia.}} \hld\ \alst{ú}tar fǿra, &
\edtext{en \alst{s}á jǫtunn \hld\ \alst{s}ína \edtext{talði}{\Afootnote{\emph{milldi} corr. \AM}}, &
\alst{l}ítla fýsi \hld\ \edtext{\alst{l}ęngra at róa}{\Afootnote{metr. emend.; \emph{at róa lęngra} \Regius\AM}}.}{\lemma{en \dots\ róa. ‘but \dots\ longer.’}\Bfootnote{Thunder’s humorous humiliation of Hymer continues with the previously spiteful ettin now forced to row against his will.}}\eva

\bvb The Lord of he-goats \ken*{= Thunder} bade the kinsman of the \inx[C]{ape}\ \ken*{\textsc{ettin} = Hymer} \\
push the launching-steed \ken{boat} further out; \\
but that ettin told of his \\
scarce wish to row longer.\evb\evg


\bvg\bva\mssnote{\Regius~14r/31, \AM~6r/27}Dró \edtrans{\alst{m}ę́rr}{famous}{\Afootnote{so \Regius; \emph{męir} ‘more, further’ \AM}} Hymir \hld\ \alst{m}óðugr hvala &
\alst{ęi}nn á \alst{ǫ}ngli \hld\ \alst{u}pp sęnn tváa; &
en \alst{a}ptr í skut \hld\ \alst{Ó}ðni sifjaðr &
\alst{V}éurr við \alst{v}élar \hld\ \alst{v}að gęrði sér.\eva

\bvb Famous, fierce Hymer pulled whales: \\
one on the hook, soon up two. \\
But back in the stern the Weden-related \\
Wighward \name{= Thunder} craftily fixed His line.\evb\evg


\bvg\bva\mssnote{\Regius~14v/1, \AM~6r/29}\alst{Ę}gnði á \alst{ǫ}ngul \hld\ sá’s \alst{ǫ}ldum bergr, &
\alst{o}rms \alst{ęi}n-bani \hld\ \alst{o}xa hǫfði; &
\alst{g}ęin við \edtrans{agni}{bait}{\Afootnote{so \AM; \emph{ǫngli} ‘hook’ \Regius}}, \hld\ sú’s \alst{g}oð fía, &
\edtext{\alst{u}mb-gjǫrð neðan \hld\ \alst{a}llra landa}{\lemma{umb-gjǫrð \dots\ allra landa ‘encircler of all lands’}\Bfootnote{This kenning occurs identically in a fragment by C9th scold Alewigh Snub (Ǫlv \emph{Þórr} in \emph{SkP} III).}}.\eva

\bvb Baited on the hook He who rescues men \ken*{= Thunder}—  \\
the Wyrm’s Lone Slayer—the ox’s head. \\
Snapped at the bait the one whom the Gods hate \ken*{= Middenyardswyrm}— \\
the encircler of all lands—from below.\evb\evg


\bvg\bva\mssnote{\Regius~14v/3, \AM~6v/1}\alst{D}ró \alst{d}jarf-liga \hld\ \alst{d}áð-rakkr Þórr &
\alst{o}rm \alst{ęi}tr-fáan \hld\ \alst{u}pp at borði; &
\alst{h}amri kníði \hld\ \edtrans{\alst{h}ǫ́-fjall skarar}{high mountain of hair \ken{head}}{\Bfootnote{A rather unfitting kenning, since serpents do not have hair.}} &
\alst{o}f-ljótt \alst{o}fan \hld\ \alst{u}lfs hnit-bróður.\eva

\bvb Bravely deed-ready Thunder pulled \\
the venom-glistening Wyrm up on the gunwale; \\
with the hammer He struck the high mountain of hair \ken{head}— \\
very hideous, from above—on the Wolf’s clash-brother \ken*{= Middenyardswyrm}.\evb\evg


\bvg\bva\mssnote{\Regius~14v/5, \AM~6v/2}\edtrans{\alst{H}raun-gǫlkn}{The lavafield-monsters}{\Bfootnote{Both mss. have \emph{hręin-}, which may mean either ‘clean’ or ‘reindeer’, neither of which fit. On the other hand \emph{hraun} \ONP: ‘stone/barren area, wasteland; lavafield’ is well attested in scoldic kennings for ettins. The precise meaning of \emph{galkn} ‘monster’ (plural \emph{gǫlkn}) is unclear; but it is attested in three scoldic verses, always in kennings of the type “troll-woman of the shield \ken{axe}”.  While the mss. spelling ‘\emph{galkn}’ (norm. \emph{gálkn}) could reflect either singular and plural, the form of the verb is plural.  This means that the word cannot be referring to the Middenyardswyrm, refuting the interpretation of \textcite{LarringtonEdda}: “the sea-wolf shrieked”.}} \edtext{\alst{h}rutu}{\Afootnote{so \AM; \emph{hlumðu} ‘dashed’ \Regius. End-rhyme is also used by the poet in st. 3/3.}}, \hld\ ęn \alst{h}ǫlkn þutu, &
\alst{f}ór hin \alst{f}orna \hld\ \alst{f}old ǫll saman; &
\edtext{[...]}{\Bfootnote{It is very likely that a line is missing here, since the stanzas in the poem otherwise consistently have four lines.  In other tellings of the myth it is at this point that Hymer cuts Thunder’s fishing line, so that is probably what has been lost.

It is of course impossible to know what exact form it had, but for the reader’s enjoyment, based on other poets and the account in \Gylfaginning\ (see introduction to the present poem) I’ve composed the following variant lines: \emph{unds vinr Hrungnis \hld\ vað Þórs of skar} ‘until the friend of Rungner \ken*{= Hymer} Thunder’s fishing-line did cut’; \emph{unds fǫlr Hymir \hld\ fekk á saxi} ‘until pale Hymer grasped the knife’.}} &
\alst{s}økkðisk \alst{s}íðan \hld\ \alst{s}á \edtrans{fiskr}{fish}{\Bfootnote{The Middenyardswyrm may also be called a “fish” in \Grimnismal\ 21; see note there.}} í mar.\eva

\bvb The lavafield-monsters \ken{ettins} bounded and the bedrock resounded; \\
the ancient earth moved all at once; \\
{[...]}; \\
sank thereafter that fish \ken*{= Middenyardswyrm} into the sea.\evb\evg


\bvg\bva\mssnote{\Regius~14v/6, \AM~6v/3}\alst{Ó}-tęitr \alst{jǫ}tunn, \hld\ es \alst{a}ptr røru, &
\edtext{[...]}{\Bfootnote{Another missing line.  As said in the previous stanza the meter usually requires four lines, and also the first half of the sentence is incomplete without a verb.}} &
svá’t \edtrans{\alst{á}r}{in the early morning}{\Bfootnote{\textcite{FinnurEdda}\ suggests \emph{svá’t at ǫ́r} ‘so that by the oar’, but this burdens the meter.  Assuming my interpretation is correct, the three would have been out fishing throughout the night.}} Hymir \hld\ \alst{ę}kki mę́lti, &
\alst{v}ęifði rǿði \hld\ \alst{v}eðrs annars til.\eva

\bvb The unmerry ettin \ken*{= Hymer}, as they rowed back, \\
{[...]}, \\
so that in early morn Hymer said nothing; \\
he pulled the oar against the wind:\evb\evg


\bvg\bva\speakernote{[Hymir:]}\mssnote{\Regius~14v/8, \AM~6v/4}„Munt of \alst{v}inna \hld\ \alst{v}erk halft við mik, &
at \alst{h}ęim \alst{h}vali \hld\ \alst{h}af til bǿjar &
eða \alst{f}lot-brúsa \hld\ \alst{f}ęstir okkarn.“\eva

\bvb\speakernoteb{[Hymer quoth:]}“Thou wilt accomplish a half work against me, \\
if thou take home the whales to the farm, \\
or our float-jar \ken{boat} do fasten.\footnoteB{Hymer tells Thunder, who having let go of the Wyrm now has nothing to show for the trip, that he can accomplish something half as good as the pulling of the whales if he carries them home or ties up the boat (by the shore).}”\evb\evg


\bvg\bva\mssnote{\Regius~14v/9, \AM~6v/6}\alst{G}ekk Hlórriði \hld\ \alst{g}ręip \edtext{á}{\Afootnote{\emph{til á} \Regius}} stafni &
vatt \edtrans{með \alst{au}stri}{with the bilge-water}{\Bfootnote{That is, the bilge-water was still inside the boat.  As anyone who has handled one knows, this water weighs very much, so this was another great work of strength.}} \hld\ \alst{u}pp lǫg-fáki; &
\alst{ęi}nn með \alst{ǫ́}rum \hld\ ok með \alst{au}st-skotu &
\alst{b}ar til \alst{b}ǿjar \hld\ \alst{b}rim-svín jǫtuns &
ok \edtext{\edtext{\alst{h}olt-riða}{\Afootnote{\emph{†holtriba†} \Regius}} \hld\ \alst{h}ver}{\lemma{holt-riða hver}\Bfootnote{An uncertain and possibly corrupt kenning.  TODO: What do other editors and translators say?}} í gegnum. \eva

\bvb Loride \name{= Thunder} went, grasped the stern, \\
hurled up the lake-nag \ken{boat} with the bilge-water; \\
alone with the oars and the bilge-bucket \\
he bore to the farm the ettin’s brim-swines \ken{whales}, \\
even through the cauldron of woodland ridges \ken{valley?}.\evb\evg


\bvg\bva\mssnote{\Regius~14v/12, \AM~6v/7}\edtext{\edtext{Ok}{\Afootnote{\emph{enn} \AM}} \alst{ę}nn \alst{jǫ}tunn \hld\ umb \alst{a}frendi, &
\alst{þ}rá-girni vanr, \hld\ við \alst{Þ}ór sęnti, &
kvað-at mann \alst{r}amman, \hld\ þótt \alst{r}óa kynni, &
\alst{k}rǫptur-ligan, \hld\ nema \alst{k}alk bryti.}{\lemma{ALL}\Bfootnote{Even after witnessing numerous great feats of strength Hymer still refuses to admit Thunder’s superiority.  He now insists on challenging him with breaking his indestructible chalice.}}\eva

\bvb And yet the ettin, used to stubbornness, \\
over strength of hand did flyte with Thunder; \\
he called no man strong—although he could row, \\
mightily—unless he broke the chalice.\evb\evg


\bvg\bva\mssnote{\Regius~14v/14, \AM~6v/9}En \alst{H}lórriði, \hld\ es at \alst{h}ǫndum kom, &
\alst{b}rátt lét \alst{b}resta \hld\ \edtrans{\alst{b}ratt-stęin glęri}{steep stone with glass}{\Bfootnote{That is, he broke the stone columns in Hymer’s house with the chalice.}}, &
\alst{s}ló \edtrans{\alst{s}itjandi}{fastened}{\Bfootnote{This word is ambiguous and can modify either Thunder (in which case it would mean “sitting”) or the columns (\emph{súlur}).  I have chosen the latter and read it as signifying their stability.}} \hld\ \alst{s}úlur í gǫgnum; &
bǫ́ru þó \alst{h}ęilan \hld\ fyr \alst{H}ymi síðan.\eva

\bvb But Loride \name{= Thunder}, when it came to his hands, \\
impatiently crushed steep stone with glass; \\
he struck right through the fastened columns; \\
it was still brought whole before Hymer afterward.\evb\evg


\bvg\bva\mssnote{\Regius~14v/16, \AM~6v/10}Unds þat hin \alst{f}ríða \hld\ \alst{f}riðla kęndi &
\alst{á}st-ráð mikit, \hld\ \alst{ęi}tt es vissi, &
„drep við \alst{h}aus \alst{H}ymis, \hld\ hann ’s \alst{h}arðari, &
\edtrans{\alst{k}ost-móðs jǫtuns}{the choice-weary ettin’s}{\Bfootnote{Presumably referring to the Gods’ having already eaten all his choicest food and slain his finest bull.}}, \hld\ \alst{k}alki hvęrjum.“\eva

\bvb Until the handsome mistress \ken*{= Tew’s mother} gave \\
a great loving counsel, the one she knew: \\
“Strike against Hymer’s skull; it is harder— \\
the choice-weary ettin’s—than every chalice.”\evb\evg


\bvg\bva\mssnote{\Regius~14v/18, \AM~6v/12}\alst{H}arðr \edtext{ręis}{\Afootnote{om. \AM}} á kné \hld\ \alst{h}afra dróttinn, &
fǿrðisk \alst{a}llra \hld\ í \alst{á}s-męgin; &
\alst{h}ęill vas karli \hld\ \alst{h}jalm-stofn ofan, &
en \alst{v}ín-fęrill \hld\ \alst{v}alr rifnaði.\eva

\bvb Hard on the knee rose the Lord of he-goats \ken*{= Thunder}; \\
He drew Himself into His highest Os-might.\footnoteB{Compare \Gylfaginning\ in its description of Thunder attempting to pull up the Wyrm: \emph{Þá varð Þórr reiðr ok fę́rðist í ás-megin} “Then Thunder became wroth, and drew himself into his os-might.”}— \\
Whole was on the churl \ken*{= Hymer} the helmet-stump \ken{head} above, \\
but the round wine-track \ken{chalice} rent apart.\evb\evg


\bvg\bva\speakernote{[Hymir kvað:]}\mssnote{\Regius~14v/20, \AM~6v/13}„\alst{M}ǫrg vęit’k \alst{m}ę́ti \hld\ \alst{m}ér gingin frá, &
\edtext{es}{\Afootnote{om. \Regius}} \alst{k}alki sé’k \hld\ \edtext{fyr}{\Afootnote{\emph{†yr†} \Regius}} \alst{k}néum hrundit,“ &
\alst{k}arl orð of \alst{k}vað: \hld\ „\edtext{\alst{k}ná’k-at sęgja &
\alst{a}ptr \alst{ę́}va-gi: \hld\ ‚þú ’st \alst{ǫ}lðr of hęitt.}{\lemma{kná’k-at \dots\ of hęitt. ‘I cannot \dots\ O ale!’}\Bfootnote{Hymer laments that with the loss of his finest vessel he will never be able to enjoy his drink again.  There is strong irony here since it was he himself who challenged Thunder to break it.}}‘\eva

\bvb\speakernoteb{[Hymer quoth:]}“I know many treasures have passed from me, \\
when I see the chalice thrown before [his] knees!”— \\
The churl spoke \ken*{= Hymer} words: “I cannot say \\
ever again: ‘Thou art brewed, O Ale!’\evb\evg


\bvg\bva\mssnote{\Regius~14v/22, \AM~6v/15}Þat ’s til \alst{k}ostar \hld\ ef \alst{k}oma mę́ttið &
\alst{ú}t ór \alst{ó}ru \hld\ \edtrans{\alst{ǫ}l-kjól}{ale-ship \ken{cauldron}}{\Bfootnote{\emph{ǫl-kjól} is the accusative form, but in this context (\CV: \emph{koma}, B) we would expect the dative \emph{ǫl-kjóli}.  The meter does not allow for this, however.}} \edtrans{hofi}{hall}{\Bfootnote{This is the only Old Norse occurrence of the word \emph{hof} in the sense ‘hall, house’; it otherwise only means ‘temple’ (\inx[C]{hove}).  The West Germanic cognates consistently mean ‘hall’, and that is probably the original sense, so it is unclear if this is an instance of foreign influence (if so, most likely Anglo-Saxon) or just a poetic archaism.}}.“ &
\alst{T}ýr lęitaði \hld\ \alst{t}ysvar hrǿra; &
stóð at \alst{h}vǫ́ru \hld\ \alst{h}verr kyrr fyrir.\eva

\bvb It would be best if ye might bring \\
the ale-ship \ken{cauldron} out of our hall.” \\
Tew attempted, twice, to move it— \\
each time stood the cauldron still before [him].\evb\evg


\bvg\bva\mssnote{\Regius~14v/24, \AM~6v/16}\alst{F}aðir Móða \hld\ \alst{f}ekk á þręmi &
ok í \alst{g}ǫgnum stęig \hld\ \alst{g}olf niðr í sal; &
\alst{h}óf sér á \alst{h}ǫfuð upp \hld\ \alst{h}ver Sifjar verr, &
en á \alst{h}ę́lum \hld\ \edtrans{\alst{h}ringar skullu}{the rings clattered}{\Bfootnote{i.e. the chain-links.  This detail is mentioned in an example sentence contrasting long and short phonemes in \FGT: \emph{heyrði til hǫddu, þá er Þórr bar hverinn} ‘the sound of the pot-links (\emph{hadda}) was heard when Thunder bore the cauldron’.  According to \textcite{FinnurEdda}\ the chain (or \emph{hadda}) on a Wiking-age cauldron would have reached across, in which case this would be a reference to the cauldron’s enormous size, with its diameter—mentioned in st. 5 as one \inx[C]{rest}—being roughly the same as Thunder’s height.}}.\eva

\bvb The father of Moody \ken*{= Thunder} grasped the brim, \\
and stepped down through the floor in the hall;\footnoteB{In the account of \Gylfaginning\ Thunder is said to have stepped through the boat when trying to pull up the Middenyardswyrm. This detail is also seen on the carving of the Altuna stone from Uppland, Sweden; it may have been transposed to this place in the narrative. TODO.} \\
Sib’s husband \ken*{= Thunder} heaved the cauldron up onto his head, \\
and at his heels the rings clattered.\evb\evg


\bvg\bva\mssnote{\Regius~14v/26, \AM~6v/18}Fóru-t \alst{l}ęngi, \hld\ áðr \alst{l}íta nam &
\alst{a}ptr \alst{Ó}ðins sonr \hld\ \alst{ęi}nu sinni; &
sá ór \alst{h}ręysum \hld\ með \alst{H}ymi austan &
\edtext{\alst{f}olk-drótt}{\lemma{folk-drótt \dots\ fjǫl-hǫfðaða ‘war-troop \dots\ many-headed’}\Bfootnote{A deviant number of body parts, especially heads, is typical of ettins.  See Introduction and note to st. 8 above.}} \alst{f}ara \hld\ \alst{f}jǫl-hǫfðaða.\eva

\bvb They journeyed not for long before Weden’s son \ken*{= Thunder} \\
took to look back a single time— \\
he saw out of stone-heaps, with Hymer from the east, \\
a war-troop coming, many-headed.\evb\evg


\bvg\bva\mssnote{\Regius~14v/28, \AM~6v/19}\alst{H}óf sér af \alst{h}ęrðum \hld\ \alst{h}ver standandi, &
vęifði \alst{M}jǫllni \hld\ \alst{m}orð-gjǫrnum framm, &
ok \alst{h}raun-\alst{h}vala \hld\ \alst{h}ann alla drap.\eva

\bvb He heaved off his shoulders the cauldron, standing; \\
he swung the murder-eager Millner forth, \\
and the rock-whales \ken{ettins} all he slew.\evb\evg


\bvg\bva\mssnote{\Regius~14v/30, \AM~6v/21}\edtext{Fóru-t \alst{l}ęngi, \hld\ áðr \alst{l}iggja nam &
\alst{h}afr \alst{H}lórriða \hld\ \alst{h}alf-dauðr fyrir, &
vas \edtext{\alst{sk}ę́r}{\Afootnote{emend. from meaningless \emph{†skirr†} \Regius\AM}} \alst{sk}ǫkuls \hld\ \alst{sk}akkr á bęini, &
en því hinn \alst{l}ę́-vísi \hld\ \alst{L}oki of olli.}{\lemma{Fóru-t \dots\ olli. ‘They journeyed \dots\ did cause.’}\Bfootnote{Lock, who is not mentioned earlier in the poem, was apparently placing curses on the returning party.  Snorre mentions this, TODO.}}\eva

\bvb They journeyed not for long before Loride’s \name{= Thunder’s} he-goat \\
took to lie half-dead before [them]; \\
the steed of the cart-pole \ken{goat} was halt in the leg, \\
and that the guile-wise Lock did cause.\evb\evg


\bvg\bva\mssnote{\Regius~14v/32, \AM~6v/22}En \edtrans{ér}{ye}{\Bfootnote{The audience.  As pointed out by \textcite{FinnurEdda} an address to the audience of this type is otherwise unparalleled in Eddic mythological poetry.  Such are however fairly common in Scaldic poetry, with which this poem shares several traits (see Introduction above).}} \alst{h}ęyrt \alst{h}afið, \hld\ \alst{h}vęrr kann umb þat &
\edtrans{\alst{g}oð-mǫ́lugra}{god-speaking}{\Bfootnote{This word is a hapax, but easily understood.  One who is \emph{goð-mǫ́lugr} is ‘able to speak about the god-lore’, i.e. ‘versed in the mythology’.}} \hld\ \alst{g}ørr at skilja, &
\alst{h}vęr af \alst{h}raun-búa \hld\ \alst{h}ann laun of fekk, &
es \alst{b}ę́ði galt \hld\ \alst{b}ǫrn sín fyrir.\eva

\bvb But ye have heard—about that can \\
any god-speaking man more clearly discern— \\
which recompense he \ken*{= Thunder} from the lavafield-dweller \ken{ettin} got, \\
as he yielded up both his own children for it.\evb\evg


\bvg\bva\mssnote{\Regius~15r/1, \AM~6v/24}\alst{Þ}rótt-ǫflugr kom \hld\ á \alst{þ}ing goða &
ok \alst{h}afði \alst{h}ver, \hld\ þann’s \alst{H}ymir átti; &
en \alst{v}éar hvęrjan \hld\ \alst{v}ęl skulu drekka &
\alst{ǫ}lðr at \alst{Ę́}gis \hld\ \edtrans{\alst{ęi}tt hǫr-męitið}{one \dots\ flax-cutting}{\Bfootnote{A very obscure kenning. \textcite{LaFargeGlossary} give several interpretations, viz. \emph{ęitr-hǫr-męitir} ‘poison-rope-cutter \ken{snake > winter}’, \emph{ęitr-orm-męiðir} ‘poison-worm-injurer’ \ken{winter}. The solution with the minimal amount of emendation is to read \emph{ęitt} ‘one’ as modifying \emph{ǫlðr} ‘ale-feast’, and \emph{hvęrjan} ‘every’ as modifying \emph{hǫr-męitiðr} ‘flax-cutting’, a compound made up of \emph{hǫrr} ‘flax, cord’ and \emph{męita} ‘to cut’, seemingly referring to an obscure harvest festival. This interpretation is by no means certain.}}.\eva

\bvb The valour-mighty one \ken*{= Thunder} came onto the \inx[C]{Thing} of the gods, \\
and had that cauldron which Hymer [had] owned; \\
but well the \inx[G]{Wighers} \name{= gods} shall drink one \\
ale-feast at Eagre’s, every flax-cutting \ken{fall?}.\evb\evg
