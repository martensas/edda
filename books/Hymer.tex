\bookStart{Lay of Hymer}[Hymiskviða]

\begin{flushright}%
\textbf{Dating} \parencite{Sapp2022}: C10th (0.694)

\textbf{Meter:} \Fornyrdislag%
\end{flushright}%

\section{Introduction}

The \textbf{Lay of Hymer} (\Hymiskvida) is attested in both \Regius\ and \AM.  The two mss. agree very well with each other; they share the same stanzas in the same order.  The most substantial difference is the title; \AM\ has \emph{Hymis kviða} ‘the lay of Hymer’ while \Regius\ instead has \emph{Þórr dró Miðgarðs-orm} ‘Thunder pulled the Middenyardswyrm’.

\subsection{Content}

At its core \Hymiskvida\ is a comedy about Thunder’s adventures in Ettinland.  This seems to have been a popular genre, which in the Poetic Edda is also represented by \Thrymskvida\ and to some degree \Harbardsljod.  Other related stories are Thunder’s journey to Outyards-Lock in \Gylfaginning\ 44–47, his fight with Rungner in \Skaldskaparmal\ 24, and his journey to Garfrith in \Skaldskaparmal\ 26 (edited in the present edition under Eddic fragments).  These tales involve fantastical events and a fair bit of humour, and usually end with Thunder having slaughtered yet more Ettins.

\subsubsection{The otherness of the Ettins}

The Ettins are very much an \emph{other} to the Gods, and this is something which \Hymiskvida\ strongly emphasizes:

\begin{itemize}
  \item They live in the far east (st. 5) in an inhospitable, frozen climate (st. 10) of mountains (sts. 2, 17) and lavafields (sts. 36, 38);
  \item they are physically deviant: misshapen (st. 10), grey-haired (st. 16), many-headed (sts. 8, 35), having bodies harder than stone (sts. 30–31);
  \item they are likened to apes (st. 20), whales (st. 36) and Danes (st. 17, see note!);
  \item they are stingy and inhospitable (sts. 9, 16);
  \item they are snide and cowardly (sts. 19–20, 25–26, 28–32).
\end{itemize}

In general the Ettins stand in direct opposition to the Old Germanic social norms, as represented by the Gods; \emph{they} live in a lush green land and are young, beautiful, generous, and brave.  The one exception in the poem is Tew’s mother in st. 8, who is blonde, beautiful, and hospitable; the mother of a god must also be godlike.

As natural inferiors and a threat to the social order the Ettins must be subjugated by the Gods, and the agent of this is Thunder.  Throughout the poem he constantly humiliaties the ettins Eagre and Hymer, recurringly through completing their challenges, which follow a similar scheme: Thunder is given a dangerous or near-impossible test of strength, but quickly accomplishes it through a combination of brawn and brain, humiliating the challenger.  The challenges consist of finding an enormously large kettle (st. 3, explicitly called Eagre’s “revenge”), wrestling one of Hymer’s oxen for bait (sts. 17–18), carrying home Hymer’s whales and boat (st. 26), breaking Hymer’s finest chalice (st. 28), and perhaps also taking away the cauldron (st. 33)—though that may just be Hymer wishing to finally be rid of the pestering gods.

In the end Thunder delivers justice by slaughtering Hymer and his troop of many-headed Ettins, probably his clansmen.

\subsubsection{The fishing expedition}

At the center of the poem stands Thunder’s fishing expedition, where he gets the Middenyardswyrm on his hook but ultimately fails to catch it.  One here finds a more reverent tone than elsewhere in the poem, especially in sts. 22–24.

This myth was very popular in the Wiking Age and is dealt with in five fragmentary Scaldic poems from the 9th or 10th centuries.  These are all found in quotations in \Skaldskaparmal; they are (by their \Skp\ 3 sigla) Bragi \emph{Þórr}, ÚlfrU \emph{Húsdr} 3–6, Ǫlv \emph{Þórr}, \emph{EVald} Þórr, and Ggnæv \emph{Þórr}.  In their present state the fragments are not complete narratives, but specifically focus on Thunder in the boat facing off against the hooked Wyrm pressed against the gunwale.  They also disagree on the course of events; in some of them the staring contest ends when the cowardly Hymer cuts the fishing line and the Wyrm sinks back unscathed into the sea (the version preferred by \Gylfaginning\ 48)—in others Thunder strikes the head off the Wyrm, slaying it.

In addition to literary sources there are also numerous pictorial depictions of the myth from the Wiking Age.  These are the Swedish runestones from Altuna (U 1611) and Linga (Sö 352), several Jutlandic picture stones from Hørdum, a Cumbrian picture stone from Gosforth, and the Gotlandic picture stone GP 21 from Ardre church.  The images depict the same scene as the Scaldic fragments: Thunder stands in the boat above the hooked Wyrm, often depicted as a fish; next to him is one companion.  Some of them have additional details like the use of the ox-head for bait (U 1611, Sö 352), or Thunder’s foot going through the boat (U 1611, Hørdum).

Other than \Hymiskvida\ the only complete retelling of the myth is found in \Gylfaginning\ 48, which may be summarized as follows:

{\small Thunder goes out alone into Middenyard in the shape of a young man (\emph{ungr dręngr}) without his goats and chariot.  In the evening he comes to the ettin Hymer and asks to stay the night.  At dawn Hymer plans to go fishing and Thunder asks to join him.  Hymer insults Thunder's small size and youth, and warns him that he usually takes long and arduous trips.  Thunder, angered, says that he will row very far, and then asks Hymer what bait they will use.  Hymer tells him to find it himself and so he turns to the flock of oxen, where he tears off the head from the greatest ox, one called Heavenrid (\emph{Himin-hrjóðr}).

The two go out to sea, and Thunder rows far past Hymer’s usual fishing waters.  Hymer, unhappy, warns him that if they row any further out they will be in danger of the Middenyardswyrm, but Thunder keeps on.  After some time he puts down the oars, readies his fishing line, hooks the ox-head and lowers it.  The Wyrm soon bites, and struggles so hard that Thunder is pressed against the gunwale.  In rage he brings himself into his Os-might (\emph{ás-męgin}) and pulls back with such force that his feet go through the bottom of the ship and press into the seabed.  The Wyrm's head goes up against the gunwale.  The two enemies ferociously stare at each other, Thunder “sharpening his eyes” and the Wyrm spitting venom.  Hymer is frightened, reaches for his bait-cutting knife, and cuts the line—the Wyrm then sinks back into the sea.  Thunder throws his hammer after it, “and men say that he struck off the monster’s head, but I think it true to tell thee that the Middenyardswyrm still lives and is lying in the outer sea.”  Thunder gives Hymer a punch to the ear so that he flies headfirst overboard; the god then wades back to land.}

This account is clearly based on multiple sources, certainly including the Scaldic fragments cited in \Skaldskaparmal.  It is hard to say whether Snorre had access to \Hymiskvida; the closest agreement is when it is said that \emph{Miðgarðs-ormr gein yfir uxa-hǫfuð’it, en ǫngull’inn vá í góm’inn orm’inum} ‘The Middenyardswyrm snapped at the ox-head and the hook went into the roof of the wyrm’s mouth’, which has some resemblance to st. 22, but it is not conclusive.  Some details must derive from now-lost texts available to Snorre: the detail of Thunder’s feet going through the boat is also found on the Swedish Altuna stone and the Danish Hørdum stone (but see note to st. 34/2 below), and the name Heavenrid is attested in \inx[C]{thule}[thules] listing names of oxen.

\sectionline

More broadly, Thunder’s fishing reflects the archetypal fight between the Storm-god and the Dragon found in a great many mythologies.  Important examples of this include Vedic Indra and Vr̥tra (\Rigveda\ 1.32 et. c.), Babylonian Marduk and Tiamat (\emph{Enūma Eliš}), Greek Zeus and Typhon, Hebrew Yahweh and Leviathan (TODO: references).  With these analogies in mind it seems that the versions where Thunder slays the Wyrm reflect an older layer of Germanic mythology, before the lethal fight between Thunder and the Wyrm had been transposed to the End Times (see \Voluspa\ 54).

\subsubsection{\Hymiskvida\ as a composite}

In \Hymiskvida\ one can roughly identify the following strands:

\begin{enumerate}
  \item 1–6 The Gods wish to drink, and Thunder goes to Eagre to make him host; Eagre in turn asks for a cauldron big enough to brew enough ale for all the Gods.
  \item 7–16 Thunder and Tew go to visit Tew’s father, the stingy ettin Hymer, who owns such a cauldron; horrified at Thunder’s great appetite during the evening he tells them that they must go fishing for food.
  \item 17–19 Thunder says that he will do it, if he is given bait; Hymer challenges him to kill one of his oxen; Thunder tears off the head from one of them.
  \item 20–25 The three go fishing; Hymer pulls up some whales; with the ox-head as bait Thunder manages to hook the Middenyardswyrm itself; he loses it.
  \item 26–27 Hymer challenges Thunder to carry the boat and whales back to his farm; he does.
  \item 28–32 Hymer challenges Thunder to break a supposedly indestructible chalice; he succeeds by smashing it against the ettin’s forehead.
  \item 33–36 Thunder and Tew depart with the cauldron; they find themselves followed by Hymer and his ettins; Thunder kills them all.
  \item 37–38 One of Thunder’s goats goes halt.
  \item 39 Thunder returns to the Gods with Hymer’s cauldron; they host a banquet.
\end{enumerate}

The fishing expedition as found in the Scaldic fragments and \Gylfaginning\ 48 is represented by 3–4.  \Hymiskvida\ is the only source that places it within the context of Thunder and Tew obtaining a huge cauldron from Hymer for the sake of brewing ale, and also scatters several other incidents throughout.  It seems inescapable to presume, both from the other sources just mentioned and broader comparative mythology, that these additional narratives originally had nothing to do with Thunder’s encounter with the Wyrm.

These strands have been woven together into a single narrative, perhaps even by the poet himself for the sake of a more entertaining and complete story. This weaving has not been entirely successful, and there are a few loose threads.  The halt goat of sts. 37–38 finds a parallel in \Gylfaginning\ 44, where it serves as the origin story of Thunder’s two servants who are to play an important part in the narrative, but it is here an entirely superfluous detail—something the poet himself anticipates in his address to the audience.  It is also strange that Lock should appear at this point, since he is never mentioned before or since.

Another loose strand is the god Tew, who plays no role at all in the fishing expedition: he is last alluded to in st. 16 where Hymer speaks of “[us] three”, and then reappears in st. 33 where he fails to lift the cauldron.  The simplest explanation for this is that he originally had nothing to do with fishing; his role is to bridge the frame-narrative of the cauldron and the fishing expedition.  In the other variants of the latter Thunder only has one companion, Hymer; this includes the pictoral depictions, which only show two figures on the boat.  Moreover, it is strange that Tew has no reaction to the murder of his father in front of him, although that paternity is in doubt; Tew is elsewhere called the son of \inx[P]{Weden} (\Skaldskaparmal\ 16), so that Hymer may perhaps be his stepfather.  This would reflect the common motif of a god mating with a beautiful ettin-woman, e.g. in \Skirnismal.

\subsection{Style}

When speaking of a composite poem, one must distinguish between a text where several separate works have been put together mostly unchanged and a text composed by a single author drawing from multiple sources.  A likely example of the former is \Havamal, but \Hymiskvida\ undoubtedly belongs to the latter category.  It has a distinct style and meter throughout which is unlike anything else in the Poetic Edda; indeed, the sharpest contrast is with the poem most similar content-wise, \Thrymskvida.  Where \Thrymskvida\ is written in a rustic style with fairly loose \Fornyrdislag\ meter and few kennings, \Hymiskvida\ uses an unusually strict meter and is filled with kennings, difficult grammatical constructions, and highly unnatural word order (see especially sts. 16, 20, and 39).

These are all traits one associates more closely with Scaldic poetry in intricate measures like \Drottkvett\ than Eddic poetry in \Fornyrdislag, and it seems clear that the anonymous poet of \Hymiskvida\ had some training in the Scaldic art and was familiar with compositions in that genre.  Two Scaldic-type kennings (17/4a \emph{brjótr berg-Dana}, 22/4 \emph{umb-gjǫrð allra landa}) are even shared identically with poems in \Drottkvett.

\subsubsection{Meter}

The meter of \Hymiskvida\ is \Fornyrdislag, but of a more strict variant than any other Eddic poem; this is especially true when it comes to the count and weight of syllables.  The poet also has a notable preference for lines of types A1s, C, and D, where the first two syllables are heavy and the third one is light, e.g. 1/4b \emph{ør-kost hvera} (type A1s), 1/2a \emph{ok sumbl-samir} (type C), and 2/4b \emph{opt sumbl gøra} (type D).  For the ambiguity between A1s and D see Suzuki (2014:116--119).%TODO: bibliography.

This preference probably explains his tendency to place the two-syllable preposition \emph{fyrir} ‘before, in front, (up) ahead’ at the end of the b-verse (never the a-verse), which he does 6 times—more frequently than in any other \Fornyrdislag\ poem of the Poetic Edda.

\sectionline

\section{The Lay of Hymer}

\bvg\bva\mssnote{\Regius~13v/26, \AM~5v/25}%
Ár \alst{v}al-tívar \hld\ \alst{v}ęiðar nǫ́mu &
ok \alst{s}umbl-\alst{s}amir \hld\ \edtrans{áðr \alst{s}aðir yrði}{before they might eat}{\Bfootnote{Lit. “might become sated”.}}, &
\edtrans{\alst{h}ristu tęina \hld\ ok á \alst{h}laut sǫ́u}{they shook the twigs and looked at the leat}{\Bfootnote{The Gods performed an augury, the means of which are not clear from this stanza alone.  The term “leat” (\emph{hlaut}) is explained in \HakonarSaga\ and \EyrbyggjaSaga\ as the sacrificial blood of the slaughtered beasts, which was sprinkled by means of “leat-twigs” (\emph{hlaut-tęinar}).  If we trust these sources the simplest explanation is that the Gods sprinkled the animal blood and interpreted the pattern formed.  In any case they found it most auspicious to feast at Eagre’s.}}, &
fundu at \alst{Ę́}gis \hld\ \alst{ø}r-kost hvera.\eva

\bvb Of yore the slain-Tews \name{Gods} had caught game, \\
and assembled at the \inx[C]{simble} before they might eat \\
they shook the twigs and looked at the \inx[C]{leat}; \\
they found at Eagre’s a great choice of cauldrons.\evb\evg


\bvg\bva\mssnote{\Regius~13v/28, \AM~5v/27}%
Sat \alst{b}erg-\alst{b}úi \hld\ \alst{b}arn-tęitr fyrir, &
\alst{m}jǫk glíkr \edtrans{\alst{m}ęgi \hld\ \alst{M}iskur-blinda}{lad of Misherblind}{\Bfootnote{An unexplained reference.  Misherblind might be another name for Firneet, Eagre’s father, in which case the line would be a tautology: “he looked much like himself”.}}, &
lęit í \alst{au}gu \hld\ \alst{Y}ggs barn í þrá: &
„þú skalt \alst{ǫ́}sum \hld\ \alst{o}pt sumbl \edtext{gøra}{\lemma{gøra ‘make’}\Afootnote{\emph{gefa} ‘give’ \AM}}!“\eva

\bvb The crag-dweller \ken*{\textsc{ettin} = Eagre} sat merry like a child before [him] \\
much alike to the lad of Misherblind. \\
Into his eyes looked the Ug’s \name{Weden’s} child \ken*{= Thunder} stubbornly: \\
“Thou shalt for the Eese oft make simbles!”\footnoteB{Having seen that Eagre has a great store of cauldrons, Thunder orders him to brew ale for the feasts of the Eese.}\evb\evg


\bvg\bva\mssnote{\Regius~13v/31, \AM~5v/29}%
\alst{Ǫ}nn fekk \alst{jǫ}tni \hld\ \alst{o}rð-bę́ginn halr, &
\alst{h}ugði at \alst{h}efndum \hld\ \alst{h}ann nę́st við goð, &
bað \alst{S}ifjar ver \hld\ \alst{s}ér fǿra hver, &
„þann’s ek \alst{ǫ}llum \alst{ǫ}l \hld\ \alst{y}ðr of hęita.“\eva

\bvb Great toil for the ettin the word-peevish man \ken*{= Thunder} caused; \\
he \ken*{= Eagre} thought of revenge, soon, against the god. \\
He bade Sib’s husband \ken*{= Thunder} bring him a cauldron, \\
“that one with which I for you all ale might warm.\footnoteB{Eagre gets back at Thunder by telling him that he needs a single cauldron which can hold enough ale to supply all the Eese.}”\evb\evg


\bvg\bva\mssnote{\Regius~14r/1, \AM~5v/30}%
Né þat \alst{m}ǫ́ttu \hld\ \alst{m}ę́rir tívar &
ok \alst{g}inn-ręgin \hld\ of \alst{g}eta hvęr-gi, &
unds af \alst{t}ryggðum \hld\ \alst{T}ýr Hlórriða &
\alst{ǫ́}st-ráð mikit \hld\ \alst{ęi}num sagði:\eva

\bvb That one could not the renowned \inx[G]{Tews} \\
and the \inx[G]{yin-Reins} anywhere get hold of— \\
until, out of loyalty, Tew to Loride \name{= Thunder} \\
a great loving counsel told alone:\evb\evg


\bvg\bva\mssnote{\Regius~14r/3, \AM~6r/2}„Býr fyr \alst{au}stan \hld\ \alst{É}li-vága &
\edtrans{\alst{h}und-víss}{hundred-wise}{\Bfootnote{Alternatively “hound-wise”; the prefix simply means “very”.}} \alst{H}ymir \hld\ at \alst{h}imins ęnda, &
á \alst{m}inn faðir \hld\ \alst{m}óðugr kętil, &
\edtext{\alst{r}úm-brugðinn}{\Afootnote{\emph{†rumbrygðan†} \AM}} hver \hld\ \alst{r}astar djúpan.“\eva

\bvb “Dwells to the east of the \inx[L]{Ilewaves} \\
the hundred-wise Hymer, at heaven’s end.\footnoteB{According to \Vafthrudnismal\ 31 the Ilewaves were the poisonous wild rushes from which the ettins emerged, and so it makes sense that they would be found in the east, where the ettins dwell.  That Hymer should dwell even to the east of them then illustrates his unusual ettin-ness.} \\
Owns my father \ken*{= Hymer}, fierce, a kettle: \\
a size-famed cauldron one \inx[C]{rest} deep.”\evb\evg


\bvg\bva\speakernote{[Þórr kvað:]}\mssnote{\Regius~14r/4, \AM~6r/4}%
„Vęitst, ef \alst{þ}iggjum \hld\ \alst{þ}ann lǫg-velli?“ &
\speakernote{[Týr kvað:]}„Ef, \alst{v}inr, \alst{v}élar \hld\ \alst{v}it gørvum til!“\eva

\bvb\speakernoteb{[Thunder quoth:]}%
“Knowest thou if we will receive that liquid-boiler \ken{cauldron}?” — \\
\speakernoteb{[Tew quoth:]}%
“If, friend, we two make use of wiles!”\footnoteB{Like elsewhere in this poem the speakers are not indicated, but it is most sensible that Thunder asks and Tew answers.}\evb\evg


\bvg\bva\mssnote{\Regius~14r/5, \AM~6r/4}%
Fóru \alst{d}rjúgum \hld\ \edtrans{\alst{d}ag þann framan}{from the beginning of the day}{\Afootnote{emend. after \textcite{FinnurEdda}; \emph{dag þann fram} ‘on that day forth’ \Regius; \emph{dag fráliga} ‘swiftly at day’ \AM}} &
\alst{Á}sgarði frá \hld\ unds til \edtrans{\alst{Ę}gils}{Eyel}{\Afootnote{so \Regius; \emph{Ę́gis} ‘Eagre’ \AM\ is probably from confusion with Eagre (the ettin) described earlier in the poem, though the shepherd may have shared his name.}} kvǫ́mu; &
\edtrans{\alst{h}irði \alst{h}afra \hld\ \alst{h}orn-gǫfgasta}{he kept the he-goats noblest of horns}{\Bfootnote{Eyel is not otherwise known but he seems to have been familiar to the original audience.  In any case he takes possession of Thunder’s two goats until he returns.}}; &
\alst{h}urfu at \alst{h}ǫllu \hld\ es \alst{H}ymir átti.\eva

\bvb They journeyed far from the beginning of the day, \\
away from Osyard, until to Eyel they came— \\
he kept the he-goats noblest of horns— \\
they turned to the hall which Hymer owned.\evb\evg


\bvg\bva\mssnote{\Regius~14r/7, \AM~6r/6}%
\alst{M}ǫgr fann ǫmmu, \hld\ \alst{m}jǫk lęiða sér, &
\edtrans{\alst{h}afði \alst{h}ǫfða \hld\ \alst{h}undruð níu}{of heads she had nine hundred}{\Bfootnote{Malformed bodies, especially with a deviant number of body parts, are typical of ettins.  Other examples include a three-headed thurse in \Skirnismal\ 31, the nine-headed ettin Thriwold (Bragi Frag 3 in \Skp\ 3), and the eight-armed Starked Eeldreng.  Cf. Introduction and st. 35 below.}}. &
en \edtrans{\alst{ǫ}nnur}{another woman}{\Bfootnote{The use of the word “son” in the following line reveals this as Tew’s mother.  The poet stresses her beauty of dress and countenance, in contrast to the grandmother.}} gekk \hld\ \alst{a}l-gullin framm &
\alst{b}rún-hvít \alst{b}era \hld\ \alst{b}jór-vęig syni:\eva

\bvb The lad \ken*{= Tew} found his grandmother very loathsome; \\
of heads she had nine hundred. \\
But another woman, all-golden, walked forth, \\
white-browed, bringing a beer-draught for [her] son \ken*{= Tew}:\evb\evg


\bvg\bva\speakernote{[Týs móðir:]}\mssnote{\Regius~14r/9, \AM~6r/8}%
„\alst{Á}tt-niðr \alst{jǫ}tna \hld\ \alst{e}k vilja’k ykkr &
\alst{h}ug-fulla tvá \hld\ und \alst{h}vera sętja; &
es \alst{m}ínn \edtrans{fríi}{lover}{\Afootnote{so \Regius; \emph{faðir} ‘father’ \AM}} \hld\ \alst{m}ǫrgu sinni &
\edtext{\alst{g}løggr við \alst{g}ęsti \hld\ \alst{g}ǫrr ills hugar}{\lemma{gløggr \dots\ hugar ‘stingy \dots\ mood’}\Bfootnote{Ettins are characteristically inhospitable, in stark opposition to the Old Germanic social norms; see Introduction to the poem above.  This statement foreshadows the later hunting expedition starting at st. 16 below.}}.“\eva

\bvb\speakernoteb{[Tew’s mother:]}“O clansman of ettins \ken*{= Tew}! I would wish to put \\
you two, full of heart, beneath the cauldrons. \\
Many a time has my lover \ken*{= Hymer} been \\
stingy with guests, quick to ill mood.”\evb\evg


\bvg\bva\mssnote{\Regius~14r/11, \AM~6r/9}%
En \alst{v}á-skapaðr \hld\ \alst{v}arð \edtrans{síð-búinn}{come late}{\Afootnote{om. \AM}}, &
\alst{h}arð-ráðr \alst{H}ymir, \hld\ \alst{h}ęim af vęiðum; &
\alst{g}ekk inn í sal, \hld\ \alst{g}lumðu \edtrans{jǫklar}{icicles}{\Bfootnote{In Hymer’s frozen beard.  In modern Icelandic the word \emph{jökull} has come to mean ‘glacier’, but its original sense (as found here) is that of its English cognate “icicle”.}}, &
vas \alst{k}arls, es \alst{k}om, \hld\ \alst{k}inn-skógr frørinn.\eva

\bvb And the misshapen one was come late, \\
hard-minded Hymer, home from the hunt. \\
He entered the hall; icicles clattered; \\
on the churl who came was the cheek-shaw \ken{beard} frozen.\evb\evg


\bvg\bva\speakernote{[Týs móðir:]}\mssnote{\Regius~14r/13, \AM~6r/11}%
„\edtext{Ves þú \alst{h}ęill, \alst{H}ymir, \hld\ í \alst{h}ugum góðum!}{\lemma{Ves þú hęill, \dots\ í hugum góðum! ‘Be thou hale \dots\ in good spirits!’}\Bfootnote{A formulaic greeting; cf. the almost identical greeting in \emph{N B380} (edited below under Galders).  Further afield cf. the type exemplified by \Beowulf\ 407a: \emph{Wæs þú, Hróðgâr, hâl} ‘Be thou, Rothgar, hale!’}} &
Nú ’s \alst{s}onr kominn \hld\ til \alst{s}ala þinna, &
sá’s \alst{v}it \alst{v}ę́ttum \hld\ af \alst{v}egi lǫngum; &
fylgir \alst{h}ǫ́num \hld\ \alst{H}róðrs and-skoti, &
\alst{v}inr \alst{v}er-liða; \hld\ \edtrans{\alst{V}éurr}{Wighward}{\Bfootnote{The guardian of \inx[C]{wigh}[wighs] (sanctuaries), a name of Thunder.}} hęitir sá.\eva

\bvb\speakernoteb{[Tew’s mother:]}“Be thou hale, Hymer, in good spirits! \\
Now the son has come to thy halls, \\
he whom we awaited, from a long way off. \\
Him follows the Rooder’s opponent \ken*{= Thunder}, \\
the friend of manly retinues—\inx[P]{Wighward} is he called.\evb\evg


\bvg\bva\mssnote{\Regius~14r/15, \AM~6r/13}\alst{S}é þú hvar \alst{s}itja \hld\ und \alst{s}alar gafli, &
\alst{s}vá \edtext{forða \alst{s}ér}{\Afootnote{\emph{forðask} \AM}}, \hld\ stęndr \edtrans{\alst{s}úl}{column}{\Afootnote{\emph{†sol†} \AM}} fyrir.“ &
\alst{S}undr stǫkk \alst{s}úla \hld\ fyr \alst{s}jón jǫtuns, &
en \edtext{\alst{a}llr}{\Afootnote{emend.; \emph{áðr} ‘earlier, before that’ \Regius\AM. TODO: elaborate, mention Finnur}} í tvau \hld\ \alst{á}ss brotnaði.\eva

\bvb See where they sit beneath the hall’s gable: \\
so they save themselves—a column stands before [them]!” \\
The column crashed down before the ettin’s gaze, \\
and all in two the roof-beam broke.\evb\evg


\bvg\bva\mssnote{\Regius~14r/17, \AM~6r/15}Stukku \alst{á}tta, \hld\ en \alst{ęi}nn af þęim &
\alst{h}verr \alst{h}arð-slęginn \hld\ \alst{h}ęill af þolli; &
\alst{f}ramm gingu þęir, \hld\ en \alst{f}orn jǫtunn &
\alst{s}jónum lęiddi \hld\ \alst{s}inn and-skota.\eva

\bvb Eight [cauldrons] crashed down, but one of them, \\
a hard-forged cauldron, [came] whole off its peg.\footnoteB{Nine cauldrons were hanging from the roof-beam supported by the column.  Eight of them broke and one remained whole, presumably the one they were looking for.} \\
Forth they went, but the ancient ettin \\
with his gaze tracked his opponent.\evb\evg


\bvg\bva\mssnote{\Regius~14r/19, \AM~6r/16}\edtrans{Sagði-t \alst{h}ǫ́num \hld\ \alst{h}ugr vęl}{His heart did not please him}{\Bfootnote{Lit. ‘his heart did not speak well to him’.}} þá’s sá &
\alst{g}ýgjar \edtrans{\alst{g}rǿti}{distresser}{\Afootnote{\emph{gę́ti} ‘keeper, warder’ \AM}} \hld\ á \alst{g}olf kominn, &
\alst{þ}ar vǫ́ru \alst{þ}jórar \hld\ \alst{þ}rír of tęknir, &
bað \edtrans{\alst{s}ęnn}{at once}{\Afootnote{\emph{sun} ‘[his] son \ken*{= Tew}?’ \AM}} jǫtunn \hld\ \alst{s}jóða ganga.\eva

\bvb His heart did not please him when he saw \\
the \inx[C]{gow}’s distresser \ken*{= Thunder} come on the floor. \\
There were three bulls a-taken: \\
the ettin bade them at once go cooking.\evb\evg


\bvg\bva\mssnote{\Regius~14r/21, \AM~6r/18}\alst{H}vęrn létu þęir \hld\ \alst{h}ǫfði skęmra &
auk á \alst{s}ęyði \hld\ \alst{s}íðan bǫ́ru, &
át \alst{S}ifjar verr \hld\ áðr \alst{s}ofa gingi, &
\alst{ęi}nn með \alst{ǫ}llu \hld\ \alst{ø}xn tvá Hymis.\eva

\bvb Each one they let shorten by a head, \\
and onto the cooking-pit then did bear: \\
Sib’s husband \ken*{= Thunder} ate—before he might go sleep— \\
alone by himself two of Hymer’s oxen.\footnoteB{Cf. \Thrymskvida\ 24 for another instance of Thunder’s great eating, which curiously also uses the kenning \emph{Sifjar verr} ‘Sib’s husband \ken*{= Thunder}’.}\evb\evg


\bvg\bva\mssnote{\Regius~14r/23, \AM~6r/19}Þótti \alst{h}ǫ́rum \hld\ \alst{H}rungnis spjalla &
\alst{v}erðr Hlórriða \hld\ \alst{v}ęl full-mikill, &
\edtext{„munum at \alst{a}ptni \hld\ \alst{ǫ}ðrum verða &
\alst{v}ið \alst{v}ęiði-mat \hld\ \alst{v}ér þrír lifa.“}{\lemma{munum \dots\ lifa. ‘the next \dots\ live.’}\Bfootnote{The poet is pushing at the limits of Old Norse syntax.  In prose word order it should be construed as: \emph{at ǫðrum aptni munum vér þrír verða lifa við vęiði-mat}, where \emph{verða} ‘have to, must’ is used like its modern German cognate \emph{werden}.

Hymer’s stinginess—he refuses to share more of his own food but instead forces his guests to go hunt—breaks all Indo-European rules of hospitality and illustrates the otherness of the Ettins.  See the Introduction above.}}\eva

\bvb To Rungner’s hoary friend \ken*{= Hymer} did seem \\
Loride’s \name{Thunder’s} eating far too great; \\
“the next evening we three will \\
on game-meat have to live.”\evb\evg


\bvg\bva\mssnote{\Regius~14r/24, \AM~6r/21}\alst{V}éurr kvaðsk \alst{v}ilja \hld\ á \alst{v}ág róa, &
ef \alst{b}allr jǫtunn \hld\ \alst{b}ęitur gę́fi. &
„\alst{H}verf þú til \edtext{\alst{h}jarðar}{\Afootnote{\emph{hallar} corr. \AM}}, \hld\ ef \alst{h}ug trúir, &
\edtrans{\alst{b}rjótr \alst{b}erg-Dana}{breaker of boulder-Danes \ken*{\textsc{ettins} > = Thunder}}{\Bfootnote{This kenning for Thunder also occurs in \Haustlong\ 18.  The ettin-kenning emphasises their otherness (see Introduction to the poem above) by equating them with ethnic foreigners.  Cf. \Thorsdrapa, where ettins are called Scots, Swedes, Danes, Ruges and Hareds; all peoples hostile to the Norwegian Earl Hathkin, at whose court that poem may have been composed.}}, \hld\ \alst{b}ęitur sǿkja.\eva

\bvb Wighward called himself willing to row on the wave, \\
if the stubborn ettin might give pieces of bait. \\
“Turn to the herd—if thou trust in thy heart, \\
O breaker of boulder-Danes \ken*{\textsc{ettins} > = Thunder}—to seek pieces of bait.\evb\evg


\bvg\bva\mssnote{\Regius~14r/26, \AM~6r/23}\alst{Þ}ess \edtext{vę́ntir mik}{\Afootnote{so \AM; \emph{vę́nti ek} \Regius}}, \hld\ at \alst{þ}ér \edtrans{myni-t}{will not}{\Afootnote{so \AM; \emph{myni} ‘will’ \Regius.  The \AM\ reading is preferable since it makes this the first of Hymer’s several challenges of strength to Thunder, which the god, to the ettin’s humiliation, easily accomplishes.}} &
\alst{ǫ}gn at \alst{o}xa \hld\ \alst{au}ð-feng vesa.“ &
\edtrans{\alst{S}vęinn}{The swain}{\Bfootnote{Thunder was apparently in the shape of a young boy.  This detail is also found in \Gylfaginning\ 48: \emph{Gekk hann út of Miðgarð svá sem ungr drengr \dots} ‘He went out about Middenyard in the shape of a young man’.}} \alst{s}ýsliga \hld\ \alst{s}vęif til skógar, &
þar’s \edtext{\alst{o}xi stóð \hld\ \alst{a}l-svartr}{\lemma{oxi \dots\ al-svartr ‘ox \dots\ all-black’}\Bfootnote{Formulaic, also occuring in \Thrymskvida\ 23; see note there for further parallels to the custom of sacrificing animals of certain colours.  It seems that all-black oxen were thought the noblest, and so Thunder’s slaying one instead of an inferior beast is probably intended to humiliate the stingy Hymer.

In \Gylfaginning\ 48 we read that: \emph{Hann tók inn mesta uxa’nn, er Himin-hrjóðr hét, ok sleit af hǫfuð’it ok fór með til sjávar.} ‘He took the greatest ox, which was called Heavenrid, and tore off its head and went with it to the sea’.}} fyrir.\eva

\bvb I ween that the baits from the ox \\
will not be an easy catch for thee!”— \\
The swain \ken*{= Thunder} swiftly turned to the wood, \\
where an ox stood, all-black, before [him].\evb\evg


\bvg\bva\mssnote{\Regius~14r/28, \AM~6r/24}Braut af \alst{þ}jóri \hld\ \alst{þ}urs ráð-bani &
\alst{h}ǫ́-tún ofan \hld\ \alst{h}orna tveggja. &
„\alst{V}erk þikkja þín \hld\ \alst{v}erri myklu &
\alst{k}jóla valdi \hld\ an \alst{k}yrr sitir.“\eva

\bvb From the bull broke the thurse’s death-planner \ken*{= Thunder} \\
the high meadow of the two horns \ken{head} from above.— \\
“Worse by far thy works do seem \\
to the wielder of ships \ken*{= Hymer = me} than if thou didst sit calm!”\evb\evg

\sectionline

{\small (The scene now shifts, and the party is out at sea.  It is possible that a stanza has here been lost, or that it would be indicated in some other way in the original performance.)}

\sectionline

\bvg\bva\mssnote{\Regius~14r/30, \AM~6r/26}%
Bað \alst{h}lunn-gota \hld\ \alst{h}afra dróttinn &
\edtext{\alst{á}tt-runn}{\Afootnote{\emph{†atrænn†} \AM}} \edtrans{\alst{a}pa}{ape}{\Bfootnote{The specific sense of \emph{api} ‘ape’ is uncertain.  It seems to generally refer to a fool, but see Encyclopedia.}} \hld\ \alst{ú}tar fǿra, &
\edtext{en \alst{s}á jǫtunn \hld\ \alst{s}ína \edtext{talði}{\Afootnote{\emph{milldi} corr. \AM}}, &
\alst{l}ítla fýsi \hld\ \edtext{\alst{l}ęngra at róa}{\Afootnote{metr. emend.; \emph{at róa lęngra} \Regius\AM}}.}{\lemma{en \dots\ róa. ‘but \dots\ longer.’}\Bfootnote{Thunder’s humorous humiliation of Hymer continues with the snide ettin now forced to row against his will.}}\eva

\bvb The Lord of He-goats \ken*{= Thunder} bade the kinsman of the \inx[C]{ape}\ \ken*{\textsc{ettin} = Hymer} \\
push the launcher-steed \ken{boat} further out, \\
but that ettin told of his \\
scarce wish to row longer.\evb\evg


\bvg\bva\mssnote{\Regius~14r/31, \AM~6r/27}Dró \edtrans{\alst{m}ę́rr}{famous}{\Afootnote{so \Regius; \emph{męir} ‘more, further’ \AM}} Hymir \hld\ \alst{m}óðugr hvala &
\alst{ęi}nn á \alst{ǫ}ngli \hld\ \alst{u}pp sęnn tváa; &
en \alst{a}ptr í skut \hld\ \alst{Ó}ðni sifjaðr &
\alst{V}éurr við \alst{v}élar \hld\ \alst{v}að gęrði sér.\eva

\bvb Famous, fierce Hymer pulled whales: \\
one on the hook, soon up two. \\
But back in the stern the Weden-related \\
Wighward craftily fixed his line.\evb\evg


\bvg\bva\mssnote{\Regius~14v/1, \AM~6r/29}\alst{Ę}gnði á \alst{ǫ}ngul \hld\ sá’s \alst{ǫ}ldum bergr, &
\alst{o}rms \alst{ęi}n-bani \hld\ \alst{o}xa hǫfði; &
\alst{g}ęin við \edtrans{agni}{bait}{\Afootnote{so \AM; \emph{ǫngli} ‘hook’ \Regius}} \hld\ sú’s \alst{g}oð fía &
\edtext{\alst{u}mb-gjǫrð neðan \hld\ \alst{a}llra landa.}{\lemma{umb-gjǫrð \dots\ allra landa ‘engirdler of all lands’}\Bfootnote{Also found in a fragment by Alewigh Snub (\Skp: Ǫlv \emph{Þórr}) quoted in \Skaldskaparmal\ 11: \emph{\alst{Ǿ}stisk \alst{a}llra landa \hld\ \alst{u}mb-gjǫrð ok sonr Jarðar.} ‘The engirdler of all lands and the son of Earth surged.’  Cf. also the Wyrm-kenning in Braye’s fragment quoted in the same chapter (\Skp: Bragi \emph{Þórr} 3): \emph{ęndi-sęiðr allra landa} ‘boundary-saithe of all lands’.

The poetic juxtaposition between the Storm-god and the Wyrm may be very old; cf. \Rigveda\ 1.32.13c: \emph{Índraś ca yád yuyudhā́tay Áhiś ca} ‘When Indra and the Wyrm (\emph{áhi}) fought each other.’}}\eva

\bvb Baited on the hook he who rescues men \ken*{= Thunder}—  \\
the Wyrm’s lone slayer—the ox’s head. \\
Snapped at the bait the one whom the Gods hate \ken*{= Middenyardswyrm}— \\
the engirdler of all lands—from below.\evb\evg


\bvg\bva\mssnote{\Regius~14v/3, \AM~6v/1}\alst{D}ró \alst{d}jarf-liga \hld\ \alst{d}áð-rakkr \edtrans{Þȯ\emph{u}rr}{Thunder}{\Bfootnote{Out of 8 three-syllable lines in \Hymiskvida, this is the only one which is present in both \Regius\ and \AM, and which cannot easily be emended by restoring an hiatus form.  In the quite strict meter (see Introduction above) observed by the poet we should expect a disyllabic form in this spot, and this may be had if we restore an archaic \emph{*Þȯurr} or \emph{*Þȯarr}.  This form is less secure than other hiatus forms, but is also required by the meter of \Hymiskvida\ 28/2b below and \Thorsdrapa\ 2/2b.  This issue is treated more fully in \textcite{Haukur2023}.}} &
\alst{o}rm \alst{ęi}tr-fá\emph{a}n \hld\ \alst{u}pp at borði; &
\alst{h}amri kníði \hld\ \edtrans{\alst{h}ǫ́-fjall skarar}{high mountain of hair \ken{head}}{\Bfootnote{A rather unfitting kenning, since serpents do not have hair.}} &
\alst{o}f-ljótt \alst{o}fan \hld\ \alst{u}lfs hnit-bróður.\eva

\bvb Bravely pulled deed-ready Thunder \\
the venom-gleaming Wyrm up on the gunwale. \\
With the hammer he struck the high mountain of hair \ken{head}— \\
very hideous, from above—on the Wolf’s clash-brother \ken*{= Middenyardswyrm}.\evb\evg


\bvg\bva\mssnote{\Regius~14v/5, \AM~6v/2}\edtrans{\alst{H}raun-gǫlkn}{The desert-monsters}{\Bfootnote{Both mss. have \emph{hręin-}, which may mean either ‘clean’ or ‘reindeer’, neither of which fit. On the other hand \emph{hraun} \ONP: ‘stone/barren area, wasteland; lavafield’ is well attested in Scaldic kennings for ettins. The precise meaning of \emph{galkn} ‘monster’ (plural \emph{gǫlkn}) is unclear; but it is attested in three Scaldic verses, always in kennings of the type “troll-woman of the shield \ken{axe}”.  While the mss. spelling ‘\emph{galkn}’ (norm. \emph{gálkn}) could reflect either singular and plural, the form of the verb is plural.  This means that the word cannot be referring to the Middenyardswyrm, refuting the interpretation of \textcite{LarringtonEdda}: “the sea-wolf shrieked”.}} \edtext{\alst{h}rutu}{\Afootnote{so \AM; \emph{hlumðu} ‘dashed’ \Regius. End-rhyme is also used by the poet in st. 3/3.}}, \hld\ ęn \alst{h}ǫlkn þutu, &
\alst{f}ór hin \alst{f}orna \hld\ \alst{f}old ǫll saman; &
\edtext{[...]}{\Bfootnote{It is very likely that a line is missing here, since the stanzas in the poem otherwise consistently have four lines.  In other tellings of the myth it is at this point that Hymer cuts Thunder’s fishing line, so that is probably what has been lost.

For the reader’s enjoyment, based on other poets and \Gylfaginning\ 48, the translator has composed the following variant lines: \emph{unds vinr Hrungnis \hld\ vað Þórs of skar} ‘until the friend of Rungner \ken*{= Hymer} Thunder’s fishing-line did cut’; \emph{unds fǫlr Hymir \hld\ fekk á saxi} ‘until pale Hymer grasped the knife’.}} &
\alst{s}økkðisk \alst{s}íðan \hld\ \alst{s}á \edtrans{fiskr}{fish}{\Bfootnote{The Middenyardswyrm may also be called a fish in \Grimnismal\ 21; see note there.  In Scaldic sources it is often called a saithe (\emph{sęiðr}).}} í mar.\eva

\bvb The desert-monsters \ken{ettins} bounded and the bedrock resounded; \\
the ancient earth moved all at once. \\
{[...]}; \\
sank thereafter that fish \ken*{= Middenyardswyrm} into the sea.\evb\evg


\bvg\bva\mssnote{\Regius~14v/6, \AM~6v/3}%
\alst{Ó}-tęitr \alst{jǫ}tunn, \hld\ es \alst{a}ptr røru, &
\edtext{[...]}{\Bfootnote{Another likely missing line.  As said in the previous stanza the meter usually requires four lines; more importantly the first half of the sentence is incomplete without a verb.}} &
svá’t \edtrans{\alst{á}r}{in early morn}{\Bfootnote{\textcite{FinnurEdda}\ suggests \emph{svá’t at ǫ́r} ‘so that by the oar’, but this burdens the strict meter.  Assuming the present interpretation is correct, the three would have been out fishing throughout the night.}} Hymir \hld\ \alst{ę}kki mę́lti, &
\alst{v}ęifði rǿði \hld\ \alst{v}eðrs annars til.\eva

\bvb The unmerry ettin \ken*{= Hymer}, as they rowed back, \\
{[...]}, \\
so that in early morn Hymer said nothing; \\
he pulled the oar against the wind:\evb\evg


\bvg\bva\speakernote{[Hymir:]}\mssnote{\Regius~14v/8, \AM~6v/4}%
„Munt of \alst{v}inna \hld\ \alst{v}erk halft við mik, &
at \alst{h}ęim \alst{h}vali \hld\ \alst{h}af til bǿjar &
eða \alst{f}lot-brúsa \hld\ \alst{f}ęstir okkarn.“\eva

\bvb\speakernoteb{[Hymer quoth:]}%
“Thou wilt accomplish a half work by me, \\
if thou bring home the whales to the farm, \\
or our float-jar \ken{boat} do fasten.\footnoteB{Hymer tells Thunder who, having let go of the Wyrm, has nothing to show for the trip, that he can accomplish something half as great as the pulling of the whales if he carries them home and ties the boat by the shore.}”\evb\evg


\bvg\bva\mssnote{\Regius~14v/9, \AM~6v/6}%
\alst{G}ekk Hlórriði \hld\ \alst{g}ręip \edtext{á}{\Afootnote{\emph{til á} \Regius}} stafni &
vatt \edtrans{með \alst{au}stri}{with the bilge-water}{\Bfootnote{That is, the bilge-water was still inside the boat; another comic work of strength.}} \hld\ \alst{u}pp lǫg-fáki; &
\alst{ęi}nn með \alst{ǫ́}rum \hld\ ok með \alst{au}st-skotu &
\alst{b}ar til \alst{b}ǿjar \hld\ \alst{b}rim-svín jǫtuns &
ok \edtext{\edtext{\alst{h}olt-riða}{\Afootnote{\emph{†holtriba†} \Regius}} \hld\ \alst{h}ver}{\lemma{holt-riða hver}\Bfootnote{An uncertain and possibly corrupt kenning.  TODO: What do other editors and translators say?}} í gegnum. \eva

\bvb Loride \name{= Thunder} went, grasped the stern, \\
hurled up the lake-nag \ken{boat} with the bilge-water. \\
Alone with the oars and the bilge-bucket \\
he bore to the farm the ettin’s brim-swines \ken{whales}, \\
even through the cauldron of woodland ridges \ken{valley?}.\evb\evg


\bvg\bva\mssnote{\Regius~14v/12, \AM~6v/7}\edtext{\edtext{Ok}{\Afootnote{\emph{Enn} \AM}} \alst{ę}nn \alst{jǫ}tunn \hld\ umb \alst{a}fr-endi, &
\alst{þ}rá-girni vanr, \hld\ við \alst{Þ}ór sęnti, &
kvað-at mann \alst{r}amman, \hld\ þótt \alst{r}óa kynni, &
\alst{k}rǫptur-ligan, \hld\ nema \alst{k}alk bryti.}{\lemma{ALL}\Bfootnote{Even after witnessing numerous great feats of strength Hymer still refuses to admit Thunder’s superiority.  He now insists on challenging him to break his indestructible chalice.}}\eva

\bvb And still the ettin, used to stubbornness, \\
over strength of hand with Thunder flyted. \\
He called no man strong—although he could row, \\
mightily—unless he broke the chalice.\evb\evg


\bvg\bva\mssnote{\Regius~14v/14, \AM~6v/9}En \alst{H}lórriði, \hld\ es at \alst{h}ǫndum kom, &
\alst{b}rátt lét \alst{b}resta \hld\ \edtrans{\alst{b}ratt-stęin glęri}{steep stone with the glass}{\Bfootnote{He probably broke the stone columns in Hymer’s house with the chalice.}}, &
\alst{s}ló \edtrans{\alst{s}itjandi}{standing}{\Bfootnote{This word is ambiguous and can modify either Thunder (in which case it would mean “sitting”) or the columns (\emph{súlur}).  I have chosen the latter and read it as signifying their stability.}} \hld\ \alst{s}úlur í gǫgnum; &
bǫ́ru þó \alst{h}ęilan \hld\ fyr \alst{H}ymi síðan,\eva

\bvb But Loride \name{= Thunder} when it came to his hands \\
impatiently crushed steep stone with the glass. \\
He struck right through the standing columns, \\
still was it brought whole before Hymer thereafter,\evb\evg


\bvg\bva\mssnote{\Regius~14v/16, \AM~6v/10}unds þat hin \alst{f}ríða \hld\ \alst{f}riðla kęndi &
\alst{ǫ́}st-ráð mikit, \hld\ \alst{ęi}tt es vissi, &
„drep við \alst{h}aus \alst{H}ymis, \hld\ hann ’s \alst{h}arðari, &
\edtrans{\alst{k}ost-móðs}{choice-weary}{\Bfootnote{The gods have destroyed eight of his nine cauldrons, eaten his choicest food, and slain his finest bull.}} jǫtuns, \hld\ \alst{k}alki hvęrjum.“\eva

\bvb until the handsome mistress \ken*{Tew’s mother} gave \\
a great loving counsel, the one she knew: \\
“Strike against Hymer’s skull! It’s harder— \\
the choice-weary ettin’s—than any chalice.”\evb\evg


\bvg\bva\mssnote{\Regius~14v/18, \AM~6v/12}%
\alst{H}arðr \edtext{ręis}{\Afootnote{om. \AM}} á kné \hld\ \alst{h}afra dróttinn, &
fǿrðisk \alst{a}llra \hld\ í \alst{á}s-męgin; &
\alst{h}ęill vas karli \hld\ \alst{h}jalm-stofn ofan, &
en \alst{v}ín-fęrill \hld\ \alst{v}alr rifnaði.\eva

\bvb Hard on the knee rose the Lord of He-goats \ken*{= Thunder}, \\
drew himself into his highest Os-might.\footnoteB{What this actually means is not entirely clear, but a likely interpretation is that Thunder gains his true form—note that he was earlier, st. 18, in the shape of a young boy.  Compare \Gylfaginning\ in its description of Thunder attempting to pull up the Wyrm: \emph{Þá varð Þórr reiðr ok fǿrðist í ás-megin} “Then Thunder turned wroth and drew himself into his Os-might.”}— \\
Whole on the churl \ken*{= Hymer} was the helm-stump \ken{head} above, \\
but the round wine-track \ken{chalice} did rend apart.\evb\evg


\bvg\bva\speakernote{[Hymir kvað:]}\mssnote{\Regius~14v/20, \AM~6v/13}%
„\alst{M}ǫrg vęit’k \alst{m}ę́ti \hld\ \alst{m}ér gingin frá, &
\edtext{es}{\Afootnote{om. \Regius}} \alst{k}alki sé’k \hld\ \edtext{fyr}{\Afootnote{\emph{†yr†} \Regius}} \alst{k}néum hrundit,“ &
\alst{k}arl orð of \alst{k}vað: \hld\ „\edtext{\alst{k}ná’k-at sęgja &
\alst{a}ptr \alst{ę́}va-gi: \hld\ ‚þú ’st \alst{ǫ}lðr of hęitt.}{\lemma{kná’k-at \dots\ of hęitt. ‘I cannot \dots\ warmed!’}\Bfootnote{Hymer laments that with the loss of his finest vessel he will never be able to enjoy his drink again.  This is ironic since it was he who challenged Thunder to break it in the first place.}}‘\eva

\bvb\speakernoteb{[Hymer quoth:]}%
“I know many treasures are gone from me, \\
when I see the chalice thrown before [my] knees!”— \\
The churl \ken*{= Hymer} spoke words: “I cannot say \\
ever again: ‘Thou art, ale, well warmed!’\evb\evg


\bvg\bva\mssnote{\Regius~14v/22, \AM~6v/15}%
Þat ’s til \alst{k}ostar \hld\ ef \alst{k}oma mę́ttið &
\alst{ú}t ór \alst{ó}ru \hld\ \edtrans{\alst{ǫ}l-kjól}{ale-vessel \ken{cauldron}}{\Bfootnote{\emph{ǫl-kjól} is the accusative of \emph{ǫl-kjóll}, but in this construction (\CV: \emph{koma}, B) we would expect the dative \emph{ǫl-kjóli}.  Since the meter does not allow for this the poet has probably taken a grammatical liberty.}} \edtrans{hofi}{hall}{\Bfootnote{This is the only Old Norse occurrence of the word \emph{hof} in the sense “hall, house”—it otherwise only means “temple” (\inx[C]{hove}).  The West Germanic cognates consistently mean “hall”, but that is probably the original sense, so it is unclear if this is an instance of foreign (if so, most likely Anglo-Saxon) influence or just a poetic archaism.}}.“ &
\alst{T}ýr lęitaði \hld\ \alst{t}ysvar hrǿra; &
stóð at \alst{h}vǫ́ru \hld\ \alst{h}verr kyrr fyrir.\eva

\bvb It would be choicest if ye might take \\
out from our hall the ale-vessel \ken{cauldron}.” \\
Tew attempted, twice, to move it— \\
each time stood the cauldron still before [him].\evb\evg


\bvg\bva\mssnote{\Regius~14v/24, \AM~6v/16}%
\alst{F}aðir Móða \hld\ \alst{f}ekk á þręmi &
ok í \alst{g}ǫgnum stęig \hld\ \alst{g}olf niðr í sal; &
\alst{h}óf sér á \alst{h}ǫfuð upp \hld\ \alst{h}ver Sifjar verr, &
en á \alst{h}ę́lum \hld\ \edtrans{\alst{h}ringar skullu}{the rings clattered}{\Bfootnote{i.e. the chain-links.  This detail is mentioned in an example sentence contrasting long and short phonemes in \FGT: \emph{heyrði til hǫddu, þá er Þórr bar hverinn} ‘the sound of the pot-links (\emph{hadda}) was heard when Thunder bore the cauldron’.  According to \textcite{FinnurEdda}\ the chain (or \emph{hadda}) on a Wiking-age cauldron would have reached across, in which case this would be a reference to the cauldron’s enormous size, with its diameter—mentioned in st. 5 as one \inx[C]{rest}—being roughly the same as Thunder’s height.}}.\eva

\bvb The father of Moody \ken*{= Thunder} grasped the brim, \\
and stepped down through the floor in the hall.\footnoteB{In the account of \Gylfaginning\ Thunder is said to have stepped through the boat when trying to pull up the Middenyardswyrm.  This detail is also seen on the carving of the Altuna stone from Uppland, Sweden; it may have been transposed to this place in the narrative. TODO.} \\
Sib’s husband \ken*{= Thunder} heaved the cauldron up on his head, \\
but by his heels the rings clattered.\evb\evg


\bvg\bva\mssnote{\Regius~14v/26, \AM~6v/18}Fóru-t \alst{l}ęngi, \hld\ áðr \alst{l}íta nam &
\alst{a}ptr \alst{Ó}ðins sonr \hld\ \alst{ęi}nu sinni; &
sá ór \alst{h}ręysum \hld\ með \alst{H}ymi austan &
\edtext{\alst{f}olk-drótt}{\lemma{folk-drótt \dots\ fjǫl-hǫfðaða ‘war-troop \dots\ many-headed’}\Bfootnote{The adjective \emph{fjǫl-hǫfðaðr} means ‘many-headed, polycephalic’ and is not referring to the size of the host.  For many-headed ettins see st. 8 and for their malformed bodies in general see Introduction.}} \alst{f}ara \hld\ \alst{f}jǫl-hǫfðaða.\eva

\bvb They journeyed not for long before Weden’s son \ken*{= Thunder} \\
took to look back a single time. \\
He saw out of stone-heaps with Hymer from the east \\
a war-troop coming, many-headed.\evb\evg


\bvg\bva\mssnote{\Regius~14v/28, \AM~6v/19}\alst{H}óf sér af \alst{h}ęrðum \hld\ \alst{h}ver standandi, &
vęifði \alst{M}jǫllni \hld\ \edtrans{\alst{m}orð-gjǫrnum}{murder-eager}{\Bfootnote{By this adjective the poet gives the Hammer something of a life of its own.  For this notion cf. \Skaldskaparmal\ 43, where the Hammer is said to always return to Thunder when thrown, and the numerous amulets where the Hammer is given eyes, most famously the Scanian silver amulet from Claes Kurck’s collection (106659 HST).}} framm, &
ok \alst{h}raun-\alst{h}vala \hld\ \alst{h}ann alla drap.\eva

\bvb He heaved from his shoulders the cauldron, standing; \\
swung the murder-eager Millner forth, \\
and the desert-whales \ken{ettins} all he slew.\evb\evg


\bvg\bva\mssnote{\Regius~14v/30, \AM~6v/21}%
\edtext{Fóru-t \alst{l}ęngi, \hld\ áðr \alst{l}iggja nam &
\alst{h}afr \alst{H}lórriða \hld\ \alst{h}alf-dauðr fyrir, &
vas \edtext{\alst{sk}ę́r}{\Afootnote{emend. from meaningless \emph{†skirr†} \Regius\AM}} \alst{sk}ǫkuls \hld\ \alst{sk}akkr á bęini, &
en því hinn \alst{l}ę́-vísi \hld\ \alst{L}oki of olli.}{\lemma{ALL}\Bfootnote{The detail of Thunder’s halt goat is also found in \Gylfaginning\ 44:

{\small Thunder and Lock were on the way to visit Outyards-Lock and stayed the night at a certain farmer’s.  For supper Thunder cut his two goats and asked the farmer and his family to eat with him.  After they had eaten he spread the goatskins before the fire and asked the housefolk to throw the bones of the goats onto them.  Thelve, the farmer’s son, secretly pried open the thigh of one of the goats and ate the marrow.  At dawn Thunder blessed the goatskins with his hammer and the goats came back to life, but one of them had a halt leg.  The farmer begged for his life and offered to give up his two children: Thelve, his son, and Wrash, his daughter.  Thunder accepted this, and the two became his servants.}

The present stanza probably references a version of the myth where Lock had a part to play in the halting of the goat, perhaps by encouraging Thelve to pry the bone open.}}\eva

\bvb They journeyed not for long before Loride’s \name{= Thunder’s} he-goat \\
took to lie half-dead before [them]. \\
The colt of the cart-pole \ken{goat} was halt in the leg, \\
and that the guile-wise Lock had caused.\evb\evg


\bvg\bva\mssnote{\Regius~14v/32, \AM~6v/22}%
En \edtrans{ér}{ye}{\Bfootnote{The listeners.  A direct address to the audience of this type is otherwise unparalleled in Eddic mythological poetry.  Such are, however, typical for the Scaldic poetry with which this poem shares several traits; see Introduction above.}} \alst{h}ęyrt \alst{h}afið, \hld\ \edtext{\alst{h}vęrr kann umb þat &
\alst{g}oð-mǫ́lugra}{\lemma{hvęrr \dots\ goð-mǫ́lugra ‘each god-speaking man’}\Bfootnote{Literally “each of the god-speaking ones”.  \emph{goð-mǫ́lugr} ‘god-speaking’ is an hapax, but easily understood as “learned in the (lore of) the gods”.}} \hld\ \alst{g}ørr at skilja, &
\alst{h}vęr af \alst{h}raun-búa \hld\ \alst{h}ann laun of fekk, &
es \alst{b}ę́ði galt \hld\ \alst{b}ǫrn sín fyrir.\eva

\bvb But ye have heard—about that can \\
each god-speaking man more clearly discern— \\
which repayments \emph{he} [Thunder] from the desert-dweller [\textsc{ettin} = the farmer] got \\
when he paid up both his children for it.\evb\evg


\bvg\bva\mssnote{\Regius~15r/1, \AM~6v/24}\alst{Þ}rótt-ǫflugr kom \hld\ á \alst{þ}ing goða &
ok \alst{h}afði \alst{h}ver, \hld\ þann’s \alst{H}ymir átti; &
en \alst{v}éar hvęrjan \hld\ \alst{v}ęl skulu drekka &
\alst{ǫ}lðr at \alst{Ę́}gis \hld\ \edtrans{\alst{ęi}tt hǫr-męitið}{an \dots\ flax-cutting}{\Bfootnote{The latter word is an \emph{hapax} and very obscure.  \textcite{LaFargeGlossary} give several suggestions based on \textsc{winter}-kennings of the type “harm of the snake”, viz. \emph{ęitr-hǫr-męitir} ‘poison-rope-cutter \ken{snake > winter}’, \emph{ęitr-orm-męiðir} ‘poison-worm-injurer’ \ken{winter}.
A solution without emendation is to read \emph{ęitt} ‘one’ n. acc. sg. as modifying \emph{ǫlðr} n. acc. ‘ale-feast’, and \emph{hvęrjan} masc. acc. sg. ‘every’ as modifying \emph{hǫr-męitiðr} masc. acc. ‘flax-cutting’, a compound made up of \emph{hǫrr} ‘flax, cord’ and \emph{męita} ‘to cut’.  The whole thing might refer to an obscure harvest festival and give the poem something of an etiological purpose.  If this interpretation is correct it is not unlikely that \Hymiskvida\ was originally composed for performance at such a festival.}}.\eva

\bvb The valour-strong man \ken*{= Thunder} came to the \inx[C]{Thing} of the Gods, \\
and had the cauldron which Hymer had owned, \\
and the \inx[G]{Wighers} \name{Gods} well shall drink \\
an ale-feast at Eagre’s, each flax-cutting \ken{fall?}.\evb\evg

\sectionline
