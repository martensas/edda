\bookStart{The Lay of Hymer}[Hymiskviða]

\begin{flushright}%
Dating \parencite{Sapp2022}: C10th (0.694)–early C11th (0.268)

Meter: \Fornyrdislag%
\end{flushright}%

% Introduction.
Attested in two manuscripts, \Regius\ and \AM. The two are surprisingly consistent; all verses are shared, and come in the same order. The title \emph{Hymiskvida} ‘the Lay of Hymer’ comes from \AM. \Regius\ instead has in the usual red ink the header \emph{Þórr dró Miðgarðsorm} ‘Thunder pulled the Middenyardsworm’.

While its meter is \Fornyrdislag, typical for Eddic poems, this poem is notable for its unusual amount of kennings and complex word-order, both of which are clearly Scoldic traits. The myth of Thunder’s fishing, likewise, is well known from a number of Skaldic poems (see TODO), with which this poem shares both kennings (e.g. 22/4 \emph{umbgjǫrð allra landa} ‘the encircler of all lands \ken*{= Middenyardsworm}’) and wording (especially). These factors suggest that \Hymiskvida\ was composed in a Scoldic environment, perhaps even by a poet by whom we have other works preserved, although that can of course not be known.

Another notable thing about this poem is its nature as a compilation of several myths. (It must here be said, that unlike \Havamal, which has clear differences of style and language between its parts, \Hymiskvida\ is clearly a stylistic and narrative whole, composed by a single poet and then transmitted faithfully!) This is most clearly seen in its analogues. Thus, the story of Thunder’s fishing is told in \Gylfaginning\ 48, but Tew is not present, and there is no mention of a cauldron. TODO!

\sectionline

\bvg
\bva\mssnote{\Regius~13v/26, \AM~5v/25}Ár \alst{v}altívar \hld\ \alst{v}ęiðar nǫ́mu &
ok \alst{s}umbl\alst{s}amir \hld\ áðr \alst{s}aðir yrði, &
\alst{h}ristu tęina \hld\ ok á \alst{h}laut sǫ́u, &
fundu at \alst{Ę́}gis \hld\ \alst{ø}rkost hvera.\eva

\bvb Of yore the slain-Tews \ken{gods} had caught game\footnoteB{Lit. ‘took game’}, and at the \inx[C]{simble} before they might eat\footnoteB{Lit. ‘might become sated’}, they shook the twigs and looked at the \inx[C]{leat}; they found at Eagre’s a great choice of cauldrons.\footnoteB{The gods sprinkled the leat (\emph{hlaut} ‘sacrificial blood’) of the beasts and interpreted the pattern; they found it most auspicious to feast at Eagre’s. TODO: reference to leat-twigs.}\evb
\evg


\bvg
\bva\mssnote{\Regius~13v/28, \AM~5v/27}Sat \alst{b}erg\alst{b}úi \hld\ \alst{b}arntęitr fyrir, &
\alst{m}jǫk glíkr \alst{m}ęgi \hld\ \alst{M}iskorblinda, &
lęit í \alst{au}gu \hld\ \alst{Y}ggs barn í þrá: &
„þú skalt \alst{ǫ́}sum \hld\ \alst{o}pt sumbl \edtext{gęra}{\lemma{gęra ‘host’}\Afootnote{\emph{gefa} ‘give’ \AM}}!“\eva

\bvb Sat the mountain-dweller \ken*{\textsc{ettin} = Eagre} there, merry like a child, much alike to the lad of Misherblind;\footnoteB{A reference to a lost myth? Unless Misherblind is an alternative name for Firneet, Eagre’s father.} into his eyes looked the child of Ug \name{= Weden} \ken*{= Thunder} stubbornly: “Thou shalt for the Ease oft host simbles!”\footnoteB{Having seen that Eagre has a great store of cauldrons, Thunder orders him to host future banquets for the Ease.}\evb
\evg


\bvg
\bva\mssnote{\Regius~13v/31, \AM~5v/29}\alst{Ǫ}nn fekk \alst{jǫ}tni \hld\ \alst{o}rðbę́ginn halr, &
\alst{h}ugði at \alst{h}efndum \hld\ \alst{h}ann nę́st við goð, &
bað hann \alst{S}ifjar ver \hld\ \alst{s}ér fǿra hver, &
„þann’s ek \alst{ǫ}llum \alst{ǫ}l \hld\ \alst{y}ðr of hęita.“\eva

\bvb Great toil for the ettin the word-peevish man \ken*{= Thunder} caused; he \ken*{= Eagre} thought of revenge, soon, against the god; he bade Sib’s husband \ken*{= Thunder} bring him a cauldron, “that one with which I for you all ale might heat.\footnoteB{Eagre gets back at Thunder by telling him that he needs a single cauldron which can hold enough ale to supply all the Ease.}”\evb
\evg


\bvg
\bva\mssnote{\Regius~14r/1, \AM~5v/30}Né þat \alst{m}ǫ́ttu \hld\ \alst{m}ę́rir tívar &
ok \alst{g}innręgin \hld\ of \alst{g}eta hvęrgi, &
unz af \alst{t}ryggðum \hld\ \alst{T}ýr Hlórriða &
\alst{á}stráð mikit \hld\ \alst{ęi}num sagði:\eva

\bvb But that one might the renowned \inx[G]{Tews} and the \inx[G]{yin-Reins} nowhere get ahold of—until, out of loyalty, a great loving counsel Tew to Loride \name{= Thunder} alone did say:\evb
\evg


\bvg
\bva\mssnote{\Regius~14r/3, \AM~6r/2}„Býr fyr \alst{au}stan \hld\ \alst{É}livága &
\alst{h}undvíss \alst{H}ymir \hld\ at \alst{h}imins ęnda, &
á \alst{m}inn faðir \hld\ \alst{m}óðugr kętil, &
\edtext{\alst{r}úmbrugðinn}{\Afootnote{\emph{†rumbrygðan†} \AM}} hver \hld\ \alst{r}astar djúpan.“\eva

\bvb “Dwells to the east of the \inx[L]{Ilewaves} the hound-wise Hymer, at heaven’s end.\footnoteB{According to \Vafthrudnismal\ 31 the Ilewaves were the poisonous wild rushes out of which the ettins emerged, and so it only makes sense that they would be found in the east, where the ettins dwell. Hymer’s dwelling even further east than them illustrates his fierceness.} Owns my father \ken*{= Hymer}, fierce, a kettle; a size-renowned cauldron, a \inx[C]{rest} deep.”\evb
\evg


\bvg {\small [Thunder quoth:]}
\bva\mssnote{\Regius~14r/4, \AM~6r/4}„Vęizt, ef \alst{þ}iggjum \hld\ \alst{þ}ann lǫgvelli?“ &
„Ef, \alst{v}inr, \alst{v}élar \hld\ \alst{v}it gørvum til!“\eva

\bvb “Knowest thou if we will receive that liquid-boiler \ken{cauldron}?” — [Tew quoth:] “If, friend, we two make use of wiles!”\footnoteB{Like elsewhere in this poem the speakers are not indicated, but it is most sensible that Thunder asks and Tew answers.}\evb
\evg

\bvg
\bva\mssnote{\Regius~14r/5, \AM~6r/4}Fóru \alst{d}rjúgum \hld\ \edtrans{\alst{d}ag þann framan}{from the beginning of the day}{\Afootnote{emend. after \textcite{FinnurEdda}; \emph{dag þann fram} ‘on that day forth’ \Regius; \emph{dag fráliga} ‘swiftly at day’ \AM}} &
\alst{Á}sgarði frá \hld\ unz til \edtrans{\alst{Ę}gils}{Agle’s [home]}{\Afootnote{so \Regius; \emph{Ę́gis} ‘Eagre’s [home]’ \AM\ is probably from confusion with Eagre (the ettin) described earlier in the poem, unless the shepherd shared his name.}} kvǫ́mu; &
\alst{h}irði \alst{h}afra \hld\ \alst{h}orngǫfgasta; &
\alst{h}urfu at \alst{h}ǫllu \hld\ es \alst{H}ymir átti.\eva

\bvb — Journeyed they with great strides from the beginning of the day, from Osyard, until to Agle’s [home] they came—he herded the horn-noblest he-goats\footnoteB{Thunder left his goats in the care of the shepherd Agle, whose identity is unclear.}—they turned to the hall which Hymer owned.\evb
\evg


\bvg
\bva\mssnote{\Regius~14r/7, \AM~6r/6}\alst{M}ǫgr fann ǫmmu, \hld\ \alst{m}jǫk lęiða sér, &
\alst{h}afði \alst{h}ǫfða \hld\ \alst{h}undruð níu. &
en \alst{ǫ}nnur gekk \hld\ \alst{a}lgollin framm &
\alst{b}rúnhvít \alst{b}era \hld\ \alst{b}jórvęig syni.\eva

\bvb The lad \ken*{= Tew} found his grandmother very loathsome; heads she had, nine hundred.—But another one stepped—all-golden—forth: white-browed, she carried a beer-draught for her son \ken*{= Tew}:\evb
\evg


\bvg
\bva\mssnote{\Regius~14r/9, \AM~6r/8}„\alst{Á}ttniðr \alst{jǫ}tna \hld\ \alst{e}k vilja’k ykkr &
\alst{h}ugfulla tvá \hld\ und \alst{h}vera sętja; &
es \alst{m}ínn \edtrans{fríi}{lover}{\Afootnote{so \Regius; \emph{faðir} ‘father’ \AM}} \hld\ \alst{m}ǫrgu sinni &
\alst{g}løggr við \alst{g}ęsti \hld\ \alst{g}ǫrr ills hugar.“\eva

\bvb “O descendant of ettins \ken*{= Tew}, \emph{I} would wish to set you high-mettled two under the cauldrons; my lover \ken*{= Hymer} has many a time been stingy toward guests, quick to ill temper.”\footnoteB{Tew’s mother hides him and Thunder, lest Hymer find them.}\evb
\evg


\bvg
\bva\mssnote{\Regius~14r/11, \AM~6r/9}En \alst{v}áskapaðr \hld\ \alst{v}arð \edtext{síðbúinn}{\lemma{síðbúinn ‘come late’}\Afootnote{om. \AM}}, &
\alst{h}arðráðr \alst{H}ymir, \hld\ \alst{h}ęim af vęiðum; &
\alst{g}ekk inn í sal, \hld\ \alst{g}lumðu jǫklar, &
vas \alst{k}arls, es \alst{k}om, \hld\ \alst{k}innskógr frørinn.\eva

\bvb But the misshapen one was come late—the hard-minded Hymer—home from the hunt. He entered the hall—icicles clattered\footnoteB{In Icelandic the word \emph{jǫkull} comes to specifically mean ‘glacier’, but this development is peculiar and its base meaning is ‘icicle’, a word with which it is also cognate. The icicles are certainly those in Hymer’s beard.}—on the churl who came \ken*{= Hymer} was the cheek-shaw \ken{beard} frozen.\evb
\evg


\bvg {\small [Tew’s mother quoth:]}
\bva\mssnote{\Regius~14r/13, \AM~6r/11}„Ves þú \alst{h}ęill, \alst{H}ymir, \hld\ í \alst{h}ugum góðum! &
Nú ’s \alst{s}onr kominn \hld\ til \alst{s}ala þinna, &
sá’s \alst{v}it \alst{v}ę́ttum \hld\ af \alst{v}egi lǫngum; &
fylgir \alst{h}ǫ́num \hld\ \alst{H}róðrs andskoti, &
\alst{v}inr \alst{v}erliða; \hld\ \alst{V}éurr hęitir sá.\eva

\bvb “Be thou hale, Hymer, in good spirits!\footnoteB{This formula is very closely paralleled in runic inscription N B380 (edited under Charms and Spells). Cf. also \Beowulf\ 407a: \emph{Wæs þú Hróðgár hál} ‘Be thou, Rothgar, hale!’} Now the son \ken*{= Tew} is come to thy halls, the one whom we two have been awaiting from a long way off. Follows him the opponent of Rooder \name{ettin}, the friend of manly retinues; \inx[P]{Wighward} \name{= Thunder} is that one called.\evb
\evg


\bvg
\bva\mssnote{\Regius~14r/15, \AM~6r/13}\alst{S}é þú hvar \alst{s}itja \hld\ und \alst{s}alar gafli, &
\alst{s}vá \edtext{forða \alst{s}ér}{\Afootnote{\emph{forðask} \AM}}, \hld\ stęndr \edtrans{\alst{s}úl}{pillar}{\Afootnote{\emph{†sol†} \AM}} fyrir.“ &
\alst{S}undr stǫkk \alst{s}úla \hld\ fyr \alst{s}jón jǫtuns, &
en \edtext{\alst{a}llr}{\Afootnote{\emph{áðr} ‘earlier, before that’ \Regius\AM. TODO: elaborate, mention Finnur}} í tvau \hld\ \alst{á}ss brotnaði.\eva

\bvb See where they sit under the hall’s gable: thus they protect themselves—a pillar stands before them!\footnoteB{Tew’s mother reveals the hiding place of the gods.}” The pillars sprang asunder before the sight of the ettin, but all in two the roof-beam was broken.\evb
\evg


\bvg
\bva\mssnote{\Regius~14r/17, \AM~6r/15}Stukku \alst{á}tta, \hld\ en \alst{ęi}nn af þęim &
\alst{h}verr \alst{h}arðslęginn \hld\ \alst{h}ęill af þolli; &
\alst{f}ramm gingu þęir, \hld\ en \alst{f}orn jǫtunn &
\alst{s}jónum lęiddi \hld\ \alst{s}inn andskota.\eva

\bvb Eight [cauldrons] sprung apart, but one of them—a hard-forged cauldron—[came] whole off its peg.\footnoteB{The cauldrons were presumably hanging on the roof-beam. Eight of them broke, but a single one remained whole.} Forth went they, but the ancient ettin with his sight closely followed his opponent \ken*{= Thunder}.\evb
\evg


\bvg
\bva\mssnote{\Regius~14r/19, \AM~6r/16}Sagði-t \alst{h}ǫ́num \hld\ \alst{h}ugr vęl þá’s sá &
\alst{g}ýgjar \edtrans{\alst{g}rǿti}{distresser}{\Afootnote{\emph{gę́ti} ‘keeper, warder’ \AM}} \hld\ á \alst{g}olf kominn, &
\alst{þ}ar vǫ́ru \alst{þ}jórar \hld\ \alst{þ}rír of tęknir, &
bað \edtrans{\alst{s}ęnn}{at once}{\Afootnote{\emph{sun} ‘[his] son \ken*{= Tew}?’ \AM}} jǫtunn \hld\ \alst{s}jóða ganga.\eva

\bvb His \ken*{Hymer’s} heart was not pleased then, when he saw the gow’s distresser \ken*{= Thunder} come on the floor. There were three bulls taken: bade the ettin at once [his servants] to go roast [them].\evb
\evg


\bvg
\bva\mssnote{\Regius~14r/21, \AM~6r/18}\alst{H}vęrn létu þęir \hld\ \alst{h}ǫfði skęmra &
auk á \alst{s}ęyði \hld\ \alst{s}íðan bǫ́ru, &
át \alst{S}ifjar verr \hld\ áðr \alst{s}ofa gingi, &
\alst{ęi}nn með \alst{ǫ}llu \hld\ \alst{ø}xn tvá Hymis.\eva

\bvb Each [bull] they let shorten by a head, and onto the fire-pit then carried: ate the husband of Sib \ken*{= Thunder}—before he might go to sleep—alone by himself two of Hymer’s oxen.\footnoteB{Cf. \Thrymskvida\ 24 for another instance of Thunder’s great eating.}\evb
\evg


\bvg
\bva\mssnote{\Regius~14r/23, \AM~6r/19}Þótti \alst{h}ǫ́rum \hld\ \alst{H}rungnis spjalla &
\alst{v}erðr Hlórriða \hld\ \alst{v}ęl fullmikill, &
„munum at \alst{a}ptni \hld\ \alst{ǫ}ðrum verða &
\alst{v}ið \alst{v}ęiðimat \hld\ \alst{v}ér þrír lifa.“\eva

\bvb To the hoary friend of Rungner \name{ettin} \ken*{= Hymer} seemed Loride’s \name{Thunder’s} eating far too great; “next evening will we three by game-meat have to live.\footnoteB{The construction is difficult, but should probably be read in prose word order as \emph{vér þrír munum at ǫðrum aptni verða lifa við vęiðimat}, where \emph{verða} has a similar use as its modern German cognate \emph{werden}. Hymer’s stinginess—he refuses to share more of his own food, forcing his guests to go hunt—breaks all Indo-European rules of hospitality and illustrates the otherness of the Ettins. See Introduction to the poem.}”\evb
\evg


\bvg
\bva\mssnote{\Regius~14r/24, \AM~6r/21}\alst{V}éurr kvaðsk \alst{v}ilja \hld\ á \alst{v}ág róa, &
ef \alst{b}allr jǫtunn \hld\ \alst{b}ęitur gę́fi. &
„\alst{H}verf þú til \edtext{\alst{h}jarðar}{\Afootnote{\emph{hallar} corr. \AM}}, \hld\ ef \alst{h}ug trúir, &
\alst{b}rjótr \alst{b}erg-Dana, \hld\ \alst{b}ęitur sǿkja.\eva

\bvb Wighward \name{= Thunder} declared himself willing to row on the wave, if the baleful ettin might give pieces of bait. “Turn to the herd if thou trust in thy heart—breaker of boulder-Danes \ken*{\textsc{ettins} > = Thunder}!—to seek pieces of bait.\evb
\evg


\bvg
\bva\mssnote{\Regius~14r/26, \AM~6r/23}\alst{Þ}ess \edtext{vę́ntir mik}{\Afootnote{so \AM; \emph{vę́nti ek} \Regius}}, \hld\ at \alst{þ}ér \edtext{mynit}{\lemma{mynit ‘will not’}\Afootnote{so \AM; \emph{myni} ‘will’ \Regius. I prefer the \AM\ reading since it makes this the first of Hymer’s several challenges to Thunder, ones which the god easily accomplishes.}} &
\alst{ǫ}gn at \alst{o}xa \hld\ \alst{au}ðfeng vesa.“ &
\edtrans{\alst{S}vęinn}{The swain}{\Bfootnote{Thunder was apparently in the shape of a young man. Cf. Snorri (TODO!) where this is attested.}} \alst{s}ýsliga \hld\ \alst{s}vęif til skógar, &
þar’s \edtext{\alst{o}xi stóð \hld\ \alst{a}lsvartr}{\lemma{oxi \dots\ alsvartr ‘ox all-black’}\Bfootnote{Formulaic, also occuring in \Thrymskvida\ 23; see note there for further parallels to this custom. All-black oxen were apparently seen as the noblest, and so Thunder’s taking of one, instead of an inferior beast, may be seen as a subtle insult towards the stingy Hymer.}} fyrir.\eva

\bvb I ween that the bait from the ox will not be an easy catch for thee.”—The swain \ken*{= Thunder} swiftly turned to the woods, there as an ox stood, all-black, before [him].\evb
\evg


\bvg
\bva\mssnote{\Regius~14r/28, \AM~6r/24}Braut af \alst{þ}jóri \hld\ \alst{þ}urs ráðbani &
\alst{h}ǫ́tún ofan \hld\ \alst{h}orna tveggja. &
„\alst{V}erk þikkja þín \hld\ \alst{v}erri myklu &
\alst{k}jóla valdi \hld\ an \alst{k}yrr sitir.“\eva

\bvb Off from the bull broke the counsel-slayer of the thurse \ken*{= Thunder} the high meadow of the two horns \ken{head} from above.—“Thy works seem worse by far to the wielder of ships \ken*{= Hymer = me} than if thou did sit calm.\footnoteB{I had originally taken this as Hymer snidely belittling Thunder’s feat of pulling the head off the ox (presumably by the horns); he would have earned greater glory had he simply sat and done nothing. However, it may also be read as a factual statement; Thunder just killed one of his finest oxen, and Hymer would certainly have preferred that he had not.}”\evb
\evg


\bvg
\bva\mssnote{\Regius~14r/30, \AM~6r/26}Bað \alst{h}lunngota \hld\ \alst{h}afra dróttinn &
\edtext{\alst{á}ttrunn}{\Afootnote{\emph{†atrænn†} \AM}} \alst{a}pa \hld\ \alst{ú}tar fǿra, &
en \alst{s}á jǫtunn \hld\ \alst{s}ína \edtext{talði}{\Afootnote{\emph{milldi} corr. \AM}}, &
\alst{l}ítla fýsi \hld\ \edtext{\alst{l}ęngra at róa}{\Afootnote{metr. emend.; \emph{at róa lęngra} \Regius\AM}}.\eva

\bvb The lord of he-goats \ken*{= Thunder} bade the kinsman of the \inx[C]{ape}\footnoteB{The specific sense of \emph{api} is uncertain. It seems to generally refer to a fool, but see Encyclopedia.}\ \ken*{\textsc{ettin} = Hymer} to push the launching-steed \ken{boat} further out; but that ettin told of his scarce wish to row longer.\footnoteB{There is some humour in the situation as Hymer, who just mocked Thunder, is now forced to do his willing by rowing.}\evb
\evg


\bvg
\bva\mssnote{\Regius~14r/31, \AM~6r/27}Dró \edtrans{\alst{m}ę́rr}{renowned}{\Afootnote{so \Regius; \emph{męirr} ‘more, further’ \AM}} Hymir \hld\ \alst{m}óðugr hvala &
\alst{ęi}nn á \alst{ǫ}ngli \hld\ \alst{u}pp sęnn tváa; &
en \alst{a}ptr í skut \hld\ \alst{Ó}ðni sifjaðr &
\alst{V}éurr við \alst{v}élar \hld\ \alst{v}að gęrði sér.\eva

\bvb Pulled renowned Hymer—fierce—whales: one on the hook, soon up two—but back in the stern the Weden-related Wighward \name{= Thunder} wilily\footnoteB{Probably because he made the fishing line behind Hymer’s back, who was distracted by the whales.} made himself a fishing-line.\evb
\evg


\bvg
\bva\mssnote{\Regius~14v/1, \AM~6r/29}\alst{Ę}gnði á \alst{ǫ}ngul \hld\ sá’s \alst{ǫ}ldum bergr, &
\alst{o}rms \alst{ęi}nbani \hld\ \alst{o}xa hǫfði; &
\alst{g}ęin við \edtrans{agni}{bait}{\Afootnote{so \AM; \emph{ǫngli} ‘hook’ \Regius}}, \hld\ sú’s \alst{g}oð fía, &
\alst{u}mbgjǫrð neðan \hld\ \alst{a}llra landa.\eva

\bvb On the hook fastened he who saves men \ken*{= Thunder}—the lone slayer of the Worm—the head of the ox. At the bait snapped the one whom the gods hate \ken*{= Middenyardsworm}—the encircler of all lands\footnoteB{This kenning occurs identically in a fragment by C9th scold Alewigh Snub (Ǫlv \emph{Þórr}, edited by Margaret Clunies Ross in \emph{SkP} III).} from below.\evb
\evg


\bvg
\bva\mssnote{\Regius~14v/3, \AM~6v/1}\alst{D}ró \alst{d}jarfliga \hld\ \alst{d}áðrakkr Þórr &
\alst{o}rm \alst{ęi}trfáan \hld\ \alst{u}pp at borði; &
\alst{h}amri kníði \hld\ \alst{h}ǫ́fjall skarar &
\alst{o}fljótt \alst{o}fan \hld\ \alst{u}lfs hnitbróður.\eva

\bvb Pulled boldly deed-ready Thunder the venom-glistening Worm up on the gunwale; with the hammer he struck the high mountain of hair\footnoteB{A rather unfitting kenning, since serpents do not have hair.} \ken{head}—very hideous, from above—on the clash-brother of the Wolf \ken*{= Middenyardsworm}.\evb
\evg


\bvg
\bva\mssnote{\Regius~14v/5, \AM~6v/2}\edtrans{\alst{H}raungǫlkn}{The lavafield-monsters}{\Bfootnote{Both mss. have \emph{hręin-}, which may mean either ‘clean’ or ‘reindeer’, neither of which fit. On the other hand \emph{hraun} \ONP: ‘stone/barren area, wasteland; lava-field’ is well attested in scoldic kennings for ettins. The precise meaning of \emph{galkn} ‘monster’ (plural \emph{gǫlkn}) is unclear; but it is attested in three scoldic verses, always in kennings of the type “troll-woman of the shield \ken{axe}”. While the mss. ‘\emph{galkn}’ (norm. \emph{gálkn}) could be both singular and plural, the form of the verb precludes the former. This means that the word cannot be referring to the Middenyardsworm, refuting the interpretation of \textcite{LarringtonEdda}: “the sea-wolf shrieked”.}} \edtext{\alst{h}rutu}{\Afootnote{so \AM; \emph{hlumðu} ‘dashed’ \Regius. End-rhyme is also used by the poet in st. 3/3.}}, \hld\ ęn \alst{h}ǫlkn þutu, &
\alst{f}ór hin \alst{f}orna \hld\ \alst{f}old ǫll saman; &
\edtext{[...]}{\Bfootnote{It is very likely that a line is missing here, since the stanzas in the poem consistently have four lines. In other texts describing this narrative Hymer cuts Thunder’s fishing line at this point, and so that is probably what it contained.

It is of course impossible to know what exact form it had, but for the reader’s enjoyment, based on other poets and the account in \Gylfaginning\ (see introduction to the present poem) I’ve composed the following variant lines: \emph{unz vinr Hrungnis \hld\ vað Þórs of skar} ‘until the friend of Rungner \ken*{= Hymer} Thunder’s fishing-line did cut’; \emph{unz fǫlr Hymir \hld\ fekk á saxi} ‘until pale Hymer grasped the knife’.}} &
\alst{s}økkðisk \alst{s}íðan \hld\ \alst{s}á \edtrans{fiskr}{fish}{\Bfootnote{The Middenyardsworm may also be called a “fish” in \Grimnismal\ 21.}} í mar.\eva

\bvb The lavafield-monsters \ken{ettins} bounded, but the bedrock resounded; moved the ancient earth all at once; [...]; sank thereafter that fish \ken*{= Middenyardsworm} into the sea.\evb
\evg


\bvg
\bva\mssnote{\Regius~14v/6, \AM~6v/3}\alst{Ó}tęitr \alst{jǫ}tunn, \hld\ es \alst{a}ptr røru, &
\edtext{[...]}{\Bfootnote{There is without doubt a line missing here; the meter usually requires four lines, and the first half of the sentence is incomplete without a verb (unless one understands an implied “was”, so that the verse would begin “Unmerry was the ettin”).}} &
svá’t \edtrans{\alst{á}r}{in the early morning’}{\Bfootnote{\textcite{FinnurEdda}\ suggests \emph{svá’t at ǫ́r} ‘so that by the oar’. Assuming my interpretation is correct, the three would have been fishing}} Hymir \hld\ \alst{ę}kki mę́lti, &
\alst{v}ęifði rǿði \hld\ \alst{v}eðrs annars til.\eva

\bvb The unmerry ettin \ken*{= Hymer}, as they rowed back, [...], so that in the early morning Hymer spoke nothing; he pulled the oar around, against the storm:\evb
\evg


\bvg {\small [Hymer quoth:]}
\bva\mssnote{\Regius~14v/8, \AM~6v/4}„Munt of \alst{v}inna \hld\ \alst{v}erk halft við mik, &
at \alst{h}ęim \alst{h}vala \hld\ \alst{h}af til bǿjar &
eða \alst{f}lotbrúsa \hld\ \alst{f}ęstir okkarn.“\eva

\bvb “Thou wilt win a half work by me if thou carry the whales home to the farm, or our float-jar \ken{boat} fasten.\footnoteB{Hymer tells Thunder, who having let go of the Worm now has nothing to show for the trip, that he can accomplish something half as good as the pulling of the whales if he carries them home, or if he fastens the boat (by the shore).}”\evb
\evg


\bvg
\bva\mssnote{\Regius~14v/9, \AM~6v/6}\alst{G}ekk Hlórriði \hld\ \alst{g}ręip \edtext{á}{\Afootnote{\emph{til á} \Regius}} stafni &
vatt með \alst{au}stri \hld\ \alst{u}pp lǫgfáki; &
\alst{ęi}nn með \alst{ǫ́}rum \hld\ ok með \alst{au}stskotu &
\alst{b}ar til \alst{b}ǿjar \hld\ \alst{b}rimsvín jǫtuns &
ok \edtext{\alst{h}oltriða}{\Afootnote{\emph{†holtriba†} \Regius}} \hld\ \alst{h}ver í gegnum. \eva

\bvb Went Loride \name{= Thunder}, grasped the stern; hurled with the bilge-water the lake-nag \ken{boat} up.\footnoteB{Thunder did not pour the bilge-water, something that makes its weight considerably heavier, out of the boat. This was a great work of strength.} Alone with the oars and the bilge-bucket he bore to the farm the ettin’s brim-swines \ken{whales}; even through the cauldron of woodland ridges\footnoteB{TODO. What do other editors and translators say?} \ken{valley?}.\evb
\evg


\bvg
\bva\mssnote{\Regius~14v/12, \AM~6v/7}\edtext{Ok}{\Afootnote{\emph{enn} \AM}} \alst{ę}nn \alst{jǫ}tunn \hld\ umb \alst{a}fręndi, &
\alst{þ}rágirni vanr, \hld\ við \alst{Þ}ór sęnti, &
kvað-at mann \alst{r}amman, \hld\ þótt \alst{r}óa kynni, &
\alst{k}rǫpturligan, \hld\ nema \alst{k}alk bryti.\eva

\bvb And yet the ettin, used to stubbornness, regarding strength of hand flyted with Thunder; he called not the man strong—although he could row, mightily—unless he broke the chalice.\footnoteB{Hymer accuses Thunder of weakness, refusing to call him strong unless he breaks a certain chalice.}\evb
\evg


\bvg
\bva\mssnote{\Regius~14v/14, \AM~6v/9}En \alst{H}lórriði, \hld\ es at \alst{h}ǫndum kom, &
\alst{b}rátt lét \alst{b}resta \hld\ \alst{b}rattstęin glęri, &
\alst{s}ló \alst{s}itjandi \hld\ \alst{s}úlur í gǫgnum; &
bǫ́ru þó \alst{h}ęilan \hld\ fyr \alst{H}ymi síðan.\eva

\bvb But Loride \name{= Thunder}, when [it] came in his hands, impatiently crashed steep stone\footnoteB{\textcite{FinnurEdda} interprets the word as referring to stone pillars.} with the glass \ken*{= chalice}; he struck right through the fastened\footnoteB{\emph{sitjandi} ‘sitting’ is ambiguous and can modify either Thunder or the (roof-bearing) pillars. I think it is more likely to modify the pillars, signifying their stability.} pillars; yet they \ken*{= Hymer’s servants?} bore it whole before Hymer afterwards.\evb
\evg


\bvg
\bva\mssnote{\Regius~14v/16, \AM~6v/10}Unz þat hin \alst{f}ríða \hld\ \alst{f}riðla kęndi &
\alst{á}stráð mikit, \hld\ \alst{ęi}tt es vissi, &
„drep við \alst{h}aus \alst{H}ymis, \hld\ hann ’s \alst{h}arðari, &
\alst{k}ostmóðs jǫtuns, \hld\ \alst{k}alki hvęrjum.“\eva

\bvb Until the handsome mistress \ken*{= Tew’s mother} gave a great loving counsel, the one she knew: “Strike against Hymer’s skull; it is harder—on the choice-weary\footnoteB{A reference to the gods having eaten up his choicest food.} ettin—than every chalice.”\evb
\evg


\bvg
\bva\mssnote{\Regius~14v/18, \AM~6v/12}\alst{H}arðr \edtext{ręis}{\Afootnote{om. \AM}} á kné \hld\ \alst{h}afra dróttinn, &
fǿrðisk \alst{a}llra \hld\ í \alst{á}smęgin; &
\alst{h}ęill vas karli \hld\ \alst{h}jalmstofn ofan, &
en \alst{v}ínfęrill \hld\ \alst{v}alr rifnaði.\eva

\bvb Hard rose on the knees the lord of he-goats \ken*{= Thunder}; he summoned his highest os-might.\footnoteB{Compare \Gylfaginning\ in its description of Thunder attempting to pull up the Worm: \emph{Þá varð Þórr reiðr ok fę́rðist í ásmegin} “Then Thunder became wroth, and summoned his os-might.”} Whole was on the churl \ken*{= Hymer} the helmet-stump \ken{head} above, but the round wine-track \ken{chalice} rent apart.\evb
\evg


\bvg {\small [Hymer quoth:]}
\bva\mssnote{\Regius~14v/20, \AM~6v/13}„\alst{M}ǫrg vęit’k \alst{m}ę́ti \hld\ \alst{m}ér gingin frá, &
\edtext{es}{\Afootnote{om. \Regius}} \alst{k}alki sé’k \hld\ \edtext{fyr}{\Afootnote{\emph{†yr†} \Regius}} \alst{k}néum hrundit,“ &
\alst{k}arl orð of kvað: \hld\ „\alst{k}ná’k-at sęgja &
\alst{a}ptr \alst{ę́}vagi: \hld\ þú ’st \alst{ǫ}lðr of hęitt.\eva

\bvb “I know many good things to be gone from me when I see the chalice thrown before [his] knees;”—the churl \ken*{= Hymer} then words did speak: “I cannot say it, ever again: ‘Thou art, ale, [well] heated!\footnoteB{Hymer laments that since his finest vessel is now broken, he will never again be able to enjoy strong drink.}’\evb
\evg


\bvg
\bva\mssnote{\Regius~14v/22, \AM~6v/15}Þat ’s til \alst{k}ostar \hld\ ef \alst{k}oma mę́ttið &
\alst{ú}t ór \alst{ó}ru \hld\ \edtrans{\alst{ǫ}lkjól}{ale-ship \ken{cauldron}}{\Bfootnote{\emph{ǫlkjól} is the accusative form, but in this sense (\CV: \emph{koma}, B) we would expect the dative \emph{ǫlkjóli}, something that the meter does not allow for.}} hofi.“ &
\alst{T}ýr lęitaði \hld\ \alst{t}ysvar hrǿra; &
stóð at \alst{h}vǫ́ru \hld\ \alst{h}verr kyrr fyrir.\eva

\bvb It would be well done, if ye might make the ale-ship \ken{cauldron} to come out of our hall.\footnoteB{\emph{hof} ‘hall’ usually means ‘hove; temple’.}” Tew attempted, twice, to move it; stood nevertheless the cauldron still before [him].\evb
\evg


\bvg
\bva\mssnote{\Regius~14v/24, \AM~6v/16}\alst{F}aðir Móða \hld\ \alst{f}ekk á þręmi &
ok í \alst{g}ǫgnum sté \hld\ \alst{g}olf niðr í sal; &
\alst{h}óf sér á \alst{h}ǫfuð upp \hld\ \alst{h}ver Sifjar verr, &
en á \alst{h}ę́lum \hld\ \alst{h}ringar skullu.\eva

\bvb The father of Moody \ken*{= Thunder} grasped the brim, and stepped down through the floor in the hall;\footnoteB{In the account of \Gylfaginning\ Thunder is said to have stepped through the boat when trying to pull up the Middenyardsworm. This detail is also seen on the carving of the Altuna stone from Uppland, Sweden; it may have been transposed to this place in the narrative.} heaved the husband of Sib \ken*{= Thunder} up onto his head the cauldron, but on his heels rings clattered.\footnoteB{The rings from the cauldron-chain; this detail is mentioned in an example sentence contrasting long and short phonemes in \FGT: \emph{heyrði til hǫddu, þá er Þórr bar hverinn} “one heard the pot-links when Thunder bore the kettle”. According to \textcite{FinnurEdda}\ this chain reached from one end of the kettle to another, in which case this would be an oblique reference to the cauldron’s size, its diameter being the same as Thunder’s height.}\evb
\evg


\bvg
\bva\mssnote{\Regius~14v/26, \AM~6v/18}Fóru-t \alst{l}ęngi, \hld\ áðr \alst{l}íta nam &
\alst{a}ptr \alst{Ó}ðins sonr \hld\ \alst{ęi}nu sinni; &
sá ór \alst{h}ręysum \hld\ með \alst{H}ymi austan &
\alst{f}olkdrótt \alst{f}ara \hld\ \alst{f}jǫlhǫfðaða.\eva

\bvb They journeyed not for long before the son of Weden \ken*{= Thunder} took to look back, a single time;—saw he out of stone-heaps, with Hymer from the east, a many-headed folk-troop \ken{= ettins} journeying.\evb
\evg


\bvg
\bva\mssnote{\Regius~14v/28, \AM~6v/19}\alst{H}óf sér af \alst{h}ęrðum \hld\ \alst{h}ver standandi, &
vęifði \alst{M}jǫllni \hld\ \alst{m}orðgjǫrnum framm, &
auk \alst{h}raun\alst{h}vala \hld\ \alst{h}ann alla drap.\eva

\bvb Heaved he off from his shoulders the cauldron, [while] standing; he swung the murder-eager Millner forth, and the rock-whales \ken{= ettins} all he slew.\evb
\evg


\bvg
\bva\mssnote{\Regius~14v/30, \AM~6v/21}Fóru-t \alst{l}ęngi, \hld\ áðr \alst{l}iggja nam &
\alst{h}afr \alst{H}lórriða \hld\ \alst{h}alfdauðr fyrir, &
vas \edtext{\alst{sk}ę́r}{\Afootnote{emend. from meaningless \emph{†skirr†} \Regius\AM}} \alst{sk}ǫkuls \hld\ \alst{sk}akkr á bęini, &
en því hinn \alst{l}ę́vísi \hld\ \alst{L}oki of olli.\eva

\bvb They journeyed not for long before the he-goat of Loride \name{= Thunder} took to lie half-dead before [them]; the steed of the cart-pole \ken{goat} was halt in the leg, but that the guile-wise Lock did cause.\footnoteB{Apparently Lock (who is not mentioned earlier in the poem) was placing curses on the returning party. Snorre mentions this, TODO.}\evb
\evg


\bvg
\bva\mssnote{\Regius~14v/32, \AM~6v/22}En ér \alst{h}ęyrt \alst{h}afið, \hld\ \alst{h}vęrr kann of þat &
\alst{g}oðmǫ́lugra \hld\ \alst{g}ørr at skilja, &
\alst{h}vęr af \alst{h}raunbúa \hld\ \alst{h}ann laun of fekk, &
es \alst{b}ę́ði galt \hld\ \alst{b}ǫrn sín fyrir.\eva

\bvb But ye have heard—each god-knowledgeable\footnoteB{\emph{goð-mǫ́lugr} ‘able to speak about the god-lore; versed in the mythology’ is a \emph{hapax}.} man knows about this more clearly discern—which rewards he \ken*{= Lock} from the rock-dweller \ken{ettin} got, as he yielded up both his own children for it.\footnoteB{As pointed out in \textcite{FinnurEdda}, a verse containing such an address to the audience is otherwise unheard of. — What myth is being referred to is unclear. TODO: What do other authors write. Check Snorre.}\evb
\evg


\bvg
\bva\mssnote{\Regius~15r/1, \AM~6v/24}\alst{Þ}róttǫflugr kom \hld\ á \alst{þ}ing goða &
ok \alst{h}afði \alst{h}ver, \hld\ þann’s \alst{H}ymir átti; &
en \alst{v}éar hvęrjan \hld\ \alst{v}ęl skulu drekka &
\alst{ǫ}lðr at \alst{Ę́}gis \hld\ \edtext{\alst{ęi}tt hǫrmęitið}{\lemma{ęitt hǫrmęitið ‘one \dots\ flax-cutting’}\Bfootnote{A very obscure kenning. \textcite{LaFargeGlossary} give several interpretations, viz. \emph{ęitr-hǫr-męitir} ‘poison-rope-cutter \ken{snake > winter}’, \emph{ęitr-orm-męiðir} ‘poison-worm-injurer’ \ken{winter}. The solution with the minimal amount of emendation is to read \emph{ęitt} ‘one’ as modifying \emph{ǫlðr} ‘ale-feast’, and \emph{hvęrjan} ‘every’ as modifying \emph{hǫr-męitiðr} ‘flax-cutting’, a compound made up of \emph{hǫrr} ‘flax, cord’ and \emph{męita} ‘to cut’, seemingly referring to an obscure harvest festival. This interpretation is by no means certain.}}.\eva

\bvb The valour-mighty one \ken*{= Thunder} came onto the \inx[C]{Thing} of the gods, and had that cauldron which Hymer owned; but the \inx[G]{Wigh-beings} \name{= gods} shall well drink one ale-feast at Eagre’s every flax-cutting \ken{fall?}.\evb
\evg
