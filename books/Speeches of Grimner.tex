\bookStart{The Speeches of Grimner}[Grímnismǫ́l]

% Introduction

The \textbf{Speeches of Grimner} are preserved whole in both \Regius\ and \AM.

The structure of the poem is mostly clear; the first three verses set the stage, repeating some of what we got in the prose. It is certain that Weden is the speaker. After this various lore is touched on, not always clearly. In this the poem aligns closely with ones such as \Vafthrudnismal\, \Sigrdrifumal\ and \Allvismal.

First are listed the halls of the gods (4–17), though the numbering does not seem to agree with the count of locations mentioned. Then the conditions and surroundings of Weden’s animals and hall are elaborated on (18–23). Mentioned are the preparation of food (18), his wolves (19) and ravens (20), the river through which dead men have to wade (21), the gate through which they have to pass (22), the count of doors in the hall (23) and the two animals who gnaw on the branches of the tree (25–26). We then have a long list of rivers (28–30) and horses ridden by the gods (31). Then is told of the conditions and animals of Ugdrassle (32–36).

Thereafter follow several discordant verses. A list of Walkirries (37), the progression of the sun and moon (38–40), the first \inx[C]{bloot} and creation of the world from Yimer’s body (41–42), the significance of the bloot for men in the present (43), the creation of the ship Shidebladner (44) and finally a list of the noblest of several categories of things and groups (45).

After all of this Weden utters an unclear verse invoking the gods (46), before listing many of his names and the circumstances in which they were used (47–50). He then turns to Garfrith, disappointed by the inhospitality and poor conduct of his former protégé, and predicts his imminent death (51–53). He finally reveals himself by his true name, daring Garfrith to face him (53). After this he repeats several of his names (54), and the poem ends.

In the final prose section we are told that Garfrith tripped and fell on his sword, after which his son Eyner ruled for a long time.



Frá sonum Hrauðungs konungs

From the sons of king Reeding

BPG
BPA Hrauðungr konungr átti tvá sonu. Hét annarr Agnarr, enn annarr Geirrøðr.
BPA Agnarr var tíu vetra enn Geirrøðr átta vetra. Þeir reru tveir á báti með dorgar sínar at smáfiski.
BPA Vindr rak þá í haf út. Í náttmyrkri brutu þeir við land ok gingu upp; fundu kotbónda einn.
BPA Þar vǫ́ru þeir um vetrinn. Kerling fostraði Agnar enn karl Geirrøð.
BPA At vári fekk karl þeim skip. Enn er þau kerling leiddu þá til strandar, þá mę́lti karl einmę́li við Geirrøð.
BPA Þeir fengu byr ok kvǫ́mu til stǫðva fǫður síns. Geirrøðr var fram í skipi.
BPA Hann hljóp upp á land enn hratt út skipinu, ok mę́lti: ”Far þú þar er smyl hafi þik.”
BPA Skipit rak út. Enn Geirrøðr gekk út til bǿjar; hánum var vel fagnat; þá var faðir hans andaðr.
BPA Var þá Geirrøðr til konungs tekinn, ok varð maðr ágę́tr.

BPB King Reeding owned two sons. One was called Eyner, and the other Garfrith.
BPB Eyner was ten winters old, and Garfrith eight winters. The two were rowing in a boat with their trolling-lines for small fishing.
BPB Wind then drove them out into the sea. In the darkness of night they crashed into land and walked up; they found a single cottage-farmer.
BPB There they were about the winter. The wife fostered Eyner, but the husband Garfrith.
BPB At spring the man gave them ships, but when they and the farmer’s wife brought them to the shore, the husband spoke privately with Garfrith.
BPB They got a good gust, and came to their father’s harbour. Garfrith was in the front of the ship.
BPB He leapt up onto land and pushed out the ship, and spoke: ”Go thou where the \inx[G]{smil} may have thee.”
BPB The ship drove out. But Garfrith walked towards the farm; he was welcomed well; his father was by then ended.
BPB Then was Garfrith taken as king, and became an excellent man.
EPG


BPG
BPA Óðinn ok Frigg sátu í Hliðskjǫlfu ok sá um heima alla.
BPA Óðinn mę́lti: Sér þú Agnar fóstra þinn, hvar hann elr bǫrn við gýgi í hellinum?
BPA En Geirrøðr, fóstri minn, er konungr ok sitr nú at landi.
BPA Frigg segir: Hann er matníðingr sá at hann kvelr gesti sína ef hánum þykkja ofmargir koma.
BPA Óðinn segir at þat er in mesta lygi. Þau veðja um þetta mál.
BPA Frigg sendi eskismey sína, Fullu, til Geirrøðar. Hon bað konung varask at eigi fyrgerði hánum fjǫlkunnigr maðr sá er þar var kominn í land ok sagði þat mark á at engi hundr var svá ólmr at á hann myndi hlaupa.
BPA En þat var inn mesti hégómi at Geirrøðr vę́ri eigi matgóðr ok þó lę́tr hann handtaka þann mann er eigi vildu hundar á ráða.
BPA Sá var í feldi blám ok nefndisk Grímnir ok sagði ekki fleira frá sér þótt hann vę́ri atspurðr.
BPA Konungr lét hann pína til sagna ok setja milli elda tveggja ok sat hann þar átta nę́tr.
BPA Geirrøðr konungr átti son tíu vetra gamlan ok hét Agnarr eftir bróður hans.
BPA Agnarr gekk at Grímni ok gaf hánum horn fullt at drekka, sagði að konungr gerði illa er hann lét pína hann saklausan.
BPA Grímnir drakk af. Þá var eldrinn svá kominn at feldrinn brann af Grímni. Hann kvað:

BPB Weden and Frie sat in \inx[G]{Litheshelf} and looked about all the Homes.
BPB Weden spoke: Seest thou Eyner thy foster-son, where he begets children with the troll-woman in the cave?
BPB But Garfrith, my foster-son, is king and now sits at land.
BPB Frie says: “He is such a meat-nithing that he tortures his guests if he judges too many are coming.”
BPB Weden says that this is the greatest lie; they make a wager about this matter.
BPB Frie sent her handmaid Full to Garfrith’s. She asked the king to be wary, that he might not be ended by that \inx[C]{feel-cunning} man who was come in the land, and said that his sign was that no hound was so fierce that he would leap at him.
BPB But that was the greatest vainglory that Garfrith would not be meat-good, and yet he has that man seized, whom the hounds would not touch.
BPB He was clad in a blue cloak, and called himself Grimner, and did not tell any more about himself, even though he was interrogated.
BPB The king had him tortured so that he would speak, and set him between two fires, and he remained there for eight nights.
BPB King Garfrith had a son ten winters old, and he was named Eyner after his brother.
BPB Eyner walked up to Grimner, and gave him a full horn to drink, saying that the king did ill as he had him tortured without cause.
BPB Grimner drank from it; then the fire had come such that the cloak burned on Grimner. He quoth:
EPG


\bvg
\bva Hęitr est hripuðr \hld\ ok hęldr til mikill, &
\ind gǫngumk firr funi. &
Loði sviðnar, \hld\ þótt á lopt bera'k; &
\ind brinnumk feldr fyrir.\eva

\bvb Hot art thou, flame, and rather too large; go far from me, fire! The woolen cape is singed though I hold it aloft; the cloak burns before me.\evb
\evg


\bvg
\bva Átta nę́tr satk \hld\ milli ęlda hér, &
\ind svát mér mangi \hld\ mat né bauð &
nema ęinn Agnarr, \hld\ es ęinn skal ráða, &
Gęirrøðar sonr, \hld\ Gotna landi.\eva

\bvb For eight nights sat I between the fires here, while no man offered me food; save for lone Eyner, who lone shall rule—the son of Garfrith—the land of the Gots!\evb
\evg


\bvg
\bva Hęill skalt, Agnarr, \hld\ alls hęilan biðr &
\ind þik Veratýr vesa; &
ęins drykkjar \hld\ þú skalt aldrigi &
\ind bętri gjǫld geta.\eva

\bvb Hale shalt thou be, Eyner, as hale Weretue <= Weden> bids thee to be; for one drink shalt thou never get a better recompense.\footnoteB{The recompense being the esoteric lore which is told starting with the following verse.}\evb
\evg


\bvg
\bva Land es hęilagt, \hld\ es liggja sé’k &
\ind ǫ́sum ok ǫlfum nę́r; &
ęn í Þrúðhęimi \hld\ skal Þórr vesa &
\ind unz of rjúfask ręgin.\eva

\bvb The land is holy, which I see lying close to the \inx[G]{Ease and Elves}; but in Thrithham shall Thunder be, until the Reins are rent.\evb
\evg


\bvg
\bva Ýdalir hęita, \hld\ þar’s Ullr of hęfr &
\ind sér of gǫrva sali; &
Alfhęim Fręy \hld\ gǫ́fu í árdaga &
\ind tívar at tannféi.\eva

\bvb Yewdales are called where Woulder has made himself a hall. Elfham to Free in days of yore the Tues as a tooth-gift\footnoteB{The gift that a child receives when he gets his first tooth.} gave.
\evg


\bvg
\bva Bǿr ’s hinn þriði, \hld\ es blíð ręgin &
\ind silfri þǫkðu sali; &
Valaskjǫlf hęitir, \hld\ es vélti sér &
\ind ǫ́ss í árdaga.\eva

\bvb Bower is the third, where the blithe Reins with silver thatched a hall. Waleshelf is called, where tricked himself, the os in days of yore.\evb
\evg


\bvg
\bva Søkkvabękkr hęitir hinn fjórði, \hld\ ęn þar svalar knegu &
\ind unnir glymja yfir; &
þar þau Óðinn ok Sága \hld\ drekka umb alla daga &
\ind glǫð ór gollnum kęrum.\eva

\bvb Sinkbench is called the fourth, but there cool waves do clash above; there Weden and Sey drink all days, gladly out of golden vats.\evb
\evg


\bvg
\bva Glaðshęimr hęitir hinn fimti \hld\ þar’s hin gollbjarta &
\ind Valhǫll víð of þrumir; &
ęn þar Hroptr \hld\ kýss hvęrjan dag &
\ind vápndauða vera.\eva

\bvb Gladsham is called the fifth, where the gold-bright Walhall—wide—stands fast; but there Roft \name{= Weden} chooses every day weapon-dead men.\evb
\evg


\bvg
\bva Mjǫk ’s auðkęnt \hld\ þęim’s til Óðins koma &
\ind salkynni at séa, &
skǫptum ’s rann rępt, \hld\ skjǫldum ’s salr þakiðr, &
\ind brynjum of bękki stráat.\eva

\bvb Very easily recognized, for those who to Weden come, is the hall to see: With shafts is the house roofed; with shields is the hall thatched; with byrnies the benches strewn.\evb
\evg


\bvg
\bva Mjǫk ’s auðkęnt \hld\ þęim’s til Óðins koma &
\ind salkynni at séa, &
vargr hangir \hld\ fyr vestan dyrr &
\ind ok drúpir ǫrn yfir.\eva

\bvb Very easily recognized, for those who to Weden come, is the hall to see: A wolf hangs before the western door, and an eagle droops over.\evb
\evg


\bvg
\bva Þrymhęimr hęitir hinn sétti, \hld\ es Þjazi bjó, &
\ind sá hinn ámátki jǫtunn; &
ęn nú Skaði byggvir, \hld\ skír brúðr goða, &
\ind fornar toptir fǫður.\eva

\bvb Thrimham is called the sixth, where Thedse dwelled, that terrifying ettin; but now Scathe bedwells—pure bride of the gods—the ancient plots of her father.\evb
\evg


\bvg
\bva Bręiðablik eru hin sjaundu, \hld\ ęn þar Baldr hęfir &
\ind sér of gǫrva sali, &
á því landi \hld\ es liggja vęit’k &
\ind fę́sta fęiknstafi.\eva

\bvb Broadblicks are the seventh, and there Balder has made for himself a hall; on that land, where I know lie the fewest staves of treachery.\footnoteB{Evil deeds.}\evb
\evg


\bvg
\bva Himinbjǫrg eru in ǫ́ttu \hld\ ęn þar Hęimdall &
\ind kveða valda véum. &
þar vǫrðr goða \hld\ drękkr í vę́ru ranni &
\ind glaðr góða mjǫð.\eva

\bvb Heavenbarrows are the eighth, and there Homedall, they say, wields over wighs. There in the tranquil house the ward of the gods \ken*{= Homedall} drinks glad the good mead.\evb
\evg


\bvg
\bva Folkvangr es inn níundi \hld\ en þar Fręyja rę́ðr &
\ind sessa kostum í sal; &
halfan val \hld\ hon kýss hvęrjan dag &
\ind ęn halfan Óðinn á.\eva

\bvb Folkwong is the ninth, and there Frow wields the choice of seats in the hall; half of the slain she chooses each day, but half Weden owns.\evb
\evg


\bvg
\bva Glitnir es inn tíundi; \hld\ hann es gulli studdr &
\ind ok silfri þakðr it sama; &
ęn þar Forseti \hld\ byggir flęstan dag &
\ind ok svę́fir allar sakir.\eva

\bvb Glitner is the tenth, it is studded by gold, and thatched by silver the same; but there Forset dwells most of the day, and resolves\footnoteB{Puts to sleep,} all [legal] matters.\evb
\evg


\bvg
\bva Nóatún eru in ęlliftu \hld\ ęn þar Njǫrðr hęfir &
\ind sér um gǫrva sali, &
manna þęngill \hld\ inn męinsvani &
\ind hátimbruðum hǫrgi rę́ðr.\eva

\bvb Nowetowns are the tenth, and there Nearth has made himself a hall. The prince of men, the guileless one, rules the high-timbered \inx[C]{harrow}.\footnoteB{Cf. \Vafthrudnismal\ 38.}\evb
\evg


\bvg
\bva Hrísi vęx \hld\ ok hǫ́u grasi &
\ind Víðars land, viði, &
ęn þar mǫgr of lę́zk \hld\ af mars baki &
\ind frǿkn at hęfna fǫður.\eva

\bvb With brushwood overgrown—and tall grass—is \inx[P]{Wider}’s land, [and] with forest;\footnoteB{lit. ‘With brushwood grows—and tall grass—Wider’s land, with forest’} but there the lad \ken*{= Wider} declares—on the back of his steed—valiant, to avenge his father \ken*{= Weden}.\footnoteB{Wider will avenge his father, Weden. See \Vafthrudnismal\ 53.}\evb
\evg


\bvg
\bva Andhrímnir \hld\ lę́tr í Ęldhrímni &
\ind Sę́hrímni soðinn, &
flęska bęzt, \hld\ ęn þat fáir vitu, &
\ind við hvat ęinhęrjar alask.\eva

\bvb Andrimner lets in Eldrimner Sowrimner be boiled. The best of meats, but few know that, by what the Ownharriers are nourished.\footnoteB{The cook Andrimner ‘face-sooty’ has the boar Sowrimner ‘sow-sooty’ boiled in the cauldron Eldrimner ‘fire-sooty’; by this meat are the Ownharriers nouished.}\evb
\evg


\bvg
\bva Gera ok Freka \hld\ sęðr gunntamiðr, &
\ind hróðigr Hęrjafǫðr, &
ęn við vín ęitt \hld\ vápngǫfugr &
\ind Óðinn ę́ lifir.\eva

\bvb The battle-accustomed, glorious Father of Hosts \ken*{= Weden} feeds Gerr and Freck; but by wine alone, the weapon-worshipful Weden ever lives.\evb
\evg


\bvg
\bva Huginn ok Muninn \hld\ fljúga hvęrjan dag &
\ind jǫrmungrund yfir; &
óumk of Hugin, \hld\ at aptr né komit; &
\ind þó séumk męir of Munin.\eva

\bvb Highen and Minden fly every day over the \inx[C]{ermin-ground} \ken{earth}. I fear for Highen, that he come not back; yet I worry more for Minden.\evb
\evg


\bvg
\bva Þýtr Þund, \hld\ unir Þjóðvitnis &
\ind fiskr flóði í; &
áarstraumr \hld\ þykkir ofmikill &
\ind valglaumi at vaða.\eva

\bvb \inx[P]{Thound} roars; Thedwitner’s fish\footnoteB{A difficult kenning to interpret, but see TODO.} dwells in the flood; the river-stream seems far too great for the noisy slain host \ken*{= Ownharriers} to wade through.\footnoteB{Presumably describing the river surrounding Walhall, which the dead have to pass over to reach the hall.}\evb
\evg


\bvg
\bva Valgrind hęitir \hld\ es stęndr vęlli á &
\ind heilǫg fyr hęlgum durum; &
forn ’s sú grind, \hld\ ęn þat fáir vitu, &
\ind hvé hon ’s í lás of lokin.\eva

\bvb \inx[L]{Walgrind}\footnoteB{‘Corpse-gate;’ the gate guarding Walhall.} is called, which stands on the plain; holy, before the holy doors. Ancient is that gate, but few know that, how it’s lock is locked.\evb
\evg


\bvg
\bva Fimm hundruð golfa \hld\ ok umb fjórum tøgum &
\ind svá hygg’k Bilskirni með bugum; &
ranna þęira, \hld\ es rępt vita’k, &
\ind míns vęit’k męst magar.\eva

\bvb With five hundred floors, and around fourty, so I judge \inx[L]{Bilshirner} altogether. Of those houses, which I might know rafted, I know my lad’s \ken*{= Thunder} to be the greatest.\evb
\evg


\bvg
\bva Fimm hundruð dura \hld\ ok umb fjórum tøgum, &
\ind svá hygg at Valhǫllu vesa; &
átta hundruð Ęinhęrja \hld\ ganga ór ęinum durum, &
\ind þá’s fara við vitni at vega.\eva

\bvb With five hundred doors, and around fourty, so I judge Walhall to be. Eight hundred \inx[G]{Ownharriers} go out of one door,\footnoteB{The hundred is probably here the long hundred (120, rather than 100), which gives a sum of \(640 * 960 = 614,400\) Ownharriers.} when they journey to fight with the wolf.\evb
\evg


\bvg
\bva Hęiðrún hęitir gęit, \hld\ es stęndr \edtext{hǫllu á}{\lemma{hǫllu á ‘on hall’}\Afootnote{TODO.}} &
\ind ok bítr af Lę́raðs limum; &
skapkęr fylla \hld\ hon skal hins skíra mjaðar, &
\ind kná-at sú vęig vanask.\eva

\bvb Heathrune is called the goat, which stands on the hall \ken*{= Walhall}, and bites off the branches of Leered. The shape-vats\footnoteB{According to \CV\ the central beer-vat, from which drinks were poured into smaller vessels.} shall she fill with the pure mead; those draughts cannot wane.\footnoteB{The mead is the goat’s milk.}\evb
\evg


\bvg
\bva Ęikþyrnir hęitir hjǫrtr \hld\ es stęndr \edtext{hǫllu á}{\lemma{hǫllu á ‘on hall’}\Afootnote{TODO. See previous v.}}&
\ind ok bítr af Lę́raðs limum; &
en af hans hornum \hld\ drýpr í Hvergęlmi &
\ind þaðan ęiga vǫtn ǫll vega:\eva

\bvb Oakthirner is called the stag, which stands on the hall \ken*{= Walhall}, and bites off the branches of Leered. But from his horns does drip into Wharyelmer; thence all waters have their ways:\footnoteB{After which several vv. of mythic river-names are listed.}\evb
\evg


\bvg
\bva TODO\eva

\bvb TODO\evb
\evg


\bvg
\bva TODO\eva

\bvb TODO\evb
\evg


\bvg
\bva Kǫrmt ok Ǫrmt \hld\ ok kęrlaugar tvę́r &
\ind þę́r skal Þórr vaða &
dag hvęrn \hld\ es dǿma fęrr &
\ind at aski Yggdrasils; &
því’t ǫ́sbrú \hld\ bręnn ǫll loga &
\ind hęilǫg vǫtn \edtext{hlóa}{\Bfootnote{A hapax. TODO.}}.\eva

\bvb Carmt and Armt, and the two Carlays, those shall Thunder wade\footnoteB{For Thunder’s association with wading cf. TODO.} every day when to judge he fares, at the ash of \inx[L]{Ugdrassle}; for the \inx[G]{ease}[os]-bridge \ken{rainbow} burns all with flame; the holy waters bellow.\evb
\evg


\bvg
\bva Glaðr ok Gyllir, \hld\ Glęr ok Skęiðbrimir, &
\ind Silfrintoppr ok Sinir, &
Gísl ok Falhófnir, \hld\ Gulltoppr ok Léttfeti, &
\ind þęim ríða ę́sir jóum &
dag hvęrn \hld\ es dǿma fara &
\ind at aski Yggdrasils.\eva

\bvb Glad and Yiller, Glare and Sheathbrimmer, Silvrentop and Sinewer, Yissel and Fallowhofner, Goldtop and Lightfeet; on those horses ride the Ease, every day when to judge they fare, at the ash of \inx[L]{Ugdrassle}.\evb
\evg


\bvg
\bva Þríar rǿtr \hld\ standa á þría vega &
\ind undan aski Yggdrasils; &
Hel býr und ęinni, \hld\ annarri hrímþursar, &
\ind þriðju męnnskir męnn. \eva

\bvb Three roots stand on three ways, from beneath the ash of Ugdrassle. Hell lives under one, [under] another the \inx[G]{Rime-Thurses}, [under] the third manly men.\evb
\evg


\bvg
\bva Ratatoskr hęitir íkorni \hld\ es rinna skal &
\ind at aski Yggdrasils; &
arnar orð \hld\ hann skal ofan bera &
\ind ok sęgja Níðhǫggvi niðr.\eva

\bvb Wratetusk is called the squirrel, who shall run at the ash of Ugdrassle. The eagle’s words he shall carry from above, and say to Nithehew below.\evb
\evg


\bvg
\bva Hirtir eru ok fjórir \hld\ þęir’s af hę́fingar &
\ind á gaghálsir gnaga, &
Dáinn ok Dvalinn, \hld\ Dúnęyrr ok Duraþrór.\eva

\bvb TODO\evb
\evg


\bvg
\bva Ormar flęiri \hld\ liggja und aski Yggdrasils &
\ind an þat of hyggi hvęrr ósviðra apa:\eva

\bvb More worms lie under the ash of Ugdrassle than each unwise \inx[C]{ape} might think:\evb
\evg


\bvg
\bva TODO\eva

\bvb TODO\evb
\evg


\bvg
\bva Askr Yggdrasils \hld\ drýgir ęrfiði &
\ind męira an męnn viti: &
Hjǫrtr bítr ofan \hld\ en á hliðu fúnar, &
\ind skęrðir Níðhǫggr neðan.\eva

\bvb The ash of Ugdrassle undergoes hardship greater than men might know: a hart bites it from above, but it rots on the side; Nithehew gnaws at it from below.\evb
\evg


\bvg
\bva TODO\eva

\bvb TODO\evb
\evg


\bvg
\bva Árvakr ok Alsviðr, \hld\ skulu upp heðan &
\ind svangir sól draga; &
en und þęira bógum \hld\ fǫ́lu blíð ręgin, &
\ind ę́sir, ísarnkol. \eva

\bvb Yorewaker and Allswith\footnoteB{These figures both appear in \Sigrdrifumal\ TODO. Along with the close formulation of the next verse, it is clear that they are closely related.} shall above hence—slender [horses]—pull the sun; but under their shoulders hid the blithe Reins—the Ease—iron-coal.\evb
\evg


\bvg
\bva Svalinn hęitir, \hld\ hann stęndr sólu fyrir, &
\ind skjǫldr skínanda goði; &
bjǫrg ok brim \hld\ vęit’k at brinna skulu, &
ef hann fęllr í frá.\eva

\bvb Swollen is [one] called, he stands before the sun; a shield [before] the shining god \ken*{= Sun}. Crags and surf\footnoteB{The mountains and seas; the whole world.} I know shall burn, if he falls away.\footnoteB{The sun-disc was apparently thought to be a translucent shield, which protected the earth from the full power of the Sun. Cf. also \Sigrdrifumal\ TODO.}\evb
\evg


\bvg
\bva Skoll hęitir ulfr, \hld\ es fylgir hinu skírlęita &
\ind goði til varna viðar, &
ęn annarr Hati, \hld\ hann ’s Hróðvitnis sonr, &
\ind sá skal fyr hęiða brúði himins.\eva

\bvb \inx[P]{Skoll} is called the wolf, which follows the pure-skinned god \ken*{= Sun} to the protection of the woods; but another one [is called] \inx[P]{Hate}, he is the son of \inx[P]{Rothwitner}, who shall [go] before the bright bride of heaven \ken*{= Sun}.\footnoteB{TODO \Gylfaginning 12.}\evb
\evg


\bvg
\bva Ór Ymis holdi \hld\ vas jǫrð of skǫpuð, &
\ind ęn ór svęita sę́r, &
bjǫrg ór bęinum, \hld\ baðmr ór hári, &
\ind ęn ór hausi himinn.\eva

\bvb Out of Yimer’s hull was the earth shaped, but out of his blood\footnoteB{In poetry \emph{svęiti}, while cognate with English ‘sweat’, almost always carries the meaning of ‘blood’. See Lexicon Poeticum TODO.} the seas; crags out of his bones, trees out of his hair, but out of his skull, heaven.\evb
\evg


\bvg
\bva En ór hans brǫ́um \hld\ gęrðu blíð ręgin &
\ind Miðgarð manna sonum, &
ęn ór hans hęila \hld\ vǫ́ru þau hin harðmóðgu &
\ind ský ǫll of skǫpuð.\eva

\bvb But out of his eyebrows the blithe \inx[G]{Reins} made \inx[L]{Middenyard} for the sons of men;\footnoteB{I agree with \textcite{FinnurEdda} in that this describes the gods enclosing Middenyard by using his eyebrows as poles. } but out of his brains were the hard-stirred skies all shaped.\evb
\evg


\bvg
\bva Ullar hylli \hld\ hęfr ok allra goða &
\ind hvęrr’s tękr fyrstr á funa, &
því’t opnir hęimar \hld\ verða of ása sonum, &
\ind þá’s hęfja af hvera.\eva

\bvb The favour of \inx[C]{Woulder}—and of all the gods—has each who first touches the fire; for the \inx[C]{Home}[Homes] become open o’er the sons of the Ease, when the cauldrons are heaved off.\footnoteB{This verse is one of the most difficult in the poem, and many interpretations have been made (for a summary see \textcite{Nordberg2005}). \textcite{FinnurEdda} and Sijmons and Gering (p. 208, TODO) interpret this verse as relating to the frame narrative, with Weden still bound between the two fires, wishing for the gods to rescue him. This, however, scarcely makes sense given its placement in the middle of various gnomic verses. I believe instead (and here I agree with \parencite{Nordberg2005}) that the verse refers to the cooking and eating of sacred stew in large cauldrons during the \inx[C]{bloot}, and Woulder’s role in the setting of the ritual fire (see Index and \parencite{afEdholm2009}). This interpretation is especially interesting in that this verse immediately follows two verses dealing with the primordial sacrifice of Yimer to create the world. This shows that the bloot was viewed as a ritual reenactment of the creation of the world by the gods (and indeed a continuation of that creation), something that is well attested comparatively (see \parencite{Lincoln1986}, especially the first two chs., for its Indo-European analogues).}\evb
\evg


\bvg
\bva Ívalda synir \hld\ gingu í árdaga &
\ind Skíðblaðni at skapa, &
skipa bazt \hld\ skírum Fręy, &
nýtum Njarðar bur.\eva

\bvb The sons of Iwald went, in days of yore, Shidebladner to shape; the best of ships for the pure Free, the useful son of Nearth \ken*{= Free}.\evb
\evg


\bvg
\bva Askr Yggdrasils, \hld\ hann es ǿztr viða &
\ind ęn Skíðblaðnir skipa, &
Óðinn ása \hld\ ęn jóa Slęipnir, &
Bilrǫst brúa \hld\ ęn Bragi skalda, &
Hábrók hauka \hld\ ęn hunda Garmr.\eva

\bvb The ash of Ugdrassle, that is the noblest of trees, but Shidebladner of ships; Weden of the Ease, but of horses Slopner; Bilrest of bridges, but Bray of scolds; Highbrook of hawks, but of hounds Garm.\evb
\evg


\bvg
\bva Svipum hęfk nú ypt \hld\ fyr sigtíva sonum, &
\ind við þat skal vilbjǫrg vaka, &
ǫllum ǫ́sum \hld\ þat skal inn koma &
\ind Ę́gis bękki á &
\ind Ę́gis drekku at.\eva

\bvb My gaze have I now lifted up before the sons of the victory-Tues \ken*{= Ease}; by that shall the willed rescue awake.\footnoteB{Weden has made the Ease aware of his identity, and thus they will come to help him.} With all the Ease shall it come in, onto the benches of Eagre, at the drinking of Eagre.\evb
\evg


...


\bvg
\bva Ǫlr est Gęirrøðr, \hld\ hęfr þú of drukkit; &
miklu est hnugginn, \hld\ es þú est mínu gęngi, &
ǫllum ęinhęrjum \hld\ ok Óðins hylli. \eva

\bvb Worse for ale art thou, Garfrith, hast thou too much drunk. Of much art thou bereft, as thou art of my support; of all the Ownharriers, and of Weden’s favour.\evb
\evg


\bvg
\bva Fjǫlð þér sagða’k, \hld\ ęn þú fátt of mant, &
\ind of þik véla vinir;
mę́ki liggja \hld\ sé’k míns vinar &
\ind allan í dręyra drifinn.\eva

\bvb Much I told thee, but thou recallest little; ’tis friends that deal with thee. The sword I see, of my friend, lying all drenched in gore.\footnoteB{Weden predicts Garfrith’s imminent death.}\evb
\evg


\bvg
\bva Ęggmóðan val \hld\ nú mun Yggr hafa, &
\ind þitt vęitk líf of liðit; &
varar ro dísir, \hld\ nú knátt Óðin séa; &
\ind nálgask mik ef þú męgir.\eva

\bvb An edge-tired corpse will Ug now have; I know thy life to be passed. Wary are the dises; now thou dost see Weden—approach me, if thou mayst!\evb
\evg


\bvg
\bva Óðinn nú hęiti’k, \hld\ Yggr áðan hét’k, &
\ind hétumk Þundr fyr þat, &
Vakr ok Skilfingr, \hld\ Vǫ́fuðr ok Hroptatýr &
\ind Gautr ok Jalkr með goðum. &
Ófnir ok Sváfnir \hld\ hygg at orðnir sé &
\ind allir at ęinum mér.\eva

\bvb Weden I am now called, Ug was I earlier called; I called myself Thound before that. Wacker and Shelfing, Waved and Roft-Tue, Geat and Gelding among the gods. Ofner and Sweefner, I ween, are become all for the one me.\evb
\evg


Geirröðr konungr sat ok hafði sverð um kné sér ok brugðit til miðs. En er hann heyrði at Óðinn var þar kominn stóð hann upp ok vildi taka Óðin frá eldinum. Sverðit slapp ór hendi hánum; vissu hjöltin niðr. Konungr drap fę́ti ok steyptiz áfram en sverðit stóð í gögnum hann ok fekk {hann}{þar af \AM} bana. {Óðinn hvarf þá.}{\emph{om.} \AM} En Agnarr {var þar}{varð \AM} konungr {lengi síðan.}{\emph{om.} \AM}

King Garfrith sat and had a sword about his knee, and it was brandished half-way up. But when he heard that Weden was come there, he stood up and would take Weden from the fire. The sword slipped out of his hand; the hilt pointed downwards. The king tripped and threw himself forth, but the sword pierced him, and he received his bane. Weden then disappeared, but Eyner was there king for a long while thence.
