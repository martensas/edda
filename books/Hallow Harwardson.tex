\bookStart{Lay of Hallow Harwardson}[Hęlgakviða Hjǫrvarðssonar]
\def\thisBookCode{HelgakvidaHjorvardssonar}

\begin{flushright}%
\textbf{Dating} \parencite{Sapp2022}: early C11th (0.385)–late C11th (0.550)

\textbf{Meter:} \Fornyrdislag%
\end{flushright}%

Heroic poem.

\sectionline

\section{From Harward and Syelind (\emph{Frá Hjǫr·varði ok Sigr·linn})}

\bpg\bpa Hjǫr·varðr hét konungr; hann átti fjórar konur.  Ein hét Alf·hildr; sonr þeira hét Heðinn.  Ǫnnur hét Sę́·reiðr; þeira sonr hét Humlungr.  In þriðja hét Sinrjóð; þeira sonr hét Hymlingr.  Hjǫr·varðr konungr hafði þess heit strengt at eiga þá konu er hann vissi vę́nsta.  Hann spurði at Sváfnir konungr átti dóttur allra\footnote{‘vęnallra’ \emph{corr.} \Regius} fegrsta; sú hét Sigr·linn.  Ið·mundr hét jarl hans; Atli var hans sonr er fór at biðja Sigr·linnar til handa konungi.  Hann dvalðisk vetr-langt með Sváfni konungi.  Frán·marr hét þar jarl, fóstri Sigr·linnar; dóttir hans hét Álǫf.  Jarl’inn réð, at meyjar var synjat, ok fór jarlinn heim.  Atli jarls sonr stóð einn dag við lund nǫkkurn, en fugl sat í limunum uppi yfir hánum ok hafði heyrt til, at hans menn kǫlluðu vę́nstar konur þę́r, er Hjǫr·varðr konungr átti.  Fugl’inn kvakaði, en Atli hlýddi, hvat hann sagði. Hann kvað:\epa

\bpb Hearward was the name of a king; he had four women.  One was called Elfhild; their son was called Headen.  Another was called Searad; their son was called Humbling.  The third was called Sindred; their son was called Himbling.  King Hearward had made a vow to have those women whom he knew the most handsome.  He learned that king Swebner had a daughter fairest of all; she was called Syelind.  Ithmund was the name of his earl; Attle was his son, who journeyed to ask for Syelind’s hand on behalf of the king.  He stayed over the winter with king Swebner.  Frenmar was the name of an earl there, the foster-father of Syelin; his daughter was called Anlab.  The bird twittered, and Attle listened to what it said.  It quoth:\epb\epg


\bvg\bva%
„\alst{S}átt-u \alst{S}igr·linn, \hld\ \alst{S}váfnis dóttur, &
męyna fęgrstu \hld\ ï \alst{m}unar-hęimi? &
Þó \alst{h}ag-ligar \hld\ \alst{H}jǫr·varðs konur &
\alst{g}umnum þykkja \hld\ at \alst{G}lasis-lundi.“\eva

\bvb “Hast thou seen Syelind Swebner’s daughter, \\
the fairest of maidens in the realm of love \ken{world}? \\
Although to mankind Hearward’s wives \\
seem handsome in Glazerslund.”\evb\evg


\bvg\bva%
„Munt við \alst{A}tla \hld\ \alst{I}ð·mundar son &
\alst{f}ugl \alst{f}róð-hugaðr \hld\ \alst{f}lęira mę́la?“ &
„Mun’k ef mik \alst{b}uðlungr \hld\ \alst{b}lóta vildi &
ok \alst{k}ýs’k þat’s ek vil \hld\ ór \alst{k}onungs garði.“\eva

\bvb “Wilt thou with Attle Idmund’s son, \\
O wise-minded fowl, speak yet further?” \\
“I will, if the prince will make me a bloot, \\
and I may choose what I wish from the house of the king.”\evb\evg


\bvg\bva%
Kjós-at-tu Hjǫr·varð \hld\ né hans sonu &
\eva

\bvb 3\evb\evg


\bvg\bva%
Hof mun’k kjósa, \hld\ hǫrga marga, &
gull-hyrndar kýr \hld\ frȧ grams búi, &
ef hǫ́num Sigr·linn \hld\ søfr ȧ armi &
ok ȯ·nauðig \hld\ jǫfri fylgir.\eva

\bvb 4\evb\evg


\bpg\bpa%
Þetta var áðr Atli fǿri. En er hann kom heim ok konungr spurði hann tíðinda, hann kvað:\epa

\bpb TODO.\epb\epg

\bvg\bva%
Hǫfum erfiði \hld\ ok ękki ørendi;\eva

\bvb 5\evb\evg


\bpg\bpa%
TODO.\epa

\bpb TODO.\epb\epg


\bpg\bpa%
TODO.\epa

\bpb TODO.\epb\epg


\bvg\bva%
6\eva

\bvb 6\evb\evg


\bvg\bva%
7\eva

\bvb 7\evb\evg


\bvg\bva%
\alst{S}verð vęit’k liggja \hld\ ï \alst{S}igars-holmi, &
\alst{f}jórum \alst{f}ę́ra \hld\ enn \alst{f}imm tǫgu; &
\alst{ęi}tt es þęira \hld\ \alst{ǫ}llum bętra &
\alst{v}íg-nesta bǫl \hld\ ok \alst{v}arið gulli.\eva

\bvb Swords I know lying in Sigarsholm: \\
four less than fifty. \\
One of them is better than all—\\
a \inx[C]{bale} of war-covers(?) \ken{shields}—and covered with gold.\evb\evg


\bvg\bva%
\edtrans{\alst{H}ringr ’s ï \alst{h}jalti}{A ring is on its hilt}{\Bfootnote{The sword is a ring-sword.  It was popular among Germanic warriors of the Migration Period to have oath-ring on their sword-hilts as a symbol of fidelity to their lords.  This custom was largely or entirely extinct by the Wiking Age, and the detail thus serves to emphasize the high age of the sword.  A well preserved Norwegian ring-sword survives from Snartemo in Vest-Agder,  dating to around 500 CE (object ID C26001); see Fig. \ref{fig:snartemo}.}}, \hld\ \alst{h}ugr ’s ï miðju, &
\alst{ó}gn ’s ï \alst{o}ddi, \hld\ þęim’s \alst{ęi}ga getr; &
liggr með \alst{ę}ggju \hld\ \alst{o}rmr dręyr-fáiðr &
en ȧ \edtrans{\alst{v}al-bǫstu}{walbast}{\Bfootnote{An unclear part of the sword-hilt; see \Sigrdrifumal\ 6.}} \hld\ \alst{v}erpr naðr hala.\eva

\bvb A ring is on its hilt; heart is in the middle; \\
terror is in the point for him who gets to own it. \\
Along the edge lies a serpent painted in blood \\
and on the walbast an adder eats its tail.\evb\evg

\begin{figure}
\centering
\includegraphics[width=\textwidth]{Snartemo-hilt}
\caption{Hilt of the Snartemo sword, front and reverse.  Migration period, ca. 500 CE.  © Eirik Irgens Johnsen, \href{https://creativecommons.org/licenses/by-sa/4.0/deed.en}{CC BY-SA 4.0}.  \url{https://www.unimus.no/portal/\#/photos/d8932af5-1082-4938-9b4b-ca6b86f2bdfb}}
\label{fig:snartemo}
\end{figure}


\bpg\bpa%
TODO.\epa

\bpb TODO.\epb\epg


TODO: many stanzas


\bpg\bpa%
Helgi ok Sváfa er sagt at vę́ri endr-borin.\epa

\bpb Hallow and Sweve, it is said, were reborn.\epb\epg

\sectionline
