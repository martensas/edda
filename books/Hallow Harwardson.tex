\bookStart{Lay of Hallow Harwardson}[Hęlgakviða Hjǫrvarðssonar]
\setBookCode{HelgakvidaHjorvardssonar}

\begin{flushright}%
\textbf{Dating} \parencite{Sapp2022}: early C11th (0.385)–late C11th (0.550)

\textbf{Meter:} \Fornyrdislag%
\end{flushright}%

Heroic poem.

\sectionline

\section{From Harward and Syelind (\emph{Frá Hjǫr·varði ok Sigr·linn})}

\bpg\bpa Hjǫr·varðr hét konungr; hann átti fjórar konur.  Ein hét Alf·hildr; sonr þeira hét Heðinn.  Ǫnnur hét Sę́·reiðr; þeira sonr hét Humlungr.  In þriðja hét Sinrjóð; þeira sonr hét Hymlingr.  Hjǫr·varðr konungr hafði þess heit strengt at eiga þá konu er hann vissi vę́nsta.  Hann spurði at Sváfnir konungr átti dóttur allra\footnote{‘vęnallra’ \emph{corr.} \Regius} fegrsta; sú hét Sigr·linn.  Ið·mundr hét jarl hans; Atli var hans sonr er fór at biðja Sigr·linnar til handa konungi.  Hann dvalðisk vetr-langt með Sváfni konungi.  Frán·marr hét þar jarl, fóstri Sigr·linnar; dóttir hans hét Álǫf.  Jarl’inn réð, at meyjar var synjat, ok fór jarlinn heim.  Atli jarls sonr stóð einn dag við lund nǫkkurn, en fugl sat í limunum uppi yfir hánum ok hafði heyrt til, at hans menn kǫlluðu vę́nstar konur þę́r, er Hjǫr·varðr konungr átti.  Fugl’inn kvakaði, en Atli hlýddi, hvat hann sagði. Hann kvað:\epa

\bpb Hearward was the name of a king; he had four women.  One was called Elfhild; their son was called Headen.  Another was called Searad; their son was called Humbling.  The third was called Sindred; their son was called Himbling.  King Hearward had made a vow to have those women whom he knew the most handsome.  He learned that king Swebner had a daughter fairest of all; she was called Syelind.  Ithmund was the name of his earl; Attle was his son, who journeyed to ask for Syelind’s hand on behalf of the king.  He stayed over the winter with king Swebner.  Frenmar was the name of an earl there, the foster-father of Syelin; his daughter was called Anlab.  The bird twittered, and Attle listened to what it said.  It quoth:\epb\epg


\bvg\bva\speakernote{Fugl kvað:}%
„\alst{S}átt-u \alst{S}igr·linn, \hld\ \alst{S}váfnis dóttur, &
męyna fęgrstu \hld\ ï \alst{m}unar-hęimi? &
Þó \alst{h}ag-ligar \hld\ \alst{H}jǫr·varðs konur &
\alst{g}umnum þykkja \hld\ at \alst{G}lasis-lundi.“\eva

\bvb “Hast thou seen Syelind Swebner’s daughter, \\
the fairest of maidens in the realm of love \ken{world}? \\
Although to mankind Hearward’s wives \\
seem handsome in Glazerslund.”\evb\evg


\bvg\bva\speakernote{Atli kvað:}%
„Munt við \alst{A}tla \hld\ \alst{I}ð·mundar son &
\alst{f}ugl \alst{f}róð-hugaðr \hld\ \alst{f}lęira mę́la?“ &
\speakernote{Fugl kvað:}%
„Mun’k ef mik \alst{b}uðlungr \hld\ \alst{b}lóta vildi &
ok \alst{k}ýs’k þat’s ek vil \hld\ ór \alst{k}onungs garði.“\eva

\bvb “Wilt thou with Attle Idmund’s son, \\
O wise-minded fowl, speak yet further?” \\
“I will, if the prince will make me a bloot, \\
and I may choose what I wish from the house of the king.”\evb\evg


\bvg\bva\speakernote{Atli kvað:}%
Kjós-at-tu \alst{H}jǫr·varð \hld\ né \alst{h}ans sonu &
né inar \alst{f}ǫgru \hld\ \alst{f}ylkis brúðir,
ęigi \alst{b}rúðir \hld\ þę́r’s \alst{b}uðlungr á;
kaupum \alst{v}el saman, \hld\ þat ’s \alst{v}ina kynni.\eva

\bvb TODO 3.\evb\evg


\bvg\bva\speakernote{Fugl kvað:}%
Hof mun’k kjósa, \hld\ hǫrga marga, &
gull-hyrndar kýr \hld\ frȧ grams búi, &
ef hǫ́num Sigr·linn \hld\ søfr ȧ armi &
ok ȯ·nauðig \hld\ jǫfri fylgir.\eva

\bvb TODO 4.\evb\evg


\bpg\bpa%
Þetta var áðr Atli fǿri. En er hann kom heim ok konungr spurði hann tíðinda, hann kvað:\epa

\bpb TODO.\epb\epg

\bvg\bva%
Hǫfum ęrfiði \hld\ ok ękki ørendi; &
mara þraut ȯra \hld\ ȧ męgin-fjalli, &
urðum síðan \hld\ Sę́morn vaða; &
þȧ vas oss synjat \hld\ Sváfnis dóttur, &
hringum gǿddrar, \hld\ es vér hafa vildum.\eva

\bvb TODO 5\evb\evg


\bpg\bpa%
Konungr bað að þeir skyldu fara annað sinn; fór hann sjálfr. En er þeir kómu upp á fjall ok sá á Svávaland landsbruna ok jóreyki stóra. Reið konungr af fjallinu fram í landið ok tók náttból við á eina. Atli helt vǫrð ok fór yfir ána. Hann fann eitt hús. Fugl mikill sat á húsinu ok gę́tti ok var sofnaðr. Atli skaut spjóti fuglinn til bana en í húsinu fann hann Sigrlinn konungs dóttur ok Álǫfu jarls dóttur ok hafði þę́r báðar braut með sér. Fránmarr jarl hafði hamazt í arnar líki ok varið þę́r fyr hernum með fjǫlkynngi. Hróðmarr hét konungr, biðill Sigrlinnar. Hann drap Svávakonung ok hafði rę́nt ok brennt landið. Hjǫrvarðr konungr fekk Sigrlinnar en Atli Álǫfar.\epa

\bpb TODO.\epb\epg


\bpg\bpa%
Hjǫrvarðr ok Sigrlinn áttu son mikinn ok vę́nan. Hann var þǫgull; ekki nafn festist við hann. Hann sat á haugi. Hann sá ríða val-kyrjur níu ok var ein gǫfug-ligust. Hon kvað:\epa

\bpb TODO.  She quoth:\epb\epg


\bvg\bva%
„Síð munt, \alst{H}elgi, \hld\ \alst{h}ringum ráða, &
\alst{r}íkr \alst{r}óg-apaldr, \hld\ né \alst{R}ǫðuls-vǫllum, &
\alst{ǫ}rn gól \alst{á}rla, \hld\ ef þú \alst{ę́} þęgir, &
þóttu \alst{h}arðan \alst{h}ug, \hld\ \alst{h}ilmir, gjaldir.“\eva

\bvb TODO 6\evb\evg


\bvg\bva\speakernote{[Hęlgi] kvað:}%
„Hvat lę́tr fylgja \hld\ Hęlga nafni, &
brúðr bjart-lituð, \hld\ alls bjóða rę́ðr? &
Hygg fyr ǫllum \hld\ at·kvę́ðum vel; &
þigg ęigi þat \hld\ nema þik hafa!“\eva

\bvb TODO 7\evb\evg


\bvg\bva\speakernote{[Val-kyrja] kvað:}%
„\alst{S}verð vęit’k liggja \hld\ ï \alst{S}igars-holmi, &
\alst{f}jórum \alst{f}ę́ra \hld\ enn \alst{f}imm tǫgu; &
\alst{ęi}tt es þęira \hld\ \alst{ǫ}llum bętra &
\alst{v}íg-nesta bǫl \hld\ ok \alst{v}arið gulli.\eva

\bvb “Swords I know lying in Sigarsholm: \\
four less than fifty. \\
One of them is better than all—\\
a \inx[C]{bale} of war-covers(?) \ken{shields}—and covered with gold.\evb\evg


\bvg\bva%
\edtrans{\alst{H}ringr ’s ï \alst{h}jalti}{A ring is on its hilt}{\Bfootnote{The sword is a ring-sword.  It was popular among Germanic warriors of the Migration Period to have oath-ring on their sword-hilts as a symbol of fidelity to their lords.  This custom was largely or entirely extinct by the Wiking Age, and the detail thus serves to emphasize the high age of the sword.  A well preserved Norwegian ring-sword survives from Snartemo in Vest-Agder,  dating to around 500 CE (object ID C26001); see Fig. \ref{fig:snartemo}.}}, \hld\ \alst{h}ugr ’s ï miðju, &
\alst{ó}gn ’s ï \alst{o}ddi, \hld\ þęim’s \alst{ęi}ga getr; &
liggr með \alst{ę}ggju \hld\ \alst{o}rmr dręyr-fáiðr &
en ȧ \edtrans{\alst{v}al-bǫstu}{walbast}{\Bfootnote{An unclear part of the sword-hilt; see \textlink{Sigrdrifumal} 6.}} \hld\ \alst{v}erpr naðr hala.“\eva

\bvb A ring is on its hilt; heart is in the middle; \\
terror is in the point for him who gets to own it. \\
Along the edge lies a serpent painted in blood \\
and on the walbast an adder eats its tail.”\evb\evg

\begin{figure}
\centering
\includegraphics[width=\textwidth]{Snartemo-hilt}
\caption{Hilt of the Snartemo sword, front and reverse.  Migration period, ca. 500 CE.  © Eirik Irgens Johnsen, \href{https://creativecommons.org/licenses/by-sa/4.0/deed.en}{CC BY-SA 4.0}.  \url{https://www.unimus.no/portal/\#/photos/d8932af5-1082-4938-9b4b-ca6b86f2bdfb}}
\label{fig:snartemo}
\end{figure}


\bpg\bpa%
Ey·limi hét konungr; dóttir hans var Sváva.  Hon var val-kyrja ok reið lopt ok lǫg.  Hon gaf Helga nafn þetta ok hlífði hǫ́num opt síðan í orrustum.  Helgi kvað:\epa

\bpb TODO.  Hallow quoth:\epb\epg


\bvg\bva%
Ert-at, Hjǫr·varðr, \hld\ hęil-ráðr konungr, &
fólks odd-viti, \hld\ þótt frę́gr séir; &
létst-u ęld eta \hld\ jǫfra byggðir &
en þeir angr við þik \hld\ ękki gørðu.\eva

\bvb TODO 10\evb\evg


\bvg\bva%
En Hróð·marr skal \hld\ hringum ráða, &
þeim es ǫ́ttu \hld\ ȯrir niðjar; &
sá sésk fylkir \hld\ fę́st at lífi, &
hyggsk al-dauðra \hld\ arfi at ráða.\eva

\bvb TODO 11\evb\evg


\bpg\bpa%
Hjǫr·varðr svaraði at hann myndi fá lið Helga ef hann vill hefna móður-fǫður síns. Þá sótti Helgi sverð’it er Sváva vísaði hánum til.  Þá fór hon ok Atli ok felldu Hróð·mar ok unnu mǫrg þrek-virki.  Hann drap Hata jǫtun er hann sat á bergi nokkuru.  Helgi ok Atli lǫ́gu skipum í Hata-firði.  Atli helt vǫrð inn fyrra hlut nę́tr’innar.  Hrím·gerðr Hata dóttir kvað:\epa

\bpb TODO.\epb\epg


\bvg\bva%
TODO.\eva

\bvb TODO 12\evb\evg


\bvg\bva%
TODO.\eva

\bvb TODO 13\evb\evg


\bvg\bva%
TODO.\eva

\bvb TODO 14\evb\evg


\bvg\bva%
TODO.\eva

\bvb TODO 15\evb\evg


\bvg\bva%
TODO.\eva

\bvb TODO 16\evb\evg


\bvg\bva%
TODO.\eva

\bvb TODO 17\evb\evg


\bvg\bva%
TODO.\eva

\bvb TODO 18\evb\evg


\bvg\bva%
TODO.\eva

\bvb TODO 19\evb\evg


\bvg\bva%
TODO.\eva

\bvb TODO 20\evb\evg


\bvg\bva%
TODO.\eva

\bvb TODO 21\evb\evg


\bvg\bva%
TODO.\eva

\bvb TODO 22\evb\evg


\bvg\bva%
TODO.\eva

\bvb TODO 23\evb\evg


\bvg\bva%
TODO.\eva

\bvb TODO 24\evb\evg


\bvg\bva%
TODO.\eva

\bvb TODO 25\evb\evg


\bvg\bva%
TODO.\eva

\bvb TODO 26\evb\evg


\bvg\bva%
TODO.\eva

\bvb TODO 27\evb\evg


\bvg\bva%
TODO.\eva

\bvb TODO 28\evb\evg


\bvg\bva%
TODO.\eva

\bvb TODO 29\evb\evg


\bvg\bva%
TODO.\eva

\bvb TODO 30\evb\evg


\bpg\bpa%
Helgi konungr var all-mikill her-maðr.  Hann kom til Ey·lima konungs ok bað Svǫ́vu, dóttur hans.  Þau Helgi ok Sváva veittusk várar ok unnusk furðu mikit.  Sváva var heima með feðr sínum en Helgi í hernaði.  Var Sváva val-kyrja enn sem fyrr.  Heðinn var heima með fǫður sínum, Hjǫr·varði konungi, í Noregi.  Heðinn fór einn saman heim ór skógi jóla-aptan ok fann troll-konu; sú reið vargi ok hafði orma at taumum ok bauð fylgð sína Heðni.  „Nei,“ sagði hann.  Hon sagði: „Þess skaltu gjalda at bragar-fulli!“ Um kveldit óru heit-strengingar; var framm leiddr sonar-gǫltr; lǫgðu menn þar á hendr sínar ok strengdu menn þá heit at bragar-fulli.  Heðinn strengði heit til Svǫ́vu Ey·lima dóttur, unnustu Helga, bróður síns, ok iðraðisk svá mjǫk at hann gekk á braut villi-stígu suðr á lǫnd ok fann Helga bróður sinn.  Helgi kvað:\epa

\bpb TODO.\epb\epg


\bvg\bva%
TODO.\eva

\bvb TODO 31\evb\evg


\bvg\bva%
TODO.\eva

\bvb TODO 32\evb\evg


\bvg\bva%
TODO.\eva

\bvb TODO 33\evb\evg


\bvg\bva%
TODO.\eva

\bvb TODO 34\evb\evg


\bpg\bpa%
Þat kvað Helgi því at hann grunaði um feigð sína ok það at fylgjur hans hǫfðu vitjat Heðins þá er hann sá konu’na ríða varginum.  Álfr hét konungr, sonr Hróð·mars, er Helga hafði vǫll haslaðan á Sigars-velli á þriggja nátta fresti.  Þá kvað Helgi:\epa

\bpb TODO.\epb\epg


\bvg\bva%
TODO.\eva

\bvb TODO 35\evb\evg


\bpg\bpa%
Þar var orrusta mikil ok fekk þar Helgi bana-sár.\epa

\bpb That was a great battle, and there Hallow got his mortal wound.\epb\epg


\bvg\bva%
TODO.\eva

\bvb TODO 36\evb\evg


\bvg\bva%
TODO.\eva

\bvb TODO 37\evb\evg


\bvg\bva%
TODO.\eva

\bvb TODO 38\evb\evg


\bvg\bva%
TODO.\eva

\bvb TODO 39\evb\evg


\bvg\bva%
TODO.\eva

\bvb TODO 40\evb\evg


\bvg\bva%
TODO.\eva

\bvb TODO 41\evb\evg


\bvg\bva%
TODO.\eva

\bvb TODO 42\evb\evg


\bvg\bva%
Kyss mik, Sváva, \hld\ køm’k ęigi áðr &
Rog-hęims ȧ vit \hld\ né Rǫðuls-fjalla &
áðr hęfnt hęfi’k \hld\ Hjǫr·varðs sonar, &
þęss es buðlungr vas \hld\ bęztr und sólu!\eva

\bvb TODO 43\evb\evg


\bpg\bpa%
Helgi ok Sváfa er sagt at vę́ri endr-borin.\epa

\bpb Hallow and Sweve, it is said, were reborn.\epb\epg

\sectionline
