Svá sęgja menn í fornum sǫgum, at ęinnhvęrr af ǫ́sum, sá es Heimdallr hét, fór fęrðar sinnar ok framm með sjóvarstrǫndu nǫkkurri, kom at ęinum húsabǿ ok nefndisk Rígr; ęptir þęiri sǫgu es kvæði þetta.

Thus men say in ancient saws†, that one of the Ease†, he who was called Homedall, went on his journey and forth along some lake-shore, came upon a lone homestead and called himself Righ. After that saw is this poem:

Ár kvǫ́ðu ganga \hld grǿnar brautir
ǫflgan ok aldinn \hld ǫ́s kunnigan,
ramman ok rǫskvan \hld Ríg stíganda. 

Of yore they said [did] walk the green paths, a powerful and aged os†, cunning; the strong and quick Righ, striding.

Gekk hann męir at þat \hld miðrar brautar,
kom hann at húsi, \hld hurð var á gætti;
inn nam at ganga, \hld eldr var á golfi,
hjón sǫ́tu þar \hld hǫ́r at arni,
Ái ok Edda \hld aldinfalda. 

Went he further at that, on the middle of the road; came he to a house, the door was wide open. H e began to walk inside, fire was on the floor. A couple sat there, hoary by the hearth: Great Grandfather and Great Grandmother, old-fashioned.
