\bookStart{Speeches of Shirner}[Skírnismǫ́l]

\begin{flushright}%
\textbf{Dating} \parencite{Sapp2022}: C10th (0.897)

\textbf{Meter:} \Ljodahattr, \Galdralag\ (TODO)%
\end{flushright}

\section{Introduction}

The whole poem is attested in both \Regius\ and \AM. The name \emph{Skírnismǫ́l} ‘\textbf{Speeches of Shirner}’ comes from \AM; \Regius\ has in the typical titular red ink \emph{Fǫr Skírnis} ‘Shirner’s journey’.

The same myth is told in prose in \Gylfaginning\ 37.  A single stanza of the present poem is quoted there, namely the last one, with some minor differences in wording that would seem to stem from oral tradition (see Note to st. 42 below).  It is unlikely that the author of \Gylfaginning\ knew of the narrative through an oral tradition which included only the last verse, chiefly since his paraphrase does not add a single detail not found in the present poem, but on the other hand condenses and abbreviates.  So, Shirner’s journey and curse (roughly sts. 10–38 here) is simply summarized in the following manner: “Then Shirner journeyed and requested the woman [i.e. Gird] for him [i.e. Free], and received her promise, that nine nights later she would come to the place which is called Barrey, and have a wedding with Free.”  The summarising of a narrative mythic poem with a single verse quotation in the form of a dialogue-stanza is something done several times in \Gylfaginning; see Eddic fragments from Snorre’s Edda below.

On the other hand, the paragraph in \Gylfaginning\ 37 corresponding to what is here P1 is much more detailed and reads: “Gymer was a man called, and his woman Earbode; she was of the lineage of mountain-risers. Their daughter is Gird, who is fairest of all women.  It was one day when Free had gone to Lithshelf and looked about all the Homes, but when he looked to the north he saw on a farm a great and fine house, and to that house walked a woman, and when she lifted her hands and closed the doors before her, then it did shine from her hands both into the air and onto the waters, and all the homes were brightened by her.  And that beauty, which he had seen in that holy seat, harmed him so that he walked away filled with pain, and when he came home he spoke nothing; he neither slept nor drank; nobody dared to get words out of him.  Then Nearth had Shirner, Free’s shoe-swain, called unto himself, and asked him to go to Free and ask him to speak, [...]”

% It seems to me that this circumstance, where the part corresponding to the poem is a short paraphrase, but the part corresponding to the prose passage is much more detailed, can only have arisen if the former already had a fixed form, whereas the latter was freer and could vary with each retelling. For this, see further TODO.

\sectionline

\section{The Speeches of Shirner}

\bpg\bpa\mssnote{\Regius~11r/10, \AM~2r/11}%
Freyr, sonr Njarðar, hafði einn dag setsk í Hlið-skjálf ok sá um heima alla; hann sá í Jǫtun-heima ok sá þar mey fagra, þá er hon gekk frá skála fǫður síns til skemmu; þar af fekk hann hug-sóttir miklar. Skírnir hét skó-sveinn Freys. Njǫrðr bað hann kveðja Frey máls. Þá mę́lti Skaði:\epa

\bpb \inx[P]{Free}, son of \inx[P]{Nearth}, had one day set himself in \inx[L]{Lithshelf} and looked about all the \inx[C]{Homes}.  He looked into the \inx[L]{Ettinhomes} and saw there a fair maiden as she walked from her father’s hall to her bower; thereof he got great heart-aches.  \inx[P]{Shirner} was called the shoe-swain of Free.  Nearth asked him to speak with Free.  Then \inx[P]{Shede} spoke:\epb\epg


\bvg\bva\mssnote{\Regius~11r/14, \AM~2r/15}%
„\edtext{Rís-tu nú Skírnir \hld\ ok gakk at bęiða}{\lemma{rís \dots\ bęiða ‘Rise \dots\ ask’}\Bfootnote{Alliteration is missing here. A simple solution would be to replace \emph{gakk} ‘go’ with a synonym like \emph{rinn} ‘run’ or \emph{ráð} ‘resolve’, but this lessens the semantic mirroring with l. 2/2 below (though, the insertion of the verb \emph{ganga} in the present stanza may in fact be due to influence from 2/2).}} &
\ind okkarn \alst{m}ála \alst{m}ǫg, &
ok þess at \alst{f}regna \hld\ hvęim hinn \alst{f}róði séi &
\ind \alst{o}f-ręiði \edtrans{\alst{a}fi}{man}{\Bfootnote{While this word usually means “father” or “grandfather”, it should here mean “man” without a connotation of old age. See further \CV.}}.“\eva

\bvb “Rise thou now, Shirner, and go to ask \\
\ind our lad for speech; \\
and to learn at whom the wise \\
\ind man might be cross.”\evb\evg


\bvg\bva\speakernote{Skírnir kvað:}\mssnote{\Regius~11r/15, \AM~2r/17}%
„\alst{I}llra \alst{o}rða \hld\ es mér \alst{ó}n at ykkrum syni, &
\ind ef ek gęng at \alst{m}ę́la við \alst{m}ǫg, &
ok þess at \alst{f}regna, \hld\ hvęim hinn \alst{f}róði séi &
\ind \alst{o}f-ręiði \alst{a}fi.“\eva

\bvb\speakernoteb{Shirner quoth:}%
“Bad words I expect from your son,  \\
\ind if I go to speak with the lad, \\
and to learn at whom the wise \\
\ind man might be cross.”\evb\evg

\sectionline

\bvg\bva\speakernote{Skírnir:}\mssnote{\Regius~11r/17, \AM~2r/18}%
„Sęg þat \alst{F}ręyr, \hld\ \alst{f}olk-valdi goða, &
\ind ok ek \alst{v}ilja \alst{v}ita, &
hví þú \alst{ęi}nn sitr \hld\ \alst{ę}nd-langa sali, &
\ind minn \alst{d}róttinn, of \alst{d}aga?“\eva

\bvb\speakernoteb{Shirner [quoth]:}%
“Tell it, O Free, troop-wielder of the gods— \\
\ind I too would wish to know, \\
why thou sittest alone in the endlong halls, \\
\ind my lord, during the days.”\evb\evg


\bvg\bva\speakernote{Fręyr:}\mssnote{\Regius~11r/19, \AM~2r/20}%
„Hví of \alst{s}ęgja’k þér, \hld\ \alst{s}ęggr hinn ungi, &
\ind \alst{m}ikinn \alst{m}óð-trega? &
því-at \edtrans{\alst{a}lf-rǫðull}{elf-wheel}{\Bfootnote{A rare poetic synonym (\emph{hęiti}) for the sun; see note to \Vafthrudnismal\ 47/1.}} \hld\ lýsir of \alst{a}lla daga &
\ind ok þęygi at \alst{m}ínum \alst{m}unum.“\eva

\bvb\speakernoteb{Free [quoth]:}%
“Why should I tell thee, O young youth, \\
\ind my great heartache? \\
For the elf-wheel \name{= Sun} shines during all days, \\
\ind and nowise to my liking.”\evb\evg


\bvg\bva\speakernote{Skírnir:}\mssnote{\Regius~11r/20, \AM~2r/21}%
„\alst{M}uni þína \hld\ hykk-a svá \alst{m}ikla vesa, &
\ind at þú mér \edtrans{\alst{s}ęggr}{youth}{\Bfootnote{This word usually means simply ‘man’, but it seems to have a specific connotation with youth. Its original meaning is ‘messenger’, and the semantic shift is thus: ‘messenger’ > ‘young man’ > ‘warrior/man’. The sense of ‘young man’ is also seen in \Volundarkvida\ 23, where it is used in reference to king Nithad’s two young sons. In the present stanza it answers Free’s addressing Shirner as \emph{sęggr hinn ungi} ‘the young youth’; Shirner points out that the two are of equal age, and so Free is as much of a young man as he.}} né \alst{s}ęgir; &
\alst{u}ngir saman \hld\ vǫ́rum í \alst{á}r-daga, &
\ind vęl mę́ttim \alst{t}vęir \alst{t}rúask.“\eva

\bvb\speakernoteb{Shirner [quoth]:}%
“Thy liking I do not think so great, \\
\ind that thou, O youth, should not tell me. \\
Young together were we in days of yore; \\
\ind we two might well trust each other.”\evb\evg


\bvg\bva\speakernote{Fręyr:}\mssnote{\Regius~11r/22, \AM~2r/23}%
„Í \alst{G}ymis gǫrðum \hld\ ek \alst{g}anga sá &
\ind \alst{m}ér tíða \alst{m}ęy; &
\alst{a}rmar lýstu, \hld\ en \alst{a}f þaðan &
\ind allt \edtrans{\alst{l}opt ok \alst{l}ǫgr}{air and sea}{\Bfootnote{Formulaic and very old, also paralleled in the Anglo-Saxon. TODO.}}.\eva

\bvb\speakernoteb{Free [quoth]:}%
“In Gymer’s yards I saw walking \\
\ind a maiden, dear to me. \\
Her arms shone and thereof \\
\ind all the air and sea.\evb\evg


\bvg\bva\mssnote{\Regius~11r/24, \AM~2r/24}%
\alst{M}ę́r ’s mér tíðari \hld\ an \alst{m}anna hvęim &
\ind \alst{u}ngum í \alst{á}r-daga; &
\alst{á}sa ok \alst{a}lfa \hld\ þat vill \alst{ę}ngi maðr, &
\ind at vit \alst{s}átt \alst{s}éim.“\eva

\bvb The maiden is dearer to me than to any man \\
\ind young in days of yore. \\
Of the \inx[F]{Eese and Elves} does no man\footnoteB{i.e. ‘person’. For other examples of gods being called men see note to final st. of \Vafthrudnismal\ 55.} wish \\
\ind that we two should be brought together.”\evb\evg


\bvg\bva\speakernote{Skírnir:}\mssnote{\Regius~11r/25, \AM~2r/25}%
„\alst{M}ar gef mér þá, \hld\ es mik of \alst{m}yrkvan beri &
\ind \alst{v}ísan \alst{v}afr-loga, &
ok þat \alst{s}verð, \hld\ es \alst{s}jalft vegisk &
\ind við \alst{jǫ}tna \alst{ę́}tt.“\eva

\bvb\speakernoteb{Shirner [quoth]:}%
“The steed then give me, which might bear me over the dark, \\
\ind wise wavering-flame; \\
and that sword, which by itself might strike \\
\ind against the line of the \inx[G]{Ettins}.”\evb\evg


\bvg\bva\speakernote{Fręyr:}\mssnote{\Regius~11r/27, \AM~2r/27}%
„\alst{M}ar þér þann gef’k, \hld\ es þik of \alst{m}yrkvan \edtext{berr &
\ind \alst{v}ísan \alst{v}afr-loga, &
auk þat \alst{s}verð, \hld\ es \alst{s}jalft mun vegask, &
\ind ef sá ’s \alst{h}orskr es \alst{h}ęfr.“}{\lemma{berr ‘bears’; mun vegask, ef sá ’s horskr es hęfr ‘will strike, if he is wise who owns it’}\Bfootnote{In his response Free replaces the subjunctive verb forms (\emph{beri} ‘might bear’, \emph{vegisk} ‘might strike’) with indicative and future forms, giving a sense of certainity and authority. The steed and sword are faultless, and if Shirner fails on the mission, it would be only due to his own fault (“if he is sharp who owns it.”).}}\eva
%TODO? Change the line numbering from 1–4 to 1, 3–4.

\bvb\speakernoteb{Free [quoth]:}%
“That steed I give thee, which bears thee over the dark, \\
\ind wise wavering-flame; \\
and that sword which by itself will strike, \\
\ind if he is wise who owns it.”\evb\evg


\bpg\bpa Skírnir mę́lti við hest’inn:\epa
\bpb Shirner spoke with the horse:\epb\epg


\bvg\bva\mssnote{\Regius~11r/29, \AM~2r/28}%
„\alst{M}yrkt es úti, \hld\ \alst{m}ál kveð’k okkr fara &
\ind \alst{ú}rig fjǫll \alst{y}fir &
\ind \edtrans{\alst{þ}ursa}{of the Thurses}{\Afootnote{so \AM; \emph{þyria} \Regius}} \alst{þ}jóð yfir; &
\alst{b}áðir vit komumk \hld\ eða okkr \alst{b}áða tękr &
\ind sá hinn \edtrans{\alst{á}m-átki \alst{jǫ}tunn}{uncanny ettin}{\Bfootnote{Formulaic. See note to \Voluspa\ 8.}}.“\eva

\bvb “’Tis dark outside; I declare it time for us to journey \\
\ind over the drizzling mountains, \\
\ind over the tribe of \inx[G]{Thurses}. \\
We will both come, or us both does take \\
\ind that uncanny ettin.\footnoteB{Shirner declares his intention not to abandon the horse given to him by his lord; they will either both make it, or both perish.}”\evb\evg


\bpg
\bpa\mssnote{\Regius~11r/31, \AM~2v/1}%
Skírnir reið i Jǫtun-heima til Gymis garða; þar váru hundar ólmir ok bundnir fyrir skíð-garðs hliði þess, er um sal Gerðar var. Hann reið at þar, er fé-hirðir sat á haugi, ok kvaddi hann: \epa

\bpb Shirner rode into the Ettinhomes, to Gymer’s yards. There were fierce hounds bound in front of the slope of the wooden fence which surrounded Gird’s\footnoteB{It is first now that we are informed of the maiden’s name.} hall. He rode to where a shepherd sat on a mound, and greeted him:\epb\epg


\bvg\bva\mssnote{\Regius~11v/2, \AM~2v/4}%
„Sęg þat \alst{h}irðir, \hld\ es á \alst{h}augi sitr &
\ind ok \alst{v}arðar alla \alst{v}ega: &
hvé ek at \alst{a}nd-spilli \hld\ komumk hins \alst{u}nga mans &
\ind fyr \alst{g}ręyjum \alst{G}ymis.“\eva

\bvb “Tell this, O herdsman, who on the mound sittest, \\
\ind and watchest all the ways, \\
how I to discourse might come with the young girl \ken*{= Gird}, \\
\ind past the greyhounds of Gymer.”\evb\evg


\bvg\bva\speakernote{[Hirðir] kvað:}\mssnote{\Regius~11v/4, \AM~2v/5}%
„Hvárt est \alst{f}ęigr, \hld\ eða est \alst{f}ramm ginginn &
\ind [...]; &
\alst{a}nd-spillis vanr \hld\ þú skalt \alst{ę́} vesa &
\ind \edtrans{\alst{g}óðrar męyjar}{good maiden}{\Bfootnote{Formulaic, carrying with it a sense of chastity.  See note to \Havamal\ 102/1 for further occurrences.}} \alst{G}ymis.“\eva

\bvb\speakernoteb{[The herdsman] quoth:}%
“Either art thou fey, or gone forth [dead]; \\
\ind {[...]}. \\
Discourse-less shalt thou always be, \\
\ind with the good maiden of Gymer \ken*{= Gird}.”\evb\evg


\bvg\bva\speakernote{[Skírnir] kvað:}\mssnote{\Regius~11v/6, \AM~2v/7}
„\edtrans{\alst{K}ostir}{Choices}{\Bfootnote{i.e. ‘alternatives, other ways’.}} ’ru bętri \hld\ \edtrans{an}{than}{\Afootnote{so \AM; \emph{hęldr an at} ‘rather than to [be]’ \Regius}} \alst{k}løkkva séi &
\ind hvęim es \alst{f}úss es \alst{f}ara, &
\alst{ęi}nu dǿgri \hld\ mér vas \alst{a}ldr of skapaðr &
\ind ok alt \alst{l}íf of \alst{l}agit.“\eva

\bvb\speakernoteb{[Shirner] quoth:}%
“Choices are better than sobbing might be \\
\ind for whomever is eager to journey. \\
In one half-day my age was shaped, \\
\ind and all my life laid down.\footnoteB{An excellent example of the fatalistic Germanic worldview, in which one’s course of life was determined (“laid down”) at birth (“in one half-day”).  Presumably after uttering these words Shirner rides through the fire surrounding the fortress. — The causative \emph{lęgja} ‘to lay (down, in place)’ is closely connected to fate; the expression is formulaic.  Cf. \Lokasenna\ 48: \emph{í ár-daga vas þér hit ljóta líf of lagit} ‘in days of yore was thy ugly life laid down’ and \Voluspa\ 19: \emph{þę́r lǫg lǫgðu} ‘they [= the Norns] laid down laws’.}”\evb\evg


\bvg\bva\speakernote{[Gęrðr] kvað:}\mssnote{\Regius~11v/7, \AM~2v/8}%
„Hvat ’s þat \alst{h}lym \alst{h}lymja \hld\ es \alst{h}lymja hęyri’k nú til &
\ind \alst{o}ssum rǫnnum \alst{í}? &
\alst{jǫ}rð bifask, \hld\ en \alst{a}llir fyr &
\ind skjalfa \alst{g}arðar \alst{G}ymis.“\eva

\bvb\speakernoteb{[Gird] quoth:}%
“What is that din of dins, which I of dins now hear \\
\ind in our halls? \\
The earth quakes, and before me tremble \\
\ind all Gymer’s yards.”\evb\evg


\bvg\bva\speakernote{Ambǫ́tt kvað:}\mssnote{\Regius~11v/9, \AM~2v/10}%
„\alst{M}aðr ’s hér úti, \hld\ stiginn af \alst{m}ars baki, &
\ind \alst{jó} lę́tr til \alst{ja}rðar taka.“\eva

\bvb\speakernoteb{A servant-woman quoth:}%
“A man is here outside, stepped down off horseback; \\
\ind he lets take his steed to the ground.\footnoteB{“He lets his horse graze.” According to \textcite{FinnurEdda} an Icelandic expression still known in his time.}”\evb\evg


\bvg\bva\speakernote{[Gęrðr] kvað:}\mssnote{\Regius~11v/10, \AM~2v/11}%
„\alst{I}nn bið þú hann ganga \hld\ í \alst{o}kkarn sal &
\ind ok drekka hinn \alst{m}ę́ra \alst{m}jǫð, &
þó ek hitt \alst{ó}umk, \hld\ at hér \alst{ú}ti séi &
\ind minn \alst{b}róður-\alst{b}ani.“\eva

\bvb\speakernoteb{[Gird] quoth:}%
“Bid thou him to go in into our hall, \\
\ind and to drink the renowned mead; \\
though I fear that here outside should be  \\
\ind my brother’s bane.”\evb\evg

\sectionline

\bvg\bva\speakernote{[Gęrðr] kvað:}\mssnote{\Regius~11v/12, \AM~2v/13}%
„Hvat ’s þat \alst{a}lfa \hld\ né \alst{á}sa sona, &
\ind né \alst{v}íssa \alst{v}ana; &
hví \alst{ęi}nn of komt \hld\ \alst{ęi}kinn fúr yfir &
\ind ór \alst{s}al-kynni at \alst{s}éa?“\eva

\bvb\speakernoteb{[Gird quoth:]}%
“What kind is that, not of Elves, nor of sons of the Eese, \\
\ind nor of wise Wanes? \\
Why camest thou alone over the raging fire, \\
\ind to see the state of our hall?”\evb\evg


\bvg\bva\speakernote{[Skírnir kvað:]}\mssnote{\Regius~11v/14}%
„\alst{E}m’k-at \alst{a}lfa \hld\ né \alst{á}sa sona &
\ind né \alst{v}íssa \alst{v}ana, &
þó \alst{ęi}nn of kom’k \hld\ \alst{ęi}kinn fúr yfir &
\ind yður \alst{s}al-kynni at \alst{s}éa.\eva

\bvb\speakernoteb{[Shirner quoth:]}%
“I am not of Elves, nor of sons of the Eese, \\
\ind nor of wise Wanes— \\
yet I came alone over the raging fire, \\
\ind to see the state of your hall.\evb\evg


\bvg\bva\mssnote{\Regius~11v/15, \AM~2v/14}%
\alst{Ę}pli \alst{ę}llifu \hld\ hér hef’k \alst{a}l-gullin, &
\ind þau mun’k þér \alst{G}ęrðr \alst{g}efa, &
\alst{f}rið at kaupa, \hld\ at þú þér \alst{F}ręy kveðir &
\ind ó·\alst{l}ęiðastan at \alst{l}ifa.“\eva

\bvb Eleven apples have I here, all-golden; \\
\ind those I will to thee, O Gird, give \\
to buy [thy] love, that thou callest Free for thee \\
\ind most unloathsome [lovely] in life.\footnoteB{\emph{at lifa} here means seems to mean ‘in life/living’ rather than the typical infinitive sense ‘to live’; cf. st. 22 \emph{at dęila} ‘in sharing’ below. This is possibly an archaism.}”\evb\evg


\bvg\bva\speakernote{[Gęrðr] kvað:}\mssnote{\Regius~11v/17, \AM~2v/15}%
„\alst{Ę}pli \alst{ę}llifu \hld\ ek þigg \alst{a}ldri-gi &
\ind at \alst{m}anns-kis \alst{m}unum, &
né vit \alst{F}ręyr, \hld\ meðan okkart \alst{f}jǫr lifir, &
\ind \alst{b}yggum \alst{b}ę́ði saman.“\eva

\bvb\speakernoteb{[Gird quoth:]}%
“Eleven apples will I never take, \\
\ind to any man’s liking; \\
nor will I and Free while our life remains \\
\ind dwell both together.”\evb\evg


\bvg\bva\speakernote{[Skírnir kvað:]}\mssnote{\Regius~11v/19, \AM~2v/17 (ll. 1–2)}%
„\alst{B}aug þér þá gef’k, \hld\ þann’s \alst{b}ręndr of vas &
\ind með \alst{u}ngum \alst{Ó}ðins syni; &
\edtext{\alst{á}tta ’ru \alst{ja}fn-hǫfgir, \hld\ es \alst{a}f drjúpa &
\ind hina \alst{n}íundu hvęrja \alst{n}ǫ́tt.“}{\lemma{átta ... nǫ́tt ‘Eight ... night.’}\Bfootnote{In \AM\ these lines and 22:1–2 are missing.  Instead 1–2 here and 22:3–4 are combined into one.}}\eva

\bvb\speakernoteb{[Shirner quoth:]}%
“The \inx[C]{bigh} I then give thee, which was burned \\
\ind with Weden’s young son \ken*{= Balder}. \\
Eight are even-heavy, which from it drip, \\
\ind every ninth night.\footnoteB{The bigh, while not named, is clearly Dreepner as known from \Gylfaginning\ 49, describing Balder’s funeral: “Weden laid on the pyre that gold ring which is called Dreepner. Its nature was such that every ninth night, eight even-heavy golden rings dripped from it.” When \inx[P]{Harmod} later comes to \inx[L]{Hell} to try to bring Balder back, Balder tells him to bring the ring back to Weden, as a token of memory.}”\evb\evg


\bvg\bva\speakernote{[Gęrðr] kvað:}\mssnote{\Regius~11v/21, \AM~2v/18 (ll. 3–4)}%
„\alst{B}aug þikk-a’k, \hld\ þótt \alst{b}ręndr séi, &
\ind með \alst{u}ngum \alst{Ó}ðins syni; &
es-a mér \alst{g}ulls vant \hld\ í \alst{g}ǫrðum \alst{G}ymis &
\ind at dęila \alst{f}é \alst{f}ǫður.“\eva

\bvb\speakernoteb{[Gird quoth:]}%
“The bigh I take not, though it may have been burned \\
\ind with Weden’s young son \ken*{= Balder}; \\
I’m not wanting gold in Gymer’s yards, \\
\ind in sharing the \inx[C]{fee} of my father.”\evb\evg


\bvg\bva\speakernote{[Skírnir kvað:]}\mssnote{\Regius~11v/23, \AM~2v/19}%
„Sér þú \alst{m}ę́ki, \alst{m}ę́r, \hld\ \alst{m}jóvan, \edtrans{\alst{m}ál-fáan}{picture-painted}{\Bfootnote{The sword is inlaid with metal (perhaps gold or silver) forming a pattern.  The expression is formulaic; cf. TODO.}}, &
\ind es \alst{h}ęf’k í \alst{h}ęndi \alst{h}ér? &
\alst{h}ǫfuð \alst{h}ǫggva \hld\ mun’k þér \alst{h}alsi af, &
\ind nema mér \alst{s}ę́tt \alst{s}ęgir.“\eva

\bvb\speakernoteb{[Shirner quoth:]}%
“Seest thou this sword, maiden—slender, pictured-painted—, \\
\ind which I have in my hand here? \\
Strike the head will I from thy neck, \\
\ind unless thou come to terms with me.”\evb\evg


\bvg\bva\speakernote{[Gęrðr kvað:]}\mssnote{\Regius~11v/25, \AM~2v/20}%
„\alst{Á}-nauð þola \hld\ vil’k \alst{a}ldri-gi &
\ind at \edtrans{\alst{m}anns-kis}{any man’s (lit. ‘no man’s)}{\Afootnote{\emph{manns ęnskis} \AM}} \alst{m}unum, &
þó hins \alst{g}et’k, \hld\ ef it \alst{G}ymir finniðsk &
\alst{v}ígs ó·trauðir \hld\ at ykkr \alst{v}ega tíði.“\eva

\bvb\speakernoteb{[Gird quoth:]}%
“Stand coercion will I never, \\
\ind to any man’s liking; \\
though I get this, if thou and Gymer meet— \\
men unreluctant of conflict—that ye two will come to fight.\footnoteB{Gird says that she will never let herself be forced to marry Free, even if that means that her father and Shirner should fight over her.}”\evb\evg


\bvg\bva\speakernote{[Skírnir kvað:]}\mssnote{\Regius~11v/27, \AM~2v/22}%
„Sér þú \alst{m}ę́ki, \alst{m}ę́r, \hld\ \alst{m}jóvan, \alst{m}ál-fáan, &
\ind es \alst{h}ęf’k í \alst{h}ęndi \alst{h}ér? &
fyr þessum \alst{ę}ggjum \hld\ hnígr sá hinn \alst{a}ldni jǫtunn, &
\ind verðr þinn \alst{f}ęigr \alst{f}aðir.\eva

\bvb\speakernoteb{[Shirner quoth:]}%
“Seest thou this sword, maiden—slender, pictured-painted—, \\
\ind which I have in my hand here? \\
By these edges sinks the aged ettin \ken*{= Gymer} down; \\
\ind \inx[C]{fey} becomes thy father.\evb\evg


\bvg\bva\mssnote{\Regius~11v/28, \AM~2v/24}%
\edtrans{\alst{T}ams-vęndi}{taming-wand}{\Bfootnote{Has been interpreted as a sword, TODO.}} þik drep’k, \hld\ ęn þik \alst{t}ęmja mun’k, &
\ind \alst{m}ę́r, at mínum \alst{m}unum, &
þar skalt \alst{g}anga \hld\ es þik \alst{g}umna synir &
\ind \alst{s}íðan ę́va \alst{s}éi.\eva

\bvb With the taming-wand I strike thee—and thee I will tame, \\
\ind O maiden, to my liking! \\
Thou shalt go where the sons of men \\
\ind never since may see thee!\evb\evg


\bvg\bva\mssnote{\Regius~11v/30, \AM~2v/26}%
\edtrans{\alst{A}ra þúfu \alst{á} \hld\ skalt \alst{á}r sitja}{On an eagle’s perch shalt thou sit at dawn}{\Afootnote{\emph{ár skalt sitja \hld\ ara þúfu á} ‘at dawn shalt thou sit on an eagle’s perch’ \AM}}, &
\ind \edtext{\alst{h}orfa \alst{h}ęimi ór; &
\ind snugga \alst{h}ęljar til}{\lemma{horfa hęimi ór; snugga hęljar til ‘turn out of the world; hanker after Hell’}\Afootnote{\emph{horfa ok snugga hęljar til} ‘turn and hanker after Hell’ \AM}\Bfootnote{i.e. “you will look toward and yearn for the underworld”.}}; &
\alst{m}atr sé þér męir lęiðr \hld\ an \alst{m}anna hvęim &
\ind hinn \alst{f}ráni ormr með \edtext{\alst{f}irum}{\Bfootnote{This is the last word of fol. 2v of \AM, after which the text cuts off.}}.\eva

\bvb On an eagle’s perch shalt thou sit at dawn; \\
\ind turn away from the world, \\
\ind hanker after \inx[L]{Hell}! \\
Let thy food be more loathsome than to any man \\
\ind the gleaming serpent \ken*{= the Middenyardswyrm} among the folk.\footnoteB{Her food will be more disgusting than the \inx[C]{Middenyardswyrm}, for which cf. \Hymiskvida\ 22.}\evb\evg


\bvg\bva\mssnote{\Regius~11v/32}%
At \alst{u}ndr-sjónum verðir \hld\ es \alst{ú}t of kømr, &
\ind á þik \alst{H}rímnir \alst{h}ari &
\ind á þik \alst{h}ot-vetna stari, &
\alst{v}íð-kunnari \alst{v}erðir \hld\ an \alst{v}ǫrðr með goðum, &
\ind \alst{g}api þú \alst{g}rindum frá.\eva

\bvb A wondrous sight be thou when thou comest out; \\
\ind at thee let Rimner ogle; \\
\ind at thee let anyone stare! \\
Be thou more widely known than the Watchman among the Gods \ken*{= Homedal}; \\
\ind may thou gape from the gates!\evb\evg


\bvg\bva\mssnote{\Regius~12r/2}%
\edtrans{\alst{T}ópi ok ópi, \hld\ \alst{t}jǫsull ok ó·þoli}{Toop and woop, tarsle and restlessness}{\Bfootnote{The first three words are magic curse words without clear meaning; I have left them untranslated.  \emph{tjǫsull} may perhaps be related to OE \emph{teors} ‘penis’ and mean ‘little phallus’.}}, &
\ind vaxi þér \alst{t}ǫ́r með \alst{t}rega; &
\alst{s}ętsk þú niðr \hld\ en mun’k \alst{s}ęgja þér &
\ind \alst{s}váran \alst{s}ús-breka, &
\ind ok \alst{t}vinnan \alst{t}rega.\eva

\bvb Toop and woop, tarsle and restlessness— \\
\ind may thy tears grow with grief! \\
Sit thyself down, and I will tell thee \\
\ind a heavy roaring-breaker, \\
\ind and a twined grief.\evb\evg


\bvg\bva\mssnote{\Regius~12r/3}%
Tramar \alst{g}nęypa \hld\ þik skulu \alst{g}ęrstan dag &
\ind \alst{jǫ}tna gǫrðum \alst{í}, &
til \alst{h}rím-þursa \alst{h}allar \hld\ þú skalt \alst{h}vęrjan dag &
\ind \alst{k}ranga \alst{k}osta-laus; &
\ind \alst{k}ranga \alst{k}osta-vǫn; &
\alst{g}rát at \alst{g}amni \hld\ skalt í \alst{g}ǫgn hafa &
\ind ok lęiða með \alst{t}ǫ́rum \alst{t}rega.\eva

\bvb Fiends shall pine thee on a gloomy day, \\
\ind in the yards of the Ettins. \\
To the hall of Rime-Thurses shalt thou every day \\
\ind crawl choice-less; \\
\ind crawl choice-lacking. \\
Weeping for joy shalt thou have in exchange, \\
\ind and nurse grief with tears.\evb\evg


\bvg\bva\mssnote{\Regius~12r/7}%
Með \edtrans{\alst{þ}ursi \alst{þ}rí-hǫfðuðum}{three-headed thurse}{\Bfootnote{Ettins often have an abnormal number of body parts.  For their “manyheadedness” see note to \Hymiskvida\ 8/2.}} \hld\ \alst{þ}ú skalt ę́ nara &
\ind eða \alst{v}er-laus \alst{v}esa, &
\ind þitt \alst{g}ęð \alst{g}rípi; &
\ind þik \alst{m}orn \alst{m}orni &
ves þú sem \alst{þ}istill, \hld\ sá’s \alst{þ}runginn vas &
\ind í \alst{o}fan-verða \alst{ǫ́}nn.\eva

\bvb With a three-headed thurse shalt thou always live, \\
\ind or be husband-less. \\
\ind May thy senses seize; \\
\ind may murrain mourn thee; \\
be thou like the thistle that was pressed \\
\ind during highest harvest!\evb\evg


\bvg\bva\mssnote{\Regius~12r/9}%
Til \alst{h}olts ek gekk \hld\ ok til \alst{h}rás viðar &
\ind \edtrans{\alst{g}amban-tęin}{gombentoe}{\Bfootnote{Perhaps “curse-twig”.  A compound consisting of the very rare word \emph{gamban} ‘magic/curse?’ and \emph{tęinn} ‘twig, branch’ (cf. \emph{mistil-tęinn} ‘mistle-toe’).  This may be the stick on which the runic curse in st. 36 below should be carved, or it is to be identified with the \emph{tams-vǫndr} ‘taming-wand’ of st. 26 above.  Cf. \Havamal\ 152, which speaks about a runic curse carved on \emph{rótum rás viðar} ‘the roots of a raw/sappy tree’.}} at \alst{g}eta &
\ind \alst{g}amban-tęin ek \alst{g}at.\eva

\bvb To the wood I went, and to the raw/sappy tree, \\
\ind the \inx[C]{gombentoe} for to get; \\
\ind the gombentoe I got.\evb\evg


\bvg\bva\mssnote{\Regius~12r/10}%
\alst{R}ęiðr ’s þér Óðinn, \hld\ \alst{r}ęiðr ’s þér Ása-bragr, &
\ind þik skal \alst{F}ręyr \alst{f}íask, &
hin \alst{f}irin-illa mę́r, \hld\ en \alst{f}ingit hęfr &
\ind \alst{g}amban-ręiði \alst{g}oða.\eva

\bvb Wroth with thee is Weden; wroth with thee is Bray of the Eese \name{= Thunder}; \\
\ind thee shall Free come to hate, \\
O most wicked maiden, if thou hast earned \\
\ind the gomben-wrath of the gods.\evb\evg


\bvg\bva\mssnote{\Regius~12r/12}%
\alst{H}ęyri jǫtnar, \hld\ \alst{h}ęyri \alst{h}rím-þursar, &
\alst{s}ynir \alst{S}uttunga, \hld\ \alst{s}jalfir ás-liðar, &
hvé \alst{f}yrir býð’k, \hld\ hvé \alst{f}yrir banna’k &
\ind \alst{m}anna glaum \alst{m}ani, &
\ind \alst{m}anna nyt \alst{m}ani.\eva

\bvb Let hear Ettins, let hear Rime-thurses, \\
sons of Sutting, the very Os-Troops \ken*{= Eese} themselves! \\
how I forbid, how I forban \\
\ind men’s fellowship from the maid, \\
\ind men’s joy from the maid!\evb\evg


\bvg\bva\mssnote{\Regius~12r/14}%
\alst{H}rím-grímnir hęitir þurs, \hld\ es þik \alst{h}afa skal &
\ind fyr \alst{n}á-grindr \alst{n}eðan, &
þar þér \alst{v}íl-męgir \hld\ á \alst{v}iðar rótum &
\ind \alst{g}ęita-hland \alst{g}efi; &
\alst{ǿ}ðri drykkju \hld\ fá þú \alst{a}ldri-gi, &
\ind \alst{m}ę́r, af þínum \alst{m}unum, &
\ind \alst{m}ę́r, at \alst{m}ínum \alst{m}unum.\eva

\bvb Rimegrimner is called the thurse who thee shall have \\
\ind down beneath Nawgrind, \\
where the lads of toil \ken{thralls} on the roots of a tree, \\
\ind goat-piss will give thee. \\
A finer drink do thou never get, \\
\ind O maiden, against thy liking, \\
\ind O maiden, to my liking!\evb\evg


\bvg\bva\mssnote{\Regius~12r/16}%
\edtrans{\alst{Þ}urs}{thurse}{\Bfootnote{Thurse is the name of the \textbf{þ}-rune (ᚦ); it is carved as part of the curse.}} ríst’k \alst{þ}ér \hld\ ok \edtrans{\alst{þ}ría stafi}{three staves}{\Bfootnote{Three runic letters (or phrases) representing the three following words (\emph{ęrgi} ‘queerness, degeneracy’ etc.).  The ritual practice of carving “three staves” is first found on the C7th Gummarp stone: \textbf{h\textsc{a}þuwol\textsc{a}fʀ s\textsc{a}te st\textsc{a}b\textsc{a} þri\textsc{a} fff} ‘Hathwolf placed three staves: fff’, where the \textbf{f}-rune (ᚠ) stands for its name \inx[C]{fee} (i.e. ‘wealth, cattle’) and is thus meant to bring wealth.}}, &
\ind \edtrans{\alst{ę}rgi ok \alst{ǿ}ði ok \alst{ó}·þola}{queerness and madness and restlessness}{\Bfootnote{Both \emph{ęrgi} ‘queerness, degeneracy’ and \emph{ó·þoli} ‘restlessness’ (here probably from strong lust) are found in the love magic charm on the rune stick B257 from Bryggen (edited below under Galders).  \emph{ęrgi} is also found in the curse-formula on the C7th Proto-Norse runestones from Stentoften and Björketorp.  See further introduction to B257.}}, &
svá ek þat \alst{a}f ríst \hld\ sem ek þat \alst{á} ręist, &
\ind ef gørask \alst{þ}arfar \alst{þ}ęss.“\eva

\bvb \inx[G]{Thurses}[Thurse] I carve for thee, and three staves: \\
\inx[C]{queerness} and madness and restlessness.— \\
\ind So I carve it \emph{off}, like I carved it \emph{on}, \\
if there be need for that.\footnoteB{Shirner has carved the curse (which will make true the curse), but tells Gird that he will scrape it off if she \ind accepts his demands. She promptly does.}”\evb\evg


\bvg\bva\speakernote{[Gęrðr kvað:]}\mssnote{\Regius~12r/19}%
„\edtext{\alst{H}ęill ves þú \alst{h}ęldr, svęinn, \hld\ ok tak við \alst{h}rím-kalki &
\ind \alst{f}ullum \alst{f}orns mjaðar,}{\lemma{Hęill \dots\ mjaðar ‘Hale \dots\ mead’}\Bfootnote{Formulaic; the same lines occur in \Lokasenna\ 53.}} &
þó hafða’k \alst{ę́}tlat, \hld\ at mynda’k \alst{a}ldri-gi &
\ind unna \edtrans{\alst{v}aningja}{the Waning \ken*{= Free}}{\Bfootnote{lit. ‘descendant of the \inx[G]{Wanes}’.  A rare word.  Its only other occurence in the Norse corpus is in a \inx[C]{thule} of boar-names.  Boars were sacred to Free, TODO.}} \alst{v}ęl.“\eva

\bvb\speakernoteb{[Gird quoth:]}%
“Hale be thou rather, swain, and receive the rime-chalice, \\
\ind full of ancient mead, \\
even though I had intended that I never would \\
\ind love the Waning \ken*{= Free} well.”\evb\evg


\bvg\bva\speakernote{[Skírnir kvað:]}\mssnote{\Regius~12r/21}%
„\alst{Ø}rendi mín \hld\ vil’k \alst{ǫ}ll vita, &
\ind áðr ríða’k \alst{h}ęim \alst{h}eðan, &
nę́r á \alst{þ}ingi \hld\ munt hinum \alst{þ}roska &
\ind \alst{n}ęnna \alst{N}jarðar syni.“\eva

\bvb\speakernoteb{[Shirner quoth:]}%
“My errands all I wish to know, \\
\ind before I ride home hence: \\
when on the \inx[C]{Thing} wilt thou with the vigorous \\
\ind son of Nearth \ken*{= Free} be joined?”\evb\evg


\bvg\bva\speakernote{[Gęrðr kvað:]}\mssnote{\Regius~12r/23}%
„\alst{B}arri hęitir, \hld\ es vit \alst{b}ę́ði vitum, &
\ind \alst{l}undr \alst{l}ogn-fara, &
en ępt \alst{n}ę́tr \alst{n}íu, \hld\ þar mun \alst{N}jarðar syni &
\ind \alst{G}ęrðr unna \alst{g}amans.“\eva

\bvb\speakernoteb{[Gird quoth:]}%
“Barrey is called—as we both know— \\
\ind a grove of calm rushes, \\
and after nine nights there will to the son of Nearth \\
\ind Gird her pleasure grant.”\evb\evg


\bpg\bpa\mssnote{\Regius~12r/24}%
Þá reið Skírnir heim. Freyr stóð úti ok kvaddi hann ok spurði tíðenda:\epa

\bpb Then Shirner rode home. Free stood outside and greeted him and asked for the tidings:\epb\epg


\bvg\bva\mssnote{\Regius~12r/25}%
„\alst{S}ęg mér, Skírnir, \hld\ áðr verpir \alst{s}ǫðli af mar &
\ind ok stígir \alst{f}eti \alst{f}ramarr, &
hvat \alst{á}rnaðir \hld\ í \alst{Jǫ}tun-hęima &
\ind þíns eða \alst{m}íns \alst{m}unar?“\eva

\bvb “Tell me, O Shirner, before thou throw the saddle off the steed, \\
\ind and take a step further: \\
what hast thou accomplished in the \inx[L]{Ettinhomes}, \\
\ind to thy or my liking?”\evb\evg


\bvg\bva\speakernote{[Skírnir kvað:]}\mssnote{\Regius~12r/27}%
„\alst{B}arri hęitir, \hld\ es vit \alst{b}áðir vitum, &
\ind \alst{l}undr \alst{l}ogn-fara, &
en ępt \alst{n}ę́tr \alst{n}íu, \hld\ þar mun \alst{N}jarðar syni &
\ind \alst{G}ęrðr unna \alst{g}amans.“\eva

\bvb\speakernoteb{[Shirner quoth:]}%
“Barrey is called—as we both know— \\
\ind a grove of calm rushes, \\
and after nine nights there will to the son of Nearth \\
\ind Gird grant her pleasure.”\evb\evg


\bvg\bva\speakernote{[Fręyr kvað:]}\mssnote{\Regius~12r/28, \GylfMS}%
\alst{L}ǫng es nǫ́tt, \hld\ \edtrans{\alst{l}angar ’u tvę́r}{long are two}{\Afootnote{\emph{lǫng es ǫnnur} ‘long is another’ \GylfMS}}, &
\ind \edtext{hvé of \alst{þ}ręyja’k \alst{þ}ríar?}{\Afootnote{\emph{hvé męga’k þręyja þríar} \GylfMS}} &
opt \alst{m}ér \alst{m}ánaðr \hld\ \alst{m}inni þótti &
\ind an sjá \alst{h}ǫlf \alst{h}ý-nǫ́tt.\eva

\bvb\speakernoteb{[Free quoth:]}%
Long is a night, long are two— \\
\ind how can I yearn for three? \\
Oft a month to me seemed less \\
\ind than this half wedding-night.\footnoteB{The wedding-night (TODO: it's a hapax so explain the etymology?) is presumably half as it is not consumated.}\evb\evg

\sectionline
