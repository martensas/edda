\bookStart{The Thule of Righ}[Rígsþula]

BPG
BPA Svá sęgja menn í fornum sǫgum, at ęinnhvęrr af ǫ́sum, sá es Heimdallr hét, fór fęrðar sinnar ok framm með sjóvarstrǫndu nǫkkurri, kom at ęinum húsabǿ ok nefndisk Rígr; ęptir þęiri sǫgu es kvę́ði þetta.EPA

BPB Thus say men in ancient saws, that one of the Ease†—he who was called Homedall—went on his journey forth along some lakeshore, came upon a lone homestead and called himself Righ. According to that saw is this poem:EPB
EPG

\bvg
\bva Ár kvǫ́ðu ganga \hld\ grǿnar brautir &
ǫflgan ok aldinn \hld\ ǫ́s kunnigan, &
ramman ok rǫskvan \hld\ Ríg stíganda.\eva

\bvb Of yore they said did walk the green paths, a mighty and aged \inx{os}, cunning; the strong and brisk Righ, striding.\evb
\evg


\bvg
\bva Gekk hann męir at þat \hld\ miðrar brautar, &
kom hann at húsi, \hld\ hurð vas á gę́tti; &
inn nam at ganga, \hld\ eldr vas á golfi, &
hjón sǫ́tu þar \hld\ hǫ́r at arni, &
Ái ok Edda \hld\ aldinfalda.\eva

\bvb Went he further at that, on the middle of the road; came he to a house; the door was wide open. He took to go inside; fire was on the floor. A couple sat there, hoary by the hearth: Great Grandfather and Great Grandmother, old-fashioned.\evb
\evg

TODO
