\bookStart{The Lay of Hildbrand}

\begin{flushright}%
\textbf{Dating:} 700s

\textbf{Meter:} \Fornyrdislag%para
\end{flushright}%

% Introduction

For the text of original poem I present the manuscript text with as few textual emendations as possible. As for the orthography, I have found it impossible to produce a normalised without too heavily distorting the received text, being as it is, a blend of several dialects (one need only observe the treatment of the name Thedric, which appears thrice, and each time in a markedly different form). Apart from my typical practice of capitalising proper names, marking prefixes with ⟨·⟩ and compounds with ⟨-⟩, and using acute accents to signify long vowels, circumflex accents to signify now-monophthongised original diphthongs, and overdots to mark nasal vowels, I have done the following changes in order to clarify etymological relationships and make the text somewhat more wieldy. Of these, 8–10 have also been noted in the apparatus where they occur:
\begin{enumerate}
  \item Consistently replaced both \emph{ƿ} (wynn) and \emph{uu} with \emph{w}.
  \item Consistently replaced \emph{c} with \emph{k}.
  \item Consistently replaced \emph{qu} with \emph{kw}.
  \item Consistently replaced \emph{t} with \emph{t̨} in positions affected by the Second Sound Shift.
  \item Replaced \emph{th} with \emph{þ}.
  \item Replaced \emph{e} with \emph{ę} when reflecting an original a-vowel affected by \emph{i}-mutation.
  \item Replaced \emph{ó} with \emph{ǫ́} where originally an \emph{a}.
  \item Removed unetymological double \emph{nn}.
  \item Restored initial \emph{h-} where etymological and/or metrically required.
  \item Removed initial \emph{h-} unetymological and/or metrically deficient.
\end{enumerate}

The punctuation of the original, entirely consisting of interpuncts, at times representing metrical breaks, at others sporadically placed, has not been retained.

Where they appear in cæsuræ, the words \emph{kwad Hilti-brant} ‘Hildbrand quoth’ (found in ll. 30, 49, and 58) replace the usual interpunct. Due to their hypermetrical nature, I had originally planned to remove these, and instead indicate the speaker in the margins—but after comparison with various Norse stanzas (e.g. \Reginsmal\ 3, wherein the words \emph{kvað Loki} ‘Lock quoth’ appear in the stanza’s first cæsura), I have come to believe that these represent an ancient oral interjection, seemingly going back as far as the Migration Period (as it seems incredulous to think that the scribe of \HildMS\ should have influenced the four centuries younger scribe of \Regius\ in such a minor point.)

% Summary

\sectionline

The poet gives a very short formulaic introduction, from which we can tell that the beginning of the poem is preserved (1–2). Hildbrand and Hathbrand, father and son, arm and dress themselves before riding into battle, each the head of an opposing host (3–6). Hildbrand asks Hathbrand about his name and lineage, saying that he knows all noble genealogies (7–13). Hathbrand gives his name, and says that the old men of his tribe have told him that his father was Hildbrand, a brave warrior. He abandoned the newborn Hathbrand in order to serve Thedric in his fight against Edwaker, but this was a long time ago, and Hathbrand doubts that he is still alive (14–29). Realising that he is facing his son, Hildbrand invokes God as witness, and as a token of loyalty offers Hathbrand a golden bigh which the Hunnish king had given him (30–35). Hathbrand exclaims that treasures must be won by struggle alone and harshly insults his father’s manhood: he calls him an old Hun, and accuses him of having survived to old age through treachery (36–41). Hathbrand then reveals that he has learned from sailors on the Mediterranean that Hildbrand is dead (42–44).

After this follow three short speeches by Hildbrand. The second one is certainly spoken by him, but the other two may be misplaced or misattributed. Hildbrand reflects on his son’s prosperity, saying that he can tell from his clothes that he has a good lord, and that he, unlike himself, has not suffered an exile’s fate (first speech: 45–48). He then calls on God, and laments that after thirty years of war he is now forced to fight against his own son; still, he tells Hathbrand that he should easily be able to kill such an old man as himself, if he has the strength to it (second speech: 49–57). Lastly, he (or Hathbrand, if we choose to emend) says that only the most queer easterner would refuse the fight when his opponent so greatly desires it. He accepts his fate and declares that when the duel is over, one of the two must win and rob the corpse of the other (third speech: 58–62).

The two men then throw their javelins, each of which gets stuck in the opposing shield, before rushing into each other, hacking away at their shields until they become worthless (63–68). The rest of the poem was continued on the now-lost, following page(s).

\sectionline

\bvg\bva[]%
Ik gi·hôrta dat̨ sęggen &
dat̨ sih \alst{u}r·hêt̨t̨un \hld\ \alst{ae}non muot̨ín: &
\alst{H}ilti-brant ęnti \alst{H}adu-brant \hld\ untar \alst{h}ęrjun t̨wêm &
\alst{s}unu-fatar·ungo \hld\ iro \alst{s}aro rihtun &
\alst{g}arutun sé iro \alst{g}u̇d-hamun \hld\ \alst{g}urtun sih iro swert ana &
\alst{h}ęlidos ubar \edtext{\alst{h}ringa}{\Afootnote{\emph{ringa} \HildMS}} \hld\ dó sie t̨ó dero \alst{h}iltu ritun.\eva

\bvb[0]I heard it said, \\
that two contenders alone did meet: \\
Hildbrand and Hathbrand, under two hosts.\footnoteB{i.e. each man was a champion of his respective army.} \\
Son and father ordered their armour, \\
readied their war-cloths, girded their swords on, \\
the heroes over the mail-coats—when to that battle they rode.\evb\evg


\bvg\bva[][6]%
\alst{H}ilti-brant \edtext{gi·mahalta}{\Afootnote{\emph{heribrantes sunu} ‘Harbrand’s son’ add. \HildMS}} \hld\ her was \alst{h}êróro man &
\alst{f}erahes \alst{f}rótóro \hld\ her \alst{f}rágén gi·stuont &
\alst{f}ôhém wortum \hld\ \edtext{hwer}{\Afootnote{\emph{wer} \HildMS}} sín \alst{f}ater wári &
\alst{f}irjo in \alst{f}olkhe \hld\ {[...]} &
{[...]} \hld\ „eddo \edtext{hwe-líhhes}{\Afootnote{\emph{welihhes} \HildMS}} \alst{k}nuosles dú sís &
ibu dú mí \alst{ê}nan sagés \hld\ ik mí de \alst{ǫ́}dre wêt &
\alst{kh}ind in \edtext{\alst{kh}unink-ríkhe}{\Afootnote{\emph{chunnincriche} \HildMS}} \hld\ \alst{kh}u̇d ist mín al irmin-deot“\eva

\bvb[0]Hildbrand spoke—he was the hoarier man, \\
more learned in life—he began to ask \\
in few words, who his father might be, \\
of men in the troop, [...] \\
“or of which lineage thou be; \\
if thou tell me one I the others will know, \\
O child, in the kingdom all great men are known to me.”\evb\evg


\bvg\bva[][13]%
\alst{H}adu-brant gi·mahalta \hld\ \alst{H}ilti-brantes sunu &
\edtext{„dat̨ sagetun mí \hld\ u̇sere liuti}{\lemma{dat \dots\ liuti}\Bfootnote{this l. breaks no rhythmic rules (cf. l. 42), but the needed alliteration is missing.}} &
\alst{a}lte anti fróte \hld\ dea \alst{ê}rhina wárun &
dat̨ \alst{H}ilti-brant haet̨t̨i mín fater \hld\ ih hęit̨t̨u \alst{H}adu-brant &
forn her \alst{ô}star \edtext{gi·węit̨}{\Afootnote{\emph{gihueit} \HildMS}} \hld\ flôh her \alst{Ô}t-akhres níd &
hina miti \alst{Þ}eot-ríhhe \hld\ ęnti sínero \alst{d}egano filu &
her fur-\alst{l}aet̨ in \alst{l}ante \hld\ \alst{l}út̨t̨ila sit̨t̨en &
\edtext{\alst{b}rút}{\Afootnote{\emph{prut} \HildMS}} in \alst{b}úre \hld\ \alst{b}arn un·wahsan &
\alst{a}rbjo-laosa \hld\ \edtext{her raet}{\Afootnote{\emph{heraet} \HildMS}} \alst{ô}star hina &
des síd \alst{D}et-ríhhe \hld\ \alst{d}arba \edtext{gi·stuontun}{\Afootnote{\emph{gistuontum} \HildMS}} &
\edtext{\alst{f}ateres}{\Afootnote{\emph{fatereres} \HildMS}} mínes \hld\ dat̨ was só \alst{f}riunt-laos man &
her was \alst{Ô}t-akhre \hld\ \alst{u}m·met̨ t̨irri &
\alst{d}egano \alst{d}ękhisto \hld\ unti \edtext{\alst{D}eot-ríkhhe}{\Afootnote{\emph{darba gistontun} add. \HildMS}} &
her was eo \alst{f}olkhes at̨ ęnte \hld\ imo was eo \edtext{\alst{f}eheta}{\Afootnote{\emph{peheta} \HildMS}} t̨i leop &
\alst{kh}u̇d was her \hld\ \edtext{\alst{kh}óném}{\Afootnote{\emph{chonnem} \HildMS}} mannum &
ni wániu ih iu líb habbe.“\eva

\bvb[0]Hathbrand spoke, Hildbrand’s son: \\
“\emph{This} \emph{our} people told me— \\
the old and learned, those who lived earlier— \\
that Hildbrand was called my father—I am called Hathbrand. \\
Long ago he turned east, he fled Edwaker’s hate, \\
hence with Thedrich and his multitude of thanes. \\
He left in the land a little one to stay: \\
a bride in the bower, a bairn ungrown, \\
inheritance-less—he rode east hence, \\
at which time Thedrich was in great need \\
of my father—that was so friendless a man! \\
He was immeasurably hostile to Edwaker, \\
the dearest of thanes under Thedrich. \\
He was always at the front of the troop; him did always the fight gladden; \\
known was he among keen men; \\
I ween not that he still have life.”\evb\evg


\bvg\bva[][29]%
„wêt̨t̨u \alst{I}rmin-got {\small (kwad Hilti-brant)} \alst{o}bana ab \edtext{hewane}{\Afootnote{\emph{heuane} \HildMS}} &
dat̨ dú neo dana halt mit sus sippan man &
dink ni gi·lęitós“ &
\alst{w}ant her dó ar arme \hld\ \alst{w}untane bauga &
\alst{kh}ęisur·ingu gi·tán \hld\ so imo sie der \alst{kh}uning gap &
\alst{h}unjo truhtin \hld\ „dat̨ ih dír it̨ nú bí \alst{h}uldí gibu“\eva

\bvb[0]“I call Ermin-god as witness above in heaven, \\
that thou never again with such a close relation lead dispute.” \\
He then unwound from his arm some twisted \inx[C]{bigh}[bighs], \\
made by a Cæsar’s man, which the king had given him, \\
the Lord of the Huns—“This I now give thee as [a sign of] \inx[C]{holdness}.\footnoteB{The giving of \emph{bighs} (armlets, torcs) in exchange for loyalty among warriors is well attested; see Encyclopedia.  This encounter is particularly reminiscent of \Harbardsljod\ 42.}”\evb\evg


\bvg\bva[][35]%
\alst{H}adu-brant gi·mahalta \hld\ \alst{H}ilti-brantes sunu: &
„\edtrans{mit \alst{g}êru skal man \hld\ \alst{g}eba in·fȧhan}{With spear shall one win gifts}{\Bfootnote{This ancient mindset was codified by the Indians as part of the \emph{kṣatra-dharma}, the code of the Warrior (\emph{kṣatriya}) caste, which explicitly forbade them from taking gifts.  So in a part of the Mahabharata (12.192.73), a Warrior King refuses a gift from a priest since “it is the duty prescribed for a Kṣatriya that he must fight and protect (people).  Kṣatriya are said to be the givers, then, how can I take (this) from you?” (\textcite{Hara1974} transl.)}} &
\alst{o}rt widar \alst{o}rte \hld\ {[...]} &
dú bist dir \alst{a}ltér hun \hld\ \alst{u}m·met̨ spáhér &
\alst{sp}ęnis mih mit díném wortun \hld\ wili mih dínu \alst{sp}eru werpan &
\edtext{bist}{\Afootnote{\emph{pist} \HildMS}} \alst{a}l-só gi·\alst{a}ltét man \hld\ só dú êwín \alst{i}n·wit fórtós &
dat̨ \alst{s}agetun mí \hld\ \alst{s}êo-lídante &
\alst{w}estar ubar \edtrans{\alst{W}ęntil-sêo}{Wendle-sea}{\Bfootnote{The Mediterranean, the name referring to the Wandals who for a time ruled North Africa.}} \hld\ dat̨ man \alst{w}ík fur·nam: &
tôt ist \alst{H}ilti-brant \hld\ \alst{H}ęri-brantes suno!“\eva

\bvb[0]Hathbrand spoke, Hildbrand’s son: \\
“With spear shall one win gifts, \\
point against point! \\
Thou art, old Hun, immeasurably clever: \\
thou dost lure me with thy words; at me wilt thou hurl thy spear! \\
Thou art thus an aged man, since thou always deceit didst work.— \\
\emph{This} told me seafarers \\
in the west over the Wendle-sea, that war took that man; \\
dead is Hildbrand, Harbrand’s son!”\evb\evg


\bvg\bva[][44]%
\alst{H}ilti-brant gi·mahalta \hld\ \alst{H}ęri-brantes suno: &
„wela gi·sihu ih in díném hrustim &
dat̨ dú \alst{h}abés \alst{h}ême \hld\ \alst{h}êrron góten &
dat̨ dú noh bí desemo \alst{r}íkhe \hld\ \alst{r}ekkhjo ni wurti“\eva

\bvb[0]Hildbrand spoke, Harbrand’s son: \\
“Well do I see from thy gear, \\
that thou hast a good lord at home, \\
that thou yet from this realm art not become an exile.”\evb\evg


\bvg\bva[][48]%
„\alst{w}elaga nú \alst{w}altant got {\small (kwad Hilti-brant)} \edtrans{\alst{w}ê-wurt}{woeful weird}{\Bfootnote{\emph{wurt} here meaning ‘inexorable course of events’, not the Old Norse norn; cf. ON \emph{grimmar urðir} ‘grim courses of events’ TODO.}} skihit &
ih wallóta \edtrans{\alst{s}umaro ęnti wintro \hld\ \alst{s}ehs-tik}{sixty summers and winters}{\Bfootnote{i.e. thirty years.  Hathbrand is then around thirty years old, while Hildbrand is in his fifties or sixties.}} ur lante &
dar man mih eo \alst{sk}ęrita \hld\ in folk \edtrans{\alst{sk}eot̨antero}{shooters}{\Bfootnote{Cf. \Beowulf\ 702, where the OE cognate \emph{sceótend} stands for “warriors” in general.}} &
só man mir at̨ \alst{b}urk ênigeru \hld\ \alst{b}anun ni gi·fasta &
nú skal mih \alst{s}wásat̨ khind \hld\ \alst{s}wertu hauwan &
\alst{b}retón mit sínu \alst{b}illju \hld\ eddo ih imo t̨i \alst{b}anin werdan. &
Doh maht dú nú \alst{ao}d-líhho \hld\ ibu dir dín \alst{ę}llen taok &
in sus \alst{h}êremo man \hld\ \alst{h}rusti gi·winnan &
\alst{r}auba \edtext{bi·\alst{r}ahanen}{\Afootnote{\emph{bihrahanen} \HildMS}} \hld\ ibu dú dar êníg \alst{r}eht habés!“\eva

\bvb[0]“Well now, O wielding God! the woeful weird comes to pass. \\
I roamed for sixty summers and winters away from the land, \\
where I always was placed in the troop of shooters, \\
as at no fortress my bane was fastened.— \\
Now shall my own child strike me with the sword, \\
beat me down with his blade—or I become his bane. \\
Yet thou mayst now easily—if thy zeal avail thee— \\
from such a hoary man win the equipment; \\
bear away the booty—if thou have any right to it!”\evb\evg


\bvg\bva[][57]%
„der sí doh nú \alst{a}rgósto {\small (kwad Hilti-brant)} \alst{ô}star-liuto &
der dir nú \alst{w}íges \alst{w}arne \hld\ nú dih es só \alst{w}el lustit &
gu̇dja gi·\alst{m}ęinun \hld\ niuse de \alst{m}ót̨t̨i &
\edtext{hwędar}{\Afootnote{\emph{werdar} \HildMS}} sih \edtext{\alst{h}iutu dêro}{\Afootnote{metr. emend.; \emph{dero hiutu} \HildMS}} \alst{h}regilo \hld\ \edtext{\alst{h}ruomen}{\Afootnote{\emph{hrumen} \HildMS}} muot̨t̨i &
\edtext{eddo}{\Afootnote{\emph{erdo} \HildMS}} desero \alst{b}runnóno \hld\ \alst{b}êdero waltan!“\eva

\bvb[0]“He be now the weakest of Easterners, \\
who should refuse thee the fight when thou so greatly cravest \\
to struggle together—try he who might, \\
which one of us today of these garments may boast, \\
or of these byrnies wield both!”\evb\evg


\bvg\bva[][62]%
Dó lét̨t̨un sé \alst{ae}rist \hld\ \alst{a}skkim skrítan &
\edtrans{\alst{sk}arpén \alst{sk}úrim}{in sharp showers}{\Bfootnote{Formulaic, also occurring in \Heliand\ 5137a.}} \hld\ dat̨ in dem \alst{sk}iltim stónt &
dó \alst{st}óptun t̨ó·samane \hld\ \alst{st}aim-bort \edtext{hludun}{\Afootnote{\emph{chludun} \HildMS}} &
\alst{h}ewun harm-líkko \hld\ \alst{h}wít̨t̨e skilti &
unti imo iro \alst{l}intún \hld\ \alst{l}út̨t̨ilo wurtun &
gi·\alst{w}igan miti \alst{w}ábnum \hld\ \edtext{[...]}{\Bfootnote{At this point the lone folio ends.  The rest of the poem would have been found on the now-lost following pages.  See Introduction to the poem.}}\eva

\bvb[0]Then let they first their ash-spears glide, \\
in sharp showers, that in the shields they stuck. \\
Then charged they into each other—the war-boards \ken{shields} resounded— \\
struck they harmfully the white shields, \\
until for them their lindens \ken{shields} became little, \\
worn down by the weapons, [...].\evb\evg

\sectionline
