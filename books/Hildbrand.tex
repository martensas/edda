\bookStart{Lay of Hildbrand}[Hildebrandslied]
\def\thisBookCode{Hildebrandslied}

\begin{flushright}%
\textbf{Dating:} C8th

\textbf{Meter:} \Fornyrdislag%para
\end{flushright}%

\section{Introduction}

For the text of original poem I present the manuscript text with as few textual emendations as possible. As for the orthography, I have found it impossible to produce a normalised without too heavily distorting the received text, being as it is, a blend of several dialects (one need only observe the treatment of the name Thedric, which appears thrice, and each time in a markedly different form). Apart from my typical practice of capitalising proper names, marking prefixes with ⟨·⟩ and compounds with ⟨-⟩, and using acute accents to signify long vowels, circumflex accents to signify now-monophthongised original diphthongs, and overdots to mark nasal vowels, I have carried out the following changes in order to clarify etymological relationships and make the text somewhat less unwieldy. Of these changes, 7–9 have also been noted in the apparatus where they occur:
\begin{enumerate}
  \item Consistently replaced both \emph{ƿ} (wynn) and \emph{uu} with \emph{w}.
  \item Consistently replaced \emph{c} with \emph{k}.
  \item Consistently replaced \emph{qu} with \emph{kw}.
  \item Consistently replaced \emph{t} with \emph{t̨} in positions affected by the Second Sound Shift.
  \item Replaced \emph{th} with \emph{þ}.
  \item Replaced \emph{e} with \emph{ę} when reflecting an original a-vowel affected by \emph{i}-mutation.
  \item Replaced unetymological double \emph{nn} with \emph{n}.
  \item Restored initial \emph{h-} where etymological and/or metrically required.
  \item Removed initial \emph{h-} where unetymological and/or metrically deficient.
\end{enumerate}

The punctuation of the original, entirely consisting of interpuncts, at times representing metrical breaks, at others sporadically placed, has not been retained.

Where it appears in the cæsura, the extrametrical interjection \emph{kwad Hilti-brant} ‘quoth Hildbrand’ (found in ll. 30, 49, and 58) replaces the usual interpunct, to indicate that the pause of the cæsura has been filled with an indication of the speaker.  Outside of \Hildebrandslied, similar interjections are found throughout early Germanic poetry in Old Norse (e.g. \Reginsmal\ 3/1, Anon \emph{Eirm} 1/1 in \Skp\ I), Old Saxon (e.g. \Heliand\ 226, \SaxonGenesis\ 1), and Old English (e.g. \Finnsburg\ 24).  The distribution of these interjections is such that it cannot be due to scribal transmission (the earliest Norse poetry was first written down in the C12th, several centuries after the alliterative meter had gone extinct in Germany), and they therefore seem to be genuine remnants of oral performance.

\subsection{Summary}

The poet begins with a short formulaic introduction; he is relating older stories (1–2).  The two duellists, Hildbrand and Hathbrand, father and son, arm themselves and ride into battle at the head of two opposing armies (3–6). They speak, and Hildbrand asks Hathbrand for his name and lineage (7–13). Hathbrand gives his name and ancestry; his father was the warrior Hildbrand, who abandoned him as a newborn. This was long ago, and Hathbrand does not think him still alive (14–29). Hearing this, Hildbrand calls on God as witness, and offers his son a golden torc as a token of loyalty (30–34). Hathbrand takes this as an insulting tricks. He proclaims that wealth should be won by struggle alone and accuses Hildbrand of having grown old through treachery (35–40); he has heard from sailors on the Mediterranean that his father is dead (41–43).

After this straight-forward narrative sequence three short speeches follow, in the ms. all spoken by Hildbrand. The second is certainly spoken by Hildbrand, but the other two may be misplaced or misattributed.

1. Hildbrand reflects on his son’s prosperity: from his clothes he can tell that he has a good lord, and that he, unlike himself, has not suffered the fate of exile (44–47).

2. Hildbrand calls on God, and laments that, after thirty years at war, he is now forced to fight against his own son. Still, Hathbrand should easily be able to kill such an old man as Hildbrand, if he has strength and fate on his side (48–56).

3. Hildbrand (or Hathbrand, and there is a case for emending here) says that only the most cowardly easterner could refuse the fight so greatly desired. Let both men fight their hardest, and when the duel is over the winner will strip the armour of the other (57–61).

The two men then throw their javelins into each other’s shield and rush at each other, hacking away at their shields until they become worthless (62–67). Here the poem abruptly ends.

\sectionline

\section{The Lay of Hildbrand}

\bvg\bva[]%
Ik gi·hôrta dat̨ sęggen &
dat̨ sih \alst{u}r·hêt̨t̨un \hld\ \alst{ae}non muot̨ín: &
\alst{H}ilti-brant ęnti \alst{H}adu-brant \hld\ \edtrans{untar \alst{h}ęrjun t̨wêm}{under two hosts}{\Bfootnote{Either man was a champion of his army.}} &
\alst{s}unu-fatar·ungo \hld\ iro \alst{s}aro rihtun &
\alst{g}arutun sé iro \alst{g}u̇d-hamun \hld\ \alst{g}urtun sih iro swert ana &
\alst{h}ęlidos ubar \edtext{\alst{h}ringa}{\Afootnote{\emph{ringa} \HildMS}} \hld\ dó sie t̨ó dero \alst{h}iltu ritun.\eva

\bvb I have heard it said \\
that two contenders alone did meet: \\
—Hildbrand and Hathbrand—under two hosts. \\
Son and father ordered their armour, \\
readied their war-cloths, girded on their swords, \\
the heroes over the mailcoats—when to that fray they rode.\evb\evg


\bvg\bva[][7]%
\alst{H}ilti-brant \edtext{gi·mahalta}{\Afootnote{\emph{heribrantes sunu} ‘Harbrand’s son’ add. \HildMS}} \hld\ —her was \alst{h}êróro man &
\edtrans{\alst{f}erạhes \alst{f}rótóro}{more learned of life}{\Bfootnote{Possibly formulaic; cf. \emph{Maldon} 317a: \emph{Ic eom fród feores.} ‘I am learned of life’.}}— \hld\ her \alst{f}rágén gi·stuont &
\alst{f}ôhém wortum \hld\ \edtext{hwer}{\Afootnote{\emph{wer} \HildMS}} sín \alst{f}ater wári &
\alst{f}irjo in \alst{f}olkhe \hld\ {[...]} &
{[...]} \hld\ „eddo \edtext{hwe-líhhes}{\Afootnote{\emph{welihhes} \HildMS}} \alst{k}nuosles dú sís &
ibu dú mí \alst{ê}nan sagés \hld\ ik mí de \alst{ȯ}dre wêt &
\alst{kh}ind in \edtext{\alst{kh}unink-ríkhe}{\Afootnote{\emph{chunnincriche} \HildMS}} \hld\ \alst{kh}u̇d ist \edtext{mí}{\Afootnote{\emph{mín} \HildMS}} al irmin-deot“\eva

\bvb Hildbrand spoke—he was the hoarier man, \\
more learned of life—he began to ask \\
in few words who his father might be \\
of men in the troop, [...] \\
{[...]} “or of which lineage thou be— \\
if thou tell me one I the others will know. \\
O child, in the kingdom, the whole tribe is known to me.”\evb\evg


\bvg\bva[][14]%
\alst{H}adu-brant gi·mahalta \hld\ \alst{H}ilti-brantes sunu: &
\edtrans{„Dat̨ sagetun mí \hld\ u̇sere liuti}{This our liegemen said to me}{\Bfootnote{The scansion of this line is inscrutable (cf. l. 42), but the needed alliteration is missing.}} &
\alst{a}lte anti fróte \hld\ dea \alst{ê}rhina wárun &
dat̨ \alst{H}ilti-brant haet̨t̨i mín fater \hld\ ih hęit̨t̨u \alst{H}adu-brant &
forn her \alst{ô}star \edtext{gi·*węit̨}{\Afootnote{\emph{gihueit} \HildMS}} \hld\ flôh her \alst{Ô}t-akhres níd &
hina miti \edtext{\emph{\alst{D}}eot-rihhe}{\Afootnote{\emph{theotrihhe} with pre-shift consonant \HildMS}} \hld\ ęnti sínero \alst{d}egano filu &
her fur·\alst{l}aet̨ in \alst{l}ante \hld\ \alst{l}út̨t̨ila sit̨t̨en &
\edtext{\emph{\alst{b}}rút}{\Afootnote{\emph{prut} \HildMS}} in \alst{b}úre \hld\ \alst{b}arn un·wahsan &
\alst{a}rbjo-laosa \hld\ \edtext{he\emph{r} raet}{\Afootnote{\emph{heraet} \HildMS}} \alst{ô}star hina &
des sïd \alst{D}e\emph{o}t-rihhe \hld\ \alst{d}arba \edtext{gi·stuontu\emph{n}}{\Afootnote{\emph{gistuontum} \HildMS}} &
\edtext{\alst{f}ater*es}{\Afootnote{\emph{fatereres} \HildMS}} mínes \hld\ dat̨ was só \alst{f}riunt-laos man &
her was \alst{Ô}t-akhre \hld\ \alst{u}m-met̨ t̨irri &
\alst{d}egano \alst{d}ękhisto \hld\ unti \edtext{\alst{D}eot-ríkhhe*}{\Afootnote{\emph{darba gistontun} add. \HildMS}} &
her was eo \alst{f}olkhes at̨ ęnte \hld\ imo was eo \edtext{\emph{\alst{f}}ehẹta}{\Afootnote{\emph{peheta} \HildMS}} t̨i leop &
\alst{kh}u̇d was her \hld\ \edtext{\alst{kh}ón*ém}{\Afootnote{\emph{chonnem} \HildMS}} mannum &
ni wániu ih iu líb habbe.“\eva

\bvb Hathbrand spoke, Hildbrand’s son: \\
“This our liegemen said to me— \\
the old and learned who earlier lived— \\
that Hildbrand my father was called—I’m called Hathbrand. \\
Long ago he turned east—he fled Edwaker’s hate— \\
away with Thedric and his multitude of thanes. \\
He left in the land a little one to stay; \\
a bride in the bower, a bairn ungrown, \\
heritance-less. He rode away east, \\
at which time Thedric was in great need \\
of my father—that was so friendless a man! \\
He was toward Edwaker utterly hostile; \\
the dearest of thanes under Thedric; \\
he was always in the front of the troop; him did always the fighting gladden; \\
known was he among keen men. \\
I do not think he still lives.”\evb\evg


\bvg\bva[][30]%
„Wêt̨t̨u \alst{I}rmin-got“ \hld[kwad Hilti-brant] „\alst{o}bana ab \edtrans{hevane}{heaven}{\Afootnote{\emph{heuane} \HildMS}\Bfootnote{A likely Old Saxon form, which merits some discussion on the relation between the synonymous \emph{himil} and \emph{hevan} in West Germanic.  The form \emph{himil} is found in both OS and OHG, but a cognate of \emph{hevan} is never found in OHG.  Further, the use of OS \emph{hevan} is unusual; it is never used in prose, and in poetry (\Heliand\ and \SaxonGenesis) its use is heavily stereotyped, being restricted to 5 cpds and 3 genitive expressions.  As a simplex, it is never used in any other form than the gen. sg.  Of course, it must have been used in some other context, since it has left descendants in modern Low German dialects.  In any case these facts pose some difficulty for the providence of the poem; if \Hildebrandslied\ were an originally OHG text (cf. Note to l. 47), translated into OS in a scribal context, it seems very strange that a translator would have replaced the neutral \emph{himil} with the rare, stereotyped \emph{hevan}.  Yet the presence of \emph{hevan} in the OHG archetype would be a major anomaly, since that form has never existed in any known variety of High German, up until the present day.}} &
dat̨ dú neo \alst{d}ana halt mit sus sippan man \hld\ \alst{d}ink ni gi·lęitós“ &
\alst{w}ant her dó ar arme \hld\ \edtrans{\alst{w}untane bauga}{twisted bighs}{\Bfootnote{The association between \inx[C]{bighs} (armlets, torcs) and a warrior’s honour is well attested; see Index.  This encounter is particularly reminiscent of \Harbardsljod\ 42.}} &
\edtrans{\alst{kh}ęisur·ingu gi·tán}{made of Caesar’s coin}{\Bfootnote{A cultural memory of the melting of Roman \emph{solidī} by Germanic smiths.}} \hld\ só imo sie der \alst{kh}uning gap &
\edtrans{\alst{h}unjo truhtin}{lord of the Huns}{\Bfootnote{Almost certainly \inx[P]{Attle}, although he is not mentioned by name in the poem.}} \hld\ „dat̨ ih dír it̨ nú bí \alst{h}uldí gibu“\eva

\bvb “I call on Ermin God as witness from heaven above, \\
that thou never henceforth with such close kin shouldst lead dispute!” \\
Then he wound from his arm twisted \inx[C]{bigh}[bighs], \\
made of Caesar’s coin, which him the king had given, \\
the lord of the Huns.—“This I now give thee out of \inx[C]{holdness}.”\evb\evg


\bvg\bva[][35]%
\alst{H}adu-brant gi·mahalta \hld\ \alst{H}ilti-brantes sunu: &
„\edtrans{mit \alst{g}êru skal man \hld\ \alst{g}eba in·fȧhan}{By his spear shall man win gifts}{\Bfootnote{This ancient mindset was codified by the Indians as part of the \emph{kṣatra-dʰarmá}, the code of the Warrior-caste (\emph{kṣatríya}), which explicitly forbade them from taking gifts.  So in \Mahabharata\ 12.192.73, a \emph{kṣatríya} king refuses a gift from a priest (\emph{brāhmaṇá}), for “it is the duty prescribed for a \emph{kṣatríya} that he must fight and protect (people).  Kṣatriya are said to be the givers, then, how can I take (this) from you?” (\textcite{Hara1974} transl., see further there.)}} &
\alst{o}rt widar \alst{o}rte! &
dú bist dir \alst{a}ltér hun \hld\ \alst{u}m-met̨ spáhér &
\alst{sp}ęnis mih mit díném wortun \hld\ wili mih dínu \alst{sp}eru werpan &
\edtext{bist}{\Afootnote{\emph{pist} \HildMS}} \alst{a}l-só gi·\alst{a}ltét man \hld\ só dú êwín \alst{i}n-wit fórtós &
dat̨ \alst{s}agetun mí \hld\ \alst{s}êo-lídante &
\alst{w}estạr ubar \edtrans{\alst{W}ęntil-sêo}{Wendle-sea}{\Bfootnote{The Mediterranean Sea, the name referring to the \emph{Vandali}, who for a time ruled North Africa.}} \hld\ dat̨ inan \alst{w}ík fur·nam: &
tôt ist \alst{H}ilti-brant \hld\ \alst{H}ęri-brantes suno!“\eva

\bvb Hathbrand spoke, Hildbrand’s son: \\
“By his spear shall man win gifts, \\
point against point! \\
Thou art for thee, old Hun, utterly clever; \\
thou dost tempt me with thy words—at me wilt thou hurl thy spear! \\
Thou art thus an aged man, since thou always didst work deceit.— \\
This seafarers said to me \\
west o’er the Wendle-sea: that war took him off— \\
dead is Hildbrand, Harbrand’s son!”\evb\evg


\bvg\bva[][44]%
\alst{H}ilti-brant gi·mahalta \hld\ \alst{H}ęri-brantes suno: &
„wela gi·sihu ih in díném hrustim &
dat̨ dú \alst{h}abés \alst{h}ême \hld\ \alst{h}êrron góten &
dat̨ dú noh bí desemo \alst{r}íkhe \hld\ \alst{r}ękkhjo ni wurti“\eva

\bvb Hildbrand spoke, Harbrand’s son: \\
“Well do I behold on thy garb, \\
that thou hast at home a good lord, \\
that thou yet in this realm hast not become an exile.”\evb\evg


\bvg\bva[][48]%
„\alst{w}elaga nú \edtrans{\alst{w}altant got}{O Ruler God!}{\Bfootnote{Cf. OE \emph{wealdend god}, OS \emph{waldand god}.  Apparently a common West Germanic poetic expression.}}“ \hld[kwad Hilti-brant] „\edtrans{\alst{w}ê-wurt}{woeful weird}{\Bfootnote{\emph{wurt} ‘weird’ here meaning ‘inexorable course of events’, not the norn; cf. ON \emph{grimmar urðir} ‘grim “weirds”’ TODO.}} skihit &
ih wallóta \edtrans{\alst{s}umaro ęnti wintro \hld\ \alst{s}ehs-tik}{sixty summers and winters}{\Bfootnote{i.e. thirty years.  Cf. \Beowulf\ 1498, 1769: \emph{hund misséra} ‘a hundred half-years’.  Hathbrand must then be thirty years old, while Hildbrand is in his fifties or sixties.}} ur lante &
dar man mih eo \alst{sk}ęrita \hld\ in folk \edtrans{\alst{sk}eot̨antero}{shooters}{\Bfootnote{Cf. \Beowulf\ 702, where the OE cognate \emph{sceótend} stands for “warriors” in general.}} &
só man mir at̨ \alst{b}urk ênigeru \hld\ \alst{b}anun ni gi·fasta &
nú skal mih \alst{s}wásat̨ khind \hld\ \alst{s}wertu hauwan &
\alst{b}retón mit sínu \alst{b}illju \hld\ eddo ih imo t̨i \alst{b}anin werdan. &
Doh maht dú nú \alst{ao}d-líhho \hld\ \edtrans{ibu dir dín \alst{ę}llen taok}{if thy zeal avail thee}{\Bfootnote{Formulaic.  Cf. \Beowulf\ 572b–573: \emph{[...] \hld\ Wyrd oft nęreð //
un-fǽgne eorl \hld\ þonne his ęllen déah.} ‘Weird often saves the un-\inx[C]{fey} \inx[C]{earl} when his zeal avails.’}} &
in sus \alst{h}êremo man \hld\ \alst{h}rusti gi·winnan &
\alst{r}auba \edtext{bi·*\alst{r}ahanen}{\Afootnote{\emph{bihrahanen} \HildMS}} \hld\ ibu dú dar êníg \alst{r}eht habés!“\eva

\bvb “Well now—O Ruler God!—the woeful weird comes to pass. \\
I roamed for sixty summers and winters from the land, \\
where I always was placed in the troop of shooters, \\
as at no fortress my bane was fastened.— \\
Now shall my very child hew at me with his sword, \\
strike me with his blade, or I become his bane. \\
Yet mayst thou now easily—if thy zeal avail thee— \\
from such a hoary man win the garb, \\
bear away the booty—if thou have any right thereto!”\evb\evg


\bvg\bva[][57]%
„der sí doh nú \alst{a}rgósto“ \hld[kwad Hilti-brant] „\alst{ô}star-liuto &
der dir nú \alst{w}íges \alst{w}arne \hld\ nú dih es só \alst{w}el lustit &
gu̇dja gi·\alst{m}ęinun \hld\ niuse de \alst{m}ót̨t̨i &
\edtext{hwędar}{\Afootnote{\emph{werdar} \HildMS}} sih \edtext{\alst{h}iutu dêro}{\Afootnote{metr. emend.; \emph{dero hiutu} \HildMS}} \edtext{\alst{h}ręgilo \hld\ \edtext{\alst{h}ruomen}{\Afootnote{\emph{hrumen} \HildMS}} muot̨t̨i &
\edtext{eddo}{\Afootnote{\emph{erdo} \HildMS}} desero \alst{b}runnóno \hld\ \alst{b}êdero waltan}{\lemma{hręgilo hruomen muot̨t̨i \dots\ desero brunnóno bêdero waltan ‘of these garments may boast \dots\ both these byrnies wield’}\Bfootnote{Like in the Iliad, the winner is expected to strip the slain of his armour.}}!“\eva

\bvb “He were now (quoth Hildbrand) the softest of Easterners, \\
who would refuse thee a fight when thou so much dost crave \\
to struggle together. Try he who might, \\
which one of us today of these garments may boast, \\
or both these byrnies wield!”\evb\evg


\bvg\bva[][62]%
Dó lét̨t̨un sé \alst{ae}rist \hld\ \edtext{\alst{a}skim}{\Afootnote{\emph{asckim} \HildMS}} skrítan &
\edtrans{\alst{sk}arpén \alst{sk}úrim}{in sharp showers}{\Bfootnote{Formulaic, also occurring in \Heliand\ 5137a.}} \hld\ dat̨ in dem \alst{sk}iltim stónt &
dó \alst{st}óptun t̨ó·samane \hld\ \alst{st}aim-bort \edtext{hludun}{\Afootnote{\emph{chludun} \HildMS}} &
\alst{h}ewun harm-líkko \hld\ \alst{h}wít̨t̨e skilti &
unti imo iro \alst{l}intún \hld\ \alst{l}út̨t̨ilo wurtun &
gi·\alst{w}igan miti \alst{w}ábnum \hld\ \edtext{[...]}{\Bfootnote{At this point the lone folio ends.  The rest of the poem would have been found on the now-lost following pages.  See Introduction to the poem.}}\eva

\bvb Then let they first their ash-spears glide, \\
in sharp showers, that in the shields they stuck. \\
Then they charged at each other—the coloured boards \ken{shields} clashed— \\
they hewed harmfully at the white shields, \\
until for them their lindens \ken{shields} became little, \\
worn down by the weapons, [...]\evb\evg

\sectionline
