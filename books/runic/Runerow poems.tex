\section{Introduction to the Rune Poems}

TODO: Acrophonic principle

The order and names of the letters in the Runic alphabets or \emph{futharks} stayed relatively consistent throughout the many centuries and countries in which they were used.  This can probably be ascribed to the \emph{rune poems}—poetic lists of the names of each rune with a short explanation, passed down orally as mnemonic devices to aid early Germanic learners, who were doubtless far more accustomed to learn by heart spoken poems than written letters.

Three such rune poems survive, from three countries: England, Norway, and Iceland.  The English rune poem documents the English \emph{futhorc}, while the Norwegian and Icelandic document the Scandinavian \emph{younger futhark}.

When compared to the Common Germanic \emph{elder futhark}, these two daughter scripts have taken opposing paths.  Whereas the English futhorc has appended several letters for new vowels to the end of the rune row, the Scandinavian futhark has instead done away with numerous runes, namely those for \emph{ng}, plosives \emph{d, g, p}, the semi-vowel \emph{w} and the vowels \emph{o} and \emph{e}, along with the obscure hook-shaped rune (TODO).  That much of this simplification was probably intentional, rather than the result of neglect or language change, is seen from the following facts.

First, several of the lost runes stood for sounds that did not undergo any major sound shifts in the North Germanic languages in the relevant time period.  For instance, all modern Scandinavian dialects still clearly distinguish between the initial consonants in the descendants of \emph{dagʀ} ‘day’ and \emph{Týr} ‘\inx[C]{Tew}’, and most even have the same articulation of these consonants as modern English.

Second, in two archaic runic inscriptions we find clear proof that the names and sound values of some of the lost runes were still remembered and passed down even after the adoption of the simplified younger futhark.  On the Swedish Rök stone (Ög 136), which is mostly composed in the younger futhark, runes of the elder futhark are used in a cipher, which works in the following way: Every younger futhark rune representing two distinct phonemes, where one of those was the sound value of that rune in the elder futhark system, and the other has been assimilated from a lost rune, is replaced by the elder futhark rune whose value it assimilated.  For instance, the \textbf{k} rune, which in the elder futhark stood for only /k/, but which in the younger futhark stands for both /k/ and /g/, is replaced with the old \textbf{g} rune.  A similar instance of two-scriptedness is found on the Ingelsta stone (Ög 43), where the old \textbf{d} rune is used in an otherwise younger futhark inscription, probably standing for its name \emph{dagʀ} ‘day’, which is also attested as a male given name.

Third, there is virtually no regional variation in which runes disappear in the transition from elder to younger futhark.  There is some variation in their shapes, but there is no region which, say, simplifies only the plosive consonants \emph{t/d, k/g, b/p} > \emph{t, k, b}, but retains the written distinction between \emph{o} and \emph{u}—they all go away at once.

These facts point away from neglect or a natural development of the script—they instead suggest deliberate reform.  Since we lack historical sources, the motivations behind such a reform can only be guessed at, but making the script simpler may have been intended to increase literacy by making it easier to learn and faster to write.  If this were the case it was certainly successful: the transition to the simplified younger futhark brings with it a huge increase in inscriptions in Scandinavia, along with interest in various ciphers, and a new tradition of inscribed stones in Denmark, where they were previously unknown.

This new system also quickly gave rise to even more simplified systems, like the “short-stave” runes found already on the C9th Rök stone, or the “staveless” runes known from northern Sweden.  Both of these variants make it even faster to write on materials like wood, wax and bone; the runes also take up less space—very useful for carvers writing on limited surfaces.

In any case, the names of the runes seem to have survived these developments.  Of the 16 runes found in both the English and Icelandic (which appears to be more conservative than the Norwegian) rune poems, 10—\textbf{f, r, h, n, i, j, s, b, m} and \textbf{l}—have etymologically identical names.  Three of the remaining six—\textbf{þ, a} and \textbf{t}—in the Icelandic stand for words with clear Heathen associations—Thurse, Os, and Tew—and so may have been changed deliberately after the conversion of England, rather than lost in the process of oral transmission.  Two more—\textbf{u} and \textbf{k}—have names which agree in form but not in meaning.  Thus it is only for the old \textbf{ʀ}-rune where there is complete disagreement about the original name.  This is easily understood, since the sound which that rune designated was lost in early Old English.

\section{The English Rune Poem}\chapterStart{}

\begin{flushright}%
\textbf{Dating:} 700s–C10th%TODO

\textbf{Meter:} \Fornyrdislag%para
\end{flushright}%

TODO: Introduction.  Preservation only in printed copy.

\sectionline


\bvg\bva%
ᚠ (feoh) byþ \alst{f}rofur \hld\ \alst{f}ira ge·hwylcum. &
Sceal ðeah \alst{m}anna ge·hwylc \hld\ \alst{m}iclun hyt dælan &
gif he wile for \alst{d}rihtne \hld\ \alst{d}ómes hleotan.\eva

\bvb TODO: TRANSLATION.\evb\evg


\bvg\bva%
ᚢ (ur) byþ \alst{â}n-mód \hld\ and \alst{o}fer-hyrned, &
\alst{f}ela-\alst{f}récne deor, \hld\ \alst{f}eohteþ mid hornum, &
\alst{m}ǽre \alst{m}ór-stapa; \hld\ þæt is \alst{m}ódig wuht.\eva

\bvb TODO: TRANSLATION.\evb\evg


\bvg\bva%
ᚦ (ðorn) byþ \alst{ð}earle scearp; \hld\ \alst{ð}egna ge·hwylcum &
\alst{a}n-feng ys \alst{y}fyl, \hld\ \alst{u}n-gemetun reþe &
\alst{m}anna ge·hwylcun \hld\ ðe him \alst{m}id resteð.\eva

\bvb TODO: TRANSLATION.\evb\evg


\bvg\bva%
ᚩ (os) byþ \alst{o}rd-fruma \hld\ \alst{æ}lcre spræce, &
\alst{w}ís-dómes \alst{w}raþu \hld\ and \alst{w}itena frofur, &
and \alst{eo}rla ge·hwam \hld\ \alst{ea}d-nys and to·hiht.\eva

\bvb TODO: TRANSLATION.\evb\evg


\bvg\bva%
ᚱ (rad) byþ on \alst{r}ecyde \hld\ \alst{r}inca ge·hwylcum &
\alst{s}efte, and \alst{s}wiþ-hwæt \hld\ ðam ðe \alst{s}itteþ on ufan &
\alst{m}eare \alst{m}ægen-heardum \hld\ ofer \alst{m}íl-paþas.\eva

\bvb TODO: TRANSLATION.\evb\evg


\bvg\bva%
ᚳ (cen) byþ \alst{c}wicera ge·hwam \hld\ \alst{c}u̇þ on fyre, &
\alst{b}lac and \alst{b}eorht-líc, \hld\ \alst{b}yrneþ oftust &
ðær hí \alst{æ}þelingas \hld\ \alst{i}nne restaþ.\eva

\bvb TODO: TRANSLATION.\evb\evg


\bvg\bva%
ᚷ (gyfu) \alst{g}umena byþ \hld\ \alst{g}leng and herenys, &
\alst{w}raþu and \alst{w}yrþ-scype, \hld\ and \alst{w}ræcna ge·hwam &
\alst{a}r and \alst{æ}twist \hld\ ðe byþ \alst{o}þra leas.\eva

\bvb TODO: TRANSLATION.\evb\evg


\bvg\bva%
ᚹ (wen) ne bruceþ \hld\ ðe can \alst{w}éana lýt, &
\alst{s}âres and \alst{s}orge, \hld\ and him \alst{s}ylfa hæfþ &
\alst{b}lǽd and \alst{b}lysse \hld\ and eac \alst{b}yrga ge·niht.\eva

\bvb TODO: TRANSLATION.\evb\evg


\bvg\bva%
ᚻ (hægl) byþ \alst{h}wítust corna; \hld\ hwyrft hit of \alst{h}eofones lyfte, &
\alst{w}ealcaþ hit \alst{w}indes scura, \hld\ weorþeþ hit to \alst{w}ætere syððan.\eva

\bvb TODO: TRANSLATION.\evb\evg


\bvg\bva%
ᚾ (nyd) byþ \alst{n}earu on breostan, \hld\ weorþeþ hi ðeah oft \alst{n}iþa bearnum &
to \alst{h}elpe and to \alst{h}æle ge·hwæþre, \hld\ gif hí his \alst{h}lystaþ æror.\eva

\bvb TODO: TRANSLATION.\evb\evg


\bvg\bva%
ᛁ (is) byþ \alst{o}fer-ceald, \hld\ \alst{u}n-ge·metum slidor, &
\alst{g}lisnaþ \alst{g}læs-hluttur, \hld\ \alst{g}immum ge·licust, &
\alst{f}lor \alst{f}orste ge·woruht, \hld\ \alst{f}æger an-sýne.\eva

\bvb TODO: TRANSLATION.\evb\evg


\bvg\bva%
ᛄ (ger) byþ \alst{g}umena hiht, \hld\ ðon \alst{G}od læteþ, &
\alst{h}âlig \alst{h}eofones cyning, \hld\ \alst{h}rusan syllan &
\alst{b}eorhte \alst{b}leda \hld\ \alst{b}eornum and ðearfum.\eva

\bvb TODO: TRANSLATION.\evb\evg


\bvg\bva%
ᛇ (eoh) byþ \alst{u}tan \hld\ \alst{u}n-smeþe treow, &
\alst{h}eard, \alst{h}rusan fæst, \hld\ \alst{h}yrde fyres, &
\alst{w}yrt-rumun under·\alst{w}reþyd, \hld\ \alst{w}ynan on éþle.\eva

\bvb TODO: TRANSLATION.\evb\evg


\bvg\bva%
ᛈ (peorð) byþ symble \hld\ \alst{p}lega and hlehter &
{[...]} \alst{w}lancum \hld\ ðar \alst{w}igan sittaþ &
on \alst{b}eor-sele \hld\ \alst{b}líþe æt·somne. \eva

\bvb TODO: TRANSLATION.\evb\evg


\bvg\bva%
ᛉ (eolhx)-secg \alst{ea}rd hæfþ \hld\ \alst{o}ftust on fenne, &
\alst{w}exeð on \alst{w}ature, \hld\ \alst{w}undaþ grimme, &
\alst{b}lode \alst{b}reneð \hld\ \alst{b}eorna ge·hwylcne &
ðe him \alst{æ}nigne \hld\ \alst{o}n-feng ge·deð.\eva

\bvb TODO: TRANSLATION.\evb\evg


\bvg\bva%
ᛋ (sigel) \alst{s}é-mannum \hld\ \alst{s}ymble biþ on hihte, &
ðonn hi hine \alst{f}eriaþ \hld\ ofer \alst{f}isces beþ, &
oþ hí \alst{b}rim-hengest \hld\ \alst{b}ringeþ to lande.\eva

\bvb TODO: TRANSLATION.\evb\evg


\bvg\bva%
ᛏ (tir) biþ \alst{t}âcna sum, \hld\ healdeð \alst{t}rywa wel &
wiþ \alst{æ}þelingas, \hld\ \alst{â} biþ on færylde, &
ofer \alst{n}ihta ge·\alst{n}ipu \hld\ \alst{n}æfre swiceþ.\eva

\bvb TODO: TRANSLATION.\evb\evg


\bvg\bva%
ᛒ (beorc) byþ \alst{b}leda leas, \hld\ \alst{b}ereþ efne swa ðeah &
\alst{t}ânas b·útan \alst{t}udder, \hld\ biþ on \alst{t}elgum wlitig, &
\alst{h}eah on \alst{h}elme \hld\ \alst{h}rysted fægere, &
ge·\alst{l}oden \alst{l}eafum, \hld\ \alst{l}yfte ge·tenge.\eva

\bvb TODO: TRANSLATION.\evb\evg


\bvg\bva%
ᛖ (eh) byþ for \alst{eo}rlum \hld\ \alst{æ}þelinga wyn, &
\alst{h}ors \alst{h}ófum wlanc, \hld\ ðær him \alst{h}æleþe ymb, &
\alst{w}elege on \alst{w}icgum, \hld\ \alst{w}rixlaþ spræce, &
and biþ \alst{u}n-styllum \hld\ \alst{æ}fre frofur.\eva

\bvb TODO: TRANSLATION.\evb\evg


\bvg\bva%
ᛗ (man) byþ on \alst{m}yrgþe \hld\ his \alst{m}agan leof; &
sceal þeah \alst{â}nra ge·hwylc \hld\ \alst{o}ðrum swícan, &
for ðam \alst{d}ryhten wyle \hld\ \alst{d}óme síne &
þæt \alst{ea}rme flæsc \hld\ \alst{eo}rþan be·tæcan.\eva

\bvb TODO: TRANSLATION.\evb\evg


\bvg\bva%
ᛚ (lagu) byþ \alst{l}eodum \hld\ \alst{l}ang-sum ge·þuht, &
gif hí sculun \alst{n}eþun \hld\ on \alst{n}acan tealtum, &
and hi \alst{s}æyþa \hld\ \alst{s}wýþe bregaþ, &
and se \alst{b}rim-hengest \hld\ \alst{b}ridles ne gymeð.\eva

\bvb TODO: TRANSLATION.\evb\evg


\bvg\bva%
ᛝ (ing) wæs \alst{æ}rest \hld\ mid Éast-Dęnum &
ge·\alst{s}ewen \alst{s}ęcgun, \hld\ oþ he \alst{s}iððan est &
ofer \alst{w}ǽg ge·\alst{w}ât, \hld\ \alst{w}æn æfter rann; &
ðus \alst{h}eardingas \hld\ ðone \alst{h}æle nęmdun.\eva

\bvb TODO: TRANSLATION.\evb\evg


\bvg\bva%
ᛟ (eþel) byþ \alst{o}fer-leof \hld\ \alst{æ}g·hwylcum men, &
gif he mot ðær \alst{r}ihtes \hld\ and ge·\alst{r}ysena on &
\alst{b}rúcan on \alst{b}lode \hld\ \alst{b}leadum oftast.\eva

\bvb TODO: TRANSLATION.\evb\evg


\bvg\bva%
ᛞ (dæg) byþ \alst{d}rihtnes sond, \hld\ \alst{d}eore mannum, &
\alst{m}ære \alst{m}etodes leoht, \hld\ \alst{m}yrgþ and to·hiht &
\alst{ea}dgum and \alst{ea}rmum, \hld\ \alst{ea}llum brice.\eva

\bvb TODO: TRANSLATION.\evb\evg


\bvg\bva%
ᚪ (ac) byþ on \alst{eo}rþan \hld\ \alst{ę}lda bearnum &
\alst{f}læsces \alst{f}odor, \hld\ \alst{f}ereþ ge·lome &
ofer \alst{g}anotes bæþ; \hld\ \alst{g}âr-sęcg fandaþ &
hwæþer \alst{â}c hæbbe \hld\ \alst{æ}þele treowe.\eva

\bvb TODO: TRANSLATION.\evb\evg


\bvg\bva%
ᚫ (æsc) biþ \alst{o}fer-heah, \hld\ \alst{ę}ldum dýre, &
\alst{st}iþ on \alst{st}aþule, \hld\ \alst{st}ede rihte hylt, &
ðeah him \alst{f}eohtan on \hld\ \alst{f}iras monige.\eva

\bvb TODO: TRANSLATION.\evb\evg


\bvg\bva%
ᚣ (yr) byþ \alst{æ}þelinga \hld\ and \alst{eo}rla ge·hwæs &
\alst{w}yn and \alst{w}yrþ-mynd, \hld\ byþ on \alst{w}icge fæger, &
\alst{f}æst-lic on \alst{f}ær-elde, \hld\ \alst{f}yrd-geatewa sum.\eva

\bvb TODO: TRANSLATION.\evb\evg


\bvg\bva%
ᛡ (iar, ior) byþ \alst{éa}-fixa, \hld\ and ðeah \alst{á} bruceþ &
\alst{f}ódres on \alst{f}oldan, \hld\ hafaþ \alst{f}ægerne eard, &
\alst{w}ætre be·\alst{w}orpen, \hld\ ðær he \alst{w}ynnum leofaþ.\eva

\bvb TODO: TRANSLATION.\evb\evg


\bvg\bva%
ᛠ (ear) byþ \alst{e}gle \hld\ \alst{e}orla ge·hwylcun, &
ðonn \alst{f}æst-lice \hld\ \alst{f}læsc on·ginneþ, &
\alst{h}raw colian, \hld\ \alst{h}rusan ceosan &
\alst{b}lac to ge·\alst{b}eddan; \hld\ \alst{b}leda ge·dreosaþ, &
\alst{w}ynna ge·\alst{w}itaþ, \hld\ \alst{w}era ge·swicaþ.\eva

\bvb TODO: TRANSLATION.\evb\evg

\sectionline

\section{The Icelandic Rune Poem}\chapterStart{}

\begin{flushright}%
\textbf{Dating:} Medieval.%TODO

\textbf{Meter:} Unclear.
\end{flushright}%

The poem is highly formulaic.  All lines begin with the respective rune’s name, followed by three kennings for it.  It is only attested in late manuscripts which often have major disagreements with each other.

\sectionline

\bvg\bva%
\alst{F}é es \alst{f}rę́nda róg \hld\ ok \alst{f}lǿðar viti &
\ind ok \alst{g}raf-sęiðs \alst{g}ata.\eva

\bvb Wealth is strife of kinsmen and beacon of the sea \\
\ind and grave-saithe’s \ken{serpent’s} street.\evb\evg


\bvg\bva%
Úr es \alst{sk}ýja grátr \hld\ ok \alst{sk}ára þvęrrir &
\ind ok \alst{h}irðis \alst{h}atr.\eva

\bvb Drizzle is weeping of clouds and ... \\
\ind and shepherd’s hatred.\evb\evg


\bvg\bva%
Þurs es \alst{k}venna \alst{k}vǫl \hld\ ok \alst{k}letta í·búi &
\ind ok \alst{v}arð-rúnar \alst{v}err.\eva

\bvb Thurse is women’s torment and indweller of hills \\
\ind and husband of the weird-whisperess \ken{giantess}.\evb\evg


\bvg\bva%
\alst{Ǫ́}ss es \alst{a}ldinn gautr \hld\ ok \alst{Ǫ́}s-garðs jǫfurr, &
\ind ok \alst{V}al-hallar \alst{v}ísi.\eva

\bvb Os is ancient Geat, and Osyard’s chief, \\
\ind and Walhall’s overseer.\evb\evg


\bvg\bva%
Ręið es \alst{s}itjandi \alst{s}ę́la \hld\ ok \alst{s}núðig fęrð &
\ind ok \alst{jó}s \alst{ę}rfiði.\eva

\bvb Chariot is sitting bliss and twirling journey \\
\ind and horse’s heavy work.\evb\evg


\bvg\bva%
Kaun es \alst{b}arna \alst{b}ǫl \hld\ ok \alst{b}ar-dagi &
\ind ok \alst{h}old-fúa \alst{h}ús.\eva

\bvb Boil is children’s curse and TODO \\
\ind and house of flesh-rot.\evb\evg


\bvg\bva%
Hagall es \alst{k}alda \alst{k}orn \hld\ ok \alst{k}nappa drífa &
\ind ok \alst{s}náka \alst{s}ótt.\eva

\bvb Hail is cold kernel and storm of beads \\
\ind and sickness of snakes.\evb\evg


\bvg\bva%
Nauð es \alst{þ}ýjar \alst{þ}rǫ́ \hld\ ok \alst{þ}ungr kostr &
\ind ok \alst{v}ás-samlig \alst{v}erk.\eva

\bvb Need is maidservant’s yearning and scant choice \\
\ind and working in wet-cold weather.\evb\evg


\bvg\bva%
\alst{Í}ss es \alst{á}ar bǫrkr \hld\ ok \alst{u}nnar þękja &
\ind ok \alst{f}ęigra manna \alst{f}ár.\eva

\bvb Ice is river’s bark and wave’s roof \\
\ind and fey men’s danger.\evb\evg


\bvg\bva%
Ár es \alst{g}umna \alst{g}óði \hld\ ok \alst{g}ótt sumar &
\ind \emph{ok} \alst{a}l-gróinn \alst{a}kr.\eva

\bvb Year is men’s boon and good summer \\
\ind (and) all-grown acre.\evb\evg


\bvg\bva%
Sól es \alst{sk}ýja \alst{sk}jǫldr \hld\ ok \alst{sk}ínandi rǫðull &
\ind ok \alst{í}sa \alst{a}ldr-tregi.\eva

\bvb Sun is the shield of clouds and shining wheel \\
\ind and ice-sheets’ life-sorrow.\evb\evg


\bvg\bva%
Týr es \alst{ęi}n-hęndr \alst{ǫ́}ss \hld\ ok \alst{u}lfs lęifar &
\ind ok \alst{h}ofa \alst{h}ilmir.\eva

\bvb Tew is the one-handed Os and the wolf’s leftovers \\
\ind and lord of hoves.\evb\evg


\bvg\bva%
Bjarkan es \alst{l}aufgat \alst{l}im \hld\ ok \alst{l}ítit tré &
\ind ok \alst{u}ng-samligr \alst{v}iðr.\eva

\bvb Birch is leafy branch and little tree \\
\ind and youthful wood.\evb\evg


\bvg\bva%
\alst{M}aðr es \alst{m}anns gaman \hld\ ok \alst{m}oldar auki &
\ind ok \alst{sk}ipa \alst{sk}ręytir.\eva

\bvb Man is man’s joy and the product of dust \\
\ind and adorner of ships.\evb\evg


\bvg\bva%
Lǫgr es \alst{v}ellanda \alst{v}atn \hld\ ok \alst{v}íðr kętill &
\ind ok \alst{g}lǫmmungr \alst{g}rund.\eva

\bvb Liquid is boiling water and wide kettle \\
\ind and TODO.\evb\evg


\bvg\bva%
Ýr es \alst{b}ęndr bogi \hld\ ok \alst{b}rot-gjarnt járn &
\ind ok \alst{f}ęnju \alst{f}lęygir.\eva

\bvb Yew is a bent bow and easily broken iron \\
\ind and arrow’s hurler.\evb\evg

\sectionline

\section{The Norwegian Rune Poem}\chapterStart{}

\begin{flushright}%
\textbf{Dating:} Medieval.%TODO

\textbf{Meter:} Unclear.
\end{flushright}%

The \textbf{Norwegian rune poem} is clearly very closely related to the Icelandic.  With the exception of runes 2 (\emph{úr} ‘slag’) and 4 (\emph{óss} ‘river-mouth’), the names of the runes are identical, as are many of the kennings used to describe them.

Still the language is unmistakably that of mediæval Norway.  As can be seen from the rhymes and alliteration the following uniquely Norwegian sound changes have occurred:
\begin{itemize}
  \item \emph{hl, hn, hr} > \emph{l, n, r} (2 \emph{lęypr} < \emph{hlęypr}; 8 \emph{nęppa} < \emph{hnęppa}; 5 \emph{rossum} < \emph{hrossum}).
  \item \emph{rst} > \emph{st} (5 \emph{vęsta} < \emph{vęrsta})
\end{itemize}

\sectionline

\bvg\bva%
ᚠ \alst{F}é vęldr \alst{f}rę́nda rógi; \hld\ \alst{f}ǿðisk ulfr í skógi.\eva

\bvb Wealth causes the strife of kinsmen; the wolf feeds itself in the wood.\evb\evg


\bvg\bva%
ᚢ \alst{Ú}r ’s af illu jarni; \hld\ \alst{o}pt lęypr ręinn á hjarni.\eva

\bvb TRANSLATION.\evb\evg


\bvg\bva%
ᚦ Þurs vęldr \alst{k}vinna \alst{k}villu; \hld\ \alst{k}átr verðr fár af illu.\eva

\bvb TRANSLATION.\evb\evg


\bvg\bva%
ᚬ Óss er \alst{f}lęstra \alst{f}ęrða \hld\ \alst{f}ǫr, en skalpr er sverða.\eva

\bvb River-mouth is the path of most journeys, and the scabbard-mouth is of swords.\evb\evg


\bvg\bva%
ᚱ \alst{R}ęið kveða \alst{r}ossum vęsta; \hld\ \alst{R}ęginn sló sverðit bęsta.\eva

\bvb Chariot they say is worst for horses; Rein struck the best sword.\evb\evg


\bvg\bva%
ᚴ Kaun er \alst{b}arna \alst{b}ǫlvan; \hld\ \alst{b}ǫl gørvir nán fǫlvan.\eva

\bvb TRANSLATION.\evb\evg


\bvg\bva%
ᚼ Hagall er \alst{k}aldastr \alst{k}orna; \hld\ \alst{K}ristr skóp hęiminn forna.\eva

\bvb Hail is coldest of kernels; Christ created the world of yore.\evb\evg


\bvg\bva%
ᚾ \alst{N}auðr gørir \alst{n}ęppa kosti; \hld\ \alst{n}øktan kęlr í frosti.\eva

\bvb TRANSLATION.\evb\evg


\bvg\bva%
ᛁ Ís kǫllum \alst{b}rú \alst{b}ręiða; \hld\ \alst{b}lindan þarf at lęiða.\eva

\bvb Ice we call a broad bridge; the blind man must be lead.\evb\evg


\bvg\bva%
ᛅ Ár er \alst{g}umna \alst{g}óði; \hld\ \alst{g}et’k at ǫrr var Fróði.\eva

\bvb Year is men’s boon; I recall that Frood was mad.\evb\evg


\bvg\bva%
ᛋ Sól er \alst{l}anda \alst{l}jómi; \hld\ \alst{l}úti’k hęlgum dómi.\eva

\bvb Sun is the light of the lands; I bow in the holy place.\evb\evg


\bvg\bva%
ᛏ Týr er \alst{ę}in-ęndr \alst{á}sa; \hld\ \alst{o}pt verðr smiðr blása.\eva

\bvb Tew is the one-handed of the Eese; the smith must often blow.\evb\evg


\bvg\bva%
ᛒ Bjarkan er \alst{l}auf-grǿnstr \alst{l}íma; \hld\ \alst{L}oki bar flę́rða tíma.\eva

\bvb TRANSLATION.\evb\evg


\bvg\bva%
ᛘ \alst{M}aðr er \alst{m}oldar auki; \hld\ \alst{m}ikil er gręip á hauki.\eva

\bvb Man is the product of dust; mighty is the grip on the hawk.\evb\evg


\bvg\bva%
ᛚ Lǫgr er er \alst{f}ęllr ór \alst{f}jalli \hld\ \alst{f}oss; en gull eru nossir.\eva

\bvb TRANSLATION.\evb\evg


\bvg\bva%
ᛦ Ýr er \alst{v}etr-grǿnstr \alst{v}iða; \hld\ \alst{v}ę́nt ’s, er brennr, at sviða. \eva

\bvb Yew is winter-greenest of trees; ’tis expected, when it burns, to get singed.\evb\evg

\sectionline
