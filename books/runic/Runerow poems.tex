\section{Introduction to the Rune Poems}

TODO: Acrophonic principle

The order and names of the letters in the Runic alphabets or \emph{futharks} stayed relatively consistent throughout the many centuries and countries in which they were used.  This can probably be ascribed to the \emph{rune poems}—poetic lists of the names of each rune with a short explanation, passed down orally as mnemonic devices to aid early Germanic learners, who were doubtless far more accustomed to learn by heart spoken poems than written letters.

Three such rune poems survive, from three countries: England, Norway, and Iceland.  The English rune poem documents the English \emph{futhorc}, while the Norwegian and Icelandic document the Scandinavian \emph{younger futhark}.

When compared to the Common Germanic \emph{elder futhark}, these two daughter scripts have taken opposing paths.  Whereas the English futhorc has appended several letters for new vowels to the end of the rune row, the Scandinavian futhark has instead done away with numerous runes, namely those for \emph{ng}, plosives \emph{d, g, p}, the semi-vowel \emph{w} and the vowels \emph{o} and \emph{e}, along with the obscure hook-shaped rune (TODO).  That much of this simplification was probably intentional, rather than the result of neglect or language change, is seen from the following facts.

First, several of the lost runes stood for sounds that did not undergo any major sound shifts in the North Germanic languages in the relevant time period.  For instance, all modern Scandinavian dialects still clearly distinguish between the initial consonants in the descendants of \emph{dagʀ} ‘day’ and \emph{Týr} ‘\inx[C]{Tew}’, and most even have the same articulation of these consonants as modern English.

Second, in two archaic runic inscriptions we find clear proof that the names and sound values of some of the lost runes were still remembered and passed down even after the adoption of the simplified younger futhark.  On the Swedish Rök stone (Ög 136), which is mostly composed in the younger futhark, runes of the elder futhark are used in a cipher, which works in the following way: Every younger futhark rune representing two distinct phonemes, where one of those was the sound value of that rune in the elder futhark system, and the other has been assimilated from a lost rune, is replaced by the elder futhark rune whose value it assimilated.  For instance, the \textbf{k} rune, which in the elder futhark stood for only /k/, but which in the younger futhark stands for both /k/ and /g/, is replaced with the old \textbf{g} rune.  A similar instance of two-scriptedness is found on the Ingelsta stone (Ög 43), where the old \textbf{d} rune is used in an otherwise younger futhark inscription, probably standing for its name \emph{dagʀ} ‘day’, which is also attested as a male given name.

Third, there is virtually no regional variation in which runes disappear in the transition from elder to younger futhark.  There is some variation in their shapes, but there is no region which, say, simplifies only the plosive consonants \emph{t/d, k/g, b/p} > \emph{t, k, b}, but retains the written distinction between \emph{o} and \emph{u}—they all go away at once.

These facts point away from neglect or a natural development of the script—they instead suggest deliberate reform.  Since we lack historical sources, the motivations behind such a reform can only be guessed at, but making the script simpler may have been intended to increase literacy by making it easier to learn and faster to write.  If this were the case it was certainly successful: the transition to the simplified younger futhark brings with it a huge increase in inscriptions in Scandinavia, along with interest in various ciphers, and a new tradition of inscribed stones in Denmark, where they were previously unknown.

This new system also quickly gave rise to even more simplified systems, like the “short-stave” runes found already on the C9th Rök stone, or the “staveless” runes known from northern Sweden.  Both of these variants make it even faster to write on materials like wood, wax and bone; the runes also take up less space—very useful for carvers writing on limited surfaces.

In any case, the names of the runes seem to have survived these developments.  Of the 16 runes found in both the English and Icelandic (which appears to be more conservative than the Norwegian) rune poems, 10—\textbf{f, r, h, n, i, j, s, b, m} and \textbf{l}—have etymologically identical names.  Three of the remaining six—\textbf{þ, a} and \textbf{t}—in the Icelandic stand for words with clear Heathen associations—Thurse, Os, and Tew—and so may have been changed deliberately after the conversion of England, rather than lost in the process of oral transmission.  Two more—\textbf{u} and \textbf{k}—have names which agree in form but not in meaning.  Thus it is only the old \textbf{ʀ}-rune where this is total disagreement about its ancient name.  This is easily understood, since the sound which that rune designated was lost in early Old English.

\bookStart{The English Rune Poem}

\begin{flushright}%
\textbf{Dating: }C8th–C10th%TODO

\textbf{Meter: }\Fornyrdislag%para
\end{flushright}%

TODO: Introduction.  Preservation only in printed copy.

\sectionline


\bvg\bva%
ᚠ (feoh) byþ frofur \hld\ fira ge·hwylcum. &
Sceal ðeah manna ge·hwylc \hld\ miclun hyt dælan &
gif he wile for drihtne \hld\ dómes hleotan.\eva

\bvb TODO: TRANSLATION.\evb\evg


\bvg\bva%
ᚢ (ur) byþ ân-mód \hld\ and ofer-hyrned, &
fela-frécne deor, \hld\ feohteþ mid hornum, &
mǽre mór-stapa; \hld\ þæt is módig wuht.\eva

\bvb TODO: TRANSLATION.\evb\evg


\bvg\bva%
ᚦ (ðorn) byþ ðearle scearp; \hld\ ðegna ge·hwylcum &
an·feng ys yfyl, \hld\ un-gemetun reþe &
manna ge·hwylcun \hld\ ðe him mid resteð.\eva

\bvb TODO: TRANSLATION.\evb\evg


\bvg\bva%
ᚩ (os) byþ ord-fruma \hld\ ælcre spræce, &
wís-dómes wraþu \hld\ and witena frofur, &
and eorla gehwam \hld\ ead-nys and to·hiht.\eva

\bvb TODO: TRANSLATION.\evb\evg


\bvg\bva%
ᚱ (rad) byþ on recyde \hld\ rinca ge·hwylcum &
sefte, and swiþ-hwæt \hld\ ðam ðe sitteþ on ufan &
meare mægen-heardum \hld\ ofer míl-paþas.\eva

\bvb TODO: TRANSLATION.\evb\evg


\bvg\bva%
ᚳ (cen) byþ cwicera ge·hwam \hld\ cu̇þ on fyre, &
blac and beorht-líc, \hld\ byrneþ oftust &
ðær hí æþelingas \hld\ inne restaþ.\eva

\bvb TODO: TRANSLATION.\evb\evg


\bvg\bva%
ᚷ (gyfu) gumena byþ \hld\ gleng and herenys, &
wraþu and wyrþ-scype, \hld\ and wræcna ge·hwam &
ar and ætwist \hld\ ðe byþ oþra leas.\eva

\bvb TODO: TRANSLATION.\evb\evg


\bvg\bva%
ᚹ (wen) ne bruceþ \hld\ ðe can wéana lýt, &
sâres and sorge, \hld\ and him sylfa hæfþ &
blǽd and blysse \hld\ and eac byrga ge·niht.\eva

\bvb TODO: TRANSLATION.\evb\evg


\bvg\bva%
ᚻ (hægl) byþ hwitust corna; \hld\ hwyrft hit of heofones lyfte, &
wealcaþ hit windes scura, \hld\ weorþeþ hit to wætere syððan.\eva

\bvb TODO: TRANSLATION.\evb\evg


\bvg\bva%
ᚾ (nyd) byþ nearu on breostan, \hld\ weorþeþ hi ðeah oft niþa bearnum &
to helpe and to hæle ge·hwæþre, \hld\ gif hí his hlystaþ æror.\eva

\bvb TODO: TRANSLATION.\evb\evg


\bvg\bva%
ᛁ (is) byþ ofer-ceald, \hld\ un-ge·metum slidor, &
glisnaþ glæs-hluttur, \hld\ gimmum ge·licust, &
flor forste ge·woruht, \hld\ fæger an-sýne.\eva

\bvb TODO: TRANSLATION.\evb\evg


\bvg\bva%
ᛄ (ger) byþ gumena hiht, \hld\ ðon God læteþ, &
hâlig heofones cyning, \hld\ hrusan syllan &
beorhte bleda \hld\ beornum and ðearfum.\eva

\bvb TODO: TRANSLATION.\evb\evg


\bvg\bva%
ᛇ (eoh) byþ utan \hld\ un-smeþe treow, &
heard, hrusan fæst, \hld\ hyrde fyres, &
wyrt-rumun under·wreþyd, \hld\ wynan on éþle.\eva

\bvb TODO: TRANSLATION.\evb\evg


\bvg\bva%
ᛈ (peorð) byþ symble \hld\ plega and hlehter &
{[...]} wlancum \hld\ ðar wigan sittaþ &
on beor-sele \hld\ blíþe æt·somne. \eva

\bvb TODO: TRANSLATION.\evb\evg


\bvg\bva%
ᛉ (eolhx)-secg eard hæfþ \hld\ oftust on fenne, &
wexeð on wature, \hld\ wundaþ grimme, &
blode breneð \hld\ beorna ge·hwylcne &
ðe him ænigne \hld\ on·feng ge·deð.\eva

\bvb TODO: TRANSLATION.\evb\evg


\bvg\bva%
ᛋ (sigel) sé-mannum \hld\ symble biþ on hihte, &
ðonn hi hine feriaþ \hld\ ofer fisces beþ, &
oþ hí brim-hengest \hld\ bringeþ to lande.\eva

\bvb TODO: TRANSLATION.\evb\evg


\bvg\bva%
ᛏ (tir) biþ tâcna sum, \hld\ healdeð trywa wel &
wiþ æþelingas, \hld\ a biþ on færylde, &
ofer nihta ge·nipu \hld\ næfre swiceþ.\eva

\bvb TODO: TRANSLATION.\evb\evg


\bvg\bva%
ᛒ (beorc) byþ bleda leas, \hld\ bereþ efne swa ðeah &
tânas b·útan tudder, \hld\ biþ on telgum wlitig, &
heah on helme \hld\ hrysted fægere, &
ge·loden leafum, \hld\ lyfte ge·tenge.\eva

\bvb TODO: TRANSLATION.\evb\evg


\bvg\bva%
ᛖ (eh) byþ for eorlum \hld\ æþelinga wyn, &
hors hofum wlanc, \hld\ ðær him hæleþe ymb, &
welege on wicgum, \hld\ wrixlaþ spræce, &
and biþ un-styllum \hld\ æfre frofur.\eva

\bvb TODO: TRANSLATION.\evb\evg


\bvg\bva%
ᛗ (man) byþ on myrgþe \hld\ his magan leof; &
sceal þeah ânra gehwylc \hld\ oðrum swícan, &
for ðam dryhten wyle \hld\ dóme síne &
þæt earme flæsc \hld\ eorþan be·tæcan.\eva

\bvb TODO: TRANSLATION.\evb\evg


\bvg\bva%
ᛚ (lagu) byþ leodum \hld\ lang-sum ge·þuht, &
gif hí sculun neþun \hld\ on nacan tealtum, &
and hi sæyþa \hld\ swýþe bregaþ, &
and se brim-hengest \hld\ bridles ne gymeð.\eva

\bvb TODO: TRANSLATION.\evb\evg


\bvg\bva%
ᛝ (ing) wæs ærest \hld\ mid Éast-Dęnum &
ge·sewen sęcgun, \hld\ oþ he siððan est &
ofer wǽg ge·wât, \hld\ wæn æfter rann; &
ðus heardingas \hld\ ðone hæle nęmdun.\eva

\bvb TODO: TRANSLATION.\evb\evg


\bvg\bva%
ᛟ (eþel) byþ ofer-leof \hld\ æg·hwylcum men, &
gif he mot ðær rihtes \hld\ and ge·rysena on &
brúcan on blode \hld\ bleadum oftast.\eva

\bvb TODO: TRANSLATION.\evb\evg


\bvg\bva%
ᛞ (dæg) byþ drihtnes sond, \hld\ deore mannum, &
mære metodes leoht, \hld\ myrgþ and to·hiht &
eadgum and earmum, \hld\ eallum brice.\eva

\bvb TODO: TRANSLATION.\evb\evg


\bvg\bva%
ᚪ (ac) byþ on eorþan \hld\ ęlda bearnum &
flæsces fodor, \hld\ fereþ ge·lome &
ofer ganotes bæþ; \hld\ gâr-sęcg fandaþ &
hwæþer ac hæbbe \hld\ æþele treowe.\eva

\bvb TODO: TRANSLATION.\evb\evg


\bvg\bva%
ᚫ (æsc) biþ ofer-heah, \hld\ ęldum dýre, &
stiþ on staþule, \hld\ stede rihte hylt, &
ðeah him feohtan on \hld\ firas monige.\eva

\bvb TODO: TRANSLATION.\evb\evg


\bvg\bva%
ᚣ (yr) byþ æþelinga \hld\ and eorla ge·hwæs &
wyn and wyrþ-mynd, \hld\ byþ on wicge fæger, &
fæst-lic on fær-elde, \hld\ fyrd-geatewa sum.\eva

\bvb TODO: TRANSLATION.\evb\evg


\bvg\bva%
ᛡ (iar, ior) byþ éa-fixa, \hld\ and ðeah a bruceþ &
fódres on foldan, \hld\ hafaþ fægerne eard, &
wætre be·worpen, \hld\ ðær he wynnum leofaþ.\eva

\bvb TODO: TRANSLATION.\evb\evg


\bvg\bva%
ᛠ (ear) byþ egle \hld\ eorla ge·hwylcun, &
ðonn fæst-lice \hld\ flæsc on·ginneþ, &
hraw colian, \hld\ hrusan ceosan &
blac to gebeddan; \hld\ bleda ge·dreosaþ, &
wynna ge·witaþ, \hld\ wera ge·swicaþ.\eva

\bvb TODO: TRANSLATION.\evb\evg

\sectionline

\bookStart{The Icelandic Rune Poem}

\begin{flushright}%
\textbf{Dating: }Medieval.%TODO

\textbf{Meter: }Unclear.
\end{flushright}%

The poem is highly formulaic.  All lines begin with the respective rune’s name, followed by three synonyms.  It is only attested in late manuscripts which often have major disagreements with each other.

\sectionline

\bvg\bva \alst{F}é es \alst{f}rę́nda róg \hld\ ok \alst{f}lǿðar viti &
\ind ok \alst{g}raf-sęiðs \alst{g}ata.\eva

\bvb Fee is strife of kinsmen and beacon of the sea \\
and grave-saithe’s \ken{serpent’s} street.\evb\evg


\bvg\bva Úr es \alst{sk}ýja grátr \hld\ ok \alst{sk}ára þvęrrir &
\ind ok \alst{h}irðis \alst{h}atr.\eva

\bvb Drizzle is weeping of clouds and ... \\
and shepherd’s hatred.\evb\evg


\bvg\bva Þurs es \alst{k}venna \alst{k}vǫl \hld\ ok \alst{k}letta í·búi &
\ind ok \alst{v}arð-rúnar \alst{v}err.\eva

\bvb Thurse is women’s torment and indweller of hills \\
and husband of the weird-whisperess \ken{giantess}.\evb\evg


\bvg\bva \alst{Ǫ́}ss es \alst{a}ldinn gautr \hld\ ok \alst{Ǫ́}s-garðs jǫfurr, &
\ind ok \alst{V}al-hallar \alst{v}ísi.\eva

\bvb Os is ancient Geat, and Osyard’s chief, \\
and Walhall’s overseer.\evb\evg


\bvg\bva Ręið es \alst{s}itjandi \alst{s}ę́la \hld\ ok \alst{s}núðig fęrð &
\ind ok \alst{jó}s \alst{ę}rfiði.\eva

\bvb Chariot is sitting bliss and twirling journey \\
and horse’s heavy work.\evb\evg


\bvg\bva Kaun es \alst{b}arna \alst{b}ǫl \hld\ ok \alst{b}ar-dagi &
\ind ok \alst{h}old-fúa \alst{h}ús.\eva

\bvb Boil is children’s curse and TODO \\
and house of flesh-rot.\evb\evg


\bvg\bva Hagall es \alst{k}alda \alst{k}orn \hld\ ok \alst{k}nappa drífa &
\ind ok \alst{s}náka \alst{s}ótt.\eva

\bvb Hail is cold kernel and storm of beads \\
and sickness of snakes.\evb\evg


\bvg\bva Nauð es \alst{þ}ýjar \alst{þ}rǫ́ \hld\ ok \alst{þ}ungr kostr &
\ind ok \alst{v}ás-samlig \alst{v}erk.\eva

\bvb Need is maidservant’s yearning and scant choice \\
and working in wet-cold weather.\evb\evg


\bvg\bva \alst{Í}ss es \alst{á}ar bǫrkr \hld\ ok \alst{u}nnar þękja &
\ind ok \alst{f}ęigra manna \alst{f}ár.\eva

\bvb Ice is river’s bark and wave’s roof \\
and fey men’s danger.\evb\evg


\bvg\bva Ár es \alst{g}umna \alst{g}óði \hld\ ok \alst{g}ótt sumar &
\ind \emph{ok} \alst{a}l-gróinn \alst{a}kr.\eva

\bvb Year is men’s boon and good summer \\
(and) all-grown acre.\evb\evg


\bvg\bva Sól es \alst{sk}ýja \alst{sk}jǫldr \hld\ ok \alst{sk}ínandi rǫðull &
\ind ok \alst{í}sa \alst{a}ldr-tregi.\eva

\bvb Sun is the shield of clouds and shining wheel \\
and life-grief of ice.\evb\evg


\bvg\bva Týr es \alst{ęi}n-hęndr \alst{ǫ́}ss \hld\ ok \alst{u}lfs lęifar &
\ind ok \alst{h}ofa \alst{h}ilmir.\eva

\bvb Tew is the one-handed Os and the wolf’s leftovers \\
and lord of hoves.\evb\evg


\bvg\bva Bjarkan es \alst{l}aufgat \alst{l}im \hld\ ok \alst{l}ítit tré &
\ind ok \alst{u}ng-samligr \alst{v}iðr.\eva

\bvb Birch is leafy branch and little tree \\
and youthful wood.\evb\evg


\bvg\bva \alst{M}aðr es \alst{m}anns gaman \hld\ ok \alst{m}oldar auki &
\ind ok \alst{sk}ipa \alst{sk}ręytir.\eva

\bvb Man is man’s joy and the product of dust \\
and adorner of ships.\evb\evg


\bvg\bva Lǫgr es \alst{v}ellanda \alst{v}atn \hld\ ok \alst{v}íðr kętill &
\ind ok \alst{g}lǫmmungr \alst{g}rund.\eva

\bvb Liquid is boiling water and wide kettle \\
and TODO.\evb\evg


\bvg\bva Ýr es \alst{b}ęndr bogi \hld\ ok \alst{b}rot-gjarnt járn &
\ind ok \alst{f}ęnju \alst{f}lęygir.\eva

\bvb Yew is a bent bow and easily broken iron \\
and arrow’s hurler.\evb\evg

\sectionline

\bookStart{The Norwegian Rune Poem}

\begin{flushright}%
\textbf{Dating: }Medieval.%TODO

\textbf{Meter: }Unclear.
\end{flushright}%

The poem is generally the same as the Icelandic, but there are some differences.

The language is clearly medieval, and has a few uniquely Norwegian sound changes.  That these are not just scribal is seen by the meter.
\begin{itemize}
  \item \emph{h-} has been lost before \emph{l, n} and \emph{r} (st. 2 \emph{lęypr} < \emph{hlęypr}; st. 8 \emph{nęppa} < \emph{hnęppa}; st. 5 \emph{rossum} < \emph{hrossum}).
  \item \emph{rst} has become \emph{st} (st. 5 \emph{vęsta} < \emph{vęrsta})
\end{itemize}

\sectionline

\bvg\bva ᚠ Fé vęldr frę́nda rógi; \hld\ fǿðisk ulfr í skógi.\eva

\bvb TRANSLATION.\evb\evg


\bvg\bva ᚢ Úr ’s af illu jarni; \hld\ opt lęypr ręinn á hjarni.\eva

\bvb TRANSLATION.\evb\evg


\bvg\bva ᚦ Þurs vęldr kvinna kvillu; \hld\ kátr verðr fár af illu.\eva

\bvb TRANSLATION.\evb\evg


\bvg\bva ᚬ Óss er flę́stra fęrða \hld\ fǫr; en skalpr er sverða.\eva

\bvb TRANSLATION.\evb\evg


\bvg\bva ᚱ Ręið kveða rossum vęsta; \hld\ Ręginn sló sverðit bęsta.\eva

\bvb TRANSLATION.\evb\evg


\bvg\bva ᚴ Kaun er barna bǫlvan; \hld\ bǫl gørvir nán fǫlvan.\eva

\bvb TRANSLATION.\evb\evg


\bvg\bva ᚼ Hagall er kaldastr korna; \hld\ Kristr skóp hęiminn forna.\eva

\bvb TRANSLATION.\evb\evg


\bvg\bva ᚾ Nauðr gørir nęppa kosti; \hld\ nøktan kęlr í frosti.\eva

\bvb TRANSLATION.\evb\evg


\bvg\bva ᛁ Ís kǫllum brú bręiða; \hld\ blindan þarf at lęiða.\eva

\bvb Ice we call a broad bridge; the blind man must be lead.\evb\evg


\bvg\bva ᛅ Ár er gumna góði; \hld\ get’k at ǫrr var Fróði.\eva

\bvb Year is men’s boon; I recall that mad was Frood.\evb\evg


\bvg\bva ᛋ Sól er landa ljómi; \hld\ lúti’k hęlgum dómi.\eva

\bvb TRANSLATION.\evb\evg


\bvg\bva ᛏ Týr er ęin-ęndr ása; \hld\ opt verðr smiðr blása.\eva

\bvb Tew is the one-handed of the Eese; TODO.\evb\evg


\bvg\bva ᛒ Bjarkan er lauf-grǿnstr líma; \hld\ Loki bar flę́rða tíma.\eva

\bvb TRANSLATION.\evb\evg


\bvg\bva ᛘ Maðr er moldar auki; \hld\ mikil er gręip á hauki.\eva

\bvb Man is the product of dust; great is the grip on the hawk..\evb\evg


\bvg\bva ᛚ Lǫgr er, fęllr ór fjalli \hld\ foss; en gull eru nossir.\eva

\bvb TRANSLATION.\evb\evg


\bvg\bva ᛦ Ýr er vetr-grǿnstr viða; \hld\ vę́nt ’s, er brennr, at sviða. \eva

\bvb TRANSLATION.\evb\evg

\sectionline
