TODO: Introduction to Swedish inscriptions

\sectionline

\section{Sm 16}

\begin{flushright}%
\textbf{Dating:} 1000s

\textbf{Meter:} \Fornyrdislag
\end{flushright}%

TODO.

\sectionline

\bvg\bva[]%
Hróstęinn auk \alst{Ęi}lífʀ, \hld\ \alst{Á}ki auk Hǫ́kon, &
ręistu þęiʀ \alst{s}vęinaʀ \hld\ ęptiʀ \alst{s}inn faður &
\alst{k}umbl \alst{k}ęnni-ligt \hld\ ęptiʀ \alst{K}ala dauðan. &
Þý mun \alst{g}óðs manns \hld\ um \alst{g}etit verða, &
með \alst{st}ęinn lifiʀ \hld\ ok \alst{st}afiʀ rúna.\eva

\bvb Rothstan and Anlif, Eke and Hathkin, \\
those lads raised after their father \\
a remarkable monument after the dead Cale. \\
Thus will the good man be spoken of, \\
while the stone lives and the staves of the runes.\evb\evg

\sectionline

\section{Sm 39}

\begin{flushright}%
\textbf{Dating:} 1000s

\textbf{Meter:} \Fornyrdislag
\end{flushright}%

A standing stone inscribed on two sides, one of which has a large cross.  The expression is formulaic; cf. Sm 44, Sö 130, U 703, U 739, and U 805.  For “\inx[C]{good of meat}”, which also occurs in \Havamal; see Encyclopedia.  The first line is not poetic.

\sectionline

\bvg\bva[]%
Gunni satti stên þęnna eptiʀ Súna, fǫður sinn, &
\alst{m}ildan orða \hld\ ok \alst{m}ataʀ góðan.\eva

\bvb Guthe set this stone after Sown, his father, \\
mild of words and good of meat.\evb\evg

\sectionline

\section{Sm 44}

\begin{flushright}%
\textbf{Dating:} 1000s

\textbf{Meter:} \Fornyrdislag
\end{flushright}%

TODO.  The expression is formulaic; cf. Sm 39, Sö 130, U 703, U 739, and U 805.

\sectionline

\bvg\bva[]%
TODO
mildan við sinna \hld\ ok mataʀ góðan, &
TODO.\eva

\bvb TODO \\
Mild with his men and good of meat. \\
TODO\evb\evg

\sectionline

\section{Sö 34–35 (Tjuvstigen)}

\begin{flushright}%
\textbf{Dating:} 1000–1100s

\textbf{Meter:} \Fornyrdislag
\end{flushright}%

Two paired stones standing next to each other.  The last line of Sö 35 is not poetic.

\sectionline

\bvg\bva[Sö 34]%
\alst{St}yrlaugʀ ok Holmbʀ \hld\ \alst{st}ęina ręistu &
at \alst{b}rǿðr sína, \hld\ \edtrans{\alst{b}rautu nę́sta}{nearest to the road}{\Bfootnote{Cf. \Havamal\ TODO.}}. &
Þęir \alst{ę}ndaðus \hld\ í \alst{au}str-vegi, &
\alst{Þ}órkęll ok Styrbjǫrn, \hld\ \alst{þ}iagnar góðir.\eva

\bvb Sturley and Holm raised the stones, \\
after their brothers, nearest to the road. \\
They were ended in the Eastway, \\
Thurkettle and Sturbern, good thanes.\evb\evg


\bvg\bva[Sö 35]%
Lét \alst{I}ngigęiʀʀ \hld\ \alst{a}nnan ręisa stęin &
at \alst{s}onu \alst{s}ína, \hld\ \alst{s}ýna giǫrði.
Guð hjalpi ǫnd þęira. Þóriʀ hjó.\eva

\bvb Inggar let raise another stone, \\
after his sons made visible. \\
God may help their spirit. Thurer hewed.\evb\evg

\sectionline

\section{Sö 56 (Fyrby)}

\begin{flushright}%
\textbf{Dating:} 1000–1100s

\textbf{Meter:} \Fornyrdislag
\end{flushright}%

TODO: INTRODUCTION.

\sectionline

\bvg\bva[]%
Iak vęit \alst{H}á-stęin \hld\ þá \alst{H}olm-stęin brǿðr &
\alst{m}ęnnr rýnasta \hld\ á \alst{M}ið-garði &
sęttu \alst{st}ęin \hld\ auk \alst{st}afa marga &
eptir \alst{F}ręy-stęin \hld\ \alst{f}ǫður sinn.\eva

\bvb I know Highstan and Holmstan, those brothers, \\
the men most rune-cunning in Middenyard; \\
they set the stone and many staves, \\
after Freestan, their father.\evb\evg

\sectionline

\section{Sö 65 (Djulefors)}

\begin{flushright}%
\textbf{Dating:} 1000–1100s

\textbf{Meter:} \Fornyrdislag\ with hendings in the b-verses
\end{flushright}%

A standing stone inscribed on one side with a large cross.  Already on the earliest depictions the stone was damaged, but an even larger part has now gone missing.  Other stones that mention \inx[L]{Longbeardland} (Lombardy) include TODO...  The meter is highly unusual for runic Swedish poetry, relying on hendings (in line 2 an ethel-hending \emph{arð- : barð-}, in line 3 a shot-hending \emph{land- : ęnd-}).  Line 2b is formulaic; see note.

\sectionline

\bvg\bva[]%
Inga ręisti stęin þannsi at Ǫ́lęif sinn \textbf{a}... &
Hann austarla \hld\ \edtrans{arði barði}{ploughed with the prow}{\Bfootnote{i.e. “sailed”.  A formulaic poetic expression shared with an anonymous line from the Third Grammatical Treatise, which reads: \emph{sá’s af Íslandi \hld\ arði barði} ‘he who [awawy] from Iceland ploughed with the prow’.}} &
auk ȧ Langbarði- \hld\ landi ęndaðis.\eva

\bvb Inge raised this stone after Anlaf, her ... . \\
Easterly he ploughed with the prow, \\
and on Longbeardland was ended.\evb\evg

\sectionline

\section{Sö 130}

\begin{flushright}%
\textbf{Dating:} 1000–1100s

\textbf{Meter:} \Fornyrdislag
\end{flushright}%

A standing stone. TODO.  The expression is formulaic; cf. Sm 39, Sm 44, U 703, U 739, and U 805.

\sectionline

\bvg\bva[]%
\alst{F}iuriʀ gęrðu \hld\ at \alst{f}ǫður góðan &
\alst{d}ýrð \alst{d}ręngi-la \hld\ at \alst{D}ómara &
\alst{m}ildan orða \hld\ ok \alst{m}ataʀ góðan. &
Þat \dots\eva

\bvb Four men made after a good father, \\
an honour, valiantly, after Doomer, \\
mild of words and good of meat. \\
This \dots\evb\evg

\sectionline

\section{Sö 179 (Gripsholm)}

\begin{flushright}%
\textbf{Dating:} 1000s

\textbf{Meter:} \Fornyrdislag
\end{flushright}%

TODO: INTRODUCTION.  The three-line stanza is a biographical addition following a typical prose memorial formula.

\sectionline

\bpg\bpa[] Tóla lét ręisa stęin þennsa at son sinn Harald, bróður Ingvars.\epa

\bpb Tool let raise this stone after his son Harold, brother of Ingwar.\epb\epg

\bvg\bva[]%
Þęiʀ \alst{f}óru dręngi-la \hld\ \alst{f}iarri at gulli &
ok \alst{au}star-la \hld\ \alst{ę}rni gǫ́fu, &
dóu \alst{s}unnar-la \hld\ á \alst{S}ęrk-landi.\eva

\bvb They journeyed valiantly far for gold, \\
and easterly gave to the eagle; \\
died southerly in Serkland.\evb\evg

\sectionline

\section{U 703}

\begin{flushright}%
\textbf{Dating:} 1000s

\textbf{Meter:} \Fornyrdislag
\end{flushright}%

A standing stone inscribed on one side.  There is no cross present, but a large four-legged beast with a long tail.  The stone is heavily damaged, but mostly readable, except for what is here taken to be the half of line 2, which is entirely lost.  The expression is formulaic; cf. Sm 39, Sm 44, Sö 130, U 739, and U 805.  For “\inx[C]{good of meat}”, which also occurs in \Havamal; see Encyclopedia.  The first line is not poetic.

\sectionline

\bvg\bva[]%
Ásvi lét ręisa stęin þennsa at Ǫrnulf, son sinn góðan. &
Hann byggi hér \hld\ ..., &
\alst{m}andr \alst{m}atar góðr \hld\ ok \alst{m}áls risinn.\eva

\bvb Oswye let raise this stone after Arnolf, her good son. \\
He dwelled here ..., \\
a man good of meat and proud of speech.\evb\evg

\sectionline

\section{U 739}

\begin{flushright}%
\textbf{Dating:} 1000s

\textbf{Meter:} \Fornyrdislag
\end{flushright}%

A standing stone inscribed on one side, with a large cross present.  There are no major difficulties with the reading.  The expression is formulaic; cf. Sm 39, Sm 44, Sö 130, U 703, and U 805.  “mild of meat” appears to be a variant of “\inx[C]{good of meat}”, which also occurs in \Havamal; see Encyclopedia.  The first line is not poetic.  For other stones raised by someone in memory of themselves, see TODO.

\sectionline

\bvg\bva[]%
Holbjǫrn lét ręisa stęin at sik sjalfan. &
Hann vaʀ \alst{m}ildr \alst{m}ataʀ \hld\ ok \alst{m}áls risinn.\eva

\bvb Holbern let raise this stone after himself. \\
He was mild of meat and proud of speech.\evb\evg

\sectionline

\section{U 805}

\begin{flushright}%
\textbf{Dating:} 1000s

\textbf{Meter:} \Fornyrdislag
\end{flushright}%

The stone has been lost, and only survives in old depictions, which makes the reading, especially two of the personal names, uncertain.  My transliteration follows Rundata.

The expression is formulaic; cf. Sm 39, Sm 44, Sö 130, U 703, and U 739.  For “\inx[C]{good of meat}”, which also occurs in \Havamal; see Encyclopedia.  The first line is not poetic.

\sectionline

\bvg\bva[]%
Fylkir lét ręisa st\emph{ęin epti}r \textbf{iel}, bróður sinn, ok Gunnmarr eptir \textbf{menk}, fǫður sinn, &
\alst{b}ónda góðan matar; \hld\ \alst{b}yggi í Víkbý.\eva

\bvb Filch let raise this stone after ..., his brother, and Guthmar after ..., his father, \\
a farmer good of meat; he lived in Wickby.\evb\evg

\sectionline
