\bookStart{Speeches of the High One}[Hávamǫ́l]

\begin{flushright}%
\textbf{Dating:} See individual sections.

\textbf{Meter:} \Ljodahattr, \Galdralag, \Fornyrdislag%para
\end{flushright}%

%Introduction.

The \textbf{Speeches of the High One} is the second poem of \Regius, which is the only medieval witness manuscript.  Several sts. are however cited or alluded to in other places, such as Eyv \emph{Hák} (TODO: formatting) 21 and \FostrbroedhraSaga\ TODO.

The poem before us does not very much seem like a single composition by one poet, but instead much more like a collection of scattered traditional poetry associated with the god Weden.  It seems to contain at least two poems of practical life advice, two mythological narratives, scattered gnomic poetry about runes, and a list of galders.  These various strands are united by their presumed speaker, namely Weden in His function as God of Wisdom.

Following previous authors, I identify the following strands, excepting various lone sts. that are probably later inserts.  In the present edition each of the following is given a separate, short introduction:

\begin{enumerate}
	\item 1–79 The Guest-strand; practical life advice, beginning with a guest arriving at a homestead
  \item 81–90 Various scattered sts. of advice
  \item 91–102 Weden’s failed seduction of Billing’s daughter
  \item 103–110 Weden’s obtaining of the Mead of Poetry
  \item 111–137 The Speeches of Loddfathomer; Weden’s advice to Loddfathomer
  \item 138–146 The Rune-tally; various sts. relating to runes and their magical use
  \item 146–165 The Leed-tally; Weden’s listing of 18 galders
\end{enumerate}

Two questions shortly arise: who was the redactor (i.e., the person who set these strands together, and gave the new work the title \emph{Háva mǫ́l}), and what was his motive?  While a detailed and sufficient answer will probably never be found, a careful reading of the final stanza, 165, gives us some clues.  By its prayer-like blessing, which brings up the Heathen dichotomy between the Gods and Ettins (the friends and enemies of Mankind, respectively) and calls the contents of the poem (which include unambiguous Heathen ritual instructions) “very useful” (\emph{all-þǫrf}); and by its reference to the process of oral transmission, the whole poem in something resembling the current form must (it seems) have been put together no later than the early 11th century, in a pre-scribal, pre-monastic, Heathen context. (Iceland converted around year 1000, but people surely clung to the old traditions for some time longer.)

As seen by the emphasis on the usefulness of the poetry, the reason for this redaction was not strictly antiquarian, but foremost utilitarian; the redactor gathered an amount of traditional poetry he found useful (whether for its life-advice or mythology) into a single poem, which could then be learned by heart by anyone.  In this he certainly achieved his goal.  The \Havamal\ is by far the greatest surviving collection of pre-Christian Norse advice poetry, and has functioned like a Noah’s Ark—or Hoardmimer’s Wood—for that genre.  Thus, those scattered stanzas which were not included by the redactor—and many must have existed—are now forever lost.

\sectionline

\section{The Guest-strand (sts. 1–79)}

The Guest-Strand (Old Norse: \emph{Gęsta-þáttr}) is one of the most interesting surviving works of Norse poetry.  Sadly, its structure has been obscured by the insertion of unrelated sts. and by poor translations.  My hope is to shed some light on the original coherence of the strand, while respecting the text as it appears in the manuscript.  As I do not think it can do each stanza justice, and since there is not exactly a clear progression of themes, I will not here attempt a stanza-by-stanza summary of this strand. Rather, I will give some important observations and then let the reader read for himself.

The Strand is a piece of advice poetry, and takes its outset in a wanderer’s arriving as a guest at a Norse farmstead.  It first (roughly sts. 1–4) discusses the mutual responsibilites between guest and host, and then moves on to broader human interactions, with a particular focus on alcohol, war, friendship and human wisdom.  While there is some coherence and nice transitions are frequently employed in order to shift from one theme to another (e.g. between sts. 4 and 5, or 10 and 11), the poem is not clearly divided into sections, nor is there (after the very first stanzas) a linear progression from one theme to another.

At all turns the poem advices caution and shrewdness.  A man should always carry his “manwit” (ON \emph{man-vit}, a word somewhat analogous with the English “common sense”) with him; he should think before he speaks

The poem moves seamlessly between various parts of life.  To do so the poet often employs transitions where a st. repeats the structure of the previous one, but with a new subject.  This is particularly evident in sts. 4–5 and 10–11.

TODO.

\sectionline

\bvg\bva\alst{G}ȧttir allar \hld\ áðr \alst{g}angi framm &
\ind \edtext{of \alst{sk}oðask \alst{sk}yli,}{\Bfootnote{om. \GylfMS}} &
\ind of \alst{sk}yggnask \alst{sk}yli; &
því-at \alst{ȯ}-víst ’s at vita, \hld\ hvar \alst{ȯ}-vinir &
\ind sitja ȧ \alst{f}lęti \alst{f}yrir.\eva

\bvb All doorways—before one might go forth \\
\ind he should spy round; \\
\ind he should pry round; \\
for it’s unsure to know where enemies \\
\ind sit on the benches within.\evb\evg


\bvg\bva \alst{G}efęndr hęilir, \hld\ \alst{g}ęstr ’s inn kominn, &
\ind hvar skal \alst{s}itja \alst{s}já? &
mjǫk es \alst{b}ráðr \hld\ sá’s \edtrans{ȧ \alst{b}rǫndum}{on the fires}{\Bfootnote{Possibly referring a Norwegian folk custom, wherein a guest would sit down on the wood-pile outside of the door, waiting until being let in; see further TODO SOME ARTICLE on this custom.  The speaker is announcing to the hosts (or “givers”) that a guest, frozen, wet and tired, is currently sitting on the wood-pile, and ought to be let in.}} skal &
\ind \edtrans{síns of \alst{f}ręista \alst{f}rama}{tempt his furtherance}{\Bfootnote{i.e. try his luck; see how far he gets.  The line is formulaic; cf. \Vafthrudnismal\ 11, 13, 15, 17.}}.\eva

\bvb O givers, hail! A guest has come in; \\
\ind where shall this one sit? \\
Very anxious is he who on the fires shall \\
\ind tempt his furtherance.\evb\evg


\bvg\bva\alst{Ę}lds es þǫrf \hld\ þęim’s \alst{i}nn es kominn &
\ind ok ȧ \alst{k}néi \alst{k}alinn, &
\alst{m}atar ok váða \hld\ es \alst{m}anni þǫrf, &
\ind þęim’s hęfr of \alst{f}jall \alst{f}arit.\eva

\bvb Of fire there is need for the one who is come in, \\
\ind and cold about the knees; \\
of food and of clothing there is need for the man \\
\ind who over the fell has fared.\evb\evg


\bvg\bva\edtext{\alst{V}ats es þǫrf \hld\ þęim’s til \alst{v}erðar kømr, &
\ind \alst{þ}ęrru ok \alst{þ}jóð-laðar, &
\alst{g}óðs of ǿðis, \hld\ —ef sér \alst{g}eta mę́tti— &
\ind \alst{o}rðs ok \alst{ę}ndr-þǫgu.}{\lemma{ALL}\Bfootnote{There is a good train of thought throughout the st.: the guest must first wash and dry himself, and then be welcomed to sit and eat at the table.  After the host has provided these amenities the responsibility shifts onto the guest, who must now speak.

The word \emph{ęndr-þaga} ‘silence in return’ leads a nice transition to the rest of the Strand, where proper social conduct (encompassed by the first word of the next stanza below, “wit”) will be discussed more broadly.  One may note that the verb \emph{þęgja} ‘shut up, be silent’ (of which \emph{*þaga}, which only appears in the present compound, is a derivative, formed in the same way as \emph{saga} ‘saw, history, story’ to \emph{sęgja} ‘say, speak’) and its derivative \emph{þǫgn} ‘silence’ are frequently used by Scoldic poets to mark the very beginning of their works (e.g. Arn \emph{Magndr} 1\textsuperscript{II}: \emph{þegi sęim-brotar} ‘may gold-breakers \ken{generous men} be silent’, Egill \emph{Berdr} 1\textsuperscript{V}: \emph{hyggi \dots\ til þagnar þinn lýðr} ‘may thy retinue focus on silence’, Glúmr \emph{Gráf} 1\textsuperscript{I}: \emph{biðjum vér þagnar} ‘we ask for silence’).}}\eva

\bvb Of water there is need for the one who comes for a meal; \\
\ind of a towel and a hearty welcome; \\
of a good reception—if he might get one— \\
\ind of speech, and silence in return.\evb\evg


\bvg\bva\alst{V}its es þǫrf \hld\ þęim’s \alst{v}íða ratar; &
\ind dę́lt es \alst{h}ęima \alst{h}vat; &
\edtrans{at \alst{au}ga-bragði}{Into a laughing-stock}{\Bfootnote{Idomatic.  \emph{auga-bragð} literally means ‘twinkling of an eye, moment’; the sense here is thus something like ‘a quick glance of derision’.}} \hld\ verðr sá’s \alst{ę}kki kann &
\ind ok með \alst{s}notrum \alst{s}itr.\eva

\bvb Of wit there is need for the one who widely roams; \\
\ind everything is easy at home. \\
Into a laughing-stock turns he who nothing knows, \\
\ind and among the clever sits.\evb\evg


\bvg\bva At \alst{h}yggjandi sinni \hld\ skyli-t maðr \alst{h}rǿsinn vesa, &
\ind hęldr \alst{g}ę́tinn at \alst{g}ęði, &
þȧ’s \alst{h}orskr ok þǫgull \hld\ kømr \alst{h}ęimis-garða til, &
\ind sjaldan verðr \alst{v}íti \alst{v}ǫrum. &
því-at \alst{ȯ}-brigðra vin \hld\ fę̇r \edtrans{maðr}{man}{\Bfootnote{In \Regius\ abbreviated with the rune ᛘ \textbf{m} “man”, the first of 45 such instances in the present poem.  While Anglo-Saxon Latin-script mss. use several runes ideographically (e.g. ᛟ \textbf{o} for OE \emph{ǿðel} ‘homeland, patrimony’), there are (to my knowledge) no Scandinavian examples with runes other than ᛘ.  The tradition of ideographic runes standing for their names is ancient and goes back to the time before Latin writing, as proven by the inscriptions from Stentoften (DR 357) and Ingelstad (Ög 43), which use the runes ᛃ \textbf{j} for \emph{ár} ‘year, good harvest’ and ᛞ \textbf{d} for \emph{dagʀ} ‘day’, respectively.  For rune names see below: Anonymous Runerow Poems.}} \alst{a}ldri-gi, &
\ind an \alst{m}an-vit \alst{m}ikit.\eva

\bvb Of his thinking should man not be boastful, \\
\ind but rather guarding of his senses \\
when sharp and silent he comes to a homestead; \\
\ind sudden harm seldom strikes the wary, \\
for an unfickler friend man never gets \\
\ind than much \inx[C]{manwit}.\evb\evg


\bvg\bva Hinn \alst{v}ari gęstr, \hld\ es til \alst{v}erðar kømr, &
\ind \edtrans{\alst{þ}unnu hljóði \alst{þ}ęgir}{shupts up and listens closely}{\Bfootnote{lit. ‘shuts up with thin (i.e. attentive) listening’.}}; &
\alst{ęy}rum hlýðir, \hld\ en \alst{au}gum skoðar, &
\ind svá \edtext{nýsisk \alst{f}róðra hvęrr \alst{f}yrir}{\lemma{nýsisk fyrir ‘looks ahead’}\Bfootnote{This verb underlies the noun \emph{for-njósn} as found in \Sigrdrifumal\ 25.}}.\eva

\bvb The wary guest—when for a meal he comes— \\
\ind shuts up and listens closely. \\
With ears he listens and with eyes he watches; \\
\ind so looks each learned man ahead.\evb\evg


\bvg\bva Hinn es \alst{s}ę́ll, \hld\ es \alst{s}ér of getr &
\ind \edtrans{\alst{l}of ok \alst{l}íkn-stafi}{praise and staves of liking}{\Bfootnote{\emph{líkn} ‘liking’ is a very interesting word.  It is defined by \ONP\ as: ‘mercy, compassion, relief, comfort, help’.  In the present poem its precise meaning seems to be something like ‘the state of being liked by your surroundings to the point where people are willing to help you out’.  Cf. its two other occurrences in the present poem: sts. 120 and especially 123 (where it is likewise paired with \emph{lof} ‘praise’).}}; &
\alst{ȯ}-dę́lla ’s við þat, \hld\ es \alst{ęi}ga skal &
\ind \alst{a}nnars brjóstum \alst{í}.\eva

\bvb This one is blessed, who for himself does get \\
\ind praise and staves of liking. \\
It’s uneasy regarding that which one shall own \\
\ind in another man’s chest.\evb\evg


\bvg\bva%
\edtrans{\alst{S}á}{That one}{\Bfootnote{Contrasting with \emph{hinn} ‘this one’ in the previous stanza.}} es \alst{s}ę́ll, \hld\ es \alst{s}jalfr of á &
\ind \alst{l}of ok vit meðan \alst{l}ifir; &
því-at \alst{i}ll rǫ́ð \hld\ hęfr maðr \alst{o}pt þęgit &
\ind \alst{a}nnars brjóstum \alst{ó}r.\eva

\bvb That one is blessed, who himself does have \\
\ind praise and wits while he lives; \\
for ill counsels has man oft taken \\
\ind out of another man’s chest.\evb\evg


\bvg\bva\alst{B}yrði \alst{b}ętri \hld\ berr-at maðr \alst{b}rautu at, &
\ind an sé \alst{m}an-vit \alst{m}ikit; &
\alst{au}ði bętra \hld\ þykkir þat í \alst{ȯ}-kunnum stað; &
\ind slíkt es \alst{v}á-laðs \alst{v}era.\eva

\bvb A better burden bears man not on the road \\
\ind than much manwit. \\
In an unknown place it seems better than wealth; \\
\ind such is the destitute man’s shelter.\evb\evg


\bvg\bva\alst{B}yrði \alst{b}ętri \hld\ berr-at maðr \alst{b}rautu at, &
\ind an sé \alst{m}an-vit \alst{m}ikit; &
\alst{v}eg-nest \alst{v}erra \hld\ \alst{v}egr-a \edtrans{\alst{v}ęlli at}{on the plain}{\Bfootnote{Formulaic, the word \emph{vǫllr} ‘plain, (uncultivated) field’ is also used in sts. 38 and 49. It is easily understood that the wild heaths and plains of Iron Age Norway were particularly unsafe places where a traveller needed to keep his wits about him, lest he fall victim to robbers or murderers (so st. 38).}}, &
\ind an sé \alst{o}f-drykkja \alst{ǫ}ls.\eva

\bvb A better burden bears man not on the road \\
\ind than much manwit. \\
Worse way-provision he drags not along on the plain \\
\ind than a too great drink of ale.\evb\evg


\bvg\bva Es-a svá \alst{g}ótt, \hld\ sęm \alst{g}ótt kveða, &
\ind \alst{ǫ}l \alst{a}lda sonum; &
því-at \alst{f}ę́ra vęit, \hld\ es \alst{f}lęira drekkr, &
\ind síns til \alst{g}ęðs \alst{g}umi.\eva

\bvb It’s not so good, as good they say, \\
\ind ale for the sons of men; \\
for the less he knows, as the more he drinks, \\
\ind man of his own senses.\evb\evg


\bvg\bva \edtrans{\alst{Ó}·minnis-hegri}{Forgetfulness-heron}{\Bfootnote{Lit. “unmemory-heron”; a rather interesting personification of drunkenness as a hovering bird.}} hęitir, \hld\ sá’s yfir \alst{ǫ}lðrum þrumir, &
\ind hann stelr \alst{g}ęði \alst{g}uma; &
þess \alst{f}ogls \alst{f}jǫðrum \hld\ ek \alst{f}jǫtraðr vas’k &
\ind í \alst{g}arði \alst{G}unnlaðar.\eva

\bvb Forgetfulness-heron is he called, who hovers over ale-feasts; \\
\ind he robs man of his senses. \\
By that bird’s feathers I was fettered \\
\ind in the yards of \inx[P]{Guthlathe}.\evb\evg


\bvg\bva\alst{Ǫ}lr ek varð, \hld\ varð \alst{o}fr-ǫlvi, &
\ind at hins \alst{f}róða \alst{F}jalars; &
því es \alst{ǫ}lðr batst, \hld\ at \alst{a}ptr of hęimtir &
\ind hvęrr sitt \alst{g}ęð \alst{g}umi.\eva

\bvb Drunk I became—I became the drunkest by far— \\
\ind at the learned Fealer’s [home].— \\
That ale-feast is best, where every man \\
\ind gets back to his senses.\evb\evg


\bvg\bva\alst{Þ}agalt ok hugalt \hld\ skyli \alst{þ}jóðans barn &
\ind ok \alst{v}íg-djarft \alst{v}esa; &
\alst{g}laðr ok ręifr \hld\ skyli \alst{g}umna hvęrr, &
\ind unds sinn \alst{b}íðr \alst{b}ana.\eva

\bvb Silent and thoughtful should the king’s child \\
\ind —and battle-bold—be. \\
Glad and cheerful should every man be, \\
\ind until he suffer his bane.\evb\evg


\bvg\bva\alst{Ó}·snjallr maðr \hld\ hyggsk munu \alst{ę}y lifa, &
\ind ef við \alst{v}íg \alst{v}arask; &
en \alst{ę}lli gefr hǫ́num \hld\ \alst{ę}ngi frið, &
\ind þótt hǫ́num \alst{g}ęirar \alst{g}efi.\eva

\bvb The unvalorous man thinks he will always live \\
\ind if he of war be wary; \\
but old age gives him no peace, \\
\ind which yet spears would give him.\footnoteB{The unvalorous man might have been spared by the spears, but death will still find him through miserable old age. Since death is unavoidable it is better to live bravely, even if one risks dying in battle, than to live cowardly and die of sickness. This connects well to the ancient view of the ‘straw-death’ (TODO).}\evb\evg


\bvg\bva\alst{K}ópir af-glapi, \hld\ es til \alst{k}ynnis kømr, &
\ind \alst{þ}ylsk hann umb eða \alst{þ}rumir; &
allt es \alst{s}ęnn, \hld\ ef \alst{s}ylg of getr, &
\ind uppi ’s þȧ \alst{g}ęð \alst{g}uma.\eva

\bvb Gapes the oaf when to visit he comes; \\
\ind he mumbles about or loiters. \\
All at once—if a sip he gets— \\
\ind exposed is the mind of the man.\evb\evg


\bvg\bva Sá ęinn \alst{v}ęit, \hld\ es \alst{v}íða ratar &
\ind ok \edtrans{hęfr \alst{f}jǫlð of \alst{f}arit}{has journeyed much}{\Bfootnote{Cf. \Vafthrudnismal\ 3, 44, et.c., where Weden repeats: \emph{Fjǫlð ek fór, \hld\ fjǫlð fręistaða’k, // fjǫlð ek ręynda ręgin} ‘Much I journeyed, much I tried, much I tested the \inx[G]{Reins}.’}}, &
hvęrju \alst{g}ęði \hld\ stýrir \alst{g}umna hvęrr, &
\ind sá es \alst{v}itandi ’s \alst{v}its.\eva

\bvb He alone knows, who widely roams, \\
\ind and has journeyed much, \\
which sort of mind every man wields, \\
\ind who is knowing of his wits.\evb\evg


\bvg\bva\edtrans{\alst{H}aldi-t maðr ȧ kęri}{Man ought not to hold onto the cask}{\Bfootnote{Perhaps referring to a toast wherein a drinking vessel would be passed around in a circle and each member would drink.  Such toasts were drunk for a long time in Northern Europe—indeed this is the origin of the Scandinavian toasting-word, \emph{skål} ‘prosit, cheers!’, lit. ‘bowl!’.  “Holding onto” the vessel (and not letting the next person drink) was surely seen as very rude; as late as 1519 a man in Jämtland was killed in an argument resulting from his refusal to pass on the bowl (see \textcite{Sjöberg1907}).  The sense is thus: “Do not refuse a toast when offered (but do not drink too much, either!)”}}, \hld\ drekki þó at \alst{h}ófi mjǫð, &
\ind \edtrans{mę́li \alst{þ}arft eða \alst{þ}ęgi}{he ought to speak the needful or shut up}{\Bfootnote{Formulaic, line occurs identically in \Vafthrudnismal\ 10/2.}}; &
\alst{ȯ}-kynnis þess \hld\ váar þik \alst{ę}ngi maðr, &
\ind at gangir \alst{s}nimma at \alst{s}ofa.\eva

\bvb Man ought not to hold onto the cask, but still drink mead in moderation; \\
\ind he ought to speak the needful or shut up. \\
For that uncouthness will no man blame thee, \\
\ind that thou go early to sleep.\evb\evg


\bvg\bva\alst{G}rǫ́ðugr halr, \hld\ nema \alst{g}ęðs viti, &
\ind \alst{e}tr sér \alst{a}ldr-trega; &
opt fę̇r \alst{h}lǿgis, \hld\ es með \alst{h}orskum kømr, &
\ind \alst{m}anni hęimskum \alst{m}agi.\eva

\bvb The gluttonous man—unless he know his sense— \\
\ind eats himself a life-sorrow. \\
Oft the belly, when among the sharp he comes, \\
\ind brings the foolish man ridicule.\evb\evg


\bvg\bva\alst{H}jarðir þat vitu, \hld\ nę́r \alst{h}ęim skulu, &
\ind ok \alst{g}anga þȧ af \alst{g}rasi; &
en \alst{ȯ}-sviðr maðr \hld\ kann \alst{ę́}va-gi &
\ind síns of \alst{m}ál \alst{m}aga.\eva

\bvb Herds know when home they shall [go], \\
\ind and then part from the grass; \\
but an unwise man never knows \\
\ind his own belly’s measure.\evb\evg


\bvg\bva\alst{V}e-sall maðr \hld\ ok \alst{i}lla skapi &
\ind \alst{h}lę́r at \alst{h}ví-vetna; &
hitt-ki hann \alst{v}ęit, \hld\ es \alst{v}ita þyrpti, &
\ind at \edtrans{hann es-a \alst{v}amma \alst{v}anr}{he is not free of blemishes}{\Bfootnote{Formulaic, cf. \Lokasenna\ 30: \emph{es-a þér vamma vant} ‘thou art not free of blemishes’.}}.\eva

\bvb The wretched man and badly turned out \\
\ind laughs at anything. \\
This he knows not, which he might need to know: \\
\ind that he is not free of blemishes.\evb\evg


\bvg\bva\alst{Ó}·sviðr maðr \hld\ vakir umb \alst{a}llar nę́tr &
\ind ok \alst{h}yggr at \alst{h}ví-vetna; &
þȧ es \alst{m}óðr, \hld\ es at \alst{m}orni kømr; &
\ind alt es \alst{v}íl sęm \alst{v}as.\eva

\bvb The unwise man is awake for all nights \\
\ind and thinks of anything. \\
Then he is weary when the morning comes: \\
\ind all the trouble is as it was.\evb\evg


\bvg\bva\alst{Ȯ}-snotr maðr \hld\ hyggr sér \alst{a}lla vesa &
\ind \alst{v}ið-hlę́jęndr \alst{v}ini; &
hitt-ki hann \alst{f}iðr, \hld\ þótt of hann \alst{f}ár lesi, &
\ind ef með \alst{s}notrum \alst{s}itr.\eva

\bvb The unclever man thinks all those \\
\ind who laugh with him his friends. \\
This he finds not, that they yet make sport in him, \\
\ind if among the clever he sits.\evb\evg


\bvg\bva\alst{Ȯ}-snotr maðr \hld\ hyggr sér \alst{a}lla vesa &
\ind \alst{v}ið-hlę́jęndr \alst{v}ini; &
\alst{þ}ȧ þat fiðr \hld\ es at \alst{þ}ingi kømr, &
\ind at \edtrans{á \alst{f}or-mę́lęndr \alst{f}áa}{has spokesmen few}{\Bfootnote{Repeated in st. 62.  He has few who are ready to take his side and speak up for him (in legal proceedings); true friends are proven in hard times, not in drunken chatter.  The Thing was the old Germanic legal assembly, where smaller disputes might easily turn into deadly feuds.}}.\eva

\bvb The unclever man thinks all those \\
\ind who laugh with him his friends. \\
Then he finds, when to the \inx[C]{Thing} he comes, \\
\ind that he has spokesmen few.\evb\evg


\bvg\bva\alst{Ȯ}-snotr maðr \hld\ þykkisk \alst{a}llt vita, &
\ind ef á sér í \edtrans{\alst{v}ǫ̇}{nook}{\Bfootnote{From earlier \emph{*vrǫ̇}; cf. Swedish \emph{vrå} ‘corner, nook’, rare English \emph{wroo} ‘id.’  The present stanza is to my knowledge the only Norse attestation of the form \emph{vǫ̇}, which features a rare Western sound change from \emph{vr-} to \emph{v-}.  The more common change \emph{vr-} to \emph{r-} yields \emph{rǫ̇}, which is the normal Norse form. — Tangentially this word is brought up in \FGT\ as an example of a word with nasal \emph{ǫ̇}, and contrasted with oral \emph{ǫ́} in \emph{rǫ́} ‘sailyard’.}} \alst{v}eru; &
hitt-ki hann \alst{v}ęit, \hld\ hvat skal \alst{v}ið kveða, &
\ind ef hans \alst{f}ręista \alst{f}irar.\eva

\bvb The unclever man seems to know everything \\
\ind if he takes shelter in a nook. \\
This he knows not, what he shall answer \\
\ind if men test him.\evb\evg


\bvg\bva\alst{Ȯ}-snotr maðr, \hld\ es með \alst{a}ldir kømr, &
\ind \alst{þ}at ’s batst at hann \alst{þ}ęgi; &
\alst{ę}ngi þat vęit, \hld\ at hann \alst{ę}kki kann, &
\ind nema hann \alst{m}ę́li til \alst{m}art. &
\alst{v}ęit-a maðr, \hld\ hinn’s \alst{v}ę́t-ki vęit, &
\ind þótt hann \alst{m}ę́li til \alst{m}art.\eva

\bvb The unclever man when among people he comes— \\
\ind it’s best that he shut up. \\
No one knows that he nothing knows, \\
\ind unless he speak too much. \\
The man knows not, who nothing knows, \\
\ind that he speak too much.\evb\evg


\bvg\bva\alst{F}róðr sá þykkisk, \hld\ es \edtext{\alst{f}regna kann, &
\ind ok \alst{s}ęgja}{\lemma{fregna \dots\ sęgja ‘ask \dots\ answer’}\Bfootnote{Perhaps specifically in the context of a riddling contest of wisdom.}} hit \alst{s}ama, &
\alst{ęy}-vitu lęyna \hld\ męgu \alst{ý}ta synir &
\ind því es \alst{g}ęngr of \alst{g}uma.\eva

\bvb Learned seems he who can ask \\
\ind and answer the same [way]. \\
In no way may the sons of men hide \\
\ind that which eludes a man.\evb\evg


\bvg\bva\alst{Ǿ}rna mę́lir, \hld\ sá’s \alst{ę́}va þęgir, &
\ind \alst{st}að-lausu \alst{st}afi; &
\edtext{\alst{h}rað-mę́lt tunga, \hld\ \edtrans{nema \alst{h}aldęndr ęigi}{unless it be held in place}{\Bfootnote{lit. ‘unless holders own it’ or ‘unless it own holders’. The ‘holders’ are perhaps the teeth which hold the tongue in place.}}, &
\ind opt sér ȯ-\alst{g}ótt of \alst{g}ęlr}{\lemma{hrað-mę́lt \dots\ of gęlr ‘A quick-spoken \dots\ for itself’}\Bfootnote{Formulaic. Cf. \Lokasenna\ 31.}}.\eva

\bvb He who never shuts up speaks plenty many \\
\ind utterings of absurdity. \\
A quick-spoken tongue—unless it be held in place— \\
\ind oft sings evil [into being] for itself.\evb\evg


\bvg\bva At \alst{au}ga-bragði \hld\ skal-a maðr \alst{a}nnan hafa, &
\ind þótt til \alst{k}ynnis \alst{k}omi; &
margr \alst{f}róðr þykkisk, \hld\ ef \alst{f}reginn es-at &
\ind ok nái \edtrans{\alst{þ}urr-fjallr}{dry-skinned}{\Bfootnote{i.e. ‘untested’, equivalent to the English idiom \emph{get one’s feet wet}.  The word \emph{fell} \char`~\ \emph{fjall} ‘skin, pelt’ is rare in Old Norse literature and only occurs in cpds, e.g. \Volundarkvida\ 11: \emph{ber-fjall} ‘bear-pelt’.  It survives in modern Swedish \emph{fjäll} ‘scale (on fish and reptiles)’}} \alst{þ}ruma.\eva

\bvb For a laughing-stock shall man not have another \\
\ind when he comes to visit. \\
Many a one seems learned if he is not asked, \\
\ind and gets to loiter about dry-skinned.\evb\evg


\bvg\bva\alst{F}róðr þykkisk \hld\ sá’s \alst{f}lótta tękr &
\ind \edtrans{\alst{g}ęstr}{guest}{\Bfootnote{The situation hinted at in this and the following stanza is that two guests—unknown to eachother—have come to the same homestead.  The sense is that when mocked by a stranger it is best not to engage, since the dealing may quickly turn violent.  Cf. sts. 122, 123, and 125.}} at \alst{g}ęst hę́ðinn; &
\alst{v}ęit-a gǫrla \hld\ sá’s of \alst{v}erði glissir, &
\ind þótt með \alst{g}rǫmum \alst{g}lami.\eva

\bvb Learned seems he who takes to flight, \\
\ind the guest, from a scoffing guest. \\
He knows not clearly, who grins over the food, \\
\ind that he be flirting with fiends.\evb\evg


\bvg\bva\alst{G}umnar margir \hld\ erusk \alst{g}agn-hollir, &
\ind en at \alst{v}irði \alst{v}rekask; &
\alst{a}ldar róg \hld\ þat mun \alst{ę́} vesa; &
\ind órir \alst{g}ęstr við \alst{g}ęst.\eva

\bvb Many men are well true to each other, \\
\ind but over food drive each other away. \\
The strife of mankind will that ever be; \\
\ind guest raves against guest.\evb\evg


\bvg\bva\alst{Á}r-liga verðar \hld\ skyli maðr \alst{o}pt fȧa, &
\ind nema til \alst{k}ynnis \alst{k}omi; &
\alst{s}itr ok \alst{s}nópir, \hld\ lę́tr sęm \alst{s}olginn sé, &
\ind ok kann \alst{f}regna at \alst{f}ǫ́u.\eva

\bvb An early meal should man oft get, \\
\ind unless he come to visit: \\
he sits and sulks, sounds as if starved, \\
\ind and can ask about little.\evb\evg


\bvg\bva\alst{A}f·hvarf mikit \hld\ es til \alst{i}lls vinar, &
\ind þótt ȧ \alst{b}rautu \alst{b}úi, &
en til \alst{g}óðs vinar \hld\ liggja \alst{g}agn-vegir, &
\ind þótt hann sé \alst{f}irr \alst{f}arinn.\eva

\bvb A great detour it’s to a bad friend, \\
\ind although he live on the road; \\
but to a good friend lie the finest ways, \\
\ind although he far gone be.\evb\evg


\bvg\bva\alst{G}anga \edtext{skal}{\Afootnote{emend.; om. \Regius}}, \hld\ skal-a \alst{g}ęstr vesa &
\ind \alst{ęy} í \alst{ęi}num stað; &
\alst{l}júfr verðr \alst{l}ęiðr, \hld\ ef \alst{l}ęngi sitr &
\ind \alst{a}nnars flętjum \alst{ȧ}.\eva

\bvb One shall go; he shall not be a guest \\
\ind forever in one place. \\
The loved becomes loathed if for long he sits \\
\ind on another man’s benches.\footnoteB{The customary length of stay in old times was three nights.  So Eyel’s saw, ch. 78: \emph{þat var engi siðr, at sitja lengr en þrjár nę́tr at kynni.} ‘it was not customary to stay longer than three nights when visiting.’  Compare a much Jutish saying: \emph{en tredje dags gjæst stinker} ‘a third day’s guest stinks’, which closely resembles a maxim attributed to Benjamin Franklin: “Guests, like fish, begin to smell after three days.”  It is probably with respect to such proverbs that Auden and Taylor translate the latter half of the present stanza “He starts to stink who outstays his welcome, / in a hall that is not his own.”}\evb\evg


\bvg\bva\edtrans{\alst{B}ú es \alst{b}ętra, \hld\ þótt lítit sé}{A dwelling is better though small it be}{\Bfootnote{The b-line is missing the necessary alliteration, but no good emendation suggests itself.}}, &
\ind \alst{h}alr es \alst{h}ęima \alst{h}vęrr; &
þótt \alst{t}vę́r gęitr ęigi \hld\ ok \alst{t}aug-ręptan sal, &
\ind þat ’s þó \alst{b}ętra an \alst{b}ǿn.\eva

\bvb A dwelling is better though small it be; \\
\ind each is a hero at home. \\
Though two goats he own and a cord-roofed hall, \\
\ind it is yet better than begging.\evb\evg


\bvg\bva\alst{B}ú es \alst{b}ętra, \hld\ þótt lítit sé, &
\ind \alst{h}alr es \alst{h}ęima \alst{h}vęrr; &
\alst{b}lóðugt es hjarta \hld\ þęim’s \alst{b}iðja skal &
\ind sér í \alst{m}ál hvęrt \alst{m}atar.\eva

\bvb A dwelling is better though small it be; \\
\ind each is a hero at home. \\
Bloody is the heart in him who shall beg \\
\ind for his every meal of food.\evb\evg


\bvg\bva%
\alst{V}ǫ́pnum sínum \hld\ skal-a maðr \edtrans{\alst{v}ęlli ȧ}{on the plain}{\Bfootnote{Formulaic, see note to st. 12.}} &
\ind \edtrans{\alst{f}eti ganga \alst{f}ramarr}{take one step further}{\Bfootnote{Formulaic. Cf. \Lokasenna\ 1: \emph{svá’t ęinu-gi feti gangir framarr} ‘so that thou not take one step further’.}}; &
því-at ȯ-\alst{v}íst ’s at \alst{v}ita, \hld\ nę́r verðr ȧ \alst{v}egum úti &
\ind \alst{g}ęirs of þǫrf \alst{g}uma.\eva

\bvb From his weapons shall man on the plain \\
\ind not take one step further; \\
for it’s unsure to know, when on the ways outside, \\
\ind man comes in need of a spear.\evb\evg


\bvg\bva%
Fann’k-a \alst{m}ildan \alst{m}ann \hld\ eða svá \edtrans{\alst{m}atar góðan}{good of meat}{\Bfootnote{A Viking Age expression; see Encyclopedia.}}, &
\ind at vę́ri-t \alst{þ}iggja \alst{þ}egit; &
eða \alst{s}íns féar \hld\ \alst{s}vá-gi \edtext{[...]}{\Bfootnote{It is doubtless that a word has been lost here; the meter and sense require it. \textcite{FinnurEdda}\ suggests \emph{gløggvan} ‘miserly, stingy’, giving a litotes ‘so unstingy’, i.e., ‘so generous’.}}, &
\ind at \alst{l}ęið sé \alst{l}aun, ef þegi.\eva

\bvb I found not a generous man or one so \inx[C]{good of meat}, \\
\ind that a gift were not accepted; \\
or one with his \inx[C]{fee} so not [...], \\
\ind that the repayments were loathed, if he accepted [them].\footnoteB{No man is so generous that he would refuse a gift presented to him, nor loathe receiving a favour as thanks for his generosity.}\evb\evg


\bvg\bva%
\alst{F}éar síns, \hld\ es \alst{f}ęngit hęfr, &
\ind skyli-t maðr \alst{þ}ǫrf \alst{þ}ola; &
opt sparir \alst{l}ęiðum \hld\ þat’s hęfr \alst{l}júfum hugat; &
\ind mart gęngr \alst{v}err an \alst{v}arir.\eva

\bvb Of his own \inx[C]{fee} which he has earned \\
\ind should man not suffer need. \\
One oft saves for the loathed what one meant for the loved; \\
\ind much goes worse than expected.\evb\evg


\bvg\bva%
\edtrans{\alst{V}ǫ́pnum ok \alst{v}ǫ́ðum}{With weapons and garments}{\Bfootnote{i.e. weapons and armour (the “garments” are probably no silks); friends are supposed to help each other and strengthen their “violence capital”.  This alliterative word-pair is formulaic and in other occurences exclusively refers to implements of war; cf. e.g. \Beowulf\ 39, where \inx[P]{Shield}’s pyre-ship is loaded with \emph{hilde-wǽpnum \alst\ ǫnd heaðo-wǽdum} ‘war-weapons and battle-garments’.}} \hld\ skulu \alst{v}inir glęðjask; &
\ind \edtrans{þat ’s ȧ \alst{s}jǫlfum \alst{s}ýnst}{that is best seen on oneself}{\Bfootnote{i.e. in one’s own experience.}}; &
\alst{v}iðr-gefęndr ok ęndr-gefęndr \hld\ erusk \alst{v}inir lęngst, &
\ind ef \edtrans{þat}{it}{\Bfootnote{The friendship.}} bíðr at \alst{v}erða \alst{v}ęl.\eva

\bvb With weapons and garments shall friends gladden each other; \\
\ind that is best seen on oneself. \\
Givers-back and givers-again are friends for the longest \\
\ind if it comes to last long.\evb\evg


\bvg\bva\alst{V}in sínum \hld\ skal maðr \alst{v}inr \alst{v}esa, &
\ind ok \alst{g}jalda \alst{g}jǫf við \alst{g}jǫf; &
\alst{h}látr við \alst{h}látri \hld\ skyli \alst{h}ǫlðar taka, &
\ind en \alst{l}ausung við \alst{l}ygi.\eva

\bvb With his friend shall man be a friend, \\
\ind and pay gift against gift; \\
laughter against laughter should men employ, \\
\ind but duplicity against lie.\evb\evg


\bvg\bva\alst{V}in sínum \hld\ skal maðr \alst{v}inr vesa, &
\ind \alst{þ}ęim ok \alst{þ}ess vin; &
en \alst{ȯ}-vinar síns \hld\ skyli \alst{ę}ngi maðr &
\ind \alst{v}inar \alst{v}inr \alst{v}esa.\eva

\bvb With his friend shall man be a friend, \\
\ind with him and his friend; \\
but his enemy’s, should no man, \\
\ind friend’s friend be.\evb\evg


\bvg\bva\alst{V}ęitst, ef \alst{v}in átt, \hld\ þann’s \alst{v}ęl trúir &
\ind ok vilt af hǫ́num \alst{g}ótt \alst{g}eta, &
\alst{g}ęði skalt við þann \hld\ ok \alst{g}jǫfum skipta, &
\ind \alst{f}ara at \alst{f}inna opt.\eva

\bvb Thou knowest, if thou have a friend whom thou well trust, \\
\ind and wilt receive good from him: \\
thoughts and gifts shalt thou trade with him; \\
\ind journey to find him oft.\footnoteB{Several lines of the present st. are shared with st. 119.}\evb\evg


\bvg\bva Ef þú \alst{á}tt \alst{a}nnan, \hld\ þann’s \alst{i}lla trúir, &
\ind vilt af hǫ́num þó \alst{g}ótt \alst{g}eta, &
\edtext{\alst{f}agrt skalt mę́la við þann, \hld\ en \alst{f}látt hyggja}{\lemma{fagrt \dots\ mę́la \dots\ flátt hyggja ‘fairly \dots\ speak \dots\ falsely think’}\Bfootnote{Formulaic, cf. sts. 90, 91.}} &
\ind ok gjalda \alst{l}ausung við \alst{l}ygi.\eva

\bvb If thou have another whom thou badly trust, \\
\ind and wilt yet receive good from him: \\
fairly shalt thou speak with him, but falsely think, \\
\ind and pay duplicity against lie.\evb\evg


\bvg\bva Þat ’s \alst{ę}nn umb þann, \hld\ es þú \alst{i}lla trúir &
\ind ok þér es \alst{g}runr at \alst{g}ęði, &
\alst{h}lę́ja skalt við þęim \hld\ ok of \alst{h}ug mę́la; &
\ind \alst{g}lík skulu \alst{g}jǫld \alst{g}jǫfum.\eva

\bvb It’s yet regarding the one whom thou trust badly, \\
\ind and whose intentions toward thee are suspect: \\
thou shalt laugh with him and speak with care; \\
\ind repayments shall be equal to gifts.\footnoteB{Equivalent to the last line of the previous st. (“pay duplicity against lie”).}\evb\evg


\bvg\bva Ungr vas’k \alst{f}orðum, \hld\ \alst{f}ór’k ęinn saman, &
\ind þȧ varð’k \alst{v}illr \alst{v}ega; &
\alst{au}ðigr þȯttumk, \hld\ es \alst{a}nnan fann’k, &
\ind \alst{m}aðr es \alst{m}anns gaman.\eva

\bvb Young was I once; I travelled alone; \\
\ind then I became lost of ways. \\
Wealthy I thought myself when another one I found; \\
\ind man is man’s pleasure.\evb\evg


\bvg\bva\alst{M}ildir frǿknir \hld\ \alst{m}ęnn batst lifa, &
\ind \alst{s}jaldan \alst{s}út ala; &
en \edtext{\alst{ȯ}-snjallr}{\linenum{|3||4}\lemma{ȯ-snjallr, gløggr ‘unvalorous, stingy’}\Bfootnote{Contrasting respectively with \emph{frǿkn, mildr} ‘brave, generous’ in the first half of the stanza; very fine parallelism.}} maðr \hld\ \alst{u}ggir hvat-vetna, &
\ind \edtext{sýtir ę́ \alst{g}løggr við \alst{g}jǫfum}{\lemma{sýtir \dots\ gjǫfum ‘the stingy man \dots\ gifts’}\Bfootnote{Cf. st. 39.  After receiving a gift, one was culturally obliged to give something back.}}.\eva

\bvb Generous, brave men live best: \\
\ind seldom they nourish sorrow— \\
but the unvalorous man is frightened by anything, \\
\ind the stingy always grieves over gifts.\evb\evg


\bvg\bva\alst{V}áðir mínar \hld\ gaf’k \alst{v}ęlli at &
\ind \alst{t}vęim \alst{t}ré-mǫnnum; &
\alst{r}ekkar þat þȯttusk, \hld\ es \alst{r}ipt hǫfðu; &
\ind \alst{n}ęiss es \alst{n}ǫkkviðr halr.\eva

\bvb My garments I gave, on the plain, \\
\ind to two tree-men. \\
Champions they seemed when cloaks they had; \\
\ind shameful is the naked hero.\footnoteB{One of the harder sts. in the poem.  The probable sense is that “the clothes make the man” (or warrior): under expensive gear a thin tree-man might be lurking, and likewise even a mighty man (the choice of the word \emph{halr} ‘hero, warrior’ (cf. sts. 36, 37) rather than the more neutral \emph{maðr} ‘man, person’ is surely intentional) can never defend himself against a heavily armoured opponent.  Without his arms, he becomes as vulnerable as the “tree-man” on the plain.}\evb\evg


\bvg\bva Hrørnar \alst{þ}ǫll, \hld\ sú’s stęndr \alst{þ}orpi ȧ, &
\ind hlýr-at hęnni \alst{b}ǫrkr né \alst{b}arr; &
svá es \alst{m}aðr, \hld\ sá’s \alst{m}ann-gi ann; &
\ind hvat skal hann \alst{l}ęngi \alst{l}ifa?\eva

\bvb Wilters the pine that stands on the yard; \\
\ind shields her not bark nor leaf. \\
So is the man who loves no man; \\
\ind why shall he live for long?\evb\evg


\bvg\bva\alst{Ę}ldi hęitari \hld\ brinnr með \alst{i}llum vinum &
\ind \alst{f}riðr \edtrans{\alst{f}imm daga}{for five days}{\Bfootnote{i.e. “for a week”, which was originally five days long.  See also st. 74 and the Encyclopedia: \inx[C]{five days}.}}, &
en þȧ \alst{sl}oknar, \hld\ es hinn \alst{s}étti kømr, &
\ind ok \alst{v}ersnar allr \alst{v}in-skapr.\eva

\bvb Hotter than fire burns love among bad friends, \\
\ind for \inx[C]{five days}; \\
but then goes out when the sixth one comes, \\
\ind and all the friendship worsens.\evb\evg


\bvg\bva\alst{M}ikit ęitt \hld\ skal-a \alst{m}anni gefa; &
\ind opt kaupir sér í \alst{l}ítlu \alst{l}of, &
með \alst{h}ǫlfum \alst{h}lęif \hld\ ok með \alst{h}ǫllu kęri &
\ind \alst{f}ekk ek mér \alst{f}é-laga.\eva

\bvb Much at once shall one not give a man; \\
\ind oft one buys oneself praise for little. \\
With half a loaf and an awry cask \\
\ind I got myself a partner.\evb\evg


\bvg\bva\edtrans{\alst{L}ítilla sanda, \hld\ \alst{l}ítilla sę́va}{Of small sands, of small seas}{\Bfootnote{Probably a partitive genitive, the sense being that man’s “horizons” are small; the universe will always be far greater than him.}}, &
\ind lítil eru \alst{g}ęð \alst{g}uma; &
\edtext{því-at \alst{a}llir męnn \hld\ \alst{u}rðu-t jafn-spakir; &
\ind \alst{h}ǫlf es ǫld \alst{h}var.}{\lemma{því-at \dots\ ǫld hvar. ‘For \dots\ every man.’}\Bfootnote{On the meaning of the second half of this stanza I find the view of \textcite{Athugasemdir1929} most convincing; namely that every man has both strengths and weaknesses in terms of wisdom.  As nobody can excel at everything, nobody is complete; every person is “half” (and it should be added that ON \emph{halfr} has a more general sense of incompleteness than its English cognate).  This interpretation fits particularly closely with sts. 71 and 132. — This stanza introduces several stanzas dealing with wisdom and foolishness.}}\eva

\bvb Of small sands, of small seas: \\
\ind small are the senses of man. \\
For all have not become evenly knowing; \\
\ind half is every man.\evb\evg


\bvg\bva\alst{M}eðal-snotr \hld\ skyli \alst{m}anna hvęrr, &
\ind ę́va til \alst{s}notr \alst{s}éi; &
þęim es \alst{f}yrða \hld\ \alst{f}ęgrst at lifa, &
\ind es \alst{v}ęl mart \alst{v}itu.\eva

\bvb Middle-clever should each man be; \\
\ind never too clever. \\
For those men it’s fairest to live, \\
\ind who know well enough.\evb\evg


\bvg\bva\alst{M}eðal-snotr \hld\ skyli \alst{m}anna hvęrr, &
\ind ę́va til \alst{s}notr \alst{s}éi; &
\alst{s}notrs manns hjarta \hld\ verðr \alst{s}jaldan glatt, &
\ind ef sá ’s \alst{a}l-snotr es \alst{á}.\eva

\bvb Middle-clever should each man be; \\
\ind never too clever. \\
The clever man’s heart is seldom glad, \\
\ind if its owner is all-clever.\evb\evg


\bvg\bva\alst{M}eðal-snotr \hld\ skyli \alst{m}anna hvęrr, &
\ind ę́va til \alst{s}notr \alst{s}éi; &
\alst{ø}r·lǫg sín \hld\ viti \alst{ę}ngi maðr fyrir; &
\ind \edtrans{þęim es \alst{s}orga-lausastr \alst{s}efi.}{his is the most sorrowless mind.}{\Bfootnote{i.e. he who is ignorant of his fate.  It is surely fitting that Weden should say this, having knowledge of the inevitable destruction of the world and himself (see \inx[L]{Rakes of the Reins}).}}\eva

\bvb Middle-clever should each man be; \\
\ind never too clever. \\
His own \inx[C]{orlay} ought no man to know ahead; \\
\ind his is the most sorrowless mind.\evb\evg


\bvg\bva\alst{B}randr af \alst{b}randi \hld\ \alst{b}rinnr unds \alst{b}runninn es, &
\ind \alst{f}uni kvęykisk af \alst{f}una; &
\alst{m}aðr af \alst{m}anni \hld\ verðr at \alst{m}áli kuðr; &
\ind en til \edtrans{\alst{d}ǿlskr}{hickish}{\Bfootnote{Derived from an ablaut variant of \emph{dalr} ‘valley, dale’ + \emph{-iskr} ‘-ish’, the sense being ‘provincial, not having left his (home) valley’.  Cf. the Icelandic tribal names like \emph{vatns-dǿlir} and \emph{lang-dǿlir} ‘inhabitants of \emph{Vatns-dalr} (Waterdale), \emph{Lang-dalr} (Longdale)’.}} af \alst{d}ul.\eva

\bvb Fire by fire burns until it is burned [out]; \\
\ind flame is quickened by flame. \\
Man by man becomes known through speech, \\
\ind but the too hickish from his folly.\evb\evg


\bvg\bva\alst{Á}r skal rísa, \hld\ sá’s \alst{a}nnars vill &
\ind \alst{f}é eða \alst{f}jǫr hafa; &
sjaldan \alst{l}iggjandi ulfr \hld\ \alst{l}ę́r of getr, &
\ind né \alst{s}ofandi maðr \alst{s}igr.\eva

\bvb Early shall he rise who another man’s \\
\ind \inx[C]{fee} or life will have. \\
Seldom gets the lying wolf the thigh, \\
\ind nor the sleeping man victory.\evb\evg


\bvg\bva\alst{Á}r skal rísa, \hld\ sá’s á \alst{y}rkjęndr fáa, &
\ind ok ganga síns \alst{v}erka ȧ \alst{v}it; &
\alst{m}art of dvęlr \hld\ þann’s umb \alst{m}orgin sefr, &
\ind \edtrans{\alst{h}alfr es auðr und \alst{h}vǫtum}{the brisk has half the wealth}{\Bfootnote{i.e. the brisk man has already claimed half of a fortune by simply choosing to wake up early.}}.\eva

\bvb Early shall he rise who has workmen few, \\
\ind and go his work to meet. \\
Much is kept back from him who in the morning sleeps; \\
\ind the brisk has half the wealth.\evb\evg


\bvg\bva\alst{Þ}urra skíða \hld\ ok \alst{þ}akinna nę́fra, &
\ind þess kann \alst{m}aðr \alst{m}jǫt, &
ok þess \alst{v}iðar, \hld\ es \alst{v}innask męgi &
\ind \edtrans{\alst{m}ál ok \alst{m}issęri}{for a season and half-year}{\Bfootnote{i.e. over nine months, presumably the ones outside of summer (June–August).}}.\eva

\bvb Of dry planks and thatching birch bark: \\
\ind of \emph{this} man knows the measure— \\
and of that firewood which he may use \\
\ind for a season and half-year.\evb\evg


\bvg\bva\edtrans{\alst{Þ}vęginn ok męttr}{washed and full}{\Bfootnote{A formulaic collocation.  Cf. \Reginsmal\ TODO: \emph{kęmbðr} ‘combed’ — \emph{þvęginn} ‘washed’ — \emph{męttr} ‘full’; \Voluspa\ 33: \emph{þó} ‘washed’ — \emph{kęmbði} ‘combed’.  These examples attest to the importance of personal hygiene in the culture, something further seen by the ubiquity of combs in pre-Christian graves.  One is reminded of a passage from \emph{Germania} (ch. 22): \emph{Statim ē somnō, quem plērumque in diem extrahunt, lavantur, saepius calidā, ut apud quōs plūrimum hiems occupat.  Lautī cibum capiunt: sēparātae singulīs sēdēs et sua cuique mēnsa.  Tum ad negōtia nec minus saepe ad convīvia prōcēdunt armātī.} ‘On waking from sleep, which they generally prolong to a late hour of the day, they take a bath, oftenest of warm water, which suits a country where winter is the longest of the seasons.  After their bath they take their meal, each having a separate seat and table of his own.  Then they go armed to business, or no less often to their festal meetings (\emph{convivia}, i.e., their Things).’}} \hld\ ríði maðr \alst{þ}ingi at, &
\ind þótt sé-t \alst{v}ę́ddr til \alst{v}ęl; &
\alst{sk}úa ok bróka \hld\ \alst{sk}ammisk ęngi maðr &
\ind né \alst{h}ęsts in \alst{h}ęldr, &
\ind \edtrans{þótt hann \alst{h}afi-t góðan}{although he has not a good one.}{\Bfootnote{\textcite{FinnurEdda} considers this a late insert, and I agree.  It seems that the inserter was not aware of the rules of the \Ljodahattr\ meter and interpreted the preceding c-verse (\emph{né hęsts in hęldr}) as an a-verse of \Fornyrdislag.}}.\eva

\bvb Washed and full ought man to ride to the \inx[C]{Thing}, \\
\ind although he be not clothed too well; \\
of his shoes and breeches ought no man to be ashamed, \\
\ind nor the more of his horse, \\
\ind although he has not a good one.\evb\evg

\sectionline

The two following sts. are written in opposite order in \Regius, but a symbol at the start of each indicates that they should switch places.

\sectionline

\bvg\bva\alst{S}napir ok gnapir, \hld\ es til \alst{s}ę́var kømr, &
\ind \alst{ǫ}rn ȧ \alst{a}ldinn mar; &
svá es \alst{m}aðr, \hld\ es með \alst{m}ǫrgum kømr &
\ind ok \edtrans{á \alst{f}or-mę́lęndr \alst{f}áa}{has spokesmen few}{\Bfootnote{Shared with st. 25.}}.\eva

\bvb Snaps and stoops—when to the sea he comes— \\
\ind the eagle on the aged ocean. \\
So is the man who among the many comes, \\
\ind and has spokesmen few.\evb\evg


\bvg\bva\alst{F}regna ok sęgja \hld\ skal \alst{f}róðra hvęrr, &
\ind sá’s vill \alst{h}ęitinn \alst{h}orskr; &
\alst{ęi}nn vita \hld\ né \alst{a}nnarr skal, &
\ind \edtrans{\alst{þ}jóð}{thirty}{\Bfootnote{Or “people, nation”; the sense is in any case “many, everybody”.  For the translation “thirty” cf. \Skaldskaparmal\ 82, a list of poetic expressions for various numerals: \emph{\emph{þjóð} eru þrír tigir} ‘a \emph{nation} is thirty’ etc.}} vęit ef \alst{þ}rír ’ru.\eva

\bvb Ask and answer shall each learned man \\
\ind who wishes to be called sharp. \\
\emph{One} shall know, another shall not; \\
\ind thirty know if there are three.\evb\evg


\bvg\bva\alst{R}íki sitt \hld\ skyli \alst{r}áð-snotra &
\ind hvęrr í \alst{h}ófi \alst{h}afa; &
\edtext{þȧ þat \alst{f}innr, \hld\ es með \alst{f}rǿknum kømr, &
\ind at \alst{ę}ngi es \alst{ęi}nna hvatastr.}{\lemma{þȧ \dots\ ęinna hvatastr ‘then \dots briskest of all’}\Bfootnote{Almost identical to \Reginsmal\ TODO/3–4, which however has \emph{flęirum} ‘more men’ instead of \emph{frǿknum} ‘the bold’.}}\eva

\bvb His own power should each counsel-clever \\
\ind man use in moderation. \\
This he then finds when among the bold he comes— \\
\ind that none is the briskest of all.\footnoteB{i.e., every man has his match.}\evb\evg


\bvg\bva\alst{O}rða þęira, \hld\ es maðr \alst{ǫ}ðrum sęgir, &
\ind opt hann \alst{g}jǫld of \alst{g}etr.\eva

\bvb For those words which man says to another \\
\ind he oft gets recompense.\evb\evg


\bvg\bva \edtrans{\alst{M}ikils til}{Much too}{\Afootnote{written as one word \emph{mikilsti} \Regius}} snimma \hld\ kom’k í \alst{m}arga staði, &
\ind en til \alst{s}íð í \alst{s}uma; &
\alst{ǫ}l vas drukkit, \hld\ sumt vas \alst{ȯ}-lagat; &
\ind sjaldan hittir \alst{l}ęiðr í \alst{l}ið.\eva

\bvb Much too early I came to many places, \\
\ind and too late to some: \\
The ale was drunk up, some was unbrewed— \\
\ind seldom finds the loathed his place.\footnoteB{i.e., “there are no wrong times, only wrong people”.}\evb\evg


\bvg\bva\alst{H}ér ok \alst{h}var \hld\ myndi mér \alst{h}ęim of boðit, &
\ind ef þyrpta’k at \alst{m}ǫ́lun-gi \alst{m}at, &
eða \alst{t}vau lę́r hęngi \hld\ at hins \alst{t}ryggva vinar, &
\ind þar’s ek hafða \alst{ęi}tt \alst{e}tit.\eva

\bvb Here and there would I to a home be invited, \\
\ind if at meal-time I needed no food; \\
or if two hams should hang at the trusty friend’s [home], \\
\ind where I had eaten one.\footnoteB{Not everyone is hospitable, especially with regards to food, which was scarce and closely watched among the Norse subsistence farmers.  The poet notes that even a “trusty friend” (possibly sarcastic) would invite him over more often if he brought more food than he ate.}\evb\evg


\bvg\bva\alst{Ę}ldr es batstr \hld\ með \alst{ý}ta sonum &
\ind ok \alst{s}ólar \alst{s}ýn, &
\alst{h}ęilyndi sitt, \hld\ ef maðr \alst{h}afa náir, &
\ind án við \alst{l}ǫst at \alst{l}ifa.\eva

\bvb Fire is best among the sons of men, \\
\ind and the sight of the sun; \\
one’s good health, if he manage to keep it— \\
\ind {[and]} living free from vice.\evb\evg


\bvg\bva\alst{E}s-at maðr \alst{a}lls \edtrans{ve-sall}{unblessed}{\Bfootnote{Or ‘woe-blessed’.  I have elsewhere translated this word as ‘wretched’, but have presently rendered it this way to show the etymological relationship.  The second element in this compound is \emph{sę́ll}, which lacks i-umlaut due to a shortening of the vowel before the umlaut became phonemic.  The ancestral Proto-Norse forms would be \emph{*sāliʀ} and \emph{*wajē-sāliʀ}.  Cf. ᚹᚨᛃᛖ-ᛗᚨᚱᛁᛉ \emph{wajē-mariʀ} ‘infamous’ on the Tjurkö bracteate, where the second element is the ancestor of ON \emph{mę́rr} ‘renowned, famous’; the expected descendant \emph{*ve-marr} is not attested.
I have chosen to translate \emph{sę́ll} as ‘blessed’, but it is not a past participle and could also be rendered as ‘lucky’ or ‘blissful’.  It carries a certain sense of innateness that is foreign to modern Western culture.  Thus a king whose land experiences bountiful harvests (\emph{ár}) is said to be \emph{ár-sę́ll} ‘blessed with harvests’, while one whose kingdom is at peace (\emph{friðr}) is said to be \emph{frið-sę́ll} ‘blessed with peace’.  In this worldview the state of the realm is not due to uncontrollable environmental or political factors, but rather arises from the very person of the king (TODO: Reference PCRN chapter).}}, \hld\ þótt sé \alst{i}lla hęill, &
\ind \alst{s}umr es af \edtext{\alst{s}onum}{\lemma{sonum \dots\ frę́ndum ‘sons \dots\ kinsmen’}\Bfootnote{Cf. st. 72 below, which stresses the importance of sons and kinsmen.}} \alst{s}ę́ll, &
sumr af \alst{f}rę́ndum, \hld\ sumr af \alst{f}é ǿrnu, &
\ind sumr af \alst{v}erkum \alst{v}ęl.\eva

\bvb Man is not all unblessed, though he of poor health be: \\
\ind someone is blessed with sons; \\
someone with kinsmen, someone with ample \inx[C]{fee}, \\
\ind someone with works done well.\evb\evg


\bvg\bva Bętra ’s \alst{l}ifðum, \hld\ \edtrans{an séi ȯ-\alst{l}ifðum}{than with the unliving}{\Afootnote{emend.; \emph{⁊ ſęl lıfðo}m \Regius.}\Bfootnote{The reading of \Regius, which would be normalized as \emph{ok sę́l-lifðum} ‘and for the blessed living’, is metrically defect since \emph{sę́l-} is strongly stressed and should carry alliteration.  For the original form of the line we may instead cf. \Fafnismal\ 30: \emph{Hvǫtum ’s bętra \hld\ an sé ȯ-hvǫtum} ‘It’s better for the brisk than it may be for the unbrisk’.  The corruption has probably happened in the following way: \emph{*en} (younger form of \emph{an} ‘than’) in the prototype was misinterpreted as \emph{en} ‘and, but’ and copied as \emph{⁊} (the tironian \emph{et}), while \emph{*séı ólıfðo}m (probably with the words cramped together) became \emph{sęl lıfðo}m.}}, &
\ind \edtrans{ęy getr \alst{k}vikr \alst{k}ú}{always gets the quick a cow}{\Bfootnote{i.e., “new opportunities always present themselves for the living”.  A reference to the cattle-based economy (see also st. 76), the cow being used as a metonym:  (cf. churchly English ‘the \emph{quick} and the dead’, i.e. ‘the \emph{living} and the dead’).}}; &
\alst{ę}ld sá’k \alst{u}pp brinna \hld\ \alst{au}ðgum manni fyr, &
\ind en úti vas \alst{d}auðr fyr \alst{d}urum.\eva

\bvb It’s better for the living than it may be for the unliving: \\
\ind ever the quick gets the cow. \\
A fire I saw burning high for a wealthy man, \\
\ind but outside he was dead before the doors.\footnoteB{The fire is presumably the man’s funeral pyre, on which a considerable amount of his wealth has been spent; according to ibn Fadlan (TODO) two thirds of a dead chieftain’s estate was spent on his funeral.  One notes the contrastive \emph{en} ‘but’ and may understand it as follows: “I saw a lavish funeral held for a man, but he was still dead.”  This interpretation is supported by the \Havamal\ 71 below, which expresses the same sentiment.}\evb\evg


\bvg\bva\alst{H}altr ríðr \alst{h}rossi, \hld\ \alst{h}jǫrð rekr \alst{h}andar vanr, &
\ind \alst{d}aufr vegr ok \alst{d}ugir; &
\alst{b}lindr es \alst{b}ętri, \hld\ an \alst{b}ręnndr séi; &
\ind \alst{n}ýtr mann-gi \alst{n}ás.\eva

\bvb A halt man rides a horse; a handless drives a herd; \\
\ind a deaf fights and avails. \\
Blind is better than be burned; \\
\ind no man has use for a corpse.\evb\evg


\bvg\bva \edtrans{\alst{S}onr es bętri}{A son is better}{\Bfootnote{i.e. it is better for a man to have a son and heir than not, even if the father should die some time before he is born. The son can further his father’s lineage and memory (as exemplified by the raising of a “beat-stone”), and as the poet says, it is rare for a non-relative to do so.}}, \hld\ þótt sé \alst{s}íð of alinn &
\ind ęptir \alst{g}inginn \alst{g}uma; &
sjaldan \edtrans{\alst{b}autar-stęinar}{beat-stones}{\Bfootnote{Large standing stones raised in memory of someone.  Numerous such stones with runic inscriptions are known from migration period Norway, often near grave fields.  Some hold only single personal names or short phrases, like the stone from Sunde in Sunnfjord, western Norway (signum \emph{KJ 90}): ᚹᛁᛞᚢᚷᚨᛊᛏᛁᛉ \textbf{widugastiʀ} ‘Woodguest’, or the one from Bø in Rogaland, southwestern Norway (signum \emph{KJ 78}): ᚺᚾᚨᛒᛞᚨᛊ ᚺᛚᚨᛁᚹᚨ \textbf{hnabdas hlaiwa} ‘Naved’s grave’.  Others hold longer inscriptions, like the one from Kjølevik in Rogaland (signum \emph{KJ 75}): ᚺᚨᛞᚢᛚᚨᛁᚲᚨᛉ ᛖᚲᚺᚨᚷᚢᛊᛏᚨᛞᚨᛉ ᚺᛚᚨᚨᛁᚹᛁᛞᛟᛗᚨᚷᚢᛗᛁᚾᛁᚾᛟ \textbf{hadulaikaz ekhagustadaz hlaaiwidomaguminino} ‘Hathlac [lies here].  I, Haystald, buried my lad.’}} \hld\ standa \alst{b}rautu nę́r, &
\ind nema ręisi \alst{n}iðr at \alst{n}ið.\eva

\bvb A son is better, though he late be born \\
\ind after a passed-on man. \\
Seldom beat-stones stand near the road, \\
\ind save by kinsman for kinsman raised.\evb\evg


\bvg\bva \edtext{\edtrans{\alst{T}vęir ’ru ęins hęrjar}{Two are of one host}{\Bfootnote{i.e. “the tongue and head belong to the same body (but the former often leads to the latter’s demise).” — \emph{hęrjar} is an inflected form of \emph{hęrr} ‘host, army’, but its function is ambiguous; it can either be (1) the gen. sg., as adopted here, or (2) the nom. pl. ‘harriers, raiders’ (cf. \emph{ęin-hęrjar} ‘\inx[G]{Oneharriers}’) which would translate as “two are the destroyers of one”, i.e. “the tongue and head often lead to the demise of the body”.}}, \hld\ \edtrans{\alst{t}unga es hǫfuðs bani}{the tongue is the head’s bane}{\Bfootnote{Formulaic or proverbial.  Cf. the Old Swedish “Heathen Law”, which describes how a duel should be conducted following an insult to a man’s honour (my norm. and trans. following \textcite{Läffler1879}): \emph{Fallr þann orð havr givit—glǿpr orða vęrstr,} tunga hovuð-bani—\emph{liggi i ú·gildum akri} ‘If he falls who has given the [insulting] word—an insult is the worst of words, \emph{the tongue the head-bane}—may he lie in an unhallowed field.’}}; &
mér ’s í \alst{h}eðin \alst{h}vęrn \hld\ \edtrans{\alst{h}andar}{a hand}{\Bfootnote{i.e. a hand holding a dagger.}} vę́ni.}{\lemma{ALL}\Bfootnote{The whole st. fits poorly in context, and the metre and style are very out of place; it is probably a later insert.}}\eva

\bvb Two are of one host: the tongue is the head’s bane; \\
in every cloak I expect a hand.\evb\evg


\bvg\bva\alst{N}ǫ́tt verðr fęginn, \hld\ sá’s \alst{n}esti trúir, &
\ind \edtrans{\alst{sk}ammar ’ru \alst{sk}ips ráar}{short are a ship’s sailyards}{\Bfootnote{TODO: Write about the varying interpretations (Finnur, Cleasby, Skp) of this line.}}, &
\ind \alst{h}verf es \alst{h}aust-gríma; &
\alst{f}jǫlð of viðrir \hld\ ȧ \edtrans{\alst{f}imm dǫgum}{five days}{\Bfootnote{i.e. “in a week” (which was originally five days long), paralleling “month” in the next line.  See note to st. 51 and Encyclopedia.}}, &
\ind en \alst{m}ęir ȧ \alst{m}ánaði.\eva

\bvb At night he rejoices, who trusts in his provisions; \\
\ind short are a ship’s sailyards; \\
\ind shifty is a stormy fall night. \\
The weather changes much in \inx[C]{five days}; \\
\ind even more in a month.\evb\evg


\bvg\bva\alst{V}ęit-a hinn, \hld\ es \alst{v}ę́tki \alst{v}ęit, &
\ind \edtrans{margr verðr \edtrans{af \alst{au}rum}{from wealth}{\Afootnote{emend. from meaningless \emph{†aflꜹðrom†} \Regius}} \alst{a}pi}{many a man turns an ape from wealth}{\Bfootnote{Cf. \Solarljod\ 34/4: \emph{margan hefr auðr apat} ‘wealth has aped many a man’, which also lends support to the emendation.}}; &
maðr es \alst{au}ðigr, \hld\ annarr \alst{ȯ}-auðigr, &
\ind skyli-t þann \alst{v}ítka \alst{v}áar.\eva

\bvb The one knows not, who nothing knows: \\
\ind many a man turns an \inx[C]{ape} from wealth. \\
A man is wealthy, another not wealthy; \\
\ind one oughtn’t to curse him for his woe.\evb\evg


\bvg\bva\alst{D}ęyr \edtext{fé}{\lemma{fé \dots\ frę́ndr ‘Fee \dots\ kinsmen’}\Bfootnote{The import of this merism may be less clear to the modern reader. In the Germanic Iron Age farming society a man’s wealth was reckoned by how many heads of cattle (and the Norman loan-word \emph{cattle} is itself the same word as \emph{capital}) he owned (cf. st. 70 above, where “a cow” is used to express “an opportunity”), and his social power by the number of able male relatives ready to side with him in conflict (cf. st. 72 above and TODO: reference?). The meaning is thus: all your power will pass away, and so too must you, but if you leave a good reputation behind it can live on. For Indo-European poetic analogues, see \textcite[99\psqq]{West2007}.}}, \hld\ \alst{d}ęyja frę́ndr, &
\ind dęyr \alst{s}jalfr hit \alst{s}ama; &
en \alst{o}rðs-tírr \hld\ dęyr \alst{a}ldri-gi &
\ind hvęim’s sér \alst{g}óðan \alst{g}etr.\eva

\bvb \inx[C]{fee}[Fee] dies, kinsmen die, \\
\ind oneself dies the same [way]; \\
but a word-glory never dies, \\
\ind for whomever gets himself a good one.\evb\evg


\bvg\bva\alst{D}ęyr fé, \hld\ \alst{d}ęyja frę́ndr, &
\ind dęyr \alst{s}jalfr hit \alst{s}ama; &
\alst{e}k vęit \alst{ęi}nn \hld\ at \alst{a}ldri-gi dęyr: &
\ind \alst{d}ómr of \alst{d}auðan hvęrn.\eva

\bvb Fee dies, kinsmen die, \\
\ind oneself dies the same [way]. \\
I know one that never dies: \\
\ind the \inx[C]{Doom} o’er each man dead.\evb\evg

\sectionline

{\small It is likely that the original Guest-Strand ended here.  The three following stanzas, especially the third, are poorly placed and seem like later inserts.}

\sectionline

\bvg\bva\alst{F}ullar grindr \hld\ sá’k fyr \alst{F}itjungs sonum, &
\ind nú bera þęir \edtrans{\alst{v}ánar \alst{v}ǫl}{the staff of hope}{\Bfootnote{A beggar’s staff.}}; &
svá es \alst{au}ðr \hld\ sęm \alst{au}ga-bragð, &
\ind hann es \alst{v}altastr \alst{v}ina.\eva

\bvb Full pens I saw for the sons of Fitting; \\
\ind now they carry the staff of hope. \\
So is wealth like the twinkling of an eye: \\
\ind it is the ficklest of friends.\evb\evg


\bvg\bva\alst{Ȯ}-snotr maðr \hld\ es \alst{ęi}gnask getr &
\ind \alst{f}é eða \alst{f}ljóðs mun-úð; &
\alst{m}etnaðr hǫ́num þróask, \hld\ en \alst{m}an-vit aldri-gi; &
\ind framm gęngr hann \alst{d}rjúgt í \alst{d}ul.\eva

\bvb The unclever man who comes to own \\
\ind fee or a girl’s loving grace: \\
his pride flourishes, but never his manwit; \\
\ind he goes forth far in folly.\evb\evg


\bvg\bva Þat ’s þȧ \alst{r}ęynt, es þú at \edtext{\alst{r}únum spyrr, \hld\ hinum \alst{r}ęgin-kunnum}{\lemma{rúnum \dots\ ręgin-kunnum ‘runes \dots\ born of the Reins’}\Bfootnote{This expression also appears on the C4th–6th Noleby stone (in the acc. sg. \emph{rúnó ragina-kundó} ‘a rune born of the Reins’), which proves that the Eddic rune-magic is (at least in part) founded in oral tradition going back to the Heathen age. See also Encyclopedia \inx[C]{rune}.}}, &
\ind \edtext{þęim’s \alst{g}ørðu \alst{g}inn-ręgin &
\ind ok \alst{f}áði \alst{F}imbul-þulr;}{\lemma{þęim’s \dots\ Fimbul-þulr ‘those which \dots\ Fimble-Thyle’}\Bfootnote{Formulaic. Cf. st. 142 where these two lines occur almost identically, but in reverse order.}} &
\ind \alst{þ}ȧ hęfr hann batst, ef hann \alst{þ}ęgir.\eva

\bvb That is then proven, which thou learnest from the runes, those born of the Reins, \\
\ind those which the \inx[G]{yin-Reins} made, \\
\ind and the Fimble-Thyle \name{= Weden} painted.— \\
\ind Then he has it best, if he shuts up.\footnoteB{This stanza, which deals with runic magic and shares expressions with sts. in the Rune-Tally section (beginning with st. 138 below), hardly fits in its current place.  The last line with its shift in person is likely to be a later insert.}\evb\evg

\sectionline

\section{Scattered stanzas of practical advice}

The following stanzas are distinguished by the prevalence of \Malahattr\ and the common subject matter.

\sectionline

\bvg\bva At \alst{k}veldi skal dag lęyfa, \hld\ \alst{k}onu es bręnnd es, &
\alst{m}ę́ki es ręyndr es, \hld\ \alst{m}ęy es gefin es, &
\alst{í}s es \alst{y}fir kømr, \hld\ \alst{ǫ}l es drukkit es.\eva

\bvb At evening shall one praise day, a woman when she is burned, \\
a sword when it is tried, a maiden when she is given,\footnoteB{i.e. in marriage.} \\
ice when one crosses over, ale when it is drunk.\evb\evg


\bvg\bva Í \alst{v}indi skal \alst{v}ið hǫggva, \hld\ \edtrans{\alst{v}eðri}{weather}{\Bfootnote{i.e. ‘in good weather’; elsewhere the word \emph{veðr} typically means ‘storm’, but that can hardly be the sense here.}} ȧ sę́ róa, &
\alst{m}yrkri við \alst{m}an spjalla— \hld\ \alst{m}ǫrg eru dags augu— &
ȧ \alst{sk}ip skal \alst{sk}riðar orka, \hld\ en ȧ \alst{sk}jǫld til hlífar, &
\alst{m}ę́ki til hǫggs, \hld\ en \alst{m}ęy til kossa.\eva

\bvb In wind shall one cut wood, in weather row at sea, \\
in darkness speak with a maiden—many are the eyes of day. \\
A ship shall one have for speed, and a shield for protection; \\
a sword for striking, and a maiden for kisses.\evb\evg


\bvg\bva Við \alst{ę}ld skal \alst{ǫ}l drekka, \hld\ en ȧ \alst{í}si skríða, &
\alst{m}agran \alst{m}ar kaupa, \hld\ en \alst{m}ę́ki saurgan, &
\alst{h}ęima \alst{h}ęst fęita, \hld\ en \alst{h}und ȧ búi.\eva

\bvb One shall drink ale by fire and skate on ice; \\
buy a starved stallion and a rusty sword; \\
fatten the horse at home and the hound in its dwelling.\evb\evg


\bvg\bva\alst{M}ęyjar orðum \hld\ skyli \alst{m}ann-gi trúa, &
\ind né því’s \alst{k}veðr \alst{k}ona; &
\edtext{\edtext{því-at}{\Afootnote{om. \FostrbroedhraSaga}} ȧ \alst{h}verfanda \alst{h}véli \hld\ \edtext{vǫ́ru}{\Afootnote{\emph{er} \FostrbroedhraSaga}} þęim \edtrans{\alst{h}jǫrtu skǫpuð}{hearts shaped}{\Afootnote{\emph{hjarta skapat} ‘heart shaped’ \FostrbroedhraSaga}}, &
\ind \edtext{\alst{b}rigð}{\Afootnote{ok brigð \FostrbroedhraSaga}} í \alst{b}rjóst of \edtext{lagit}{\Afootnote{\emph{laginn} \FostrbroedhraSaga}}.}{\lemma{þvít \dots\ lagið}\Bfootnote{Quoted in slightly divergent form in \FostrbroedhraSaga\ (Thott 1768 4°\textsuperscript{x}, fol. 210r) introduced with the words: \emph{Kom honum þá í hug kviðlingr sá, er kveðinn hafði verit um lausungar-konur:} ‘And then he remembered the ditty which had been composed about loose women:’}}\eva

\bvb A maiden’s words should no man trust, \\
\ind nor that which a woman speaks. \\
For on a whirling wheel their hearts were shaped; \\
\ind fickleness laid in their breasts.\evb\evg


\bvg\bva\alst{B}restanda \alst{b}oga, \hld\ \alst{b}rinnanda loga, &
\alst{g}ínanda ulfi, \hld\ \alst{g}alandi krǫ́ku, &
\alst{r}ýtanda svíni, \hld\ \alst{r}ót-lausum viði, &
\alst{v}axanda \alst{v}ági, \hld\ \alst{v}ellanda katli,\eva

\bvb In bursting bow, in burning flame, \\
in yawning wolf, in crowing crow, \\
in roaring swine, in rootless tree, \\
in waxing wave, in boiling kettle,\evb\evg


\bvg\bva\alst{f}ljúganda \alst{f}lęini, \hld\ \alst{f}allandi bǫ́ru, &
\alst{í}si \alst{ęi}n-nę́ttum, \hld\ \alst{o}rmi hring-lęgnum, &
\alst{b}rúðar \alst{b}ęð-mǫ́lum \hld\ eða \alst{b}rotnu sverði, &
\alst{b}jarnar lęiki \hld\ eða \alst{b}arni konungs,\eva

\bvb in flying spear, in falling billow, \\
in one-night old ice, in coiled-up serpent, \\
in bride’s bed-speech, or in broken sword, \\
in bear’s play, or in king’s child,\evb\evg


\bvg\bva\alst{s}júkum kalfi, \hld\ \alst{s}jalf-ráða þrę́li, &
\edtrans{\alst{v}ǫlu \alst{v}il-mę́li}{in wallow’s pleasing speech}{\Bfootnote{i.e. in a favourable prophecy (\inx[C]{spae}).}}, \hld\ \alst{v}al ný-fęldum.\eva

\bvb in sick calf, in self-willing thrall, \\
in wallow’s pleasing speech, in newly felled corpses,\evb\evg

\sectionline

In \Regius\ the following two sts. come in the opposite order, but it seems probable from its \Malahattr\ meter and the dative case of the words that 89 should follow 87.  On the other hand st. 88, with its \Ljodahattr\ meter and self-enclosed form seems a separate composition, and was probably inserted after 87 due to its first line (\emph{akri ár-sǫ́num}), which is also in the dative.

\sectionline

\bvg\bva[89]\alst{b}róður-\alst{b}ana sínum \hld\ þótt ȧ \alst{b}rautu mǿti, &
\alst{h}úsi \alst{h}alf-brunnu, \hld\ \alst{h}ęsti al-skjótum, &
þȧ ’s \alst{jó}r \alst{ȯ}-nýtr, \hld\ ef \alst{ęi}nn fótr brotnar; &
verðr-it maðr svá \alst{t}ryggr \hld\ at þessu \alst{t}rúi ǫllu!\eva

\bvb in one’s brother’s bane—though on the road ye meet— \\
in half-burned house, in all-fleet horse— \\
the steed is useless if one foot breaks. \\
No man be so trusting that he trust in all this!\evb\evg\stepcounter{stanza}


\bvg\bva[88]\alst{A}kri \alst{á}r-sǫ́num \hld\ trúi \alst{ę}ngi maðr, &
\ind né til \alst{s}nimma \alst{s}yni; &
\alst{v}eðr rę́ðr akri, \hld\ en \alst{v}it syni; &
\ind \alst{h}ę́tt es þęira \alst{h}várt.\eva

\bvb In an early sown field ought no man to trust, \\
\ind nor too soon in a son. \\
The weather rules the field and the wits the son: \\
\ind there is risk to them both.\evb\evg\stepcounter{stanza}


\bvg\bva Svá ’s \alst{f}riðr kvinna \hld\ þęira’s \alst{f}látt hyggja, &
sęm \alst{a}ki \alst{jó} ȯ-bryddum \hld\ ȧ \alst{í}si hǫ́lum &
\alst{t}ęitum, \alst{t}vé-vetrum \hld\ ok sé \alst{t}amr illa, &
eða í \alst{b}yr óðum \hld\ \alst{b}ęiti stjórn-lausu, &
eða skyli \alst{h}altr \alst{h}ęnda \hld\ \alst{h}ręin \edtrans{í þá-fjalli}{on a thawing fell}{\Bfootnote{i.e. in springtime, when the melting ice on the ground is most slippery.}}.\eva

\bvb So is those women’s love who falsely think \\
like one rode an unshod horse on slippery ice— \\
a merry one, two winters old, and badly tamed— \\
or in mad wind tacked a rudderless [ship], \\
or a halt man should catch a reindeer on a thawing fell.\evb\evg

\sectionline

\section{Weden’s failed seduction of Billing’s daughter}

The following sts. are united by their meter, \Ljodahattr\ (unlike most of the preceding sts., see introduction to them above), style and content.  The strand begins with general maxims about love and relations between the sexes, before moving on to the narrative about Billing’s daughter.

\sectionline

\bvg\bva\alst{B}ęrt nú mę́li’k, \hld\ því-at \edtrans{\alst{b}ę́ði}{both}{\Bfootnote{i.e. both sides, both sexes.  The (male) poet declares that he will not attack the fair sex unfairly; he is also aware of men’s faults.}} vęit’k, &
\ind brigðr es \alst{k}arla hugr \alst{k}onum, &
\edtext{þȧ \alst{f}ęgrst mę́lum, \hld\ es \alst{f}lást hyggjum}{\lemma{fęgrst mę́lum \dots\ flást hyggjum ‘speak fairest \dots\ think falsest’}\Bfootnote{Formulaic.  Cf. st. 45.}}; &
\ind \edtrans{þat tę́lir \alst{h}orska \alst{h}ugi}{that entraps sharp minds}{\Bfootnote{i.e., love (or sexual infatuation—the poet does not distinguish between them) turns even wise men into liars or otherwise dishonest persons.  Cf. \Malshattakvadi\ 20/1–2, which is probably partly based on this stanza:
\emph{Ást-blindir ’ru seggir svá \hld\ sumir, at þykkja mjǫk fás gá;} \\
\emph{þannig verðr um man-sǫng mę́lt: \hld\ marga hefr þat hyggna tę́lt.}
‘Some men are so love-blind, that they seem to heed very little; // for that sake it is said about love-song: many thinking men has it entrapped.’}}.\eva

\bvb Plainly I now speak, for I know both: \\
\ind fickle is men’s thought towards women. \\
We then speak fairest when we think falsest; \\
\ind that entraps sharp minds.\evb\evg


\bvg\bva \edtrans{\alst{F}agrt skal mę́la}{Fairly shall speak}{\Bfootnote{Formulaic. Cf. st. 45.}} \hld\ ok \alst{f}é bjóða, &
\ind sá’s vill \alst{f}ljóðs ǫ́st \alst{f}ȧa, &
\alst{l}íki \alst{l}ęyfa \hld\ hins \alst{l}jósa mans, &
\ind \edtrans{sá \alst{f}ę̇r, es \alst{f}ríar}{he gets, who woos}{\Bfootnote{i.e., “he who courts her gets her”.}}.\eva

\bvb Fairly shall speak, and offer \inx[C]{fee}, \\
\ind he who will get a woman’s love; \\
praise the body of the bright girl; \\
\ind he gets, who woos.\evb\evg


\bvg\bva\alst{Á}star firna \hld\ skyli \alst{ę}ngi maðr &
\ind \alst{a}nnan \alst{a}ldri-gi; &
opt fȧa ȧ \alst{h}orskan, \hld\ es ȧ \alst{h}ęimskan né fȧa, &
\ind \edtrans{\alst{l}ost-fagrir \alst{l}itir}{lust-fair hues}{\Bfootnote{i.e. a (woman with a) countenance so beautiful that men cannot help but lust after her.}}.\eva

\bvb For [matters of] love should no man \\
\ind ever blame another; \\
oft they seize the sharp when they seize not the foolish, \\
\ind the lust-fair hues.\evb\evg


\bvg\bva\alst{Ęy}-vitar firna, \hld\ es maðr \alst{a}nnan skal, &
\ind þess es of margan \alst{g}ęngr \alst{g}uma; &
\alst{h}ęimska ór \alst{h}orskum \hld\ gęrir \alst{h}ǫlða sonu &
\ind sá hinn \alst{m}átki \alst{m}unr.\eva

\bvb In no way shall man blame another \\
\ind for that which happens to many a man; \\
from sharp to fools are the sons of men made \\
\ind by that mighty thing, love.\evb\evg


\bvg\bva\alst{H}ugr ęinn þat vęit, \hld\ es býr \alst{h}jarta nę́r, &
\ind ęinn es hann \alst{s}ér of \alst{s}efa; &
øng es \alst{s}ótt verri \hld\ hvęim \alst{s}notrum manni &
\ind an sér \alst{ø}ngu at \alst{u}na.\eva

\bvb The mind alone knows what dwells close to the heart; \\
\ind it is alone with its thoughts. \\
No sickness is worse for any clever man \\
\ind than with nothing to be content.\evb\evg


\bvg\bva Þat þȧ \alst{r}ęynda’k, \hld\ es í \alst{r}ęyri sat’k, &
\ind ok vę̇tta’k \alst{m}íns \alst{m}unar, &
\alst{h}old ok \alst{h}jarta \hld\ vas mér hin \alst{h}orska mę́r, &
\ind þęygi hana at \alst{h}ęldr \alst{h}ęf’k.\eva

\bvb I experienced it then, as I sat in the reed, \\
\ind and awaited my love. \\
My flesh and heart was that sharp maiden— \\
\ind I have her none the more.\evb\evg


\bvg\bva\alst{B}illings \edtrans{męy}{maiden}{\Bfootnote{i.e. unmarried (virgin) daughter.}} \hld\ ek fann \alst{b}ęðjum ȧ &
\ind \alst{s}ól-hvíta \alst{s}ofa; &
\alst{ja}rls \alst{y}nði \hld\ þȯtti mér \alst{ę}kki vesa &
\ind nema við þat \alst{l}ík at \alst{l}ifa.\eva

\bvb Billing’s maiden I found on the beds, \\
\ind sun-white, asleep. \\
An earl’s pleasure seemed me naught to be, \\
\ind save living alongside that body.\evb\evg


\bvg\bva\speakernote{[Billings mę́r:]}„\alst{Au}k nę́r \alst{a}ptni \hld\ skalt \alst{Ó}ðinn koma, &
\ind ef vilt þér \alst{m}ę́la \alst{m}an, &
\alst{a}llt eru \alst{ȯ}-skǫp, \hld\ nema \alst{ęi}n vitim &
\ind \alst{s}likan lǫst \alst{s}aman.“\eva

\bvb\speakernoteb{[Billing’s maiden:]}
“And by evening shalt thou, Weden, come, \\
\ind if thou wilt get for thee the girl [me]; \\
everything’s misshapen unless we alone should know, \\
\ind such a vice together.”\evb\evg


\bvg\bva\alst{A}ptr ek hvarf \hld\ ok \alst{u}nna þȯttumk &
\ind \edtrans{\alst{v}ísum \alst{v}ilja frȧ}{away from my wise will}{\Bfootnote{i.e., “against my better judgment”; the wise choice would have been to walk away.}}; &
\alst{h}itt ek \alst{h}ugða, \hld\ at \alst{h}afa mynda’k &
\ind \alst{g}ęð hęnnar allt ok \alst{g}aman.\eva

\bvb Back I turned—and thought myself in love— \\
\ind away from my wise will; \\
\emph{this} I thought: that I would have \\
\ind her senses all, and pleasure.\evb\evg


\bvg\bva Svá kom’k \alst{n}ę́st, \hld\ at hin \edtrans{\alst{n}ýta}{useful}{\Bfootnote{Sarcastic. Billing’s daughter had apparently summoned a lynch mob.}} vas &
\ind \alst{v}íg-drótt ǫll of \alst{v}akin, &
með \alst{b}rinnǫndum ljósum \hld\ ok \edtrans{\alst{b}ornum viði}{carried sticks}{\Bfootnote{lit. ‘carried wood’; the mob was armed with clubs.}}, &
\ind svá vas mér \edtrans{\alst{v}íl-stígr}{sad path}{\Bfootnote{Ambiguous, referring either to the beating he would have received at the hands of the mob, or to his walk of shame away from the hall.  The latter is perhaps more likely.}} of \alst{v}itaðr.\eva

\bvb So I came next, as the useful \\
\ind war-troop was all awake; \\
with burning lights and with carried sticks; \\
\ind so a sad path was marked out for me.\evb\evg


\bvg\bva \edtrans{\alst{Au}k nę́r morni}{And by morning}{\Bfootnote{Mirroring the beginning of st. 97 above.}}, \hld\ es vas’k \alst{ę}nn of kominn, &
\ind þȧ vas \alst{s}al-drótt of \alst{s}ofin; &
\edtrans{\alst{g}ręy ęitt}{A lone bitch}{\Bfootnote{The insult is clearly understood; Weden is compared to a horny dog, and mockingly asked to make love to one—“this is all you get, you dog!”}} þȧ fann’k \hld\ hinnar \edtrans{\alst{g}óðu}{good}{\Bfootnote{Possibly not sarcastic, but rather referring to her chastity.}} konu &
\ind \alst{b}undit \alst{b}ęðjum ȧ.\eva

\bvb And by morning when I had come again, \\
\ind then was the hall-troop asleep. \\
A lone bitch I then found, by the good woman \\
\ind bound on the beds.\evb\evg


\bvg\bva Mǫrg es \edtrans{\alst{g}óð mę́r}{good maiden}{\Bfootnote{A formulaic expression; the “goodness” here refers to faithfulness and chastity.  Cf. \Skirnismal\ 12, TODO.}}, \hld\ ef \alst{g}ǫrva kannar, &
\ind \alst{h}ug-brigð við \alst{h}ali; &
þȧ þat \alst{r}ęynda’k, \hld\ es hit \alst{r}áð-spaka &
\ind tęygða’k ȧ \alst{f}lę́rðir \alst{f}ljóð; &
\alst{h}ǫ́ðungar \alst{h}vęrrar \hld\ lęitaði mér hit \alst{h}orska man &
\ind ok hafða’k þess \alst{v}ę́t-ki \alst{v}ífs.\eva

\bvb Many a good maiden—if one comes to know her well— \\
\ind is heart-fickle towards men. \\
I found that out when the counsel-clever \\
\ind lady into sins I lured: \\
all kinds of disgraces that sharp girl sought out for me, \\
\ind and I had naught of the woman.\evb\evg

\sectionline

\section{Weden’s theft of the Mead of Poetry (104–110)}

The intricate myth of how Weden came to own the Mead of Poetry is told more fully in \Skaldskaparmal\ 5–6. That narrative goes as follows, with minor details left out:
After the war between the Eese and Wanes, the two tribes of gods reconcile through spitting into a vat. Not wanting to discard this token of their truce, they instead create a man out of the spit, calling him \inx[P]{Quasher}; he is so wise that he can answer any question posed to him, and so travels around the world in order to share his wisdom with humans.
Quasher eventually comes to the dwelling of two dwarfs, Fealer and Galer. They kill him and drain his blood into three vessels: two vats named Soon and Bothem, and a kettle named \inx[P]{Woderearer}. Through mixing the blood with honey they make a mead, with the power to turn anyone who drinks from it “a scold or man of learning (\emph{skald eða frǿða-maðr})”. The dwarfs then lie to the Eese about the murder, telling them that Quasher drowned in his own wisdom.
Some time later, the dwarfs murder an ettin named \inx[P]{Gilling} and his wife. Gilling’s son, \inx[P]{Sutting}, learns of this and prepares to drown the dwarfs. In exchange for their lives and as recompense for his father’s slaying, the dwarfs offer Sutting the “dear mead” (\emph{mjǫðinn dýra}; cf. here sts. 105 and 140). Sutting accepts the ransom and takes the mead home with him. He makes his daughter \inx[P]{Guthlathe} guard it.
Some time later, Weden is out journeying, and finds nine thralls mowing hay. He sharpens their scythes with a special whetstone, and the mowing improves greatly. He then throws it in the air and the thralls shortly kill each other over it. By evening Weden comes to the owner of the thralls, Bigh, Sutting’s brother. Bigh laments the death of his workmen, and so Weden, who calls himself \inx[P]{Baleworker}, offers to do the work of the thralls over the summer, in exchange for one drink of Sutting’s mead. Bigh tells him that Sutting alone owns the mead, but that he will accompany Baleworker to Sutting to ask for the drink.
The two arrive at Sutting, who as expected refuses to give any part of the mead away. Baleworker then tells Bigh that he will get to it anyway; he takes out the drill \inx[P]{Rate}, and tells Bigh to drill through the mountain, into the room where the mead is stored. Bigh first attempts to trick him by only drilling halfway, but eventually creates a narrow passage. Baleworker turns himself into a snake and crawls through it; as he does, Bigh tries to strike him the drill, but misses.
After coming through, Baleworker sees Guthlathe watching over the mead. He goes on to sleep with her for three nights, after which she promises him three sips of the mead. With each sip he swallows the contents of one of the three vessels, so that all of the mead ends up in his belly.
Having taken the mead, he dons his eagle-hame and flies away from the mountain. Sutting sees him, takes his own eagle-hame, and gives chase. The Eese see Weden in flight, and set out several large vat on the ground, into which Weden, still flying, spits out the mead. At this point Sutting has almost caught up with him, and so Weden “sends back” (\emph{sęnda aptr}, usually interpreted being sent out from the anus) some of the mead, presumably into his face. This portion becomes the lot of foolish poets (\emph{skald-fífla hlutr}), while the rest of the mead is given to the Eese and to skilled poets (\emph{þęim mǫnnum, er yrkja kunnu} ‘those men who can compose [poetry]’).

The core of this many-twisted myth is old. A close parallel is found in \Rigveda\ hymns 4.26–27. In these two hymns the \emph{soma} plant (who in the Vedic mythology is not just the plant and its resulting drink, but also a god, perhaps somewhat like Quasher) is first held within “a hundred iron forts” (4.27.1c: \emph{śatám púraḥ ā́yasīḥ}) by the archer \emph{Kr̥şānu}, before being stolen by a sweeping falcon. The falcon brings \emph{Soma} to \emph{Manu}, the ancestor of the Aryans and first sacrificer.

The resemblance to the last part of the \Skaldskaparmal\ account should be obvious, but, notably, the detail of the falcon is not found in any of the sts. below. This shows that the narrative of \Skaldskaparmal\ cannot be exclusively based on the sts. here below, but instead also relies on other, now-lost sources. This is also supported by the present sts. leaving out the narratives about Quasher, the two dwarfs, and Baye, along with some subtler narrative differences.

The order of the present sts. follows that of \Regius, their main witness manuscript. The strand begins with some social advice (103), after which the narrative follows (104–110). It is narrated in the first person by Weden himself. The sts. do not tell the myth in chronological order and leave much up to the listener; they are surely composed for an audience that already knows the story. The following narrative details are given:

\begin{enumerate}
	\setcounter{enumi}{103}
	\item Weden visits Sutting’s home, but does not receive a good reception.
	\item Guthlate falls in love with Weden, and gives him a drink of the Mead.
	\item Weden has to bore through the mountains with the drill Rate.
	\item Weden has “bought [the Mead] well”; possibly a euphemistic reference to sleeping with Guthlathe for it.
	\item Guthlathe indeed does sleep with Weden, though not expressely in exchange for the Mead.
	\item The following day (\emph{hins hindra dags}, see note to this word in the edited text below), a group of Rime-Thurses come to Weden’s hall, to ask him whether a Baleworker is among the Gods, or if he has been slain by Sutting.
	\item Switching to the third person (which may indicate that this is his answer to the Rime-Thurses), Weden says that he “thinks” that Weden has sworn an oath, but that his words cannot be trusted. After the “simble” (i.e. drinking feast, banquet; probably referring to the drink of the Mead), Weden betrayed Sutting and made Guthlathe weep.
\end{enumerate}

The underlying narrative seems to generally agree with that of \Skaldskaparmal, but unlike its more transactional affair, we here find a stronger emphasis on Weden’s cruel betrayal of Guthlathe. A notable detail not found in \Skaldskaparmal\ is Weden’s oath in st. 109. The content of the oath was most likely that Weden would marry Guthlathe, something supported by the language used (see note to st. 108: \emph{hins hindra dags}). The recipient of the oath, which Weden clearly broke, was either Sutting or Guthlathe. That Weden swore it to Sutting, and thus asked him for Guthlathe’s hand in marriage, may be suggested by the description of Sutting as \emph{svikvinn} ‘betrayed’ in st. 109. This view, however, has an internal narrative problem: in st. 103 Weden describes his interaction with Sutting as poor, and in st. 105 Weden is said to have had to bore through the mountains, but this may just have been to reach Sutting, rather than Guthlathe as in \Skaldskaparmal.
The recipient of the oath being Guthlathe would agree better with the \Skaldskaparmal\ narrative, and Sutting’s betrayer would instead be her.

\sectionline

\bvg\bva Hęima \alst{g}laðr \alst{g}umi \hld\ ok við \alst{g}ęsti ręifr, &
\ind \alst{s}viðr skal of \alst{s}ik vesa; &
\alst{m}innigr ok \alst{m}ǫ́lugr, \hld\ ef vill \alst{m}arg-fróðr vesa; &
\ind opt skal \alst{g}óðs \alst{g}eta; &
\alst{f}imbul-\alst{f}ambi hęitir, \hld\ sá’s \alst{f}átt kann sęgja; &
\ind þat es \alst{ȯ}-snotrs \alst{a}ðal.\eva

\bvb At home shall man be glad and giving with the guest, \\
\ind wise about himself. \\
Of good memory and speech, if he wishes to be many-learned; \\
\ind oft shall he speak of good. \\
A fimble-fool is he called who little can say; \\
\ind that is the unclever man’s nature.\evb\evg


\bvg\bva Hinn \alst{a}ldna \alst{jǫ}tun sótta’k, \hld\ nú em’k \alst{a}ptr of kominn; &
\ind fátt gat’k \alst{þ}ęgjandi \alst{þ}ar; &
\alst{m}ǫrgum orðum \hld\ \alst{m}ę́lta’k í minn frama &
\ind í \alst{S}uttungs \alst{s}ǫlum.\eva

\bvb The old ettin \name{= Sutting} I sought, now am I come back; \\
\ind I got little hearing there. \\
Many words I spoke to my furtherance, \\
\ind in the halls of Sutting.\evb\evg


\bvg\bva\alst{G}unn-lǫð mér of \alst{g}af \hld\ \alst{g}ullnum stóli ȧ &
\ind \alst{d}rykk hins \alst{d}ýra mjaðar; &
\alst{i}ll \alst{i}ð-gjǫld \hld\ lét’k hana \alst{ę}ptir hafa &
\ind síns hins \alst{h}ęila \alst{h}ugar, &
\ind síns hins \alst{s}vára \alst{s}efa.\eva

\bvb \inx[P]{Guthlathe} did give me, on the golden throne, \\
\ind a drink of the dear mead; \\
evil recompense I let her have afterwards, \\
\ind for her whole heart, \\
\ind for her severe affection.\evb\evg


\bvg\bva\alst{R}ata munn \hld\ létumk \alst{r}úms of fȧa &
\ind ok of \alst{g}rjót \alst{g}naga; &
\alst{y}fir ok \alst{u}ndir \hld\ stóðumk \alst{jǫ}tna vegir, &
\ind svá \alst{h}ę́tta’k \alst{h}ǫfði til.\eva

\bvb Rate’s mouth I made to bring me room, \\
\ind and gnaw away at the rocks. \\
Over and under me stood the roads of the ettins \ken{mountains}; \\
\ind so I risked my head.\evb\evg


\bvg\bva \edtext{\alst{V}ęl kęypts hlutar \hld\ hęf’k \alst{v}ęl notit; &
\ind \alst{f}ás es \alst{f}róðum vant; &
því-at \edtrans{\alst{Ó}ð-rǿrir}{Woderearer}{\Bfootnote{One of the vessels in with the Mead of Poetry was held (see introduction to the present section above), here standing in for all the Mead.}} \hld\ es nú \alst{u}pp kominn &
\ind ȧ \alst{a}lda vés \edtrans{\alst{ja}ðar}{rim}{\Bfootnote{metr. emend.; \emph{jarðar} \Regius\ has a long root-syllable, and does not fit grammatically.}}.}{\lemma{Vęl \dots\ jaðar}\Bfootnote{Taken on its own this st. would be somewhat difficult, but in context the import is clear: Weden says that He has made good use of the Mead of Poetry by bringing it to earth, making poetry (and surely likewise other intellectual disciplines) available to men.}}\eva

\bvb The well bought thing \ken*{Mead of Poetry} have I used well— \\
\ind little do the learned lack, \\
for Woderearer is now come up \\
\ind over the rim of the \inx[C]{wigh} of men \ken*{= Middenyard}.\evb\evg


\bvg\bva\alst{I}fi ’s mér \alst{ȧ}, \hld\ at vę́ra’k \alst{ę}nn kominn &
\ind \alst{jǫ}tna gǫrðum \alst{ó}r, &
ef \alst{G}unn-laðar né nyta’k, \hld\ hinnar \alst{g}óðu konu, &
\ind es lǫgðumk \alst{a}rm \alst{y}fir.\eva

\bvb There is doubt in me, if I would yet be come \\
\ind out of the yards of the Ettins, \\
if Guthlathe I had not used, that good woman \\
\ind whom I laid my arm over.\evb\evg


\bvg\bva \edtrans{\alst{H}ins \alst{h}indra dags}{The following day}{\Bfootnote{This is the only occurrence of the comparative \emph{hindra} ‘following, next’ in the Norse (i.e. ‘belonging to Norway and its colonies’) literature. The superlative \emph{hindstr} ‘last, final’ does occur more often (e.g. \emph{indsta sinni} ‘the last time’, with loss of the \emph{h-}; see \CV: \emph{hindri}), and the possible derivative \emph{hindar-dags} ‘day after tomorrow, two days after’ is found twice, both times in the \Gulatingslog, chh. 37 and 266.  If we, however, search in the broader Scandinavian sphere, we find in the Swedish provicial laws an exact equivalent of the present phrase, namely OSwe. \emph{hindra-dagher}, a law-word referring specifically to the ‘day after the wedding’, used both on its own and in the expression \emph{hindra-dags gięf} ‘morning gift’.  If this is indeed the sense in the present stanza, two interpretations are possible: it either (i) refers sarcastically to Weden’s sleeping with Guthlathe (as would be done on the wedding night), or (ii) means that Weden actually married, or promised to marry, Guthlathe.  The latter interpretation may find support in st. 109, see notes there.}} \hld\ gingu \alst{h}rím-þursar &
\alst{H}áva ráðs at fregna, \hld\ \alst{H}áva \alst{h}ǫllu í, &
at \alst{B}ǫl-verki spurðu, \hld\ ef vę́ri með \alst{b}ǫndum kominn &
\ind eða hęfði hǫ́num \alst{S}uttungr of \alst{s}óit.\eva

\bvb The following day went the Rime-Thurses \\
\ind to ask for the High One’s counsel, in the High One’s hall. \\
About Baleworker \name{= Weden} they asked, if he were come among the bonds \ken{gods}, \\
\ind or if Sutting had slain him.\evb\evg


\bvg\bva \edtext{Baug-ęið \alst{Ó}ðinn \hld\ hygg at \alst{u}nnit hafi, &
\ind hvat skal hans \alst{t}ryggðum \alst{t}rúa? &
\alst{S}uttung \alst{s}vikvinn \hld\ hann lét \alst{s}umbli frȧ &
\ind ok \alst{g}rǿtta \alst{G}unn-lǫðu}{\lemma{Baug-ęið \dots\ Gunn-lǫðu ‘A bigh-oath \dots\ brought to tears™}\Bfootnote{The exact narrative referred to in the stanza is hard to pin down, but I find the following most likely: Weden swore an oath on a bigh, its contents being that he would marry Guthlathe. Sutting then hosted a simble (banquet, drinking feast) for the new couple (cf. \emph{hins hindra dags} in st. 108), and Weden slept with her, but after. \emph{svikvinn} ‘betrayed’ and \emph{grǿtta} ‘brought to tears’ are (respectively masc. and fem.) acc. sg. past participles of the transitive verbs \emph{svíkva} ‘to betray’ and \emph{grǿta} ‘to make weep, bring to tears’. I read \emph{lét} as meaning ‘left, abandoned, forsook’.}}.\eva

\bvb A \inx[C]{bigh-oath} I ween that Weden has sworn— \\
\ind how shall one trust his truces? \\
Away from the \inx[C]{simble} he left Sutting betrayed, \\
\ind and Guthlathe, made to weep.\evb\evg

\sectionline

\section{The Speeches of Loddfathomer}

ON \emph{Loddfáfnis mǫ́l}.

A series of advice stanzas addressed to \inx[P]{Loddfathomer}, an otherwise unknown figure who is clearly mythological.  The name is a compound: the first element, \emph{lodd-}, is related to ON \emph{loddari} ‘juggler, tramp’, OE \emph{loddere} ‘pauper, beggar’; the second, \emph{Fáfnir} (\inx[P]{Fathomer}), is the name of a famous Wyrm and literally means ‘embracer’.  This name gives a picture of an archetypal “bumbling fool”; he is taught by Weden, his opposite.

The section division is found in \Regius.  Stanza 111 has a large initial \emph{M}, albeit smaller than those which introduce new chapters and poems, and the beginning of the following section, the \emph{Rune-Tally}, is also clearly marked by an initial.

\sectionline

\bvg\bva Mál ’s at \alst{þ}ylja \hld\ \alst{þ}ular stóli ȧ; &
\ind \alst{U}rðar brunni \alst{a}t &
\alst{s}á’k ok þagða’k, \hld\ \alst{s}á’k ok hugða’k, &
\ind hlýdda’k ȧ \alst{m}anna \alst{m}ál; &
of \alst{r}únar hęyrða’k dǿma, \hld\ né umb \alst{r}ǫ́ðum þǫgðu &
\ind \alst{H}áva \alst{h}ǫllu at, &
\ind \alst{H}áva \alst{h}ǫllu í &
\ind hęyrða’k \alst{s}ęgja \alst{s}vá:\eva

\bvb It’s time to \inx[C]{thill}, upon the \inx[C]{thyle}’s chair. \\
\ind At the \inx[L]{Well of Weird} \\
I saw and shut up; I saw and I thought; \\
\ind I heeded the matters of men. \\
Of runes I heard them speak, nor did they shut up about counsels, \\
\ind at the High One’s hall, \\
\ind in the High One’s hall, \\
\ind I heard them say so:\footnoteB{The speaker, describing himself as a thyle (\emph{þulr} ‘sage, chanter of memorized poetry’), says that he will relate what he has heard said in Walhall. Considering the location, it seems almost certain that the giver of this advice was its owner, \inx[P]{Weden}. The receiver of the advice, \inx[P]{Loddfathomer} (see Encyclopedia for etymologies), is otherwise unknown.}\evb\evg


\bvg\bva\alst{R}ǫ́ðumk þér Loddfáfnir, \hld\ at \alst{r}ǫ́ð nemir, &
\ind \alst{n}jóta munt ef \alst{n}emr, &
\ind þér munu \alst{g}óð ef \alst{g}etr: &
\alst{n}ǫ́tt þú rís-at, \hld\ nema ȧ \alst{n}jósn séir, &
\ind eða \edtrans{lęitir þér \alst{i}nnan \alst{ú}t staðar}{or thou look for thy place outside}{\Bfootnote{Lit. word-for-word “or thou look for thee from within out a place”, which becomes nonsensical.  \emph{lęita sér staðar} ‘look for one’s place’ is a euphemism, i.e. “to relieve oneself”, which was done outside.}}.\eva

\bvb I counsel thee, O Loddfathomer, that thou learn the counsels; \\
\ind thou wilt have use if thou learn, \\
\ind they will be good for thee if thou get: \\
At night do not rise, unless thou be scouting, \\
\ind or thou look for thy place outside.\evb\evg


\bvg\bva\alst{R}ǫ́ðumk þér Loddfáfnir, \hld\ at \alst{r}ǫ́ð nemir, &
\ind \alst{n}jóta munt ef \alst{n}emr, &
\ind þér munu \alst{g}óð ef \alst{g}etr: &
\alst{f}jǫl-kunnigri konu \hld\ skal-at-tu í \alst{f}aðmi sofa, &
\ind svá’t hon \alst{l}yki þik \alst{l}iðum.\eva

\bvb I counsel thee, O Loddfathomer, that thou learn the counsels; \\
\ind thou wilt have use if thou learn, \\
\ind they will be good for thee if thou get: \\
By a \inx[C]{many-cunning} woman’s bosom shalt thou never sleep, \\
\ind lest she lock thee in [her?] limbs.\evb\evg


\bvg\bva Hǫ́n svá \alst{g}ørir \hld\ at \edtrans{\alst{g}ȧir}{heed}{\Bfootnote{The nasal vowel here is based on Elfdalian \emph{gą̊}.}} ęigi &
\ind \alst{þ}ings né \alst{þ}jóðans máls; &
\alst{m}at þú vill-at \hld\ né \alst{m}anns-kis gaman &
\ind fęrr þú \alst{s}orga-fullr at \alst{s}ofa.\eva

\bvb She makes it so that thou heed not \\
\ind \inx[C]{Thing}’s or ruler’s speech; \\
thou hast no wish for food nor any man’s pleasure; \\
\ind thou goest sorrowful to sleep.\evb\evg


\bvg\bva\alst{R}ǫ́ðumk þér Loddfáfnir, \hld\ at \alst{r}ǫ́ð nemir, &
\ind \alst{n}jóta munt ef \alst{n}emr, &
\ind þér munu \alst{g}óð ef \alst{g}etr: &
\alst{a}nnars konu \hld\ tęyg þér \alst{a}ldri-gi &
\ind \edtrans{\alst{ęy}ra-rúnu}{ear-whisperer \ken{lover}}{\Bfootnote{This word is also used in \Voluspa\ 38, in which male seducers of married women are among those being forced to wade through “heavy streams” in the afterlife.}} \alst{a}t.\eva

\bvb I counsel thee, O Loddfathomer, that thou learn the counsels; \\
\ind thou wilt have use if thou learn, \\
\ind they will be good for thee if thou get: \\
Another man’s woman do never tug \\
\ind into becoming thy ear-whisperer \ken{lover}.\evb\evg


\bvg\bva\alst{R}ǫ́ðumk þér Loddfáfnir, \hld\ en \alst{r}ǫ́ð nemir, &
\ind \alst{n}jóta munt ef \alst{n}emr, &
\ind þér munu \alst{g}óð ef \alst{g}etr: &
\edtrans{\alst{f}jalli eða \alst{f}irði}{on fell or firth}{\Bfootnote{i.e. ‘hiking through mountains or travelling at sea’; a very Norwegian expression.  This word pair is a formulaic merism; this is its only poetic attestation, but it is found a few times in the Old Norwegian laws.}}, \hld\ ef þik \alst{f}ara tíðir, &
\ind fȧsk-tu at \alst{v}irði \alst{v}ęl.\eva

\bvb I counsel thee, O Loddfathomer—and thou oughtst to learn the counsels; \\
\ind thou wilt have use if thou learn, \\
\ind they will be good for thee if thou get: \\
on fell or firth—if thou desire to journey— \\
\ind furnish thyself well with food.\evb\evg


\bvg\bva\alst{R}ǫ́ðumk þér Loddfáfnir, \hld\ en \alst{r}ǫ́ð nemir, &
\ind \alst{n}jóta munt ef \alst{n}emr, &
\ind þér munu \alst{g}óð ef \alst{g}etr: &
\alst{i}llan mann \hld\ lát \alst{a}ldri-gi &
\ind \edtext{\alst{ȯ}-hǫpp at þér \alst{v}ita}{\Bfootnote{An unambiguous instance of \emph{v} alliterating with a vowel.}}, &
því-at af \alst{i}llum manni \hld\ fę̇r \alst{a}ldri-gi &
\ind \alst{g}jǫld hins \alst{g}óða hugar.\eva

\bvb I counsel thee, O Loddfathomer—and thou oughtst to learn the counsels; \\
\ind thou wilt have use if thou learn, \\
\ind they will be good for thee if thou get: \\
An evil man do never let \\
\ind know of thy misfortunes; \\
for from an evil man gettest thou never \\
\ind rewards for thy good will.\evb\evg


\bvg\bva\edtrans{\alst{O}far-la}{Sorely}{\Bfootnote{Contraction of \emph{ofar-liga} ‘\CV: high up, in the upper part’, presumably meaning that the words were particularly grievous or insulting, i.e., they “got to him”.  Whether he was murdered or committed suicide is not clear.}} bíta \hld\ sá’k \alst{ęi}num hal &
\ind \alst{o}rð \alst{i}llrar konu, &
\edtrans{\alst{f}lá-rǫ́ð tunga}{a false-counseling tongue}{\Bfootnote{Cf. \Lokasenna\ 31/1: \emph{flǫ́ ’s þér tunga} ‘false is thy tongue’.}} \hld\ varð hǫ́num at \alst{f}jǫr-lagi &
\ind ok þęygi of \alst{s}anna \alst{s}ǫk.\eva

\bvb Sorely biting I saw at a lonely man \\
\ind the words of an evil woman; \\
a false-counseling tongue brought his life to its end, \\
\ind and in no way over a truthful charge.\evb\evg


\bvg\bva\alst{R}ǫ́ðumk þér Loddfáfnir, \hld\ en \alst{r}ǫ́ð nemir, &
\ind \alst{n}jóta munt ef \alst{n}emr, &
\ind þér munu \alst{g}óð ef \alst{g}etr: &
\alst{v}ęitst, ef \alst{v}in átt, \hld\ þann’s \alst{v}ęl trúir, &
\ind \alst{f}ar þú at \alst{f}inna opt; &
því-at \edtrans{\alst{h}rísi vęx \hld\ ok \alst{h}ǫ́u grasi}{with brushwood and with tall grass grows}{\Bfootnote{Identical to \Grimnismal\ 17/1.}} &
\ind \alst{v}egr, es \alst{v}ę́t-ki trøðr.\eva

\bvb I counsel thee, O Loddfathomer—and thou oughtst to learn the counsels; \\
\ind thou wilt have use if thou learn, \\
\ind they will be good for thee if thou get: \\
Thou knowest, if thou have a friend whom thou well trust: \\
\ind journey to find him oft; \\
for with brushwood and tall grass grows \\
\ind the way which no one treads.\evb\evg


\bvg\bva\alst{R}ǫ́ðumk þér Loddfáfnir, \hld\ en \alst{r}ǫ́ð nemir, &
\ind \alst{n}jóta munt ef \alst{n}emr, &
\ind þér munu \alst{g}óð ef \alst{g}etr: &
\alst{g}óðan mann \hld\ tęyg þér at \edtrans{\alst{g}aman-rúnum}{pleasure-runes}{\Bfootnote{Here “rune” appears to carry its root meaning of ‘whisper, counsel, speech’, thus ‘pleasing speech’.  Cf. st. 129 where this word reoccurs.}} &
\ind ok nem \edtrans{\alst{l}íknar-galdr}{liking-galders}{\Bfootnote{i.e. ways of speaking which will make one liked or popular.  For \emph{líkn} ‘liking’ see sts. 8 (with note) and 123.}} meðan \alst{l}ifir.\eva

\bvb I counsel thee, O Loddfathomer—and thou oughtst to learn the counsels; \\
\ind thou wilt have use if thou learn, \\
\ind they will be good for thee if thou get: \\
A good man do tug toward thee with pleasure-runes, \\
\ind and learn liking-galders while thou livest.\evb\evg


\bvg\bva\alst{R}ǫ́ðumk þér Loddfáfnir, \hld\ en \alst{r}ǫ́ð nemir, &
\ind \alst{n}jóta munt ef \alst{n}emr, &
\ind þér munu \alst{g}óð ef \alst{g}etr: &
\alst{v}in þínum \hld\ \alst{v}es aldri-gi &
\ind \alst{f}yrri at \alst{f}laum-slitum. &
\alst{s}org etr hjarta, \hld\ ef þú \edtext{\alst{s}ęgja né náir &
\ind \alst{ęi}n-hvęrjum \alst{a}llan hug}{\lemma{sęgja \dots\ ęin-hvęrjum allan hug ‘tell anyone thy whole mind’}\Bfootnote{Cf. st. 123 which uses almost the same expression.}}.\eva

\bvb I counsel thee, O Loddfathomer—and thou oughtst to learn the counsels; \\
\ind thou wilt have use if thou learn, \\
\ind they will be good for thee if thou get: \\
With thy friend be thou never the first \\
\ind to tear the relation apart. \\
Sorrow will eat thy heart if thou canst not tell \\
\ind anyone thy whole mind.\evb\evg


\bvg\bva\alst{R}ǫ́ðumk þér Loddfáfnir, \hld\ en \alst{r}ǫ́ð nemir, &
\ind \alst{n}jóta munt ef \alst{n}emr, &
\ind þér munu \alst{g}óð ef \alst{g}etr: &
\edtext{\alst{o}rðum skipta \hld\ skalt \alst{a}ldri-gi &
\ind við \edtrans{\alst{ȯ}-svinna \alst{a}pa}{unwise apes}{\Bfootnote{Formulaic; cf. \Grimnismal\ 33, \Fafnismal\ 11.}}}{\lemma{orðum \dots\ apa ‘Words \dots\ apes’}\Bfootnote{Cf. st. 125 which gives similar advice.}},\eva

\bvb I counsel thee, O Loddfathomer—and thou oughtst to learn the counsels; \\
\ind thou wilt have use if thou learn, \\
\ind they will be good for thee if thou get: \\
Words shalt thou never exchange \\
\ind with unwise apes,\evb\evg


\bvg\bva \edtext{því-at af \alst{i}llum manni \hld\ munt \alst{a}ldri-gi &
\ind \alst{g}óðs laun of \alst{g}eta}{\lemma{því-at \dots\ geta ‘For \dots\ praise’}\Bfootnote{Cf. st. 117/6–7.}}, &
en \alst{g}óðr maðr \hld\ mun þik \alst{g}ørva męga &
\ind \edtrans{\alst{l}íkn-fastan}{steadfast in liking}{\Bfootnote{The first element \emph{líkn} ‘liking’ is somewhat difficult; see sts. 8 (with note) and 120.  For the present cpd \textcite{LaFargeGlossary} give a tentative ‘assured of favour’, while \CV\ gives ‘fast in goodwill, beloved’.}} at \alst{l}ofi.\eva

\bvb for from an evil man wilt thou never \\
\ind get a reward for thy goodness, \\
but a good man will know to make thee \\
\ind steadfast in liking by [his] praise.\evb\evg


\bvg\bva\alst{S}ifjum ’s þȧ blandit \hld\ hvęrr es \edtext{\alst{s}ęgja rę́ðr &
\ind \alst{ęi}num \alst{a}llan hug}{\lemma{sęgja \dots\ \alst{ęi}num \alst{a}llan hug ‘tell one man his whole mind’}\Bfootnote{Cf. st. 121 which uses almost the same expression.}}; &
alt es \alst{b}ętra \hld\ an sé \alst{b}rigðum at vesa: &
es-a sá \alst{v}inr ǫðrum \hld\ es \alst{v}ilt ęitt sęgir.\eva

\bvb Kinship is blended whereever one resolves to tell \\
\ind one man his whole mind. \\
Everything is better than to be with the fickle; \\
he is no friend to another who says only that which is wanted.\evb\evg


\bvg\bva\alst{R}ǫ́ðumk þér Loddfáfnir, \hld\ en \alst{r}ǫ́ð nemir, &
\ind \alst{n}jóta munt ef \alst{n}emr, &
\ind þér munu \alst{g}óð ef \alst{g}etr: &
\edtrans{þrimr \alst{o}rðum}{With three words}{\Bfootnote{i.e. ‘not even with three words’. If one understands \emph{orð} to mean ‘speech’, it may be interpreted as that if one says something (the first speech) to which another man responds insultingly (the second speech), one should not respond a third time and turn it into a fight.}} sęnna \hld\ skal-at-tu þér við \alst{v}erra mann; &
\ind opt hinn \alst{b}ętri \alst{b}ilar, &
\ind þȧ’s hinn \alst{v}erri \alst{v}egr.\eva

\bvb I counsel thee, O Loddfathomer—and thou oughtst to learn the counsels; \\
\ind thou wilt have use if thou learn, \\
\ind they will be good for thee if thou get: \\
With three words shalt thou not flyte with a worse man; \\
\ind oft the better man breaks \\
\ind when the worse man strikes.\footnoteB{Cf. st. 121.}\evb\evg


\bvg\bva\alst{R}ǫ́ðumk þér Loddfáfnir, \hld\ en \alst{r}ǫ́ð nemir, &
\ind \alst{n}jóta munt ef \alst{n}emr, &
\ind þér munu \alst{g}óð ef \alst{g}etr: &
\alst{sk}ó-smiðr þú vesir \hld\ né \alst{sk}ępti-smiðr, &
\ind nema \alst{s}jǫlfum þér \alst{s}éir. &
\alst{Sk}ór ’s \alst{sk}apaðr illa \hld\ eða \alst{sk}apt sé rangt, &
\ind þȧ ’s þér \alst{b}ǫls \alst{b}eðit.\eva

\bvb I counsel thee, O Loddfathomer—and thou oughtst to learn the counsels; \\
\ind thou wilt have use if thou learn, \\
\ind they will be good for thee if thou get: \\
Be not a shoe-maker nor shaft-maker, \\
\ind unless thou be one for thyself. \\
The shoe is shaped badly or the shaft be crooked— \\
\ind then for thee a \inx[C]{bale} is bid.\footnoteB{i.e. the customer will place a curse on you if he dislikes the wares.}\evb\evg


\bvg\bva\alst{R}ǫ́ðumk þér Loddfáfnir, \hld\ en \alst{r}ǫ́ð nemir, &
\ind \alst{n}jóta munt ef \alst{n}emr, &
\ind þér munu \alst{g}óð ef \alst{g}etr: &
hvar’s \alst{b}ǫl kant, \hld\ kveð þér \alst{b}ǫlvi at &
\ind ok gef-at þínum \alst{f}jǫ́ndum \alst{f}rið.\eva

\bvb I counsel thee, O Loddfathomer—and thou oughtst to learn the counsels; \\
\ind thou wilt have use if thou learn, \\
\ind they will be good for thee if thou get: \\
Wherever thou knowest a bale, call it a bale against thee, \\
\ind and give not thy enemies peace.\footnoteB{i.e. “if somebody puts a curse on you, do not ignore it, but respond decisively”.  This st. has often been interpreted as a command to call out evil, even when committed towards somebody else, and while there is nothing in it that speaks clearly against that interpretation, it does not agree with the general spirit of the \Havamal, which is one of caution and shrewdness.}\evb\evg


\bvg\bva\alst{R}ǫ́ðumk þér Loddfáfnir, \hld\ en \alst{r}ǫ́ð nemir, &
\ind \alst{n}jóta munt ef \alst{n}emr, &
\ind þér munu \alst{g}óð ef \alst{g}etr: &
\alst{i}llu fęginn \hld\ ves \alst{a}ldri-gi, &
\ind \edtrans{en lát þér at \alst{g}óðu \alst{g}etit}{but [rather] let thyself be pleased by good}{\Bfootnote{This construction is equivalent to \CV: \emph{geta}, A. IV. with acc.}}.\eva

\bvb I counsel thee, O Loddfathomer—and thou oughtst to learn the counsels; \\
\ind thou wilt have use if thou learn, \\
\ind they will be good for thee if thou get: \\
Rejoicing in evil be thou never, \\
\ind but let thyself be pleased by good.\evb\evg


\bvg\bva\alst{R}ǫ́ðumk þér Loddfáfnir, \hld\ en \alst{r}ǫ́ð nemir, &
\ind \alst{n}jóta munt ef \alst{n}emr, &
\ind þér munu \alst{g}óð ef \alst{g}etr: &
\alst{u}pp líta \hld\ skal-at-tu í \alst{o}rrostu; &
—\alst{g}jalti \alst{g}líkir \hld\ verða \alst{g}umna synir— &
\ind síðr þitt of \alst{h}ęilli \alst{h}alir.\eva

\bvb I counsel thee, O Loddfathomer—and thou oughtst to learn the counsels; \\
\ind thou wilt have use if thou learn, \\
\ind they will be good for thee if thou get: \\
Up shalt thou not look in battle \\
—alike to a madman become the sons of men— \\
\ind lest men bewitch thy [sense/life/face].\footnoteB{A very difficult st. \CV\ explains \emph{gjalti} as an old dative of \emph{gǫltr} ‘boar, hog’, and thus sees the closely related phrase \emph{verða at gjalti} as “‘to be turned into a hog’, i.e. ‘to turn mad with terror’, esp. in a fight”. The vowel breaking is however unexpected here, since \emph{gǫltr} (< Proto-Norse \emph{*galtuʀ}) is an u-stem, which makes the stem-vowel in the dat. sg. \emph{gęlti} (< \emph{*galtiu}, cf. \textbf{kunimudiu}, dat. sg. of \emph{*Kunimunduʀ}, on the Tjurkö 1 bracteate) the result of i-umlaut rather than an original short \emph{*e}.

\textcite{LaFargeGlossary} instead explain the word as a borrowing from Old Irish \emph{geilt} ‘insane, mad’. \textcite{PettitEdda} follows this, and argues that the whole theme of the st. probably be of Celtic origin, giving several examples from Celtic literature of warriors going mad upon looking up into the sky during battle. In this case the men (\emph{halir}, which word seems to have an association with warriors; cf. 36–37, 49) would be to quote Pettit some sort of “supernatural sky warriors”, in my opinion most likely the \inx[G]{Oneharriers}.}\evb\evg


\bvg\bva\alst{R}ǫ́ðumk þér Loddfáfnir, \hld\ en \alst{r}ǫ́ð nemir, &
\ind \alst{n}jóta munt ef \alst{n}emr, &
\ind þér munu \alst{g}óð ef \alst{g}etr: &
Ef vilt þér \alst{g}óða konu \hld\ kvęðja at \edtrans{\alst{g}aman-rúnum}{pleasure-runes}{\Bfootnote{While easily interpreted as ‘sexual intercourse’, the word is used in st. 120 with a decidedly non-sexual meaning.  Its base meaning is probably ‘good conversation’.}} &
\ind ok \alst{f}ȧa \alst{f}ǫgnuð af, &
\alst{f}ǫgru skalt hęita \hld\ ok láta \alst{f}ast vesa; &
\ind lęiðisk mann-gi \alst{g}ótt ef \alst{g}etr.\eva

\bvb I counsel thee, O Loddfathomer—and thou oughtst to learn the counsels; \\
\ind thou wilt have use if thou learn, \\
\ind they will be good for thee if thou get: \\
If thou wilt for thyself greet a good woman to pleasure-runes, \\
\ind and get good cheer from her; \\
fair things shalt thou promise, and let it be fast; \\
\ind no man loathes a good thing if he gets it.\evb\evg


\bvg\bva\alst{R}ǫ́ðumk þér Loddfáfnir, \hld\ en \alst{r}ǫ́ð nemir, &
\ind \alst{n}jóta munt ef \alst{n}emr, &
\ind þér munu \alst{g}óð ef \alst{g}etr: &
\alst{v}aran bið’k þik \alst{v}esa \hld\ ok ęigi of·\alst{v}aran, &
ves við \alst{ǫ}l varastr, \hld\ ok við \alst{a}nnars konu &
ok við \alst{þ}at hit \alst{þ}riðja, \hld\ at \alst{þ}jófar né lęiki.\eva

\bvb I counsel thee, O Loddfathomer—and thou oughtst to learn the counsels; \\
\ind thou wilt have use if thou learn, \\
\ind they will be good for thee if thou get: \\
Wary I ask thee to be, and not over-wary; \\
be thou wariest with ale, and with another man’s woman, \\
and with the third, that thieves do not outplay [thee].\evb\evg


\bvg\bva\alst{R}ǫ́ðumk þér Loddfáfnir, \hld\ en \alst{r}ǫ́ð nemir, &
\ind \alst{n}jóta munt ef \alst{n}emr, &
\ind þér munu \alst{g}óð ef \alst{g}etr: &
at \alst{h}áði né \alst{h}látri \hld\ \alst{h}af aldri-gi &
\ind \alst{g}ęst né \alst{g}anganda.\eva

\bvb I counsel thee, O Loddfathomer—and thou oughtst to learn the counsels; \\
\ind thou wilt have use if thou learn, \\
\ind they will be good for thee if thou get: \\
In scorn or laughter do never have \\
\ind a guest or wanderer.\evb\evg


\bvg\bva\alst{O}pt vitu \alst{ȯ}-gǫrla, \hld\ þęir’s sitja \alst{i}nni fyrir, &
\ind hvęrs þęir ’ru \alst{k}yns es \alst{k}oma; &
es-at maðr svá \alst{g}óðr \hld\ at \alst{g}alli né fylgi, &
\ind né svá \alst{i}llr at \alst{ęi}nu-gi dugi.\eva

\bvb Oft they know unclearly, those who sit further within, \\
\ind of what kind are those who come; \\
there is no man so good that no flaw follows, \\
\ind nor so bad that he for nothing avails.\evb\evg


\bvg\bva\alst{R}ǫ́ðumk þér Loddfáfnir, \hld\ en \alst{r}ǫ́ð nemir, &
\ind \alst{n}jóta munt ef \alst{n}emr, &
\ind þér munu \alst{g}óð ef \alst{g}etr: &
at \alst{h}ǫ́rum þul \hld\ \alst{h}lę́ aldri-gi, &
\ind opt ’s \alst{g}ótt þat’s \alst{g}amlir kveða, &
opt ór \alst{sk}ǫrpum bęlg \hld\ \alst{sk}ilin orð koma &
\ind þęim’s \alst{h}angir með \alst{h}ǫ́um &
\ind ok \alst{sk}ollir með \alst{sk}rǫ́um, &
\ind ok \alst{v}áfir með \alst{v}íl-mǫgum.\eva

\bvb I counsel thee, O Loddfathomer—and thou oughtst to learn the counsels; \\
\ind thou wilt have use if thou learn, \\
\ind they will be good for thee if thou get: \\
At a hoary thyle do never laugh; \\
\ind oft is good that which old men sing. \\
Oft from scorched leather come discerning words; \\
\ind from him who hangs with hides, \\
\ind and dangles with dry skins, \\
\ind and sways among lads of toil \ken{thralls}.\footnoteB{TODO: Some note. \emph{vil-mǫgum} meaning ‘veal-stomachs’? Cf. Crawford’s video and Finnur on this.}\evb\evg


\bvg\bva\alst{R}ǫ́ðumk þér Loddfáfnir, \hld\ en \alst{r}ǫ́ð nemir, &
\ind \alst{n}jóta munt ef \alst{n}emr, &
\ind þér munu \alst{g}óð ef \alst{g}etr: &
\alst{g}ęst þú né \alst{g}ęyj-a \hld\ \edtrans{né ȧ \alst{g}rind hrę́kir}{nor spit at the gate}{\Bfootnote{The guest is presumably standing behind gate waiting for the farmer to open it and let him in.}}; &
\ind get þú \alst{v}ǫ́-luðum \alst{v}ęl.\eva

\bvb I counsel thee, O Loddfathomer—and thou oughtst to learn the counsels; \\
\ind thou wilt have use if thou learn, \\
\ind they will be good for thee if thou get: \\
At a guest bark not, nor spit at the gate; \\
\ind furnish the destitute well.\evb\evg


\bvg\bva\alst{R}ammt es þat tré, \hld\ es \alst{r}íða skal &
\ind \alst{ǫ}llum at \alst{u}pp-loki; &
\alst{b}aug þú gef \hld\ eða þat \alst{b}iðja mun &
\ind þér \alst{l}ę́s hvęrs ȧ \alst{l}iðu.\eva

\bvb Strong is that wood which shall swing \\
\ind to open up for all.\footnoteB{i.e. the beam of the gate in front of the farm.} \\
Do give a bigh, or it will bid \\
\ind every kind of guile onto thy limbs.\evb\evg


\bvg\bva\alst{R}ǫ́ðumk þér Loddfáfnir, \hld\ en \alst{r}ǫ́ð nemir, &
\ind \alst{n}jóta munt ef \alst{n}emr, &
\ind þér munu \alst{g}óð ef \alst{g}etr: &
hvar’s \alst{ǫ}l drekkir \hld\ kjós þér \alst{ja}rðar męgin, &
því-at \alst{jǫ}rð tękr við \alst{ǫ}lðri, \hld\ en \alst{ę}ldr við sóttum, &
\alst{ęi}k við \alst{a}bbindi, \hld\ \alst{a}x við fjǫl-kyngi, &
\alst{h}ǫll við \alst{h}ýrógi; \hld\ \edtrans{\alst{h}ęiptum skal Mána kvęðja}{in feuds shall one hail Moon}{\Bfootnote{Cf. \Voluspa\ 5 which mentions the “Moon’s might”; for which He is presumably here invoked.  For \emph{kvęðja} ‘hail, invoke’ cf. \Lokasenna\ P3.}}, &
\alst{b}ęiti við \alst{b}it-sóttum, \hld\ en við \alst{b}ǫlvi rúnar; &
\ind \alst{f}old skal við \alst{f}lóði taka.\eva

\bvb I counsel thee, O Loddfathomer, that thou learn the counsels; \\
\ind thou wilt have use if thou learn, \\
\ind they will be good for thee if thou get: \\
Wherever thou drinkest ale choose thee Earth’s might, \\
for earth takes against drunkenness, and fire against sicknesses; \\
oak against dysentery; the ear [of corn] against sorcery; \\
bearded rye against hernia—in feuds shall one hail Moon— \\
heather against bite-sicknesses, and \inx[C]{rune}[runes] against a \inx[C]{bale};\footnoteB{cf. sts. 126, 152.} \\
\ind fold \ken{earth} shall one have against flood.\evb\evg

\sectionline

\section{The Rune-Tally}

This group of stanzas is introduced by a large initial in \Regius, marking the beginning of a new section.  In younger paper manuscripts they have the header \emph{Rúna-tals þáttr} ‘Strand of the Rune-Tally’, and generally give an archaic, mystic impression; at times one gets a feeling that they were drawn from the lips of an Odinic priest.

Apart from these stanzas there are a few other manuscript attestations of similar Runic magic.  Closest at hand is st. 80 above, which would fit seamlessly into the present section.  Outside of \Havamal\ there is \Sigrdrifumal\ 5–17, also preserved in \Regius.

\sectionline

\bvg\bva\alst{V}ęit’k at ek hekk \hld\ \alst{v}indga męiði ȧ &
\ind \alst{n}ę́tr allar \alst{n}íu, &
\alst{g}ęiri undaðr \hld\ ok \alst{g}efinn Óðni, &
\ind \alst{s}jalfr \alst{s}jǫlfum mér, &
ȧ þęim \alst{m}ęiði, \hld\ es \alst{m}ann-gi vęit, &
\ind hvęrs af \alst{r}ótum \alst{r}innr.\eva

\bvb I know that I hung on the windy beam, \\
\ind for nine nights all; \\
wounded by spear and given to Weden— \\
\ind myself to myself— \\
on that beam, which no man knows, \\
\ind of whose roots it runs.\evb\evg


\bvg\bva Við \edtext{\alst{h}lęifi mik sǿldu-t \hld\ né við \alst{h}orni-gi}{\lemma{hlęifi \dots\ horni-gi ‘loaf \dots\ horn’}\Bfootnote{i.e. “I got neither bread nor drink.”}}; &
\alst{n}ýsta ek \alst{n}iðr, \hld\ \alst{n}am’k upp rúnar, &
\alst{ǿ}pandi nam, \hld\ fell’k \alst{a}ptr þaðan.\eva

\bvb With loaf they relieved me not, nor with any horn. \\
I peered down; I took up the runes; \\
screaming I took; I fell back thence.\evb\evg


\bvg\bva\edtrans{\alst{F}imbul-ljóð níu}{Nine fimble-leeds}{\Bfootnote{Nine very great chants or spells (\inx[C]{galders}), compare the eighteen leeds below (st. 147 onward).  It is unclear what this has to do with Weden’s Hanging; this stanza may be an insert.}} \hld\ nam’k af \edtext{hinum \alst{f}rę́gja syni &
\ind \alst{B}ǫlþorns, \alst{B}ęstlu fǫður,}{\lemma{hinum frę́gja syni Bǫlþorns, Bęstlu fǫður ‘the famous son of Balethorn, Bestle’s father’}\Bfootnote{According to \Gylfaginning\ 6, Byre got Bestle for a wife, the daughter of the ettin Balethorn.  By her he fathered three sons: Weden, Will and Wigh.  The “famous son of Balethorn” would then be Weden’s maternal uncle.  This reflects the old Indo-European custom of sending sons away to be fostered by the male relations of the mother.  Cf. TODO: some reference.}} &
ok ek \alst{d}rykk of gat \hld\ hins \alst{d}ýra mjaðar &
\ind \alst{au}sinn \alst{Ó}ð-rǿri.\eva

\bvb Nine \inx[C]{fimble}-leeds I learned from the famous son \\
\ind of \inx[P]{Balethorn}, \inx[P]{Bestle}’s father— \\
and a drink I got, of that dear mead \\
\ind poured [from] \inx[P]{Woderearer}.\evb\evg


\bvg\bva Þȧ \edtrans{nam’k \alst{f}rę́vask}{I began to flourish}{\Bfootnote{A notorious mistranslation popularized by \textcite{Greenberg1988} has rendered these words as “I took semen”.  They would supposedly reference Weden stealing the ejaculate from hanged men in order to replenish his own powers—something not otherwise attested.  This preposterous notion makes no sense in the context of the text and has no philological grounding.  While Old Norse \emph{frę́} does mean “seed”, it only refers to the seeds of plants, not the seed animals or men.  Regardless, \emph{frę́vask} is without doubt a reflexive verb literally meaning something like ‘cultivate oneself’.}} \hld\ ok \alst{f}róðr vesa &
\ind ok \alst{v}axa ok \alst{v}ęl hafask; &
\edtext{\alst{o}rð mér af \alst{o}rði \hld\ \alst{o}rðs lęitaði &
\alst{v}erk mér af \alst{v}erki \hld\ \alst{v}erks lęitaði.}{\lemma{orð \dots\ lęitaði. ‘My word \dots sought out.’}\Bfootnote{i.e. “Every good speech led to another; every good deed likewise.”}}\eva

\bvb Then I began to flourish, and be learned, \\
\ind and grow and have it well. \\
My word from a word a word sought out; \\
my work from a work a work sought out.\evb\evg


\bvg\bva \edtext{\alst{R}únar munt finna \hld\ ok \alst{r}áðna stafi}{\lemma{Rúnar \dots\ ok ráðna stafi ‘Runes \dots\ and interpreted staves’}\Bfootnote{Formulaic.  Cf. the long-line on the medieval runestone N 13 (excerpt): \emph{rúnar ek ríst \hld\ ok ráðna stafi} ‘runes I carve, and interpreted staves.’}}, &
\ind mjǫk \alst{st}óra \alst{st}afi, &
\ind mjǫk \alst{st}inna \alst{st}afi, &
\ind es \alst{f}áði \alst{F}imbul-þulr &
\ind ok \alst{g}ørðu \alst{g}inn-ręgin &
\ind ok \alst{r}ęist Hroptr \edtrans{\alst{r}agna}{of the Reins}{\Afootnote{\emph{‘rǫgna’} \Regius}}.\eva

\bvb \inx[C]{rune}[Runes] wilt thou find, and interpreted staves: \\
\ind very large staves, \\
\ind very stiff staves, \\
\ind which \inx[P]{Fimble-Thyle} \name{= Weden} painted, \\
\ind and the \inx[G]{yin-Reins} made, \\
\ind and Roft \name{= Weden} of the Reins carved.\evb\evg


\bvg\bva\alst{Ó}ðinn með \alst{ǫ̇}sum, \hld\ en fyr \alst{ǫ}lfum Dáinn, &
\ind \alst{D}valinn \alst{d}vergum fyrir, &
\ind \alst{Á}sviðr \alst{jǫ}tnum fyrir, &
\ind \edtrans{ek}{I}{\Bfootnote{The identity of the speaker is unclear; one would expect it to be Weden, but He is already named in line 1.}} ręist \alst{s}jalfr \alst{s}umar.\eva

\bvb \inx[P]{Weden} among the \inx[G]{Eese} and \inx[P]{Dowen} for the \inx[G]{Elves}; \\
\ind \inx[P]{Dwollen} for the \inx[G]{Dwarfs}; \\
\ind \inx[P]{Oswith} for the Ettins; \\
\ind I myself carved some.\evb\evg


\bvg\bva Vęitst, hvé \alst{r}ísta skal? \hld\ Vęitst, hvé \alst{r}áða skal? &
Vęitst, hvé \alst{f}áa skal? \hld\ Vęitst, hvé \alst{f}ręista skal? &
Vęitst, hvé \alst{b}iðja skal? \hld\ Vęitst, hvé \alst{b}lóta skal? &
Vęitst, hvé \alst{s}ęnda skal? \hld\ Vęitst, hvé \alst{s}óa skal?\eva

\bvb Knowest thou how one shall carve? Knowest thou how one shall read? \\
Knowest thou how one shall paint? Knowest thou how one shall try? \\
Knowest thou how one shall bid? Knowest thou how one shall \inx[C]{bloot}? \\
Knowest thou one shall send? Knowest thou how one shall \inx[C]{soo}?\footnoteB{A neat semantic structure would be found if the former four verbs referred to \inx[C]{rune}[runes]: carving, interpreting, painting (with blood?), and divining; and the latter four referred to sacrifice: asking for boons, worshipping, sending (the sacrifice or the prayer; making sure the gods receive it), and slaying the victim. This may be supported by the following stanza, which repeats the last four verbs here in what looks like a sacrificial context. See further relevant Encyclopedia entries.}\footnoteB{The meter of this st. is unusual, but bears some resemblance to Vg 216 (the Högstena galder). TODO: Elaborate.}\evb\evg


\bvg\bva\alst{B}ętra ’s ȯ-\alst{b}eðit \hld\ an sé of·\alst{b}lótit, &
\ind ęy sér til \alst{g}ildis \alst{g}jǫf; &
bętra ’s ȯ-\alst{s}ęnt \hld\ an sé of·\alst{s}óit; &
\edtext{[...]}{\Bfootnote{For metrical reasons it is very likely that a line has been lost here.}}\eva

\bvb It’s better unbid than over\inx[C]{bloot}[blooted]; \\
\ind a gift always sees repayment. \\
It’s better unsent than over\inx[C]{soo}[sooed]; \\
{[...]}.\footnoteB{An identical progression of four verbs suggests a close relation with the previous st. — The sense seems to be that it is better not to sacrifice at all than to sacrifice in excess, since even a small gift (to the gods) will be rewarded. A ritual cycle of gifts and rewards between men and the gods is also seen in other Indo-European pagan literatures. Compare the Sanskrit \emph{Dehí me, dádāmi te} ‘Give to me, I give to thee’ and Latin \emph{dō ut dēs} ‘I give that thou might give’.}\evb\evg


\bvg\bva Svá \alst{Þ}undr of ręist \hld\ fyr \alst{þ}jóða rǫk, &
þar’s \alst{u}pp of ręis, \hld\ es \alst{a}ptr of kom.\eva

\bvb So \inx[P]{Thound} \name{= Weden} did carve for the rakes of nations, \\
where up he rose as back he came.\footnoteB{TODO: A very cryptic st.}\evb\evg

\sectionline

\section{The Leed-Tally (147–165)}

This section of \Havamal, the so-called the Leed-Tally (\emph{Ljóðatal}), is not separated from the preceding section (which is marked out with a large initial), but is usually taken as separate since it is a self-contained list not much concerned with runes.  The speaker (certainly Weden) recounts eighteen spells, apparently to Loddfathomer.  The spells themselves are not listed; only their use and effects.  They are aristocratic and Odinic in character, and deal with such things as battle (3, 4, 5, 8, 11, 13), healing (spell 2, 12), countering sorcery (6, 10), controlling the elements (7, 9), and seduction (16, 17).  The eighteenth and last spell must remain mysterious; not even its purpose is told, and it is known only to Weden and his lover.

The eighteen have some similarities with other known spells and lists of spells.  The fourth bears a strong likeness to \Grougaldr\ 10, and its effect (removing fetters) is shared with the High German \MerseburgOne, where such a spell is actually found.

\sectionline

\bvg\bva Ljóð \alst{þ}au kann’k, \hld\ es kann-at \alst{þ}jóðans kona &
\ind ok \alst{m}anns-kis \alst{m}ǫgr. &
\alst{H}jǫlp hęitir ęitt, \hld\ þat þér \alst{h}jalpa mun &
\ind við \alst{s}orgum ok \edtrans{\alst{s}ǫkum}{sakes}{\Bfootnote{Legal charges, the first element of English \emph{sakeless}.}}, \hld\ ok \alst{s}útum gǫrv-ǫllum.\eva

\bvb Those \inx[C]{leed}[leeds] I know, which knows no king’s woman, \\
\ind and no man’s lad. \\
Help is called one, it will help thee \\
\ind against sorrows and sakes, and all kinds of griefs.\footnoteB{TODO: elaborate on translatioon}\evb\evg


\bvg\bva Þat kann’k \alst{a}nnat, \hld\ es \edtrans{þurfu \alst{ý}ta synir}{the sons of men need}{\Bfootnote{Cf. the similar wording in 166/2.}}, &
\ind þęir’s vilja \alst{l}ę́knar \alst{l}ifa.\eva

\bvb I know another, which the sons of men need, \\
\ind those who wish to live as leechers.\evb\evg


\bvg\bva Þat kann’k \alst{þ}riðja, \hld\ ef mér verðr \alst{þ}ǫrf mikil &
\ind \alst{h}apts við mína \alst{h}ęipt-mǫgu, &
\alst{ę}ggjar dęyfi’k \hld\ minna \alst{a}nd-skota, &
\ind bíta-t þęim \alst{v}ǫ́pn né \edtrans{\alst{v}ęlir}{staffs}{\Bfootnote{plural of \emph{vǫlr}, a magic staff used by witches and warlocks.  The word \emph{vǫlva} ‘\inx[C]{wallow}’ (seeress, prophetess) derives from this word.  The reading \emph{vélir} ‘wiles, tricks, deceits’ must be excluded for metrical reasons since a \Ljodahattr\ c-verse cannot end in a trochée.}}.\eva

\bvb I know the third, if I come in great need \\
\ind of hindrance against my feud-lads \ken{enemies}; \\
I dull the edges of my opponents; \\
\ind for them bite not weapons nor staffs.\evb\evg


\bvg\bva Þat kann’k \alst{f}jórða, \hld\ ef mér \alst{f}yrðar bera &
\ind \alst{b}ǫnd at \alst{b}óg-limum, &
svá ek \alst{g}ęl, \hld\ at \alst{g}anga má’k, &
\ind sprettr mér af \alst{f}ótum \alst{f}jǫturr, &
\ind en af \alst{h}ǫndum \alst{h}apt.\eva

\bvb I know the fourth, if men bear \\
\ind bonds onto my shoulder-limbs: \\
\emph{so} I gale that I may walk; \\
\ind springs from my feet the fetter, \\
\ind and from my hands the bond.\footnoteB{Cf. \Grougaldr\ 10, which is very similar to the present stanza, and \MerseburgOne\ (edited below under Galders), a galder that seems to have actually been used for the purpose of removing fetters.}\evb\evg


\bvg\bva Þat kann’k \alst{f}imta, \hld\ ef sé’k af \alst{f}ári skotinn &
\ind \alst{f}lęin í \alst{f}olki vaða, &
flýgr-a svá \alst{st}int, \hld\ at \alst{st}ǫðvi’g-a’k, &
\ind ef hann \alst{s}jónum of \alst{s}é’k.\eva

\bvb I know the fifth, if I see a dangerously shot \\
\ind arrow in the troop wading: \\
it flies not so stiff that I may not stop it, \\
\ind if I see it with my sights.\evb\evg


\bvg\bva Þat kann’k \alst{s}étta, \hld\ \edtext{ef mik \alst{s}ę́rir þegn &
\ind ȧ \alst{r}ótum \edtrans{\alst{r}ás}{raw/sappy}{\Bfootnote{The normal form of this word is \emph{*hrár} (cf. \Skirnismal\ 32), but the required alliteration with \emph{rótum} makes it impossible here.}} viðar}{\lemma{ef mik sę́rir þegn ȧ rótum rás viðar ‘if a thane wounds me on the roots of a raw/sappy tree’}\Bfootnote{i.e., “if someone carves a runic curse directed against me”.  The sappy wood was apparently thought to be important for the curse to work.  Cf. \Grettissaga\ 79, where a hag curses Gretter in the following way: after finding a small tree and planing a small smooth surface onto a burnt side of it, she carves runes in its roots and reddens them with her own blood.  She then chants \inx[C]{galder}[galders] while walking counter-clockwise around it.  She last pushes it out to sea, praying for it to drift to Gretter’s homestead, cursing him.  Cf. also \Skirnismal\ 32 where a \emph{hrár viðr} ‘raw/sappy tree’ occurs in the context of a curse.}}, &
þann \alst{h}al, \hld\ es mik \alst{h}ęipta kveðr, &
\ind þann eta \alst{m}ęin hęldr an \alst{m}ik.\eva

\bvb I know the sixth, if a thane wounds me \\
\ind on the roots of a raw/sappy tree: \\
\emph{that man} who sings hatred against me, \\
\ind \emph{him} the harms eat, rather than me.\evb\evg


\bvg\bva Þat kann’k \alst{s}jaunda, \hld\ ef \alst{s}é’k hǫ́van loga &
\ind \alst{s}al of \alst{s}ess-mǫgum, &
\alst{b}rinnr-at svá \alst{b}ręitt, \hld\ at hǫ́num \alst{b}jargi’g-a’k; &
\ind þann kann’k \alst{g}aldr at \alst{g}ala.\eva

\bvb I know the seventh, if I see a high hall \\
\ind blazing over seat-lads \ken{warriors}: \\
it burns not so broadly that I may not save it\footnoteB{i.e. “if I see a hall burning with men trapped inside, no matter how large the flame is I can save both the hall and the men.”}— \\
\ind that galder I can gale.\evb\evg


\bvg\bva Þat kann’k \alst{á}tta, \hld\ es \alst{ǫ}llum es &
\ind \alst{n}yt-sam-ligt at \alst{n}ema, &
\alst{h}var’s \edtrans{\alst{h}atr}{hatred}{\Bfootnote{i.e. with regard to the father’s inheritance.}} vęx \hld\ með \alst{h}ildings sonum, &
\ind þat má’k \alst{b}ǿta \alst{b}rátt.\eva

\bvb I know the eighth, which for all men is \\
\ind useful to learn: \\
wherever hatred grows among a prince’s sons, \\
\ind it I may shortly mend.\evb\evg


\bvg\bva Þat kann’k \alst{n}íunda, \hld\ ef mik \alst{n}auðr of stęndr &
\ind at bjarga \alst{f}ari mínu ȧ \alst{f}loti, &
\alst{v}ind ek kyrri \hld\ \alst{v}ági ȧ &
\ind ok \alst{s}vę́fi’k allan \alst{s}ę́.\eva

\bvb I know the ninth, if I am in need \\
\ind to save my ride on a floater \ken{ship}: \\
the wind I calm on the wave, \\
\ind and put all the sea asleep.\evb\evg


\bvg\bva Þat kann’k \alst{t}íunda, \hld\ ef sé’k \alst{t}ún-riður &
\ind \alst{l}ęika \alst{l}opti ȧ, &
ek svá \alst{v}inn’k, \hld\ at \edtrans{þę́r \alst{v}illar fara}{they (\emph{fem.}) go astray}{\Bfootnote{emend.; \emph{þęir villir fara} ‘they (\emph{masc.}) go astray’ \Regius}} &
\ind sinna \alst{h}ęim-\alst{h}ama &
\ind sinna \alst{h}ęim-\alst{h}uga.\eva

\bvb I know the tenth, if I see \inx[G]{town-rideresses} \\
\ind playing aloft: \\
I accomplish it so that they go astray \\
\ind from their home-\inx[C]{hame}[hames]; \\
\ind from their home-minds.\footnoteB{The \emph{riður} ‘(female) riders’ were witches who would leave their original human shapes or skins (\emph{hamir}) in order to fly around in the air tormenting and poisoning villagers.  Their original bodies would then be lying in a coma-like state, in something resembling that which is today called astral projection.  Yet, it was not the case that their whole mental faculties would disconnect from their bodies, but rather they would leave behind something of their humanity, which was thought to be inextricably linked to their human bodies.  Weden was through his second sight able to see these riders, and could then use his superior magical skill to confuse them so that they would not be able to return to their human “home”-shapes or minds, but were instead forced to stray as tormented disentagled ghosts; a cruel fate. — Weden likewise brags about tricking riders in \Harbardsljod\ 20.}\evb\evg


\bvg\bva Þat kann’k \alst{ę}llipta, \hld\ ef skal’k til \alst{o}rrostu &
\ind \alst{l}ęiða \alst{l}ang-vini, &
und \alst{r}andir gęl’k, \hld\ en þęir með \alst{r}íki fara, &
\ind \alst{h}ęilir \alst{h}ildar til, &
\ind \alst{h}ęilir \alst{h}ildi frȧ, &
\ind koma þęir \alst{h}ęilir \alst{h}vaðan.\eva

\bvb I know the eleventh, if I shall into war \\
\ind lead old friends: \\
beneath the shields I gale, and they go with power \\
\ind healthy to the battle, \\
\ind healthy from the battle; \\
\ind they return healthy anywhence.\evb\evg


\bvg\bva Þat kann’k \alst{t}olpta, \hld\ ef sé’k ȧ \alst{t}ré uppi &
\ind \alst{v}áfa \alst{v}irgil-ná, &
svá ek \alst{r}íst \hld\ ok í \alst{r}únum fá’k, &
\ind at sá \alst{g}ęngr \alst{g}umi. &
\ind ok \alst{m}ę́lir við \alst{m}ik.\eva

\bvb I know the twelfth, if I see high up on a tree \\
\ind a gallow-corpse dangling: \\
so I carve and paint in the runes, \\
\ind that that man walks \\
\ind and speaks with me.\evb\evg


\bvg\bva Þat kann’k \alst{þ}rettánda \hld\ \edtext{ef skal’k \alst{þ}egn ungan &
\ind \alst{v}erpa \alst{v}atni ȧ,}{\lemma{ef skal’k þegn ungan verpa vatni ȧ ‘if on a young thane I shall sprinkle water’}\Bfootnote{A reference to the Heathen name-giving ceremony in which the infant would be sprinkled with water; cf. the attestations in \Rigsthula\ 7, 21, 34.}} &
mun-at hann \alst{f}alla \hld\ þótt í \alst{f}olk komi, &
\ind \alst{h}nígr-a sá \alst{h}alr fyr \alst{h}jǫrum.\eva

\bvb I know the thirteenth, if on a young thane \\
\ind I shall sprinkle water: \\
he will not fall though he should come into battle; \\
\ind that warrior sinks not down before swords.\evb\evg


\bvg\bva Þat kann’k \alst{f}jórtánda, \hld\ ef skal’k \alst{f}yrða liði &
\ind \alst{t}ęlja \alst{t}íva fyr, &
\alst{ȧ}sa ok \alst{a}lfa \hld\ ek kann \alst{a}llra \edtrans{skil}{discernments}{\Bfootnote{Cf. \Hymiskvida\ 38, where the corresponding verb \emph{skilja} ‘to discern, understand’ is used in the context of god-lore.}}, &
\ind fár kann ȯ-\alst{s}notr \alst{s}vá.\eva

\bvb I know the fourteenth, if before a retinue of men \\
\ind I shall count forth the Tews: \\
of all the Eese and Elves I know the discernments; \\
\ind few unwise men can do so.\evb\evg


\bvg\bva\alst{Þ}at kann’k fimtánda, \hld\ es gól \alst{Þ}jóð-rǿrir &
\ind \alst{d}vergr fyr \alst{D}ęllings \alst{d}urum, &
\alst{a}fl gól \alst{ǫ́}sum, \hld\ en \alst{ǫ}lfum frama, &
\ind \alst{h}yggju \alst{H}ropta-týi.\eva

\bvb I know the fifteenth, which Thedrearer galed, \\
\ind the dwarf, before Delling’s doors. \\
He galed strength for the Eese and fame for the Elves; \\
\ind thought for Roft-Tew \name{= Weden}.\evb\evg


\bvg\bva Þat kann’k \alst{s}extánda, \hld\ ef vil’k hins \alst{s}vinna mans &
\ind hafa \alst{g}ęð allt ok \alst{g}aman, &
\alst{h}ugi \alst{h}vęrfi’k \hld\ \alst{h}vit-armri konu &
\ind ok \alst{s}ný’k hęnnar ǫllum \alst{s}efa.\eva

\bvb I know the sixteenth, if I will from the wise girl \\
\ind have her senses all, and pleasure; \\
the heart I change of the white-armed woman, \\
\ind and I twist all her mind.\evb\evg


\bvg\bva Þat kann’k \alst{s}jautjánda \hld\ at mik \alst{s}ęint mun firrask &
\ind hit \alst{m}an-unga \alst{m}an.\eva

\bvb I know the seventeenth, that the girl-young girl \\
\ind will lately shun me.\evb\evg


\bvg\bva\alst{L}jóða þessa \hld\ munt \alst{L}oddfáfnir &
\ind lengi \alst{v}anr \alst{v}esa; &
\ind þó sé þér \alst{g}óð ef \alst{g}etr, &
\ind \alst{n}ýt ef \alst{n}emr, &
\ind \alst{þ}ǫrf ef \alst{þ}iggr.\eva

\bvb These leeds wilt thou, Loddfathomer, \\
\ind long be lacking! \\
Though they would be good for thee if thou get, \\
\ind useful if thou learn, \\
\ind needful if thou receive.\evb\evg


\bvg\bva Þat kann’k \alst{á}tjánda, \hld\ es \alst{ę́}va kęnni’k &
\ind \alst{m}ęy né \alst{m}anns konu, &
—\alst{a}llt es bętra \hld\ es \alst{ęi}nn of kann, &
\ind þat fylgir \alst{l}jóða \alst{l}okum— &
nema þęiri \alst{ęi}nni, \hld\ es \edtrans{mik \alst{a}rmi vęrr}{with her arm guards me}{\Bfootnote{A similar expression is also used \Volundarkvida\ 2.  The one who wraps Weden in her arm may be His wife, \inx[P]{Frie}.}}, &
\ind eða mín \alst{s}ystir \alst{s}éi.\eva

\bvb I know the eighteenth, which I never teach \\
\ind a maiden nor man’s woman— \\
everything is better when one alone can do it; \\
\ind that follows the end of the leeds— \\
save for her alone who with her arm guards me, \\
\ind or who is my sister.\evb\evg

\sectionline

\bvg\bva Nú eru \alst{H}áva mǫ́l kveðin \hld\ \alst{H}áva \alst{h}ǫllu í; &
\ind \alst{a}ll-þǫrf \alst{ý}ta sonum, &
\ind \alst{ȯ}-þǫrf \edtrans{\alst{jǫ}tna}{ettins}{\Afootnote{corrected in margin from \emph{ýta} ‘men’ \Regius}} sonum; &
hęill sá’s \edtext{\alst{k}vað, \hld\ hęill sá’s \alst{k}ann, &
\ind \alst{n}jóti sá’s \alst{n}am, &
\ind \alst{h}ęilir þęir’s \alst{h}lýddu}{\lemma{kvað, kann, nam, hlýddu ‘sang, knows, learned, heeded’}\Bfootnote{The implied subject is the speeches, i.e. ‘hail he who sang them, hail he who knows them,’ et.c.}}.\eva

\bvb Now are the High One’s speeches sung in the High One’s hall; \\
\ind of great use for the sons of men; \\
\ind of harm for the sons of ettins. \\
Hail he who sang; hail he who knows; \\
\ind may he benefit who learned; \\
\ind hail those who heeded!\evb\evg

\sectionline
