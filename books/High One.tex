and thou oughtst to learn the counsels\bookStart{The Speeches of the High One}[Hávamǫ́l]

%Introduction.

The \textbf{Speeches of the High One} is the second poem of \Regius, which is also the only ancient manuscript in which it is attested. Several sts. are however cited or alluded to in other places, such as Eyv \emph{Hák} (TODO: formatting) 21 and \FostrbroedhraSaga\ TODO.

The poem as it currently comes down to us hardly seems like a single composition, much rather like a grab bag of traditional poetic sts. associated with the god Weden. It combines two separate advice-poems with sts. concerning Weden’s love adventures, runes and spells. Little unites these various strands other than their speaker.

Following previous authors, I identify several such strands, excepting various lone sts. that are probably later inserts. In the present edition each of them is given a separate, short introduction:

\begin{longtabu} to \textwidth {|c c c c c c|}
	\hline
  1–79 & The Guest-strand; practical life advice placed within the context of a guest arriving at a homestead. \\
  81–89 & Other sts. of advice, mostly composed in \Fornyrdislag. \\
  90–101 & Weden’s failed seduction of Billing’s daughter. \\
  102–109 & Weden’s obtaining of the Mead of Poetry \\
  110–135 & The Speeches of Loddfathomer; Weden’s advice to Loddfathomer. \\
  136–144 & The Rune-tally; various sts. relating to runes. \\
  145–163 & The Leed-tally; Weden’s listing of 18 spells. \\
  164 & Final st., composed when the strands above were collected. \\ [1ex]
  \hline
\end{longtabu}

Whatever their origins, it is clear from the final st. that they have been thought of as a single work, but it is notable that this st., which also contains the title \emph{Hávamǫ́l} ‘Speeches of the High One’, is highly metrically irregular. It has likely been composed by the person who assembled the disparate elements listed above into one text.

\sectionline

\section{The Guest-strand}

The Guest-Strand (Old Norse: \emph{Gęstaþáttr}) is possibly the finest work in Norse poetry. Sadly, its structure has been obscured by various inserted and possibly displaced sts. My hope is to shed some light on the original vision behind the poem, while as usual not changing the order of sts. as they appear in the only surviving witness manuscript.

The poem moves through many elements of life, but in a poetically almost seamless way. To move from one topic to another, the poet often employs transitions where a st. recalls the structure of the previous one, but with a new subject. This is particularly evident in sts. 4–5 and 10–11.

The strand begins with a st. encouraging travellers to be wary of entering strange houses without first spying out who is inside (1), after which a voice inside of a farmstead (possibly Weden?) announces that a guest is waiting to be let in (2). The same speaker then lists several things which the newly arrived guest needs from the host, namely: fire, food and clothes (3), water, a towel, a great welcome, a good reception, an opportunity to speak and silence in return (4).

After this focus shifts to the conduct of the wanderer, with an introductory st. explaining that he needs wit (specifically \inx[C]{manwit} (\emph{manvit}); see Encyclopedia), lest he become a laughing-stock (5). He should be silent but attentive, and choose his words carefully (6–7). He should be confident in himself and his own decisions, and not rely too much on the opinions of others (8–9), since there is nothing better one may bring along on the journey than much manwit (10).

Here the advice moves to the subject alcohol. Where the best thing one may bring along on the journey is manwit, the worst is too much ale (11). It is not as good as men call it (12) since it “robs [them] of their senses”; it is even personified as a “heron of forgetfulness” (13). A drinking round is best when the participants do not drink too much, but rather regain their senses afterwards (14).

St. 15 contains some general advice; a royal child should be silent, thoughtful and bold in battle, and all men should stay happy, until they die.

TODO.

\sectionline

\bvg
\bva\alst{G}áttir allar \hld\ áðr \alst{g}angi framm &
\ind \edtext{of \alst{sk}oðask \alst{sk}yli,}{\Bfootnote{om. \GylfMS}} &
\ind of \alst{sk}yggnask \alst{sk}yli; &
því-at ó·\alst{v}íst ’s at \alst{v}ita, \hld\ hvar ó·\alst{v}inir &
\ind sitja á \alst{f}lęti \alst{f}yrir.\eva

\bvb All doorways—before one might go forth— \\
should be watched, \\
should be spied at; \\
for uncertain ’tis to know, where enemies \\
sit on the benches within.\evb
\evg


\bvg
\bva \edtrans{\alst{G}efęndr}{the givers}{\Bfootnote{The hosts.}} hęilir, \hld\ \alst{g}ęstr ’s inn kominn, &
\ind hvar skal \alst{s}itja \alst{s}já? &
mjǫk es \alst{b}ráðr \hld\ sá’s \edtrans{á \alst{b}rǫndum}{on the fires}{\Bfootnote{Possibly referring a Norwegian folk custom, wherein a guest would sit down on the wood-pile outside of the door, waiting until being let in. See further TODO SOME ARTICLE on this custom. The speaker thus announces to the hosts that a frozen, wet and tired guest has arrived and currently sits impatiently on the wood-pile, and ought to be taken in.}} skal &
\ind \edtrans{síns of \alst{f}ręista \alst{f}rama}{try his distinction}{\Bfootnote{Formulaic, also occurring in TODO other places.}}.\eva

\bvb Hail the givers, a guest is come in! \\
Where shall this one sit? \\
Very impatient is he, who on the fires shall \\
try his distinction.\evb
\evg


\bvg
\bva\alst{Ę}lds es þǫrf \hld\ þęim’s \alst{i}nn es kominn &
\ind ok á \alst{k}néi \alst{k}alinn, &
\alst{m}atar ok váða \hld\ es \alst{m}anni þǫrf, &
\ind þęim’s hęfr of \alst{f}jall \alst{f}arit.\eva

\bvb Of fire is there need for the one who is come in, \\
and cold about the knees; \\
of food and of clothing is there need for that man \\
who over the fell has fared.\evb
\evg


\bvg
\bva\alst{V}ats es þǫrf \hld\ þęim’s til \alst{v}erðar kømr, &
\ind \alst{þ}ęrru ok \alst{þ}jóð-laðar, &
\alst{g}óðs of ǿðis, \hld\ —ef sér \alst{g}eta mę́tti— &
\ind \alst{o}rðs ok \alst{ę}ndr-þǫgu.\eva

\bvb Of water is there need for the one who comes for a meal; \\
of a towel and of a great welcome; \\
of a good reception—if he might get one— \\
of speech, and of silence in return.\footnoteB{There is a well thought-out linear progression throughout this st.: The guest must first wash, then dry himself with a towel, then be welcomed to sit and eat at the table and speak with the host. The host has done his part, and now it is the guest’s turn. This nicely leads the transition to the following sts., where the proper conduct of the guest (first in speech, and then in various other areas) is discussed.}\evb
\evg


\bvg
\bva\alst{V}its es þǫrf \hld\ þęim’s \alst{v}íða ratar; &
\ind dę́lt es \alst{h}ęima \alst{h}vat; &
at \alst{au}ga-bragði \hld\ verðr sá’s \alst{ę}kki kann &
\ind ok með \alst{s}notrum \alst{s}itr.\eva

\bvb Of wit is there need for the one who widely roams; \\
everything is easy at home. \\
A laughing-stock\footnoteB{An idiom, \emph{auga-bragð} lit. ‘twinkling of an eye, moment’.} becomes he who nothing knows, \\
and among the clever sits.\evb
\evg


\bvg
\bva At \alst{h}yggjandi sinni \hld\ skyli-t maðr \alst{h}rǿsinn vesa, &
\ind hęldr \alst{g}ę́tinn at \alst{g}ęði, &
þá’s \alst{h}orskr ok þǫgull \hld\ kømr \alst{h}ęimis-garða til, &
\ind sjaldan verðr \alst{v}íti \alst{v}ǫrum. &
því-at \alst{ó}-brigðra vin \hld\ fę́r maðr \alst{a}ldri-gi, &
\ind an \alst{m}an-vit \alst{m}ikit.\eva

\bvb Of his thinking should man not be boastful; \\
rather guarding of his senses, \\
when sharp and silent he comes to a homestead; \\
sudden injury seldom strikes the wary, \\
for an unfickler friend man never gets \\
than much \inx[C]{manwit}.\evb
\evg


\bvg
\bva Hinn \alst{v}ari gęstr, \hld\ es til \alst{v}erðar kømr, &
\ind \alst{þ}unnu hljóði \alst{þ}ęgir; &
\alst{ęy}rum hlýðir, \hld\ en \alst{au}gum skoðar, &
\ind svá \edtext{nýsisk \alst{f}róðra hvęrr \alst{f}yrir}{\lemma{nýsisk \dots\ fyrir ‘looks \dots\ ahead’}\Bfootnote{Verb underlying the noun \emph{for-njósn} as found in \Sigrdrifumal\ 24.}}.\eva

\bvb The wary guest—when for a meal he comes— \\
with thin listening shuts up.\footnoteB{i.e. is in attentive silence.} \\
With ears he listens, but with eyes he observes; \\
so looks each learned man ahead.\evb
\evg


\bvg
\bva Hinn es \alst{s}ę́ll, \hld\ es \alst{s}ér of getr &
\ind \edtrans{\alst{l}of ok \alst{l}íkn-stafi}{praise and staves of liking}{\Bfootnote{\emph{líkn} ‘liking’ is a very interesting word. It is defined by \ONP\ as: ‘mercy, compassion, relief, comfort, help’. In the present poem its precise meaning seems to be something like ‘the state of being liked by your surroundings to the point where people are willing to help you out’. Cf. its two other occurrences in the present poem: sts. 120 and especially 123 (where it is likewise paired with \emph{lof} ‘praise’).}}; &
\alst{ó}-dę́lla ’s við þat, \hld\ es \alst{ęi}ga skal &
\ind \alst{a}nnars brjóstum \alst{í}.\eva

\bvb The one is blessed, who for himself gets \\
praise and staves of liking. \\
’Tis uneasy regarding that which one shall own \\
in another man’s breast.\evb
\evg


\bvg
\bva\alst{S}á es \alst{s}ę́ll, \hld\ es \alst{s}jalfr of á &
\ind \alst{l}of ok vit meðan \alst{l}ifir; &
því-at \alst{i}ll rǫ́ð \hld\ hęfr maðr \alst{o}pt þęgit &
\ind \alst{a}nnars brjóstum \alst{ó}r.\eva

\bvb He is blessed, who himself does own \\
praise and wits while he lives, \\
for ill counsels has man oft taken \\
out of another man’s breast.\evb
\evg


\bvg
\bva\alst{B}yrði \alst{b}ętri \hld\ berr-at maðr \alst{b}rautu at, &
\ind an sé \alst{m}an-vit \alst{m}ikit; &
\alst{au}ði bętra \hld\ þykkir þat í \alst{ó}-kunnum stað; &
\ind slíkt es \alst{v}á-laðs \alst{v}era.\eva

\bvb A better burden bears man not on the road \\
than much manwit. \\
In an unknown place it seems better than wealth; \\
such is the destitute man’s shelter.\evb
\evg

% TODO: NEW SECTION (Alcohol)

\bvg
\bva\alst{B}yrði \alst{b}ętri \hld\ berr-at maðr \alst{b}rautu at, &
\ind an sé \alst{m}an-vit \alst{m}ikit; &
\alst{v}eg-nest \alst{v}erra \hld\ \alst{v}egr-a \edtrans{\alst{v}ęlli at}{on the plain}{\Bfootnote{Formulaic, the word \emph{vǫllr} ‘plain, (uncultivated) field’ is also used in sts. 38 and 49. It is easily understood that the wild heaths and plains of Iron Age Norway were particularly unsafe places where a traveller needed to keep his wits about him, lest he fall victim to robbers or murderers (so st. 38).}}, &
\ind an sé \alst{o}f-drykkja \alst{ǫ}ls.\eva

\bvb A better burden bears man not on the road \\
than much manwit. \\
Worse way-provision he drags not along on the plain \\
than a too great drink of ale.\evb
\evg


\bvg
\bva Es-a svá \alst{g}ótt, \hld\ sęm \alst{g}ótt kveða, &
\ind \alst{ǫ}l \alst{a}lda sonum; &
því-at \alst{f}ę́ra vęit, \hld\ es \alst{f}lęira drekkr, &
\ind síns til \alst{g}ęðs \alst{g}umi.\eva

\bvb ’Tis not so good, as good they say, \\
ale for the sons of men; \\
for the less he knows, as the more he drinks, \\
man of his own senses.\evb
\evg


\bvg
\bva \edtrans{\alst{Ó}-minnis-hegri}{Forgetfulness-heron}{\Bfootnote{Lit. “unmemory-heron”; a rather interesting personification of drunkenness as a hovering bird.}} hęitir, \hld\ sá’s yfir \alst{ǫ}lðrum þrumir, &
\ind hann stelr \alst{g}ęði \alst{g}uma; &
þess \alst{f}ogls \alst{f}jǫðrum \hld\ ek \alst{f}jǫtraðr vas’k &
\ind í \alst{g}arði \alst{G}unnlaðar.\eva

\bvb Forgetfulness-heron is called he who over ale-feasts hovers: \\
he robs man of his senses. \\
With that bird’s feathers was I fettered \\
in the yards of \inx[P]{Guthlathe}.\evb
\evg


\bvg
\bva\alst{Ǫ}lr ek varð, \hld\ varð \alst{o}fr-ǫlvi, &
\ind at hins \alst{f}róða \alst{F}jalars; &
því es \alst{ǫ}lðr batst, \hld\ at \alst{a}ptr of hęimtir &
\ind hvęrr sitt \alst{g}ęð \alst{g}umi.\eva

\bvb Drunk I became—I became the drunkest by far— \\
at the learned Fealer’s [home].— \\
That ale-feast is best, where every man \\
fetches back his senses.\evb
\evg

% TODO: NEW SECTION (War)

\bvg
\bva\alst{Þ}agalt ok hugalt \hld\ skyli \alst{þ}jóðans barn &
\ind ok \alst{v}íg-djarft \alst{v}esa; &
\alst{g}laðr ok ręifr \hld\ skyli \alst{g}umna hvęrr, &
\ind unds sinn \alst{b}íðr \alst{b}ana.\eva

\bvb Silent and thoughtful should the ruler’s child \\
—and battle-bold—be. \\
Glad and cheerful should each man [be], \\
until he suffer his bane.\evb
\evg


\bvg
\bva\alst{Ó}-snjallr maðr \hld\ hyggsk munu \alst{ę}y lifa, &
\ind ef við \alst{v}íg \alst{v}arask; &
en \alst{ę}lli gefr hǫ́num \hld\ \alst{ę}ngi frið, &
\ind þótt hǫ́num \alst{g}ęirar \alst{g}efi.\eva

\bvb The unvalorous man thinks he will always live \\
if he of war be wary; \\
but old age gives him no peace, \\
although spears would give him.\footnoteB{The unvalorous man might have been spared by the spears, but death will still find him through miserable old age. Since death is unavoidable it is better to live bravely, even if one risks dying in battle, than to live cowardly and die of sickness. This connects well to the ancient view of the ‘straw-death’ (TODO).}\evb
\evg


\bvg
\bva\alst{K}ópir af-glapi, \hld\ es til \alst{k}ynnis kømr, &
\ind \alst{þ}ylsk hann umb eða \alst{þ}rumir; &
alt es \alst{s}ęnn, \hld\ ef \alst{s}ylg of getr, &
\ind uppi ’s þá \alst{g}ęð \alst{g}uma.\eva

\bvb Gapes the oaf when to visit he comes; \\
he mumbles about or loiters. \\
All at once—if a sip he gets— \\
are the senses of the man exposed.\evb
\evg


\bvg
\bva Sá ęinn \alst{v}ęit, \hld\ es \alst{v}íða ratar &
\ind ok hęfr \edtrans{\alst{f}jǫlð of \alst{f}arit}{journeyed much}{\Bfootnote{Formulaic, also occuring in \Vafthrudnismal\ 3, 44, and so on in the fixed lines spoken by Weden: \emph{Fjǫlð ek fór, \hld\ fjǫlð fręistaða’k, // fjǫlð ek ręynda ręgin} ‘Much I journeyed, much I tried, much I tested the \inx[G]{Reins}.’.}}, &
hvęrju \alst{g}ęði \hld\ stýrir \alst{g}umna hvęrr, &
\ind sá es \alst{v}itandi ’s \alst{v}its.\eva

\bvb He alone knows, who widely roams, \\
and has journeyed much: \\
his own senses does each man control, \\
who is knowing of his wits.\evb
\evg


\bvg
\bva\alst{H}aldi-t maðr á kęri, \hld\ drekki þó at \alst{h}ófi mjǫð, &
\ind \edtrans{mę́li \alst{þ}arft eða \alst{þ}ęgi}{he ought to speak the needful or shut up}{\Bfootnote{Formulaic, line occurs identically in \Vafthrudnismal\ 10/2.}}; &
\alst{ó}-kynnis þess \hld\ váar þik \alst{ę}ngi maðr, &
\ind at gangir \alst{s}nimma at \alst{s}ofa.\eva

\bvb Man ought not to hold onto the cask, yet drink mead in moderation;\footnoteB{Drinking horns at this time could not be set down, and so to “hold onto” may have been an expression for not drinking. The st. may also be referring to the toasting ritual wherein a single vessel would be passed around and drunk from by each person (indeed this is the origin of the Scandinavian toasting-word, \emph{skål} ‘prosit, cheers!’, lit. ‘bowl!’). At such celebrations “holding onto” the vessel and refusing to drink was very rude; as late as 1519 a man in Jämtland was killed in an argument resulting from his refusal to pass on to the bowl (see \textcite{Sjöberg1907}).} \\
he ought to speak the needful or shut up. \\
For that uncouthness will no man blame thee, \\
that thou go early to sleep.\evb
\evg


\bvg
\bva\alst{G}rǫ́ðugr halr, \hld\ nema \alst{g}ęðs viti, &
\ind \alst{e}tr sér \alst{a}ldr-trega; &
opt fę́r \alst{h}lǿgis, \hld\ es með \alst{h}orskum kømr, &
\ind \alst{m}anni hęimskum \alst{m}agi.\eva

\bvb The gluttonous man—unless he know his sense— \\
eats himself a life-sorrow. \\
Oft the belly, when among the sharp he comes, \\
brings a foolish man ridicule.\evb
\evg


\bvg
\bva\alst{H}jarðir þat vitu, \hld\ nę́r \alst{h}ęim skulu, &
\ind ok \alst{g}anga þá af \alst{g}rasi; &
en \alst{ó}-sviðr maðr \hld\ kann \alst{ę́}va-gi &
\ind síns of \alst{m}ál \alst{m}aga.\eva

\bvb Herds know when homewards they shall [turn], \\
and then part from the grass; \\
but an unwise man never knows \\
his own belly’s measure.\evb
\evg


\bvg
\bva\alst{V}e-sall maðr \hld\ ok \alst{i}lla skapi &
\ind \alst{h}lę́r at \alst{h}ví-vetna; &
hit-ki hann \alst{v}ęit, \hld\ es \alst{v}ita þyrpti, &
\ind at \edtrans{hann es-a \alst{v}amma \alst{v}anr}{he is not free of blemishes}{\Bfootnote{Formulaic, cf. \Lokasenna\ 30: \emph{es-a þér vamma vant} ‘thou art not free of blemishes’.}}.\eva

\bvb The wretched man and badly tempered \\
laughs at anything. \\
This he knows not, which he might need to know: \\
that he is not free of blemishes.\evb
\evg


\bvg
\bva\alst{Ó}-sviðr maðr \hld\ vakir umb \alst{a}llar nę́tr &
\ind ok \alst{h}yggr at \alst{h}ví-vetna; &
þá es \alst{m}óðr, \hld\ es at \alst{m}orni kømr; &
\ind alt es \alst{v}íl sęm \alst{v}as.\eva

\bvb The unwise man is awake during all nights, \\
and thinks of anything. \\
Then he is weary when the morning comes: \\
all the trouble is as it was.\evb
\evg


\bvg
\bva\alst{Ó}-snotr maðr \hld\ hyggr sér \alst{a}lla vesa &
\ind \alst{v}ið-hlę́jęndr \alst{v}ini; &
hit-ki hann \alst{f}iðr, \hld\ þótt of hann \alst{f}ár lesi, &
\ind ef með \alst{s}notrum \alst{s}itr.\eva

\bvb The unclever man thinks all to be \\
who laugh with him his friends. \\
This he finds not, that they still see flaws in him, \\
if among the clever he sits.\evb
\evg


\bvg
\bva\alst{Ó}-snotr maðr \hld\ hyggr sér \alst{a}lla vesa &
\ind \alst{v}ið-hlę́jęndr \alst{v}ini; &
\alst{þ}á þat fiðr \hld\ es at \alst{þ}ingi kømr, &
\ind at \edtrans{á \alst{f}or-mę́lęndr \alst{f}áa}{has spokesmen few}{\Bfootnote{Repeated in st. 62. He has few who are ready to take his side and speak up for him; the sense is that true friends are proven in conflict, not in easy things like laughing. The Thing was the old Germanic legal assembly, and so the specific reference here is to legal disputes, which, however, could easily turn into deadly feuds.}}.\eva

\bvb The unclever man thinks all to be \\
who laugh with him his friends. \\
Then he finds it, when to the \inx[C]{Thing} he comes, \\
that he has spokesmen few.\evb
\evg


\bvg
\bva\alst{Ó}-snotr maðr \hld\ þykkisk \alst{a}lt vita, &
\ind ef á sér i \alst{v}ǫ́ \alst{v}eru; &
hit-ki hann \alst{v}ęit, \hld\ hvat skal \alst{v}ið kveða, &
\ind ef hans \alst{f}ręista \alst{f}irar.\eva

\bvb The unclever man seems to know everything \\
if he finds shelter in a nook. \\
This he knows not, what he shall say in return \\
if men test him.\evb
\evg


\bvg
\bva\alst{Ó}-snotr maðr, \hld\ es með \alst{a}ldir kømr, &
\ind \alst{þ}at ’s batst at hann \alst{þ}ęgi; &
\alst{ę}ngi þat vęit, \hld\ at hann \alst{ę}kki kann, &
\ind nema hann \alst{m}ę́li til \alst{m}art. &
\alst{v}ęit-a maðr, \hld\ hinn’s \alst{v}ę́t-ki vęit, &
\ind þótt hann \alst{m}ę́li til \alst{m}art.\eva

\bvb The unclever man, when among people he comes, \\
’tis best that he shut up. \\
Noone knows that he nothing knows, \\
unless he speak too much. \\
The man knows not, who nothing knows, \\
that he speak too much.\evb
\evg


\bvg
\bva\alst{F}róðr sá þykkisk, \hld\ es \alst{f}regna kann, &
\ind ok \alst{s}ęgja hit \alst{s}ama, &
\alst{ęy}-vitu lęyna \hld\ męgu \alst{ý}ta synir &
\ind því es \alst{g}ęngr of \alst{g}uma.\eva

\bvb Learned seems he who can ask \\
and answer likewise. \\
Naught may the sons of men conceal \\
of that [gossip] which goes about a man.\evb
\evg


\bvg
\bva\alst{Ǿ}rna mę́lir, \hld\ sá’s \alst{ę́}va þęgir, &
\ind \alst{st}að-lausu \alst{st}afi; &
\edtext{\alst{h}rað-mę́lt tunga, \hld\ \edtrans{nema \alst{h}aldęndr ęigi}{unless it be held in place}{\Bfootnote{lit. ‘unless holders own it’ or ‘unless it own holders’. The ‘holders’ are perhaps the teeth which hold the tongue in place.}}, &
\ind opt sér ó·\alst{g}ótt of \alst{g}ęlr}{\lemma{hrað-mę́lt \dots\ of gęlr ‘A quick-spoken \dots\ for itself’}\Bfootnote{Formulaic. Cf. \Lokasenna\ 31.}}.\eva

\bvb Plenty enough speaks he who never shuts up \\
utterings of absurdity. \\
A quick-spoken tongue—unless it be held in place— \\
oft sings evil [into being] for itself.\evb
\evg


\bvg
\bva At \alst{au}ga-bragði \hld\ skal-a maðr \alst{a}nnan hafa, &
\ind þótt til \alst{k}ynnis \alst{k}omi; &
margr \alst{f}róðr þykkisk, \hld\ ef \alst{f}reginn es-at &
\ind ok nái \edtrans{\alst{þ}urr-fjallr}{dry-skinned}{\Bfootnote{i.e. ‘untested’, equivalent to the English idiom \emph{get one’s feet wet}. The word \emph{fell} \char`~ \emph{fjall} ‘skin, pelt’ is rare in Old Norse literature and only occurs in cpds, e.g. \Volundarkvida\ 11: \emph{ber-fjall} ‘bear-pelt’. Cf. however Swedish \emph{fjäll} ‘scale (on fish and reptiles)’}} \alst{þ}ruma.\eva

\bvb As a laughing-stock shall man not have another \\
when he comes to visit. \\
Many a one seems learned if he is not asked, \\
and manages to loiter about dry-skinned.\evb
\evg


\bvg
\bva\alst{F}róðr þykkisk \hld\ sá’s \alst{f}lótta tękr &
\ind \edtrans{\alst{g}ęstr}{guest}{\Bfootnote{Here probably ‘stranger’; when being mocked by a stranger it is best not to engage, since the conversation can quickly turn violent. Cf. sts. 122–123 and 125.}} at \alst{g}ęst hę́ðinn; &
\alst{v}ęit-a gǫrla \hld\ sá’s of \alst{v}erði glissir, &
\ind þótt með \alst{g}rǫmum \alst{g}lami.\eva

\bvb Learned seems he who takes to flight, \\
the guest, from a scoffing guest. \\
Clearly knows not he who grins over the food, \\
that he with fiends be prattling.\evb
\evg


\bvg
\bva\alst{G}umnar margir \hld\ erusk \alst{g}agn-hollir, &
\ind en at \alst{v}irði \alst{v}rekask; &
\alst{a}ldar róg \hld\ þat mun \alst{ę́} vesa; &
\ind órir \alst{g}ęstr við \alst{g}ęst.\eva

\bvb Many men are \inx[C]{hold} to each other, \\
but over a meal drive each other away. \\
The strife of mankind will that ever be; \\
guest raves against guest.\evb
\evg


\bvg
\bva\alst{Á}r-liga verðar \hld\ skyli maðr \alst{o}pt fáa, &
\ind nema til \alst{k}ynnis \alst{k}omi; &
\alst{s}itr ok \alst{s}nópir, \hld\ lę́tr sęm \alst{s}olginn sé, &
\ind ok kann \alst{f}regna at \alst{f}ǫ́u.\eva

\bvb An early meal should man oft get, \\
unless he come to visit: \\
he sits and idles haplessly, makes as if starved, \\
and can ask about little.\evb
\evg


\bvg
\bva\alst{A}f-hvarf mikit \hld\ es til \alst{i}lls vinar, &
\ind þótt á \alst{b}rautu \alst{b}úi, &
en til \alst{g}óðs vinar \hld\ liggja \alst{g}agn-vegir, &
\ind þótt hann sé \alst{f}irr \alst{f}arinn.\eva

\bvb A great detour ’tis to a bad friend, \\
although he on the highway live; \\
but to a good friend lie the finest ways, \\
although he far gone be.\evb
\evg


\bvg
\bva\alst{G}anga \edtext{skal}{\Afootnote{emend.; om. \Regius}}, \hld\ skal-a \alst{g}ęstr vesa &
\ind \alst{ęy} í \alst{ęi}num stað; &
\alst{l}júfr verðr \alst{l}ęiðr, \hld\ ef \alst{l}ęngi sitr &
\ind \alst{a}nnars flętjum \alst{á}.\eva

\bvb Go shall one; shall not be a guest \\
forever in one place. \\
The loved becomes loathed if for long he sits \\
on another man’s benches.\evb
\evg


\bvg
\bva\alst{B}ú es \alst{b}ętra, \hld\ þótt lítit sé, &
\ind \alst{h}alr es \alst{h}ęima \alst{h}vęrr; &
þótt \alst{t}vę́r gęitr ęigi \hld\ ok \alst{t}aug-ręptan sal, &
\ind þat ’s þó \alst{b}ętra an \alst{b}ǿn.\eva

\bvb A dwelling is better, though small it be: \\
each is a warrior at home. \\
Though two goats he own, and a cord-roofed hall, \\
that is yet better than begging.\evb
\evg


\bvg
\bva\alst{B}ú es \alst{b}ętra, \hld\ þótt lítit sé, &
\ind \alst{h}alr es \alst{h}ęima \alst{h}vęrr; &
\alst{b}lóðugt es hjarta \hld\ þęim’s \alst{b}iðja skal &
\ind sér í \alst{m}ál hvęrt \alst{m}atar.\eva

\bvb A dwelling is better, though small it be: \\
each is a warrior at home. \\
Bloody is the heart of the one who shall beg \\
for himself each meal of food.\evb
\evg


\bvg
\bva\alst{V}ǫ́pnum sínum \hld\ skal-a maðr \edtrans{\alst{v}ęlli á}{on the plain}{\Bfootnote{Formulaic, see note to st. 12.}} &
\ind \edtrans{\alst{f}eti ganga \alst{f}ramarr}{take one step further}{\Bfootnote{Formulaic. Cf. \Lokasenna\ 1: \emph{svát ęinu-gi feti gangir framarr} ‘so that thou not take one step further’.}}; &
því-at ó·\alst{v}íst ’s at \alst{v}ita, \hld\ nę́r verðr á \alst{v}egum úti &
\ind \alst{g}ęirs of þǫrf \alst{g}uma.\eva

\bvb From his weapons shall man not on the plain \\
take one step further; \\
for uncertain ’tis to know, when on the ways outside, \\
man comes in need of a spear.\evb
\evg


\bvg
\bva Fann’k-a \alst{m}ildan \alst{m}ann \hld\ eða svá \edtrans{\alst{m}atar góðan}{good of meat}{\Bfootnote{A Viking Age expression; see Encyclopedia.}}, &
\ind at vę́ri-t \alst{þ}iggja \alst{þ}egit; &
eða \alst{s}íns féar \hld\ \alst{s}vá-gi \edtext{[...]}{\Bfootnote{It is doubtless that a word has been lost here; the meter and sense require it. \textcite{FinnurEdda}\ suggests \emph{gløggvan} ‘miserly, stingy’, giving a litotes ‘so not stingy’, i.e., ‘so generous’.}}, &
\ind at \alst{l}ęið sé \alst{l}aun, ef þegi.\eva

\bvb I found not a generous man, or one so \inx[C]{good of meat}, \\
that a gift were not accepted; \\
or one of his \inx[C]{fee} so not [...], \\
that the rewards were loathed, if he accepted [them].\footnoteB{No man is so generous that he would refuse a gift presented to him, nor loathe receiving a favour as thanks for his generosity.}\evb
\evg


\bvg
\bva\alst{F}éar síns, \hld\ es \alst{f}ęngit hęfr, &
\ind skyli-t maðr \alst{þ}ǫrf \alst{þ}ola; &
opt sparir \alst{l}ęiðum \hld\ þat’s hęfr \alst{l}júfum hugat; &
\ind mart gęngr \alst{v}err an \alst{v}arir.\eva

\bvb Of his own \inx[C]{fee}, which he has earned, \\
should man not suffer need. \\
Oft one saves for the loathed what was meant for the loved;\\
many a thing goes worse than one expects.\evb
\evg


\bvg
\bva\alst{V}ǫ́pnum ok \alst{v}ǫ́ðum \hld\ skulu \alst{v}inir glęðjask; &
\ind þat ’s á \alst{s}jǫlfum \alst{s}ýnst; &
\alst{v}iðr-gefęndr ok ęndr-gefęndr \hld\ erusk \alst{v}inir lęngst, &
\ind ef þat bíðr at \alst{v}erða \alst{v}ęl.\eva

\bvb With weapons and garments shall friends gladden each other; \\
that is most seen on oneself.\footnoteB{i.e. in one’s own lived experience.} \\
Mutual givers and return-givers are friends for the longest, \\
if it\footnoteB{The friendship.} is to last long.\evb
\evg


\bvg
\bva\alst{V}in sínum \hld\ skal maðr \alst{v}inr \alst{v}esa, &
\ind ok \alst{g}jalda \alst{g}jǫf við \alst{g}jǫf; &
\alst{h}látr við \alst{h}látri \hld\ skyli \alst{h}ǫlðar taka, &
\ind en \alst{l}ausung við \alst{l}ygi.\eva

\bvb With his friend shall man be a friend, \\
and pay gift against gift; \\
laughter against laughter should men employ, \\
but duplicity against lie.\evb
\evg


\bvg
\bva\alst{V}in sínum \hld\ skal maðr \alst{v}inr vesa, &
\ind \alst{þ}ęim ok \alst{þ}ess vin; &
en \alst{ó}-vinar síns \hld\ skyli \alst{ę}ngi maðr &
\ind \alst{v}inar \alst{v}inr \alst{v}esa.\eva

\bvb With his friend shall man be a friend, \\
with him and his friend; \\
but with his enemy’s, should no man, \\
friend’s friend be.\evb
\evg


\bvg
\bva\alst{V}ęitst, ef \alst{v}in átt, \hld\ þann’s \alst{v}ęl trúir &
\ind ok vilt af hǫ́num \alst{g}ótt \alst{g}eta, &
\alst{g}ęði skalt við þann \hld\ ok \alst{g}jǫfum skipta, &
\ind \alst{f}ara at \alst{f}inna opt.\eva

\bvb Know, if thou have a friend, one on which thou well trust, \\
and wilt receive good from him: \\
thoughts and gifts shalt thou exchange with him; \\
journey to find him oft.\footnoteB{Several lines of the present st. are shared with st. 119.}\evb
\evg


\bvg
\bva Ef þú \alst{á}tt \alst{a}nnan, \hld\ þann’s \alst{i}lla trúir, &
\ind vilt af hǫ́num þó \alst{g}ótt \alst{g}eta, &
\edtext{\alst{f}agrt skalt mę́la við þann, \hld\ en \alst{f}látt hyggja}{\lemma{fagrt \dots\ mę́la \dots\ flátt hyggja ‘fairly \dots\ speak \dots\ falsely think’}\Bfootnote{Formulaic, cf. sts. 90, 91.}} &
\ind ok gjalda \alst{l}ausung við \alst{l}ygi.\eva

\bvb If thou have another, one on which thou badly trust, \\
and wilt yet receive good from: \\
fairly shalt thou speak with him, but falsely think, \\
and pay duplicity against lie.\evb
\evg


\bvg
\bva Þat ’s \alst{ę}nn umb þann, \hld\ es þú \alst{i}lla trúir &
\ind ok þér es \alst{g}runr at \alst{g}ęði, &
\alst{h}lę́ja skalt við þęim \hld\ ok of \alst{h}ug mę́la; &
\ind \alst{g}lík skulu \alst{g}jǫld \alst{g}jǫfum.\eva

\bvb ’Tis yet regarding that one, on which thou badly trustest, \\
and who causes thy senses doubt:\footnoteB{lit. “and for thee is doubt in senses”.} \\
laugh shalt thou with him, and speak thoughtfully; \\
payments shall be equal to gifts.\footnoteB{Equivalent to the last line of the previous st. (“pay duplicity against lie”).}\evb
\evg


\bvg
\bva Ungr vas’k \alst{f}orðum, \hld\ \alst{f}ór’k ęinn saman, &
\ind þá varð’k \alst{v}illr \alst{v}ega; &
\alst{au}ðigr þóttumk, \hld\ es \alst{a}nnan fann’k, &
\ind \alst{m}aðr es \alst{m}anns gaman.\eva

\bvb Young was I once, I travelled alone; \\
then I became lost about the ways. \\
Wealthy I thought myself when another one I found; \\
man is man’s pleasure.\evb
\evg


\bvg
\bva\alst{M}ildir frǿknir \hld\ \alst{m}ęnn batst lifa, &
\ind \alst{s}jaldan \alst{s}út ala; &
\alst{ó}-snjallr maðr \hld\ \alst{u}ggir hvat-vetna, &
\ind sýtir ę́ \alst{g}løggr við \alst{g}jǫfum.\eva

\bvb Generous, bold men live the best; \\
seldom they nourish grief. \\
The unvalorous man is frightened by anything; \\
ever the stingy man grieves over gifts.\footnoteB{Refer back to st. 39; after receiving a gift, one was culturally obliged to give something back.}\evb
\evg


\bvg
\bva\alst{V}áðir mínar \hld\ gaf’k \alst{v}ęlli at &
\ind \alst{t}vęim \alst{t}ré-mǫnnum; &
\alst{r}ekkar þat þóttusk, \hld\ es \alst{r}ipt hǫfðu; &
\ind \alst{n}ęiss es \alst{n}ǫkkviðr halr.\eva

\bvb My garments I gave, on the plain, \\
to two tree-men. \\
Champions they seemed when cloaks they had; \\
shameful is the naked warrior.\footnoteB{One of the hardest sts. in the poem. After much thought I consider the probable sense to be that “the clothes make the man”. Under expensive gear a thin tree-man might be hiding, and likewise even a strong man (I see the choice of the word \emph{halr} ‘warrior’ rather than the more neutral \emph{maðr} ‘man, person’ as intentional) when naked and facing a heavily armoured opponent becomes as vulnerable as the ‘tree-man’ on a plain.}\evb
\evg


\bvg
\bva Hrørnar \alst{þ}ǫll, \hld\ sú’s stęndr \alst{þ}orpi á, &
\ind hlýr-at hęnni \alst{b}ǫrkr né \alst{b}arr; &
svá es \alst{m}aðr, \hld\ sá’s \alst{m}ann-gi ann; &
\ind hvat skal hann \alst{l}ęngi \alst{l}ifa?\eva

\bvb Wilters the pine that stands on the yard; \\
shields her not bark nor needle. \\
So is the man who loves no man; \\
for what shall he live for long?\evb
\evg


\bvg
\bva\alst{Ę}ldi hęitari \hld\ brinnr með \alst{i}llum vinum &
\ind \alst{f}riðr \alst{f}imm daga, &
en þá \alst{sl}oknar, \hld\ es hinn \alst{s}étti kømr, &
\ind ok \alst{v}ersnar allr \alst{v}in-skapr.\eva

\bvb Hotter than fire burns love among bad friends, \\
for \inx[C]{five days};\footnoteB{A reference to the five-day week (see also st. 74); the number is symbolic. See further Encyclopedia.} \\
but then goes out when the sixth one comes, \\
and all the friendship worsens.\evb
\evg


\bvg
\bva\alst{M}ikit ęitt \hld\ skal-a \alst{m}anni gefa; &
\ind opt kaupir sér í \alst{l}ítlu \alst{l}of, &
með \alst{h}ǫlfum \alst{h}lęif \hld\ ok með \alst{h}ǫllu kęri &
\ind \alst{f}ekk ek mér \alst{f}é-laga.\eva

\bvb Much at once shall one not give a man; \\
oft one buys oneself praise for little. \\
With half a loaf and an awry cask, \\
I got me a companion.\evb
\evg


\bvg
\bva\alst{L}ítilla sanda, \hld\ \alst{l}ítilla sę́va, &
\ind lítil eru \alst{g}ęð \alst{g}uma; &
því-at \alst{a}llir męnn \hld\ \alst{u}rðu-t jafn-spakir; &
\ind \alst{h}ǫlf es ǫld \alst{h}var.\eva

\bvb Of small sands, of small seas; \\
small are the senses of man. \\
For all have not become evenly knowing; \\
half is every man.\footnoteB{The genitive “of small sands, of small seas” is probably a partitive, the sense being that man’s horizons are small; the universe is far greater than he and always will be. On the meaning of the second half of the st. I find that of \textcite{Athugasemdir1929} most convincing, namely that every man has both strengths and weaknesses. As nobody can excel at everything, nobody is complete; every person is “half” (and it should be added that ON \emph{halfr} has a more general sense of incompleteness than its English cognate). This interpretation fits particularly closely with sts. 71 and 132.}\evb
\evg


\bvg
\bva\alst{M}eðal-snotr \hld\ skyli \alst{m}anna hvęrr, &
\ind ę́va til \alst{s}notr \alst{s}éi; &
þęim es \alst{f}yrða \hld\ \alst{f}ęgrst at lifa, &
\ind es \alst{v}ęl mart \alst{v}itu.\eva

\bvb Middle-clever should each man be; \\
never too clever. \\
For those men ’tis fairest to live \\
who know well enough.\evb
\evg


\bvg
\bva\alst{M}eðal-snotr \hld\ skyli \alst{m}anna hvęrr, &
\ind ę́va til \alst{s}notr \alst{s}éi; &
\alst{s}notrs manns hjarta \hld\ verðr \alst{s}jaldan glatt, &
\ind ef sá ’s \alst{a}lsnotr es \alst{á}.\eva

\bvb Middle-clever should each man be; \\
never too clever. \\
The clever man’s heart is seldom gladdened, \\
if he is all-clever that owns [it].\evb
\evg


\bvg
\bva\alst{M}eðal-snotr \hld\ skyli \alst{m}anna hvęrr, &
\ind ę́va til \alst{s}notr \alst{s}éi; &
\alst{ø}r-lǫg sín \hld\ viti \alst{ę}ngi fyr; &
\ind þęim es \alst{s}orga-lausastr \alst{s}efi.\eva

\bvb Middle-clever should each man be; \\
never too clever. \\
His own \inx[C]{orlay} ought none to know ahead; \\
his is the most sorrowless mind.\footnoteB{Who knows not his fate. It is fitting that Weden should say this, having knowledge of the inevitable destruction of the world and hisself.}\evb
\evg


\bvg
\bva\alst{B}randr af \alst{b}randi \hld\ \alst{b}rinnr unds \alst{b}runninn es, &
\ind \alst{f}uni kvęykisk af \alst{f}una; &
\alst{m}aðr af \alst{m}anni \hld\ verðr at \alst{m}áli kuðr; &
\ind en til \alst{d}ǿlskr af \alst{d}ul.\eva

\bvb Fire by fire burns until it is burnt [out]; \\
flame is quickened by flame. \\
Man by man becomes known through speech, \\
but the too hickish from delusion.\evb
\evg


\bvg
\bva\alst{Á}r skal rísa, \hld\ sá’s \alst{a}nnars vill &
\ind \alst{f}é eða \alst{f}jǫr hafa; &
sjaldan \alst{l}iggjandi ulfr \hld\ \alst{l}ę́r of getr, &
\ind né \alst{s}ofandi maðr \alst{s}igr.\eva

\bvb Early shall he rise who another man’s \\
\inx[C]{fee} or life will have. \\
Seldom gets the lying wolf the thigh, \\
nor the sleeping man victory.\evb
\evg


\bvg
\bva\alst{Á}r skal rísa, \hld\ sá’s á \alst{y}rkjęndr fáa, &
\ind ok ganga síns \alst{v}erka á \alst{v}it; &
\alst{m}art of dvęlr \hld\ þann’s umb \alst{m}orgin sefr, &
\ind \alst{h}alfr es auðr und \alst{h}vǫtum.\eva

\bvb Early he shall rise who owns workers few, \\
and go his work to meet. \\
Much is kept back from him who in the morning sleeps; \\
a half wealth is under the brisk.\footnoteB{The brisk man has already obtained a “half wealth” just by putting his work before his comfort (and sleeping in).}\evb
\evg


\bvg
\bva\alst{Þ}urra skíða \hld\ ok \alst{þ}akinna nę́fra, &
\ind þess kann \alst{m}aðr \alst{m}jǫt, &
ok þess \alst{v}iðar, \hld\ es \alst{v}innask męgi &
\ind \alst{m}ál ok \alst{m}issęri.\eva

\bvb Dry planks and of thatching birch bark: \\
of this man knows the measure— \\
and of that firewood which he may use \\
for a season and half-year.\footnoteB{i.e. over the winter.}\evb
\evg


\bvg
\bva\alst{Þ}vęginn ok męttr \hld\ ríði maðr \alst{þ}ingi at, &
\ind þótt sé-t \alst{v}ę́ddr til \alst{v}ęl; &
\alst{sk}úa ok bróka \hld\ \alst{sk}ammisk ęngi maðr &
\ind né \alst{h}ęsts in \alst{h}ęldr. \hld\ (\edtext{þótt hann \alst{h}afi-t góðan}{\lemma{þótt \dots\ góðan ‘although \dots\ good one’}\Bfootnote{As \textcite{FinnurEdda} points out this line is surely a late insert. The inserter was not aware of the rules of the \Ljodahattr\ meter and interpreted the c-verse as an a-verse in \Fornyrdislag.}}).\eva

\bvb Washed and full\footnoteB{A collocation. Cf. \Reginsmal\ TODO: \emph{kęmbðr} ‘combed’ — \emph{þvęginn} ‘washed’ — \emph{męttr} ‘full’; \Voluspa\ 33: \emph{þó} ‘washed’ — \emph{kęmbði} ‘combed’. These examples attest to the importance of personal hygiene in the culture, something further seen by the ubiquity of combs in pre-Christian graves. Cf. also Tacitī Germania 22: \emph{Statim e somno, quem plerumque in diem extrahunt, lavantur, saepius calida, ut apud quos plurimum hiems occupat. Lauti cibum capiunt: separatae singulis sedes et sua cuique mensa. Tum ad negotia nec minus saepe ad convivia procedunt armati.} ‘On waking from sleep, which they generally prolong to a late hour of the day, they take a bath, oftenest of warm water, which suits a country where winter is the longest of the seasons. After their bath they take their meal, each having a separate seat and table of his own. Then they go armed to business, or no less often to their festal meetings.’} ought a man to ride to the Thing, \\
although he be not clothed too well; \\
of his shoes and his breeches ought no man to be ashamed, \\
nor the more of his horse. (although he has not a good one.)\evb
\evg

The two following sts. were written in opposite order in \Regius, but a symbol at the start of each indicates that they should switch places; hence it has been done in the present edition.

\bvg
\bva\alst{S}napir ok gnapir, \hld\ es til \alst{s}ę́var kømr, &
\ind \alst{ǫ}rn á \alst{a}ldinn mar; &
svá es \alst{m}aðr, \hld\ es með \alst{m}ǫrgum kømr &
\ind ok \edtrans{á \alst{f}or-mę́lęndr \alst{f}áa}{has spokesmen few}{\Bfootnote{Shared with st. 25.}}.\eva

\bvb Snaps and stoops—when to the sea it comes— \\
the eagle on the aged ocean. \\
So is the man who among the many comes, \\
and has spokesmen few.\evb
\evg


\bvg
\bva\alst{F}regna ok sęgja \hld\ skal \alst{f}róðra hvęrr, &
\ind sá’s vill \alst{h}ęitinn \alst{h}orskr; &
\alst{ęi}nn vita \hld\ né \alst{a}nnarr skal, &
\ind \alst{þ}jóð vęit ef \alst{þ}rír ’ru.\eva

\bvb Ask and speak shall each learned man \\
who wishes to be called sharp. \\
\emph{One} shall know, but not another; \\
thirty\footnoteB{\emph{þjóð} lit. ‘people, nation’; cf. \Skaldskaparmal\ (TODO): \emph{þjóð eru þrír tigir} ‘thirty are a \emph{people}’.} know if there are three.\evb
\evg


\bvg
\bva\alst{R}íki sitt \hld\ skyli \alst{r}áð-snotra &
\ind hvęrr í \alst{h}ófi \alst{h}afa; &
\edtext{þá þat \alst{f}innr, \hld\ es með \alst{f}rǿknum kømr, &
\ind at \alst{ę}ngi es \alst{ęi}nna hvatastr.}{\lemma{þá \dots\ ęinna hvatastr ‘then \dots briskest of all’}\Bfootnote{Almost identical to \Reginsmal\ TODO/3–4, which however has \emph{flęirum} ‘more men’ for \emph{frǿknum} ‘the bold’.}}\eva

\bvb His own power should each counsel-clever \\
man use in moderation; \\
then he finds it—when among the bold he comes— \\
that none is the briskest of all.\footnoteB{i.e., every man has his match.}\evb
\evg


\bvg
\bva\alst{O}rða þęira, \hld\ es maðr \alst{ǫ}ðrum sęgir, &
\ind opt hann \alst{g}jǫld of \alst{g}etr.\eva

\bvb For those words which man to another says, \\
he oft gets recompense.\evb
\evg


\bvg
\bva \edtrans{\alst{M}ikils til}{Much too}{\Afootnote{written as one word \emph{mikilsti} \Regius}} snimma \hld\ kom’k í \alst{m}arga staði, &
\ind en til \alst{s}íð í \alst{s}uma; &
\alst{ǫ}l vas drukkit, \hld\ sumt vas \alst{ó}-lagat; &
\ind sjaldan hittir \alst{l}ęiðr í \alst{l}ið.\eva

\bvb Much too early I came to many places, \\
but too late to some. \\
Ale was drunk, some was unbrewed; \\
seldom finds the loathed one his place.\evb
\evg


\bvg
\bva\alst{H}ér ok \alst{h}var \hld\ myndi mér \alst{h}ęim of boðit, &
\ind ef þyrpta’k at \alst{m}ǫ́lun-gi \alst{m}at, &
eða \alst{t}vau lę́r hęngi \hld\ at hins \alst{t}ryggva vinar, &
\ind þar’s ek hafða \alst{ęi}tt \alst{e}tit.\eva

\bvb Here and there would I to a home be invited, \\
if at no meal-time I needed food; \\
or [if] two hams should hang at the trusty friend’s [home], \\
where I had eaten one.\footnoteB{Not everyone is hospitable, especially with regards to food, which was scarce and closely watched among subsistence farmers. The speaker notes that even a “trusty friend” (possibly sarcastic) would invite him more often if he could increase the amount of food rather than decrease it.}\evb
\evg


\bvg
\bva\alst{Ę}ldr es batstr \hld\ með \alst{ý}ta sonum &
\ind ok \alst{s}ólar \alst{s}ýn, &
\alst{h}ęilyndi sitt, \hld\ ef maðr \alst{h}afa náir, &
\ind án við \alst{l}ǫst at \alst{l}ifa.\eva

\bvb Fire is best among the sons of men, \\
and the sight of the sun; \\
one’s good health, if he manage to keep it— \\
{[and]} not living by vice.\evb
\evg


\bvg
\bva\alst{E}s-at maðr \alst{a}lls \edtrans{ve-sall}{unblessed}{\Bfootnote{Or ‘woe-blessed’. I have elsewhere translated this word as ‘wretched’, but I have presently rendered it this way to show the etymological relationship. The second element in this word is \emph{sę́ll}, but lacks i-umlaut due to Proto-Norse shortening of the vowel before the umlaut occurred or became phonemic. The ancestral Proto-Norse forms would be \emph{*sāliʀ} and \emph{*wajē-sāliʀ}. Cf. here ᚹᚨᛃᛖ-ᛗᚨᚱᛁᛉ \emph{wajē-mariʀ} ‘infamous’ on the Tjurkö bracteate, where the second element is the ancestor of ON \emph{mę́rr} ‘renowned, famous’. The expected descendant \emph{*ve-marr} is not attested.}\Bfootnote{I have chosen to translate \emph{sę́ll} as ‘blessed’, but it is not a past participle and could also be rendered as ‘lucky’. It carries with it a certain sense of innateness, in a way that modern Westerners may find foreign. So a king whose reign is one of peace (\emph{friðr}) is said to be \emph{frið-sę́ll} ‘blessed with peace’, while one who reigns during good harvests (\emph{ár}) is said to be \emph{ár-sę́ll} ‘blessed with harvests’. The harvests and peace are not due to environmental or political factors outside of his control, but rather spring from the king himself (TODO: Reference PCRN chapter).}}, \hld\ þótt sé \alst{i}lla hęill, &
\ind \alst{s}umr es af \edtext{\alst{s}onum}{\lemma{sonum \dots\ frę́ndum ‘sons \dots\ kinsmen’}\Bfootnote{Cf. st. 72 below, which stresses the importance of sons and kinsmen.}} \alst{s}ę́ll, &
sumr af \alst{f}rę́ndum, \hld\ sumr af \alst{f}é ǿrnu, &
\ind sumr af \alst{v}erkum \alst{v}ęl.\eva

\bvb Man is not all unblessed, though he of poor health be: \\
someone is blessed with sons; \\
someone with kinsmen, someone with ample \inx[C]{fee}, \\
someone with works done well.\evb
\evg


\bvg
\bva Bętra ’s \alst{l}ifðum, \hld\ \edtrans{an séi ó·\alst{l}ifðum}{than with the unliving}{\Bfootnote{emend.; \emph{⁊ ſęl lıfðo}m \Regius. The normalized reading \emph{ok sę́l-lifðum} ‘and for the blessed living’ is metrically defect, since \emph{sę́l-} is strongly stressed and thus should carry alliteration. For the original form of the line we may instead compare \Fafnismal\ 30: \emph{Hvǫtum ’s bętra \hld\ an sé óhvǫtum} ‘For the brisk ’tis better than it may be for the unbrisk’. The corruption is understandable; \emph{*en} (younger form of \emph{an}) ‘than’ was interpreted as \emph{en} ‘and, but’ and copied as \emph{⁊} (the tironian \emph{et}), while \emph{*séı ólıfðo}m (probably with the words cramped together) became \emph{sęl lıfðo}m.}}, &
\ind ęy getr \alst{k}vikr \alst{k}ú; &
\alst{ę}ld sá’k \alst{u}pp brinna \hld\ \alst{au}ðgum manni fyr, &
\ind en úti vas \alst{d}auðr fyr \alst{d}urum.\eva

\bvb ’Tis better for the living than it may be for the unliving: \\
ever gets the quick a cow.\footnoteB{A reference to the cattle-based economy (see also st. 76), the cow being used as a metonym: “new opportunities always present themselves for the living” (cf. churchly English ‘the \emph{quick} and the dead’, i.e. ‘the \emph{living} and the dead’).} \\
A fire I saw burning high for a wealthy man, \\
but outside he was dead before the doors.\footnoteB{The fire is probably the man’s funeral pyre. It is notable that his wealth is mentioned; according to Ibn Fadlan (TODO) two thirds of a great chieftain’s wealth was spent on his funeral. One notes the contrastive \emph{en} ‘but’, and may paraphrase it as something like “I saw a lavish funeral, \emph{but} the burning man was dead \emph{anyway}.” This interpretation is supported by the following st. (\Havamal\ 70, especially the second half), which expresses the same sentiment.”}\evb
\evg


\bvg
\bva\alst{H}altr ríðr \alst{h}rossi, \hld\ \alst{h}jǫrð rekr \alst{h}andar vanr, &
\ind \alst{d}aufr vegr ok \alst{d}ugir; &
\alst{b}lindr es \alst{b}ętri, \hld\ an \alst{b}ręndr séi; &
\ind \alst{n}ýtr mann-gi \alst{n}ás.\eva

\bvb A halt man rides a horse; a handless drives a herd; \\
a deaf fights and avails. \\
Blind is better than be burnt; \\
no man has use for a corpse.\evb
\evg


\bvg
\bva \edtrans{\alst{S}onr es bętri}{A son is better}{\Bfootnote{i.e. it is better for a man to have a son and heir than not, even if the father should die some time before he is born. The son can further his father’s lineage and memory (as exemplified by the raising of a “beat-stone”), and as the poet says, it is rare for a non-relative to do so.}}, \hld\ þótt sé \alst{s}íð of alinn &
\ind ęptir \alst{g}inginn \alst{g}uma; &
sjaldan \edtrans{\alst{b}autar-stęinar}{beat-stones}{\Bfootnote{Large memorial stones (menhirs), later and especially in Sweden decorated with Runic inscriptions.}} \hld\ standa \alst{b}rautu nę́r, &
\ind nema ręisi \alst{n}iðr at \alst{n}ið.\eva

\bvb A son is better, though he late be born \\
after a passed-on man; \\
seldom beat-stones near the highway stand, \\
save by kinsman for kinsman raised.\evb
\evg


\bvg
\bva \edtext{\alst{T}vęir ’ru ęins hęrjar, \hld\ \edtrans{\alst{t}unga ’s hǫfuðs bani}{the tongue is the head’s bane}{\Bfootnote{Formulaic or proverbial. Cf. the Old Swedish Heathen Law (my norm. following \textcite{Läffler1879}): \emph{Fallr þann orð havr givit—glǿpr orða vęrstr, tunga hovuð-bani—liggi i ú·gildum akri} ‘If he falls who has given the word (of insult)—wickedness is the worst of words, the tongue the head-bane-man—may he lie in an invalid (i.e. not properly enclosed) field.’}}; &
mér ’s í \alst{h}eðin \alst{h}vęrn \hld\ \alst{h}andar vę́ni.}{\lemma{Tvęir \dots\ vę́ni}\Bfootnote{The whole st. is undoubtedly a later insert as seen from the divergent meter and style.}}\eva

\bvb Two are of one host;\footnoteB{\emph{hęrjar} gen. sg. of \emph{hęrr} ‘host, army’ may alternatively be read as the nom. pl. meaning ‘harriers, raiders,’ present in \emph{ęinhęrjar} (\inx[G]{Ownharriers}). Thus ‘two are the destroyers of one (i.e. the person)’.} the tongue is the head’s bane;\footnoteB{The tongue and the head are part of the same body and need each other, yet the former often leads to the demise of the latter.} \\
in every cloak I expect a hand.\evb
\evg


\bvg
\bva\alst{N}ǫ́tt verðr fęginn, \hld\ sá’s \alst{n}esti trúir, &
\ind \alst{sk}ammar ’ru \alst{sk}ips ráar, &
\ind \alst{h}verf es \alst{h}aust-gríma; &
\alst{f}jǫlð of viðrir \hld\ á \alst{f}imm dǫgum, &
\ind en \alst{m}ęir á \alst{m}ánaði.\eva

\bvb At night rejoices he who trusts in his provisions; \\
short are the ship’s sailyards;\footnoteB{TODO: Write about the varying interpretations (Finnur, Cleasby, Skp) of this line.} \\
ever-shifting is the autumn night. \\
The weather shifts much in \inx[C]{five days},\footnoteB{See note to st. 51 and Encyclopedia.} \\
but more in a month.\evb
\evg


\bvg
\bva\alst{V}ęit-a hinn, \hld\ es \alst{v}ę́tki \alst{v}ęit, &
\ind margr verðr \edtrans{af \alst{au}rum}{by treasures}{\Afootnote{emend. from \emph{†aflꜹðrom†} \Regius}} \alst{a}pi; &
maðr es \alst{au}ðigr, \hld\ annarr \alst{ó}-auðigr, &
\ind skyli-t þann \alst{v}ítka \alst{v}áar.\eva

\bvb The one knows not, who nothing knows: \\
many a man becomes by treasures an \inx[C]{ape}. \\
A man is wealthy, another not wealthy; \\
one oughtn’t to curse him for his woe.\evb
\evg


\bvg
\bva\alst{D}ęyr \edtext{fé}{\lemma{fé \dots\ frę́ndr ‘Fee \dots\ kinsmen’}\Bfootnote{The import of this merism may be less clear to the modern reader. In the Germanic Iron Age farming society a man’s wealth was reckoned by how many heads of cattle (and the Norman loan-word \emph{cattle} is itself the same word as \emph{capital}) he owned (cf. st. 70 above, where “a cow” is used to express “an opportunity”), and his social power by the number of able male relatives ready to side with him in conflict (cf. st. 72 above and TODO: reference?). The meaning is thus: all your power will pass away, and so too must you, but if you leave a good reputation behind it can live on. For Indo-European poetic analogues, see \textcite[99\psqq]{West2007}.}}, \hld\ \alst{d}ęyja frę́ndr, &
\ind dęyr \alst{s}jalfr hit \alst{s}ama; &
en \alst{o}rðs-tírr \hld\ dęyr \alst{a}ldri-gi &
\ind hvęim’s sér \alst{g}óðan \alst{g}etr.\eva

\bvb \inx[C]{fee}[Fee] dies, kinsmen die, \\
oneself dies likewise; \\
but a word-glory never dies, \\
for whomever gets himself a good one.\evb
\evg


\bvg
\bva\alst{D}ęyr fé, \hld\ \alst{d}ęyja frę́ndr, &
\ind dęyr \alst{s}jalfr hit \alst{s}ama; &
\alst{e}k vęit \alst{ęi}nn \hld\ at \alst{a}ldri-gi dęyr: &
\ind \alst{d}ómr of \alst{d}auðan hvęrn.\eva

\bvb Fee dies, kinsmen die, \\
oneself dies likewise. \\
I know one that never dies: \\
the \inx[C]{Doom} o’er each man dead.\evb
\evg

\sectionline

It is likely that the original \emph{Gęsta-þáttr} ended here. The three following stanzas, especially the third, are poorly placed and seem like later inserts.

\sectionline

\bvg
\bva\alst{F}ullar grindr \hld\ sá’k fyr \alst{F}itjungs sonum, &
\ind nú bera þęir \alst{v}ánar \alst{v}ǫl; &
svá es \alst{au}ðr \hld\ sęm \alst{au}ga-bragð, &
\ind hann es \alst{v}altastr \alst{v}ina.\eva

\bvb Full pens I saw for the sons of Fitting; \\
now they carry the staff of hope.\footnoteB{A beggar’s staff.} \\
So is wealth like the twinkling of an eye: \\
it is the ficklest of friends.\evb
\evg


\bvg
\bva\alst{Ó}-snotr maðr \hld\ es \alst{ęi}gnask getr &
\ind \alst{f}é eða \alst{f}ljóðs mun-úð; &
\alst{m}etnaðr hǫ́num þróask, \hld\ en \alst{m}an-vit aldri-gi; &
\ind framm gęngr hann \alst{d}rjúgt í \alst{d}ul.\eva

\bvb The unclever man who comes to own \\
fee or a girl’s grace: \\
his conceit flourishes, but his manwit never; \\
he goes forth far into delusion.\evb
\evg


\bvg
\bva Þat ’s þá \alst{r}ęynt, \hld\ es þú at \edtext{\alst{r}únum spyrr, \hld\ hinum \alst{r}ęgin-kunnum}{\lemma{rúnum \dots\ ręgin-kunnum ‘runes \dots\ born of the Reins’}\Bfootnote{This expression also appears on the C4th–6th Noleby stone (in the acc. sg. \emph{rúnó ragina-kundó} ‘a rune born of the Reins’), which proves that the Eddic rune-magic is (at least in part) founded in oral tradition going back to the Heathen age. See also Encyclopedia \inx[C]{rune}.}}, &
\ind \edtext{þęim’s \alst{g}ørðu \alst{g}inn-ręgin &
\ind ok \alst{f}áði \alst{F}imbul-þulr;}{\lemma{þęim’s \dots\ Fimbul-þulr ‘those which \dots\ Fimble-Thyle’}\Bfootnote{Formulaic. Cf. st. 142 where these two lines occur almost identically, but in reverse order.}} &
\ind (\alst{þ}á hęfr hann batst, ef hann \alst{þ}ęgir.)\eva

\bvb That is then proven, which from the runes thou learnest, [from] the ones born of the Reins, \\
{[from]} those which the \inx[G]{yin-Reins} made, \\
and the Fimble-Thyle \name{= Weden} painted. \\
(Then he has it best, if he shuts up.)\footnoteB{This stanza, which deals with runic magic, and shares expressions with sts. in the Rune-Tally section (beginning with st. 138 below), hardly fits in its current placing. The last line with its shift in person is likely to be a later insert.}\evb
\evg

\sectionline

\section{Stanzas of practical advice, mostly in \Fornyrdislag.}

\bvg
\bva At \alst{k}veldi skal dag lęyfa, \hld\ \alst{k}onu es bręnnd es, &
\alst{m}ę́ki es ręyndr es, \hld\ \alst{m}ęy es gefin es, &
\alst{í}s es \alst{y}fir kømr, \hld\ \alst{ǫ}l es drukkit es.\eva

\bvb At evening shall one praise day, a woman when she is burned, \\
a sword when it is tried, a maiden when she is given,\footnoteB{i.e. in marriage.} \\
ice when one crosses over, ale when it is drunk.\evb
\evg


\bvg
\bva Í \alst{v}indi skal \alst{v}ið hǫggva, \hld\ \alst{v}eðri á sę́ róa, &
\alst{m}yrkri við \alst{m}an spjalla, \hld\ \alst{m}ǫrg eru dags augu, &
á \alst{sk}ip skal \alst{sk}riðar orka, \hld\ en á \alst{sk}jǫld til hlífar, &
\alst{m}ę́ki til hǫggs, \hld\ en \alst{m}ęy til kossa.\eva

\bvb In wind shall one cut wood, in weather row at sea, \\
in darkness speak with a maiden—many are the eyes of day. \\
A ship shall one have for speed, and a shield for protection; \\
a sword for striking, and a maiden for kisses.\evb
\evg


\bvg
\bva Við \alst{ę}ld skal \alst{ǫ}l drekka, \hld\ en á \alst{í}si skríða, &
\alst{m}agran \alst{m}ar kaupa, \hld\ en \alst{m}ę́ki saurgan, &
\alst{h}ęima \alst{h}ęst fęita, \hld\ en \alst{h}und á búi.\eva

\bvb By fire shall one drink ale, and skate on ice; \\
buy a meager stallion, and a rusty sword; \\
at home fatten the horse, and the hound in the dwelling.\evb
\evg


\bvg
\bva\alst{M}ęyjar orðum \hld\ skyli \alst{m}anngi trúa, &
\ind né því’s \alst{k}veðr \alst{k}ona; &
\edtext{\edtext{því-at}{\Afootnote{om. \FostrbroedhraSaga}} á \alst{h}verfanda \alst{h}véli \hld\ \edtext{vǫ́ru}{\Afootnote{\emph{er} \FostrbroedhraSaga}} þęim \edtrans{\alst{h}jǫrtu skǫpuð}{hearts shaped}{\Afootnote{\emph{hjarta skapat} ‘heart shaped’ \FostrbroedhraSaga}}, &
\ind \edtext{\alst{b}rigð}{\Afootnote{ok brigð \FostrbroedhraSaga}} í \alst{b}rjóst of \edtext{lagit}{\Afootnote{\emph{laginn} \FostrbroedhraSaga}}.}{\lemma{þvít \dots\ lagið}\Bfootnote{Quoted in slightly divergent form in \FostrbroedhraSaga\ (Thott 1768 4°\textsuperscript{x}, fol. 210r) introduced with the words: \emph{Kom honum þá í hug kviðlingr sá, er kveðinn hafði verit um lausungar-konur:} ‘And then he remembered the ditty which had been composed about loose women:’}}\eva

\bvb A maiden’s words should no man trust, \\
nor that which a woman speaks. \\
For on a spinning wheel were their hearts shaped; \\
fickleness in their breasts was laid.\evb
\evg

\sectionline

\bvg
\bva\alst{B}restanda \alst{b}oga, \hld\ \alst{b}rinnanda loga, &
\alst{g}ínanda ulfi, \hld\ \alst{g}alandi krǫ́ku, &
\alst{r}ýtanda svíni, \hld\ \alst{r}ót-lausum viði, &
\alst{v}axanda \alst{v}ági, \hld\ \alst{v}ellanda katli,\eva

\bvb In the bursting bow, in the burning flame, \\
in the yawning wolf, in the crowing crow, \\
in the roaring swine, in the rootless tree, \\
in the waxing wave, in the swelling kettle,\evb
\evg


\bvg
\bva\alst{f}ljúganda \alst{f}lęini, \hld\ \alst{f}allandi bǫ́ru, &
\alst{í}si \alst{ęi}n-nę́ttum, \hld\ \alst{o}rmi hring-lęgnum, &
\alst{b}rúðar \alst{b}ęð-mǫ́lum \hld\ eða \alst{b}rotnu sverði, &
\alst{b}jarnar lęiki \hld\ eða \alst{b}arni konungs, &
\alst{s}júkum kalfi, \hld\ \alst{s}jalf-ráða þrę́li, &
\alst{v}ǫlu \alst{v}il-mę́li, \hld\ \alst{v}al ný-fęldum.\eva

\bvb in the flying spear, in the falling billow, \\
in one-night old ice, in the coiled-up serpent, \\
in the bed-speeches of a bride or in the broken sword, \\
in the play of a bear or in the child of a king, \\
in the sick calf, in the self-ruling thrall, \\
in the pleasing speech of a wallow, in newly felled corpses,\evb
\evg

\sectionline

In \Regius\ the following two sts. come in the opposite order, but it is clear from its \Fornyrdislag\ meter and the dative case of the words that 88 should follow 86. On the other hand st. 87, with its \Ljodahattr\ meter and self-enclosed form seems a separate composition, and was probably inserted after 86 due to its first line, which is also in the dative.

\sectionline

\bvg
\bva[88]\alst{b}róður-\alst{b}ana sínum \hld\ þótt á \alst{b}rautu mǿti, &
\alst{h}úsi \alst{h}alf-brunnu, \hld\ \alst{h}ęsti al-skjótum, &
þá ’s \alst{jó}r \alst{ó}-nýtr, \hld\ ef \alst{ęi}nn fótr brotnar; &
verðr-it maðr svá \alst{t}ryggr \hld\ at þessu \alst{t}rúi ǫllu!\eva

\bvb in his brother’s bane-man—though on the highway they meet— \\
in the half-burned house, in the all-fleet horse: \\
then is the steed useless, if one foot breaks.— \\
There will be no man so trusting, that he trust in all this!\evb
\evg\stepcounter{stanza}


\bvg
\bva[87]\alst{A}kri \alst{á}r-sǫ́num \hld\ trúi \alst{ę}ngi maðr, &
\ind né til \alst{s}nimma \alst{s}yni; &
\alst{v}eðr rę́ðr akri, \hld\ en \alst{v}it syni; &
\ind \alst{h}ę́tt es þęira \alst{h}várt.\eva

\bvb In an early sown field ought no man to trust, \\
nor too soon in a son. \\
The weather rules the field, and the wits the son; \\
there is risk to them both.\evb
\evg\stepcounter{stanza}


\bvg
\bva Svá ’s \alst{f}riðr kvinna \hld\ þęira’s \alst{f}látt hyggja, &
sęm \alst{a}ki \alst{jó} ó·bryddum \hld\ á \alst{í}si hǫ́lum &
\alst{t}ęitum, \alst{t}vé-vetrum \hld\ ok sé \alst{t}amr illa, &
eða í \alst{b}yr óðum \hld\ \alst{b}ęiti stjórn-lausu, &
eða skyli \alst{h}altr \alst{h}ęnda \hld\ \alst{h}ręin í þá-fjalli.\eva

\bvb So is the love of women—those who falsely think— \\
like one rode an unshod horse on slippery ice: \\
a merry one, two winters old, and badly tamed— \\
or in mad wind tacked a rudderless [ship], \\
or [as if] a halt man should catch a reindeer on a thawing mountain.\evb
\evg

\sectionline

\section{Weden’s failed seduction of Billing’s daughter.}

The following sts. are united by their meter, \Ljodahattr\ (unlike most of the preceding sts., see introduction to them above) and by their logical progression, beginning with general maxims about love and relations between the sexes, before moving into the narrative about Billing’s daughter. The narrator is securely identified as Weden in st. 97.

\sectionline

\bvg
\bva\alst{B}ęrt nú mę́li’k, \hld\ því-at \edtrans{\alst{b}ę́ði}{both}{\Bfootnote{i.e. “both sides, both sexes”. The poet, a man, declares that he is not setting out to unfairly attack women; he is also aware of the faults of his own sex.}} vęit’k, &
\ind brigðr es \alst{k}arla hugr \alst{k}onum, &
\edtext{þá \alst{f}ęgrst mę́lum, \hld\ es \alst{f}lást hyggjum}{\lemma{fęgrst mę́lum \dots\ flást hyggjum ‘most fairly speak \dots\ most falsely we think’}\Bfootnote{Formulaic. Cf. st. 45.}}; &
\ind þat tę́lir \alst{h}orska \alst{h}ugi.\eva

\bvb Plainly I now speak, for I know both: \\
fickle is men’s thought towards women. \\
We then most fairly speak, when most falsely we think; \\
that entices sharp minds.\evb
\evg


\bvg
\bva \edtrans{\alst{F}agrt skal mę́la}{Fairly shall speak}{\Bfootnote{Formulaic. Cf. st. 45.}} \hld\ ok \alst{f}é bjóða, &
\ind sá’s vill \alst{f}ljóðs ǫ́st \alst{f}áa, &
\alst{l}íki \alst{l}ęyfa \hld\ hins \alst{l}jósa mans, &
\ind sá \alst{f}ę́r, es \alst{f}ríar.\eva

\bvb Fairly shall speak, and offer \inx[C]{fee}, \\
he who will earn a girl’s love; \\
{[he shall]} praise the body of the light maiden; \\
he gets, who woos.\footnoteB{i.e., ‘he who woos her gets her’.}\evb
\evg


\bvg
\bva\alst{Á}star firna \hld\ skyli \alst{ę}ngi maðr &
\ind \alst{a}nnan \alst{a}ldri-gi; &
opt fáa á \alst{h}orskan, \hld\ es á \alst{h}ęimskan né fáa, &
\ind \alst{l}ost-fagrir \alst{l}itir.\eva

\bvb For [his] love should no man \\
ever blame another; \\
oft they seize the sharp one, when they seize not the foolish one, \\
lust-fair looks.\footnoteB{Looks so fair that they cause great lust.}\evb
\evg


\bvg
\bva\alst{Ęy}-vitar firna, \hld\ es maðr \alst{a}nnan skal, &
\ind þess es of margan \alst{g}ęngr \alst{g}uma; &
\alst{h}ęimska ór \alst{h}orskum \hld\ gęrir \alst{h}ǫlða sonu &
\ind sá hinn \alst{m}átki \alst{m}unr.\eva

\bvb For nothing shall man ever blame another, \\
which happens to many a man; \\
fools out of sharp ones makes the sons of men \\
that mighty liking \ken{love}.\evb
\evg


\bvg
\bva\alst{H}ugr ęinn þat vęit, \hld\ es býr \alst{h}jarta nę́r, &
\ind ęinn es hann \alst{s}ér of \alst{s}efa; &
øng es \alst{s}ótt verri \hld\ hvęim \alst{s}notrum manni &
\ind an sér \alst{ø}ngu at \alst{u}na.\eva

\bvb The mind alone knows what lives close to the heart, \\
it is alone with its thoughts. \\
No sickness is worse for any clever man \\
than to with nothing be content.\evb
\evg


\bvg
\bva Þat þá \alst{r}ęynda’k, \hld\ es í \alst{r}ęyri sat’k, &
\ind ok vę́tta’k \alst{m}íns \alst{m}unar, &
\alst{h}old ok \alst{h}jarta \hld\ vas mér hin \alst{h}orska mę́r, &
\ind þęygi hana at \alst{h}ęldr \alst{h}ęf’k.\eva

\bvb That I then discovered, as I sat in the reed, \\
and awaited my pleasure. \\
My flesh and heart was that sharp maiden; \\
I hold her none the more.\evb
\evg


\bvg
\bva\alst{B}illings \edtrans{męy}{maiden}{\Bfootnote{i.e. ‘unmarried (virgin) daughter’.}} \hld\ ek fann \alst{b}ęðjum á &
\ind \alst{s}ól-hvíta \alst{s}ofa; &
\alst{ja}rls \alst{y}nði \hld\ þótti mér \alst{ę}kki vesa &
\ind nema við þat \alst{l}ík at \alst{l}ifa.\eva

\bvb Billing’s maiden I found on the beds, \\
sun-white, sleeping. \\
An earl’s pleasure seemed me naught to be, \\
except living alongside that body.\evb
\evg


\bvg
\bva „\alst{Au}k nę́r \alst{a}ptni \hld\ skalt \alst{Ó}ðinn koma, &
\ind ef vilt þér \alst{m}ę́la \alst{m}an, &
\alst{a}lt eru \alst{ó}-skǫp, \hld\ nema \alst{ęi}n vitim &
\ind \alst{s}likan lǫst \alst{s}aman.“\eva

\bvb {[Billing’s daughter:]} \\
“And by evening shalt thou, Weden, come, \\
if thou wilt get for thee the girl [me]; \\
all is misshapen, if we may not know, \\
alone, such a vice together.”\evb
\evg


\bvg
\bva\alst{A}ptr ek hvarf \hld\ ok \alst{u}nna þóttumk &
\ind \edtrans{\alst{v}ísum \alst{v}ilja frá}{away from my wise will}{\Bfootnote{i.e., the wise choice would have been to walk away, rather than return.}}; &
\alst{h}itt ek \alst{h}ugða, \hld\ at \alst{h}afa mynda’k &
\ind \alst{g}ęð hęnnar allt ok \alst{g}aman.\eva

\bvb Back I turned—and thought myself in love— \\
away from my wise will; \\
this I thought, that I would have \\
her senses all, and pleasure.\evb
\evg


\bvg
\bva Svá kom’k \alst{n}ę́st, \hld\ at hin \edtrans{\alst{n}ýta}{useful}{\Bfootnote{Sarcastic. Billing’s daughter had apparently summoned a lynch mob.}} vas &
\ind \alst{v}íg-drótt ǫll of \alst{v}akin; &
með \alst{b}rinnǫndum ljósum \hld\ ok \edtrans{\alst{b}ornum viði}{carried sticks}{\Bfootnote{lit. ‘carried wood’; the mob was armed.}}, &
\ind svá vas mér \edtrans{\alst{v}íl-stígr}{sad path}{\Bfootnote{Ambiguous, either referring to the beating he would have received at the hands of the mob, or to his walk of shame away from the hall. The latter is perhaps more likely.}} of \alst{v}itaðr.\eva

\bvb So I came next, as was the useful \\
war-troop all awake; \\
with burning lights and with carried sticks; \\
so was for me a sad path marked out.\evb
\evg


\bvg
\bva \edtrans{\alst{Au}k nę́r morni}{And by morning}{\Bfootnote{Mirroring the beginning of st. 97 above.}}, \hld\ es vas’k \alst{ę}nn of kominn, &
\ind þá vas \alst{s}al-drótt of \alst{s}ofin; &
\edtrans{\alst{g}ręy ęitt}{a lone bitch}{\Bfootnote{The insult is easily understood: Weden is being asked to make love to the dog, “this is all you get!”}} þá fann’k \hld\ hinnar \edtrans{\alst{g}óðu}{good}{\Bfootnote{Possibly not sarcastic, but rather referring to her chastity.}} konu &
\ind \alst{b}undit \alst{b}ęðjum á.\eva

\bvb And by morning when I had come again, \\
then was the hall-troop asleep. \\
A lone bitch I then found, by the good woman \\
bound on the bed.\evb
\evg


\bvg
\bva Mǫrg es \alst{g}óð mę́r, \hld\ ef \alst{g}ǫrva kannar, &
\ind \alst{h}ug-brigð við \alst{h}ali; &
þá þat \alst{r}ęynda’k, \hld\ es hit \alst{r}áð-spaka &
\ind tęygða’k á \alst{f}lę́rðir \alst{f}ljóð; &
\alst{h}ǫ́ðungar \alst{h}vęrrar \hld\ lęitaði mér hit \alst{h}orska man &
\ind ok hafða’k þess \alst{v}ę́t-ki \alst{v}ífs.\eva

\bvb Many a good maiden—if one comes to know her well— \\
is heart-fickle towards men; \\
then I found that out, as into sins I lured \\
the counsel-clever maid. \\
All sorts of disgraces this sharp girl sought out for me, \\
and I had naught of that woman.\evb
\evg

\sectionline

\section{Weden’s obtaining of the Mead of Poetry}

The intricate myth of how Weden came to own the Mead of Poetry is told more fully in \Skaldskaparmal\ 5–6. That narrative goes as follows, with minor details left out:
After the war between the Eese and Wanes, the two tribes of gods reconcile through spitting into a vat. Not wanting to discard this token of their truce, they instead create a man out of the spit, calling him \inx[P]{Quasher}; he is so wise that he can answer any question posed to him, and so travels around the world in order to share his wisdom with humans.
Quasher eventually comes to the dwelling of two dwarfs, Fealer and Galer. They kill him and drain his blood into three vessels: two vats named Soon and Bothem, and a kettle named \inx[P]{Woderearer}. Through mixing the blood with honey they make a mead, with the power to turn anyone who drinks from it “a scold or man of learning (\emph{skald eða frǿða-maðr})”. The dwarfs then lie to the Eese about the murder, telling them that Quasher drowned in his own wisdom.
Some time later, the dwarfs murder an ettin named \inx[P]{Gilling} and his wife. Gilling’s son, \inx[P]{Sutting}, learns of this and prepares to drown the dwarfs. In exchange for their lives and as recompense for his father’s slaying, the dwarfs offer Sutting the “dear mead” (\emph{mjǫðinn dýra}; cf. here sts. 104 and 138). Sutting accepts the ransom and takes the mead home with him. He makes his daughter \inx[P]{Guthlathe} guard it.
Some time later, Weden is out journeying, and finds nine thralls mowing hay. He sharpens their scythes with a special whetstone, and the mowing improves greatly. He then throws it in the air and the thralls shortly kill each other over it. By evening Weden comes to the owner of the thralls, Bigh, Sutting’s brother. Bigh laments the death of his workmen, and so Weden, who calls himself \inx[P]{Baleworker}, offers to do the work of the thralls over the summer, in exchange for one drink of Sutting’s mead. Bigh tells him that Sutting alone owns the mead, but that he will accompany Baleworker to Sutting to ask for the drink.
The two arrive at Sutting, who as expected refuses to give any part of the mead away. Baleworker then tells Bigh that he will get to it anyway; he takes out the drill \inx[P]{Rate}, and tells Bigh to drill through the mountain, into the room where the mead is stored. Bigh first attempts to trick him by only drilling halfway, but eventually creates a narrow passage. Baleworker turns himself into a snake and crawls through it; as he does, Bigh tries to strike him the drill, but misses.
After coming through, Baleworker sees Guthlathe watching over the mead. He goes on to sleep with her for three nights, after which she promises him three sips of the mead. With each sip he swallows the contents of one of the three vessels, so that all of the mead ends up in his belly.
Having taken the mead, he dons his eagle-hame and flies away from the mountain. Sutting sees him, takes his own eagle-hame, and gives chase. The Eese see Weden in flight, and set out several large vat on the ground, into which Weden, still flying, spits out the mead. At this point Sutting has almost caught up with him, and so Weden “sends back” (\emph{sęnda aptr}, usually interpreted being sent out from the anus) some of the mead, presumably into his face. This portion becomes the lot of foolish poets (\emph{skald-fífla hlutr}), while the rest of the mead is given to the Eese and to skilled poets (\emph{þęim mǫnnum, er yrkja kunnu} ‘those men who can compose [poetry]’).

The core of this many-twisted myth is old. A close parallel is found in \Rigveda\ hymns 4.26–27. In these two hymns the \emph{soma} plant (who in the Vedic mythology is not just the plant and its resulting drink, but also a god, perhaps somewhat like Quasher) is first held within “a hundred iron forts” (4.27.1c: \emph{śatám púraḥ ā́yasīḥ}) by the archer \emph{Kr̥şānu}, before being stolen by a sweeping falcon. The falcon brings \emph{Soma} to \emph{Manu}, the ancestor of the Aryans and first sacrificer.

The resemblance to the last part of the \Skaldskaparmal\ account should be obvious, but, notably, the detail of the falcon is not found in any of the sts. below. This shows that the narrative of \Skaldskaparmal\ cannot be exclusively based on the sts. here below, but instead also relies on other, now-lost sources. This is also supported by the present sts. leaving out the narratives about Quasher, the two dwarfs, and Baye, along with some subtler narrative differences.

The order of the present sts. follows that of \Regius, their main witness manuscript. The strand begins with some social advice (102), after which the narrative follows (103–109). It is narrated in the first person by Weden himself. The sts. do not tell the myth in chronological order and leave much up to the listener; they are surely composed for an audience that already knows the story. The following narrative details are given:

\begin{enumerate}
	\setcounter{enumi}{103}
	\item Weden visits Sutting’s home, but does not receive a good reception.
	\item Guthlate falls in love with Weden, and gives him a drink of the Mead.
	\item Weden has to bore through the mountains with the drill Rate.
	\item Weden has “bought [the Mead] well”; possibly a euphemistic reference to sleeping with Guthlathe for it.
	\item Guthlathe indeed does sleep with Weden, though not expressely in exchange for the Mead.
	\item The following day (\emph{hins hindra dags}, see note to this word in the edited text below), a group of Rime-Thurses come to Weden’s hall, to ask him whether a Baleworker is among the Gods, or if he has been slain by Sutting.
	\item Switching to the third person (which may indicate that this is his answer to the Rime-Thurses), Weden says that he “thinks” that Weden has sworn an oath, but that his words cannot be trusted. After the “simble” (i.e. drinking feast, banquet; probably referring to the drink of the Mead), Weden betrayed Sutting and made Guthlathe weep.
\end{enumerate}

The underlying narrative seems to generally agree with that of \Skaldskaparmal, but unlike its more transactional affair, we here find a stronger emphasis on Weden’s cruel betrayal of Guthlathe. A notable detail not found in \Skaldskaparmal\ is Weden’s oath in st. 109. The content of the oath was most likely that Weden would marry Guthlathe, something supported by the language used (see note to st. 108: \emph{hins hindra dags}). The recipient of the oath, which Weden clearly broke, was either Sutting or Guthlathe. That Weden swore it to Sutting, and thus asked him for Guthlathe’s hand in marriage, may be suggested by the description of Sutting as \emph{svikvinn} ‘betrayed’ in st. 109. This view, however, has an internal narrative problem: in st. 103 Weden describes his interaction with Sutting as poor, and in st. 105 Weden is said to have had to bore through the mountains, but this may just have been to reach Sutting, rather than Guthlathe as in \Skaldskaparmal.
The recipient of the oath being Guthlathe would agree better with the \Skaldskaparmal\ narrative, and Sutting’s betrayer would instead be her.

\sectionline

\bvg
\bva Hęima \alst{g}laðr \alst{g}umi \hld\ ok við \alst{g}ęsti ręifr, &
\ind \alst{s}viðr skal of \alst{s}ik vesa; &
\alst{m}innigr ok \alst{m}ǫ́lugr, \hld\ ef vill \alst{m}arg-fróðr vesa; &
\ind opt skal \alst{g}óðs \alst{g}eta; &
\alst{f}imbul-\alst{f}ambi hęitir, \hld\ sá’s \alst{f}átt kann sęgja; &
\ind þat es \alst{ó}-snotrs \alst{a}ðal.\eva

\bvb At home shall man be glad and giving with the guest, \\
wise about himself; \\
{[he shall be]} of good memory and speech, if he wishes to be many-learned; \\
oft shall he speak of good. \\
A fimble-fool is he called who little can say; \\
that is an unclever man’s nature.\evb
\evg


\bvg
\bva Hinn \alst{a}ldna \alst{jǫ}tun sótta’k, \hld\ nú em’k \alst{a}ptr of kominn; &
\ind fátt gat’k \alst{þ}ęgjandi \alst{þ}ar; &
\alst{m}ǫrgum orðum \hld\ \alst{m}ę́lta’k í minn frama &
\ind í \alst{S}uttungs \alst{s}ǫlum.\eva

\bvb The old ettin \name{= Sutting} I sought, now am I come back; \\
I got little audience there. \\
Many words I spoke to my furtherance, \\
in the halls of Sutting.\evb
\evg


\bvg
\bva\alst{G}unn-lǫð mér of \alst{g}af \hld\ \alst{g}ullnum stóli á &
\ind \alst{d}rykk hins \alst{d}ýra mjaðar; &
\alst{i}ll \alst{i}ð-gjǫld \hld\ lét’k hana \alst{ę}ptir hafa &
\ind síns hins \alst{h}ęila \alst{h}ugar, &
\ind síns hins \alst{s}vára \alst{s}efa.\eva

\bvb \inx[P]{Guthlathe} did give me, on the golden throne, \\
a drink of the dear mead; \\
evil recompense I let her have afterwards, \\
for her whole heart, \\
for her severe affection.\evb
\evg


\bvg
\bva\alst{R}ata munn \hld\ létumk \alst{r}úms of fáa &
\ind ok of \alst{g}rjót \alst{g}naga; &
\alst{y}fir ok \alst{u}ndir \hld\ stóðumk \alst{jǫ}tna vegir, &
\ind svá \alst{h}ę́tta’k \alst{h}ǫfði til.\eva

\bvb Rate’s mouth I made to bring me room, \\
and gnaw away at the rocks. \\
Over and under me stood the roads of the ettins \ken{mountains}; \\
so I risked my head.\evb
\evg


\bvg
\bva \edtext{\alst{V}ęl kęypts hlutar \hld\ hęf’k \alst{v}ęl notit; &
\ind \alst{f}ás es \alst{f}róðum vant; &
því-at \edtrans{\alst{Ó}ð-rǿrir}{Woderearer}{\Bfootnote{One of the vessels in with the Mead of Poetry was held (see introduction to the present section above), here standing in for all the Mead.}} \hld\ es nú \alst{u}pp kominn &
\ind á \alst{a}lda vés \edtrans{\alst{ja}ðar}{rim}{\Bfootnote{metr. emend.; \emph{jarðar} \Regius\ has a long root-syllable, and does not fit grammatically.}}.}{\lemma{Vęl \dots\ jaðar}\Bfootnote{Taken on its own this st. would be somewhat difficult, but in context the import is clear: Weden says that He has made good use of the Mead of Poetry by bringing it to earth, making poetry (and surely likewise other intellectual disciplines) available to men.}}\eva

\bvb The well bought thing \ken*{Mead of Poetry} have I used well— \\
little is lacking for the learned, \\
for Woderearer is now come up \\
over the rim of the \inx[C]{wigh} of men \ken*{= Middenyard}.\evb
\evg


\bvg
\bva\alst{I}fi ’s mér \alst{á}, \hld\ at vę́ra’k \alst{ę}nn kominn &
\ind \alst{jǫ}tna gǫrðum \alst{ó}r, &
ef \alst{G}unn-laðar né nyta’k, \hld\ hinnar \alst{g}óðu konu, &
\ind es lǫgðumk \alst{a}rm \alst{y}fir.\eva

\bvb There is doubt in me, that I would yet be come \\
out of the yards of the Ettins, \\
if I had not used Guthlathe, that good woman \\
whom I laid my arm over.\evb
\evg


\bvg
\bva \edtrans{\alst{H}ins \alst{h}indra dags}{The following day}{\Bfootnote{This is the only occurrence of the comparative \emph{hindra} ‘following, next’ in the Norse (i.e. ‘belonging to Norway and its colonies’) literature. The superlative \emph{hindstr} ‘last, final’ does occur more often (e.g. \emph{indsta sinni} ‘the last time’, with loss of the \emph{h-}; see \CV: \emph{hindri}), and the possible derivative \emph{hindar-dags} ‘day after tomorrow, two days after’ is found twice, both times in the \Gulatingslog, chh. 37 and 266. If we look at the broader Scandinavian sphere, we find in the Swedish provicial laws an exact equivalent of the present phrase, namely OSwe. \emph{hindra-dagher}, a law-term referring specifically to the ‘day after the (consumation of the) wedding’, used both on its own and in the expression \emph{hindra-dags gięf} ‘morning gift’. If this is indeed the sense in the present stanza, two interpretations are possible. It either refers sarcastically to Weden’s sleeping with Guthlathe (as would be done on the wedding night), or it means that Weden married, or promised to marry, Guthlathe. The latter interpretation has some support in st. 109, see notes there.}} \hld\ gingu \alst{h}rím-þursar &
\alst{H}áva ráðs at fregna, \hld\ \alst{H}áva \alst{h}ǫllu í, &
at \alst{B}ǫl-verki spurðu, \hld\ ef vę́ri með \alst{b}ǫndum kominn &
\ind eða hęfði hǫ́num \alst{S}uttungr of \alst{s}óit.\eva

\bvb The following day went the Rime-Thurses \\
to ask for the High One’s counsel, in the High One’s hall. \\
About Baleworker \name{= Weden} they asked, whether he were come among the bonds \ken{gods}, \\
or if Sutting had slain him.\evb
\evg


\bvg
\bva \edtext{Baug-ęið \alst{Ó}ðinn \hld\ hygg at \alst{u}nnit hafi, &
\ind hvat skal hans \alst{t}ryggðum \alst{t}rúa? &
\alst{S}uttung \alst{s}vikvinn \hld\ hann lét \alst{s}umbli frá &
\ind ok \alst{g}rǿtta \alst{G}unn-lǫðu}{\lemma{Baug-ęið \dots\ Gunn-lǫðu ‘A bigh-oath \dots\ brought to tears™}\Bfootnote{The exact narrative referred to in the stanza is hard to pin down, but I find the following most likely: Weden swore an oath on a bigh, its contents being that he would marry Guthlathe. Sutting then hosted a simble (banquet, drinking feast) for the new couple (cf. \emph{hins hindra dags} in st. 108), and Weden slept with her, but after. \emph{svikvinn} ‘betrayed’ and \emph{grǿtta} ‘brought to tears’ are (respectively masc. and fem.) acc. sg. past participles of the transitive verbs \emph{svíkva} ‘to betray’ and \emph{grǿta} ‘to make weep, bring to tears’. I read \emph{lét} as meaning ‘left, abandoned, forsook’.}}.\eva

\bvb A \inx[C]{bigh-oath} I ween that Weden has sworn— \\
how shall one trust his truces? \\
Away from the \inx[C]{simble} he left Sutting, betrayed, \\
and Guthlathe, brought to tears.\evb
\evg

\sectionline

\section{The Speeches of Loddfathomer}

\emph{Loddfáfnismǫ́l}. Advice given to Loddfathomer. In \Regius\ stanza 110 begins with a large initial \emph{M} in the margin, smaller than those of individual named poems, but larger than the typical initials for sts.

\sectionline

\bvg
\bva Mál ’s at \alst{þ}ylja \hld\ \alst{þ}ular stóli á; &
\ind \alst{U}rðar brunni \alst{a}t &
\alst{s}á’k ok þagða’k, \hld\ \alst{s}á’k ok hugða’k, &
\ind hlýdda’k á \alst{m}anna \alst{m}ál; &
of \alst{r}únar hęyrða’k dǿma, \hld\ né umb \alst{r}ǫ́ðum þǫgðu &
\ind \alst{H}áva \alst{h}ǫllu at, &
\ind \alst{H}áva \alst{h}ǫllu í &
\ind hęyrða’k \alst{s}ęgja \alst{s}vá:\eva

\bvb ’Tis time to \inx[C]{thill}, upon the \inx[C]{thyle}’s chair. \\
At the well of Weird \\
I saw and I shut up: I saw and I thought: \\
I heeded the matters of men. \\
Of runes I heard them speak, nor did they shut up about counsels, \\
at the High One’s \name{= Weden’s} hall \ken*{= Walhall}, \\
in the High One’s hall, \\
I heard [them] say thus:\footnoteB{The speaker, describing himself as a thyle (\emph{þulr} ‘sage, chanter of memorized poetry’), says that he will relate what he has heard said in Walhall. Considering the location, it seems almost certain that the giver of this advice was its owner, \inx[P]{Weden}. The receiver of the advice, \inx[P]{Loddfathomer} (see Encyclopedia for etymologies), is otherwise unknown.}\evb
\evg


\bvg
\bva\alst{R}ǫ́ðumk þér Loddfáfnir, \hld\ at \alst{r}ǫ́ð nemir, &
\ind \alst{n}jóta munt ef \alst{n}emr, &
\ind þér munu \alst{g}óð ef \alst{g}etr: &
\alst{n}ǫ́tt þú rís-at, \hld\ nema á \alst{n}jósn séir, &
\ind eða \edtrans{lęitir þér \alst{i}nnan \alst{ú}t staðar}{thou must look for thy place, [going] out from within}{\Bfootnote{A difficult line to translate faithfully, owing to \emph{innan út} ‘[going] out from within’ and the euphemistic expression \emph{lęita sér staðar} ‘look for one’s place’ for ‘shit’, something which at the time was done outside. The meaning of the line is thus ‘or if you are leaving your house to relieve yourself’.}}.\eva

\bvb I counsel thee, O Loddfathomer, that thou learn the counsels; \\
thou wilt have use if thou learn [them], \\
they will be good for thee if thou get [them]: \\
At night thou rise not, unless thou be scouting, \\
or [if] thou must look for thy place, [going] out from within.\evb
\evg


\bvg
\bva\alst{R}ǫ́ðumk þér Loddfáfnir, \hld\ at \alst{r}ǫ́ð nemir, &
\ind \alst{n}jóta munt ef \alst{n}emr, &
\ind þér munu \alst{g}óð ef \alst{g}etr: &
\alst{f}jǫl-kunnigri konu \hld\ skal-at-tu í \alst{f}aðmi sofa, &
\ind svá’t hon \alst{l}yki þik \alst{l}iðum.\eva

\bvb I counsel thee, O Loddfathomer, that thou learn the counsels; \\
thou wilt have use if thou learn [them], \\
they will be good for thee if thou get [them]: \\
In the bosom of a \inx[C]{many-cunning} woman shalt thou never sleep, \\
lest she might lock you in [her?] limbs.\evb
\evg


\bvg
\bva Hón svá \alst{g}ęrir \hld\ at \alst{g}áir ęigi &
\ind \alst{þ}ings né \alst{þ}jóðans máls; &
\alst{m}at þú vill-at \hld\ né \alst{m}anns-kis gaman &
\ind fęrr þú \alst{s}orga-fullr at \alst{s}ofa.\eva

\bvb She makes it so that thou heed not \\
the \inx[C]{Thing}, nor the ruler’s speech: \\
thou wilt [then] not have food, nor any man’s pleasure; \\
thou goest full of sorrows to sleep.\evb
\evg\stepcounter{stanza}


\bvg
\bva\alst{R}ǫ́ðumk þér Loddfáfnir, \hld\ at \alst{r}ǫ́ð nemir, &
\ind \alst{n}jóta munt ef \alst{n}emr, &
\ind þér munu \alst{g}óð ef \alst{g}etr: &
\alst{a}nnars konu \hld\ tęyg þér \alst{a}ldri-gi &
\ind \alst{ęy}ra-rúnu \alst{a}t.\eva

\bvb I counsel thee, O Loddfathomer, that thou learn the counsels; \\
thou wilt have use if thou learn [them], \\
they will be good for thee if thou get [them]: \\
Never lure another man’s woman \\
into [becoming] thy ear-whisperer \ken{lover}.\evb
\evg


\bvg
\bva\alst{R}ǫ́ðumk þér Loddfáfnir, \hld\ en \alst{r}ǫ́ð nemir, &
\ind \alst{n}jóta munt ef \alst{n}emr, &
\ind þér munu \alst{g}óð ef \alst{g}etr: &
á \edtrans{\alst{f}jalli eða \alst{f}irði}{fell or firth}{\Bfootnote{i.e. ‘hiking through the mountains or travelling at sea’; a very Norse expression. This word pair is a formulaic merism, which occurs a few times in the Norwegian laws, but not elsewhere in poetry.}}, \hld\ ef þik \alst{f}ara tíðir, &
\ind fásk-tu at \alst{v}irði \alst{v}ęl.\eva

\bvb I counsel thee, O Loddfathomer—and thou oughtst to learn the counsels; \\
thou wilt have use if thou learn [them], \\
they will be good for thee if thou get [them]: \\
on the fell or firth—if thou desire to journey— \\
furnish thyself well with food.\evb
\evg


\bvg
\bva\alst{R}ǫ́ðumk þér Loddfáfnir, \hld\ en \alst{r}ǫ́ð nemir, &
\ind \alst{n}jóta munt ef \alst{n}emr, &
\ind þér munu \alst{g}óð ef \alst{g}etr: &
\alst{i}llan mann \hld\ lát \alst{a}ldri-gi &
\ind \edtext{\alst{ó}-hǫpp at þér \alst{v}ita}{\Bfootnote{Excluding some corrpution (but there seems not to be any) this line is probably one the few undisputed cases of \emph{v-} alliterating with a vowel.}}. &
því-at af \alst{i}llum manni \hld\ fę́r \alst{a}ldri-gi &
\ind \alst{g}jǫld hins \alst{g}óða hugar.\eva

\bvb I counsel thee, O Loddfathomer—and thou oughtst to learn the counsels; \\
thou wilt have use if thou learn [them], \\
they will be good for thee if thou get [them]: \\
An evil man let thou never \\
know of thy misfortunes, \\
for from an evil man gettest thou never \\
recompense for thy good heart.\evb
\evg


\bvg
\bva\alst{O}far-la bíta \hld\ sá’k \alst{ęi}num hal &
\ind \alst{o}rð \alst{i}llrar konu, &
\alst{f}lá-rǫ́ð tunga \hld\ varð hǫ́num at \alst{f}jǫr-lagi &
\ind ok þęygi of \alst{s}anna \alst{s}ǫk.\eva

\bvb Sorely I saw biting, on one man, \\
an evil woman’s words; \\
a false-counseling tongue brought his life to its end, \\
and in no way over a truthful charge.\footnoteB{Cf. \Lokasenna\ 31/1: \emph{flǫ́ ’s þér tunga} ‘false is thy tongue’. — The evil woman’s words bit the man \emph{ofarla}, contraction of \emph{ofar-liga} ‘\CV: high up, in the upper part’, presumably here meaning that the words were particularly grievous or insulting; they “got to him”. Whether he was murdered or committed suicide is not clear.}\evb
\evg


\bvg
\bva\alst{R}ǫ́ðumk þér Loddfáfnir, \hld\ en \alst{r}ǫ́ð nemir, &
\ind \alst{n}jóta munt ef \alst{n}emr, &
\ind þér munu \alst{g}óð ef \alst{g}etr: &
\alst{v}ęitst, ef \alst{v}in átt, \hld\ þann’s \alst{v}ęl trúir, &
\ind \alst{f}ar þú at \alst{f}inna opt; &
því-at \edtrans{\alst{h}rísi vęx \hld\ ok \alst{h}ǫ́u grasi}{with brushwood and with tall grass grows’}{\Bfootnote{Identical with \Grimnismal\ 17/1.}} &
\ind \alst{v}egr, es \alst{v}ę́t-ki trøðr,\eva

\bvb I counsel thee, O Loddfathomer—and thou oughtst to learn the counsels; \\
thou wilt have use if thou learn [them], \\
they will be good for thee if thou get [them]: \\
Know, if thou have a friend, one on which thou well trust, \\
journey to find him oft; \\
for with brushwood and tall grass grows \\
the way which no man treads.\evb
\evg


\bvg
\bva\alst{R}ǫ́ðumk þér Loddfáfnir, \hld\ en \alst{r}ǫ́ð nemir, &
\ind \alst{n}jóta munt ef \alst{n}emr, &
\ind þér munu \alst{g}óð ef \alst{g}etr: &
\alst{g}óðan mann \hld\ tęyg þér at \alst{g}aman-rúnum &
\ind ok nem \edtrans{\alst{l}íknar-galdr}{liking-galder}{\Bfootnote{i.e. ways of speaking which will make one liked or popular. For \emph{líkn} see sts. 8 (with note) and 123.}} meðan \alst{l}ifir.\eva

\bvb I counsel thee, O Loddfathomer—and thou oughtst to learn the counsels; \\
thou wilt have use if thou learn [them], \\
they will be good for thee if thou get [them]: \\
Lure a good man to thee through pleasure-runes,\footnoteB{Pleasurable conversation. Cf. st. 128.} \\
and learn liking-galder while thou livest.\evb
\evg


\bvg
\bva\alst{R}ǫ́ðumk þér Loddfáfnir, \hld\ en \alst{r}ǫ́ð nemir, &
\ind \alst{n}jóta munt ef \alst{n}emr, &
\ind þér munu \alst{g}óð ef \alst{g}etr: &
\alst{v}in þínum \hld\ \alst{v}es aldri-gi &
\ind \alst{f}yrri at \alst{f}laum-slitum. &
\alst{s}org etr hjarta, \hld\ ef þú \edtext{\alst{s}ęgja né náir &
\ind \alst{ęi}n-hvęrjum \alst{a}llan hug}{\lemma{sęgja \dots\ ęin-hvęrjum allan hug ‘tell anyone thy whole mind’}\Bfootnote{Cf. st. 124 which uses almost the same expression.}}.\eva

\bvb I counsel thee, O Loddfathomer—and thou oughtst to learn the counsels; \\
thou wilt have use if thou learn [them], \\
they will be good for thee if thou get [them]: \\gettest: \\
With thy friend be thou never the first \\
to tear apart the company. \\
Sorrow eats thy heart if thou cannot tell \\
anyone thy whole mind.\evb
\evg


\bvg
\bva\alst{R}ǫ́ðumk þér Loddfáfnir, \hld\ en \alst{r}ǫ́ð nemir, &
\ind \alst{n}jóta munt ef \alst{n}emr, &
\ind þér munu \alst{g}óð ef \alst{g}etr: &
\edtext{\alst{o}rðum skipta \hld\ skalt \alst{a}ldri-gi &
\ind við \edtrans{\alst{ó}-svinna \alst{a}pa}{unwise apes}{\Bfootnote{Formulaic. Cf. TODO.}}.}{\lemma{orðum \dots\ apa ‘Words \dots\ apes’}\Bfootnote{Cf. st. 125 which gives similar advice.}}\eva

\bvb I counsel thee, O Loddfathomer—and thou oughtst to learn the counsels; \\
thou wilt have use if thou learn [them], \\
they will be good for thee if thou get [them]: \\
Words shalt thou never exchange \\
with unwise apes.\evb
\evg


\bvg
\bva því-at af \alst{i}llum manni \hld\ munt \alst{a}ldri-gi &
\ind \alst{g}óðs laun of \alst{g}eta, &
en \alst{g}óðr maðr \hld\ mun þik \alst{g}ørva męga &
\ind \edtrans{\alst{l}íkn-fastan}{fast in liking}{\Bfootnote{The first element \emph{líkn} is somewhat difficult; see note to st. 8 and cf. st. 120. For the present cpd \textcite{LaFargeGlossary} give a tentative ‘assured of favour’, while \CV\ gives ‘fast in goodwill, beloved’.}} at \alst{l}ofi.\eva

\bvb For from an evil man wilt thou never \\
get a reward for thy goodness, \\
but a good man will know to make thee \\
fast in liking by [his] praise.\evb
\evg


\bvg
\bva\alst{S}ifjum ’s þá blandit \hld\ hvęrr es \edtext{\alst{s}ęgja rę́ðr &
\ind \alst{ęi}num \alst{a}llan hug}{\lemma{sęgja \dots\ \alst{ęi}num \alst{a}llan hug ‘tell one man his whole mind’}\Bfootnote{Cf. st. 121 which uses almost the same expression.}}; &
alt es \alst{b}ętra \hld\ an sé \alst{b}rigðum at vesa: &
es-a sá \alst{v}inr ǫðrum \hld\ es \alst{v}ilt ęitt sęgir.\eva

\bvb Kinship is then blended, when any man decides to tell \\
one man his whole mind. \\
Everything is better than to be with the fickle; \\
he is no friend to another who says only that which is wanted.\evb
\evg


\bvg
\bva\alst{R}ǫ́ðumk þér Loddfáfnir, \hld\ en \alst{r}ǫ́ð nemir, &
\ind \alst{n}jóta munt ef \alst{n}emr, &
\ind þér munu \alst{g}óð ef \alst{g}etr: &
\edtrans{þrimr orðum}{With three words}{\Bfootnote{i.e. ‘not even with three words’. If one understands \emph{orð} to mean ‘speech’, it may be interpreted as that if one says something (the first speech) to which another man responds insultingly (the second speech), one should not respond a third time and turn it into a fight.}} sęnna \hld\ skal-at-tu þér við verra mann; &
\ind opt hinn \alst{b}ętri \alst{b}ilar, &
\ind þá’s hinn \alst{v}erri \alst{v}egr.\eva

\bvb I counsel thee, O Loddfathomer—and thou oughtst to learn the counsels; \\
thou wilt have use if thou learn [them], \\
they will be good for thee if thou get [them]: \\
With three words shalt thou not flyte with a worse man; \\
oft the better man breaks \\
when the worse man strikes.\footnoteB{Cf. st. 122.}\evb
\evg


\bvg
\bva\alst{R}ǫ́ðumk þér Loddfáfnir, \hld\ en \alst{r}ǫ́ð nemir, &
\ind \alst{n}jóta munt ef \alst{n}emr, &
\ind þér munu \alst{g}óð ef \alst{g}etr: &
\alst{sk}ó-smiðr þú vesir \hld\ né \alst{sk}ępti-smiðr, &
\ind nema \alst{s}jǫlfum þér \alst{s}éir. &
\alst{Sk}ór ’s \alst{sk}apaðr illa \hld\ eða \alst{sk}apt sé rangt, &
\ind þá ’s þér \alst{b}ǫls \alst{b}eðit.\eva

\bvb I counsel thee, O Loddfathomer—and thou oughtst to learn the counsels; \\
thou wilt have use if thou learn [them], \\
they will be good for thee if thou get [them]: \\
Be not a shoe-maker nor shaft-maker, \\
unless thou be one for thyself. \\
{[If]} the shoe is shaped badly or the shaft be crooked, \\
then for thee a \inx[C]{bale} is bidden.\footnoteB{i.e. ‘the customer will place a curse on you if he dislikes the wares’.}\evb
\evg


\bvg
\bva\alst{R}ǫ́ðumk þér Loddfáfnir, \hld\ en \alst{r}ǫ́ð nemir, &
\ind \alst{n}jóta munt ef \alst{n}emr, &
\ind þér munu \alst{g}óð ef \alst{g}etr: &
hvar’s \alst{b}ǫl kant, \hld\ kveð þér \alst{b}ǫlvi at &
\ind ok gef-at þínum \alst{f}jǫ́ndum \alst{f}rið.\eva

\bvb I counsel thee, O Loddfathomer—and thou oughtst to learn the counsels; \\
thou wilt have use if thou learn [them], \\
they will be good for thee if thou get [them]: \\
Whereever thou dost know a bale, call it a bale against thee, \\
and give not thy enemies peace.\footnoteB{i.e. “if somebody puts a curse on you, do not ignore it, but respond decisively”.  This st. has often been interpreted as a command to call out evil, even when committed towards somebody else, and while there is nothing in it that speaks clearly against that interpretation, it does not agree with the general spirit of the \Havamal, which is one of caution and shrewdness.}\evb
\evg


\bvg
\bva\alst{R}ǫ́ðumk þér Loddfáfnir, \hld\ en \alst{r}ǫ́ð nemir, &
\ind \alst{n}jóta munt ef \alst{n}emr, &
\ind þér munu \alst{g}óð ef \alst{g}etr: &
\alst{i}llu fęginn \hld\ ves \alst{a}ldri-gi, &
\ind \edtrans{en lát þér at \alst{g}óðu \alst{g}etit}{but [rather] let thyself be pleased by good}{\Bfootnote{This construction is equivalent to \CV: \emph{geta}, A. IV. with acc.}}.\eva

\bvb I counsel thee, O Loddfathomer—and thou oughtst to learn the counsels; \\
thou wilt have use if thou learn [them], \\
they will be good for thee if thou get [them]: \\
Rejoicing in evil be thou never, \\
but [rather] let thyself be pleased by good.\evb
\evg


\bvg
\bva\alst{R}ǫ́ðumk þér Loddfáfnir, \hld\ en \alst{r}ǫ́ð nemir, &
\ind \alst{n}jóta munt ef \alst{n}emr, &
\ind þér munu \alst{g}óð ef \alst{g}etr: &
\alst{u}pp líta \hld\ skal-at-tu í \alst{o}rrostu; &
—\alst{g}jalti \alst{g}líkir \hld\ verða \alst{g}umna synir— &
\ind síðr þitt of \alst{h}ęilli \alst{h}alir.\eva

\bvb I counsel thee, O Loddfathomer—and thou oughtst to learn the counsels; \\
thou wilt have use if thou learn [them], \\
they will be good for thee if thou get [them]: \\
Up shalt thou not look in battle \\
—alike to a madman become the sons of men— \\
lest men bewitch thy [sense/life/face].\footnoteB{A very difficult st. \CV\ explains \emph{gjalti} as an old dative of \emph{gǫltr} ‘boar, hog’, and thus sees the closely related phrase \emph{verða at gjalti} as “‘to be turned into a hog’, i.e. ‘to turn mad with terror’, esp. in a fight”. The vowel breaking is however unexpected here, since \emph{gǫltr} (< Proto-Norse \emph{*galtuʀ}) is an u-stem, which makes the stem-vowel in the dat. sg. \emph{gęlti} (< \emph{*galtiu}, cf. \textbf{kunimudiu}, dat. sg. of \emph{*Kunimunduʀ}, on the Tjurkö 1 bracteate) the result of i-umlaut rather than an original short \emph{*e}.

\textcite{LaFargeGlossary} instead explain the word as a borrowing from Old Irish \emph{geilt} ‘insane, mad’. \textcite{PettitEdda} follows this, and argues that the whole theme of the st. probably be of Celtic origin, giving several examples from Celtic literature of warriors going mad upon looking up into the sky during battle. In this case the men (\emph{halir}, which word seems to have an association with warriors; cf. 36–37, 49) would be to quote Pettit some sort of “supernatural sky warriors”, in my opinion most likely the \inx[G]{Ownharriers}.}\evb
\evg


\bvg
\bva\alst{R}ǫ́ðumk þér Loddfáfnir, \hld\ en \alst{r}ǫ́ð nemir, &
\ind \alst{n}jóta munt ef \alst{n}emr, &
\ind þér munu \alst{g}óð ef \alst{g}etr: &
Ef vilt þér \alst{g}óða konu \hld\ kvęðja at \edtrans{\alst{g}aman-rúnum}{pleasure-runes}{\Bfootnote{While easily interpreted as ‘sexual intercourse’, the word is used in st. 120 with a decidedly non-sexual meaning. Its base meaning is probably ‘good, light-hearted conversation’.}} &
\ind ok \alst{f}áa \alst{f}ǫgnuð af, &
\alst{f}ǫgru skalt hęita \hld\ ok láta \alst{f}ast vesa; &
\ind lęiðisk mann-gi \alst{g}ótt ef \alst{g}etr.\eva

\bvb I counsel thee, O Loddfathomer—and thou oughtst to learn the counsels; \\
thou wilt have use if thou learn [them], \\
they will be good for thee if thou get [them]: \\
If thou wilt for thee greet a good woman to pleasure-runes, \\
and receive good cheer from [her]; \\
fair things shalt thou promise, and let it be fast; \\
no man loathes a good thing if he gets it.\evb
\evg


\bvg
\bva\alst{R}ǫ́ðumk þér Loddfáfnir, \hld\ en \alst{r}ǫ́ð nemir, &
\ind \alst{n}jóta munt ef \alst{n}emr, &
\ind þér munu \alst{g}óð ef \alst{g}etr: &
\alst{v}aran bið’k þik \alst{v}esa \hld\ ok ęigi of-\alst{v}aran, &
ves við \alst{ǫ}l varastr, \hld\ ok við \alst{a}nnars konu &
ok við \alst{þ}at hit \alst{þ}riðja, \hld\ at \alst{þ}jófar né lęiki.\eva

\bvb I counsel thee, O Loddfathomer—and thou oughtst to learn the counsels; \\
thou wilt have use if thou learn [them], \\
they will be good for thee if thou get [them]: \\
Wary I ask thee to be, and not over-wary; \\
be thou wariest with ale, and with another man’s woman, \\
and with the third, that thieves do not outplay [thee].\evb
\evg


\bvg
\bva\alst{R}ǫ́ðumk þér Loddfáfnir, \hld\ en \alst{r}ǫ́ð nemir, &
\ind \alst{n}jóta munt ef \alst{n}emr, &
\ind þér munu \alst{g}óð ef \alst{g}etr: &
at \alst{h}áði né \alst{h}látri \hld\ \alst{h}af aldri-gi &
\ind \alst{g}ęst né \alst{g}anganda.\eva

\bvb I counsel thee, O Loddfathomer—and thou oughtst to learn the counsels; \\
thou wilt have use if thou learn [them], \\
they will be good for thee if thou get [them]: \\
In mockery or laughter have thou never \\
a guest nor wanderer.\evb
\evg


\bvg
\bva\alst{O}pt vitu \alst{ó}-gǫrla, \hld\ þęir’s sitja \alst{i}nni fyrir, &
\ind hvęrs þęir ’ru \alst{k}yns es \alst{k}oma; &
es-at maðr svá \alst{g}óðr \hld\ at \alst{g}alli né fylgi, &
\ind né svá \alst{i}llr at \alst{ęi}nu-gi dugi.\eva

\bvb Oft they know unclearly, those who sit further within, \\
of what kind are those who come; \\
there is no man so good that him follows no flaw, \\
nor so bad that he to nothing avails.\evb
\evg


\bvg
\bva\alst{R}ǫ́ðumk þér Loddfáfnir, \hld\ en \alst{r}ǫ́ð nemir, &
\ind \alst{n}jóta munt ef \alst{n}emr, &
\ind þér munu \alst{g}óð ef \alst{g}etr: &
at \alst{h}ǫ́rum þul \hld\ \alst{h}lę́ aldri-gi, &
\ind opt ’s \alst{g}ótt þat’s \alst{g}amlir kveða, &
opt ór \alst{sk}ǫrpum bęlg \hld\ \alst{sk}ilin orð koma &
\ind þęim’s \alst{h}angir með \alst{h}ǫ́um &
\ind ok \alst{sk}ollir með \alst{sk}rǫ́um, &
\ind ok \alst{v}áfir með \alst{v}íl-mǫgum.\eva

\bvb I counsel thee, O Loddfathomer—and thou oughtst to learn the counsels; \\
thou wilt have use if thou learn [them], \\
they will be good for thee if thou get [them]: \\
At a hoary thyle laugh thou never; \\
oft is good that which old men sing. \\
Oft out of a scorched leather discerning words come; \\
out of that one that hangs with hides, \\
and dangles with dry skins, \\
and sways among lads of toil \ken{thralls}.\footnoteB{TODO: Some note. \emph{vil-mǫgum} meaning ‘veal-stomachs’? Cf. Crawford’s video and Finnur on this.}\evb
\evg


\bvg
\bva\alst{R}ǫ́ðumk þér Loddfáfnir, \hld\ en \alst{r}ǫ́ð nemir, &
\ind \alst{n}jóta munt ef \alst{n}emr, &
\ind þér munu \alst{g}óð ef \alst{g}etr: &
\alst{g}ęst þú né \alst{g}ęyj-a \hld\ né á \alst{g}rind hrę́kir; &
\ind get þú \alst{v}ǫ́-luðum \alst{v}ęl.\eva

\bvb I counsel thee, O Loddfathomer—and thou oughtst to learn the counsels; \\
thou wilt have use if thou learn [them], \\
they will be good for thee if thou get [them]: \\
Bark not at a guest, nor spit at the gate;\footnoteB{Behind which the guest stands, waiting for the farmer to open.} \\
furnish the destitute well.\evb
\evg


\bvg
\bva\alst{R}amt es þat tré, \hld\ es \alst{r}íða skal &
\ind \alst{ǫ}llum at \alst{u}pp-loki; &
\alst{b}aug þú gef \hld\ eða þat \alst{b}iðja mun &
\ind þér \alst{l}ę́s hvęrs á \alst{l}iðu.\eva

\bvb Strong is that wood which shall swing \\
to open for all.\footnoteB{i.e. the beam of the gate in front of the farm.} \\
Give a bigh, or it will bid \\
every kind of guile on thy limbs.\evb
\evg


\bvg
\bva\alst{R}ǫ́ðumk þér Loddfáfnir, \hld\ en \alst{r}ǫ́ð nemir, &
\ind \alst{n}jóta munt ef \alst{n}emr, &
\ind þér munu \alst{g}óð ef \alst{g}etr: &
hvar’s \alst{ǫ}l drekkir \hld\ kjós þér \alst{ja}rðar męgin, &
því-at \alst{jǫ}rð tękr við \alst{ǫ}lðri, \hld\ en \alst{ę}ldr við sóttum, &
\alst{ęi}k við \alst{a}bbindi, \hld\ \alst{a}x við fjǫl-kyngi, &
\alst{h}ǫll við \alst{h}ýrógi; \hld\ \alst{h}ęiptum skal mána kvęðja, &
\alst{b}ęiti við \alst{b}it-sóttum, \hld\ en við \alst{b}ǫlvi rúnar; &
\ind \alst{f}old skal við \alst{f}lóði taka.\eva

\bvb I counsel thee, O Loddfathomer, that thou learn the counsels; \\
thou wilt have use if thou learn [them], \\
they will be good for thee if thou get [them]: \\
Wherever thou drinkest ale, choose for thee Earth’s might, \\
for earth takes against drunkenness, but fire against sicknesses; \\
oak against dysentery, the ear [of corn] against sorcery, \\
bearded rye against hernia—in conflicts shall one invoke Moon\footnoteB{According to \Voluspa\ 5, the moon has some sort of power, and based on \Lokasenna\ P3 \emph{kvęðja} ‘greet, call’ seems to be the word used for invoking in prayer.}— \\
heather against bite-sicknesses; but \inx[C]{rune}[runes] against a \inx[C]{bale};\footnoteB{cf. sts. 124, 149.} \\
fold \ken{earth} shall one employ against flood.\evb
\evg

\sectionline

\section{The Rune-Tally}

These scattered sts. are introduced by a larger initial in \Regius, marking the beginning of a new section. They have the header \emph{Rúna-tals þáttr} ‘Strand of the Rune-Tally’ in younger paper mss. and generally give an archaic, mystic impression; it is as if they were drawn from the lips of an Odinic priest.

Apart from these stanzas, there are a few other instances of Runic magic. Closest at hand is st. 80 above, which would fit seamlessly into the present section. Outside of \Havamal\ there is \Sigrdrifumal\ 4–16, also preserved in \Regius.

\sectionline

\bvg
\bva\alst{V}ęit’k at ek hekk \hld\ \alst{v}indga męiði á &
\ind \alst{n}ę́tr allar \alst{n}íu, &
\alst{g}ęiri undaðr \hld\ ok \alst{g}efinn Óðni, &
\ind \alst{s}jalfr \alst{s}jǫlfum mér, &
á þęim \alst{m}ęiði, \hld\ es \alst{m}ann-gi vęit, &
\ind hvęrs af \alst{r}ótum \alst{r}innr.\eva

\bvb I know that I hung on the windy beam, \\
for nine nights all; \\
wounded by spear and given to Weden— \\
myself to myself— \\
on that beam, which no man knows, \\
of whose roots it runs.\evb
\evg


\bvg
\bva Við \edtext{\alst{h}lęifi mik sǿldu-t \hld\ né við \alst{h}orni-gi}{\lemma{hlęifi \dots\ horni-gi ‘loaf \dots\ horn’}\Bfootnote{i.e. “I was given neither food nor drink”.}}; &
\alst{n}ýsta ek \alst{n}iðr, \hld\ \alst{n}am’k upp rúnar, &
\alst{ǿ}pandi nam, \hld\ fell’k \alst{a}ptr þaðan.\eva

\bvb With loaf they relieved me not, nor with any horn. \\
I peered down, I took up the runes, \\
screaming I took; I fell back thence.\evb
\evg


\bvg
\bva\alst{F}imbul-ljóð níu \hld\ nam’k af hinum \alst{f}rę́gja syni &
\ind \alst{B}ǫlþorns, \alst{B}ęstlu fǫður, &
ok ek \alst{d}rykk of gat \hld\ hins \alst{d}ýra mjaðar &
\ind \alst{au}sinn \alst{Ó}ð-rǿri.\eva

\bvb Nine \inx[C]{fimble-leeds} I learned from the famous son \\
of \inx[P]{Balethorn}, \inx[P]{Bestle}’s father— \\
and a drink I got, of that dear mead \\
poured [from] \inx[P]{Woderearer}.\footnoteB{This st. fits poorly here and seems like an insert. It mentions \emph{ljóð} ‘leeds; (magical) songs, incantations’ rather than runes, and has nothing to do with Weden’s hanging on the tree. Bestle was Weden’s mother and Balethorn his maternal grandfather. The famous son of Balethorn would then be his maternal uncle. The custom of sending sons away to be fostered by their maternal uncles or grandfathers (which seems to be what is going on here) was quite common in Germanic society, cf. TODO.}\evb
\evg


\bvg
\bva Þá \edtrans{nam’k \alst{f}rę́vask}{I took to thrive}{\Bfootnote{A notorious mistranslation (TODO: source) has rendered these words as ‘I took semen’, seeing in them a reference to Weden taking the seed from hanged men in order to replenish his own powers, something never attested elsewhere. This notion, surely based on the word \emph{frę́} ‘seed’, has no philological grounding. \emph{frę́vask} is wo. doubt a reflexive verb, and regardless \emph{frę́} is used of plant seeds, not ejaculate.}} \hld\ ok \alst{f}róðr vesa &
\ind ok \alst{v}axa ok \alst{v}ęl hafask; &
\alst{o}rð mér af \alst{o}rði \hld\ \alst{o}rðs lęitaði &
\ind \alst{v}erk mér af \alst{v}erki \alst{v}erks.\eva

\bvb Then I began to flourish, and be learned, \\
and grow and have it well. \\
My word from a word a word sought out; \\
my work from a work a work.\footnoteB{Each good speech and deed quickly led to another.}\evb
\evg


\bvg
\bva \edtext{\alst{R}únar munt finna \hld\ ok \alst{r}áðna stafi}{\lemma{Rúnar \dots\ ok ráðna stafi}\Bfootnote{Formulaic. Cf. the long-line on the medieval runestone N 13 (excerpt): \emph{rúnar ek ríst \hld\ ok ráðna stafi} ‘runes I carve, and interpreted staves’.}}, &
\ind mjǫk \alst{st}óra \alst{st}afi, &
\ind mjǫk \alst{st}inna \alst{st}afi, &
\ind es \alst{f}áði \alst{F}imbul-þulr &
\ind ok \alst{g}ørðu \alst{g}inn-ręgin &
\ind ok \alst{r}ęist Hroptr \edtrans{\alst{r}agna}{of the Reins}{\Afootnote{\emph{‘rǫgna’} \Regius}}.\eva

\bvb \inx[C]{rune}[Runes] wilt thou find, and interpreted staves: \\
very large staves, \\
very stiff staves, \\
which \inx[P]{Fimble-Thyle} \name{= Weden} painted, \\
and the \inx[G]{yin-Reins} made, \\
and Roft \name{= Weden} of the Reins carved.\evb
\evg


\bvg
\bva\alst{Ó}ðinn með \alst{ǫ́}sum, \hld\ en fyr \alst{ǫ}lfum Dáinn, &
\ind \alst{D}valinn \alst{d}vergum fyrir, &
\ind \alst{Á}sviðr \alst{jǫ}tnum fyrir, &
\ind ek ręist \alst{s}jalfr \alst{s}umar.\eva

\bvb \inx[P]{Weden} among the \inx[G]{Eese}, but for the \inx[G]{Elves} \inx[P]{Dowen}; \\
\inx[P]{Dwollen} for the \inx[G]{Dwarfs}; \\
\inx[P]{Oswood} for the Ettins; \\
I myself carved some.\footnoteB{The identity of the speaker is not clear. One would expect him to be Weden.}\evb
\evg


\bvg
\bva Vęitst, hvé \alst{r}ísta skal? \hld\ Vęitst, hvé \alst{r}áða skal? &
Vęitst, hvé \alst{f}áa skal? \hld\ Vęitst, hvé \alst{f}ręista skal? &
Vęitst, hvé \alst{b}iðja skal? \hld\ Vęitst, hvé \alst{b}lóta skal? &
Vęitst, hvé \alst{s}ęnda skal? \hld\ Vęitst, hvé \alst{s}óa skal?\eva

\bvb Knowest thou how one shall carve? Knowest thou how one shall read? \\
Knowest thou how one shall paint? Knowest thou how one shall try? \\
Knowest thou how one shall bid? Knowest thou how one shall \inx[C]{bloot}? \\
Knowest thou one shall send? Knowest thou how one shall \inx[C]{soo}?\footnoteB{A neat semantic structure would be found if the former four verbs referred to \inx[C]{rune}[runes]: carving, interpreting, painting (with blood?), and divining; and the latter four referred to sacrifice: asking for boons, worshipping, sending (the sacrifice or the prayer; making sure the gods receive it), and slaying the victim. This may be supported by the following stanza, which repeats the last four verbs here in what looks like a sacrificial context. See further relevant Encyclopedia entries.}\footnoteB{The meter of this st. is unusual, but bears some resemblance to Vg 216 (the Högstena galder). TODO: Elaborate.}\evb
\evg


\bvg
\bva\alst{B}ętra ’s ó·\alst{b}eðit \hld\ an sé of-\alst{b}lótit, &
\ind ęy sér til \alst{g}ildis \alst{g}jǫf; &
bętra ’s ó·\alst{s}ęnt \hld\ an sé of-\alst{s}óit; &
\edtext{[...]}{\Bfootnote{It is almost certain that a line be missing here, which is very unfortunate.}}\eva

\bvb ’Tis better unbid than over\inx[C]{bloot}[blooted]; \\
a gift always sees repayment. \\
’Tis better unsent than over\inx[C]{soo}[sooed]; \\
{[...]}.\footnoteB{An identical progression of four verbs suggests a close relation with the previous st. — The sense seems to be that it is better not to sacrifice at all than to sacrifice in excess, since even a small gift (to the gods) will be rewarded. A ritual cycle of gifts and rewards between men and the gods is also seen in other Indo-European pagan literatures. Compare the Sanskrit \emph{Dehí me, dádāmi te} ‘Give to me, I give to thee’ and Latin \emph{dō ut dēs} ‘I give that thou might give’.}\evb
\evg


\bvg
\bva Svá \alst{Þ}undr of ręist \hld\ fyr \alst{þ}jóða rǫk, &
þar’s \alst{u}pp of ręis, \hld\ es \alst{a}ptr of kom.\eva

\bvb Thus \inx[P]{Thound} \name{= Weden} did carve for the rakes of nations, \\
where up he rose as back he came.\footnoteB{TODO: A very cryptic st.}\evb
\evg

\sectionline

\section{The Leed-Tally}

This section of \Havamal, the so-called the Leed-Tally (\emph{Ljóðatal}), is not separated from the preceding section (which is marked out with a large initial), but is usually taken as separate since it is a unified whole not much concerned with runes. The speaker (certainly Weden) recounts eighteen spells, aristocratic and Odinic in character; they deal with such things as healing (spell 2, 12), battle (3, 4, 5, 8, 11, 13), countering sorcery (6, 10), stilling the elements (7, 9), and seduction (16, 17).

In particular the fourth spell bears a strong likeness to the first Merseburg charm.


\bvg
\bva Ljóð \alst{þ}au kann’k, \hld\ es kann-at \alst{þ}jóðans kona &
\ind ok \alst{m}anns-kis \alst{m}ǫgr. &
\alst{H}jǫlp hęitir ęitt, \hld\ þat þér \alst{h}jalpa mun &
\ind við \alst{s}orgum ok \alst{s}ǫkum, \hld\ ok \alst{s}útum gǫrv-ǫllum.\eva

\bvb Those \inx[C]{leed}[leeds] I know, as knows not the ruler’s woman, \\
and no man’s lad: \\
Help is called one, it will help thee \\
against sorrows and sakes,\footnoteB{Legal proceedings.} and all kinds of griefs.\footnoteB{TODO: elaborate on translatioon}\evb
\evg


\bvg
\bva Þat kann’k \alst{a}nnat, \hld\ es þurfu \alst{ý}ta synir, &
\ind þęir’s vilja \alst{l}ę́knar \alst{l}ifa.\eva

\bvb I know another, which the sons of men need;\footnoteB{Identical wording to 164/2.} \\
those who wish to live as leechers.\evb
\evg


\bvg
\bva Þat kann’k \alst{þ}riðja, \hld\ ef mér verðr \alst{þ}ǫrf mikil &
\ind \alst{h}apts við mína \alst{h}ęipt-mǫgu, &
\alst{ę}ggjar dęyfi’k \hld\ minna \alst{a}nd-skota, &
\ind bíta-t þęim \alst{v}ǫ́pn né \edtrans{\alst{v}ęlir}{staffs}{\Bfootnote{This word cannot be \emph{vélir} ‘wiles’ due to the meter. It may refer to magical staffs. (TODO.)}}.\eva

\bvb I know the third, if I come in great need \\
of hindrance against my conflict-lads \ken{enemies}; \\
I dull the edges of my opponents; \\
for them bite not weapons nor staffs.\evb
\evg


\bvg
\bva Þat kann’k \alst{f}jórða, \hld\ ef mér \alst{f}yrðar bera &
\ind \alst{b}ǫnd at \alst{b}óg-limum, &
svá ek \alst{g}ęl, \hld\ at \alst{g}anga má’k, &
\ind sprettr mér af \alst{f}ótum \alst{f}jǫturr, &
\ind en af \alst{h}ǫndum \alst{h}apt.\eva

\bvb I know the fourth, if men should bear \\
bonds onto my shoulder-limbs \ken{arms}: \\
so I gale that I may walk; \\
springs off my feet the fetter, \\
and off my hands the bond.\footnoteB{Cf. \MerseburgOne\ (edited below under Charms and Spells), a galder that seems to have actually been used for the purpose of removing fetters.}\evb
\evg


\bvg
\bva Þat kann’k \alst{f}imta, \hld\ ef sé’k af \alst{f}ári skotinn &
\ind \alst{f}lęin í \alst{f}olki vaða, &
flýgr-a svá \alst{st}int, \hld\ at \alst{st}ǫðvi’g-a’k, &
\ind ef hann \alst{s}jónum of \alst{s}é’k.\eva

\bvb I know the fifth, if I see a dangerously shot \\
arrow wading in the troop; \\
it flies not so stiffly that I may not hinder it, \\
if I see it with my sights.\evb
\evg


\bvg
\bva Þat kann’k \alst{s}étta, \hld\ ef mik \alst{s}ę́rir þegn &
\ind á \alst{r}ótum \alst{r}ás viðar, &
þann \alst{h}al, \hld\ es mik \alst{h}ęipta kveðr, &
\ind þann eta \alst{m}ęin hęldr an \alst{m}ik.\eva

\bvb I know the sixth, if a thane should injure me \\
on the roots of a raw/sappy tree;\footnoteB{i.e., if he carves harmful magic runes into the roots. See note to \Skirnismal\ 32, where \emph{hrár viðr} ‘raw/sappy tree’ also occurs in a context of curse-magic.} \\
that man who sings hatred against me, \\
him eat the harms rather than me.\evb
\evg


\bvg
\bva Þat kann’k \alst{s}jaunda, \hld\ ef \alst{s}é’k hǫ́van loga &
\ind \alst{s}al of \alst{s}ess-mǫgum, &
\alst{b}rinnr-at svá \alst{b}ręitt, \hld\ at hǫ́num \alst{b}jargi’g-a’k; &
\ind þann kann’k \alst{g}aldr at \alst{g}ala.\eva

\bvb I know the seventh, if I see a high hall \\
burning over seat-lads \ken{warriors}: \\
it burns not so broadly that I do not save it\footnoteB{i.e. ‘if I see a hall burning with men trapped inside, no matter how large the flame is I can save both the hall and the men’.}— \\
that galder I can gale.\evb
\evg


\bvg
\bva Þat kann’k \alst{á}tta, \hld\ es \alst{ǫ}llum es &
\ind \alst{n}yt-sam-ligt at \alst{n}ema, &
\alst{h}var’s \edtrans{\alst{h}atr}{hatred}{\Bfootnote{i.e. with regard to the father’s inheritance.}} vęx \hld\ með \alst{h}ildings sonum, &
\ind þat má’k \alst{b}ǿta \alst{b}rátt.\eva

\bvb I know the eighth, which for all men is \\
useful to learn: \\
wherever hatred grows among a prince’s sons, \\
it I may shortly mend.\evb
\evg


\bvg
\bva Þat kann’k \alst{n}íunda, \hld\ ef mik \alst{n}auðr of stęndr &
\ind at bjarga \alst{f}ari mínu á \alst{f}loti, &
\alst{v}ind ek kyrri \hld\ \alst{v}ági á &
\ind ok \alst{s}vę́fi’k allan \alst{s}ę́.\eva

\bvb I know the ninth, if I am in need \\
to save my friend on a floater \ken{ship}: \\
the wind I calm on the wave, \\
and put all the sea asleep.\evb
\evg


\bvg
\bva Þat kann’k \alst{t}íunda, \hld\ ef sé’k \alst{t}ún-riður &
\ind \alst{l}ęika \alst{l}opti á, &
ek svá \alst{v}inn’k, \hld\ at \edtrans{þę́r \alst{v}illar fara}{they (\emph{fem.}) go astray}{\Bfootnote{emend.; \emph{þęir villir fara} ‘they (\emph{masc.}) go astray’ \Regius}} &
\ind sinna \alst{h}ęim-\alst{h}ama &
\ind sinna \alst{h}ęim-\alst{h}uga.\eva

\bvb I know the tenth, if I see \inx[G]{town-riders} \\
playing aloft: \\
I accomplish it so that they go astray \\
from their home-\inx[C]{hame}[hames]; \\
from their home-minds.\footnoteB{The \emph{riður} ‘(female) riders’ were witches who were thought to leave their hames (\emph{hamir} ‘skins, shapes’) in a form of astral projection in order to fly around in the air, tormenting villagers. Their original bodies would of course be lying in a comatose state, and with the bodies their original minds; their humanness. Weden was through his second sight able to see these riders, and could use his superior magical abilities in order to confuse them so that they were not able to return to their original hames or minds (but were instead forced to wander astray); a cruel fate. — Weden likewise brags about tricking riders in \Harbardsljod\ 20.}\evb
\evg


\bvg
\bva Þat kann’k \alst{ę}llipta, \hld\ ef skal’k til \alst{o}rrostu &
\ind \alst{l}ęiða \alst{l}ang-vini, &
und \alst{r}andir gęl’k, \hld\ en þęir með \alst{r}íki fara, &
\ind \alst{h}ęilir \alst{h}ildar til, &
\ind \alst{h}ęilir \alst{h}ildi frá, &
\ind koma þęir \alst{h}ęilir \alst{h}vaðan.\eva

\bvb I know the eleventh, if I shall into war \\
lead old friends: \\
beneath the shields I gale, and they go with power \\
healthy to the battle, \\
healthy from the battle; \\
they return healthy anywhence.\evb
\evg


\bvg
\bva Þat kann’k \alst{t}olpta, \hld\ ef sé’k á \alst{t}ré uppi &
\ind \alst{v}áfa \alst{v}irgil-ná, &
svá ek \alst{r}íst \hld\ ok í \alst{r}únum fá’k, &
\ind at sá \alst{g}ęngr \alst{g}umi. &
\ind ok \alst{m}ę́lir við \alst{m}ik.\eva

\bvb I know the twelfth, if I see high up on a tree \\
a gallow-corpse dangling: \\
so I carve and paint in the runes, \\
that that man walks \\
and speaks with me.\evb
\evg


\bvg
\bva Þat kann’k \alst{þ}rettánda \hld\ ef skal’k \alst{þ}egn ungan &
\ind \alst{v}erpa \alst{v}atni á, &
mun-at hann \alst{f}alla \hld\ þótt í \alst{f}olk komi, &
\ind \alst{h}nígr-a sá \alst{h}alr fyr \alst{h}jǫrum.\eva

\bvb I know the thirteenth, if I shall upon a young thane \\
throw water:\footnoteB{Describing the Heathen ritual of pouring water on a newborn child. Cf. \Rigsthula\ 7, 21, 34.} he will not fall though he should come into battle; \\
that warrior sinks not down before swords.\evb
\evg


\bvg
\bva Þat kann’k \alst{f}jórtánda, \hld\ ef skal’k \alst{f}yrða liði &
\ind \alst{t}ęlja \alst{t}íva fyr, &
\alst{á}sa ok \alst{a}lfa \hld\ ek kann \alst{a}llra skil, &
\ind fár kann ó·\alst{s}notr \alst{s}vá.\eva

\bvb I know the fourteenth, if before a retinue of men \\
I shall count forth the Tews: \\
of all the Eese and Elves I know the discernments;\footnoteB{Cf. \Hymiskvida\ 38, where the corresponding verb \emph{skilja} is used in the context of god-knowledge.} \\
few unwise men can do so.\evb
\evg


\bvg
\bva\alst{Þ}at kann’k fimtánda, \hld\ es gól \alst{Þ}jóð-rǿrir &
\ind \alst{d}vergr fyr \alst{D}ęllings \alst{d}urum, &
\alst{a}fl gól \alst{ǫ́}sum, \hld\ en \alst{ǫ}lfum frama, &
\ind \alst{h}yggju \alst{H}ropta-týi.\eva

\bvb I know the fifteenth, which Thedrearer galed, \\
the dwarf, before Delling’s doors. \\
Power he galed for the Eese, but for the Elves distinction; \\
thought for Roft-Tew \name{= Weden}.\evb
\evg


\bvg
\bva Þat kann’k \alst{s}extánda, \hld\ ef vil’k hins \alst{s}vinna mans &
\ind hafa \alst{g}ęð allt ok \alst{g}aman, &
\alst{h}ugi \alst{h}vęrfi’k \hld\ \alst{h}vit-armri konu &
\ind ok \alst{s}ný’k hęnnar ǫllum \alst{s}efa.\eva

\bvb I know the sixteenth, if I will from the wise girl \\
have her senses all, and pleasure; \\
the heart I change of the white-armed woman, \\
and I twist all her mind.\evb
\evg


\bvg
\bva Þat kann’k \alst{s}jautjánda \hld\ at mik \alst{s}ęint mun firrask &
\ind hit \alst{m}an-unga \alst{m}an.\eva

\bvb I know the seventeenth, that the girl-young girl \\
will lately shun me.\evb
\evg


\bvg
\bva\alst{L}jóða þessa \hld\ munt \alst{L}oddfáfnir &
\ind lengi \alst{v}anr \alst{v}esa; &
\ind þó sé þér \alst{g}óð ef \alst{g}etr, &
\ind \alst{n}ýt ef \alst{n}emr, &
\ind \alst{þ}ǫrf ef \alst{þ}iggr.\eva

\bvb These leeds wilt thou, Loddfathomer, \\
long be lacking! \\
Though they would be good for thee if thou get [them], \\
useful if thou learn [them], \\
needful if thou receive [them].\evb
\evg


\bvg
\bva Þat kann’k \alst{á}tjánda, \hld\ es \alst{ę́}va kęnni’k &
\ind \alst{m}ęy né \alst{m}anns konu, &
—\alst{a}lt es bętra \hld\ es \alst{ęi}nn of kann, &
\ind þat fylgir \alst{l}jóða \alst{l}okum— &
nema þęiri \alst{ęi}nni, \hld\ es mik \alst{a}rmi vęrr, &
\ind eða mín \alst{s}ystir \alst{s}éi.\eva

\bvb I know the eighteenth, which I never teach \\
a maiden nor man’s woman— \\
everything is better when one alone can do it; \\
that follows the end of the leeds— \\
save for her alone who wraps me in her arm,\footnoteB{This interesting expression is also used \Volundarkvida\ 2. — The one who wraps Weden in her arm may be his wife, Frie. He has no known sister.} or who my sister is.\evb
\evg


\bvg
\bva Nú eru \alst{H}áva mǫ́l kveðin \hld\ \alst{H}áva \alst{h}ǫllu í; &
\ind \alst{a}ll-þǫrf \alst{ý}ta sonum, &
\ind \alst{ó}-þǫrf \edtrans{\alst{jǫ}tna}{ettins}{\Afootnote{\emph{ýta} ‘men’ (corrected in margin) \Regius}} sonum; &
hęill sá’s \alst{k}vað, \hld\ hęill sá’s \alst{k}ann, &
\ind \alst{n}jóti sá’s \alst{n}am, &
\ind \alst{h}ęilir þęir’s \alst{h}lýddu.\eva

\bvb Now are the High One’s speeches sung, in the High One’s hall; \\
of great need for the sons of men, \\
of harm for the sons of ettins. \\
Hail he who sang; hail he who knows; \\
may he use who learned; \\
hail those who heeded!\evb
\evg
