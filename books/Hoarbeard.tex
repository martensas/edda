\bookStart{Leeds of Hoarbeard}[Hárbarðsljóð]
\def\thisBookCode{Harbardsljod}

\begin{flushright}%
\textbf{Dating} \parencite{Sapp2022}: early C11th (0.578)–late C11th (0.377)

\textbf{Meter:} Unclear (TODO)%
\end{flushright}

\section{Introduction}

The \textbf{Leeds of Hoarbeard} (\Harbardsljod) is preserved in full in \Regius, and in part in \AM.  The poem might be seen as an allegory on class relations, namely between the self-owning yeomen farmers and the warlike earls, represented through their patron gods.

Of all Eddic poems \Harbardsljod\ is probably the strangest in terms of form. Verse length varies greatly, and many of the lines (see especially the final verse) are of an obscene length reminiscent of late continental Germanic poems like the Heliand; some simply have no metrical qualities at all. The young clitic definite is (uniquely) employed frequently throughout the poem. These criteria would seem to point towards a late origin for the poem (though not later than the late C13th, when \Regius\ was written).

Against this late origin speaks the presence of rare words (e.g. \emph{ǫgurr} v. 13) and a thorough understanding of the personalities of the two gods which would seem unlikely to stem from several centuries after the conversion of Iceland. The model devised by Sapp gives the poem a 57.8\% likelihood of being from the early C11th, and a 37.7\% likelihood of being from the late 11th. These scores are most similar to those obtained by \Gripisspa, a poem that on the surface seems much more archaic.

What could we then be dealing with? It may of course be that the poem is heavily corrupt, but there is no good evidence for this (apart from the above-mentioned irregularities). Most lines are readily understandable and fit well both within their respective context and the poem as a whole. I think a better solution to this problem is to assume that the poem has been acted out as a sort of carnivalesque theatre, with two masked actors, each playing one of the gods. This would explain the variations in meter and line length, and the prose; some lines were simply shouted out, and the lack of alliteration in them would then have a kind of discordant effect.

This is shown also by uses of the word ‘here’ in sts. 9 and 14. TODO: mention concept of "double scene" by Lars Lönnroth?

\section{The Leeds of Hoarbeard}

\bpg\bpa\mssnote{\Regius~12r/30}%
Þórr fór ór austr-vegi ok kom at sundi einu. Ǫðrum megum sundsins var ferju-karlinn með skipit. Þórr kallaði:\epa

\bpb Thunder journeyed from the Eastern Way and came to a sound. At the other side of the sound was the ferryman with the ship. Thunder called out:\epb\epg


\bvg\bva\mssnote{\Regius~12r/32}%
„Hvęrr ’s sá \alst{s}vęinn \alst{s}vęina \hld\ es stęndr fyr \alst{s}undit handan?“\eva

\bvb “Who is that swain of swains, standing here across the sound?”\evb\evg


\bvg\bva\speakernote{Hann svaraði:}\mssnote{\Regius~12v/1}%
„Hvęrr ’s sá \alst{k}arl \alst{k}arla \hld\ es \alst{k}allar of váginn?“\eva

\bvb\speakernoteb{He answered:}%
“Who is that churl of churls, calling out over the wave?”\evb\evg


\bvg\bva\mssnote{\Regius~12v/2}%
„\alst{F}ęr þú mik of sundit, \hld\ \alst{f}ǿði’k þik á morgun; &
\alst{m}ęis hęfi’k á baki, \hld\ verðr-a \alst{m}atr inn bętri. &
Át’k í \alst{h}víld \hld\ áðr ek \alst{h}ęiman fór, &
\alst{s}íldr ok \edtrans{hafra}{oatmeal/he-goats}{\Bfootnote{(1) The easiest reading is the acc. pl. of \emph{hafr} ‘he-goat’.  Thunder also eats his goats in \Gylfaginning\ 44, where he butchers and cooks them in the evening and brings them back to life by blessing them with his hammer at dawn.  \textcite{FinnurEdda} and \textcite{PettitEdda} prefer this. (2) Other scholars instead read an acc. pl. of \emph{hafri} ‘oat’, i.e. ‘porridge, oatmeal’.  Stiles (forthcoming TODO) connects this with the porridge-eating of the Vedic god Pūṣán (\Rigveda\ 6.56.1, 57.2), who is “partner and yokemate” (\Rigveda\ 6.56.2) of Índra, Thunder’s vedic equivalent.  Another similarity Stiles notes between Thunder and Pūṣán is that both have chariots driven by goats (e.g. 6.57.3: “Goats are the draft-animals for the one”, 58.2: “Having goats as his horses”).  Whether the Vedic tradition has split the Thunder-god in two or whether the Germanic Thunder has absorbed elements of his yokemate is hard to say.}}; \hld\ \alst{s}aðr em’k ęnn þęss.“\eva

\bvb\speakernoteb{[Thunder quoth:]} \\
“Ferry me over the sound, I feed thee in the morning! \\
A basket have I on my back; better food will not be found. \\
I ate for a while before I journeyed from home, \\
herring and oatmeal/he-goats; I am still full from that.”\evb\evg


\bvg\bva\mssnote{\Regius~12v/5}%
„Ár-ligum \alst{v}erkum hrósar þú, \alst{v}ęrði’num; \hld\ \alst{v}ęitst-at-tu fyr gǫrla, &
\alst{d}ǫpr ’ru þín hęim-kynni, \hld\ \alst{d}auð hygg’k at þín móðir sé.“\eva

\bvb “Of early works boastest thou; of eating!\footnoteB{TODO. This is pretty difficult. From the previous stanza \emph{vęrðinum} seems to be referring to eating.} Thou seest not clearly ahead: \\
dire is the state of thy home—I think that thy mother is dead!”\evb\evg


\bvg\bva\mssnote{\Regius~12v/6}%
„\alst{Þ}at sęgir þú nú \hld\ es hvęrjum \alst{þ}ikkir &
\alst{m}ęst at vita— \hld\ at mín \alst{m}óðir dauð sé.“\eva

\bvb “Thou now sayest that which to every man seems \\
of most weight to know—that my mother is dead!”\evb\evg


\bvg\bva\mssnote{\Regius~12v/8}%
„\alst{Þ}ęygi ’s sem \alst{þ}ú \hld\ \alst{þ}rjú bú ęigir góð; &
\alst{b}ęr-\alst{b}ęinn þú stęndr \hld\ ok hęfir \alst{b}rautinga gørvi, \hld\ þat-ki at þú hafir \alst{b}rę́kr þínar.“\eva

\bvb “It’s hardly as if thou own three good farms— \\
bare-legged thou standest, and hast the gear of a tramp; it is not even as if thou own thy breeches!”\evb\evg


\bvg\bva\mssnote{\Regius~12v/9}%
„\alst{St}ýr-ðu hingat ęikjunni, \hld\ ek mun þér \alst{st}ǫðna kęnna &
eða \alst{h}vęrr á skipit \hld\ es þú \alst{h}ęldr við landit?“\eva

\bvb “Steer hither the boat! I will show thee to the harbour— \\
or who owns the ship which thou holdest by the shore?”\evb\evg


\bvg\bva\mssnote{\Regius~12v/11}%
„\alst{H}ildólfr sá \alst{h}ęitir \hld\ es mik \alst{h}alda bað, &
\alst{r}ekkr inn \alst{r}áð-svinni \hld\ es býr í \alst{R}áðs-ęyjar-sundi; &
bað-at hann \alst{h}lęnni-męnn flytja \hld\ eða \alst{h}rossa-þjófa, &
\alst{g}óða ęina \hld\ ok þá’s ek \alst{g}ørva kunna; &
\alst{s}ęg-ðu til nafns þíns \hld\ ef þú vill of \alst{s}undit fara.“\eva

\bvb “Hildolf is he called who asked me to hold it, \\
the counsel-wise man who lives in Redeseysound. \\
He bade me not ferry highwaymen nor horsethieves; \\
good men only, and those I know well— \\
speak to thy name if thou wilt go over the sound!”\evb\evg


\bvg\bva\mssnote{\Regius~12v/15}%
„\alst{S}ęgja mun’k til nafns míns \hld\ þótt ek \alst{s}ękr sjá’k &
ok til \alst{a}lls \alst{ø}ðlis: \hld\ Ek em \alst{Ó}ðins sonr, &
\alst{M}ęila bróðir \hld\ ęn \alst{M}agna faðir, &
\alst{þ}rúð-valdr goða \hld\ við \alst{Þ}ór knátt-u hér dǿma! &
\alst{H}ins vil’k nú spyrja, \hld\ hvat þú \alst{h}ęitir.“\eva

\bvb “I will speak to my name—even though I should be charged— \\
and to all my origin: I am Weden’s son, \\
Male’s brother and Main’s father, \\
the strength-wielder of the Gods; with Thunder dost thou here speak! \\
Now I will ask this, what thou art called.”\evb\evg


\bvg\bva\mssnote{\Regius~12v/18}„\alst{H}ár-barðr ek \alst{h}ęiti, \hld\ \alst{h}yl’k of nafn sjaldan.“\eva

\bvb “Hoarbeard I am called; I seldom conceal my name.”\evb\evg


\bvg\bva\mssnote{\Regius~12v/18}„Hvat skalt-u of \alst{n}afn hylja \hld\ \alst{n}ema þú sakar ęigir?“\eva

\bvb “Why shalt thou conceal thy name, unless thou have charges?”\evb\evg


\bvg\bva\mssnote{\Regius~12v/19}„En þótt ek \alst{s}akar ęiga, \hld\ fyr \alst{s}líkum sem þú est &
þá mun’k \alst{f}orða \alst{f}jǫrvi mínu \hld\ nema ek \alst{f}ęigr sé.“\eva

\bvb “Even though I had charges—for such a one as thou art \\
I would then protect my life, unless I be \inx[C]{fey}.”\evb\evg


\bvg\bva\mssnote{\Regius~12v/21}„Harm ljótan mér þikkir í því &
at \alst{v}aða of \alst{v}áginn til þín \hld\ ok \alst{v}ę́ta \edtrans{ǫgur}{burden}{\Bfootnote{The sense of this word is not clear, though it is probably the same as the first element of the compound \emph{ǫgur-stund} ‘burdensome hour’, found in \Volundarkvida\ 42.  Some authors have read it as a crude euphemism for “penis”, which would not stand out much in this poem.  Another interpretation is that it refers to the food Thunder carries on his back (st. 3).}} mínn; &
skylda’k launa \alst{k}ǫgur-svęini \hld\ þínum \alst{k}angin-yrði \hld\ ef ek \alst{k}omumk yfir sundit.“\eva

\bvb “An ugly harm it seems to me \\
to wade o’er the wave to thee, and wet my burden. \\
I would repay thee, swaddle-swain, for thy mocking words, if I could bring myself over the sound.”\evb\evg


\bvg\bva\mssnote{\Regius~12v/23}„\alst{H}ér mun’k standa \hld\ ok þín \alst{h}eðan bíða; &
fannt-a-tu mann inn \alst{h}arðara \hld\ at \alst{H}rungni dauðan.“\eva

\bvb “Here will I stand and hence await thee; \\
thou foundest not a harder man since \inx[P]{Rungner} died!\footnoteB{Rungner was a famous ettin slain by Thunder in a fierce battle.  Hoarbeard’s mention of that battle sets off a long argument over their respective accomplishments.}”\evb\evg


\bvg\bva\mssnote{\Regius~12v/25}„\alst{H}ins vilt-u nú geta \hld\ es vit \alst{H}rungnir dęildum, &
sá inn \alst{st}ór-úðgi jǫtunn, \hld\ es ór \alst{st}ęini vas hǫfuðit á, &
þó lét’k hann \alst{f}alla \hld\ ok \alst{f}yrir hníga; &
\ind hvat vannt-u þá meðan, Hárbarðr?“\eva

\bvb “Of this wilt thou now speak, when I and Rungner dealt with each other, \\
that great-minded ettin on whom the head was of stone.  \\
Yet I made him fall, and kneel down before [me]— \\
what didst thou then meanwhile, Hoarbeard?”\evb\evg


\bvg\bva\mssnote{\Regius~12v/27}%
„Vas’k með \alst{F}jǫl-vari \hld\ \alst{f}imm vetr alla &
í \alst{ęy} þęiri \hld\ es \alst{A}l-grǿn hęitir; &
\alst{v}ega vér þar knǫ́ttum \hld\ ok \alst{v}al fęlla, &
\alst{m}args at fręista, \hld\ \alst{m}ans at kosta.“\eva

\bvb “I was with Felwar for five winters all \\
in that island which is called Allgreen. \\
There we did fight and fell the slain, \\
many a girl tempt and win.\footnoteB{I read \emph{margs} ‘many a’ as modifying \emph{mans} ‘girl’.}”\evb\evg


\bvg\bva\mssnote{\Regius~12v/30}„Hversu snúnuðu yðr konur yðrar?“\eva

\bvb “How did your women pleasure (TODO!!!) you?.\footnoteB{Seemingly a prose line; see Introduction.}”\evb\evg


\bvg\bva\mssnote{\Regius~12v/30}„\alst{Sp}arkar ǫ́ttum vér konur \hld\ ef oss at \alst{sp}ǫkum yrði; &
\alst{h}orskar ǫ́ttum vér konur \hld\ ef oss \alst{h}ollar vę́ri, &
þę́r ór \alst{s}andi \hld\ \alst{s}íma undu &
\ind ok ór \alst{d}ali \alst{d}júpum &
\ind \alst{g}rund of \alst{g}rófu; &
varð’k þęim ęinn \alst{ǫ}llum \hld\ \alst{ø}fri at rǫ́ðum; &
\ind hvílda’k hjá \alst{s}ystrum \alst{s}jau &
\ind ok hafða’k \alst{g}ęð þęira allt ok \alst{g}aman; &
\ind hvat vannt-u þá meðan, Þórr?“\eva

\bvb “We had smart women if we found them pleasing; \\
we had clever women if they were \inx[C]{hold} toward us. \\
They wound a rope out of the sand, \\
\ind and out of a deep dale \\
\ind dug up the ground. \\
I alone became superior to them all in counsels, \\
\ind I rested beside those sisters seven, \\
\ind and had their senses all, and pleasure— \\
\ind what didst thou then meanwhile, Thunder?”\evb\evg


\bvg\bva\mssnote{\Regius~13r/2, \AM~1r/1 (l. 4b ff.)}%
„Ek drap \alst{Þ}jatsa, \hld\ hinn \alst{þ}rúð-móðga jǫtun, &
\alst{u}pp ek varp \alst{au}gum \hld\ \alst{A}ll-valda sonar &
\ind á þann hinn \alst{h}ęiða \alst{h}imin; &
þau ’ru \alst{m}ęrki \alst{m}ęst \hld\ \alst{m}inna verka, &
\ind þau’s allir męnn \edtext{\alst{s}íðan}{\Afootnote{om. \AM}} of \alst{s}é\emph{a}; &
\ind hvat vannt-u þá meðan, Hárbarðr?“\eva

\bvb “I slew \inx[C]{Thedse}, the strength-minded ettin; \\
Up I threw the eyes of Allwald’s son \ken*{= Thedse} \\
\ind onto the clear heaven. \\
Those are the greatest marks of my works, \\
\ind those which all men since may see\footnoteB{Here we seem to have a rare example of native Germanic star-lore. Is the exact constellation identifiable? TODO.}— \\
\ind what didst thou then meanwhile, Hoarbeard?”\evb\evg


\bvg\bva\mssnote{\Regius~13r/5, \AM~1r/1}%
„\alst{M}iklar \alst{m}an-vélar \hld\ hafða’k við \alst{m}yrk-riður &
\ind þá’s ek \alst{v}élta þę́r frá \alst{v}erum. &
\alst{H}arðan jǫtun \hld\ hugða’k \alst{H}lébarð vesa; &
\ind \alst{g}af hann mér \alst{g}amban-tęin &
\ind en ek \alst{v}élta hann ór \alst{v}iti.“\eva

\bvb “Great girl-tricks I had against \inx[C]{mirk-rideresses}, \\
\ind when I lured them away from men.\footnoteB{Alternatiely ‘away from [their] husbands’.  The \emph{riður} ‘(female) riders’ were witches thought to torment people and cause disease and suffering. See \Havamal\ 156 for discussion.} \\
A hard ettin I judged Leebeard to be; \\
\ind he gave me a \inx[C]{gombentoe}, \\
\ind but I tricked him out of his wits.”\evb\evg


\bvg\bva\mssnote{\Regius~13r/7, \AM~1r/3}%
„Illum huga launaðir þú \edtext{þá}{\Afootnote{om. \AM}} \alst{g}óðar \alst{g}jafar.“\eva

\bvb “With an evil heart didst thou then repay the good gift.”\evb\evg


\bvg\bva\mssnote{\Regius~13r/8, \AM~1r/4}%
„Þat hęfir \alst{ęi}k \hld\ es af \alst{a}nnarri skęfr; &
\ind umb \alst{s}ik es hvęrr í \alst{s}líku— &
\ind hvat vannt-u þá meðan, Þórr?“\eva

\bvb “The oak has that which it chafes from the other; \\
\ind each man is for himself in such— \\
\ind what didst thou then meanwhile, Thunder?”\evb\evg


\bvg\bva\mssnote{\Regius~13r/9, \AM~1r/4}%
„Ek vas \alst{au}str \hld\ ok \alst{jǫ}tna barða’k &
\alst{b}rúðir \alst{b}ǫl-vísar \hld\ es til \alst{b}jargs gingu; &
mikil myndi \alst{ę́}tt \alst{jǫ}tna \hld\ ef \alst{a}llir lifði, &
vę́tr myndi \alst{m}anna \hld\ undir \alst{M}ið-garði— &
\ind hvat vannt-u þá meðan, Hárbarðr?\eva

\bvb “I was in the east and bashed Ettins, \\
bale-wise brides who walked to the mountain. \\
Great would the line of ettins be if all lived, \\
naught would remain of men within Middenyard\footnoteB{Thunder is the defender of Middenyard (the home of men) against the Ettins.  For Thunder’s killing of women cf. sts. 37–39 below and \textcite{Lindow1988}.}— \\
what didst thou then meanwhile, Hoarbeard?”\evb\evg


\bvg\bva\mssnote{\Regius~13r/11, \AM~1r/6}%
„\alst{V}as’k á \alst{V}allandi \hld\ ok \alst{v}ígum fylgða’k, &
\alst{a}tta ek \alst{jǫ}frum \hld\ en \alst{a}ldri sę́tta’k; &
\alst{Ó}ðinn á \alst{ja}rla \hld\ þá’s í \alst{v}al falla &
\ind en \alst{Þ}órr á \alst{þ}rę́la kyn.“\eva

\bvb “I was in \inx[L]{Walland} and followed battles; \\
I provoked princes, but I never reconciled them. \\
Weden owns the earls which fall among the slain, \\
but Thunder owns the race of thralls.\footnoteB{Weden expresses an aristocratic disregard for lower life and life as mere life; where Thunder boasts of saving men, Weden sarcastically responds that he made them slay each other so that he could have the best of them for himself.}”\evb\evg


\bvg\bva\mssnote{\Regius~13r/13, \AM~1r/8}%
„\alst{Ó}-jafnt skipta \hld\ es þú myndir með \edtext{\alst{ǫ́}sum}{\Afootnote{\emph{ása} \AM}} liði &
\ind ef þú ę́ttir \alst{v}il-gi mikils \alst{v}ald.“\eva

\bvb “Thou wouldst unfairly deal out troops among the Eese,\\
\ind if thou hadst great enough power.” \evg


\bvg\bva\mssnote{\Regius~13r/14, \AM~1r/9}%
„Þórr á \alst{a}fl \alst{ǿ}rit \hld\ en \alst{ę}kki hjarta; &
af \alst{h}rę́ðslu ok \alst{h}ug-blęyði \hld\ \edtext{vas þér}{\Afootnote{\emph{þér vas} \Regius}} í \alst{h}andska troðit &
\ind ok \alst{þ}óttisk-a þú \alst{þ}á \alst{Þ}órr vesa; &
\alst{h}vár-ki þú þá þorðir \hld\ fyr \alst{h}rę́ðslu þinni &
\edtrans{hnjósa né \alst{f}ísa}{sneeze or fart}{\Afootnote{\emph{físa né hnjósa} ‘fart or sneeze’ \AM}} \hld\ svá’t \alst{F}jalarr hęyrði.“\eva

\bvb “Thunder has strength enough, but no heart. \\
For fear and heart-softness didst thou tread into a glove, \\
\ind and then seemedest thou not to be Thunder. \\
Thou daredest not—for thy fear— \\
sneeze or fart lest Feller should hear.\footnoteB{This story is also referenced in \Lokasenna\ 60, and is told in full in \Gylfaginning\ 45: Lock, Thunder, and his servants Thelve and Wrash had journeyed east for a long time when they came upon a large hall, with an opening on one end as wide as the building.  They rested inside, but in the middle of the night they were awakened by a great earthquake.  Thunder rose and led the party to a side-room to the right in the middle of the hall. He stayed closest to the opening with his hammer ready, while the terrified others were further inside.  At daybreak they left the hall and found the huge ettin \emph{Skrymir} (\inx[P]{Shrimer}) asleep outside.  His snoring had caused the earth-quakes, and the hall was his mitten; the side-room was its thumb.}”\evb\evg


\bvg\bva\mssnote{\Regius~13r/17, \AM~1r/11}%
„\alst{H}ár-barðr hinn ragi, \hld\ ek munda þik í \alst{h}ęl drepa &
\ind ef ek mę́tta \alst{s}ęilask of \edtext{\alst{s}und}{\Afootnote{\emph{sundit} \AM}}.“\eva

\bvb “O Hoarbeard the \inx[C]{queer}! I would strike thee into \inx[L]{Hell}, \\
\ind if I might sail o’er the sound!”\evb\evg


\bvg\bva\mssnote{\Regius~13r/18, \AM~1r/12}%
„Hvat \edtext{skyldir}{\Afootnote{\emph{skalt-u} \AM}} of \alst{s}und \alst{s}ęilask \hld\ es \edtext{\alst{s}akir}{\Afootnote{\emph{sakar} \AM}} ’ru alls øngar? &
\ind hvat vannt-u þá meðan, Þórr?“\eva

\bvb “Why should thou sail o’er the sound when the charges are none?— \\
\ind what didst thou then meanwhile, Thunder?”\evb\evg


\bvg\bva\mssnote{\Regius~13r/19, \AM~1r/13}%
„Ek vas \alst{au}str \hld\ ok \alst{á}na varða’k &
þá’s \edtext{mik \alst{s}óttu \hld\ þęir}{\Afootnote{\emph{þęir sóttu mik} \AM}} \alst{S}várangs synir; &
\alst{g}rjóti mik bǫrðu, \hld\ \alst{g}agni urðu \edtext{þó}{\Afootnote{om. \AM}} lítt fęgnir, &
þó urðu mik \alst{f}yrri \hld\ \alst{f}riðar at biðja— &
\ind hvat vannt-u þá meðan, Hárbarðr?“\eva

\bvb “I was in the east and guarded the river \\
when I was set upon by Sweering’s sons. \\
With rocks they bashed me, still they rejoiced little in victory; \\
still they had to beg me first for peace— \\
\ind what didst thou then meanwhile, Hoarbeard?”\evb\evg


\bvg\bva\mssnote{\Regius~13r/22, \AM~1r/15}%
„Ek vas \alst{au}str \hld\ ok við \edtext{\alst{ęi}n-hvęrja}{\Afootnote{\emph{‘æinhæriu’} \AM}} dǿmða’k, &
\alst{l}ék’k við ina \alst{l}ind-hvítu \hld\ ok \edtrans{\alst{l}aun-þing}{secret trysts}{\Afootnote{so \AM; \emph{laung þing} ‘long trysts’ \Regius}} háða’k, &
\alst{g}ladda’k ina \edtrans{\alst{g}ull-bjǫrtu}{gold-bright}{\Afootnote{\emph{gull-hvítu} ‘gold-white’ \AM}}, \hld\ \alst{g}amni mę́r unði.“\eva

\bvb “I was in the east and spoke with a certain woman; \\
I played with the linen-white, and held secret trysts: \\
I gladdened the gold-bright—the maiden enjoyed pleasure.”\evb\evg


\bvg\bva\mssnote{\Regius~13r/24, \AM~1r/17}„Góð ǫ́ttu þęir man-kynni þar þá.“\eva

\bvb “Then they had good girl-visits there.”\evb\evg


\bvg\bva\mssnote{\Regius~13r/24, \AM~1r/17}„\alst{L}iðs þíns \edtext{vę́ra’k}{\Afootnote{\emph{vas’k} \AM}} þá þurfi, Þórr, \hld\ at ek hęlda þęiri inni \alst{l}ín-hvítu męy.“\eva

\bvb “Of thy help would I have been in need then, Thunder, that I might hold that linen-white maiden.”\evb\evg


\bvg\bva\mssnote{\Regius~13r/25, \AM~1r/18}%
„Ek mynda þér \edtext{þá þat}{\Afootnote{\emph{þat þá} \AM}} \alst{v}ęita \hld\ ef ek \alst{v}iðr of \edtext{kǿmumk}{\Afootnote{\emph{kǿmisk} \Regius}}.“\eva

\bvb “I would then have granted thee that, if I were able.”\evb\evg


\bvg\bva\mssnote{\Regius~13r/26, \AM~1r/18}%
„Ek mynda þér þá \alst{t}rúa, \hld\ nema mik í \alst{t}ryggð véltir.“\eva

\bvb “I would then have trusted thee, unless thou wouldst betray my trust.”\evb\evg


\bvg\bva\mssnote{\Regius~13r/27, \AM~1r/19}%
„Em’k-at ek sá \alst{h}ę́l-bítr \hld\ sem \alst{h}úð-skór forn á vár.“\eva

\bvb “I’m not such a heel-biter as an old hide-shoe in spring.\footnoteB{Proverbial (a heel-biter being someone who betrays his companions); the old leather becoming stiff and chafed over the winter.}”\evb\evg


\bvg\bva\mssnote{\Regius~13r/28, \AM~1r/20}%
„Hvat vannt-u þá meðan, Þórr?“\eva

\bvb “What didst thou then meanwhile, Thunder?”\evb\evg


\bvg\bva\mssnote{\Regius~13r/28, \AM~1r/20}„\alst{B}rúðir \alst{b}er-sęrkja \hld\ \alst{b}arða’k í \edtext{Hlés-ęyju}{\Afootnote{\emph{Hlés-ęy} \AM}}; &
þę́r hǫfðu \alst{v}ęrst unnit, \hld \alst{v}élta þjóð alla.“\eva

\bvb “The brides of bearserks I bashed in Leesey; \\
they had done the worst thing: betrayed the whole nation.”\evb\evg


\bvg\bva\mssnote{\Regius~13r/29, \AM~1r/21}„\alst{K}lę́ki vannt-u þá, Þórr, \hld\ es þú \edtext{á}{\Afootnote{\emph{‘ǽ’} corr. \AM}} \alst{k}onum barðir.“\eva

\bvb “A disgrace didst thou then, Thunder, when thou didst bash women.”\evb\evg


\bvg\bva\mssnote{\Regius~13r/30, \AM~1r/22}%
„\alst{V}argynjur \edtext{vǫ́ru þę́r}{\Afootnote{\emph{þat vǫ́ru} \AM}} \hld\ en \alst{v}ar-la konur, &
\alst{sk}ęlldu \alst{sk}ip mitt \hld\ es \alst{sk}orðat hafða’k, &
\alst{ǿ}gðu \edtext{mér}{\Afootnote{add. \emph{þęim} \AM}} \alst{já}rn-lurki \hld\ en \alst{ę}ltu Þjálfa— &
\ind hvat vannt-u þá meðan, Hárbarðr?“\eva

\bvb “She-wolves were they, and hardly women; \\
they overturned my ship which I had propped, \\
terrorised me with an iron cudgel and chased \inx[P]{Thelve} around— \\
\ind what didst thou then meanwhile, Hoarbeard?”\evb\evg


\bvg\bva\mssnote{\Regius~13r/32, \AM~1r/23}%
„Ek vas’k í \alst{h}ęr’num \hld\ es \alst{h}ingat gørðisk &
\alst{g}nę́fa \alst{g}unn-fana, \hld\ \alst{g}ęir at rjóða.“\eva

\bvb “I was in the warband, when it readied itself hither \\
to raise the war-standard, to redden the spear.”\evb\evg


\bvg\bva\mssnote{\Regius~13v/1, \AM~1r/24}%
„Þęss vilt-u nú geta, es þú fórt oss \edtext{ó-ljúfan}{\Afootnote{\emph{‘óliyfan’} \AM; \emph{†olubann†} \Regius}} at bjóða!“\eva

\bvb “This wilt thou now mention, that thou didst journey to hurt us!”\evb\evg


\bvg\bva\mssnote{\Regius~13v/2, \AM~1r/25}%
„\alst{B}ǿta skal þér \edtext{þat þá}{\Afootnote{om. \AM}} \hld\ munda \alst{b}augi &
sem \alst{ja}fnęndr \alst{u}nnu \hld\ \edtext{þęir’s \alst{o}kkr vilja sę́tta}{\Afootnote{\emph{þęir’s okkr vilja sę́tt hafa} \AM}}.“\eva

\bvb “Then I shall repay thee for that with a hand-bigh, \\
bestowed by the mediators who wish to reconcile us two.”\evb\evg


\bvg\bva\mssnote{\Regius~13v/3, \AM~1r/26}%
„\alst{H}var namt þęssi \hld\ in \alst{h}nǿfi-ligu orð &
es \alst{h}ęyrða’k aldri-gi \hld\ \edtext{in}{\Afootnote{so \AM; om. \Regius}} \alst{h}nǿfi-ligri?“\eva

\bvb “Where didst thou learn these sarcastic words, \\
which I never heard more sarcastic?”\evb\evg


\bvg\bva\mssnote{\Regius~13v/5, \AM~1r/27}%
„Nam’k at \edtext{mǫnnum}{\Afootnote{om. \AM}} þęim inum aldr-ǿnum es búa í hęimis-skógum.“\eva

\bvb “I learned them from the old men who dwell in homely forests.”\evb\evg


\bvg\bva\mssnote{\Regius~13v/5, \AM~1v/1}%
„Þó gefr þú gótt nafn \edtrans{dysjum}{poor cairns}{\Bfootnote{A reference to Weden’s waking the dead, as attested e.g. in \Voluspa\ and \Baldrsdraumar.}}, es þú kallar þat hęimis-skóga.“\eva

\bvb “Yet thou givest a good name to poor cairns, when thou callest them homely forests.”\evb\evg


\bvg\bva\mssnote{\Regius~13v/6, \AM~1v/2}%
„Svá dǿmi’k \edtext{of}{\Afootnote{om. \AM}} slíkt far.“\eva

\bvb “So I speak about such matters.”\evb\evg


\bvg\bva\mssnote{\Regius~13v/7, \AM~1v/2}%
„\alst{O}rð-kringi þín \hld\ mun þér \alst{i}lla koma &
\ind ef ek rę́ð á \alst{v}ág at \alst{v}aða; &
\alst{u}lfi hę́rra \hld\ hygg’k \edtext{at \alst{ǿ}pa mynir}{\Afootnote{\emph{þik ǿpa munu} \AM}} &
\ind ef \alst{h}lýtr af \alst{h}amri \alst{h}ǫgg.“\eva

\bvb “Thy glibness of word will bring thee ill \\
\ind if I decide to wade on the wave! \\
Higher than a wolf I think thou wilt scream, \\
\ind if thou get a strike from the hammer.”\evb\evg


\bvg\bva\mssnote{\Regius~13v/9, \AM~1v/4}%
„Sif á \edtrans{\alst{h}ó}{lover}{\Bfootnote{Most translators take this acc. sg. word as an alternative form of \emph{hórr} m. ‘adulterer’ (gen. \emph{hórs}), containing the same root as \emph{hóra} f. ‘whore, prostitute’, \emph{hór} n. ‘adultery, fornication’, ModEngl. whore. The \emph{-r} has presumably been interpreted as the masc. nom. sg. ending, giving nom. \emph{*hór}, gen. \emph{*hós}. Further, this accusation is also found in \Lokasenna\ TODO, where Lock says that he has been Sib’s lover (\emph{hórr}). Notably, \CV\ interprets this word as the unrelated \emph{hór} m. ‘pot-hook’, “insinuating that Thor busied himself with cooking and dairy-work.” This seems very unlikely when considering Thunder’s response in the next verse: “I think that thou liest!” and the parallel in \Lokasenna.}} \alst{h}ęima, \hld\ \alst{h}ans munt fund vilja, &
\alst{þ}ann munt \alst{þ}ręk drýgja, \hld\ \alst{þ}at ’s þér \edtext{skyldara}{\Afootnote{\emph{skyldra} \AM}}.“\eva

\bvb “Sib has a lover at home; \emph{him} wilt thou wish to meet! \\
On him shalt thou use thy strength—that is more urgent for thee!”\evb\evg


\bvg\bva\mssnote{\Regius~13v/10, \AM~1v/5}%
„\alst{M}ę́lir þú at \alst{m}unns ráði \hld\ svá’t \alst{m}ér skyldi vęrst þikkja, &
\alst{h}alr inn \alst{h}ug-blauði, \hld\ \alst{h}ygg’k at þú ljúgir.“\eva

\bvb “Thou speakest to thy mouth’s counsel what should seem worst to me; \\
O heart-soft hero, I think thou liest!”\evb\evg


\bvg\bva\mssnote{\Regius~13v/12, \AM~1v/6}%
„\alst{S}att hygg’k \edtext{mik}{\Afootnote{\emph{þik} \AM}} \alst{s}ęgja, \hld\ \alst{s}ęinn ert at fǫr þinni, &
\alst{l}angt myndir nú kominn, Þórr, \hld\ ef þú \edtrans{\alst{l}itum fǿrir}{changed colour}{\Bfootnote{Unclear expression.}}.“\eva

\bvb “I think myself to speak truly, thou art late on thy journey; \\
far wouldst thou now be come, Thunder, if thou hadst changed colour.”\evb\evg


\bvg\bva\mssnote{\Regius~13v/14, \AM~1v/8}%
„\alst{H}árbarðr inn ragi, \hld\ \alst{h}ęldr hęfir nú mik \edtext{dvalðan}{\Afootnote{\emph{dvalit} \AM}}!“\eva

\bvb “Hoarbeard the queer; thou hast now much delayed me!”\evb\evg


\bvg\bva\mssnote{\Regius~13v/14, \AM~1v/8}%
„\edtext{\alst{Á}sa-Þórs}{\Afootnote{\emph{Ása-Þór} \AM}} \hld\ hugða’k \alst{a}ldri-gi myndu &
\ind glępja \alst{f}é-hirði \alst{f}arar.“\eva

\bvb “Eese-Thunder’s journey I never thought \\
\ind that a shepherd would divert.”\evb\evg


\bvg\bva\mssnote{\Regius~13v/15, \AM~1v/9}%
„\alst{R}áð mun’k þér nú \alst{r}áða: \hld\ \alst{r}ó hingat bátinum, &
\alst{h}ę́ttum \alst{h}ǿtingi, \hld\ \alst{h}itt fǫður Magna!“\eva

\bvb “I will now counsel thee a counsel: row the boat hither, \\
let us cease the taunting; meet the father of Main \ken*{= Thunder = me}!”\evb\evg


\bvg\bva\mssnote{\Regius~13v/17, \AM~1v/10}%
„\alst{F}ar þú \edtext{\alst{f}irr}{\Afootnote{\emph{frá} \AM}} sundi, \hld\ þér skal \alst{f}ars synja!“\eva

\bvb “Go far away from the sound; passage shall be denied thee!”\evb\evg


\bvg\bva\mssnote{\Regius~13v/17, \AM~1v/11}%
„\alst{V}ísa þú mér \edtext{nú}{\Afootnote{om. \AM}} lęiðina \hld\ alls þú vill mik ęigi of \alst{v}áginn fęrja!“\eva

\bvb “Show me now the way, since thou wilt not ferry me o’er the wave!”\evb\evg


\bvg\bva\mssnote{\Regius~13v/18, \AM~1v/11}%
„\alst{L}ítit ’s \edtext{at}{\Afootnote{om. \Regius}} synja, \hld\ \alst{l}angt ’s at fara; &
\alst{st}und ’s til \edtext{\alst{st}okks’ins}{\Afootnote{\emph{stokks} \AM}}, \hld\ ǫnnur til \edtext{\alst{st}ęins’ins}{\Afootnote{\emph{stęins} \AM}}, &
halt svá til \alst{v}instra \edtext{\alst{v}egs’ins}{\Afootnote{\emph{vegs} \AM}} \hld\ unds þú hittir \edtrans{\alst{V}er-land}{Wereland}{\Afootnote{\emph{Valland} \AM}\Bfootnote{The land of men.}}; &
\alst{þ}ar mun Fjǫrgyn \hld\ hitta \alst{Þ}ór, son sinn, &
ok mun hǫ́n kęnna hǫ́num \alst{ǫ́}ttunga brautir \hld\ til \alst{Ó}ðins landa.“\eva

\bvb “It is little to deny; it is long to journey: \\
an hour to the log, another to the stone;  \\
hold thus to the left road until thou findest Wereland;  \\
there will Firgyn find Thunder, her son, \\
and she will show him the ancestral roads to Weden’s lands \ken*{= Osyard}.”\evb\evg


\bvg\bva\mssnote{\Regius~13v/22, \AM~1v/14}%
„Mun’k taka þangat \edtext{í dag}{\Afootnote{\emph{á dęgi} \AM}}?“\eva

\bvb “Will I get there today?”\evb\evg


\bvg\bva\mssnote{\Regius~13v/22, \AM~1v/14}%
„Taka við víl \edtext{ok}{\Afootnote{\emph{við} \AM}} \alst{ę}rfiði \hld\ at \edtext{\alst{u}pp-vesandi}{\Afootnote{\emph{upp-rennandi} \AM}} sólu &
\ind es ek get þána.“\eva

\bvb “[Thou wilt] get there with toil and hardship at the rising of the sun, \\
\ind since I guess it be thawing.”\evb\evg


\bvg\bva\mssnote{\Regius~13v/23, \AM~1v/15}%
„\alst{Sk}ammt mun nú mál okkat vesa, \hld\ alls þú mér \alst{sk}ǿtingu ęinni svarar; &
launa mun ek þér \alst{f}ar-synjun \hld\ ef vit \alst{f}innumsk í sinn annat. &
Far þú nú þar’s þik hafi allan gramir!“\eva

\bvb “Short will now our speech be, since thou answerest me with scoffing alone. \\
I will reward thee for this ferry-denial if we meet another time. \\
Go now whither the fiends may have thee whole!”\evb\evg

\sectionline
