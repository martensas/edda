\bookStart{The Leed of Hoarbeard}[Hárbarðsljóð]

\begin{flushright}%
Dating \parencite{Sapp2022}: early C11th (0.578)–late C11th (0.377)

Meter: Unclear (TODO)%
\end{flushright}

In my opinion the poem can be seen as an allegory on class relations, namely between the self-owning Norwegian and later Icelandic farmers, and the warlike Norwegian earls.

Of all Eddic poems this one is probably the strangest in terms of form. Verse length varies greatly, and many of the lines (see especially the final verse) are of an obscene length reminiscent of late continental Germanic poems like the Heliand; some simply have no metrical qualities at all. The young clitic definite is (uniquely) employed frequently throughout the poem. These criteria would seem to point towards a late origin for the poem (though not later than the late C13th, when \Regius\ was written).

Against this late origin speaks the presence of rare words (e.g. \emph{ǫgurr} v. 13) and a thorough understanding of the personalities of the two gods which would seem unlikely to stem from several centuries after the conversion of Iceland. The model devised by Sapp gives the poem a 57.8\% likelihood of being from the early C11th, and a 37.7\% likelihood of being from the late 11th. These scores are most similar to those obtained by \Gripisspa, a poem that on the surface seems much more archaic.

What could we then be dealing with? It may of course be that the poem is heavily corrupt, but there is no good evidence for this (apart from the above-mentioned irregularities). Most lines are readily understandable and fit well both within their respective context and the poem as a whole. I think a better solution to this problem is to assume that the poem has been acted out as a sort of carnivalesque theatre, with two masked actors, each playing one of the gods. This would explain the variations in meter and line length, and the prose; some lines were simply shouted out, and the lack of alliteration in them would then have a powerful, discordant effect.

This is shown also by uses of the word ‘here’ in vv. 9 and 14. TODO: mention concept of "double scene" by Lars Lönnroth?


\sectionline


\bpg
\bpa\mssnote{\Regius~12r/30}Þórr fór ór austr-vegi ok kom at sundi einu. Ǫðrum megum sundsins var ferju-karlinn með skipit. Þórr kallaði:\epa

\bpb Thunder journeyed from the Eastern Way and came to a sound. At the other side of the sound was the ferryman with the ship. Thunder called out:\epb
\epg


\bvg
\bva\mssnote{\Regius~12r/32}„Hvęrr ’s sá svęinn svęina \hld\ es stęndr fyr sundit handan?“\eva

\bvb “Who is that swain of swains, that stands across the sound?”\evb
\evg


\bvg
\bva\mssnote{\Regius~12v/1}„Hvęrr ’s sá karl karla \hld\ es kallar of váginn?“\eva

\bvb “Who is that churl of churls, that calls out over the wave?”\evb
\evg


\bvg
\bva\mssnote{\Regius~12v/2}„Fęr þú mik of sundit, \hld\ fǿði’k þik á morgun; &
męis hęfi’k á baki, \hld\ verðr-a matrinn bętri. &
Át’k í hvíld \hld\ áðr ek hęiman fór, &
síldr ok \edtrans{hafra}{oatmeal/he-goats}{\Bfootnote{The easiest reading here is the acc. pl. of \emph{hafr} ‘he-goat’. Thunder also eats his goats in \Gylfaginning\ 44, where he butchers and cooks them in the evening and brings them back to life by blessing them with his hammer at dawn. \textcite{FinnurEdda} and \textcite{PettitEdda} prefer this reading; see also note to next stanza.—Many other scholars have here read an accusative plural of \emph{hafri} ‘oat’, i.e. ‘porridge, oatmeal’. Stiles (forthcoming TODO) connects this with Indrá’s (who is the Vedic equivalent of Thunder) “partner and yokemate” (\Rigveda\ 6.56.2) Pūṣán’s eating porridge (e.g. 6.56.1, 57.2). Another similarity Stiles notes between Thunder and Pūṣan is that both have chariots driven by goats (e.g. 6.57.3: “Goats are the draft-animals for the one”, 58.2: “Having goats as his horses”). Whether the Vedic tradition has split an original god into two or whether Thunder has absorbed elements of another god is hard to say.}}; \hld\ saðr em’k ęnn þęss.“\eva

\bvb {[Thunder quoth:]} “Ferry me over the sound, I feed thee in the morning! \\
A basket have I on my back; the food does not get better.\footnoteB{i.e. ‘you will not get better food than that.’} \\
I ate for a while before I journeyed from home, \\
herring and oatmeal/he-goats; I am still full from that.”\evb
\evg


\bvg
\bva\mssnote{\Regius~12v/5}„Ár-ligum verkum \hld\ hrósar þú, vęrðinum; \hld\ vęizt-at-tu fyr gǫrla, &
dǫpr ’ru þín hęim-kynni, \hld\ dauð hygg’k at þín móðir sé.“\eva

\bvb “Of early works boastest thou; of eating!\footnoteB{TODO. This is pretty difficult. From the previous stanza \emph{vęrðinum} seems to be referring to eating.} Thou knowest not clearly [what lies] before [thee]: \\
dismal is the state of thy home—dead I ween thy mother be!”\evb
\evg


\bvg
\bva\mssnote{\Regius~12v/6}„Þat sęgir þú nú \hld\ es hvęrjum þikkir &
męst at vita— \hld\ at mín móðir dauð sé.“\eva

\bvb “Thou now sayest that which to each man seems \\
most [important] to know: that my mother be dead!”\evb
\evg


\bvg
\bva\mssnote{\Regius~12v/8}„Þęygi ’s sem þú \hld\ þrjú bú ęigir góð; &
bęr-bęinn þú stęndr \hld\ ok hęfir brautinga gørvi, \hld\ þat-ki at þú hafir brę́kr þínar.“\eva

\bvb “’Tis hardly as if thou might own three good homesteads; \\
bare-legged thou standest, and hast the gear of a tramp; ’tis not even as if thou have thy own breeches!”\evb
\evg


\bvg
\bva\mssnote{\Regius~12v/9}„Stýrðu hingat ęikjunni, \hld\ ek mun þér stǫðna kęnna &
eða hvęrr á skipit \hld\ es þú hęldr við landit?“\eva

\bvb “Steer hither the boat! I will show thee to the harbour— \\
or who owns the ship which thou holdest by the shore?”\evb
\evg


\bvg
\bva\mssnote{\Regius~12v/11}„Hildólfr sá hęitir \hld\ es mik halda bað, &
rekkr inn ráð-svinni \hld\ es býr í Ráðs-ęyjar-sundi; &
bað-at hann hlęnni-męnn flytja \hld\ eða hrossa-þjófa, &
góða ęina \hld\ ok þá’s ek gørva kunna; &
sęg-ðu til nafns þíns \hld\ ef þú vill of sundit fara.“\eva

\bvb “Hildolf is he called who asked me to hold it, \\
the counsel-wise man who lives in Redeseysound. \\
He bade me not take highwaymen nor horse-thiefs; \\
good men only, and those whom I know well— \\
state thy name if thou wilt fare o’er the sound!”\evb
\evg


\bvg
\bva\mssnote{\Regius~12v/15}„Sęgja mun’k til nafns míns \hld\ þótt ek sękr sjá’k &
ok til alls øðlis: \hld\ Ek em Óðins sonr, &
Męila bróðir \hld\ ęn Magna faðir, &
þrúð-valdr goða \hld\ við Þór knátt-u hér dǿma! &
Hins vil’k nú spyrja \hld\ hvat þú hęitir?“\eva

\bvb “I will state my name—[and would] even if I were charged— \\
and all my origin: I am Weden’s son, \\
Male’s brother and Main’s father, \\
the strength-wielder of the Gods; with Thunder dost thou here speak! \\
This will I now ask, what thou art called?”\evb
\evg


\bvg
\bva\mssnote{\Regius~12v/18}„Hárbarðr ek hęiti, \hld\ hyl’k of nafn sjaldan.“\eva

\bvb “Hoarbeard I am called, seldom I conceal my name.”\evb
\evg


\bvg
\bva\mssnote{\Regius~12v/18}„Hvat skalt-u of nafn hylja \hld\ nema þú sakar ęigir?“\eva

\bvb “Why shalt thou conceal thy name, unless thou have charges?”\evb
\evg


\bvg
\bva\mssnote{\Regius~12v/19}„En þótt ek sakar ęiga \hld\ fyr slíkum sem þú est &
þá mun’k forða fjǫrvi mínu \hld\ nema ek fęigr sé.“\eva

\bvb “Even if I should have charges, for such a one as thou art \\
would I still protect by life, lest I be \inx[C]{fey}.”\evb
\evg


\bvg
\bva\mssnote{\Regius~12v/21}„Harm ljótan mér þikkir í því &
at vaða of váginn til þín \hld\ ok vę́ta \edtrans{ǫgur}{burden}{\Bfootnote{The sense of this word is not clear, though it is probably the same as the first element of the compound \emph{ǫgur-stund} ‘burdensome hour’, found in \Volundarkvida\ 42. Some authors have read it as a crude euphemism for ‘penis’, which would not be out of character for this poem. I however consider the best interpretation to be that of an author whose name I've forgotten (TODO!), namely that Thunder is referring to the food he carries on his back (cf. v. 3).}} minn; &
skylda’k launa kǫgur-sveini \hld\ þínum kangin-yrði \hld\ ef ek komumk yfir sundit.“\eva

\bvb “An ugly harm it seems to me \\
to wade o’er the wave to thee, and wet my burden. \\
I would repay thee, swaddle-swain, for thy mocking words, if myself I could bring over the sound.”\evb
\evg


\bvg
\bva\mssnote{\Regius~12v/23}„Hér mun’k standa \hld\ ok þín heðan bíða; &
fannt-a-tu mann inn harðara \hld\ at Hrungni dauðan.“\eva

\bvb “\emph{Here} will I stand, and \emph{hence} await thee;  \\
thou foundest not a harder man since the death of \inx[P]{Rungner}!\footnoteB{Rungner was an ettin slain by Thunder, TODO. Hoarbeard’s mentioning of him sets off a long interchange, wherein the two boast of their deeds, and ask what the other one was doing meanwhile.}”\evb
\evg


\bvg
\bva\mssnote{\Regius~12v/25}„Hins vilt-u nú geta \hld\ es vit Hrungnir dęildum, &
sá inn stór-úðgi jǫtunn, \hld\ es ór stęini vas hǫfuðit á, &
þó lét’k hann falla \hld\ ok fyr hníga; &
\ind hvat vannt-u þá meðan, Hárbarðr?“\eva

\bvb “This wilt thou now mention, of when I and Rungner dealt with each other, \\
that great-minded ettin on whom the head was made of stone.  \\
Yet I let him fall, and sink down before [me]— \\
what didst thou then meanwhile, Hoarbeard?”\evb
\evg


\bvg
\bva\mssnote{\Regius~12v/27}„Vas’k með Fjǫl-vari \hld\ fimm vetr alla &
í ęy þeiri \hld\ er Algrǿn hęitir; &
vega vér þar knǫ́ttum \hld\ ok val fęlla, &
margs at fręista, \hld\ mans at kosta.“\eva

\bvb “I was with Felwar for all of five winters \\
in that island which Allgreen is called. \\
There we knew to fight, and fell corpses; \\
many to tempt, a girl to win.\footnoteB{I read \emph{margs} ‘many a’ as modifying \emph{mans} ‘girl’, thus giving ‘(we knew) to tempt and to win many a girl’.}”\evb
\evg


\bvg
\bva\mssnote{\Regius~12v/30}„Hversu snúnuðu yðr konur yðrar?“\eva

\bvb “How did your women pleasure (TODO!!!) you?.\footnoteB{Seemingly a prose line; see Introduction.}”\evb
\evg


\bvg
\bva\mssnote{\Regius~12v/30}„Sparkar ǫ́ttum vér konur \hld\ ef oss at spǫkum yrði; &
horskar ǫ́ttum vér konur \hld\ ef oss hollar vę́ri, &
þę́r ór sandi \hld\ síma undu &
\ind ok ór dali djúpum &
\ind grund of grófu; &
varð’k þęim ęinn ǫllum \hld\ øfri at rǫ́ðum; &
\ind hvílda’k hjá systrum sjau &
\ind ok hafða’k gęð þęira allt ok gaman;
\ind hvat vannt-u þá meðan, Þórr?“\eva

\bvb “We \ken*{I} owned frisky women, if they were pleasing towards us \ken*{me}; \\
we \ken*{I} owned wise women, if they were \inx[C]{hold} towards us \ken*{me}; \\
out of the sand a rope they wound, \\
and out of a deep dale \\
dug up the ground; \\
I alone became superior to all of them in counsels; \\
I rested by those sisters seven, \\
and had their senses all, and pleasure— \\
what didst thou then meanwhile, Thunder?”\evb
\evg


\bvg
\bva\mssnote{\Regius~13r/2, \AM~1r/1 (l. 4b ff.)}„Ek drap Þjaza, \hld\ hinn þrúð-móðga jǫtun, &
upp ek varp augum \hld\ Allvalda sonar &
\ind á þann hinn hęiða himin; &
þau ’ru męrki męst \hld\ minna verka, &
\ind þau’s allir męnn síðan of sé; &
\ind hvat vannt-u þá meðan, Hárbarðr?“\eva

\bvb “I slew \inx[C]{Thedse}, the strength-minded ettin; \\
up I threw the eyes of Allwald’s son \ken*{= Thedse} \\
onto that bright heaven; \\
those are the greatest marks of my works, \\
those that all men since do see\footnoteB{Here we seem to have a rare example of native Germanic star-lore. Is the exact constellation identifiable? TODO.}— \\
what didst thou then meanwhile, Hoarbeard?”\evb
\evg


\bvg
\bva\mssnote{\Regius~13r/5, \AM~1r/1}„Miklar man-vélar \hld\ hafða’k við myrk-riður &
\ind þá’s ek vélta þę́r frá verum; &
harðan jǫtun \hld\ hugða’k Hlébarð vesa; &
\ind gaf hann mér gamban-tęin &
\ind en ek vélta hann ór viti.“\eva

\bvb “Great girl-tricks I used against \inx[C]{mirk-riders}, \\
when I tricked them away from their husbands.\footnoteB{Alternatiely ‘away from men’. The \emph{riður} ‘(female) riders’ were witches thought to torment people and cause disease and suffering. See \Havamal\ 156 for discussion.} \\
A hard ettin I judged Leebeard to be; he gave me a \inx[C]{gombentoe}, but I tricked him out of his wits.”\evb
\evg


\bvg
\bva\mssnote{\Regius~13r/7, \AM~1r/3}„Illum huga launaðir þú þá góðar gjafar.“\eva

\bvb “With an evil mind rewardedst thou that good gift.”\evb
\evg


\bvg
\bva\mssnote{\Regius~13r/8, \AM~1r/4}„Þat hęfir ęik \hld\ es af annarri skęfr; &
\ind umb sik es hvęrr í slíku; &
\ind hvat vannt-u þá meðan, Þórr?“\eva

\bvb “An oak has that which it shaves from another; \\
each [man] is for himself in such [a matter]— \\
what didst thou then meanwhile, Thunder?”\evb
\evg


\bvg
\bva\mssnote{\Regius~13r/9, \AM~1r/4}„Ek vas austr \hld\ ok jǫtna barða’k &
brúðir bǫl-vísar \hld\ es til bjargs gingu; &
mikil myndi ę́tt jǫtna \hld\ ef allir lifði, &
vę́tr myndi manna \hld\ undir Mið-garði; &
\ind hvat vannt-u þá meðan, Hárbarðr?\eva

\bvb “I was in the east, and ettins I fought; bale-wise brides who walked to the mountain. Great would the lineage of ettins be if all lived; naught would remain of men within Middenyard\footnoteB{A remarkable clear statement of purpose. This conception is far from unique to this verse; in \Hymiskvida\ 11, for instance, Thunder is described as “the opponent of Rooder”, “the friend of manly retinues” and “Wighward”, attesting his role in the slaying of ettins and the protection of men and their sanctuaries (\inx[C]{wigh}[wighs]). kenned as the wigh-ward (sanctuary-defender) of Middenyard. For Thunder’s killing of women cf. vv. 37–39 below and also}—what didst thou then meanwhile, Hoarbeard?”\evb
\evg


\bvg
\bva\mssnote{\Regius~13r/11, \AM~1r/6}„Vas’k á Vallandi \hld\ ok vígum fylgða’k, &
atta ek jǫfrum \hld\ en aldrigi sę́tta’k; &
Óðinn á jarla \hld\ þá’s í val falla &
\ind en Þórr á þrę́la kyn.“\eva

\bvb “I was in \inx[L]{Walland} and followed conflicts; I goaded princes on, but never reconciled them. Weden owns the earls which fall among the slain, but Thunder owns the kin of thralls.\footnoteB{We see here a sort of aristocratic, Odinic disregard for lower life and life as a good in itself; where Thunder boasts of saving men, Weden sarcastically responds that he caused the deaths of men so that he could have them for himself.}”\evb
\evg


\bvg
\bva\mssnote{\Regius~13r/13, \AM~1r/8}„Ójafnt skipta \hld\ es þú myndir með ǫ́sum liði &
\ind ef þú ę́ttir vilgi mikils vald.“\eva

\bvb “Translation.”\evb%TODO: There’s something very weird going on here.
\evg


\bvg
\bva\mssnote{\Regius~13r/14, \AM~1r/9}„Þórr á afl ǿrit \hld\ en ękki hjarta; &
af hrę́ðslu ok hug-blęyði \hld\ þér vas í hanzka troðit &
\ind ok þóttisk-a þú þá Þórr vesa; &
hvárki þá þorðir \hld\ fyr hrę́ðslu þinni &
hnjósa né físa \hld\ svá’t Fjalarr hęyrði.“\eva

\bvb “Thunder owns ample strength, but no heart; out of fear and mind-softness didst thou tread into a glove, and then seemedest thou not to be Thunder. Thou daredest neither—for thy fear—to sneeze nor to fart so that Feller might hear [it].\footnoteB{This story is also referenced in \Lokasenna\ TODO. It is elaborated heavily on in \Gylfaginning\ 45: Thunder, Lock, and the siblings Thelve and Wrash had travelled east for a long time when they discovered a large hall, with an opening on one end, as wide as the building. They took rest inside, but in the middle of the night there was a great earthquake and the ground beneath them trembled. Thunder rose and led the party to a side-room to the right in the middle of the hall. He sat closest to the opening with his hammer ready, while the others sat terrified further inside. At daybreak they left the hall and found a huge ettin named \emph{Skrymir} (\inx[P]{Shrimer}) sleeping next to them. His snoring had caused the earth-quakes, and the hall was his mitten; the side-room was the thumb-part.}”\evb
\evg


\bvg
\bva\mssnote{\Regius~13r/17, \AM~1r/11}„Hárbarðr hinn ragi, \hld\ munda’k þik í Hęl drepa &
\ind ef mę́tta’k sęilask of sund.“\eva

\bvb “Hoarbeard the \inx[C]{degenerate}, I would strike thee into \inx[L]{Hell}, if I might sail o’er the sound!”\evb
\evg


\bvg
\bva\mssnote{\Regius~13r/18, \AM~1r/12}„Hvat skyldir of sund sęilask \hld\ es sakir ’ru allz øngar? &
\ind hvat vannt-u þá meðan, Þórr?“\eva

\bvb “Why should thou sail o’er the sound when there are no offenses?—what didst thou then meanwhile, Thunder?”\evb
\evg


\bvg
\bva\mssnote{\Regius~13r/19, \AM~1r/13}„Ek vas austr \hld\ ok ána varða’k &
þá’s mik sóttu \hld\ þęir Svárangs synir; &
grjóti mik bǫrðu, \hld\ gagni urðu þó lítt fęgnir, &
þó urðu mik fyrri \hld\ friðar at biðja. &
\ind hvat vannt-u þá meðan, Hárbarðr?“\eva

\bvb “I was in the east, and warded the river, when the sons of Sweering attacked me. With rocks they fought me, yet they rejoiced little in victory; yet they earlier had to beg me for peace—what didst thou then meanwhile, Hoarbeard?”\evb
\evg


\bvg
\bva\mssnote{\Regius~13r/22, \AM~1r/15}„Ek vas austr \hld\ ok við ęin-hvęrja dǿmða’k, &
lék’k við ina lind-hvítu \hld\ ok lǫng þing háða’k, &
gladda’k ina gull-bjǫrtu, \hld\ gamni mę́r unði.“\eva

\bvb “I was in the east, and with a certain woman conversed; I played with the linen-white one, and held long-lasting trysts:\footnoteB{\emph{þing} (see \inx[C]{Thing}) usually means ‘legal assembly’, but clearly not here.} I gladdened the gold-bright one; the maiden enjoyed pleasure.”\evb
\evg


\bvg
\bva\mssnote{\Regius~13r/24, \AM~1r/17}„Góð ǫ́ttu þęir man-kynni þar þá.“\eva

\bvb “Then they had good girl-visits there.”\evb
\evg


\bvg
\bva\mssnote{\Regius~13r/24, \AM~1r/17}„Liðs þíns vę́ra’k þá þurfi, Þórr, \hld\ at hęlda’k þęiri inni lín-hvítu męy.“\eva

\bvb “Of thy help I might have been in need then, Thunder, that I might hold that linen-white maiden.”\evb
\evg


\bvg
\bva\mssnote{\Regius~13r/25, \AM~1r/18}„Ek mynda þér þat þá vęita \hld\ ef ek viðr of kę́misk.“\eva

\bvb “I would then have granted thee that, if I were able.”\evb
\evg


\bvg
\bva\mssnote{\Regius~13r/26, \AM~1r/18}„Ek mynda þér þá trúa, \hld\ nema mik í tryggð véltir.“\eva

\bvb “I would then have trusted thee, unless thou betrayed my trust.”\evb
\evg


\bvg
\bva\mssnote{\Regius~13r/27, \AM~1r/19}„Em’k-at ek sá hę́lbítr \hld\ sem húð-skór forn á vár.“\eva

\bvb “I am not such a heel-biter as an old hide-shoe in spring.\footnoteB{Proverbial (a heel-biter being someone who betrays his companions); the leather of a shoe would become very stiff and chafing over the winter.}”\evb
\evg


\bvg
\bva\mssnote{\Regius~13r/28, \AM~1r/20}\ind Hvat Shed þá meðan, Þórr?“\eva

\bvb “What didst thou then meanwhile, Thunder?”\evb
\evg


\bvg
\bva\mssnote{\Regius~13r/28, \AM~1r/20}„Brúðir ber-sęrkja \hld\ barða’k í Hlés-ęyju; &
þę́r hǫfðu vęrst unnit, \hld vélta þjóð alla.“\eva

\bvb “The brides of bearserks I fought in Leesie; they had done the worst thing: deceived a whole people.”\evb
\evg


\bvg
\bva\mssnote{\Regius~13r/29, \AM~1r/21}„Klę́ki  þá, Þórr, \hld\ es þú á konum barðir.“\eva

\bvb “A great disgrace didst thou then, Thunder, when thou foughtest women.”\evb
\evg


\bvg
\bva\mssnote{\Regius~13r/30, \AM~1r/22}„Vargynjur vǫ́ru þę́r \hld\ en varla konur, &
skęlldu skip mitt \hld\ es ek skorðat hafða’k, &
ǿgðu mér járn-lurki \hld\ en ęltu Þjálfa. &
\ind hvat vannt-u þá meðan, Hárbarðr?“\eva

\bvb “She-wolves were they, but hardly women; they knocked my ship which I had propped; frightened me with an iron-cudgel, but chased Thelve around—what didst thou then meanwhile, Hoarbeard?”\evb
\evg


\bvg
\bva\mssnote{\Regius~13r/32, \AM~1r/23}„Ek vas’k í hęrnum \hld\ es hingat gjǫrðisk &
gnę́fa gunn-fana, \hld\ gęir at rjóða.“\eva

\bvb “I was in the army, as hence it made ready to raise the war-standard, to redden the spear.”\evb
\evg


\bvg
\bva\mssnote{\Regius~13v/1, \AM~1r/24}„Þess vilt-u nú geta, es þú fórt oss \edtext{ó-ljúfan}{\Bfootnote{oliyfan \AM; †olubann† \Regius}} at bjóða!“\eva

\bvb “This wilt thou now mention, as thou wentest to bid us \ken*{= the Ease} hatred!”\evb
\evg


\bvg
\bva\mssnote{\Regius~13v/2, \AM~1r/25}„Bǿta skal þér þat þá \hld\ munda baugi &
sem jafnęndr unnu \hld\ þęir’s okkr vilja sę́tta.“\eva

\bvb “I will then restore thee for that with a hand-bigh, like the settlers [have] considered, those who wish to reconcile us two.”\evb
\evg


\bvg
\bva\mssnote{\Regius~13v/3, \AM~1r/26}„Hvar namt þęssi \hld\ in hnǿfi-ligu orð &
es hęyrða’k aldrigi \hld\ hnǿfi-ligri?“\eva

\bvb “Where learnedst thou these sarcastic words, which I never heard more sarcastic?”\evb
\evg


\bvg
\bva\mssnote{\Regius~13v/5, \AM~1r/27}„Nam’k at mǫnnum þęim inum aldrǿnum es búa í hęimis-skógum.“\eva

\bvb “I learned them from the old men who dwell in the home-forests.”\evb
\evg


\bvg
\bva\mssnote{\Regius~13v/5, \AM~1v/1}„Þó gefr þú gótt nafn dysjum, es þú kallar þat hęimis-skóga.“\eva

\bvb “Yet thou givest a good name to poor cairns,\footnoteB{cf. his waking the dead in various poems TODO.} as thou callest them home-forests.”\evb
\evg


\bvg
\bva\mssnote{\Regius~13v/6, \AM~1v/2}„Svá dǿmi’k of slíkt far.“\eva

\bvb “So I speak about such matters.”\evb
\evg


\bvg
\bva\mssnote{\Regius~13v/7, \AM~1v/2}„Orð-kringi þín \hld\ mun þér illa koma &
\ind ef ek rę́ð á vág at vaða; &
ulfi hę́ra \hld\ hygg’k at ǿpa mynir &
\ind ef hlýtr af hamri hǫgg.“\eva

\bvb “Thy word-glibness will bring thee evil, if I resolve to wade on the wave; higher than a wolf I think that thou wilt scream, if thou suffer a strike from the hammer.”\evb
\evg


\bvg
\bva\mssnote{\Regius~13v/9, \AM~1v/4}„Sif á \edtrans{hó}{lover}{\Bfootnote{Most translators take this acc. sg. word as an alternative form of \emph{hórr} m. ‘adulterer’ (gen. \emph{hórs}), containing the same root as \emph{hóra} f. ‘whore, prostitute’, \emph{hór} n. ‘adultery, fornication’, ModEngl. whore. The \emph{-r} has presumably been interpreted as the masc. nom. sg. ending, giving nom. \emph{*hór}, gen. \emph{*hós}. Further, this accusation is also found in \Lokasenna\ TODO, where Lock says that he has been Sib’s lover (\emph{hórr}). Notably, \CV\ interprets this word as the unrelated \emph{hór} m. ‘pot-hook’, “insinuating that Thor busied himself with cooking and dairy-work.” This seems very unlikely when considering Thunder’s response in the next verse: “I think that thou liest!” and the parallel in \Lokasenna.}} hęima, \hld\ hans munt fund vilja, &
þann munt þręk drýgja, \hld\ þat ’s þér skyldara.“\eva

\bvb “Sib has a lover at home; him wilt thou wish to meet! On that one shalt thou use thy strength—that befits thee more!”\evb
\evg


\bvg
\bva\mssnote{\Regius~13v/10, \AM~1v/5}„Mę́lir þú at munns ráði \hld\ svá’t mér skyldi vęrst þikkja, &
halr inn hug-blauði, \hld\ hygg’k at þú ljúgir.“\eva

\bvb “Thou speakest to the counsel of thy mouth that which would seem to me the worst; heart-soft man, I think that thou liest!”\evb
\evg


\bvg
\bva\mssnote{\Regius~13v/12, \AM~1v/6}„Satt hygg’k mik sęgja, \hld\ sęinn est at fǫr þinni, &
langt myndir nú kominn, Þórr, \hld\ ef þú \edtrans{litum fǿrir}{brought thy colours}{\Bfootnote{Very unclear expression. \emph{fǿra litum} TODO.}}.“\eva

\bvb “I think myself to speak truly: late art thou in thy journey; far would thou now be come, Thunder, if thou had brought thy colours.”\evb
\evg


\bvg
\bva\mssnote{\Regius~13v/14, \AM~1v/8}„Hárbarðr inn ragi, \hld\ hęldr hęfir nú mik dvalðan!“\eva

\bvb “Hoarbeard the degenerate; thou hast now delayed me greatly!”\evb
\evg


\bvg
\bva\mssnote{\Regius~13v/14, \AM~1v/8}„Ása-Þórs \hld\ hugða’k aldrigi myndu &
\ind glępja fé-hirði farar.“\eva

\bvb “The journey of Thunder of the Ease I never thought that a shepherd \ken*{= I} would divert.”\evb
\evg


\bvg
\bva\mssnote{\Regius~13v/15, \AM~1v/9}„Ráð mun’k þér nú ráða: \hld\ Ró þú hingat bátinum, &
hę́ttum hǿtingi, \hld\ hitt fǫður Magna!“\eva

\bvb “I will now give thee a counsel: Row hither the boat; seize with the taunting; come to the father of Main \ken*{= Thunder = me}!”\evb
\evg


\bvg
\bva\mssnote{\Regius~13v/17, \AM~1v/10}„Far þú firr sundi, \hld\ þér skal fars synja!“\eva

\bvb “Go far from the sound; the ferry shall be denied thee!”\evb
\evg


\bvg
\bva\mssnote{\Regius~13v/17, \AM~1v/11}„Vísa þú mér nú lęiðina \hld\ allz þú vill mik ęigi of váginn fęrja!“\eva

\bvb “Show me now the path, as thou wilt not ferry me o’er the wave!”\evb
\evg


\bvg
\bva\mssnote{\Regius~13v/18, \AM~1v/11}„Lítit ’s at synja, \hld\ langt ’s at fara; &
stund ’s til stokksins, \hld\ ǫnnur til stęinsins, &
halt svá til vinstra vegsins \hld\ unz þú hittir Ver-land; &
þar mun Fjǫrgyn \hld\ hitta Þór, son sinn, &
ok mun hǫ́n kęnna hǫ́num ǫ́ttunga brautir \hld\ til Óðins landa.“\eva

\bvb “’Tis little to deny, ’tis long to journey: an hour to the log, another to the stone; hold thus to the left road, until thou findest Wereland; there will Firgyn find Thunder, her son, and she will show him to the highways of her ancestors, to Weden’s lands \ken*{= Osyard}.”\evb
\evg


\bvg
\bva\mssnote{\Regius~13v/22, \AM~1v/14}„Mun’k taka þangat í dag?“\eva

\bvb “Will I come thither today?”\evb
\evg


\bvg
\bva\mssnote{\Regius~13v/22, \AM~1v/14}„Taka við víl ok ęrfiði \hld\ at upp-vesandi sólu &
es ek get þána.“\eva

\bvb “[Thou wilt] come with toil and hardship at the rising of the sun, as I think it is thawing.”\evb
\evg


\bvg
\bva\mssnote{\Regius~13v/23, \AM~1v/15}„Skammt mun nú mál okkat vesa, \hld\ allz þú mér skǿtingu ęinni svarar; &
launa mun ek þér far-synjun \hld\ ef vit finnumk í sinn annat. &
Far þú nú þar’s þik hafi allan gramir!“\eva

\bvb “Short will now our speech be, as thou answerest me with scoffing alone; I will reward thee for this ferry-denial if we meet another time. Now go, whither the fiends may have all of thee!”\evb
\evg
