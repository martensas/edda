\bookStart{First Lay of Guthrun}[Guð·rúnarkviða fyrsta]
\setBookCode{GudrunOne}

\begin{flushright}%
\textbf{Dating} \parencite{Sapp2022}: C10th (0.988)

\textbf{Meter:} \Fornyrdislag%
\end{flushright}

\section{Introduction}

After Siward’s death Guthrun is so upset that she cannot make herself weep.

\sectionline

\section{From the Death of Siward (\emph{Frá dauða Sig·urðar})}

\bpg\bpa Hér er sagt í þessi kviðu frá dauða Sig·urðar ok víkr hér svá til sem þeir drę́pi hann úti. En sumir segja svá at þeir drę́pi hann inni í rekkju sinni sofanda. En þýðverskir menn segja svá at þeir drę́pi hann úti í skógi ok svá segir í Guð·rúnar kviðu inni fornu at Sig·urðr ok Gjúka synir hefði til þings riðit þá er hann var drepinn—en þat segja allir einnig at þeir sviku hann í tryggð ok vógu at hánum liggjanda ok ó·búnum.
Guð·rún sat yfir Sig·urði dauðum. Hon grét eigi sem aðrar konur en hon var búin til at springa af harmi. Til gengu bę́ði konur ok karlar at hugga hana en þat var eigi auð-velt. Þat er sǫgn manna at Guð·rún hefði etit af Fáfnis hjarta ok hon skilði því fugls rǫdd. Þetta er enn kveðit um Guð·rúnu:\epa

\bpb Here it is said in this lay about the death of Siward, and it is at this point that they slew him outside. But some say that they slew him inside in his chamber asleep. But German men say that they slew him outside in the forest, and so it says in the Ancientr Lay of Guthrun that Siward and the sons of Yivick had ridden to the Thing when he was slain—but this they all say in agreement that they betrayed him while he trusted them, and struck at him lying and unarmed.
Guthrun sat over Siward, dead. She did not weep like other women, but she was ready to burst apart from sorrow. Both women and men came to her to console her, but that was not easily done. It is the saying of men that Guthrun had eaten of Fathomer’s heart, and she therefore understood the speech of birds. This is further said about Guthrun:\epb\epg

\sectionline

\section{The First Lay of Guthrun}

\bvg\bva%
Ár vas þat’s \alst{G}uð·rún \hld\ \alst{g}ørðisk at dęyja, &
es hǫ́n \alst{s}at \alst{s}org-full \hld\ yfir \alst{S}ig·urði, &
gørði-t hǫ́n \alst{h}júfra \hld\ né \alst{h}ǫndum sláa &
né \alst{k}vęina of \hld\ sem \alst{k}onur aðrar.\eva

\bvb It was of yore that Guthrun made ready to die \\
as she sat sorrowful over Siward. \\
She did not pant nor beat her hands \\
nor wail over him like other women.\evb\evg


\bvg\bva%
Gingu \alst{ja}rlar \hld\ \alst{a}l-snotrir framm, &
þęir’s \alst{h}arðs \alst{h}ugar \hld\ \alst{h}ana lǫttu; &
þęygi \alst{G}uð·rún \hld\ \alst{g}ráta mátti, &
svá vas hǫ́n \alst{m}óðug; \hld\ \alst{m}undi hǫ́n springa.\eva

\bvb Earls went all-clever forth, \\
they who would loosen her hard heart. \\
Yet nowise could Guthrun weep, \\
so moody was she—she would burst apart.\evb\evg


\bvg\bva%
Sǫ́tu \alst{í}trar \hld\ \alst{ja}rla brúðir &
\alst{g}olli búnar \hld\ fyr \alst{G}uð·rúnu; &
hvęr \alst{s}agði þęira \hld\ \alst{s}ínn of-trega &
þann’s \alst{b}itrastan \hld\ of \alst{b}eðit hafði.\eva

\bvb The splendid brides of the earls sat \\
adorned with gold before Guthrun. \\
Each one of them told her own great sorrow, \\
the bitterest one that she had suffered.\evb\evg


\bvg\bva%
Þȧ kvað \alst{G}jaf·laug, \hld\ \alst{G}júka systir: &
„\alst{M}ik vęit’k á \alst{m}oldu \hld\ \alst{m}unar-lausasta; &
hęfi’k \alst{f}imm vera \hld\ \alst{f}or-spell beðit, &
tvęggja dǿtra, \hld\ þriggja systra, &
\alst{á}tta brǿðra, \hld\ þó ek \alst{ęi}n lifi.“\eva

\bvb Then quoth Yeflie, Yivick’s sister: \\
“I know myself on the earth to be the most joyless. \\
Of five husbands have I suffered the loss, \\
of two daughters, three sisters, \\
eight brothers—yet I alone live.”\evb\evg


\bvg\bva%
Þęygi \alst{G}uð·rún \hld\ \alst{g}ráta mátti; &
svá vas hǫ́n \alst{m}óðug \hld\ at \alst{m}ǫg dauðan &
ok \alst{h}arð-\alst{h}uguð \hld\ of \alst{h}rør fylkis.\eva

\bvb Yet nowise could Guthrun weep; \\
so moody was she after the lad’s death, \\
and hard-hearted over the marshal’s corpse.\evb\evg


\bvg\bva%
Þȧ kvað þat \alst{H}ęr·borg, \hld\ \alst{H}úna-lands dróttning: &
„\alst{H}ęfi’k \alst{h}arðara \hld\ \alst{h}arm at sęgja: &
mínir \alst{s}jau \alst{s}ynir \hld\ \alst{s}unnan lands, &
\alst{v}err inn átti, \hld\ ï \alst{v}al fellu.\eva

\bvb Then quoth this Harburg, Hunland’s queen: \\
“I have a harder harm to tell. \\
My seven sons to the south of their land, \\
—my husband eighth—in battle fell.”\evb\evg


\bvg\bva%
\alst{F}aðir ok móðir, \hld\ \alst{f}jórir brǿðr, &
þau ȧ \alst{v}ági \hld\ \alst{v}indr of lék, &
\alst{b}arði \alst{b}ára \hld\ við \alst{b}orð-þili.\eva

\bvb My father and mother, four brothers— \\
them on the wave the wind outplayed; \\
the breaker beat against the ship-side.\evb\evg


\bvg\bva%
Sjǫlf skylda’k \alst{g}ǫfga, \hld\ sjǫlf skylda’k \alst{g}ǫtva, &
sjǫlf skylda’k \alst{h}ǫndla, \hld\ \edtext{\alst{h}ęl-fǫr}{\Afootnote{emend.; \emph{hęr-fǫr} \Regius}} þęira; &
þat ek \alst{a}llt of bęið \hld\ \alst{ęi}n misseri &
svá’t \alst{m}ér \alst{m}aðr ęngi \hld\ \alst{m}unar lęitaði.\eva

\bvb I alone had to honour them; I alone had to bury them; \\
I alone had to handle their hell-journey \ken{death}. \\
This all I suffered in one half-year, \\
while noone found me any joy.\evb\evg


\bvg\bva%
Þȧ varð’k \alst{h}apta \hld\ ok \alst{h}ęr-numa &
\alst{s}ams misseris \hld\ \alst{s}íðan-verða; &
\alst{sk}ylda’k \alst{sk}ręyta \hld\ ok \alst{sk}úa binda &
\alst{h}ęrsis kván \hld\ \alst{h}vęrjan morgin.\eva

\bvb Then I became a captive and taken in war, \\
in the latter part of that same half-year. \\
I had to dress and bind the shoes \\
of the ruler’s wife every morning.\evb\evg


\bvg\bva%
Hǫ́n \alst{ǿ}gði mér \hld\ af \alst{a}f-brýði &
ok \alst{h}ǫrðum mik \hld\ \alst{h}ǫggum kęyrði; &
fann’k \alst{h}ús-guma \hld\ \alst{h}vęrgi inn bętra &
en \alst{h}ús-fręyju \hld\ \alst{h}vęrgi verri.“\eva

\bvb She tortured me out of jealousy \\
and with hard blows drove me on. \\
A husband nowhere I’ve met better, \\
but a housewife nowhere worse.”\evb\evg


\bvg\bva%
Þęygi \alst{G}uð·rún \hld\ \alst{g}ráta mátti; &
svá vas hǫ́n \alst{m}óðug \hld\ at \alst{m}ǫg dauðan &
ok \alst{h}arð-\alst{h}uguð \hld\ of \alst{h}rør fylkis.\eva

\bvb Yet nowise could Guthrun weep; \\
so moody was she after the lad’s death, \\
and hard-hearted over the marshal’s corpse.\evb\evg


\bvg\bva%
Þȧ kvað þat \alst{G}ullrǫnd, \hld\ \alst{G}júka dóttir: &
„\alst{F}ǫ́ kannt, \alst{f}óstra, \hld\ þótt \alst{f}róð séir, &
\alst{u}ngu vífi \hld\ \alst{a}nd-spjǫll bera.“ &
Varaði hǫ́n at \alst{h}ylja \hld\ of \alst{h}rør fylkis.\eva

\bvb Then quoth this Goldrand, Yivick’s daughter: \\
“Little canst thou, foster-mother—though thou be wise— \\
to a young wife give answers.”— \\
She bade them uncover the marshal’s corpse.\evb\evg


\bvg\bva%
\alst{S}vipti hǫ́n blę́ju \hld\ af \alst{S}ig·urði &
ok \alst{v}att \alst{v}ęngi \hld\ fyr \alst{v}ífs \edtext{kn\emph{éu}m}{\Afootnote{metr. emend. by restoration of old hiatus form; \emph{knjám} \Regius}}: &
„\alst{L}ít-tu ȧ \alst{l}júfan, \hld\ \alst{l}ęgg þú munn við grǫn &
sem þú \alst{h}alsaðir \hld\ \alst{h}ęilan stilli.“\eva

\bvb She drew the shroud off of Siward \\
and turned his cheeks before the wife’s knees: \\
“Look on thy beloved! Lay thy mouth to his lips \\
like thou didst embrace the hale prince.”\evb\evg


\bvg\bva\alst{Ȧ} lęit Guð·rún \hld\ \alst{ęi}nu sinni; &
sá hǫ́n \alst{d}ǫglings skǫr \hld\ \alst{d}ręyra runna, &
\alst{f}ránar sjónir \hld\ \alst{f}ylkis liðnar, &
\alst{h}ug-borg jǫfurs \hld\ \alst{h}jǫrvi skorna.\eva

\bvb On him looked Guthrun a single time; \\
she saw the noble’s locks run with blood, \\
the gleaming gaze of the marshal gone, \\
the heart-fort \ken{chest} of the ruler cut by sword.\evb\evg


\bvg\bva%
Þȧ \alst{h}né Guð·rún \hld\ \alst{h}ǫll við bólstri; &
\alst{h}addr losnaði, \hld\ \alst{h}lýr roðnaði &
en \alst{r}egns dropi \hld\ \alst{r}ann niðr umb kné.\eva

\bvb Then Guthrun sank down, slooped against the bolster; \\
her hair loosened, her cheek reddened, \\
and a rain drop ran down to her knee.\evb\evg


\bvg\bva%
Þȧ \alst{g}rét \alst{G}uð·rún, \hld\ \alst{G}júka dóttir, &
svá’t \alst{t}ǫ́r flugu \hld\ \edtrans{\alst{t}resk}{veil(?)}{\Bfootnote{A guess translation; this word is an unexplained \emph{hapax}.}} ï gǫgnum &
ok \alst{g}ullu við \hld\ \alst{g}ę̇ss ï túni, &
\alst{m}ę́rir fuglar \hld\ es \alst{m}ę́r átti.\eva

\bvb Then wept Guthrun, Yivick’s daughter, \\
so that the tears flew through her veil(?) \\
and in response shrieked the geese in the yard, \\
the famèd fowls which the maiden owned.\evb\evg


\bvg\bva%
Þȧ kvað þat \alst{G}ullrǫnd, \hld\ \alst{G}júka dóttir: &
„\alst{y}kkar vissa’k \hld\ \alst{ȧ}stir męstar &
\alst{m}anna allra \hld\ fyr \alst{m}old ofan; &
\alst{u}nðir þú hvárki \hld\ \alst{ú}ti né inni, &
\alst{s}ystir mín, \hld\ nema hjá \alst{S}ig·urði.“\eva

\bvb Then quoth this Goldrand, Yivick’s daughter: \\
“I knew the love between you two was the greatest \\
of all men above the earth. \\
Thou wast never content outside or inside, \\
O sister of mine, save by Siward’s side.”\evb\evg


\bvg\bva%
„\edtext{\alst{S}vá vas mïnn \alst{S}ig·urðr \hld\ hjá \alst{s}onum Gjúka &
sęm vę́ri \edtrans{\alst{g}ęir-laukr}{garlic}{\Bfootnote{Or ‘spear-leek’. I have opted for this translation based on etymology (cf. OE \emph{gâr-léac} ‘spear-leek’), but the botanical identity is unclear. \textlink{GudrunTwo} 2 has \emph{grǿnn laukr} ‘green leek’ instead. For the cultural importance of leeks and onions see note to \textlink{Voluspa} 4.}} \hld\ ór \alst{g}rasi vaxinn,}{\lemma{Svá vas \dots\ vaxinn ‘So was \dots\ grown’}\Bfootnote{These two lines are almost identical to \textlink{GudrunTwo} 2/1–2. Since the present poem is probably older \parencite{Sapp2022}, it is likely the source.}} &
\edtext{eða vę́ri \alst{b}jartr stęinn \hld\ ȧ \alst{b}and dręginn: &
\alst{j}arkna-stęinn \hld\ yfir \alst{ǫ}ðlingum.}{\lemma{eða vę́ri \dots\ ǫðlingum. ‘or were \dots\ athlings.’}\Bfootnote{Beaded necklaces were commonly worn by Scandinavian women of the time, and the beads were mostly of opaque coloured glass.  Siward is likened to a bright crystal, the sons of Yivick to dull glass.}}\eva

\bvb\speakernoteb{[Guthrun quoth:]}%
“So was my Siward beside the sons of Yivick \\
like were a garlic out of grass grown, \\
or were a bright stone on a string drawn: \\
an \inx[C]{arkenstone} over the athlings.\evb\evg


\bvg\bva%
Ek \alst{þ}ȯtta auk \hld\ \alst{þ}jóðans rekkum &
\alst{h}vęrri \alst{h}ę́rri \hld\ \alst{H}ęrjans dísi; &
nú em’k svá \alst{l}ítil \hld\ sem \alst{l}auf séa &
\alst{o}pt ï \alst{jǫ}lstrum \hld\ at \alst{jǫ}fur dauðan.\eva

\bvb I also seemed to the ruler’s champions \\
higher than any of the Lord of Hosts’ dises \ken{walkirries}. \\
Now am I so small as if a leaf I were, \\
high in the willows, after the ruler’s death.\evb\evg


\bvg\bva%
\alst{S}akna’k ï \alst{s}essi \hld\ ok ï \alst{s}ę́ingu &
\alst{m}ïns \alst{m}ál-vinar— \hld\ valda \alst{m}ęgir Gjúka; &
valda \alst{m}ęgir Gjúka \hld\ \alst{m}ínu bǫlvi &
ok \alst{s}ystr \alst{s}innar \hld\ \alst{s}ǫ́rum gráti.\eva

\bvb I miss in the seat and in the bed \\
my confidant—at fault are the lads of Yivick; \\
the lads of Yivick are at fault for my bale \\
and for their sister’s [my] bitter weeping.\evb\evg


\bvg\bva%
Svá ér of \alst{l}ýða \hld\ \alst{l}andi ęyðið &
sem ér of \alst{u}nnuð \hld\ \alst{ęi}ða svarða; &
man-a þú, \alst{G}unn·arr, \hld\ \alst{g}olls of njóta; &
\edtrans{þęir munu þér \alst{b}augar \hld\ at \alst{b}ana verða}{those bighs will for thee become the bane}{\Bfootnote{I.e. “the wealh will be the end of you”.  Formulaic; cf. \textlink{Fafnismal} 9, 20.}} &
es þú \alst{S}ig·urði \hld\ \alst{s}varðir ęiða.\eva

\bvb So may ye make the land deserted by folk \\
like ye treated the sworn oaths! \\
Thou wilt not, Guther, enjoy the gold; \\
those bighs will for thee become the bane \\
on which thou to Siward didst swear the oaths.\evb\evg


\bvg\bva%
Opt vas ï \alst{t}u̇ni \hld\ \alst{t}ęiti męiri &
þȧ’s mïnn \alst{S}ig·urðr \hld\ \alst{s}ǫðladi Grana. &
ok þęir \alst{B}ryn·hildar \hld\ \alst{b}iðja fóru. &
\alst{a}rmrar vę́ttar \hld\ \alst{i}llu hęilli.“\eva

\bvb Oft in the courtyard there was greater cheer \\
when my Siward saddled Grane \\
and they journeyed to ask for Byrnhild’s hand, \\
that wretched wight of ill omen.”\evb\evg


\bvg\bva%
Þȧ kvað þat \alst{B}ryn·hildr \hld\ \alst{B}uðla dóttir: &
„\alst{v}ǫn sé sú \alst{v}ę́ttr \hld\ \alst{v}ers ok barna &
es þik \alst{G}uð·ru̇n \hld\ \alst{g}ráts of bęiddi &
ok þér ï \alst{m}orgun \hld\ \alst{m}ál-ru̇nar gaf.“\eva

\bvb Then quoth this Byrnhild, Budle’s daughter: \evb\evg


\bvg\bva%
Þȧ kvað þat \alst{G}ull·rǫnd \hld\ \alst{G}júka dóttir: &
„\alst{þ}ęgi þú, \alst{þ}jóð-lęið, \hld\ \alst{þ}ęira orða! &
\alst{U}rðr \alst{ǫ}ðlinga \hld\ hęfir þú \alst{ę́} vesit; &
rekr þik \alst{a}lda hvęrr \hld\ \alst{i}llrar skępnu &
\alst{s}org \alst{s}ára \hld\ \alst{s}jau konunga &
ok \alst{v}in-spell \hld\ \alst{v}ífa męst.“\eva

\bvb Then quoth this Goldrand, Yivick’s daughter:\evb\evg


\bvg\bva%
Þȧ kvað þat \alst{B}ryn·hildr \hld\ \alst{B}uðla dóttir: &
„vęldr ęinn \alst{A}tli \hld\ \alst{ǫ}llu bǫlvi &
of \alst{b}orinn \alst{B}uðla \hld\ \alst{b}róðir mïnn &
þȧ’s vit ï \alst{h}ǫll \hld\ \alst{h}u̇nskrar þjóðar &
\alst{ę}ld ȧ \alst{jǫ}fri \hld\ \alst{o}rm-bęðs litum; &
þęss hęfi’k \alst{g}angs \hld\ \alst{g}oldit síðan &
þęirar \alst{s}ýnar \hld\ \alst{s}ǫ́umk ęy.“\eva

\bvb Then quoth this Byrnhild, Budle’s daughter: \evb\evg


\bvg\bva%
\alst{St}óð hȯn und \alst{st}oð \hld\ \alst{st}ręngði hȯn ęlfi, &
\alst{b}rann \alst{B}rynhildi \hld\ \alst{B}uðla dóttur &
\alst{ę}ldr ór \alst{au}gum; \hld\ \alst{ęi}tri fnę́sti &
es hȯn \alst{s}ǫ́r of lęit \hld\ ȧ \alst{S}ig·urði.\eva

\bvb She stood beneath a pillar, strengthened her anger; \\
on Byrnhild Budle’s daughter burned \\
a fire from her eyes; she spit venom \\
when she beheld the wounds upon Siward.\evb\evg


\bpg\bpa%
Guð·rún gekk þaðan á braut til skógar á eyði-merkr ok fór allt til Danmarkar ok var þar með Þóru, Hákonar dóttur, sjau misseri.  Bryn·hildr vildi eigi lifa eptir Sig·urð.  Hon lét drepa þrę́la sína átta ok fimm ambóttir, þá lagði hon sik sverði til bana svá sem segir í Sig·urðar kviðu inni skǫmmu.\epa

\bpb Guthrun went away thence to the wood in the wilderness and journeyed all the way to Denmark and stayed there with Thure, Hathkin’s daughter, for seven half-years.  Byrnhild did not want to live after Siward.  She had her eight thralls and five handmaids slain; then she ran herself through with a sword unto her death, as it says in the Short Lay of Siward.\epb\epg

\sectionline
