\bookStart{The First Lay of Guthrun}[Guðrúnarkviða fyrsta]

\begin{flushright}%
\textbf{Dating} \parencite{Sapp2022}: C10th (0.988)

\textbf{Meter:} \Fornyrdislag%
\end{flushright}

After Siward’s death Guthrun is so upset that she cannot make herself weep.

\sectionline

\bvg\bva Ár vas þat’s \alst{G}uðrún \hld\ \alst{g}ørðisk at dęyja, &
es hǫ́n \alst{s}at \alst{s}org-full \hld\ yfir \alst{S}igurði, &
gørði-t hǫ́n \alst{h}júfra \hld\ né \alst{h}ǫndum sláa &
né \alst{k}vęina umb \hld\ sem \alst{k}onur aðrar.\eva

\bvb TODO.\evb\evg


\bvg\bva Gingu \alst{ja}rlar \hld\ \alst{a}l-snotrir framm, &
þęir’s \alst{h}arðs \alst{h}ugar \hld\ \alst{h}ana lǫttu; &
þęygi \alst{G}uðrún \hld\ \alst{g}ráta mátti, &
svá vas hǫ́n \alst{m}óðug; \hld\ \alst{m}undi hǫ́n springa.\eva

\bvb TODO... \\
Nowise could Guthrun weep, \\
so moody was she—she would burst apart.\evb\evg


\bvg\bva Sǫ́tu \alst{í}trar \hld\ \alst{ja}rla brúðir &
\alst{g}olli búnar \hld\ fyr \alst{G}uðrúnu; &
hvęr \alst{s}agði þęira \hld\ \alst{s}ínn of-trega &
þann’s \alst{b}itrastan \hld\ of \alst{b}eðit hafði.\eva

\bvb Sat the splendid brides of earls \\
adorned with gold, before Guthrun. \\
Each one of them told her great sorrow, \\
the most bitter one she had suffered.\evb\evg


\bvg\bva Þá kvað Gjaflaug, \hld\ Gjúka systir: &
„Mik vęit’k á moldu \hld\ munar-lausasta; &
hęfi’k fimm vera \hld\ for-spell beðit, &
tvęggja dǿtra, \hld\ þriggja systra, &
átta brǿðra, \hld\ þó ek ęin lifi.“\eva

\bvb Then quoth Yeflie, Yivick’s sister: \\
“I know myself on the earth the most joyless; \\
of five husbands I have suffered the loss, \\
of two daughters, three sisters, \\
eight brothers—yet I alone live.”\evb\evg


\bvg\bva Þęygi Guðrún \hld\ gráta mátti; &
svá vas hǫ́n móðug \hld\ at mǫg dauðan &
ok harð-huguð \hld\ um hrør fylkis.\eva

\bvb Nowise could Guthrun weep; \\
so moody was she after the lad’s death, \\
and hard-minded over the marshaller’s corpse.\evb\evg


\bvg\bva Þá kvað þat Hęrborg, \hld\ Húna lands dróttning: &
„Hęfi’k harðara \hld\ harm at sęgja: &
mínir sjau synir \hld\ sunnan lands, &
verr inn átti, \hld\ í val fellu.\eva

\bvb Then quoth this Harbury, queen of Hunland: \\
“I have a harder harm to tell. \\
My seven sons south of the land, \\
—my husband the eighth—in battle fell.”\evb\evg


\bvg\bva Faðir ok móðir, \hld\ fjórir brǿðr,
þau á vági \hld\ vindr of lék,
barði bára \hld\ við borð-þili.\eva

\bvb My father and mother, four brothers— \\
them on the wave the wind outplayed; \\
the breaker beat over the ship-sides.\evb\evg


\bvg\bva Sjǫlf skylda’k gǫfga, \hld\ sjǫlf skylda’k gǫtva, &
sjǫlf skylda’k hǫndla, \hld\ hęr-fǫr þęira; &
þat ek allt of bęið \hld\ ęin misseri &
svá’t mér maðr ęngi \hld\ munar lęitaði.\eva

\bvb I alone had to honour them; I alone had to bury them; \\
I alone had to handle their hell-journey \ken{death}. \\
This all I suffered in one half-year, \\
when no man found me any joy.\evb\evg


\bvg\bva Þá varð’k hapta \hld\ ok hęr-numa &
sams misseris \hld\ síðan verða; &
skylda’k skręyta \hld\ ok skúa binda &
hęrsis kván \hld\ hvęrjan morgin.\eva

\bvb Then I became a captive and war-taken, \\
in the same half-year afterwards. \\
I had to dress and bind the shoes \\
of the ruler’s wife every morning.\evb\evg


\bvg\bva Hon ǿgði mér \hld\ af af-brýði &
ok hǫrðum mik \hld\ hǫggum kęyrði; &
fann’k hús-guma \hld\ hvęrgi inn bętra &
en hús-fręyju \hld\ hvęrgi verri.“\eva

\bvb She tortured me out of jealousy, \\
and with hard blows drove me on; \\
a husband I never found better, \\
and a housewife never worse.”\evb\evg


\bvg\bva Þęygi Guðrún \hld\ gráta mátti; &
svá vas hǫ́n móðug \hld\ at mǫg dauðan &
ok harð-huguð \hld\ um hrør fylkis.\eva

\bvb Nowise could Guthrun weep; \\
so moody was she after the lad’s death, \\
and hard-minded over the marshaller’s corpse.\evb\evg


\bvg\bva Þá kvað þat Gullrǫnd, \hld\ Gjúka dóttir: &
„Fǫ́ kannt, fóstra, \hld\ þótt fróð séir, &
ungu vífi \hld\ and-spjǫll bera.“ &
Varaði hǫ́n at hylja \hld\ umb hrør fylkis.\eva

\bvb TODO.\evb\evg


\bvg\bva \alst{S}vipti hǫ́n blę́ju \hld\ af \alst{S}igurði &
ok \alst{v}att \alst{v}ęngi \hld\ fyr \alst{v}ífs knjám: &
„\alst{L}ít-tu á \alst{l}júfan, \hld\ \alst{l}ęgg þú munn við grǫn &
sem þú \alst{h}alsaðir \hld\ \alst{h}ęilan stilli.“\eva

\bvb She cast the cover off of Siward \\
and turned his cheeks before the wife’s knees: \\
“Look upon the loved one; lay your mouth against his lip \\
like thou didst embrace the hale prince.”\evb\evg


\bvg\bva \alst{Á} lęit Guðrún \hld\ \alst{ęi}nu sinni; &
sá hǫ́n \alst{d}ǫglings skǫr \hld\ \alst{d}ręyra runna, &
\alst{f}ránar sjónir \hld\ \alst{f}ylkis liðnar, &
\alst{h}ug-borg jǫfurs \hld\ \alst{h}jǫrvi skorna.\eva

\bvb On him looked Guthrun a single time; \\
she saw the noble’s locks run with blood, \\
the gleaming gaze of the marshaller gone, \\
the heart-fort \ken{chest} of the ruler cut by the sword.\evb\evg


\bvg\bva Þá hné Guðrún \hld\ hǫll við bólstri; &
haddr losnaði, \hld\ hlýr roðnaði &
en regns dropi \hld\ rann niðr umb kné.\eva

\bvb Then Guthrun sank down, slooped against the bolster; \\
her hair loosened, her cheek reddened, \\
and a drop of rain ran down to her knee.\evb\evg


\bvg\bva Þá grét Guðrún, \hld\ Gjúka dóttir, &
svá’t tǫ́r flugu \hld\ tresk í gǫgnum &
ok gullu við \hld\ gę̇ss í túni, &
mę́rir fuglar \hld\ es mę́r átti.\eva

\bvb Then wept Guthrun, Yivick’s daughter, \\
so that the tears flew through the ... \\
and in response shrieked the geese in the yard, \\
the famous fowls which the maiden owned.\evb\evg


\bvg\bva Þá kvað þat Gullrǫnd, \hld\ Gjúka dóttir: &
„ykkar vissa’k \hld\ ȧstir męstar &
manna allra \hld\ fyr mold ofan; &
unðir þú hvárki \hld\ úti né inni, &
systir mín, \hld\ nema hjá Sigurði.“\eva

\bvb Then quoth this Goldrand, Yivick’s daughter:
“I knew the love of you two to be the greatest \\
of all men above the earth. \\
Thou wast never content, not outside nor inside, \\
O my sister, save next to Siward.”\evb\evg


\bvg\bva%
„\edtext{\alst{S}vá vas mínn \alst{S}igurðr \hld\ hjá \alst{s}onum Gjúka &
sęm vę́ri \edtrans{\alst{g}ęir-laukr}{garlic}{\Bfootnote{or ‘spear-leek’. I have opted for this translation based on etymology (cf. OE \emph{gâr-léac} ‘spear-leek’), but the botanical identity is unclear. \GudrunTwo\ 2 has \emph{grǿnn laukr} ‘green leek’ instead. For the cultural importance of leeks and onions see note to \Voluspa\ 4.}} \hld\ ór \alst{g}rasi vaxinn,}{\lemma{Svá vas \dots\ vaxinn ‘So was \dots\ grown’}\Bfootnote{These two lines are almost identical to \GudrunTwo\ 2/1–2. Since the present poem is probably older \parencite{Sapp2022}, it is likely the source.}} &
\edtext{eða vę́ri \alst{b}jartr stęinn \hld\ ȧ \alst{b}and dręginn: &
\alst{j}arkna-stęinn \hld\ yfir \alst{ǫ}ðlingum.}{\lemma{eða vę́ri \dots\ ǫðlingum. ‘or were \dots\ athlings.’}\Bfootnote{Beaded necklaces were commonly worn by Scandinavian women of the time, and the beads were mostly of opaque coloured glass. Siward is thus likened to a bright crystal, the sons of Yivick to (dull) glass.}}\eva

\bvb\speakernoteb{[Guthrun quoth:]}So was my Siward by the sons of Yivick \\
like were a garlic out of grass grown, \\
or were a bright stone drawn on a band: \\
an \inx[C]{arkenstone} over the athlings.\evb\evg


\bvg\bva%
Ek \alst{þ}ȯtta auk \hld\ \alst{þ}jóðans rekkum &
\alst{h}vęrri \alst{h}ę́rri \hld\ \alst{H}ęrjans dísi; &
nú em’k svá \alst{l}ítil \hld\ sem \alst{l}auf séa &
\alst{o}pt í \alst{jǫ}lstrum \hld\ at \alst{jǫ}fur dauðan.\eva

\bvb I seemed even to the prince’s champions \\
higher than each of the Lord of Hosts’ ladies \ken{walkirries}. \\
Now I am as small as if a leaf I were, \\
high in the willows, after the ruler’s death.\evb\evg

TODO...

\sectionline
