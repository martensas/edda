\bookStart{The Hellride of Byrnhild}[Hęlręið Brynhildar]

\begin{flushright}%
Dating \parencite{Sapp2022}: late C11th (0.650), C13th (0.215), early C11th (0.135)

Meter: \Fornyrdislag
\end{flushright}%

TODO: INTRODUCTION.

\sectionline

\bpg\bpa Eptir dauða Brynhildar vóru gør bál tvau: annat Sigurði, ok brann þat fyrr, en Brynhildr var á ǫðru brennd ok var hon í reið þeiri er guð-vefjum var tjǫlduð. Svá er sagt at Brynhildr ók með reiðinni á helveg ok fór um tún þar er gýgr nǫkkur bjó. Gýgrin kvað:\epa

\bpb After Byrnhild’s death two pyres were made: one for Siward, and it burned earlier, but Byrnhild was burned on another, and she was in that chariot which was tent-roofed with good fabric. So is said, that Byrnhild drove with the chariot onto the Hellway, and passed by a plot where a certain gow lived. The gow quoth:\epb\epg

...

\bvg\bva „Skalt í gǫgnum \hld\ ganga ęigi &
grjóti studda \hld\ garða mína; &
bętr sǿmði þér \hld\ borða at rękja &
hęldr an vitja \hld\ vers annarar.“\eva

\bvb TRANSLATION.\evb\evg


\bvg\bva Hvat skalt vitja \hld\ af Vallandi, &
hvar-fúst hǫfuð, \hld\ húsa minna? &
Þú hęfir, Vár gullz, \hld\ ef þik vita lystir, &
mild, af hǫndum \hld\ mannz blóð þvegit.\eva

\bvb TRANSLATION.\evb\evg


\bvg\bva Breg þú ęigi mér, \hld\ brúðr ór stęini, &
þótt ek vę́ra’k \hld\ í víkingu; &
ek mun okkur \hld\ ǿðri þikkja &
hvar’s męnn ęðli \hld\ okkart kunna.\eva

\bvb TRANSLATION.\evb\evg


\bvg\bva Þú vart, Brynhildr, \hld\ Buðla dóttir, &
hęilli verstu \hld\ í hęim borin; &
þú hęfir Gjúka \hld\ of glatað bǫrnum &
ok búi þęira \hld\ brugðið góðu.\eva

\bvb TRANSLATION.\evb\evg


\bvg\bva Ek mun sęgja þér, \hld\ svinn, ór ręiðu &
vit-laussi mjök, \hld\ ef þik vita lystir: &
hvé gørðu mik \hld\ Gjúka arfar &
ásta-lausa \hld\ ok ęið-rofa.\eva

\bvb TRANSLATION.\evb\evg


\bvg\bva Lét hami vára \hld\ hugfullr konungr, &
átta systra, \hld\ undir ęik borið; &
vas’k vetra tólf, \hld\ ef þik vita lystir, &
es ungum gram \hld\ ęiða sęlda’k.\eva

\bvb TRANSLATION.\evb\evg


\bvg\bva Hétu mik allir \hld\ í Hlymdǫlum &
Hildi und hjalmi, \hld\ hvęrr es kunni.\eva

\bvb TRANSLATION.\evb\evg


\bvg\bva Þá lét’k gamlan \hld\ á Goðþjóðu &
Hjalm-Gunnar nę́st \hld\ hęljar ganga; &
gaf’k ungum sigr \hld\ Auðu bróður; &
þar varð mér Óðinn \hld\ of-ręiðr um þat.\eva

\bvb TRANSLATION.\evb\evg


\bvg\bva Lauk hann mik skjǫldum \hld\ í Skatalundi, &
rauðum ok hvítum, \hld\ randir snurtu; &
þann bað hann slíta \hld\ svefni mínum &
es hvęr-gi lands \hld\ hrę́ðask kynni.\eva

\bvb TRANSLATION.\evb\evg


\bvg\bva Lét umb sal minn \hld\ sunnan-verðan &
hávan brenna \hld\ hęr allz viðar; &
þar bað hann ęinn þegn \hld\ yfir at ríða, &
þann’s mér fǿrði gull \hld\ þat’s und Fáfni lá.\eva

\bvb TRANSLATION.\evb\evg


\bvg\bva Ręið góðr Grana \hld\ gull-miðlandi &
þar’s fóstri minn \hld\ flętjum stýrði; &
ęinn þótti hann þar \hld\ ǫllum bętri, &
víkingr Dana, \hld\ í verðungu.\eva

\bvb TRANSLATION.\evb\evg


\bvg\bva Svǫ́fu vit ok unðum \hld\ í sę́ing ęinni &
sem hann minn bróðir \hld\ of borinn vę́ri; &
hvárt-ki knátti \hld\ hǫnd yfir annat &
átta nóttum \hld\ okkart lęggja.\eva

\bvb We slept and loved in a single bed, as if he were born my brother; neither one of us could\evb\evg


\bvg\bva Því brá mér Guðrún, \hld\ Gjúka dóttir, &
at ek Sigurði \hld\ svę́fa’k á armi; &
þar varð’k þęss vís \hld\ es vildi’g-a’k &
at þau véltu mik \hld\ í ver-fangi.\eva

\bvb TRANSLATION.\evb\evg


\bvg\bva Munu við of-stríð \hld\ allz til lęngi &
konur ok karlar \hld\ kvikkvir fǿðask; &
vit skulum okkrum \hld\ aldri slíta, &
Sigurðr, saman. \hld\ Søkks-tu, gýgjar-kyn!“\eva

\bvb TODO—Sink thou down, Oh gow-kin!”\evb\evg
