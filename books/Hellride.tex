\bookStart{The Hell-ride of Byrnhild}[Hęlręið Brynhildar]

\begin{flushright}%
\textbf{Dating} \parencite{Sapp2022}: late C11th (0.650)

\textbf{Meter:} \Fornyrdislag
\end{flushright}%

TODO: INTRODUCTION.

\sectionline

\bpg\bpa Eptir dauða Brynhildar vóru gǫr bǫ́l tvau: annat Sigurði, ok brann þat fyrr, en Brynhildr var á ǫðru brennd ok var hon \edtrans{í reið þeiri er guð-vefjum var tjǫlduð}{in that wagon which was covered with godweb}{\Bfootnote{The tent-covering of the wagon was made of precious garments.  For the burial of women in wagons, cf. TODO (Oseberg ship?).}}.  Svá er sagt at \edtrans{Brynhildr ók með reið’inni á hel-veg}{Byrnhild drove with the wagon on the Hellway}{\Bfootnote{This gives us some interesting insight into old afterlife beliefs. After Byrnhild is burnt she ends up between the worlds of the dead and the living, the so-called “Hell-way”, or road to Hell (the underworld); she is buried in a wagon so that she will be able to travel comfortably. We may presume that the animals driving the wagon were slaughtered and burnt with her on the pyre.}} ok fór um tún þar er gýgr nǫkkur bjó.  Gýgr’in kvað:\epa

\bpb After Byrnhild’s death two pyres were made: one for Siward, and it burned earlier; but Byrnhild was burned on the other, and she was in that wagon which was covered with \inx[C]{godweb}.  It is said that Byrnhild drove with the wagon onto the Hellway and passed through a plot where there lived a certain \inx[C]{gow}. The gow quoth:\epb\epg


\bvg\bva „Skalt í \alst{g}ǫgnum \hld\ \alst{g}anga ęigi &
\alst{g}rjóti studda \hld\ \alst{g}arða mína; &
\alst{b}ętr sǿmði þér \hld\ \alst{b}orða at rękja &
hęldr an \alst{v}itja \hld\ \alst{v}ers annarar.\eva

\bvb “Thou shalt in no way go through \\
these rock-supported yards of mine; \\
it befit thee better to weave tapestries, \\
rather than visit another woman’s man.\evb\evg


\bvg\bva Hvat skalt \alst{v}itja \hld\ af \alst{V}al-landi, &
\alst{h}var-fu̇st \alst{h}ǫfuð, \hld\ \alst{h}úsa minna? &
Þú hęfir, \alst{V}ǫ́r gulls, \hld\ ef þik \alst{v}ita lystir, &
\alst{m}ild, af hǫndum \hld\ \alst{m}anns blóð þvegit.“\eva

\bvb Why shalt thou visit from Walland, \\
O straying head, these houses of mine? \\
Thou hast, mild \inx[P]{Ware} of gold, if thou hast lust to know, \\
washed a man’s blood off thy hands.”\evb\evg

Byrnhild answers:

\bvg\bva „\alst{B}regð ęigi mér, \hld\ \alst{b}rúðr ór stęini, &
þótt ek \alst{v}ę́ra’k \hld\ í \alst{v}íkingu; &
\alst{e}k mun \alst{o}kkur \hld\ \alst{ǿ}ðri þikkja &
hvar’s męnn \alst{ę}ðli \hld\ \alst{o}kkart kunna.“\eva

\bvb “Upbraid me not, O bride from the stone, \\
though I may have been in the sea-raid; \\
of us two will I seem the nobler, \\
wherever men know our lineages.”\evb\evg

The gow:

\bvg\bva „Þú vast, \alst{B}ryn-hildr, \hld\ \alst{B}uðla dóttir, &
\alst{h}ęilli verstu \hld\ í \alst{h}ęim borin; &
þú hęfir \alst{G}júka \hld\ of \alst{g}latat bǫrnum &
ok \alst{b}úi þęira \hld\ \alst{b}rugðit góðu.“\eva

\bvb “Thou wast, O Byrnhild, Budle’s daughter, \\
with the worst luck born into the world; \\
thou hast destroyed Yivick’s children, \\
and deprived their house of good.”\evb\evg

Byrnhild:

\bvg\bva „Ek mun \alst{s}ęgja þér, \hld\ \alst{s}vinn, ór ręiðu &
\alst{v}it-laussi mjǫk, \hld\ ef þik \alst{v}ita lystir: &
hvé \alst{g}ørðu mik \hld\ \alst{G}júka arfar &
\alst{ȧ}sta-lausa \hld\ ok \alst{ęi}ð-rofa.\eva

\bvb “I will tell thee, wise from my wagon, \\
O very witless one, if thou hast lust to know, \\
how Yivick’s heirs did make me \\
loveless, and an oath-breakeress.\evb\evg


\bvg\bva Lét hami vára \hld\ hug-fullr konungr, &
átta systra, \hld\ undir ęik borit; &
vas’k vetra tólf, \hld\ ef þik vita lystir, &
es ungum gram \hld\ ęiða sęlda’k.\eva

\bvb TODO. \\
I was twelve winters old, if thou hast lust to know, \\
when to the young prince I swore oaths.\evb\evg


\bvg\bva Hétu mik allir \hld\ í Hlym-dǫlum &
Hildi und hjalmi, \hld\ hvęrr es kunni.\eva

\bvb They all called me in the Limdales, \\
a Hild ’neath the helmet, whoever knew me.\evb\evg


\bvg\bva Þá lét’k gamlan \hld\ á Goð-þjóðu &
Hjalm-Gunnar nę́st \hld\ hęljar ganga; &
gaf’k ungum sigr \hld\ Auðu bróður; &
þar varð mér Óðinn \hld\ of-ręiðr um þat.\eva

\bvb Then I next among the Gots \\
made old Helm-Guther go the way of Hell; \\
I gave victory to Ead’s young brother; \\
there Weden was furious with me for that.\evb\evg


\bvg\bva Lauk hann mik skjǫldum \hld\ í Skata-lundi, &
rauðum ok hvítum, \hld\ randir snurtu; &
þann bað hann slíta \hld\ svefni mínum &
es hvęr-gi lands \hld\ hrę́ðask kynni.\eva

\bvb He locked me in with shields in Shatelund, \\
with red ones and white; their rims clasped. \\
He bade that one end my sleep, \\
who of no land could be frightened.\evb\evg


\bvg\bva Lét umb sal minn \hld\ sunnan-verðan &
hávan brenna \hld\ hęr alls viðar; &
þar bað hann ęinn þegn \hld\ yfir at ríða, &
þann’s mér fǿrði gull \hld\ þat’s und Fáfni lá.\eva

\bvb He made around my hall a south-facing, \\
high host of all wood \ken{fire} burn; \\
there he bade one thane ride over, \\
he who brought me the gold which ’neath Fathomer lay.\evb\evg


\bvg\bva Ręið góðr Grana \hld\ gull-miðlandi &
þar’s fóstri minn \hld\ flętjum stýrði; &
ęinn þótti hann þar \hld\ ǫllum bętri, &
víkingr Dana, \hld\ í verðungu.\eva

\bvb On Grane rode the good gold-dealer, \\
where my foster-son ruled the benches; \\
alone he seemed there better than all, \\
the Wiking of Danes, in the warband.\evb\evg


\bvg\bva Svǫ́fu vit ok unðum \hld\ í sę́ing ęinni &
sem hann minn bróðir \hld\ of borinn vę́ri; &
hvárt-ki knátti \hld\ hǫnd yfir annat &
átta nóttum \hld\ okkart lęggja.\eva

\bvb We slept and loved in one bed, \\
as if he were born my brother: \\
neither one laid a hand o’er the other \\
for eight nights, of us two.\evb\evg


\bvg\bva Því brá mér Guðrún, \hld\ Gjúka dóttir, &
at ek Sigurði \hld\ svę́fa’k á armi; &
þar varð’k þęss vís \hld\ es vildi’g-a’k &
at þau véltu mik \hld\ í ver-fangi.\eva

\bvb Thus Guthrun upbraided me, Yivick’s daughter, \\
that I slept on Siward’s arm; \\
there I became wise of that which I wanted not, \\
that those two had tricked me in the catch of man.\evb\evg


\bvg\bva Munu við of-stríð \hld\ alls til lęngi &
konur ok karlar \hld\ kvikkvir fǿðask; &
vit skulum okkrum \hld\ aldri slíta, &
Sigurðr, saman. \hld\ Søkks-tu, gýgjar-kyn!“\eva

\bvb In great strife for far too long \\
will men and women alive be born. \\
We two shall end our age, \\
I and Siward, together.—Sink down, thou gow’s kin!”\evb\evg

\sectionline
