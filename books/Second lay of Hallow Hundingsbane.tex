Helgi fekk Sigrúnar ok áttu þau sonu; var Helgi eigi gamall. Dagr Hǫgna sonr blótaði Óðin til fǫðurhefnda. Óðinn léði Dag geirs síns. Dagr fann Helga, mág sinn, þar sem heitir at Fjǫturlundi. Hann lagði í gǫgnum Helga með geirnum. Þar fell Helgi en Dagr reið til fjalla ok sagði Sigrúnu tíðindi: 

Hallow got Sighrun, and they owned sons; Hallow was not old. Day, son of Hain, blooted† to Weden to take revenge for his father. Weden lent Day his spear. Day found Hallow, his brother-in-law, at a place called Fetterlund; he laid the spear through Hallow. There fell Hallow, but Day rode to the fells and told Sighrun the news:

“Trauðr em ek, systir, \hld trega þér at sęgja
þvíat ek hęfi nauðigr \hld nipti grætta:
Fell í morgun \hld und Fjǫturlundi
buðlungr sá’s vas \hld bęztr í hęimi
ok hildingum \hld á halsi stóð.”

“Regretful am I, sister, to grieve thee by saying — for forced must I cause my kinswoman to cry: This morning fell, ’neath Fetterlund, that prince who was in the world the best, and on the throats of rulers stood.”

...

Fyrr vil’k kyssa \hld konung ólifðan
an þú blóðugri \hld brynju kastir;
hár es þitt, Helgi, \hld hélu þrungit,
allr es vísi \hld valdǫgg slęginn,
hęndr úrsvalar \hld Hǫgna mági;
hvé skal’k þér, buðlungr, \hld þess bót of vinna? 

“Sooner would I kiss the unliving king, than thou the bloody byrnie mightst cast away. Thy hair is, Hallow, with hoarfrost thick: the prince is all with corpse-dew whipped: the hands wet-cold on the kinsman of Hain. How shall I for thee, lord, remedy that?”

Ęin vęldr þú, Sigrún \hld frá Sefafjǫllum,
es Hęlgi es \hld harmdǫgg slęginn:
Grætr þú, gullvarit, \hld grimmum tǫ́rum,
sólbjǫrt suðrǿn, \hld áðr þú sofa gangir,
hvęrt fęllr blóðugt \hld á brjóst grami,
úrsvalt, innfjalgt \hld ękka þrungit.

“Thou alone causest, Sighrun from the Sevefells, that Hallow be by harm-dew whipped; thou criest, gold-covered, bitter tears, sun-bright southern lady, before thou to sleep mightst go. Each one falls bloody on the breast of the ruler, wet-cold and stifled, pressed forth by grief.”
