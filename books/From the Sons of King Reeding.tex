\bookStart{From the Sons of King Reeding}[Frá sonum Hrauðungs konungs]

BPA Hrauðungr konungr átti tvá sonu. Hét annarr Agnarr, enn annarr Geirrøðr.
BPA Agnarr var tíu vetra enn Geirrøðr átta vetra. Þeir reru tveir á báti með dorgar sínar at smáfiski.
BPA Vindr rak þá í haf út. Í náttmyrkri brutu þeir við land ok gingu upp; fundu kotbónda einn.
BPA Þar vǫ́ru þeir um vetrinn. Kerling fostraði Agnar enn karl Geirrøð.
BPA At vári fekk karl þeim skip. Enn er þau kerling leiddu þá til strandar, þá mælti karl einmæli við Geirrøð.
BPA Þeir fengu byr ok kvǫ́mu til stǫðva fǫður síns. Geirrøðr var fram í skipi.
BPA Hann hljóp upp á land enn hratt út skipinu, ok mælti: ”Far þú þar er smyl hafi þik.”
BPA Skipit rak út. Enn Geirrøðr gekk út til bǿjar; hánum var vel fagnat; þá var faðir hans andaðr.
BPA Var þá Geirrøðr til konungs tekinn, ok varð maðr ágætr.

BPB King Reeding owned two sons. One was called Eyner, and the other Garfrith.
BPB Eyner was ten winters old, and Garfrith eight winters. The two were rowing in a boat with their trolling-lines for small fishing.
BPB Wind then drove them out into the sea. In the darkness of night they crashed into land and walked up; they found a single cottage-farmer.
BPB There they were about the winter. The wife fostered Eyner, but the husband Garfrith.
BPB At spring the man gave them ships, but when they and the farmer’s wife brought them to the shore, the husband spoke privately with Garfrith.
BPB They got a good gust, and came to their father’s harbour. Garfrith was in the front of the ship.
BPB He leapt up onto land and pushed out the ship, and spoke: ”Go thou where the \inx{smil}[G] might have thee.”
BPB The ship drove out. But Garfrith walked towards the farm; he was welcomed well; his father had by then drawn his final breath.
BPB Then was Garfrith taken as king, and became an excellent man.

BPA Óðinn ok Frigg sátu í Hliðskjǫlfu ok sá um heima alla.
BPA Óðinn mælti: Sér þú Agnar fóstra þinn, hvar hann elr bǫrn við gýgi í hellinum?
BPA En Geirrøðr, fóstri minn, er konungr ok sitr nú at landi.
BPA Frigg segir: Hann er matníðingr sá at hann kvelr gesti sína ef hánum þykkja ofmargir koma.
BPA Óðinn segir at þat er in mesta lygi. Þau veðja um þetta mál.
BPA Frigg sendi eskismey sína, Fullu, til Geirrøðar. Hon bað konung varask at eigi fyrgerði hánum fjǫlkunnigr maðr sá er þar var kominn í land ok sagði þat mark á at engi hundr var svá olmr at á hann myndi hlaupa.
BPA En þat var inn mesti hégómi at Geirrøðr væri eigi matgóðr ok þó lætr hann handtaka þann mann er eigi vildu hundar á ráða.
BPA Sá var í feldi blám ok nefndisk Grímnir ok sagði ekki fleira frá sér þótt hann væri atspurðr.
BPA Konungr lét hann pína til sagna ok setja milli elda tveggja ok sat hann þar átta nætr.
BPA Geirrøðr konungr átti son tíu vetra gamlan ok hét Agnarr eftir bróður hans.
BPA Agnarr gekk at Grímni ok gaf hánum horn fullt at drekka, sagði að konungr gerði illa er hann lét pína hann saklausan.
BPA Grímnir drakk af. Þá var eldrinn svá kominn at feldrinn brann af Grímni. Hann kvað:

BPB Weden and Frie sat in Litheshelf and looked about all the Homes.
BPB Weden spoke: Seest thou Eyner thy foster-son, where he begets children with the troll-woman in the cave?
BPB But Garfrith, my foster-son, is king and now sits at land.
BPB Frie says: He is such a meat-nithing that he tortures his guests if he thinks there are too many of them.
BPB Weden says that this is the greatest lie; they make a bet about this matter.
BPB Frie sent her handmaid Full to Garfrith’s. She asked the king to be wary, that he might not be ended by that \inx{feelcunning}[C] man who was come in the land, and said that his mark was that no hound were so fierce that he would leap onto him.
BPB But that was the greatest vainglory that Garfrith would not be meat-good, and yet he has that man seized, whom the dogs would not touch.
BPB He was clad in a blue cloak, and called himself Grimen, and did not tell any more about himself, even though he was interrogated.
BPB The king had him tortured so that he would speak, and set him between two fires, and he remained there for eight nights.
BPB King Garfrith had a son ten winters old, and he was named Eyner after his brother.
BPB Eyner walked up to Grimen, and gave him a full horn to drink, saying that the king did ill as he had him tortured without cause.
BPB Grimen drank from it; then the fire had come such that the cloak burned on Grimen. He quoth:
