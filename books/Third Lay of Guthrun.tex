\bookStart{The Third Lay of Guthrun}[Guðrúnarkviða þriðja]

\begin{flushright}%
Dating \parencite{Sapp2022}: C10th (0.731), early C11th (0.178)

Meter: \Fornyrdislag%
\end{flushright}

A very short narrative poem, depicting a single minor legendary event. It is especially notable for its depiction of a trial by ordeal and the mention of a woman being drowned in a bog.

Herch, one of Attle’s concubines tells Attle that she has seen his wife Guthrun sleeping with Thedric. Attle becomes distressed upon hearing this (P1). Guthrun asks him what is wrong (1), and he responds that Herch has accused her of sleeping with Thedric (2). Guthrun promises to to prove her innocence through a trial by ordeal involving picking up a white stone from boiling water (3). She further says that while she and Thedric did sit down together, they did so in mutual grief over the deaths of her brothers (4–5). She tells Attle to summon a German lord named Saxe, who knows how to carry out the trial. Seven hundred men arrive to witness the event (6). Before picking up the stone, Guthrun laments over her brothers’ deaths, saying that they would have disputed the accusation through violence, but that she must now prove her innocence by herself (7). She then puts her hand in the boiling water, and unscathed takes out the stones. She holds it up and shows it to the witnesses (8). Attle laughs, knowing that his wife has been faithful, and orders Herch to pick up the stone (9). She does so, but her hands are horribly scorched, and men lead her to a “foul bog”, presumably to be drowned (see above). The poet ends by laconically stating that Guthrun in such a way was “reconstituted for her affronts”.

\sectionline

\bpg
\bpa Herkja hét ambǫ́tt Atla; hón hafði verit frilla hans. Hón sagði Atla at hón hefði sét Þjóðrek ok Guðrúnu bę́ði saman. Atli var þá allókátr. Þá kvað Guðrún:\epa

\bpb Herch was named the female thrall of Attle; she had been his concubine. She told Attle that she had seen Thedric and Guthrun both together. Attle was then wholly displeased. Then Guthrun quoth:\epb
\epg


\bvg
\bva „Hvat ’s þér, Atli? \hld\ ę́, Buðla sonr, &
es þér hryggt í hug; \hld\ hví hlę́r þú ę́va? &
Hitt myndi ǿðra \hld\ jǫrlum þykkja &
at við męnn mę́ltir \hld\ ok mik sę́ir.“\eva

\bvb “What is with thee, Attle? Always, son of Bodle, art thou sad at heart; why laughest thou never? TODO.”\evb
\evg


\bvg
\bva „Tregr mik þat, Guðrún, \hld\ Gjúka dóttir, &
mér í hǫllu \hld\ Hęrkja sagði &
at þit Þjóðrekr \hld\ undir þaki svę́fið &
ok léttliga \hld\ líni vęrðið.“\eva

\bvb “It troubles me, Guthrun, Yivick’s daughter, as in the hall Herch has said me: that thou and Thedric beneath thatched roof slept, and ye lightly warded the linen.\footnoteB{i.e., they threw off their clothes and slept together.}”\evb
\evg


\bvg
\bva „Þér mun’k alls þęss \hld\ ęiða vinna &
at inum hvíta \hld\ hęlga stęini, &
at ek við Þjóðmar \hld\ þat-ki átta’k, &
es vǫrðr né verr \hld\ vinna knátti,—\eva

\bvb “To thee I will swear oaths regarding all of that—by the white, holy stone—that I did not do such a thing with Thedmar,\footnoteB{Historically, Thedmar was the father of Thedric, who took over the kingdom after his father’s death (see Encyclopedia). Thedmar may here be a scribal error for Thedric, a scribal error for “Thedmar’s son”, or a nickname due to conflation of the father and son.} which neither watchman nor warrior has been able to swear upon,—\footnoteB{Guthrun says that she will prove her innocence through a trial by ordeal (that is, by lifting “the white holy stone” out of boiling water; see v. 8). She further strengthens her position by pointing out that no reliable man has sworn an oath attesting to her guilt.}”\evb
\evg


\bvg
\bva Nema ek halsaða \hld\ hęrja stilli, &
jǫfur ónęisinn, \hld\ ęinu sinni; &
aðrar vǫ́ru \hld\ okkrar spękjur &
es vit hǫrmug tvau \hld\ hnigum at rúnum.\eva

\bvb Unless I embraced the stiller of hosts \ken*{\textsc{ruler} = Thedmar}—the unshamed prince—a single time. Different were our dealings, when we two distressed ones [Guthrun and Thedric] reclined in private conversation.\evb
\evg


\bvg
\bva Hér kom Þjóðrekr \hld\ með þrjá tøgu, &
lifa þęir né ęinir, \hld\ þriggja tega manna; &
hrinktu mik at brǿðrum \hld\ ok at brynjuðum, &
hrinktu mik at ǫllum \hld\ á hǫfuðniðjum.\eva

\bvb Here came Thedric with thirty; not one of those thirty men still live. Surround\footnoteB{\emph{hrinktu} consisting of \emph{hring}, 2nd sg. imper. of \emph{hringja} ‘surround, encircle’ + \emph{þú} ‘thou’. The clitic form \emph{-tu} has caused devoicing.} me with my brothers, and with byrnied men; surround me with all my close kinsmen.\evb
\evg


\bvg
\bva Sęnd at Saxa, \hld\ sunnmanna gram; &
hann kann hęlga \hld\ hver vellanda;“ &
sjau hundruð manna \hld\ í sal gingu &
áðr kvę́n konungs \hld\ í kętil tǿki.\eva

\bvb Send for Saxe, lord of the southmen; he knows how to hallow a swelling cauldron!” Seven hundred men went into the hall, before the wife of the king might touch the kettle.\evb
\evg


\bvg
\bva „Kemr-a nú Gunnarr, \hld\ kalli’k-a Hǫgna, &
sé’k-a síðan \hld\ svása brǿðr; &
sverði myndi Hǫgni \hld\ slíks harms reka, &
nú verð’k sjǫlf fyr mik \hld\ synja lýta.“\eva

\bvb “Now Guther comes not, I can not call on Hain; I see not thereafter [my] beloved brothers. With a sword would Hain avenge such an affront; now I will for myself disprove the slanders.”\evb
\evg


\bvg
\bva Brá hón til botns \hld\ bjǫrtum lófa &
ok hón upp of tók \hld\ jarknastęina: &
„Sé nú sęggir \hld\ —sykn em ek orðin &
hęilagliga— \hld\ hvé sjá hverr velli.“\eva

\bvb Brought she the bright palms to the bottom, and she up did take the earkenstones: “See now, men—I am proven innocent, through holy means—how this cauldron boils!”\evb
\evg


\bvg
\bva Hló þá Atla \hld\ hugr í brjósti &
es hann hęilar sá \hld\ hęndr Guðrúnar: &
„Nú skal Hęrkja \hld\ til hvers ganga, &
sú’s Guðrúnu \hld\ grandi vę́nti.“\eva

\bvb Then laughed the heart in Attle’s chest, when he saw unscathed the hands of Guthrun: “Now shall Herch go to the cauldron, she who to Guthrun hoped to cause harm.”\evb
\evg


\bvg
\bva Sá-at maðr armligt, \hld\ hvęrr es þat sá at, &
hvé þar á Hęrkju \hld\ hęndr sviðnuðu; &
lęiddu þá męy \hld\ í mýri fúla, &
svá þá Guðrún \hld\ sinna harma.\eva

\bvb Each man saw not something so pitiful, who saw that: how there on Herch the hands were scorched. Led they the maiden into the foul bog; thus was Guthrun reconstituted for her affronts.\evb
\evg
