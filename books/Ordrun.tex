\bookStart{Weeping of Ordrun}[Oddrúnargrátr]
\def\thisBookCode{Oddrunargratr}

\begin{flushright}%
\textbf{Dating} \parencite{Sapp2022}: C10th (0.954)

\textbf{Meter:} \Fornyrdislag%
\end{flushright}%

\section{Introduction}

The \textbf{Weeping of Ordrun} (\Oddrunargratr) is another heroic poem.  The following edition and translation is by no means complete.

\section{From Burgny and Ordrun (\emph{Frá Borgnýju ok Oddrúnu})}

\bpg\bpa Heiðrekr hét konungr; dóttir hans hét Borgný. Vilmundr hét sá er var friðill hennar. Hȯn mátti eigi fǿða bǫrn áðr til kom Oddrún, Atla systir; hȯn hafði verit unnusta Gunnars, Gjúka sonar. Um þessa sǫgu er hér kveðit:\epa

\bpb {\huge H}eathric was a king called, his daughter was called Burgny. Wilmund was he called who was her lover. She could not bear children before Ordrun, Attle’s sister, came to her. She had been the lover of Guther, Yivick’s son. Of this saw is here sung:\epb\epg


\bvg\bva%
Hęyrða’k \alst{s}ęgja \hld\ ï \alst{s}ǫgum fornum &
hvé \alst{m}ę́r of kom \hld\ til \alst{M}orna-lands; &
\alst{ę}ngi mátti \hld\ fyr \alst{jǫ}rð ofan &
\alst{H}ęiðreks dóttur \hld\ \alst{h}jalpir vinna.\eva

\bvb {\huge I} heard it said in ancient saws\footnoteB{Probably formulaic; cf. \Hildebrandslied\ 1: \emph{ik gi-hórta dat seggen} ‘I heard it said’ which likewise uses the 1sg pret. of ‘hear’ and the infinitive of ‘say’. Both would go back to a Proto-Northwest Germanic phrase \emph{*ek (ga-)hauʀidō (þat) sagjaną}.} \\
how a maiden came to Mornland; \\
noone could—above the earth— \\
find help for Heathric’s daughter \ken*{= Burgny}.\evb\evg


\bvg\bva%
Þat frȧ \alst{O}ddrún, \hld\ \alst{A}tla systir, &
at sú \alst{m}ę́r hafði \hld\ \alst{m}iklar sóttir; &
brá hȯn af \alst{st}alli \hld\ \alst{st}jórn-bitluðum &
ok ȧ \alst{s}vartan \hld\ \alst{s}ǫðul of lagði.\eva

\bvb This learned Ordrun, Attle’s sister, \\
that the maiden \ken*{= Burgny} had great ailments; \\
she grabbed from the stable a rudder-bitted steed, \\
and a black saddle on [it] did lay.\evb\evg


\bvg\bva%
Lét hȯn \alst{m}ar fara \hld\ \alst{m}old-veg sléttan &
unds at \alst{h}ári kom \hld\ \alst{h}ǫll standandi; &
\edtext{ok hȯn \alst{i}nn of gekk \hld\ \alst{ę}nd-langan sal;}{\lemma{ok hȯn \dots\ sal ‘and she ... house’}\Bfootnote{The whole line is formulaic, see note to \Volundarkvida\ 8.}} &
\alst{s}vipti hȯn \alst{s}ǫðli \hld\ af \alst{s}vǫngum jó &
\edtext{ok hȯn þat \alst{o}rða \hld\ \alst{a}lls fyrst of kvað:}{\lemma{ok \dots\ of kvað ‘and ... did say’}\Bfootnote{The whole line is formulaic, see note to \Thrymskvida\ 2.}}\eva

\bvb She let the steed travel the smooth soil-way \ken{earth} \\
until she came to the high standing hall \\
and she inside did go the endlong house. \\
She cast the saddle off the slender horse \\
and she this word first of all did say:\evb\evg

TODO: More stanzas...

\sectionline
