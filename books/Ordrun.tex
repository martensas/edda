\bookStart{The Weeping of Ordrun}[Oddrúnargrátr]

\begin{flushright}%
Dating \parencite{Sapp2022}: C10th (0.954)

Meter: \Fornyrdislag%
\end{flushright}%

% Introduction

\section{From Burgny and Ordrun (\emph{Frá Borgnýju ok Oddrúnu})}

\bpg\bpa Heiðrekr hét konungr; dóttir hans hét Borgný. Vilmundr hét sá er var friðill hennar. Hon mátti eigi fǿða bǫrn áðr til kom Oddrún, Atla systir; hon hafði verit unnusta Gunnars, Gjúka sonar. Um þessa sǫgu er hér kveðit:\epa

\bpb Heathric was a king called, his daughter was called Burgny. Wilmund was he called who was her lover. She could not bear children befrore Ordrun arrived, Attle’s sister. She had been the lover of Guther, Yivick’s son. About this saw is here sung:\epb\epg


\bvg
\bva Hęyrða ek sęgja \hld\ í sǫgum fornum &
hvé mę́r of kom \hld\ til Morna-lands; &
engi mátti \hld\ fyr jǫrð ofan &
Hęiðreks dóttur \hld\ hjalpir vinna.\eva

\bvb I heard [it] said in ancient saws,\footnoteB{Probably formulaic; cf. \Hildebrandslied\ 1: \emph{ik gi-hórta dat seggen} ‘I heard it said’ which likewise uses the 1sg pret. of ‘hear’ and the infinitive of ‘say’. Both go back to a Proto-Northwest Germanic phrase \emph{*ek (ga-)hauʀidō (þat) sagjaną}.} how a maiden came to Mornland; no man could—above the earth—find help for Heathric’s daughter \ken*{= Burgny}.\evb
\evg


\bvg
\bva Þat frá Oddrún, \hld\ Atla systir, &
at sú mę́r hafði \hld\ miklar sóttir; &
brá hon af stalli \hld\ stjórn-bitluðum &
ok á svartan \hld\ sǫðul of lagði.\eva

\bvb This learned Ordrun, Attle’s sister, that the maiden \ken*{= Burgny} had great ailments; she seized from the stable a rudder-bitted steed, and a black saddle on [it] did lay.\evb
\evg


\bvg
\bva Lét hon mar fara \hld\ mold-veg sléttan &
unz at hári kom \hld\ hǫll standandi; &
\edtext{ok hon inn of gekk \hld\ ęnd-langan sal;}{\lemma{ok hon \dots\ sal ‘and she ... hall’}\Bfootnote{The whole line is formulaic, see note to \Volundarkvida\ 8.}} &
svipti hon sǫðli \hld\ af svǫngum jó &
\edtext{ok hon þat orða \hld\ allz fyrst of kvað:}{\lemma{ok \dots\ of kvað ‘and ... did say’}\Bfootnote{The whole line is formulaic, see note to \Thrymskvida\ 2.}}\eva

\bvb She let the steed journey on the smooth soil-way \ken{earth}, until she came to the high standing hall, and she inside did go the endlong hall. She drew the saddle of the slender horse, and she that word first of all did say:\evb
\evg

TODO: More verses.
