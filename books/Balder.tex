\bookStart{Dreams of Balder}[Baldrs draumar]

% Introduction.
\begin{flushright}%
\textbf{Dating} \parencite{Sapp2022}: C10th (0.890)

\textbf{Meter:} \Fornyrdislag%
\end{flushright}

\section{Introduction}

The \textbf{Dreams of Balder} (\Baldrsdraumar) are not preserved in \Regius, but rather in the early C14th ms. \AM.  The poem also survives in much younger paper mss., where a number of post-mediæval stanzas have been inserted.

The poem begins \emph{in medias res}; \inx[P]{Balder} has been having nightmares, which the Gods meet at the Thing to discuss (1).  \inx[P]{Weden} rides to \inx[L]{Hell}, where he has an encounter with a bloody hound; he passes it and continues to “the high house of \inx[P]{Hell}” (2–3), from which he rides west, to the grave of a certain \inx[C]{wallow} whom he revives using magic (4). She asks which man has forced her out of the grave (5), and Weden introduces himself as Waytame, before asking for whom the benches of Hell are covered with gold (6). The wallow responds that barrels of mead stand brewed for Balder and that the gods are very anxious (7). Weden asks her who will slay Balder (8), and she responds that it is Hath, carrying a “high fame-beam” (9).  Weden asks who will avenge Balder’s death (10), the wallow responds that \inx[P]{Rind} will give birth to Weden’s son \inx[P]{Wonnel}, who will slay Hath when only one night old (11).  Weden then asks about some mysterious maidens (12), which apparently betrays his identity.  The wallow announces that she now knows that it is Weden, who in turn retorts that she is not a wallow, but rather the “mother of three thurses” (13). The wallow tells him to ride home and “be famous” and taunts him over his unavoidable death at the \inx[L]{Rakes of the Reins} (14).

\sectionline

\section{The Dreams of Balder}

\bvg\bva\mssnote{\AM~1v/18}%
\edtext{Sęnn vǫ́ru \alst{ę́}sir \hld\ \alst{a}llir á þingi &
ok \alst{ǫ́}synjur \hld\ \alst{a}llar á máli, &
ok umb þat \alst{r}éðu \hld\ \alst{r}íkir tívar:}{\lemma{Sęnn \dots\ tívar ‘Soon \dots\ Tews’}\Bfootnote{Formulaic, identically shared with \Thrymskvida\ 14/1–3.  See also \inx[L]{Thing of the Gods}.}} &
hví vę́ri \alst{B}aldri \hld\ \alst{b}allir draumar?\eva

\bvb Soon were the \inx[G]{Eese} all at the \inx[C]{Thing}, \\
and the \inx[G]{Ossens} all at speech, \\
and of this counseled the mighty \inx[G]{Tews}: \\
Why did Balder have troubling dreams?\evb\evg


\bvg\bva\mssnote{\AM~1v/19}%
\alst{U}pp ręis \alst{Ó}ðinn, \hld\ \edtext{\alst{a}ldinn}{\Afootnote{emend.; \emph{alda} \AM}} gautr, &
ok hann á \alst{S}lęipni \hld\ \alst{s}ǫðul of lagði, &
ręið \alst{n}iðr þaðan \hld\ \alst{n}ifl-hęljar til; &
mǿtti \edtrans{\alst{h}velpi, \hld\ þęim’s ór \alst{h}ęlju kom}{the whelp that came out of Hell}{\Bfootnote{An otherwise unknown dog, sometimes identified with \inx[P]{Garm}.  The “hellhound” guarding the underworld is well known from world mythology, most famously the Greek \emph{Kérberos}.}}.\eva

\bvb Up rose Weden, the ancient Geat, \\
and he on \inx[P]{Slapner} the saddle did lay; \\
rode down thence to \inx[L]{Nivelhell}; \\
met the whelp that came out of Hell.\evb\evg


\bvg\bva\mssnote{\AM~1v/21}%
Sá vas \alst{b}lóðugr \hld\ of \alst{b}rjóst framan, &
ok \alst{g}aldrs fǫður \hld\ \edtext{\alst{g}ól of}{\Afootnote{\emph{golv} \AM}} lęngi, &
\alst{f}ramm ręið Óðinn, \hld\ \edtrans{\alst{f}old-vegr dunði}{the fold-way \ken{earth} resounded}{\Bfootnote{Cf. the description of \inx[P]{Thunder}’s riding in \Haustlong\ 14: \emph{dunði \dots\ mána vegr und hǫ́num} ‘the moon’s way \ken{sky/heaven} \dots\ resounded beneath him’); see further \Thrymskvida\ 21.}}, &
\alst{h}ann kom at \alst{h}ǫ́u \hld\ \alst{H}ęljar ranni.\eva

\bvb It was bloody on the front of its chest, \\
and at the father of \inx[C]{galder} \ken*{= Weden} for a long time bayed.— \\
Forth rode Weden—the fold-way \ken{earth} resounded— \\
he came to the high house of Hell.\evb\evg


\bvg\bva\mssnote{\AM~1v/22}%
Þá ręið \alst{Ó}ðinn \hld\ fyr \alst{au}stan dyrr, &
þar’s hann \alst{v}issi \hld\ \alst{v}ǫlu lęiði; &
nam hann \alst{v}ittugri \hld\ \edtrans{\alst{v}al-galdr}{slain-galder}{\Bfootnote{i.e. a galder to quicken the dead, in this case the wallow.  Cf. \Havamal\ 158 where Weden tells how He can bring hanged men back to life with runes.}} kveða, &
unds \alst{n}auðug ręis, \hld\ \alst{n}ás orð of kvað:\eva

\bvb Then rode Weden east from the door, \\
there as he knew the wallow’s grave. \\
He began for the cunning woman to sing a slain-\inx[C]{galder}, \\
until forced she rose, a corpse’s words quoth:\evb\evg


\bvg\bva\mssnote{\AM~1v/24}%
„Hvat ’s \alst{m}anna þat \hld\ \alst{m}ér ó·kunnra, &
es mér hęfr \alst{au}kit \hld\ \edtrans{\alst{ę}rfitt sinni}{this toilsome journey}{\Bfootnote{i.e. the journey out of the grave.}}? &
\edtext{Vas’k \alst{s}nifin \alst{s}njóvi, \hld\ ok \alst{s}lęgin regni, &
ok \alst{d}rifin \alst{d}ǫggu, \hld\ \alst{d}auð vas’k lęngi.}{\lemma{Vas’k snifin \dots\ lęngi. ‘I was snowed \dots\ long.’}\Bfootnote{Cf. the similar description of a buried person in \HelgakvidaTwo\ 47–48 (TODO).}}“\eva

\bvb “What sort of man is this, to me unknown, \\
who has caused for me this toilsome journey? \\
I was snowed by snow and struck by rain, \\
and bespattered with dew—dead was I for long.”\evb\evg


\bvg\bva\mssnote{\AM~1v/25}\speakernote{[Óðinn kvað:]}%
„\alst{V}eg-tamr ek hęiti, \hld\ sonr em’k \alst{V}al-tams, &
sęg þú mér ór \alst{h}ęlju, \hld\ ek man ór \alst{h}ęimi; &
hvęim eru \alst{b}ękkir \hld\ baugum sánir, &
\alst{f}lęt \alst{f}agrliga \hld\ \alst{f}lóuð gulli?“\eva

\bvb\speakernoteb{[Weden quoth:]}%
“Waytame am I called, I am Waltame’s son; \\
tell me [the tidings] from Hell—I will [tell those] from the world. \\
For whom are the benches sown with \inx[C]{bigh}[bighs], \\
the floors fairly flooded with gold?”\evb\evg


\bvg\bva\mssnote{\AM~1v/27}\speakernote{[Vǫlva kvað:]}%
„Hér stęndr \alst{B}aldri \hld\ of \alst{b}rugginn mjǫðr, &
\alst{sk}írar vęigar, \hld\ \edtrans{liggr \alst{sk}jǫldr yfir}{a shield lies over [them]}{\Bfootnote{Shields covering casks of mead is a common trope. Cf. TODO.}}, &
en \alst{ǫ́}s-męgir \hld\ í \alst{o}f-vę́ni; &
\alst{n}auðug sagða’k, \hld\ \alst{n}ú mun’k þęgja.“\eva

\bvb\speakernoteb{[The wallow quoth:]}%
“Here for Balder mead stands brewed, \\
pure draughts—a shield lies over them; \\
but the os-lads \ken*{= Eese} [stand] in great suspense— \\
forced I spoke, now I will shut up!”\evb\evg


\bvg\bva\mssnote{\AM~1v/29}\speakernote{[Óðinn kvað:]}%
„\alst{Þ}ęgj-at-tu vǫlva, \hld\ \alst{þ}ik vil’k fregna, &
\alst{u}nds \alst{a}l-kunna, \hld\ vil’k \alst{ę}nn vita: &
hvęrr man \alst{B}aldri \hld\ at \alst{b}ana verða, &
ok \alst{Ó}ðins son \hld\ \alst{a}ldri rę́na?“\eva

\bvb\speakernoteb{[Weden quoth:]}%
“Shut not up, wallow—thee I wish to ask! \\
Until all is known I wish yet to know: \\
Who will become Balder’s bane, \\
and rob Weden’s son \ken*{= Balder} of age?”\evb\evg


\bvg\bva\mssnote{\AM~2r/1}\speakernote{[Vǫlva kvað:]}%
„\alst{H}ǫðr berr \alst{h}ǫ́van \hld\ \edtext{\alst{h}róðr-baðm}{\Afootnote{emend.; \emph{hróðr-barm} \AM}} þinig, &
hann man \alst{B}aldri \hld\ at \alst{b}ana verða, &
ok \alst{Ó}ðins son \hld\ \alst{a}ldri rę́na; &
\alst{n}auðug sagða’k, \hld\ \alst{n}ú mun’k þęgja.“\eva

\bvb\speakernoteb{[The wallow quoth:]}%
“\inx[P]{Hath} bears the high glory-beam \ken{mistletoe} thither; \\
he will Balder’s bane become \\
and Weden’s son \ken*{= Balder} rob of life— \\
forced I spoke, now I will shut up!”\evb\evg


\bvg\bva\mssnote{\AM~2r/3}\speakernote{[Óðinn kvað:]}%
„\alst{Þ}ęgj-at-tu vǫlva, \hld\ \alst{þ}ik vil’k fregna, &
\alst{u}nds \alst{a}l-kunna, \hld\ vil’k \alst{ę}nn vita, &
hvęrr man \alst{h}ęipt \alst{H}ęði \hld\ \alst{h}ęfnt of vinna, &
eða \alst{B}aldrs \alst{b}ana \hld\ ȧ \alst{b}ál vega?“\eva

\bvb\speakernoteb{[Weden quoth:]}%
“Shut not up, wallow—thee I wish to ask! \\
Until all is known I wish yet to know: \\
Who will avenge that evil on Hath, \\
or cast on the pyre Balder’s bane \ken*{= Hath}?”\evb\evg


\bvg\bva\mssnote{\AM~2r/4}\speakernote{[Vǫlva kvað:]}%
„Rindr berr \edtext{\emph{\alst{V}ála}}{\Afootnote{required by alliteration; om. \AM}} \hld\ í \alst{v}estr-sǫlum, &
\edtext{sá man \alst{Ó}ðins sonr \hld\ \alst{ęi}n-nę́ttr vega; &
\alst{h}ǫnd of þvę́r-\edtext{\emph{at}}{\Afootnote{om. \AM}} \hld\ né \alst{h}ǫfuð kęmbir, &
áðr ȧ \alst{b}ál of \alst{b}err \hld\ \alst{B}aldrs and-skota;}{\lemma{sá \dots\ and-skota ‘that son \dots\ opponent’}\Bfootnote{These lines are, apart from the verb tense, identical to \Voluspa\ 32/4–33/2.  It is possible that both are building on a now-lost third poem; or that one has got these lines from the other.  (For discussion on the myth itself see introduction to \Voluspa\ 31–34.)}} &
\alst{n}auðug sagða’k, \hld\ \alst{n}ú mun’k þęgja.“\eva

\bvb\speakernoteb{[The wallow quoth:]}%
“Rind bears \inx[P]{Wonnel} in the western halls: \\
he will, Weden’s son, one night old, fight. \\
He washes not his hand nor combs his head \\
before onto the pyre he bears Balder’s opponent \ken*{= Hath}— \\
forced I spoke, now I will shut up.”\evb\evg


\bvg\bva\mssnote{\AM~2r/6}\speakernote{[Óðinn kvað:]}%
„\alst{Þ}ęgj-at-tu vǫlva, \hld\ \alst{þ}ik vil’k fregna, &
\alst{u}nds \alst{a}l-kunna, \hld\ vil’k \alst{ę}nn vita, &
hvęrjar ’ru \alst{m}ęyjar, \hld\ es at \alst{m}uni gráta &
ok á \alst{h}imin verpa \hld\ \alst{h}alsa-skautum?“\eva

\bvb\speakernoteb{[Weden quoth:]}%
“Shut not up, wallow—thee I wish to ask! \\
Until all is known I wish yet to know: \\
Which are the maidens that heartily weep, \\
and onto heaven throw the front-sheets?\footnoteB{According to \Gylfaginning\ 49 Hell promised to give Balder back to the Eese if “all things in the world, living and dead, cry for him”. The Eese relayed this message, and “the men and the animals and the earth and the stones and trees and all metals” cried for Balder. It may be that these maidens were included among the grievers (perhaps they were the walkirries, and this is what reveals Weden’s identity?), but their identity is otherwise unknown.  They may perhaps be identified with the maidens in \Vafthrudnismal\ 49.}”\evb\evg


\bvg\bva\mssnote{\AM~2r/8}\speakernote{[Vǫlva kvað:]}%
„\alst{E}rt-at Veg-tamr, \hld\ sem \alst{e}k hugða, &
hęldr ert \alst{Ó}ðinn, \hld\ \alst{a}ldinn gautr!“ &
\speakernote{[Óðinn kvað:]}„Ert-at \alst{v}ǫlva \hld\ né \alst{v}ís kona, &
hęldr ert \alst{þ}riggja \hld\ \alst{þ}ursa móðir!“\eva

\bvb\speakernoteb{[The wallow quoth:]}%
“Thou art not Waytame as I thought, \\
rather art thou Weden, the ancient Geat!”— \\
\speakernoteb{[Weden quoth:]}%
“Thou art no \inx[C]{wallow} nor wise woman, \\
rather art thou three \inx[G]{Thurses}’ mother!”\evb\evg


\bvg\bva\mssnote{\AM~2r/9}\speakernote{[Vǫlva kvað:]}%
„\alst{H}ęim ríð Óðinn \hld\ \edtrans{ok ves \alst{h}róðigr}{and be renowned}{\Bfootnote{A sarcastic taunt, the sense being: “Your fame, Weden, will not save you!”}}, &
svá komi-t \alst{m}anna \hld\ \alst{m}ęirr aptr ȧ vit, &
es \alst{l}auss \alst{L}oki \hld\ \alst{l}íðr ór bǫndum &
ok \alst{r}agna \alst{r}ǫk \hld\ \edtrans{\alst{r}júfęndr}{rippers}{\Bfootnote{Presumably Surt and Lock with his children, as described in \Voluspa\ 40 ff.  The verb \emph{rjúfa} ‘\CV: to break, rip up, break a hole in’ is used in the same context in the formulaic \emph{þá’s rjúfask ręgin} ‘when the \inx[G]{Reins} are ripped’ (\Vafthrudnismal\ 52), \emph{unds (of) rjúfask ręgin} ‘until the Reins are ripped’ (\Grimnismal\ 4, \Lokasenna\ 41 and \Sigrdrifumal\ 17).
One may also compare the similar sounding (but not or only very distantly related) verb \emph{rifna} ‘be riven, rent apart’; see Introduction to Runic inscription Sö 154 (Skarpåker, Sweden).}} koma.“\eva
%NOTE: If printing mythological Eddic poems separately One may also compare the similar sounding (but not or only very distantly related) verb \emph{rifna} ‘be riven, rent apart’ used in reference to the destruction of the world in Runic inscription Sö 154 (Skarpåker), and Arn \emph{Hryn} (in \Skp\ II pp. 185–6, ll. 3/7–8, see also note there): \emph{meiri verði þinn an þeira \hld\ þrifnuðr allr, unds himinn rifnar.} ‘greater than theirs may thy whole wealth be, until heaven is riven.’}} koma.“

\bvb\speakernoteb{[The wallow quoth:]}%
“Ride home, Weden, and be renowned! \\
So may no man come again to visit, \\
when loose Lock slips out of his bonds,\\
and [at] the \inx[L]{Rakes of the Reins} the rippers come!”\evb\evg


%TODO Late stanzas in paper manuscripts.

\sectionline
