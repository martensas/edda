I will shut up!\bookStart{The Dreams of Balder}[Baldrs draumar]

% Introduction.
\begin{flushright}%
Dating \parencite{Sapp2022}: C9th (0.110)–C10th (0.890)

Meter: \Fornyrdislag%
\end{flushright}

In ancient manuscripts only preserved in \AM, but the poem also survives in later manuscripts with a few extra stanzas (see below). It follows the structure of a riddle contest.

The poem begins \emph{in medias res}; \inx[P]{Balder} has been having nightmares, and so the gods meet at the Thing to figure out why (1). \inx[P]{Weden} rides to \inx[L]{Hell}, where he has an encounter with a bloody dog (2). It barks for a long time at him, but he passes it and continues to “the high house of \inx[P]{Hell}” (3), from which he rides west, to the grave of a certain \inx[C]{wallow}, whom he revives using magic (4). She asks which man has forced her out of the grave (5), and Weden introduces himself as Waytame, before asking for whom the benches of Hell are covered with gold (6). The wallow responds that barrels of mead stand brewed for Balder and that the gods are very anxious (7). Weden asks her who will slay Balder (8), and she responds that it is Hath, carrying a “high fame-beam” (9). Weden then asks her who will avenge Balder’s death by slaying Hath (10). The wallow responds that \inx[P]{Rind} will give birth to Weden’s son \inx[P]{Wonnel}, who will slay Hath when only one night old (11). Weden then asks about some mysterious maidens (12; see Note), which betrays his identity. The wallow tells him that she now knows his true identity, to which Weden responds that he does as well: she is not a wallow, but rather the “mother of three thurses” (13). The wallow tells him to ride home and “be famous”; he must still die at the \inx[L]{Rakes of the Reins} (14).

\sectionline

\bvg\bva\mssnote{\AM~1v/18}\edtext{Sęnn vǫ́ru \alst{ę́}sir \hld\ \alst{a}llir á þingi &
ok \alst{ǫ́}synjur \hld\ \alst{a}llar á máli, &
ok umb þat \alst{r}éðu \hld\ \alst{r}íkir tívar:}{\lemma{Sęnn \dots\ tívar ‘Soon \dots\ Tews’}\Bfootnote{Formulaic, identically shared with \Thrymskvida\ 14/1–3.  For the \inx[C]{Thing} of the Gods see \inx[G]{All Gods}.}} &
hví vę́ri \alst{B}aldri \hld\ \alst{b}allir draumar?\eva

\bvb Soon were the \inx[G]{Eese} all at the \inx[C]{Thing}, \\
and the \inx[G]{Ossens} all at speech, \\
and of this counseled the mighty \inx[G]{Tews}: \\
Why did Balder have troubling dreams?\evb
\evg


\bvg\bva\mssnote{\AM~1v/19}\alst{U}pp ręis \alst{Ó}ðinn, \hld\ \alst{a}ldinn gautr, &
ok hann á \alst{S}lęipni \hld\ \alst{s}ǫðul of lagði, &
ręið \alst{n}iðr þaðan \hld\ \alst{n}ifl-hęljar til; &
mǿtti \alst{h}velpi, \hld\ þęim’s ór \alst{h}ęlju kom.\eva

\bvb Up rose Weden, the ancient Geat, \\
and he on \inx[P]{Slapner} the saddle did lay; \\
rode down thence to \inx[L]{Nivelhell}; \\
met the whelp that came out of Hell.\evb
\evg


\bvg\bva\mssnote{\AM~1v/21}Sá vas \alst{b}lóðugr \hld\ of \alst{b}rjóst framan, &
ok \alst{g}aldrs fǫður \hld\ \alst{g}ól oflęngi, &
\alst{f}ramm ręið Óðinn, \hld\ \alst{f}old-vegr dunði, &
kom at \alst{h}ǫ́u \hld\ \alst{H}ęljar ranni.\eva

\bvb That one was bloody on the front of the chest, \\
and at the father of \inx[C]{galder} \ken*{= Weden} for a long time bayed.— \\
Forth rode Weden, the fold-way \ken{earth} resounded;\footnoteB{A similarity may be noted with the description of \inx[P]{Thunder}’s riding in \Haustlong\ 14: \emph{dunði \dots\ mána vegr und hǫ́num} ‘the moon’s way \ken{sky/heaven} \dots\ resounded beneath him’) and \Thrymskvida\ 20 (see also note there).} \\
he came to the high house of Hell.\evb
\evg


\bvg\bva\mssnote{\AM~1v/22}Þá ręið \alst{Ó}ðinn \hld\ fyr \alst{au}stan dyrr, &
þar’s hann \alst{v}issi \hld\ \alst{v}ǫlu lęiði; &
nam hann \alst{v}ittugri \hld\ \edtrans{\alst{v}al-galdr}{slain-galder}{\Bfootnote{i.e. an incantation to wake the slain (in this case the wallow); cf. \Havamal\ 157 where Weden lists a galder which can revive hanged men.}} kveða, &
unds \alst{n}auðug ręis, \hld\ \alst{n}ás orð of kvað:\eva

\bvb Then rode Weden east from the door, \\
there as he knew the wallow’s grave; \\
he took to sing a slain-\inx[C]{galder} for the cunning woman, \\
until forced she rose, a corpse’s words quoth:\evb
\evg


\bvg\bva\mssnote{\AM~1v/24}„Hvat ’s \alst{m}anna þat \hld\ \alst{m}ér ó·kunnra, &
es mér hęfr \alst{au}kit \hld\ \edtrans{\alst{ę}rfitt sinni}{this toilsome journey}{\Bfootnote{i.e. out of the grave.}}; &
\edtext{vas’k \alst{s}nifin \alst{s}nę́vi, \hld\ ok \alst{s}lęgin regni, &
ok \alst{d}rifin \alst{d}ǫggu, \hld\ \alst{d}auð vas’k lęngi.}{\lemma{vas’k snifin \dots\ lęngi. ‘I was snowed \dots\ dead.’}\Bfootnote{Cf. the similar description of a buried person in \HelgakvidaTwo\ 47–48 (TODO).}}“\eva

\bvb “What sort of man is this, unknown to me, \\
who has caused for me this toilsome journey? \\
I was snowed by snow and struck by rain, \\
and bespattered with dew—long was I dead.”\evb
\evg


\bvg\bva\mssnote{\AM~1v/25}\speakernote{[Óðinn kvað:]}
„\alst{V}eg-tamr hęiti’k, \hld\ sonr em’k \alst{V}al-tams, &
sęg mér ór \alst{h}ęlju, \hld\ ek ór \alst{h}ęimi mun; &
hvęim eru \alst{b}ękkir \hld\ baugum sánir? &
\alst{f}lęt \alst{f}agrliga \hld\ \alst{f}lóuð eru gulli.“\eva

\bvb\speakernoteb{[Weden quoth:]}
“Waytame am I called, I am Waltame’s son; \\
tell me [the tidings] from Hell—I will [tell those] from the world. \\
For whom are the benches sown with \inx[C]{bigh}[bighs]? \\
Fairly are the floors flooded with gold.”\evb
\evg


\bvg\bva\mssnote{\AM~1v/27}\speakernote{[Vǫlva kvað:]}
„Hér stęndr \alst{B}aldri \hld\ of \alst{b}rugginn mjǫðr, &
\alst{sk}írar vęigar, \hld\ \edtrans{liggr \alst{sk}jǫldr yfir}{a shield lies over [them]}{\Bfootnote{Shields covering casks of mead is a common trope. Cf. TODO.}}, &
en \alst{á}s-męgir \hld\ í \alst{o}f-vę́ni; &
\alst{n}auðug sagða’k, \hld\ \alst{n}ú mun’k þęgja.“\eva

\bvb\speakernoteb{[The wallow quoth:]}
“Here stands brewed for Balder mead: \\
pure draughts—a shield lies over [them]; \\
but the os-lads \ken*{= Eese} [stand] in great suspense— \\
forced I spoke, now I will shut up!”\evb
\evg


\bvg\bva\mssnote{\AM~1v/29}\speakernote{[Óðinn kvað:]}
„\alst{Þ}ęgj-at vǫlva, \hld\ \alst{þ}ik vil’k fregna, &
\alst{u}nds es \alst{a}l-kunna, \hld\ vil’k \alst{ę}nn vita, &
hvęrr mun \alst{B}aldri \hld\ at \alst{b}ana verða, &
ok \alst{Ó}ðins son \hld\ \alst{a}ldri rę́na?“\eva

\bvb\speakernoteb{[Weden quoth:]}
“Shut thou not up, wallow; thee I wish to ask! \\
Until all is known I wish to know further: \\
Who will become Balder’s bane, \\
and rob Weden’s son \ken*{= Balder} of age?”\evb
\evg


\bvg\bva\mssnote{\AM~2r/1}\speakernote{[Vǫlva kvað:]}
„\alst{H}ǫðr berr \alst{h}ǫ́van \hld\ \alst{h}róðr-baðm þinig, &
hann mun \alst{B}aldri \hld\ at \alst{b}ana verða, &
ok \alst{Ó}ðins son \hld\ \alst{a}ldri rę́na; &
\alst{n}auðug sagða’k, \hld\ \alst{n}ú mun’k þęgja.“\eva

\bvb\speakernoteb{[The wallow quoth:]}
“\inx[P]{Hath} bears the high fame-beam \ken{mistletoe} thither; \\
he will become Balder’s bane, \\
and rob Weden’s son \ken*{= Balder} of age— \\
forced I spoke, now I will shut up!”\evb
\evg


\bvg\bva\mssnote{\AM~2r/3}\speakernote{[Óðinn kvað:]}
„\alst{Þ}ęgj-at vǫlva, \hld\ \alst{þ}ik vil’k fregna, &
\alst{u}nds es \alst{a}l-kunna, \hld\ vil’k \alst{ę}nn vita, &
hvęrr mun \alst{h}ęipt \alst{H}ęði \hld\ \alst{h}ęfnt of vinna, &
eða \alst{B}aldrs \alst{b}ana \hld\ á \alst{b}ál vega?“\eva

\bvb\speakernoteb{[Weden quoth:]}
“Shut thou not up, wallow; thee I wish to ask! \\
Until all is known I wish to know further: \\
Who will avenge that evil doing on Hath, \\
or drag onto the pyre Balder’s bane \ken*{= Hath}?”\evb
\evg


\bvg\bva\mssnote{\AM~2r/4}\speakernote{[Vǫlva kvað:]}
„Rindr berr \alst{V}ála \hld\ í \alst{v}estr-sǫlum, &
sá mun \alst{Ó}ðins sonr \hld\ \alst{ęi}n-nę́ttr vega; &
\alst{h}ǫnd of þvę́r-at \hld\ né \alst{h}ǫfuð kęmbir, &
áðr á \alst{b}ál of \alst{b}err \hld\ \alst{B}aldrs and-skota; &
\alst{n}auðug sagða’k, \hld\ \alst{n}ú mun’k þęgja.“\eva

\bvb\speakernoteb{[The wallow quoth:]}
“Rind bears \inx[P]{Wonnel} in the western halls: \\
he will—Weden’s son, one night old—fight. \\
He washes not his hand nor combs his head, \\
before onto the pyre he bears Balder’s opponent \ken*{= Hath}— \\
forced I spoke, now I will shut up.\footnoteB{The similarity in wording to the treatment of this myth in \Voluspa\ is striking; apart from the tense, ll. 2–4 here are near-identical to 32/4–33/2 there (for discussion on the narrative see introduction to \Voluspa\ 31–34). The irregularity of the stanza length might suggest that a line has been inserted in analogy with the aforementioned poem.}”\evb
\evg


\bvg\bva\mssnote{\AM~2r/6}\speakernote{[Óðinn kvað:]}
„\alst{Þ}ęgj-at vǫlva, \hld\ \alst{þ}ik vil’k fregna, &
\alst{u}nds es \alst{a}l-kunna, \hld\ vil’k \alst{ę}nn vita, &
hvęrjar ’ru \alst{m}ęyjar, \hld\ es at \alst{m}uni gráta &
ok á \alst{h}imin verpa \hld\ \alst{h}alsa-skautum?“\eva

\bvb\speakernoteb{[Weden quoth:]}
“Shut thou not up, wallow; thee I wish to ask! \\
Until all is known I wish to know further: \\
Which are the maidens that weep heartily, \\
and onto heaven cast the front sheets?\footnoteB{According to \Gylfaginning\ 49 Hell promised to give Balder back to the Eese if “all things in the world, living and dead, cry for him”. The Eese relayed this message, and “the men and the animals and the earth and the stones and trees and all metals” cried for Balder. It may be that these maidens were included among the grievers (perhaps they were the walkirries, and this is what reveals Weden’s identity?), but their identity is otherwise unknown.}”\evb
\evg


\bvg\bva\mssnote{\AM~2r/8}\speakernote{[Vǫlva kvað:]}
„\alst{E}rt-at Veg-tamr, \hld\ sem \alst{e}k hugða, &
hęldr est \alst{Ó}ðinn, \hld\ \alst{a}ldinn gautr.“ &
\speakernote{[Óðinn kvað:]}„est-at \alst{v}ǫlva \hld\ \alst{n}é vís kona, &
hęldr est \alst{þ}riggja \hld\ \alst{þ}ursa móðir.“\eva

\bvb\speakernoteb{[The wallow quoth:]}
“Thou art not Waytame as I thought, \\
rather art thou Weden, the ancient Geat!”— \\
\speakernoteb{[Weden quoth:]}
“Thou art no \inx[C]{wallow} nor wise woman, \\
rather art thou of three \inx[G]{Thurses} the mother!”\evb
\evg


\bvg\bva\mssnote{\AM~2r/9}\speakernote{[Vǫlva kvað:]}„\alst{H}ęim ríð Óðinn \hld\ \edtrans{ok \alst{h}róðigr ves}{and be renowned}{\Bfootnote{A sarcastic, taunting statement, the sense being: “Your fame, Weden, will not save you!”}}, &
svá komi-t \alst{m}anna \hld\ \alst{m}ęirr aptr á vit, &
es \alst{l}auss \alst{L}oki \hld\ \alst{l}íðr ór bǫndum &
ok \alst{r}agna \alst{r}ǫk \hld\ \edtrans{\alst{r}júfęndr}{the rippers}{\Bfootnote{Presumably Surt and Lock with his children, as described in \Voluspa\ 40 ff.  The verb \emph{rjúfa} ‘\CV: to break, rip up, break a hole in’ is also used in the context in the formulaic \emph{þá’s rjúfask ręgin} ‘when the \inx[G]{Reins} are ripped’ (\Vafthrudnismal\ 52), \emph{unds (of) rjúfask ręgin} ‘until the Reins are ripped’ (\Grimnismal\ 4, \Lokasenna\ TODO and \Sigrdrifumal\ TODO).
One may further compare the similar sounding verb \emph{rifna} ‘be riven’, also used with reference to the destruction of the world in Runic inscription Sö 154: \emph{Jǫrð sal rifna \hld\ ok upp-himinn} ‘Earth shall be riven, and Up-heaven’, and Arn \emph{Hryn} (in \Skp\ II pp. 185–6, ll. 3/7–8, see also note there): \emph{meiri verði þinn an þeira \hld\ þrifnuðr allr, unds himinn rifnar.} ‘greater than theirs may thy whole wealth be, until heaven is riven.’}} koma.“\eva

\bvb\speakernoteb{[The wallow quoth:]}
“Ride home Weden, and be renowned! \\
So may no other man come again to visit [me], \\
when loose, Lock slips out of his bonds,\\
and [at] the \inx[P]{Rakes of the Reins} the rippers come!”\evb
\evg


%TODO Late stanzas in paper manuscripts.
