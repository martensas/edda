\bookStart{Second Lay of Guthrun}[Guðrúnarkviða aðra]

\begin{flushright}%
\textbf{Dating} \parencite{Sapp2022}: early C11th (0.759)–late C11th (0.199)

\textbf{Meter:} \Fornyrdislag%
\end{flushright}

\section{Introduction}

TODO.

\sectionline

\section{The Slaying of the Nivlings (\emph{Dráp Niflunga})}

\bpg\bpa Gunnarr ok Hǫgni tóku þá gullit allt, Fáfnis arf. Ó-friðr var þá milli Gjúkunga ok Atla; kenndi hann Gjúkungum vǫld um and-lát Brynhildar. Þat var til sę́tta, at þeir skyldu gipta hánum Guðrúnu, ok gáfu henni ó·minnis-veig at drekka áðr hon játti at giptast Atla. Synir Atla vóru þeir Erpr ok Eitill, en Svanhildr var Sigurðar dóttir ok Guðrúnar. Atli konungr bauð heim Gunnari ok Hǫgna, ok sendi Vinga eða Knéfrøð. Guðrún vissi vélar ok sendi með rúnum orð at þeir skyldu eigi koma ok til jar-tegna sendi hon Hǫgna hringinn Andvaranaut ok knýtti í vargs-hár. Gunnarr hafði beðit Oddrúnar, systur Atla, ok gat eigi; þá fekk hann Glaumvarar, en Hǫgni átti Kostberu. Þeira synir vóru þeir Sólarr ok Snę́varr ok Gjúki. En er Gjúkungar kómu til Atla, þá bað Guðrún sonu sína at þeir bę́ði Gjúkungum lífs en þeir vildu eigi. Hjarta var skorit ór Hǫgna en Gunnarr settr í orm-garð. Hann sló hǫrpu ok svę́fði ormana, en naðra stakk hann til lifrar. Þjóðrekr konungr var með Atla ok hafði þar látit flesta alla menn sína. Þjóðrekr ok Guðrún kę́rðu harma sín á milli. Hon sagði hánum ok kvað:\epa

\bpb Guther and Hain then took all the gold, Fathomer’s inheritance.  Hatred was then between the Yivickings and Attle; he blamed the Yivickings for Byrnhild’s passing.  These were their terms, that they would marry off to him Guthrun; and they gave her a forgetfulness-draught to drink before she agreed to be married off to Attle. The sons of Attle were Earp and Oatle, and Swanhild was Siward’s daughter and Guthrun’s. Attle invited to his home Guther and Hain, and sent Winge or \inx[P]{Kneefrith}. Guthrun knew his wiles and sent a word with runes, that they should not come, and as a sign she sent Hain the ring Andwaresneat, and tied through it a wolf’s hair. Guther had asked for Ordrun’s hand, Attle’s sister, and did not get her; then he got Gleamware, and Hain had Costbeare. Their sons were Solwer and Snower and Yivick. And when the Yivickings came to Attle, then Guthrun asked her sons that they should ask for the life of the Yivickings, but they would not. The heart was cut out of Hain, and Guther set in the serpent-yard. He struck his harp and soothed the serpents, but an adder stung him unto the liver. King Thedric was with Attle, and had there lost almost all of his men. Thedric and Guthrun recounted their griefs to each other. She spoke to him and quoth:\epb\epg

\sectionline

\section{The Second Lay of Guthrun}

\bvg\bva%
„\alst{M}ę́r vas’k \alst{m}eyja; \hld\ \alst{m}óðir mik fǿddi, &
\alst{b}jǫrt í \alst{b}úri; \hld\ unna’k vęl \alst{b}rǿðrum— &
unds mik \alst{G}júki \hld\ \alst{g}ulli ręifði, &
\alst{g}ulli ręifði, \hld\ \alst{g}af Sigurði.\eva

\bvb “A maiden was I of maidens; my mother raised me \\
bright in the bowers; I loved well my brothers— \\
until Yivick with gold endowed me, \\
with gold endowed me, and gave to Siward.\evb\evg


\bvg\bva%
\edtext{\alst{S}vá vas \alst{S}igurðr \hld\ uf \alst{s}onum Gjúka &
sem vę́ri \edtrans{\alst{g}rǿnn laukr}{green leek}{\Bfootnote{The leek was a highly valued plant.  Compare \Voluspa\ 4 where the \emph{grǿnn laukr} ‘green leek’ is said to have grown the first Golden Age.  See also note there about its mythological significance.}} \hld\ ór \alst{g}rasi vaxinn, &
eða \alst{h}jǫrtr \alst{h}ǫ́-bęinn \hld\ um \alst{h}vǫssum dýrum, &
eða \alst{g}ull \alst{g}lóð-rautt \hld\ af \alst{g}rǫ́u silfri.}{\lemma{ALL}\Bfootnote{Cf. \GudrunOne\ 18, which shares the first two lines with only small differences, and the very similar description of Hallow in \HelgakvidaTwo\ TODO: \emph{Svá bar Hęlgi · af hildingum...}}}“\eva

\bvb So was Siward over the sons of Yivick, \\
like were a green leek out of grass grown, \\
or a hart, high-legged, amidst coarse beasts, \\
or gold, glowing-red, beside grey silver—\evb\evg


\bvg\bva%
unds mér fyr·\alst{m}unðu \hld\ \alst{m}ínir brǿðr &
at ek \alst{ę́}tta ver \hld\ \alst{ǫ}llum fręmra; &
\alst{s}ofa þęir né mǫ́ttu-t \hld\ né of \alst{s}akar dǿma &
áðr þęir \alst{S}igurð \hld\ \alst{s}vęlta létu.\eva

\bvb until my brothers begrudged me, \\
that I had a husband better than all; \\
sleep could they not, nor speak of anything, \\
before they made Siward die.\evb\evg


\bvg\bva%
\alst{G}rani rann at þingi, \hld\ \alst{g}nýr vas at hęyra, &
en þá \alst{S}igurðr \hld\ \alst{s}jalfr ęigi kom; &
ǫll vǫ́ru \edtrans{\alst{s}ǫðul-dýr}{saddle-beasts \ken{horses}}{\Bfootnote{This kenning also occurs in a loose stanza by Norse King Anlaf “the Holy” Haraldson.}} \hld\ \alst{s}vęita stokkin &
ok of \alst{v}anið \alst{v}ási \hld\ of \alst{v}egǫndum.\eva

\bvb Grane ran from the Thing—a din was to be heard— \\
but then Siward himself came not. \\
All were the saddle-beasts \ken{horses} with sweat covered, \\
and trained to toil under heavy men.\evb\evg


\bvg\bva%
\alst{G}ekk ek \alst{g}rátandi \hld\ við \alst{G}rana rǿða, &
\alst{ú}rug-hlýra, \hld\ \alst{jó} frá’k spjalla; &
hnipnaði \alst{G}rani þá, \hld\ drap í \alst{g}ras hǫfði; &
\alst{jó}r þat vissi: \hld\ \alst{ęi}gendr né lifðu-t.\eva

\bvb I went, weeping, with Grane to speak, \\
teary-cheeked, the horse I asked for news. \\
Drooped Grane then; dropped his head in the grass; \\
the horse knew this: its owners lived not.\evb\evg


\bvg\bva%
Lęngi hvarf-at, \hld\ lęngi hugir dęildusk &
áðr of frę́gja’k \hld\ folk-vǫrð at gram; &
hnipnaði Gunnarr, \hld\ sagði mér Hǫgni &
frá Sigurðar \hld\ sǫ́rum dauða:\eva

\bvb Long time passed not—long my thoughts were torn— \\
before I did ask the folk-ward about the prince. \\
Drooped Guther; Hain told me \\
of Siward’s sore death.\evb\evg


\bvg\bva%
Liggr of hǫggvinn \hld\ fyr handan ver &
Guðþorms bani, \hld\ of gefinn ulfum; &
lít-tu þar Sigurð \hld\ á suðr-vega, &
þá hęyrir þú \hld\ hrafna gjalla, &
ǫrnu gjalla, \hld\ ę́zli fegna, &
varga þjóta \hld\ umb veri þínum.\eva

\bvb TODO. \\
Guthorm’s bane, given to the wolves. \\
Behold there Siward on the southern ways; \\
then hearest thou ravens shrieking; \\
eagles shrieking, of carrion rejoicing; \\
wolves howling around thy husband.\evb\evg

...TODO...

\sectionline
