\bookStart{The Second Lay of Guthrun}[Guðrúnarkviða aðra]

\begin{flushright}%
Dating \parencite{Sapp2022}: C10th (0.731), early C11th (0.178)

Meter: \Fornyrdislag%
\end{flushright}

TODO.

\sectionline

\section{The Slaying of the Nivlings (\emph{Dráp Niflunga})}

\bpg\bpa Gunnarr ok Hǫgni tóku þá gullit allt, Fáfnis arf. Ó-friðr var þá milli Gjúkunga ok Atla; kenndi hann Gjúkungum vǫld um and-lát Brynhildar. Þat var til sę́tta, at þeir skyldu gipta hánum Guðrúnu, ok gáfu henni ó·minnis-veig at drekka áðr hon játti at giptast Atla. Synir Atla vóru þeir Erpr ok Eitill, en Svanhildr var Sigurðar dóttir ok Guðrúnar. Atli konungr bauð heim Gunnari ok Hǫgna, ok sendi Vinga eða Knéfrøð. Guðrún vissi vélar ok sendi með rúnum orð at þeir skyldu eigi koma ok til jar-tegna sendi hon Hǫgna hringinn Andvaranaut ok knýtti í vargs-hár. Gunnarr hafði beðit Oddrúnar, systur Atla, ok gat eigi; þá fekk hann Glaumvarar, en Hǫgni átti Kostberu. Þeira synir vóru þeir Sólarr ok Snę́varr ok Gjúki. En er Gjúkungar kómu til Atla, þá bað Guðrún sonu sína at þeir bę́ði Gjúkungum lífs en þeir vildu eigi. Hjarta var skorit ór Hǫgna en Gunnarr settr í orm-garð. Hann sló hǫrpu ok svę́fði ormana en naðra stakk hann til lifrar. Þjóðrekr konungr var með Atla ok hafði þar látit flesta alla menn sína. Þjóðrekr ok Guðrún kę́rðu harma sín á milli. Hon sagði hánum ok kvað:\epa

\bpb Guther and Hain took all the gold, Fathomer’s inheritance. There was then enmity between the Yivickings and Attle; he blamed the Yivickings for Byrnhild’s passing. They came to terms that they would marry away Guthrun to him, and TODO. She spoke to him and quoth:\epb\epg


\bvg
\bva „\alst{M}ę́r vas’k \alst{m}eyja; \hld\ \alst{m}óðir mik fǿddi, &
\alst{b}jǫrt í \alst{b}úri; \hld\ unna’k vęl \alst{b}rǿðrum— &
unds mik \alst{G}júki \hld\ \alst{g}ulli ręifði, &
\alst{g}ulli ręifði, \hld\ \alst{g}af Sigurði.\eva

\bvb “A maiden was I of maidens; my mother raised me bright in the bowers; I loved well my brothers—until Yivick with gold endowed me, with gold endowed me, and gave [me] to Siward.\evb
\evg


\bvg
\bva „\alst{S}vá vas \alst{S}igurðr \hld\ uf \alst{s}onum Gjúka &
sem vę́ri \edtrans{\alst{g}rǿnn laukr}{green leek}{\Bfootnote{This st. shows that the leek was held to be the noblest of plants, something also seen by \Voluspa\ 4, where \emph{grǿnn laukr} it specifically mentioned as growing in the world’s very first days. See note there for its mythological significance.}} \hld\ ór \alst{g}rasi vaxinn, &
eða \alst{h}jǫrtr \alst{h}á-bęinn \hld\ um \alst{h}vǫssum dýrum, &
eða \alst{g}ull \alst{g}lóð-rautt \hld\ af \alst{g}rǫ́u silfri.“\eva

\bvb “So was Siward above the sons of Yivick, as were a green leek grown out of grass, or a high-boned hart in the midst of wild beasts, or glowing-red gold from grey silver.\evb
\evg
