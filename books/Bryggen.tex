\bookStart{Galders from Bryggen}

These galders were found as part of the cache of medieval rune-inscribed sticks found at Bryggen in the city of Bergen, Norway.  For the sake of simplicity, they are here listed in ascending order of their respective runological signum.

\sectionline

\section{B 257}

\begin{flushright}%
Dating: c. 1335

Meter: \Galdralag
\end{flushright}%

A stick inscribed on four carved sides, with the end broken off (though it seems that not much has been lost, since the).  The inscription is clearly a “love-charm” (that is, a piece of sexually coercive magic), addressed—as shown by the feminine dative adjective \emph{sjalfri} ‘self’ on side D—to a woman.  The language is particularly close to that of \Skirnismal\ 36, in which Shirner, Free’s servant, threatens to carve a runic inscription which will curse the ettin-woman Gird with \emph{ęrgi} ‘degeneracy’, \emph{ǿði} ‘madness’, and \emph{ó·þoli} ‘restlessness, impatience’ unless she sleep with his master.  It seems that we are here dealing with just such a surviving runic curse, and that \Skirnismal\ 36 is reflecting an  authentic form of Norse “love magic” (for it is unlikely that the present inscription should derive directly from that poem) by which a woman is cursed with sexual restlessness until she succumb to the will of the male curser.

A more distant parallel may be seen in the curse-formula found on the two C7th runic inscriptions from Stentoften and Björketorp (see TODO), wherein the destroyer of the respective monuments is cursed to become \emph{herma-lausaʀ argjú} ‘restless (a different root from \emph{ó·þoli} above!) with degeneracy’, i.e. ‘incessantly randy’.

Another thing of note is that side D ends with a string of fake-Latin gibberish, indicating post-conversion influence on the Old Norse-Germanic magic tradition.

\bvg
\bva[A]Ríst ek \alst{b}ót-rúnar \hld\ ríst ek \alst{b}jarg-rúnar &
\ind \alst{ei}n-falt við \alst{ǫ}lfum &
\ind \alst{t}ví-falt við \alst{t}rollum &
\ind \alst{þ}rí-falt við \alst{þ}u\emph{rsum}\eva

\bvb I carve cure-runes, I carve rescue-runes: \\
onefold against elves, \\
twofold against trolls, \\
threefold against thurses.\evb
\evg


\bvg
\bva[B]Við inni \alst{sk}ǿðu \hld\ \alst{sk}ag-val-kyrju &
svá’t \alst{ei} megi \hld\ þó-at \alst{ę́} vili &
\alst{l}ę́-vís kona \hld\ \alst{l}ífi þínu g\emph{randa}.\eva

\bvb Against the scatheful shag-walkirrie, \\
so that she may not—though she always wants to— \\
that guile-wise woman—harm thy life.\evb
\evg


\bvg
\bva[C]Ek \alst{s}endir þér \hld\ ek \alst{s}é á þér &
\alst{y}lgjar \alst{e}rgi \hld\ ok \alst{ó}·þola; &
á þér hríni \alst{ó}·þoli \hld\ ok \alst{jǫ}tuns móð\emph{r}; &
\alst{s}it-tu aldri, \hld\ \alst{s}op-tu aldri.\eva

\bvb I send to thee, I see on thee \\
a she-wolf’s degeneracy and restlessness; \\
may restlessness stick on thee, and an ettin’s wrath! \\
Never sit, never sleep!\evb
\evg


\bvg
\bva[D]Ant mér sem sjalfri þér. &
\textbf{†Beirist rubus rabus et arantabus laus abus rosa gava†}\eva

\bvb Love me like thy self. \\
\emph{Beirist rubus rabus et arantabus laus abus rosa gava}.\evb
\evg

\sectionline

\section{B 380}

\begin{flushright}%
Dating: ?

Meter: \Galdralag
\end{flushright}%

A short little charm explicitly invoking the two most important Heathen Gods, Thunder and Weden.  The inscription postdates the conversion of Norway by over a century, and it is therefore an open question whether the two mentioned gods should still have been seen in a positive light (in which case the inscription is only well-wishing, assuming that the receiver was of like mind to the sender), or whether they had already assimilated into the Christian complex of demons and devils (in which case the inscriber may have had more sinister intent than the first line lets on).  Judging from the first line, and from the half-Heathen contents of many other inscriptions found at Bryggen (some from as late as the C14th), I prefer the first option.

\bvg
\bva[]\alst{H}ęill sé þú \hld\ ok í \alst{h}ugum góðum; &
\ind \alst{Þ}órr þik \alst{þ}iggi, &
\ind \edtrans{\alst{Ó}ðinn þik \alst{ęi}gi}{may Weden own thee}{\Bfootnote{See note to \Voluspa\ 23.}}.\eva

\bvb May thou be hale and in good spirits;\footnoteB{Formulaic, the same line is attested in \Hymiskvida\ 41; see note there for parallels.} \\
may Thunder receive thee, \\
may Weden own thee.\evb
\evg
