\book{The Speeches of Webthrithen. (Vafþrúðnismǫ́l)}\bookStart

\begin{verse}
(Óðinn kvað:) \\%M
\bva Ráð mér nú \alst{F}rigg \hld alls mik \alst{f}ara tíðir \\%M
\ind at \alst{v}itja \alst{V}afþrúðnis; \\%M
\alst{f}orvitni mikla \hld kveð'k mér á \alst{f}ornum stǫfum \\%M
\ind við þann hinn \alst{a}lsvinna \alst{jǫ}tun.\\%E
\end{verse}

\bvb \textbf{Weden} quoth: \\ “Counsel me now, \textbf{Frie}, as I desire to travel to visit \textbf{Webthrithen}; great curiosity I say to myself of ancient staves\footnotemark[1], by that all-wise \textbf{ettin}." \\
\footnotetext[1]{Ancient (pieces of) lore; cf. v. 55. — Meaning (from \emph{great} onwards) is clear, but form is very confused.} 

\begin{verse}
(Frigg kvað:) \\%M
\bva \alst{H}ęima lętja \hld mynda'k \alst{H}ęrjafǫðr \\%M
\ind í \alst{g}ǫrðum \alst{g}oða; \\%M
\alst{ę}ngi \alst{jǫ}tun \hld hugða'k \alst{ja}fnramman \\%M
\ind sęm \alst{V}afþrúðni \alst{v}esa.\\%E
\end{verse}

\bvb Frie quoth: \\ “I would encourage the \textbf{Leader of Armies} to [stay at] home in the yards of the gods, for I've judged no ettin be as strong as\footnotemark[3] Webthrithen." \\
\footnotetext[3]{Lit. ‘equal-strong'.}

\begin{verse}
(Óðinn kvað:) \\%M
\bva Fjǫlð ek fór, \hld fjǫlð fręistaða'k, \\%M
\ind fjǫlð ek ręynda ręgin; \\%M
hitt vil'k vita, \hld hvé Vafþrúðnis \\%M
\ind salakynni séi.\\%E
\end{verse}

\bvb Weden quoth: \\ “Much I travelled, much I tried, much I tested the \textbf{Powers}\footnotemark[4]. \emph{This} I want to know, how the condition of the halls of Webthrithen might be?" \\
\footnotetext[4]{The gods.}

\begin{verse}
(Frigg kvað:) \\%M
\bva Hęill þú farir, \hld hęill þú aptr komir, \\%M
\ind hęill á sinnum séir; \\%M
ǿði þér dugi \hld hvar's skalt, Aldafǫðr, \\%M
\ind orðum mæla jǫtun.\\%E
\end{verse}

\bvb Frie quoth: \\ “Whole may thou travel, whole may thou return, whole may thou be on thy paths! May thy wisdom suffice, \textbf{Leader of Men}, when thou go to exchange words with the ettin." \\

\begin{verse}
\bva Fór þá Óðinn \hld at fręista orðspęki \\%M
\ind þess hins alsvinna jǫtuns; \\%M
at hǫllu hann kom, \hld es\footnotemark[1] átti Íms faðir; \\%M
\ind inn gekk Yggr þegar.\\%E
\end{verse}
\footnotetext[1]{Ms. \emph{ok} corrected to \emph{es}. Alliteration is lacking in this line, for which reason FJ emends \emph{Íms} to \emph{Hymis}.}

\bvb Then went Weden, to try the word-wisdom of that all-wise ettin. To the hall he came, which the father of \textbf{Ime}\footnotemark[5] owned; shortly the \textbf{Frightener}\footnotemark[6] walked in. \\
\footnotetext[5]{Webthrithen.}
\footnotetext[6]{Weden.}

\begin{verse}
(Óðinn kvað:) \\%M
\bva Hęill þú nú, Vafþrúðnir, \hld nú em'k í hǫll kominn \\%M
\ind á þik sjalfan séa; \\%M
hitt vilk fyrst vita, \hld ef fróðr séir \\%M
\ind eða alsviðr, jǫtunn.\\%E
\end{verse}

\bvb Weden quoth: \\ “Hail thee now, Webthrithen; now I have come into the hall, to see thee thyself. \emph{This} I want to know first, if knowing thou might be, or all-wise, ettin!" \\

\begin{verse}
(Vafþrúðnir kvað:) \\%M
\bva Hvat's þat manna, \hld es í mínum sal \\%M
\ind verpumk orði á? \\%M
út þú né kømr \hld órum hǫllum frá. \\%M
\ind nema þú inn snotrari séir.\\%E
\end{verse}

\bvb Webthrithen quoth: \\ “What is that of men\footnotemark[10], that in \emph{my} hall throws words at me? Thou will not come \emph{out}, from \emph{our}\footnotemark[11] halls, unless thou be the wiser [of us two]." \\
\footnotetext[10]{Ie., ‘what man is that'. The use of the neuter pronoun \emph{hvat} by Web-str. may be seen as an insult or a way of belittling the guest.}
\footnotetext[11]{Prob. again meaning ‘my', unless Web-str. has allies present in the hall, but no such indication is given.}

\begin{verse}
(Óðinn kvað:) \\%M
\bva Gagnráðr\footnotemark[5] hęiti'k, \hld nú em'k af gǫngu kominn, \\%M
\ind þyrstr til þinna sala; \\%M
laðar þurfi \hld hęf'k lęngi farit \\%M
\ind ok þinna andfanga, jǫtunn.\\%E
\end{verse}
\footnotetext[5]{R's \emph{Gagnráðr} ‘Gain-adviser', is attested as Gangráðr ‘Journey-adviser' in \emph{Gylf}.}

\bvb Weden quoth: \\ “\textbf{Gain-adviser} I am called, I am come from the journey, thirsty to thy halls. I have travelled for a long time in need of hospitality, and of thy reception, ettin!" \\

\begin{verse}
(Vafþrúðnir kvað:) \\%M
\bva Hví þú þá, Gagnráðr, \hld mælisk af golfi fyrir? \\%M
\ind far þú í sess í sal; \\%M
þá skal fręista, \hld hvárr flęira viti, \\%M
\ind gęstr eða hinn gamli þulr.\\%E
\end{verse}

\bvb Webthrithen quoth: \\ “Why then, Gain-adviser, art thou speaking from the floor before [me]? Take a seat in the hall! Then it shall be proven, which of the two might know more; the guest, or the old \textbf{thyle}." \\

\begin{verse}
(Gagnráðr kvað:) \\%M
\bva Óauðigr maðr, \hld es til auðigs kømr, \\%M
\ind mæli þarft eða þęgi; \\%M
ofrmælgi mikil \hld hygg at illa geti \\%M
\ind hvęim's við kaldrifjaðan kømr.\\%E
\end{verse}

\bvb Gain-adviser quoth: \\ “An unwealthy man, who comes to a wealthy [one], ought to speak what is needed, or be silent.\footnotemark[14] Much over-speaking\footnotemark[15], I judge, will be bad for the one who comes to a cold-ribbed\footnotemark[16] [man]." \\
\footnotetext[14]{Line identical to \emph{High} 18/2. The whole verse strongly reminds of verses from the \emph{Guest-thread} portion of said poem.}
\footnotetext[15]{“Speaking too much".}
\footnotetext[16]{That is, ‘cold-hearted', ‘cunning'.}

\begin{verse}
(Vafþrúðnir kvað:) \\%M
\bva Sęg mér, Gagnráðr, \hld alls á golfi vill \\%M
\ind þíns of fręista frama, \\%M
hvé hęstr hęitir, \hld sá's hvęrjan dręgr \\%M
\ind dag of dróttmǫgu.\\%E
\end{verse}

\bvb Webthrithen quoth: \\ “Say to me, Gain-adviser, since on the floor I will to try thy fame: What is the horse called, which drags each \emph{day} above the sons of the retinue\footnotemark[20]?" \\
\footnotetext[20]{Kenning for ‘men', ‘mankind'.}

\begin{verse}
(Gagnráðr kvað:) \\%M
\bva Skinfaxi hęitir, \hld es hinn skíra dręgr \\%M
\ind dag of dróttmǫgu; \\%M
hęsta baztr \hld þykkir með Hręiðgotum; \\%M
\ind ęy lýsir mǫn af mari.\\%E
\end{verse}

\bvb Gain-adviser quoth: \\ “\textbf{Shining-fax} [that one] is called, who drags the bright day above the sons of the retinue. The best of horses he seems among the \textbf{Rode-goths}; the mane of that stallion ever shines." \\

\begin{verse}
(Vafþrúðnir kvað:) \\%M
\bva Sęg þat, Gagnráðr, \hld alls á golfi vill \\%M
\ind þíns of fręista frama, \\%M
hvé jór hęitir, \hld sá's austan dręgr \\%M
\ind nótt of nýt ręgin.\\%E
\end{verse}

\bvb Webthrithen quoth: \\ “Say this, Gain-adviser, since on the floor I will to try thy fame: What is the horse called, which from the east drags night above the useful \textbf{Powers}?" \\

\begin{verse}
(Gagnráðr kvað:) \\%M
\bva Hrímfaxi hęitir, \hld es hvęrja dręgr \\%M
\ind nótt of nýt ręgin; \\%M
méldropa fęllir \hld morgin hvęrjan; \\%M
\ind þaðan kømr dǫgg of dala.\\%E
\end{verse}

\bvb Gain-adviser quoth: \\ “\textbf{Frost-fax} [that one] is called, who drags each night above the useful Powers. Every morning he lets foam fall from his bit\footnotemark[26]; thence comes dew in the valleys." \\
\footnotetext[26]{Lit. “he fells bit-drops".}

\begin{verse}
(Vafþrúðnir kvað:) \\%M
\bva Sęg þat, Gagnráðr, \hld alls á golfi vill \\%M
\ind þíns of fręista frama, \\%M
hvé ǫ́ hęitir, \hld sú's dęilir með jǫtna sonum \\%M
\ind grund ok með goðum.\\%E
\end{verse}

\bvb Webthrithen quoth: \\ “Say this, Gain-adviser, since on the floor I will to try thy fame; How the river is called, which divides the ground between the sons of ettins and the gods?" \\

\begin{verse}
(Gagnráðr kvað:) \\%M
\bva Ífing hęitir ǫ́, \hld es dęilir með jǫtna sonum \\%M
\ind grund ok með goðum; \\%M
opin rinna \hld hón skal um aldrdaga; \\%M
\ind verðr-at íss á ǫ́.\\%E
\end{verse}

\bvb Gain-adviser quoth: \\ “\textbf{Iving} the river is called, which divides the ground between the sons of ettins and the gods. Throughout [her] life-days she shall flow open; ice forms not on the river." \\

\begin{verse}
(Vafþrúðnir kvað:) \\%M
\bva Sęg þat, Gagnráðr, \hld alls á golfi vill \\%M
\ind þíns of fręista frama, \\%M
hvé vǫllr hęitir, \hld es finnask vigi at \\%M
\ind Surtr ok hin svǫ́su goð.\\%E
\end{verse}

\bvb Webthrithen quoth: \\ “Say this, Gain-adviser, since on the floor I will to try thy fame: How that valley is called, where \textbf{Surt} and the excellent gods find each other at war?" \\

\begin{verse}
(Gagnráðr kvað:) \\%M
\bva Vígríðr hęitir vǫllr, \hld es finnask vígi at \\%M
\ind Surtr ok hin svǫ́su goð; \\%M
hundrað rasta \hld hann's á hvęrjan veg; \\%M
\ind sá's þęim vǫllr vitaðr.\\%E
\end{verse}

\bvb Gain-adviser quoth: \\ “\textbf{Battle-rider} is the valley called, where Surt and the cheerful gods find each other at war. A hundred rests\footnotemark[30], he stretches in each direction; that valley is known for them.\footnotemark[31]" \\
\footnotetext[30]{An old unit of length, from its name prob. the length a horse could travel before resting.}
\footnotetext[31]{That is, known for its great size.}

\begin{verse}
(Vafþrúðnir kvað:) \\%M
\bva Fróðr est nú gęstr, \hld far á bękk jǫtuns, \\%M
\ind mælumk í sessi saman; \\%M
hǫfði vęðja \hld vit skulum hǫllu í \\%M
\ind gęstr, of gęðspęki.\\%E
\end{verse}

\bvb Webthrithen quoth: \\ “Knowing art thou now, guest, sit down on the ettin's bench; let us speak while sitting together. In the hall we shall wager a head, guest, over [our] mind-wisdom." \\

\begin{verse}
(Gagnráðr kvað:) \\%M
\bva Sęg þat hit ęina, \hld ef þitt ǿði\footnotemark[10] dugir \\%M
\ind ok þú Vafþrúðnir vitir, \\%M
hvaðan jǫrð of kom \hld eða upphiminn \\%M
\ind fyrst, hinn fróði jǫtunn.\\%E
\end{verse}
\footnotetext[10]{Starting with \emph{ǿði}, the poem is also preserved in 748.}

\bvb Gain-adviser quoth: \\ “Say the first\footnotemark[32], if thy wisdom suffices, and thou, Webthrithen, might know: Whence, O knowing ettin, the earth first came, or \textbf{up-heaven}?" \\
\footnotemark[32]{Lit. ‘one'.}

\begin{verse}
(Vafþrúðnir kvað:) \\%M
\bva Ór Ymis holdi \hld vas jǫrð of skǫpuð, \\%M
\ind ęn ór bęinum bjǫrg, \\%M
himinn ór hausi \hld hins hrimkalda jǫtuns, \\%M
\ind ęn ór svęita sær.\\%E
\end{verse}

\bvb Webthrithen quoth: \\ “Out of \textbf{Yime's} hull\footnotemark[35], the earth was created, but the mountains out of his bones. Heaven out of the skull of the frost-cold ettin, but the sea out of his sweat.\footnotemark[36]" \\
\footnotetext[35]{His body.}
\footnotetext[36]{\emph{svęiti} ‘sweat' is a common kenning for blood. — This v. closely resembles \emph{Grím} 40.}

\begin{verse}
(Gagnráðr kvað:) \\%M
\bva Sęg þat annat, \hld ef þitt ǿði dugir \\%M
\ind ok þú Vafþrúðnir vitir, \\%M
hvaðan máni of kom, \hld svát fęrr menn yfir, \\%M
\ind eða sól hit sama.\\%E
\end{verse}

\bvb Gain-adviser quoth: \\ “Say the second, if thy wisdom suffices, and thou, Webthrithen, might know: Whence the moon came, so that it travels over men, or likewise the sun?" \\

\begin{verse}
(Vafþrúðnir kvað:) \\%M
\bva Mundilfari hęitir, \hld hann's Mána faðir \\%M
\ind ok svá Solar hit sama; \\%M
himin hverfa \hld þau skulu hvęrjan dag \\%M
\ind ǫldum at ártali.\\%E
\end{verse}

\bvb Webthrithen quoth: \\ “\textbf{Moundelfare} [that one] is called, he is the father of the Moon, and likewise of the Sun. They shall circle in the heavens every day, for men to reckon time\footnotemark[40]." \\
\footnotetext[40]{Lit. “for men to year-reckoning".}

\begin{verse}
(Gagnráðr kvað:) \\%M
\bva Sęg þat þriðja, \hld alls þik svinnan kveða \\%M
\ind ok þú Vafþrúðnir vitir, \\%M
hvaðan dagr of kom, \hld sá's fęrr drótt yfir, \\%M
\ind eða nótt með niðum.\\%E
\end{verse}

\bvb Gain-adviser quoth: \\ “Say the third, since [they] call thee wise, and thou, Webthrithen, might know: Whence the day came, the one that travels over the rettinue, or night with the moon-phases?" \\

\begin{verse}
(Vafþrúðnir kvað:) \\%M
\bva Dęllingr hęitir, \hld hann's Dags faðir, \\%M
\ind ęn Nótt vas Nǫrvi borin; \\%M
ný ok nið \hld skópu nýt ręgin \\%M
\ind ǫldum at ártali.\\%E
\end{verse}

\bvb Webthrithen quoth: \\ “\textbf{Delling} [that one] is called, he is the father of \textbf{Day}, but \textbf{Night} was born to \textbf{Nare}. The waxing and waning [of the moon], the useful Powers created, for men to reckon time." \\

\begin{verse}
(Gagnráðr kvað:) \\%M
\bva Sęg þat fjórða, \hld alls þik fróðan kveða, \\%M
\ind ok þú Vafþrúðnir vitir, \\%M
hvaðan vetr of kom \hld eða varmt sumar \\%M
\ind fyrst með fróð ręgin.\\%E
\end{verse}

\bvb Gain-adviser quoth: \\ “Say the fourth, since [they] call thee knowing, and thou, Webthrithen, might know: Whence winter first came, or the warm summer, among the knowing Powers?" \\

\begin{verse}
(Vafþrúðnir kvað:) \\%M
\bva Vindsvalr hęitir, \hld hann's Vetrar faðir, \\%M
\ind ęn Svǫ́suðr Sumars.\footnotemark[15]\\%E
\end{verse}
\footnotetext[15]{Second half of the v. seems missing.}

\bvb Webthrithen quoth: \\ “\textbf{Wind-cool} [that one] is called, he is the father of \textbf{Winter}, but \textbf{Delightful} of \textbf{Summer}." \\

\begin{verse}
(Gagnráðr kvað:) \\%M
\bva Sęg þat fimta, \hld alls þik fróðan kveða, \\%M
\ind ok þú Vafþrúðnir vitir, \\%M
hvęrr ása ęlztr \hld eða Ymis niðja \\%M
\ind yrði í árdaga.\\%E
\end{verse}

\bvb Gain-adviser quoth: \\ “Say the fifth, since [they] call thee knowing, and thou, Webthrithen, might know: Who, in days of yore became the eldest of the \textbf{Anses}, or of the descendants of Yime?" \\

\begin{verse}
(Vafþrúðnir kvað:) \\%M
\bva Ørófi vetra \hld áðr væri jǫrð skǫpuð, \\%M
\ind þá vas Bergęlmir borinn, \\%M
Þrúðgęlmir \hld vas þess faðir, \\%M
\ind ęn Aurgęlmir afi.\\%E
\end{verse}

\bvb Webthrithen quoth: \\ “Uncountable winters before the earth would be created, then \textbf{Bear-yeller} was born. \textbf{Strength-yeller} was \emph{that one's} father, and \textbf{Mud-yeller} the grandfather." \\

\begin{verse}
(Gagnráðr kvað:) \\%M
\bva Sęg þat sétta, \hld alls þik svinnan kveða, \\%M
\ind ok þú Vafþrúðnir vitir, \\%M
hvaðan Aurgęlmir kom \hld með jǫtna sonum \\%M
\ind fyrst, hinn fróði jǫtunn.\\%E
\end{verse}

\bvb Gain-adviser quoth: \\ “Say the sixth, since [they] call thee wise, and thou, Webthrithen, might know: Whence, O knowing ettin, Mud-yeller first came among the sons of ettins?" \\

\begin{verse}
(Vafþrúðnir kvað:) \\%M
\bva Ór Élivǫ́gum \hld stukku ęitrdropar, \\%M
\ind svá óx unz ór varð jǫtunn; \\%M
órar ættir \hld kómu þar allar saman; \\%M
\ind því's þat æ alt til atalt.\footnotemark[20]\\%E
\end{verse}
\footnotetext[20]{Lines 3–4 missing in R and 748, but quoted in \emph{Gylf}.}

\bvb Webthrithen quoth: \\ “From the \textbf{Ell-waves}, poison-drops splashed; thus [it] grew until an ettin emerged. \emph{Our} family lines all together originated there, therefore our race\footnotemark[45] is forever fierce against all.\footnotemark[46]" \\
\footnotetext[45]{Lit. ‘it' or ‘that'.}
\footnotetext[46]{Somewhat strange phrasing, but the line does not appear damaged. It is clearly an explanation of the fierce and maleficent nature of the ettins, as their first ancestors were created from poison.}

\begin{verse}
(Gagnráðr kvað:) \\%M
\bva Sęg þat sjaunda, \hld alls þik svinnan kveða, \\%M
\ind ok þú Vafþrúðnir vitir, \\%M
hvé sá bǫrn gat \hld hinn baldni\footnotemark[25] jǫtunn, \\%M
\ind es hann hafði-t gýgjar gaman.\\%E
\end{verse}
\footnotetext[25]{R has \emph{aldni}, ‘aged, old'. This breaks alliteration; \emph{baldni} ‘bold, defiant' has been substituted from 748.}

\bvb Gain-adviser quoth: \\ “Say the seventh, since [they] call thee wise, and thou, Webthrithen, might know: How did that one, the defiant ettin, beget children, when he did not enjoy the [carnal] pleasure of a troll-woman?" \\

\begin{verse}
(Vafþrúðnir kvað:) \\%M
\bva Und hęndi vaxa \hld kvǫ́ðu hrímþursi \\%M
\ind męy ok mǫg saman; \\%M
fótr við fǿti \hld gat hins fróða jǫtuns \\%M
\ind sexhǫfðaðan son.\\%E
\end{verse}

\bvb Webthrithen quoth: \\ “Neath the hand\footnotemark[50] on the \textbf{frost-thurse}, [they] said that a maiden and lad grew together. A foot against a foot begot, for the knowing ettin, a six-headed son." \\
\footnotetext[50]{\emph{hęndi} (dative of \emph{hǫnd}) means ‘hand', but might here be a poetic circumlocution for ‘arm'.}

\begin{verse}
(Gagnráðr kvað:) \\%M
\bva Sęg þat áttunda, \hld alls þik fróðan kveða, \\%M
\ind ok þú Vafþrúðnir vitir, \\%M
hvat fyrst of mant \hld eða fręmst of vęizt, \\%M
\ind þú est alsviðr jǫtunn.\\%E
\end{verse}

\bvb Gain-adviser quoth: \\ “Say the eigth, since [they] call thee knowing, and thou, Webthrithen, might know: What dost thou first remember, or earliest know?\footnotemark[55] Thou art all-wise, ettin." \\
\footnotetext[55]{Cf. Vsp 1.}

\begin{verse}
(Vafþrúðnir kvað:) \\%M
\bva Ørófi vetra \hld áðr væri jǫrð of skǫpuð, \\%M
\ind þá vas Bergęlmir borinn; \\%M
þat fyrst um man'k, \hld es hinn fróði jǫtunn \\%M
\ind á vas lúðr of lagiðr.\footnotemark[30]\\%E
\end{verse}
\footnotetext[30]{This verse is quoted in \emph{Gylf}.}

\bvb Webthrithen quoth: \\ “Uncountable winters before the earth would be created, then Bear-yeller was born. \emph{That} I first remember, when the knowing ettin\footnotemark[60] was laid down on the funeral-bed\footnotemark[61]." \\
\footnotetext[60]{That is, Bear-yeller. Cf. v. 29.}
\footnotetext[61]{\emph{lúðr}, a tricky word.}

\begin{verse}
(Gagnráðr kvað:) \\%M
\bva Sęg þat níunda, \hld alls þik svinnan kveða, \\%M
\ind ok þú Vafþrúðnir vitir, \\%M
hvaðan vindr of kømr \hld svát fęrr vág yfir, \\%M
\ind æ męnn hann sjalfan of séa.\\%E
\end{verse}

\bvb Gain-adviser quoth: \\ “Say the ninth, since [they] call thee wise, and thou, Webthrithen, might know: Whence the wind comes, so that he travels over the wave; forever men see him himself.\footnotemark[65]" \\
\footnotetext[65]{Perhaps a negation has been lost here; the wind is never seen by men.}

\begin{verse}
(Vafþrúðnir kvað:) \\%M
\bva Hræsvęlgr hęitir, \hld es sitr á himins ęnda, \\%M
\ind jǫtunn í arnar ham; \\%M
af hans vængjum \hld kveða vind koma \\%M
\ind alla męnn yfir.\\%E
\end{verse}

\bvb Webthrithen quoth: \\ “\textbf{Corpse-swallower} [that one] is called, which sits at the end of the heavens, an ettin in the shape of an eagle. From his wings, they say [that] the wind comes over all men." \\

\begin{verse}
(Gagnráðr kvað:) \\%M
\bva Sęg þat tíunda, \hld alls þú tíva rǫk \\%M
\ind ǫll Vafþrúðnir vitir, \\%M
hvaðan Njǫrðr of kom \hld með niðjum ása. \\%M
Hófum ok hǫrgum \hld hann ræðr hundmǫrgum \\%M
\ind ok varð-at hann ǫ́sum alinn.\\%E
\end{verse}

\bvb Gain-adviser quoth: \\ “Say the \emph{tenth}, since thou, Webthrithen, of all the fates of the \textbf{Tues} might know: Whence \textbf{Nearth} came into the company of the kinsmen of the \textbf{Anses}? He rules an immense number\footnotemark[68] of \textbf{hoves} and \textbf{heargs}, and he was not begotten among the Anses." \\
\footnotetext[68]{Lit. “he rules hound-many".}

\begin{verse}
(Vafþrúðnir kvað:) \\%M
\bva Í Vanahęimi \hld skópu hann vís ręgin \\%M
\ind ok sęldu at gíslingu goðum, \\%M
í aldar rǫk \hld hann mun aptr koma \\%M
\ind hęim með vísum vǫnum.\\%E
\end{verse}

\bvb Webthrithen quoth: \\ “In \textbf{Wane-Home}, the wise \textbf{Powers}\footnotemark[69] created him, and sold him as a hostage to the gods. In the fate of the age, he will come back, home among the wise \textbf{Wanes}." \\
\footnotetext[69]{Though \emph{ręgin} usually serves as a direct synonym of \emph{goð} 'god(s)', it here seems to refer specifically to the Wanes, in contrast with the \textbf{Eses} or gods.}

\begin{verse}
(Gagnráðr kvað:) \\%M
\bva Sęg þat ęllipta, \hld hvar ýtar túnum í \\%M
\ind hǫggvask hvęrjan dag; \\%M
val þęir kjósa \hld ok ríða vígi frá, \\%M
\ind sitja męir of sáttir saman.\footnotemark[35]\\%E
\end{verse}
\footnotetext[35]{This and the next v. are damaged in both R and 748; R has only this verse, but splits it in two (the 2nd starting with \emph{val}), while 748 has 40:1 (Ms.: \emph{S. þ. e. XI}) and then jumps to the answer v. 41. They have here been reconstructed, but it is possible some lines are still missing.}

\bvb Gain-adviser quoth: \\ “Say the eleventh, Where men in yards, hew away at each other every day? They choose those destined to die in war, and ride [away] from battle; [then] they sit more content together." \\

\begin{verse}
(Vafþrúðnir kvað:) \\%M
\bva Allir ęinhęrjar \hld Óðins túnum í \\%M
\ind hǫggvask hvęrjan dag, \\%M
val þeir kjósa \hld ok ríða vígi frá, \\%M
\ind sitja męir of sáttir saman.\\%E
\end{verse}

\bvb Webthrithen quoth: \\ “In Weden's yards, all the \textbf{Lone Warriors} hew away at each other every day. They choose those destined to die in war, and ride [away] from battle; [then] they sit more content together." \\

\begin{verse}
(Gagnráðr kvað:) \\%M
\bva Sęg þat tolpta, \hld hví þú tíva rǫk \\%M
\ind ǫll Vafþrúðnir vitir, \\%M
frá jǫtna rúnum \hld ok allra goða \\%M
\ind þú hit sannasta sęgir, \\%M
\ind hinn alsvinni jǫtunn.\\%E
\end{verse}

\bvb Gain-adviser quoth: \\ “Say the twelfth, Why thou, Webthrithen, shouldst know all the fates of the \textbf{Tues}\footnotemark[73]? From the \textbf{runes} of the ettins and of all the gods, thou, the all-wise ettin, speakest most truly." \\
\footnotetext[73]{The gods. Formation identical to \emph{ragna rǫk} ‘the fates of the Powers'.}

\begin{verse}
(Vafþrúðnir kvað:) \\%M
\bva Frá jǫtna rúnum \hld ok allra goða \\%M
\ind ek kann sęgja satt, \\%M
þvíat hvęrn hęf'k \hld heim of komit, \\%M
níu kom'k hęima \hld fyr niflhęl neðan; \\%M
\ind hinig dęyja ór hęlju halir.\\%E
\end{verse}

\bvb Webthrithen quoth: \\ “From the runes of the ettins and of all the gods I can speak truly, for I have been about each \textbf{Home}. I was about nine Homes beneath Nivelhell; this way men die out of Hell\footnotemark[1]." \\
\footnotetext[1]{A difficult verse. Finnur considers \emph{ór hęlju} “out of Hell” a later interpolation.}

\begin{verse}
(Gagnráðr kvað:) \\%M
\bva Fjǫlð ek fór, \hld fjǫlð fręistaða'k, \\%M
\ind fjǫlð ek ręynda ręgin; \\%M
hvat lifir manna, \hld þá's hinn mæra líðr \\%M
\ind fimbulvetr með firum?\\%E
\end{verse}

\bvb Gain-adviser quoth: \\ “Much I travelled, much I tried, much I tested the \textbf{Powers}.\footnotemark[80] What remains\footnotemark[79] of men, when the famous \textbf{fimbol-winter} passes among firs\footnotemark[81]?” \\
\footnotetext[79]{Lit. “lives".}
\footnotetext[80]{Here begins the repetition of the same “mantra" used in v. 3, which continues until the final question (v. 54).}
\footnotetext[81]{Among men.}

\begin{verse}
(Vafþrúðnir kvað:) \\%M
\bva Líf ok Lífþrasir, \hld ęn þau lęynask munu \\%M
\ind í holti Hoddmímis; \\%M
morgindǫggvar \hld þau sér at mat hafa; \\%M
\ind þaðan af aldir alask.\\%E
\end{verse}

\bvb Webthrithen quoth: \\ “Life and Lifethrasher, and they will hide themselves in the wood of Hoard-Mime\footnotemark[85]. Morning-dew they [will] have as food; thereof generations [will] be bred.” \\
\footnotetext[85]{Prob. the same as Uggdrassle.}}

\begin{verse}
(Gagnráðr kvað:) \\%M
\bva Fjǫlð ek fór, \hld fjǫlð fręistaða'k, \\%M
\ind fjǫlð ek ręynda ręgin; \\%M
hvaðan kømr sól \hld á hinn slétta himin, \\%M
\ind es þessa hęfr Fęnrir farit?\\%E
\end{verse}

\bvb Gain-adviser quoth: \\ “Much I travelled, much I tried, much I tested the Powers. Whence comes sun onto the smooth heaven, when \textbf{Fenner} has killed this one\footnotemark[1]?" \\
\footnotetext[1]{That is, the current incarnation of the sun, as explained in the next v.}

\begin{verse}
(Vafþrúðnir kvað:) \\%M
\bva Ęina dóttur \hld berr alfrǫðull, \\%M
\ind áðr hana Fęnrir fari; \\%M
sú skal ríða, \hld þá's ręgin dęyja, \\%M
\ind móður brautir mær.\\%E
\end{verse}

\bvb Webthrithen quoth: \\ “One daughter the elf-wheel\footnotemark[1] bears, before \textbf{Fenner} might kill her. When the Powers die, that one, the maiden, shall ride the paths of the mother.” \\
\footnotetext[1]{The sun.}

\begin{verse}
(Gagnráðr kvað:) \\%M
\bva Fjǫlð ek fór, \hld fjǫlð fręistaða'k, \\%M
\ind fjǫlð ek ręynda ręgin; \\%M
hvęrjar 'ro męyjar, \hld es líða mar yfir, \\%M
\ind fróðgęðjaðar fara.\\%E
\end{verse}

\bvb Weden quoth: \\ “Much I travelled, much I tried, much I tested the Powers. Which are the maidens that pass over the ocean; wise-minded they go?” \\

\begin{verse}
(Vafþrúðnir kvað:) \\%M
\bva Þríar þjóðár \hld falla þorp yfir \\%M
\ind męyja Mǫgþrasis; \\%M
hamingjur ęinar \hld þær’s í hęimi eru, \\%M
\ind þó þær með jǫtnum alask.\\%E
\end{verse}

\bvb Webthrithen quoth: \\ “Three great rivers fall over the settlement of the maidens of Maythrasher; the only Hamings that are in the Home,\footnotemark[1] though they are raised among the ettins\footnotemark[2]." \\
\footnotetext[1]{Either in Ettinhome, or in the entire world.}
\footnotetext[2]{See index entry Maythrasher.}

\begin{verse}
(Gagnráðr kvað:) \\%M
\bva Fjǫlð ek fór, \hld fjǫlð fręistaða'k, \\%M
\ind fjǫlð ek ręynda ręgin; \\%M
hvęrir ráða æsir \hld ęignum goða, \\%M
\ind þá's sloknar Surtalogi?\\%E
\end{verse}

\bvb Gain-adviser quoth: \\ “Much I travelled, much I tried, much I tested the Powers. Which properties of the gods will the \textbf{Anses} [still] rule\footnotemark[105], when the flame of \textbf{Surt} burns out?" \\
\footnotetext[105]{Or ‘control'.}

\begin{verse}
(Vafþrúðnir kvað:) \\%M
\bva Víðarr ok Váli \hld byggva vé goða, \\%M
\ind þá's sloknar Surtalogi; \\%M
Móði ok Magni \hld skulu Mjǫlni hafa \\%M
\ind Vingnis at vígþroti.\\%E
\end{verse}

\bvb Webthrithen quoth: \\ “\textbf{Wider} and \textbf{Weel} [will] inhabit the sanctuaries of the gods, when the \textbf{flame of Surt} burns out. \textbf{Mood} and \textbf{Main} will own \textbf{Mealen}, when \textbf{Wingen} can no longer fight\footnotemark[110]." \\
\footnotetext[110]{Lit. “at Wingen's fight-exhaustion", referring to his death.}

\begin{verse}
(Gagnráðr kvað:) \\%M
\bva Fjǫlð ek fór, \hld fjǫlð fręistaða'k, \\%M
\ind fjǫlð ek ręynda ręgin; \\%M
hvat verðr Óðni \hld at aldrlagi, \\%M
\ind þá's rjúfask ręgin?\\%E
\end{verse}

\bvb Gain-adviser quoth: \\ “Much I travelled, much I tried, much I tested the Powers. What happens to Weden at the end of the age, when the Powers are broken?" \\

\begin{verse}
(Vafþrúðnir kvað:) \\%M
\bva Ulfr glęypa \hld mun Aldafǫðr, \\%M
\ind þess mun Víðarr vreka; \\%M
kalda kjapta \hld hann klyfja mun \\%M
\ind vitnis vígi at.\\%E
\end{verse}

\bvb Webthrithen quoth: \\ “The wolf will swallow the \textbf{Leader of Men}; Wider will avenge that. He will cleave the cold jaws of the wolf at the battle." \\

\begin{verse}
(Gagnráðr kvað:) \\%M
\bva Fjǫlð ek fór, \hld fjǫlð fręistaða'k, \\%M
\ind fjǫlð ek ręynda ręgin; \\%M
hvat mælti Óðinn, \hld áðr á bál stigi, \\%M
\ind sjalfr í ęyra syni?\\%E
\end{verse}

\bvb Gain-adviser quoth: \\ “Much I travelled, much I tried, much I tested the Powers. What spoke Weden himself, before [he]\footnotemark[115] would step onto the funeral pyre, into the ear of the son?" \\
\footnotetext[115]{Prob. Weden's son, that is \textbf{Balder}.}

\begin{verse}
(Vafþrúðnir kvað:) \\%M
\bva Ęy mann-gi\footnotemark[40] vęit, \hld hvat þú í árdaga \\%M
\ind sagðir í ęyra syni; \\%M
fęigum munni \hld mælta'k mína forna stafi \\%M
\ind ok of ragna rǫk. \\%M
Nú við Óðin \hld dęilda'k mína orðspęki; \\%M
\ind þú est æ vísastr vera.\\%E
\end{verse}
\footnotetext[40]{Emended from R, 748 \emph{manni}; the word must be in the nominative, but \emph{manni} is dative.}

\bvb Webthrithen quoth: \\ “No man may ever know\footnotemark[119], what thou in days of yore said into the ear of the son. With a death-doomed\footnotemark[120] mouth, I spoke my ancient staves, and about the \textbf{fates of the Powers}\footnotemark[121]. Now against Weden, I have shared my word-wisdom\footnotemark[122]; thou art forever the wisest of men\footnotemark[123]." \\
\footnotetext[119]{Lit. “eternally no man knows".}
\footnotetext[120]{Or “\emph{fey} mouth". Webthrithen here realizes that he was bound to die from the moment (v. 19) he proposed the wager; no man, nor god, nor ettin can outwit Weden.}
\footnotetext[121]{“fates of the Powers", that is, \emph{ragna rǫk}.}
\footnotetext[122]{The same word-wisdom Weden in v. 5 set out to try.}
\footnotetext[123]{Word used is \emph{ver} ‘husband', ‘man'. Perhaps in the broader sense of ‘male being'.}
