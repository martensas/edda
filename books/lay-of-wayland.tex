Níðuðr hét konungr í Svíþjóð. Hann átti tvá sonu ok eina dóttur. Hon hét Böðvildr. Bræðr váru þrír, synir Finnakonungs. Hét einn Slagfiðr, annarr Egill, þriði Völundr. Þeir skriðu ok veiddu dýr. Þeir kómu í Úlfdali ok gerðu sér þar hús. Þar er vatn, er heitir Úlfsjár. Snemma of morgin fundu þeir á vatnsströndu konur þrjár, ok spunnu lín. Þar váru hjá þeim álftarhamir þeira. Þat váru valkyrjur. Þar váru tvær dætr Hlöðvés konungs, Hlaðguðr svanhvít ok Hervör alvitr, in þriðja var Ölrún Kjársdóttir af Vallandi. Þeir höfðu þær heim til skála með sér. Fekk Egill Ölrúnar, en Slagfiðr Svanhvítrar, en Völundr Alvitrar. Þau bjuggu sjau vetr. Þá flugu þær at vitja víga ok kómu eigi aftr. Þá skreið Egill at leita Ölrúnar, en Slagfiðr leitaði Svanhvítrar, en Völundr sat í Úlfdölum. Hann var hagastr maðr, svá at menn viti, í fornum sögum. Níðuðr konungr lét hann höndum taka, svá sem hér er um kveðit:\\%E

\begin{verse}
\bva Męyjar flugu sunnan
Myrkvið í gǫgnum
alvitr ungar,
ørlǫg drýgja;
þær á sævarstrǫnd
sęttusk at hvílask
drósir suðrœnar,
dýrt lín spunnu.\\%E
\end{verse}

\bvb 1

\begin{verse}
\bva Ęin nam þęira
Ęgil at vęrja
fǫgr mær fira
faðmi ljósum.
Ǫnnur vas Svanhvít,
svanfjaðrar dró,
— — — —
ęn hin þriðja
þęira systir
varði hvítan
hals Vǫlundar.\\%E
\end{verse}

\bvb 2

\begin{verse}
\bva Sǫ́tu síðan
sjau vetr at þat,
ęn hinn átta
allan þrǫ́ðu,
ęn hinn níunda
nauðr of skilði,
męyjar fýstusk
á myrkvan við.
Alvitr ungar
ørlǫg drýgja.\\%E
\end{verse}

\bvb 3

\begin{verse}
\bva Kom þar af vęiði
veðręygr skyti
Vǫlundr líðandi
of langan veg,
Slagfiðr ok Ęgill,
sali fundu auða,
gingu út ok inn
ok umb sǫ́usk.\\%E
\end{verse}

\bvb 4

\begin{verse}
\bva Austr skręið Ęgill
at Ǫlrúnu,
ęn suðr Slagfiðr
at Svanhvítu,
ęn ęinn Vǫlundr
sat í Ulfdǫlum.\\%E
\end{verse}

\bvb 5

\begin{verse}
\bva Hann sló goll rautt
við gim fastan,
lukði hann alla
lind baugum vęl;
svá bęið hann
sinnar ljóssar
kvánar, ef hǫ́num
of koma gęrði.\\%E
\end{verse}

\bvb 6

\begin{verse}
\bva Þat spyrr Níðuðr,
Níara dróttinn,
at ęinn Vǫlundr
sat í Ulfdǫlum;
nóttum fóru sęggir,
nęglðar vǫ́ru brynjur,
skildir bliku þęira
við hinn skarða mána.\\%E
\end{verse}

\bvb 7

\begin{verse}
\bva Stigu ór sǫðlum
at salar gafli,
gingu inn þaðan
ęndlangan sal,
sǫ́u þęir á bast
bauga dręgna,
sjau hundruð allra,
es sá sęggr átti.\\%E
\end{verse}

\bvb 8

\begin{verse}
\bva Ok þęir af tóku
ok þęir á létu
fyr ęinn útan,
es af létu;
kom þar af vęiði
veðręygr skyti
Vǫlundr líðandi
of langan veg.\\%E
\end{verse}

\bvb 9

\begin{verse}
\bva Gekk brátt inni
beru hold stęikja,
ár brann hrísi
allþurru fúrr,
viðr hinn vindþurri,
fyr Vǫlundi.\\%E
\end{verse}

\bvb 10

\begin{verse}
\bva Sat á berfjalli,
bauga talði
alfa ljóði,
ęins saknaði.
hugði at hęfði *
Hlǫðvés dóttir, *
Alvitr unga, *
væri aptr komin. *
 
\bvb 11

Sat hann svá lęngi,
at hann sofnaði,
ok hann viljalauss
of vaknaði;
vissi sér á hǫndum
hǫfgar nauðir,
ęn á fótum
fjǫtur of spęntan.\\%E
\end{verse}

\bvb 12

\begin{verse}
\bva (Vǫlundr kvað)
Hvęrir ro jǫfrar
þęir's á lǫgðu
bęstisíma
ok bundu mik?\\%E
\end{verse}

\bvb 13

\begin{verse}
\bva (Kallaði Níðuðr,
Níara dróttinn):
Hvar gazt Vǫlundr,
vísi alfa,
óra aura,
í Ulfdǫlum?
Goll vas þar ęigi
á Grana lęiðu,
fjarri hugðak várt land
fjǫllum Rínar.\\%E
\end{verse}

\bvb 14

\begin{verse}
\bva Mank at męiri
mæti ǫ́ttum,
es vér hęil hjú
hęima vǫ́rum.
Hlaðguðr ok Hervǫr *
borin vas Hlǫðvé, *
kunn vas Ǫlrún *
Kíars dóttir. *\\%E
\end{verse}

\bvb 15

\begin{verse}
\bva [Úti stóð kunnig
kvǫ́n Níðaðar],
hón inn of gekk
ęndlangan sal,
stóð á golfi,
stilti rǫddu:
esa sá nú hýrr,
es ór holti fęrr.\\%E
\end{verse}

\bvb 16

\begin{verse}
\bva Ǫ́mun eru augu
ormi hinum frána,
tęnn hǫ́num tęygjask,
es tét es sverð
ok hann Bǫðvildar *
baug of þękkir, *
sníðið hann sina
sinna magni,
sętið hann síðan
í Sævarstǫð.\\%E
\end{verse}

\bvb 17

\begin{verse}
\bva Svá var gǫrt, at skornar váru sinar í knésfótum ok settr í holm einn, er þar var fyrir landi, er hét Sævarstaðr. Þar smíðaði hann konungi allskyns gǫrsimar; engi maðr þorði at fara til hans, nema konungr einn. Vǫlundr kvað:\\%E
\end{verse}

\bvb B

\begin{verse}
\bva Sék Níðaði
sverð á linda,
þats ek hvęsta
sęm hagast kunnak
ok ek hęrðak
sęm hœgst þótti;
sá ’s mér fránn mækir
æ fjarri borinn.
sékka þann Vǫlundi *
til smiðju borinn. *\\%E
\end{verse}

\bvb 18

\begin{verse}
\bva Nú berr Bǫðvildr *
brúðar minnar, *
bíðka þess bót, *
bauga rauða. *\\%E
\end{verse}

\bvb 19

\begin{verse}
\bva Sat hann né svaf ávalt
ok sló hamri;
vél gęrði hęldr
hvatt Níðaðí;
drifu ungir tvęir
á dýr séa
synir Níðaðar
í Sævarstǫð.\\%E
\end{verse}

\bvb 20

\begin{verse}
\bva Kómu til kistu,
krǫfðu lukla,
opin vas illúð,
es í sǫ́u,
fjǫlð vas þar męina,
es mǫgum sýndisk
at væri goll rautt
ok gǫrsimar.\\%E
\end{verse}

\bvb 21

\begin{verse}
\bva Komið ęinir tvęir,
komið annars dags;
ykr lætk þat goll
of gefit verða;
sęgiða męyjum
né salþjóðum,
manni ęngum,
at mik fyndið.\\%E
\end{verse}

\bvb 22

\begin{verse}
\bva Snimma kallaði
sęggr á annan,
bróðir á bróður:
gǫngum baug séa.
Kómu til kistu,
krǫfðu lukla,
opin vas illúð
es í litu.\\%E
\end{verse}

\bvb 23

\begin{verse}
\bva Snęið af hǫfuð
húna þęira
ok und fęn fjǫturs
fœtr of lagði,
ęn þær skálar,
es und skǫrum vǫ́ru,
svęip útan silfri,
sęldi Níðaði.\\%E
\end{verse}

\bvb 24

\begin{verse}
\bva Ęn ór augum
jarknastęina
sęndi kunnigri
kvǫ́n Níðaðar;
ęn ór tǫnnum
tvęggja þęira
sló brjóstkringlur,
sęndi Bǫðvildi.\\%E
\end{verse}

\bvb 25

\begin{verse}
\bva Þá nam Bǫðvildr
baugi at hrósa
— — — —
es brotit hafði,
»þorigak sęgja,
nema þér ęinum.« \\%E
\end{verse}

\bvb 26

\begin{verse}
\bva Ek bœti svá
brest á golli,
at fęðr þínum
fęgri þykkir,
ok mœðr þinni
miklu bętri,
ok sjalfri þér
at sama hófi.« \\%E
\end{verse}

\bvb 27

\begin{verse}
\bva Bar hann hána bjóri,
þvíat hann bętr kunni,
svát hón í sessi
of sofnaði.
»Nú hęfk hęfnt
harma minna
allra nema ęinna
íviðgjǫrnum.\\%E
\end{verse}

\bvb 28

\begin{verse}
\bva Vęl ek, kvað Vǫlundr,
verðak á fitjum,
þęims mik Níðaðar
nǫ́mu rekkar.« 
Hlæjandi Vǫlundr
hófsk at lopti,
grátandi Bǫðvildr
gekk ór ęyju.
tregði fǫr friðils *
ok fǫður reiði. *\\%E
\end{verse}

\bvb 29

\begin{verse}
\bva Úti stóð kunnig
kvǫ́n Níðaðar,
ok hón inn of gekk
ęndlangan sal,
— ęn hann á salgarð
sęttisk at hvílask —,
»Vakir þú Níðuðr,
Níara dróttinn?« \\%E
\end{verse}

\bvb 30

\begin{verse}
\bva Vaki'k ávalt
viljalauss,
sofna'k minst,
síz sonu dauða,
kęll mik í hǫfuð,
kǫld erumk rǫ́ð þín,
vilnumk þess nú,
at við Vǫlund dœma'k.\\%E
\end{verse}

\bvb 31

\begin{verse}
\bva Sęg mér þat Vǫlundr,
vísi alfa,
af hęilum hvat varð
húnum mínum?\\%E
\end{verse}

\bvb 32

\begin{verse}
\bva Ęiða skalt mér áðr
alla vinna,
at skips borði
ok at skjaldar rǫnd,
at mars bœgi
ok at mækis ęgg
at þú kvęljat
kvǫ́n Vǫlundar,
né brúði minni
at bana verðir,
þótt kvǫ́n ęigim,
þás ér kunnið,
eða jóð ęigim
innan hallar.\\%E
\end{verse}

\bvb 33

\begin{verse}
\bva Gakk til smiðju,
es gęrðir þú,
þar fiðr þú bęlgi
blóði stokna,
snęiðk af hǫfuð
húna þinna
ok und fęn fjǫturs
fœtr of lagðak.\\%E
\end{verse}

\bvb 34

\begin{verse}
\bva Ęn þær skálar,
es und skǫrum vǫ́ru,
svęipk útan silfri,
sęldak Níðaði,
ęn ór augum
jarknastęina,
sęndak kunnigri
kvǫ́n Níðaðar.\\%E
\end{verse}

\bvb 35

\begin{verse}
\bva Ęn ór tǫnnum
tvęggja þęira
sló'k brjóstkringlur,
sęnda'k Bǫðvildi;
nú gęngr Bǫðvildr
barni aukin,
ęingadóttir
ykkur bęggja.\\%E
\end{verse}

\bvb 36

\begin{verse}
\bva Mæltir-a þú þat mál,
es mik męir tregi,
né þik vilja'k Vǫlundr
verr of níta;
es-at svá maðr hór,
at þik af hęsti taki,
né svá ǫflugr,
at þik neðan skjóti.
þars þú skollir
við ský uppi.\\%E
\end{verse}

\bvb 37

\begin{verse}
\bva Hlæjandi Vǫlundr
hófsk at lopti,
ęn ókátr Níðuðr
þá ęptir sat.\\%E
\end{verse}

\bvb 38

\begin{verse}
\bva Upp rís Þakkráðr,
þræll minn bazti,
bið Bǫðvildi,
męy hina bráhvítu,
gangi fagrvarið
við fǫður rœða.\\%E
\end{verse}

\bvb 39

\begin{verse}
\bva Es þat satt Bǫðvildr,
es sǫgðu mér,
sǫ́tuð it Vǫlundr
saman í holmi?\\%E
\end{verse}

\bvb Is it true, Beadhild, what they said to me: Sat thou and Wayland together on the island?”

\begin{verse}
\bva Satt ’s þat Níðuðr \hld es sagði þér.
Sǫ́tum vit Vǫlundr \hld saman í holmi
ęina ǫgurstund, \hld æva skyldi;
ek vætr hǫ́num \hld vinna kunnak,
ek vætr hǫ́num \hld vinna máttak.
\end{verse}

\bvb It is true, Nithad, what \emph{he} said to thee; I and Wayland sat together on the island, for a heavy moment; it should never [have been]. I knew nought struggle against him; I could nought struggle against him.
