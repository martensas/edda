% The Old Norse version currently used for this poem is that of Finnur Jónsson. It should be treated as such.

Níðuðr hét konungr í Svíþjóð. Hann átti tvá sonu ok eina dóttur. Hon hét Böðvildr. Bræðr váru þrír, synir Finnakonungs. Hét einn Slagfiðr, annarr Egill, þriði Völundr. Þeir skriðu ok veiddu dýr. Þeir kómu í Úlfdali ok gerðu sér þar hús. Þar er vatn, er heitir Úlfsjár. Snemma of morgin fundu þeir á vatnsströndu konur þrjár, ok spunnu lín. Þar váru hjá þeim álftarhamir þeira. Þat váru valkyrjur. Þar váru tvær dætr Hlöðvés konungs, Hlaðguðr svanhvít ok Hervör alvitr, in þriðja var Ölrún Kjársdóttir af Vallandi. Þeir höfðu þær heim til skála með sér. Fekk Egill Ölrúnar, en Slagfiðr Svanhvítrar, en Völundr Alvitrar. Þau bjuggu sjau vetr. Þá flugu þær at vitja víga ok kómu eigi aftr. Þá skreið Egill at leita Ölrúnar, en Slagfiðr leitaði Svanhvítrar, en Völundr sat í Úlfdölum. Hann var hagastr maðr, svá at menn viti, í fornum sögum. Níðuðr konungr lét hann höndum taka, svá sem hér er um kveðit: \\%E

\begin{verse}
\bva Męyjar flugu sunnan \hld Myrkvið í gǫgnum
alvitr ungar, \hld ørlǫg drýgja;
þær á sævarstrǫnd \hld sęttusk at hvílask
drósir suðrǿnar, \hld dýrt lín spunnu. \\%E
\end{verse}

\bvb 1

\begin{verse}
\bva Ęin nam þęira \hld Ęgil at vęrja
fǫgr mær fira \hld faðmi ljósum.
Ǫnnur vas Svanhvít, \hld svanfjaðrar dró,
ęn hin þriðja \hld þęira systir
varði hvítan \hld hals Vǫlundar. \\%E
\end{verse}

\bvb 2

\begin{verse}
\bva Sǫ́tu síðan \hld sjau vetr at þat,
ęn hinn átta \hld allan þrǫ́ðu,
ęn hinn níunda \hld nauðr of skilði,
męyjar fýstusk \hld á myrkvan við.
Alvitr ungar \hld ørlǫg drýgja. \\%E
\end{verse}

\bvb 3

\begin{verse}
\bva Kom þar af vęiði \hld veðręygr skyti
Vǫlundr líðandi \hld of langan veg,
Slagfiðr ok Ęgill, \hld sali fundu auða,
gingu út ok inn \hld ok umb sǫ́usk. \\%E
\end{verse}

\bvb 4

\begin{verse}
\bva Austr skręið Ęgill \hld at Ǫlrúnu,
ęn suðr Slagfiðr \hld at Svanhvítu,
ęn ęinn Vǫlundr \hld sat í Ulfdǫlum. \\%E
\end{verse}

\bvb 5

\begin{verse}
\bva Hann sló goll rautt \hld við gim fastan,
lukði hann alla \hld lind baugum vęl;
svá bęið hann \hld sinnar ljóssar
kvánar, ef hǫ́num \hld of koma gęrði. \\%E
\end{verse}

\bvb 6

\begin{verse}
\bva Þat spyrr Níðuðr, \hld Níara dróttinn,
at ęinn Vǫlundr \hld sat í Ulfdǫlum;
nóttum fóru sęggir, \hld nęglðar vǫ́ru brynjur,
skildir bliku þęira \hld við hinn skarða mána. \\%E
\end{verse}

\bvb 7

\begin{verse}
\bva Stigu ór sǫðlum \hld at salar gafli,
gingu inn þaðan \hld ęndlangan sal,
sǫ́u þęir á bast \hld bauga dręgna,
sjau hundruð allra, \hld es sá sęggr átti. \\%E
\end{verse}

\bvb 8

\begin{verse}
\bva Ok þęir af tóku \hld ok þęir á létu
fyr ęinn útan, \hld es af létu;
kom þar af vęiði \hld veðręygr skyti
Vǫlundr líðandi \hld of langan veg. \\%E
\end{verse}

\bvb 9

\begin{verse}
\bva Gekk brátt inni \hld beru hold stęikja,
ár brann hrísi \hld allþurru fúrr,
viðr hinn vindþurri, \hld fyr Vǫlundi. \\%E
\end{verse}

\bvb 10

\begin{verse}
\bva Sat á berfjalli, \hld bauga talði
alfa ljóði, \hld ęins saknaði.
hugði at hęfði \hld Hlǫðvés dóttir,
Alvitr unga, \hld væri aptr komin.
 
\bvb 11

Sat hann svá lęngi, \hld at hann sofnaði,
ok hann viljalauss \hld of vaknaði;
vissi sér á hǫndum \hld hǫfgar nauðir,
ęn á fótum \hld fjǫtur of spęntan. \\%E
\end{verse}

\bvb 12

\begin{verse}
(Vǫlundr kvað)
\bva Hvęrir ro jǫfrar \hld þęir's á lǫgðu
bęstisíma \hld ok bundu mik? \\%E
\end{verse}

\bvb 13

\begin{verse}
\bva (Kallaði Níðuðr, \hld Níara dróttinn):
Hvar gazt Vǫlundr, \hld vísi alfa,
óra aura, \hld í Ulfdǫlum?
Goll vas þar ęigi \hld á Grana lęiðu,
fjarri hugðak várt land \hld fjǫllum Rínar. \\%E
\end{verse}

\bvb 14

\begin{verse}
\bva Mank at męiri \hld mæti ǫ́ttum,
es vér hęil hjú \hld hęima vǫ́rum.
Hlaðguðr ok Hervǫr \hld borin vas Hlǫðvé,
kunn vas Ǫlrún \hld Kíars dóttir.  \\%E
\end{verse}

\bvb 15

\begin{verse}
\bva [Úti stóð kunnig \hld kvǫ́n Níðaðar],
hón inn of gekk \hld ęndlangan sal,
stóð á golfi, \hld stilti rǫddu:
es-a sá nú hýrr, \hld es ór holti fęrr. \\%E
\end{verse}

\bvb 16

\begin{verse}
\bva Ǫ́mun eru augu \hld ormi hinum frána,
tęnn hǫ́num tęygjask, \hld es tét es sverð
ok hann Bǫðvildar \hld baug of þękkir,
sníðið hann sina \hld sinna magni,
sętið hann síðan \hld í Sævarstǫð. \\%E
\end{verse}

\bvb 17

\begin{verse}
\bva Svá var gǫrt, at skornar váru sinar í knésfótum ok settr í holm einn, er þar var fyrir landi, er hét Sævarstaðr. Þar smíðaði hann konungi allskyns gǫrsimar; engi maðr þorði at fara til hans, nema konungr einn. Vǫlundr kvað: \\%E
\end{verse}

\bvb B

\begin{verse}
\bva Sék Níðaði \hld sverð á linda,
þats ek hvęsta \hld sęm hagast kunnak
ok ek hęrðak \hld sęm hǿgst þótti;
sá 's mér fránn mækir \hld æ fjarri borinn.
sékka þann Vǫlundi \hld til smiðju borinn. \\%E
\end{verse}

\bvb 18

\begin{verse}
\bva Nú berr Bǫðvildr \hld brúðar minnar,
bíðka þess bót, \hld bauga rauða. \\%E
\end{verse}

\bvb 19

\begin{verse}
\bva Sat hann né svaf ávalt \hld ok sló hamri;
vél gęrði hęldr \hld hvatt Níðaðí;
drifu ungir tvęir \hld á dýr séa
synir Níðaðar \hld í Sævarstǫð. \\%E
\end{verse}

\bvb 20

\begin{verse}
\bva Kómu til kistu, \hld krǫfðu lukla,
opin vas illúð, \hld es í sǫ́u,
fjǫlð vas þar męina, \hld es mǫgum sýndisk
at væri goll rautt \hld ok gǫrsimar. \\%E
\end{verse}

\bvb 21

\begin{verse}
\bva Komið ęinir tvęir, \hld komið annars dags;
ykr lætk þat goll \hld of gefit verða;
sęgið-a męyjum \hld né salþjóðum,
manni ęngum, \hld at mik fyndið. \\%E
\end{verse}

\bvb 22

\begin{verse}
\bva Snimma kallaði \hld sęggr á annan,
bróðir á bróður: \hld gǫngum baug séa.
Kómu til kistu, \hld krǫfðu lukla,
opin vas illúð \hld es í litu. \\%E
\end{verse}

\bvb 23

\begin{verse}
\bva Snęið af hǫfuð \hld húna þęira
ok und fęn fjǫturs \hld fǿtr of lagði,
ęn þær skálar, \hld es und skǫrum vǫ́ru,
svęip útan silfri, \hldsęldi Níðaði. \\%E
\end{verse}

\bvb 24

\begin{verse}
\bva Ęn ór augum \hld jarknastęina
sęndi kunnigri \hld kvǫ́n Níðaðar;
ęn ór tǫnnum \hld tvęggja þęira
sló brjóstkringlur, \hld sęndi Bǫðvildi. \\%E
\end{verse}

\bvb 25

\begin{verse}
\bva Þá nam Bǫðvildr \hld baugi at hrósa
— — — \hld es brotit hafði,
»þorigak sęgja, \hld nema þér ęinum.«  \\%E
\end{verse}

\bvb 26

\begin{verse}
\bva Ek bǿti svá \hld brest á golli,
at fęðr þínum \hld fęgri þykkir,
ok mǿðr þinni \hld miklu bętri,
ok sjalfri þér \hld at sama hófi.«  \\%E
\end{verse}

\bvb 27

\begin{verse}
\bva Bar hann hána bjóri, \hld þvíat hann bętr kunni,
svát hón í sessi \hld of sofnaði.
»Nú hęfk hęfnt \hld harma minna
allra nema ęinna \hld íviðgjǫrnum. \\%E
\end{verse}

\bvb 28

\begin{verse}
\bva Vęl ek, kvað Vǫlundr, \hld verðak á fitjum,
þęims mik Níðaðar \hld nǫ́mu rekkar.« 
Hlæjandi Vǫlundr \hld hófsk at lopti,
grátandi Bǫðvildr \hld gekk ór ęyju.
tregði fǫr friðils \hld ok fǫður reiði. \\%E
\end{verse}

\bvb 29

\begin{verse}
\bva Úti stóð kunnig \hld kvǫ́n Níðaðar,
ok hón inn of gekk \hld ęndlangan sal,
— ęn hann á salgarð \hld sęttisk at hvílask —,
»Vakir þú Níðuðr, \hld Níara dróttinn?«  \\%E
\end{verse}

\bvb 30

\begin{verse}
\bva Vaki'k ávalt \hld viljalauss,
sofna'k minst, \hld síz sonu dauða,
kęll mik í hǫfuð, \hld kǫld erumk rǫ́ð þín,
vilnumk þess nú, \hld at við Vǫlund dǿma'k. \\%E
\end{verse}

\bvb 31

\begin{verse}
\bva Sęg mér þat Vǫlundr, \hld vísi alfa,
af hęilum hvat varð \hld húnum mínum? \\%E
\end{verse}

\bvb 32

\begin{verse}
\bva Ęiða skalt mér áðr \hld alla vinna,
at skips borði \hld ok at skjaldar rǫnd,
at mars bǿgi \hld ok at mækis ęgg
at þú kvęlj-at \hld kvǫ́n Vǫlundar,
né brúði minni \hld at bana verðir,
þótt kvǫ́n ęigim, \hld þás ér kunnið,
eða jóð ęigim \hld innan hallar. \\%E
\end{verse}

\bvb 33

\begin{verse}
\bva Gakk til smiðju, \hld es gęrðir þú,
þar fiðr þú bęlgi \hld blóði stokna,
snęiðk af hǫfuð \hld húna þinna
ok und fęn fjǫturs \hld fǿtr of lagðak. \\%E
\end{verse}

\bvb 34

\begin{verse}
\bva Ęn þær skálar, \hld es und skǫrum vǫ́ru,
svęipk útan silfri, \hld sęldak Níðaði,
ęn ór augum \hld jarknastęina,
sęndak kunnigri \hld kvǫ́n Níðaðar. \\%E
\end{verse}

\bvb 35

\begin{verse}
\bva Ęn ór tǫnnum \hld tvęggja þęira
sló'k brjóstkringlur, \hld sęnda'k Bǫðvildi;
nú gęngr Bǫðvildr \hld barni aukin,
ęingadóttir \hld ykkur bęggja. \\%E
\end{verse}

\bvb 36

\begin{verse}
\bva Mæltir-a þú þat mál, \hld es mik męir tregi,
né þik vilja'k Vǫlundr \hld verr of níta;
es-at svá maðr hór, \hld at þik af hęsti taki,
né svá ǫflugr, \hld at þik neðan skjóti.
þars þú skollir \hld við ský uppi. \\%E
\end{verse}

\bvb 37

\begin{verse}
\bva Hlæjandi Vǫlundr \hld hófsk at lopti,
ęn ókátr Níðuðr \hld þá ęptir sat. \\%E
\end{verse}

\bvb 38

\begin{verse}
\bva Upp rís Þakkráðr, \hld þræll minn bazti,
bið Bǫðvildi, \hld męy hina bráhvítu,
gangi fagrvarið \hld við fǫður rǿða. \\%E
\end{verse}

\bvb 39

\begin{verse}
\bva Es þat satt Bǫðvildr, \hld es sǫgðu mér,
sǫ́tuð it Vǫlundr \hld saman í holmi? \\%E
\end{verse}

\bvb Is it true, Beadhild, what they said to me: Sat thou and Wayland together on the island?” \\

\begin{verse}
\bva Satt 's þat Níðuðr \hld es sagði þér.
Sǫ́tum vit Vǫlundr \hld saman í holmi
ęina ǫgurstund, \hld æva skyldi;
ek vætr hǫ́num \hld vinna kunnak,
ek vætr hǫ́num \hld vinna máttak.
\end{verse}

\bvb It is true, Nithad, what \emph{he} said to thee; I and Wayland sat together on the island, for a heavy moment; it should never [have been]. I knew nought struggle against him; I could nought struggle against him.
