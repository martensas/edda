\book{The Greenlandish Lay of Ettle. (Atlakviða in grǿnlenzka)}\bookStart

Guthrun, the daughter of Givick, avenged her brothers as has become famous. She first killed the sons of Ettle, and after that she killed Ettle, and burned the hall and the whole hird. Regarding that this lay is wrought.

\begin{verse}
\bva Atli sęndi \hld ár til Gunnars
kunnan sęgg at ríða, \hld Knéfrøðr vas sá hęitinn;
at gǫrðum kom hann Gjúka \hld ok at Gunnars hǫllu,
bękkjum aringręypum \hld ok at bjóri svásum. 
\end{verse}

\bvb Early, Ettle sent to Guthhere a well-known man to ride; Kneefrod that one was called. To the estate of Giveck he arrived, and to the hall of Guthhere, to the hearth-surrounding benches, and to the lovely beer.

\begin{verse}
\bva Drukku þar dróttmęgir \hld —ęn dyljęndr þǫgðu—
vín í valhǫllu, \hld vręiði sásk þęir Húna;
kallaði þá Knéfrøðr \hld kaldri rǫddu,
sęggr inn suðrǿni \hld sat hann á bekk hám: 
\end{verse}

\bvb There the dright-lads drank—but the concealed ones were silent—wine in the walhall; they were wary of the wrath of the Huns. Then Kneefrod, the southern man, called with cold voice; he sat on a high bench:

\begin{verse}
\bva “Atli mik hingat sęndi \hld ríða ørendi,
mar inum mélgręypa, \hld Myrkvið inn ókunna
at biðja yðr, Gunnarr, \hld at it á bękk kǿmið
með hjǫlmum aringręypum \hld at sǿkja hęim Atla. 
\end{verse}

\bvb “Ettle sent me hither, to ride an errand, with the bit-champing horse through the uncharted Mirkwood, to ask you, Guthhere, that ye two might come onto the bench, with hearth-surrounding helmets, to visit the home of Ettle.

\begin{verse}
\bva Skjǫldu kneguð þar vęlja \hld ok skafna aska,
hjalma gullroðna \hld ok Húna męngi,
silfrgyllt sǫðulklæði, \hld sęrki valrauða,
dafar, darraða, \hld drǫsla mélgręypa. 
\end{verse}

\bvb There ye might choose shields, and smooth spears, gold-reddened helmets, and the multitude of the Huns, silver-gilt saddle-cloths, walred serks, dafs, standards, bit-champing steeds.

\begin{verse}
\bva Vǫll lézk ykkr ok myndu gefa \hld víðrar Gnitahęiðar
af gęiri gjallanda \hld ok af gylltum stǫfnum,
stórar męiðmar \hld ok staði Danpar,
hrís þat it mæra \hld es meðr Myrkvið kalla. 
\end{verse}

\begin{verse}
\bva Hǫfði vatt þá Gunnarr \hld ok Hǫgna til sagði:
Hvat ræðr þú okkr, sęggr inn ǿri, \hld allz vit slíkt heyrum?
Gull vissa ek ekki \hld á Gnitahęiði,
þat es vit ættim-a \hld annat slíkt.
\end{verse}

\bvb Then Guthhere turned his head, and said to Hayn: “What does thou advise us, younger man, as we hear such things? I knew of no gold on the Gnitheath, that we did not own as much of.

\begin{verse}
\bva Sjau ęigu vit salhús \hld sverða full,
hvęrju eru þęira \hld hjǫlt ór gulli;
mínn vęit ek mar bęztan \hld ęn mæki hvassastan,
boga bękksǿma \hld ęn brynjur ór gulli. 
\end{verse}

\bvb We own seven hallhouses, filled with swords—each one of them has a hilt of gold; I know my horse to be the best, and my sword the sharpest; my bow bench-fit, and my byrnies of gold.

\begin{verse}
\bva Hjalm ok skjǫld hvítastan, \hld kominn ór hǫll Kjárs;
ęinn es mínn bętri \hld ęn sé allra Húna. 
\end{verse}

\bvb Helmet and whitest shield, come from the hall of Chear; mine alone is better, than all of the Huns might be.”

\begin{verse}
\bva Hvat hyggr þú brúði bęndu \hld þá es hón okkr baug sęndi,
varinn váðum hęiðingja? \hld Hygg ek at hón vǫrnuð byði!
Hár fann ek hęiðingja \hld riðit í hring rauðum;
ylfskr es vegr okkarr \hld at ríða ørendi. 
\end{verse}

\bvb “What does thou think the bride meant, when she sent us an armlet, wrapped in the weeds of a heathing [WOLF]? I think that she bid us a warning! I found the hair of a heathing wrapped round the red ring; wolfish is our road, riding the errand.”

\begin{verse}
\bva Niðjar-gi hvǫttu Gunnar \hld né náungr annarr,
rýnendr né ráðendr, \hld né þęir es ríkir vǫ́ru;
kvaddi þá Gunnarr \hld sęm konungr skyldi,
mærr í mjǫðranni \hld af móði stórum: 
\end{verse}

\bvb Kinsmen did not urge Guthhere, nor any other close one, nor advisors nor counselors, nor those who were powerful. Guthhere then responded, as a king should, renowned in the mead-house, with great courage:

\begin{verse}
\bva Rís-tu nú, Fjǫrnir, \hld lát-tu á flęt vaða
gręppa gullskálir \hld með gumna hǫndum! 
\end{verse}

\bvb “Rise now, Feren. Let on the floorboards wade forth the golden bowls of warriors, along the hands of men!

\begin{verse}
\bva Ulfr mun ráða \hld arfi Niflunga,
gamlir granvarðir, \hld ef Gunnars missir,
birnir blakkfjallir \hld bíta þreftǫnnum,
gamna gręystóði, \hld ef Gunnarr né kømr-at. 
\end{verse}

\bvb The wolf will rule the inheritance of the Niflings, the old grey guardians, if Guthhere is not there. Black-haired bears [will] bite with wrangling teeth, amusing the pack of bitches, if Guthhere does not come.”

\begin{verse}
\bva Lęiddu landrǫgni \hld lýðar ónęisir,
grátęndr, gunnhvatan, \hld ór garði Húna;
þá kvað þat inn ǿri \hld ęrfivǫrðr Hǫgna:
Hęilir farið nú ok horskir \hld hvar’s ykkr hugr tęygir! 
\end{verse}

\begin{verse}
\bva Fetum létu frǿknir \hld um fjǫll at þyrja
marina mélgręypu, \hld Myrkvið inn ókunna;
hristisk ǫll Húnmǫrk \hld þar es harðmóðgir fóru,
vrǫ́ku þęir vannstyggva \hld vǫllu algrǿna. 
\end{verse}

\begin{verse}
\bva Land sá þęir Atla \hld ok liðskjalfar djúpar
Bikka greppar standa \hld á borg inni há—
sal um suðrþjóðum, \hld slęginn sessmeiðum,
bundnum rǫndum, \hld blęikum skjǫldum,
dafar, darraða; \hld en þar drakk Atli
vín í valhǫllu; \hld vęrðir sǫ́tu úti
at varða þęim Gunnari \hld ef þęir hér vitja kvǿmi
með gęiri gjallanda \hld at vękja gram hildi. 
\end{verse}

\begin{verse}
\bva Systir fann þęira snemmst \hld at þęir í sal kvǫ́mu,
brǿðr hennar báðir, \hld bjóri var hón lítt drukkin:
Ráðinn ert-u nú, Gunnarr, \hld hvat munt-u, ríkr, vinna
við Húna harmbrǫgðum? \hld Hǫll gakk þú ór snemma! 
\end{verse}

\begin{verse}
\bva Betr hęfðir þú, bróðir, \hld at þú í brynju fǿrir,
sem hjǫlmum aringręypum \hld at sjá, hęim Atla;
sætir þú í sǫðlum \hld sólhęiða daga,
nái nauðfǫlva \hld létir nornir gráta. 
\end{verse}

\begin{verse}
\bva Húna skjaldmęyjar \hld herfi kanna
ęn Atla sjalfan \hld létir þú í ormgarð koma;
nú es sá ormgarðr \hld ykkr um folginn.
\end{verse}

\begin{verse}
\bva Seinað es nú, systir, \hld at samna Niflungum,
langt es at lęita \hld lýða sinnis til,
of rosmufjǫll Rínar, \hld rekka ónęissa. 
\end{verse}

\begin{verse}
\bva Fengu þęir Gunnar \hld ok í fjǫtur sęttu,
vinir Borgunda, \hld ok bundu fastla;
sjau hjó Hǫgni \hld sverði hvǫssu
en inum átta hratt hann \hld í eld hęitan;
svá skal frǿkn \hld fjándum vęrjask! 
\end{verse}

\begin{verse}
\bva Hǫgni varði \hld hęndr Gunnars;
frǫ́gu frǿknan \hld ef fjǫr vildi
Gotna þjóðann \hld gulli kaupa. 
\end{verse}

\bva Hjarta skal mér Hǫgna \hld í hęndi liggja  \\%M
blóðugt, ór brjósti \hld skorit baldriða, \\%M
saxi slíðrbęitu, \hld syni þjóðans.\\%E
\end{verse}

\bvb (Guthhere quoth:) \\ “The heart of Hayne shall in my hands lie, bloody out of the breast, cut from the bold rider; with a razor-sharp seax, from the son of the sovereign.”

\begin{verse}
\bva Skǫ́ru þęir hjarta \hld Hjalla ór brjósti \\%M
blóðugt ok á bjóð lǫgðu \hld ok bǫ́ru þat fyr Gunnar.\\%E
\end{verse}

\bvb They cut the heart of Helle out of the breast; bloody on a platter they laid it, and carried it before Guthhere.

\begin{verse}
\bva Þá kvað þat Gunnarr, \hld gumna dróttinn: \\%M
Hér hęfi ek hjarta \hld Hjalla ins blauða, \\%M
ólíkt hjarta \hld Hǫgna ins frǿkna, \\%M
es mjǫk bifask \hld es á bjóði liggr; \\%M
bifðisk hǫlfu męirr \hld es í brjósti lá!\\%E
\end{verse}

\bvb Then Guthhere quoth this, the lord of men: “Here have I the heart of Helle the soft—unlike the heart of Hayne the bold!—which much trembles when on the platter it lies; it trembled twice as much when in the breast it lay.”

\begin{verse}
\bva Hló þá Hǫgni \hld es til hjarta skǫ́ru \\%M
kvikvan kumblasmið \hld kløkkva hann sízt hugði \\%M
blóðugt þat á bjóð lǫgðu \hld ok bǫ́ru fyr Gunnar.\\%E
\end{verse}

\bvb Hayne laughed then, when to the heart they cut; the living landmark-smith thought least of sobbing. Bloody on a platter they laid it, and carried it before Guthhere.

\begin{verse}
\bva Mærr kvað þat Gunnarr, \hld Gęir-Niflungr: \\%M
Hér hęfi ek hjarta \hld Hǫgna ins frǿkna, \\%M
ólíkt hjarta \hld Hjalla ins blauða, \\%M
es lítt bifask \hld es á bjóði liggr; \\%M
bifðisk svági mjǫk \hld þá’s í brjósti lá!\\%E
\end{verse}

\bvb Renowned Guthhere quoth this, the Gore-Nifling: “Here have I the heart of Hayne the bold—unlike the heart of Helle the soft!—which little trembles when on the platter it lies; it trembled even less when in the breast it lay.

\begin{verse}
\bva Svá skaltu, Atli, \hld augum fjarri \\%M
sęm munt \hld męnjum verða; \\%M
es und ęinum mér \hld ǫll of folgin \\%M
hodd Niflunga: \hld Lifir-a nú Hǫgni!\\%E
\end{verse}

\bvb Thus shalt thou, Ettle, be as far from the eyes, as thou wilt from the neck-rings. With me alone it is all hidden, the hoard of the Niflings; now Hayne does not live.

\begin{verse}
\bva Ęy vas mér týja \hld  meðan vit tvęir lifðum, \\%M
nú es mér ęngi \hld es ęinn lifi’k; \\%M
Rín skal ráða \hld rógmalmi skatna, \\%M
svinn, ǫ́skunna \hld arfi Niflunga.\\%E
\end{verse}

\bvb I was always doubtful, when we two lived; now I am not, when I alone live. The Rhine shall rule the strife-ore of princes; wisely the os-born inheritance of the Niflings.

\begin{verse}
\bva Í veltanda vatni \hld lýsask valbaugar \\%M
hęldr an á hǫndum gull \hld skíni Húna bǫrnum.\\%E
\end{verse}

\bvb In tumbling water the Welsh rings [will] gleam, rather than [as] gold on hands shine on the children of Huns.”
