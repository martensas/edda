\bookStart{The Speeches of Rein}[Ręginsmǫ́l]

\begin{flushright}%
Dating \parencite{Sapp2022}: C10th (0.666)–early C11th (0.259)

Meter: \Ljodahattr, \Fornyrdislag%
\end{flushright}

Like other poems from this section, it is better defined as a prosimetrum. The differing meter of the verses might suggest that they are taken from different poems.

\sectionline

\bpg\bpa Sigurðr gekk til stóðs Hjálpreks ok kaus sér af hest einn er Grani var kallaðr síðan. Þá var kominn Reginn til Hjálpreks, sonr Hreiðmars. Hann var hverjum manni hagari ok dvergr of vǫxt. Hann var vitr, grimmr ok fjǫlkunnigr. Reginn veitti Sigurði fóstr ok kennslu ok elskaði hann mjǫk. Hann sagði Sigurði frá forellri sínu ok þeim atburðum at Óðinn ok Hǿnir ok Loki hǫfðu komið til Andvarafors; í þeim forsi var fjǫlði fiska. Einn dvergr hét Andvari; hann var lǫngum í forsinum í geddu líki ok fekk sér þar matar. „Otr hét bróðir várr,“ kvað Reginn, „er oft fór í forsinn í otrs líki. Hann hafði tekið einn lax ok sat á árbakkanum ok át blundandi. Loki laust hann með steini til bana. Þóttust ę́sir mjǫk heppnir verið hafa ok flógu belg af otrinum. Þat sama kveld sóttu þeir gisting til Hreiðmars ok sýndu veiði sína. Þá tóku vér þá hǫndum ok lǫgðum þeim fjǫrlausn at fylla otrbelginn með gulli ok hylja útan ok með rauðu gulli. Þá sendu þeir Loka at afla gullsins. Hann kom til Ránar ok fekk net hennar ok fór þá til Andvarafors ok kastaði netinu fyr gedduna en hon hljóp í netið. Þá mę́lti Loki:\epa

\bpb Siward went to Helpric’s stable and chose one horse, which was thereafter called Grane. Then Rein, son of Rethmar, was come to Helpric. He was more skilled than any man and a dwarf in stature. He was wise, cruel and feel-cunning. Rein fostered and taught Siward and love him very much. He told Siward about his own parents, and about the events that Weden, Heener and Lock had come to Andwareforce; in that force was a multitude of fish. A dwarf was named Andware; he was for a long time in the force in the likeness of a pike and got his food there. “Otter was our brother called,” said Rein, “who often journeyed in the force in the likeness of an otter. He had caught a salmon and sat on the riverbank and ate it with closed eyes Lock struck him with a stone unto his death. The Ease thought themselves to have been very lucky, and flayed the skin off the otter. The same evening they sought to pass the night at Rethmare’s house, and showed their catch. Then we bound them and proposed to them as a life-ransom that they would fill the otter-skin with gold, and also coat the outside with red gold. Then they sent Lock to get ahold of the gold. He came to Ran and got her net and then journeyed to Andwareforce and threw the net before the pike, and it jumped into the net. Then Lock spoke:\epb\epg


\bvg
\bva „Hvat ’s þat fiska \hld\ es renn flóði í &
\ind kann-at sér við víti varask; &
hǫfuð þitt \hld\ lęys-tu hęlju ór &
\ind finn mér lindar loga!“\eva

\bvb “TODO.”\evb
\evg


\bvg
\bva „Andvári ec heiti
oiɴ het miɴ faþir
\ind margan hefi ec forſ vm fariþ.
ꜹmlig norn
ſcop os i ardaga
\ind at ec ſcꝩlda i vatni vaþa.“\eva

\bvb “TODO.”\evb
\evg


\bvg
\bva „Sęg-ðu þat, Andvari, (kvað Loki) ef þú ęiga vill &
\ind líf í lýða sǫlum: &
Hvęr gjǫld \hld\ fáa gumna synir &
\ind ef hǫggvask orðum á?“\eva

\bvb “Say that, Andware—quoth Lock—if thou wilt have life in the halls of men:
Which recompense do the sons of men get, if they hew at each other with words?”\evb
\evg


\bvg
\bva „Ofrgjǫld \hld\ fáa gumna synir &
\ind þęir’s Vaðgęlmi vaða; &
ósaðra orða \hld\ hvęrr’s á annan lýgr, &
\ind of lęngi lęiða limar.“\eva

\bvb “Overwhelming recompense do the sons of men get, those who wade in \inx[L]{Wadyelmer}. By the ramifications of untrue words is each who lies to another long followed.\footnoteB{Watery torment in the afterlife for oath-breakers and liars is well attested in the Germanic corpus (including in other poetic stanzas in the pres. ed.). See further note to \Voluspa\ 39.}”\evb
\evg


\bpg\bpa Loki sá allt gull þat er Andvari átti. En er hann hafði fram reitt gullit, þá hafði hann eftir einn hring ok tók Loki þann af hánum. Dvergrinn gekk inn í steininn ok mę́lti:\epa

\bpb Lock saw all the gold which Andware owned. But when he had brought forth all the gold, then he had one ring left, and Lock took it off him. The dwarf went into the stone and spoke:\epb\epg


\bvg
\bva „Þat skal gull \hld\ es Gustr átti &
brǿðrum tvęim \hld\ at bana verða &
ok ǫðlingum \hld\ átta at rógi; &
mun míns féar \hld\ manngi njóta.\footnoteB{Note the change of meter in this st.; it certainly does not originally belong with the previous sts.}“\eva

\bvb “TODO.”\evb
\evg



TODO


\bvg
\bva Kęmbðr ok þvęginn \hld\ skal kǿnna hvęrr &
\ind ok at morni męttr. &
því-at ósýnt es \hld\  hvar at aptni kømr; &
\ind illt ’s fyr hęill at hrapa.\eva

\bvb Combed and washed shall each keen man be, and full in morning,—for unknown it is where he will come by evening; ’tis bad to rush before one’s luck.\footnoteB{The wording of the first half of this stanza is very close to \Havamal\ 61 and \Voluspa\ 33; for discussion on personal hygiene and bathing see note to the former.}\evb
\evg
