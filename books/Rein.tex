\bookStart{Speeches of Rein}[Ręginsmǫ́l]
\def\thisBookCode{Reginsmal}

\begin{flushright}%
\textbf{Dating} \parencite{Sapp2022}: C10th (0.666)–early C11th (0.259)

\textbf{Meter:} \Ljodahattr, \Fornyrdislag%
\end{flushright}

\section{Introduction}

\textbf{The Speeches of Rein} (\Reginsmal) are preserved in \Regius, where they follow \Gripisspa and are introduced with a large initial and a near-illegible title.  The text clearly serves as the basis for \VolsungaSaga\ 14–15 and 17–18 (for ch. 16 see \Gripisspa), where sts. 1–2, 6 and 18 are cited.

In \Regius, \Reginsmal\ is the first of a group of three very similar “poems” in an unbroken narrative sequence which also includes \Fafnismal\ and \Sigrdrifumal, for which reason the whole group will be shortly discussed here.

The existence of these three “poems”—indeed their very names—is entirely a product of later philology, and to paraphrase Bellows, it is doubtful whether it is logically sound.  Although \Fafnismal\ is introduced by a title and large initial and thus separated from \Reginsmal, the distinction between \Fafnismal\ and \Sigrdrifumal\ is entirely arbitrary, and the two are continuous in the ms.  More importantly, none of the three poems is a unit, but throughout them one finds the same amalgamation of narrative prose and stanzas in \Fornyrdislag\ and \Ljodahattr.  It may be noted that the style of the \Ljodahattr\ stanzas is very similar throughout, and this may also be the case for the \Fornyrdislag-stanzas, so that we appear to be dealing with at least two long separate cycles treating the same overlapping story.  A particularly transparent example of overlap between sources is the speech of the tits in \Fafnismal\ (TODO: stanza numbers), where there is a perfect logical progression of thought if one only reads the stanzas in one meter, but which is lost if one reads both.

Since they are not three distinct poems (unlike say \Voluspa, \Grimnismal\ and \Vafthrudnismal), the whole group should be understood as a continuous narrative saw or \emph{prosimetrum}, where the redactor tells the story primarily through prose, with the stanzas are reserved for direct speech.  It is not improbable that this reflects some convention of oral storytelling.  In any case, this division into three poems has been retained in the present edition for reasons of convention and accessibility, but the reader is strongly encouraged to read the entire sequence in order.

\section{The Speeches of Rein}

\bpg\bpa Sig·urðr gekk til stóðs Hjálp-reks ok kaus sér af hest einn er Grani var kallaðr síðan. Þá var kominn Reginn til Hjálp-reks, sonr Hreið-mars. Hann var hverjum manni hagari ok dvergr of vǫxt. Hann var vitr, grimmr ok fjǫl-kunnigr. Reginn veitti Sig·urði fóstr ok kennslu ok elskaði hann mjǫk. Hann sagði Sig·urði frá for·ellri sínu ok þeim at·burðum at Óðinn ok Hǿnir ok Loki hǫfðu komit til And-vara-fors; í þeim forsi var fjǫlði fiska. Einn dvergr hét And-vari; hann var lǫngum í forsinum í geddu líki ok fekk sér þar matar. „Otr hét bróðir várr,“ kvað Reginn, „er oft fór í forsinn í otrs líki. Hann hafði tekit einn lax ok sat á ár-bakkanum ok át blundandi. Loki laust hann með steini til bana. Þóttust ę́sir mjǫk heppnir verit hafa ok flógu belg af otrinum. Þat sama kveld sóttu þeir gisting til Hreið-mars ok sýndu veiði sína. Þá tóku vér þá hǫndum ok lǫgðum þeim fjǫr-lausn at fylla otr-belginn með gulli ok hylja útan ok með rauðu gulli. Þá sendu þeir Loka at afla gullsins. Hann kom til Ránar ok fekk net hennar ok fór þá til And-vara-fors ok kastaði netinu fyr gedduna en hon hljóp í netit. Þá mę́lti Loki:\epa

\bpb {\huge S}iward went to Helpric’s stable and thereof chose for himself one horse which was thenceforth called Grane. Then Rein, son of Rethmar, was come to Helpric. He was craftier than every man and a dwarf in stature; he was clever, cruel and \inx[C]{many-cunning}. Rein granted Siward fosterage and teaching, and loved him much. He told Siward about his parentage, and about the events that Weden, Heener and Lock had come to Andwaresforce; in that force was a multitude of fish. One dwarf was called Andware; he was for a long time in the force in the likeness of a pike and got his food there. “Otter was our brother called,” said Rein, “who often went forth in the force in the likeness of an otter. He had taken a salmon and sat on the riverbank and ate it with his eyes closed. Lock beat him with a stone to his death. The Eese thought themselves to have been very lucky and flayed the skin from the otter. The same evening they sought lodgings at Rethmar’s house, and showed their catch. Then we bound them and gave them as a life-ransom to fill the otter-skin with gold and cover even the outside with red gold. Then they sent Lock to procure the gold. He came to Ran and got her net, and then journeyed to Andwaresforce and threw the net in front of the pike, and it jumped into the net. Then spoke Lock:\epb\epg


\bvg\bva%
„Hvat ’s þat \alst{f}iska \hld\ es rinn \alst{f}lóði ï; &
\ind kann-at sér við \alst{v}íti \alst{v}arask? &
\alst{H}ǫfuð þitt \hld\ lęys-tu \alst{h}ęlju ór; &
\ind finn mér \alst{l}indar \alst{l}oga!“\eva

\bvb “{\huge W}hat kind of fish is this that runs in the flood? \\
\ind It cannot ward itself from harm. \\
Redeem thy head out of Hell; \\
\ind find me the linden’s flame \ken{gold}!”\evb\evg


\bvg\bva%
„\alst{A}nd-vari ek hęiti, \hld\ \alst{Ó}inn hét minn faðir, &
\ind margan hęfi’k \alst{f}ors of \alst{f}arit. &
\alst{Au}mlig norn \hld\ skóp oss ï \alst{á}r-daga &
\ind at ek skylda ï \alst{v}atni \alst{v}aða.“\eva

\bvb “Andware I am called; Owen was my father called; \\
\ind through many a force I have fared. \\
A wretched norn shaped for us in days of yore \\
\ind that I should in the water wade.”\evb\evg


\bvg\bva%
„Sęg-ðu þat, \alst{A}nd-vari,“ \hld[kvað Loki,] „ef þú \alst{ęi}ga vill &
\ind \alst{l}íf ï \alst{l}ýða sǫlum: &
Hvęr \alst{g}jǫld \hld\ fȧa \alst{g}umna synir &
\ind ef hǫggvask \alst{o}rðum \alst{ȧ}?“\eva

\bvb “Tell this, Andware—quoth Lock—if thou wilt own \\
\ind life in the halls of men: \\
Which recompense do the sons of men get, \\
\ind if they hew at each other with words?”\evb\evg


\bvg\bva%
„Ofr-\alst{g}jǫld \hld\ fȧa \alst{g}umna synir &
\ind þęir’s \alst{V}að-gęlmi \alst{v}aða; &
\alst{ȯ}-saðra orða \hld\ hvęrr’s á \alst{a}nnan lýgr, &
\ind of \alst{l}ęngi \alst{l}ęiða \alst{l}imar.“\eva

\bvb “Great recompense do the sons of men get, \\
\ind those who in \inx[L]{Wadyelmer} wade. \\
By the branches of untrue words is each \\
\ind who lies to another long followed.\footnoteB{Watery torment in the afterlife for oath-breakers and liars is well attested in the Germanic sources. See note to \Voluspa\ 39 for discussion.}”\evb\evg


\bpg\bpa Loki sá allt gull þat er And-vari átti. En er hann hafði fram reitt gullit, þá hafði hann eptir einn hring ok tók Loki þann af hánum. Dvergrinn gekk inn í steininn ok mę́lti:\epa

\bpb Lock saw all the gold which Andware owned. But when he had readied all the gold, then he still had one ring, and Lock took it from him. The dwarf went into the stone and spoke:\epb\epg


\bvg\bva%
„Þat skal \alst{g}ull \hld\ es \alst{G}ustr átti &
\alst{b}rǿðrum tvęim \hld\ at \alst{b}ana verða &
ok \alst{ǫ}ðlingum \hld\ \alst{á}tta at rógi; &
\alst{m}un \alst{m}íns féar \hld\ \alst{m}ann-gi njóta.“\eva

\bvb “That gold which Gust owned shall \\
for two brothers become the bane, \\
and for eight nobles the [cause of] strife; \\
of my wealth will no man benefit.”\evb\evg


\bpg\bpa Ę́sir reiddu Hreið-mari féit ok tráðu upp otr-belginn ok reistu á fǿtr; þá skyldu ę́sirnir hlaða upp gullinu ok hylja. En er þat var gørt gekk Hreið-marr framm ok sá eitt grana-hár ok bað hylja. Þá dró Óðinn framm hringinn And-vara-naut ok hulði hárit.\epa

\bpb The Eese readied the wealth for Rethmar and stuffed the otter-skin and raised it on its feet. Then the Eese should fill it up with gold and cover it. But when that was done Rethmar stepped forth, and saw a single whisker-strand and bade it be covered. Then Weden drew forth the ring Andwaresgift and covered the strand.\epb\epg


\bvg\bva%
„\alst{G}ull ’s þér nú ręitt“, \hld[kvað Loki,] „en þú \alst{g}jǫld hęfir &
\ind \alst{m}ikil \alst{m}íns hǫfuðs; &
\alst{s}yni þínum \hld\ verðr-a \alst{s}ę́la skǫpuð; &
\ind þat verðr ykkarr \alst{b}ęggja \alst{b}ani!“\eva

\bvb “The gold is now readied for thee—quoth Lock—and thou hast the great \\
\ind payment for my head. \\
For thy son no welfare will be made; \\
\ind it will be the bane of you both!”\evb\evg

Hreiðmarr sagði:

\bvg\bva%
„\alst{G}jafar þú \alst{g}aft— \hld\ \alst{g}aft-at ǫ̇st-gjafar, &
\ind gaft-at af \alst{h}ęilum \alst{h}ug! &
\alst{F}jǫrvi yðru \hld\ skylduð ér \alst{f}irrðir vesa &
\ind ef vissa’k þat \alst{f}ár \alst{f}yrir.“\eva

\bvb “Thou gavest a gift—gavest not a gift of love; \\
\ind gavest not out of true heart! \\
From your lives would ye be far taken, \\
\ind if I had known that danger before!”\evb\evg


\bvg\bva%
„Enn es \alst{v}erra, \hld\ þat \alst{v}ita þikkjumk, &
\ind \alst{n}iðja stríð um \alst{n}ept; &
\alst{jǫ}fra \alst{ó}-borna \hld\ hygg þá \alst{e}nn vesa &
\ind es þat ’s til \alst{h}atrs \alst{h}ugat.“\eva

\bvb “TODO.”\evb\evg


\bvg\bva%
„\alst{R}auðu gulli“, \hld[kvað Hreiðmarr,] „hygg ek mik \alst{r}áða munu &
\ind svá \alst{l}engi sem ek \alst{l}ifi; &
\alst{h}ót þín \hld\ \alst{h}rę́ðumk ękki lyf &
\ind ok \alst{h}aldið \alst{h}ęim \alst{h}eðan!“\eva

\bvb “The red gold—quoth Rethmar—I think that I will rule \\
\ind so long as I live. \\
Thy threats I fear not at all (TODO) \\
and hold home from hence!”\evb\evg


\bpg\bpa Fáfnir ok Reginn krǫfðu Hreið-mar nið-gjalda eptir Otr, bróður sinn. Hann kvað nei við. En Fáfnir lagði sverði Hreið-mar, fǫður sinn, sofanda. Hreið-marr kallaði á dǿtr sínar:\epa

\bpb Fathomer and Rein demanded from Rethmar the kin-payment after Otter, their brother. He said no to it. But Fathomer ran the sword through Rethmar, his father, sleeping. Rethmar called on his daughters:\epb\epg


\bvg\bva%
„\alst{L}yng-hęiðr ok \alst{L}ofn-hęiðr, \hld\ vitið mínu \alst{l}ífi farit! &
\ind \edtrans{Mart ’s þat’s \alst{þ}ǫrf \alst{þ}éar!}{Much does need compel!}{\Bfootnote{Or “Much is required by neccessity”.  Rethmar refers to the duty of his daughters to avenge him, even by killing their own brother.}}“ &
\speakernote{Lyngheiðr svaraði:}„\alst{F}ǫ́ mun systir, \hld\ þótt \alst{f}ǫður missi, &
\ind \alst{h}ęfna \alst{h}lýra \alst{h}arms!“\eva

\bvb “O Lingheath and Lovenheath, witness my life destroyed! \\
\ind Much does need compel!” \\
\speakernoteb{Lingheath answered:}“Few a sister, though she miss her father, \\
\ind will avenge her brother’s harm!\evb\evg


\bvg\bva%
„Al þú þó \alst{d}óttur“, \hld[kvað Hreiðmarr,] „\alst{d}ís úlf-huguð, &
ef þú getr-at \alst{s}on \hld\ við \alst{s}iklingi; &
fȧ þú \alst{m}ęy \edtext{\alst{m}anni \hld\ \alst{m}ęgin-þarfar}{\Afootnote{\emph{mann imeginþarfar} \Regius}}, &
þá mun \alst{þ}ęirar sonr \hld\ \alst{þ}íns harms vreka.“\eva

\bvb “Beget yet a daughter—quoth Rethmar—a wolf-minded lady, \\
if thou gettest no son by the prince. \\
Wed that maiden to a man of great need, \\
then \emph{her} son will avenge thy harm!\footnoteB{Rethmar’s last words foretell the life of Siward, whose mother, Hardise, would then be Lingheath’s daughter.}”\evb\evg


\bpg\bpa Þá dó Hreið-marr, en Fáfnir tók gullit allt. Þá beiddisk Reginn at hafa fǫður-arf sinn, en Fáfnir galt þar nei við. Þá leitaði Reginn ráða við Lyng-heiði, systur sína, hvernig hann skyldi heimta fǫður-arf sinn. Hon kvað:\epa

\bpb Then Rethmar died and Fathomer took all the gold. Then Rein begged to have his father’s inheritance, but Fathomer gave back a no. Then Rein sought counsel from Lingheath, his sister, over how he should take his father’s inheritance. She quoth:\epb\epg


\bvg\bva%
„\edtrans{\alst{B}rúðar}{From the bride}{\Bfootnote{“From me.”  It seems that Lingheath here offers Rein her part of the inheritance.}} kvęðja \hld\ skalt \alst{b}líð-liga &
\ind \alst{a}rfs ok \alst{ǿ}ðra hugar; &
es-a þat \alst{h}ǿft \hld\ at þú \alst{h}jǫrvi skylir &
\ind kvęðja \alst{F}áfni \alst{f}éar!“\eva

\bvb “From the bride shalt thou blithely call \\
\ind for heritance and nobler thoughts; \\
it is not fitting that thou shouldst by sword \\
\ind call for Fathomer’s wealth!”\evb\evg


\bpg\bpa Þessa hluti sagði Reginn Sig·urði. Einn dag, er hann kom til húsa Regins, var hánum vel fagnat. Reginn kvað:\epa

\bpb These things Rein told Siward. One day when he came to Rein’s house he was greeted heartily. Rein quoth:\epb\epg


\bvg\bva%
„\alst{K}ominn ’s hingat \hld\ \alst{k}onr Sig-mundar, &
\alst{s}ęggr inn \alst{s}nar-ráði, \hld\ til \alst{s}ala várra; &
\alst{m}óð hęfir \alst{m}ęira \hld\ an \alst{m}aðr gamall, &
ok es mér \alst{f}angs vǫ́n \hld\ at \alst{f}rekum ulfi.\eva

\bvb “Hither is come the son of Syemund \ken*{= Siward}, \\
the youth of quick counsel to our halls! \\
He has greater heart than an old man, \\
and I expect a catch from the hungry wolf.\evb\evg


\bvg\bva%
Ek mun \alst{f}ǿða \hld\ \alst{f}olk-djarfan gram; &
nú ’s \alst{y}ngva konr \hld\ með \alst{o}ss kominn; &
sjá mun \alst{r}ę́sir \hld\ \alst{r}íkstr und sólu, &
\edtext{þrymr um \alst{ǫ}ll lǫnd \hld\ \alst{ø}r·lǫg-símu}{\lemma{þrymr \dots\ ør·lǫg-símu ‘he fastens \dots\ orlay-strands’}\Bfootnote{“His fate is being fixed through all lands.”  Cf. the first four sts. of \HelgakvidaOne.}}.“\eva

\bvb I will raise the troop-bold prince; \\
now the son of the king is come amidst us! \\
This ruler will become mightiest under the sun; \\
he fastens through all lands his orlay-strands!”\evb\evg


\bpg\bpa Sig·urðr var þá jafnan með Regin ok sagði hann Sig·urði at Fáfnir lá á Gnita-heiði ok var í orms líki. Hann átti ǿgis-hjalm er ǫll kvikvendi hrę́ddusk við. Reginn gerði Sig·urði sverð er Gramr hét. Þat var svá hvasst at hann brá því ofan í Rín ok lét reka ullar-lagð fyr straumi ok tók í sundr lagðinn sem vatnit. Því sverði klauf Sig·urðr í sundr steðja Regins. Eptir þat eggjaði Reginn Sig·urð at vega Fáfni. Hann sagði:\epa

\bpb Thereafter Siward was always with Rein, and he told Siward that Fathomer lay on the Gnit-heath and was in a Wyrm’s likeness; he owned the helm of awe by which all living things were frightened. Rein made Siward the sword called Gram; it was so sharp that he plunged it down into the Rhine, and let a lock of wool float down the stream, and it split the lock like it did the water. With that sword Siward split asunder the anvil of Rein; after that Rein urged Siward to slay Fathomer. He said:\epb\epg


\bvg\bva%
„\alst{H}ǫ́tt munu \alst{h}lę́ja \hld\ \alst{H}undings synir &
þęir’s \alst{Ęy}-lima \hld\ \alst{a}ldrs synjuðu, &
ef \alst{m}ęirr tiggja \hld\ \alst{m}unar at sǿkja &
\alst{h}ringa rauða \hld\ an \alst{h}ęfnd fǫður.“\eva

\bvb “Loudly laugh will Hunding’s sons \\
—they who denied Eanlime’s old age— \\
if the chief is more eager to seek \\
red rings than to avenge his father.”\evb\evg


\bpg\bpa Hjálp-rekr konungr fekk Sig·urði skipa-lið til fǫður-hefnda. Þeir fengu storm mikinn ok beittu fyr bergs-nǫs nakkvara. Maðr einn stóð á berginu ok kvað:\epa

\bpb Helpric got Siward a ship-retinue for the avenging of his father. They caught a great storm, and tacked the ships before a group of crags. A lone man stood on the crag and quoth:\epb\epg


\bvg\bva%
„Hvęrir \alst{r}íða þar \hld\ \alst{R}ę́fils hestum &
\alst{h}ávar unnir, \hld\ \alst{h}af glymjanda? &
\alst{S}egl-vigg eru \hld\ \alst{s}vęita stokkin, &
mun-at \alst{v}ág-marar \hld\ \alst{v}ind of standask.“\eva

\bvb “Which men ride there Revil’s horses \ken{ships} \\
on the high waves, the roaring sea? \\
The sail-steeds are spattered with blood; \\
the wave-chargers will not bear the wind!”\evb\evg


\bvg\bva%
„Hér eru vér \alst{S}ig-urðr \hld\ á \alst{s}ę́-tréum; &
es oss \alst{b}yrr gefinn \hld\ við \alst{b}ana sjalfan; &
fellr \alst{b}rattr \alst{b}reki \hld\ \alst{b}rǫndum hę́ri, &
\alst{h}lunn-vigg \alst{h}rapa— \hld\ \alst{h}vęrr spyrr at því?“\eva

\bvb “Here are we, Siward [and his men], on sea-trees \ken{ships}; \\
we are given a gust toward death itself! \\
The steep breaker falls higher than flames; \\
the launcher-steeds rush forth—who asks of this?”\evb\evg


\bvg\bva%
„\alst{H}nikar hétu mik \hld\ þá’s \edtrans{\alst{H}ugin gladdi}{gladdened Highen}{\Bfootnote{A variant of the extremely common motif “feed the raven”, i.e., by the corpses of slain foes on the battlefield.}} &
\edtrans{\alst{V}ǫlsungr ungi}{young Walsing}{\Bfootnote{Siward’s grandfather, the founder of the Walsing dynasty.}} \hld\ ok \alst{v}egit hafði; &
nú mátt \alst{k}alla \hld\ \alst{k}arl af bergi, &
\alst{F}ęng eða \alst{F}jǫlni; \hld\ \alst{f}ar vil’k þiggja.“\eva

\bvb “Nicker they called me when young Walsing \\
gladdened Highen and had conquered. \\
Now mayst thou call me churl-from-the-crag, \\
Feng or Fillner—I wish to beg passage.”\evb\evg


\bpg\bpa Þeir viku at landi, ok gekk karl á skip, ok lę́gði þá veðrit.\epa

\bpb They turned to land and the man went on the ship, and then the weather calmed down.\epb\epg


\bvg\bva%
„Sęg mér þat, \alst{H}nikarr, \hld\ alls \alst{h}vár-tvęggja vęitst, &
\ind \alst{g}oða hęill ok \alst{g}uma: &
hvęr \alst{b}ǫzt eru \hld\ ef \alst{b}ęrjask skal, &
\ind hęill at \alst{s}verða \alst{s}vipun?“\eva

\bvb “Tell me this, Nicker, as thou knowest both \\
\ind the charms of gods and men: \\
Which are the best—if one shall fight— \\
\ind charms in the swinging of swords?”\evb\evg


\bvg\bva%
„Mǫrg eru \alst{g}óð \hld\ ef \alst{g}umar vissi, &
\ind hęill at \alst{s}verða \alst{s}vipun; &
\alst{d}yggja fylgju \hld\ hygg ins \alst{d}økkva vesa &
\ind at \alst{h}rotta-męiði \alst{h}rafns.\eva

\bvb “There are many good—if men knew them— \\
\ind charms in the swinging of swords. \\
A good followeress I judge the dark one \\
TODO..”\evb\evg


\bvg\bva%
Þat es \alst{a}nnat \hld\ ef ert \alst{ú}t of kominn &
\ind ok est á \alst{b}raut \alst{b}úinn: &
\alst{t}vá þú lítr \hld\ á \alst{t}ái standa &
\ind \alst{h}róðr-fúsa \alst{h}ali.\eva

\bvb “This is the other, if thou art come out \\
\ind and art ready on the road: \\
thou beholdest two standing on their toes \\
\ind glory-eager heroes.”\evb\evg


\bvg\bva%
Þat ’s it \alst{þ}riðja \hld\ ef \alst{þ}jóta hęyrir &
\ind \alst{u}lf und \alst{a}sk-limum, &
\alst{h}ęilla auðit \hld\ verðr þér af \alst{h}jalm-stǫfum &
\ind ef sér þá \alst{f}yrri \alst{f}ara.\eva

\bvb “This is the third, if thou hear howling \\
\ind a wolf beneath ashen branches \\
TODO..”\evb\evg


\bvg\bva%
Ęngr skal \alst{g}umna \hld\ í \alst{g}ǫgn vega &
\alst{s}íð skínandi \hld\ \alst{s}ystur mána; &
þęir \alst{s}igr hafa \hld\ es \alst{s}éa kunnu, &
\alst{h}jǫr-lęiks \alst{h}vatir, \hld\ eða \edtrans{\alst{h}amalt fylkja}{draw up the flying wedge}{\Bfootnote{This formation, known as the swine-array (\emph{svín-fylking}), was favoured by the Germanic peoples.  It is mentioned already in Tacitus Germania ch. 6: \emph{acies per cuneos componitur} ‘their line of battle is drawn up in a wedge-like formation’.
In the legendary saws it has a particular association with Weden; according \Sogubrot\ it was taught by Weden to the Danish king Harold Hildtooth, who went on to win great victories with it.  At last his rival, the Swedish king Siward Ring, was also taught it, and went on to slay Harold at the battle of the Browolds (\emph{Brávęllir}).  Cf. \Sogubrot\ 8:
\emph{Brúni segir: „Svá lítst mér sem Hringr muni búinn at berjask ok hans lið. Hann hefir undarliga fylkt. Hann hefir svín-fylkt her sínum, ok mun eigi gott at berjask við hann.“
Þá segir Haraldr konungr: „Hverr mun Hringi hafa kennt hamalt at fylkja? Ek hugða engan kunna nema mik ok Óðin, eða mun Óðinn vilja skjǫplast í sigr-gjǫfinni við mik? [...]“}
‘Brown says: “It seems to me that Ring is ready to fight, and his troop too. He has drawn up them in a wondersome way; he has drawn up his host in the swine-shape, and it will not be good to fight against him.
Then says king Harold: “Who will have taught Ring to draw up the flying wedge? I thought noone knew it save for me and Weden; or will Weden wish to fail in his giving me victory? [...]”’}}.\eva

\bvb No man shall fight facing \\
in evening the shining sister of Moon \ken{sun}. \\
They have the victory who can see \\
—men brisk in sword-play \ken{battle}—or draw up the flying wedge.\evb\evg


\bvg\bva%
Þat ’s \alst{f}ár mikit \hld\ ef \alst{f}ǿti drepr &
\ind þar’s þú at \alst{v}ígi \alst{v}ęðr; &
\alst{t}álar dísir \hld\ standa þér á \alst{t}vę́r hliðar &
\ind ok vilja þik \alst{s}áran \alst{s}éa.\eva

\bvb It is a great peril if thou stumble thy foot \\
\ind where you wade forth in war. \\
Treacherous dises stand on both sides of thee \\
\ind and wish to see thee harmed.\evb\evg


\bvg\bva%
\alst{K}ęmbðr ok þvęginn \hld\ skal \alst{k}ǿnna hvęrr &
\ind ok at \alst{m}orni \alst{m}ęttr, &
því’t \alst{ó}-sýnt es \hld\ hvar at \alst{a}ptni kømr; &
\ind illt ’s fyr \alst{h}ęill at \alst{h}rapa.\eva

\bvb Combed and washed shall each keen man be, \\
\ind and by morning full, \\
for ’tis unseen where by evening he comes; \\
\ind ’tis bad to rush ahead of the charms!\footnoteB{The wording of the first half of this stanza is very close to \Havamal\ 61 and \Voluspa\ 33; for discussion on personal hygiene and bathing see note to the former.}\evb\evg

\sectionline

\bpg\bpa Sig·urðr átti orrustu mikla við Lyngva Hundings son ok brǿðr hans. Þar fell Lyngvi ok þeir þrír brǿðr. Eptir orrustu kvað Reginn:\epa

\bpb Siward had a great battle with Ling Hunding’s son and his brothers. There fell Ling and three of his brothers. After the battle Rein quoth:\epb\epg


\bvg\bva%
Nú ’s \alst{b}lóðugr ǫrn \hld\ \alst{b}itrum hjǫrvi &
\alst{b}ana Sigmundar \hld\ á \alst{b}aki ristinn; &
øngr es \alst{f}ręmri, \hld\ sá’s \alst{f}old ryði, &
\alst{h}ilmis arfi \hld\ ok \edtrans{\alst{H}ugin gladdi}{has gladdened Highen}{\Bfootnote{i.e. “has fed the raven (with corpses).”}}!\eva

\bvb Now the bloody eagle with a bitter sword \\
is carved on the back of Syemund’s bane. \\
No chieftain’s heir is more successful, \\
who clears the earth and has gladdened Highen!\evb\evg


\bpg\bpa Heim fór Sig·urðr til Hjálpreks. Þá eggjaði Reginn Sig·urð til at vega Fáfni. Sig·urðr ok Reginn fóru upp á Gnitaheiði ok hittu þar slóð Fáfnis þá er hann skreið til vats. Þar gørði Sig·urðr grǫf mikla á veginum ok gekk Sig·urðr þar í. En er Fáfnir skreið af gullinu blés hann eitri ok hraut þat fyr ofan hǫfuð Sig·urði. En er Fáfnir skreið yfir grǫfina þá lagði Sig·urðr hann með sverði til hjarta. Fáfnir hristi sik ok barði hǫfði ok sporði. Sig·urðr hljóp ór grǫfinni ok sá þá hvárr annan. Fáfnir kvað:\epa

\bpb Siward journeyed home to Helpric. Then Rein incited Siward to smite Fathomer. Siward and Rein journeyed up on the Gnit-heath and found there Siward’s trail as he was slithering to water. There Siward made a great trench in the way, and Siward went down into it. And when Fathomer slithered off the gold he blew venom, and it flew over Siward’s head. But when Fathomer slithered over the trench, then Siward ran him through with the sword to the heart. Fathomer shook himself and struck his head and spurned. Siward leapt out of the trench, and then each of them saw the other. Fathomer quoth:\epb\epg
