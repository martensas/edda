\bookStart{The Speeches of Rein}[Ręginsmǫ́l]

\begin{flushright}%
\textbf{Dating} \parencite{Sapp2022}: C10th (0.666)–early C11th (0.259)

\textbf{Meter:} \Ljodahattr, \Fornyrdislag%
\end{flushright}

The title of this poem (or, better, prosimetrum) is editorial.  Itmost closely The differing meter of the stanzas might suggest that they are taken from different poems.

\sectionline

\bpg\bpa Sigurðr gekk til stóðs Hjálp-reks ok kaus sér af hest einn er Grani var kallaðr síðan. Þá var kominn Reginn til Hjálp-reks, sonr Hreið-mars. Hann var hverjum manni hagari ok dvergr of vǫxt. Hann var vitr, grimmr ok fjǫl-kunnigr. Reginn veitti Sigurði fóstr ok kennslu ok elskaði hann mjǫk. Hann sagði Sigurði frá for·ellri sínu ok þeim at·burðum at Óðinn ok Hǿnir ok Loki hǫfðu komit til And-vara-fors; í þeim forsi var fjǫlði fiska. Einn dvergr hét And-vari; hann var lǫngum í forsinum í geddu líki ok fekk sér þar matar. „Otr hét bróðir várr,“ kvað Reginn, „er oft fór í forsinn í otrs líki. Hann hafði tekit einn lax ok sat á ár-bakkanum ok át blundandi. Loki laust hann með steini til bana. Þóttust ę́sir mjǫk heppnir verit hafa ok flógu belg af otrinum. Þat sama kveld sóttu þeir gisting til Hreið-mars ok sýndu veiði sína. Þá tóku vér þá hǫndum ok lǫgðum þeim fjǫr-lausn at fylla otr-belginn með gulli ok hylja útan ok með rauðu gulli. Þá sendu þeir Loka at afla gullsins. Hann kom til Ránar ok fekk net hennar ok fór þá til And-vara-fors ok kastaði netinu fyr gedduna en hon hljóp í netit. Þá mę́lti Loki:\epa

\bpb Siward went to Helpric’s stable and thereof chose for himself one horse who was henceforth called Grane. Then Rein, son of Rethmar, was come to Helpric. He was more crafty than any man and a dwarf in stature; he was clever, cruel and \inx[C]{many-cunning}. Rein fostered and taught Siward and love him very much. He told Siward about his own parents, and about the events that Weden, Heener and Lock had come to Andwareforce; in that force was a multitude of fish. A dwarf was named Andware; he was for a long time in the force in the likeness of a pike and got his food there. “Otter was our brother called,” said Rein, “who often journeyed in the force in the likeness of an otter. He had caught a salmon and sat on the riverbank and ate it with closed eyes Lock struck him with a stone unto his death. The Eese thought themselves to have been very lucky, and flayed the skin off the otter. The same evening they sought to pass the night at Rethmar’s house, and showed their catch. Then we bound them and proposed to them as a life-ransom that they would fill the otter-skin with gold, and also cover the outside with red gold. Then they sent Lock to procure the gold. He came to Ran and got her net, and then journeyed to Andwareforce and threw the net in front of the pike, and it jumped into the net. Then spoke Lock:\epb\epg


\bvg\bva „Hvat ’s þat fiska \hld\ es rinn flóði í, &
\ind kann-at sér við víti varask. &
Hǫfuð þitt \hld\ lęys-tu hęlju ór; &
\ind finn mér lindar loga!“\eva

\bvb “What kind of fish is it who runs in the flood? \\
It cannot protect itself from harm. \\
Ransom thy head out of Hell; \\
find me the flame of the linden \ken{gold}!”\evb\evg


\bvg\bva „And-vari ek hęiti, \hld\ Óinn hét minn faðir, &
\ind margan hęfi’k fors of farit. &
Aumlig norn \hld\ skóp oss í ár-daga &
\ind at skylda í vatni vaða.“\eva

\bvb “Andware I am called; Owen was called my father; \\
through many a force have I fared. \\
A wretched norn shaped for us in days of yore, \\
that I should in the water wade.”\evb\evg


\bvg\bva „Sęg-ðu þat, And-vari, \small{(kvað Loki)} ef þú ęiga vill &
\ind líf í lýða sǫlum: &
Hvęr gjǫld \hld\ fȧa gumna synir &
\ind ef hǫggvask orðum á?“\eva

\bvb “Say this, Andware—quoth Lock—if thou wilt own \\
life in the halls of men: \\
Which recompense do the sons of men get, \\
if they hew at each other with words?”\evb\evg


\bvg\bva „Ofr-gjǫld \hld\ fȧa gumna synir &
\ind þęir’s Vaðgęlmi vaða; &
ó·saðra orða \hld\ hvęrr’s á annan lýgr, &
\ind of lęngi lęiða limar.“\eva

\bvb “Great recompense do the sons of men get, \\
those who in \inx[L]{Wadyelmer} wade. \\
By the ramifications of untrue words is each \\
who lies to another long followed.\footnoteB{Watery torment in the afterlife for oath-breakers and liars is well attested in the Germanic sources. See note to \Voluspa\ 39 for discussion.}”\evb\evg


\bpg\bpa Loki sá allt gull þat er And-vari átti. En er hann hafði fram reitt gullit, þá hafði hann eftir einn hring ok tók Loki þann af hánum. Dvergrinn gekk inn í steininn ok mę́lti:\epa

\bpb Lock saw all the gold which Andware owned. But when he had brought forth all the gold, then he had one ring left, and Lock took it off him. The dwarf went into the stone and spoke:\epb\epg


\bvg\bva „Þat skal gull \hld\ es Gustr átti &
brǿðrum tvęim \hld\ at bana verða &
ok ǫðlingum \hld\ átta at rógi; &
mun míns féar \hld\ mann-gi njóta.“\eva

\bvb “That gold which Gust owned shall \\
for two brothers become the bane, \\
and for eight nobles the [cause of] strife; \\
of my wealth will no man benefit.”\evb\evg


\bpg\bpa Ę́sir reiddu Hreið-mari féit ok tráðu upp otr-belginn ok reistu á fǿtr; þá skyldu ę́sirnir hlaða upp gullinu ok hylja. En er þat var gørt gekk Hreið-marr framm ok sá eitt grana-hár ok bað hylja. Þá dró Óðinn framm hringinn And-vara-naut ok hulði hárit.\epa

\bpb The Eese prepared the wealth for Rethmar and stuffed the otter-skin and raised it on its feet. Then the Eese should fill it up with gold and cover it. But when that was done Rethmar stepped forth, and saw a single whisker-strand and bade it be covered. Then Weden drew forth the ring Andwaresgift and covered the strand.\epb\epg


\bvg\bva „Gull ’s þér nú ręitt (kvað Loki) en þú gjǫld hęfir &
\ind mikil míns hǫfuðs; &
syni þínum \hld\ verðr-a sę́la skǫpuð; &
\ind þat verðr ykkarr bęggja bani!“\eva

\bvb “TODO.”\evb\evg


\bvg\bva „Gjafar þú gaft— \hld\ gaft-at ǫ́st-gjafar, &
\ind gaft-at af hęilum hug! &
Fjǫrvi yðru \hld\ skylduð ér firrðir vesa &
\ind ef vissa’k þat fár fyrir.“\eva

\bvb “Thou gavest a gift—gavest not a gift of love; \\
gavest not out of a true heart! \\
From your lives would ye be removed, \\
if I had known that danger before!”\evb\evg


\bvg\bva „Enn es verra, \hld\ þat vita þikkjumk, &
\ind niðja stríð um nept; &
jǫfra ó·borna \hld\ hygg þá enn vesa &
\ind es þat ’s til hatrs hugat.“\eva

\bvb “TODO.”\evb\evg


\bvg\bva „Rauðu gulli (kvað Hreiðmarr) hygg ek mik ráða munu &
\ind svá lengi sem ek lifi; &
hót þín \hld\ hrę́ðumk ękki lyf &
\ind ok haldið hęim heðan!“\eva

\bvb “The red gold—quoth Rethmar—I think that I will rule \\
for as long as live. \\
Thy threats TODO.”\evb\evg


\bpg\bpa Fáfnir ok Reginn krǫfðu Hreið-mar nið-gjalda eptir Otr, bróður sinn. Hann kvað nei við. En Fáfnir lagði sverði Hreið-mar, fǫður sinn, sofanda. Hreið-marr kallaði á dǿtr sínar:\epa

\bpb Fathomer and Rein demanded from Rethmar the kinsman-payment after Otter, their brother. He said no to it. But Fathomer ran the sword through Rethmar, his father, sleeping. Rethmar called upon his daughters:\epb\epg


\bvg\bva „Lyng-hęiðr ok Lofn-hęiðr, \hld\ vitið mínu lífi farit! &
\ind Mart ’s þat’s þǫrf þéar!“\eva

\bvb “Lingheath and Lovenheath, witness my destroyed life! \\
TODO.”\evb\evg


\bvg\bva\speakernote{Lyngheiðr svaraði:}„Fá mun systir, \hld\ þótt fǫður missi, &
\ind hęfna hlýra harms!“\eva

\bvb Lingheath answered: \\
“Not many a sister, although she misses her father, \\
will avenge her brother’s harm!”\evb\evg


\bvg\bva „Al þú þó dóttur, (kvað Hreiðmarr) dís úlf-huguð, &
ef þú getr-at son \hld\ við siklingi; &
fá þú męy mann \hld\ í megin-þarfar,
þá mun þęirar sonr \hld\ þíns harms reka.“\eva

\bvb “TODO.”\evb\evg


\bpg\bpa Þá dó Hreið-marr en Fáfnir tók gullit allt. Þá beiddisk Reginn at hafa fǫður-arf sinn, en Fáfnir galt þar nei við. Þá leitaði Reginn ráða við Lyng-heiði, systur sína, hvernig hann skyldi heimta fǫður-arf sinn. Hon kvað:\epa

\bpb Then Rethmar died, and Fathomer took all the gold. Then Rein asked to have his father’s inheritance, but Fathomer gave back a no. Then Rein looked for counsel from Lingheath, his sister, over how he should get his father’s inheritance. She quoth:\epb\epg


\bvg\bva „Brúðar kvęðja \hld\ skalt blíð-liga &
\ind arfs ok ǿðra hugar; &
es-a þat hǿft \hld\ at þú hjǫrvi skylir &
\ind kvęðja Fáfni féar!“\eva

\bvb “TODO.”\evb\evg


\bpg\bpa Þessa hluti sagði Reginn Sigurði. Einn dag, er hann kom til húsa Regins, var hánum vel fagnat. Reginn kvað:\epa

\bpb These things Rein said to Siward. One day when he came to Rein’s house he was greeted well. Rein quoth:\epb\epg


\bvg\bva „Kominn ’s hingat \hld\ konr Sig-mundar, &
sęggr inn snar-ráði, \hld\ til sala várra; &
móð hęfir męira \hld\ en maðr gamall, &
ok es mér fangs vǫ́n \hld\ at frekum ulfi.\eva

\bvb “Hither is come the son of Syemund \ken*{= Siward}, \\
the quick-counselling youth, to our halls; \\
he has greater courage than an old man, \\
and I expect a catch from the hungry wolf!\evb\evg


\bvg\bva Ek mun fǿða \hld\ folk-djarfan gram; &
nú ’s yngva konr \hld\ með oss kominn; &
sjá mun rę́sir \hld\ ríkstr und sólu, &
\edtext{þrymr um ǫll lǫnd \hld\ ør·lǫg-símu}{\lemma{þrymr \dots\ ør·lǫg-símu ‘he fastens \dots\ orlay-strands’}\Bfootnote{i.e. “his fate is being fixed throughout all lands”. Cf. the first four sts. of \HelgakvidaOne.}}.“\eva

\bvb I will raise the troop-bold prince, \\
now the son of a king is come among us! \\
This ruler will become mightiest under the sun, \\
he fastens through all lands his orlay-strands!”\evb\evg


\bpg\bpa Sigurðr var þá jafnan með Regin ok sagði hann Sigurði at Fáfnir lá á Gnita-heiði ok var í orms líki. Hann átti ǿgis-hjalm er ǫll kvikvendi hrę́ddusk við. Reginn gerði Sigurði sverð er Gramr hét. Þat var svá hvasst at hann brá því ofan í Rín ok lét reka ullar-lagð fyr straumi ok tók í sundr lagðinn sem vatnit. Því sverði klauf Sigurðr í sundr steðja Regins. Eptir þat eggjaði Reginn Sigurð at vega Fáfni. Hann sagði:\epa

\bpb Then Siward was always with Rein, and he told Siward that Fathomer lay on the Gnit-heath in a Wyrm’s likeness; he owned the helm of awe by which all living things were frightened. Rein made for Siward the sword which is called Gram; it was so sharp that he plunged it down into the Rhine, and floated a lock of wool down the stream, and it split the lock like it did the water. With that sword Siward split asunder Rein’s anvil; after that Rein urged Siward to slay Fathomer. He said:\epb\epg


\bvg\bva „Hátt munu hlę́ja \hld\ Hundings synir &
þęir’s Ęy-lima \hld\ aldrs synjuðu, &
ef męirr tiggja \hld\ munar at sǿkja &
hringa rauða \hld\ en hęfnd fǫður.“\eva

\bvb “TODO.”\evb\evg


\bpg\bpa Hjálp-rekr konungr fekk Sigurði skipa-lið til fǫður-hefnda. Þeir fengu storm mikinn ok beittu fyr bergs-nǫs nakkvara. Maðr einn stóð á berginu ok kvað:\epa

\bpb Helpric got Siward a ship-retinue in order to avenge his father. They caught a great storm, and tacked the ships through some rocky cliffs. A lone man stood on the cliff and quoth:\epb\epg


\bvg\bva „Hvęrir ríða þar \hld\ Rę́fils hestum &
hávar unnir, \hld\ haf glymjanda? &
Segl-vigg eru \hld\ svęita stokkin, &
mun-at vág-marar \hld\ vind of standask.“\eva

\bvb “TODO.”\evb\evg


\bvg\bva „Hér eru vér Sig-urðr \hld\ á sę́-tréum; &
es oss byrr gefinn \hld\ við bana sjalfan; &
fellr brattr breki \hld\ brǫndum hę́ri, &
hlunn-vigg hrapa— \hld\ hvęrr spyrr at því?“\eva

\bvb “TODO.”\evb\evg


\bvg\bva „Hnikar hétu mik \hld\ þá’s Hugin gladdi &
Vǫlsungr ungi \hld\ ok vegit hafði; &
nú mátt kalla \hld\ karl af bergi, &
Feng eða Fjǫlni; \hld\ far vil’k þiggja.“\eva

\bvb “Nicker they called me, when the young Walsing \\
gladdened Highen, and had fought; \\
now thou mayst call me man of the cliff, \\
Fang or Fillner—I wish to take passage!”\evb\evg


\bpg\bpa Þeir viku at landi, ok gekk karl á skip, ok lę́gði þá veðrit.\epa

\bpb They turned toward land and the man stepped onto the ship, and then the weather calmed down.\epb\epg


\bvg\bva „Sęg mér þat, Hnikarr, \hld\ alls hvár-tvęggja vęitst, &
\ind goða hęill ok guma: &
hvęr bǫzt eru \hld\ ef bęrjask skal, &
\ind hęill at sverða svipun?“\eva

\bvb “TODO.”\evb\evg


\bvg\bva „Mǫrg eru góð \hld\ ef gumar vissi, &
\ind hęill at sverða svipun; &
dyggja fylgju \hld\ hygg ins døkkva vesa &
\ind at hrotta-męiði hrafns.\eva

\bvb “TODO.”\evb\evg


\bvg\bva Þat es annat \hld\ ef est út of kominn &
\ind ok est á braut búinn: &
tvá þú lítr \hld\ á tái standa &
\ind hróðr-fúsa hali.\eva

\bvb “TODO.”\evb\evg


\bvg\bva Þat ’s it þriðja \hld\ ef þjóta hęyrir &
\ind ulf und ask-limum, &
hęilla auðit \hld\ verðr þér af hjalm-stǫfum &
\ind ef sér þá fyrri fara.\eva

\bvb “TODO.”\evb\evg


\bvg\bva Ęngr skal gumna \hld\ í gǫgn vega &
síð skínandi \hld\ systur mána; &
þęir sigr hafa \hld\ es séa kunnu, &
hjǫr-lęiks hvatir, \hld\ eða hamalt fylkja.\eva

\bvb “TODO.”\evb\evg


\bvg\bva Þat ’s fár mikit \hld\ ef fǿti drepr &
\ind þar’s þú at vígi vęðr; &
tálar dísir \hld\ standa þér á tvę́r hliðar &
\ind ok vilja þik sáran séa.\eva

\bvb “TODO.”\evb\evg


\bvg\bva Kęmbðr ok þvęginn \hld\ skal kǿnna hvęrr &
\ind ok at morni męttr, &
því-at ó·sýnt es \hld\ hvar at aptni kømr; &
\ind illt ’s fyr hęill at hrapa.\eva

\bvb Combed and washed shall each keen man be, \\
and by morning full, \\
for ’tis unseen where by evening he comes; \\
’tis bad to rush before one’s luck.\footnoteB{The wording of the first half of this stanza is very close to \Havamal\ 61 and \Voluspa\ 33; for discussion on personal hygiene and bathing see note to the former.}\evb\evg

\sectionline

\bpg\bpa Sigurðr átti orrustu mikla við Lyngva Hundings son ok brǿðr hans. Þar fell Lyngvi ok þeir þrír brǿðr. Eptir orrustu kvað Reginn:\epa

\bpb Siward had a great battle with Ling Hunding’s son and his brothers. There fell Ling and three of his brothers. After the battle Rein quoth:\epb\epg


\bvg\bva Nú ’s \alst{b}lóðugr ǫrn \hld\ \alst{b}itrum hjǫrvi &
\alst{b}ana Sigmundar \hld\ á \alst{b}aki ristinn; &
øngr es \alst{f}ręmri, \hld\ sá’s \alst{f}old ryði, &
\alst{h}ilmis arfi \hld\ ok \edtrans{\alst{H}ugin gladdi}{has gladdened Highen}{\Bfootnote{i.e. “has fed the raven (with corpses).”}}!\eva

\bvb Now is the bloody eagle with a biting sword \\
carved on the back of Syemund’s bane. \\
No chieftain’s heir is more successful, \\
who clears the earth and has gladdened Highen!\evb\evg


\bpg\bpa Heim fór Sigurðr til Hjálpreks. Þá eggjaði Reginn Sigurð til at vega Fáfni. Sigurðr ok Reginn fóru upp á Gnitaheiði ok hittu þar slóð Fáfnis þá er hann skreið til vats. Þar gørði Sigurðr grǫf mikla á veginum ok gekk Sigurðr þar í. En er Fáfnir skreið af gullinu blés hann eitri ok hraut þat fyr ofan hǫfuð Sigurði. En er Fáfnir skreið yfir grǫfina þá lagði Sigurðr hann með sverði til hjarta. Fáfnir hristi sik ok barði hǫfði ok sporði. Sigurðr hljóp ór grǫfinni ok sá þá hvárr annan. Fáfnir kvað:\epa

\bpb Siward journeyed home to Helpric. Then Rein incited Siward to smite Fathomer. Siward and Rein journeyed up on the Gnit-heath and found there Siward’s trail as he was slithering to water. There Siward made a great trench in the way, and Siward went down into it. And when Fathomer slithered off the gold he blew venom, and it flew over Siward’s head. But when Fathomer slithered over the trench, then Siward ran him through with the sword to the heart. Fathomer shook himself and struck his head and spurned. Siward leapt out of the trench, and then each of them saw the other. Fathomer quoth:\epb\epg
