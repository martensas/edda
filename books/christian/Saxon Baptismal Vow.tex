\bookStart{Old Saxon Baptismal Vow}

\begin{flushright}%
\textbf{Dating:} ?

\textbf{Meter:} None
\end{flushright}%

\section{Introduction}

While not an alliterative poem in the slightest, this short text is important for its mention of Saxon Heathen Gods, for which reason I have here set it before the Christian poetry, in order to give some relevant cultural context.

The format of the text is straight-forward and resembles the modern Catholic questions posed to participants during the Sacrament of Confirmation (TODO: reference).  The person to be baptised is to respond positively to three denying and three affirming questions; first to forsake the Devil, all “Devil-yields” (i.e. non-Christian rituals, see note to that word), and all the Devil’s works and words and followers, among which are listed the three Germanic-Saxon gods Thunder, Weden, and Saxneet; second to profess belief in each member of the Trinity: God the Almighty Father, Christ, son of God, and the Holy Ghost (P6).

\sectionline

\section{Text}

\subsection{Old Saxon Baptismal Vow}

\bpg\bpa[0]%
„For·sachistu diobole?“ et respondeat: „ec for·sacho diabole“\epa

\bpb “Forsakest thou the Devil?” \emph{and he should respond:} “I forsake the Devil.”\epb\epg


\bpg\bpa[0][2]%
„end allum \edtrans{diobol-gelde}{devil-yields}{\Bfootnote{An obvious calque of OE TODO, which means TODO.}}?“ respondeat: „end ec for·sacho allum diobol-gelde.“\epa

\bpb “And all devil-yields?” \emph{he should respond:} “I forsake all devil-yields.”\epb\epg


\bpg\bpa[0][4]%
„End allum dioboles wercum?“ respondeat „end ec for·sacho allum dioboles wercum and wordum, Thuner ende Wóden ende Sax-nôte ende allem them un·holdum the hira ge·nôtas sint.“\epa

\bpb “And all the Devil’s works” \emph{he should respond:} “and I forsake all the works and words of the Devil; Thunder and Weden and Saxneet and all those unhold ones who are their fellows.”\epb\epg


\bpg\bpa[0][7]%
„Ge·lôbistu in Got ala-męhtigun fader?“ „Ec ge·lôbo in Got ala-męhtigun fader.“\epa

\bpb “Believest thou in God, the almighty father?” “I believe in God, the almighty father.”\epb\epg


\bpg\bpa[0][9]%
„Ge·lôbistu in Crist Godes suno?“ „Ec ge·lôbo in Crist Gotes suno.“\epa

\bpb “Believest thou in Christ, God’s son?” “I believe in Christ, God’s son.”\epb\epg


\bpg\bpa[0][11]%
„Ge·lôbistu in hâlogan gâst?“ „Ec ge·lôbo in hâlogan gâst.“\epa

\bpb “Believest thou in the Holy Ghost?” “I believe in the Holy Ghost.”\epb\epg

\sectionline
