\bookStart{Old Saxon Baptismal Vow}

\begin{flushright}%
\textbf{Dating:} ?

\textbf{Meter:} None
\end{flushright}%

\section{Introduction}

While not an alliterative poem in the slightest, the \textbf{Old Saxon Baptismal Vow} is an extremely important historical document.  It was under just this vow many an Old Saxon would have gone during his forced baptism in the late 8th century, in many cases directly before his execution.  For this reason I have here placed it before the two Old Saxon Biblical epics, in order to give some relevant cultural context.

The format of the short text is straight-forward and resembles the Catholic questions posed to participants during the Sacrament of Confirmation (TODO: reference).  The person to be baptised is to respond positively to three denying and three affirming questions.  He is to forsake the Devil, all “devil-yields” (pagan rituals), all the Devil’s works and words and his followers, among which the three Old Saxon gods Thunder, Weden, and Saxneet are mentioned by name.  He is then to profess belief in each member of the Trinity: God the Father, Jesus Christ, and the Holy Ghost.

\sectionline

\section{Text}

\bpg\bpa%
„For·sachistu diobole?“ \\
et respondeat: „ec for·sacho diabole“\epa

\bpb “Forsakest thou the Devil?” \\
\emph{and he should respond:} “I forsake the Devil.”\epb\epg


\bpg\bpa%
„end allum \edtrans{diobol-gelde}{devil-yields}{\Bfootnote{An obvious calque of OE \emph{déofol-ġield} ‘idolatrous acts of worship’.}}?“ \\
respondeat: „end ec for·sacho allum diobol-gelde.“\epa

\bpb “And all devil-yields?” \\
\emph{he should respond:} “I forsake all devil-yields.”\epb\epg


\bpg\bpa%
„End allum dioboles wercum?“ \\
respondeat „end ec for·sacho allum dioboles wercum and wordum, Thuner ende Wóden ende Sax-nôte ende allem them un·holdum the hira ge·nôtas sint.“\epa

\bpb “And all the Devil’s works?” \\
\emph{he should respond:} “and I forsake all the Devil’s works and words, Thunder and Weden and Saxneat and all the unhold ones which are their fellows.”\epb\epg


\bpg\bpa%
„Ge·lôbistu in Got ala-męhtigun fader?“ \\
„Ec ge·lôbo in Got ala-męhtigun fader.“\epa

\bpb “Believest thou in God, the Almighty Father?” \\
“I believe in God, the Almighty Father.”\epb\epg


\bpg\bpa%
„Ge·lôbistu in Crist Godes suno?“ \\
„Ec ge·lôbo in Crist Gotes suno.“\epa

\bpb “Believest thou in Christ, God’s son?” \\
“I believe in Christ, God’s son.”\epb\epg


\bpg\bpa%
„Ge·lôbistu in hâlogan gâst?“ \\
„Ec ge·lôbo in hâlogan gâst.“\epa

\bpb “Believest thou in the Holy Ghost?” \\
“I believe in the Holy Ghost.”\epb\epg

\sectionline
