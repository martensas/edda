\bookStart{Wessobrunn Hymn}
\setBookCode{Wessobrunn}

\begin{flushright}%
\textbf{Dating:} late 700s

\textbf{Meter:} \Fornyrdislag%para
\end{flushright}%

\section{Introduction}

The so-called \textbf{Wessobrunn Hymn} is found in a late C8th Bavarian manuscript with the Latin heading \emph{De poeta} ‘By the poet’.  The text was divided by the scribe into three parts, each introduced by a capital letter adorned with dots of red ink.  The first two parts are poetic (“the poem”), and the third is in prose (“the prayer”).

The poem consists of 9 long-lines in alliterative meter, detailing the earliest beginning of the world.  The first five lines describe “the greatest of wonders”, namely that the universe was once void, without earth or heaven, wood or mountain, sun or moon or sea.  These lines are very similar to pre-Christian Norse stanzas about the creation of the world, and in fact contain formulaic word-pairs also found in those stanzas (see notes to ll. 2, 3), suggesting a repurposing of older Heathen motifs and expressions in the new, Christian context.  With this in mind, the latter four lines constitute a subversion of the earlier Heathen tradition, by placing in this early emptiness the Almighty God, Jesus Christ, and His many ghosts—presumably the Heavenly Host or the Angels.  This is the Christian creation \emph{ex nihilo}, rather than the Indo-European creation \emph{ex materia} through sacrifice of a primordial being (see note to \textlink{Vafthrudnismal}[21], \textlink{Grimnismal}[41]–42).

The prayer is in prose.  The speaker first thanks God for creating the earth and heaven—this is presumably why the poem was included—and for giving boons to mankind.  He then asks for faith, strength, and wisdom to help him in his mission.

\newpage

\section{Wessobrunn Hymn}

\bvg\bva%
Dat \edtext{ga·\alst{f}ręgin}{\Afootnote{\emph{ga-} is abbrev. by the rune-like symbol \emph{ᛡ}.  This symbol is used for all other occurrences of \emph{ga-} in the present text except for \emph{ga·náda} and \emph{ga·laupa} in P1 below.}} ih mit \alst{f}irahim \hld\ \alst{f}iri-wizzó męista, &
dat \edtext{\edtext{\alst{e}r\emph{d}o}{\Afootnote{\emph{ero} ms.}} ni was \hld\ noh \alst{ú}f-himil}{\lemma{erdo \dots\ úf-himil ‘Earth \dots\ Up-heaven’}\Bfootnote{A formulaic merism attested in numerous Germanic languages, expressing the totality of the universe.  Cf. especially \textlink{Vafthrudnismal}[21], where the god Weden asks the ettin Webthrithner about the origin of “Earth and Up-heaven”, and \textlink{Voluspa}[3]/3, where it is said, about the time before the World existed, that “Earth and Up-heaven were never found”.}} &
\edtext{noh \alst{p}aum \hld\ noh \alst{p}erek ni was}{\lemma{paum \dots\ perek ‘wood \dots\ mountain’}\Bfootnote{The same word-pair is found in the OHG \textlink{Muspilli} 50 (describing the Christian destruction of the world prior to the Judgment) and in the ON \textlink{Grimnismal}[40] (describing the creation of the world from Yimer’s body by the Gods). — For metrical reasons the line is clearly defective; \emph{noh paum} is not an acceptable a-verse.}} &
ni [...] nohh-ęinig \hld\ noh sunna ni skęin &
noh \alst{m}áno ni liuhta \hld\ noh der \alst{m}árjo sêo.\eva

\bvb%
{\huge I} \textsc{have learned among men} this greatest of wonders, \\
that Earth was not, nor Up-heaven, \\
nor wood, nor was there mountain, \\
nor did any [...], nor did the sun shine, \\
nor the moon give off light, nor the glittering sea.\evb\evg


\bvg\bva[][6]%
Dó dar ni·\alst{w}iht ni \alst{w}as \hld\ ęntjó ni \alst{w}ęntjó, &
ęnti dó was der \alst{ęi}no \hld\ \alst{a}l-mahtiko kot, &
\alst{m}anno \alst{m}iltisto, \hld\ ęnti dar wárun auh \alst{m}anaké mit inan &
\alst{k}ót-líhhé \alst{g}ęistá, \hld\ ęnti \alst{k}ot hęilak.\eva

\bvb%
Then there was no kind of end or border, \\
and then was the one Almighty God, \\
the Mildest of Men, and there were also many \\
glorious ghosts with Him, and Holy God.\evb\evg


\bpg\bpa Kot al-mahtiko, dú himil ęnti erda ga·worahtós, ęnti dú mannun só manak kót for·gápi,
for·gip mir in dína ga·náda rehta ga·laupa, ęnti kótan willjon; wís-tóm ęnti spáhida ęnti kraft tiuflun za widar·stantanne, ęnti ark za pi·wísanne, ęnti dínan willjon za ga·wurkhanne.\epa

\bpb O God almighty! Thou wroughtest heaven and earth and Thou gavest men so much good.
Give me in Thy mercy right belief and good will, wisdom and foresight and power, to withstand devils and to reproach queerness and to work thy will.\epb\epg

\sectionline
