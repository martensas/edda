\bookStart{Cadman’s Hymn}
\setBookCode{Cadman}

\begin{flushright}%
\textbf{Dating:} C7th

\textbf{Meter:} \Fornyrdislag%para
\end{flushright}%

\section{Introduction}

\textbf{Cadman’s Hymn} is a short Old English poem found in numerous recensions of Bede’s English history, attributed to the illiterate shepherd Cadman (OE \emph{Cædmon}).  It employs several traditional Germanic poetic formulae and clearly draws on earlier, now-lost pagan compositions.

The hymn has two discernible parts.  Lines 1–4 serve as an introduction and statement of purpose, while 5–9 describe the creation, first of the Heaven (5–6) and second of Middenyard and the Earth (7–9).

Based on Kaluza’s law in the younger \Beowulf\ the pres. ed. marks etymologically long unstressed vowels.

\section{Cadman’s Hymn}

\bvg\bva[]%
Nú scylun \alst{h}ęrgan \hld\ \alst{h}ebæn-rícæs ward, &
\alst{m}etudæs \alst{m}æhti \hld\ end his \alst{m}ód-gi·þanc, &
\alst{w}erc \alst{w}uldur-fadur, \hld\ swé hé \alst{w}undrá gi·hwæs, &
\alst{é}cí dryhtin \hld\ \alst{ó}r ȧ·stęlidǽ. &
Hé \alst{ǽ}rist scóp \hld\ \alst{æ}ldá barnum &
\alst{h}ebæn til \alst{h}rófǽ, \hld\ \alst{h}âlig scęppend. &
Þȧ \alst{m}iddun-geard \hld\ \alst{m}ǫn-cynnæs ward, &
\alst{é}cí dryhtin \hld\ \alst{æ}fter tíadǽ, &
\alst{f}írum \alst{f}oldú \hld\ \alst{f}réá al-mæhtig.\eva

\bvb%
{\huge N}\textsc{ow shall we praise} the heavenly realm’s Guardian, \\
the Measurer’s might and His thinking mind, \\
the works of the Glory-Father, as He every wonder, \\
the everlasting Lord, in the beginning set up. \\
He first created for the children of men \\
the heaven as a roof, the holy Creator. \\
Then Middenyard did Mankind’s Guardian, \\
the everlasting Lord afterwards make: \\
the land for humans, the Lord Almighty.\evb\evg
