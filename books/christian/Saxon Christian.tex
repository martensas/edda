\bookStart{Introduction to Old Saxon Christian Poetry}

The forced conversion of the Saxons to Christianity under Charlemagne was a notably violent process, involving 32 years of war from 772 to 804 CE.  TODO: historical discussion.

It was only a generation after the Saxon Wars that the two poems edited here—the \textbf{Healend} (abbrev.~\Heliand) and the \textbf{Saxon Genesis} (abbrev.~\SaxonGenesis) were composed under the guidance of the Frankish king.  Their political purpose is clear enough: to depict Biblical stories and figures in the language of Germanic epic, and thereby to allow the aristocratic Saxon audience to identify with the but recently imposed religion.

Together \textlink{Heliand} and \textlink{SaxonGenesis} form the totality of the Old Saxon poetic corpus.  Both are thoroughly Christian, and although written in the language of traditional epic, surely for a noble audience, they launch a pointed Christian attack on the fatalistic and warlike worldview of Germanic Paganism.

Before the two poems I present the \textbf{Old Saxon Baptismal Vow} as an important piece of historical context.

\sectionline
