\bookStart{The Two Merseburg Galders}

\begin{flushright}%
Dating: TODO.

Meter: \Fornyrdislag, \Galdralag%
\end{flushright}

These two galders, preserved in a manuscript (TODO) are some of the only surviving examples of genuine Heathen galders from the continent. The two share a common two-part structure, each beginning with an \emph{historiola} (a pseudo-historical account describing the successful effects of the galder in the mythic past), followed by an \emph{imperative}, commanding that the willed effects take place in the present.

The first galder begins with an historiola describing a group of supernatural women in the midst of a battle who placed soldiers in fetters, hindering an army. The imperative then commands that some fetters in the present be destroyed so that captive(s) can escape.

The second galder begins with an historiola describing a group of Gods riding through the woods. Among them is Balder, whose horse sprains its foot. Three Gods are said to have sung (see Note to \emph{bi·guol} below) a healing-galder each over the horse in order to heal it. First sang the goddess Sithguth, then the goddess Sun, and finally the god Weden. The imperative (apparently the same as was sung over Balder’s horse) then commands that a sprain in the present be healed.

\sectionline

\bvg
\bva Ęiris \alst{s}ázun idisi \hld\ \alst{s}ázun hera duo der; &
suma \alst{h}apt \alst{h}ęptidun \hld\ suma \alst{h}ęri lęzidun &
suma \alst{k}lubodun \hld\ umbi \alst{k}uonjo-widi &
\alst{i}n·sprink hapt-bandun \hld\ \alst{i}n·far fígandun &
\edtext{.H.}{\Bfootnote{The meaning of this letter, which is very clear and written in the same hand as the galders, is uncertain. To me, the most convincing suggestion is that it be read as \emph{.N.}, short for Latin \emph{nomen} ‘name’, presumably the name for the person whom the singer wishes to free from the fetters.}}\eva

\bvb Of yore sat dises, sat here, then there: \\
some fastened fetters, some hindered armies, \\
some cleaved shackles (TODO!).— \\
Destroy the fetter-bonds, flee the fiends! \\
.H.\evb
\evg


\bvg
\bva \edtext{\alst{F}ol}{\Afootnote{\emph{Phol} ms.}} ęnde Wódan \hld\ \alst{f}uorun zi holza &
dú wart demo Balderes \alst{f}olon \hld\ sín \alst{f}uoz bi·ręnkit &
þú \edtrans{bi·guol}{begale}{\Bfootnote{third past singular of \emph{bi·galan} ‘begale’, transitive of \emph{galan} ‘gale, sing a galder’. This verb is important as it is the origin of the verbal noun “galder” (literally ‘something galed’), which is thus shown to describe the charm.}} en \edtext{\alst{S}inthgunt}{\lemma{Sinthgunt}\Afootnote{\emph{Sinhtgunt} ms.}} \hld\ \alst{S}unna era swister &
þú bi·guol en \alst{F}rija \hld\ \alst{F}olla era swister &
þú bi·guol en \alst{W}ódan \hld\ só hé \alst{w}ola konda &
só-se \alst{b}ên-ręnkí \hld\ só-se \alst{b}luot-ręnkí \hld\ só-se lidi-ręnkí &
\ind \alst{b}ên zi \alst{b}êna &
\ind \alst{b}luot zi \alst{b}luoda &
\alst{l}id zi ge·\alst{l}iden \hld\ só-se ge·\alst{l}imida sín!\eva

\bvb Phol and Weden journeyed in the woods; \\
then was the foot of Balder’s foal sprained. \\
Then \inx[C]{begale}[begaled] him \inx[P]{Sithguth}, \inx[P]{Sun} her sister; \\
then begaled him \inx[P]{Frie}, \inx[P]{Full} her sister; \\
then begaled him Weden, as he knew well: \\
Like bone-sprain, like blood-sprain, like joint-sprain! \\
Bone to bone, \\
blood to blood, \\
joint to joints, like were they glued together!\evb
\evg
