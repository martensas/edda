\bookStart{The Speeches of Fathomer}[Fáfnismǫ́l]

\begin{flushright}%
Dating \parencite{Sapp2022}: C10th (0.442), early C11th (0.402), late C11th (0.155)

Meter: \Ljodahattr\ (TODO)%
\end{flushright}

\sectionline

\bvg {\small [Fathomer quoth:]}
\bva „Svęinn ok svęinn! \hld\ Hvęrjum estu svęini of borinn? &
\ind Hvęrra estu manna mǫgr? &
es þú á Fáfni rautt \hld\ þínn hinn frána mę́ki; &
\ind stǫndumk til hjarta hjǫrr!“\eva

\bvb “Swain and swain! To which swain art thou born; of which men art thou the son? As thou on Fathomer hast reddened thy gleaming blade, the sword stands to my the heart!”\evb
\evg


BPG
BPA Sigurðr dulði nafns síns fyr því at þat var trúa þeira í forneskju at orð feigs manns mę́tti mikit ef hann bǫlvaði óvin sínum með nafni. Hann kvað:EPA

BPB Siward concealed his name, because it was their belief in ancient times that the word of a \inx[C]{fey} man could do much if he baled his enemy by his name. He \ken*{= Siward} quoth:EPB
EPG


\bvg
\bva „Gǫfugt dýr ek hęiti \hld\ en ek gęngit hef’k &
\ind hinn móðurlausi mǫgr, &
fǫður ek á’kk-a \hld\ sem fira synir, &
\ind gęng ek ęinn saman.“\eva

\bvb “Noble beast I am called, but I have walked as the motherless lad. A father I own not, like the sons of men do; I walk alone.”\evb
\evg


\bvg {\small [Fathomer quoth:]}
\bva „Vęizt, ef fǫður né átt-at \hld\ sem fira synir, &
\ind af hvęrju vastu undri alinn?“\eva

\bvb “Knowest thou, if thou haddest not a father like the sons of men, by which wonder thou wast born?”\evb
\evg


\bvg {\small [Siward quoth:]}
\bva „Ę́tterni mitt \hld\ kveð’k þér ókunnigt vesa &
\ind ok mik sjalfan hit sama: &
Sigurðr ek hęiti \hld\ Sigmundr hét minn faðir &
\ind es hęf’k þik vápnum vegit.“\eva

\bvb “My lineage I say is unknown to thee, and my self the same.\footnoteB{The meaning is that Fathomer would not recognize Siward’s lineage (i.e. his father) or name, since he is an orphan who up until this point has not won any glory. He is not saying that he is lineage is unknown even to himself, since \emph{sjalfan mik} ‘my self’ is accusative, not dative.} Siward I am called—Sighmund was called my father—who with weapons have struck thee.”\evb
\evg


\bvg {\small [Fathomer quoth:]}
\bva „Hvęrr þik hvatti, \hld\ hví hvętjask lézt, &
\ind mínu fjǫrvi at fara? &
Hinn fránęygi svęinn, \hld\ þú áttir fǫður bitran, &
\ind ábornu skjór á skęið.“\eva

\bvb “Who goaded thee—why didst thou let thyself be goaded—my life for to destroy? Gleaming-eyed swain, thou haddest a sharp father; inborn traits show quickly.\footnoteB{The original is unclear. \emph{á skęið} means roughly ‘rapidly, quickly’; thus \emph{ríða á skęið} \CV: ‘to ride at full speed’, but the other words are uncertain. \textcite{LaFargeGlossary} read ‘your innate qualities show quickly’, suggesting two unattested words: an adjective \emph{*áborinn} ‘innate, inborn’ and a verb \emph{*skjóa} ‘to show’. Yet the lack of i-umlaut in the supposed 3rd sg. pres. ind. \emph{skjór} is difficult. We would expect \emph{**skýr}, as in \emph{skjóta} ‘to shoot,’ with 2nd/3rd sg. pres. ind \emph{skýtr}. A solution here would be reading a 2nd sg. pres. subj. \emph{skjóir}, with a vowel TODO}”\evb
\evg

TODO: More verses...
