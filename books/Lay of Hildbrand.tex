\bookStart{The Lay of Hildbrand}

% Introduction

For the text of original poem I generally present the manuscript text. I have found it impossible to produce a normalization without too heavily distorting the received text, being as it is, a blend of several dialects. I have, however, added acute accents to signify long vowels, capitalized proper names, consistently replaced \emph{ƿ} (wynn) and \emph{uu} with \emph{w}, and made minor corrections where the manuscript is clearly in error—these are noted in the critical apparatus. The punctuation of the original, entirely consisting of interpuncts, at times representing line breaks and cæsuræ and at others sporadically placed, has not been retained.

Where they appear in cæsuræ, the words \emph{quad Hiltibrant} ‘Hildbrand quoth’ (found in ll., 30, 49, and 58) replace the usual interpunct. I had originally planned to remove these as hypermetrical, instead indicating the speaker above the verse, but after comparison with \Reginsmal\ 3, wherein the words \emph{kvað Loki} ‘Lock quoth’ appear in the first cæsura of the verse, I have come to believe that these represent an ancient oral indication, seemingly going back as far as the Migration Period (as it seems incredulous to think that the scribe of \HildMS\ would have influenced the scribe of \Regius\ four centuries later in such a minor point.)


\bvg
\bva[0]Ik gihórta dat seggen &
dat sih \alst{u}rhettun \hld\ aenon muotín &
\alst{H}iltibrant enti \alst{H}adubrant \hld\ untar \alst{h}eriun twém &
\alst{s}unufatarungo \hld\ iro \alst{s}aro rihtun &
\alst{g}arutun se iro \alst{g}údhamun \hld\ \alst{g}urtun sih iro swert ana &
\alst{h}elidos ubar \edtext{\alst{h}ringa}{\lemma{hringa}\Afootnote{ringa \HildMS}} \hld\ dó sie to dero \alst{h}iltiu ritun\eva

\bvb[0] I heard it said, that two contenders alone did meet: Hildbrand and Hathbrand, under two hosts.\footnoteB{i.e. each man was a champion of his respective army.} Son and father ordered their armour, readied their war-cloth, girded their swords on, the heroes over the mail, when to that battle they rode.\evb
\evg


\bvg\setlinenum{6}
\bva[0]\alst{H}iltibrant \edtext{gimahalta}{\Afootnote{\emph{add.} heribrantes sunu “Harbrand’s son” \HildMS}} \hld\ her was \alst{h}éróro man &
\alst{f}erahes \alst{f}rótóro \hld\ her \alst{f}rágén gistuont &
\alst{f}óhém wortum \hld\ \edtext{hwer}{\Afootnote{wer \HildMS}} sín \alst{f}ater wári &
\alst{f}ireo in \alst{f}olche \hld\ {[...]} &
{[...]} \hld\ „eddo \edtext{hwelíhhes}{\Afootnote{welihhes \HildMS}} \alst{c}nuosles dú sís &
ibu dú mí \alst{é}nan sagés \hld\ ik mí de \alst{o}dre wét &
\alst{ch}ind in \edtext{\alst{ch}unincríche}{\lemma{chunincríche}\Afootnote{chunnincriche \HildMS}} \hld\ \alst{ch}úd ist mín al irmindeot“\eva

\bvb[0] Hildbrand spoke—he was the hoarier man, more learned in life—he began to ask, with few words, who his father might be, of men in the troop, [...] “or of which lineage thou be; if thou me one say, I the others will know; child, in the kingdom, known to me are all great men.”\evb
\evg


\bvg\setlinenum{13}
\bva[0]Hadubrant gimahalta \hld\ Hiltibrantes sunu &
\edtext{„dat sagetun mí \hld\ úsere liuti}{\lemma{dat ... liuti}\Bfootnote{this l. breaks no rhythmic rules (cf. l. 42), but the needed alliteration is missing.}} &
\alst{a}lte anti fróte \hld\ dea \alst{é}rhina wárun &
dat \alst{H}iltibrant haetti mín fater \hld\ ih heittu \alst{H}adubrant &
forn her \alst{ó}star \edtext{giweit}{\Afootnote{gihueit \HildMS}} \hld\ flóh her \alst{Ó}tachres níd &
hina miti \alst{Th}eotríhhe \hld\ enti sínero \alst{d}egano filu &
her fur\alst{l}aet in \alst{l}ante \hld\ \alst{l}uttila sitten &
\edtext{\alst{b}rút}{\lemma{brút}\Afootnote{prut \HildMS}} in \alst{b}úre \hld\ \alst{b}arn unwahsan &
\alst{a}rbeolaosa \hld\ \edtext{her raet}{\Afootnote{heraet \HildMS}} \alst{ó}star hina &
det síd \alst{D}etríhhe \hld\ \alst{d}arba gistuontum &
\edtext{\alst{f}ateres}{\lemma{fateres}\Afootnote{fatereres \HildMS}} mínes \hld\ dat was só \alst{f}riuntlaos man &
her was \alst{Ó}tachre \hld\ \alst{u}mmet tirri &
\alst{d}egano \alst{d}echisto \hld\ unti \edtext{\alst{D}eotríchhe}{\lemma{Deotríchhe}\Afootnote{\emph{add.} darba gistontun \HildMS}} &
her was eo \alst{f}olches at ente \hld\ imo was eo \edtext{\alst{f}ehta}{\lemma{fehta}\Afootnote{peheta \HildMS}} ti leop &
\alst{ch}úd was her \hld\ \edtext{\alst{ch}óném}{\lemma{chóném}\Afootnote{chonnem \HildMS}} mannum &
ni wániu ih iu líb habbe“\eva

\bvb[0] Hathbrand spoke, Hildbrand’s son: “It told me our people, the old and learned, those who earlier lived, that Hildbrand was called my father — I am called Hathbrand. Long ago he hurried east — he fled Edwaker’s hate — thither with Thedrich, and his great many thanes. He left in the land a little one to stay, a bride in the bower, a bairn ungrown, without inheritance; he rode east thither, as Thedrich was in great need of my father; — that was so friendless a man. He was to Edwaker exceptionally hostile, the dearest of thanes under Thedrich. He was ever at the front of the troop, ever did the fight gladden him, known was he among keen men; I ween not that he have life.”\evb
\evg


\bvg\setlinenum{29}
\bva[0]„wettu \alst{i}rmingot {\small (quad Hiltibrant)} \alst{o}bana ab \edtext{hebane}{\Afootnote{heuane \HildMS}} &
dat dú neo dana halt mit sus sippan man &
dinc ni gileitós“ &
\alst{w}ant her dó ar arme \hld\ \alst{w}untane bauga &
\alst{ch}eisuringu gitán \hld\ so imo sie der \alst{ch}uning gap &
\alst{h}uneo truhtin \hld\ „dat ih dir it nú bí \alst{h}uldí gibu“\eva

\bvb[0] “I call on Ermin-god as witness, above in heaven, that thou never with such a close man once more lead dispute.” Unwound he then from his arm some twisted bighs\footnoteA{Armlets used as currency during the Migration Period; ON \emph{baugr}, OE \emph{béag}. — The giving of rings and armlets in exchange for loyalty was common across all of Germanic Europe, as seen in the many ruler-kennings of the type “breaker of rings” (like \emph{béaga brytta} “the breaker of bighs” \Beowulf\ ll. 35, 352, 1487.) This is also connected with the oath-ring, and the famous ring-swords. TODO? reference some literature on this.}, made from imperial coin, which the king once gave him, the lord of the Huns—“This I now give thee as pledge.”\evb
\evg


\bvg\setlinenum{35}
\bva[0]\alst{H}adubrant gimahalta \hld\ \alst{H}iltibrantes sunu &
„mit \alst{g}éru scal man \hld\ \alst{g}eba infáhan &
\alst{o}rt widar \alst{o}rte \hld\ [...] &
dú bist dir \alst{a}ltér hun \hld\ \alst{u}mmet spáhér &
\alst{sp}enis mih mit díném wortun \hld\ wili mih dínu \alst{sp}eru werpan &
\edtext{bist}{\Afootnote{pist \HildMS}} alsó gialtét man \hld\ só dú éwín inwit fórtós &
dat \alst{s}agetun mí \hld\ \alst{s}éolídante &
\alst{w}estar ubar \alst{W}entilséo \hld\ dat man \alst{w}íc furnam &
tót ist \alst{H}iltibrant \hld\ \alst{H}eribrantes suno“\eva

\bvb[0] Hathbrand spoke, Hildbrand’s son: “With spear shall one earn gifts, point against point! Thou art, old Hun, exceptionally clever; thou lurest me with thy words, wilt thou at me thy spear hurl! Thou art thus old, though thou ever deceit didst work. — It told me seafarers, heading west o’er the Wendle-sea\footnoteA{The Mediterranean, referring to the Vandals in North Africa.}, that war took that man: — dead is Hildbrand, Harbrand’s son!”\evb
\evg


\bvg\setlinenum{44}
\bva[0]\alst{H}iltibrant gimahalta \hld\ \alst{H}eribrantes suno &
„wela gisihu ih \hld\ in díném hrustim &
dat dú \alst{h}abés \alst{h}éme \hld\ \alst{h}érron góten &
dat dú noh bí desemo \alst{r}íche \hld\ \alst{r}eccheo ni wurti“\eva

\bvb[0] Hildbrand spoke, Harbrand’s son: “I see well on thy equipment, that thou hast a good lord at home, that thou still in this reign didst not become an exile.”\evb
\evg


\bvg\setlinenum{48}
\bva[0]„\alst{w}elaga nú \alst{w}altant got {\small (quad Hiltibrant)} \alst{w}éwurt skihit &
ih wallóta \alst{s}umaro enti wintro \hld\ \alst{s}ehstic ur lante &
dar man mih eo \alst{sc}erita \hld\ in folc \alst{sc}eotantero &
só man mir at \alst{b}urc énigeru \hld\ \alst{b}anun ni gifasta &
nú scal mih \alst{s}wásat chind \hld\ \alst{s}wertu hauwan &
\alst{b}retón mit sínu \alst{b}illiu \hld\ eddo ih imo ti \alst{b}anin werdan &
doh maht dú nú \alst{ao}dlíhho \hld\ ibu dir dín \alst{e}llen taoc &
in sus \alst{h}éremo man \hld\ \alst{h}rusti giwinnan &
\alst{r}auba \edtext{bi\alst{r}ahanen}{\lemma{birahanen}\Afootnote{bihrahanen \HildMS}} \hld\ ibu dú dar éníg \alst{r}eht habés“\eva

\bvb[0] “Well now, wielding God! woeful Weird\footnoteA{The personification of fate, in this case most likely just a noun. OE \emph{Wyrd} (\Beowulf\ 455: \emph{Gǽð á Wyrd swá hío scel} “Ever goes Weird as she must”), ON \emph{Urðr} ‘one of the norns’.} comes to pass. I wallowed for summers and winters sixty out of the land, where one ever set me in the troop of shooters; thus one at no fortress my bane did inflict. Now shall my own child hew at me with sword; beat down with his blade, or I his bane become. Yet canst thou now easily—if thy zeal avail thee—from such a hoary man win the equipment; bear away the booty, if thou thereto have any right.”\evb
\evg


\bvg\setlinenum{57}
\bva[0]„der sí doh nú \alst{a}rgósto {\small (quad Hiltibrant)} \alst{ó}starliuto &
der dir nú \alst{w}íges \alst{w}arne \hld\ nú dih es só \alst{w}el lustit &
gúdea gi\alst{m}einun \hld\ niuse de \alst{m}ótti &
\edtext{hwedar}{\Afootnote{werdar \HildMS}} sih \edtext{\alst{h}iutu déro}{\lemma{hiutu déro}\Afootnote{dero hiutu \HildMS}} \alst{h}regilo \hld\ \edtext{\alst{h}ruomen}{\lemma{hruomen}\Afootnote{hrumen \HildMS}} muotti &
\edtext{eddo}{\Afootnote{erdo \HildMS}} desero \alst{b}runnóno \hld\ \alst{b}édero waltan“\eva

\bvb[0] “He be now the weakest of the eastern peoples, who refuse thee the fight, when thou so greatly cravest to struggle together; — try he who might, which of us today of these garments may boast, or both of these byrnies wield!”\evb
\evg


\bvg\setlinenum{62}
\bva[0]dó lettun se \alst{ae}rist \hld\ \alst{a}sckim scrítan &
\alst{sc}arpén \alst{sc}úrim \hld\ dat in dem \alst{sc}iltim stónt &
dó \alst{st}óptun tosamane \hld\ \alst{st}aimbort \edtext{hlúdun}{\Afootnote{chludun \HildMS}} &
\alst{h}ewun harmlícco \hld\ \alst{h}wítte scilti &
unti imo iro \alst{l}intún \hld\ \alst{l}uttilo wurtun &
gi\alst{w}igan miti \alst{w}ábnum \hld\ [...]\eva

\bvb[0] Then let they first their ash-spears glide, in harsh torrents, that in the shields they stuck. Then charged they into each other—the war-boards \ken{shields} resounded—struck they bitterly the white shields, until for them their lindens \ken{shields} became little, worn down by the weapons, [...]\evb
\evg
