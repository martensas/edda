\book{\emph{Vǫluspǫ́} — The Spae of the Wallow}\bookStart

% Introduction.

{\small The “\inx{Spae} of the \inx{Wallow}” is attested in full in two principal versions. The earlier is the Codex Regius of the Poetic Edda, \Regius\ (GKS 2365 4to; 1270s), where it is the first poem, found on folios 1r–3r. The later is Hawksbook, \Hauksbok\ (AM 544 4to; 1300–75), where it is found at 20r–21r, in the middle of a large collection of saws and Catholics works. Many verses are also cited in \Gylfaginning, which here has the general siglum \GylfMS; to avoid confusion, it is only used when all employed witness mss. agree. The noted mss. of it are:\begin{enumerate} %TODO: move this list to introduction since plenty more poems are quoted in it.
	\item The Codex Regius of the Prose Edda \RegiusProse\ (GKS 2367 4to; 1300-1350)
	\item The Codex Trajectinus \Trajectinus\ (Traj 1374; a c. 1595 paper copy of a ms. closely related to \RegiusProse.)
	\item The Codex Wormianus \Wormianus\ (AM 242 fol.; 1340–70)
	\item The Codex Upsaliensis \Upsaliensis\ (DG 11; 1300–25)
\end{enumerate}

Thus, \GylfMS\ is equivalent to \RegiusProse\Trajectinus\Wormianus\Upsaliensis. — I further refer to the introduction of Finnur 1931 (p. III ff.).}

\bvg {\small Greeting to the audience, bidding of Weden.}
\bva\ledleftnote{\Regius\Hauksbok}\alst{H}ljóðs bið’k allar \hld \edtext{\alst{h}ęlgar}{\lemma{hęlgar}\Afootnote{\emph{om.} \Regius}} kindir, &%nvl
\alst{m}ęiri ok \alst{m}inni \hld \alst{m}ǫgu Hęimdallar; &%nvl
\alst{v}ildu at, \alst{V}alfǫðr, \hld \alst{v}ęl fram tęlja’k &%nvl
\alst{f}orn spjǫll \alst{f}ira, \hld þau’s \alst{f}ręmst of man?\eva

\bvb Of hearing I bid all holy kinds, the greater and lesser sons of \inx{Homedall}\footnoteB{Cf. \Rigsthula, wherein Homedall, under the name Righ, sires the three castes (earls, churls and thralls).}. Wilt thou, Leader of the Slain, that I well tell forth the ancient sayings of firs\footnoteB{Men.}, those I foremost recall?\footnoteB{Cf. \Vafthrudnismal\ 34, 35 with very similar phrasing.}\evb
\evg


\bvg {\small Wallow reckons what she recalls; the creation and ordering of the world.}
\bva\ledleftnote{\Regius\Hauksbok}Ek man jǫtna \hld ár of borna, &%nvl
þá es forðum \hld mik fǿdda hǫfðu; &%nvl
níu man’k hęima, \hld níu \edtext{íviðjur}{\Afootnote{\emph{Previously read} íviði, \emph{but closer study of} \Regius\ \emph{has disproven this. See Gripla 3, pp. 227–28.}}}, &%nvl
mjǫtvið mæran \hld fyr mold neðan.\eva

\bvb I recall \inx{ettins}, born of yore, those who anciently had nourished me. Nine \inx{homes} I recall, nine \inx{inwithies}, the renowned \inx{Metwood} beneath the earth.\footnoteB{Certainly Ugdrassle, “beneath the earth” likely referring to it still being a seed.}\evb
\evg


\bvg
\bva\ledleftnote{\Regius\Hauksbok\GylfMS}Ár vas alda \hld \edtext{þar’s Ymir byggði}{\Afootnote{þat’s ekki vas “[of elds], that which nothing was” \GylfMS}}, &%nvl
vas-a sandr né sær, \hld né svalar unnir; &%nvl
jǫrð fansk æva \hld né upphiminn; &%nvl
gap vas ginnunga, \hld ęn gras \edtext{hvęrgi}{\Afootnote{ekki \Hauksbok}}.\eva

\bvb It was the beginning of \inx{elds}, there where \inx{Yime} dwelled; was there not sand nor sea, nor cool waves. The earth was never found, nor \inx{up-heaven}; a gap was of ginnings\footnoteB{See index Gap of Ginnings. \emph{ginnungr} means ‘hawk’ in the Scoldish poetry, but that meaning is strange here, unless it be an obscure sky-kenning (referring to the void).}, but grass nowhere.\evb
\evg


\bvg
\bva Áðr Burs synir \hld bjǫðum of ypðu, &%nvl
þęir es Miðgarð \hld mæran skópu; &%nvl 
sól skein sunnan \hld á salar stęina; &%nvl
þá vas grund gróin \hld grǿnum lauki.\eva

\bvb Before the sons of Bur the flatlands did upwards lift, they who the renowned Middenyard shaped. Sun shone from the south on the stones of the hall; then was the ground grown with green leek.\footnoteB{The sons of Bur, that is Weden, Will and Wigh (cf. \Gylfaginning\ TODO), lift the lands out of the primordial chaos (the Gap of Ginnings).}\evb
\evg


\bvg
\bva\ledleftnote{\Regius\Hauksbok\GylfMS}\edtext{Sól varp sunnan, \hld sinni mána, &
hęndi hinni hǿgri \hld \edtext{umb himinjǫður}{\Afootnote{vm himin iodyr \Regius\ of ioður \Hauksbok}}}{\lemma{Sól ... himinjǫður}\Afootnote{\emph{om.} \GylfMS}}; &
sól þat né vissi, \hld hvar hón sali átti; &%nvl
\edtext{stjǫrnur þat né vissu, \hld hvar þær staði ǫ́ttu}{\lemma{stjǫrnur ... ǫ́ttu}\Bfootnote{In \GylfMS\ follows 5, so that order is sun, moon, stars.}}; &%nvl
máni þat né vissi, \hld hvat hann męgins átti.\eva

\bvb Sun cast from the south, — the companion of Moon\footnoteB{At times translated as “its moon”; this cannot be correct, as \emph{máni} ‘moon’ is masculine, while \emph{sinni}, dative singular of \emph{sínn} ‘its (reflexive)’ is feminine.}, — her right hand about heaven’s rim;\footnoteB{The sun heaved herself up over the horizon and rose for the first time?} Sun knew not, where halls she owned; stars knew not, where steads they owned; Moon knew not, what of might he owned.\evb
\evg


\bvg
\bva Þá gingu ręgin ǫll \hld á rǫkstóla, &%nvl
ginnhęilǫg goð, \hld ok gættusk umb þat.\eva

\bvb Then went the Powers all onto the rake-seats\footnoteB{Judgment-seats; first element \emph{rǫk} defined by \CV\ as ‘reason, ground, origin’.}: the gin-holy gods, and from each other took counsel about that.\footnoteB{10, 23, 25 would suggest two \inx{long-lines} be missing here.}\evb
\evg


\bvg
\bva Nótt ok niðjum \hld nǫfn of gǫ́fu, &%nvl
morgin hétu \hld ok miðjan dag, &%nvl
undurn ok aptan, \hld ǫ́rum at tęlja.\eva

\bvb To night and the moon’s phases, names did they give; morning they called, and middle day; afternoon and evening, the years for to tally.\footnoteB{Cf. \emph{Web} 23, 25.}\evb
\evg


\bvg
\bva Hittusk æsir \hld á Iðavęlli, &%nvl
\edtext{þęir’s hǫrg ok hof \hld hǫ́ timbruðu}{\lemma{þęir’s ... timbruðu}\Afootnote{afls kostuðu / allz freistuðu “[their] strength they tried; everything they tempted:” \Hauksbok}}; &%nvl
afla lǫgðu, \hld auð smíðuðu, &%nvl
tangir skópu \hld ok tól gęrðu.\eva

\bvb The Ease found each other on the \inx{Idewolds}, they who \inx{harrows} and \inx{hoves} high timbered: hearths they laid, wealth they smithed, tongs they shaped, and tools they made.\evb
\evg


\bvg
\bva Tęflðu í túni, \hld tęitir vǫ́ru, &%nvl
vas þeim véttugis \hld vant ór golli, &%nvl
unz þríar kvǫ́mu \hld þursa męyjar, &%nvl
ámátkar mjǫk, \hld ór Jǫtunhęimum.\eva

\bvb They played \inx{Tavel} in the yards, joyous were they: was for them no lack of gold\footnoteB{Cf. v. 59.}, until three came, maidens of \inx{thurses}, greatly terrifying, out of \inx{Ettinham}.\footnoteB{These are immediately forgotten and not again mentioned (unless they are taken to be the norns in v. 21, but they would then be introduced twice). — Clearly, there is something missing between this verse and the next, detailing the reason for creation of dwarves.}\evb
\evg


\bvg {\small Creation of dwarves.}
\bva\ledleftnote{\Regius\Hauksbok\GylfMS}Þá gingu ręgin ǫll \hld á rǫkstóla, &%nvl
ginnhęilǫg goð, \hld ok gættusk umb þat: &%nvl
\edtext{hvęrr skyldi dverga}{\Afootnote{\emph{thus} \Regius\Wormianus\Upsaliensis; at skyldi dverga “That they would [shape the troop] of dwarves” \RegiusProse\Trajectinus; hverir skyldu dvergar “Which dwarves would [shape the people]” \Hauksbok}} \hld \edtext{drótt of}{\Afootnote{\emph{thus} \GylfMS; drotin (\emph{late definite wo. doubt not original}) \Regius; dróttir “the people” \Hauksbok}} \edtext{skępja}{\Afootnote{spekia “soothe [the troop]” \Upsaliensis}} &
\edtext{ór \edtext{brimi blóðgu}{\Afootnote{\emph{thus} \Hauksbok\RegiusProse\Wormianus\Upsaliensis; Brimis blóði “[out of] the blood of Brimmer” \Regius\Trajectinus}} \hld ok ór \edtext{blǫ́um lęggjum}{\Afootnote{\emph{metr. emend}; blám leggiom “id.” \Regius; Bláins lęggjum “the legs of Blown” \Hauksbok\Wormianus; Bláms lęggjum (\emph{wo. doubt corrupt form of former}) \RegiusProse\Trajectinus\Upsaliensis}}}{\lemma{ór brimi ... lęggjum}\Bfootnote{I think that the poem simply telling of “the bloody surf” and “the black legs” fits better with its general allusive style, but this choice may be somewhat controversial.}}?\eva

\bvb — Then went the Powers all onto the rake-seats: the gin-holy gods, and from each other took counsel about that: Who would shape the troops of \inx{dwarves}, out of the bloody surf, and out of the blue-black legs\footnoteB{Gurevich (\emph{Skp} 2017, p. 693) (employing the translation of \FaulkesEdda\ p. 16) interprets the “legs of Blown (\emph{a dwarf})” as a kenning for ‘stone’, but this disagrees with the prose in \Gylfaginning\ (TODO), which states that the dwarves first originated as maggots in Yime’s rotting corpse.}?\evb
\evg


\bvg
\bva\ledleftnote{\Regius\Hauksbok\GylfMS}\edtext{\edtext{Þar vas Móðsognir}{\Afootnote{\emph{thus} \Hauksbok; Þar mótſognir vitnir “there Mootsown wolf” \emph{wo. doubt corrupt} \Regius\ — \emph{The prose of} \Gylfaginning\ \emph{confirms reading Móðsognir.}}} \hld mæztr of orðinn &
dverga allra, \hld ęn Durinn annarr;}{\lemma{Þar ... annarr}\Afootnote{\emph{om.} \GylfMS}} &
\edtext{þęir manlíkun \hld mǫrg of gęrðu,}{\lemma{þęir ... gęrðu}\Afootnote{\emph{thus} \Regius\Hauksbok\Upsaliensis; þar manlíkun / mǫrg of gęrðusk (\emph{norm.}) “There man-likenesses many were made” \RegiusProse\Trajectinus\Wormianus}} &
dvergar í \edtext{jǫrðu}{\Afootnote{iorðum “the earths” \Wormianus}}, \hld \edtext{sęm Durinn sagði}{\Afootnote{\emph{thus} \Regius\Hauksbok\RegiusProse\Wormianus; sem dur menn sagdi “as door-men said” \Trajectinus; sem þeim dyrinn kendi “as the animals taught them” \Upsaliensis}}.\eva

\bvb There was Moodsown become the worthiest of all dwarves, but Dorn [was] second. They made man-likenesses many; dwarves out of the earth, as Dorn said.\footnoteB{\emph{manlíkan} ‘man-likeness’ only appears here. Were the lower dwarves shaped out of soil by the higher dwarves Moodsown and Dorn? “as Dorn said” suggests that Dorn (and Moodsown?) had a role in the creation, but this is not stated in the prose of \Gylfaginning; see note to last v.}\evb
\evg

%TODO: mvoe these verses
\bvg {\small Two lists of dwarves. That both belonged to the original poem is impossible, since several names (Oakenshield, Great-grandfather) appear twice. The three following verses seem to belong together, since there is no repetition of names. From the last verse of the middle one, it seems that it should have been placed at the end of the group.}
\bva Nýi ok Niði, \hld Norðri, Suðri, &%nvl
Austri, Vestri, \hld Alþjófr, Dvalinn, &%nvl
Bívurr, Bávurr, \hld Bǫmburr, Nóri, &%nvl
Ánn ok Ánarr, \hld Ái, Mjǫðvitnir.\eva

\bvb — New and Nithe, Norther and Suther, Easter and Wester, Allthief, Dwollen, Bewer, Bower, Bamber, Noor, Own and Owner, Great-grandfather, Meadwitten.\evb
\evg


\bvg
\bva Vęigr ok Gandalfr, \hld Vindalfr, Þráinn, &%nvl
Þękkr ok Þorinn, \hld Þrór, Vitr ok Litr, &%nvl
Nár ok Nýráðr, \hld nú hęf’k dverga, &%nvl
Ręginn ok Ráðsviðr, \hld rétt of talða.\eva

\bvb Wey and Gandelf, Windelf, Thrown, Thetch and Thorn, Throo, Wit and Lit, Nee and Newred; — now have I the dwarves, — Rain and Redswith, — rightly tallied.\evb
\evg


\bvg
\bva Fíli, Kíli, \hld Fundinn, Náli, &%nvl
Hępti, Víli, \hld Hannarr, Svíurr, &%nvl
Frár, Hornbori, \hld Frægr ok Lóni, &%nvl
Aurvangr, Jari, \hld Ęikinskjaldi.\eva

\bvb Filer, Chiler, Founden and Needler, Heft, Wiler, Hanner, Swigher, Frew, Hornborer, Fray and Looner, Earwong, Erer, Oakenshield.\evb
\evg


\bvg
\bva Mál es dverga \hld í Dvalins liði &%nvl
ljóna kindum \hld til Lofars tęlja, &%nvl
\edtext{þęir}{\Afootnote{þeim \Hauksbok}} es sóttu \hld frá salar stęini &%nvl
aurvanga sjǫt \hld til Jǫruvalla.\eva

\bvb — ’Tis time to tally the dwarves in Dwollen’s host [back] to Loffer, for the kins of men\footnoteB{A standard genealogical introduction (compare \Haleygjatal\ 1). The line of dwarves is to be counted to their progenitor, Loffer. This possibly disagrees with the earlier introduction (“There was ...”), where Moodsown is said to be the foremost of the dwarves, and Loffer is not mentioned.}; they who sought, from the stone of the hall, the abode of \inx{Earwongs}\footnoteB{\CV\ \emph{aurvangr} ‘a loamy field’, and indeed this fits etymologically.} to the \inx{Erwolds}.\footnoteB{\Gylfaginning\ (TODO): “But these came from Swornshigh (\emph{Svarinshaugr}) to the Earwongs on the Erwolds, and thence Lofer is come; these are their names: Sherper (\emph{Skirpir}), Werper (\emph{Virpir}), Showfind, Great-grandfather, Elf and Ing (\emph{Ingi}), Oakenshield, Fale (\emph{Falr}), Frost, Finn, Ginner.”}\evb
\evg


\bvg
\bva Þar vas Draupnir \hld ok Dolgþrasir, &%nvl
Hár, Haugspori, \hld Hlévangr, Glói, &%nvl
Skirfir, Virfir, \hld Skáfiðr, Ái, &%nvl
Alfr ok Yngvi, \hld Ęikinskjaldi, &%nvl
Fjalarr ok Frosti, \hld Finnr ok Ginnarr; &%nvl
Þat mun \edtext{æ}{\Afootnote{\emph{om.} \Regius}} uppi, \hld meðan ǫld lifir, &%nvl
langniðja-tal \hld \edtext{til}{\Afootnote{\emph{om.} \Hauksbok}} Lofars hafat.\eva

\bvb There was Dreepen and Dollowthrasher, High, Highspurer, Leewong, Glower, Sherver, Werver, Showfind, Great-grandfather, Elf and Ing, Oakenshield, Feller and Frost, Finn and Ginner: That will ever be remembered, while the \inx{eld} lives\footnoteB{Two archaic formulae. The first literally “that will ever up above”, cf. \Hervarar\ TODO: “We two are cursed, brother, thy bane am I become! That will ever be remembered (\emph{þat mun æ uppi}, but both mss. \emph{þat mun enn uppi}), evil is the doom of the norns!”. The second is found in a runic inscription, U 323 (980–1015): “Ever will lie, while the eld lives (\textbf{meþ + altr + lifiʀ} \emph{með aldr lifir}), the hard-hammered bridge, broad, after a good man.”}, the tally of descendants, heaved to Lofer.\evb
\evg


\bvg {\small Creation of first men.}
\bva Unz þrír kvǫ́mu \hld \edtext{ór því liði}{\Afootnote{þussa brúðir “brides of thurses” (\emph{wo. doubt corrupt}) \Hauksbok}} &%nvl
\edtext{ǫflgir ok ástkir}{\Afootnote{ástkir ok ǫflgir \Hauksbok}} \hld æsir at húsi; &%nvl
fundu á landi \hld lítt męgandi &%nvl
Ask ok Emblu \hld ørlǫglausa.\eva

\bvb — Until three came out of that host: mighty and loving Ease along the houses; they found on land the little availing \inx{Ash} and \inx{Emble}, \inx{orlay}-less.\footnoteB{For, according to \Gylfaginning\ (TODO: reference), they were pieces of driftwood.}\evb
\evg


\bvg
\bva Ǫnd þau né ǫ́ttu, \hld óð þau né hǫfðu, &%nvl
lǫ́ né læti \hld né litu góða; &%nvl
ǫnd gaf Óðinn, \hld óð gaf Hǿnir, &%nvl
lǫ́ gaf Lóðurr \hld ok litu góða.\eva

\bvb Breath they owned not, \inx{wode} they had not, not craft nor sound, nor good complexion. Breath gave \inx{Weden}, wode gave \inx{Heen}, craft gave \inx{Lother}, and good complexion.\evb
\evg


\bvg {\small The ash of Ugdrassle and its three norns.}
\bva\ledleftnote{\Regius\Hauksbok\GylfMS}Ask vęit’k \edtext{standa}{\lemma{standa “[does] stand”}\Afootnote{\emph{thus} \Regius\Hauksbok\Upsaliensis; ausinn “[is] poured” \RegiusProse\Trajectinus\Wormianus}}, \hld hęitir \edtext{Yggdrasill}{\Afootnote{Yggdrasils \RegiusProse}}, &%nvl
hǫ́r \edtext{baðmr}{\lemma{baðmr “beam”}\Afootnote{borinn “born” (\emph{wo. doubt corrupt}) \Upsaliensis}}, \edtext{ausinn}{\lemma{ausinn “poured”}\Afootnote{hęilagr (\emph{norm.}) “holy” \GylfMS}} \hld hvíta auri; &%nvl
þaðan koma dǫggvar \hld \edtext{þær’s}{\Afootnote{er “which” \RegiusProse\Trajectinus}} í dala falla; &%nvl
\edtext{stęndr}{\Afootnote{\emph{add.} hann \RegiusProse\Trajectinus}} \edtext{æ}{\Afootnote{\emph{om.} \Upsaliensis}} yfir \edtext{grǿnn}{\Afootnote{grvnn \RegiusProse; grein \Upsaliensis}} \hld Urðar brunni.\eva

\bvb — I know an ash [does] stand, \inx{Ugdrassle} ’tis called: a high beam\footnoteB{Tree.}, poured with white mud\footnoteB{Compare perhaps with the Indian ritual pouring of beverages onto the \emph{lingam}. — For the whole passage compare 27.}. Thence come the dew-drops which in the dales fall; it stands ever green over the \inx{Well of Weird}.\evb
\evg


\bvg
\bva Þaðan koma męyjar \hld margs vitandi &%nvl
þríar ór þeim \edtext{sæ}{\Afootnote{sal “[out of that] hall” \Hauksbok}}, \hld es \edtext{und}{\Afootnote{á “on [the pine]” \Hauksbok}} þolli stendr; &%nvl
Urð hétu ęina, \hld aðra Verðandi, &%nvl
skǫ́ru á skíði, \hld Skuld hina þriðju &%nvl
þær lǫg lǫgðu, \hld þær líf køru, &%nvl
alda bǫrnum, \hld ørlǫg \edtext{sęggja}{\Afootnote{at segia “[the orlay] to say.” \Hauksbok}}.\eva

\bvb Thence come maidens, much knowing: three out of that lake, which stands beneath the pine\footnoteB{But here simply meaning ‘tree’; perhaps the same applies for “ash” earlier.}: Weird they called one, the other Werthing—they carved onto planks—Shild the third. Laws they laid; lives they chose: for the children of mortals, the \inx{orlay} of men.\evb
\evg


\bvg {\small The origin of the Wallow.}
\bva Þat man hón folkvíg \hld fyrst í hęimi, &
es Gollvęigu \hld gęirum studdu &
ok í hǫll Háars \hld hána bręnndu, &
\edtext{þrysvar bręnndu}{\Afootnote{\emph{repeated twice} \Hauksbok}} \hld þrysvar borna, &
opt ósjaldan, \hld þó hón ęnn lifir.\eva

\bvb — That troop-conflict she recalls\footnoteB{While appealing to read \emph{folk-víg} ‘folk-conflict’ as meaning ‘ethnic conflict’, thus describing the war between the Ease and Wanes, \emph{folk} almost certainly here carries its earlier meaning of ‘troop, group of warriors’.}, the first in the \inx{home}, as Goldwey with spears they goaded, and in the hall of \inx{Higher}\footnoteB{The hall of Weden; Walhall.} burned: thrice they burned the thrice born; often unseldom, though she yet lives.\footnoteB{Very cryptic. TODO: double check Snorri. Goldwey was apparently burned three times “often unseldom” (in short succession?) by the Ease, which yet did not kill her?}\evb
\evg


\bvg
\bva \edtext{Hęiði}{\Afootnote{\emph{metr. emend.} Héidi hána \Regius; Heiði hana \Hauksbok}} hétu, \hld hvar’s til húsa kom, &%nvl
\edtext{vǫlu}{\Afootnote{ok vǫlu \Hauksbok}} \edtext{velspáa}{\Afootnote{\emph{metr. emend.} uel spá \Regius; vel spa \Hauksbok}}, \hld vitti hón ganda; &%nvl
sęið \edtext{hvar’s kunni}{\Afootnote{hon kvnni \Regius; hon hvars hvn kunni \Hauksbok}}, \hld sęið \edtext{hug lęikinn}{\Afootnote{hon leikinn \Regius; hon hugleikin \Hauksbok}}; &%nvl
æ vas hón angan \hld illrar brúðar.\eva

\bvb Heath they called her, where to houses she came: a well-spaeing\footnoteB{Gifted at soothsaying.} \inx{wallow}, she bewitched \inx{gands}. She soth\footnoteB{Past tense of sithe (ON \emph{síða}) ‘to enchant, bewitch’.} where she could, she soth deluded minds; ever was she the love of the evil bride.\evb
\evg


\bvg {\small War between Ease and Wanes.}
\bva Þá gingu ręgin ǫll \hld á rǫkstóla, &%nvl
ginnhęilǫg goð, \hld ok gættusk umb þat: &%nvl
hvárt skyldi æsir \hld afráð gjalda, &%nvl
eða skyldi goð ǫll \hld gildi ęiga?\eva


\bvb Then went the Powers all onto the rake-seats: the gin-holy gods, and from each other took counsel about that: whether the Ease should tribute yield, or should the gods all a banquet hold?\evb
\evg

\bvg
\bva Flęygði Óðinn \hld ok í folk of skaut; &%nvl
þat vas ęnn folkvíg \hld fyrst í hęimi; &%nvl
brotinn vas borðvęggr \hld borgar ása, &%nvl
knǫ́ttu vanir vígspǫ́ \hld vǫllu sporna.\eva

\bvb Weden flung [a spear], and into the opposing army did shoot; that was yet the first folk-war\footnoteB{\emph{folk} probably in its earlier sense, ‘troop’, though reading it as ‘people, folk’ is attractive, since it would give \emph{folkvíg} the meaning ‘ethnic conflict’.} in the \inx{home}. Broken was the board-wall\footnoteB{Wall made of planks.} of the fortification of the Ease; the Wanes did by \inx{wigh-spae} tread the fields.\footnoteB{The Wanes used magic spells to defeat the Ease.}\evb
\evg


\bvg {\small Building of the wall by the ettin.}
\bva Þá gingu ręgin ǫll \hld á rǫkstóla, &%nvl
ginnhęilǫg goð, \hld ok gættusk umb þat: &%nvl
hvęrr hęfði lopt alt \hld lævi blandit &%nvl
eða ætt jǫtuns \hld Óðs męy gefna.\eva

\bvb Then went the Powers all onto the rake-seats: the gin-holy gods, and from each other took counsel about that: Who had the air all with treason blended, or, to the ettin’s \inx{aught}, given \inx{Wode}’s maiden\footnoteB{That is, promised Frie to the ettin NAME. TODO: relate with what Snorri writes about the building of the wall.}?\evb
\evg


\bvg {\small Thunder slays him.}
\bva Þórr ęinn þar vá \hld þrunginn móði, &%nvl
hann sjaldan sitr, \hld es slíkt of fregn; &%nvl
á gingusk ęiðar, \hld orð ok sǿri, &%nvl
mǫ́l ǫll męginlig, \hld es á meðal \edtext{fóru}{\Afootnote{voru “[between them] were.” \Hauksbok}}.\eva

\bvb Thunder alone fought there, pressed by wrath; he seldom sits, when of such he learns. Trampled were oaths, vows and sworn words; the mighty treaties all, which between them had gone.\evb
\evg


\bvg {\small Homedall’s hearing hidden beneath Ugdrassle.}
\bva Vęit hón Hęimdallar \hld hljóð of folgit &%nvl
und hęiðvǫnum \hld hęlgum baðmi; &%nvl
á sér hón ausask \hld aurgum forsi &%nvl
af veði Valfǫðrs. \hld Vituð ér ęnn eða hvat?\eva

\bvb — Knows she the hearing of Homedall hidden, ’neath a shady\footnoteB{\emph{hęiðvanr}, literally ‘clear-, bright-less’.}, hallowed beam\footnoteB{The tree must be Ugdrassle.}. On it she sees being poured a muddy torrent\footnoteB{Literally “on she sees being poured with a muddy torrent”, which should be the same mud as in v. 19. However, if ms. \emph{á} is read as \emph{ǫ́} ‘river’, it would mean “A river she sees being fed by a muddy waterfall, from ...”}, from the pledge of the \inx{Father of the Slain}; — Know ye yet, or what?\footnoteB{“Do ye (Weden) know enough now, or what?” — repeated in 28, 33, 34, 38, 40, 47, 60, 61.}”\evb
\evg


\bvg {\small Weden sought out the wallow. — The following two verses are written together as one in \Regius.}
\bva\ledleftnote{\Regius}Ęin sat hón úti, \hld þá’s hinn aldni kom &%nvl
yggjungr ása \hld ok í augu lęit; &%nvl
hvęrs fregnið mik? \hld hví fręistið mín?\eva

\bvb — Lone sat she outside, when the old one came: the Terrifier of the Ease\footnoteB{Weden.}, and into [her] eyes looked. “Why inquirest thou me? Why temptest thou me?\footnoteB{The Wallow speaks.}\evb
\evg

\bvg
\bva\ledleftnote{\Regius\GylfMS}Alt vęit’k, Óðinn, \hld hvar auga falt &%nvl
\edtext{í hinum mæra}{\Afootnote{\emph{thus} \Wormianus; þitt (\emph{with points marking as error}) i enom męra \Regius í þęim hinum meira (“id.”) (\emph{norm.}) \Trajectinus\Upsaliensis; vr þeim envm mæra “out of the renowned” \RegiusProse}} \hld Mímis brunni; &%nvl
drekkr mjǫð Mímir \hld morgin hvęrjan &%nvl
af \edtext{veði}{\lemma{veði “pledge”}\Afootnote{veiþi “hunting”}} Valfǫðrs. \hld Vituð ér ęnn eða hvat?\eva

\bvb I know it all, Weden; where thine eye thou hidst: in the renowned \inx{Well of Mime}, [there] drinks Mime mead every morning, from the pledge of the \inx{Father of the Slain}; — Know ye yet, or what?”\evb
\evg


\bvg
\bva Valði hęnni Hęrfǫðr \hld hringa ok męn; &%nvl
\edtext{féspjǫll spaklig}{\lemma{“wise wealth-spells”}\Bfootnote{By some authors (see Haukur Þorgeirsson 2020, p. 51 ff.) emended to \emph{fekk spjǫll spaklig} “he (= Weden) received wise tidings”}} \hld ok spáganda; &%nvl
sá hón vítt ok umb vítt \hld of verǫld hvęrja.\eva

\bvb Host-father chose for her, rings and necklaces, wise wealth-spells, and spae-gands\footnoteB{The meaning of a \emph{gand} not fully clear. In this verse perhaps staffs used in ritual?}; saw she widely and widely about, o’er every world.\evb
\evg


\bvg {\small The Walkirries.}
\bva Sá hón valkyrjur \hld vítt of komnar, &%nvl
gǫrvar at ríða \hld til goðþjóðar. &%nvl
\edtext{Skuld hęlt skildi, \hld ęn Skǫgul ǫnnur, &%nvl
Gunnr, Hildr, Gǫndul \hld ok Gęirskǫgul; &%nvl
nú eru talðar \hld nǫnnur Hęrjans, &%nvl
gǫrvar at ríða \hld grund valkyrjur.}{\lemma{Skuld ... valkyrjur}\Bfootnote{These four lines, especially from the out-of-place ending (\emph{nú eru talðar}), seem to be a latter insert from a thule listing walkirries.}}\eva

\bvb Saw she walkirries, widely come, ready to ride to \inx{Godthede}. Shild held a shield, and Shagle another; Guth, Hild, Gandle, and Goreshagle; now are tallied the women of Harn: ready to ride the ground, \inx{walkirries}.\evb
\evg


\bvg {\small The fate of Bolder.}
\bva Ek sá Baldri, \hld blóðgum tívi, &%nvl
Óðins barni, \hld ørlǫg folgin; &%nvl
stóð of vaxinn \hld vǫllum hæri &%nvl
mjór ok mjǫk fagr \hld mistiltęinn.\eva

\bvb — I saw Bolder’s, the bloody tue’s, the child of Weden’s, \inx{orlay} sealed\footnoteB{Notably, \emph{fela} ‘hide, conceal’ is used to describe burial in mounds, as in \Ynglingatal\ 24, Öl 1 (900s): “hidden (\textbf{fulkin} \emph{folginn}) in this mound lies he whom the greatest deeds followed...”}; grown did stand, higher than the fields, slender and greatly fair, the mistletoe.\footnoteB{Told allusively in the following three verses is the death of Bolder at the hands of his blind brother Hath. \Gylfaginning\ TODO}\evb
\evg


\bvg
\bva Varð af męiði, \hld þęim’s mær sýndisk, &%nvl
harmflaug hættlig, \hld Hǫðr nam skjóta. &%nvl
Baldrs bróðir vas \hld of borinn snimma, &%nvl
sá nam, Óðins sonr, \hld ęinnættr vega;\eva

\bvb Became of that beam, which meager seemed, a baneful harm-flier; Hath began to shoot. Bolder’s brother was born early; that one began, Weden’s son, one night old, to slay.\evb
\evg


\bvg
\bva þó hann æva hęndr \hld né hǫfuð kęmbði, &%nvl
áðr á bál of bar \hld Baldrs andskota. &%nvl
Ęn Frigg of grét \hld í Fęnsǫlum &%nvl
vǫ́ Valhallar. \hld Vituð ér ęnn eða hvat?\eva

\bvb Washed he never hands, nor head combed, before onto the pyre, he did bear Bolder’s opponent. But Frie did lament, in the Fenhalls, the woe of Walhall; — Know ye yet, or what?\evb
\evg


\bvg {\small The imprisoned Lock.}
\bva Hapt sá hón liggja \hld und Hveralundi &%nvl
lægjarns líki \hld Loka áþękkjan; &%nvl
Þa kná Váli \hld vígbǫnd snúa &
hęldr vǫ́ru harðgǫr \hld hǫpt ór þǫrmum &
þar sitr Sigyn \hld þęygi of sínum &%nvl
veri vęl glýjuð. \hld Vitud ér ęnn eða hvat?\eva

\bvb A captive she saw lying, beneath Wharlund: a bodily likeness of guileful Lock. Then did Woal the war-bonds turn; very were they sturdy, fetters made of intestines. There sits Sighyn, not at all cheerful, above her husband;\footnoteB{See \FraLoka.} — Know ye yet, or what?\evb
\evg


\bvg
\bva Ǫ́ fęllr austan \hld of ęitrdala &%nvl
sǫxum ok sverðum, \hld Slíðr heitir sú.\eva

\bvb A river falls from the east, above the venom-dales, with saxes and swords; Slide is that one called.\evb
\evg


\bvg {\small Two halls.}
\bva Stóð fyr norðan \hld á Niðavǫllum &%nvl
salr ór golli \hld Sindra ættar, &%nvl
ęn annarr stóð \hld á Ókólni, &%nvl
bjórsalr jǫtuns, \hld ęn sá Brimir hęitir.\eva

\bvb Stood to the north on the Nithewolds a hall out of gold, owned by the \inx{aught} of Sinder; but another one stood, on Uncoalen, the beer-hall of an ettin, and that one is called Brimmer.\evb
\evg


\bvg {\small The worst hall.}
\bva Sal sá hón standa \hld sólu fjarri &%nvl
Nástrǫndu á, \hld norðr horfa dyrr; &%nvl
falla ęitrdropar \hld inn um ljóra, &%nvl
sá ’s undinn salr \hld orma hryggjum.\eva

\bvb A hall saw she standing, far from the sun, on Neestrand, north face the doors; fall venom-drops in through the smoke-vent, that hall is wound by the spines of snakes.\evb
\evg


\bvg
\bva Sér hón þar vaða \hld þunga strauma &%nvl
męnn męinsvara \hld ok morðvarga &%nvl
ok þann’s annars glępr \hld ęyrarúnu. &%nvl
Þar sýgr Níðhǫggr \hld nái framgingna; &%nvl
slítr vargr vera. \hld Vituð ér ęnn eða hvat?\eva

\bvb There she sees wade, through heavy streams, oath-breaking men and murderwargs, and the one who confounds another’s understanding\footnoteB{Literally “who confounds another’s ear-rune,” probably referring to false counsellors.}. There sucks Nithehew from corpses passed-on; the warg tears men; — Know ye yet, or what?\evb
\evg


\bvg {\small The hag nourishes the destroyers in Ironwood.}
\bva\ledleftnote{\Regius\Hauksbok\GylfMS}Austr \edtext{býr}{\Afootnote{\emph{Thus} \Hauksbok\GylfMS\ sat “stayed [the old]” \Regius}} hin \edtext{aldna}{\Afootnote{arma “the wretched woman” \Upsaliensis}} \hld í \edtext{Járnviði}{\Afootnote{jarnuidiom “[in] Ironwoods” \Trajectinus}} &%nvl
ok \edtext{fǿðir}{\Afootnote{\emph{Thus} \Hauksbok\GylfMS; fǿddi “nourished” \Regius}} þar \hld Fęnris kindir; &%nvl
verðr \edtext{af}{\Afootnote{ór “out of [them] \Trajectinus\RegiusProse}} þeim ǫllum \hld ęinna nøkkurr &%nvl
tungls \edtext{tjúgari}{\lemma{tjúgari}\Afootnote{tuigan \Trajectinus\ \emph{wo. doubt corrupt}; tregari “griever [of the moon]” \Upsaliensis\ — As the young agentive suffix \emph{-ari} is found only here in the poem, it is possible that this word is corrupt. In that case, it must have occurred quite early in the transmission, as reflexes of \emph{*tiugari} are found in all surviving mss.}} \hld í trolls hami.\eva

\bvb In the east dwells the old woman, in \inx{Ironwood}, and nourishes there the kinds of \inx{Fenner}; from them all becomes one most particular: a seizer of the moon, in the \inx{hame} of a troll.\footnoteB{The old hag raises the offspring of the wolf Fenner, of which one will swallow the moon (and according to \Gylfaginning\ TODO the other the sun). See note to the next v.}\evb
\evg


\bvg
\bva\ledleftnote{\Regius\Hauksbok\GylfMS}Fyllisk fjǫrvi \hld fęigra manna, &%nvl
rýðr ragna sjǫt \hld rauðum dręyra, &%nvl
svǫrt var þá sólskin \hld umb sumur ęptir, &%nvl
veðr ǫll válynd. \hld Vituð ér ęnn eða hvat?\eva

\bvb He\footnoteB{The wolf.} fills himself with the life of \inx{fey} men; he reddens the abode of the \inx{Powers} with red gore. Black becomes the sunshine about the summers afterwards\footnoteB{After the sun is swallowed. But since the wallow does not tell us that this is a different wolf (it seems rather it be one and the same), it may reflect an earlier version of the myth, where one son of Fenner swallowed both the sun and moon. Yet, according to \Vafthrudnismal\ 36-37 it is Fenner himself who will swallow the sun (and thus likely the moon as well,) unless it there be taken as a general \inx{hote} for ‘wolf’ (which undoubtedly is its original meaning). TODO}; the storms all woeful; — Know ye yet, or what?\evb
\evg


\bvg {\small Edgethew struck harp; a fair-red cock crowed.}
\bva Sat þar á haugi \hld ok sló hǫrpu &%nvl
gýgjar hirðir, \hld glaðr Ęggþér; &%nvl
gól of hǫ́num \hld í gaglviði &%nvl
fagrrauðr hani, \hld sá’s Fjalarr hęitir.\eva

\bvb Sat there on the \inx{high} and struck the harp, the troll-woman’s keeper, glad \inx{Edgethew}. Above him crowed, in Gallowwood\footnoteB{Probably the same as Ironwood.}, a fair-red cock, that one who Feller is called.\evb
\evg


\bvg {\small A golden cock crowed in Osyard; a soot-red in Hell.}
\bva Gól of ǫ́sum \hld Gollinkambi, &%nvl
sá vękr hǫlða \hld at Hęrjafǫðrs, &%nvl
ęn annarr gęlr \hld fyr jǫrð neðan &%nvl
sótrauðr hani \hld at sǫlum Hęljar.\eva

\bvb Above the Ease crowed Goldencombe: he wakes men at the Father of Hosts’s [estate]; but another one crows below the earth: a soot-red cock, at the halls of Hell.\evb
\evg


\bvg
\bva Gęyr Garmr mjǫk \hld fyr Gnipahęlli, &%nvl
fęstr mun slitna, \hld ęn Freki rinna; &%nvl
fjǫlð vęit hón frǿða, \hld framm sé’k lęngra &%nvl
of ragna rǫk, \hld rǫmm sigtíva.\eva

\bvb Barks Garm loudly before the Gnip-caverns; the rope will tear, and Freck run. Much she knows of wisdom, forth I see yet further; about the rakes of the Powers: the mighty [fates] of the victory-tues.\evb
\evg


\bvg {\small Degeneration of man.}
\bva Brǿðr munu bęrjask \hld ok at bǫnum verða, &%nvl
munu systrungar \hld sifjum spilla, &%nvl
hart ’s í hęimi, \hld hórdómr mikill, &%nvl
skęggǫld, skalmǫld, \hld skildir ’ró klofnir, &%nvl
vindǫld, vargǫld, \hld áðr verǫld stęypisk, &%nvl
mun ęngi maðr \hld ǫðrum þyrma.\eva

\bvb Brothers will fight one another, and become slayers; sister’s sons will spill their kinship.\footnoteB{Whether through incest or treachery. TODO: literary evidence of the phrase \emph{spilla sifjum}.} ’Tis hard in the \inx{home}, great whoredom: halberd-eld, short-sword-eld; shields are split! Wind-eld, warg-eld; before the world\footnoteB{\emph{ver-ǫld} ‘world’ might perhaps be better translated as ‘man-eld’, ‘the eld of man’ with the other elds preceding it.} tumbles down, will no man another spare.\evb
\evg


\bvg {\small Prophesied events come to pass.}
\bva Lęika Míms synir, \hld ęn mjǫtuðr kyndisk &%nvl
at hinu galla \hld Gjallarhorni; &%nvl
hǫ́tt blæss Hęimdallr, \hld horn ’s á lopti; &%nvl
mælir Óðinn \hld við Míms hǫfuð.\eva

\bvb The sons of Mime play, and the Metted is kindled, at [the sounding of] the shrill Horn of Yell. Loud blows Homedall, the horn is aloft; speaks Weden with the head of Mime.\evb
\evg


\bvg
\bva \edtext{Skęlfr Yggdrasils \hld askr standandi, &
ymr it aldna tré, \hld ęn jǫtunn losnar;}{\lemma{Skęlfr ... losnar} \Afootnote{\emph{thus} \Hauksbok; \emph{In} \Regius\ \emph{the two long-lines are reversed.}}} &
\edtext{hræðask allir \hld á hęlvegum &
áðr Surtar þann \hld sefi of glęypir.}{\lemma{hræðask ... glęypir} \Afootnote{\emph{om.} \Regius.}} \eva

\bvb Quakes the ash of Ugdrassle, standing; groans the old tree, and the ettin loosens. All are frightened on the Hell-ways, before Surt’s kinsman \emph{it} does devour.\evb
\evg


\bvg
\bva Hvat ’s með ǫ́sum? \hld hvat ’s með ǫlfum? &
gnýr allr Jǫtunhęimr, \hld æsir ’ro á þingi, &
stynja dvergar \hld fyr stęindurum &
vęggbergs vísir — \hld vituð ér ęnn eða hvat?\footnoteB{In \Regius\ this v. follows v. 50 (Kjóll fęrr austan ...)}\eva

\bvb — What is with Ease? What is with Elves? Roars all Ettinham, Ease are at the Thing. Dwarves groan before gates of stone, the princes of the mountain-wall; — Know ye yet, or what?\evb
\evg


\bvg
\bva Gęyr nú Garmr mjǫk \hld fyr Gnipahęlli, &%nvl
fęstr mun slitna, \hld ęn Freki rinna; &%nvl
fjǫlð vęit hón frǿða, \hld framm sé’k lęngra &%nvl
of ragna rǫk, \hld rǫmm sigtíva.\eva

\bvb Barks now Garm loudly before the Gnip-caverns; the rope will tear, and Freck run. Much she knows of wisdom, forth I see yet further; about the rakes of the Powers: the mighty [fates] of the victory-tues.\evb
\evg


\bvg {\small The enemies of the gods assemble.}
\bva Hrymr ękr austan, \hld hęfsk lind fyrir, &%nvl
snýsk Jǫrmungandr \hld í jǫtunmóði; &%nvl
ormr knýr unnir, \hld ęn ari hlakkar, &%nvl
slítr nái neffǫlr; \hld Naglfar losnar.\eva

\bvb Rim drives from the east, holding the shield before himself. Ermingand writhes about himself in ettin-wrath: the worm propels the waves, but the eagle screams: the pale-beak tears corpses; Nailfare loosens.\evb
\evg


\bvg
\bva Kjóll fęrr austan \hld koma munu Múspells &%nvl
of lǫg lýðir, \hld ęn Loki stýrir; &%nvl
fara fíflmęgir \hld með Freka allir, &%nvl
þęim es bróðir \hld Býlęists í fǫr.\eva

\bvb A keel travels from the east—come will Muspell’s subjects by sea—but Lock steers it. Travel the warlocks all with Freck; with them fares the brother of Bylest along.\evb
\evg


\bvg {\small Surt comes; the final battle begins.}
\bva\ledleftnote{\Regius\Hauksbok\GylfMS}\edtext{Surtr}{\Afootnote{Svartr \Upsaliensis}} fęrr sunnan \hld með sviga lævi, &%nvl
skínn af sverði \hld sól valtíva; &%nvl
grjótbjǫrg gnata, \hld ęn \edtext{gífr rata}{\Afootnote{guðar hrata “[but] the gods stagger” (\emph{wo. doubt corrupt, young masc. pl. is proof enough.}) \Upsaliensis}}, &%nvl
troða halir hęlveg, \hld ęn himinn klofnar.\eva

\bvb Surt comes from the south, with the switch-bane\footnoteB{According to \CV\ ‘fire’.}; from the sword shines the sun of the slain-tues; boulders clash, but the fiends reel; men march on the \inx{Hell-ways}, but heaven is sundered.\evb
\evg


\bvg {\small Weden falls to the Wolf and Free to Surt.}
\bva Þá kømr Hlínar \hld harmr annarr framm, &%nvl
es Óðinn fęrr \hld við ulf vega, &%nvl
ęn bani Bęlja \hld bjartr at Surti; &%nvl
þá mun Friggjar \hld falla \edtext{angan}{\Afootnote{\emph{thus} \Hauksbok\ angantyr \Regius}}.\eva

\bvb Then comes \inx{Line}’s second sorrow to pass, as Weden goes to strike against the wolf; but the bane of \inx{Bellow}\footnoteB{\inx{Free}.}, bright, [goes] against Surt; then will Frie’s beloved\footnoteB{Weden, her husband.} fall.\evb
\evg


\bvg {\small Wither avenges Weden and slays the Wolf.}
\bva \edtext{Þá kømr hinn mikli \hld mǫgr Sigfǫður}{\lemma{Þá kømr ... Sigfǫður}\Afootnote{Gęngr Óðins sonr / við ulf vega \Gylfaginning}}, &%nvl
Víðarr \edtext{vega}{\Afootnote{of veg \Gylfaginning}} \hld at valdýri; &%nvl
lætr hann męgi Hveðrungs \hld mund of standa &%nvl
hjǫr til hjarta; \hld þá ’s hefnt fǫður.\eva

\bvb Then comes the great lad of \inx{Sighfather}, Wither, to strike at the murderous beast; he lets his hand plunge the sword into the heart of \inx{Whethring}’s lad\footnoteB{The son of Lock; the wolf.}; then is the father avenged.\evb
\evg


\bvg {\small Thunder and the Worm kill each other.}
\bva \edtext{Þá kømr}{\Afootnote{Gęngr \Gylfaginning}} hinn mæri \hld mǫgr Hlǫðynjar &%nvl
\edtext{gęngr Óðins sonr \hld ormi mǿta.}{\lemma{gęngr ... mǿta}\Afootnote{\emph{om.} \Gylfaginning}} &%nvl
\edtext{Drepr af móði \hld Miðgarðs véurr; &
munu halir allir \hld hęimstǫð ryðja; &
gęngr fet níu \hld Fjǫrgynjar burr &
nęppr frá naðri, \hld níðs ókvíðinn.}{\lemma{Drepr ... ókviðinn} \Afootnote{neppr af naðri / niðs ókvíðnum / munu halir allir / heimstǫð ryðja, / er af móði drepr / Miðgarðs véurr “[Goes the renowned lad of Lathyn,] pained, away from the loathsome adder. All men will empty their homesteads, when Middenyard’s wigh-ward strikes out of wrath.” \GylfMS}}\eva

\bvb Then comes the renowned lad of Lathyn: the son of Weden goes the \inx{worm} to meet. Middenyard’s wigh-ward strikes out of wrath; all men will their homesteads empty.\footnoteB{It seems likely that the order found in \Gylfaginning\ is original. After Thunder dies, farming becomes impossible, and thus they must leave their homes.} The son of Firgyn goes nine paces, pained, away from the loathsome adder.\footnoteB{Thunder, mortally wounded, struggles nine steps away from the serpent before he falls. TODO: Snorri’s account}\evb
\evg


\bvg {\small Culmination.}
\bva Sól tér sortna, \hld sígr fold í mar, &%nvl
hverfa af himni \hld hęiðar stjǫrnur; &%nvl
gęisar ęimi \hld við aldrnara; &%nvl
lęikr hǫ́r hiti \hld við himin sjalfan.\eva

\bvb The sun does blacken, descends the fold into the sea; disappear from heaven the clear stars. Rages smoke from the nourisher of life\footnoteB{Fire.}; licks the high heat heaven itself.\evb
\evg


\bvg
\bva Gęyr nú Garmr mjǫk \hld fyr Gnipahęlli, &%nvl
fęstr mun slitna, \hld ęn Freki rinna; &%nvl
fjǫlð vęit hón frǿða, \hld framm sé’k lęngra &%nvl
of ragna rǫk, \hld rǫmm sigtíva.\eva

\bvb Barks now Garm loudly before the Gnip-caverns; the rope will tear, and Freck run. Much she knows of wisdom, forth I see yet further; about the rakes of the Powers: the mighty [fates] of the victory-tues.\evb
\evg


\bvg {\small The world is reborn.}
\bva Sér hón upp koma \hld ǫðru sinni &%nvl
jǫrð ór ægi \hld iðjagrǿna —; &%nvl
falla forsar, \hld flýgr ǫrn yfir, &%nvl
sá’s á fjalli \hld fiska vęiðir.\eva

\bvb Sees she come up, a second time: the earth out of the sea, ever green anew. Torrents fall, flies an eagle above, the one who on the fells fish does catch.\evb
\evg


\bvg
\bva Finnask æsir \hld á Iðavęlli &%nvl
ok of moldþinur \hld mǫ́tkan dǿma, &%nvl
ok minnask þar \hld á męgindóma &%nvl
ok á Fimbultýs \hld fornar rúnar.\eva

\bvb The Ease find each other on the Idewolds, and about the mighty earth-strip\footnoteB{The Middenyardsworm.} converse, and remember there mighty judgements, and Fimbletue’s ancient runes.\evb
\evg

\bvg {\small A new golden age.}
\bva Þar munu ęptir \hld undrsamligar &%nvl
gollnar tǫflur \hld í grasi finnask, &%nvl
þær’s í árdaga \hld áttar hǫfðu.\eva

\bvb There will afterwards wondrous golden Tavel-bricks in the grass be found: those which in days of yore they had owned.\footnoteB{Cf. v. 9. The rediscovering of the golden game pieces symbolizes a new golden age.}\evb
\evg


\bvg
\bva Munu ósánir \hld akrar vaxa; &%nvl
bǫls mun alls batna \hld mun Baldr koma; &%nvl
búa Hǫðr ok Baldr \hld Hropts sigtoptir &%nvl
(vęl valtívar, \hld Vituð ér ęnn eða hvat?)\eva

\bvb Will unsown fields grow: evil will all be bettered: Bolder will come. Hath and Bolder dwell in the building-plots of Roft: happily, the slain-Tues; — Know ye yet, or what?\evb
\evg


\bvg
\bva Þá kná Hǿnir \hld hlautvið kjósa &%nvl
ok burir byggva \hld brǿðra Tvęggja &%nvl
vindhęim víðan. \hld Vituð ér ęnn eða hvat?\eva

\bvb Then does Heen choose the \inx{leat}-wood, and the sons of the brothers of Tway build in the wide wind-home\footnoteB{TODO. What does Snorri write about the wind-home?}; — Know ye yet, or what?\evb
\evg


\bvg
\bva Sal sér hón standa \hld sólu fęgra, &%nvl
golli þakðan, \hld á \edtext{Gimléi}{\Afootnote{\emph{metr. emend.} Gimlé \Regius; Gimle \Hauksbok}}; &%nvl
þar munu dyggvar \hld dróttir byggva &%nvl
ok um aldrdaga \hld ynðis njóta.\eva

\bvb A hall she sees standing, fairer than the sun, thatched with gold, on Gemlee; there the dutiful \inx{drights} will dwell, and in their days of life delights enjoy.\evb
\evg


\bvg {\small The dragon still lives; the wallow descends.}
\bva Þar kømr hinn dimmi \hld dręki fljúgandi, &%nvl
naðr fránn neðan \hld frá Niðafjǫllum; &%nvl
berr sér í fjǫðrum \hld — flýgr vǫll yfir — &%nvl
Níðhǫggr nái; \hld nú mun hón søkkvask.\eva

\bvb — Then comes the shadowy dragon flying; the gleaming serpent down below from the \inx{Nithfells}. Nithehew bears in his feathers—flying over the field—corpses.” — Now she will sink!\footnoteB{The wallow, referring to herself in third person, descends back down into her grave, whence Weden woke her.}\evb
\evg
