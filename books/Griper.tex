\bookStart{Spae of Griper}[Grípisspǫ́]
\def\thisBookCode{Gripisspa}

\begin{flushright}%
\textbf{Dating} \parencite{Sapp2022}: early C11th (0.616)–late C11th (0.313).

\textbf{Meter:} \Fornyrdislag%
\end{flushright}

\section{Introduction}

TODO: Introduction.

This poem is very regular and well preserved; every single one of its 53 \Fornyrdislag\ stanzas is four lines long.

\section{From the Death of Sinfittle (\emph{Frá dauða Sinfjǫtla})}

\sectionline

\bpg\bpa Sigmundr Vǫlsungs sonr var konungr á Frakklandi. Sinfjǫtli var elztr hans sona, annarr Helgi, þriði Hámundr. Borghildr, kona Sigmundar, átti bróður er hét... en Sinfjǫtli, stjúp-sonr hennar, ok... báðu einnar konu báðir ok fyr þá sǫk drap Sinfjǫtli hann. En er hann kom heim þá bað Borghildr hann fara á brot en Sigmundr bauð henni fé-bǿtr ok þat varð hón at þiggja. En at erfi’nu bar Borghildr ǫl. Hon tók eitr mikit, horn fullt, ok bar Sinfjǫtla.  En er hann sá í horn’it skilði hann at eitr var í ok mę́lti til Sigmundar: „Gjǫr-óttr er drykkr’inn, ái!“  Sigmundr tók horn’it ok drakk af.  Svá er sagt at Sigmundr var harð-gǫrr at hvárki mátti hánum eitr granda útan né innan.  En allir synir hans stóðusk eitr á hǫrund útan.  Borghildr bar annat horn Sinfjǫtla ok bað drekka ok fór allt sem fyrr.  Ok enn it þriðja sinn bar hon hánum horn’it ok þó á-mę́lis-orð með ef hann drykki eigi af.  Hann mę́lti enn sem fyrr við Sigmund; hann sagði: „Láttu grǫn sía þá, sonr!“  Sinfjǫtli drakk ok varð þegar dauðr.  Sigmundr bar hann langar leiðir í fangi sér ok kom at firði einum mjóvum ok lǫngum ok var þar skip eitt lítit ok maðr einn á.  Hann bauð Sigmundi far of fjǫrð’inn.  En er Sigmundr bar lík’it út á skip’it þá var bátr’inn hlaðinn.  Karl mę́lti at Sigmundr skyldi fara fyr inn á fjǫrð’inn.  Karl hratt út skip’inu ok hvarf þegar.  Sigmundr konungr dvalðisk lengi í Danmǫrk í ríki Borghildar síðan er hann fekk hennar.  Fór Sigmundr þá suðr í Frakkland til þess ríkis er hann átti þar.  Þá fekk hann Hjǫrdísar, dóttur Eylima konungs.  Þeira sonr var Sigurðr.  Sigmundr konungr fell í orrustu fyr Hundings sonum.  En Hjǫrdís giptisk þá Álfi, syni Hjálpreks konungs.  Óx Sigurðr þar upp í barn-ǿsku.  Sigmundr ok allir synir hans vóru langt um fram alla menn aðra um afl ok vǫxt ok hug ok alla at-gørvi.  Sigurðr var þá allra framarstr ok hann kalla allir menn í forn-frǿðum um alla menn fram ok gǫfgastan her-konunga.\epa

\bpb TODO.\epb\epg


\bpg\bpa Grípir hét sonr Ęylima, bróðir Hjǫrdísar.  Hann réð lǫndum ok vas allra manna vitrastr ok fram-víss.  Sigurðr ręið ęinn saman ok kom til hallar Grípis.  Sigurðr vas auð-kęnndr.  Hann hitti mann at máli úti fyr hǫll’inni; sá nęfndisk Gęitir.  Þá kvaddi Sigurðr hann máls, ok spyrr:\epa

\bpb Griper was called the son of Ilime, Hardise’s brother.  He ruled lands and was wisest of all men, and forthwise.  Siward rode alone and came to Griper’s hall.  Siward was easily recognized.  He approached a man for speech outside of the hall; he was named Goater.  Then Siward greeted him with a speech, and asks:\epb\epg

\section{The Spae of Griper}

\bvg\bva „Hvęrr \alst{b}yggir hér \hld\ \alst{b}orgir þessar? &
Hvat þann \alst{þ}jóð-konung \hld\ \alst{þ}egnar nefna?“ &
„\alst{G}rípir hęitir \hld\ \alst{g}umna stjóri, &
sá’s \alst{f}astri rę́ðr \hld\ \alst{f}oldu ok þegnum.“\eva

\bvb “Who bedwells here these forts? \\
What is this great king called by thanes?” \\
“Griper is called the steerer of men \\
who rules the steadfast land and thanes.”\evb\evg


\bvg\bva \alst{M}ę́la nǫ́mu \hld\ ok \alst{m}argt hjala &
þá’s \alst{r}áð-spakir \hld\ \alst{r}ekkar fundusk. &
„Sęg-ðu \alst{m}ér ef þú vęizt, \hld\ \alst{m}óður-bróðir, &
hvé mun \alst{S}igurði \hld\ \alst{s}núna ę́vi?“\eva

\bvb They took to speak and chatter much, \\
when the council-wise champions found each other. \\
“Tell me, if thou knowest, O mother’s brother: \\
how will Siward’s age turn out?”\evb\evg


\bvg\bva „Þú \alst{m}unt \alst{m}aðr vesa \hld\ \alst{m}ę́ztr und sólu &
ok \alst{h}ę́str borinn \hld\ \alst{h}vęrjum jǫfri; &
\alst{g}jǫfull af \alst{g}ulli \hld\ en \alst{g}løggr flugar, &
\alst{í}tr á-liti \hld\ ok í \alst{o}rðum spakr.“\eva

\bvb „Thou wilt be a man noblest neath the sun, \\
and borne higher than every ruler, \\
giving with gold but stingy of flight, \\
radiant of hue and wise in words.“\evb\evg

TODO.

\bvg\bva Es-a með \alst{l}ǫstum \hld\ \alst{l}ǫgð ę́vi þér; &
lát-tu, inn \alst{í}tri, \hld\ þat, \alst{ǫ}ðlingr, nemask &
því at \alst{u}ppi mun \hld\ meðan \alst{ǫ}ld lifir, &
\alst{n}add-éls boði, \hld\ \alst{n}afn þitt vera.\eva

\bvb TODO. \\
For remembered will while mankind lives, \\
O beseecher of the sword-storm \ken{battle > warrior}, thy name be.\evb\evg

TODO.

\bvg\bva Þú munt \alst{h}víla, \hld\ \alst{h}ęrs odd-viti, &
\alst{m}ę́rr hjá \alst{m}ęyju \hld\ sem þín \alst{m}óðir sé; &
því mun \alst{u}ppi \hld\ meðan \alst{ǫ}ld lifir, &
\alst{þ}jóðar \alst{þ}ęngill, \hld\ \alst{þ}itt nafn vera.\eva

\bvb Thou wilt rest, O point-knower of the host \ken{warrior}, \\
renowned beside a maiden like she were thy mother. \\
For that will remembered while mankind lives, \\
O prince of the nation, thy name be.\evb\evg

TODO.

\bvg\bva Því skal \alst{h}ugga þik, \hld\ \alst{h}ęrs odd-viti, &
sú mun \alst{g}ipt lagit \hld\ á \alst{g}rams ę́vi; &
mun-at \alst{m}ę́tri \alst{m}aðr \hld\ á \alst{m}old koma &
und \alst{s}ólar \alst{s}jǫt \hld\ an, \alst{S}igurðr, þikkir.\eva

\bvb For that [she] shall soothe thee, O point-knower of the host; \\%TODO: "soothe"??
she will have laid venom in the ruler’s age. \\
No nobler man will come onto the earth \\
neath the sun’s seat \ken{sky/heaven}, than thou, Siward, seemest!\evb\evg


\bvg\bva \alst{Sk}iljumk hęilir; \hld\ mun-at \alst{sk}ǫpum vinna! &
Nú hęfir þú, Grípir, vęl \hld\ gørt sem bęiddak; &
fljótt myndir þú \hld\ fríðri sęgja &
mína ę́vi \hld\ ef þú mę́ttir þat!\eva

\bvb Let us part healthy; one will not withstand the \inx[C]{shape}[shapes]! \\
Now hast thou, Griper, well done as I asked; \\
shortly wouldst thou fairer speak \\
of my age, if thou couldst do that!\evb\evg

\sectionline
