\bookStart{Old Saxon Genesis}

\begin{flushright}%
\textbf{Dating:} C9th

\textbf{Meter:} \Fornyrdislag%para
\end{flushright}%

\section{Introduction}

The normalization follows that adapted for \Heliand.  There is only one ms., Palatinus latinus 1447 (\textbf{V}, https://digi.vatlib.it/view/MSS\_Pal.lat.1447/0005), where the poem is found written on a few fragmentary pages between Latin theological texts.  In this ms. a small fragment of the \Heliand\ is also found.

Much of the poem, including parts not extant in \textbf{V}, was closely translated into Old English and later inserted into an English poem on Genesis.  The translation is called \emph{Genesis B}, and will be edited below.  Lines 1–26 of the present poem correspond almost exactly with lines 791–817 of that poem.

\sectionline

\section{After the Fall}

\bvg\bva\mssnote{\textbf{V} 1r/TODO}
„Wela, þat þú nú, \alst{É}wa, havas,“ kwad Adam, „\alst{u}vilo gi·marạkot &
unkaro \alst{s}elvaro \alst{s}ïd. \hld\ Nú maht þú \edtext{sehan}{\Afootnote{\emph{sean} \textbf{V}}} þia \alst{s}warton hęll &
\alst{g}inon \alst{g}rádaga; \hld\ nú þú sia \alst{g}rimman maht &
\alst{h}inana gi·\alst{h}ôrjan, \hld\ nis \alst{h}evan-ríki &
ge·\alst{l}íhk sulíkaro \alst{l}ógnun: \hld\ þit was alloro \alst{l}ando skônjust, &
þat wit hier þuruh unkas \alst{h}êrran þank \hld\ \alst{h}ębbjan muostun &
þár þú þem ni \alst{h}ôrdis \hld\ þie unk þesan \alst{h}arạm gi·ried, &
þat wit \alst{w}aldandas \hld\ \alst{w}ord far·brákun, &
\alst{h}evan-kuningas. \hld\ Nú wit \alst{h}riwig mugon &
\alst{s}orogon for þem \alst{s}ïda, \hld\ wand hé \edtext{unk}{\Afootnote{\emph{hunk} \textbf{V}}} \alst{s}elvo gi·bôd, &
þat \alst{w}it \edtext{unk}{\Afootnote{\emph{hunk} \textbf{V}}} su-lik \alst{w}íti \hld\ \alst{w}ardon skoldin, &
\alst{h}arạmo mêstan— \hld\ nú þwingit mí giu \alst{h}ungar endi þrust, &
\alst{b}itter \alst{b}alo-werẹk, \hld\ þero wáron wit êr \alst{b}êdero tuom. &
Hú skulun wit nu \alst{l}ibbjan, \hld\ efto hú skulun wit an þesum \alst{l}iahta wesan, &
nu hier hwílum \alst{w}ind kumit \hld\ \alst{w}estan efto ôstan, &
\alst{s}u̇ðan efto nordan? \hld\ gi·\alst{s}werẹk upp drívit, &
kumit \alst{h}aglas skion \hld\ \alst{h}imile bi·tengi, &
\alst{f}ęrid \alst{f}ord an gi·mang \hld\ (þat is \alst{f}irinum kald): &
\alst{h}wílum þanne fan \alst{h}imile \hld\ \alst{h}êto skínit, &
\alst{b}líkit þiu \alst{b}erahto sunna: \hld\ wit hier þus \alst{b}ara standat, &
un·\alst{w}ęrid mid gi·\alst{w}ádi: \hld\ nis unk hier \alst{w}iht bi·foran &
ni te \alst{sk}adowa ni te \alst{sk}úra, \hld\ unk nis hier \alst{sk}attas wiht &
te \alst{m}ęti gi·\alst{m}arkot: \hld\ wit hębbjat unk gi·duan \alst{m}ahtigna god, &
\alst{w}aldand \alst{w}rêdan. \hld\ Te hwí skulun wit \alst{w}erdan nu? &
Nu mag mí þat \alst{h}reuwan, \hld\ þat ik is io bad \alst{h}evan-ríkjan god, &
\edtext{\alst{w}aldand þ{[...]}}{\Bfootnote{The bottom part of \textbf{V} 1r has been trimmed, resulting in the loss of a few lines.  For the continuation cf. \emph{Genesis B} 817 ff., which translates this and the following lines.}}\eva

\bvb TODO.\evb\evg

\sectionline

\section{After Cain’s slaying of Abel}

\bvg\bva\mssnote{\textbf{V} 2v/TODO}
\alst{S}ïdoda im þuȯ te \alst{s}ęlidon, \hld\ habda im \alst{s}undja gi·warạht &
\alst{b}ittra an is \alst{b}ruodar; \hld\ liet ina undar \alst{b}aka liggjan &
an ênam \alst{d}iapun \alst{d}ala \hld\ \alst{d}rôr-wóragana, &
\alst{l}íbas \alst{l}ôsan, \hld\ \alst{l}egar-bedd waran, &
\alst{g}uman an \alst{g}riata. \hld\ Þuȯ sprak im \alst{g}od selbo tuo, &
\alst{w}aldand mid is \alst{w}ordun \hld\ (was im \alst{w}rêd an is hugi, &
þem \alst{b}anan gi·\alst{b}olgan), \hld\ frágoda hwar he habdi is \alst{b}ródar þuȯ &
\alst{k}ind-jungan guman. \hld\ Þó sprak im eft \alst{K}ain an·gęgen &
-habda im mid is \alst{h}andun \hld\ \alst{h}arạm-werẹk mikil &
\alst{w}am-dádjun gi·\alst{w}arạht, \hld\ þius \alst{w}erold was só swído &
be·\alst{s}mitin an \alst{s}undjun-: \hld\ „Ni ik þes \alst{s}orọgun ni skal,“ kwad he, &
„\alst{g}ômjan hwar hie \alst{g}anga, \hld\ ni it mi \alst{g}od ni gi·bôd, &
þat is \alst{h}węrigin \alst{h}ier \hld\ \alst{h}uodjan þorọfti, &
\alst{w}ardon an þesaro \alst{w}eroldi.“ \hld\ \alst{W}ánde he swído, &
þat he bi·\alst{h}elan mahti \hld\ \alst{h}êrran sínum, &
þia \alst{d}ádi bi·\alst{d}ęrnjan. \hld\ Þuȯ sprak im eft u̇sa \alst{d}rohtin tuo: &
„All habas þu só gi·\alst{w}erẹkot,“ kwad he, \hld\ „só þí ti þínaro \alst{w}er-oldi mag &
wesan þín \alst{h}ugi \alst{h}riuwig, \hld\ þes þu mid þínum \alst{h}andon gi·dedos, &
þat þú wurdi þínes \alst{b}ruodar \alst{b}ano: \hld\ nu he \alst{b}luodig ligit, &
\alst{w}undun \alst{w}órig; \hld\ þes ni habda he êniga ge·\alst{w}urụhte te þi, &
\alst{s}undja gi·\alst{s}uohta, \hld\ þoh þu ina nu a·\alst{s}lagan hębbjas, &
\alst{d}ôdan gi·\alst{d}uanan. \hld\ Is \alst{d}rôr sinkit nu an erda, &
\alst{s}wêt \alst{s}undar ligit; \hld\ þiu \alst{s}eola hwarọbat &
þie \alst{g}êst \alst{g}jámar-muod \hld\ an \alst{g}odas willjan; &
\alst{d}rôr hruopit is te \alst{d}rohtina selbun \hld\ endi sagat hwe þea \alst{d}ádi frumida, &
þat \alst{m}ên an þesun \alst{m}iddil-gardun: \hld\ ni mag im ênig \alst{m}ann þan swídor &
\alst{w}ero far·\alst{w}irịkjan \hld\ an \alst{w}erold-ríkja &
an \alst{b}ittron \alst{b}alo-dádjon, \hld\ þan þú an þínum \alst{b}ruodar habas &
\alst{f}irin-werẹk gi·\alst{f}ręmid.“ \hld\ Þuȯ an \alst{f}orạhtun ward &
\alst{K}ain aftar þem \alst{k}widjun drohtinas, \hld\ \alst{k}wad þat hie wisse garwo, &
þat is ni mahti werdan \alst{w}aldand \alst{w}iht, \hld\ an \alst{w}erold-stundu &
\alst{d}ádjo bi·\alst{d}ęrnid, \hld\ „só ik is nu mag \alst{d}rubundjan hugi,“ kwad he, &
„\alst{b}eran an mínun \alst{b}reostun \hld\ þes ik mínan \alst{b}ruodar sluog &
þuru mín \alst{h}and-męgin. \hld\ Nu wêt ik, þat ik skal an þínum \alst{h}ęti libbjan, &
\alst{f}ord an þínum \alst{f}íund-skępi, \hld\ nu ik mí þesa \alst{f}irina gi·deda, &
\alst{s}ó mí mína \alst{s}undja nu \hld\ \alst{s}wídaron þunkjat, &
\alst{m}is-dád \alst{m}êra, \hld\ þan þín \alst{m}ildi hugi, &
só ik þes nu \alst{w}irdig ni bium, \hld\ \alst{w}aldand þie guodo, &
þat þú mí a·\alst{l}átas \hld\ \alst{l}êdas þingas, &
\alst{t}ianono a·\alst{t}uemjas. \hld\ Nu ik ni welda mína \alst{t}riuwa haldan, &
\alst{h}ugi wid þem þínum \alst{h}lutron muoda, \hld\ nu wêt ik, þat ik hier ni mag êniga \alst{h}wíla libbjan, &
hwand mí ant·\alst{w}irikit, \hld\ só hwat só mi an þisun \alst{w}ega findit, &
a·\alst{s}lęhit mi bi þesun \alst{s}undjun.“ \hld\ Þuȯ sprak im eft \alst{s}elbo an·gegin &
\alst{h}evanes waldand: \hld\ „\alst{H}ier skalt þu noh nu“, kwad he, &
„\alst{l}ibbjan \alst{l}ango hwíla. \hld\ Þo þu sus a·\alst{l}êdit sís, &
mid \alst{f}irinum bi·\alst{f}angan, \hld\ þoh will ik þi \alst{f}riðu sęttjan, &
\alst{t}ôgjan su-lik \alst{t}êkjan, \hld\ só þu an \alst{t}reuwa maht &
\alst{w}esan an þesero \alst{w}erolde, \hld\ þoh þu is \alst{w}irdik ni sís: &
\alst{f}luhtik skalt þu þoh endi \alst{f}rêdig \hld\ \alst{f}ord-wardas nu &
\alst{l}ibbjan an þesum \alst{l}anda, \hld\ só lango só þu þit \alst{l}iaht waros; &
for·\alst{h}wátan skulun þi \alst{h}luttra liudi, \hld\ þu ni salt io furður kuman te þínes \alst{h}êrron spráko, &
\alst{w}esljan þár mid \alst{w}ordon þínon: \hld\ \alst{w}aldandi stêt &
þínes \alst{b}ródor wráka \hld\ \alst{b}ittẹr an hęlli.“\eva

\bvb TODO.\evb\evg


\bvg\bva[][54]\mssnote{\textbf{V} 2v/TODO}%
\edtext{Þó}{\Afootnote{Introduced with large initial.}} géng im þanan mid \alst{g}rimmo hugi, \hld\ habda ina \alst{g}od selbo &
\alst{s}wído far·\alst{s}akanan. \hld\ \alst{S}orọga warð þár þuȯ gi·ku̇dit &
\alst{A}dama endi \alst{É}wun, \hld\ \alst{i}n-widd mikil, &
iro \alst{k}indes \alst{k}walm, \hld\ þat he ni muosta \alst{k}wik libbjan. &
Þes ward \alst{A}damas hugi \hld\ \alst{i}nnan breostun &
\alst{s}wído an \alst{s}orogun, \hld\ þuȯ he wissa is \alst{s}unu dôdan: &
só ward is ôk þiu \alst{m}uodar, \hld\ þe þana \alst{m}agu fuodda, &
\alst{b}arn bi iro \alst{b}reostun. \hld\ Þuȯ siu \alst{b}luodag wuosk &
\alst{h}rêu-gi·wádi, \hld\ þuȯ ward iro \alst{h}ugi sêrag. &
\alst{B}êþo was im þó an sorogun \hld\ iak iro \alst{b}arnas dôd, &
þes \alst{h}ęliðas \alst{h}in-fard, \hld\ iak þat im mid is \alst{h}andun for·dæda &
\alst{K}ain an su-likun \alst{k}walma: \hld\ siu ni habdun þuȯ noh \alst{k}indo þan mêr &
\alst{l}ibbendero an þem \alst{l}iahta, \hld\ botan þana ênna, þie þuȯ a·\alst{l}êdit was &
\alst{w}aldanda be is far·\alst{w}urọhtjun: \hld\ þár ni habdun siu êniga \alst{w}unja tuo &
\alst{n}iud-líko gi·\alst{n}uman, \hld\ wand hie su-likan \alst{n}íd a·huof, &
þat he ward is \alst{b}ruodar \alst{b}ano. \hld\ Þes im þuȯ \alst{b}êðjun ward, &
\alst{s}in-híun twêm \hld\ \alst{s}êr umbi herta. &
Oft siu þes \alst{g}ornunde \hld\ an \alst{g}riata gi·stuodun, &
\alst{s}in-híun \alst{s}amad, \hld\ kwádun, þat sia wissin, þat im þat iro \alst{s}undja gi·dedin, &
þat im ni muostin \alst{a}ftar \hld\ \alst{ę}rẹbi-wardos &
\alst{þ}egnas \alst{þ}ían. \hld\ \alst{Þ}olodun siu bêðju &
\alst{m}ikila \alst{m}ord-kwála, \hld\ unt þat im eft \alst{m}ahtig god, &
\alst{h}êr \alst{h}evanes ward \hld\ iro \alst{h}ugi buotta, &
þat im wurðun \alst{ô}dana \hld\ \alst{ę}rẹbi-wardos, &
\alst{þ}egnos endi \alst{þ}iornun, \hld\ \alst{þ}igun aftar wel, &
\alst{w}óhsun \alst{w}án-líko, \hld\ ge·\alst{w}itt línodun, &
\alst{sp}áha \alst{sp}ráka. \hld\ \alst{Sp}uodda þie mahta &
is \alst{h}and-gi·werẹk, \hld\ \alst{h}êlag \edtext{drohtin}{\Afootnote{Here the poem ends on fol. 2v; it picks back up on fol. 10v.}}, &
þat im ward \alst{s}unu gi·boran; \hld\ þem skuopun siu \alst{S}eð te naman &
\alst{w}árom \alst{w}ordum: \hld\ þem \alst{w}astom lêh &
\alst{h}evanas waldand \hld\ endi \alst{h}ugi guodan, &
\alst{g}am-likan \alst{g}ang \hld\ -he was \alst{g}oda wirðig-, &
\alst{m}ildi was hie im an is \alst{m}uoda. \hld\ Só þana is \alst{m}anno wel, &
þie io mið su-likaro \alst{h}uldi muot \hld\ \alst{h}êrron þionun. &
Hie \alst{l}ovoda þuȯ mêst \hld\ \alst{l}iodjo barnun, &
\alst{g}odas huldi: \hld\ \alst{g}umun þanan kwámun &
guoda mann, \hld\ . . . . . . . . . . &
\alst{w}ordun \alst{w}ísa, \hld\ ge·\alst{w}itt línodun, &
\alst{þ}egnos gi·\alst{þ}ȧhte \hld\ endi \alst{þ}igun aftar wel. &
Þann \alst{k}wámun eft fan \alst{K}aina \hld\ \alst{k}raftaga liudi, &
\alst{h}ęlidos \alst{h}ard-muoda, \hld\ habdun im \alst{h}ugi strangan, &
\alst{w}rêdan willjan, \hld\ wí weldun \alst{w}aldandas &
\alst{l}êra \alst{l}êstjan, \hld\ ak habdun im \alst{l}êdan stríd; &
\alst{w}uohsun im \alst{w}risi-líko: \hld\ þat was þiu \alst{w}irsa gi·burd, &
\alst{k}uman fan \alst{K}aina. \hld\ Bi·gunnun im \alst{k}ôpun þuȯ &
\alst{w}eros \alst{w}íb undor twisk: \hld\ þas ward a·\alst{w}erðit sán &
\alst{S}eðas ge·\alst{s}ïdi, \hld\ warð \alst{s}ęggjo folk &
\alst{m}ênu gi·\alst{m}ęngid \hld\ endi wurðun \alst{m}anno barn, &
\alst{l}iudi \alst{l}êða, \hld\ þem þitt \alst{l}ioht gi·skuop, &
botan þat iro \alst{ê}n habda \hld\ \alst{e}rlas gi·hugdi, &
\alst{þ}egạn-líka gi·\alst{þ}ȧht; \hld\ was im gi·\alst{þ}ungin mann, &
\alst{w}ís endi \alst{w}ord-spáh, \hld\ habda gi·\alst{w}itt mikil: &
\alst{E}nokh was hie hêtan. \hld\ Þie hier an \alst{e}rðu warð &
\alst{m}annum te \alst{m}árðum \hld\ obar þesan \alst{m}iddil-garð, &
þat ina hier só \alst{k}wikana \hld\ \alst{k}uningo þie bętsto, &
\alst{l}ibbendjan an is \alst{l}ík-haman, \hld\ só hie io an þesun \alst{l}iahta ni starạf - &
ak só gi·\alst{h}aloda ina \alst{h}ier \hld\ \alst{h}evanas waldand &
endi ina þár gi·\alst{s}ętta, \hld\ þár hie \alst{s}imlon muot &
\alst{w}esan an \alst{w}unnjon, \hld\ untat ina eft an þesa \alst{w}erold sęndit &
\alst{h}êr \alst{h}evanas ward \hld\ \alst{h}ęliðo barnum, &
\alst{l}iodjun te \alst{l}êro. \hld\ Þann hier ôk þie \alst{l}êdo kumit, &
þat hier \alst{A}nti-krist \hld\ \alst{a}lla þioda, &
\alst{w}erod a·\alst{w}erðit, \hld\ þann he mid \alst{w}ápnu skal &
werðan \alst{E}nokha te banon, \hld\ \alst{ę}ggjun skarạpun &
þuruh is \alst{h}and-męgin; \hld\ \alst{h}wirịbit þiu sêola, &
þie \alst{g}êst an \alst{g}uodan weg, \hld\ endi \alst{g}odas ęngil kumit, &
\alst{w}rikit ina, \alst{w}amm-skaðon \hld\ \alst{w}ápnas ęggjun: &
wirðit \alst{A}nti-krist \hld\ \alst{a}ldru bi·lôsid, &
þie \alst{f}íund \edtext{bi·\alst{f}ęllid}{\Afootnote{\emph{biuellid} \textbf{V}}}. \hld\ \alst{F}olk wirðit eft gi·hworọvan &
te \alst{g}odas ríkja, \hld\ \alst{g}umuno gi·sïði &
\alst{l}anga hwíla, \hld\ endi stéd im sídor þit \alst{l}and gi·sund.\eva

\bvb TODO.\evb\evg

\sectionline

\section{The Destruction of Sodom}

\bvg\bva\mssnote{\textbf{V} 2r/1}%
Þuȯ habdun im eft só \alst{s}wíðo \hld\ \alst{S}odomo-liudi, &
\alst{w}eros só far·\alst{w}erkot, \hld\ þat im was u̇sa \alst{w}aldand gram, &
\alst{m}ahtig drohtin, \hld\ wand sia \alst{m}ên drivun, &
\alst{f}ręmidun \alst{f}irin-dâdi, \hld\ habdun im só uilu \alst{f}íunda barn &
\alst{w}ammas ge·\alst{w}ísid: \hld\ þuȯ ni welda þat \alst{w}aldand god, &
\alst{þ}iadan \alst{þ}olojan, \hld\ ak hiet sie \alst{þ}rea faran, &
is \alst{ę}ngelos \alst{ô}stan \hld\ an is \alst{á}rundi, &
\alst{s}ïðon te \alst{S}odoma, \hld\ endi was im \alst{s}elvo þar mið. &
Þuȯ sea ovar \alst{M}ambra \hld\ \alst{m}ahtige fuorun, &
þuȯ fundun sia \alst{A}brahama \hld\ bi ênum \alst{a}la standan, &
\alst{w}aran ênna \alst{w}ih-stędi, \hld\ endi skolda u̇sas \alst{w}aldandas &
\alst{g}eld gi·frummjan, \hld\ endi skolda þar \alst{g}oda þeonan &
an \alst{m}iddjan dag \hld\ \alst{m}anna þie bętsto. &
Þuȯ ant·\alst{k}ęnda hé \alst{k}raft godas, \hld\ só he sea \alst{k}uman gi·sakh: &
\alst{g}éng im þuȯ ti·\alst{g}egnes \hld\ endi \alst{g}oda selvun hnêg, &
\alst{b}ôg endi \alst{b}edode \hld\ endi \alst{b}ad gerno, &
þat hie is \alst{h}uldi forð \hld\ \alst{h}ębbjan muosti: &
„\alst{w}arod \alst{w}ilþu nu, \hld\ \alst{w}aldand, frô mín, &
\alst{a}lo-mahtig fadar? \hld\ ik biun þín \alst{ê}gan skalk, &
\alst{h}old endi gi·\alst{h}ôrig; \hld\ þú bist mí \alst{h}êrro só guod, &
\alst{m}êðmo só \alst{m}ildi: \hld\ wilþu \alst{m}ínas wiht, &
\alst{d}rohtin, hębbjan? \hld\ Hwat, it all an þínum \alst{d}uoma stéd, &
ik \alst{l}ibbjo bi þínum \alst{l}êhene, \hld\ endi ik gi·\alst{l}ôbi an þi, &
\alst{f}rô mín þe guoda: \hld\ muot ik þi \alst{f}rágon nu, &
warod þu \alst{s}igi-drohtin \hld\ \alst{s}ïðon willjas?“ &
Þuȯ kwam im eft te·\alst{g}egnes \hld\ \alst{g}odas and-wordi, &
\alst{m}ahtig \alst{m}uotta: \hld\ „Ni willi ik is þi \alst{m}íðan nu,“ kwað he, &
„\alst{h}elan \alst{h}oldan man, \hld\ hú mín \alst{h}ugi gęngit. &
\alst{S}ïðan skulun wí \alst{s}u̇ðar hinan: \hld\ hębbjat him umbi \alst{S}odoma-land &
\alst{w}eros só for·\alst{w}erkot. \hld\ Nú hruopat \edtext{þeæ \alst{w}ardas}{\Afootnote{\emph{þe æuuardas \textbf{V}}}} te mí &
\alst{d}ages endi nahtes, \hld\ þe þe iro \alst{d}ádi tęlljat, &
\alst{s}ęggjat hiro \alst{s}undjon. \hld\ Nú willi ik \alst{s}elvo witan, &
ef þia \alst{m}ann under him \hld\ su-lík \alst{m}ên fręmmjat, &
\alst{w}eros \alst{w}am-dádi. \hld\ Þanna skal sea \alst{w}allande &
\alst{f}iur bi·\alst{u}allan, \hld\ skulun sia hira \alst{f}irin-sundjon &
\alst{s}wára bi·\alst{s}ęnkjan: \hld\ \alst{s}weval fan himile &
\alst{f}allit mid \alst{f}iure, \hld\ \alst{f}êknja sterẹvat, &
\alst{m}ên-dádige \alst{m}ęn, \hld\ reht só \alst{m}organ kumit.“ &
\alst{A}braham þuȯ gi·mahalda \hld\ (habda im \alst{ę}lljan guod, &
\alst{w}ísa \alst{w}ord-kwidi), \hld\ endi wiðer is \alst{w}aldand sprak: &
„Hwat! þu \alst{g}ódas só uilu,“ \hld\ kwat hie, „\alst{g}od hevan-ríki, &
\alst{d}rohtin gi·\alst{d}uomis, \hld\ all bi þínun \alst{d}ádjun stéd &
þius \alst{w}erold an þínum \alst{w}illjan; \hld\ þu gi·\alst{w}ald habas &
ovar þesan \alst{m}iddil-gard \hld\ \alst{m}anna kunnjas, &
só þat gio \alst{w}erðan ni skal, \hld\ \alst{w}aldand frô mín, &
þat þú þar te \edtext{\alst{ê}num}{\Afootnote{\emph{henum} \textbf{V}}} duoas \hld\ \alst{u}vila endi guoda, &
\alst{l}iova endi \alst{l}êða, \hld\ wand sia gi·\alst{l}íka ni sind. &
Þu \alst{r}uomes só \alst{r}ehtæs, \hld\ \alst{r}íki drohtin, &
só þu ni wili, þat þar ant·\alst{g}eldan \hld\ \alst{g}uod-willige mann &
\alst{w}am-skaðono \alst{w}erẹk, \hld\ þoh þu is gi·\alst{w}ald haves &
te gi·\alst{f}rummjanna. \hld\ Muot ik þi \alst{f}rágon nu, &
só þú mí þiu \alst{g}ramara ni sís, \hld\ \alst{g}od hevan-ríki? &
ef þú þar \alst{f}ïðis \alst{f}ïftig \hld\ \alst{f}erạhtaro manno, &
\alst{l}iuvigaro \alst{l}iodo, \hld\ muot þanna þat \alst{l}and gi·sund, &
\alst{w}aldand, and þínum \alst{w}illjan \hld\ gi·\alst{w}erid standan?“ &
Þuȯ kwam im eft te·\alst{g}egnes \hld\ \alst{g}odas and-wordi: &
„Ef ik þar \alst{f}indo \alst{f}ïftig,“ kwað he, \hld\ „\alst{f}erạhtara manno, &
\alst{g}uodaro \alst{g}umono, \hld\ þea te \alst{g}oda hębbjan &
\alst{f}asto gi·\alst{f}angan, \hld\ þanna willi ik im iro \alst{f}erạh far·gevan &
þuru þat ik þea \alst{h}luttron man \hld\ \alst{h}aldan wille.“ &
\alst{A}braham þuȯ gi·mahalda \hld\ \alst{ȧ}ðar sïðe, &
\alst{f}orð \alst{f}rágoda \hld\ \alst{f}râhon sínan: &
„Hwat \alst{d}uos þu is þanna,“ kwað he, \hld\ „\alst{d}rohtin frô mín, &
ef þu þar \alst{þ}rítig maht \hld\ \alst{þ}egno fïðan, &
\alst{w}am-lôsa \alst{w}eros? \hld\ \alst{w}ilþu sia noh þanna &
\alst{l}átan te \alst{l}íva, \hld\ þat sia muotin þat \alst{l}and waran?“ &
Þuȯ im þe \alst{g}uoda, \hld\ \alst{g}od hevan-ríki, &
\alst{s}niumo gi·\alst{s}agda, \hld\ þat hie \alst{s}ó weldi &
\alst{l}êstjan an þen \alst{l}anda: \hld\ „Ef ik þar \alst{l}ubigaro mahg,“ kwað he, &
„\alst{þ}rítig undar þero \alst{þ}iodo \hld\ \alst{þ}egno fïðan &
\alst{g}od-forọhta \alst{g}umon: \hld\ þanna willi ik im far·\alst{g}evan allum &
þat \alst{m}ên endi þea \alst{m}is-dád \hld\ endi látan þat \alst{m}anno folk &
\alst{s}ittjan umbi \alst{S}odoma \hld\ endi ge·\alst{s}und wesan.“ &
\alst{A}braham þuȯ gi·mahalda \hld\ \alst{a}galêt-líko &
-\alst{f}olgoda is \alst{f}rôjan-, \hld\ \alst{f}ilo worda gi·sprak: &
„Nu skal ik is þi \alst{b}iddjan“, kwað he, \hld\ „þat þu þi ni \alst{b}elges ti mi, &
\alst{f}rô mín þie guoda, \hld\ hú ik sus \alst{f}ilu mahlja, &
\alst{w}eslja wiðer þi mid mínum \alst{w}ordum: \hld\ ik wêt, þat ik þas \alst{w}irðig ni bium &
ni sí þat þu it willjas bi þínaro \alst{g}uodi, \hld\ \alst{g}od hevan-ríki &
\alst{þ}iadan, gi·\alst{þ}olojan: \hld\ mí is \alst{þ}arạf mikil &
te \alst{w}itanna þínne \alst{w}illjan, \hld\ hweðer þat \alst{w}erad gi·sund &
\alst{l}ibbjan muoti, \hld\ þe sea \alst{l}iggjan skulun, &
\alst{f}êgja bi·\alst{u}allan: \hld\ hwat wilis þu is þanna, \alst{f}rô mín, duoan, &
ef þu þar \alst{t}ehani \hld\ \alst{t}reu-hafte maht &
\alst{f}ïðan under þemo \alst{f}olka \alst{f}erahtera manno \hld\ wilþu im þanna hiro \alst{f}erh far·gevan, &
þat sia umbi \alst{S}odoma-land \hld\ \alst{s}ittjan muotin &
\alst{b}úan an þem \alst{b}urụgjum, \hld\ só þu im a·\alst{b}olgan ni sís?“ &
Þuȯ kwam im eft te·\alst{g}egnes \hld\ \alst{g}odas and-wordi: &
„Ef ik þar \alst{t}ehani,“ kwað he, \hld\ „\alst{t}reu-haftera mag &
an þem \alst{l}ande noh \hld\ \alst{l}iodjo fïðan, &
þanna látu ik sia alla þuru þie \alst{f}erạhtun man \hld\ \alst{f}erẹhas brúkan.“ &
Þuȯ ni \alst{d}orste Abraham lęng \hld\ \alst{d}rohtin sínan &
\alst{f}urður \alst{f}rágon, \hld\ hak he \alst{f}ell im after te bedu &
an \alst{k}neo \alst{k}raftag, \hld\ \alst{k}wað he gerno &
is \alst{g}eld \alst{g}ęrẹwedi \hld\ endi \alst{g}ode þeonodi, &
\alst{w}arạhti after is \alst{w}illjan. \hld\ Gi·\alst{w}êt im eft þanan &
\alst{g}angan te is \alst{g}ęst-sęli; \hld\ \alst{g}odes ęngilos fort &
\alst{s}ïðodun te \alst{S}odoma, \hld\ so im \alst{s}elvo ge·bôd &
\alst{w}aldand mid is \alst{w}ordo, \hld\ þuȯ hie sea hiet an þana \alst{w}eg faran.\eva

\bvb TODO.\evb\evg


\bvg\bva[][100]\mssnote{\textbf{V} 2r/36}%
\edtext{Skoldun}{\Afootnote{Introduced by large initial.}} sie be·\alst{f}ïðan, \hld\ \edtext{hwat þár}{\Afootnote{\emph{huattar} \textbf{V}}} \alst{f}erạhtera &
umbi \alst{S}odoma-burg, \hld\ \alst{s}undjono tuomera &
\alst{m}anna wári, \hld\ þie ni habdin \alst{m}ênes filu, &
\alst{f}irin-werko gi·\alst{f}rumid. \hld\ Þȯ gi·hôrdun siæ \alst{f}êgero karm &
an allaro \alst{s}ęliðu gi·hwen, \hld\ \alst{s}undiga liudi &
\alst{f}irin-werk \alst{f}ręmmjan: \hld\ was þar \alst{f}íundo gi·mang, &
\alst{w}rêðaro \alst{w}ihtjo, \hld\ þea an þat \alst{w}am habdun &
þea \alst{l}iudi far·\alst{l}êdid: \hld\ þat \alst{l}ôn was þuȯ hat handum &
\alst{m}ikil mið \alst{m}orðu, \hld\ þat sia oft \alst{m}ên drivun. &
Þanna sat im þar an \alst{i}nnan \hld\ \alst{a}ðal-burdig man, &
\alst{L}oth mið þem \alst{l}iudjum, \hld\ þie oft \alst{l}of godas &
\alst{w}arạhte an þesaro \alst{w}eroldi: \hld\ habda im þar \alst{w}elono gi·nuog, &
\alst{g}uodas gi·wunnan: \hld\ he was \alst{g}ode wirðig. &
He was \alst{A}brahamas \hld\ \alst{a}ðali-knóslas, &
his \alst{b}róðer \alst{b}arn: \hld\ ni was \alst{b}ętara man &
umbi \alst{G}iordanas staðos \hld\ mið \alst{g}um-kustjum, &
gi·\alst{w}erid mið ge·\alst{w}ittjo: \hld\ him was u̇sa \alst{w}aldand hold &
Þuȯ te \alst{s}edla hnêg \hld\ \alst{s}unna þiu hwíta, &
alloro \alst{b}ôkno \alst{b}erạhtost, \hld\ þuȯ stuond hie fore þes \alst{b}urụges dore. &
Þuȯ gi·sah hé an \edtext{\alst{á}vand}{\Afootnote{\emph{haband} \textbf{V}}} \hld\ \alst{ę}ngilos twêne &
\alst{g}angan an þea \alst{g}ardos, \hld\ só sea fan \alst{g}ode kwámun &
ge·\alst{w}eride mid ge·\alst{w}ittjo; \hld\ þuȯ sprak he im sán mid is \alst{w}ordum tuo. &
\alst{G}éng þuȯ te·\alst{g}egnes \hld\ endi \alst{g}ode þankade, &
\alst{h}evan-kuninga, \hld\ þes hé im þea \alst{h}elpa fer·lêkh, &
þat he muosta sea mið is \alst{ô}gum \hld\ \alst{a}n luokojan, &
iak he sea an \alst{k}neo \alst{k}usta \hld\ endi \alst{k}u̇sko bad, &
þat sea \alst{s}uohtin his \alst{s}ęliða: \hld\ kwat þat he im \alst{s}elbas duom &
\alst{g}áui su-líkas \alst{g}uodas, \hld\ só im \alst{g}od habdi &
far·\alst{l}iwen an þem \alst{l}anda: \hld\ sea ni wurðun te \alst{l}ata hwerigin, &
ak se \alst{g}engun im an is \alst{g}ęst-sęli, \hld\ endi he im \alst{g}iungar-duom &
\alst{f}ręmide \alst{f}eraht-líka, \hld\ sea im \alst{f}ilo sagdun &%NOTE: sea] first word on 2v
\alst{w}áraro \alst{w}ordu. \hld\ Þár he an \alst{w}ahtu sat, &
\alst{h}eld is \alst{h}êrran bodan \hld\ \alst{h}êlag-líka, &
\alst{g}odas ęngilos. \hld\ Sia him \alst{g}uodas só filo, &
\alst{s}uȯðas gi·\alst{s}agdun. \hld\ \alst{S}wart furður skrêd, &
\alst{n}arowa \alst{n}aht an skion, \hld\ \alst{n}áhida moragan &
an \alst{a}llara sęliða gi·hwem. \hld\ \alst{U}ht-fugal sang &
fora \edtext{\alst{d}aga-h\emph{r}uom\emph{a}}{\Afootnote{emend.; \emph{‘daga huoam’} \textbf{V}}}. \hld\ Þȯ habdun u̇sas \alst{d}rohtinas bodon &
þea \alst{f}irina bi·\alst{f}undan, \hld\ þea þar \alst{f}ręmidun mên &
umbi \alst{S}odoma-burụg. \hld\ Þȯ \alst{s}agdun sia Loða, &
þat þar \alst{m}orð \alst{m}ikil \hld\ \alst{m}anno barno, &
skolda þera \alst{l}iodjo \edtext{werðan}{\Afootnote{\emph{‘huuerthan’} \textbf{V}}} \hld\ endi ôk þes \alst{l}andas só samo. &
Hietun ina þuȯ \alst{g}ęrẹwjan, \hld\ endi hietun þȯ \alst{g}angan þanan, &
\alst{f}irrjan hina fon þem \alst{f}íundum \hld\ endi lêdjan is \alst{f}rí mið him, &
\alst{i}dis \alst{a}ðal-borana. \hld\ He ni habda þar his \edtext{\alst{a}ðaljas}{\Afootnote{\emph{‘hadalias’} \textbf{V}}} þan mêr, &
botan is \alst{d}ohtar twá, \hld\ mid þem gi·hietun sie, þat hie êr \alst{d}aga wári &
an ênum \alst{b}erga uppan, \hld\ þat hina \alst{b}rinnandi &
\alst{f}iur ni bi·\alst{u}engi. \hld\ Þȯ he te þere \alst{f}ęrði warð &
\alst{g}ȧhun gi·\alst{g}ęrewid, \hld\ \alst{g}engun ęngilos, &
\alst{h}abdun hina bi \alst{h}andum \hld\ \alst{h}evan-kuningas bodon, &
\alst{l}êddun hina endi \alst{l}êrdun \hld\ \alst{l}ango hwíla, &
untat sea ina gi·\alst{b}râhtun \hld\ bi þera \alst{b}urụg útan. &
\alst{H}ietun, þat siæ io ni ge·\alst{h}ôrdin \hld\ sulik ge·\alst{h}lunn mikil &
\alst{b}rakon an þem \alst{b}urụgjum, \hld\ þat sia io under \alst{b}ak sâwen, &
an þiu þie sea an þem \alst{l}andæ \hld\ \alst{l}ibbjan weldin. &
Þuȯ \edtext{\alst{\emph{h}}wuruvun}{\Afootnote{metr. emend.; \emph{uuruƀun} \textbf{V}}} eft wiðer \hld\ \alst{h}êlega wardos, &
\alst{g}odas ęngilos, \hld\ \alst{g}engun sniumo, &
\alst{s}ïðodun te \alst{S}odomo: \hld\ þanan \alst{s}u̇ðar fuor &
\alst{L}oth þoro hira \alst{l}êra, \hld\ flôh þera \alst{l}iodjo gi·mang, &
\alst{d}ęrẹvjoro manno: \hld\ þȯ warð \alst{d}ag kuman. &
Þuȯ warð þar gi·\alst{h}lunn mikil \hld\ \alst{h}imile bi·tęngi, &
\alst{b}rast endi \alst{b}rakoda, \hld\ warð þero \alst{b}urụgjo gi·\emph{h}wilík &
\alst{r}ôkas gi·fullit, \hld\ warð þar fan \alst{r}adura só uilu &
\alst{f}iures gi·\alst{f}allin, \hld\ warð \alst{f}êgero karm, &
\alst{l}êðaro \alst{l}iodjo: \hld\ \alst{l}ogna all bi·ueng &
\alst{b}rêd \alst{b}urụgu-gi·setu: \hld\ \alst{b}ran all samað, &
\alst{st}ên endi erða, \hld\ endi só manag \alst{st}rídin man &
\alst{s}wultun endi \alst{s}unkun: \hld\ \alst{s}weval brinnandi &
\alst{w}el after \alst{w}íkjom; \hld\ \alst{w}arạgas þolodun &
\alst{l}êðas \alst{l}ôn-geld. \hld\ Þat \alst{l}and inn bi·sank, &
þiu \alst{e}rða an \alst{a}f-grundi; \hld\ \alst{a}l warð far·spildit &
\alst{S}odoma-ríki, \hld\ þat is ênig \alst{s}ęg ni gi·nas, &
iak só bi·\alst{d}ôðit an \alst{d}ôð-sêu, \hld\ so it noh te \alst{d}aga stęndit &
\alst{f}luodas gi·\alst{f}ullit. \hld\ Þuȯ habdun hiro \alst{f}irin-dádi &
all \alst{S}odomo-þiod \hld\ \alst{s}êro ant·goldan, &
botan þat þar iro \alst{ê}nna \hld\ \alst{ú}t ent·lêdde &
\alst{w}aldand an is \alst{w}illjan \hld\ endi þiu \alst{w}íf mid im, &
\alst{þ}riu mið þem \alst{þ}egna. \hld\ Þȯ gi·hôrdun sea þero \alst{þ}iodo kwalm, &
\alst{b}urụgi \alst{b}rinnan. \hld\ Þȯ þar under \alst{b}ak bi·sakh &
\alst{i}dis \alst{a}ðal-boren \hld\ -siu ni welde þera \alst{ę}ngilo &
\alst{l}êra \alst{l}êstjan; \hld\ þat was \alst{L}ohthas brúd, &
þan \alst{l}ang þe siu an þem \alst{l}anda \hld\ \alst{l}ibbjan muosta- &
þuȯ siu an þem \alst{b}erẹga gi·stuod \hld\ endi under \alst{b}ak bi·sakh, &
þuȯ warð siu te \alst{st}êne, \hld\ þar siu \alst{st}andan skal &
\alst{m}annum te \alst{m}árðu \hld\ ovar \alst{m}iddil-gard &
\alst{a}fter te \alst{ê}wan-dage, \hld\ só lango só þius \alst{e}rða \edtext{lêvot.}{\Afootnote{add. \emph{EXPL} \textbf{V}}\Bfootnote{The \emph{EXPL} in the ms. stands for ‘explicit’, customarily placed at the end of a text in medieval mss.  This line also serves as a fitting conclusion to the poem.}}\eva

\bvb TODO.\evb\evg

\sectionline
