Þ\bookStart{Old Saxon Genesis}

\begin{flushright}%
\textbf{Dating:} C9th

\textbf{Meter:} \Fornyrdislag%para
\end{flushright}%

The normalization follows the \Heliand.  There is only one ms., Palatinus latinus 1447: https://digi.vatlib.it/view/MSS_Pal.lat.1447/0005.

\sectionline

\bvg\bva%NOTE: 1r of the ms.
„Wela, þat þú nú, \alst{Ê}wa, habas,“ kwad Adam, „\alst{u}bilo gi·marakot &
unkaro \alst{s}elbaro \alst{s}ïd. \hld\ Nú maht þú \edtext{sehan}{\Afootnote{\emph{sean} ms.}} þia \alst{s}warton hęll &
\alst{g}inon \alst{g}rádaga; \hld\ nú þú sia \alst{g}rimman maht &
\alst{h}inana gi·\alst{h}ôrjan, \hld\ nis \alst{h}eban-ríki &
ge·\alst{l}íhk sulíkaro \alst{l}ógnun: \hld\ þit was alloro \alst{l}ando skônjust, &
þat wit hier þuruh unkas \alst{h}êrran þank \hld\ \alst{h}ębbjan muostun &
þar þú þem ni \alst{h}ôrdis \hld\ þie unk þesan \alst{h}aram gi·ried, &
þat wit \alst{w}aldandas \hld\ \alst{w}ord far·brâkun, &
\alst{h}eban-kuningas. \hld\ Nú wit \alst{h}riwig mugon &
\alst{s}orogon for þem \alst{s}ïda, \hld\ wand hé \edtext{unk}{\Afootnote{\emph{hunk} ms.}} \alst{s}elbo gi·bôd, &
þat \alst{w}it \edtext{unk}{\Afootnote{\emph{hunk} ms.}} su-lik \alst{w}íti \hld\ \alst{w}ardon skoldin, &
\alst{h}aramo mêstan— \hld\ nú þwingit mí giu \alst{h}ungar endi þrust, &
\alst{b}itter \alst{b}alo-werek, \hld\ þero wáron wit êr \alst{b}êdero tuom. &
Hú skulun wit nu \alst{l}ibbjan, \hld\ efto hú skulun wit an þesum \alst{l}iahta wesan, &
nu hier hwílum \alst{w}ind kumit \hld\ \alst{w}estan efto ôstan, &
\alst{s}u̇ðan efto nordan? \hld\ gi·\alst{s}werek upp dríbit &
-kumit \alst{h}aglas skion \hld\ \alst{h}imile bi·tengi-, &
\alst{f}ęrid \alst{f}ord an gi·mang \hld\ (þat is \alst{f}irinum kald): &
\alst{h}wílum þanne fan \alst{h}imile \hld\ \alst{h}êto skínit, &
\alst{b}líkit þiu \alst{b}erahto sunna: \hld\ wit hier þus \alst{b}ara standat, &
un·\alst{w}ęrid mid gi·\alst{w}âdi: \hld\ nis unk hier \alst{w}iht bi·foran &
ni te \alst{sk}adowa ni te \alst{sk}úra, \hld\ unk nis hier \alst{sk}attas wiht &
te \alst{m}ęti gi·\alst{m}arkot: \hld\ wit hębbjat unk gi·duan \alst{m}ahtigna god, &
\alst{w}aldand \alst{w}rêdan. \hld\ Te hwí skulun wit \alst{w}erdan nu? &
Nu mag mí þat \alst{h}rewan, \hld\ þat ik is io bad \alst{h}eban-ríkjan god, &
waldand þ {[...]}\eva

\bvb TODO.\evb\evg

\sectionline

\bvg\bva%NOTE: 2r
Sïdoda im þuo te sęlidon, \hld\ habda im sundja gi·waraht
bittra an is bruodar; \hld\ liet ina undar baka liggjan
an ênam diapun dala \hld\ drôr-wóragana,
líbas lôsan, \hld\ legar-bedd waran,
guman an griata. \hld\ Þuo sprak im god selbo tuo,
waldand mid is wordun \hld\ (was im wrêd an is hugi,
þem banan gi·bolgan), \hld\ frâgoda hwar he habdi is brôdar þuo
kind-jungan guman. \hld\ Þô sprak im eft Kain an·gęgen
-habda im mid is handun \hld\ haram-werek mikil
wam-dádjun gi·waraht, \hld\ þius werold was só swído
be·smitin an sundjun-: \hld\ „Ni ik þes sorogun ni skal,“ kwad he,
„gômian hwar hie ganga, \hld\ ni it mi god ni gibôd,
þat is hwęrigin hier \hld\ huodian þorofti,
wardon an þesaro weroldi.“ \hld\ Wánde he swído,
þat he bi·helan mahti \hld\ hêrran sínum,
þia dâdi bi·dęrnjan. \hld\ Þuo sprak im eft ûsa drohtin tuo:
„All habas þu sô gi·werekot,“ quad he, \hld\ „sô þi ti þínaro weroldi mag
wesan þín hugi hriuwig, \hld\ þes þu mid þínum handon gi·dedos,
þat þu wurdi þínes bruodar bano: \hld\ nu he bluodig ligit,
wundun wôrig; \hld\ þes ni habda he êniga ge·wuruhte te þi,
sundja gi·suohta, \hld\ þoh þu ina nu a·slagan hębbjas,
dôdan gi·duanan. \hld\ Is drôr sinkit nu an erda,
swêt sundar ligit; \hld\ þiu seola hwarobat
þie gêst giámar-muod \hld\ an godas willjan;
drôr hruopit is te drohtina selbun \hld\ endi sagat hwe þea dâdi frumida,
þat mên an þesun middil-gardun: \hld\ ni mag im ênig mann þan swídor
wero far·wirikjan \hld\ an werold-ríkja
an bittron balo-dádjon, \hld\ þan þu an þínum bruodar habas
firin-werek gi·fręmid.“ \hld\ Þuo an forahtun ward
Kain aftar þem quidiun drohtinas, \hld\ quad þat hie wisse garoo,
þat is ni mahti werdan waldand wiht, \hld\ an werold-stundu
dâdeo bidernid, \hld\ „sô ik is nu mag drubundjan hugi,“ quad he,
„beran an mínun breostun \hld\ þes ik mínan bruodar sluog
þuru mín hand-męgin. \hld\ Nu wêt ik, þat ik skal an þínum hęti libbjan,
ford an þínum fíund-skępi, \hld\ nu ik mi þesa firina gi·deda,
só mi mína sundja nu \hld\ swídaron þunkjat,
misdâd mêra, \hld\ þan þín mildi hugi,
sô ik þes nu wirdig ni bium, \hld\ waldand þie guodo,
þat þu mi alâtas \hld\ lêdas þingas,
tianono atuemeas. \hld\ Nu ik ni welda mína triuwa haldan,
hugi wid þem þínum hlutron muoda, \hld\ nu wêt ik, þat ik hier ni mag êniga hwíla libbjan,
huand mí ant·wirikit, \hld\ sô hwat sô mi an þisun wega findit,
a·slęhit mi bi þesun sundjun.“ \hld\ Þuo sprak im eft selbo an·gegin
hebanes waldand: \hld\ „Hier skalt þu noh nu“, quad he,
„libbjan lango hwíla. \hld\ Þo þu sus a·lêdit sís,
mid firinum bifangan, \hld\ þoh will ik þi friðu settean,
tôgean sulik têkean, \hld\ sô þu an treuwa maht
wesan an þesero werolde, \hld\ þoh þu is wirdik ni sís:
fluhtik skalt þu þoh endi frêdig \hld\ fordwardas nu
libbean an þesum landa, \hld\ sô lango sô þu þit liaht waros;
forhuâtan skulun þi hluttra liudi, \hld\ þu ni salt io furður kuman te þínes hêrron sprâko,
weslean þar mid wordon þínon: \hld\ unaldandi stêt
þínes brôdor wrâka \hld\ bitter an helli.“
Þô geng im þanan mid grimmo hugi, \hld\ habda ina god selbo
swído far·sakanan. \hld\ Soroga warð þar þuo gi·ku̇dit
Adama endi Êwun, \hld\ in-widd mikil,
iro kindes kwalm, \hld\ þat he ni muosta kwik libbjan.
Þes ward damas hugi \hld\ innan breostun
swído an sorogun, \hld\ þuo he wissa is sunu dôdan:
só ward is ôk þiu muodar, \hld\ þe þana magu fuodda,
barn bi iro breostun. \hld\ Þuo siu bluodag wuosk
hrêu-gi·wâdi, \hld\ þuo ward iro hugi sêrag.
Bêþo was im þô an sorogun \hld\ iak iro barnas dôd,
þes heliðas hin-fard, \hld\ iak þat im mid is handun fordæda
Kain an sulikun qualma: \hld\ siu ni habdun þuo noh kindo þan mêr
libbendero an þem liahta, \hld\ botan þana ênna, þie þuo alêdit was
waldanda be is far·wurohtjun: \hld\ þar ni habdun siu êniga wunja tuo
niud-líko gi·numan, \hld\ wand hie sulikan níd a·huof,
þat he ward is bruodar bano. \hld\ Þes im þuo bêðjun ward,
sin-híun tuêm \hld\ sêr umbi herta.
Oft siu þes gornunde \hld\ an griata gi·stuodun,
sin-híun samad, \hld\ quâdun, þat sia wissin, þat im þat iro sundja gi·dedin,
þat im ni muostin aftar \hld\ ęrebi-wardos
þegnas þían. \hld\ Þolodun siu bêðiu
mikila mord-kwâla, \hld\ unt þat im eft mahtig god,
hêr hebanes ward \hld\ iro hugi buotta,
þat im wurðun ôdana \hld\ ęrebi-wardos,
þegnos endi þiornun, \hld\ þigun aftar wel,
wôhsun wân-líko, \hld\ ge·witt línodun,
spâha sprâka. \hld\ Spuodda þie mahta
is hand-gi·werek, \hld\ hêlag drohtin,
þat im ward sunu giboran; \hld\ þem skuopun siu Seð te naman
wárom wordum: \hld\ þem wastom lêh
hebanas waldand \hld\ endi hugi guodan,
gamlikan gang - \hld\ he was goda wirðig -,
mildi was hie im an is muoda. \hld\ Sô þana is manno wel,
þie io mið sulikaro huldi muot \hld\ hêrron þionun.
Hie loboda þuo mêst \hld\ liodio barnun,
godas huldi: \hld\ gumun þanan quâmun
guoda mann, \hld\ . . . . . . . . . .
wordun wísa, \hld\ ge·witt línodun,
þegnos giþâhte \hld\ endi þigun aftar wel.
Þann quâmun eft fan Kaina \hld\ kraftaga liudi,
helidos hardmuoda, \hld\ habdun im hugi strangan,
wrêdan willjan, \hld\ wí weldun waldandas
lêra lêstjan, \hld\ ak habdun im lêdan stríd;
wuohsun im wrisilíko: \hld\ þat was þiu wirsa giburd,
kuman fan Kaina. \hld\ Bigunnun im kôpun þuo
weros wíb undor twisk: \hld\ þas ward a·werðit sân
Seðas ge·sïdi, \hld\ warð sęggjo folk
mênu gi·męngid \hld\ endi wurðun manno barn,
liudi lêða, \hld\ þem þitt lioht gi·skuop,
botan þat iro ên habda \hld\ erlas gi·hugdi,
þegan-líka gi·þȧht; \hld\ was im gi·þungin mann,
wís endi word-spáh, \hld\ habda gi·witt mikil:
Enokh was hie hêtan. \hld\ Þie hier an erðu warð
mannum te mârðum \hld\ obar þesan middilgarð,
þat ina hier sô quikana \hld\ kuningo þie bezto,
libbendjan an is lík-haman, \hld\ sô hie io an þesun liahta ni staraf -
ak sô gi·haloda ina hier \hld\ hebanas waldand
endi ina þar gi·sętta, \hld\ þar hie simlon muot
wesan an wunnjon, \hld\ untat ina eft an þesa werold sendit
hêr hebanas warð \hld\ heliðo barnum,
liodiun te lêro. \hld\ Þann hier ôk þie lêdo kumit,
þat hier Antikrist \hld\ alla þioda,
werod awerðit, \hld\ þann he mid wâpnu skal
werðan Enokha te banon, \hld\ eggiun skarapun
þuruh is hand-męgin; \hld\ hwiribit þiu sêola,
þie gêst an guodan weg, \hld\ endi godas engil kumit,
wrikit ina, wamm-skaðon \hld\ wâpnas eggiun:
wirðit Antikrist \hld\ aldru bi·lôsid,
þie fíund bi·uellid. \hld\ Folk wirðit eft gi·hworoban
te godas ríkja, \hld\ gumuno gi·sïði
langa hwíla, \hld\ endi stêd im sídor þit land gi·sund.\eva

\bvb TODO.\evb\evg

\sectionline
