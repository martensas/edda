\book{\emph{Vǫluspǫ́}}\bookStart

\bva \alst{H}ljóðs bið’k allar \hld \alst{h}ęlgar kindir, \\%M
\alst{m}ęiri ok \alst{m}inni \hld \alst{m}ǫgu Hęimdallar; \\%M
\alst{v}ildu at, \alst{V}alfǫðr, \hld \alst{v}ęl fram tęlja’k \\%M
\alst{f}orn spjǫll \alst{f}ira, \hld þau’s \alst{f}ręmst of man?\eva

\bva Ek man jǫtna \hld ár of borna, \\%M
þá es forðum \hld mik fǿdda hǫfðu; \\%M
níu man’k hęima, \hld níu íviðjur\footnotemark[1], \\%M
mjǫtvið mæran \hld fyr mold neðan.\eva
\footnotetext[1]{Previously read \emph{íviði}, but closer study of R has disproven this. See \emph{Gripla} 3, pp. 227–28.}

\bva Ár vas alda \hld þar’s Ymir byggði,\footnotemark[1] \\%M
vas-a sandr né sær, \hld né svalar unnir; \\%M
jǫrð fansk æva \hld né upphiminn; \\%M
gap vas ginnunga, \hld ęn gras hvęrgi.\eva
\footnotetext[1]{\Gylfaginning\ \emph{þat’s ekki vas}.}

\bva Áðr Burs synir \hld bjǫðum of ypðu, \\%M
þęir es Miðgarð \hld mæran skópu; \\%M
sól skein sunnan \hld á salar stęina; \\%M
þá vas grund gróin \hld grǿnum lauki.\eva

\bva Sól varp sunnan, \hld sinni mána,\footnotemark[1] \\%M
hęndi hinni hǿgri \hld um himinjǫður; \\%M
sól þat né vissi, \hld hvar hon sali átti; \\%M
(stjǫrnur þat né vissu, \hld hvar þær staði ǫ́ttu); \\%M
máni þat né vissi, \hld hvat hann męgins átti.\eva
\footnotetext[1]{At times translated as “its moon”; this cannot be correct, as \emph{máni} ‘moon’ is masculine, while \emph{sinni}, dative singular of \emph{sínn} ‘its (reflexive)’ is feminine.}

\bva Þá gingu ręgin ǫll \hld á rǫkstóla, \\%M
ginnhęilǫg goð, \hld ok gættusk umb þat.\footnotemark[1]\eva
\footnotetext[1]{9:1–4, 23:1–4, 25:1–4 would suggest two \inx{long-lines} be missing here.}

\bva Nótt ok niðjum \hld nǫfn of gǫ́fu, \\%M
morgin hétu \hld ok miðjan dag, \\%M
undurn ok aptan, \hld ǫ́rum at tęlja.\footnotemark[1]\eva
\footnotetext[1]{Cf. \emph{Web} 23, 25.}

\bva Hittusk æsir \hld á Iðavęlli, \\%M
þęir’s hǫrg ok hof \hld hótimbruðu; \\%M
afla lǫgðu, \hld auð smíðuðu, \\%M
tangir skópu \hld ok tól gęrðu.\eva

\bva Tęflðu í túni, \hld tęitir vǫ́ru, \\%M
vas þeim véttugis \hld vant ór golli, \\%M
unz þríar kvǫ́mu \hld þursa męyjar, \\%M
ámátkar mjǫk, \hld ór Jǫtunhęimum.\eva

\bva — Þá gingu ręgin ǫll \hld á rǫkstóla, \\%M
ginnhęilǫg goð, \hld ok gættusk umb þat: \\%M
hverr skyldi dverga \hld dróttir skępja \\%M
ór Brimis blóði \hld ok ór Bláins lęggjum?\footnotemark[1]\eva
\footnotetext[1]{The final two long-lines vary substantially. \Regius\ has \emph{hverr scyldi duerga drotin scepia or brimis bloði oc or blám leggiom.} \Hauksbok\ has \emph{huerer skylldu duergar drottir skepia or brimi bloðgv ok or Blains leggivm.}}

\bva Þar vas Móðsognir\footnotemark[1] \hld mæztr of orðinn \\%M
dverga allra, \hld en Durinn annarr; \\%M
þęir manlíkun \hld mǫrg of gęrðu, \\%M
dvergar í jǫrðu, \hld sęm Durinn sagði.\eva
\footnotetext[1]{R. \emph{mótsognir}, H. \emph{móðsognir}}

\bva Nýi ok Niði, \hld Norðri, Suðri, \\%M
Austri, Vestri, \hld Alþjófr, Dvalinn, \\%M
Bívurr, Bávurr, \hld Bǫmburr, Nóri, \\%M
Ánn ok Ánarr, \hld Ái, Mjǫðvitnir.\footnotemark[1]\eva
\footnotetext[1]{The three following verses seem to belong together, since there is no repetition of names. From the last verse of the middle one, it seems that it should have been placed at the end of the list.}

\bva Vęigr ok Gandalfr, \hld Vindalfr, Þráinn, \\%M
Þękkr ok Þorinn, \hld Þrór, Vitr ok Litr, \\%M
Nár ok Nýráðr, \hld nú hęf’k dverga, \\%M
Ręginn ok Ráðsviðr, \hld rétt of talða.\eva

\bva Fíli, Kíli, \hld Fundinn, Náli, \\%M
Hęptifíli, \hld Hannarr, Svíurr, \\%M
Frár, Hornbori, \hld Frægr ok Lóni, \\%M
Aurvangr, Jari, \hld Ęikinskjaldi.\eva

\bva — Mál es dverga \hld í Dvalins liði \\%M
ljóna kindum \hld til Lofars tęlja, \\%M
þęir es sóttu \hld frá salar stęini \\%M
Aurvanga sjǫt \hld til Jǫruvalla.\footnotemark[1]\eva
\footnotetext[1]{From the repeated names (Ęikinskjaldi, Ái), and the out-of-place introduction (\emph{mál es dverga} “now is to be spoken of dwarves”), it is clear that this verse and the following are originally separate from the previous three, and are an addition to the original poem.}

\bva Þar vas Draupnir \hld ok Dolgþrasir, \\%M
Hár, Haugspori, \hld Hlévangr, Glói, \\%M
Skirfir, Virfir, \hld Skáfiðr, Ái, \\%M
Alfr ok Yngvi, \hld Ęikinskjaldi, \\%M
Fjalarr ok Frosti, \hld Finnr ok Ginnarr; \\%M
Þat mun æ uppi, \hld meðan ǫld lifir, \\%M
langniðja-tal \hld til Lofars hafat.\eva

\bva — Unz þrír kvǫ́mu \hld ór því liði \\%M
ǫflgir ok ástkir \hld æsir at húsi; \\%M
fundu á landi \hld lítt męgandi \\%M
Ask ok Emblu \hld ørlǫglausa.\eva

\bva Ǫnd þau né ǫ́ttu, \hld óð þau né hǫfðu, \\%M
lǫ́ né læti \hld né litu góða; \\%M
ǫnd gaf Óðinn, \hld óð gaf Hǿnir, \\%M
lǫ́ gaf Lóðurr \hld ok litu góða.\eva

\bva Ask vęit’k standa, \hld hęitir Yggdrasill, \\%M
hǫ́r baðmr, ausinn \hld hvíta auri; \\%M
þaðan koma dǫggvar \hld þær’s í dala falla; \\%M
stęndr æ yfir grǿnn \hld Urðar brunni.\eva

\bva Þaðan koma męyjar \hld margs vitandi \\%M
þríar ór þeim sæ\footnotemark[1], \hld es und þolli stendr; \\%M
Urð hétu ęina, \hld aðra Verðandi, \\%M
skǫ́ru á skíði, \hld Skuld hina þriðju \\%M
þær lǫg lǫgðu, \hld þær líf køru, \\%M
alda bǫrnum, \hld ørlǫg sęggja.\eva
\footnotetext[1]{\Hauksbok\ \emph{sal}}

\bva — Þat man hón folkvíg \hld fyrst í hęimi, \\%M
es Gollvęigu \hld gęirum studdu \\%M
ok í hǫll Háars \hld hána bręndu, \\%M
þrysvar bręndu \hld þrysvar borna, \\%M
(opt ósjaldan, \hld þó hon ęnn lifir).\footnotemark[1]\eva
\footnotetext[1]{Here begins the introduction of the wallow of the title.}

\bva Hęiði hétu, \hld hvar’s til húsa kom, \\%M
vǫlu vęlspáa, \hld vitti hon ganda; \\%M
sęið, hvars kunni, \hld sęið hug lęikinn; \\%M
æ vas hon angan \hld illrar brúðar.\eva

\bva Þá gingu ręgin ǫll \hld á rǫkstóla, \\%M
ginnhęilǫg goð, \hld ok gættusk umb þat: \\%M
hvárt skyldi æsir \hld afráð gjalda, \\%M
eða skyldi goð ǫll \hld gildi ęiga?\eva

\bva Flęygði Óðinn \hld ok í folk of skaut; \\%M
þat vas ęnn folkvíg \hld fyrst í hęimi; \\%M
brotinn vas borðvęggr \hld borgar ása, \\%M
knǫ́ttu vanir vígspǫ́ \hld vǫllu sporna.\eva

\bva Þá gingu ręgin ǫll \hld á rǫkstóla, \\%M
ginnhęilǫg goð, \hld ok gættusk umb þat: \\%M
hvęrr hęfði lopt alt \hld lævi blandit \\%M
eða ætt jǫtuns \hld Óðs męy gefna.\eva

\bva Þórr ęinn þar vá \hld þrunginn móði, \\%M
hann sjaldan sitr, \hld es slíkt of fregn; \\%M
á gingusk ęiðar, \hld orð ok sǿri, \\%M
mǫ́l ǫll męginlig, \hld es á meðal fóru.\eva

\bva Vęit hon Hęimdallar \hld hljóð of folgit \\%M
und hęiðvǫnum \hld hęlgum baðmi; \\%M
á sér hon ausask \hld aurgum forsi \\%M
af veði Valfǫðrs. \hld Vituð ér ęnn eða hvat?\eva

\bva Ęin sat hon úti, \hld þá’s hinn aldni kom \\%M
yggjungr ása \hld ok í augu lęit — \\%M
»hvęrs fregnið mik? \hld hví fręistið mín? \\%M
Alt vęit’k, Óðinn, \hld hvar auga falt \\%M
í hinum mæra \hld Mímis brunni;« \\%M
drekkr mjǫð Mímir \hld morgin hvęrjan \\%M
af veði Valfǫðrs. \hld Vituð ér ęnn eða hvat?\eva

\bva Valði hęnni Hęrfǫðr \hld hringa ok męn; \\%M
fekk spjǫll spaklig \hld ok spáganda; \\%M
sá vítt ok of vítt \hld of verǫld hvęrja.\eva

\bva Sá hon valkyrjur \hld vítt of komnar, \\%M
gǫrvar at ríða \hld til goðþjóðar. \\%M
Skuld hęlt skildi, \hld ęn Skǫgul ǫnnur, \\%M
Gunnr, Hildr, Gǫndul \hld ok Gęirskǫgul; \\%M
nú eru talðar \hld nǫnnur Hęrjans, \\%M
gǫrvar at ríða \hld grund valkyrjur.\eva


\bva Ek sá Baldri, \hld blóðgum tívur, \\%M
Óðins barni, \hld ørlǫg folgin; \\%M
stóð of vaxinn \hld vǫllum hæri \\%M
mjór ok mjǫk fagr \hld mistiltęinn.\eva

\bva Varð af męiði, \hld þęim’s mær sýndisk, \\%M
harmflaug hættlig, \hld Hǫðr nam skjóta. \\%M
Baldrs bróðir vas \hld of borinn snimma, \\%M
sá nam, Óðins sonr, \hld ęinnættr vega;\eva

\bva þó hann æva hęndr \hld né hǫfuð kęmbði, \\%M
áðr á bál of bar \hld Baldrs andskota. \\%M
Ęn Frigg of grét \hld í Fęnsǫlum \\%M
vǫ́ Valhallar. \hld Vituð ér ęnn eða hvat?\eva

\bva Hapt sá hon liggja \hld und Hveralundi \\%M
lægjarns líki \hld Loka áþękkjan; \\%M
þar sitr Sigyn \hld þęygi of sínum \\%M
veri vęl glýjuð. \hld Vitud ér ęnn eða hvat?\eva

\bva Ǫ́ fęllr austan \hld of ęitrdala \\%M
sǫxum ok sverðum, \hld Slíðr heitir sú.\eva

\bva Stóð fyr norðan \hld á Niðavǫllum \\%M
salr ór golli \hld Sindra ættar, \\%M
ęn annarr stóð \hld á Ókólni, \\%M
bjórsalr jǫtuns, \hld ęn sá Brimir hęitir.\eva

\bva Sal sá hon standa \hld sólu fjarri \\%M
Nástrǫndu á, \hld norðr horfa dyrr; \\%M
falla ęitrdropar \hld inn um ljóra, \\%M
sá ’s undinn salr \hld orma hryggjum.\eva

\bva Sér hón þar vaða \hld þunga strauma \\%M
męnn męinsvara \hld ok morðvarga \\%M
ok þann’s annars glępr \hld ęyrarúnu. \\%M
Þar sýgr Níðhǫggr \hld nái framgingna; \\%M
slítr vargr vera. \hld Vituð ér ęnn eða hvat?\eva

\bva Austr sat hin aldna \hld í Járnviði \\%M
ok fǿddi þar \hld Fęnris kindir; \\%M
verðr af þeim ǫllum \hld ęinna nøkkurr \\%M
tungls tjúgari \hld í trolls hami.\eva

\bva Fyllisk fjǫrvi \hld fęigra manna, \\%M
rýðr ragna sjǫt \hld rauðum dręyra, \\%M
svǫrt var þá sólskin \hld um sumur ęptir, \\%M
veðr ǫll válynd. \hld Vituð ér ęnn eða hvat?\eva

\bva Sat þar á haugi \hld ok sló hǫrpu \\%M
gýgjar hirðir, \hld glaðr Ęggþér; \\%M
gól of hǫ́num \hld í gaglviði \\%M
fagrrauðr hani, \hld sá’s Fjalarr hęitir.\eva

\bva Gól of ǫ́sum \hld Gollinkambi, \\%M
sá vękr hǫlða \hld at Hęrjafǫðrs, \\%M
ęn annarr gęlr \hld fyr jǫrð neðan \\%M
sótrauðr hani \hld at sǫlum Hęljar.\eva

\bva Gęyr Garmr mjǫk \hld fyr Gnipahęlli, \\%M
fęstr mun slitna, \hld ęn Freki rinna. \\%M
fjǫlð vęit hǫ́n frǿða, \hld framm sé’k lęngra \\%M
of ragna rǫk, \hld rǫmm sigtíva.\eva

\bva Brǿðr munu bęrjask \hld ok at bǫnum verða, \\%M
munu systrungar \hld sifjum spilla, \\%M
hart ’s í hęimi, \hld hórdómr mikill, \\%M
skęggǫld, skalmǫld, \hld skildir ’ró klofnir, \\%M
vindǫld, vargǫld, \hld áðr verǫld stęypisk, \\%M
mun ęngi maðr \hld ǫðrum þyrma.\eva

\bva Lęika Míms synir, \hld ęn mjǫtuðr kyndisk \\%M
at hinu galla \hld Gjallarhorni \\%M
hótt blæss Hęimdallr, \hld horn ’s á lopti; \\%M
mælir Óðinn \hld við Míms hǫfuð.\eva

\bva Skęlfr Yggdrasils \hld askr standandi, \\%M
ymr aldit tré, \hld ęn jǫtunn losnar; \\%M
hræðask allir \hld á hęlvegum \\%M
áðr Surtar þann \hld sefi of glęypir.\eva

\bva Hvat ’s með ǫ́sum? \hld hvat ’s með ǫlfum? \\%M
gnýr allr Jǫtunhęimr, \hld æsir ’ro á þingi, \\%M
stynja dvergar \hld fyr stęindurum \\%M
vęggbergs vísir — \hld vituð ér ęnn eða hvat?\eva

\bva Gęyr nú Garmr mjǫk \hld fyr Gnipahęlli, \\%M
fęstr mun slitna, \hld ęn freki rinna, \\%M
fjǫlð vęit’k frǿða, \hld framm sé’k lęngra \\%M
of ragna rǫk, \hld rǫmm sigtíva.\eva

\bva Hrymr ękr austan, \hld hęfsk lind fyrir, \\%M
snýsk Jǫrmungandr \hld í jǫtunmóði; \\%M
ormr knýr unnir, \hld ęn ari hlakkar, \\%M
slítr nái niðfǫlr; \hld Naglfar losnar.\eva

\bva Kjóll fęrr austan \hld koma munu Múspells \\%M
of lǫg lýðir, \hld ęn Loki stýrir; \\%M
fara fíflmęgir \hld með freka allir, \\%M
þęim es bróðir \hld Býlęists í fǫr.\eva

\bva Surtr\footnotemark[1] fęrr sunnan \hld með sviga lævi, \\%M
skínn af sverði \hld sól valtíva; \\%M
grjótbjǫrg gnata, \hld ęn gífr\footnotemark[2] rata, \\%M
troða halir hęlveg, \hld ęn himinn klofnar.\eva
\footnotetext[1]{\Snorri\ \emph{Svartr}}
\footnotetext[2]{\Snorri\ \emph{guðar} ‘gods’.}

\bva Þá kømr Hlínar \hld harmr annarr framm, \\%M
es Óðinn fęrr \hld við ulf vega, \\%M
ęn bani Bęlja \hld bjartr at Surti; \\%M
þá mun Friggjar \hld falla angan.\eva

\bva Þá kømr hinn mikli \hld mǫgr Sigfǫður, \\%M
Víðarr vega \hld at valdýri; \\%M
lætr hann męgi Hveðrungs \hld mund of standa \\%M
hjǫr til hjarta; \hld þá ’s hefnt fǫður.\eva

\bva Þá kømr hinn mæri \hld mǫgr Hlǫðynjar \\%M
gęngr Óðins sonr \hld ormi mǿta. \\%M
Drepr af móði \hld Miðgarðs véurr; \\%M
munu halir allir \hld hęimstǫð ryðja; \\%M
gęngr fet níu \hld Fjǫrgynjar burr \\%M
nęppr frá naðri, \hld níðs ókvíðinn.\footnotemark[1]\eva
\footnotetext[1]{This verse, from what we can read in the very damaged part of the manuscript, is very different in \Hauksbok: \emph{[...] [fet niu] / [...] [ior] [...] / [nepr fra] naðr [...] / [...] / munv halir [al] [...] / [...] ydia / [er] [...] / mið [...]}. From what we have in this version, we might translate something like: “[The son of F]er[gin goes] nine paces, pained, away from [the loathsome] adder. [All] men will [their homesteads e]mpty, when Mid[denyard’s wigh-guardian] (falls?).” Particularly the last line is very uncertain. TODO: check scans of the original manuscript.}

\bva Sól tér sortna, \hld søkkr fold í mar, \\%M
hverfa af himni \hld hęiðar stjǫrnur; \\%M
gęisar ęimi \hld við aldrnara; \\%M
lęikr hór hiti \hld við himin sjalfan.\eva

\bva Gęyr Garmr mjǫk \hld fyr Gnipahęlli, \\%M
fęstr mun slitna, \hld ęn freki rinna, \\%M
fjǫlð vęit’k frǿða, \hld framm sé’k lęngra \\%M
of ragna rǫk, \hld rǫmm sigtíva.\eva

\bva Sér hon upp koma \hld ǫðru sinni \\%M
jǫrð ór ægi \hld iðjagrǿna —; \\%M
falla forsar, \hld flýgr ǫrn yfir, \\%M
sás á fjalli \hld fiska vęiðir.\eva

\bva Finnask æsir \hld á Iðavęlli \\%M
ok of moldþinur \hld mǫ́tkan dǿma, \\%M
ok minnask þar \hld á męgindóma \\%M
ok á Fimbultýs \hld fornar rúnar.\eva

\bva Þar munu ęptir \hld undrsamligar \\%M
gollnar tǫflur \hld í grasi finnask, \\%M
þær’s í árdaga \hld áttar hǫfðu.\eva

\bva Munu ósánir \hld akrar vaxa; \\%M
bǫls mun alls batna \hld mun Baldr koma; \\%M
búa Hǫðr ok Baldr \hld Hropts sigtoptir \\%M
(vęl valtívar, \hld Vituð ér ęnn eða hvat?)\eva

\bva Þá kná Hǿnir \hld hlautvið kjósa \\%M
ok burir byggva \hld brǿðra Tvęggja \\%M
vindhęim víðan. \hld Vituð ér ęnn eða hvat?\eva

\bva Sal sér hon standa \hld sólu fęgra, \\%M
golli þakðan, \hld á Gimléi; \\%M
þar munu dyggvar \hld dróttir byggva \\%M
ok of aldrdaga \hld ynðis njóta.\eva

\bva Þar kømr hinn dimmi \hld dręki fljúgandi, \\%M
naðr fránn neðan \hld frá Niðafjǫllum; \\%M
berr sér í fjǫðrum \hld — flýgr vǫll yfir — \\%M
Níðhǫggr nái; \hld nú mun hón søkkvask.\eva
