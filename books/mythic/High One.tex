\bookStart{Speeches of the High One}[Háva mǫ́l]
\setBookCode{Havamal}

\begin{flushright}%
\textbf{Dating:} See individual sections.

\textbf{Meter:} \Ljodahattr\ (2–61/2, 62–72, 74/4–79/4, 84, 88, 91–105/2, 106–108, 109/3–111/4, 112/4–5, 113–4/5), \Galdralag\ (1/1–3, ?61/3–5, 74/1–3, 80, 105/3–5, 111/5–112/3, 113/1–3), \Malahattr\ (73, 81–83, 85–87, 89–90, 109/1–2)
\end{flushright}%

\section{Introduction}

\subsection{Preservation}

The \textbf{Speeches of the High One} (abbrev. \textlink{Havamal}) is the second poem of \Regius, where it follows \textlink{Voluspa} and is followed by \textlink{Vafthrudnismal}.  \Regius\ is the only mediæval manuscript witness for the whole poem, but several sts. (e.g. 1, 58, 84, 76–77) are cited or alluded to in other texts.

\subsection{Contents}

\textlink{Havamal} is, as it comes down to us in \Regius, a varied collection.  It contains two or three poems of practical life advice, two mythological narratives, scattered gnomic poetry about runes, and a list of galders.  These materials are chiefly united by their attribution to the high god Weden, or as he is called in 109, 111, and 166, \emph{Hǫ́vi} ‘the High One’.

Following philological tradition I identify the following major strands, excluding various isolated sts. (e.g. 80) that are probably later inserts.  In the present edition each is given its own short introduction:

\begin{enumerate}
	\item The Guests’ Strand (1–77)
  \item Various scattered sts. of advice (81–90)
  \item Weden’s tryst with Billing’s daughter (91–102)
  \item Weden’s obtaining of the Mead of Poetry (103–110)
  \item The Speeches of Loddfathomer (111–137)
  \item The Rune-tally; sts. about runes and ritual (138–146)
  \item The Leed-tally; Weden’s listing of 18 galders (146–165)
\end{enumerate}

It cannot be claimed for certain that each strand was originally its own poem.  Weden’s two romantic adventures (91–102, 103–110), for instance, have much in common stylistically and seem too short to stand on their own.

On the other hand it seems highly unlikely that the Guests’ Strand (1–77) and the Speeches of Loddfathomer (111–137) were originally united.  They differ greatly
in tone—the former being down to earth and sceptical, the latter placing great emphasis on magical or even superstitious ideas;
in style—the former never making use of the second imperative (and only thrice of the verb \emph{skalt} ‘(thou) shalt’, 44–46), the latter very frequently;
and in structure—the former having a perfectly fitting conclusion in sts. 76–77, the latter being much more varied and concluding with a long list of folk magical remedies (st. 137).  There is also some repetition between them (most notably sts. 44, 119), which would seem rather redundant if both were originally a single work.

Since the full \textlink{Havamal}, then, appears to consist of at least a few originally separate compositions, two questions naturally arise: \emph{how} were these materials redacted into a single poem, and \emph{why}?  Any answers must needs be speculative, and so the following is only my speculation.

To answer either question, we first need to determine in what context the redaction took place; whether in an oral or scribal tradition, in a Heathen hove or a Catholic monastery.  St. 166, given that it explicitly mentions the title of the poem, must probably be associated with the final layer of the redaction, and is thus of particular use.  Its blessing of reciter, hearers, and learners indicates that the poem was to be chanted and learned by heart, and its description of the contents of the poem (which includes unambiguous Heathen ritual advice like st. 145) as \emph{all-þǫrf} ‘most useful’ to Men and \emph{ȯ·þǫrf} ‘harmful’ to Ettins invokes the Heathen dichotomy between the Gods and Ettins as friends and enemies of Mankind, respectively.  With this in mind, the poem was probably redacted into something very close to its present form no later than the early 11th century, in an oral, Heathen context.

Moving on to the \emph{how}, it is certain that in an oral transmission additions and inserts need not have happened all at once, but could have taken place successively in the form of layers appended to an original core.  Thus the original Guests’ Strand probably ended at st. 77, but sts. 78–90 may have been added shortly afterwards, later the two narratives about Weden’s romantic escapades, thereafter the Speeches of Loddfathomer, the Rune-tally, and the Leeds.  Even after the basic structure was obtained, stanzas such as 73 could have been inserted where they were felt most fitting in order to make the poem more “complete” in the eyes of the inserter.  These inserts may well have continued into the period of scribal transmission.

For the \emph{why}, we should consider what reason someone would have for redacting numerous materials into a single poem.  St. 166, as discussed above, suggests that the main reason was utilitarian rather than antiquarian, and a picture then emerges of a redactor arranging a corpus of traditional poetry, selected both for its traditional attribution to the god Weden and for its \emph{usefulness}—whether as practical life advice or as mythological and religious lore—into a single long poem meant to be learned by heart as a whole and recited for magical purposes.  In practice this final redaction served as sort of Odinic “ark” (or “Hoardmimer’s wood”) in which the bulk of surviving pre-Christian Norse advice poetry was transmitted until it could be written down.  Forever lost were whichever stanzas were not included in it—and many such must have existed.

\subsection{Dating}

Having determined that the redaction of the poem most likely took place in pagan times we should look at more detailed data.

On the purely linguistic side \Havamal\ (or at least the Guests’ Strand) is highly archaic.  Old \emph{vr-} alliterates with \emph{v-} 2 times in non-formulaic word pairs (sts. 26/2, 32/2) and the particle \emph{of} is highly frequent, including 3 times before nominals (4/3: \emph{of ǿðis}, 21/4: \emph{of mál}, 38/4: \emph{of þǫrf}).

Content-based criteria have already been discussed above, but a few more can be mentioned shortly.  As \textcite{Males2024} points out, the poem twice treats being “burned” as synonymous with being “dead” (71, 81).  This convincingly suggests a pagan date, since cremation was strongly taboo in mediæval Christianity and reserved for criminals and heretics (cf. \textlink{Atlamal}[84] n.), and as such the neutral statement of “praising a woman when she is burned” could hardly have been composed by a Christian wishing to give advice.

\newpage

\section{Text}

\subsection{The Guests’ Strand (1–79)}

The Guests’ Strand (Old Norse: \emph{Gęsta-þáttr}) is a wisdom poem, taking its outset in the scene of a lone wanderer’s arriving as a stranger at a farmstead.  It begins by discussing the mutual responsibilites between guest and host, before moving on to describing proper conduct in broader human interactions with a particular focus on drinking, speech, and friendship.

While there are some fine transitions employed in order to move from one theme to another (e.g. between sts. 4–5, or 10–11), there is no clear division into thematic sections, and previous subjects often appear again after having been dropped for a few stanzas.  The spirit of the advice is in any case very consistent throughout, and the poem at all turns advices caution and shrewdness.  Of particular importance is the idea of “manwit” (ON \emph{man-vit}), a word somewhat analogous with the English “common sense” or “street wisdom”.

It seems very likely that the original Guests’ Strand ended at st. 77.  This finds strong support in \Hakonarmal\ 21, the final st. of that poem, which likewise begins with the first two lines \emph{dęyr fé \hld\ dęyja frę́ndr}.

\bvg\bva\mssnote{\Regius~3r/4}%
\alst{G}ȧttir allar \hld\ áðr \alst{g}angi framm &
\ind \edtext{of \alst{sk}oða⸗sk \alst{sk}yli,}{\Bfootnote{om. \RegiusProse\Trajectinus\Upsaliensis\Wormianus}} &
\ind of \alst{sk}yggna⸗sk \alst{sk}yli; &
því’t \alst{ȯ}·víst ’s at vita, \hld\ hvar \alst{ȯ}·vinir &
\ind sitja ȧ \alst{f}lęti \alst{f}yrir.\eva

\bvb {\huge A}\textsc{ll doorways}—before one might go forth— \\
\ind he should spy round; \\
\ind he should pry round, \\
for it is unsure to know where enemies \\
\ind sit on the benches within.\evb\evg


\bvg\bva\mssnote{\Regius~3r/6}%
\Ballnote{The speaker announces to the hosts (the “givers”) that a guest, frozen, wet and tired, is sitting outside waiting to be let in.  With this stanza the frame of the guest arriving at a farmstead begins; it is from this scenario that the following advice gradually branches out.}%
\alst{G}efęndr hęilir, \hld\ \alst{g}ęstr ’s inn kominn, &
\ind hvar skal \alst{s}itja \alst{s}já? &
mjǫk es \alst{b}ráðr \hld\ sá’s \edtrans{ȧ \alst{b}rǫndum}{on the wood-pile}{\Bfootnote{Many interpretations have been offered for this line, chiefly owing to the varied senses of ON \emph{brandr} ‘fire, firewood, sword’.  I agree with \textcite[77]{Evans1986} that the best explanation is attained by reference to a Norwegian folk custom attested from much later times where the guest would \emph{sitja i brondo}, i.e., sit down on the wood-pile outside the door and wait until being let in.}} skal &
\ind \edtrans{sïns of \alst{f}ręista \alst{f}rama}{test his furtherance}{\Bfootnote{Try his luck, see how far he gets.  The same line is also found in \textlink{Vafthrudnismal}[11], 13, 15, 17.}}.\eva

\bvb O givers, hail! A guest is come in; \\
\ind where shall this one sit? \\
Most anxious is he who on the wood-pile shall \\
\ind test his furtherance.\evb\evg


\bvg\bva\mssnote{\Regius~3r/8}%
\alst{Ę}lds es þǫrf \hld\ þęim’s \alst{i}nn es kominn &
\ind ok ȧ \alst{k}néi \alst{k}alinn, &
\alst{m}atar ok váða \hld\ es \alst{m}anni þǫrf, &
\ind þęim’s hęfr of \alst{f}jall \alst{f}arit.\eva

\bvb Of fire there is need for him who has come inside \\
\ind and is cold about his knees. \\
Of food and of clothing there is need for the man \\
\ind who over the fell has fared.\evb\evg


\bvg\bva\mssnote{\Regius~3r/10}%
\Ballnote{There is a good train of thought throughout the st.: the guest must first wash and dry himself, and then be welcomed to sit and eat at the table.  After the host has furnished him with these amenities the need for proper conduct now shifts onto the guest, who must speak and speak wisely.}%
\alst{V}ats es þǫrf \hld\ þęim’s til \alst{v}erðar kømr, &
\ind \alst{þ}ęrru ok \alst{þ}jóð-laðar, &
\alst{g}óðs \edtrans{of ǿðis}{mood}{\Bfootnote{An instance of prenominal \emph{of}, a rare feature.  \emph{of ǿði} appears once in a Scaldic poem, Egill \emph{Arkv} 2 (in \Skp\ 5).  Cf. st. 21/4 below.}}, \hld\ —ef sér \alst{g}eta mę́tti— &
\ind \alst{o}rðs ok \edtrans{\alst{ę}ndr-þǫgu}{silence in return}{\Bfootnote{One may note that the verb \emph{þęgja} ‘shut up, hush, be silent’—of which \emph{*þaga}, which only appears in the present cpd., is a derivative formed in the same way as \emph{saga} ‘saw, history’ to \emph{sęgja} ‘say, speak’—and the related noun \emph{þǫgn} ‘silence’ are frequently used at the beginning of Scaldic poems (e.g. Arn \emph{Magndr} 1: \emph{þegi sęim-brotar} ‘may gold-breakers \ken{generous men} be silent’, Egill \emph{Berdr} 1: \emph{hyggi \dots\ til þagnar þïnn lýðr} ‘may thy retinue focus on silence’,
Glúmr \emph{Gráf} 1: \emph{biðjum vér þagnar} ‘we ask for silence’).}}.\eva

\bvb Of water there is need for him who comes for a meal, \\
\ind of a towel and a hearty welcome, \\
of a good mood—if he might get it— \\
\ind of a word, and of silence in return.\evb\evg


\bvg\bva\mssnote{\Regius~3r/12}%
\alst{V}its es þǫrf \hld\ þęim’s \alst{v}íða ratar; &
\ind dę́lt es \alst{h}ęima \alst{h}vat; &
\edtrans{at \alst{au}ga-bragði}{Into a laughing-stock}{\Bfootnote{Idomatic.  \emph{auga-bragð} literally means ‘twinkling of an eye, moment’; the sense here is thus something like ‘a quick glance of derision’.}} \hld\ verðr sá’s \alst{ę}kki kann &
\ind ok með \alst{s}notrum \alst{s}itr.\eva

\bvb Of wit there is need for him who widely roams; \\
\ind everything is easy at home. \\
Into a laughing-stock turns he who nothing knows, \\
\ind and among the clever sits.\evb\evg


\bvg\bva\mssnote{\Regius~3r/14}%
At \alst{h}yggjandi sïnni \hld\ skyli⸗t maðr \alst{h}rǿsinn vesa, &
\ind hęldr \alst{g}ę́tinn at \alst{g}ęði, &
þȧ’s \alst{h}orskr ok þǫgull \hld\ kømr \alst{h}ęimis-garða til, &
\ind sjaldan verðr \alst{v}íti \alst{v}ǫrum. &
því’t \alst{ȯ}·brigðra vin \hld\ fę̇r \edtrans{maðr}{man}{\Bfootnote{In \Regius\ abbreviated with the rune ᛘ \textbf{m} “man”, the first of 45 such instances in the present poem.  Whereas Anglo-Saxon Latin-script mss. use several runes ideographically (e.g. ᛟ \textbf{o} for OE \emph{ǿðel} ‘homeland, patrimony’), there do not seem to be any Scandinavian examples with runes other than ᛘ.  The tradition of ideographic runes goes back to the Runic period itself, as shown by the pre-Christian inscriptions from Stentoften (DR 357) and Ingelstad (Ög 43); DR 357 uses the rune ᛃ \textbf{j} for \emph{ár} ‘year, good harvest’ and Ög 43 uses ᛞ \textbf{d} for \emph{dagʀ} ‘day’.  For the names of the runes see the Three Rune Poems, edited below under Miscellaneous Runic Poetry.}} \alst{a}ldri⸗gi, &
\ind an \alst{m}an-vit \alst{m}ikit.\eva

\bvb Of his thinking should man not be boastful, \\
\ind but rather guarding of his senses \\
when sharp and silent he comes to a homestead; \\
\ind sudden harm seldom strikes the wary, \\
for an unfickler friend man never gets \\
\ind than great \inx[C]{manwit}.\evb\evg


\bvg\bva\mssnote{\Regius~3r/17}%
Hinn \alst{v}ari gęstr \hld\ es til \alst{v}erðar kømr, &
\ind \edtrans{\alst{þ}unnu hljóði}{with sharp hearing}{\Bfootnote{Or ‘with thin silence’.}} \alst{þ}ęgir; &
\alst{ęy}rum hlýðir, \hld\ en \alst{au}gum skoðar, &
\ind svá \edtext{nýsi⸗sk \alst{f}róðra hvęrr \alst{f}yrir}{\lemma{nýsi⸗sk fyrir ‘looks ahead’}\Bfootnote{This verb underlies the noun \emph{for·njósn} as found in \textlink{Sigrdrifumal}[25].}}.\eva

\bvb The wary guest who comes for a meal \\
\ind with sharp hearing shuts up. \\
With ears he listens and with eyes he watches; \\
\ind so looks each learned man ahead.\evb\evg


\bvg\bva\mssnote{\Regius~3r/19}%
Hinn es \alst{s}ę́ll, \hld\ es \alst{s}ér of getr &
\ind \edtrans{\alst{l}of ok \alst{l}íkn-stafi}{praise and staves of liking}{\Bfootnote{\emph{líkn} ‘liking’ is a very interesting word.  It is defined by \ONP\ as: ‘mercy, compassion, relief, comfort, help’.  In the present poem its precise meaning seems to be something like ‘the state of being liked by your surroundings to the point where people are willing to help you out’.  Cf. its two other occurrences in the present poem: sts. 120 and especially 123 (where it is likewise paired with \emph{lof} ‘praise’).}}; &
\alst{ȯ}·dę́lla ’s við þat, \hld\ es \alst{ęi}ga skal &
\ind \alst{a}nnars brjóstum \alst{ï}.\eva

\bvb This one is blessed, who for himself does get \\
\ind praise and staves of liking. \\
It is uneasy regarding that which one shall own \\
\ind in another man’s breast.\evb\evg


\bvg\bva\mssnote{\Regius~3r/20}%
\edtrans{\alst{S}á}{That one}{\Bfootnote{Contrasting with \emph{hinn} ‘this one’ in the previous stanza.}} es \alst{s}ę́ll, \hld\ es \alst{s}jalfr of á &
\ind \alst{l}of ok vit meðan \alst{l}ifir; &
því’t \alst{i}ll rǫ́ð \hld\ hęfr maðr \alst{o}pt þęgit &
\ind \alst{a}nnars brjóstum \alst{ó}r.\eva

\bvb That one is blessed, who himself does have \\
\ind praise and wits while he lives, \\
for ill counsel has man oft taken \\
\ind out of another man’s breast.\evb\evg


\bvg\bva\mssnote{\Regius~3r/22}%
\alst{B}yrði \alst{b}ętri \hld\ berr⸗at maðr \alst{b}rautu at, &
\ind an séi \alst{m}an-vit \alst{m}ikit; &
\alst{au}ði bętra \hld\ þykkir þat ï \alst{ȯ}·kunnum stað; &
\ind slíkt es \alst{v}á-laðs \alst{v}era.\eva

\bvb A better burden man bears not on the road \\
\ind than be it much manwit. \\
In an unknown place it seems better than wealth; \\
\ind such is the destitute man’s shelter.\evb\evg


\bvg\bva\mssnote{\Regius~3r/24}%
\alst{B}yrði \alst{b}ętri \hld\ berr⸗at maðr \alst{b}rautu at, &
\ind an séi \alst{m}an-vit \alst{m}ikit; &
\alst{v}eg-nest \alst{v}erra \hld\ vegr⸗a \edtrans{\alst{v}ęlli at}{on the plain}{\Bfootnote{Formulaic, the word \emph{vǫllr} ‘plain, (uncultivated) field’ is also used in sts. 38 and 49. It is easily understood that the wild heaths and plains of Iron Age Norway were particularly unsafe places where a traveller needed to keep his wits about him, lest he fall victim to robbers or murderers (so st. 38).}}, &
\ind an séi \alst{o}f·drykkja \alst{ǫ}ls.\eva

\bvb A better burden man bears not on the road \\
\ind than be it much manwit. \\
Worse way-provision he drags not along on the plain \\
\ind than a too great drink of ale.\evb\evg


\bvg\bva\mssnote{\Regius~3r/25}%
Es⸗a svá \alst{g}ótt, \hld\ sęm \alst{g}ótt kveða, &
\ind \alst{ǫ}l \alst{a}lda sonum; &
því’t \alst{f}ę́ra vęit, \hld\ es \alst{f}lęira drekkr, &
\ind sïns til \alst{g}ęðs \alst{g}umi.\eva

\bvb It is not so good as they call it good, \\
\ind ale, for the sons of men, \\
for the less he knows as the more he drinks \\
\ind man of his own sense.\evb\evg


\bvg\bva\mssnote{\Regius~3r/27}%
\edtrans{\alst{Ȯ}·minnis-hęgri}{Forgetfulness-heron}{\Bfootnote{Lit. ‘unmemory-heron’, the personification of drunkenness as a hovering bird.}} hęitir, \hld\ sá’s yfir \alst{ǫ}lðrum þrumir, &
\ind hann stelr \alst{g}ęði \alst{g}uma; &
\edtext{þęss \alst{f}ogls \alst{f}jǫðrum \hld\ ek \alst{f}jǫtraðr vas’k &
\ind ï \alst{g}arði \alst{G}unn·laðar.}{\lemma{þęss fogls fjǫðrum \hld\ ek fjǫtraðr vas’k ï garði Gunn·laðar ‘By that bird’s feathers I was fettered in Guthlathe’s yard.’}\Bfootnote{Weden stole the Mead of Poetry from Sutting’s daughter, Guthlathe, who was placed by her father to guard it.  For this myth see introduction to sts. 103–110 below.  In the version told in \Skaldskaparmal\ Weden does indeed drink all of the Mead, but soon spits it out again and shows no adverse effects.  On the other hand the conception behind the present stanza seems to be that the Mead has the drawbacks of normal alcohol.  If this is the case it might lend support to the theory that the Guests’ Strand and the later parts of \textlink{Havamal} were originally separate compositions, since no such drunkenness is found in \textlink{Havamal}[103]–110.  Cf. the following stanza.}}\eva

\bvb Forgetfulness-heron is he called who hovers over ale-feasts; \\
\ind he robs man of his senses. \\
By that bird’s feathers I was fettered \\
\ind in \inx[P]{Guthlathe}’s yard.\evb\evg


\bvg\bva\mssnote{\Regius~3r/29}%
\edtext{\alst{Ǫ}lr ek varð, \hld\ varð \alst{o}fr·ǫlvi, &
\ind at hins \alst{f}róða \alst{F}jalars}{\lemma{Ǫlr ek varð \dots\ at hins fróða Fjalars ‘Drunk I became \dots\ at the learned Fealer’s’}\Bfootnote{Possibly another reference to the Mead of Poetry, for Fealer was one of the two dwarfs who slew Quasher and made the mead.  Again the sense seems to be that Weden got drunk on it, but since Weden (in the attested versions of the myth) never meets the two dwarfs it may be metaphorical.  Fealer may also be a variant name of Sutting, Guthlathe’s father.}}; &
því es \alst{ǫ}lðr batst, \hld\ at \alst{a}ptr of hęimtir &
\ind hvęrr sitt \alst{g}ęð \alst{g}umi.\eva

\bvb Drunk I became—became the greatest drunkard— \\
\ind at the learned Fealer’s. \\
So that ale-feast is best where every man \\
\ind gets back to his senses.\evb\evg


\bvg\bva\mssnote{\Regius~3r/31}%
\alst{Þ}agalt ok hugalt \hld\ skyli \alst{þ}jóðans barn &
\ind ok \alst{v}íg-djarft \alst{v}esa; &
\alst{g}laðr ok ręifr \hld\ skyli \alst{g}umna hvęrr, &
\ind und’s sïnn \alst{b}íðr \alst{b}ana.\eva

\bvb Silent and thoughtful should the king’s child \\
\ind —and battle-bold—be. \\
Glad and cheerful should every man be \\
\ind until he suffers his bane.\evb\evg


\bvg\bva\mssnote{\Regius~3v/1}%
\Ballnote{The coward may have been spared pain by the spears, but he cannot avoid the suffering of infirm old age.  The subtext is that since death is unavoidable it is better to live an honourable life and die young than a cowardly one and die of old age.  A related concept is the negative view of the “straw-death” (TODO), that suffered by the old person who dies of an ailment in his bed.  A strong contempt for life is common in the heroic literature (cf. e.g. \textlink{Fafnismal}[10], \textlink{Atlakvida}[23]–27), written as it was to celebrate young kings and warriors, although \Havamal\ is uniquely nuanced in this regard (cf. sts. 69–71 below).}%
\alst{Ȯ}·snjallr maðr \hld\ hygg⸗sk munu \alst{ę}y lifa, &
\ind ef við \alst{v}íg \alst{v}ara⸗sk; &
en \alst{ę}lli gefr hǫ̇num \hld\ \alst{ę}ngi frið, &
\ind þó’tt hǫ̇num \alst{g}ęirar \alst{g}efi.\eva

\bvb The unvalorous man thinks he will forever live \\
\ind if he of war be wary, \\
but old age gives him no peace \\
\ind although the spears might give.\evb\evg


\bvg\bva\mssnote{\Regius~3v/3}%
\alst{K}ópir af-glapi, \hld\ es til \alst{k}ynnis kømr, &
\ind \alst{þ}yl⸗sk hann umb eða \alst{þ}rumir; &
allt es \alst{s}ęnn, \hld\ ef \alst{s}ylg of getr, &
\ind uppi ’s þȧ \alst{g}ęð \alst{g}uma.\eva

\bvb The oaf gapes when he comes to visit; \\
\ind he mumbles about or loiters. \\
All at once if a sip he gets \\
\ind exposed is the mind of the man.\evb\evg


\bvg\bva\mssnote{\Regius~3v/5}%
Sá ęinn \alst{v}ęit, \hld\ es \alst{v}íða ratar &
\ind ok \edtrans{hęfr \alst{f}jǫlð of \alst{f}arit}{has journeyed much}{\Bfootnote{Cf. \textlink{Vafthrudnismal}[3], 44, et.c., where Weden repeats: \emph{Fjǫlð ek fór, \hld\ fjǫlð fręistaða’k, // fjǫlð ek ręynda ręgin} ‘Much I journeyed, much I tried, much I tested the \inx[P]{Reins}.’}}, &
hvęrju \alst{g}ęði \hld\ stýrir \alst{g}umna hvęrr, &
\ind sá es \alst{v}itandi ’s \alst{v}its.\eva

\bvb He alone knows who widely roams \\
\ind and has journeyed much, \\
which sort of mind every man wields, \\
\ind who is knowing of his wits.\evb\evg


\bvg\bva\mssnote{\Regius~3v/7}%
\edtrans{\alst{H}aldi⸗t maðr ȧ kęri}{Man ought not to hold onto the cask}{\Bfootnote{Perhaps referring to a toast wherein the drinking vessel would be passed around in a circle and each recipient would drink in turn.  Such toasts were drunk for a long time in Northern Europe—indeed this is the origin of the Scandinavian toasting-word, \emph{skål} ‘prosit, cheers!’, lit. ‘bowl!’.  “Holding onto” the vessel (and not letting the next person drink) was surely seen as very rude; indeed, in 1519 a man in Jämtland was killed in an argument resulting from his refusal to pass on the bowl \parencite{Sjöberg1907}.  The sense is thus: “Do not refuse a toast when offered, but do not drink too much.”}}, \hld\ drekki þó at \alst{h}ófi mjǫð, &
\ind \edtrans{mę́li \alst{þ}arft eða \alst{þ}ęgi}{ought to speak the needful or shut up}{\Bfootnote{Formulaic, line occurs identically in \textlink{Vafthrudnismal}[10]/2.}}; &
\alst{ȯ}·kynnis þęss \hld\ váar þik \alst{ę}ngi maðr, &
\ind at þú gangir \alst{s}nimma at \alst{s}ofa.\eva

\bvb Man ought not to hold onto the cask; ought yet to drink mead in moderation; \\
\ind ought to speak the needful or shut up. \\
For that uncouthness will no man blame thee \\
\ind that thou go early to sleep.\evb\evg


\bvg\bva\mssnote{\Regius~3v/9}%
\alst{G}rǫ́ðugr halr, \hld\ nema \alst{g}ęðs viti, &
\ind \edtrans{\alst{e}tr sér \alst{a}ldr-trega}{eats himself a life-sorrow}{\Bfootnote{Or, ‘eats himself to death.’}}; &
opt fę̇r \alst{h}lǿgis, \hld\ es með \alst{h}orskum kømr, &
\ind \alst{m}anni hęimskum \alst{m}agi.\eva

\bvb The gluttonous man—unless he know his sense— \\
\ind eats himself a life-sorrow. \\
Oft the belly when among the sharp he comes \\
\ind brings the foolish man ridicule.\evb\evg


\bvg\bva\mssnote{\Regius~3v/11}%
\alst{H}jarðir þat vitu, \hld\ nę́r \alst{h}ęim skulu, &
\ind ok \alst{g}anga þȧ af \alst{g}rasi; &
en \alst{ȯ}·sviðr maðr \hld\ kann \alst{ę́}va-gi &
\ind sïns of \alst{m}ál \alst{m}aga.\eva

\bvb Herds know when home they shall turn \\
\ind and then part from the grass, \\
but the unwise man never knows \\
\ind his own belly’s measure.\evb\evg


\bvg\bva\mssnote{\Regius~3v/13}%
\Ballnote{The sense is that the wretched man laughs at the flaws of others, not realizing that these are found just as much in himself.}%
\alst{V}e-sall maðr \hld\ ok \alst{i}lla skapi &
\ind \alst{h}lę́r at \alst{h}ví-vetna; &
hitt⸗ki hann \alst{v}ęit, \hld\ es \alst{v}ita þyrpti, &
\ind at \edtrans{hann es⸗a \alst{v}amma \alst{v}anr}{he is not free of blemishes}{\Bfootnote{Formulaic, cf. \textlink{Lokasenna}[30]: \emph{es⸗a þér vamma vant} ‘thou art not free of blemishes’.}}.\eva

\bvb The wretched man and ill turned out \\
\ind laughs at anything. \\
This he knows not, which he would need to know: \\
\ind that he is not free of blemishes.\evb\evg


\bvg\bva\mssnote{\Regius~3v/14}%
\alst{Ȯ}·sviðr maðr \hld\ vakir umb \alst{a}llar nę́tr &
\ind ok \alst{h}yggr at \alst{h}ví-vetna; &
þȧ es \alst{m}óðr, \hld\ es at \alst{m}orni kømr; &
\ind alt es \alst{v}íl sęm \alst{v}as.\eva

\bvb The unwise man is awake for all nights \\
\ind and thinks of anything. \\
Then he is weary when the morning comes; \\
\ind all the trouble is as it was.\evb\evg


\bvg\bva\mssnote{\Regius~3v/16}%
\alst{Ȯ}·snotr maðr \hld\ hyggr sér \alst{a}lla vesa &
\ind \alst{v}ið-hlę́jęndr \alst{v}ini; &
hitt⸗ki hann \alst{f}iðr, \hld\ þó’tt of hann \alst{f}ár lesi, &
\ind ef með \alst{s}notrum \alst{s}itr.\eva

\bvb The unclever man thinks all those \\
\ind who laugh with him his friends. \\
He finds it not though they make sport of him, \\
\ind if among the clever he sits.\evb\evg


\bvg\bva\mssnote{\Regius~3v/18}%
\alst{Ȯ}·snotr maðr \hld\ hyggr sér \alst{a}lla vesa &
\ind \alst{v}ið-hlę́jęndr \alst{v}ini; &
\alst{þ}ȧ þat fiðr \hld\ es at \alst{þ}ingi kømr, &
\ind at \edtrans{á \alst{f}or·mę́lęndr \alst{f}áa}{has spokesmen few}{\Bfootnote{Repeated in st. 62.  The Thing was the Germanic legal assembly, where small disputes could easily turn into deadly feuds, so the import is that true friends are proven in conflict, not in drunken revelry.}}.\eva

\bvb The unclever man thinks all those \\
\ind who laugh with him his friends. \\
Then he finds when to the \inx[C]{Thing} he comes \\
\ind that he has spokesmen few.\evb\evg


\bvg\bva\mssnote{\Regius~3v/20}%
\alst{Ȯ}·snotr maðr \hld\ þykki⸗sk \alst{a}llt vita, &
\ind ef á sér ï \edtrans{\alst{v}ǫ̇}{nook}{\Bfootnote{From earlier \emph{*vrǫ̇}; cf. Swedish \emph{vrå} ‘corner, nook’, rare English \emph{wroo} ‘id.’  The present stanza is to my knowledge the only Norse attestation of this word in the form \emph{vǫ̇}, featuring an irregular West Norse sound change from \emph{vr-} > \emph{v-}.
The normal change \emph{vr-} > \emph{r-} yields \emph{rǫ̇}, which is the only form this word ever takes outside of the present instance.  This includes \FGT\ where \emph{rǫ̇} is brought up as an example of a word with nasal \emph{ǫ̇} and contrasted with the oral \emph{ǫ́} in \emph{rǫ́} ‘sailyard’.

It is plausible that \emph{vǫ̇} be a corruption from earlier \emph{*vrǫ̇} after alliteration between \emph{vr-} and \emph{v-} had become impossible due to the sound change \emph{vr-} > \emph{r-}.  Since old \emph{vr-} is never found in dateable Scaldic poetry composed after ca. 1000, and only twice in poetry composed ca. 900 (Egill Frag 1 and Eil \emph{Þdr} 22, both in \Skp\ 3) this provides a solid dating criterion for the present stanza, even moreso since there is nothing to suggest that the word pair \emph{*vrǫ̇} and \emph{vera} was ever formulaic.
For another instance of alliterating \emph{vr-} and \emph{v-} cf. \textlink{Havamal}[32]/2 below; for a summary of discussion about this criteria, especially as it relates to the present poem see \textcite[87--92]{Males2024}.}} \alst{v}eru; &
hitt⸗ki hann \alst{v}ęit, \hld\ hvat skal \alst{v}ið kveða, &
\ind ef hans \alst{f}ręista \alst{f}irar.\eva

\bvb The unclever man seems to know everything \\
\ind if he takes shelter in a nook. \\
He knows it not, what he shall answer \\
\ind if men test him.\evb\evg


\bvg\bva\mssnote{\Regius~3v/21}%
\alst{Ȯ}·snotr maðr \hld\ es með \alst{a}ldir kømr, &
\ind \alst{þ}at ’s batst at hann \alst{þ}ęgi; &
\alst{ę}ngi þat vęit, \hld\ at hann \alst{ę}kki kann, &
\ind nema hann \alst{m}ę́li til \alst{m}art. &
\alst{v}ęit⸗a maðr, \hld\ hinn’s \alst{v}ę́t⸗ki vęit, &
\ind þó’tt hann \alst{m}ę́li til \alst{m}art.\eva

\bvb The unclever man who comes amidst folk— \\
\ind it is best that he shut up. \\
No one knows that he nothing knows, \\
\ind unless he speak too much. \\
The man knows not, who nothing knows, \\
\ind that he speak too much.\evb\evg


\bvg\bva\mssnote{\Regius~3v/24}%
\alst{F}róðr sá þykki⸗sk, \hld\ es \edtext{\alst{f}regna kann, &
\ind ok \alst{s}ęgja}{\lemma{fregna \dots\ sęgja ‘ask \dots\ answer’}\Bfootnote{Perhaps specifically in the context of a riddling contest of wisdom.}} hit \alst{s}ama, &
\edtext{\alst{ęy}-vitu lęyna \hld\ męgu \alst{ý}ta synir &
\ind því es \alst{g}ęngr of \alst{g}uma.}{\lemma{ęy-vitu \dots\ guma. ‘In no way \dots\ earthlings.’}\Bfootnote{I.e., ‘in no way may man hide his ignorance,’ when asked a question to which he does not know the answer.}}\eva

\bvb Learned seems he who can ask \\
\ind and answer in the same way. \\
In no way may the sons of men hide \\
\ind that which eludes earthlings.\evb\evg


\bvg\bva\mssnote{\Regius~3v/26}%
\alst{Ǿ}rna mę́lir, \hld\ sá’s \alst{ę́}va þęgir, &
\ind \alst{st}að-lausu \alst{st}afi; &
\edtext{\alst{h}rað-mę́lt tunga, \hld\ \edtrans{nema \alst{h}aldęndr ęigi}{unless it be held in place}{\Bfootnote{Lit. ‘unless holders own it’ or ‘unless it own holders’; the “holders” perhaps being the teeth which hold the tongue in place.}}, &
\ind opt sér ȯ·\alst{g}ótt of \alst{g}ęlr}{\lemma{hrað-mę́lt \dots\ of gęlr ‘A quick-spoken \dots\ for itself’}\Bfootnote{Formulaic. Cf. \textlink{Lokasenna}[31].}}.\eva

\bvb He who never shuts up speaks plenty many \\
\ind utterings of absurdity. \\
A quick-spoken tongue—unless it be held in place— \\
\ind oft sings evil [into being] for itself.\evb\evg


\bvg\bva\mssnote{\Regius~3v/28}%
At \alst{au}ga-bragði \hld\ skal⸗a maðr \alst{a}nnan hafa, &
\ind þó’tt til \alst{k}ynnis \alst{k}omi; &
margr \alst{f}róðr þykki⸗sk, \hld\ ef \alst{f}reginn es⸗at &
\ind ok nái \edtrans{\alst{þ}urr-fjallr}{dry-skinned}{\Bfootnote{i.e. ‘untested’, equivalent to the English idiom \emph{get one’s feet wet}.  The word \emph{fell} \char`~\ \emph{fjall} ‘skin, pelt’ is rare in Old Norse literature and only occurs in cpds, e.g. \textlink{Volundarkvida}[11]: \emph{ber-fjall} ‘bear-pelt’.  It survives in modern Swedish \emph{fjäll} ‘scale (on fish and reptiles)’}} \alst{þ}ruma.\eva

\bvb For a laughing-stock shall man not have another \\
\ind when he comes to visit. \\
Many a one seems learned if he is not asked, \\
\ind and gets to loiter about dry-skinned.\evb\evg


\bvg\bva\mssnote{\Regius~3v/30}%
\alst{F}róðr þykki⸗sk \hld\ sá’s \alst{f}lótta tękr &
\ind \edtrans{\alst{g}ęstr}{guest}{\Bfootnote{The situation hinted at in this and the following stanza is that two guests—unknown to eachother—have come to the same homestead.  The sense is that when mocked by a stranger it is best not to engage, since the dealing may quickly turn violent.  Cf. sts. 122, 123, and 125.}} at \alst{g}ęst hę́ðinn; &
\alst{v}ęit⸗a gǫr⸗la \hld\ sá’s of \alst{v}erði glissir, &
\ind þó’tt með \alst{g}rǫmum \alst{g}lami.\eva

\bvb Learned seems he who takes to flight, \\
\ind the guest, from a scoffing guest. \\
He knows not clearly when he grins over the food, \\
\ind though he be flirting with fiends.\evb\evg


\bvg\bva\mssnote{\Regius~4r/1}%
\alst{G}umnar margir \hld\ eru⸗sk \alst{g}agn-hollir, &
\ind en \edtrans{at \alst{v}irði \emph{\alst{v}}reka⸗sk}{over food drive each other away}{\Bfootnote{The archaic initial \emph{vr-} in \emph{vreka⸗sk} must be restored for metrical reasons.  this really is quite an archaic feature since the pairing \emph{verðr} ‘food, a meal’, \emph{vreka} ‘drive away’ does not appear to be formulaic and \emph{reka} (< \emph{vreka}) alliterates exclusively with \emph{r-} in the extant Scaldic corpus.  Cf. \textlink{Havamal}[26]/2 n. above.}}; &
\alst{a}ldar róg \hld\ þat mun \alst{ę́} vesa; &
\ind órir \alst{g}ęstr við \alst{g}ęst.\eva

\bvb Many men are well true to each other, \\
\ind but over food drive each other away. \\
The strife of mankind will that ever be; \\
\ind guest raves against guest.\evb\evg


\bvg\bva\mssnote{\Regius~4r/3}%
\alst{Á}r⸗liga verðar \hld\ skyli maðr \alst{o}pt fȧa, &
\ind nema til \alst{k}ynnis \alst{k}omi; &
\alst{s}itr ok \alst{s}nópir, \hld\ lę́tr sęm \alst{s}olginn séi, &
\ind ok kann \alst{f}regna at \alst{f}ǫ́u.\eva

\bvb An early meal should man oft get \\
\ind unless he come to visit; \\
he sits and sulks, sounds as if starved, \\
\ind and can ask about little.\evb\evg


\bvg\bva\mssnote{\Regius~4r/4}%
\alst{A}f-hvarf mikit \hld\ es til \alst{i}lls vinar, &
\ind þó’tt ȧ \alst{b}rautu \alst{b}úi, &
en til \alst{g}óðs vinar \hld\ liggja \alst{g}agn-vegir, &
\ind þó’tt hann séi \alst{f}irr \alst{f}arinn.\eva

\bvb A great offroad it is to a bad friend, \\
\ind though on the road he live, \\
but to a good friend lie pleasant ways, \\
\ind though he be far gone.\evb\evg


\bvg\bva\mssnote{\Regius~4r/6}%
\Ballnote{It is best not to outstay one’s welcome.  The customary length of stay in old times was three nights, as noted in \EgilsSaga, ch. 78: \emph{þat var engi siðr, at sitja lengr en þrjár nę́tr at kynni} ‘it was not customary to stay longer than three nights when visiting.’  Compare a much more recent Jutlandish saying: \emph{en tredje dags gjæst stinker} ‘a third day’s guest stinks’, which closely resembles a contemporary American maxim popularly attributed to Benjamin Franklin: “Guests, like fish, begin to smell after three days.”  It is probably inspired by such proverbs that Auden and Taylor translate the last two lines of this stanza as “He starts to stink who outstays his welcome, / in a hall that is not his own.”}%
\alst{G}anga \edtext{\emph{skal}}{\Afootnote{emend.; om. \Regius}}, \hld\ skal⸗a \alst{g}ęstr vesa &
\ind \alst{ęy} ï \alst{ęi}num stað; &
\alst{l}júfr verðr \alst{l}ęiðr, \hld\ ef \alst{l}ęngi sitr &
\ind \alst{a}nnars flętjum \alst{ȧ}.\eva

\bvb One shall go; he shall not be a guest \\
\ind forever in one place. \\
The loved becomes loathed if for long he sits \\
\ind on another man’s benches.\evb\evg


\bvg\bva\mssnote{\Regius~4r/7}%
\edtrans{\alst{B}ú es \alst{b}ętra, \hld\ þó’tt lítit séi}{A dwelling is better though small it be}{\Bfootnote{The b-verse is missing the necessary alliteration, but no good emendation suggests itself.}}, &
\ind \alst{h}alr es \alst{h}ęima \alst{h}vęrr; &
þó’tt \alst{t}vę́r gęitr ęigi \hld\ ok \alst{t}aug-ręptan sal, &
\ind þat ’s þó \alst{b}ętra an \alst{b}ǿn.\eva

\bvb A dwelling is better though small it be; \\
\ind each is a hero at home. \\
Though two goats he own and a cord-roofed hall, \\
\ind it is yet better than begging.\evb\evg


\bvg\bva\mssnote{\Regius~4r/9}%
\alst{B}ú es \alst{b}ętra, \hld\ þó’tt lítit séi, &
\ind \alst{h}alr es \alst{h}ęima \alst{h}vęrr; &
\alst{b}lóðugt es hjarta \hld\ þęim’s \alst{b}iðja skal &
\ind sér ï \alst{m}ál hvęrt \alst{m}atar.\eva

\bvb A dwelling is better though small it be; \\
\ind each is a hero at home. \\
Bloody is the heart in him who shall beg \\
\ind for his every meal of food.\evb\evg


\bvg\bva\mssnote{\Regius~4r/10}%
\alst{V}ǫ́pnum sïnum \hld\ skal⸗a maðr \edtrans{\alst{v}ęlli ȧ}{on the plain}{\Bfootnote{Formulaic, see note to st. 11.}} &
\ind \edtrans{\alst{f}eti ganga \alst{f}ramarr}{take one step further}{\Bfootnote{Formulaic c-line, also occurring in \textlink{Lokasenna}[1]/2 (\emph{feti gangir framarr}) and \textlink{Skirnismal}[40]/2 (\emph{stígir feti framarr}).}}, &
því’t ȯ·\alst{v}íst ’s at \alst{v}ita, \hld\ nę́r verðr ȧ \alst{v}egum úti &
\ind \alst{g}ęirs of þǫrf \alst{g}uma.\eva

\bvb From his weapons shall man on the plain \\
\ind not take one step further, \\
for it is unsure to know, when on the ways outside, \\
\ind man comes in need of a spear.\evb\evg


\bvg\bva\mssnote{\Regius~4r/12}%
\Ballnote{No man is so generous that he would refuse a gift formally presented to him or loathe receiving a favour as thanks for his generosity.}%
Fann’k⸗a \alst{m}ildan \alst{m}ann \hld\ eða svá \edtrans{\alst{m}atar góðan}{good of meat}{\Bfootnote{A Wiking Age expression with parallels on Swedish runestones; see Index.}}, &
\ind at vę́ri⸗t \alst{þ}iggja \alst{þ}egit; &
eða \alst{s}ïns \edtrans{féar}{money}{\Bfootnote{In the present poem English “money” always translates ON \emph{fé} ‘money, movable property, cattle’; see Index: \inx[C]{fee}.}} \hld\ \alst{s}vá-gi \edtext{[...]}{\Bfootnote{It is doubtless that a word has been lost here; the meter and sense require it. \textcite{FinnurEdda}\ suggests \emph{gløggvan} ‘miserly, stingy’, giving a litotes ‘so unstingy’, i.e., ‘so generous’.}}, &
\ind at \alst{l}ęið séi \alst{l}aun, ef þegi.\eva

\bvb I found not a generous man or one so \inx[C]{good of meat} \\
\ind that a gift were not accepted; \\
or one with his money so not [...], \\
\ind that the repayments were loathed, if he accepted [them].\evb\evg


\bvg\bva\mssnote{\Regius~4r/14}%
\alst{F}éar sïns, \hld\ es \alst{f}ęngit hęfr, &
\ind skyli⸗t maðr \alst{þ}ǫrf \alst{þ}ola; &
opt sparir \alst{l}ęiðum \hld\ þat’s hęfr \alst{l}júfum hugat; &
\ind mart gęngr \alst{v}err an \alst{v}arir.\eva

\bvb Of his \inx[C]{money} which he has earned \\
\ind should man not suffer need. \\
Oft he saves for the loathed what he had meant for the loved; \\
\ind much goes worse than he expects.\evb\evg


\bvg\bva\mssnote{\Regius~4r/16}%
\edtrans{\alst{V}ǫ́pnum ok \alst{v}ǫ́ðum}{With weapons and garments}{\Bfootnote{i.e. weapons and armour (the “garments” are probably no silks); friends are supposed to help each other and strengthen their “violence capital”.  This alliterative word-pair is formulaic and in other occurences exclusively refers to implements of war; cf. e.g. \Beowulf\ 39, where \inx[P]{Shield}’s pyre-ship is loaded with \emph{hilde-wǽpnum \alst\ ǫnd heaðo-wǽdum} ‘war-weapons and battle-garments’.}} \hld\ skulu \alst{v}inir glęðja⸗sk; &
\ind \edtrans{þat ’s ȧ \alst{s}jǫlfum \alst{s}ýnst}{that is best seen on oneself}{\Bfootnote{I.e. in your own lived experience.}}; &
\alst{v}iðr-gefęndr ok ęndr-gefęndr \hld\ eru⸗sk \alst{v}inir lęngst, &
\ind ef \edtrans{þat}{it}{\Bfootnote{The friendship.}} bíðr at \alst{v}erða \alst{v}ęl.\eva

\bvb With weapons and garments shall friends gladden each other; \\
\ind that is best seen on oneself. \\
Givers-back and givers-again are friends for the longest \\
\ind if it awaits to turns out well.\evb\evg


\bvg\bva\mssnote{\Regius~4r/18}%
\alst{V}in sïnum \hld\ skal maðr \alst{v}inr \alst{v}esa, &
\ind ok \alst{g}jalda \alst{g}jǫf við \alst{g}jǫf; &
\alst{h}látr við \alst{h}látri \hld\ skyli \alst{h}ǫlðar taka, &
\ind en \alst{l}ausung við \alst{l}ygi.\eva

\bvb With his friend shall man be a friend, \\
\ind and pay gift against gift; \\
laughter for laughter should men employ, \\
\ind but duplicity for lie.\evb\evg


\bvg\bva\mssnote{\Regius~4r/19}%
\alst{V}in sïnum \hld\ skal maðr \alst{v}inr vesa, &
\ind \alst{þ}ęim ok \alst{þ}ess vin; &
en \alst{ȯ}·vinar sïns \hld\ skyli \alst{ę}ngi maðr &
\ind \alst{v}inar \alst{v}inr \alst{v}esa.\eva

\bvb With his friend shall man be a friend, \\
\ind with him and with \emph{his} friend; \\
but his enemy’s, should no man, \\
\ind friend’s friend be.\evb\evg


\bvg\bva\mssnote{\Regius~4r/21}%
\Ballnote{Lines 1 and 4 are repeated near-identically in st. 119 below.}%
\alst{V}ęitst, ef þú \alst{v}in átt, \hld\ þann’s \alst{v}ęl trúir &
\ind ok vilt af hǫ̇num \alst{g}ótt \alst{g}eta, &
\alst{g}ęði skalt við þann \hld\ ok \alst{g}jǫfum skipta, &
\ind \alst{f}ara at \alst{f}inna opt.\eva

\bvb Thou knowest, if thou hast a friend whom thou trustest well, \\
\ind and wilt get good from him: \\
thoughts and gifts shalt thou exchange with him; \\
\ind journey to find him oft.\evb\evg


\bvg\bva\mssnote{\Regius~4r/23}%
Ef þú \alst{á}tt \alst{a}nnan, \hld\ þann’s \alst{i}lla trúir, &
\ind vilt af hǫ̇num þó \alst{g}ótt \alst{g}eta, &
\edtext{\alst{f}agrt skalt mę́la við þann, \hld\ en \alst{f}látt hyggja}{\lemma{fagrt \dots\ mę́la, flátt hyggja ‘fairly \dots\ speak, falsely think’}\Bfootnote{Formulaic, cf. sts. 90, 91.}} &
\ind ok gjalda \alst{l}ausung við \alst{l}ygi.\eva

\bvb If thou hast another whom thou trustest badly, \\
\ind and wilt yet get good from him: \\
fairly shalt thou speak with him, but falsely think, \\
\ind and pay duplicity for lie.\evb\evg


\bvg\bva\mssnote{\Regius~4r/25}%
Þat ’s \alst{ę}nn umb þann, \hld\ es þú \alst{i}lla trúir &
\ind ok þér es \alst{g}runr at \alst{g}ęði, &
\alst{h}lę́ja skalt við þęim \hld\ ok umb \alst{h}ug mę́la; &
\ind \alst{g}·lík skulu \alst{g}jǫld \alst{g}jǫfum.\eva

\bvb This is yet about him whom thou trustest badly, \\
\ind and about whom thou hast doubt: \\
laugh shalt thou with him, and speak with care; \\
\ind repayments shall be equal to gifts.\evb\evg


\bvg\bva\mssnote{\Regius~4r/28}%
Ungr vas’k \alst{f}orðum, \hld\ \alst{f}ór’k ęinn saman, &
\ind þȧ varð’k \alst{v}illr \alst{v}ega; &
\alst{au}ðigr þȯtt⸗umk, \hld\ es \alst{a}nnan fann’k, &
\ind \alst{m}aðr es \alst{m}anns gaman.\eva

\bvb Young was I once, I wandered alone; \\
\ind then I became lost of ways. \\
Wealthy I thought me when another I found; \\
\ind man is man’s pleasure.\evb\evg


\bvg\bva\mssnote{\Regius~4r/29}%
\alst{M}ildir frǿknir \hld\ \alst{m}ęnn batst lifa, &
\ind \alst{s}jaldan \alst{s}út ala; &
en \edtext{\alst{ȯ}·snjallr}{\linenum{|3||4}\lemma{ȯ·snjallr, gløggr ‘unvalorous, stingy’}\Bfootnote{Contrasting respectively with \emph{frǿkn, mildr} ‘brave, generous’ in the first half of the stanza; very fine parallelism.}} maðr \hld\ \alst{u}ggir hvat-vetna, &
\ind \edtrans{sýtir ę́ \alst{g}løggr við \alst{g}jǫfum}{the stingy always grieves over gifts}{\Bfootnote{After receiving a gift, one was culturally obliged to give something back.  Cf. sts. 39, 145.}}.\eva

\bvb Generous, brave men live best; \\
\ind seldom they nourish sorrow— \\
but the unvalorous man is frightened by anything, \\
\ind the stingy always grieves over gifts.\evb\evg


\bvg\bva\mssnote{\Regius~4r/31}%
\Ballnote{I picture the scene in the following way: The wanderer comes walking along the plain when he sees two unadorned “tree-men”. Taking pity for the sorry-looking stick figures, he lends them some clothes, and from a distance they now look like fine chaps. Just such a frail, freezing figure, he argues, is man in his naked state; it is his clothes that afford the hero his status, and even the weak stick-man can look like a champion.  Clearly this is quite a different view from the pre-Christian Greek celebration of the naked body, but in the cold Northern climes there was seemingly not much room for public nakedness.}%
\alst{V}áðir mïnar \hld\ gaf’k \alst{v}ęlli at &
\ind \alst{t}vęim \edtrans{\alst{t}ré-mǫnnum}{tree-men}{\Bfootnote{Man-shaped wooden figures.  Much has been made of their appearance here, including seeing them as cultic idols, but whatever the case, the tone in the stanza is more pessimistic than reverent.  Cf. the three stanzas spoken by a tree-man in \emph{Ragn} (\emph{Ragn} 38–40 in \Skp\ 8) and notes there.}}; &
\alst{r}ekkar þat þȯttu⸗sk, \hld\ es \alst{r}ipt hǫfðu; &
\ind \alst{n}ęiss es \alst{n}ǫkkviðr \edtrans{halr}{hero}{\Bfootnote{The use of \emph{halr} ‘hero, warrior’ (cf. sts. 36, 37) rather than the more neutral \emph{maðr} ‘man, person’ is probably intentional.}}.\eva

\bvb My garments I gave on the plain \\
\ind to two tree-men. \\
Champions they seemed when cloaks they had; \\
\ind shameful is the naked hero.\evb\evg


\bvg\bva\mssnote{\Regius~4r/33}%
Hrørnar \alst{þ}ǫll, \hld\ sú’s stęndr \alst{þ}orpi ȧ, &
\ind \edtext{hlýr⸗a\emph{t}}{\Afootnote{\emph{‘hlyrar’} \Regius}} hęnni \alst{b}ǫrkr né \alst{b}arr; &
svá es \alst{m}aðr, \hld\ sá’s \alst{m}ann-gi ann; &
\ind hvat skal hann \alst{l}ęngi \alst{l}ifa?\eva

\bvb Withers the pine that stands on the yard; \\
\ind her shields no bark nor leaf. \\
So is the man who loves no man— \\
\ind why shall he live for long?\evb\evg


\bvg\bva\mssnote{\Regius~4v/2}%
\alst{Ę}ldi hęitari \hld\ brinnr með \alst{i}llum vinum &
\ind \alst{f}riðr \edtrans{\alst{f}imm daga}{for five days}{\Bfootnote{I.e. “for a week”, which was originally five days long.  The sense is that the bad friends quickly tire of each other when staying together for an extended period of time.  See also st. 74 and Index: \inx[C]{five days}.}}, &
en þȧ \alst{s}loknar, \hld\ es hinn \alst{s}étti kømr, &
\ind ok \alst{v}ersnar allr \alst{v}in-skapr.\eva

\bvb Hotter than fire among bad friends burns \\
\ind love, for \inx[C]{five days}, \\
but then goes out when the sixth one comes \\
\ind and all the friendship worsens.\evb\evg


\bvg\bva\mssnote{\Regius~4v/4}%
\alst{M}ikit ęitt \hld\ skal⸗a \alst{m}anni gefa; &
\ind opt kaupir sér ï \alst{l}ítlu \edtrans{\alst{l}of}{goodwill}{\Bfootnote{Or “praise”, but \emph{lof} here carries the specific sense of the favour or goodwill earned through generous acts.}}; &
með \alst{h}ǫlfum \alst{h}lęif \hld\ ok með \alst{h}ǫllu kęri &
\ind \alst{f}ekk ek mér \edtrans{\alst{f}é-laga}{fellow}{\Bfootnote{A business partner or companion.}}.\eva

\bvb Much at once shall one not give a man; \\
\ind oft one buys himself goodwill for little. \\
With half a loaf and a sloping cask \\
\ind I got myself a fellow.\evb\evg


\bvg\bva\mssnote{\Regius~4v/6}%
\Ballnote{With this stanza the subject of the advice moves on from friendship to wisdom.}%
\edtrans{\alst{L}ítilla sanda, \hld\ \alst{l}ítilla sę́va}{Of small sands, of small seas}{\Bfootnote{Most likely a partitive genitive, but the sense is not certain; in any case, the genitive excludes the translation “where sands are small, seas are small”.  I find the most likely reading to be a declaration of the smallness of man’s horizons; the world will always be far greater than him, and there will always be much of which he is unwise.}}, &
\ind lítil eru \alst{g}ęð \alst{g}uma; &
\edtext{því’t \alst{a}llir męnn \hld\ urðu⸗t \alst{ja}fn-spakir; &
\ind \alst{h}ǫlf es ǫld \alst{h}var.}{\lemma{því’t allir męnn \hld\ urðu⸗t jafn-spakir; hǫlf es ǫld hvar. ‘For all men have not become evenly wise; half is every person.’}\Bfootnote{I find the interpretation of \textcite{Athugasemdir1929} most convincing: intellectual faculties have not been distributed evenly among men, and so every one has his own strengths and weaknesses; all men are “half” (or “incomplete”, for it should be noted that ON \emph{halfr} ‘half’ has a sense of “incompleteness” not always found in its modern English cognate).  This interpretation accords well with sts. 71 and 132 below.  In the hyperspecialized modern world it is probably truer than ever.}}\eva

\bvb Of small sands, of small seas: \\
\ind small are the senses of man. \\
For all men have not become evenly wise; \\
\ind half is every person.\evb\evg


\bvg\bva\mssnote{\Regius~4v/7}%
\alst{M}eðal-snotr \hld\ skyli \alst{m}anna hvęrr, &
\ind ę́va til \alst{s}notr \alst{s}éi; &
þęim es \alst{f}yrða \hld\ \alst{f}ęgrst at lifa, &
\ind es \alst{v}ęl mart \alst{v}itu.\eva

\bvb Middle-clever should each man be; \\
\ind never too clever. \\
For those men it is fairest to live, \\
\ind who know well enough.\evb\evg


\bvg\bva\mssnote{\Regius~4v/9}%
\alst{M}eðal-snotr \hld\ skyli \alst{m}anna hvęrr, &
\ind ę́va til \alst{s}notr \alst{s}éi; &
\alst{s}notrs manns hjarta \hld\ verðr \alst{s}jaldan glatt, &
\ind ef sá ’s \alst{a}l-snotr es \alst{á}.\eva

\bvb Middle-clever should each man be; \\
\ind never too clever. \\
The clever man’s heart is seldom glad, \\
\ind if its owner is all-clever.\evb\evg


\bvg\bva\mssnote{\Regius~4v/10}%
\alst{M}eðal-snotr \hld\ skyli \alst{m}anna hvęrr, &
\ind ę́va til \alst{s}notr \alst{s}éi; &
\edtrans{\alst{ø}r·lǫg}{orlay}{\Bfootnote{One’s predetermined fate or course of life.  See \textlink{Voluspa}[19] n.}} sïn \hld\ viti \alst{ę}ngi maðr fyrir; &
\ind \edtrans{þęim es \alst{s}orga-lausastr \alst{s}efi.}{his is the most sorrowless mind.}{\Bfootnote{i.e. he who is ignorant of his fate.  It is surely fitting that Weden should say this, having knowledge of the inevitable destruction of the world and himself (see \inx[P]{Rakes of the Reins}).}}\eva

\bvb Middle-clever should each man be; \\
\ind never too clever. \\
His own \inx[C]{orlay} ought no man to know ahead; \\
\ind his is the most sorrowless mind.\evb\evg


\bvg\bva\mssnote{\Regius~4v/11}%
\alst{B}randr af \alst{b}randi \hld\ \alst{b}rinnr und’s \alst{b}runninn es, &
\ind \alst{f}uni kvęyki⸗sk af \alst{f}una; &
\alst{m}aðr af \alst{m}anni \hld\ verðr at \alst{m}áli kuðr; &
\ind en til \edtrans{\alst{d}ǿlskr}{hickish}{\Bfootnote{Derived from an ablaut variant of \emph{dalr} ‘valley, dale’ + \emph{-iskr} ‘-ish’, the sense being ‘provincial, not having left his (home) valley’; cf. English hillbilly.  The form is related to Icelandic words like \emph{vatns-dǿlir} and \emph{lang-dǿlir} ‘inhabitants of Waterdale (\emph{Vatns-dalr}), Longdale (\emph{Lang-dalr})’.}} af \alst{d}ul.\eva

\bvb Fire from fire burns until it is burned; \\
\ind flame is quickened by flame. \\
Man from man becomes wise through speech, \\
\ind but the too hickish from folly.\evb\evg


\bvg\bva\mssnote{\Regius~4v/13}%
\Ballnote{A close analogue to this stanza is found in \textcite{Saxo} 5.7.3: \emph{Pernox enim et pervigil esse debet alienum appetens culmen. Nemo stertendo victoriam cepit, nec luporum quisquam cubando cadaver invenit.} ‘Whoever intends to scale another’s pinnacle must be watchful and wakeful. Nobody has ever won victory by snoring, nor has any sleeping wolf found a carcass.’}%
\alst{Á}r skal rísa, \hld\ sá’s \alst{a}nnars vill &
\ind \edtrans{\alst{f}é eða \alst{f}jǫr}{money or life}{\Bfootnote{A formulaic word-pair found over 30 times in Norse prose, especially in laws.  It is also found in mediæval English and Frisian laws as OE \emph{feoh and feorh}, OF \emph{fia ande ferech}.}} hafa; &
sjaldan \alst{l}iggjandi ulfr \hld\ \alst{l}ę́r of getr, &
\ind né \alst{s}ofandi maðr \alst{s}igr.\eva

\bvb Early shall he rise who another man’s \\
\ind \inx[C]{money} or life will have. \\
Seldom the lying wolf gets the thigh, \\
\ind or the sleeping man victory.\evb\evg


\bvg\bva\mssnote{\Regius~4v/15}%
\alst{Á}r skal rísa, \hld\ sá’s á \alst{y}rkjęndr fáa, &
\ind ok ganga sïns \alst{v}erka ȧ \alst{v}it; &
\alst{m}art of dvęlr \hld\ þann’s umb \alst{m}orgin sefr, &
\ind \edtrans{\alst{h}alfr es auðr und \alst{h}vǫtum}{a half wealth is due the brisk}{\Bfootnote{The brisk man has already claimed a half fortune by waking up early.}}.\eva

\bvb Early shall he rise who has workmen few, \\
\ind and go his work to meet. \\
Much is kept back from him who in the morning sleeps; \\
\ind a half wealth is due the brisk.\evb\evg


\bvg\bva\mssnote{\Regius~4v/17}%
\alst{Þ}urra skíða \hld\ ok \alst{þ}akinna nę́fra, &
\ind þęss kann \alst{m}aðr \alst{m}jǫt, &
ok þęss \alst{v}iðar, \hld\ es \alst{v}inna⸗sk męgi &
\ind \edtrans{\alst{m}ál ok \alst{m}issęri}{for a season and a half-year}{\Bfootnote{Over nine months.}}.\eva

\bvb Of dry billets and thatching birch bark— \\
\ind of this man knows the measure, \\
and of that firewood which he may use \\
\ind for a season and a half-year.\evb\evg


\bvg\bva\mssnote{\Regius~4v/19}%
\edtrans{\alst{Þ}vęginn ok męttr}{Washed and fed}{\Bfootnote{A formulaic collocation.  Cf. \textlink{Reginsmal}[25] (\emph{kęmbðr} ‘combed’, \emph{þvęginn} ‘washed’, \emph{męttr} ‘fed’) and \textlink{Voluspa}[33]: (\emph{þó} ‘washed’, \emph{kęmbði} ‘combed’).  These examples attest to the importance of personal hygiene in the culture, something further seen by the ubiquity of combs in pre-Christian graves (TODO: archeological reference).
The stanza reminds of the following passage from Tacitus \emph{Germania} ch. 22: \emph{Statim ē somnō, quem plērumque in diem extrahunt, lavantur, saepius calidā, ut apud quōs plūrimum hiems occupat.  Lautī cibum capiunt: sēparātae singulīs sēdēs et sua cuique mēnsa.  Tum ad negōtia nec minus saepe ad convīvia prōcēdunt armātī.} ‘On waking from sleep, which they generally prolong to a late hour of the day, they take a bath, oftenest of warm water, which suits a country where winter is the longest of the seasons.  After their bath they take their meal, each having a separate seat and table of his own.  Then they go armed to business, or no less often to their festal meetings (\emph{convivia}, i.e., the Things).’}} \hld\ ríði maðr \alst{þ}ingi at, &
\ind þó’tt séi⸗t \alst{v}ę́ddr til \alst{v}ęl; &
\alst{sk}úa ok bróka \hld\ \alst{sk}ammi⸗sk ęngi maðr &
\ind né \alst{h}ęsts in \alst{h}ęldr, &
\ind \edtrans{þó’tt hann \alst{h}afi⸗t góðan}{although he haven’t a good one}{\Bfootnote{A difficult line metrically.  Without it, line 4 can be scanned straightforwardly as a c-verse, but then this line comes off as an isolated b-verse.  \textcite{FinnurEdda} explains it away by considering this line an interpolation, which is certainly a possibility since its content is entirely superfluous.  In that case the interpolator would have interpreted line 4 (the c-verse) as an a-verse and added line 5 as a corresponding b-verse.}}.\eva

\bvb Washed and fed ought man to ride to the \inx[C]{Thing}, \\
\ind although he be not clothed too well; \\
of his shoes and breeches ought no man to be ashamed, \\
\ind nor the more of his horse, \\
\ind although he haven’t a good one.\evb\evg


\bvg\bva\mssnote{\Regius~4v/22}%
\alst{S}napir ok gnapir, \hld\ es til \alst{s}ę́var kømr, &
\ind \alst{ǫ}rn ȧ \alst{a}ldinn mar; &
svá es \alst{m}aðr, \hld\ es með \alst{m}ǫrgum kømr &
\ind ok \edtrans{á \alst{f}or·mę́lęndr \alst{f}áa}{has spokesmen few}{\Bfootnote{Shared with st. 25.}}.%
\Aallnote{The two following sts. are written in opposite order in \Regius, but a symbol at the start of each indicates that they should switch places.}\eva

\bvb It snaps and stoops when to the sea it comes, \\
\ind the eagle on the ancient ocean. \\
So is the man who comes among the many \\
\ind and has spokesmen few.\evb\evg


\bvg\bva\mssnote{\Regius~4v/21}%
\alst{F}regna ok sęgja \hld\ skal \alst{f}róðra hvęrr, &
\ind sá’s vill \alst{h}ęitinn \alst{h}orskr; &
\alst{ęi}nn vita \hld\ né \alst{a}nnarr skal, &
\ind \edtrans{\alst{þ}jóð}{thirty}{\Bfootnote{Or “the people, nation”; the sense is in any case “many, all”.  For the translation “thirty” cf. \Skaldskaparmal\ 82, a list of poetic expressions for various numerals: \emph{\emph{þjóð} eru þrír tigir} ‘a \emph{nation} is thirty’ etc.}} vęit ef \alst{þ}rír ’ro.\eva

\bvb Ask and answer shall each learned man \\
\ind who wishes to be called sharp. \\
\emph{One} shall know—not another; \\
\ind thirty know if there are three.\evb\evg


\bvg\bva\mssnote{\Regius~4v/24}%
\Ballnote{A powerful man should not abuse his power since no height of political power or physical strength can make him invincible.  In Germanic legend Siward, remembered as the strongest and tallest hero of the Migration Period, was killed in his sleep; Ermenric, who ruled the vast realm of the Gots with an iron fist (\textlink{Deor}[5]), was maimed by two young boys (\textlink{Gudrunarhvot}, \textlink{Hamdismal}).  Consider the expression of \textcite{Hobbes1996}, ch. 13: “Nature hath made men so equall, in the faculties of body, and mind; as that though there bee found one man sometimes manifestly stronger in body, or of quicker mind then another; yet when all is reckoned together, the difference between man, and man, is not so considerable, as that one man can thereupon claim to himselfe any benefit, to which another may not pretend, as well as he. For as to the strength of body, the weakest has strength enough to kill the strongest, either by secret machination, or by confederacy with others, that are in the same danger with himselfe.”}%
\alst{R}íki sitt \hld\ skyli \alst{r}áð-snotra &
\ind \alst{h}vęrr ï \alst{h}ófi \alst{h}afa; &
\edtext{þȧ þat \alst{f}innr, \hld\ es með \alst{f}rǿknum kømr, &
\ind at \alst{ę}ngi es \alst{ęi}nna hvatastr.}{\lemma{þȧ \dots\ ęinna hvatastr ‘then \dots\ boldest of all’}\Bfootnote{Almost identical to \textlink{Fafnismal}[17]/3–4, which however has \emph{flęirum} ‘more men’ instead of \emph{frǿknum} ‘the brave’.}}\eva

\bvb His own power should each counsel-clever \\
\ind man use in moderation. \\
This he then finds when among the brave he comes— \\
\ind that none is boldest of all.\evb\evg


\bvg\bva\mssnote{\Regius~4v/25}%
\alst{O}rða þęira, \hld\ es maðr \alst{ǫ}ðrum sęgir, &
\ind opt hann \alst{g}jǫld of \alst{g}etr.\eva

\bvb For those words which man says to another \\
\ind he oft gets recompense.\evb\evg


\bvg\bva\mssnote{\Regius~4v/26}%
\Ballnote{There was nothing wrong with the ale—the problem was with the people themselves.  There are no wrong times, only wrong people, and it is bad to waste your time with those who dislike you.}%
\edtrans{\alst{M}ikils ti\emph{l}}{Much too}{\Afootnote{emend.; \emph{mikilsti} \Regius}} snimma \hld\ kom’k ï \alst{m}arga staði, &
\ind en til \alst{s}íð ï \alst{s}uma; &
\alst{ǫ}l vas drukkit, \hld\ sumt vas \alst{ȯ}·lagat; &
\ind sjaldan hittir \alst{l}ęiðr ï \alst{l}ið.\eva

\bvb Much too early I came to many places, \\
\ind but too late to some: \\
The ale was drunk up, some was unbrewed— \\
\ind seldom finds the loathed his place.\evb\evg


\bvg\bva\mssnote{\Regius~4v/28}%
\Ballnote{People are often stingy, especially with food, which was scarce and closely watched among the Norse subsistence farmers.  The poet sarcastically notes that even the “trusty friend” would invite him over oftener if he brought more food than he ate; how good of a friend is he, if he’s not willing to share his food?}%
\alst{H}ér ok \alst{h}var \hld\ myndi mér \alst{h}ęim of boðit, &
\ind ef þyrpta’k at \alst{m}ǫ́lun-gi \alst{m}at, &
eða \alst{t}vau lę́r hęngi \hld\ at hins \alst{t}ryggva vinar, &
\ind þar’s ek hafða \alst{ęi}tt \alst{e}tit.\eva

\bvb Here and there would I be invited to a home \\
\ind if at meal-time I needed no food, \\
or if two hams should hang at the trusty friend’s, \\
\ind where I had eaten one.\evb\evg


\bvg\bva\mssnote{\Regius~4v/30}%
\Ballnote{The poet celebrates simplicity.  The best pleasures in life are not power and wealth, but a hot hearth, the bright rays of the sun in springtime, and good health.}%
\alst{Ę}ldr es batstr \hld\ með \alst{ý}ta sonum &
\ind ok \alst{s}ólar \alst{s}ẏn, &
\alst{h}ęil·yndi sitt, \hld\ ef maðr \alst{h}afa náir, &
\ind ȧn við \edtrans{\alst{l}ǫst}{vice}{\Bfootnote{Used of an illicit sexual encounter in st. 98 below.  It may also refer to a physical blemish.}} at \alst{l}ifa.\eva

\bvb Fire is best among the sons of men, \\
\ind and the sight of the sun, \\
one’s good health if he gets to keep it, \\
\ind {[and]} living free from vice.\evb\evg


\bvg\bva\mssnote{\Regius~4v/32}%
\Ballnote{The strain of thought continues from the previous st.  Even someone whose health is failing can find some joy.}%
\alst{E}s⸗at maðr \alst{a}lls \edtrans{ve-sall}{unblessed}{\Bfootnote{I have elsewhere translated \emph{ve-sall} as ‘wretched’, but in the present stanza I render it literally in order to show the etymological relationship to \emph{sę́ll} ‘blessed’ used elsewhere in the stanza.  The form \emph{-sall} lacks i-umlaut due to a shortening of the vowel before the umlaut became phonemic; the ancestral Proto-Norse form would be \emph{*wajē-sāliʀ}, for which cf. ᚹᚨᛃᛖ-ᛗᚨᚱᛁᛉ \emph{wajē-mariʀ} ‘infamous’ on the Tjurkö bracteate, where the second element is the ancestor of ON \emph{mę́rr} ‘renowned, famous’; the expected descendant \emph{*ve-marr} is not attested. —
I translate \emph{sę́ll} as ‘blessed’, but it is not a past participle and could also be rendered as ‘lucky’ or ‘happy’; the translation ‘blessed’ is based on the fact that it carries a certain sense of innateness that is perhaps foreign to post-Enlightenment Western culture.  Compare here the idea of the king’s ‘luck’ (ON \emph{gipt}), which is thought to emanate from his person and shine over his land;
in this vein a king whose land experiences bountiful harvests (\emph{ár}) is said to be \emph{ár-sę́ll} ‘blessed with harvests’, while one whose reign is one of peace (\emph{friðr}) is said to be \emph{frið-sę́ll} ‘blessed with peace’.
Thus the state of the realm is not due to uncontrollable environmental or political factors, nor due to the king’s personal choices, but rather arises from the kingly person to the degree that he is favoured and blessed by the Gods.  To a lesser degree this is thought true also of the private person’s life.
This worldview is by no means exclusive Germanic, but is on the contrary shared with many other peoples, e.g. the Chinese, in whose political history the “mandate of Heaven” has been hugely important.  (TODO: Reference PCRN chapter).}}, \hld\ þó’tt séi \alst{i}lla hęill, &
\ind \alst{s}umr es af \edtext{\alst{s}onum}{\lemma{sonum \dots\ frę́ndum ‘sons \dots\ kinsmen’}\Bfootnote{Cf. st. 72 below, which stresses the importance of sons and kinsmen.}} \alst{s}ę́ll, &
sumr af \alst{f}rę́ndum, \hld\ sumr af \alst{f}é ǿrnu, &
\ind sumr af \alst{v}erkum \alst{v}ęl.\eva

\bvb Man is not all unblessed though he be of poor health: \\
\ind someone is blessed with sons, \\
someone with kinsmen, someone with ample kine, \\
\ind someone with works done well.\evb\evg


\bvg\bva\mssnote{\Regius~5r/2}%
Bętra ’s \alst{l}ifðum, \hld\ \edtext{\emph{an} sé\emph{i ȯ}-\alst{l}ifðum}{\Afootnote{emend.; ‘\emph{⁊ ſęl lıfðo}m’ \Regius.}\lemma{\emph{an} sé\emph{i ȯ·}lifðum ‘than it may be for the unliving’}\Bfootnote{The reading of \Regius, which would be normalized as \emph{ok sę́l-lifðum} ‘and for the blessed living’, is metrically defect since \emph{sę́l-} is strongly stressed and should carry alliteration.
For the original form of the line we have a close parallel in \textlink{Fafnismal}[30]: \emph{Hvǫtum ’s bętra \hld\ an sé ȯ·hvǫtum} ‘It is better for the brisk than it may be for the unbrisk’, on which the pres. ed. is based.  The corruption has probably happened in the following way: \emph{*en} (younger form of \emph{an} ‘than’) in the prototype was misinterpreted as \emph{en} ‘and, but’ and copied as \emph{⁊} (the tironian \emph{et}), while \emph{*séı ólıfðo}m (probably with the words cramped together) became \emph{sęl lıfðo}m.}}, &
\ind \edtrans{ęy getr \alst{k}vikr \alst{k}ú}{ever the quick gets the cow}{\Bfootnote{I.e., “new opportunities always present themselves for the living”.  A reference to the cattle-based economy (see also st. 76), the cow being used as a metonym.  For “quick” cf. churchly English “the quick and the dead”, i.e. “the \emph{living} and the dead”.}}; &
\edtext{\alst{ę}ld sá’k \alst{u}pp brinna \hld\ \alst{au}ðgum manni fyr, &
\ind en úti vas \alst{d}auðr fyr \alst{d}urum.}{\lemma{ęld \dots\ durum. ‘A fire \dots\ the doors.’}\Bfootnote{The fire is probably the man’s funeral pyre burning on his farm, on which a considerable amount of his wealth has been spent—in ibn Fadlan’s account of the Rus (TODO), two thirds of a dead chieftain’s estate were spent on his lavish funeral, but in spite of which he is just as dead.  The next stanza continues this thought.}}\eva

\bvb It is better for the living than it may be for the unliving: \\
\ind ever the quick gets the cow. \\
A fire I saw burning high for a wealthy man, \\
\ind but outside he was dead before the doors.\evb\evg


\bvg\bva\mssnote{\Regius~5r/3}%
\alst{H}altr ríðr \alst{h}rossi, \hld\ \alst{h}jǫrð rekr \alst{h}andar vanr, &
\ind \alst{d}aufr vegr ok \alst{d}ugir; &
\alst{b}lindr es \alst{b}ętri, \hld\ an \alst{b}ręnndr séi; &
\ind \alst{n}ýtr mann-gi \alst{n}ás.\eva

\bvb A halt man rides a horse; a handless drives a herd; \\
\ind a deaf fights and avails. \\
Blind is better than be burned; \\
\ind no man has use for a corpse.\evb\evg


\bvg\bva\mssnote{\Regius~5r/5}%
\Ballnote{It is a boon and a blessing for a man to have a son, even if he should die before his birth.  The son will carry on his father’s name, lineage, and memory; as exemplified by the raising of a stone memorial it is rare for non-family to tend a grave.  In a broader context we might consider how it is only a tiny few out of the great human masses who will leave behind any kind of lasting personal legacy beyond their direct familiar relations.  The current generation must soon die off, and even those who in their lifetimes see influence and success in their fields and careers are soon forgotten by posterity and relegated to the footnotes of the annals of history if they have no descendants to carry on their names into future ages.}%
\alst{S}onr es bętri, \hld\ þó’tt séi \alst{s}íð of alinn &
\ind ęptir \alst{g}inginn \alst{g}uma; &
sjaldan \edtrans{\alst{b}autar-stęinar}{beat-stones}{\Bfootnote{Large standing stones (menhirs) raised as memorials.  They were usually unadorned, but were in some places and periods adorned with runic inscriptions.  In Norway a large number of inscribed stones survive from about the 2nd to 5th centuries, often raised near grave fields.  Some hold only single personal names or genitive phrases, like KJ 90 from Sunde in Sunnfjord, western Norway: ᚹᛁᛞᚢᚷᚨᛊᛏᛁᛉ \textbf{widugastiʀ} ‘Woodguest’ or KJ 78 from Bø in Rogaland, southwestern Norway: ᚺᚾᚨᛒᛞᚨᛊ ᚺᛚᚨᛁᚹᚨ \textbf{hnabdas hlaiwa} ‘Naved’s grave’, while others have longer inscriptions, like KJ 75 from Kjølevik, also in Rogaland: ᚺᚨᛞᚢᛚᚨᛁᚲᚨᛉ ᛖᚲᚺᚨᚷᚢᛊᛏᚨᛞᚨᛉ ᚺᛚᚨᚨᛁᚹᛁᛞᛟᛗᚨᚷᚢᛗᛁᚾᛁᚾᛟ \textbf{hadulaikaz ekhagustadaz hlaaiwidomaguminino} ‘Handlac [lies here].  I, Haystald, buried my lad.’}} \hld\ standa \alst{b}rautu nę́r, &
\ind nema ręisi \alst{n}iðr at \alst{n}ið.\eva

\bvb A son is better, though he late be born \\
\ind after a passed man. \\
Seldom beat-stones near the highway stand, \\
\ind save by kinsman for kinsman raised.\evb\evg


\bvg\bva\mssnote{\Regius~5r/7}%
\Ballnote{A problematic stanza in \Malahattr, unlike the surrounding \Ljodahattr\ sts.  The style is also unusual and the content fits poorly in context.  It is probably a later insert.}%
\edtrans{\alst{T}vęir ’ro ęins hęrjar}{Two are of one host}{\Bfootnote{The tongue and head belong to the same body, but the former often leads to the demise of the latter and thereby itself. — \emph{hęrjar} is an inflected form of \emph{hęrr} ‘host, army’, but its function is ambiguous; it can either be (1) the gen. sg., as adopted here, or (2) the nom. pl. ‘harriers, raiders’ (cf. \emph{ęin-hęrjar} ‘\inx[P]{Oneharriers}’) which would translate as “two are the destroyers of one”, i.e. “the tongue and head often lead to the demise of the body”.}}, \hld\ \edtrans{\alst{t}unga es hǫfuðs bani}{the tongue is the head’s bane}{\Bfootnote{Formulaic or proverbial.  Cf. the Old Swedish “Heathen Law”, which describes how a duel should be conducted following an insult to a man’s honour (my norm. and trans. following \textcite{Läffler1879}): \emph{Fallr þann orð havr givit—glǿpr orða vęrstr,} tunga hovuð-bani—\emph{liggi í ú-gildum akri} ‘If he falls who has given the [insulting] word—an insult is the worst of words, \emph{the tongue the head-bane}—may he lie in an unhallowed field.’}}; &
mér ’s ï \alst{h}eðin \alst{h}vęrn \hld\ \edtrans{\alst{h}andar}{a hand}{\Bfootnote{i.e. a hand holding a dagger.}} vę̇ni.\eva

\bvb Two are of one host: the tongue is the head’s bane; \\
in every cloak I expect a hand.\evb\evg


\bvg\bva\mssnote{\Regius~5r/8}%
\alst{N}ǫ́tt verðr fęginn, \hld\ sá’s \alst{n}esti trúir, &
\ind \edtrans{\alst{sk}ammar ’ro \alst{sk}ips ráar}{short are a ship’s sailyards}{\Bfootnote{TODO: Write about the varying interpretations (Finnur, Cleasby, Skp) of this line.}}, &
\ind \alst{h}verf es \alst{h}aust-gríma; &
\edtrans{\alst{f}jǫlð of viðrir}{The winds blow far}{\Bfootnote{I.e., ‘the weather changes much’; \emph{viðra} being a causative verb derived from \emph{veðr} ‘wind, storm’.  Consider Weden’s name \emph{Viðrir} ‘Withrer; Stormer, One of the Storm’, which can be explained as an agent noun formed to this verb.}} \hld\ ȧ \edtrans{\alst{f}imm dǫgum}{five days}{\Bfootnote{i.e. “in a week” (which was originally five days long), paralleling “month” in the next line.  See note to st. 51 and Index.}}, &
\ind en \alst{m}ęir ȧ \alst{m}ȧnaði.\eva

\bvb At night he rejoices who trusts in his provisions; \\
\ind short are a ship’s sailyards; \\
\ind shifty is a stormy fall night. \\
The winds blow far in \inx[C]{five days}; \\
\ind even more in a month.\evb\evg


\bvg\bva\mssnote{\Regius~5r/10}%
\alst{V}ęit⸗a hinn, \hld\ es \alst{v}ę́t⸗ki \alst{v}ęit, &
\ind \edtrans{margr verðr \edtrans{af \alst{au}rum}{from wealth}{\Afootnote{emend. from meaningless \emph{†aflꜹðrom†} \Regius}} \alst{a}pi}{many a man turns an ape from wealth}{\Bfootnote{Cf. \Solarljod\ 34/4: \emph{margan hefr auðr apat} ‘wealth has aped many a man’, which also lends support to the emendation.}}; &
maðr es \alst{au}ðigr, \hld\ \alst{a}nnarr ȯ·auðigr, &
\ind skyli⸗t þann \alst{v}ítka \alst{v}áar.\eva

\bvb The one knows not who nothing knows: \\
\ind many a man turns an \inx[C]{ape} from wealth. \\
A man is wealthy, another not wealthy; \\
\ind one oughtn’t to curse him for his woe.\evb\evg


\bvg\bva\mssnote{\Regius~5r/12}%
\Ballnote{It is likely that sts. 76–77 concluded the original Guests’ Strand.  This is supported internally by their tone of finality and their reflections on death, and externally by the fact that the C10th \Hakonarmal\ borrows the first line of its final stanza (\emph{dęyr fé \hld\ dęyja frę́ndr}) from these two sts.}
\edtrans{\alst{D}ęyr \edtext{fé}{\lemma{fé, frę́ndr ‘Kine, kinsmen’}\Bfootnote{In the Germanic Iron Age farming society a man’s wealth was reckoned by how many heads of cattle (and the Norman loan-word \emph{cattle} is itself the same word as \emph{capital}) he owned, and his social power by the number of able male relatives ready to side with him in conflict (cf. st. 72 above and TODO: reference?).  All one’s earthly power will pass away, and so too oneself, but a good reputation, fame and glory can linger on.  For Indo-European poetic analogues, see \textcite[99\psqq]{West2007}.}}, \hld\ \alst{d}ęyja frę́ndr}{Kine die, kinsmen die}{\Bfootnote{This line is also found in the final st. (21) of \Hakonarmal, a funerary eulogy composed ca. 961.}}, &
\ind dęyr \alst{s}jalfr hit \alst{s}ama; &
en \alst{o}rðs-tírr \hld\ dęyr \alst{a}ldri⸗gi &
\ind hvęim’s sér \alst{g}óðan \alst{g}etr.\eva

\bvb Kine die, kinsmen die, \\
\ind oneself dies the same. \\
But the word-glory never dies \\
\ind for whomever gets himself a good one.\evb\evg


\bvg\bva\mssnote{\Regius~5r/13}%
\alst{D}ęyr fé, \hld\ \alst{d}ęyja frę́ndr, &
\ind dęyr \alst{s}jalfr hit \alst{s}ama; &
\alst{e}k vęit \alst{ęi}nn \hld\ at \alst{a}ldri⸗gi dęyr: &
\ind \edtrans{\alst{d}ómr}{Doom}{\Bfootnote{Here meaning ‘judgment, glory’.  See Index.}} umb \alst{d}auðan hvęrn.\eva

\bvb Kine die, kinsmen die, \\
\ind oneself dies the same. \\
I know one that never dies: \\
\ind the \inx[C]{Doom} o’er each man dead.\evb\evg


\bvg\bva\mssnote{\Regius~5r/14}%
\Ballnote{Sts. 78–80 are poorly placed and seem like later inserts.  78–79 at least resemble the general content of the Guests’ Strand, but 80 is a true enigma.}%
\alst{F}ullar grindr \hld\ sá’k fyr \edtrans{\alst{F}itjungs sonum}{Fitting’s sons}{\Bfootnote{Entirely unknown figures.}}, &
\ind nú bera þęir \edtrans{\alst{v}ȧnar \alst{v}ǫl}{the staff of hope}{\Bfootnote{A beggar’s staff.}}; &
svá es \alst{au}ðr \hld\ sęm \alst{au}ga-bragð, &
\ind hann es \alst{v}altastr \alst{v}ina.\eva

\bvb Full pens I saw for Fitting’s sons; \\
\ind now they carry the staff of hope. \\
So is wealth like the twinkling of an eye: \\
\ind it is the ficklest of friends.\evb\evg


\bvg\bva\mssnote{\Regius~5r/16}%
\alst{Ȯ}·snotr maðr \hld\ es \alst{ęi}gna⸗sk getr &
\ind \alst{f}é eða \alst{f}ljóðs mun-úð; &
\alst{m}etnaðr hǫ̇num þróa⸗sk, \hld\ en \alst{m}an-vit aldri⸗gi; &
\ind framm gęngr hann \alst{d}rjúgt ï \alst{d}ul.\eva

\bvb The unclever man who comes to own \\
\ind money or a maid’s loving grace: \\
his pride flourishes, but never his manwit; \\
\ind he goes forth far in folly.\evb\evg


\bvg\bva\mssnote{\Regius~5r/18}%
\Ballnote{This st. with its strange meter and its subject of runic magic does not fit well in its current place.  It would have fit better in the Rune-Tally (\textlink{Havamal}[138]–146), with whose stanzas it also shares formulaic expressions.  The last line with its shift in person is especially curious and possibly a later insert.}%
Þat ’s þȧ \alst{r}ęynt, \hld\ es þú at \edtext{\alst{r}u̇num spyrr, &
\ind hinum \alst{r}ęgin-kunnum}{\lemma{ru̇num \dots\ hinum ręgin-kunnum ‘the runes born of the Reins’}\Bfootnote{‘Runes of Godly origin’, namely through Weden’s Self-Hanging (sts. 138–139 below).  The expression is formulaic and very old, for it also appears on the C4th–6th Noleby stone (in the acc. sg. \emph{rúnó ragina-kundó} ‘a rune born of the Reins’)—an undeniable proof of the antiquity of some of the runic lore preserved in Norse poetry.  See also Index: \inx[C]{rune}. — The line is unusually long but need not be corrupt; cf. the similar form of \textlink{Harbardsljod}[4]/1, 6/2, 13/3.}}, &
\ind \edtext{þęim’s \alst{g}ørðu \edtrans{\alst{g}inn-ręgin}{Yin-Reins}{\Bfootnote{The ‘vast, broad Gods’, a pantheistic word.}} &
\ind ok \alst{f}áði \edtrans{\alst{F}imbul·þulr}{Fimblethyle}{\Bfootnote{‘The great thyle’, i.e. ‘the great chanter’; a name for Weden in his role as loremaster.  For \emph{þulr} ‘thyle’ cf. st. 111 below, \textlink{Vafthrudnismal}[9]/4 n.}};}{\lemma{þęim’s gørðu ginn-ręgin / ok fáði Fimbul·þulr ‘those which the Yin-Reins made and the Fimblethyle \name{= Weden} painted’}\Bfootnote{Cf. st. 142 where these two lines occur almost identically, but in reverse order.}} &
\ind \alst{þ}ȧ hęfr hann batst, ef hann \alst{þ}ęgir.\eva

\bvb Proven is then what thou learnest from the runes born of the Reins— \\
\ind from those which the \inx[P]{Yin-Reins} made \\
\ind and the Fimblethyle \name{= Weden} painted. \\
\ind Then he has it best, if he shuts up.\evb\evg

\sectionline

\subsection{Scattered stanzas of practical advice (81–90)}

The following stanzas are distinguished by a common subject matter and a prevalence of \Malahattr.

\sectionline

\bvg\bva\mssnote{\Regius~5r/20}%
At \alst{k}veldi skal dag lęyfa, \hld\ \alst{k}onu es bręnnd es, &
\alst{m}ę́ki es ręyndr es, \hld\ \alst{m}ęy es \edtrans{gefin}{given}{\Bfootnote{In marriage.}} es, &
\alst{í}s es \alst{y}fir kømr, \hld\ \alst{ǫ}l es drukkit es.\eva

\bvb Come evening shall one praise day, a woman when she is burned, \\
a sword when it is tried, a maiden when she is given, \\
ice when one comes over it, ale when it is drunk.\evb\evg


\bvg\bva\mssnote{\Regius~5r/22}%
İ \alst{v}indi skal \alst{v}ið hǫggva, \hld\ \edtrans{\alst{v}eðri}{good weather}{\Bfootnote{The word \emph{veðr} typically means ‘storm’, but that can hardly be the sense here.}} ȧ sę́ róa, &
\alst{m}yrkri við \alst{m}an spjalla \hld\ —\alst{m}ǫrg eru dags augu; &
ȧ \alst{sk}ip skal \alst{sk}riðar orka, \hld\ en ȧ \alst{sk}jǫld til hlífar, &
\alst{m}ę́ki til hǫggs, \hld\ en \alst{m}ęy til kossa.\eva

\bvb In wind shall one cut wood, in good weather row at sea, \\
in darkness speak with a maiden—many are the eyes of day. \\
A ship shall one have for speed and a shield for protection, \\
a sword for striking and a maiden for kisses.\evb\evg


\bvg\bva\mssnote{\Regius~5r/24}%
Við \alst{ę}ld skal \alst{ǫ}l drekka, \hld\ en ȧ \alst{í}si skríða, &
\alst{m}agran \edtext{\alst{m}ar kaupa, \hld\ en \alst{m}ę́ki}{\lemma{mar \dots\ mę́ki ‘steed \dots\ sword’}\Bfootnote{Formulaic pair, also occurring in \textlink{Lokasenna}[12]/1, \textlink{Volundarkvida}[33]/3, \textlink{Atlakvida}[7]/3.}} saurgan, &
\alst{h}ęima \alst{h}ęst fęita, \hld\ en \alst{h}und ȧ búi.\eva

\bvb By fire shall one drink ale, but skate on ice, \\
buy a starved steed and a rusty sword, \\
fatten the horse at home and the hound in its dwelling.\evb\evg


\bvg\bva\mssnote{\Regius~5r/26}%
\alst{M}ęyjar orðum \hld\ skyli \alst{m}ann-gi trúa, &
\ind né því’s \alst{k}veðr \alst{k}ona; &
\edtext{\edtext{því’t}{\Afootnote{om. \FostrbroedhraSaga}} ȧ \alst{h}verfanda \alst{h}véli \hld\ \edtext{vǫ́ru}{\Afootnote{\emph{er} \FostrbroedhraSaga}} þęim \edtrans{\alst{h}jǫrtu skǫpuð}{hearts shaped}{\Afootnote{\emph{hjarta skapat} ‘heart shaped’ \FostrbroedhraSaga}}, &
\ind \edtext{\alst{b}rigð}{\Afootnote{ok brigð \FostrbroedhraSaga}} ï \alst{b}rjóst of \edtext{lagit}{\Afootnote{\emph{laginn} \FostrbroedhraSaga}}.}{\lemma{því’t \dots\ lagið}\Bfootnote{Quoted in slightly divergent form in \FostrbroedhraSaga\ (Thott 1768 4°\textsuperscript{x}, fol. 210r) introduced with the words: \emph{Kom honum þá í hug kviðlingr sá, er kveðinn hafði verit um lausungar-konur:} ‘And then he remembered the ditty which had been composed about loose women:’}}\eva

\bvb A maiden’s words should no man trust, \\
\ind nor that which a woman speaks. \\
For on a whirling wheel their hearts were shaped; \\
\ind fickleness laid in their breasts.\evb\evg


\bvg\bva\mssnote{\Regius~5r/28}%
\alst{B}ristanda \alst{b}oga, \hld\ \alst{b}rinnanda loga, &
\alst{g}ïnanda ulfi, \hld\ \alst{g}alandi krǫ́ku, &
\alst{r}ýtanda svïni, \hld\ \alst{r}ót-lausum viði, &
\alst{v}axanda \alst{v}ági, \hld\ \alst{v}ellanda katli,\eva

\bvb In bursting bow, in burning flame, \\
in yawning wolf, in crowing crow, \\
in roaring swine, in rootless tree, \\
in waxing wave, in boiling kettle,\evb\evg


\bvg\bva\mssnote{\Regius~5r/30}%
\alst{f}ljúganda \alst{f}lęini, \hld\ \alst{f}allandi bǫ́ru, &
\alst{í}si \alst{ęi}n-nę́ttum, \hld\ \alst{o}rmi hring-lęgnum, &
\alst{b}rúðar \alst{b}ęð-mǫ́lum \hld\ eða \alst{b}rotnu sverði, &
\alst{b}jarnar lęiki \hld\ eða \alst{b}arni konungs,\eva

\bvb in flying spear, in falling billow, \\
in one-night old ice, in coiled-up serpent, \\
in bride’s bed-speech, or in broken sword, \\
in bear’s play, or in king’s child,\evb\evg


\bvg\bva\mssnote{\Regius~5r/32}%
\alst{s}júkum kalfi, \hld\ \alst{s}jalf-ráða þrę́li, &
\edtrans{\alst{v}ǫlu \alst{v}il-mę́li}{in wallow’s pleasing speech}{\Bfootnote{I.e. in a favourable prophecy (\inx[C]{spae}).}}, \hld\ \alst{v}al ný-fęldum.\eva

\bvb in sick calf, in self-willing thrall, \\
in wallow’s pleasing speech, in newly felled corpses,\evb\evg


\bvg\bva[89]\mssnote{\Regius~5v/2}%
\Ballnote{The numbering of the sts. in the pres. ed. follows \Regius, where sts. 88, 89 come in the opposite order.  They have reversed as it seems apparent from its \Malahattr\ meter and the dative case of the words that 89 should follow 87, and it nicely concludes the sequence 85–87.  On the other hand st. 88 with its \Ljodahattr\ meter and self-enclosed form seems to be a separate composition; it has probably been inserted after 87 due to its first line (\emph{akri ár-sǫ́num} ‘In an early sown field’), which in the dat. sg. superficially resembles the structure of 85–87, 89.}%
\alst{b}róður-\alst{b}ana sïnum \hld\ þó’tt ȧ \alst{b}rautu mǿti, &
\alst{h}úsi \alst{h}alf-brunnu, \hld\ \alst{h}ęsti al-skjótum, &
þȧ ’s \alst{jó}r \alst{ȯ}·nýtr, \hld\ ef \alst{ęi}nn fótr brotnar; &
verðr-it maðr svá \alst{t}ryggr \hld\ at þęssu \alst{t}rúi ǫllu!\eva

\bvb in one’s brother’s bane—though on the road ye meet— \\
in half-burned house, in all-fleet horse— \\
the steed is useless if one foot breaks. \\
No man be so trusting that he trust in all this!\evb\evg


\bvg\bva[88]\mssnote{\Regius~5r/33}%
\alst{A}kri \alst{á}r-sǫ́num \hld\ trúi \alst{ę}ngi maðr, &
\ind né til \alst{s}nimma \alst{s}yni; &
\alst{v}eðr rę́ðr akri, \hld\ en \alst{v}it syni; &
\ind \alst{h}ę̇tt es þęira \alst{h}várt.\eva

\bvb In an early sown field ought no man to trust, \\
\ind nor too soon in a son. \\
The weather rules the field and the wits the son: \\
\ind there is risk to them both.\evb\evg


\bvg\bva\mssnote{\Regius~5v/4}%
Svá ’s \alst{f}riðr kvinna \hld\ þęira’s \alst{f}látt hyggja, &
sęm \alst{a}ki \alst{jó} \alst{ȯ}·bryddum \hld\ ȧ \alst{í}si hǫ́lum &
\alst{t}ęitum, \alst{t}vé-vetrum \hld\ ok séi \alst{t}amr illa, &
eða ï \alst{b}yr óðum \hld\ \alst{b}ęiti stjórn-lausu, &
eða skyli \alst{h}altr \alst{h}ęnda \hld\ \alst{h}ręin \edtrans{ï þá-fjalli}{on a thawing fell}{\Bfootnote{I.e. in springtime, when the melting ice on the ground is most slippery.}}.\eva

\bvb So is the love of those women who falsely think \\
like one rode an unshod horse on slippery ice— \\
a merry one, two winters old and ill-tamed— \\
or in mad wind tacked a rudderless ship, \\
or a halt man should catch a reindeer on a thawing fell.\evb\evg

\sectionline

\subsection{Weden’s tryst with Billing’s daughter (91–102)}

The following two sections (91–102, 103–110) are united by their allusive narrative style, their composition in \Ljodahattr, and their content; together they constitute a clear subgroup in \Havamal.  Both sections concern Weden’s romantic adventures and each begins with general advice about love and social interactions before turning to the situation at hand.

The whole group begins with a stanza describing how men can be as fickle and dishonest towards women as women towards men (st. 91), but this is not illustrated until 103–110.  Like the two human sexes, the two sections form a complementary pairing; sts. 91–102 describe how a man (Weden) is deceived by a woman (Billing’s daughter), while in 103–110 the roles are reversed and it is the man (Weden) who deceives the woman (Guthlathe).

The first section begins with general maxims about love and relations between the sexes (91–95) before moving on to the narrative about Billing’s daughter (96–102).  The underlying myth—if one existed—is completely unknown, and Billing and his daughter are not known from any other source (for the name \emph{Billingr} cf. 97/1 n.)  Attempts to connect the myth to natural phenomena or later heroic ballads have not been convincing.

\sectionline

\bvg\bva\mssnote{\Regius~5v/7}%
\alst{B}ęrt nú mę́li’k, \hld\ því’t \edtrans{\alst{b}ę́ði}{them both}{\Bfootnote{The natures of both sexes; \emph{bę́ði} is neutr. pl., which in ON is used for mixed-sex groups.  The (male) poet declares that he will not attack the fair sex unfairly; he is also aware of men’s faults.}} vęit’k, &
\ind brigðr es \alst{k}arla hugr \alst{k}onum, &
\edtext{þȧ \alst{f}ęgrst mę́lum, \hld\ es \alst{f}lást hyggjum}{\lemma{fęgrst mę́lum \dots\ flást hyggjum ‘speak fairest \dots\ think falsest’}\Bfootnote{Formulaic.  Cf. st. 45.}}; &
\ind \edtrans{þat tę́lir \alst{h}orska \alst{h}ugi}{that entraps sharp minds}{\Bfootnote{Love (or sexual infatuation—the poet does not distinguish between them) turns even wise men into liars or otherwise dishonest persons.  Cf. \Malshattakvadi\ 20/1–2, which is probably partly based on this stanza:
\emph{Ást-blindir ’ro seggir svá \hld\ sumir, at þykkja mjǫk fás gá; // þannig verðr um man-sǫng mę́lt: \hld\ marga hefr þat hyggna tę́lt.}
‘Some men are so love-blind that they seem to heed very little; // for that sake it is said about love-song: many thinking men has it entrapped.’}}.\eva

\bvb Plainly I now speak, for I know them both: \\
\ind fickle is men’s mind towards women. \\
Fairest we speak when falsest we think; \\
\ind that entraps sharp minds.\evb\evg


\bvg\bva\mssnote{\Regius~5v/9}%
\edtrans{\alst{F}agrt skal mę́la}{Fairly shall speak}{\Bfootnote{Formulaic. Cf. st. 45.}} \hld\ ok \alst{f}é bjóða, &
\ind sá’s vill \alst{f}ljóðs ǫ̇st \alst{f}ȧa, &
\alst{l}íki \alst{l}ęyfa \hld\ hins \alst{l}jósa mans, &
\ind \edtrans{sá \alst{f}ę̇r, es \alst{f}ríar}{he wins, who woos}{\Bfootnote{Only he who courts her will win her hand.}}.\eva

\bvb Fairly shall he speak and offer money, \\
\ind whoso will win a lady’s love: \\
praise the body of the bright girl— \\
\ind he wins, who woos.\evb\evg


\bvg\bva\mssnote{\Regius~5v/11}%
\alst{Ȧ}star firna \hld\ skyli \alst{ę}ngi maðr &
\ind \alst{a}nnan \alst{a}ldri⸗gi; &
opt fȧa ȧ \alst{h}orskan, \hld\ es ȧ \alst{h}ęimskan né fȧa, &
\ind \edtrans{\alst{l}ost-fagrir \alst{l}itir}{lust-fair hues}{\Bfootnote{A woman with a countenance so beautiful that men cannot help but lust after her.}}.\eva

\bvb For [matters of] love should no man \\
\ind ever blame another; \\
oft they seize the sharp when they seize not the foolish, \\
\ind the lust-fair hues.\evb\evg


\bvg\bva\mssnote{\Regius~5v/12}%
\alst{Ęy}-vitar firna, \hld\ es maðr \alst{a}nnan skal, &
\ind þęss es of margan \alst{g}ęngr \alst{g}uma; &
\alst{h}ęimska ór \alst{h}orskum \hld\ gęrir \alst{h}ǫlða sonu &
\ind sá hinn \alst{m}ǫ́ttki \alst{m}unr.\eva

\bvb In no way shall man blame another \\
\ind for that which happens to many a man; \\
from sharp to fools are the sons of men \\
\ind made by this mighty thing, love.\evb\evg


\bvg\bva\mssnote{\Regius~5v/14}%
\edtrans{\alst{H}ugr}{The mind}{\Bfootnote{ON \emph{hugr} refers to the seat of emotions in the breast, which English “mind” does not entirely capture.  Normally it could be translated by English “heart”, but since the present stanza uses \emph{hjarta} ‘heart’ to refer specifically to the organ that would be very confusing for the reader.}} ęinn þat vęit, \hld\ es býr \alst{h}jarta nę́r, &
\ind ęinn es hann \alst{s}ér umb \alst{s}efa; &
øng es \alst{s}ótt verri \hld\ hvęim \alst{s}notrum manni &
\ind an sér \alst{ø}ngu at \alst{u}na.\eva

\bvb The mind alone knows what dwells close to the heart; \\
\ind it is alone with its thoughts. \\
No sickness is worse for each clever man \\
\ind than with nothing to be content.\evb\evg


\bvg\bva\mssnote{\Regius~5v/16}%
Þat þȧ \alst{r}ęynda’k, \hld\ es ï \alst{r}ęyri sat’k, &
\ind ok vę̇tta’k \alst{m}ïns \alst{m}unar, &
\alst{h}old ok \alst{h}jarta \hld\ vas mér hin \alst{h}orska mę́r, &
\ind þęy⸗gi hana at \alst{h}ęldr \alst{h}ęf’k.\eva

\bvb That I found out when I sat in the reeds \\
\ind and awaited my love. \\
My flesh and heart was that sharp maiden— \\
\ind I have her none the more.\evb\evg


\bvg\bva\mssnote{\Regius~5v/18}%
\edtrans{\alst{B}illings}{Billing’s}{\Bfootnote{An unidentified dwarf or ettin.
In the \Hauksbok\ manuscript of \textlink{Voluspa}[13] Billing appears as a dwarf, and he may also be mentioned on the cryptic C9th Malt runestone (SJy 38): \textbf{bilikikʀ} (error for \textbf{*bilikʀ} \emph{Billingʀ}?).
He also appears in a Scaldic kenning for \textsc{poetry} (Ormr \emph{Woman} 2, \Skp\ 3: \emph{full burar Billings} ‘the cup of Billing’s son’), but this does not narrow down his identity since he is merely standing in as a generic \textsc{dwarf}/\textsc{etiin} \parencite[428]{Meissner1921}.
Support for his being an ettin is found in the fact that his name is clearly similar to Gilling, an ettin who plays a part in the story of the Mead of Poetry (see introduction to sts. 103–110 below), and that a god lusting for a dwarf-maiden (to the degree they even exist) is a nonexistent motif, whereas the gods very frequently lust for those of the ettins.}} \edtrans{męy}{maiden}{\Bfootnote{I.e. unmarried (virgin) daughter.}} \hld\ ek fann \alst{b}ęðjum ȧ &
\ind \alst{s}ól-hvíta \alst{s}ofa; &
\alst{ja}rls \alst{y}nði \hld\ þȯtti mér \alst{ę}kki vesa &
\ind nema við þat \alst{l}ík at \alst{l}ifa.\eva

\bvb Billing’s maiden I found on the beds, \\
\ind sun-white, asleep. \\
An earl’s pleasure seemed me naught to be, \\
\ind save living alongside that body.\evb\evg


\bvg\bva\mssnote{\Regius~5v/20}\speakernote{[Billings mę́r:]}%
„\alst{Au}k nę́r \alst{a}ptni \hld\ skalt \alst{Ó}ðinn koma, &
\ind ef vilt þér \alst{m}ę́la \alst{m}an, &
\edtrans{\alst{a}llt eru \alst{ȯ}·skǫp}{all is misshapen}{\Bfootnote{Or, ‘the shapes (i.e. fates, destinies) are all awry’.  See Index: \inx[C]{shape}.}}, \hld\ nema \alst{ęi}n vitim &
\ind \alst{s}likan lǫst \alst{s}aman.“\eva

\bvb\speakernoteb{[Billing’s maiden:]}
“And by evening shalt thou, Weden, come, \\
\ind if thou wilt get for thee the girl [me]; \\
all is misshapen unless we alone should know \\
\ind such a vice together!”\evb\evg


\bvg\bva\mssnote{\Regius~5v/22}%
\alst{A}ptr ek hvarf \hld\ ok \alst{u}nna þȯtt⸗umk &
\ind \edtrans{\alst{v}ísum \alst{v}ilja frȧ}{away from my wise will}{\Bfootnote{‘Against my better judgment’.  The wise choice would have been to walk away, rather than fall into her trap.}}; &
\alst{h}itt ek \alst{h}ugða, \hld\ at \alst{h}afa mynda’k &
\ind \alst{g}ęð hęnnar allt ok \alst{g}aman.\eva

\bvb Back I turned—and thought myself in love— \\
\ind away from my wise will; \\
\emph{this} I thought, that I would have \\
\ind her senses all, and pleasure.\evb\evg


\bvg\bva\mssnote{\Regius~5v/23}%
Svá kom’k \alst{n}ę́st, \hld\ at hin \edtrans{\alst{n}ýta}{useful}{\Bfootnote{Sarcastic. Billing’s daughter had apparently summoned a lynch mob.}} vas &
\ind \alst{v}íg-drótt ǫll of \alst{v}akin, &
með \alst{b}rinnǫndum ljósum \hld\ ok \edtrans{\alst{b}ornum viði}{carried sticks}{\Bfootnote{The mob was armed with clubs.}}, &
\ind svá vas mér \edtrans{\alst{v}íl-stígr}{sad path}{\Bfootnote{Ambiguous, referring either to the beating he would have received at the hands of the mob, or to his walk of shame away from the hall.  The latter is perhaps more likely.}} of \alst{v}itaðr.\eva

\bvb So next I came after the useful \\
\ind war-troop had all awakened \\
with burning lights and with carried sticks; \\
\ind so for me a sad path was marked.\evb\evg


\bvg\bva\mssnote{\Regius~5v/25}%
\edtrans{\alst{Au}k nę́r morni}{And by morning}{\Bfootnote{Mirroring the beginning of st. 97 above.}}, \hld\ es vas’k \alst{ę}nn of kominn, &
\ind þȧ vas \alst{s}al-drótt of \alst{s}ofin; &
\edtrans{\alst{g}ręy ęitt}{A lone bitch}{\Bfootnote{The insult is clearly understood; Weden is likened to a horny dog and mockingly asked to relieve his lusts on it instead of Billing’s daughter—‘this is all you get, you dog!’}} þȧ fann’k \hld\ hinnar \edtrans{\alst{g}óðu}{good}{\Bfootnote{Possibly not sarcastic, but rather referring to her chastity.}} konu &
\ind \alst{b}undit \alst{b}ęðjum ȧ.\eva

\bvb And by morning when I had come again, \\
\ind then was the hall-troop asleep. \\
A lone bitch I then found, by the good woman \\
\ind bound upon the beds.\evb\evg


\bvg\bva\mssnote{\Regius~5v/27}%
Mǫrg es \edtrans{\alst{g}óð mę́r}{good maiden}{\Bfootnote{The “goodness” here refers to faithfulness and chastity.  Cf. \textlink{Skirnismal}[12], TODO.}}, \hld\ ef \alst{g}ǫrva kannar, &
\ind \alst{h}ug-brigð við \alst{h}ali; &
þȧ þat \alst{r}ęynda’k, \hld\ es hit \alst{r}áð-spaka &
\ind tęygða’k ȧ \alst{f}lę́rðir \alst{f}ljóð; &
\alst{h}ǫ́ðungar \alst{h}vęrrar \hld\ lęitaði mér hit \alst{h}orska man &
\ind ok hafða’k þęss \alst{v}ę́t⸗ki \alst{v}ífs.\eva

\bvb Many a good maiden—if one comes to know her well— \\
\ind is heart-fickle towards men. \\
I found that out when the counsel-clever \\
\ind lady into sins I lured; \\
every disgrace that sharp girl sought out for me, \\
\ind and I had naught of the woman.\evb\evg

\sectionline

\subsection{Weden’s theft of the Mead of Poetry (103–110)}

Sts. 103–110 contain Weden’s second “love adventure” in \Havamal\ and deal with his theft of the Mead of Poetry from the ettin Sutting and his daughter Guthlathe.

Unlike the previous adventure (sts. 91–102), the underlying myth of this one is very well known.  It therefore merits more extensive discussion.  The narrative is laid out in full in \Skaldskaparmal\ 5–6, which may be summarized as follows, with minor details left out:

\begin{quote}\begin{small}
	\emph{Chapter 5:} After the war between the Eese and Wanes, the two tribes of gods reconcile through spitting into a vat. Not wanting to discard this token of their truce, they instead make a man out of the spit and call him \inx[P]{Quasher} (ON \emph{Kvasir}).  He is so wise that he can answer any question posed to him, and so he travels around the world in order to share his learning with men.

	Quasher eventually comes to the dwelling of two dwarfs, Fealer and Galer (\emph{Fjalarr ok Galarr}). They kill him and drain his blood into three vessels: two vats named Soon (\emph{Són}) and Bothem (\emph{Boðn}), and a cauldron named \inx[P]{Woderearer} (\emph{Óð·rǿrir}).  They mix the blood into honey, and from the mixture they brew a mead which can make whomever drinks from it “a scold or man of learning (\emph{skald eða frǿða-maðr})”.  The dwarfs lie to the Eese about the murder, telling them that Quasher drowned in his own wisdom for a lack of good questions.

	Some time later, the dwarfs murder the ettin \inx[P]{Gilling} (\emph{Gillingr}) and his wife. Gilling’s son, \inx[P]{Sutting} (\emph{Suttungr}), learns of this and prepares to drown the dwarfs. In exchange for their lives and as weregild for his parents, the dwarfs offer Sutting the “dear mead” (\emph{mjǫð’inn dýra}; cf. here sts. 105 and 140). Sutting accepts the payment and takes the mead home with him. He places his daughter \inx[P]{Guthlathe} (\emph{Gunn·lǫð}) in a cave to guard it.

	\emph{Chapter 6:} Weden is wandering through the world when he finds nine thralls mowing hay.  He lends them aid by sharpening their scythes with a special whetstone, and they now cut much faster.  He throws the whetstone in the air and the greedy thralls fight to the death over it, leaving none alive.  By evening Weden comes to their master, Baye (\emph{Baugi}), Sutting’s brother.  Baye laments the death of his workmen, and so Weden, calling himself \inx[P]{Baleworker} (\emph{Bǫl·verkr} ‘evil-doer’), offers to do their work over the summer in exchange for one drink of the mead.  Baye tells him that Sutting alone owns the mead, but that he will accompany him to Sutting’s to ask.

	In autumn the two arrive at Sutting’s, who as expected refuses to give any part of the mead away.  Weden then tells Baye that he will get to it anyway.  He takes out the drill \inx[P]{Rate} (\emph{Rati}) and tells Baye to drill through the mountains into the cave where the mead is stored.  Baye first attempts to trick him by only drilling halfway through, but eventually creates a narrow passage.  Weden turns himself into a snake and crawls through it; as he does, Baye tries to strike him with the drill, but misses.

	On the other side Weden finds Guthlathe watching over the mead.  He seduces her, and she promises him three sips of the mead in exchange for sleeping with her for three nights.  Weden sleeps with her and then drinks.  With each sip he swallows the contents of one of the three vessels, so that all the mead ends up in his belly.

	Having drunk the mead, he dons his eagle-hame and flies away from the mountain.  Sutting sees him, takes his own eagle-hame, and gives chase.  The Eese see the chase overhead and set out several large vats on the ground, into which Weden, still flying, spits out the mead. At this point Sutting has almost caught up with him, and so Weden “sends back” (\emph{sęnda aptr}, viz. from behind) some of the mead, presumably into Sutting’s face.

	The mead caught in the vats is given to the Eese and to skilled poets (\emph{þęim mǫnnum, er yrkja kunnu} ‘those men who can compose verse’) but the portion which was “sent back” becomes the lot of bad poets (\emph{skald-fífla hlutr}).
\end{small}\end{quote}

Surprisingly, the narrative core of this longwinded myth appears to go back at least to the Bronze Age.  A close parallel is found in the Vedic myth of the origin of the psychoactive ritual drink \emph{sṓma}, which in Vedic mythology is also a god (\emph{Sṓma}) in its own right.  The earliest version of the story is found in the two hymns \Rigveda\ 4.26 and 27, which tell how \emph{Sṓma} is held inside ‘a hundred bronze forts’ (4.27.1c: \emph{şatám púras ā́yasīs}) by the archer \emph{Kr̥şā́nu}, before being stolen by the sweeping, mighty Eagle who brings it to \emph{Mánu}, the first human sacrificer and ancestor of the Aryans.  The guardian \emph{Kr̥şā́nu} does not himself give chase, but shoots his arrows at the Eagle, missing.

Later Vedic texts clearly identify the Eagle with \emph{Agní} (the god of fire), specifically in his form as the \emph{gāyatrī́} meter \parencite{Bloomfield1896}.  One text in particular (\AitareyaBrahmana\ 3.25–27) is interesting in its etiological function: “What (the \emph{gāyatrī}) seized with her right foot, that became the morning pressure (\emph{prātaḥsavana}). \dots\ What she seized with her left foot became the noon pressure (\emph{mādhyaṁdinaṁ savanam}). \dots\ What she seized became the third pressure (\emph{tṛtīyam savanam}).” \parencite[6]{Bloomfield1896}.
I may here note that \citeauthor{Bloomfield1896} offers a naturalistic explanation of the myth: the Eagle is \emph{Agní} in the form of lightning, who shoots forth “from the womb of the cloud; as the lightning shoots from the cloud, the heavenly fluid, the Soma, streams down upon the earth.”

The close relation between the Norse and Vedic origin myths for the sacred drink may be explained in two ways; either through common inheritance from the Indo-Europeans or through Iranian steppe influence, which may have taken place as late as the Migration Period (probably by way of the Sarmatians, cf. the loanword \emph{path} which must have entered Germanic after Grimm’s Law).  It should here be said that linguistic and botanical facts preclude the Indo-Iranian \emph{sṓma} cult’s being originally Indo-European; it instead appears to have been borrowed in the late 3rd millenium BCE from the Central Asian Bactria-Margian Archaeological Complex or BMAC \parencite[3--4, 6]{Lubotsky2001}.

The original Proto-Indo-European drink was probably the Drink of Undying (OI \emph{a·mŕ̥ta}, Greek \textgreek{ἀ·μβροσία} < PIE \emph{*n̥mr̥tós} ‘undying, immortal’; Greek also \textgreek{νέκ·τᾰρ} ‘death-overcoming’), which in the Greek mythology is brought to Zeús (\textgreek{Ζεύς}) by doves (\emph{Odyssey} 12.62–63: \textgreek{πέλειαι τρήρωνες, ταί τ' ἀμβροσίην Διὶ πατρὶ φέρουσιν} ‘the trembling doves which bring ambrosia to Father Zeús’) and in the Irish mythology was brewed for the Gods (the \emph{tuath dé danann}) by the smith \emph{Goibniu} (TODO: source).
It seems evident from the Greek, Norse, and Irish attestations that this Drink was originally mythological and believed to be the mystery behind the immortality of the Gods.  In this vein it would probably have been invoked in the symbolic poetic language of the ritual toasts drunk by the PIE chieftains (probably originally consisting of fermented honey-wine—mead).

It was in the Indo-Iranian tradition that this symbologic language, together with the myth of the eagle stealing it from the mountains, was uniquely integrated into the borrowed cult of the \emph{sṓma}, for which \emph{a·mŕ̥ta} ‘undying’ became a poetic epithet (e.g. \Rigveda\ 5.2.3c).
Perhaps unlike the earlier alcoholic drinks, this one was believed to really make its human drinkers immortal, as famously seen in the ecstatic \Rigveda\ 8.48.3: \emph{Á·pāma sṓmam, a·mŕ̥tā a·bhūma, / á·ganma jyótir, á·vidāma dēvā́n. / Kíṃ nūnám asmā́n kr̥ṇavad árātiḥ / kím u dʰūrtír a·mr̥ta mártiyasya?} ‘We have drunk the soma; we have become immortal; we have gone to the light; we have found the gods.  What can hostility do to us now, and what the malice of a mortal, o immortal one?’%NOTE: OI

Having discussed non-Norse parallels to this myth we may now turn to its form in the present stanzas and how it differs from \Skaldskaparmal\ 5–6.  To begin, the biggest difference between the treatment of the myth in \Skaldskaparmal\ and \Havamal\ is that the latter is very opaque.  Far from being a linear retelling of events, the narrative thread is actually quite difficult to follow, especially without the help of \Skaldskaparmal.
Notably, one of the key details shared between \Skaldskaparmal\ and the Vedic hymns—the eagle—is not found in \Havamal.  Other important \Skaldskaparmal\ elements not found in the \textlink{Havamal} version are Quasher, the two dwarfs, and Baye.  It is thus clear that Snorre’s narrative cannot be exclusively based on \textlink{Havamal}, but must also rely on other, now-lost sources.  This hypothesis is supported by the large number of kennings for \textsc{poetry} found in the Scaldic corpus which rely on the narrative as told in \Skaldskaparmal, but which predate that text (ca. 1220) by up to four centuries.
Scaldic kennings reference the following details from \Skaldskaparmal\ not found in \Havamal\ (\cite[427--430]{Meissner1921}; all citations from \Skp):
\begin{itemize}
\item Quasher’s blood (Eskál \emph{Vell} 1: \emph{Kvasis dreyra} ‘Quasher’s blood \ken{poetry}’);
\item the ransom of the two dwarfs (Eyv \emph{Hál} 1: \emph{gjǫld Gillings} ‘the payments for Gilling \ken{poetry}’, Anon (\emph{SnE}) 1: \emph{sęin-fyrnð skip dverga} ‘the late forgotten ships of dwarfs \ken{poems}’);
\item Weden’s companion Baye (Egill \emph{Arkv} 22: \emph{sǫku-nautr Sónar hvinna} ‘adversary of the thieves of Soon <mythical vat> [= Weden and Baye (\emph{Bauga} gen. sg. = \emph{bauga} gen. pl. ‘rings’) > \textsc{generous man}]’—this kenning is not mentioned by \citeauthor{Meissner1921} due to his reliance on an earlier ed.);
\item the eagle’s cargo (Egill \emph{Berdr} 1: \emph{ǫrð arnar kjapta} ‘produce of the eagle’s beak \ken{poetry}’);
\item and its “sending back” of a certain part of the mead (Þstf Lv 3: \emph{lęirr ara ins gamla} ‘dung of the ancient eagle \ken{bad poetry}’).
\end{itemize}

On the other hand there are elements found in \textlink{Havamal}[103]–110 which do not appear in \Skaldskaparmal\ 5–6.  The focus of \textlink{Havamal} is squarely on Weden’s visit to (and particularly his betrayal of) Sutting and his daughter Guthlathe, and the emphasis \textlink{Havamal} places on Weden’s \emph{betrayal} contrasts sharply with the transactional and seemingly unemotional three-night affair in \Skaldskaparmal.  It is possible that the version of the myth underlying \textlink{Havamal} even saw Weden marry Guthlathe, receiving the mead as a dowry.
This is supported by the archaic legal expression \emph{hins hindra dags} (st. 109), and would explain Weden’s broken oath (110), which is not mentioned in \Skaldskaparmal.  The recipient of the oath may even have been Sutting, the father of the bride himself, as suggested by the description of him as \emph{svikvinn} ‘betrayed’ and by the possibility that he hosted a banquet for Weden (110).  An internal problem with that view is that Weden is still said to have had to bore through the mountains (107), presumably to reach Guthlathe, in which case it comes off as unlikely that he would \emph{then} have asked Sutting for her hand, rather than have simply seduced her then-and-there in her chamber; further, the betrayer of Sutting need not be Weden directly but could of course also be Guthlathe.  Two other motifs to be considered in relation to this myth are the beautiful ettin’s daughter coveted by a god (cf. Billing’s maiden above, Gird in \textlink{Skirnismal}, and Rind in \citeauthor{Saxo} (\textlink{Voluspa}[31] n.)) and the horny ettin’s daughter who attempts to seduce a young hero (cf. \HelgakvidaHjorvardssonar\ P6–30).  See further notes to the individual stanzas.

\textlink{Havamal}[103]–110 begin with some social advice (103), after which the narrative is retold in non-linear fashion by Weden himself.  He visits Sutting’s home, but does not receive a good reception (104).  Guthlathe falls in love with him and gives him a drink of the Mead, for which he cruelly repays her (105).  In order to get to her chamber Weden has to bore through the mountains with the drill Rate (106).  Weden has “bought” the Mead “well”; probably a euphemistic reference to sleeping with Guthlathe for it, and given it to mankind (107).  Weden slept with Guthlathe and worries that he would not have made it out alive without her aid (108).  “The following day” a group of Rime-Thurses come to Weden’s hall to ask him whether Baleworker (presumably the name he gave Guthlathe) is among the Gods or whether he has been slain by Sutting (109).  Weden, talking about himself in the third person, answers that he “thinks” that Weden has sworn an oath, but that his words cannot be trusted; after the simble he betrayed Sutting and made Guthlathe weep (110).

\sectionline

\bvg\bva\mssnote{\Regius~5v/30}%
Hęima \alst{g}laðr \alst{g}umi \hld\ ok við \alst{g}ęsti ręifr, &
\ind \alst{s}viðr skal umb \alst{s}ik vesa; &
\alst{m}innigr ok \alst{m}ǫ́lugr, \hld\ ef vill \alst{m}arg-fróðr vesa; &
\ind opt skal \alst{g}óðs \alst{g}eta; &
\alst{f}imbul-\alst{f}ambi hęitir, \hld\ sá’s \alst{f}átt kann sęgja; &
\ind þat es \alst{ȯ}·snotrs \alst{a}ðal.\eva

\bvb At home shall man be glad and giving with his guest, \\
\ind wise about himself. \\
Of good memory and speech if he wishes to be many-learned; \\
\ind oft shall he speak of good. \\
A fimble-fool is he called who can say little; \\
\ind that is the unclever man’s nature.\evb\evg


\bvg\bva\mssnote{\Regius~5v/33}%
Hinn \alst{a}ldna \alst{jǫ}tun sótta’k, \hld\ nú em’k \alst{a}ptr of kominn; &
\ind fátt gat’k \alst{þ}ęgjandi \alst{þ}ar; &
\alst{m}ǫrgum orðum \hld\ \alst{m}ę́lta’k ï mïnn frama &
\ind ï \alst{S}uttungs \alst{s}ǫlum.\eva

\bvb The old ettin \ken*{= Sutting} I sought out; now am I come back; \\
\ind I got little hearing there. \\
Many words I spoke to my furtherance, \\
\ind in the halls of Sutting.\evb\evg


\bvg\bva\mssnote{\Regius~6r/2}%
\alst{G}unn·lǫð mér of \alst{g}af \hld\ \alst{g}ollnum stóli ȧ &
\ind \alst{d}rykk hins \alst{d}ýra mjaðar; &
\alst{i}ll \alst{i}ð-gjǫld \hld\ lét’k hana \alst{ę}ptir hafa &
\ind sïns hins \alst{h}ęila \alst{h}ugar, &
\ind sïns hins \alst{s}vára \alst{s}efa.\eva

\bvb \inx[P]{Guthlathe} gave me on the golden throne \\
\ind a drink of the dear mead; \\
evil recompense I let her have afterwards, \\
\ind for her whole heart, \\
\ind for her severe affection.\evb\evg


\bvg\bva\mssnote{\Regius~6r/4}%
\edtrans{\alst{R}ata}{Rate’s}{\Bfootnote{The drill used by Weden to bore through the mountain into the room where Guthlathe sat over the mead.}} munn \hld\ lét⸗umk \alst{r}úms of fȧa &
\ind ok umb \alst{g}rjót \alst{g}naga; &
\alst{y}fir ok \alst{u}ndir \hld\ stóð⸗umk \alst{jǫ}tna vegir, &
\ind svá \alst{h}ę̇tta’k \alst{h}ǫfði til.\eva

\bvb Rate’s mouth I made to bring me room \\
\ind and gnaw away at the rocks. \\
Over and under me stood the roads of ettins \ken{mountains}; \\
\ind so I risked my head.\evb\evg


\bvg\bva\mssnote{\Regius~6r/6}%
\edtrans{\alst{V}ęl kęypts hlutar}{The well bought thing}{\Bfootnote{The Mead of Poetry; it was “well bought” in that the price Weden paid for it was three nights with Guthlathe.}} \hld\ hęf’k \alst{v}ęl notit; &
\ind \alst{f}ás es \alst{f}róðum vant; &
því’t \edtrans{\alst{Ó}ð·rǿrir}{Woderearer}{\Bfootnote{One of the vessels in with the Mead of Poetry was held (see introduction to sts. 103–110 above), here standing as a \emph{pars pro toto} for all the Mead.}} \hld\ es nú \alst{u}pp kominn &
\ind ȧ \alst{a}lda vés \edtrans{\alst{ja}ðar}{rim}{\Bfootnote{metr. emend.; \emph{jarðar} \Regius\ has a long root-syllable, and does not fit grammatically.}}.\eva

\bvb The well bought thing have I used well; \\
\ind little do the learned lack, \\
for Woderearer has now come up \\
\ind over the rim of the \inx[C]{wigh} of men \ken*{= Middenyard}.\evb\evg


\bvg\bva\mssnote{\Regius~6r/8}%
\edtrans{\alst{I}fi ’s mér \alst{ȧ}}{I harbour doubt}{\Bfootnote{Lit. “There is doubt upon me”.}}, \hld\ at vę́ra’k \alst{ę}nn kominn &
\ind \alst{jǫ}tna gǫrðum \alst{ó}r, &
ef \alst{G}unn·laðar né nyta’k, \hld\ hinnar \alst{g}óðu konu, &
\ind es lǫgð⸗umk \alst{a}rm \alst{y}fir.\eva

\bvb I harbour doubt that I would have come back \\
\ind out of the yards of the Ettins, \\
if Guthlathe I had not used, that good woman \\
\ind over whom I laid my arm.\evb\evg


\bvg\bva\mssnote{\Regius~6r/9}%
\edtrans{\alst{H}ins \alst{h}indra dags}{The following day}{\Bfootnote{This is the only occurrence of the comparative \emph{hindra} ‘following, next’ in the whole Old Norse-Icelandic corpus.  The superlative \emph{hindstr} ‘last, final’ does occur (e.g. \emph{indsta sinni} ‘the last time’, with loss of the \emph{h-}; see \CV: \emph{hindri}), and the possible derivative \emph{hindar-dags} ‘day after tomorrow, two days after’ is found twice, both times in the \Gulatingslog, chh. 37 and 266.  In the broader Scandinavian sphere, however, we find in the Swedish provicial laws an exact equivalent of the present phrase, namely OSwe. \emph{hindra-daghẹr}, a law-word referring specifically to the ‘day after \emph{the wedding night}’, used both on its own and in the expression \emph{hindra-dags gięf} ‘morning gift’ (\textciteshorttitle{LMNL}).  If ‘the day after the wedding night’ is in fact the sense of \emph{hindra dagr} in the present stanza, two interpretations are possible: (1) Weden refers sarcastically to the day after he slept with Guthlathe, as would be done on a wedding night.  (2) Weden actually married, or promised to marry, Guthlathe.  The latter interpretation may find further support in the “bigh-oath” of st. 110, but Guthlathe is never referred to in any surviving Norse sources as Weden’s wife, only his lover (Steinþ Frag 1 (\Skp\ 3)).}} \hld\ gingu \alst{h}rím-þursar &
\alst{H}ǫ́va ráðs at fregna, \hld\ \alst{H}ǫ́va \alst{h}ǫllu ï, &
at \alst{B}ǫl·verki spurðu, \hld\ ef vę́ri með \alst{b}ǫndum kominn &
\ind eða hęfði hǫ̇num \alst{S}uttungr of \alst{s}óit.\eva

\bvb The following day the Rime-Thurses went \\
to ask for the High One’s counsel, in the High One’s hall. \\
About Baleworker they asked if he were come among the Bonds \name{Gods}, \\
\ind or if Sutting had slain him.\evb\evg


\bvg\bva\mssnote{\Regius~6r/12}%
\Ballnote{The exact narrative referred to in the stanza is hard to pin down.  One possibility is that Weden swore an oath on a bigh (an armring) to marry Guthlathe; Sutting then hosted a banquet (simble) for the couple (cf. \emph{hins hindra dags} in st. 109), and Weden slept with her and then stole the mead.  The mention of Sutting as betrayed could however also be a reference to Guthlathe’s betrayal of him, so that the sense is that Weden left both the father and daughter upset and weeping: the father furious over the loss of his mead and the harlotry of his daughter, the daughter heartbroken over the flight of her false lover.}%
Baug-ęið \alst{Ó}ðinn \hld\ hygg at \alst{u}nnit hafi, &
\ind hvat skal hans \alst{t}ryggðum \alst{t}rúa? &
\alst{S}uttung \alst{s}vikvinn \hld\ hann lét \alst{s}umbli frȧ &
\ind ok \alst{g}rǿtta \alst{G}unn·lǫðu.\eva

\bvb A \inx[C]{bigh-oath} I judge that Weden has sworn— \\
\ind how shall one trust his truces? \\
Leaving the \inx[C]{simble} he left Sutting betrayed \\
\ind and Guthlathe made to weep.\evb\evg

\sectionline

\subsection{The Speeches of Loddfathomer (111–137)}

The so-called \textbf{Speeches of Loddfathomer} (ON \emph{Loddfáfnismǫ́l}) is a series of advice stanzas addressed to \inx[P]{Loddfathomer}, an otherwise unknown figure who is clearly fictional.  His name is a compound of \emph{lodd-,} apparently related to ON \emph{loddari} ‘juggler, tramp’, OE \emph{loddere} ‘pauper, beggar’ + \emph{Fáfnir} ‘\inx[P]{Fathomer}’, literally ‘embracer’, the name of a famous \inx[C]{wyrm}.  The name thus paints a picture of an archetypal greedy fool in desperate need of the wisdom taught by Weden, his (intellectual) superior.  Loddfathomer reappears in st. 164 which reveals that the galders of 147–165 are also addressed to him, but he is not found anywhere outside of \textlink{Havamal}.

The content of the advice is generally similar and sometimes identical to that found in the Guests’ Strand (1–79) above, but some items are of a superstitious or religious nature, something the Guests’ Strand tends to avoid (e.g. 113–114, 126, 129, 137).

In \Regius\ stanza 111 has a noticably larger initial \emph{M}, albeit smaller than the initials which introduce new chapters and poems.

\sectionline

\bvg\bva\mssnote{\Regius~6r/14}%
Mál ’s at \edtrans{\alst{þ}ylja}{thill}{\Bfootnote{To ‘recite, chant’, the verb corresponding to \emph{þulr} ‘thyle’.}} \hld\ \edtrans{\alst{þ}ular}{thyle}{\Bfootnote{The reciter, chanter of ancient lore.  See Index.}} stóli ȧ; &
\ind \alst{U}rðar brunni \alst{a}t &
\alst{s}á’k ok þagða’k, \hld\ \alst{s}á’k ok hugða’k, &
\ind hlýdda’k ȧ \alst{m}anna \alst{m}ál; &
umb \alst{r}u̇nar hęyrða’k dǿma, \hld\ né umb \alst{r}ǫ́ðum þǫgðu &
\ind \alst{H}ǫ́va \alst{h}ǫllu at, &
\ind \alst{H}ǫ́va \alst{h}ǫllu ï &
\ind hęyrða’k \alst{s}ęgja \alst{s}vá:\eva

\bvb It is time to \inx[C]{thill} upon the chair of the \inx[C]{thyle}. \\
\ind At the \inx[P]{Well of Weird} \\
I saw and I shut up; I saw and I thought; \\
\ind I listened to the matters of men. \\
Of runes I heard them speak nor shut up about counsel \\
\ind at the High One’s hall, \\
\ind in the High One’s hall, \\
\ind I heard them say so:\evb\evg


\bvg\bva\mssnote{\Regius~6r/17}%
\edtrans{\alst{R}ǫ́ð⸗umk}{I counsel}{\Bfootnote{The use of the reflexive is unusual.  It may derive from a hypercorrection of 1pl. pres. ind. \emph{rǫ́ðum}, which is not uncommonly used in a singular sense in poetry.}} þér Lodd·fáfnir, \hld\ at \alst{r}ǫ́ð nemir, &
\ind \alst{n}jóta munt ef \alst{n}emr, &
\ind þér munu \alst{g}óð ef \alst{g}etr: &
\alst{n}ǫ́tt þú rís⸗at, \hld\ nema ȧ \alst{n}jósn séir, &
\ind eða \edtrans{lęitir þér \alst{i}nnan \alst{ú}t staðar}{looking to relieve thyself outside}{\Bfootnote{Lit. “looking for thy place outside”.  To \emph{lęita sér staðar} ‘look for one’s place’ is euphemistic; the same expression is used by Snorre in \YnglingaSaga\ 11.}}.\eva

\bvb \emph{I counsel thee, Loddfathomer, that thou learn the counsels; \\
\ind thou wilt profit if thou learnest, \\
\ind they will be good for thee if thou gettest:} \\
Rise not at night unless thou be scouting \\
\ind or looking to relieve thyself outside.\evb\evg


\bvg\bva\mssnote{\Regius~6r/19}%
\alst{R}ǫ́ð⸗umk þér Lodd·fáfnir, \hld\ at \alst{r}ǫ́ð nemir, &
\ind \alst{n}jóta munt ef \alst{n}emr, &
\ind þér munu \alst{g}óð ef \alst{g}etr: &
\alst{f}jǫl-kunnigri konu \hld\ skal⸗at-tu ï \alst{f}aðmi sofa, &
\ind svá’t hǫ̇n \alst{l}yki þik \alst{l}iðum.\eva

\bvb \emph{I counsel thee, Loddfathomer, that thou learn the counsels; \\
\ind thou wilt profit if thou learnest, \\
\ind they will be good for thee if thou gettest:} \\
In a \inx[C]{many-cunning} woman’s bosom shalt thou never sleep \\
\ind lest she lock thee in her limbs.\evb\evg


\bvg\bva\mssnote{\Regius~6r/21}%
Hǫ̇n svá \alst{g}ørir \hld\ at \edtrans{\alst{g}ȧir}{heed}{\Bfootnote{The existence of a nasal vowel in this verb is attested by Elfdalian \emph{gą̊}.}} ęigi &
\ind \alst{þ}ings né \alst{þ}jóðans máls; &
\alst{m}at þú vill⸗at \hld\ né \alst{m}anns⸗kis gaman &
\ind fęrr þú \alst{s}orga-fullr at \alst{s}ofa.\eva

\bvb She makes it so that thou nowise heedest \\
\ind the \inx[C]{Thing} or the ruler’s speech; \\
thou wilt not have food nor any man’s pleasure; \\
\ind thou goest sorrowful to sleep.\evb\evg


\bvg\bva\mssnote{\Regius~6r/22}%
\alst{R}ǫ́ð⸗umk þér Lodd·fáfnir, \hld\ at \alst{r}ǫ́ð nemir, &
\ind \alst{n}jóta munt ef \alst{n}emr, &
\ind þér munu \alst{g}óð ef \alst{g}etr: &
\alst{a}nnars konu \hld\ tęyg þér \alst{a}ldri⸗gi &
\ind \edtrans{\alst{ęy}ra-ru̇nu}{ear-whisperer \ken{lover}}{\Bfootnote{This word is also used in \textlink{Voluspa}[38], in which male seducers of married women are among those being forced to wade through “heavy streams” in the afterlife.}} \alst{a}t.\eva

\bvb \emph{I counsel thee, Loddfathomer, that thou learn the counsels; \\
\ind thou wilt profit if thou learnest, \\
\ind they will be good for thee if thou gettest:} \\
Another man’s wife do never tug \\
\ind into becoming thy ear-whisperer \ken{lover}.\evb\evg


\bvg\bva\mssnote{\Regius~6r/23}%
\alst{R}ǫ́ð⸗umk þér Lodd·fáfnir, \hld\ en \alst{r}ǫ́ð nemir, &
\ind \alst{n}jóta munt ef \alst{n}emr, &
\ind þér munu \alst{g}óð ef \alst{g}etr: &
\edtrans{ȧ \alst{f}jalli eða \alst{f}irði}{on fell or firth}{\Bfootnote{Hiking through mountains or sailing at sea; an expression just as well at home on Iceland as in Norway.  This word pair is a formulaic merism, and although this is the only poetic attestation it is also found a few times in the Old Norwegian laws (TODO: reference).}}, \hld\ ef þik \alst{f}ara tíðir, &
\ind fȧ⸗sk-tu at \alst{v}irði \alst{v}ęl.\eva

\bvb \emph{I counsel thee, Loddfathomer—and thou oughtst to learn the counsels; \\
\ind thou wilt profit if thou learnest, \\
\ind they will be good for thee if thou gettest:} \\
on fell or firth—if thou desire to journey— \\
\ind furnish thyself well with food.\evb\evg


\bvg\bva\mssnote{\Regius~6r/24}%
\alst{R}ǫ́ð⸗umk þér Lodd·fáfnir, \hld\ en \alst{r}ǫ́ð nemir, &
\ind \alst{n}jóta munt ef \alst{n}emr, &
\ind þér munu \alst{g}óð ef \alst{g}etr: &
\alst{i}llan mann \hld\ lát \alst{a}ldri⸗gi &
\ind \edtext{\alst{ȯ}·hǫpp at þér \alst{v}ita}{\Bfootnote{An unambiguous instance of \emph{v} alliterating with a vowel.}}, &
því’t af \alst{i}llum manni \hld\ fę̇r \alst{a}ldri⸗gi &
\ind \alst{g}jǫld hins \alst{g}óða hugar.\eva

\bvb \emph{I counsel thee, Loddfathomer—and thou oughtst to learn the counsels; \\
\ind thou wilt profit if thou learnest, \\
\ind they will be good for thee if thou gettest:} \\
Never let an evil man \\
\ind know of thy mishaps, \\
for from an evil man wilt thou never get \\
\ind recompense for thy good heart.\evb\evg


\bvg\bva\mssnote{\Regius~6r/26}%
\edtrans{\alst{O}far⸗la}{Sorely}{\Bfootnote{Contraction of \emph{ofar⸗liga} ‘\CV: high up, in the upper part’, presumably meaning that the words were particularly grievous or insulting, i.e., they “got to him”.  Whether the man was murdered or committed suicide is not clear.}} bíta \hld\ sá’k \alst{ęi}num hal &
\ind \alst{o}rð \alst{i}llrar konu, &
\edtrans{\alst{f}lá-rǫ́ð tunga}{a false-counseling tongue}{\Bfootnote{Cf. \textlink{Lokasenna}[31]/1: \emph{flǫ́ ’s þér tunga} ‘false is thy tongue’.}} \hld\ varð hǫ̇num at \alst{f}jǫr-lagi &
\ind ok þęy⸗gi umb \alst{s}anna \alst{s}ǫk.\eva

\bvb Sorely biting I saw at one man \\
\ind the words of an evil woman; \\
a false-counseling tongue brought his life to its end \\
\ind and yet nowise over a truthful charge.\evb\evg


\bvg\bva\mssnote{\Regius~6r/28}%
\alst{R}ǫ́ð⸗umk þér Lodd·fáfnir, \hld\ en \alst{r}ǫ́ð nemir, &
\ind \alst{n}jóta munt ef \alst{n}emr, &
\ind þér munu \alst{g}óð ef \alst{g}etr: &
\edtext{\alst{v}ęitst, ef \alst{v}in átt, \hld\ þann’s \alst{v}ęl trúir, &
\ind \alst{f}ar þú at \alst{f}inna opt}{\lemma{vęitst \dots\ oft ‘Thou knowest \dots\ oft’}\Bfootnote{Near-identical to st. 44/1, 4 above.}}; &
því’t \edtrans{\alst{h}rísi vęx \hld\ ok \alst{h}ǫ́u grasi}{with brushwood and with tall grass grows}{\Bfootnote{Identical to \textlink{Grimnismal}[17]/1.}} &
\ind \alst{v}egr, es \alst{v}ę́t⸗ki trøðr.\eva

\bvb \emph{I counsel thee, Loddfathomer—and thou oughtst to learn the counsels; \\
\ind thou wilt profit if thou learnest, \\
\ind they will be good for thee if thou gettest:} \\
Thou knowest, if thou hast a friend whom thou trustest well \\
\ind journey to find him oft, \\
for with brushwood and with tall grass grows \\
\ind the way which no one treads.\evb\evg


\bvg\bva\mssnote{\Regius~6r/30}%
\alst{R}ǫ́ð⸗umk þér Lodd·fáfnir, \hld\ en \alst{r}ǫ́ð nemir, &
\ind \alst{n}jóta munt ef \alst{n}emr, &
\ind þér munu \alst{g}óð ef \alst{g}etr: &
\alst{g}óðan mann \hld\ tęyg þér at \edtrans{\alst{g}aman-ru̇num}{pleasure-runes}{\Bfootnote{Here “rune” appears to carry its root meaning of ‘whisper, counsel, speech’, thus ‘pleasing speech’.  Cf. st. 129 where this word reoccurs.}} &
\ind ok nem \edtrans{\alst{l}íknar-galdr}{liking-galders}{\Bfootnote{Ways of speaking which will make one liked or popular.  For \emph{líkn} ‘liking’ see sts. 8 (with note) and 123.}} meðan \alst{l}ifir.\eva

\bvb \emph{I counsel thee, Loddfathomer—and thou oughtst to learn the counsels; \\
\ind thou wilt profit if thou learnest, \\
\ind they will be good for thee if thou gettest:} \\
A good man do tug toward thee with pleasure-runes \\
\ind and learn liking-galders while thou livest.\evb\evg


\bvg\bva\mssnote{\Regius~6r/31}%
\alst{R}ǫ́ð⸗umk þér Lodd·fáfnir, \hld\ en \alst{r}ǫ́ð nemir, &
\ind \alst{n}jóta munt ef \alst{n}emr, &
\ind þér munu \alst{g}óð ef \alst{g}etr: &
\alst{v}in þïnum \hld\ \alst{v}es aldri⸗gi &
\ind \alst{f}yrri at \alst{f}laum-slitum. &
\alst{s}org etr hjarta, \hld\ ef þú \edtext{\alst{s}ęgja né náir &
\ind \alst{ęi}n-hvęrjum \alst{a}llan hug}{\lemma{sęgja \dots\ ęin-hvęrjum allan hug ‘tell anyone thy whole mind’}\Bfootnote{Cf. st. 124 which uses almost the same expression.}}.\eva

\bvb \emph{I counsel thee, Loddfathomer—and thou oughtst to learn the counsels; \\
\ind thou wilt profit if thou learnest, \\
\ind they will be good for thee if thou gettest:} \\
With thy friend never be the first \\
\ind to tear the relation apart. \\
Sorrow eats thy heart if thou canst not tell \\
\ind anyone thy whole mind.\evb\evg


\bvg\bva\mssnote{\Regius~6r/33}%
\Ballnote{Cf. st. 125 below which gives similar advice.}%
\alst{R}ǫ́ð⸗umk þér Lodd·fáfnir, \hld\ en \alst{r}ǫ́ð nemir, &
\ind \alst{n}jóta munt ef \alst{n}emr, &
\ind þér munu \alst{g}óð ef \alst{g}etr: &
\alst{o}rðum skipta \hld\ skalt \alst{a}ldri⸗gi &
\ind við \edtrans{\alst{ȯ}·svinna \alst{a}pa}{unwise apes}{\Bfootnote{Formulaic; cf. \textlink{Grimnismal}[33], \textlink{Fafnismal}[11].}},\eva

\bvb \emph{I counsel thee, Loddfathomer—and thou oughtst to learn the counsels; \\
\ind thou wilt profit if thou learnest, \\
\ind they will be good for thee if thou gettest:} \\
Words shalt thou never exchange \\
\ind with unwise apes,\evb\evg


\bvg\bva\mssnote{\Regius~6r/34}%
\edtext{því’t af \alst{i}llum manni \hld\ munt \alst{a}ldri⸗gi &
\ind \alst{g}óðs laun of \alst{g}eta}{\lemma{því’t \dots\ geta ‘For \dots\ praise’}\Bfootnote{Cf. st. 117/6–7.}}, &
ęn \alst{g}óðr maðr \hld\ mun þik \alst{g}ørva mega &
\ind \edtrans{\alst{l}íkn-fastan}{steadfastly liked}{\Bfootnote{The first element \emph{líkn} ‘liking’ is somewhat difficult; see sts. 8 n. and 120.  For the present cpd \textcite{LaFargeGlossary} give a tentative ‘assured of favour’, while \CV\ gives ‘fast in goodwill, beloved’.}} at \alst{l}ofi.\eva

\bvb for from an evil man wilt thou never \\
\ind get a reward for thy goodness, \\
but a good man will know to make thee \\
\ind steadfastly liked through his praise.\evb\evg


\bvg\bva\mssnote{\Regius~6v/2}%
\alst{S}ifjum ’s þȧ blandit \hld\ hvęrr es \edtext{\alst{s}ęgja rę́ðr &
\ind \alst{ęi}num \alst{a}llan hug}{\lemma{sęgja \dots\ ęinum allan hug ‘tell one man his whole mind’}\Bfootnote{Cf. st. 121 which uses almost the same expression.}}; &
alt es \alst{b}ętra \hld\ an séi \alst{b}rigðum at vesa: &
es⸗a sá \alst{v}inr ǫðrum \hld\ es \alst{v}ilt ęitt sęgir.\eva

\bvb Kinship is then blended whenever one resolves to tell \\
\ind one man his whole mind. \\
Everything is better than to be with the fickle; \\
he’s no friend to another who speaks pleasantries alone.\evb\evg


\bvg\bva\mssnote{\Regius~6v/4}%
\alst{R}ǫ́ð⸗umk þér Lodd·fáfnir, \hld\ en \alst{r}ǫ́ð nemir, &
\ind \alst{n}jóta munt ef \alst{n}emr, &
\ind þér munu \alst{g}óð ef \alst{g}etr: &
\edtrans{þrimr \alst{o}rðum}{With three words}{\Bfootnote{I.e. ‘not even with three words’. If one understands \emph{orð} to mean ‘speech’ (a valid sense), we may understand that if one man says something (the first speech) to which another responds with an insult (the second speech), the first man should not retaliate (the third speech) and escalate the dispute.}} sęnna \hld\ skal⸗at-tu þér við \alst{v}erra mann; &
\ind opt hinn \alst{b}ętri \alst{b}ilar, &
\ind þȧ’s hinn \alst{v}erri \alst{v}egr.\eva

\bvb \emph{I counsel thee, Loddfathomer—and thou oughtst to learn the counsels; \\
\ind thou wilt profit if thou learnest, \\
\ind they will be good for thee if thou gettest:} \\
With three words shalt thou not flyte with a worse man; \\
\ind oft the better one breaks \\
\ind when the worse one strikes.\evb\evg


\bvg\bva\mssnote{\Regius~6v/5}%
\Ballnote{The plain sense is that the customer will place a curse on the maker if he dislikes his wares.}%
\alst{R}ǫ́ð⸗umk þér Lodd·fáfnir, \hld\ en \alst{r}ǫ́ð nemir, &
\ind \alst{n}jóta munt ef \alst{n}emr, &
\ind þér munu \alst{g}óð ef \alst{g}etr: &
\alst{sk}ó-smiðr þú vesir \hld\ né \alst{sk}ępti-smiðr, &
\ind nema \alst{s}jǫlfum þér \alst{s}éir. &
\alst{Sk}ór ’s \alst{sk}apaðr illa \hld\ eða \alst{sk}apt séi rangt, &
\ind þȧ ’s þér \alst{b}ǫls \alst{b}eðit.\eva

\bvb \emph{I counsel thee, Loddfathomer—and thou oughtst to learn the counsels; \\
\ind thou wilt profit if thou learnest, \\
\ind they will be good for thee if thou gettest:} \\
Be not a shoe-maker nor a shaft-maker \\
\ind unless thou be one for thyself. \\
The shoe is shaped badly or the shaft be crooked— \\
\ind then for thee a \inx[C]{bale} is bid!\evb\evg


\bvg\bva\mssnote{\Regius~6v/7}%
\Ballnote{If somebody puts a curse on you, acknowledge it and act decisively.  This st. has often been interpreted as a command to call out evil or injustice even when committed towards somebody else, and while there is nothing in it that speaks decisively against such a reading, it certainly does not agree with the general spirit of the \textlink{Havamal} which is one of caution and shrewdness.}%
\alst{R}ǫ́ð⸗umk þér Lodd·fáfnir, \hld\ en \alst{r}ǫ́ð nemir, &
\ind \alst{n}jóta munt ef \alst{n}emr, &
\ind þér munu \alst{g}óð ef \alst{g}etr: &
hvar’s \alst{b}ǫl kant, \hld\ kveð þér \alst{b}ǫlvi at &
\ind ok gef⸗at þïnum \alst{f}íǫndum \alst{f}rið.\eva

\bvb \emph{I counsel thee, Loddfathomer—and thou oughtst to learn the counsels; \\
\ind thou wilt profit if thou learnest, \\
\ind they will be good for thee if thou gettest:} \\
Wherever thou knowest a bale call it baleful against thee, \\
\ind and give thy foes no peace.\evb\evg


\bvg\bva\mssnote{\Regius~6v/8}%
\alst{R}ǫ́ð⸗umk þér Lodd·fáfnir, \hld\ en \alst{r}ǫ́ð nemir, &
\ind \alst{n}jóta munt ef \alst{n}emr, &
\ind þér munu \alst{g}óð ef \alst{g}etr: &
\alst{i}llu fęginn \hld\ ves \alst{a}ldri⸗gi, &
\ind \edtrans{en lát þér at \alst{g}óðu \alst{g}etit}{but rather let thyself be pleased by good}{\Bfootnote{This construction is equivalent to \CV: \emph{geta}, A. IV. with acc.}}.\eva

\bvb \emph{I counsel thee, Loddfathomer—and thou oughtst to learn the counsels; \\
\ind thou wilt profit if thou learnest, \\
\ind they will be good for thee if thou gettest:} \\
In evil do never rejoice, \\
\ind but rather let thyself be pleased by good.\evb\evg


\bvg\bva\mssnote{\Regius~6v/9}%
\Ballnote{An obscure superstition; the interpretation hinges on the word \emph{gjalti} ‘madman’ dat. sg., which must be compared with the closely related phrase \emph{verða at gjalti} ‘to be turned into a “gelt”’.
(a) \CV\ explains it as an old dative of \emph{gǫltr} ‘boar, hog’.  This necessitates an irregular breaking of \emph{ja} < \emph{ę}, since \emph{gǫltr} (< Proto-Norse \emph{*galtuʀ}) is an u-stem and should have dat. sg. \emph{gęlti} (< \emph{*galtiu}, cf. \textbf{kunimudiu}, dat. sg. of \emph{*Kunimunduʀ}, on the Tjurkö 1 bracteate).
(b) The generally accepted explanation in modern scholarship seems to be a borrowing from Old Irish \emph{geilt} ‘insane, mad’ (so \textcite{LaFargeGlossary} and others).  A close Irish parallel to the present stanza is found in the C12th or 13th Gaelic tale of Suibhne mac Colmáin, who was cursed by saint Rónán Finn to become mad; the curse took effect when he looked into the sky during a battle, after which he was known as Suibhne \emph{geilt}.  Since earlier versions of the same story are attested as early as the 840s \parencite[100]{Males2024} this word cannot be used to argue for a C13th dating for the Speeches of Loddfathomer section of \textlink{Havamal}, but its Irish roots suggest an Icelandic rather than Norwegian origin for this part of the poem.}%
\alst{R}ǫ́ð⸗umk þér Lodd·fáfnir, \hld\ en \alst{r}ǫ́ð nemir, &
\ind \alst{n}jóta munt ef \alst{n}emr, &
\ind þér munu \alst{g}óð ef \alst{g}etr: &
\alst{u}pp líta \hld\ skal⸗at-tu ï \alst{o}rrostu; &
—\alst{g}jalti \alst{g}·líkir \hld\ verða \alst{g}umna synir— &
\ind síðr þitt of \alst{h}ęilli \edtrans{\alst{h}alir}{warriors}{\Bfootnote{Some sort of “supernatural sky warriors” to quote \textcite{PettitEdda}—perhaps even the \inx[P]{Oneharriers}.}}.\eva

\bvb \emph{I counsel thee, Loddfathomer—and thou oughtst to learn the counsels; \\
\ind thou wilt profit if thou learnest, \\
\ind they will be good for thee if thou gettest:} \\
Look upward shalt thou not in battle \\
—alike to a madman become the sons of men— \\
\ind lest warriors bewitch thee.\evb\evg


\bvg\bva\mssnote{\Regius~6v/11}%
\alst{R}ǫ́ð⸗umk þér Lodd·fáfnir, \hld\ en \alst{r}ǫ́ð nemir, &
\ind \alst{n}jóta munt ef \alst{n}emr, &
\ind þér munu \alst{g}óð ef \alst{g}etr: &
Ef vilt þér \alst{g}óða konu \hld\ kvęðja at \edtrans{\alst{g}aman-ru̇num}{pleasure-runes}{\Bfootnote{While easily interpreted as ‘sexual intercourse’, the word is used in st. 120 in a decidedly non-sexual sense.  Its base meaning is probably ‘good conversation’.}} &
\ind ok \alst{f}ȧa \alst{f}ǫgnuð af, &
\alst{f}ǫgru skalt hęita \hld\ ok láta \alst{f}ast vesa; &
\ind lęiði⸗sk mann-gi \alst{g}ótt ef \alst{g}etr.\eva

\bvb \emph{I counsel thee, Loddfathomer—and thou oughtst to learn the counsels; \\
\ind thou wilt profit if thou learnest, \\
\ind they will be good for thee if thou gettest:} \\
If thou wilt for thyself greet a good woman to pleasure-runes \\
\ind and get good cheer from her, \\
fair things shalt thou promise and let it stand firm; \\
\ind no one loathes a good thing if he gets it.\evb\evg


\bvg\bva\mssnote{\Regius~6v/13}%
\alst{R}ǫ́ð⸗umk þér Lodd·fáfnir, \hld\ en \alst{r}ǫ́ð nemir, &
\ind \alst{n}jóta munt ef \alst{n}emr, &
\ind þér munu \alst{g}óð ef \alst{g}etr: &
\alst{v}aran bið’k þik \alst{v}esa \hld\ ok ęigi of·\alst{v}aran, &
ves við \alst{ǫ}l varastr, \hld\ ok við \alst{a}nnars konu &
ok við \alst{þ}at hit \alst{þ}riðja, \hld\ at \alst{þ}jófar né lęiki.\eva

\bvb \emph{I counsel thee, Loddfathomer—and thou oughtst to learn the counsels; \\
\ind thou wilt profit if thou learnest, \\
\ind they will be good for thee if thou gettest:} \\
Wary I ask thee to be, and not too wary; \\
be wariest with ale, and with another man’s woman, \\
and with this the third, that thieves do not play thee.\evb\evg


\bvg\bva\mssnote{\Regius~6v/15}%
\alst{R}ǫ́ð⸗umk þér Lodd·fáfnir, \hld\ en \alst{r}ǫ́ð nemir, &
\ind \alst{n}jóta munt ef \alst{n}emr, &
\ind þér munu \alst{g}óð ef \alst{g}etr: &
at \alst{h}áði né \alst{h}látri \hld\ \alst{h}af aldri⸗gi &
\ind \alst{g}ęst né \alst{g}anganda.\eva

\bvb \emph{I counsel thee, Loddfathomer—and thou oughtst to learn the counsels; \\
\ind thou wilt profit if thou learnest, \\
\ind they will be good for thee if thou gettest:} \\
In scorn or laughter never hold \\
\ind a guest or wanderer.\evb\evg


\bvg\bva\mssnote{\Regius~6v/16}%
\alst{O}pt vitu \alst{ȯ}·gǫr⸗la, \hld\ þęir’s sitja \alst{i}nni fyrir, &
\ind hvęrs þęir ’ro \alst{k}yns es \alst{k}oma; &
es⸗at maðr svá \alst{g}óðr \hld\ at \alst{g}alli né fylgi, &
\ind né svá \alst{i}llr at \alst{ęi}nu-gi dugi.\eva

\bvb Oft they who sit inside know not clearly ahead \\
\ind of what kind are those who come; \\
there is no man so good that no flaw follows \\
\ind nor so bad that he for nothing avails.\evb\evg


\bvg\bva\mssnote{\Regius~6v/17}%
\alst{R}ǫ́ð⸗umk þér Lodd·fáfnir, \hld\ en \alst{r}ǫ́ð nemir, &
\ind \alst{n}jóta munt ef \alst{n}emr, &
\ind þér munu \alst{g}óð ef \alst{g}etr: &
at \alst{h}ǫ́rum þul \hld\ \alst{h}lę́ aldri⸗gi, &
\ind opt ’s \alst{g}ótt þat’s \alst{g}amlir kveða, &
opt ór \edtrans{\alst{sk}ǫrpum bęlg}{from a scorched leather-bag}{\Bfootnote{Metaphor for an ancient, thick-wrinkled man.}} \hld\ \alst{sk}ilin orð koma &
\ind þęim’s \alst{h}angir með \alst{h}ǫ́um &
\ind ok \alst{sk}ollir með \alst{sk}rǫ́um, &
\ind ok \alst{v}áfir með \edtrans{\alst{v}il-mǫgum}{calf-stomachs}{\Bfootnote{A compound of \emph{vil} ‘cattle-entrails’ + \emph{magar} ‘stomachs’.  A reading \emph{víl-mǫgum} ‘lads of toil \ken{thralls}’ is possible but gives no reasonable sense that adds to the description of a hanging leather-bag.}}.\eva

\bvb \emph{I counsel thee, Loddfathomer—and thou oughtst to learn the counsels; \\
\ind thou wilt profit if thou learnest, \\
\ind they will be good for thee if thou gettest:} \\
At a hoary thyle never laugh; \\
\ind oft is good that which old men sing; \\
oft from a scorched leather-bag come discerning words; \\
\ind from him who hangs amidst hides \\
\ind and dangles amidst dry skins \\
\ind and sways amidst calf-stomachs.\evb\evg


\bvg\bva\mssnote{\Regius~6v/20}%
\alst{R}ǫ́ð⸗umk þér Lodd·fáfnir, \hld\ en \alst{r}ǫ́ð nemir, &
\ind \alst{n}jóta munt ef \alst{n}emr, &
\ind þér munu \alst{g}óð ef \alst{g}etr: &
\alst{g}ęst þú \edtrans{né \alst{g}ęyj⸗a}{At the guest do not bark}{\Bfootnote{Note the occurrence of the archaic double verbal negation \emph{né ... ⸗a(t)} which otherwise sees no use in \Havamal\ (probably for metrical-syntactical reasons, since \emph{né} cannot occur at the start of a line or clause).  \emph{né} is the original Germanic negation placed directly before the verb (cf. Gothic \emph{ni}, OE \emph{ne}) while \emph{⸗a(t)} (also \emph{-t}) is a West Norse innovation probably originally deriving from a shortening of \emph{vę́ttr} ‘wight, thing’ (cf. English not < \emph{n·á·wiht} ‘nothing’, originally \emph{ne} + \emph{á-wiht}).

In an instance of Jespersen’s cycle (whereby a strengthening adverb eventually displaces the original negative; e.g. archaic French \emph{je ne sais} ‘I do not know’ > modern standard \emph{je ne sais pas} ‘id.’, lit. ‘I do not know a step’ > modern colloquial \emph{je sais pas} ‘I don’t know) ON \emph{⸗a(t)} comes to be the main productive negator in poetry, whereas both older \emph{né} and intermediate \emph{né \dots\ ⸗a(t)} are marginalized  \parencite{Jespersen2025}.  For an overview of Norse negations in poetry see \textcite[83--89]{Sapp2022}.}} \hld\ \edtrans{né ȧ \alst{g}rind hrę́kir}{nor spit at the gate}{\Bfootnote{The guest is presumably standing in front of the gate waiting for the farmer to open it and let him in.}}; &
\ind get þú \alst{v}ǫ́-luðum \alst{v}ęl.\eva

\bvb \emph{I counsel thee, Loddfathomer—and thou oughtst to learn the counsels; \\
\ind thou wilt profit if thou learnest, \\
\ind they will be good for thee if thou gettest:} \\
At the guest do not bark nor spit at the gate; \\
\ind furnish the destitute well.\evb\evg


\bvg\bva\mssnote{\Regius~6v/21}%
\Ballnote{This stanza is rather difficult, but it must relate to the advice in the previous one.  The sense seems to be that one’s house (symbolized by its gate) will be strengthened by generosity, but weakened and cursed by greed.}%
\alst{R}ammt es þat tré, \hld\ es \alst{r}íða skal &
\ind \alst{ǫ}llum at \alst{u}pp-loki; &
\alst{b}aug þú gef \hld\ eða þat \alst{b}iðja mun &
\ind þér \alst{l}ę́s hvęrs ȧ \alst{l}iðu.\eva

\bvb Strong is that wood which shall swing \\
\ind to open up for all. \\
Give a bigh or it will bid \\
\ind every kind of guile onto thy limbs.\evb\evg


\bvg\bva\mssnote{\Regius~6v/23}%
\Ballnote{This stanza gives a rare glimpse into Wiking Age folk medicine.  The exact application of the listed cures is highly uncertain; is the drunk man, for example, supposed to ingest earth, rub it against his belly, or invoke the personified Earth in some way?}%
\alst{R}ǫ́ð⸗umk þér Lodd·fáfnir, \hld\ en \alst{r}ǫ́ð nemir, &
\ind \alst{n}jóta munt ef \alst{n}emr, &
\ind þér munu \alst{g}óð ef \alst{g}etr: &
hvar’s \alst{ǫ}l drekkir \hld\ kjós þér \alst{ja}rðar męgin, &
því’t \alst{jǫ}rð tękr við \alst{ǫ}lðri, \hld\ en \alst{ę}ldr við sóttum, &
\alst{ęi}k við \alst{a}bbindi, \hld\ \alst{a}x við fjǫl-kyngi, &
\alst{h}ǫll við \alst{h}ýrógi; \hld\ \edtrans{\alst{h}ęiptum skal Mȧna kvęðja}{in feuds shall one hail Moon}{\Bfootnote{That the Moon had a certain “might” is also attested in \textlink{Voluspa}[5]; it is presumably for this might which he is invoked here, that he may give strength to the man in conflict.  For \emph{kvęðja} ‘hail, invoke’ cf. \textlink{Lokasenna}[P3].}}, &
\alst{b}ęiti við \alst{b}it-sóttum, \hld\ en við \alst{b}ǫlvi ru̇nar; &
\ind \alst{f}old skal við \alst{f}lóði taka.\eva

\bvb \emph{I counsel thee, Loddfathomer, that thou learn the counsels; \\
\ind thou wilt profit if thou learnest, \\
\ind they will be good for thee if thou gettest:} \\
Wherever thou drinkest ale choose for thee the earth’s might, \\
for earth takes against drunkenness and fire against sicknesses, \\
oak against dysentery, the ear of corn against sorcery, \\
bearded rye against hernia—in feuds shall one hail Moon— \\
heather against bite-sicknesses and \inx[C]{rune}[runes] against a \inx[C]{bale}; \\
\ind earth shall be taken against flood.\evb\evg

\sectionline

\subsection{The Rune-Tally (138–146)}

This group of stanzas is introduced by a large initial in \Regius, marking the beginning of a new section.  In younger paper manuscripts they have the header \emph{Rúna-tals þáttr} ‘Strand of the Rune-Tally’.  They give an ancient mystical impression; one feels that it is not unlikely they were drawn from the lips of an Odinic priest.

Outside of \Havamal\ (cf. st. 80 above which would fit seamlessly into the present section) there are a few other manuscript attestations of similar Runic magic, but nothing quite like them.  \textlink{Sigrdrifumal}[5]–17 is also preserved in \Regius, although there are signs that point towards it being a later antiquarian composition, something sts. 138–146 are almost certainly not.

\sectionline

\bvg\bva\mssnote{\Regius~6v/27}%
\Ballnote{The myth of Weden’s self-sacrifice (“the Hanging”) is told only here and nowhere else.  Still, the god has a strong association with hanging throughout the Norse corpus.  He is known as \emph{Hangi} ‘Hanged One’ (Tindr \emph{Hákdr} 1/5 (in \Skp\ 1), cf. note there), \emph{Hanga-týr} ‘Tew of Hanged Ones’ (\Skaldskaparmal\ 7) and \emph{Hanga-guð} ‘God of Hanged Ones’ (\Gylfaginning\ 20); in st. 158 below he speaks of bringing a hanged man’s corpse back to life.

The method of the sacrifice described in the present st. bears close resemblance to an episode in \GautreksSaga\ 7.  In exchange for good wind king Wiker and his men, with whom is Starked, agree to hang one of themselves as a sacrifice to Weden.  They draw lots and they fall upon king Wiker.  Troubled by this they agree to perform a mock hanging, and Starked, whom Weden has secretly given a spear disguised as a reed, leads it.  The rest is best repeated in full:

\emph{Þar stóð fura ein hjá þeim ok stofn einn hár nę́r furu’nni.  Neðar⸗liga af furu’nni stóð einn kvistr mjór ok tók í lim’it upp.  Þá bjuggu þjónustu-sveinar mat manna, ok var kálfr einn skorinn ok krufðr.  Starkaðr lét taka kálfs-þarma’na.  Síðan steig Starkaðr upp á stofn’inn ok sveigði ofan þann inn mjóva kvist’inn ok knýtti þar um kálfs-þǫrmu’num.  Þá mę́lti Starkaðr til konungs:  „Nú er hér búinn þér gálgi, konungr, ok mun sýnast eigi all-mann-hę́ttligr.  Nú gakk-tu hingat, ok mun ek leggja snøru á háls þér.“ Konungr mę́lti: „Sé þessi um·búð ekki mér hę́ttu⸗ligri en mér sýnist, þá vę́nti ek, at mik skaði þetta ekki, en ef ǫðru-vís er, þá mun auðna ráða, hvat at gerist.“  Síðan steig hann upp á stofn’inn, ok lagði Starkaðr virgul’inn um háls honum ok steig síðan ofan af stofni’num.  Þá stakk Starkaðr sprota’num á konungi ok mę́lti: „Nú gef ek þik Óðni.“ Þá lét Starkaðr lausan furu-kvist’inn.  Reyr-sprot’inn varð at geir, ok stóð í gegnum konung’inn. Stofninn fell undan fótum honum, en kálfs-þarmar’nir urðu at viðju sterkri, en kvistr’inn reis upp ok hóf upp konung’inn við limar, ok dó hann þar.}

‘There stood a tall pine close to them and a tall stump next to the pine.  At the lower part of the pine there was a thin branch and it curved upwards into the leafage.  Then the servant boys made food for the men and a calf was cut in its throat and opened up.  Starked had the calf-intestines taken out.  Thereafter Starked stepped onto the stump and pulled down the thin branch and tied around it the calf-intestines.  Then Starked spoke to the king: “Now a gallows is readied for thee, O king, and it will not seem dangerous to any man.  Now go hither and I will lay the rope onto thy neck.” The king spoke: “If this undertaking be of as little harm to me as it looks then I expect that this will not harm me, but if it is otherwise then fortune will decide what happens afterwards.”  Thereafter he stepped onto the stump, and Starked laid the noose around his neck and thereafter stepped down from the stump.  Then Starked stabbed the reed into the king and spoke: “Now I give thee to Weden!” Then Starked let go of the pine-branch.  The reed turned into a spear and it pierced through the king.  The stump fell away from under his feet, but the calf-intestines became a strong withy, but the branch rose up and lifted the king up into the leafage, and there he died.’

Several motifs from \GautreksSaga\ 7 are attested in other Norse sources: in \StyrbjarnarThattr\ 2 Weden gives king Eric a reed which upon the uttering of a devotional formula turns into a spear and inagurates a large-scale human sacrifice in the form of a landslide burying an enemy host (cf. \textlink{Voluspa}[23] n.); in \textlink{HelgakvidaTwo}[P11] Weden lends Day his spear; in \textcite[77]{Saxo} 1.8.27 the Odinic king Harding ends his life by hanging himself before his people (for the Odinic nature of this sacrifice cf. \textciteshorttitle{PCRN-HS} III:42, pp. 1158, 1162, 1174–76); and in \YnglingaSaga\ 9 Weden marks himself with a spear on his deathbed and is followed by Nearth, who continues the practice by before dying “marking” himself (presumably with a spear) to Weden.

With these parallels in mind it seems likely that the method of sacrifice described in the present st. has historical ritual bases.  It is only fitting that a sacrifice to Weden—even if the victim be the god himself—should be carried out in the way such sacrifices ought to be, and the past mythological actions of gods tend to serve as the basis for present human ritual (\textlink{Voluspa}[7] n., \textlink{Grimnismal}[41]–43 n.).  Further instances of hanging-sacrifices include Adam of Bremen (TODO: reference) who mentions the hanging of nine men in a grove at the Temple of Upsal, and although the god to whom they are sacrificed is not named, it can only really be Weden.  For discussion of the devotion of armies and war-captives to Weden, cf. \textlink{Voluspa}[23] n.}%
\edtext{\alst{V}ęit’k}{\Afootnote{\emph{V} has a descending initial with a height of two lines.}} at ek hekk \hld\ \edtrans{\alst{v}indga-męiði}{the windswept tree}{\Bfootnote{Generally understood to be a form of \emph{vinga-męiðr} ‘gallows tree’, a word found in three Scaldic stanzas (\Skp: Egill Lv 14, Eyv \emph{Hál} 5, Anon (\emph{FoGT}) 17).
The form \emph{vindga-} ‘windswept, windy’ is most likely a folk etymological substitution for \emph{vinga-}, which has nothing to do at all with wind but is better interpreted as ‘swaying’ (related to words like Swedish \emph{vingla} ‘sway, wobble’, German \emph{wanken} ‘sway; stagger’, cf. \textlink{Thrymskvida}[1]/1 n.).  The original sense of \emph{vinga-męiðr} is thus the ‘tree which sways back and forth’, viz. under the weight of the hanged man’s body.}} ȧ &%NOTE: not in Kroonen2013.
\ind \alst{n}ę́tr allar \alst{n}íu, &
\alst{g}ęiri undaðr \hld\ ok \alst{g}efinn Óðni, &
\ind \alst{s}jalfr \alst{s}jǫlfum mér, &
ȧ \edtext{þęim \alst{m}ęiði, \hld\ es \alst{m}ann-gi vęit, &
\ind hvęrs af \alst{r}ótum \alst{r}innr.}{\lemma{ȧ þęim męiði, es mann-gi vęit, hvęrs af rótum rinnr ‘on that tree where no man knows of whose roots it runs.’}\Bfootnote{Probably \inx[P]{Ugdrassle’s Ash}, which is named after this hanging, being the “ash-tree of Ug’s \name{Weden} gallows”. The unknown origin of its roots clearly adds to the mystery of the self-sacrifice.}}\eva

\bvb {\huge I} \textsc{know that I hung} on the windswept tree \\
\ind for whole nights nine, \\
wounded by a spear and given to Weden, \\
\ind Myself to Myself \\
on that tree where no man knows \\
\ind of whose roots it runs.\evb\evg


\bvg\bva\mssnote{\Regius~6v/29}%
\edtrans{Við \alst{h}lęifi mik sǿldu⸗t \hld\ né við \alst{h}orni-gi}{With no loaf they relieved me, nor with any horn}{\Bfootnote{I.e. “I got neither bread to eat nor ale to drink.”}}; &
\alst{n}ýsta ek \alst{n}iðr, \hld\ \alst{n}am’k upp ru̇nar, &
\alst{ǿ}pandi nam, \hld\ fell’k \alst{a}ptr þaðan.\eva

\bvb With no loaf they relieved Me, nor with any horn. \\
I peered down; I took up the runes; \\
screaming I took—I fell back thence.\evb\evg


\bvg\bva\mssnote{\Regius~6v/31}%
\Ballnote{Here the poem moves away from the subject of the Hanging to the subject of how Weden learned his galders (ll. 1–2) and poetry (3–4).}%
\edtrans{\alst{F}imbul-ljóð níu}{Nine fimble-leeds}{\Bfootnote{Nine very great chants or spells (\inx[C]{galders}); compare the eighteen (9 times 2) leeds in \textlink{Havamal}[147]–165 below.}} \hld\ nam’k af \edtext{hinum \alst{f}rę́gja syni &
\ind \alst{B}ǫl·þorns, \alst{B}ęstlu fǫður,}{\lemma{hinum frę́gja syni Bǫl·þorns, Bęstlu fǫður ‘the famed son of Balethorn, Bestle’s father’}\Bfootnote{According to \Gylfaginning\ 6: \emph{[Borr] fekk þeirar konu, er Bestla hét, dóttir Bǫl·þorns jǫtuns, ok fengu þau þrjá sonu; hét einn Óðinn, annarr Vili, þriði Vé [\dots]} ‘[Byre] got for his wife the woman called Bestle, the daughter of the ettin Balethorn, and they had three sons.  One was called Weden, the other Will, the third Wigh.’  Balethorn’s son is thereby Weden’s maternal uncle, an instance of the old Indo-European custom of sending sons away to be fostered by the mother’s male relations.  Cf. TODO: some reference on this practice.}} &
ok ek \alst{d}rykk of gat \hld\ hins \alst{d}ýra mjaðar &
\ind \alst{au}sinn \alst{Ó}ð·rǿri.\eva

\bvb Nine \inx[C]{fimble}-leeds I learned from the famed son \\
\ind of \inx[P]{Balethorn}, \inx[P]{Bestle}’s father— \\
and a drink I got of the dear mead \\
\ind poured from \inx[P]{Woderearer}.\evb\evg


\bvg\bva\mssnote{\Regius~6v/33}%
Þȧ \edtrans{nam’k \alst{f}rę́va⸗sk}{I began to flourish}{\Bfootnote{A notorious mistranslation popularized by \textcite{Greenberg1988} has rendered these words as “I took semen”.  They would supposedly reference Weden stealing the ejaculate from hanged men in order to replenish his own powers—something not otherwise attested.  This preposterous notion makes no sense in the context of the text and has no philological grounding.  While Old Norse \emph{frę́} does mean “seed”, it only refers to the seeds of plants, not the seed animals or men.  Regardless, \emph{frę́va⸗sk} is without doubt a reflexive verb literally meaning something like ‘cultivate oneself’.}} \hld\ ok \alst{f}róðr vesa &
\ind ok \alst{v}axa ok \alst{v}ęl hafa⸗sk; &
\edtext{\alst{o}rð mér af \alst{o}rði \hld\ \alst{o}rðs lęitaði &
\alst{v}erk mér af \alst{v}erki \hld\ \alst{v}erks lęitaði.}{\lemma{orð \dots\ lęitaði. ‘My word \dots\ sought out.’}\Bfootnote{Every good speech led to another; every good deed likewise.}}\eva

\bvb Then I took to flourish and be wise, \\
\ind and grow and have it well. \\
My word from a word a word sought out; \\
My work from a work a work sought out.\evb\evg


\bvg\bva\mssnote{\Regius~6v/35}%
\edtrans{\alst{R}u̇nar munt finna \hld\ ok \alst{r}áðna stafi}{Runes wilt thou find and counselled staves’}{\Bfootnote{A strong resemblance is found in the long-line on the mediæval runestone N 13: \emph{ru̇nar ek ríst \hld\ ok ráðna stafi} ‘runes I carve, and counselled staves.’}}, &
\ind mjǫk \alst{st}óra \alst{st}afi, &
\ind mjǫk \alst{st}inna \alst{st}afi, &
\ind es \alst{f}áði \alst{F}imbul·þulr &
\ind ok \alst{g}ørðu \alst{g}inn-ręgin &
\ind ok \alst{r}ęist Hroptr \edtrans{\alst{r}\emph{a}gna}{of the Reins}{\Afootnote{\emph{‘rǫgna’} \Regius}}.\eva

\bvb \inx[C]{rune}[Runes] wilt thou find and counselled staves: \\
\ind very great staves, \\
\ind very stiff staves, \\
\ind which the \inx[P]{Fimblethyle} \name{= Weden} painted, \\
\ind and the \inx[P]{Yin-Reins} made, \\
\ind and Roft of the Reins carved.\evb\evg


\bvg\bva\mssnote{\Regius~7r/2}%
\alst{Ó}ðinn með \alst{ǫ̇}sum, \hld\ en fyr \alst{ǫ}lfum Dáinn, &
\ind \alst{D}valinn \alst{d}vergum fyrir, &
\ind \alst{Á}sviðr \alst{jǫ}tnum fyrir, &
\ind \edtrans{ek}{I}{\Bfootnote{The identity of the speaker is unclear, but judging by line 1 is apparently not Weden.}} ręist \alst{s}jalfr \alst{s}umar.\eva

\bvb \inx[P]{Weden} among the \inx[P]{Eese} but \inx[P]{Dowen} for the \inx[P]{Elves}, \\
\ind \inx[P]{Dwollen} for the \inx[P]{Dwarfs}, \\
\ind \inx[P]{Oswith} for the Ettins; \\
\ind I myself carved some.\evb\evg


\bvg\bva\mssnote{\Regius~7r/3}%
\Ballnote{The first four verbs refer to runes—carving, interpreting, colouring (with blood?), and divining, the latter four to sacrifice—praying, worshipping, sending (the sacrifice or the prayer), and killing the victim.  See further relevant Index entries: bloot, soo. — The meter of this st. is unusual, but bears some resemblance to Vg 216 (the Högstena galder). TODO: Elaborate.}%
Vęitst, hvé \alst{r}ísta skal? \hld\ Vęitst, hvé \alst{r}áða skal? &
Vęitst, hvé \alst{f}áa skal? \hld\ Vęitst, hvé \alst{f}ręista skal? &
Vęitst, hvé \alst{b}iðja skal? \hld\ Vęitst, hvé \alst{b}lóta skal? &
Vęitst, hvé \alst{s}ęnda skal? \hld\ Vęitst, hvé \alst{s}óa skal?\eva

\bvb Knowest thou how one shall carve? Knowest thou how one shall read? \\
Knowest thou how one shall paint? Knowest thou how one shall try? \\
Knowest thou how one shall bid? Knowest thou how one shall \inx[C]{bloot}? \\
Knowest thou how one shall send? Knowest thou how one shall \inx[C]{soo}?\evb\evg


\bvg\bva\mssnote{\Regius~7r/5}%
\Ballnote{An identical progression of four verbs suggests a close relation with the previous st. — I agree with \textcite{Males2024} on the interpretation of this stanza: since a gift always requires recompense, an excessive sacrifice could be seen as offensive and upset the relationship with the god.  Males draws the analogy with an episode in \EgilsSaga, where a rival poet leaves an expensive shield for Eyel and rides off; the latter understands this as a demand to compose a poem about the shield and is greatly insulted.

The gift cycle between Gods and men is very important in Indo-European pagan religions, and the present st. is a Norse attestation; cf. in OI (\emph{Taittirīya Saṃhitā} 1.8.4.1, during the \emph{Rāja-sūya}, the consecration of a king): \emph{Dēhí mē, dádāmi tē; ní mē dʰēhi, ní te dadʰē; ni·hā́ram ín ní mē harā, ni·hā́ram} ‘Give to me, I give to thee; bestow upon me, I bestow upon thee; bring gain indeed to me, [I bring] gain [to thee],’ in Latin (TODO: source) \emph{dō ut dēs} ‘I give that thou might give’.}%
\alst{B}ętra ’s ȯ·\alst{b}eðit \hld\ an séi of·\alst{b}lótit, &
\ind ęy sér til \alst{g}ildis \alst{g}jǫf; &
bętra ’s ȯ·\alst{s}ęnt \hld\ an séi of·\alst{s}óit; &
\edtext{[...]}{\Bfootnote{For metrical reasons it is very likely that a line has been lost here.}}\eva

\bvb It is better unbid than over-\inx[C]{bloot}[blooted]; \\
\ind a gift always looks for recompense. \\
It is better unsent than over-\inx[C]{soo}[sooed]; \\
{[...]}\evb\evg


\bvg\bva\mssnote{\Regius~7r/7}%
\Ballnote{This stanza is obscure and the section to which it belongs is unclear; \emph{svá} ‘so, thus’ may be referring back to the preceding sts. or to the ones ahead.  Regardless of which section it is referring to, it describes its contents as the verses Weden wrote down immediately after learning the runes during the Hanging (sts. 138–139 above) and places this at a time long predating human history.}%
Svá \edtrans{\alst{Þ}undr}{Thound}{\Bfootnote{A common poetic name for Weden with an opaque etymology.  Due to the genitive \emph{Þundar} it must be an \emph{i}- or \emph{u}-stem (< PGmc. \emph{*Þundiz} or \emph{*Þunduz}).  I find the most promising etymology to be a derivative of the IE root \emph{*ten-} ‘stretch, extend’ (surviving in the causative as ON \emph{þęnja} ‘to stretch’) extended with \emph{-d-} (cf. Latin \emph{tendō} ‘to stretch’).
A direct cognate for \emph{*Þunduz} is obtained in Lithuanian \emph{tandùs} ‘lazy, sloppy’ < \emph{*ténd-u-s} \char`~\ \emph{*tn̥d-éw-s}, with Germanic regularizing the zero-grade stem of the oblique in all cases (for which cf. e.g. PGmc. \emph{*burþiz} \char`~\ \emph{*burþīz} ‘birth’ < PIE \emph{*bʰértis} \char`~\ \emph{*bʰr̥téys}).  The sense of the name would thus be ‘stretched’, probably referring to his neck during the hanging.}} of ręist \hld\ fyr \alst{þ}jóða rǫk, &
þar’s \alst{u}pp of ręis, \hld\ es \alst{a}ptr of kom.\eva

\bvb Thus did \inx[P]{Thound} \name{= Weden} carve before the histories of the nations \\
where he rose up when he came back.\evb\evg

\sectionline

\subsection{The Leed-Tally (147–165)}

This section of \textlink{Havamal}, the so-called the Leed-Tally (\emph{Ljóðatal}), is not separated from the preceding section (which is marked out with a large initial), but is usually taken as separate since it is a self-contained list not much concerned with runes.  The speaker, certainly Weden, addresses Loddfathomer and lists eighteen galders.  Of course, the spells themselves are not given, but only their purpose.  They are aristocratic and Odinic in character and deal with such things as battle (galders 3, 4, 5, 8, 11, 13), healing (1, 2, 12), countering sorcery (6, 10), controlling the elements (7, 9), and seduction (16, 17).  The 18th and last spell is a mystery; not even its purpose is told, and it is known only by Weden and the women closest to him.

Uniquely in this section we find Weden as a healer—a side of the god which is otherwise largely absent from the Old Norse literary corpus, where he instead tends to take on the role of War-God.  On the other hand this aligns well with his personification in West Germanic healing charms (the OE \NineHerbs\ and the OHG \MerseburgTwo).  Still, we need not look in vain for West Germanic influence in the present section; this difference is most likely simply a matter of which genres of texts have survived in the different traditions.

On the level of each numbered leed several of them are highly reminiscent of other known Germanic galders.  The fourth (st. 150) bears a strong likeness to \Grougaldr\ 10, and its effect (removing fetters) is shared with the High German \textlink{MerseburgOne}, apparently an actual galder of that type.  The mysterious eighteenth spell (st. 165) finds an interesting parallel in the unknowable eighteenth question posed by Weden in \textlink{Vafthrudnismal}[54].

\sectionline

\bvg\bva\mssnote{\Regius~7r/8}%
Ljóð \alst{þ}au kann’k, \hld\ es kann⸗at \alst{þ}jóðans kona &
\ind ok \alst{m}anns⸗kis \alst{m}ǫgr. &
\alst{H}jǫlp hęitir ęitt, \hld\ þat þér \alst{h}jalpa mun &
við \alst{s}orgum ok \edtrans{\alst{s}ǫkum}{sakes}{\Bfootnote{Legal charges, the first element of English \emph{sakeless}.}}, \hld\ ok \alst{s}útum gǫrv-ǫllum.\eva

\bvb Those \inx[C]{leed}[leeds] I know which the king’s wife knows not, \\
\ind and no man’s lad. \\
Help is one called; it will help thee \\
against sorrows and sakes and all griefs entire.\evb\evg


\bvg\bva\mssnote{\Regius~7r/10}%
Þat kann’k \alst{a}nnat, \hld\ es \edtrans{þurfu \alst{ý}ta synir}{those sons of men need}{\Bfootnote{Cf. the similar wording in 166/2.}}, &
\ind þęir’s vilja \edtrans{\alst{l}ę́knar}{leechers}{\Bfootnote{Physicians.  A relation to ‘leech’ is likely but the direction is unclear \parencite[331]{Kroonen2013}; are the physicians named after the tool of their trade (leeches, for bloodletting), or are the leeches named after the physicians?}} \alst{l}ifa.\eva

\bvb This I know second, which those sons of men need \\
\ind who wish to live as leechers.\evb\evg


\bvg\bva\mssnote{\Regius~7r/11}%
Þat kann’k \alst{þ}riðja, \hld\ ef mér verðr \alst{þ}ǫrf mikil &
\ind \alst{h}apts við mïna \alst{h}ęipt-mǫgu, &
\alst{ę}ggjar dęyfi’k \hld\ mïnna \alst{a}nd·skota, &
\ind bíta⸗t þęim \alst{v}ǫ́pn né \edtrans{\alst{v}ęlir}{staffs}{\Bfootnote{Plural of \emph{vǫlr}, here referring to the magic staff or sceptre used by witches and warlocks; the word \emph{vǫlva} ‘\inx[C]{wallow}’ (seeress, prophetess) is probably derived from this word.  The reading \emph{vélir} ‘wiles, tricks, deceits’ must be excluded for metrical reasons, since a c-verse in \Ljodahattr\ cannot end in a trochée.}}.\eva

\bvb This I know third, if I come in great need \\
\ind of hindrance against my feud-lads \ken{foes}: \\
I dull the blades of my enemies; \\
\ind for them bite not weapons nor staffs.\evb\evg


\bvg\bva\mssnote{\Regius~7r/13}%
\Ballnote{Cf. \Grougaldr\ 10, which is very similar to the present stanza, and \textlink{MerseburgOne} (edited below under Galders), a galder that seems actually to have been used for loosening fetters.}%
Þat kann’k \alst{f}jórða, \hld\ ef mér \alst{f}yrðar bera &
\ind \alst{b}ǫnd at \alst{b}óg-limum, &
svá ek \alst{g}ęl, \hld\ at \alst{g}anga má’k, &
\ind sprettr mér af \alst{f}ótum \alst{f}jǫturr, &
\ind en af \alst{h}ǫndum \alst{h}apt.\eva

\bvb This I know fourth, if men bear \\
\ind bonds onto my shoulder-limbs \ken{arms}: \\
\emph{so} do I gale that I may walk; \\
\ind springs from my feet the fetter, \\
\ind and from my hands the bond.\evb\evg


\bvg\bva\mssnote{\Regius~7r/15}%
Þat kann’k \alst{f}imta, \hld\ ef sé’k af \alst{f}ári skotinn &
\ind \alst{f}lęin ï \alst{f}olki vaða, &
flýgr⸗a svá \alst{st}int, \hld\ at \alst{st}ǫðvi’g⸗a’k, &
\ind ef hann \alst{s}jȯnum of \alst{s}é’k.\eva

\bvb This I know fifth, if I see a dangerously shot \\
\ind arrow in the troop wading: \\
it flies not so stiff that I cannot stop it \\
\ind if I see it in my sights.\evb\evg


\bvg\bva\mssnote{\Regius~7r/16}%
Þat kann’k \alst{s}étta, \hld\ \edtext{ef mik \alst{s}ę́rir þegn &
\ind ȧ \alst{r}ótum \edtrans{\alst{r}ás}{raw/sappy}{\Bfootnote{The sappy wood was apparently of importance for the curse; cf. the curious account of \Grettissaga\ 79, where a hag curses Gretter in the following way: after finding a small tree and planing a small smooth surface onto a scorched side of it, she carves runes in its roots and reddens them with her own blood.  She then chants \inx[C]{galder}[galders] while walking counter-clockwise around it.  Lastly she pushes it out to sea, praying for it to drift to Gretter’s homestead and curse him.  Cf. also \textlink{Skirnismal}[32] where Shirner goes to a \emph{hrár viðr} ‘raw/sappy tree’ to get a certain magic stick. —
The normal form of this word is \emph{hrár} (so \textlink{Skirnismal}[32]), but the required alliteration with \emph{rótum} makes that an impossibility.  Another alternative is to emend \emph{*v-} in \emph{*vrótum} for a line \emph{**ȧ \alst{v}rótum hrás \alst{v}iðar}.
The emendation relies on an etymological relation between ON \emph{rót} and OE \emph{wyrt}, but as \textcite[597, 601]{Kroonen2013} points out \emph{rót} has an initial \emph{r-} even in East Nordic which otherwise consistently preserves old \emph{vr-}.  Naturally this calls the existence of an OWN form \emph{*vrót} into question.  Further, \emph{rót} alliterates with \emph{r-} in st. 138 above.}} viðar}{\lemma{ef mik sę́rir þegn ȧ rótum rás viðar ‘if a thane wounds me on the roots of a raw/sappy tree’}\Bfootnote{I.e., “if a man carves a runic curse against me”.}}, &
\edtrans{þann \alst{h}al}{that man}{\Afootnote{\emph{ok þann hal} ‘and that man’ \Regius}}, \hld\ es mik \alst{h}ęipta kveðr, &
\ind þann eta \alst{m}ęin hęldr an \alst{m}ik.\eva

\bvb This I know sixth, if a thane wounds me \\
\ind on the roots of a raw/sappy tree: \\
\emph{that man} who sings hatred against me, \\
\ind \emph{him} the evils eat instead of me.\evb\evg


\bvg\bva\mssnote{\Regius~7r/18}%
Þat kann’k \alst{s}jaunda, \hld\ ef \alst{s}é’k hǫ́van \edtrans{loga}{ablaze}{\Bfootnote{The word order makes this word look like the noun \emph{logi} ‘flame’ (“if I see a high flame”), but the noun modified by the adj. \emph{hǫ́van} ‘high’ is in fact \emph{sal} ‘hall’, and \emph{loga} is a verb ‘to burn, be ablaze’.}} &
\ind \alst{s}al of \alst{s}ess-mǫgum, &
\alst{b}rinnr⸗at svá \alst{b}ręitt, \hld\ at hǫ̇num \alst{b}jargi’g⸗a’k; &
\ind þann kann’k \edtrans{\alst{g}aldr}{galder}{\Bfootnote{The use of this word here makes the synonymity of “galder” and “leed” (\emph{ljóð}) clear.}} at \alst{g}ala.\eva

\bvb This I know seventh, if I see a high hall \\
\ind ablaze over seat-lads \ken{warriors}: \\
it burns not so broadly that I cannot save it— \\
\ind that galder I can gale.\evb\evg


\bvg\bva\mssnote{\Regius~7r/20}%
Þat kann’k \alst{á}tta, \hld\ es \alst{ǫ}llum es &
\ind \alst{n}yt-sam⸗ligt at \alst{n}ema, &
\alst{h}var’s \edtrans{\alst{h}atr}{hatred}{\Bfootnote{Naturally with regard to their father’s inheritance.  As any reader of European history will know, such conflicts were a constant source of war both in the mediæval Germanic-founded kingdoms and in the Roman Empire before them.}} vęx \hld\ með \alst{h}ildings sonum, &
\ind þat má’k \alst{b}ǿta \alst{b}rátt.\eva

\bvb This I know eighth, which for all men is \\
\ind useful to learn: \\
wherever hatred grows among a prince’s sons, \\
\ind it I may shortly mend.\evb\evg


\bvg\bva\mssnote{\Regius~7r/22}%
Þat kann’k \alst{n}íunda, \hld\ ef mik \alst{n}auðr of stęndr &
\ind at bjarga \alst{f}ari mïnu ȧ \alst{f}loti, &
\alst{v}ind ek kyrri \hld\ \alst{v}ági ȧ &
\ind ok \alst{s}vę́fi’k allan \alst{s}ę́.\eva

\bvb This I know ninth, if I come in need \\
\ind of saving my ship where it floats: \\
the wind I calm upon the wave, \\
\ind and put all the sea asleep.\evb\evg


\bvg\bva\mssnote{\Regius~7r/23}%
Þat kann’k \alst{t}íunda, \hld\ ef sé’k \edtrans{\alst{t}ún-riður}{town-rideresses}{\Bfootnote{The \emph{riður} ‘rideresses’ were witches believed to leave their original human shapes or skins (\emph{hamir}) in order to fly (“ride”) in the air tormenting and injuring the townsfolk.  When they were out riding their original bodies would be lying in a coma-like state, but it was not the case that their whole mental faculties would disconnect from their bodies; indeed they would leave something of their soul behind, which was thought to be inextricably linked to the body.  Through his second sight Weden could see these rideresses, and through his superior magical skill he could confuse them so that they would not be able to return to their original forms or minds, instead being doomed to stray as tormented “homeless” ghosts.  Weden brags about doing this in \textlink{Harbardsljod}[20].}} &
\ind \alst{l}ęika \alst{l}opti ȧ, &
ek svá \alst{v}inn’k, \hld\ at \edtrans{þę́r \alst{v}illar fara}{they (\emph{fem.}) go astray}{\Afootnote{emend.; \emph{þęir villir fara} ‘they (\emph{masc.}) go astray’ \Regius}} &
\ind sinna \alst{h}ęim-\alst{h}ama &
\ind sinna \alst{h}ęim-\alst{h}uga.\eva

\bvb This I know tenth, if I see \inx[P]{town-rideresses} \\
\ind playing aloft: \\
I work it so that they go astray \\
\ind of their home-\inx[C]{hame}[hames], \\
\ind of their home-minds.\evb\evg


\bvg\bva\mssnote{\Regius~7r/25}%
Þat kann’k \alst{ę}llipta, \hld\ ef skal’k til \alst{o}rrostu &
\ind \alst{l}ęiða \edtrans{\alst{l}ang-vini}{old friends}{\Bfootnote{In Germanic paganism the followers and protégés of a god are his friends (\emph{vinir}).  Already in \Beowulf\ we see that the Shieldings are called the \emph{Ing-wine} ‘friends of \inx[P]{Ing}’, and in \textlink{Hymiskvida}[11] Thunder is called the \emph{vinr ver-liða} ‘friend of manly retinues’.  Two other places where it is used of Weden’s followers in particular are \textlink{Grimnismal}[54] and \Sonatorrek\ 22, where Eyel speaks about his friendship (\emph{vin-átt}) with Weden.}}, &
und \alst{r}andir gęl’k, \hld\ en þęir með \alst{r}íki fara, &
\ind \alst{h}ęilir \alst{h}ildar til, &
\ind \alst{h}ęilir \alst{h}ildi frȧ, &
\ind koma þęir \alst{h}ęilir \alst{h}vaðan.\eva

\bvb This I know eleventh, if I shall into the fray \\
\ind lead old friends: \\
beneath the shield-rims I gale, and they go with power \\
\ind hale to the battle, \\
\ind hale from the battle; \\
\ind they come hale anywhence.\evb\evg


\bvg\bva\mssnote{\Regius~7r/27}%
Þat kann’k \alst{t}olpta, \hld\ ef sé’k ȧ \alst{t}ré uppi &
\ind \alst{v}áfa \alst{v}irgil-ná, &
svá ek \alst{r}íst \hld\ ok ï \alst{r}u̇num fá’k, &
\ind at sá \alst{g}ęngr \alst{g}umi. &
\ind ok \alst{m}ę́lir við \alst{m}ik.\eva

\bvb This I know twelfth, if I see in a tree up high \\
\ind sway a gallow-corpse: \\
so I carve and paint in the runes, \\
\ind that that man walks \\
\ind and speaks with me.\evb\evg


\bvg\bva\mssnote{\Regius~7r/29}%
Þat kann’k \alst{þ}rėt-tȧnda \hld\ \edtext{ef skal’k \alst{þ}egn ungan &
\ind \alst{v}erpa \alst{v}atni ȧ,}{\lemma{ef skal’k þegn ungan verpa vatni ȧ ‘if on a young thane I shall sprinkle water’}\Bfootnote{A reference to the Heathen name-giving ceremony in which the infant would be sprinkled with water; cf. the attestations in \textlink{Rigsthula}[7], 21, 34.}} &
mun⸗at hann \alst{f}alla \hld\ þó’tt ï \alst{f}olk komi, &
\ind \alst{h}nígr⸗a sá \alst{h}alr fyr \alst{h}jǫrum.\eva

\bvb This I know thirteenth, if on a young thane \\
\ind I shall sprinkle water: \\
he will not fall though he come into battle; \\
\ind that hero will not sink before swords.\evb\evg


\bvg\bva\mssnote{\Regius~7r/31}%
Þat kann’k \alst{f}jór-tȧnda, \hld\ ef skal’k \alst{f}yrða liði &
\ind \alst{t}ęlja \alst{t}íva fyr, &
\alst{ȧ}sa ok \alst{a}lfa \hld\ ek kann \alst{a}llra \edtrans{skil}{discernments}{\Bfootnote{Their unique attributes.  Cf. \textlink{Hymiskvida}[38], where the corresponding verb \emph{skilja} ‘to discern, understand’ is used in the context of god-lore.}}, &
\ind fár kann ȯ·\alst{s}notr \alst{s}vá.\eva

\bvb This I know fourteenth, if before the troop of men \\
\ind I shall count forth the Tews: \\
of the Eese and Elves all I know the discernments; \\
\ind few unwise men can do so.\evb\evg


\bvg\bva\mssnote{\Regius~7r/33}%
\alst{Þ}at kann’k fimm-tȧnda, \hld\ es gól \alst{Þ}jóð·rǿrir &
\ind \alst{d}vergr fyr \alst{D}ęllings \alst{d}urum, &
\alst{a}fl gól \alst{ǫ̇}sum, \hld\ en \alst{ǫ}lfum frama, &
\ind \alst{h}yggju \edtrans{\alst{H}ropta-tý}{Tew of the Rofts \name{= Weden}}{\Bfootnote{The older dative \emph{-tívi} (\Haustlong\ 8/1a, \Thorsdrapa\ 20/3a) is not allowed for metrical reasons as a \Ljodahattr\ c-verse cannot end in a long syllable followed by another syllable.}}.\eva

\bvb This I know fifteenth, which Thedrearer galed, \\
\ind the dwarf, before Delling’s doors. \\
Strength he galed for the Eese, but fame for the Elves, \\
\ind thought for Tew of the Rofts \name{= Weden}.\evb\evg


\bvg\bva\mssnote{\Regius~7r/35}%
Þat kann’k \alst{s}ex-tȧnda, \hld\ ef vil’k hins \alst{s}vinna mans &
\ind hafa \alst{g}ęð allt ok \alst{g}aman, &
\alst{h}ugi \alst{h}vęrfi’k \hld\ \alst{h}vit-armri konu &
\ind ok \alst{s}ný’k hęnnar ǫllum \alst{s}efa.\eva

\bvb This I know sixteenth, if I will from the smart girl \\
\ind have her senses all, and pleasure: \\
the heart I change in the white-armed woman, \\
\ind and I twist her whole mind.\evb\evg


\bvg\bva\mssnote{\Regius~7v/2}%
Þat kann’k \alst{s}jau-tjȧnda \hld\ at mik \alst{s}ęint mun firra⸗sk &
\ind hit \alst{m}an-unga \alst{m}an.\eva

\bvb This I know seventeenth, that she’ll lately shun me, \\
\ind that girl-young girl.\evb\evg


\bvg\bva\mssnote{\Regius~7v/2}%
\alst{L}jóða þęssa \hld\ munt \alst{L}odd·fáfnir &
\ind lęngi \alst{v}anr \alst{v}esa; &
\ind þó séi þér \alst{g}óð ef \alst{g}etr, &
\ind \alst{n}ýt ef \alst{n}emr, &
\ind \alst{þ}ǫrf ef \alst{þ}iggr.\eva

\bvb (These leeds wilt thou, Loddfathomer, \\
\ind long be lacking! \\
\ind Though they might be good for thee if thou gettest, \\
\ind profitable if thou learnest, \\
\ind needful if thou takest.)\evb\evg


\bvg\bva\mssnote{\Regius~7v/4}%
Þat kann’k \alst{á}t-tjȧnda, \hld\ es \alst{ę́}va kęnni’k &
\ind \alst{m}ęy né \alst{m}anns konu, &
—\alst{a}llt es bętra \hld\ es \alst{ęi}nn of kann, &
\ind þat fylgir \alst{l}jóða \alst{l}okum— &
nema þęiri \alst{ęi}nni, \hld\ es \edtrans{mik \alst{a}rmi vęrr}{holds me in her arms}{\Bfootnote{A similar expression is also used \textlink{Volundarkvida}[2].  The one who wraps Weden in her arm may be his wife, \inx[P]{Frie}.}}, &
\ind eða mïn \alst{s}ystir \alst{s}éi.\eva

\bvb This I know eighteenth, which I will never teach \\
\ind a maiden nor man’s woman, \\
(everything is better which one alone knows; \\
\ind that follows the last of the leeds!) \\
save for her alone who holds me in her arms, \\
\ind or is my sister.\evb\evg

\sectionline

\bvg\bva\mssnote{\Regius~7v/7}%
Nú eru \alst{H}ǫ́va mǫ́l kveðin \hld\ \alst{H}ǫ́va \alst{h}ǫllu ï; &
\ind \alst{a}ll-þǫrf \alst{ý}ta sonum, &
\ind \alst{ȯ}·þǫrf \edtrans{\alst{jǫ}tna}{ettins}{\Afootnote{corr. by other hand from \emph{ýta} ‘men’ \Regius}} sonum; &
hęill sá’s \edtext{\alst{k}vað, \hld\ hęill sá’s \alst{k}ann, &
\ind \alst{n}jóti sá’s \alst{n}am, &
\ind \alst{h}ęilir þęir’s \alst{h}lýddu}{\lemma{kvað, kann, nam, hlýddu ‘sang, knows, learned, heeded’}\Bfootnote{The implicit object is the speeches.  These verbs all indicate a fully oral cultural context.}}.\eva

\bvb Now have the High One’s speeches been sung in the High One’s hall, \\
\ind most useful for the sons of men; \\
\ind harmful for the sons of ettins. \\
Hail him who sang; hail him who knows; \\
\ind may he use who learned; \\
\ind hail those who heeded!\evb\evg

\sectionline
