\bookStart{Lay of Thrim}[Þryms kviða]
\setBookCode{Thrymskvida}

\begin{flushright}%
\textbf{Dating} \parencite{Sapp2022}: C9th (0.741)

\textbf{Meter:} \Fornyrdislag%
\end{flushright}

\section{Introduction}

The \textbf{Lay of Thrim} (\textlink{Thrymskvida}) is only found in \Regius, where it follows \textlink{Lokasenna} and precedes \textlink{Volundarkvida}.  It has oft been considered the oldest poem in the \Regius\ collection, and \textcite{Sapp2022}’s model agrees with that judgment.

Comedic stories involving Thunder and his ettin-bashing seem to have been very popular in Wiking age Norway and Iceland, and the god himself is not infrequently the butt of the joke in them.  Apart from \textlink{Thrymskvida} there are also the Eddic poems \textlink{Hymiskvida} and \textlink{Harbardsljod}, and the Scaldic poems \Haustlong and \Thorsdrapa.  Fragment from a lost Eddic poem about Thunder’s fight with the ettin Garfrith and his daughters also survive in \Gylfaginning; see \textlink{EddicFragments}[F6][6] below.

\sectionline

\section{Lay of Thrim}

\bvg\bva\mssnote{\Regius~17r/13}%
\edtext{\emph{\alst{V}}ręiðr}{\Afootnote{\emph{Reiðr} \Regius}\Bfootnote{Initial \emph{v-} is restored for the sake of better alliteration but is not strictly metrically necessary; cf. st 13 where it definitely does not alliterate.  \emph{vr-} is justified since \textlink{Thrymskvida}, which is generally considered to be the oldest Eddic poem, probably predates the West Norse sound change \emph{vr-} > \emph{r-}; cf. \textlink{Havamal}[32]/2 n.}} vas þȧ \edtrans{\alst{V}ing-Þȯrr}{Wing-Thunder}{\Bfootnote{A rare poetic synonym for Thunder; it only elsewhere occurs in \textlink{Alvissmal}[6].  See Index for etymology.}} \hld\ es hann \alst{v}aknaði &
ok \alst{s}ïns hamars \hld\ of \alst{s}aknaði, &
\edtext{\alst{sk}ęgg nam at hrista, \hld\ \alst{sk}ǫr nam at dýja}{\lemma{skęgg \dots\ dýja ‘beard \dots\ pull’}\Bfootnote{Apparently formulaic. Cf. \textlink{Brot}[13]/1.}}, &
réð \alst{Ja}rðar burr \hld\ \alst{u}mb at þręifask.\eva

\bvb {\huge W}\textsc{roth was then \inx[P]{Wing-Thunder}} when he woke, \\
and of his hammer was bereaved. \\
His beard he took to rustle, his locks he took to rip; \\
the son of Earth resolved to grope about.\evb\evg


\bvg\bva\mssnote{\Regius~17r/15}%
\edtrans{\alst{O}k hann þat \alst{o}rða \hld\ \alst{a}lls fyrst of kvað}{And he this word first of all did say}{\Bfootnote{The whole line is formulaic, occuring in five other places: sts. 3, 9 and 12 of the present poem, \textlink{Oddrunargratr}[3]/5, \textlink{Brot}[5]/2.}}: &
„\alst{H}ęyr-ðu nú, Loki, \hld\ \alst{h}vat ek nú mę́li &
es \alst{ęi}gi vęit \hld\ \edtext{\alst{ja}rðar hvęr-gi &
né \alst{u}pp-himins}{\lemma{jarðar \dots\ upp-himins ‘earth \dots\ up-heaven’}\Bfootnote{The whole cosmos.  Formulaic, see Index: \inx[F]{Earth and Up-heaven}.}}: \hld\ \alst{ǫ̇}ss es stolinn hamri!“\eva

\bvb And he this word first of all did say: \\
“Hear thou now, Lock, what I now speak, \\
which no man knows anywhere on earth \\
nor in up-heaven: the \inx[G]{Eese}[os] \ken*{= Thunder = I} is robbed of His hammer!”\evb\evg


\bvg\bva\mssnote{\Regius~17r/17}%
Gingu þęir \alst{f}agra \hld\ \alst{F}ręyju tu̇na &
\alst{o}k \edtrans{hann}{he}{\Bfootnote{The speaker is Thunder, since he speaks about “my hammer”.}} þat \alst{o}rða \hld\ \alst{a}lls fyrst of kvað: &
„Munt-u mér, \alst{F}ręyja, \hld\ \edtrans{\alst{f}jaðr-hams}{feather-hame}{\Bfootnote{A “feather-skin” by which the wearer can transform or fly like a bird.}} léa &
ef ek \alst{m}ïnn hamar \hld\ \alst{m}ę́tta’k hitta?“\eva

\bvb Went they to the fair yards of \inx[P]{Frow}, \\
and he this word first of all did say: \\
“Wilt thou me, O Frow, the \inx[C]{feather-hame} lend, \\
if I my hammer might find?”\evb\evg


\bvg\bva\mssnote{\Regius~17r/19}\speakernote{Fręyja kvað:}%
„Þó mynda’k \alst{g}efa þér \hld\ þó’tt ór \alst{g}olli vę́ri &
ok þó \edtrans{\alst{s}ęlja}{hand}{\Bfootnote{\emph{sęlja}, cognate of English \emph{sell}, here has its older sense of ‘hand over’, cf. Gotish \emph{saljan} ‘\emph{opfern}; \textgreek{θύειν}’ \parencite[116]{Streitberg}.}} \hld\ at vę́ri ór \alst{s}ilfri.“\eva

\bvb\speakernoteb{[Frow quoth:]}%
“Yet would I give it to thee though it were golden, \\
and yet hand it to thee if it were silvern.”\evb\evg


\bvg\bva\mssnote{\Regius~17r/20}%
\alst{F}ló þȧ \edtrans{Loki}{Lock}{\Bfootnote{Though Thunder is the one asking for the feather-hame (3/4 “if I \emph{my} hammer might find”), it is Lock who takes off flying.}}, \hld\ \alst{f}jaðr-hamr dunði, &
und’s fyr \alst{ú}tan kom \hld\ \alst{ȧ}sa garða &
ok fyr \alst{i}nnan kom \hld\ \alst{jǫ}tna hęima.\eva

\bvb Then flew Lock—the feather-hame rustled— \\
until he came outside the \inx[L]{Osyard}[Yards of the Eese], \\
and he came inside the \inx[L]{Ettinham}[Homes of the Ettins].\evb\evg


\bvg\bva\mssnote{\Regius~17r/22}%
\alst{Þ}rymr \edtrans{sat ȧ haugi}{sat on the mound}{\Bfootnote{Meditating on mounds was a common pastime for the ancients.  See \textlink{Voluspa} 41 for other attestations.}}, \hld\ \edtrans{\alst{þ}ursa dróttinn}{lord of Thurses}{\Bfootnote{This formulaic expression also occurs in several Runic charms against such thursen lords (see below under Galders); an example of the close connection between mythology and ritual.}}, &
\edtext{\alst{g}ręyjum sïnum \hld\ \alst{g}oll-bǫnd snøri &
ok \alst{m}ǫrum sïnum}{\lemma{gręyjum sïnum \dots\ mǫrum sïnum ‘his greyhounds \dots\ his steeds’}\Bfootnote{Thrim sits surrounded by dogs and horses.  The scene is reminiscent of the ancient “master of animals” motif, especially as attested on panel A of the Gundestrup cauldron.}} \hld\ \alst{m}ǫn jafnaði.\eva

\bvb Thrim sat on the mound, the lord of \inx[G]{Thurses}: \\
on his greyhounds the golden leashes he twirled, \\
and on his steeds the manes he cut even.\evb\evg


\bvg\bva\mssnote{\Regius~17r/23}%
\speakernote{[Þrymr kvað:]}%
„\edtrans{Hvat ’s með \alst{ǫ̇}sum? \hld\ Hvat ’s með \alst{ǫ}lfum?}{What is with the Eese? What is with the Elves?}{\Bfootnote{Formulaic, the same line occurs in \textlink{Voluspa}[46]/1.}} &
Hví est \alst{ęi}nn kominn \hld\ ï \alst{jǫ}tun-hęima?“ &
\speakernote{[Loki kvað:]}%
„\alst{I}llt ’s með \alst{ǫ̇}sum, \hld\ \edtext{\emph{\alst{i}llt ’s með \alst{ǫ}lfum}}{\Afootnote{required by sense and meter; om. \Regius}}! &
\alst{H}ęfir þú \edtrans{\alst{H}lór·riða}{Loride}{\Bfootnote{“The Loud Rider”, Thunder.}} \hld\ \alst{h}amar of folginn?“\eva

\bvb\speakernoteb{[Thrim quoth:]}%
“What’s with the Eese? What’s with the Elves? \\
Why art thou alone come into the \inx[L]{Ettinham}[Ettin-homes]?”— \\
\speakernoteb{[Lock quoth:]}%
“’Tis ill with the Eese! ’Tis ill with the Elves! \\
Hast thou the hammer of Loride \name{= Thunder} hidden?”\evb\evg


\bvg\bva\mssnote{\Regius~17r/25}%
\speakernote{[Þrymr kvað:]}%
„Ek \alst{h}ęfi \alst{H}lór·riða \hld\ \alst{h}amar of folginn &
\edtrans{\alst{á}tta rǫstum}{eight rests}{\Bfootnote{Eight leagues; a “rest” being an old distance measurement. See Index.}} \hld\ fyr \alst{jǫ}rð neðan; &
hann \alst{ę}ngi maðr \hld\ \alst{a}ptr of hęimtir &
nema \alst{f}ǿri mér \hld\ \edtrans{\alst{F}ręyju}{Frow}{\Bfootnote{Frow, who is probably originally the Indo-European dawn-goddess.  The Ettins apparently have a great desire to obtain her; in \Gylfaginning\ 42 the ettin who builds the wall of Osyard asks specifically for Frow in return, along with the Sun and the Moon (for which see \textlink{Voluspa}[24]).}} at kvę̇n.“\eva

\bvb\speakernoteb{[Thrim quoth:]}%
“I have the hammer of Loride hidden \\
eight \inx[C]{rest}[rests] beneath the earth. \\
It no man might fetch back, \\
unless he bring me Frow for a wife.”\evb\evg


\bvg\bva\mssnote{\Regius~17r/27}%
\alst{F}ló þȧ Loki, \hld\ \alst{f}jaðr-hamr dunði, &
und’s fyr \alst{ú}tan kom \hld\ \alst{jǫ}tna hęima &
ok fyr \alst{i}nnan kom \hld\ \alst{ȧ}sa garða; &
\edtrans{\alst{m}ǿtti hann Þȯr}{He met Thunder}{\Bfootnote{This line is compatible with the reconstructed disyllabic form \emph{*Þȯar} if the pronoun \emph{hann} is excised.  For that form see note to \textlink{Hymiskvida} 23/1.}} \hld\ \alst{m}iðra garða &
\alst{o}k \edtext{\emph{hann þat}}{\Afootnote{emend. (cf. st. 2 n.); \emph{þat hann} with elsewhere unprecedented word order \Regius}} \alst{o}rða \hld\ \alst{a}lls fyrst of kvað:\eva

\bvb Then flew Lock—the feather-hame rustled— \\
until he came outside the Homes of the Ettins \\
and he came inside the Yards of the Eese. \\
He met Thunder in the middle yards, \\
and he \ken*{= Thunder} this word first of all did say:\evb\evg


\bvg\bva\mssnote{\Regius~17r/29}%
„\edtrans{Hęfir þú \alst{ø}rendi \hld\ sem \alst{ę}rfiði?}{Hast thou an errand of hardship?}{\Bfootnote{Thunder asks Lock whether he is the bearer of ill tidings.  The rhyming pair \emph{ørendi} ‘errand’ \dots\ \emph{ęrfiði} ‘trouble, hardship’ is formulaic and occurs in X other (TODO!!) places, including \textlink{HelgakvidaHjorvardssonar} 5.}} &
Seg-ðu ȧ \alst{l}opti \hld\ \alst{l}ǫng tíðendi! &
\edtext{Opt \alst{s}itjanda \hld\ \alst{s}ǫgur of fallask, &
ok \alst{l}iggjandi \hld\ \alst{l}ygi of bęllir.}{\lemma{Opt sitjanda \hld\ sǫgur of fallask, // ok liggjandi · lygi of bęllir. ‘Oft the sitting man’s stories fail each other // and the lying down blows up his lie.’}\Bfootnote{Proverbial. If one waits and mulls over bad news after receiving them, details will be left out and excuses thought up.  It is therefore best that Lock immediately tell Thunder what he has learned.  ON \emph{liggja} ‘recline’ and \emph{ljúga} ‘speak untruth’ are entirely different verbs; it is very unfortunate that they sound the same in English.}}“\eva

\bvb “Hast thou an errand of hardship? \\
Tell thou the long tidings aloft! \\
Oft the sitting man’s stories fail each other \\
and the lying down blows up his lie.”\evb\evg


\bvg\bva\mssnote{\Regius~17r/31}\speakernote{[Loki kvað:]}%
„Hefi’k \alst{ø}rendi, \hld\ \alst{ę}rfiði ok: &
\alst{Þ}rymr hęfir þïnn hamar, \hld\ \alst{þ}ursa dróttinn; &
hann \alst{ę}ngi maðr \hld\ \alst{a}ptr of hęimtir &
nema hǫ̇num \alst{f}ǿri \hld\ \alst{F}ręyju at kvę̇n.“\eva

\bvb\speakernoteb{[Lock quoth:]}%
“I have an errand, hardship also: \\
Thrim has thy hammer, the lord of Thurses. \\
It no man will fetch back, \\
unless he bring him Frow for a wife.”\evb\evg


\bvg\bva\mssnote{\Regius~17r/33}%
Ganga þęir \alst{f}agra \hld\ \alst{F}ręyju at hitta &
\alst{o}k \edtrans{hann}{he}{\Bfootnote{The speaker is either Thunder or Lock.}} þat \alst{o}rða \hld\ \alst{a}lls fyrst of kvað: &
„\alst{B}itt-u þik, Fręyja, \hld\ \edtrans{\alst{b}rúðar lïni!}{bridal linen}{\Bfootnote{The dress of the bride.}} &
Vit skulum \alst{a}ka tvau \hld\ ï \alst{jǫ}tun-hęima.“\eva

\bvb Go they the fair Frow to find, \\
and he this word first of all did say: \\
“Bind thyself, Frow, with bridal linen! \\
We two shall drive into the Ettin-homes.”\evb\evg


\bvg\bva\mssnote{\Regius~17v/1}%
\emph{V}ręið varð þȧ \alst{F}ręyja \hld\ ok \alst{f}nasaði, &
\alst{a}llr \alst{ȧ}sa salr \hld\ \alst{u}ndir bifðisk, &
stǫkk þat it \alst{m}ikla \hld\ \edtrans{\alst{m}ęn brísinga}{Necklace of Blazes}{\Bfootnote{The \emph{Brísinga-męn} (second element \emph{męn} ‘neckring, necklace, jewel’) is a jewel worn by Frow.  It is probably a symbol of fire and light (especially if Frow is identified with the Dawn, for which cf. \textlink{Voluspa}[24]) as shown by the first element \emph{brísingr}, which is listed as a poetic synonym for fire in Þul \emph{Elds} 4 (\Skp\ 3) and appears with the sense “blaze” in Norwegian dialects.
This is further supported by the obscure myth wherein the two fire-figures Lock (identified in folklore with the Ash-Lad) and Homedal (\emph{Hęim·dallr} ‘World-Brightener’, the Watchman of the Gods (\textlink{Grimnismal}[13]), the White Os, Goldentooth and Haldenshid (\textlink{EddicFragments}[F3][3]:P1)) fight over it (\Skaldskaparmal\ 15, 23; ÚlfrU \emph{Húsdr} 2 (\Skp\ 3));
cf. \Haustlong\ 9 where Lock is called \emph{girði-þjófr brísings} ‘girdle-thief of the blaze’, i.e. ‘thief of the Blaze-girdle [= the Necklace of Blazes]’.

Archeologically the Blaze-Necklace seems to appear in various finds depicting ladies wearing oversized brooches \parencite[79--84]{Arrhenius1962}.  It has further been identified with several large “disc-on-bow”-type brooches from Wiking Age Scandinavia, especially Sweden, which are some of the finest pieces of jewelry surviving from the period \parencite[87]{Arrhenius1962}.  These brooches, which with their gilt bronze and \emph{cloisonée}-fitted red garnets would given off a real fire-like lustre, are doubtless cultic objects.  They are far too big and heavy to have been worn daily by any woman—the one from Kårsta, Uppland, Sweden is as long as 28.5 cm—and are further found exclusively in hoards and as single finds \parencite[84--85, 93--94, 97]{Arrhenius1962}.  I include a photograph of a particularly beautiful such brooch from the village Othemars in Othem parish, Gotland (item number 453312 HST), Figure \ref{fig:othemars}.}}: &
„Mik \alst{v}ęitst \edtrans{\alst{v}erða \hld\ \alst{v}er-gjarnasta}{become most eager of men}{\Bfootnote{Presumably Frow is speaking out of self-awareness of her lustful inclinations, i.e., she will be gripped by uncontrollable lust if surrounded by strange men.  It is also possible that she worries about being accused of promiscuity by the other gods but that is not the literal sense of the words.  For Frow’s alleged looseness cf. \textlink{Lokasenna}[30]; in \textlink{Lokasenna}[26] Frie is likewise called \emph{ver-gjǫrn} ‘eager of men’.}} &
ef ek \alst{ę}k með þér \hld\ ï \alst{jǫ}tun-hęima.“\eva

\bvb Wroth became Frow then, and snorted; \\
all the hall of the Eese under her shook; \\
down crashed the great \inx[P]{Necklace of Blazes}— \\
“Thou knowest that I will become most eager for men \\
if I drive with thee into the Ettin-homes!”\evb\evg

\begin{figure}[b]
\centering
\includegraphics[width=\textwidth]{Othemars}
\caption{The disc-on-bow brooch from Othemars, Gotland.  Wiking Age, ca. 900 CE.  © Ola Myrin, Historiska museet/SHM, \href{https://creativecommons.org/licenses/by/4.0/deed.en}{CC BY 4.0}.  \url{https://samlingar.shm.se/object/AECAAAD2-4C29-4164-BB36-E6F589FDE1A5}}
\label{fig:othemars}
\end{figure}

\bvg\bva\mssnote{\Regius~17v/3}%
\edtext{Sęnn vǫ́ru \alst{ę̇}sir \hld\ \alst{a}llir ȧ þingi &
ok \alst{ǫ̇}synjur \hld\ \alst{a}llar ȧ máli, &
ok umb þat \alst{r}éðu \hld\ \alst{r}íkir tívar:}{\lemma{Sęnn \dots\ tívar ‘Soon \dots\ Tews’}\Bfootnote{The very same three lines also occur \textlink{Baldrsdraumar}[1]/1–3; see Note there.}} &
\alst{h}vé þęir \alst{H}lór·riða \hld\ \alst{h}amar of sǿtti?\eva

\bvb Soon were the \inx[G]{Eese} all at the \inx[C]{Thing}, \\
and the \inx[G]{Ossens} all at speech, \\
and of this counseled the mighty \inx[G]{Tews}, \\
how they Loride’s hammer might get.\evb\evg


\bvg\bva\mssnote{\Regius~17v/5}%
Þȧ kvað þat \alst{H}ęim·dallr, \hld\ \alst{h}vítastr ȧsa, &
\edtrans{\alst{v}issi \alst{v}ęl framm}{he foreknew well}{\Bfootnote{i.e. saw the future.  Compare the derived adjective \emph{fram-víss} ’forth-wise, prescient.’}} \hld\ sęm \alst{v}anir aðrir: &
„\alst{B}indu vér Þȯr þȧ \hld\ \alst{b}rúðar lïni; &
hafi hann it \alst{m}ikla \hld\ \alst{m}ęn brísinga!\eva

\bvb Then quoth this \inx[P]{Homedal}, whitest of the Eese; \\
he foreknew well like the other \inx[G]{Wanes}: \\
“Let us bind Thunder, then, with bridal linen; \\
let him have the great Necklace of Blazes!\evb\evg


\bvg\bva\mssnote{\Regius~17v/6}%
\Ballnote{A unique description of Wiking Age bridal dress.  Cf. the description’s of dress in \textlink{Rigsthula}, which is, however, a much younger poem than \textlink{Thrymskvida}.  Being the mistress of the household, keys were the mark of a respectable married woman.  The “broad stones” on the breast may be tortoise brooches (also mentioned in \textlink{Volundarkvida} 25, 36.) or beads in a large necklace.  The “tipping” of the head refers to some sort of bridal hat which would have included a veil (cf. st. 27 below).}%
Lǫ́tum und \alst{h}ǫ́num \hld\ \alst{h}rynja lukla &
ok \alst{k}ven-váðir \hld\ umb \alst{k}né falla &
en ȧ \alst{b}rjósti \hld\ \alst{b}ręiða stęina &
ok \alst{h}ag-liga \hld\ umb \alst{h}ǫfuð typpum!“\eva

\bvb Let us by his side hang jingling keys, \\
and women’s garments to fall about his knees, \\
but on the breast broad stones, \\
and skillfully let us tip his head.”\evb\evg


\bvg\bva\mssnote{\Regius~17v/8}%
Þȧ kvað þat \alst{Þ}ȯrr, \hld\ \alst{þ}rúðugr ǫ̇ss: &
„Mik munu \alst{ę̇}sir \hld\ \alst{a}rgan kalla &
ef ek \alst{b}indask lę́t \hld\ \alst{b}rúðar lïni!“\eva

\bvb Then quoth this Thunder, the mighty Os: \\
“Me will the Eese call \inx[C]{queer} \\
if I let me be bound with bridal linen!”\evb\evg


\bvg\bva\mssnote{\Regius~17v/9}%
Þȧ kvað þat \alst{L}oki \hld\ \alst{L}auf·ęyjar sonr: &
„\edtrans{\alst{Þ}ęgi þú, \alst{Þ}ȯrr, \hld\ \alst{þ}ęira orða!}{Shut up thou, Thunder, with those words!}{\Bfootnote{Formulaic line; cf. \textlink{GuthrunOne}[24]/2: \emph{Þęgi þú, þjóð-lęið, \hld\ þęira orða}.}} &
\edtext{Þegar munu \alst{jǫ}tnar \hld\ \alst{Ǫ̇}s-garð búa &
nema \alst{þ}ú \alst{þ}ïnn hamar \hld\ \alst{þ}ér of hęimtir.}{\lemma{Þegar \dots\ hęimtir. ‘Shortly \dots\ dost fetch!’}\Bfootnote{Guarding the lands of the Gods and men from transgressive and destructive forces is Thunder’s task, and the Hammer his most important tool.  Cf. \textlink{Voluspa}[24]–25, \textlink{Harbardsljod}[23]–24, \textlink{Hymiskvida}[22], and a couplet by the obscure poet Thurbern Dise-scold, cited in \Skaldskaparmal\ 11 (Þdís \emph{Þórr} in \Skp\ 3): \emph{Þȯrr hęfr Yggs með ǫ́rum \hld\ Ǫ̇s-garð af þrek varðan.} ‘Thunder has with the messengers of Ug \ken{gods} mightily guarded Osyard.’}}“\eva

\bvb Then quoth this Lock, Leafie’s son: \\
“Shut up thou, Thunder, with those words! \\
Shortly the Ettins will settle Osyard, \\
unless thou thy hammer for thyself dost fetch!”\evb\evg


\bvg\bva\mssnote{\Regius~17v/11}%
\alst{B}undu þęir Þȯr þȧ \hld\ \alst{b}rúðar lïni &
ok inu \alst{m}ikla \hld\ \alst{m}ęni brísinga, &
létu und \alst{h}ǫ́num \hld\ \alst{h}rynja lukla &
ok \alst{k}ven-váðir \hld\ umb \alst{k}né falla &
en ȧ \alst{b}rjósti \hld\ \alst{b}ręiða stęina &
ok \alst{h}ag-liga \hld\ of \alst{h}ǫfuð typpðu.\eva

\bvb They bound Thunder then with bridal linen, \\
and with the great Necklace of Blazes. \\
They by his side set keys to jingle, \\
and women’s garments to fall about the knees, \\
but on the breast broad stones, \\
and skillfully they tipped his head.\evb\evg


\bvg\bva\mssnote{\Regius~17v/13}%
Þȧ kvað þat \alst{L}oki \hld\ \alst{L}auf·ęyjar sonr: &
„Mun’k \alst{au}k með þér \hld\ \alst{a}mbǫ́tt vesa, &
\edtext{vit skulum \alst{a}ka tvau}{\lemma{vit \dots\ tvau ‘we two’}\Bfootnote{It is a fundamental characteristic of Thunder in the Norse mythology that he is very seldom alone on his adventures, but almost always has a travel companion or sidekick \parencite[123]{Lindow1988}. —
\emph{tvau} ‘two’ is here in the neuter, which is used for mixed-sex groups.  This is either an error due to mindless copying of st. 11, or a backhanded insult against Thunder by Lock.}} \hld\ ï \alst{jǫ}tun-hęima.“\eva

\bvb Then quoth this Lock, Leafie’s son: \\
“I too will with thee be a handmaid; \\
we two shall drive into the Ettin-homes.”\evb\evg


\bvg\bva\mssnote{\Regius~17v/14}%
Sęnn vǫ́ru \edtrans{\alst{h}afrar}{he-goats}{\Bfootnote{Thunder’s chariot was driven by his two goats; cf. the kenning \emph{hafra dróttinn} ‘Lord of He-goats’ (\textlink{Hymiskvida}[20], 31).}} \hld\ \alst{h}ęim of \emph{v}reknir, &
\alst{sk}yndir at \alst{sk}ǫklum, \hld\ \alst{sk}yldu vęl rinna; &
\edtrans{\alst{b}jǫrg \alst{b}rotnuðu, \hld\ \alst{b}rann jǫrð loga}{Crags burst, the earth burned with flame}{\Bfootnote{Thunder’s driving is often heralded by cosmic disturbance.  So, his arrival in \textlink{Lokasenna}[55] is signalled by the mountains quaking.  The description most similar to the present stanza is found in Thedwolf’s \Haustlong\ 14–16, where crags (\emph{bjǫrg}) burst asunder and fires rage before him as he drives to fight \inx[P]{Rungner}.
A possibly Indo-European parallel is the Vedic myth of Índra breaking the mountains and releasing the rivers (as described most famously in \Rigveda\ 1.32).  Cf. also \textlink{Baldrsdraumar}[3] where the ground rumbles beneath the riding Weden.}}; &
\alst{ó}k \alst{Ó}ðins sonr \hld\ ï \alst{jǫ}tun-hęima.\eva

\bvb Soon were the \inx[C]{he-goats} driven home, \\
hastened onto the cart-poles—they were to run well. \\
Crags burst, the earth burned with flame; \\
Weden’s son \ken*{= Thunder} drove to the Ettin-homes.\evb\evg


\bvg\bva\mssnote{\Regius~17v/16}%
Þȧ kvað þat \alst{Þ}rymr, \hld\ \alst{þ}ursa dróttinn: &
„\alst{St}andið upp, jǫtnar, \hld\ ok \alst{st}ráið bękki! &
Nú \alst{f}ǿrið mér \hld\ \alst{F}ręyju at kvǫ̇n, &
\alst{N}jarðar dóttur \hld\ ór \alst{N}óa-tu̇num.\eva

\bvb Then quoth this Thrim, the lord of Thurses: \\
“Stand up, ye ettins, and strew the benches! \\
Now bring ye to me Frow for a wife, \\
\inx[P]{Nearth}’s daughter from the \inx[L]{Nowetons}!\evb\evg


\bvg\bva\mssnote{\Regius~17v/18}%
\alst{G}anga hér at \alst{g}arði \hld\ \edtext{\alst{g}oll-hyrnðar kýr, &
\alst{ø}xn \alst{a}l-svartir}{\lemma{goll-hyrnðar kýr, øxn al-svartir ‘golden-horned kine, all-black oxen’}\Bfootnote{Two releated formulae.  Together they emphasize the great value and sacrifical purpose of the oxen; both have Indo-European parallels and appear together in Latvian \emph{Daina} 33863 (\emph{Melni vērši, zelta ragi, / nāk par jūru baurodami} ‘Black oxen, golden horns, come bellowing over the sea.’) \parencite[37--40]{Calin1996}.

\emph{goll-hyrnðar kýr} is also found in \textlink{HelgakvidaHjorvardssonar}[4]/2a, and the term \emph{gollin-horni} ‘golden-horned one’ appears as a poetic synonym for “ox” in the Þul \emph{Øxna} 3. —
The singular of \emph{øxn al-svartir} appears in \textlink{Hymiskvida}[18] (\emph{oxi al-svartr} ‘an all-black ox’).  Compare \textcite{Saxo} 1.8.12, where the hero Hadding has to atone for his slaying of a heavenly being by a sacrifice of dark-coloured victims (\emph{furvae hostiae}): \emph{Siquidem propiciandorum numinum gratia Frø deo rem diuinam furuis hostiis fecit. Quem litationis morem annuo feriarum circuitu repetitum posteris imitandum reliquit. Frøblod Sueones uocant.} ‘In order to mollify the divinities he [= Hadding] did indeed make a holy sacrifice of dark-coloured victims to the god Frø.  He repeated this mode of propitiation at an annual festival and left it to be imitated by his descendants.  The Swedes call it Frøblot.’  This ancient ritual taboo finds parallel even in the Tanakh, where animals dedicated to YHWH were to be \texthebrew{תָּמִ֖ים} (‘unblemished’, Leviticus 1:3).

TODO: write about the IE parallels}}, \hld\ \alst{jǫ}tni at gamni, &
fjǫlð á’k \alst{m}ęiðma, \hld\ fjǫlð á’k \alst{m}ęnja; &
\alst{ęi}nnar mér Fręyju \hld\ \alst{ȧ}·vant þykkir.“\eva

\bvb Here march to the courtyard golden-horned kine, \\
all-black oxen to the ettin’s [my] pleasure. \\
A multitude I own of treasures, a multitude I own of torcs; \\
only Frow I think me missing.”\evb\evg


\bvg\bva\mssnote{\Regius~17v/20}%
Vas þar at \alst{k}veldi \hld\ of \alst{k}omit snimma &
\alst{o}k fyr \alst{jǫ}tna \hld\ \alst{ǫ}l framm borit; &
\edtext{\alst{ęi}nn át \alst{o}xa, \hld\ \alst{á}tta laxa, &
\alst{k}rásir allar, \hld\ þę́r’s \alst{k}onur skyldu, &
drakk \edtrans{\alst{S}ifjar verr}{Sib’s husband}{\Bfootnote{It is curious that the same kenning is used in \textlink{Hymiskvida}[15] which also describes Thunder’s eating; it is perhaps a borrowing from the present stanza.}} \hld\ \alst{s}ǫ́ld þrjú mjaðar.}{\lemma{Ęinn \dots\ mjaðar. ‘He alone \dots\ of mead.’}\Bfootnote{Thunder is renowned for his great appetite; cf. \textlink{Hymiskvida}[15] where he eats two of Hymer’s oxen and \Gylfaginning\ 46–47 where he drinks a large part of the sea.}}\eva

\bvb There was the evening come early, \\
and for the ettins ale brought forth. \\
He \ken*{= Thunder} alone ate an ox, eight salmons, \\
all the dainties meant for the women; \\
Sib’s husband \ken*{= Thunder} drank three sieves of mead.\evb\evg


\bvg\bva\mssnote{\Regius~17v/23}%
Þȧ kvað þat \alst{Þ}rymr, \hld\ \alst{þ}ursa dróttinn: &
„Hvar sátt-u \alst{b}rúðir \hld\ \alst{b}íta hvassara? &
Sá’k-a \alst{b}rúðir \hld\ \alst{b}íta ęnn \alst{b}ręiðara &
né ęnn \alst{m}ęira \alst{m}jǫð \hld\ \alst{m}ęy of drekka!“\eva

\bvb Then quoth this Thrim, the lord of Thurses: \\
“Where sawest thou brides bite sharper? \\
I never saw brides bite yet broader; \\
nor yet more mead a maiden drink!”\evb\evg


\bvg\bva\mssnote{\Regius~17v/25}%
Sat in \alst{a}l-snotra \hld\ \alst{a}mbǫ́tt fyrir &
es \alst{o}rð of fann \hld\ við \alst{jǫ}tuns máli: &
„\alst{Á}t \alst{v}ę́tr Fręyja \hld\ \alst{á}tta nǫ́ttum, &
svá vas hǫ̇n \alst{ó}ð-fu̇s \hld\ ï \alst{jǫ}tun-hęima.“\eva

\bvb Sat the all-clever handmaid \ken*{= Lock} in front, \\
who found a word against the ettin’s speech: \\
“Frow ate naught for eight nights; \\
so madly she longed for the Ettin-homes.”\evb\evg


\bvg\bva\mssnote{\Regius~17v/27}%
\alst{L}aut und \edtrans{\alst{l}ïnu}{linen}{\Bfootnote{The bridal veil.}}, \hld\ \alst{l}ysti at kyssa, &
en hann \alst{ú}tan stǫkk \hld\ \alst{ę}nd-langan sal: &
„Hví eru \alst{ǫ}ndótt \hld\ \alst{au}gu Fręyju? &
\edtrans{Þykki mér \alst{ó}r \hld\ \alst{au}gum brenna!}{Methinks it burning from the eyes!}{\Bfootnote{The meter of this line is very poor: the first half-line is only three syllables long, and the alliteration falls on \emph{ór} ‘from’, which has no reason to be stressed.  It would be much improved by inserting \emph{ęldar} ‘fires’ between \emph{augum} ‘eyes’ and \emph{brenna} ‘burns’, and this expression is actually attested in \Gylfaginning\ 51: \emph{Eldar brenna ór augum hans ok nǫsum} ‘Fires burn from his eyes and nostrils’.}}“\eva

\bvb He \ken*{= Thrim} looked ’neath the linen, lusted to kiss— \\
but flung back out across the length of the hall— \\
“Why are the eyes of Frow blazing? \\
Methinks it burning from the eyes!”\evb\evg


\bvg\bva\mssnote{\Regius~17v/29}%
Sat in \alst{a}l-snotra \hld\ \alst{a}mbǫ́tt \edtext{fyrir}{\Afootnote{add. \emph{†ſ.†} \Regius.}} &
es \alst{o}rð of fann \hld\ við \alst{jǫ}tuns máli: &
„Svaf \alst{v}ę́tr Fręyja \hld\ \alst{á}tta nǫ́ttum, &
svá vas hǫ̇n \alst{ó}ð-fu̇s \hld\ ï \alst{jǫ}tun-hęima.“\eva

\bvb Sat the all-clever handmaid in front, \\
who found a word against the ettin’s speech: \\
“Frow slept naught for eight nights; \\
so madly she longed for the Ettin-homes.”\evb\evg


\bvg\bva\mssnote{\Regius~17v/30}%
\Ballnote{The sister, who was apparently the one who asked for the Hammer, now has the audacity to ask Thunder (disguised as Frow) to give her the very rings on his hands.}%
\alst{I}nn kom in \alst{a}rma \hld\ \alst{jǫ}tna systir, &
hin’s \edtrans{\alst{b}rúð-féar}{the bride-fee}{\Bfootnote{Thunder’s hammer.}} \hld\ \alst{b}iðja þorði: &
„Lát þér af \alst{h}ǫndum \hld\ \alst{h}ringa rauða &
ef þú \alst{ǫ}ðlask vill \hld\ \alst{ȧ}stir mïnar, &
\edtrans{\alst{ȧ}stir mïnar, \hld\ \alst{a}lla hylli}{my love; all [my] holdness’}{\Bfootnote{Possibly formulaic.  There are no preserved parallels in poetry, but there may be one in \Gylfaginning\ 49 (excerpt, following the death of Balder):
\emph{En er goðin vitkuðust, þá mę́lti Frigg ok spurði, hverr sá vę́ri með ǫ́sum, er \textbf{eignast vildi „allar ástir mínar}} (so \Trajectinus\Wormianus; \emph{ástir hennar} ‘her loves’ \RegiusProse\Upsaliensis) \emph{\textbf{ok hylli}, ok vili hann ríða á hel-veg ok freista, ef hann fái fundit Baldr, ok bjóða Helju út·lausn, ef hón vill láta fara Baldr heim í Ás-garð.“} ‘But when the gods came back to their wits, then Frie spoke and asked which one among the Eese \textbf{would have “all my affections and holdness} and would ride on the \inx[L]{Hellway} and try whether he may find Balder and offer Hell a ransom if she will let Balder come home to Osyard.”’ — We can tell from the citation of a \Ljodahattr\ stanza at the end of ch. 49 (\textlink{EddicFragments}[F5][5]:1 below) that Snorre had access to now-lost Eddic poetry concerning Balder’s death, and it may thus be the case that one of these poems contained the same two long-lines as the present ll. 4–5.  For such a sharing of several lines cf. st. 14/1–3 above, which are identical to \textlink{Baldrsdraumar}[1]/1–3.}}!“\eva

\bvb In came the wretched sister of the ettins, \\
she who for the bride-fee had dared ask: \\
“Slide off from thy hands the red rings, \\
if thou wilt win my affections, \\
my affections, all [my] \inx[C]{holdness}.”\evb\evg


\bvg\bva\mssnote{\Regius~17v/32}%
Þȧ kvað þat \alst{Þ}rymr, \hld\ \alst{þ}ursa dróttinn: &
„\alst{B}erið inn hamar \hld\ \alst{b}rúði at vígja, &
lęggið \alst{M}jǫllni \hld\ ï \alst{m}ęyjar kné, &
\alst{v}ígið okkr saman \hld\ \edtrans{\alst{V}árar}{Ware}{\Bfootnote{According to \Gylfaginning\ 35 \emph{Vǫ́r} is the goddess who governs vows between men and women.  Here she is apparently invoked as a witness in the wedding ceremony, perhaps a reflex of authentic pagan wedding customs of the Wiking Age.}} hęndi!“\eva

\bvb Then quoth this Thrim, the lord of Thurses: \\
“Bear ye in the hammer the bride for to bless; \\
lay ye Millner in the maiden’s knee; \\
bless us two together by the hand of \inx[P]{Ware}!”\evb\evg


\bvg\bva\mssnote{\Regius~17v/34}%
\edtrans{\alst{H}ló \alst{H}lór·riða \hld\ \alst{h}ugr ï brjósti}{Laughed Loride’s heart in his chest}{\Bfootnote{Cf. \textlink{GudrunThree}[9]/1: \emph{Hló þȧ Atla \hld\ hugr ï brjósti} ‘Then laughed Attle’s heart in his chest’.}} &
es \alst{h}arð-\alst{h}ugaðr \hld\ \alst{h}amar of þękkði; &
\alst{Þ}rym drap hann fyrstan, \hld\ \alst{þ}ursa dróttin, &
ok \alst{ę́}tt \alst{jǫ}tuns \hld\ \alst{a}lla lamði.\eva

\bvb Laughed Loride’s heart in his chest, \\
when, hard-hearted, he recognised the hammer. \\
Thrim he smote first, the lord of Thurses, \\
and all the ettin’s lineage he beat lame.\evb\evg


\bvg\bva\mssnote{\Regius~18r/1}%
Drap hann ina \alst{ǫ}ldnu \hld\ \alst{jǫ}tna systur, &
hin’s \alst{b}rúð-féar \hld\ of \alst{b}eðit hafði; &
hǫ̇n \alst{sk}ell of hlaut \hld\ fyr \alst{sk}illinga, &
en \alst{h}ǫgg \alst{h}amars \hld\ fyr \alst{h}ringa fjǫlð. &
Svá kom \alst{Ó}ðins sonr \hld\ \alst{ę}ndr at hamri.\eva

\bvb He smote the aged sister of the ettins, \\
she who for the bride-fee had asked. \\
She got a smiting for shillings, \\
and a blow of the hammer for a multitude of rings. \\
So came Weden’s son back to his hammer.\evb\evg

\sectionline
