\bookStart{Leeds of Hoarbeard}[Hár·barðs ljóð]
\setBookCode{Harbardsljod}

\begin{flushright}%
\textbf{Dating} \parencite{Sapp2022}: early C11th (0.578)–late C11th (0.377)

\textbf{Meter:} Unclear (TODO)%
\end{flushright}

\section{Introduction}

The \textbf{Leeds of Hoarbeard} (\textlink{Harbardsljod}) is preserved in full in \Regius, and in part in \AM.  The poem might be seen as an allegory on class relations, namely between the self-owning yeomen farmers and the warlike earls, represented through their patron gods.

Of all Eddic poems \textlink{Harbardsljod} is probably the strangest in terms of form. Verse length varies greatly, and many of the lines (see especially the final verse) are of an obscene length reminiscent of late continental Germanic poems like the Heliand; some simply have no metrical qualities at all. The young clitic definite is (uniquely) employed frequently throughout the poem. These criteria would seem to point towards a late origin for the poem (though not later than the late C13th, when \Regius\ was written).

Against a late origin speaks the presence of rare words (e.g. \emph{ǫgurr} v. 13) and a thorough understanding of the personalities of the two gods which would seem unlikely to stem from several centuries after the conversion of Iceland. The model devised by Sapp gives the poem a 57.8\% likelihood of being from the early C11th, and a 37.7\% likelihood of being from the late 11th. These scores are most similar to those obtained by \textlink{Gripisspa}, a poem that on the surface seems much more archaic.

What could we then be dealing with? It may of course be that the poem is heavily corrupt, but there is no good evidence for this (apart from the above-mentioned irregularities). Most lines are readily understandable and fit well both within their respective context and the poem as a whole. I think a better solution to this problem is to assume that the poem has been acted out as a sort of carnivalesque theatre, with two masked actors, each playing one of the gods. This would explain the variations in meter and line length, and the prose; some lines were simply shouted out, and the lack of alliteration in them would then have a kind of discordant effect.

This is shown also by uses of the word ‘here’ in sts. 9 and 14. TODO: mention concept of "double scene" by Lars Lönnroth?

\section{The Leeds of Hoarbeard}

\bpg\bpa\mssnote{\Regius~12r/30}%
Þórr fór ór austr-vegi ok kom at sundi einu. Ǫðrum megum sunds’ins var ferju-karl’inn með skip’it. Þórr kallaði:\epa

\bpb {\huge T}\textsc{hunder journeyed} from the Eastern Way and came to a sound. At the other side of the sound was the ferryman with the ship. Thunder called out:\epb\epg


\bvg\bva\mssnote{\Regius~12r/32}%
„Hvęrr ’s sá \alst{s}vęinn \alst{s}vęina \hld\ es stęndr fyr \alst{s}und’it handan?“\eva

\bvb%
“Who is that swain of swains, standing here across the sound?”\evb\evg


\bvg\bva\mssnote{\Regius~12v/1}%
\speakernote{Hann svaraði:}%
„Hvęrr ’s sá \alst{k}arl \alst{k}arla \hld\ es \alst{k}allar of vág’inn?“\eva

\bvb\speakernoteb{He answered:}%
“Who is that churl of churls, calling out o’er the wave?”\evb\evg


\bvg\bva\mssnote{\Regius~12v/2}%
\speakernote{[Þórr kvað:]}%
„\alst{F}ęr þú mik of sund’it, \hld\ \alst{f}ǿði’k þik á morgun; &
\alst{m}ęis hęfi’k á baki, \hld\ verðr⸗a \alst{m}atr inn bętri. &
Át’k í \alst{h}víld \hld\ áðr ek \alst{h}ęiman fór, &
\alst{s}íldr ok \edtrans{hafra}{oatmeal/he-goats}{\Bfootnote{(1) The easiest reading is an acc. pl. of \emph{hafr} ‘he-goat’; Thunder also eats his goats in \Gylfaginning\ 44, where he butchers and cooks them in the evening and brings them back to life at dawn by blessing them with his hammer (cf. \textlink{Hymiskvida}[37]–38).  \textcite{FinnurEdda} and \textcite{PettitEdda} prefer this.
(2) Other scholars instead read an acc. pl. of \emph{hafri} ‘oat’, i.e. ‘porridge, oatmeal’.  Stiles (forthcoming TODO) connects this with the porridge-eating of the Vedic god Pūṣán (\Rigveda\ 6.56.1, 57.2), who is “partner and yokemate” (\Rigveda\ 6.56.2) of Índra, Thunder’s vedic equivalent.  Another similarity Stiles notes between Thunder and Pūṣán is that both have chariots driven by goats (e.g. 6.57.3: “Goats are the draft-animals for the one”, 58.2: “Having goats as his horses”).  Whether the Vedic tradition has split the Thunder-god in two or whether the Germanic Thunder has absorbed elements of an earlier yokemate is hard to say.  Like Índra, Thunder is frequently joined by companions, as is in fact one of his most fundamental characteristics \parencite[123]{Lindow1988}.}}; \hld\ \alst{s}aðr em’k ęnn þęss.“\eva

\bvb\speakernoteb{[Thunder quoth:]}%
“Ferry me over the sound, I’ll feed thee in the morning! \\
A basket have I on my back; better food will not be found. \\
I ate for a while before I journeyed from home \\
herring and oatmeal/he-goats—I am still full from that.”\evb\evg


\bvg\bva\mssnote{\Regius~12v/5}%
\speakernote{[Hár·barðr kvað:]}%
„Ár⸗ligum \alst{v}erkum \hld\ hrósar þú, \alst{v}ęrði’num; &
\ind \alst{v}ęitst⸗at-tu fyr gǫr⸗la, &
\alst{d}ǫpr ’ro þín hęim-kynni, \hld\ \alst{d}auð hygg’k at þín móðir sé.“\eva

\bvb%
“Of early works thou boastest, of thy eating!\footnoteB{TODO. This is pretty difficult. From the previous stanza \emph{vęrði’num} seems to be referring to eating.} \\
\ind Thou seest not clearly ahead: \\
dire is the state of thy home—I think that thy mother is dead!”\evb\evg


\bvg\bva\mssnote{\Regius~12v/6}%
\speakernote{[Þórr kvað:]}%
„\alst{Þ}at sęgir þú nú \hld\ es hvęrjum \alst{þ}ikkir &
\alst{m}ęst at vita— \hld\ at mín \alst{m}óðir dauð sé.“\eva

\bvb%
“Now thou sayest what to every man seems \\
of most weight to know—that my mother is dead!”\evb\evg


\bvg\bva\mssnote{\Regius~12v/8}%
\speakernote{[Hár·barðr kvað:]}%
„\alst{Þ}ęy⸗gi ’s sem \alst{þ}ú \hld\ \alst{þ}rjú bú góð ęigir; &
\alst{b}ęr-\alst{b}ęinn þú stęndr \hld\ ok hęfir \alst{b}rautinga gørvi, &
\ind þat-ki at þú hafir \alst{b}rę́kr þïnar.“\eva

\bvb%
“Yet it’s not as if \emph{thou} own three good farms— \\
bare-legged thou standests and hast the gear of a tramp; \\
\ind it’s hardly as if thou own thy breeches!”\evb\evg


\bvg\bva\mssnote{\Regius~12v/9}%
\speakernote{[Þórr kvað:]}%
„\alst{St}ýr-ðu hingat ęikju’nni, \hld\ ek mun þér \alst{st}ǫðna kęnna &
eða \alst{h}vęrr á skip’it \hld\ es þú \alst{h}ęldr við land’it?“\eva

\bvb%
“Steer hither the boat! I will show thee to the harbour— \\
or who owns the ship which thou holdest by the shore?”\evb\evg


\bvg\bva\mssnote{\Regius~12v/11}%
\speakernote{[Hár·barðr kvað:]}%
„\alst{H}ild·ólfr sá \alst{h}ęitir \hld\ es mik \alst{h}alda bað, &
\alst{r}ekkr inn \alst{r}áð-svinni \hld\ es býr í \alst{R}áðs-ęyjar-sundi; &
bað⸗at hann \alst{h}lęnni-męnn flytja \hld\ eða \alst{h}rossa-þjófa, &
\alst{g}óða ęina \hld\ ok þá’s ek \alst{g}ǫrva kunna; &
\alst{s}ęg-ðu til nafns þíns \hld\ ef þú vill of \alst{s}und’it fara.“\eva

\bvb%
“Hildolf is he called who bade me hold [it], \\
the counsel-wise man who lives in Redeseysound. \\
He bade me not ferry highwaymen nor horse-thieves; \\
good ones only, and those I know well— \\
speak to thy name if thou wilt pass over the sound!”\evb\evg


\bvg\bva\mssnote{\Regius~12v/15}%
\speakernote{[Þórr kvað:]}%
„\alst{S}ęgja mun’k til nafns míns \hld\ þó’tt ek \alst{s}ękr sjá’k &
ok til \alst{a}lls \alst{ø}ðlis: \hld\ Ek em \alst{Ó}ðins sonr, &
\alst{M}ęila bróðir \hld\ ęn \alst{M}agna faðir, &
\alst{þ}rúð-valdr goða \hld\ við \alst{Þ}ȯr knátt-u hér dǿma! &
\alst{H}ins vil’k nú spyrja, \hld\ \alst{h}vat þú hęitir.“\eva

\bvb%
“I will speak to my name—although I might be charged— \\
and to all my origin: I am Weden’s son, \\
Male’s brother and Main’s father, \\
the strength-wielder of the Gods—with Thunder dost thou here speak! \\
One thing will I now ask: what art thou called?”\evb\evg


\bvg\bva\mssnote{\Regius~12v/18}%
\speakernote{[Hár·barðr kvað:]}%
„\alst{H}ár·barðr ek \alst{h}ęiti, \hld\ \alst{h}yl’k umb nafn sjaldan.“\eva

\bvb%
“Hoarbeard I am called; I seldom conceal my name.”\evb\evg


\bvg\bva\mssnote{\Regius~12v/18}%
\speakernote{[Þórr kvað:]}%
„Hvat skalt-u umb \alst{n}afn hylja \hld\ \alst{n}ema þú sakar ęigir?“\eva

\bvb%
“Why shalt thou conceal thy name—unless thou have charges?”\evb\evg


\bvg\bva\mssnote{\Regius~12v/19}%
\speakernote{[Hár·barðr kvað:]}%
„En þó’tt ek \alst{s}akar ęiga, \hld\ fyr \alst{s}líkum sem þú est &
þá mun’k \alst{f}orða \alst{f}jǫrvi mínu \hld\ nema ek \alst{f}ęigr sé.“\eva

\bvb%
“Yet even though I had charges, for such a one as thou art \\
I would then protect my life—unless I should be \inx[C]{fey}.”\evb\evg


\bvg\bva\mssnote{\Regius~12v/21}%
\speakernote{[Þórr kvað:]}%
\ind „Harm ljótan mér \alst{þ}ikkir ï \alst{þ}ví &
\ind at \alst{v}aða of \alst{v}áginn til þïn
\ind ok \alst{v}ę́ta \edtrans{ǫgur}{burden}{\Bfootnote{The sense of this word is not clear, though it is probably the same as the first element of the compound \emph{ǫgur-stund} ‘burdensome hour’, found in \textlink{Volundarkvida}[42].  My favoured interpretation is that it refers to the food Thunder carries on his back (st. 3), which would explain.  Some authors have read it as a euphemism for “bollocks”, which admittedly would not stand out much in \Harbardsljod, although one would not expect it in the mouth of Thunder who is very much the “straight man” throughout the poem.}} mïnn; &
\edtext{skylda’k launa \alst{k}ǫgur-svęini \hld\ þïnum \alst{k}angin-yrði}{\Bfootnote{Cf. \Gylfaginning\ 45: \emph{Ekki munu hirð-menn Útgarða-Loka vel þola því·líkum kǫgur-sveinum kǫpur-yrði.} ‘TODO.’}} &
\ind ef ek \alst{k}om⸗umk yfir sund’it.“\eva

\bvb%
“An ugly harm it seems to me \\
to wade o’er the wave to thee, and wet my burden. \\
I would repay thee, swaddle-swain, for thy mocking words, \\
\ind if I might get myself over the sound.”\evb\evg


\bvg\bva\mssnote{\Regius~12v/23}%
\Ballnote{Rungner was a famous ettin slain by Thunder in a fierce battle (\Skaldskaparmal\ TODO, \Haustlong\ TODO).  Hoarbeard’s mention of it sets off a long argument over their respective accomplishments.}
\speakernote{[Hár·barðr kvað:]}%
„\alst{H}ér mun’k standa \hld\ ok þïn \alst{h}eðan bíða; &
fannt⸗a-tu mann inn \alst{h}arðara \hld\ at \alst{H}rungni dauðan.“\eva

\bvb%
“\emph{Here} will I stand, and from \emph{here} await thee. \\
Thou hast not found a harder man since \inx[P]{Rungner} died!”\evb\evg


\bvg\bva\mssnote{\Regius~12v/25}%
\speakernote{[Þórr kvað:]}%
„\alst{H}ins vilt-u nú geta \hld\ es vit \alst{H}rungnir dęildum, &
sá inn \alst{st}ór-úðgi jǫtunn, \hld\ \edtrans{es ór \alst{st}ęini vas hǫfuð’it ȧ}{on whom the head was of stone}{\Bfootnote{Cf. \textlink{Hymiskvida} 29–30, where the ettin Hymer’s head is harder than stone.  This characteristic part of ettin-physiology can probably be explained by reference to Germanic cosmology.  In numerous Indo-European cosmologies the Firmament is believed to be made of stone, as is seen in the PIE root \emph{*h₂éḱmō} whose descendants can mean both ‘heaven, sky’ and ‘stone’, sometimes varying even within languages \parencites[342]{West2007}[3]{Calin1996}; cf. e.g. Sanskrit \emph{áşman} ‘stone, rock’ with Old Persian \emph{asman} ‘sky, heaven’ and ON \emph{himinn} ‘sky, heaven’ with the derivative \emph{hamarr} ‘cliff, rock’ \parencite[220,206--207]{Kroonen2013}.  In the Germanic cosmology this “Stoney Heaven” was originally the skull of \inx[P]{Yimer}, the primordial ettin sacrificed by the Gods (\textlink{Grimnismal}[41], \textlink{Vafthrudnismal} 21), and as the ancestor of the Ettins he thus passed his stone-skull on to his descendants.}}, &
þó lét’k hann \alst{f}alla \hld\ ok \alst{f}yrir hníga; &
\ind hvat vannt-u þá meðan, Hár·barðr?“\eva

\bvb%
“Of this wilt thou now speak, when I and Rungner dealt with each other, \\
that great-minded ettin on whom the head was of stone. \\
Yet I made him fall and kneel down before me— \\
\ind what didst thou accomplish meanwhile, Hoarbeard?”\evb\evg


\bvg\bva\mssnote{\Regius~12v/27}%
\speakernote{[Hár·barðr kvað:]}%
„Vas’k með \alst{F}jǫl·vari \hld\ \alst{f}imm vetr alla &
ï \alst{ęy} þęiri \hld\ es \alst{A}l-grǿn hęitir; &
\alst{v}ega vér þar knǫ́ttum \hld\ ok \alst{v}al fęlla, &
\edtrans{\alst{m}args at fręista, \hld\ \alst{m}ans at kosta}{many a girl did we tempt and win}{\Bfootnote{I read \emph{margs} ‘many a’ as modifying \emph{mans} ‘girl’.}}.“\eva

\bvb%
“I was with Felwar for five winters all \\
in that island which is called Allgreen. \\
There we did fight and fell the slain, \\
many a girl did we tempt and win.”\evb\evg


\bvg\bva\mssnote{\Regius~12v/30}%
\speakernote{[Þórr kvað:]}%
„Hversu snúnuðu yðr konur yðrar?“\eva

\bvb%
“How did your women pleasure (TODO!!!) you?.”\evb\evg


\bvg\bva\mssnote{\Regius~12v/30}%
\speakernote{[Hár·barðr kvað:]}%
„\alst{Sp}arkar ǫ́ttum vér konur \hld\ ef oss at \alst{sp}ǫkum yrði; &
\alst{h}orskar ǫ́ttum vér konur \hld\ ef oss \alst{h}ollar vę́ri, &
þę́r ór \alst{s}andi \hld\ \alst{s}íma undu &
\ind ok ór \alst{d}ali \alst{d}júpum &
\ind \alst{g}rund of \alst{g}rófu; &
varð’k þęim ęinn \alst{ǫ}llum \hld\ \alst{ø}fri at rǫ́ðum; &
\ind hvílda’k hjá \alst{s}ystrum \alst{s}jau &
\ind ok hafða’k \alst{g}ęð þęira allt ok \alst{g}aman; &
\ind hvat vannt-u þá meðan, Þȯrr?“\eva

\bvb%
“We had smart women if they were to our enjoyment; \\
we had wise women if they were \inx[C]{hold} toward us. \\
Out of the sand they unwound a rope, \\
\ind and out of a deep dale \\
\ind dug up the ground. \\
I alone became superior to them all in counsel; \\
\ind I rested beside those sisters seven \\
\ind and had their senses all, and pleasure— \\
\ind what didst thou accomplish meanwhile, Thunder?”\evb\evg


\bvg\bva\mssnote{\Regius~13r/2, \AM~1r/1 (l. 4b ff.)}%
\Ballnote{For the slaying of Thedse cf. \textlink{Lokasenna}[50] n.  Here we seem to have a rare example of native Germanic star-lore.  TODO: Is the exact constellation identifiable?}%
\speakernote{Þórr kvað:}%
„Ek drap \alst{Þ}jatsa, \hld\ hinn \alst{þ}rúð-móðga jǫtun, &
\alst{u}pp ek varp \alst{au}gum \hld\ \alst{A}ll·valda sonar &
\ind ȧ þann inn \alst{h}ęiða \alst{h}imin; &
þau ’ro \alst{m}ęrki \alst{m}ęst \hld\ \alst{m}inna verka, &
\ind þau’s allir męnn \edtext{\alst{s}íðan}{\Afootnote{om. \AM}} of \alst{s}é\emph{a}; &
\ind hvat vannt-u þá meðan, Hár·barðr?“\eva

\bvb%
“I slew \inx[C]{Thedse}, the strength-minded ettin; \\
I threw up the eyes of Allwald’s son \ken*{= Thedse} \\
\ind onto the clear heaven. \\
Those are the greatest marks of my works, \\
\ind those which all men since do see— \\
\ind what didst thou accomplish meanwhile, Hoarbeard?”\evb\evg


\bvg\bva\mssnote{\Regius~13r/5, \AM~1r/1}%
\speakernote{Hár·barðr kvað:}%
„\alst{M}iklar \alst{m}an-vélar \hld\ hafða’k við \edtrans{\alst{m}yrk-riður}{mirk-rideresses}{\Bfootnote{The \emph{riður} ‘rideresses, fevers’ were witches thought to torment people and cause disease and delirium.  See \textlink{Havamal}[156] for discussion.}} &
\ind þȧ’s ek \alst{v}élta þę́r \edtrans{frȧ \alst{v}erum}{from mankind}{\Bfootnote{Alternatively ‘from their husbands’, but that makes less sense in context.  For \emph{verar} ‘men, husbands’ in the general sense ‘mankind’ cf. st. 53/3 below, \textlink{Lokasenna}[24]/3.}}. &
\alst{H}arðan jǫtun \hld\ hugða’k \alst{H}lé·barð vesa; &
\ind \alst{g}af hann mér \edtrans{\alst{g}amban-tęin}{gombentoe}{\Bfootnote{An unknown magical object also occuring in \textlink{Skirnismal}[32]/2.  Etymologically it is clearly some type of stick, probably for carving runes into.}} &
\ind en ek \alst{v}élta hann ór \alst{v}iti.“\eva

\bvb%
“Great girl-tricks I had against \inx[C]{mirk-rideresses}, \\
\ind when I lured them away from men. \\
A hard ettin I judged Leebeard to be; \\
\ind he gave me a \inx[C]{gombentoe}, \\
\ind but I tricked him out of his wits.”\evb\evg


\bvg\bva\mssnote{\Regius~13r/7, \AM~1r/3}%
\speakernote{Þórr kvað:}%
„Illum huga launaðir þú \edtext{þá}{\Afootnote{om. \AM}} \alst{g}óðar \alst{g}jafar.“\eva

\bvb%
“With an evil heart didst thou then repay the good gifts.”\evb\evg


\bvg\bva\mssnote{\Regius~13r/8, \AM~1r/4}%
\speakernote{Hár·barðr kvað:}%
„\edtrans{Þat hęfir \alst{ęi}k \hld\ es af \alst{a}nnarri skęfr}{An oak has that which it chafes from the other}{\Bfootnote{Proverbial, aso appearing in \Malshattakvadi\ TODO.}}; &
\ind umb \alst{s}ik es hvęrr ï \alst{s}líku— &
\ind hvat vannt-u þȧ meðan, Þȯrr?“\eva

\bvb%
“An oak has that which it chafes from the other; \\
\ind each man is for himself in such— \\
\ind what didst thou accomplish meanwhile, Thunder?”\evb\evg


\bvg\bva\mssnote{\Regius~13r/9, \AM~1r/4}%
\Ballnote{Thunder is the defender of both the Gods and Middenyard (the Home of Men) against the Ettins, for which cf. \textlink{Voluspa}[25], 53, \textlink{Thrymskvida}[18].  For Thunder’s killing of ettin-women in particular cf. sts. 37–39 below, \textlink{EddicFragments}[F6][6], and \textcite{Lindow1988}.}%
\speakernote{Þórr kvað:}%
„\alst{E}k vas \alst{au}str \hld\ ok \alst{jǫ}tna barða’k &
\alst{b}rúðir \alst{b}ǫl-vísar \hld\ es til \alst{b}jargs gingu; &
mikil myndi \alst{ę́}tt \alst{jǫ}tna \hld\ ef \alst{a}llir lifði, &
vę́tr myndi \alst{m}anna \hld\ undir \alst{M}ið-garði— &
\ind hvat vannt-u þá meðan, Hár·barðr?\eva

\bvb%
“I was in the east and bashed Ettins, \\
bale-wise brides who walked to the mountain. \\
Great would the line of Ettins be if they all had lived, \\
naught would remain of Men within Middenyard— \\
\ind what didst thou accomplish meanwhile, Hoarbeard?”\evb\evg


\bvg\bva\mssnote{\Regius~13r/11, \AM~1r/6}%
\Ballnote{Weden expresses characteristic aristocratic disregard for lower life and life as mere life.  Where the moral Thunder boasts of protecting mankind, the ambivalent Weden sarcastically responds that he incited war and conflict in order to take the best of them for himself.}%
\speakernote{Hár·barðr kvað:}%
„\alst{V}as’k ȧ \alst{V}allandi \hld\ ok \alst{v}ígum fylgða’k, &
\alst{a}tta ek \alst{jǫ}frum \hld\ en \alst{a}ldri sę́tta’k; &
\alst{Ó}ðinn á \alst{ja}rla \hld\ þȧ’s ï \alst{v}al falla &
\ind en \alst{Þ}ȯrr á \alst{þ}rę́la kyn.“\eva

\bvb%
“I was in \inx[L]{Walland} and followed wars; \\
I incited princes and never reconciled them. \\
Weden owns the earls which fall among the slain, \\
\ind but Thunder owns the race of thralls.”\evb\evg


\bvg\bva\mssnote{\Regius~13r/13, \AM~1r/8}%
\speakernote{Þórr kvað:}%
„\alst{Ȯ}·jafnt skipta \hld\ es þú myndir með \edtext{\alst{ǫ́}sum}{\Afootnote{\emph{ása} \AM}} liði &
\ind ef þú ę́ttir \alst{v}il-gi mikils \alst{v}ald.“\eva

\bvb%
“Unfairly wouldst thou deal out troops among the Eese,\\
\ind if thou hadst unrestrained power.”\evg


\bvg\bva\mssnote{\Regius~13r/14, \AM~1r/9}%
\Ballnote{The same story is referenced in \textlink{Lokasenna}[60] and told in full in \Gylfaginning\ 45: Lock, Thunder, and his servants Thelve and Wrash had journeyed east for a long time when they came upon a large hall, with an opening on one end as wide as the building.  They rested inside, but in the middle of the night they were awakened by a great earthquake.  Thunder rose and led the party to a side-room to the right in the middle of the hall. He stayed closest to the opening with his hammer ready, while the terrified others were further inside.  At daybreak they left the hall and found the huge ettin \emph{Skrymir} (\inx[P]{Shrimer}) asleep outside.  His snoring had caused the earth-quakes, and the hall was his mitten; the side-room was its thumb.}%
\speakernote{Hár·barðr kvað:}%
„Þȯrr á \alst{a}fl \alst{ǿ}rit \hld\ ęn \alst{ę}kki hjarta; &
af \alst{h}rę́ðslu ok \alst{h}ug-blęyði \hld\ \edtext{vas þér}{\Afootnote{\emph{þér vas} \Regius}} ï \alst{h}andska troðit &
\ind ok \alst{þ}ȯtti⸗sk⸗a þú \alst{þ}ȧ \alst{Þ}ȯrr vesa; &
\alst{h}vár⸗ki þú þá þorðir \hld\ fyr \alst{h}rę́ðslu þinni &
\edtrans{hnjósa né \alst{f}ísa}{sneeze or fart}{\Afootnote{\emph{físa né hnjósa} ‘fart or sneeze’ \AM}} \hld\ svá’t \alst{F}jalarr hęyrði.“\eva

\bvb%
“Thunder has ample strength, but little heart. \\
Out of fear and heart-softness wast thou pushed into a glove, \\
\ind and then seemedest thou not to be Thunder. \\
Thou daredest not—for thy fear— \\
sneeze or fart lest Feller should hear.”\evb\evg


\bvg\bva\mssnote{\Regius~13r/17, \AM~1r/11}%
\speakernote{Þórr kvað:}%
„\alst{H}ár·barðr inn ragi, \hld\ ek munda þik ï \alst{h}ęl drepa &
\ind ef ek mę́tta \alst{s}ęilask of \edtext{\alst{s}und}{\Afootnote{\emph{sundit} \AM}}.“\eva

\bvb%
“O Hoarbeard the \inx[C]{queer}! I would strike thee into \inx[L]{Hell}, \\
\ind if I could sail o’er the sound!”\evb\evg


\bvg\bva\mssnote{\Regius~13r/18, \AM~1r/12}%
\speakernote{Hár·barðr kvað:}%
„Hvat \edtext{skyldir}{\Afootnote{\emph{skalt-u} \AM}} of \alst{s}und \alst{s}ęilask \hld\ es \edtext{\alst{s}akir}{\Afootnote{\emph{sakar} \AM}} ’ro alls øngar? &
\ind hvat vannt-u þá meðan, Þórr?“\eva

\bvb%
“Why shouldst thou sail o’er the sound when there are not at all any charges? \\
\ind what didst thou accomplish meanwhile, Thunder?”\evb\evg


\bvg\bva\mssnote{\Regius~13r/19, \AM~1r/13}%
\speakernote{Þórr kvað:}%
„\alst{E}k vas \alst{au}str \hld\ ok \alst{ǫ́}’na varða’k &
\ind þȧ’s \edtext{mik \alst{s}óttu þęir}{\Afootnote{\emph{þęir sóttu mik} \AM}} \alst{S}várangs \alst{s}ynir; &
\alst{g}rjóti mik bǫrðu, \hld\ \alst{g}agni urðu \edtext{þó}{\Afootnote{om. \AM}} lítt fęgnir, &
þó urðu mik \alst{f}yrri \hld\ \alst{f}riðar at biðja— &
\ind hvat vannt-u þá meðan, Hár·barðr?“\eva

\bvb%
“I was in the east and guarded the river \\
\ind when I was set upon by Sweering’s sons. \\
With rocks they bashed me, they rejoiced yet little in victory; \\
yet they soon had to beg me for peace— \\
\ind what didst thou accomplish meanwhile, Hoarbeard?”\evb\evg


\bvg\bva\mssnote{\Regius~13r/22, \AM~1r/15}%
\speakernote{Hár·barðr kvað:}%
„\alst{E}k vas \alst{au}str \hld\ ok við \edtext{\alst{ęi}n-hvęrja}{\Afootnote{\emph{‘æinhæriu’} \AM}} dǿmða’k, &
\alst{l}ék’k við ina \alst{l}ind-hvítu \hld\ ok \edtrans{\alst{l}aun-þing}{secret trysts}{\Afootnote{so \AM; \emph{laung þing} ‘long trysts’ \Regius}} háða’k, &
\alst{g}ladda’k ina \edtrans{\alst{g}oll-bjǫrtu}{gold-bright}{\Afootnote{\emph{goll-hvítu} ‘gold-white’ \AM}}, \hld\ \alst{g}amni mę́r unði.“\eva

\bvb%
“I was in the east and flirted with a certain someone; \\
I played with the linen-white and held secret trysts: \\
I gladdened the gold-bright—the maiden enjoyed pleasure.”\evb\evg


\bvg\bva\mssnote{\Regius~13r/24, \AM~1r/17}%
\speakernote{Þórr kvað:}%
„\edtext{Góð ǫ́ttu þęir man-kynni þar þȧ.}{\Bfootnote{Clearly prose.}}“\eva

\bvb%
“Then they had good girls there.”\evb\evg


\bvg\bva\mssnote{\Regius~13r/24, \AM~1r/17}%
\speakernote{Hár·barðr kvað:}%
„\alst{L}iðs þíns \edtext{vę́ra’k}{\Afootnote{\emph{vas’k} \AM}} þȧ þurfi, Þórr, \hld\ at ek hęlda þęiri inni \alst{l}ín-hvítu męy.“\eva

\bvb%
“I might have needed thy aid then, Thunder, that I might hold that linen-white maiden.”\evb\evg


\bvg\bva\mssnote{\Regius~13r/25, \AM~1r/18}%
\speakernote{Þórr kvað:}%
„Ek mynda þér \edtext{þȧ þat}{\Afootnote{\emph{þat þá} \AM}} \alst{v}ęita \hld\ ef ek \alst{v}iðr of \edtext{kǿm⸗umk}{\Afootnote{so \AM\ (\emph{‘kæmvmz’}); \emph{kǿmist} \Regius}}.“\eva

\bvb%
“I would then have given it to thee, if I were able.”\evb\evg


\bvg\bva\mssnote{\Regius~13r/26, \AM~1r/18}%
\speakernote{Hár·barðr kvað:}%
„Ek mynda þér þá \alst{t}rúa, \hld\ nema mik ï \alst{t}ryggð véltir.“\eva

\bvb%
“I would then have trusted thee, unless thou wouldst betray my trust.”\evb\evg


\bvg\bva\mssnote{\Regius~13r/27, \AM~1r/19}%
\speakernote{Þórr kvað:}%
„Em’k⸗at ek sá \alst{h}ę̇l-bítr \hld\ sem \alst{h}úð-skór forn ȧ vár.“\eva

\bvb%
“I am not such a heel-biter as an old hide-shoe in spring.\footnoteB{Proverbial (a heel-biter being someone who betrays his companions), the old leather having become stiff over the winter.}”\evb\evg


\bvg\bva\mssnote{\Regius~13r/28, \AM~1r/20}%
\speakernote{Hár·barðr kvað:}%
„Hvat vannt-u þá meðan, Þórr?“\eva

\bvb%
“what didst thou accomplish meanwhile, Thunder?”\evb\evg


\bvg\bva\mssnote{\Regius~13r/28, \AM~1r/20}%
\speakernote{Þórr kvað:}%
„\alst{B}rúðir \alst{b}er-sęrkja \hld\ \alst{b}arða’k ï \edtext{Hlés-ęyju}{\Afootnote{\emph{Hlés-ęy} \AM}}; &
þę́r hǫfðu \alst{v}ęrst unnit, \hld \alst{v}élta þjóð alla.“\eva

\bvb%
“The brides of bearserks I bashed in Leesey; \\
they had done the worst thing: tricked all the people.”\evb\evg


\bvg\bva\mssnote{\Regius~13r/29, \AM~1r/21}%
\speakernote{Hár·barðr kvað:}%
„\alst{K}lę́ki vannt-u þá, Þórr, \hld\ es þú \edtext{ȧ}{\Afootnote{\emph{‘ǽ’} with small corr. above \AM}} \alst{k}onum barðir.“\eva

\bvb%
“A disgrace didst thou accomplish, Thunder, when thou bashedest women!”\evb\evg


\bvg\bva\mssnote{\Regius~13r/30, \AM~1r/22}%
\speakernote{Þórr kvað:}%
„\alst{V}argynjur \edtext{vǫ́ru þę́r}{\Afootnote{\emph{þat vǫ́ru} \AM}} \hld\ en \alst{v}ar⸗la konur, &
\alst{sk}ęlldu \alst{sk}ip mitt \hld\ es \alst{sk}orðat hafða’k, &
\alst{ǿ}gðu \edtext{mér}{\Afootnote{add. \emph{þęim} \AM}} \alst{já}rn-lurki \hld\ en \alst{ę}ltu Þjálfa— &
\ind hvat vannt-u þá meðan, Hár·barðr?“\eva

\bvb%
“She-wolves were they, and hardly women; \\
they overturned my ship which I had propped, \\
terrorised me with an iron cudgel and chased \inx[P]{Thelve} around— \\
\ind what didst thou accomplish meanwhile, Hoarbeard?”\evb\evg


\bvg\bva\mssnote{\Regius~13r/32, \AM~1r/23}%
\speakernote{Hár·barðr kvað:}%
„Ek vas’k ï \alst{h}ęr’num \hld\ es \alst{h}ingat gørði⸗sk &
\alst{g}nę́fa \alst{g}unn-fana, \hld\ \alst{g}ęir at rjóða.“\eva

\bvb%
“I was in the warband when it readied itself hither \\
to raise the war-standard, to redden the spear.”\evb\evg


\bvg\bva\mssnote{\Regius~13v/1, \AM~1r/24}%
\speakernote{Þórr kvað:}%
„Þęss vilt-u nú geta, es þú fórt oss \edtext{ȯ·ljúfan}{\Afootnote{\emph{‘óliyfan’} \AM; \emph{†olubann†} \Regius}} at bjóða!“\eva

\bvb%
“This wilt thou now mention, that thou didst journey to make war upon us!”\evb\evg


\bvg\bva\mssnote{\Regius~13v/2, \AM~1r/25}%
\speakernote{Hár·barðr kvað:}%
„\alst{B}ǿta skal þér \edtext{þat þá}{\Afootnote{om. \AM}} \hld\ munda \alst{b}augi &
sem \alst{ja}fnęndr \alst{u}nnu \hld\ \edtext{þęir’s \alst{o}kkr vilja sę́tta}{\Afootnote{\emph{þęir’s okkr vilja sę́tt hafa} \AM}}.“\eva

\bvb%
“Then I shall repay thee for that with a hand-bigh, \\
bestowed by the mediators who wish to reconcile us two.”\evb\evg


\bvg\bva\mssnote{\Regius~13v/3, \AM~1r/26}%
\speakernote{Þórr kvað:}%
„\alst{H}var namt þęssi \hld\ in \alst{h}nǿfi⸗ligu orð &
es \alst{h}ęyrða’k aldri⸗gi \hld\ \edtext{in}{\Afootnote{so \AM; om. \Regius}} \alst{h}nǿfi⸗ligri?“\eva

\bvb%
“Where didst thou learn these sarcastic words, \\
when I never heard more sarcastic ones?”\evb\evg


\bpg\bpa\mssnote{\Regius~13v/5, \AM~1r/27}%
\speakernote{Hár·barðr kvað:}%
„Nam’k at \edtext{mǫnnum}{\Afootnote{om. \AM}} þęim inum aldr-ǿnum es búa í hęimis-skógum.“\epa

\bpb%
“I learned them from the elderly who dwell in homely forests.”\epb\epg


\bpg\bpa\mssnote{\Regius~13v/5, \AM~1v/1}%
\speakernote{Þórr kvað:}%
„Þó gefr þú gótt nafn \edtrans{dysjum}{poor cairns}{\Bfootnote{A reference to Weden’s consultation of the dead for knowledge as attested e.g. in \textlink{Voluspa} and \textlink{Baldrsdraumar}.}}, es þú kallar þat hęimis-skóga.“\epa

\bpb%
“Yet thou givest a good name to poor cairns, when thou callest them ‘homely forests’.”\epb\epg


\bpg\bpa\mssnote{\Regius~13v/6, \AM~1v/2}%
\speakernote{Hár·barðr kvað:}%
„Svá dǿmi’k \edtext{umb}{\Afootnote{om. \AM}} slíkt far.“\epa

\bpb%
“So do I speak about such matters.”\epb\epg


\bvg\bva\mssnote{\Regius~13v/7, \AM~1v/2}%
\speakernote{Þórr kvað:}%
„\alst{O}rð-kringi þín \hld\ mun þér \alst{i}lla koma &
\ind ef ek rę́ð ȧ \alst{v}ág at \alst{v}aða; &
\alst{u}lfi hę́rra \hld\ hygg’k \edtext{at \alst{ǿ}pa mynir}{\Afootnote{\emph{þik ǿpa munu} \AM}} &
\ind ef \alst{h}lýtr af \alst{h}amri \alst{h}ǫgg.“\eva

\bvb%
“Thy glibness of word will bring thee ill \\
\ind if I decide to wade on the wave. \\
Louder than a wolf I think thou wilt scream \\
\ind if thou gettest a strike from the hammer!”\evb\evg


\bvg\bva\mssnote{\Regius~13v/9, \AM~1v/4}%
\speakernote{Hár·barðr kvað:}%
„Sif á \edtrans{\alst{h}ó}{lover}{\Bfootnote{Most translators take this acc. sg. word as an alternative form of \emph{hórr} m. ‘adulterer’ (gen. \emph{hórs}), containing the same root as \emph{hóra} f. ‘whore, prostitute’, \emph{hór} n. ‘adultery, fornication’, ModEngl. whore. The \emph{-r} has presumably been interpreted as the masc. nom. sg. ending, giving nom. \emph{*hór}, gen. \emph{*hós}. Further, this accusation is also found in \textlink{Lokasenna} TODO, where Lock says that he has been Sib’s lover (\emph{hórr}). Notably, \CV\ interprets this word as the unrelated \emph{hór} m. ‘pot-hook’, “insinuating that Thor busied himself with cooking and dairy-work.” This seems very unlikely when considering Thunder’s response in the next verse: “I think that thou liest!” and the parallel in \textlink{Lokasenna}.}} \alst{h}ęima, \hld\ \alst{h}ans munt fund vilja, &
\alst{þ}ann munt \alst{þ}ręk drýgja, \hld\ \alst{þ}at ’s þér \edtext{skyldara}{\Afootnote{\emph{skyldra} \AM}}.“\eva

\bvb%
“Sib has a lover at home; \emph{him} wilt thou wish to seek! \\
On \emph{him} wilt thou use thy strength—that is more urgent for thee!”\evb\evg


\bvg\bva\mssnote{\Regius~13v/10, \AM~1v/5}%
\speakernote{Þórr kvað:}%
„\alst{M}ę́lir þú at \alst{m}unns ráði \hld\ svá’t \alst{m}ér skyldi vęrst þikkja, &
\alst{h}alr inn \alst{h}ug-blauði, \hld\ \alst{h}ygg’k at þú ljúgir.“\eva

\bvb%
“Thou speakest after thy mouth’s counsel what should seem worst to me— \\
O heart-soft man, I think thou liest!”\evb\evg


\bvg\bva\mssnote{\Regius~13v/12, \AM~1v/6}%
\speakernote{Hár·barðr kvað:}%
„\alst{S}att hygg’k \edtext{mik}{\Afootnote{\emph{þik} \AM}} \alst{s}ęgja, \hld\ \alst{s}ęinn est at fǫr þinni, &
\alst{l}angt myndir nú kominn, Þórr, \hld\ ef þú \edtrans{\alst{l}itum fǿrir}{changed colour}{\Bfootnote{Unclear expression.}}.“\eva

\bvb%
“I think I speak truly. Thou art late on thy journey; \\
far wouldst thou have come by now, Thunder, if thou hadst changed colour.”\evb\evg


\bvg\bva\mssnote{\Regius~13v/14, \AM~1v/8}%
\speakernote{Þórr kvað:}%
„\alst{H}ár·barðr inn ragi, \hld\ \alst{h}ęldr hęfir nú mik \edtext{dvalðan}{\Afootnote{\emph{dvalit} \AM}}!“\eva

\bvb%
“O Hoarbeard the queer! Thou hast now greatly delayed me!”\evb\evg


\bvg\bva\mssnote{\Regius~13v/14, \AM~1v/8}%
\speakernote{Hár·barðr kvað:}%
„\edtext{\alst{Ȧ}sa-Þȯrs}{\Afootnote{\emph{Ása-Þór} \AM}} \hld\ hugða’k \alst{a}ldri⸗gi myndu &
\ind glępja \alst{f}é-hirði \alst{f}arar.“\eva

\bvb%
“Eese-Thunder’s journeys I never thought \\
\ind a shepherd would divert.”\evb\evg


\bvg\bva\mssnote{\Regius~13v/15, \AM~1v/9}%
\speakernote{Þórr kvað:}%
„\alst{R}áð mun’k þér nú \alst{r}áða: \hld\ \alst{r}ó hingat bátinum, &
\alst{h}ę̇ttum \alst{h}ǿtingi, \hld\ \alst{h}itt fǫður Magna!“\eva

\bvb%
“I will now counsel thee counsel!  Row the boat hither; \\
let us cease the taunting; get to the father of Main \ken*{= Thunder = me}!”\evb\evg


\bvg\bva\mssnote{\Regius~13v/17, \AM~1v/10}%
\speakernote{Hár·barðr kvað:}%
„\alst{F}ar þú \edtext{\alst{f}irr}{\Afootnote{\emph{frá} \AM}} sundi, \hld\ þér skal \alst{f}ars synja!“\eva

\bvb%
“Go far away from the sound; passage shall be denied thee!”\evb\evg


\bvg\bva\mssnote{\Regius~13v/17, \AM~1v/11}%
\speakernote{Þórr kvað:}%
„\alst{V}ísa þú mér \edtext{nú}{\Afootnote{om. \AM}} lęið’ina \hld\ alls þú vill mik ęigi of \alst{v}ág’inn fęrja.“\eva

\bvb%
“Show me now the way, as thou wilt not ferry me o’er the wave.”\evb\evg


\bvg\bva\mssnote{\Regius~13v/18, \AM~1v/11}%
\speakernote{Hár·barðr kvað:}%
„\alst{L}ítit ’s \edtext{at}{\Afootnote{om. \Regius}} synja, \hld\ \alst{l}angt ’s at fara; &
\alst{st}und ’s til \edtext{\alst{st}okks’ins}{\Afootnote{\emph{stokks} \AM}}, \hld\ ǫnnur til \edtext{\alst{st}ęins’ins}{\Afootnote{\emph{stęins} \AM}}, &
halt svá til \alst{v}instra \edtext{\alst{v}egs’ins}{\Afootnote{\emph{vegs} \AM}} \hld\ und’s þú hittir \edtrans{\alst{V}er-land}{Wereland}{\Afootnote{\emph{Valland} \AM}\Bfootnote{The Land of Men.}}; &
\alst{þ}ar mun Fjǫrgyn \hld\ hitta \alst{Þ}ȯr, son sinn, &
ok mun hǫ̇n kęnna hǫ̇num \alst{ǫ́}ttunga brautir \hld\ til \alst{Ó}ðins landa.“\eva

\bvb%
“It is a little thing to deny. It is long to journey: \\
an hour to the log, another to the stone; \\
hold so to the left road until thou findest Wereland. \\
There will Firgyn find Thunder, her son, \\
and she will show him the ancestral roads to Weden’s lands \ken*{= Osyard}.”\evb\evg


\bvg\bva\mssnote{\Regius~13v/22, \AM~1v/14}%
\speakernote{Þórr kvað:}%
„Mun’k taka þangat \edtext{ï dag}{\Afootnote{\emph{ȧ dęgi} \AM}}?“\eva

\bvb%
“Will I get thither today?”\evb\evg


\bvg\bva\mssnote{\Regius~13v/22, \AM~1v/14}%
\speakernote{Hár·barðr kvað:}%
„Taka við víl \edtext{ok}{\Afootnote{\emph{við} \AM}} \alst{ę}rfiði \hld\ at \edtext{\alst{u}pp-vesandi}{\Afootnote{\emph{upp-rennandi} \AM}} sólu &
\ind es ek get þána.“\eva

\bvb%
“[Thou wilt] get thither with toil and hardship by the rising of the sun, \\
\ind as I believe it is thawing.”\evb\evg


\bvg\bva\mssnote{\Regius~13v/23, \AM~1v/15}%
\speakernote{Þórr kvað:}
„\alst{Sk}ammt mun nú mál okkat vesa, \hld\ alls þú mér \alst{sk}ǿtingu ęinni svarar; &
launa mun ek þér \alst{f}ar-synjun \hld\ ef vit \alst{f}innum⸗sk ï sinn annat.“\eva

\bvb%
“Short will now our speech be since thou answerest me with scoffing alone. \\
I will reward thee for this ferry-denial if we meet another time!”\evb\evg


\bvg\bva\mssnote{\Regius~13v/25, \AM~1v/17}%
\speakernote{Hár·barðr kvað:}%
\Ballnote{This line is separated with a speech marker from st. 59 in \AM; in \Regius\ it belongs to 59.}%
„Far þú nú þar’s \edtrans{þik hafi allan gramir}{the fiends may have thee whole}{\Bfootnote{The \emph{gramir} ‘fiends’, lit. ‘wroth, cross ones’ are some sort of (male) dæmons.  The adjective \emph{gramr} is commonly used specifically with a connotation of divine wrath and they may thus be supernatural avengers for severe crimes.  The same curse is found in \textlink{Brot}[10].}}!“\eva

\bvb%
“Go now whither the fiends may have thee whole!”\evb\evg

\sectionline
