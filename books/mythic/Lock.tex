\bookStart{Flyting of Lock}[Loka sęnna]
\setBookCode{Lokasenna}

\begin{flushright}%
\textbf{Dating} \parencite{Sapp2022}: C10th (0.965)

\textbf{Meter:} \Ljodahattr%
\end{flushright}

\section{Introduction}

The \textbf{Flyting of Lock} (\Lokasenna) is only preserved in \Regius, where it follows \textlink{Hymiskvida} and foregoes \textlink{Thrymskvida}.  In \Regius\ it is tied together into a continuous narrative with \textlink{Hymiskvida} by the prose passage “From Eagre and the Gods”, but the two poems are so drastically different in style that they are certainly distinct compositions.  In \AM, \textlink{Hymiskvida} stands alone with no trace of a frame narrative while no trace of \Lokasenna\ survives, so that the frame narrative.

A stanza that appears to belong to \textlink{Lokasenna} is found in \Gylfaginning\ 20; it is edited below following the end of the poem.  The poem—and especially its introductory prose (\Lokasenna\ P1, P2)—is also alluded to in \Skaldskaparmal\ 41.

The poem has often (TODO) been interpreted as a blasphemous composition belonging to the period after conversion, with the reasoning that no pious pagan would have written a poem insulting his own gods.  On the other hand its archaic language and the breadth of mythological knowledge point to the pagan period, nor is the attack on the gods something the poet necessarily agrees with; after all, Lock is punished by the most popular god of the Wiking Age, Thunder.

From Lock’s insult of Bray in st. 11 until Thunder’s arrival in st. 57 the poem has a very regular structure; Lock and a god take turns insulting each other for one or two pairs of stanzas before another god comes to the defense of the previous one and becomes the new target for Lock’s insults.

\emph{Overview of speakers, first occurrence in bold:}

St. 1 \textbf{Lock}, 2 \textbf{Elder}, 3 Lock, 4 Elder, 5 Lock —
6–7 Lock, 8 \textbf{Bray} —
9 Lock, 10 \textbf{Weden} —
11 Lock, 12 Bray, 13 Lock, 14 Bray, 15 Lock —
16 \textbf{Idun}, 17 Lock, 18 Idun —
19 \textbf{Yiven}, 20 Lock —
21 Weden, 22 Lock, 23 Weden, 24 Lock —
25 \textbf{Frie}, 26 Lock, 27 Frie, 28 Lock —
29 \textbf{Frow}, 30 Lock, 31 Frow, 32 Lock —
33 \textbf{Nearth}, 34 Lock, 35 Nearth, 36 Lock —
37 \textbf{Tew}, 38 Lock, 39 Tew, 40 Lock —
41 \textbf{Free}, 42 Lock —
43 \textbf{Bewer}, 44 Lock, 45 Bewer, 46 Lock —
47 \textbf{Homedal}, 48 Lock —
49 \textbf{[Shede]}, 50 Lock, 51 [Shede], 52 Lock —
53 \textbf{Sib}, 54 Lock —
55 \textbf{Beal}, 56 Lock —
57 \textbf{Thunder}, 58 Lock, 59 Thunder, 60 Lock, 61 Thunder, 62 Lock, 63 Thunder, 64–65 Lock


\newpage

\section{From Eagre and the Gods (\emph{Frá Ę́gi ok goðum})}

\bpg\bpa Ę́gir, er ǫðru nafni hét Gymir, hann hafði búit ásum ǫl þá er hann hafði fengit ketil inn mikla \edtrans{sem nú er sagt}{as was just told}{\Bfootnote{In immediately preceding \textlink{Hymiskvida}.}}.  Til þeirar veitslu kom Óðinn ok Frigg kona hans. Þórr kom eigi því at hann var í austr-vegi. Sif var þar, kona Þórs; Bragi, ok Iðunn kona hans. Týr var þar, hann var ein-hendr; \edtrans{Fenrisulfr sleit hǫnd af hánum, þá er hann var bundinn}{the Fenrerswolf tore his hand off when it was bound}{\Bfootnote{This detail is presumably brought up on the basis of sts. 38–39; for the myth see there.}}.  Þar var Njǫrðr ok kona hans Skaði; Freyr ok Freyja; Víðarr son Óðins. Loki var þar, ok þjónustu-menn Freys, Byggvir ok Beyla. Mart var þar ása ok alfa.\epa

\bpb \inx[P]{Eagre}[{\huge E}\textsc{agre}], who by another name was called \inx[P]{Gymer}—he had prepared an ale-feast for the Eese when he had got the great kettle as was just told.
To that gathering came \inx[P]{Weden} and \inx[P]{Frie} his wife.
\inx[P]{Thunder} came not, for he was on the \inx[L]{Eastern Way}. Sib was there, Thunder’s wife;
\inx[P]{Bray} and \inx[P]{Idun} his wife.
\inx[P]{Tew} was there, he was one-handed; the \inx[P]{Fenrerswolf} tore his hand off when it was bound.
\inx[P]{Nearth} was there and his wife \inx[P]{Shede},
\inx[P]{Free} and \inx[L]{Frow},
\inx[P]{Wider} the son of \inx[P]{Weden}.
\inx[P]{Lock} was there,
and the servants of Free \inx[P]{Bew} and \inx[P]{Beal}.
There was a great deal of \inx[G]{Eese} and \inx[G]{Elves}.\epb\epg


\bpg\bpa Ę́gir átti tvá þjónustu-menn, Fimafengr ok Eldir. Þar var lýsi-gull haft fyr elds-ljós; sjalft barsk þar ǫl. Þar var \edtrans{griða-stadr mikill}{place of grith}{\Bfootnote{A place wherein all violence was forbidden, see Index: grith.  This detail comes to play an important later in the narrative.}}. Menn lofuðu mjǫk hversu góðir þjónustu-menn Ę́gis vóru. Loki mátti eigi heyra þat, ok drap hann Fimafeng. \edtrans{Þá skóku ę́sir skjǫldu sína ok ǿptu}{Then the Eese shook their shields and screamed}{\Bfootnote{Apparently in some sort of ancient war dance.  Cf. the Old Swedish Heathen Law: “He screams three nithing-screams TODO”.}} at Loka, ok eltu hann braut til skógar, en þeir fóru at drekka. Loki hvarf aptr ok hitti úti Eldi; Loki kvaddi hann:\epa

\bpb Eagre had two servants, \inx[P]{Femfinger} and \inx[P]{Elder}. There glowing gold was used instead of fire; the ale there carried itself. It was a great place of \inx[C]{grith}. Men greatly praised how good the servants of Eagre were; Lock could not stand to hear it, and he slew Femfinger. Then the Eese shook their shields and screamed at Lock, and chased him away to the woods—but they went [back] to drinking.  Lock turned back and hit upon Elder outside.  Lock greeted him:\epb\epg


\section{The Flyting of Lock}

\bvg\bva\mssnote{\Regius~15r/17}%
„Sęg þú þat, \alst{Ę}ldir, \hld\ \edtext{svá’t \alst{ęi}nu-gi &
\ind \alst{f}eti gangir \alst{f}ramarr}{\lemma{svá’t \dots\ framarr ‘so that \dots\ further’}\Bfootnote{Shared with \textlink{Havamal} 38.}}, &
hvat hér \alst{i}nni \hld\ \edtrans{hafa at \alst{ǫ}l-mǫ́lum}{they say over the ale}{\Bfootnote{Lit. “they have for their ale-speeches”.}} &
\ind \alst{s}ig-tíva \alst{s}ynir.“\eva

\bvb “{\huge T}\textsc{ell this, Elder}, so that thou not \\
\ind takest one step further: \\
What here within they say over the ale, \\
\ind the sons of the victory-Tews \ken{gods}.”\evb\evg


\bvg\bva\mssnote{\Regius~15r/18}%
\speakernote{Ęldir:}%
„Of \alst{v}ǫ́pn sïn dǿma \hld\ ok of \alst{v}íg-risni sïna &
\ind \alst{s}ig-tíva \alst{s}ynir; &
\alst{ȧ}sa ok \alst{a}lfa, \hld\ es hér \alst{i}nni eru, &
\ind \edtrans{mann-gi ’s þér ï \alst{o}rði \alst{v}inr.}{none is thee a friend in words.}{\Bfootnote{I.e., “nobody says anything good about you.” — The alliteration here is notable, and also occurs in st. 10 (\emph{Víðarr} : \emph{ulfs}, see note there).  There are no signs of corruption, and so there are two possible explanations.  Either (1) the semi-vowel \emph{v} (\textipa{/w/}) is participating in vowel-alliteration with \emph{o}— such alliteration between \emph{v} and true vowels is never encountered in Scaldic poetry, but there are some examples from Eddic styles—or (2) the poem (or the relevant lines) was composed before the North Germanic loss of \emph{v} before rounded vowels.  (2) finds support in the notable fact that in both the present st. and st. 10 the words \emph{orð} ‘word’ and \emph{ulfr} ‘wolf’ originally began with \emph{v}; in the case of the word \emph{ulfr} this consonant is attested in old Scandinavian runic inscriptions.  For metrical reasons the lines must postdate the syncope of most unstressed short vowels, but on the basis of the three closely related C7th runestones from Blekinge (DR 357–359, from Stentoften, Gummarp, and Istaby) the loss of \emph{w} before rounded vowels is shown to have occurred later; so DR 359 \textbf{h\textsc{a}þuwulafʀ} \emph{Haþuwulᵃfʀ}.  If the alliteration indeed should fall on \emph{v}, this would not require dating the whole \textlink{Lokasenna} to the late Proto-Norse period (indeed, according to the analysis done by \textcite{Sapp2022}, it is not even the linguistically oldest poem preserved); the older forms could, for instance, reflect archaic poetic formulae.  A C7th Proto-Norse form of this c-line might be: \emph{*mann-gí ’s þéʀ in worðé winiʀ}.}}“\eva

\bvb\speakernoteb{Elder quoth:}%
“Of their weapons they speak and of their battle-prowess, \\
\ind the sons of the victory-Tews \ken{gods}. \\
Of the Eese and Elves which are here within \\
\ind none is thee a friend in words.”\evb\evg


\bvg\bva\mssnote{\Regius~15r/20}%
\speakernote{Loki kvað:}%
„\alst{I}nn skal ganga \hld\ \alst{Ę́}gis hallir ï &
\ind ȧ þat \edtrans{\alst{s}umbl}{simble}{\Bfootnote{The Germanic word for “feast, banquet”.}} at \alst{s}éa, &
\edtrans{\alst{jǫ}ll ok \alst{ǫ́}fu}{scorn and hatred}{\Bfootnote{Two rare words to which the present translation hardly does justice.  The former occurs nowhere else, while the latter only otherwise occurs in \textlink{Sigurdskamma} 33.  They have been interpreted in a variety of ways: \CV\ sees the first word as \emph{jóll} ‘wild angelica’, whereas the second is taken to be an error for \emph{áfr} (“a beverage [...] translated by Magnaeus by \emph{sorbitio avenacea}, a sort of common ale brewed of oats”).  TODO: What do other editors say? Esp. Kommentar.}} \hld\ fǿri’k \alst{ȧ}sa sonum &
\ind ok \edtext{blęnd’k þęim svá \alst{m}ęini \alst{m}jǫð}{\lemma{blęnd’k \dots\ męini mjǫð ‘I mix \dots\ mead with evil’}\Bfootnote{Formulaic, cf. \textlink{Sigrdrifumal} 8 (and others TODO).}}.“\eva

\bvb\speakernoteb{Lock quoth:}%
“I shall go into Eagre’s halls, \\
\ind on that \inx[C]{simble} for to see. \\
Scorn and hatred I bring the sons of the Eese, \\
\ind and so I mix their mead with evil.”\evb\evg


\bvg\bva\mssnote{\Regius~15r/22}%
\speakernote{Ęldir kvað:}%
„Vęitst, ef \alst{i}nn gęngr \hld\ \alst{Ę́}gis hallir ï &
\ind ȧ þat \alst{s}umbl at \alst{s}éa, &
\alst{h}rópi ok rógi \hld\ ef ęyss ȧ \alst{h}oll ręgin, &
\ind ȧ \alst{þ}ér munu þau \alst{þ}ęrra \alst{þ}at.“\eva

\bvb\speakernoteb{Elder quoth:}%
“Thou knowest, if thou goest into Eagre’s halls, \\
\ind on that simble for to see— \\
if slander and strife thou pourest on the \inx[C]{hold} \inx[G]{Reins}, \\
\ind on \emph{thee} will they dry it off!”\evb\evg


\bvg\bva\mssnote{\Regius~15r/24}%
\speakernote{Loki kvað:}%
„Vęitst þat \alst{Ę}ldir, \hld\ ef \alst{ęi}nir skulum &
\ind \edtrans{\alst{s}ár-yrðum \alst{s}akask}{bandy wounding words}{\Bfootnote{Reoccuring at st. 19/2.}}, &
\alst{au}ðigr verða \hld\ mun’k ï \alst{a}nd-svǫrum, &
\ind \edtrans{ef þú \alst{m}ę́lir til \alst{m}art!}{if thou speak too much!}{\Bfootnote{Formulaic; cf. \textlink{Havamal} 27.}}“\eva

\bvb\speakernoteb{Lock quoth:}%
“Thou knowest, Elder, if one-on-one we shall \\
\ind bandy wounding words, \\
wealthy will I in my answers grow \\
\ind if thou speak too much!”\evb\evg


\bpg\bpa\mssnote{\Regius~15r/25}%
Síðan gekk Loki inn í hǫll’ina; en er þeir sá, er fyrir váru, hverr inn var kominn, þǫgnuðu þeir allir.\epa

\bpb Thereafter Lock went into the hall, but when those who were further within saw who had come inside they all turned silent.\epb\epg


\bvg\bva\mssnote{\Regius~15r/27}%
\speakernote{Loki kvað:}%
„\alst{Þ}yrstr ek kom \hld\ \alst{þ}ęssar hallar til &
\ind \alst{L}optr of \alst{l}angan veg, &
\alst{ǫ̇}su at biðja, \hld\ at mér \alst{ęi}nn gefi &
\ind \edtrans{\alst{m}ę́ran drykk \alst{m}jaðar}{famed drink of mead}{\Bfootnote{Formulaic language for describing mead; cf. \textlink{Havamal} 105, 140, \textlink{Skirnismal} 16. TODO: more parallels.}}.\eva

\bvb\speakernoteb{Lock quoth:}%
“Thirsty I came to these halls, \\
\ind Loft \name{= Lock}, over a long way, \\
to bid the Eese that they give me but one \\
\ind famed drink of mead.\evb\evg


\bvg\bva\mssnote{\Regius~15r/28}%
Hví \alst{þ}ęgið ér svá \hld\ \alst{þ}rungin goð, &
\ind at \alst{m}ę́la né \alst{m}ęguð; &
\edtext{\alst{s}essa ok staði \hld\ vęlið mér \alst{s}umbli at, &
\ind eða \alst{h}ęitið mik \alst{h}eðan!}{\lemma{sessa \dots\ heðan! ‘Choose \dots\ hence!’}\Bfootnote{That is, “Cease your dallying; give me a seat or tell me to leave!”}}“\eva

\bvb Why shut ye up so, ye pressed Gods, \\
\ind that ye cannot speak? \\
Choose seats and places for me at the simble, \\
\ind or call me away hence!”\evb\evg


\bvg\bva\mssnote{\Regius~15r/30}%
\speakernote{Bragi:}%
„\alst{S}essa ok staði \hld\ vęlja þér \alst{s}umbli at &
\ind \alst{ę̇}sir \alst{a}ldri⸗gi; &
því’t \alst{ę̇}sir vitu \hld\ \edtrans{hvęim \alst{a}lda}{which man}{\Bfootnote{Here “person, being”.  See note to \textlink{Vafthrudnismal} 55/6.}} skulu &
\ind \edtrans{\alst{g}amban-sumbl}{gomben-simble}{\Bfootnote{\emph{gamban} ‘gomben’ being an obscure prefix which only occurs in \textlink{Lokasenna}, \textlink{Skirnismal} and \textlink{Harbardsljod}.  \CV\ suggest it means something like “costly”.}} of \alst{g}eta.“\eva

\bvb\speakernoteb{Bray [quoth]:}%
“Choose seats and places for thee at the simble \\
\ind the Eese will never do, \\
for the Eese know for which man they shall \\
\ind prepare the gomben-simble.”\evb\evg


\bvg\bva\mssnote{\Regius~15r/32}%
\Ballnote{Lock turns to Weden, chief of the Eese, and reminds him of an oath of blood-brotherhood the two undertook in the early days of the world.  The circumstances of this oath are otherwise entirely unknown.}%
\speakernote{[Loki:]}%
„Mant þat \alst{Ó}ðinn, \hld\ es vit ï \alst{á}r-daga &
\ind \alst{b}lendum \alst{b}lóði saman? &
\alst{ǫ}lvi bęrgja \hld\ létsk \alst{ęi}gi mundu, &
\ind nema okkr vę́ri \alst{b}ǫ́ðum \alst{b}orit.“\eva

\bvb\speakernoteb{[Lock quoth:]}%
“Recallest thou, Weden, when we in days of yore \\
\ind blended our blood together? \\
Thou didst declare thou wouldst ne’er taste ale, \\
\ind unless it were for us both borne forth!”\evb\evg


\bvg\bva\mssnote{\Regius~15r/34}%
\speakernote{[Óðinn:]}%
„\edtrans{Rís þȧ \alst{V}íðarr \hld\ ok lát \alst{u}lfs fǫður}{Rise thou, Wider, and let the Wolf’s father}{\Bfootnote{For the alliteration see note to st. 2/4.  A C7th Proto-Norse form of the line might be: \emph{*Rís þan Wíðarʀ · auk lát wulfs fǫður}.}} &
\ind \alst{s}itja \alst{s}umbli at, &
síðr oss \alst{L}oki \hld\ kvęði \alst{l}asta-stǫfum &
\ind \alst{Ę́}gis hǫllu \alst{ï}.“\eva

\bvb\speakernoteb{[Weden quoth:]}%
“Then rise, O Wider, and let the Wolf’s father \ken*{= Lock} \\
\ind sit at the simble, \\
lest Lock greet us with words of blame \\
\ind in Eagre’s hall.”\evb\evg


\bpg\bpa\mssnote{\Regius~15v/1}%
Þá stóð Víðarr upp ok skenkti Loka, en áðr hann drykki, kvaddi hann ásuna:\epa

\bpb Then Wider stood up and poured a drink to Lock, but before he \ken*{= Lock} drank, he greeted the Eese:\epb\epg


\bvg\bva\mssnote{\Regius~15v/3}%
\Ballnote{The first line is identical to the prayer in \textlink{Sigrdrifumal}[4].  An intriguing possibility suggests itself, viz. that this formula for hailing the Gods may actually have been used in Heathen drinking-toasts, where the second helming (ll. 3–4) would follow the hailing by asking for gifts.  If this is the case it would greatly increase the dramatic effect of Lock’s insult in ll. 3–4, since he, instead of asking for a boon, subverts the prayer by insulting one of the gods present, something that would have have come off as shocking to the Heathen audience expecting a typical prayer.}%
„Hęilir \alst{ę̇}sir, \hld\ hęilar \alst{ǫ̇}synjur &
\ind ok ǫll \alst{g}inn-hęilǫg \alst{g}oð, &
nema sá \alst{ęi}nn \alst{ǫ̇}ss \hld\ es \alst{i}nnar sitr &
\ind \alst{B}ragi \alst{b}ękkjum ȧ.“\eva

\bvb “Hail the \inx[G]{Eese}! Hail the \inx[G]{Ossens}, \\
\ind and all \inx[C]{yin-holy} Gods! \\
But for that one \inx[G]{Eese}[os] who sits further within: \\
\ind Bray, upon the benches.”\evb\evg


\bvg\bva\mssnote{\Regius~15v/4}%
\speakernote{[Bragi] kvað:}%
„\edtrans{\alst{M}ar ok \alst{m}ę́ki}{Steed and sword}{\Bfootnote{Formulaic pair; cf. \textlink{Havamal} 83/2.}} \hld\ gef’k þér \alst{m}ïns féar &
\ind ok \edtrans{\alst{b}ǿtir þér svá \alst{b}augi}{restores thee with a bigh}{\Bfootnote{Cf. the comment in \textlink{Harbardsljod}[42].}} \alst{B}ragi, &
síðr þú \alst{ǫ̇}sum \hld\ \alst{ǫ}fund of gjaldir; &
\ind \alst{g}ręm þú ęigi \alst{g}oð at þér!“\eva

\bvb\speakernoteb{{[Bray]} quoth:}%
“Steed and sword I give thee of my own wealth \\
\ind and so Bray restores thee with a \inx[C]{bigh}, \\
lest thou with envy repay the Eese; \\
\ind anger not the Gods against thee!”\evb\evg


\bvg\bva\mssnote{\Regius~15v/6}%
\speakernote{[Loki] kvað:}%
„\alst{Jó}s ok \alst{a}rm-bauga \hld\ munt \alst{ę́} vesa &
\ind \alst{b}ęggja vanr \alst{B}ragi, &
\alst{ȧ}sa ok \alst{a}lfa, \hld\ es hér \alst{i}nni eru, &
\ind þú est við \alst{v}íg \alst{v}arastr, &
\ind ok \alst{sk}jarrastr við \alst{sk}ot.“\eva

\bvb\speakernoteb{{[Lock]} quoth:}%
“Of horse and arm-bighs wilt thou always be \\
\ind both lacking, Bray! \\
Of the Eese and Elves which are here within \\
\ind thou art wariest of war \\
\ind and shiest of shot.”\evb\evg


\bvg\bva\mssnote{\Regius~15v/8}%
\speakernote{[Bragi] kvað:}%
„\edtext{Vęit’k, ef fyr \alst{ú}tan vę́ra’k, \hld\ svá sem fyr \alst{i}nnan em’k, &
\ind \alst{Ę́}gis hǫll \alst{o}f kominn}{\lemma{Vęit’k, ef fyr útan vę́ra’k, \hld\ svá sem fyr innan em’k, / Ę́gis hǫll of kominn ‘I know if I were outside, as I am inside come into Eagre’s hall’}\Bfootnote{As said in P1, the law of \inx[C]{grith} (a truce of non-violence, even between enemies) applies inside the hall, and Bray and the other gods are honour-bound not to injure Lock.}}, &
\alst{h}ǫfuð þitt \hld\ bę́ra’k ï \alst{h}ęndi mér; &
\ind\edtext{\alst{l}ít’k þér þat fyr \alst{l}ygi}{\Bfootnote{\emph{‘litt ec þer þat fyr lygi’} \Regius.  A variety of emendations have been proposed for this line.  Simplest would be \emph{lítt es þér þat fyr lygi} ‘that is little [punishment] for thee for lying’. Based on the similarity of \emph{ꞇ̇} (= \emph{tt}) and \emph{c} \textcite{FinnurEdda} gives \emph{lyka’k þér þat fyr lygi} ‘that I would bring thee for thy lie’.}}.“\eva

\bvb\speakernoteb{{[Bray]} quoth:}%
“I know if I were without, as I am within \\
\ind come into Eagre’s hall, \\
that head of thine would I hold in my hands; \\
\ind this I see for thy lie.”\evb\evg


\bvg\bva\mssnote{\Regius~15v/10}%
\speakernote{[Loki] kvað:}%
\Ballnote{Lock attacks Bray’s excuse; a truly brave man would break even the grith-truce to avenge such a personal insult.}%
„\alst{S}njallr est ï \alst{s}essi, \hld\ skal⸗at-tu \alst{s}vá gęra, &
\ind \alst{B}ragi \alst{b}ękk-skrautuðr; &
\alst{v}ega þú gakk \hld\ ef \alst{v}ręiðr séir; &
\ind \alst{h}yggsk vę́tr \alst{h}vatr fyrir.“\eva

\bvb\speakernoteb{{[Lock]} quoth:}%
“Valiant art thou in the seat; thou shalt not do so, \\
\ind Bray the bench-adorner! \\
Go to strike if thou art wroth; \\
\ind the bold thinks not ahead.”\evb\evg


\bvg\bva\mssnote{\Regius~15v/11}%
\speakernote{[Iðunn] kvað:}%
„\alst{B}ið ek, \alst{B}ragi, \hld\ \alst{b}arna-sifjar duga &
\ind ok allra \alst{ȯ}sk-maga, &
at þú \alst{L}oka \hld\ kveðir⸗a \alst{l}asta-stǫfum &
\ind \alst{Ę́}gis hǫllu \alst{ï}.“\eva

\bvb\speakernoteb{[Idun] quoth:}%
“I bid thee, Bray, to respect the ties of children \\
\ind and all adopted sons, \\
that thou not greet Lock with words of blame \\
\ind in Eagre’s hall.”\evb\evg


\bvg\bva\mssnote{\Regius~15v/13}%
\speakernote{[Loki] kvað:}%
„Þęgi þú, \alst{I}ðunn, \hld\ þik kveð’k \alst{a}llra kvinna &
\ind \alst{v}er-gjarnasta \alst{v}esa &
síðst þú \alst{a}rma þïna \hld\ lagðir \alst{í}tr-þvęgna &
\ind umb þinn \alst{b}róður-\alst{b}ana.“\eva

\bvb\speakernoteb{{[Lock]} quoth:}%
“Shut up, Idun! I call thee of all women \\
\ind the most man-eager, \\
since thy brightly washed arms thou didst lay \\
\ind around thy brother’s bane.”\evb\evg


\bvg\bva\mssnote{\Regius~15v/15}%
\speakernote{[Iðunn] kvað:}%
„\alst{L}oka ek kveð’k⸗a \hld\ \alst{l}asta-stǫfum &
\ind \alst{Ę́}gis hǫllu \alst{ï}; &
\alst{B}raga ek kyrri \hld\ \alst{b}jór-ręifan, &
\ind \alst{v}il’k⸗at at it \alst{v}ręiðir \alst{v}egisk.“\eva

\bvb\speakernoteb{{[Idun]} quoth:}%
“I am not greeting Lock with words of blame \\
\ind in Eagre’s hall; \\
I am calming Bray, made rowdy from beer— \\
\ind I wish not that ye two wroth men should fight.”\evb\evg


\bvg\bva\mssnote{\Regius~15v/16}%
\speakernote{[Gęfjun] kvað:}%
„Hví it \alst{ę̇}sir tvęir \hld\ skuluð \alst{i}nni hér &
\ind \alst{s}ár-yrðum \alst{s}aka⸗sk? &
\edtrans{\alst{L}opt-ki þat vęit}{Loft knows not}{\Bfootnote{For the rare poetic construction where a verb is negated by suffixing \emph{-gi} to its subject cf. sts. 29/4 and 39/3 below.}} \hld\ at hann \alst{l}ęikinn es &
\ind ok hann \edtrans{\alst{f}jǫrg \emph{ǫ}ll \emph{\alst{f}ía}}{and all the Farrows \name{= Gods} hate him}{\Afootnote{\emph{‘fiorgvall fría’} \Regius}\Bfootnote{\emph{‘fiorgvall’} \Regius\ is taken as an error for \emph{fiorg avll}, norm. \emph{fjǫrg ǫll}, where \emph{fjǫrg} ‘Farrows’ (sg. \emph{fjarg}) is a rare word for ‘Gods’. —
\emph{fría} means ‘love’ but is emended to its antonym \emph{fía} ‘hate’ since the statement that all the Gods love Lock can hardly be defended.  Cf. \textlink{Hymiskvida}[22] for the Gods’ hatred of one of Lock’s children, the Middenyardswyrm.}}.”\eva

\bvb\speakernoteb{{[Yiven]} quoth:}
“Why shall ye two Eese here within \\
\ind bandy wounding words? \\
Loft \name{= Lock} knows not that he is crazed, \\
\ind and all the Farrows \name{= Gods} hate him.”\evb\evg


\bvg\bva\mssnote{\Regius~15v/18}%
\speakernote{[Loki] kvað:}%
„\alst{Þ}ęgi þú, Gęfjun, \hld\ \alst{þ}ęss mun’k nú geta &
\ind es þik \alst{g}lapði at \alst{g}ęði: &
\alst{s}vęinn inn hvíti \hld\ es þér \edtrans{\alst{s}igli}{necklace}{\Bfootnote{A very rare poetic word ultimately derived from Latin \emph{sigillum}.  It only occurs in two other places in the Old Norse corpus: \textlink{Sigurdskamma}[49]/3 and KormǪ Lv 56 (\Skp\ 5).}} gaf &
\ind ok þú \alst{l}agðir \alst{l}ę́r yfir.“\eva

\bvb\speakernoteb{{[Lock]} quoth:}%
“Shut up, Yiven! \emph{Him} will I now mention, \\
\ind who seduced thy senses: \\
the white-hued swain who gave thee a necklace, \\
\ind and thou laidest o’er him thy leg!”\evb\evg


\bvg\bva\mssnote{\Regius~15v/20}%
\speakernote{[Óðinn kvað] þat:}%
„\edtext{\alst{Ǿ}rr est, Loki, \hld\ ok \alst{ø}r·viti,}{\lemma{Ǿrr \dots\ ok ør·viti ‘Mad \dots\ and out of thy wits’}\Bfootnote{Formulaic line, also occuring in \textlink{HelgakvidaTwo}[34] and \textlink{Oddrunargratr}[15].  Cf. also st. 47 and G1 below.}} &
\ind es þú fę̇r þér \alst{G}ęfjun at \alst{g}ręmi &
því’t \alst{a}ldar \alst{ø}r·lǫg \hld\ hygg at \alst{ǫ}ll of viti &
\ind \alst{ja}fn-gǫrla sem \alst{e}k.“\eva

\bvb\speakernoteb{{[Weden quoth]} this:}%
“Mad art thou, Lock, and out of thy wits \\
\ind when thou rousest Yiven’s wrath against thee, \\
for all the orlays of men I think she knows \\
\ind quite as clearly as I.”\evb\evg


\bvg\bva\mssnote{\Regius~15v/22}%
\speakernote{[Loki] kvað:}%
„Þęgi þú, \alst{Ó}ðinn, \hld\ þú kunnir \alst{a}ldri⸗gi &
\ind dęila \alst{v}íg með \alst{v}erum; &
opt þú \alst{g}aft \hld\ þęim’s \alst{g}efa skyldir⸗a, &
\ind inum \alst{s}lę́vurum, \alst{s}igr.“\eva

\bvb\speakernoteb{{[Lock]} quoth:}%
“Shut up, Weden! Thou couldst never \\
\ind deal out war amidst men— \\
oft thou gavest to those thou shouldst not have given— \\
\ind to the duller men—victory.”\evb\evg


\bvg\bva\mssnote{\Regius~15v/24}%
\speakernote{[Óðinn] kvað:}%
„Vęitst ef ek \alst{g}af \hld\ þęim’s \alst{g}efa né skylda, &
\ind inum \alst{s}lę́vurum, \alst{s}igr, &
\alst{á}tta vetr \hld\ vast fyr \alst{jǫ}rð neðan &
\ind \edtrans{\alst{k}ýr mólkandi}{a milch cow}{\Bfootnote{May also be read as “milking cows”, the nom. sg. \emph{kýr} being identical to the acc. pl. \emph{kýr}, and \emph{mólka} meaning both ‘to milk’ and ‘to give milk’.  “Milch cow” is preferable for several reasons.  Firstly, the phrase is followed by \emph{ok kona} ‘and a woman’ rather than \emph{sem kona} ‘as a woman’ or similar; secondly, it conforms to a known pattern of insults in flyting where the insultee is equated with a woman and said to have been impregnated (cf. \textlink{HelgakvidaOne}), cows, of course, only giving milk after calving; thirdly, it agrees well with another instance where Lock gives birth in the form of a female animal, namely the episode of the building of the wall around Osyard as told in \Gylfaginning\ 42.}} ok \alst{k}ona &
\ind ok hęfir þar \alst{b}ǫrn of \alst{b}orit &
\ind ok hugða’k þat \alst{a}rgs \alst{a}ðal.“\eva

\bvb\speakernoteb{{[Weden]} quoth:}%
“Thou knowest, if I gave to to those I should not have given— \\
\ind to the duller men—victory, \\
for eight winters wast thou beneath the earth \\
\ind a milch cow and a woman, \\
\ind and thou hast there borne children, \\
\ind and I’ve judged that a \inx[C]{queer}’s nature!”\evb\evg


\bvg\bva\mssnote{\Regius~15v/26}%
\Ballnote{For Weden taking the shape of a witch or warlock cf. the story of Wrind (\textlink{Voluspa}[31] n.)  TODO: elaborate.}%
\speakernote{[Loki] kvað:}%
„Ęn þik \alst{s}íga kóðu \hld\ \edtrans{\alst{S}ȧms-ęyju}{Samsey}{\Bfootnote{\emph{Samsø} in Denmark, which features prominently in later legendary literature.  TODO: elaborate.}} ï &
\ind ok \edtrans{drapt ȧ \alst{v}ett sem \alst{v}ǫlur}{didst beat on the drum like wallows}{\Bfootnote{A reference to “shaman”-like drums as attested in the later Lappish religion.  TODO: elaborate.}}, &
\alst{v}itka líki \hld\ fórt \alst{v}er-þjóð yfir, &
\ind ok hugða’k þat \alst{a}rgs \alst{a}ðal.“\eva

\bvb\speakernoteb{{[Lock]} quoth:}%
“But thou, they said, didst sink down into Samsey, \\
\ind and didst beat on the drum like wallows. \\
In a warlock’s likeness didst thou journey o’er the folk of men \\
\ind and I’ve judged \emph{that} a queer’s nature!”\evb\evg


\bvg\bva\mssnote{\Regius~15v/28}%
\speakernote{[Frigg kvað:]}%
„\alst{Ø}r·lǫgum \alst{y}kkrum \hld\ skylið \alst{a}ldri⸗gi &
\ind \alst{s}ęgja \alst{s}ęggjum frȧ, &
hvat it \alst{ę̇}sir tvęir \hld\ drýgðuð ï \alst{á}r-daga; &
\ind \alst{f}irrisk ę́ \alst{f}orn rǫk \alst{f}irar.“\eva

\bvb\speakernoteb{[Frie quoth:]}%
“Of your orlays should ye two never \\
\ind speak to the youth. \\
No matter what ye two Eese did in days of yore, \\
\ind let ancient tales be ever shunned by folk.”\evb\evg


\bvg\bva\mssnote{\Regius~15v/30}%
\speakernote{[Loki kvað:]}%
„Þęgi þú, \alst{F}rigg, \hld\ þú est \alst{F}jǫrgyns mę́r &
\ind ok hęfir ę́ \alst{v}er-gjǫrn \alst{v}esit, &
es þȧ \edtrans{\alst{V}éa ok Vilja}{Will and Wigh}{\Bfootnote{Weden’s brothers.  The event alluded to here is a myth where Weden is exiled from the Eese and his wife, Frie, instead marries his two brothers.  So \YnglingaSaga\ 3: \emph{Óðinn átti tvá brǿðr, hét annarr Vé, en annarr Vili; þeir brǿðr hans stýrðu ríki’nu, þá er hann var í brottu.  Þat var eitt sinn, þá er Óðinn var farinn langt í brott ok hafði lengi dvaltsk, at ǫ́sum þótti ør·vę́nt hans heim; þá tóku brǿðr hans at skipta arfi hans, en konu hans Frigg géngu þeir báðir at eiga.  En litlu síðar kom Óðinn heim, tók hann þá við konu sinni.} ‘Weden had two brothers; one was called Wigh, the other Will.  Those brothers of his ruled the realm when he was departed.  It was one time when Weden had journeyed far away and had tarried for long that the Eese thought his homecoming unlikely.  Then his brothers took to divide his inheritance, but his wife Frie they both went to own.  But a short while later Weden came home; then he took back his wife.’}} \hld\ létst þér, \alst{V}iðris kvę̇n, &
\ind \alst{b}áða ï \alst{b}aðm of tękit.“\eva

\bvb\speakernoteb{[Lock quoth:]}%
“Shut up, Frie! Thou art Firgyn’s maiden \\
\ind and hast always been eager of men, \\
{[like]} when thou Wigh and Will—O Withrer’s wife— \\
\ind both in thy bosom didst take.”\evb\evg


\bvg\bva\mssnote{\Regius~15v/32}%
\speakernote{[Frigg kvað:]}%
„Vęitst ef \alst{i}nni \alst{ę́}tta’k \hld\ \alst{Ę́}gis hǫllum ï &
\ind \alst{B}aldri líkan \alst{b}ur &
\alst{ú}t né kvę̇mir \hld\ frȧ \alst{ȧ}sa sonum &
\ind ok vę́ri þȧ at þér \alst{v}ręiðum \alst{v}egit.“\eva

\bvb\speakernoteb{[Frie quoth:]}%
“Thou knowest, if I had within Eagre’s halls \\
\ind a boy like Balder, \\
thou camest out from the sons of the Eese, \\
\ind and wouldst, wroth man, be fought!”\evb\evg


\bvg\bva\mssnote{\Regius~16r/1}%
\Ballnote{Lock caused the death of Balder, as alluded to in \textlink{Voluspa}[31]–33 (see note there) and \textlink{Baldrsdraumar}[8]–11 and described in depth in \Gylfaginning\ 49.  It is probably this admission of guilt that seals Lock’s fate.}
\speakernote{[Loki kvað:]}%
„Ęnn vill þú, \alst{F}rigg, \hld\ at ek \alst{f}lęiri tęlja &
\ind \alst{m}ïna \alst{m}ęin-stafi: &
ek því \alst{r}éð \hld\ es þú \alst{r}íða sér⸗at &
\ind \alst{s}íðan Baldr at \alst{s}ǫlum.“\eva

\bvb\speakernoteb{[Lock quoth:]}
“Still wilt thou, Frie, that I recount more \\
\ind of my evil deeds: \\
It was my doing that thou wilt not henceforth see \\
\ind Balder ride to the halls.”\evb\evg


\bvg\bva\mssnote{\Regius~16r/3}%
\speakernote{[Fręyja kvað:]}%
„\alst{Ǿ}rr est, Loki, \hld\ es þú \alst{y}ðra tęlr &
\ind \alst{l}jóta \alst{l}ęið-stafi; &
\alst{ø}r·lǫg Frigg \hld\ hygg at \alst{ǫ}ll viti &
\ind þó’tt hǫ̇n \alst{s}jǫlf⸗gi \alst{s}ęgi.“\eva

\bvb\speakernoteb{[Frow quoth:]}
“Mad art thou, Lock, when thou dost recount \\
\ind your ugly, loathsome deeds! \\
All orlays I think that Frie might know, \\
\ind though she tell them not herself.”\evb\evg


\bvg\bva\mssnote{\Regius~16r/4}%
\Ballnote{For Frow’s promiscuity cf. \textlink{Thrymskvida}[13]/4 n.}%
\speakernote{[Loki kvað:]}%
„Þęgi þú, \alst{F}ręyja, \hld\ þik kann’k \alst{f}ull-gǫrva; &
\ind es⸗a þér \edtrans{\alst{v}amma \alst{v}ant}{free of blemishes}{\Bfootnote{Formulaic, cf. \textlink{Havamal}[22]/4: \emph{hann es⸗a vamma vanr} ‘he is not free of blemishes’.}}: &
\alst{ȧ}sa ok \alst{a}lfa, \hld\ es hér \alst{i}nni eru, &
\ind \alst{h}vęrr \alst{h}ęfir þinn \alst{h}ór vesit.“\eva

\bvb\speakernoteb{[Lock quoth:]}
“Shut up, Frow! I know thee full well— \\
\ind thou art not free of blemishes. \\
Of the Eese and Elves which are here within \\
\ind each has been thy lover!”\evb\evg


\bvg\bva\mssnote{\Regius~16r/6}%
\speakernote{[Fręyja kvað:]}%
\edtext{„\alst{F}lǫ́ ’s þér tunga, \hld\ hygg at þér \alst{f}ręmr myni &
\ind ȯ·\alst{g}ótt of \alst{g}ala;}{\lemma{Flǫ́ \dots\ gala ‘False \dots\ thee’}\Bfootnote{The language is strikingly similar to \textlink{Havamal}, particularly 29/3–4 and 116/3–4.}} &
vręiðir ’ro þér \alst{ę̇}sir \hld\ ok \alst{ǫ̇}synjur, &
\ind \edtrans{\alst{h}ryggr munt \alst{h}ęim fara}{grieved wilt thou journey home}{\Bfootnote{Frow foresees the future; Lock will come to regret his insults.}}.“\eva

\bvb\speakernoteb{[Frow quoth:]}%
“False is thy tongue, I think it further will \\
\ind sing evil [into being] for thee. \\
Wroth with thee are the Eese and the Ossens: \\
\ind grieved wilt thou journey home.”\evb\evg


\bvg\bva\mssnote{\Regius~16r/8}%
\Ballnote{Lock accuses Frow of committing incest with her brother (Free) while bewitched by the Gods.  Incest between siblings is particularly associated with the Wanes and Nearth is said to have begotten Free by his sister in st. 36 below.}%
\speakernote{Loki:}%
„Þęgi þú, \alst{F}ręyja, \hld\ þú est \alst{f}or·dę́ða &
\ind ok \alst{m}ęini blandin \alst{m}jǫk, &
síðst-u at \alst{b}rǿðr þïnum \hld\ siðu \edtrans{\alst{b}líð ręgin}{the blithe Reins}{\Bfootnote{A reverent formulaic epithet for the Gods, here used blasphemously by Lock.  It otherwise occurs in \textlink{Grimnismal}[6]/1, 38/3 and 42/1.  For another instance of Lock parodying reverent language cf. st. 11 above.}} &
\ind ok myndir \edtrans{þȧ}{then}{\Bfootnote{In the act of (incestual) coitus, an especially graphic insult.}}, \alst{F}ręyja, \alst{f}rata.“\eva

\bvb\speakernoteb{Lock [quoth]:}
“Shut up, Frow! Thou art an evil-working woman, \\
\ind and mixed with much evil, \\
since towards thy brother the blithe Reins bewitched thee \\
\ind and then wouldst thou, O Frow, fart.”\evb\evg


\bvg\bva\mssnote{\Regius~16r/10}%
\speakernote{Njǫrðr:}%
„Þat ’s \alst{v}ǫ́-lítit \hld\ þó’tt sér \alst{v}arðir \alst{v}ers fȧi, &
\ind \alst{h}ós eða \alst{h}várs; &
hitt ’s \alst{u}ndr, es \alst{ǫ̇}ss ragr \hld\ es hér \alst{i}nn of kominn &
\ind ok hęfir sá \alst{b}ǫrn of \alst{b}orit.“\eva

\bvb\speakernoteb{Nearth [quoth]:}
“It is little woe that women get themselves a man, \\
\ind a lover or whomever else, \\
yet it is a wonder when a queer os is come here within, \\
\ind and this man has borne children!”\evb\evg


\bvg\bva\mssnote{\Regius~16r/12}%
\Ballnote{For Nearth as a hostage cf. \textlink{Vafthrudnismal}[39].}%
\speakernote{Loki:}%
„\alst{Þ}ęgi þú, Njǫrðr, \hld\ \alst{þ}ú vast austr heðan &
\ind \alst{g}ísl of sęndr at \alst{g}oðum; &
\alst{H}ymis męyjar \hld\ hǫfðu þik at \alst{h}land-trogi &
\ind ok þér ï \alst{m}unn \alst{m}igu.“\eva

\bvb\speakernoteb{Lock [quoth]:}%
“Shut up, Nearth! Thou wast to the east hence \\
\ind sent as hostage for the Gods. \\
Hymer’s daughters had thee for a lant-trough \\
\ind and pissed thee in the mouth!”\evb\evg


\bvg\bva\mssnote{\Regius~16r/14}%
\speakernote{Njǫrðr:}%
„Sú es⸗umk \alst{l}íkn \hld\ es vas’k \alst{l}angt heðan &
\ind \alst{g}ísl of sęndr at \alst{g}oðum: &
þȧ ek \edtext{\alst{m}ǫg gat \hld\ þann’s \alst{m}ann-gi fíar}{\lemma{mǫg \dots\ þann’s mann-gi fíar ‘the lad whom no man hates’}\Bfootnote{Free.}}, &
\ind ok þikkir sá \alst{ȧ}sa \alst{ja}ðarr.“\eva

\bvb\speakernoteb{Nearth [quoth]:}%
“This is my relief since I was far-away hence \\
\ind sent as hostage for the Gods: \\
I thereafter begot the lad whom no man hates \\
\ind and he seems the peak of the Eese.”\evb\evg


\bvg\bva\mssnote{\Regius~16r/16}%
\speakernote{Loki:}%
„\alst{H}ę̇tt-u nú, Njǫrðr, \hld\ haf ȧ \alst{h}ófi þik; &
\ind mun’k⸗a því \alst{l}ęyna \alst{l}ęngr: &
við \alst{s}ystur þinni \hld\ gatst \alst{s}líkan mǫg, &
\ind ok es⸗a þó \alst{ȯ}nu \alst{v}err.“\eva

\bvb\speakernoteb{Lock [quoth]:}
“Cease now, Nearth; restrain thyself! \\
\ind I will no longer hide it: \\
by thy sister didst thou beget such a lad \\
\ind and naught can be expected worse.”\evb\evg


\bvg\bva\mssnote{\Regius~16r/17}%
\speakernote{Týr:}%
„Fręyr ’s \alst{b}ętstr \hld\ allra \alst{b}all-riða &
\ind \alst{ȧ}sa gǫrðum \alst{ï}; &
\alst{m}ęy né grǿtir \hld\ né \alst{m}anns konu, &
\ind ok lęysir ór \alst{h}ǫptum \alst{h}vęrn.“\eva

\bvb\speakernoteb{Tew [quoth]:}%
“Free is the best of all bold riders \\
\ind in the yards of the Eese; \\
he makes no maiden weep nor any man’s woman, \\
\ind and loosens each from his bonds!”\evb\evg


\bvg\bva\mssnote{\Regius~16r/19}%
\speakernote{Loki:}%
„\alst{Þ}ęgi þú, Týr, \hld\ \alst{þ}ú kunnir aldri⸗gi &
\ind \edtrans{bera \alst{t}ilt með \alst{t}vęim}{settle strife between two}{\Bfootnote{Uncertain. TODO.}}; &
\edtext{\alst{h}andar ennar \alst{h}ǿgri \hld\ mun’k \alst{h}innar geta &
\ind es þér slęit \alst{F}ęnrir \alst{f}rȧ.}{\lemma{handar \dots\ frȧ. ‘the right \dots\ tore.’}\Bfootnote{As told in \Gylfaginning\ TODO.}}“\eva

\bvb\speakernoteb{Lock [quoth]:}%
“Shut up, Tew! Thou couldst never \\
\ind settle strife between two; \\
the right hand will I mention next, \\
\ind which from thee Fenrer tore.”\evb\evg


\bvg\bva\mssnote{\Regius~16r/21}%
\speakernote{Týr:}%
„\alst{H}andar em’k vanr \hld\ ęn þú \alst{h}róðrs vitnis; &
\ind \alst{b}ǫl es \alst{b}ęggja þrá\emph{a}; &
\edtext{ulf⸗gi hęfir ok vel \hld\ es ï bǫndum skal}{\Bfootnote{Alliteration is absent from this otherwise metrically acceptable line and no obvious emendation suggests itself.}} &
\ind bíða \alst{r}agna \alst{r}økrs.“\eva

\bvb\speakernoteb{Tew [quoth]:}%
“A hand am I missing, but thou the Famous Wolf; \\
\ind both yearnings are a bale! \\
Nor does the Wolf have it well which in its bonds \\
\ind shall await the Twilight of the Reins.”\evb\evg


\bvg\bva\mssnote{\Regius~16r/23}%
\speakernote{Loki:}%
\Ballnote{This event is entirely unsubstantiated, and Tew’s nameless wife is not known from any other source.}%
„\alst{Þ}ęgi þú, Týr, \hld\ \alst{þ}at varð þïnni konu &
\ind at hǫ̇n átti \alst{m}ǫg við \alst{m}ér! &
\edtrans{\alst{Ǫ}ln}{ell [of wool]}{\Bfootnote{Wool, measured in ells, was often used for barter in Iceland and Norway.}} né pęnning \hld\ hafðir þęss \alst{a}ldri⸗gi &
\ind \alst{v}an-réttis, \alst{v}ę-sall.“\eva

\bvb\speakernoteb{Lock [quoth]:}%
“Shut up, Tew! It happened to thy woman \\
\ind that she had a son by me! \\
No ell [of wool] nor penny hadst thou ever for that \\
\ind injustice, O wretch!”\evb\evg


\bvg\bva\mssnote{\Regius~16r/25}%
\speakernote{Fręyr:}%
„\alst{U}lf sé’k liggja \hld\ \alst{á}ar-ósi fyr &
\ind und’s \alst{r}júfask \alst{r}ęgin; &
því munt \alst{n}ę́st, \hld\ nema \alst{n}ú þęgir, &
\ind \alst{b}undinn, \alst{b}ǫlva smiðr!“\eva

\bvb\speakernoteb{Free [quoth]:}%
“The Wolf I see lying before the river-mouth \\
\ind until the Reins are ripped; \\
there wilt thou next—unless thou now shut up— \\
\ind be bound, O smith of bales!”\evb\evg


\bvg\bva\mssnote{\Regius~16r/26}%
\speakernote{Loki:}%
\Ballnote{Lock alludes to the events of \textlink{Skirnismal}: Free gave his sword to his servant Shirner and sent him on a mission to convince Gird, Gymer’s daughter, to sleep with him.  The mission was successful and Free and Gird were united, but he did not get back his sword.  This comes back to bite Free at his fight with Surt, for which see \textlink{Voluspa}[51]/3 n.}%
„\alst{G}olli kęypta \hld\ létst \alst{G}ymis dóttur &
\ind ok \alst{s}ęldir þitt \alst{s}vá \alst{s}verð, &
en es \alst{M}úspells synir \hld\ ríða \alst{M}yrk-við yfir &
\ind \alst{v}ęitst⸗a þȧ, \alst{v}ę-sall, hvé \alst{v}egr!“\eva

\bvb\speakernoteb{Lock [quoth]:}%
“With gold thou hadst Gymer’s daughter \ken*{= Gird} bought, \\
\ind and didst so sell thy sword, \\
but when Muspell’s sons ride over Mirkwood \\
\ind knowest thou not, O wretch, how to fight!”\evb\evg


\bvg\bva\mssnote{\Regius~16r/29}%
\speakernote{Byggvir:}%
\Ballnote{Free’s servant Bewer (\emph{Byggvir}, < \emph{bygg} ‘barley’) comes to his defence.}%
„Vęitst ef \edtrans{\alst{ø}ðli}{pedigree}{\Bfootnote{Free is the son of Nearth and the legendary ancestor of the Ingling dynasty originally based at Upsal.}} \alst{ę́}tta’k \hld\ sem \edtrans{\alst{I}ngunar Fręyr}{Ingwin-Free}{\Bfootnote{A rare (probably cultic) name for the god Free, whose usual name \emph{Fręyr} originally simply means ‘Lord’.

\emph{Ingunar-} appears to be the gen. of a name \emph{*Ingunn} and is probably related to OE \emph{Ing-winas} ‘friends of Ing \ken{danes}’ (\Beowulf\ 1044, 1319) and Latin \emph{Ingaevōnēs} ‘a Germanic tribe around the North Sea’ (Tacitus, \emph{Germania} 2).
It clearly contains the same root as \emph{Yngvi} ‘Ing’, an earlier name for Free found in the C1st male given name \emph{Inguiomerus} (Tacitus, \emph{Annals} 1.60, 1.68, 2.17 et c.), the ON compound \emph{Yngvi-Fręyr} ‘Ing-Free’ (i.e. “Lord Ing”), and the OE Rune Poem (rune ᛝ, st. 22, where it is again associated with the Danes).}}, &
\ind ok \alst{s}vá \alst{s}ę́l-ligt \alst{s}etr: &
\alst{m}ęrgi smę́ra \hld\ \alst{m}ølða’k þȧ \alst{m}ęin-krǫ́ku &
\ind ok \alst{l}ęmða alla ï \alst{l}iðu.“\eva

\bvb\speakernoteb{Bewer [quoth]:}%
“Thou knowest, if I had pedigree like Ingwin-Free \\
\ind and such blessed pasture— \\
finer than bone-meal would I mill this harm-crow, \\
\ind and beat all his limbs lame!”\evb\evg


\bvg\bva\mssnote{\Regius~16r/31}%
\speakernote{Loki:}%
„Hvat ’s þat it \alst{l}itla \hld\ es þat \edtrans{\alst{l}ǫggra}{wagging its tail}{\Bfootnote{A hapax; cognate with Danish and Norwegian \emph{logre} ‘wag (one’s tail)’.}} sé’k &
\ind ok \alst{s}nap-víst \alst{s}napir? &
At \alst{ęy}rum Fręys \hld\ munt \alst{ę́} vesa &
\ind ok und \alst{k}vęrnum \alst{k}laka.“\eva

\bvb\speakernoteb{Lock [quoth]:}%
“What’s this little thing I see wagging its tail \\
\ind and snap-wisely snapping? \\
At Free’s ears wilt thou always be \\
\ind and chirping under mills!”\evb\evg


\bvg\bva\mssnote{\Regius~16r/32}%
\speakernote{[Byggvir kvað:]}%
„\edtext{\alst{B}\emph{y}ggvir}{\Afootnote{\emph{‘Beyggv\emph{ir}’} \Regius}} ek hęiti, \hld\ en mik \alst{b}ráðan kveða &
\ind \edtext{\alst{g}oð ǫll ok \alst{g}umar}{\lemma{goð \dots\ ok gumar ‘Gods and men’}\Bfootnote{This pairing also occurs in \textlink{Lokasenna}[55]/4 and \textlink{Reginsmal} 19.}}; &
því em’k \alst{h}ér \alst{h}róðugr \hld\ at drekka \alst{H}ropts męgir &
\ind \alst{a}llir \alst{ǫ}l saman.“\eva

\bvb\speakernoteb{[Bewer quoth:]}%
“Bewer I am called, and hurried call me \\
\ind all Gods and men. \\
Therefore I am honoured here when Roft’s lads \ken*{the \textsc{eese}} drink \\
\ind ale all together.”\evb\evg


\bvg\bva\mssnote{\Regius~16v/1}%
\speakernote{[Loki kvað:]}%
„\alst{Þ}ęgi þú, Byggvir, \hld\ \alst{þ}ú kunnir aldri⸗gi &
\ind dęila með \alst{m}ǫnnum \alst{m}at; &
ok þik ï \alst{f}lęts strái \hld\ \alst{f}inna né mǫ́ttu &
\ind þȧ’s \alst{v}ǫ́gu \alst{v}erar.“\eva

\bvb\speakernoteb{[Lock quoth:]}
“Shut up, Bewer! Thou couldst never \\
\ind deal out food amidst men, \\
and in the bench-straw they could not find thee \\
\ind when the warriors fought.”\evb\evg


\bvg\bva\mssnote{\Regius~16v/3}%
\speakernote{[Hęim·dallr kvað:]}%
„\alst{Ǫ}lr est, Loki \hld\ svá’t es \alst{ø}r·viti, &
\ind hví né \alst{l}ętsk⸗a þú, \alst{L}oki? &
því’t \alst{o}f-drykkja \hld\ vęldr \alst{a}lda hvęim &
\ind es sïna \alst{m}ę́lgi né \alst{m}an⸗at.“\eva

\bvb\speakernoteb{[Homedal quoth:]}%
“Drunk art thou, Lock, so that thou art out of thy wits; \\
\ind why holdest thou not back, Lock? \\
For over-drinking makes every man \\
\ind no more mind his speech.”\evb\evg


\bvg\bva\mssnote{\Regius~16v/5}%
\speakernote{[Loki kvað:]}%
„\alst{Þ}ęgi þú, Hęim·dallr, \hld\ \alst{þ}ér vas ï ár-daga &
\ind it \alst{l}jóta \edtrans{\alst{l}íf of \alst{l}agit}{life laid down}{\Bfootnote{His course of life was decreed by the Norns.  Formulaic language; cf. \textlink{Skirnismal}[13]/4.}}; &
\edtrans{\alst{ǫ}rgu}{stiff}{\Bfootnote{\emph{‘ꜹrgo’} \Regius\ is ambiguous since \emph{ꜹ} can represent both \emph{ǫ} and \emph{au}.  It is here read as a variant of \emph{ǫrðgu}, neutr. dat. sg. of \emph{ǫrðigr} ‘upright, arduous, harsh’, but it can also be read as \emph{aurgu} ‘muddy’.  The former is thought to give better sense since it specifies Homedal’s \emph{ljóta líf} ‘ugly life’;  “all that standing must hurt your back, Watchman of the Gods!”}} baki \hld\ munt \alst{ę́} vesa &
\ind ok \alst{v}aka \edtrans{\alst{v}ǫrðr goða}{Watchman of the Gods}{\Bfootnote{Formulaic epithet of Homedal, who had to guard the rainbow bridge of the Gods against their enemies.  See \textlink{Grimnismal}[13] n.}}.“\eva

\bvb\speakernoteb{[Lock quoth:]}%
“Shut up, Homedal! For \emph{thee} in days of yore \\
\ind was thy ugly life laid down. \\
With a stiff back wilt thou always be \\
\ind and waking, O Watchman of the Gods.”\evb\evg


\bvg\bva\mssnote{\Regius~16v/7}%
\speakernote{[Skaði kvað:]}%
\Ballnote{The speaker of sts. 49 and 51 is not indicated in \Regius\ and is not directly named in Lock’s answers, but is certainly Shede.  She would otherwise be the only deity mentioned in P1 without a speaking role, and Lock’s mention of the killing of Thedse (st. 50) is only effective if it relates personally to whomever he is attacking, which is only the case for Shede.}%
„\alst{L}ėtt ’s þér, Loki; \hld\ mun⸗at-tu \alst{l}ęngi svá &
\ind \alst{l}ęika \alst{l}ausum hala, &
\edtext{því’t þik ȧ \alst{h}jǫrvi skulu \hld\ ins \alst{h}rïm-kalda magar &
\ind \alst{g}ǫrnum binda \alst{g}oð.}{\lemma{því at þik ȧ hjǫrvi skulu \hld\ ins hrïm-kalda magar / gǫrnum binda goð. ‘for on a sword with thy rime-cold lad’s / guts the Gods shall bind thee’}\Bfootnote{See P8 below.}}“\eva

\bvb\speakernoteb{[Shede quoth:]}%
“Thou takest it lightly, Lock—thou wilt not so for long \\
\ind play with a loose tail, \\
for on a sword with thy rime-cold lad’s \\
\ind guts the Gods shall bind thee.”\evb\evg


\bvg\bva\mssnote{\Regius~16v/9}%
\speakernote{[Loki kvað:]}%
„Vęitst ef mik ȧ \alst{h}jǫrvi skulu \hld\ ins \alst{h}rïm-kalda magar &
\ind \alst{g}ǫrnum binda \alst{g}oð, &
\alst{f}yrstr ok øfstr \hld\ vas’k at \alst{f}jǫr-lagi &
\ind \edtrans{\alst{þ}ar’s vér ȧ \alst{Þ}jatsa \alst{þ}rifum}{where we on Thedse laid hands}{\Bfootnote{A reference to a longwinded myth told most fully in \Skaldskaparmal\ 2–4 and \Haustlong\ 2–13.  After \inx[P]{Idun} was abducted by the ettin Thedse, the Eese forced Lock to recover her, which he set out to do by flying to his home in the shape of a hawk.  When he found Idun he turned her into a nut, took her in his claws, and turned back to Osyard.  Thedse quickly spotted him, set chase in the shape of an eagle, and was soon closing the distance.  Standing in Osyard, the Eese saw the chase from afar and hurriedly threw wood shavings on the ground; just as Lock had flown over them they set fire to the shavings; the fire rose and scorched the wings of Thedse, who fell down to the ground and was killed.  Not long thereafter, Shede, Thedse’s daughter, came to Osyard to avenge her father, but the gods convinced her to settle with them, after which she married Nearth.  It is most sensible that Lock brings this myth up in order to insult Shede; cf. note to the previous st.}}.“\eva

\bvb\speakernoteb{[Lock quoth:]}%
“Thou knowest, if on a sword with my rime-cold lad’s \\
\ind guts the Gods shall bind me, \\
first and highest was I in life-taking \\
\ind when we on \inx[P]{Thedse} laid hands.”\evb\evg


\bvg\bva\mssnote{\Regius~16v/11}%
\speakernote{[Skaði kvað:]}%
„Vęitst ef \alst{f}yrstr ok øfstr \hld\ vast at \alst{f}jǫr-lagi &
\ind \alst{þ}ȧ’s ér ȧ \alst{Þ}jatsa \alst{þ}rifuð, &
\edtrans{frȧ mïnum \alst{v}éum \hld\ ok \alst{v}ǫngum}{from my wighs and wongs}{\Bfootnote{From her cultic sites; viz. her sanctuaries and sacred meadows.}} skulu &
\ind þér ę́ \alst{k}ǫld rǫ́ð \alst{k}oma.“\eva

\bvb\speakernoteb{[Shede quoth:]}%
“Thou knowest that if first and highest thou wast in life-taking \\
\ind when ye laid hands on Thedse: \\
from my wighs and wongs shall for thee \\
\ind ever cold counsels come.”\evb\evg


\bvg\bva\mssnote{\Regius~16v/12}%
\speakernote{[Loki kvað:]}%
„\alst{L}ėttari ï mǫ́lum \hld\ vast við \alst{L}auf·ęyjar son &
\ind þȧ’s létsk mér ȧ \alst{b}ęð þinn \alst{b}oðit; &
\alst{g}etit verðr oss slíks \hld\ ef vér \alst{g}ǫrva skulum &
\ind tęlja \alst{v}ǫmmin \alst{v}ǫ̇r.“\eva

\bvb\speakernoteb{[Lock quoth:]}%
“Lighter in speech wast thou with Leafie’s son \ken*{= Lock, me} \\
\ind when thou didst summon me to thy bed. \\
Such will be said of us if we clearly shall \\
\ind recount our blemishes.\evb\evg


\bpg\bpa\mssnote{\Regius~16v/15}%
Þá gekk \edtext{\emph{Sif}}{\Afootnote{replaced with a \emph{;}-like symbol \Regius}\Bfootnote{That Sib is the speaker is supported by P1 and by the fact that st. 54 mentions Thunder, her husband.}} framm ok byrlaði Loka í hrím-kalki mjǫð ok mę́lti:\epa

\bpb Then Sib stepped forth and poured for Lock mead in a \inx[C]{rime-chalice}, and spoke:\epb\epg


\bvg\bva\mssnote{\Regius~16v/12}%
\edtext{„\alst{H}ęill ves þú nú, Loki, \hld\ ok tak við \alst{h}rïm-kalki &
\ind \alst{f}ullum \alst{f}orns mjaðar}{\lemma{Hęill \dots\ mjaðar ‘Hale \dots\ mead’}\Bfootnote{Formulaic; repeated identically in \textlink{Skirnismal}[37]/1–2.}}, &
hęldr þú \edtrans{hana}{her}{\Bfootnote{Sib speaks in the third person.}} \alst{ęi}na \hld\ látir með \alst{ȧ}sa sonum &
\ind \alst{v}amma-lausa \alst{v}esa.“\eva

\bvb “Hale be thou now, Lock, and receive this rime-chalice, \\
\ind full of ancient mead, \\
but thou oughtst to let me alone among the sons of the Eese \\
\ind remain blemish-less.”\evb\evg


\bpg\bpa\mssnote{\Regius~16v/17}%
Hann tók við horni ok drakk af:\epa

\bpb He received the horn and drank thereof:\epb\epg


\bvg\bva\mssnote{\Regius~16v/18}%
„\alst{Ęi}n þú vę́rir \hld\ \alst{e}f þú svá vę́rir, &
\ind \alst{v}ǫr ok grǫm at \alst{v}eri; &
ęinn ek \alst{v}ęit, \hld\ svá’t ek \alst{v}ita þikkjumk, &
\ind \alst{h}ór ok af \edtrans{\alst{H}lór·riða}{Loride}{\Bfootnote{“The Loud Rider”; Thunder.}}, &
\ind ok vas þat sá inn \edtrans{\alst{l}ę́-vísi \alst{L}oki}{guile-wise Lock}{\Bfootnote{Formulaic, also occuring in \textlink{Hymiskvida}[37].  Cf. also \textlink{Voluspa}[35] where Lock is called \emph{lę́-gjarn} ‘guile-eager’ and \textlink{Voluspa}[17] where Lother (possibly to be identified with Lock) gives men \emph{lǫ́}, which may be an accusative form of \emph{lę́}.}}.“\eva

\bvb “Alone wouldst thou remain, \emph{if} thou hadst remained \\
\ind wary and wroth against menfolk. \\
I know one—whom I think me to know— \\
\ind adulterer behind even \inx[P]{Loride}’s back— \\
\ind and that was the guile-wise Lock!”\evb\evg


\bvg\bva\mssnote{\Regius~16v/20}%
\speakernote{[Bęyla kvað:]}%
\Ballnote{It seems that Lock’s mere mention of Thunder invites his presence, probably a suggestion of the efficacy of invoking this god.
The same thing occurs when the ettin Rungner is at the banquet of the Eese in \Skaldskaparmal\ 24: \emph{En er ǫ́sum leiddu⸗sk ofr·yrði hans, þá nefna þeir Þór.  Því nę́st kom Þórr í hǫll’ina ok hafði á lopti hamar’inn ok var all-reiðr ok spyrr, hverr því rę́ðr, er jǫtnar hund-vísir skulu þar drekka, eða hverr seldi Hrungni grið at vera í Val-hǫll eða hví Freyja skal skenkja honum sem at gildi ása.} ‘But when the Eese grew tired of his boasting words, then they say Thunder’s name.  Just thereafter Thunder came into the hall and held his hammer aloft and was all-wroth, asking whose doing it is that hundred-wise ettins shall drink there, or who gave Rungner grith to be in Walhall, or why Frow shall pour drinks for him like at the banquets of the Eese.’}%
„\edtrans{\alst{F}jǫll ǫll skjalfa}{The fells all quake}{\Bfootnote{The movement of gods, especially Thunder, is often signalled by cosmic disturbance.  Cf. \textlink{Thrymskvida}[21] n.}}, \hld\ hygg ȧ \alst{f}ǫr vesa &
\ind \alst{h}ęiman \alst{H}lór·riða; &
hann \alst{r}ę́ðr \alst{r}ó \hld\ þeim’s \alst{r}ǿgir hér &
\ind \alst{g}oð ǫll ok \alst{g}uma!“\eva

\bvb\speakernoteb{[Beal quoth:]}%
“The fells all quake—I think on his journey \\
\ind from home Loride must be. \\
He will bring to rest him who here maligns \\
\ind all the Gods and men!”\evb\evg


\bvg\bva\mssnote{\Regius~16v/21}%
\speakernote{[Loki kvað:]}%
„Þęgi þú, \alst{B}ęyla, \hld\ þú est \alst{B}yggvis kvę̇n &
\ind ok \alst{m}ęini blandin \alst{m}jǫk; &
\alst{ȯ}·kynja’n męira \hld\ kom⸗a með \alst{ȧ}sa sonum; &
\ind \edtrans{ǫll est, \alst{d}ęigja, \alst{d}ritin}{thou art all, dough-girl, dungy}{\Bfootnote{\emph{dęigja} ‘dough-girl’ is a derivative of \emph{dęigr} ‘dough’ and refers to a young maid at a farm who carries out tasks like kneading dough, milking the cows, and carrying water.  Lock insults her; she’s still covered with cow dung.}}.“\eva

\bvb\speakernoteb{[Lock quoth:]}%
“Shut up, Beal! Thou art Bewer’s wife, \\
\ind and mixed with much evil. \\
A greater disgrace came not amidst the sons of the Eese; \\
\ind thou art all, dough-girl, dungy!”\evb\evg


\bpg\bpa\mssnote{\Regius~16v/23}%
Þá kom Þórr at ok kvað:\epa

\bpb Then Thunder arrived and quoth:\epb\epg


\bvg\bva\mssnote{\Regius~16v/24}%
„\alst{Þ}ęgi þú, rǫg vę́ttr, \hld\ þér skal mïnn \edtrans{\alst{þ}rúð-hamarr}{thrith-hammer}{\Bfootnote{“Strength-hammer”, \emph{þrúðr} ‘thrith’ being an obsolete word for strength used only in connection with Thunder or ettins.  \emph{Þrúðr} ‘\inx[P]{Thrith}’ is also the name of Thunder’s daughter.}}, &
\ind \alst{M}jǫllnir, \alst{m}ál fyr·nema! &
\alst{H}ęrða klett \hld\ drep’k þér \alst{h}alsi af, &
\ind ok verðr þȧ þïnu \alst{f}jǫrvi of \alst{f}arit.“\eva

\bvb “Shut up, thou \inx[C]{queer} wight! Thee shall my thrith-hammer \\
\ind Millner, deprive of speech! \\
The rock of thy shoulders \ken{head} I will cut from thy neck, \\
\ind and then is thy life destroyed!”\evb\evg


\bvg\bva\mssnote{\Regius~16v/26}%
\speakernote{[Loki kvað:]}%
„\alst{Ja}rðar burr \hld\ es hér nú \alst{i}nn kominn; &
\ind hví \alst{þ}rasir þú svá, \alst{Þ}ȯrr? &
En þȧ þorir \alst{ę}kki \hld\ \edtext{es skalt við \alst{u}lf’inn vega &
\ind ok \alst{s}velgr hann allan \alst{S}ig-fǫður}{\lemma{es skalt við ulf’inn vega / ok svelgr hann allan Sig-fǫður ‘when thou shalt fight the Wolf / and he swallows Syefather \name{= Weden} whole.’}\Bfootnote{A reference to the Rakes of the Reins, where \inx[P]{Weden} is slain by the Wolf.  Thunder, meanwhile, dies while slaying the Wyrm; see \textlink{Voluspa}[51]–53, \textlink{Vafthrudnismal}[53].}}.“\eva

\bvb\speakernoteb{[Lock quoth:]}%
“Earth’s Son is here now come inside, \\
\ind why thrashest thou so, Thunder? \\
But thou wilt nowise dare when thou shalt fight the Wolf \\
\ind and he swallows Syefather \name{= Weden} whole.”\evb\evg


\bvg\bva\mssnote{\Regius~16v/28}%
\speakernote{[Þȯrr kvað:]}%
„\alst{Þ}ęgi þú, rǫg vę́ttr, \hld\ þér skal mïnn \alst{þ}rúð-hamarr, &
\ind \alst{M}jǫllnir, \alst{m}ál fyr·nema! &
\alst{U}pp ek þér verp \hld\ ok ȧ \edtrans{\alst{au}str-vega}{eastern ways}{\Bfootnote{The desolate eastern wastelands where Thunder hunts ettins, troll-women and other unnatural creatures.}}, &
\ind \alst{s}íðan þik mann-gi \alst{s}ér.“\eva

\bvb\speakernoteb{[Thunder quoth:]}%
“Shut up, thou queer wight! Thee shall my thrith-hammer \\
\ind Millner, deprive of speech! \\
I will throw thee up and onto the eastern ways, \\
\ind whereafter no man will see thee!”\evb\evg


\bvg\bva\mssnote{\Regius~16v/29}%
\speakernote{[Loki kvað:]}%
„\alst{Au}str-fǫrum þïnum \hld\ skalt \alst{a}ldri⸗gi &
\ind \edtrans{\alst{s}ęgja \alst{s}ęggjum frȧ}{speak to the youth}{\Bfootnote{Lock here borrows from Frie’s use of this expression in st. 25 above.}} &
síðst \edtrans{ï \alst{h}anska þumlungi \hld\ \alst{h}núkðir þú}{since into a glove’s thumb thou crawledest}{\Bfootnote{Sts. 60 and 62 allude to Thunder’s encounter with the ettin Shrimer, which is retold in \Gylfaginning\ 45.  A closely related narrative is mentioned in \textlink{Harbardsljod}[26], although the ettin is there called Feller.}}, Ęin-hęri, &
\ind ok \alst{þ}ȯttisk⸗a \alst{þ}ȧ \alst{Þ}ȯrr vesa!“\eva

\bvb\speakernoteb{[Lock quoth:]}%
“From thy eastern journeys shalt thou never \\
\ind speak to the youth, \\
since into a glove’s thumb thou crawledest, Oneharrier, \\
\ind and didst not seem to be Thunder then!”\evb\evg


\bvg\bva\mssnote{\Regius~16v/31}%
\speakernote{[Þȯrr kvað:]}%
„\alst{Þ}ęgi þú, rǫg vę́ttr, \hld\ þér skal mïnn \alst{þ}rúð-hamarr, &
\ind \alst{M}jǫllnir, \alst{m}ál fyr·nema! &
\alst{h}ęndi inni \alst{h}ǿgri \hld\ drep’k þik \alst{H}rungnis bana, &
\ind svá’t þér \alst{b}rotnar \alst{b}ęina hvat.“\eva

\bvb\speakernoteb{[Thunder quoth:]}%
“Shut up, thou queer wight! Thee shall my thrith-hammer \\
\ind Millner, deprive of speech! \\
With my right hand I will beat thee with Rungner’s bane \ken*{= Millner} \\
\ind so that every bone in thee breaks!”\evb\evg


\bvg\bva\mssnote{\Regius~16v/32}%
\speakernote{[Loki kvað:]}%
„\alst{L}ifa ę́tla’k mér \hld\ \alst{l}angan aldr &
\ind þó’tt \alst{h}ǿtir \alst{h}amri mér; &
\alst{sk}arpar ȧlar \hld\ þȯttu þér \alst{Sk}rymis vesa &
\ind ok máttir⸗a þȧ \alst{n}ęsti \alst{n}áa &
\ind ok svaltsk þȧ \alst{h}ungri \alst{h}ęill.“\eva

\bvb\speakernoteb{[Lock quoth:]}%
“I intend for myself to live a long life \\
\ind although thou mighst threaten me with the hammer. \\
Sharp seemed Shrimer’s straps to thee, \\
\ind and then couldst thou not reach thy provisions, \\
\ind and then wast thou dying, healthy, of hunger!”\evb\evg


\bvg\bva\mssnote{\Regius~17r/1}%
\speakernote{[Þȯrr kvað:]}%
„\alst{Þ}ęgi þú, rǫg vę́ttr, \hld\ þér skal mïnn \alst{þ}rúð-hamarr, &
\ind \alst{M}jǫllnir, \alst{m}ál fyr·nema! &
\alst{H}rungnis bani \hld\ mun þér ï \alst{h}ęl koma &
\ind fyr \alst{N}á-grindr \alst{n}eðan.“\eva

\bvb\speakernoteb{[Thunder quoth:]}%
“Shut up, thou queer wight! Thee shall my thrith-hammer \\
\ind Millner, deprive of speech! \\
Rungner’s bane will take thee to hell, \\
\ind down beneath Neegrind!”\evb\evg


\bvg\bva\mssnote{\Regius~17r/2}%
\speakernote{[Loki kvað:]}%
„Kvað’k fyr \alst{ǫ̇}sum, \hld\ kvað’k fyr \alst{ȧ}sa sonum, &
\ind þat’s mik \alst{h}vatti \alst{h}ugr, &
en fyr þér \alst{ęi}num \hld\ mun’k \alst{ú}t ganga &
\ind því’t ek \alst{v}ęit at þú \alst{v}egr.\eva

\bvb\speakernoteb{[Lock quoth:]}
“I spoke before the Eese; I spoke before the sons of the Eese \\
\ind whatever my heart did goad me, \\
but for thee alone will I walk out \\
\ind for I know that thou dost strike.\evb\evg


\bvg\bva\mssnote{\Regius~17r/4}%
\alst{Ǫ}l gørðir þú, \alst{Ę́}gir, \hld\ en þú \alst{a}ldri munt &
\ind \alst{s}íðan \alst{s}umbl of gøra; &
\alst{ęi}ga þïn \alst{ǫ}ll, \hld\ es hér \alst{i}nni es, &
\ind \alst{l}ęiki yfir \alst{l}ogi &
\ind ok \alst{b}renni þér ȧ \alst{b}aki.“\eva

\bvb Ale hast thou made, Eagre, but thou wilt never \\
\ind henceforth make a simble! \\
All thy estate which is here within— \\
\ind let flame play over it \\
\ind and burn thee in the back!”\evb\evg


\section{From Lock (\emph{Frá Loka})}

\bpg\bpa\mssnote{\Regius~17r/6}%
\Ballnote{The myth of the binding of Lock is alluded to in \textlink{Voluspa}[34] and told at length in \Gylfaginning\ 50, which does not entirely agree with \textlink{Lokasenna} P8.  According to \Gylfaginning, the Eese captured two of Lock’s sons, Wonnel and “Nare or Narve”. They turned Wonnel into a wolf (\emph{vargr}, which also means ‘outlaw’) and forced him to tear his brother Narve apart.  They took Narve’s intestines used them to bind Lock on top of three sharp stones with one digging into his shoulder-blades, the other into his loins, the third into his houghs. The intestines hardened into iron and Lock was bound fast.  Since the author of \Gylfaginning\ knew \textlink{Voluspa}, it is possible that he combined a text similar to P8 with \textlink{Voluspa}[H1], interpreting \emph{Vȧla víg-bǫnd} as ‘Wonnel’s war-bonds’.  Wonnel is otherwise only known as the son of Weden, and there is no reason as to why he could not have bound Lock.}%
En eptir þetta falst Loki í Fránangrs-forsi í lax líki. Þar tóku ę́sir hann. Hann var bundinn með þǫrmum sonar Nara; en Narfi, sonr hans, varð at vargi. Skaði tók eitr-orm ok festi upp yfir and-lit Loka; draup þar ór eitr. Sigyn, kona Loka, sat þar ok helt munn-laug undir eitr’it, en er munn-laugin var full bar hón út eitr’it, en meðan draup eitr’it á Loka. Þá kipptist hann svá hart við, at þaðan af skalf jǫrð ǫll; þat eru nú kallaðir land-skjálftar.\epa

\bpb {\huge B}\textsc{ut after this} Lock hid himself in Freenanger’s Force in the likeness of a salmon.  There the Eese took him.  He was bound with the intestines of his son Nare, but his son Narve was made a wolf/outlaw.  Shede took a venomous serpent and fastened it up above Lock’s face; from it ran venom.  Syein, Lock’s wife, sat there and held a basin under the venom and when the basin was full she carried out the venom, but meanwhile the venom ran onto Lock.  Then he struggled so hard that thereof all the earth quaked; such are now called earth-quakes.\epb\epg

\section{Stanza from \Gylfaginning}

\bvg\bva[G1]%
\Ballnote{In \Gylfaginning\ 20 this stanza is cited as proof of Frie’s foresight regarding the orlays of men.  It is introduced by the words \emph{svá sem hér er sagt, at Óðinn mę́lti sjalfr við þann ás, er Loki heitir} ‘just as it is said here, that Weden himself spoke to that Os who is called Lock’. — The text looks like an amalgamation of several \textlink{Lokasenna} stanzas (which is why it has been placed here, rather than under \textlink{EddicFragments}); l. 1 corresponds to st. 21/1 (spoken by Weden), l. 2 to st. 47/2 (spoken by Homedal), and ll. 3–4 to st. 29/3–4 (spoken by Frow).  It is possible that it derives from an alternate version of \textlink{Lokasenna}, but it could also have been formed due to Snorre’s misremembering the rest of the stanza after the first line, which is also attributed to Weden in st. 21.}%
„\alst{Ǿ}rr est, Loki, \hld\ ok \alst{ø}r·viti, &
\ind hví né \alst{l}ęt⸗sk⸗a þú, \alst{L}oki? &
\alst{ø}r·lǫg Frigg \hld\ hygg at \alst{ǫ}ll viti &
\ind þó’tt hǫ̇n \alst{s}jǫlf⸗gi \alst{s}ęgi.“\eva

\bvb “Mad art thou, Lock, and out of thy wits, \\
\ind why holdest thou not back, O Lock? \\
All orlays I think that Frie might know, \\
\ind though she tell them not herself.”\evb\evg

\sectionline
