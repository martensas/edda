\bookStart{Lay of Wayland}[Vǫlundar kviða]
\setBookCode{Volundarkvida}

\begin{flushright}%
\textbf{Dating} \parencite{Sapp2022}: C10th (0.428)–early C11th (0.475)

\textbf{Meter:} \Fornyrdislag%
\end{flushright}%

\section{Introduction}

The \textbf{Lay of Wayland} (\textlink{Volundarkvida}) is a psychologically complex, well wrought poem.

\subsection{Preservation}

\Volundarkvida\ only survives in full in \Regius, but the beginning of the foreword is found on the very last page of \AM\ which shows that it was also found there.  Although technically a heroic poem, it is part of the mythological section in \Regius, for which reason it is placed under Norse Mythic Poetry here.

\subsection{Content}

\textlink{Volundarkvida} is a narrative poem telling the story of Wayland the Smith (ON \emph{Vǫlundr}, OE \emph{Wéland} or \emph{Wélund}, MHG \emph{*Welent}).  Wayland was one of the most popular figures in early mediæval Germanic legend, and independent versions of his tale are found in Germany, England, and Iceland.

In his archetypal form, Wayland is an exceptionally talented smith who is taken captive and hamstrung by the greedy tyrant king Nithad (ON \emph{Níð·uðr}, OE \emph{Níþhad}, MHG \emph{*Nídung}).  Nithad forces him to make jewels for him and his family, but Wayland plots a cruel revenge against the king’s household: he murders his two sons and rapes his daughter, Beadhild (ON \emph{Bǫðv·ildr}, OE \emph{Beaduhild}, MHG \emph{*Botil}), making her pregnant.  At last, he makes a flight-suit out of feathers, typically (though not in \Volundarkvida) with the help of his brother Eyel (ON \emph{Ęgill}, OE \emph{Ægili}).  Having regained his mobility through ingenuity he makes his escape, leaving the grieving household behind.

There is a strong parallelism throughout the narrative, which is developed most strongly in the present poem (but cf. the treatment in the OE \textlink{Deor}[1]–2).  Wayland gets his revenge on the whole royal household by doing to them what was done to him.  Like he was taken from waiting for his wife, so he murders Nithad’s two young sons and ends his male lineage.  Like he was made powerless, he defangs Nithad’s \emph{kunnig kvǫ̇n} ‘cunning wife’ by reducing her once powerful counsels to cold words.  Like he lost his wife’s golden ring, so he rapes Beadhild, depriving her of her maidenhood and value as a marriage prospect.
In this way the royal household is reduced to the very same state of abject powerlessness that Wayland himself experienced, something signalled by the repetition of the adjective \emph{vilja-lauss} ‘powerless’.  In st. 12 it describes Wayland after he wakes in shackles, but in st. 31 Nithad uses it to refer to his own mental state after the deaths of his sons.  This sense of hopelessness concludes the poem in Beadhild’s haunting words: “In no way did I know how to withstand him; in no way could I withstand him.”

\subsubsection{Parallels}

As mentioned above, \Volundarkvida\ is not the only retelling of Wayland’s story.  The other relevant retellings are the 9th century OE \textlink{Deor} and the 13th century Norse \ThidreksSaga; the former represents the older, Anglo-Norse version, while the latter represents the younger Low German version.

Wayland is alluded to in numerous other sources, from which we learn a little about the fate of Beadhild.  After Wayland abandoned her she gave birth to a son, Woody (OE \emph{Wudga}, \ThidreksSaga\ \emph{Viðga}, in Danish ballads \emph{Vidrik Verlandsøn}).  He went on to become a great hero, and in the later heroic ballads far eclipses his father in fame.  His birth seems heavily foreshadowed by Wayland forcing Nithad to swear an oath on the unborn child in st. 33, but he is nowhere directly mentioned in the poem, probably for artistic reasons: dwelling on the tragic image of Beadhild is certainly a powerful ending.

\begin{figure}[b]
\centering
\includegraphics[width=\textwidth]{Franks-casket}
\caption{The front panel of the Franks casket.  Early Mediæval, ca. 725 CE.  © John W. Schulze. \href{https://creativecommons.org/licenses/by/2.0/deed.en}{CC BY 2.0}.  \url{https://commons.wikimedia.org/wiki/File:Franks_Casket_front.jpg}}
\label{fig:franks}
\end{figure}

\textlink{Deor} shortly relates the lives of five figures from Germanic legend.  Stanzas 1–2 deal with Wayland and Beadhild, and the underlying story is identical to that of \Volundarkvida: Wayland is imprisoned by Nithad, then kills his sons and rapes Beadhild.  Like \Volundarkvida, \textlink{Deor} particularly emphasises the powerlessness felt by Wayland and how his revenge in turn inflicted the same on Beadhild.

An Anglo-Saxon pictoral depiction of the story is found on the early C8th Franks casket, which shows Wayland drugging Beadhild, Nithad’s decapitated sons, and his brother Eyel collecting swan-feathers for the flight-suit (fig.~\ref{fig:franks}).  The fact that the Wayland-Story is as prominent as the depiction of the Adoration of the Magi is a testament to the weight with which it was regarded in early mediæval England.

The other full retelling of Wayland’ life is found in the much younger \ThidreksSaga.  While written in Old Norse, it is clear from the linguistic forms of the proper names and content in it that it is based on Low German sources, probably heroic ballads.  Thus the native Norse forms \emph{Vǫlundr} and \emph{Níð·uðr} are replaced with the Low German \emph{Velent} [\emph{sic}] and \emph{Níðungr}.  The 13th century author makes an interesting note which reveals that native Norse stories about Wayland were not yet entirely extinct in his time: he mentions “\emph{Velent}, the excellent smith, whom Warrings (\emph{væringjar} ‘Varangians’) call \emph{Vǫlundr}. [...] \emph{Velent} is so famous in the whole northern hemisphere that all men seem to praise his workmanship in such a way that the craftsman of any piece which is crafted better than others is called a \emph{Vǫlundr} with regards to craftmanship.”

It is not merely in dialect that \ThidreksSaga\ departs from the older retellings; far more important is the difference in tone.  The psychological complexity and tragedy of \Volundarkvida\ and \textlink{Deor} are entirely gone, and Wayland is no longer a mysterious supernatural wild man, but a chivalrous knight who can escape from any peril through his ingenuity and skill.  He is not kidnapped out of Nithad’s greed nor hamstrung on the suspicions of his cruel wife, but is instead a loyal servant of Nithad’s, banished from the kingdom after defending himself against the king’s corrupt steward and hamstrung after being caught attempting to poison the king’s food.

Yet the biggest change in the narrative is without doubt that the personality of Beadhild is entirely expunged.  Far from the Anglo-Norse version she has here become the anonymous “king’s daughter”, a nameless \emph{jungfrú} ‘maiden’ (a borrowing from Low German) whom Wayland seduces by means of a vulgar sex joke (“Now the king’s daughter comes into the smithy and asks Wayland to mend the ring, but he says that he firsts wants to ‘forge’ something else.”) and who quickly falls in love with him.

Likewise gone is the person of Nithad’s cunning wife, and the murder of his sons no longer ends his lineage, since he has another son who survives him and takes over the kingdom. Wayland still flies away laughing after telling Nithad what he has done, but only four years later reconciliates with Nithad’s living son, retrieves Beadhild and their three-year old son and lives a long and happy life as a famous smith.

By the time of the \ThidreksSaga\ the old story of Wayland had been heavily bowlderized, a tragic victim of chivalric sensibilities.  This younger version does not have any high literary value, but is of course still of interest since it shows the wide reception and variation of the narrative.

\subsubsection{Origins}

If we are to search for the origins of the Wayland-story we can hardly look in Scandinavia.  As we have seen above the story was particularly popular in Anglo-Saxon England, and it is probably hence that it made its way to Scandinavia.

\textlink{Volundarkvida} does indeed show signs of Old English influence in its vocabulary (1/2: \emph{al-vitr}, 11/2: \emph{ljóði}, 36/3: \emph{aukin}), and the proper names of its characters (particularly Beadhild and Thankred, the thrall) do not seem North Germanic.  Of course, it cannot be a direct translation from OE.  The impossibility of that is shown by the use of formulaic Norse lines like 16/1–2, 30/1–2 and by lines requiring North Germanic sound changes to alliterate like 14/3 \emph{ȯra aura · ï Ulf-dǫlum} ‘our ounces in the Wolfdales’ (reconstructed OE: \emph{*u̇re éaras · in Wulf-dalum}, note that OE \emph{*éaras} ‘ounces’ is not even a real word!)

Even older versions of the story must needs remain forever forlorn.  It is clear from their names that Wayland and Nithad are fairy-tale characters, not historical figures.  Wayland’s name means ‘Crafty Hand’ (< PGmc \emph{*wéla-handuz}, \cite{Brate1908}) and Nithad’s means ‘Shameful Fighter’ (< PGmc \emph{*níþa-haduz}.

Comparatively there are striking similarities between Wayland and the Greek smith \textgreek{Δαίδαλος}.  Both are legendary inventors trapped on an island by an evil tyrant for whom they have been forced to work, and both make their escape together with a near relation (Germanic Eyel, Greek \textgreek{Ἴκαρος}) by means of a flight-suit.  The difficulty with these parallels is that the Wayland-story, anchored as it is in Germanic legend, is attested long before knowledge of Greek mythology can have existed in Northern Europe.  We may therefore be dealing with much older folklore, perhaps even going back to the Bronze Age (as the image of the Labyrinth definitely does) or the Neolithic European Farmers, although such connections cannot but remain speculation.

\newpage

\section{Text}

\subsection{From Wayland (\emph{Frá Vǫlundi})}

\bpg\bpa\mssnote{\Regius~18r/4, \AM~6v/26}%
Níð·uðr hét konungr í Sví-þjóð.
Hann átti tvá sonu ok eina dóttur; \edtrans{hón hét}{she was called}{\Afootnote{ok hét hón ‘and she was called’ \AM}} Bǫðv·ildr.
Brǿðr \edtrans{vǫ́ru}{were}{\Afootnote{so \AM; om. \Regius}} þrír, synir Finna konungs. Hét einn Slag·fiðr, annarr Egill, þriði Vǫlundr.
Þeir skriðu ok veiddu dýr. Þeir kvǫ́mu í Ulf-dali ok \edtrans{gerðu}{made}{\Afootnote{after this word the rest of the ms. ends for loss of following foll. \AM}} sér þar hús.
Þar er vatn, er heitir Úlfsjár.
Snemma of morgin fundu þeir á vats-strǫndu konur þrjár, ok spunnu lín. Þar vǫ́ru hjá þeim álftar-hamir þeira; þat vǫ́ru val-kyrjur.
Þar vǫ́ru tvę́r dǿtr Hlǫð·vés konungs: Hlað·guðr svan-hvít ok Her·vǫr al·vitr. In þriðja var Ǫl·rún \edtext{Kjárs dóttir af Val-landi}{\lemma{Kjárs \dots\ af Val-landi ‘Coser of Walland’}\Bfootnote{I.e. ‘Caesar of Rome/France’.  Of course no specific historical emperor can be identified by this name, but rather it is an anachronistic reflex of the position, a title which has been misunderstood as a proper name.  Cf. \textlink{Widsith}[4]/3, where Coser (OE \emph{Câsere}) is said to rule the Greeks.   For the form see st. 15/4 n., for other occurrences see Index.}}.
Þeir hǫfðu þę́r heim til skála með sér. Fekk Egill Ǫl·rúnar, en Slag·fiðr svan-hvítrar, en Vǫlundr al·vitrar.
Þau bjuggu sjau vetr. Þá flugu þę́r at vitja víga ok kvǫ́mu eigi aptr.
Þá skreið Egill at leita Ǫl·rúnar, en Slag·fiðr leitaði svan-hvítrar, en Vǫlundr sat í Ulf-dǫlum.
Hann var hagastr maðr, svá at menn viti í fornum sǫgum.
Níð·uðr konungr lét hann hǫndum taka, svá sem hér er um kveðit:\epa

\bpb Nithad was the name of a king in Sweden.
He had two sons and one daughter; she was called Beadhild.
Three brothers were there, the sons of a king of the Finns.  One was called Slayfinn, the other Eyel, the third Wayland.
They fared on skis and hunted wild beasts. They came into the Wolfdales and made for themselves houses there.
There is a lake there which is called the Wolfsea.
Early in the morning they found by the lake-shore three women and they were spinning linen. There beside were them their swan-\inx[C]{hame}[hames]; those were Walkirries.
There were two daughters of king Ludwigh: Ladguth Swanwhite and Harware Elwight.  The third was Alerune, daughter of \inx[P]{Coser} of \inx[G]{Walland}.
The men took the women to their halls with them.  Eyel got Alerune and Slayfinn Swanwhite and Wayland the Elwight.
The couples lived there for seven winters; then the women left to attend battles, and did not come back.
Then Eyel fared on skis to search for Alerune and Slayfinn searched for Swanwhite, but Wayland stayed in the Wolfdales.
He was the most skilled craftsman whom men know of in the ancient saws. King Nithad had him taken captive, about which it is here sung:\epb\epg

\sectionline

\subsection{The Lay of Wayland}

\bvg\bva\mssnote{\Regius~18r/19}%
\alst{M}ęyjar flugu sunnan \hld\ \edtrans{\alst{M}yrk-við}{Mirkwood}{\Bfootnote{A great border forest, surely referenced for its association with the war-ravaged lands of the Gots and Huns; a natural environment for \inx[G]{Walkirries}.}} ï gǫgnum &
\edtrans{\alst{a}l-vitr}{elwights}{\Bfootnote{“Strange beings, foreign wights”, from Proto-Norse \emph{*alja-wihtiz} or borrowed from OE \emph{æl-wihta} pl. ‘strange creatures, monsters’.}} \alst{u}ngar, \hld\ \edtrans{\alst{ø}r·lǫg drýgja}{fulfill orlay}{\Bfootnote{That is, to fulfill their preordained destinies, viz. to attend battles as Walkirries and choose which men to take away (cf. P1 and st. 3).
\textcite[103]{MCR2005}, \textcite[319]{LaFargeGlossary} and others see these words as a sign of English influence and translate \emph{drýgja ør·lǫg} as ‘engage in war’, considering \emph{ør·lǫg} a semantic borrowing from OE \emph{or·lęge} ‘war, strife’ (cf. Dutch \emph{oorlog} ‘war’).  This is unnecessary.  ON \emph{ør·lǫg} otherwise only means ‘fate, destiny’, as may OE \emph{or·lęg} as seen by an equivalent phrase found in l. 29 of a poem on the Christian Doomsday (TODO?), where a man going to Hell for his sins \emph{þǫnne â tó ealdre \hld\ or·lęg dreógeð} ‘then for ever and ever suffers his orlay’.}}; &
þę́r ȧ \alst{s}ę́var-strǫnd \hld\ \alst{s}ęttu⸗sk at hvíla⸗sk, &
\alst{d}rósir suð-rǿnar \hld\ \alst{d}ýrt lín spunnu.\eva

\bvb Maidens flew from the south through \inx[L]{Mirkwood} \\
—young elwights—to fulfill \inx[C]{orlay}. \\
They on the lake-shore set down to rest; \\
the southern ladies span costly linen.\evb\evg


\bvg\bva\mssnote{\Regius~18r/21}%
\alst{Ęi}n nam þęira \hld\ \alst{Ę}gil at vęrja &
\alst{f}ǫgr mę́r \alst{f}ira \hld\ \alst{f}aðmi ljósum; &
ǫnnur vas \alst{S}vanhvít, \hld\ \alst{s}van-fjaðrar dró, &
\edtext{[...]}{\Bfootnote{A line mentioning Slayfinn has probably been lost here.}} &
en in \alst{þ}riðja \hld\ \alst{þ}ęira systir &
varði \edtrans{\alst{h}vítan}{white}{\Bfootnote{Pale skin being a sign of noble ancestry; cf. 17/3.}} \hld\ \alst{h}als Vǫlundar.\eva

\bvb One of them took to embrace Eyel \\
—the fair maiden among men—in her pale bosom. \\
Second was Swanwhite; her swan-feathers she rustled, \\
{[...]} \\
And the third sister among them \\
embraced the white throat of Wayland.\evb\evg


\bvg\bva\mssnote{\Regius~18r/24}%
\alst{S}ǫ́tu \alst{s}íðan \hld\ \alst{s}jau vetr at þat, &
en hinn \alst{á}tta \hld\ \alst{a}llan þrǫ́ðu, &
en hinn \alst{n}íunda \hld\ \alst{n}auðr of skilði, &
\alst{m}ęyjar fẏstu⸗sk \hld\ ȧ \alst{m}yrkvan við, &
\alst{a}l-vitr \alst{u}ngar \hld\ \alst{ø}r·lǫg drýgja.\eva

\bvb They stayed then seven winters after that, \\
and all the eighth they yearned, \\
and the ninth did need divorce them. \\
The maidens longed for the Mirky Wood: \\
the young elwights, to fulfill orlay.\evb\evg


\bvg\bva\mssnote{\Regius~18r/26}%
Kom þar af \alst{v}ęiði \hld\ \alst{v}eðr-ęygr skyti &
\edtext{\emph{Vǫlundr \alst{l}íðandi \hld\ of \alst{l}angan veg,}}{\lemma{Vǫlundr \dots\ veg ‘Wayland \dots\ way’}\Afootnote{emend. based on st. 9/3–4; om. \Regius}} &
\alst{S}lag·fiðr ok Ęgill, \hld\ \alst{s}ali fundu auða, &
gingu \alst{ú}t ok \alst{i}nn \hld\ ok \alst{u}mb sǫ́u⸗sk.\eva

\bvb Came there from the hunt the stormy-eyed shooter: \\
Wayland passing over a long way. \\
Slayfinn and Eyel found the halls deserted; \\
they walked out and in, and looked about.\evb\evg


\bvg\bva\mssnote{\Regius~18r/27}%
\alst{Au}str skręið \alst{Ę}gill \hld\ at \alst{Ǫ}l·ru̇nu, &
en \alst{s}uðr \alst{S}lag·fiðr \hld\ at \alst{S}van-hvítu, &
en \alst{ęi}nn Vǫlundr \hld\ sat ï \alst{U}lf-dǫlum.\eva

\bvb East skied Eyel after Alerune, \\
and south Slayfinn after Swanwhite, \\
and alone Wayland stayed in the Wolfdales.\evb\evg


\bvg\bva\mssnote{\Regius~18r/29}%
Hann \edtrans{sló \alst{g}oll rautt \hld\ við \alst{g}im fastan}{struck red gold against fastened gemstone}{\Bfootnote{A description of the \emph{cloisonée} technique wherein red garnet is encased in thin gold plates.  Finds featuring this technique are very common in the period between ca. 300–700 CE, but almost non-existent afterwards \parencite[33–37]{Nerman1931}.  As \Volundarkvida\ was most likely composed one or two centuries after 700 this reflects a memory of the material culture of an earlier period.  For a similar instance cf. \textlink{HelgakvidaHjorvardssonar}[9].}}, &
\alst{l}ukði alla \hld\ \edtrans{\alst{l}in\emph{n}-bauga}{serpent-bighs}{\Afootnote{emend.; \emph{‘lind bauga’} \Regius}\Bfootnote{It is unclear whether this word refers to rings actually fitted with snake-heads or is merely a poetic description of twisted rings or spirals.  Archeological examples of the former include the so-called “snake-head rings” (German \emph{Schlangenkopfringe}, Swedish \emph{ormhuvudringar}) from the early Migration Period \parencite[38]{Nerman1931}, and the snake- or dragon-shaped armlet from the Wiking Age found in a hoard in Undrom, Ångermanland, northern Sweden (item number 108822 HST; \url{https://samlingar.shm.se/object/5C5658C4-0813-4DFF-947F-E5E4C4BAB965}).}} vel; &
\alst{s}vá bęið hann \hld\ \alst{s}innar ljóssar &
\alst{k}vȧnar, ef hǫ̇num \hld\ \alst{k}oma gęrði.\eva

\bvb He struck red gold against fastened gemstone; \\
he enclosed all the serpent-\inx[C]{bigh}[bighs] well; \\
so he awaited his own bright wife, \\
if to him she might come.\evb\evg


\bvg\bva\mssnote{\Regius~18r/31}%
Þat spyrr \alst{N}íð·uðr, \hld\ \edtrans{\alst{N}íara}{the Nears}{\Bfootnote{An obscure tribe, perhaps the residents of \emph{Närke}, an ancient province of Sweden. See Index.}} dróttinn, &
at \alst{ęi}nn Vǫlundr \hld\ sat ï \alst{U}lf-dǫlum; &
\alst{n}ǫ́ttum fóru sęggir, \hld\ \edtrans{\alst{n}ęglðar vǫ́ru brynjur}{nailed were their byrnies}{\Bfootnote{The “byrnies” here are some kind of plate armour.}}, &
\alst{sk}ildir bliku þęira \hld\ við inn \alst{sk}arða mána.\eva

\bvb That learns Nithad, lord of the \inx[G]{Nears}, \\
that alone Wayland was staying in the Wolfdales. \\
At night journeyed warriors—nailed were their byrnies— \\
their shields gleamed by the sickle moon.\evb\evg


\bvg\bva\mssnote{\Regius~18r/33}%
Stigu ór \alst{s}ǫðlum \hld\ at \alst{s}alar gafli, &
\edtext{gingu \alst{i}nn þaðan \hld\ \alst{ę}nd-langan sal}{\lemma{gingu \dots\ sal ‘went \dots\ hall’}\Bfootnote{The line appears to be a variant of the formulaic \emph{hann/hǫ̇n inn of gekk \hld\ ęnd-langan sal} ‘he/she went inside the endlong hall’ which also occurs in below \textlink{Volundarkvida}[16]/2 and 30/2 and in \textlink{Oddrunargratr}[3]/3.  For \emph{ęnd-langan sal} ‘endlong hall’ in general cf. \textlink{Thrymskvida}[27]/2 n.}}, &
sǫ́u ȧ \alst{b}ast \hld\ \alst{b}auga dręgna, &
\alst{s}jau hundruð allra, \hld\ es sá \alst{s}ęggr átti.\eva

\bvb They stepped off their saddles by the hall’s gables, \\
went thence inside the endlong hall. \\
They saw on a bast-rope bighs drawn up, \\
seven hundred in all, which that man owned.\evb\evg


\bvg\bva\mssnote{\Regius~18v/2}%
Ok þęir \alst{a}f tóku \hld\ ok þęir \alst{ȧ} létu &
\edtrans{fyr \alst{ęi}nn \alst{ú}tan, \hld\ es \alst{a}f létu}{save for one, which off they slid}{\Bfootnote{This bigh is probably the one mentioned in sts. 17 and 26, since Beadhild has it already when Wayland is brought back after being captured. It may have been kept for its particular beauty. \textcite{FinnurEdda}\ writes (\emph{my translation from the Danish}): “The ring which Nithad kept must have had special properties, and distinguished itself before others.  There is no doubt that the ring is a flight ring; whether this was clear to the poet is however questionable.  This much is certain, that Wayland seems to be able to fly away only after he has got back the ring; that is, the one which Beadhild brings him.”  This is by no means certain.  Wayland was a craftsman of legendary skill and could certainly have built wings for himself without a magical flight-ring.  That is what he does in the Low German version; it is also what happens in the related Daidalos myth.  For both of these see the introduction to the present poem.}}. &
Kom þar af \alst{v}ęiði \hld\ \alst{v}eðr-ęygr skyti &
Vǫlundr \alst{l}íðandi \hld\ of \alst{l}angan veg.\eva

\bvb And they took them off and they slid them [back] on, \\
save for one which they slid off.— \\
Came there from the hunt the stormy-eyed shooter: \\
Wayland passing over a long way.\evb\evg


\bvg\bva\mssnote{\Regius~18v/4}%
Gekk hann \alst{b}ru̇nni \hld\ \alst{b}eru hold stęikja; &
\edtext{\alst{á}r}{\Afootnote{metr. and sens. emend.; \emph{hár} \Regius}} brann hrísi \hld\ \alst{a}ll-þurr fura, &
\alst{v}iðr inn \alst{v}ind-þurri, \hld\ fyr \alst{V}ǫlundi.\eva

\bvb Went he the brown she-bear’s flesh to roast; \\
in early morn burned the twigs of all-dry pine— \\
the wood wind-dry—before Wayland.\evb\evg


\bvg\bva\mssnote{\Regius~18v/5}%
Sat ȧ \alst{b}er-fjalli, \hld\ \edtrans{\alst{b}auga talði}{bighs he counted}{\Bfootnote{Wayland’s grief and loneliness are skilfully illustrated by his counting all seven hundred rings, something which had apparently become a habit for him.}}, &
\edtrans{\alst{a}lfa ljóði}{prince of elves}{\Bfootnote{Probably referring to Wayland’s nature as a Wild Man, something also seen by his hunting of bears, skiing, and fierce gaze, all associated with his Finnish or Saami ancestry.  Cf. 14/2b and 32/1b, where Nithad calls him \emph{vísi alfa} ‘chief of elves’.}} \hld\ \alst{ęi}ns saknaði; &
\alst{h}ugði at \alst{h}ęfði \hld\ \alst{H}lǫð·vés dóttir, &
\alst{a}l-vitr \alst{u}nga \hld\ vę́ri \alst{a}ptr komin.\eva

\bvb Sat he on the bear-pelt, bighs he counted— \\
the prince of elves was missing one! \\
Thought he that Ludwigh’s daughter \ken*{= Harware} might have it, \\
that the young elwight might be come back.\evb\evg


\bvg\bva\mssnote{\Regius~18v/7}%
\alst{S}at \alst{s}vá lęngi, \hld\ at \alst{s}ofnaði, &
ok \alst{v}aknaði \hld\ \alst{v}ilja-lauss; &
vissi sér á \alst{h}ǫndum \hld\ \alst{h}ǫfgar nauðir, &
en ȧ \alst{f}ótum \hld\ \alst{f}jǫtur of spęnntan.\eva

\bvb Sat he so long that asleep he fell, \\
and he awoke, powerless. \\
He knew on his hands heavy restraints, \\
and on his feet a fetter tightened.\evb\evg


\bvg\bva\mssnote{\Regius~18v/9}%
\speakernote{[Vǫlundr kvað:]}%
„Hvęrir ’ro \alst{jǫ}frar \hld\ þęir’s \alst{ȧ} lǫgðu &
\alst{b}ęsti-síma \hld\ ok \alst{b}undu mik?“\eva

\bvb\speakernoteb{[Wayland quoth:]}%
“Which are the princes that have laid on \\
the bast-cordage, and bound me?”\evb\evg


\bvg\bva\mssnote{\Regius~18v/10}%
\Ballnote{Grane was the horse of the legendary hero \inx[P]{Siward}, who slew the dragon \inx[P]{Fathomer} and took his gold (see \textlink{Fafnismal}); these events are here, as in the later German \emph{Nibelung} tradition, supposed to have taken place around the Rhine.  Nithad’s speech is sarcastic: “How did you get so rich, Wayland? Was there a dragon’s hoard in the Wolfdales? Clearly not, so you must have stolen from me.”}%
Kallaði \alst{n}ú \alst{N}íð·uðr, \hld\ \alst{N}íara dróttinn: &
„Hvar gatst, \alst{V}ǫlundr, \hld\ \alst{v}ísi alfa, &
\alst{ȯ}ra \alst{au}ra, \hld\ ï \alst{U}lf-dǫlum? &
\alst{G}oll vas þar ęigi \hld\ ȧ \alst{G}rana lęiðu, &
\alst{f}jarri hugða’k vȧrt land \hld\ \alst{f}jǫllum Rínar.“\eva

\bvb Now called Nithad, lord of the Nears: \\
“Where didst thou, Wayland, chief of elves, \\
get \emph{our} ounces in the Wolfdales? \\
Gold was there not on \inx[P]{Grane}’s path; \\
far I thought our land from the fells of the Rhine.”\evb\evg


\bvg\bva\mssnote{\Regius~18v/13}%
\speakernote{[Vǫlundr kvað:]}%
\Ballnote{Wayland responds rather cryptically and almost seems to be speaking to himself.  By asserting the noble lineages of the three swan-wives he gives a legitimate origin for his wealth, but is at the same time aware that Nithad neither believes him nor cares.}%
„\alst{M}an’k at \alst{m}ęiri \hld\ \alst{m}ę́ti ǫ́ttum, &
es vér \alst{h}ęil \alst{h}jú \hld\ \alst{h}ęima vǫ́rum: &
\alst{H}lað·guðr ok \alst{H}ęr·vǫr \hld\ borin vas \alst{H}lǫð·vé, &
\alst{k}unn vas Ǫl·ru̇n \hld\ \edtrans{\alst{K}íars}{Coser’s}{\Bfootnote{For the sense of this name see note to P1 above. —
For the sake of meter the uncontracted hiatus form \emph{Kíarr} must be restored from \emph{Kjárr}; the verse is otherwise only three syllables long.  The uncontracted form is also metrically necessary in \textlink{Atlakvida}[8]/1b and Edáð \emph{Banddr} 2 (\Skp\ 1). —
The development from Latin \emph{Caesar} to ON \emph{Kíarr} \char`~\ \emph{Kjárr} requires some elucidation.  The word was first borrowed from Classical Latin into PGmc. as \emph{*kaisaraz} (cf. OHG \emph{khęisur-} in \textlink{Hildebrandslied} 33, OE \emph{Câsere} in \textlink{Widsith}[4]/3),
which would be expected to yield ON \emph{**kęisarr} were it not for an obscure Norse sound change causing the debuccalization of \emph{*s} > \emph{*h} > ∅ in the first syllable of 3-syllable words when followed by \emph{r} in a following syllable (cf. ON \emph{ȯrr} \char`~\ \emph{vȧrr} ‘our’ < \emph{*ȯarr} < \emph{*ȯsarr} < PN \emph{*onsaraʀ} < PGmc. \emph{*unseraz}; ON \emph{járn} \char`~\ \emph{éarn} < \emph{ísarn}), so that
\emph{*kęisarr} > \emph{kíarr} (\char`~\ \emph{*kéarr}).  The loss of the diphthong is not unexpected, \emph{**-ęi(j)a-} being an utterly illegal vowel glide.
The later ON \emph{keisari, keiseri} is undoubtedly borrowed from Low German.}} dóttir.“\eva

\bvb\speakernoteb{[Wayland quoth:]}%
“I recall that we owned a greater treasure \\
when we a whole household were at home. \\
Ladguth and Harware were born to Ludwigh; \\
known was Alerune, Coser’s daughter.”\evb\evg


\bvg\bva\mssnote{\Regius~18v/15}%
\edtrans{\emph{Úti stóð \alst{k}unnig \hld\ \alst{k}vǫ̇n Níð·aðar}}{Outside stood the cunning wife of Nithad}{\Afootnote{emend. required by the sense (cf. st. 30/1–2); om. \Regius}\Bfootnote{Formulaic long-line template; cf. \textlink{HelgakvidaOne}[48]/3, \textlink{Brot}[5]/1.  That it in fact a formulaic line is shown by its combination with other formulaic lines, both here and in \textlink{Brot}[5].}} &
\edtrans{\emph{ok} hǫ̇n \alst{i}nn of gekk \hld\ \alst{ę}nd-langan sal}{and she went inside the endlong hall}{\Bfootnote{Formulaic; cf. st. 8/2 n. above.}}, &
\alst{st}óð ȧ golfi, \hld\ \alst{st}ilti rǫddu: &
„es⸗a sá nú \alst{h}ýrr, \hld\ es ór \alst{h}olti fęrr.“\eva

\bvb Outside stood the cunning wife of Nithad; \\
and she went inside the endlong hall, \\
stood on the floor, steered her voice: \\
“He is not mild now, who comes out of the wood.”\evb\evg


\bpg\bpa\mssnote{\Regius~18v/16}%
Níð·uðr konungr gaf dóttur sinni Bǫðv·ildi gull-hring þann er hann tók af basti’nu at Vǫlundar, en hann sjalfr bar sverð’it er Vǫlundr átti.  En dróttning kvað:\epa

\bpb King Nithad gave his daughter Beadhild the golden ring which he took from the bast rope in Wayland’s hall, but he himself carried the sword which Wayland had owned.  But the queen quoth:\epb\epg


\bvg\bva\mssnote{\Regius~18v/19}%
„\alst{T}ęnn hǫ̇num \alst{t}ęygja⸗sk \hld\ es hǫ̇num ’s \alst{t}ét sverð, &
ok hann \alst{B}ǫðv·ildar \hld\ \alst{b}aug of þękkir, &
\alst{ǫ́}mun eru \alst{au}gu \hld\ \alst{o}rmi inum frȧna; &
\alst{s}níðið ér hann \hld\ \alst{s}ina magni, &
ok \alst{s}ętið hann \alst{s}íðan \hld\ ï \alst{S}ę́var-stǫð.“\eva

\bvb “His teeth are bared when he is shown the sword, \\
and Beadhild’s bigh he recognizes; \\
reminiscent are his eyes to the gleaming serpent’s. \\
Snithe ye from him the might of his sinews, \\
and set him thereafter on Seastead!”\evb\evg


\bpg\bpa\mssnote{\Regius~18v/21}%
Svá var gǫrt, at skornar vǫ́ru sinar í knés-fótum ok settr í holm einn, er þar var fyrir landi, er hét Sę́var-staðr. Þar smíðaði hann konungi alls-kyns gǫr-simar; engi maðr þorði at fara til hans, nema konungr einn. Vǫlundr kvað:\epa

\bpb So it was done that the sinews in his houghs were cut and he was placed on the lonely islet which lay there before the land and which was called Seastead.  There he forged for the king every kind of jewel.  No man dared go to him, save the king alone.  Wayland quoth:\epb\epg


\bvg\bva\mssnote{\Regius~18v/24}%
„\edtrans{Skínn}{shines}{\Bfootnote{Metrically deficient, since \emph{sk-} and \emph{s-} cannot alliterate.  A possible emendation is \emph{se’k} ‘I see’.}} Níð·aði \hld\ sverð á linda, &
þat’s ek \alst{h}vęsta \hld\ sęm \alst{h}agast kunna’k &
ok ek \alst{h}ęrða’k \hld\ sęm \alst{h}ǿgst þȯtti; &
sá ’s mér \alst{f}rȧnn mę́kir \hld\ ę́ \alst{f}jarri borinn; &
\alst{s}é’k⸗a þann Vǫlundi \hld\ til \alst{s}miðju borinn.\eva

\bvb “The sword shines on Nithad’s belt, \\
which I sharpened as most handily I could, \\
and I hardened as most pleasingly seemed. \\
That gleaming blade is ever further from me carried; \\
I see it not for Wayland to the smithy carried!\evb\evg


\bvg\bva\mssnote{\Regius~18v/27}%
Nú \alst{b}err \alst{B}ǫðv·ildr \hld\ \alst{b}rúðar minnar &
—\alst{b}íð’k⸗a þęss \alst{b}ót— \hld\ \alst{b}auga rauða.“\eva

\bvb Now does Beadhild bear my bride’s \\
—I await no recompense for that—red bighs.”\evb\evg


\bvg\bva\mssnote{\Regius~18v/28}%
\edtrans{\alst{S}at—né \alst{s}vaf ȧ·valt—}{He sat—never slept—}{\Bfootnote{Compare \textlink{Gudrunarhvot}[13]/3: \emph{hófu mik—né drękkðu—} ‘[they] lifted me—drowned [me] not—’.}} \hld\ ok \alst{s}ló hamri; &
vél gęrði \alst{h}ęldr \hld\ \alst{h}vatt Níð·aði; &
\alst{d}rifu ungir tvęir \hld\ ȧ \alst{d}ýr séa &
\alst{s}ynir Níð·aðar \hld\ ï \alst{S}ę́var-stǫð.\eva

\bvb He sat—never slept—and struck the hammer; \\
wiles he most boldly planned for Nithad. \\
Two young ones were drifting to see costly things: \\
Nithad’s sons, to Seastead.\evb\evg


\bvg\bva\mssnote{\Regius~18v/30}%
\alst{K}vǫ́mu til \alst{k}istu, \hld\ \alst{k}rǫfðu lukla, &
\alst{o}pin vas \alst{i}ll-úð, \hld\ es þęir \alst{ï} sǫ́u, &
fjǫlð vas þar \alst{m}ęina, \hld\ es \alst{m}ǫgum sẏndi⸗sk &
at vę́ri \alst{g}oll rautt \hld\ ok \alst{g}ǫr-simar.\eva

\bvb Came they to the chest, demanded the keys; \\
open was the evil when inside they looked. \\
A host was there of harms, which to the lads seemed \\
like were they red gold and jewelry.\evb\evg


\bvg\bva\mssnote{\Regius~18v/33}%
\speakernote{[Vǫlundr kvað:]}%
„Komið \alst{ęi}nir tvęir, \hld\ komið \alst{a}nnars dags; &
ykkr lę́t’k þat \alst{g}oll \hld\ of \alst{g}efit verða; &
\alst{s}ęgið-a męyjum \hld\ né \alst{s}al-þjóðum, &
\alst{m}anni øngum, \hld\ at \alst{m}ik fyndið.“\eva

\bvb\speakernoteb{[Wayland quoth:]}%
“Come alone ye two, come another day; \\
to you, I say, this gold will be given. \\
Tell no maidens nor hall-folk \\
—not a man!—that \emph{me} ye met.”\evb\evg


\bvg\bva\mssnote{\Regius~19r/1}%
\alst{S}nimma kallaði \hld\ \alst{s}ęggr ȧ annan, &
\alst{b}róðir ȧ \alst{b}róður: \hld\ „gǫngum \alst{b}aug séa!“ &
\alst{K}vǫ́mu til \alst{k}istu, \hld\ \alst{k}rǫfðu lukla, &
\alst{o}pin vas \alst{i}ll-úð \hld\ es þęir \alst{ï} litu.\eva

\bvb Early called one youth to another, \\
brother to brother: “Let us go see the bighs!” \\
Came they to the chest, demanded the keys; \\
open was the evil when inside they gazed.\evb\evg


\bvg\bva\mssnote{\Regius~19r/3}%
Snęið af \alst{h}ǫfuð \hld\ \edtrans{\alst{h}u̇na}{bear-cubs}{\Bfootnote{An affectionate term for young boys, perhaps relating to warrior-initiations done in bear-skins.  This word is repeated by Nithad in st. 32 and mirrored by Wayland in st. 34.}} þęira &
ok und \edtrans{\alst{f}ęn \alst{f}jǫturs}{the fetter’s fen}{\Bfootnote{Unclear.  The smithy or islet may be Wayland’s “fetter”, in which case he buried them in a fen on the island.}} \hld\ \alst{f}ǿtr of lagði, &
ęn \edtrans{þę́r \alst{sk}álar, \hld\ es und \alst{sk}ǫrum vǫ́ru}{those bowls which were under their curls}{\Bfootnote{Their skulls.}}, &
\alst{s}vęip útan \alst{s}ilfri, \hld\ \alst{s}ęldi Níð·aði.\eva

\bvb He sliced off the heads of those bear-cubs, \\
and under the fetter’s fen their feet he laid. \\
And the bowls which were under their curls \\
he coated with silver, gave to Nithad.\evb\evg


\bvg\bva\mssnote{\Regius~19r/5}%
\alst{Ę}n ór \alst{au}gum \hld\ \edtrans{\alst{ja}rkna-stęina}{arkenstones}{\Bfootnote{Probably round crystals.}} &
sęndi \alst{k}unnigri \hld\ \alst{k}vǫ̇n Níð·aðar; &
ęn ór \alst{t}ǫnnum \hld\ \alst{t}vęggja þęira &
\alst{s}ló brjóst-kringlur, \hld\ \alst{s}ęndi Bǫðv·ildi.\eva

\bvb And from the eyes arkenstones \\
he sent to the cunning wife of Nithad. \\
And from the teeth of the two \\
he struck breast-brooches, sent to Beadhild.\evb\evg


\bvg\bva\mssnote{\Regius~19r/7}%
\Ballnote{Something appears to be missing before this stanza, but the narrative can be gleaned.  Beadhild breaks the bigh given to her by Nithad (mentioned above in sts. 10—see note there—and 17), and fears her father’s anger.  She goes to Wayland in secret and begs him to fix it.  The sight of the ring reminds Wayland of his wife and he is furious; he decides to begin with his revenge, and rapes Beadhild.}%
Þȧ nam \alst{B}ǫðv·ildr \hld\ \alst{b}augi at hrósa &
\edtext{[...]}{\Bfootnote{The meter requires a half-line here, perhaps containing a repetition of 1a: \emph{baugi at hrósa} ‘the bigh to praise’.}}\ \hld\ es brotit hafði, &
„\alst{þ}ori’g⸗a’k sęgja, \hld\ nema \alst{þ}ér ęinum.“\eva

\bvb Then Beadhild began the bigh to praise, \\
{[...]} which she had broken, \\
“I dare not tell, save to thee alone.”\evb\evg


\bvg\bva\mssnote{\Regius~19r/8}%
\speakernote{Vǫlundr kvað:}%
„Ek \alst{b}ǿti svá \hld\ \alst{b}rest ȧ golli, &
at \alst{f}ęðr þïnum \hld\ \alst{f}ęgri þykkir, &
ok \alst{m}ǿðr þinni \hld\ \alst{m}iklu bętri, &
ok \alst{s}jalfri þér \hld\ at \alst{s}ama hófi.“\eva

\bvb\speakernoteb{Wayland quoth:}
“I will so mend the crack on the gold, \\
that to thy father it fairer seems, \\
and to thy mother even better, \\
and to thyself of the same rank.”\evb\evg


\bvg\bva\mssnote{\Regius~19r/10}%
\alst{B}ar hȧna \alst{b}jóri, \hld\ \edtrans{því’t \alst{b}ętr kunni}{for he knew better}{\Bfootnote{He was more cunning than her.}}, &
\alst{s}vá’t hǫ̇n ï \alst{s}essi \hld\ of \alst{s}ofnaði. &
\speakernote{[Vǫlundr kvað:]}%
„Nú \alst{h}ęfi’k \alst{h}ęfnt \hld\ \alst{h}arma minna &
\alst{a}llra \edtrans{nema \alst{ęi}nna}{save one}{\Bfootnote{Presumably the deprivation of his mobility due to the hamstringing, which he resolves by crafting his flight suit.}} \hld\ \edtrans{\alst{ï}·við-gjarna}{insidious ones}{\Bfootnote{King Nithad and his house.}}.“\eva

\bvb He overcame her with beer—for he knew better— \\
so that she in the seat did fall asleep. \\
“Now have I avenged my harms, \\
all, save one, on the insidious ones.”\evb\evg


\bvg\bva\mssnote{\Regius~19r/12}%
„\alst{V}ęl ek,“ kvað \alst{V}ǫlundr, \hld\ „\alst{v}erða’k ȧ \edtrans{fitjum}{paddles}{\Bfootnote{\CV: \emph{fit} ‘the webbed foot of water-birds’, here a reference to the flight-suit which allows Wayland to regain his freedom.}}, &
þęim’s mik \alst{N}íð·aðar \hld\ \alst{n}ǫ́mu rekkar.“ &
\alst{H}lę́jandi Vǫlundr \hld\ \alst{h}óf⸗sk at lopti, &
\alst{g}rátandi Bǫðv·ildr \hld\ \alst{g}ekk ór ęyju, &
tregði \alst{f}ǫr \alst{f}riðils \hld\ ok \alst{f}ǫður ręiði.\eva

\bvb “Well I”, quoth Wayland, “fall on my paddles; \\
those of which Nithad’s men bereaved me!” \\
Laughing, Wayland threw himself in the air; \\
weeping, Beadhild went from the island, \\
grieved the lover’s flight and the father’s wrath.\evb\evg

\sectionline

\bvg\bva\mssnote{\Regius~19r/14}%
Úti \edtrans{st\emph{óð}}{stood}{\Afootnote{emend.; \emph{stęndr} \Regius}} \alst{k}unnig \hld\ \alst{k}vǫ̇n Níð·aðar, &
ok hǫ̇n \alst{i}nn of gekk \hld\ \alst{ę}nd-langan sal, &
ęn hann ȧ \alst{s}al-garð \hld\ \alst{s}ętti⸗sk at hvíla⸗sk, &
„Vakir þú \alst{N}íð·uðr, \hld\ \alst{N}íara dróttinn?“\eva

\bvb Outside stood the cunning wife of Nithad \\
and she went inside the endlong hall, \\
but he on the courtyard set down to rest— \\
“Art thou awake, O Nithad, lord of the Nears?”\evb\evg


\bvg\bva\mssnote{\Regius~19r/17}%
\speakernote{[Níð·uðr kvað:]}%
„\edtrans{\alst{V}aki’k ȧ·valt \hld\ \alst{v}ilja-lauss}{I am always awake, powerless}{\Bfootnote{This line references sts. 12 and 20, but there Wayland was the powerless man who never slept.  By his revenge the suffering has been transferred onto Nithad.}}, &
\alst{s}ofna’k minst, \hld\ síðst \alst{s}onu dauða, &
\alst{k}ęll mik ï hǫfuð, \hld\ \edtrans{\alst{k}ǫld eru⸗mk rǫ́ð þín}{cold seem thy counsels}{\Bfootnote{A severe insult to a woman of power, for such counsels to her husband was how she would influence worldly affairs.  In this way Wayland’s revenge reaches also Nithad’s wife.}}, &
\alst{v}ilnumk þęss nú, \hld\ at við \alst{V}ǫlund dǿma’k.“\eva

\bvb\speakernoteb{[Nithad quoth:]}%
“I am always awake, powerless; \\
I sleep the least since my sons died. \\
My head turns cold; cold seem thy counsels— \\
I would now but that I with Wayland may speak.”\evb\evg

\sectionline

\bvg\bva\mssnote{\Regius~19r/19}%
\speakernote{[Níð·uðr kvað:]}%
„Sęg mér þat \alst{V}ǫlundr, \hld\ \alst{v}ísi alfa, &
af \alst{h}ęilum \alst{h}vat \hld\ varð \alst{h}u̇num mïnum?“\eva

\bvb\speakernoteb{[Nithad quoth:]}%
“Tell me this, Wayland, chief of elves: \\
what became of my healthy bear-cubs?”\evb\evg


\bvg\bva\mssnote{\Regius~19r/20}%
\speakernote{[Vǫlundr kvað:]}%
„\alst{Ęi}ða skalt mér \alst{á}ðr \hld\ \alst{a}lla vinna, &
\edtext{at \alst{sk}ips borði \hld\ ok at \alst{sk}jaldar rǫnd, &
at \alst{m}ars bǿgi \hld\ ok at \alst{m}ę́kis ęgg}{\lemma{at skips \dots\ ęgg ‘by side \dots\ of sword’}\Bfootnote{Nithad must swear oaths by the tools of trade of the warrior i.e. on his martial honour.  Cf. \textlink{HelgakvidaTwo}, where broken oaths are to come back “biting” the oath-breaker by cursing his ship, horse, and sword, in that order.}} &
at þú \edtrans{\alst{k}vęlj⸗at}{shalt not torment}{\Bfootnote{A negative imperative.  The 2nd. sg. imper. of \emph{kvęlja} ‘torment’ is \emph{kvęl}, but the negative clitic \emph{⸗at} causes the \emph{-j-} of the stem to reappear in a rare \emph{liaison} effect.  This indicates that forms like \emph{kvęl} were still understood to contain \emph{-j}, just no longer pronounced, but which could reappear in the correct circumstance.}} \hld\ \edtext{\alst{k}vǫ̇n Vǫlundar, &
né \alst{b}rúði minni}{\lemma{kvǫ̇n Vǫlundar ‘wife of Wayland’, brúði minni ‘my bride’}\Bfootnote{Beadhild, who is now pregnant.}} \hld\ at \alst{b}ana verðir, &
þó’tt kvǫ̇n \alst{ęi}gim, \hld\ þá’s \alst{é}r kunnið, &
eða \alst{jó}ð \alst{ęi}gim \hld\ \alst{i}nnan hallar.\eva

\bvb\speakernoteb{[Wayland quoth:]}%
“Oaths shalt thou first all swear to me— \\
by side of ship and rim of shield, \\
by bough of steed and edge of sword— \\
that thou shalt not torment the wife of Wayland, \\
nor of my bride become the bane, \\
though a wife we might own whom ye might know; \\
or a babe might own within the hall.\evb\evg


\bvg\bva\mssnote{\Regius~19r/24}%
\alst{G}akk til smiðju, \hld\ þęirar’s \alst{g}ørðir, &
þar fiðr \alst{b}ęlgi \hld\ \alst{b}lóði stokna, &
snęið’k af \alst{h}ǫfuð \hld\ \alst{h}u̇na þinna &
ok und \alst{f}ęn \alst{f}jǫturs \hld\ \alst{f}ǿtr of lagða’k.\eva

\bvb Go to the smithy which thou madest; \\
there wilt thou find bellows blood-besprinkled. \\
I sliced off the heads of thy bear-cubs, \\
and under the fetter’s fen their feet I laid.\evb\evg


\bvg\bva\mssnote{\Regius~19r/26}%
En þę́r \alst{sk}álar, \hld\ es und \alst{sk}ǫrum vǫ́ru, &
\alst{s}vęip’k útan \alst{s}ilfri, \hld\ \alst{s}ęlda’k Níð·aði, &
\alst{e}n ór \alst{au}gum \hld\ \alst{ja}rkna-stęina, &
sęnda’k \alst{k}unnigri \hld\ \alst{k}vǫ̇n Níð·aðar.\eva

\bvb And the bowls which were under their curls, \\
I coated with silver, gave to Nithad. \\
And from the eyes arkenstones \\
I sent to the cunning wife of Nithad.\evb\evg


\bvg\bva\mssnote{\Regius~19r/28}%
Ęn ór \alst{t}ǫnnum \hld\ \alst{t}vęggja þęira &
\alst{s}ló’k brjóst-kringlur, \hld\ \alst{s}ęnda’k Bǫðv·ildi; &
nú gęngr \alst{B}ǫðv·ildr \hld\ \alst{b}arni aukin, &
\edtrans{\alst{ęi}nga dóttir \hld\ \alst{y}kkur bęggja.}{the only daughter of you both}{\Bfootnote{Formulaic, near-identical to \HervararSaga\ st. 25/1–2: (\emph{Vaki, Angantýr, \hld\ vękr þik Hęrvǫr, // ęinga dóttir \hld\ ykkur Svǫ́fu.} ‘Wake, Ongentew: Harware awakes thee, the only daughter of thee and Sweve.’ Cf. also \Beowulf\ 375a, 2997b: \emph{ángan dohtor} ‘only daughter (accusative)’.)}}“\eva

\bvb And from the teeth of the two \\
I struck breast-brooches, sent to Beadhild. \\
Now goes Beadhild swollen with child; \\
the only daughter of you both.”\evb\evg


\bvg\bva\mssnote{\Regius~19r/30}\speakernote{[Níð·uðr kvað:]}%
„\alst{M}ę́ltir⸗a þat \alst{m}ál, \hld\ es mik \alst{m}ęirr tregi, &
né þik \alst{v}ilja’k \alst{V}ǫlundr \hld\ \alst{v}err of níta; &
es⸗at svá maðr \alst{h}ǫ́r, \hld\ at þik af \alst{h}ęsti taki, &
\alst{n}é svá ǫflugr, \hld\ at þik \alst{n}eðan skjóti, &
þar’s þú \alst{sk}ollir \hld\ við \alst{sk}ý uppi.“\eva

\bvb\speakernoteb{[Nithad quoth:]}%
“Thou couldst not have spoken a speech which would grieve me more; \\
nor could I worse wish, Wayland, to deny thee. \\
There is no man so high that he might take thee from a horse, \\
nor so strong that he might shoot thee from below, \\
where thou dost jeer by the clouds above!”\evb\evg


\bvg\bva\mssnote{\Regius~19v/1}%
\alst{H}lę́jandi Vǫlundr \hld\ \alst{h}óf⸗sk at lopti, &
en \alst{ȯ}·kȧtr Níð·uðr \hld\ sat þȧ \alst{ę}ptir.\eva

\bvb Laughing, Wayland threw himself in the air; \\
but, gloomy, Nithad stayed behind.\evb\evg

\sectionline

\bvg\bva\mssnote{\Regius~19v/2}%
\speakernote{[Níð·uðr kvað:]}%
„Upp rís \edtrans{\alst{Þ}akk·ráðr}{Thankred}{\Bfootnote{A German name never found elsewhere in ON, but equivalent to MHG \emph{Dancrát}.}}, \hld\ \alst{þ}rę́ll minn batsti, &
\alst{b}ið \alst{B}ǫðv·ildi, \hld\ \edtext{męy ina \alst{b}rá-hvítu, &
gangi \alst{f}agr-varið}{\lemma{męy hina brá-hvítu \dots\ fagr-varið ‘the brow-white maiden \dots\ fair-clothed’}\Bfootnote{Nithad still has some doubt in his heart and by these words tries to convince himself of the innocence of his daughter (\emph{mę́r} ‘maiden, virgin’).}} \hld\ við \alst{f}ǫður rǿða.“\eva

\bvb\speakernoteb{[Nithad quoth:]}%
“Rise up, Thankred, my best thrall; \\
bid Beadhild, the brow-white maiden, \\
to go, fair-clothed, with her father to counsel.”\evb\evg

\sectionline

\bvg\bva\mssnote{\Regius~19v/3}%
\speakernote{[Níð·uðr kvað:]}%
„Es þat \alst{s}att Bǫðv·ildr, \hld\ es \alst{s}ǫgðu mér, &
\alst{s}ǫ́tuð it Vǫlundr \hld\ \alst{s}aman í holmi?“\eva

\bvb\speakernoteb{[Nithad quoth:]}%
“Is it true, Beadhild, as they told me— \\
stayed thou and Wayland together on the islet?”\evb\evg


\bvg\bva\mssnote{\Regius~19v/4}\speakernote{[Bǫðv·ildr kvað:]}%
„\alst{S}att ’s þat Níð·uðr \hld\ es \edtrans{\alst{s}agði}{\emph{he} told}{\Bfootnote{Beadhild knows that Wayland is the only one aware of the rape and thus deduces that \emph{he} told her father.  She makes a subtle change in the conjugation from her father’s general third person plural (“what they told”), to the specific singular form (“what \emph{he} told”).}} þér: &
\alst{s}ǫ́tum vit Vǫlundr \hld\ \alst{s}aman í holmi &
\alst{ęi}na \alst{ǫ}gur-stund, \hld\ \alst{ę́}va skyldi; &
ek \alst{v}ę́tr hǫ̇num \hld\ \edtext{\alst{v}inna}{\Afootnote{metr. and sens. emend.; om. \Regius}} \edtext{kunna’k, &
ek \alst{v}ę́tr hǫ̇num \hld\ \alst{v}inna mátta’k}{\lemma{kunna’k ‘knew’, mátta’k ‘could’}\Bfootnote{Beadhild could defend herself neither mentally (\emph{kunna} ‘to know, understand’) nor physically (\emph{mega} ‘to have strength to do, avail’).  A powerful final stanza.}}.“\eva

\bvb\speakernoteb{[Beadhild quoth:]}%
“True it is, Nithad, as \emph{he} told thee— \\
I and Wayland stayed together on the islet \\
for one heavy hour—it should never have been. \\
I nowise knew withstand him; \\
I nowise could withstand him.”\evb\evg

\sectionline
