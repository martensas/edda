\bookStart{Speeches of Shirner}[Skírnis mǫ́l]
\setBookCode{Skirnismal}

\begin{flushright}%
\textbf{Dating} \parencite{Sapp2022}: C10th (0.897)

\textbf{Meter:} \Ljodahattr, \Galdralag\ (TODO)%
\end{flushright}

\section{Introduction}

The \textbf{Speeches of Shirner} (\textlink{Skirnismal}) is attested in full in both \Regius\ and \AM.  The name \emph{Skírnis-mǫ́l} ‘Speeches of Shirner’ comes from \AM; \Regius\ instead has \emph{Fǫr Skírnis} ‘Shirner’s journey’.

\subsection{The \Gylfaginning\ paraphrase}

The narrative of \textlink{Skirnismal} is summarised in \Gylfaginning\ 37, which also quotes st. 42.  \Gylfaginning\ 37 begins with a long introduction corresponding to \textlink{Skirnismal}[P1]–2:

\begin{quote}{\small ‘Gymer was the name of a man, and his woman was Earbode; she was of the lineage of mountain-risers.  Their daughter is Gird who is fairest of all women.

It was one day when Free had gone to the Lithshelf and looked out over all the Homes.  And when he looked north he saw on a farm a large and fair house, and to that house walked a woman, and when she lifted her hands and closed the doors behind her it shone from her hands into both the air and the waters, and all the homes were brightened by her.

onAnd the beauty which he had beheld from the holy seat harmed him so greatly that he walked away filled with grief, and when he came home he said nothing; he neither slept nor drank.  No one dared to get words out of him.’}\end{quote}

After this it paraphrases sts. 3–9, describing Shirner’s interaction with Free:

\begin{quote}{\small ‘Then Nearth had Shirner, Free’s shoe-swain, summoned before him and asked him to go to Free and bid him to speak and ask at whom he was so wroth that he would not speak with men.  And Shirner said that he would go, although not eagerly, and said that he expected ill answers from him.

And when he came to Free he asked why Free was so downcast and spoke nothing with men.  Then Free answers, and said that he had seen a fair woman and for her sakes was he so full of grief that he would not live long if he should not reach her, “and now shalt thou journey to ask for her hand on my behalf, and bring her home hither whether her father wants to or not, and I shall reward thee well for that.”

Then Shirner answers; said so, that he will go on the errand-journey, but Free shall give him his sword; it was such a good sword that it struck by itself.  And Free did not deny him that and gave him the sword.’}\end{quote}

The rest of the poem (sts. 10–38) is summarised very succinctly:

\begin{quote}{\small ‘Then Shirner journeyed and asked for the woman’s hand for him [Free] and got her promise that nine nights later she would come to that place which is called Barrey and have a wedding with Free.  And when Shirner told Free his errand then he quoth this:’}\end{quote}

After which the text cites a closely related variant of stanza 42.  It lastly explains that \emph{Þessi sǫk er til þess, er Freyr var svá vápn-lauss, er hann barðist við Belja ok drap hann með hjartar-horni.} ‘This event (viz. Free’s giving of his sword) is the reason for why Free was so unarmed when he fought against \inx[P]{Bellower} and slew him with a hart’s antler.’

It seems near-certain that the author of \Gylfaginning\ had access to \textlink{Skirnismal} directly rather than a prose retelling of the story.  There is no detail in his paraphrase that is not found in the \Regius-version of the poem, although the introductory prose differs a fair bit and Shirner’s curse is entirely omitted.  These circumstances are easily explained if the version of \textlink{Skirnismal} underlying \Gylfaginning\ were written down based on oral tradition; the poetry, being in bound form, would remain largely stable, while the introductory prose would vary with each retelling.  To sum up a narrative mythic poem in prose form and then quote one or two stanzas is something probably done elsewhere in \Gylfaginning; see \textlink{EddicFragments} below.

\section{Text}

\subsection{The Speeches of Shirner}

\bpg\bpa\mssnote{\Regius~11r/10, \AM~2r/11}%
Freyr, sonr Njarðar, hafði einn dag setsk í \edtrans{Hlið-skjálf}{Lithshelf}{\Bfootnote{The heavenly lookout point of the Gods.}} ok sá um heima alla; hann sá í Jǫtun-heima ok sá þar mey fagra, þá er hón gekk frá skála fǫður síns til skemmu; þar af fekk hann hug-sóttir miklar. Skírnir hét \edtrans{skó-sveinn}{shoeswain}{\Bfootnote}{A young boy who ties the shoes for his master.} Freys.  Njǫrðr bað hann kveðja Frey máls. Þá mę́lti Skaði:\epa

\bpb \inx[P]{Free}[{\huge F}\textsc{ree}], son of \inx[P]{Nearth}, had one day sat down in the \inx[L]{Lithshelf}, and saw throughout all the \inx[C]{Homes}.  He looked into the \inx[L]{Ettinhomes} and saw there a fair maiden as she walked from her father’s hall to her bower; thereof he got great heart-aches.  \inx[P]{Shirner} was the name of Free’s shoe-swain; Nearth asked him to get Free to speak.  Then \inx[P]{Shede} spoke:\epb\epg


\bvg\bva\mssnote{\Regius~11r/14, \AM~2r/15}%
„\edtext{Rís-tu nú Skírnir \hld\ ok gakk at bęiða}{\lemma{rís \dots\ bęiða ‘Rise \dots\ beg’}\Bfootnote{Alliteration is missing here. A simple solution would be to replace \emph{gakk} ‘go’ with a synonym like \emph{rinn} ‘run’ or \emph{ráð} ‘resolve’, but this lessens the semantic mirroring with l. 2/2 below (though, the insertion of the verb \emph{ganga} in the present stanza may in fact be due to influence from 2/2).}} &
\ind okkarn \alst{m}ála \alst{m}ǫg, &
ok þęss at \alst{f}regna \hld\ hvęim hinn \alst{f}róði séi &
\ind \alst{o}f-ręiði \edtrans{\alst{a}fi}{man}{\Bfootnote{While this word usually means “father” or “grandfather”, it should here mean “man” without a connotation of old age. See further \CV.}}.“\eva

\bvb%
“{\huge R}\textsc{ise now, Shirner}, and go to beg \\
\ind our lad for speech, \\
and to ask at whom the wise \\
\ind man might be cross.”\evb\evg


\bvg\bva\speakernote{Skírnir kvað:}\mssnote{\Regius~11r/15, \AM~2r/17}%
„\alst{I}llra \alst{o}rða \hld\ es mér \alst{ȯ}n at ykkrum syni, &
\ind ef ek gęng at \alst{m}ę́la við \alst{m}ǫg, &
ok þęss at \alst{f}regna, \hld\ hvęim hinn \alst{f}róði séi &
\ind \alst{o}f-ręiði \alst{a}fi.“\eva

\bvb\speakernoteb{Shirner quoth:}%
“Ill words I expect from your son \\
\ind if I go to speak with the lad, \\
and to ask at whom the wise \\
\ind man might be cross.”\evb\evg


\bvg\bva\speakernote{Skírnir:}\mssnote{\Regius~11r/17, \AM~2r/18}%
„Sęg þat \alst{F}ręyr, \hld\ \alst{f}olk-valdi goða, &
\ind ok ek \alst{v}ilja \alst{v}ita, &
hví þú \alst{ęi}nn sitr \hld\ \alst{ę}nd-langa sali, &
\ind mïnn \alst{d}róttinn, of \alst{d}aga?“\eva

\bvb\speakernoteb{Shirner [quoth]:}%
“Tell this, O Free, troop-wielder of the gods— \\
\ind I too would wish to know \\
why thou sittest alone in the endlong halls, \\
\ind my lord, during the days.”\evb\evg


\bvg\bva\speakernote{Fręyr:}\mssnote{\Regius~11r/19, \AM~2r/20}%
„Hví of \alst{s}ęgja’k þér, \hld\ \alst{s}ęggr hinn ungi, &
\ind \alst{m}ikinn \alst{m}óð-trega? &
því’t \edtrans{\alst{a}lf-rǫðull}{Elf-wheel}{\Bfootnote{A rare poetic synonym (\emph{hęiti}) for the sun; see note to \textlink{Vafthrudnismal} 47/1.}} \hld\ lýsir of \alst{a}lla daga &
\ind ok þęy⸗gi at \alst{m}ïnum \alst{m}unum.“\eva

\bvb\speakernoteb{Free [quoth]:}%
“Why should I tell thee, O young youth, \\
\ind of my great heartache? \\
For the Elf-wheel \ken{sun} shines during all days \\
\ind and nowise to my liking.”\evb\evg


\bvg\bva\speakernote{Skírnir:}\mssnote{\Regius~11r/20, \AM~2r/21}%
„\alst{M}uni þïna \hld\ hykk⸗a svá \alst{m}ikla vesa, &
\ind at þú mér \edtrans{\alst{s}ęggr}{youth}{\Bfootnote{This word usually means simply ‘man’, but it seems to have a specific connotation with youth. Its original meaning is ‘messenger’, and the semantic shift is thus: ‘messenger’ > ‘young man’ > ‘warrior/man’. The sense of ‘young man’ is also seen in \textlink{Volundarkvida} 23, where it is used in reference to king Nithad’s two young sons. In the present stanza it answers Free’s addressing Shirner as \emph{sęggr hinn ungi} ‘the young youth’; Shirner points out that the two are of equal age, and so Free is as much of a young man as he.}} né \alst{s}ęgir; &
\alst{u}ngir saman \hld\ vǫ́rum ï \alst{á}r-daga, &
\ind vęl mę́ttim \alst{t}vęir \alst{t}rúask.“\eva

\bvb\speakernoteb{Shirner [quoth]:}%
“Thy liking I do not think so great \\
\ind that thou, O youth, should not tell me. \\
Young together were we in days of yore; \\
\ind we two might well trust each other.”\evb\evg


\bvg\bva\speakernote{Fręyr:}\mssnote{\Regius~11r/22, \AM~2r/23}%
„İ \alst{G}ymis gǫrðum \hld\ ek \alst{g}anga sá &
\ind \alst{m}ér tíða \alst{m}ęy; &
\alst{a}rmar lýstu, \hld\ en \alst{a}f þaðan &
\ind allt \edtrans{\alst{l}opt ok \alst{l}ǫgr}{air and sea}{\Bfootnote{Formulaic and very old, also paralleled in the Anglo-Saxon. TODO.}}.\eva

\bvb\speakernoteb{Free [quoth]:}%
“In Gymer’s yards I saw walking \\
\ind a maiden dear pleasing to me. \\
Her arms shone, and thereof [shone] \\
\ind all the air and sea.\evb\evg


\bvg\bva\mssnote{\Regius~11r/24, \AM~2r/24}%
\alst{M}ę́r ’s mér tíðari \hld\ an \alst{m}anna hvęim &
\ind \alst{u}ngum ï \alst{á}r-daga; &
\alst{ȧ}sa ok \alst{a}lfa \hld\ þat vill \edtrans{\alst{ę}ngi maðr}{no one}{\Bfootnote{Lit. ‘no man’, where “man” just means person.  Cf. note to final st. of \textlink{Vafthrudnismal} 55.}}, &
\ind at vit \alst{s}ǫ́tt \alst{s}éim.“\eva

\bvb The maiden is more pleasing to me than to any young \\
\ind man in days of yore. \\
Of the \inx[F]{Eese and Elves} does no one wish \\
\ind that we two should be agreed.”\evb\evg


\bvg\bva\speakernote{Skírnir:}\mssnote{\Regius~11r/25, \AM~2r/25}%
„\alst{M}ar gef mér þȧ, \hld\ es mik of \alst{m}yrkvan beri &
\ind \alst{v}ísan \alst{v}afr-loga, &
ok þat \alst{s}verð, \hld\ es \alst{s}jalft vegisk &
\ind við \alst{jǫ}tna \alst{ę́}tt.“\eva

\bvb\speakernoteb{Shirner [quoth]:}%
“The steed then give me which might bear me over the dark, \\
\ind wise wavering-flame, \\
and that sword which by itself might strike \\
\ind against the race of \inx[G]{Ettins}.”\evb\evg


\bvg\bva\speakernote{Fręyr:}\mssnote{\Regius~11r/27, \AM~2r/27}%
„\alst{M}ar þér þann gef’k, \hld\ es þik of \alst{m}yrkvan \edtext{berr &
\ind \alst{v}ísan \alst{v}afr-loga, &
auk þat \alst{s}verð, \hld\ es \alst{s}jalft mun vegask, &
\ind ef sá ’s \alst{h}orskr es \alst{h}ęfr.“}{\lemma{berr ‘bears’; mun vegask, ef sá ’s horskr es hęfr ‘will strike, if he is wise who has it’}\Bfootnote{In his response Free replaces the subjunctive verb forms (\emph{beri} ‘might bear’, \emph{vegisk} ‘might strike’) with indicative and future forms, giving a sense of certainity and authority. The steed and sword are faultless, and if Shirner fails on the mission, it would be only due to his own fault (“if he is sharp who owns it.”).}}\eva
%TODO? Change the line numbering from 1–4 to 1, 3–4.

\bvb\speakernoteb{Free [quoth]:}%
“That steed I give thee which bears thee over the dark, \\
\ind wise wavering-flame, \\
and that sword which by itself will strike \\
\ind if he is wise who has it.”\evb\evg


\bpg\bpa Skírnir mę́lti við hest’inn:\epa
\bpb Shirner spoke with the horse:\epb\epg


\bvg\bva\mssnote{\Regius~11r/29, \AM~2r/28}%
\Ballnote{Shirner expresses his resolute loyalty to Free.  He will not abandon the horse given to him by his lord; either they both make it or both perish.}%
„\alst{M}yrkt es úti, \hld\ \alst{m}ál kveð’k okkr fara &
\ind \alst{ú}rig fjǫll \alst{y}fir &
\ind \edtrans{\alst{þ}ursa}{of the Thurses}{\Afootnote{so \AM; \emph{þyria} \Regius}} \alst{þ}jóð yfir; &
\alst{b}áðir vit komumk \hld\ eða okkr \alst{b}áða tękr &
\ind sá hinn \edtrans{\alst{á}m-átki \alst{jǫ}tunn}{uncanny ettin}{\Bfootnote{Formulaic; the adjective \emph{ám-áttigr} ‘uncanny’ is used exclusively for evil supernatural beings.  See note to \textlink{Voluspa} 8.}}.“\eva

\bvb “It is dark outside; I call it time for us to journey \\
\ind over the drizzling mountains, \\
\ind over the tribe of \inx[G]{Thurses}. \\
{[Either]} we both come through or us both does take \\
\ind that uncanny ettin.”\evb\evg


\bpg\bpa\mssnote{\Regius~11r/31, \AM~2v/1}%
Skírnir reið i Jǫtun-heima til Gymis garða; þar vǫ́ru hundar ólmir ok bundnir fyrir skíð-garðs hliði þess, er um sal \edtrans{Gerðar}{Gird}{\Bfootnote{It is only now that we find out the maiden’s name.}} var. Hann reið at þar, er fé-hirðir sat á haugi, ok kvaddi hann: \epa

\bpb Shirner rode into the Ettinhomes to Gymer’s yards.  Thereat there were hounds fierce and bound before the slope of the paled fence which surrounded Gird’s hall.  He rode up to where a shepherd sat on a mound and greeted him:\epb\epg


\bvg\bva\mssnote{\Regius~11v/2, \AM~2v/4}%
„Sęg þat \alst{h}irðir, \hld\ es ȧ \alst{h}augi sitr &
\ind ok \alst{v}arðar alla \alst{v}ega: &
hvé ek at \alst{a}nd-spilli \hld\ komumk hins \alst{u}nga mans &
\ind fyr \alst{g}ręyjum \alst{G}ymis.“\eva

\bvb “Tell this, O herdsman who sittest on the mound \\
\ind and watchest all the ways, \\
how I to discourse might come with the young girl \ken*{= Gird} \\
\ind past the greyhounds of Gymer.”\evb\evg


\bvg\bva\speakernote{[Hirðir] kvað:}\mssnote{\Regius~11v/4, \AM~2v/5}%
„Hvárt est \alst{f}ęigr, \hld\ eða est \alst{f}ramm ginginn &
\ind [...]; &
\alst{a}nd-spillis vanr \hld\ þú skalt \alst{ę́} vesa &
\ind \edtrans{\alst{g}óðrar męyjar}{good maiden}{\Bfootnote{Formulaic, carrying with it a sense of chastity.  See note to \textlink{Havamal}[102]/1 for further occurrences.}} \alst{G}ymis.“\eva

\bvb\speakernoteb{[The herdsman] quoth:}%
“Whether thou art fey, or gone forth [dead] \\
\ind {[...]} \\
discourse-less shalt thou always be \\
\ind with the good maiden of Gymer \ken*{= Gird}.”\evb\evg


\bvg\bva\speakernote{[Skírnir] kvað:}\mssnote{\Regius~11v/6, \AM~2v/7}
\Ballnote{An excellent formulation of the Old Germanic fatalism, according to which one’s course of life is determined at birth.  Presumably after uttering these words Shirner rides through the fire surrounding the fortress.}%
„\edtrans{\alst{K}ostir}{Choices}{\Bfootnote{i.e. ‘alternatives, other ways’.}} ’ro bętri \hld\ \edtrans{an}{than}{\Afootnote{so \AM; \emph{hęldr an at} ‘rather than to [be]’ \Regius}} \alst{k}løkkva séi &
\ind hvęim es \alst{f}úss es \alst{f}ara, &
\edtrans{\alst{ęi}nu dǿgri \hld\ mér vas \alst{a}ldr of skapaðr}{In one half-day my age was shaped}{\Bfootnote{Formulaic.}} &
\ind ok \edtrans{alt \alst{l}íf of \alst{l}agit}{all my life laid down}{\Bfootnote{The causative \emph{lęgja} ‘to lay (down, in place)’ is closely connected to fate; the expression is formulaic.  Cf. \textlink{Lokasenna} 48: \emph{ï ár-daga vas þér hit ljóta líf of lagit} ‘in days of yore was thy ugly life laid down’ and \textlink{Voluspa} 19: \emph{þę́r lǫg lǫgðu} ‘they [= the Norns] laid down laws’.}}.“\eva

\bvb\speakernoteb{[Shirner] quoth:}%
“Choices are better than sobbing might be \\
\ind for whomever is eager to journey. \\
In one half-day my age was shaped, \\
\ind and all my life laid down.”\evb\evg


\bvg\bva\speakernote{[Gęrðr] kvað:}\mssnote{\Regius~11v/7, \AM~2v/8}%
„Hvat ’s þat \alst{h}lym \alst{h}lymja \hld\ es \alst{h}lymja hęyri’k nú til &
\ind \alst{o}ssum rǫnnum \alst{ï}? &
\alst{jǫ}rð bifask, \hld\ en \alst{a}llir fyr &
\ind skjalfa \alst{g}arðar \alst{G}ymis.“\eva

\bvb\speakernoteb{[Gird] quoth:}%
“What is this din of dins which I now hear dinning \\
\ind in our houses? \\
The earth trembles and before us quake \\
\ind all Gymer’s yards.”\evb\evg


\bvg\bva\speakernote{Ambǫ́tt kvað:}\mssnote{\Regius~11v/9, \AM~2v/10}%
„\alst{M}aðr ’s hér úti, \hld\ stiginn af \alst{m}ars baki, &
\ind \edtrans{\alst{jó} lę́tr til \alst{ja}rðar taka}{he lets his steed graze the earth}{\Bfootnote{Lit. “he lets his steed take to the earth”.  According to \textcite{FinnurEdda} this expression is still used in modern Icelandic (or was, at his time).}}.“\eva

\bvb\speakernoteb{A servant-woman quoth:}%
“A man is here outside stepped down from horse-back; \\
\ind he lets his steed graze the earth.”\evb\evg


\bvg\bva\speakernote{[Gęrðr] kvað:}\mssnote{\Regius~11v/10, \AM~2v/11}%
„\alst{I}nn bið hann ganga \hld\ ï \alst{o}kkarn sal &
\ind ok drekka hinn \alst{m}ę́ra \alst{m}jǫð, &
þó ek hitt \alst{ó}umk, \hld\ at hér \alst{ú}ti séi &
\ind mïnn \alst{b}róður-\alst{b}ani.“\eva

\bvb\speakernoteb{[Gird] quoth:}%
“Bid him to go in into our hall \\
\ind and drink the renowned mead; \\
though I fear that here outside might be \\
\ind my brother’s bane.”\evb\evg


\bvg\bva\speakernote{[Gęrðr] kvað:}\mssnote{\Regius~11v/12, \AM~2v/13}%
„Hvat ’s þat \alst{a}lfa \hld\ né \alst{ȧ}sa sona, &
\ind né \alst{v}íssa \alst{v}ana; &
hví \alst{ęi}nn of komt \hld\ \alst{ęi}kinn fúr yfir &
\ind ór \alst{s}al-kynni at \alst{s}éa?“\eva

\bvb\speakernoteb{[Gird quoth:]}%
“What kind is this of Elves, nor of sons of the Eese, \\
\ind nor of wise Wanes? \\
Why camest thou alone o’er the raging fire, \\
\ind to see the state of our hall?”\evb\evg


\bvg\bva\speakernote{[Skírnir kvað:]}\mssnote{\Regius~11v/14}%
„\alst{E}m’k⸗at \alst{a}lfa \hld\ né \alst{ȧ}sa sona &
\ind né \alst{v}íssa \alst{v}ana, &
þó \alst{ęi}nn of kom’k \hld\ \alst{ęi}kinn fúr yfir &
\ind yður \alst{s}al-kynni at \alst{s}éa.\eva

\bvb\speakernoteb{[Shirner quoth:]}%
“I am not of Elves, nor of sons of the Eese, \\
\ind nor of wise Wanes— \\
yet I came alone o’er the raging fire, \\
\ind to see the state of your hall.\evb\evg


\bvg\bva\mssnote{\Regius~11v/15, \AM~2v/14}%
\Ballnote{Shirner begins his coercion of Gird, which takes up sts. 19–36.
It has been noted by \textcite{Brate1913} that the structure of this coercion bears some resemblance to Weden’s seduction of Wrind in \textcite{Saxo} 3.4.1–8.  Shirner first offers Gird gold (sts. 19–22), then threatens her with violence (23–25) and finally curses her with a long spell (26–36), which breaks her.  In Weden’s seduction of Wrind he disguises as a warrior, then as a gold-smith, then again as a warrior but is spurned each time.  Finally he makes out to be a witch, and having tied Wrind down, proceeds to rape her (cf. \textlink{Voluspa}[31] n.)  If the first disguise is omitted, we see in both texts a three-fold progression consisting of giving gold, using force of arms, and at last employing magic.}%
\edtrans{\alst{Ę}pli \alst{ę}llifu}{Eleven apples}{\Bfootnote{Probably the apples of Idun, which grant perpetual youth to the gods.  It is not clear whether Shirner has the apples in his possession or whether he is merely promising big.}} \hld\ hér hef’k \alst{a}l-gollin, &
\ind þau mun’k þér \alst{G}ęrðr \alst{g}efa, &
\alst{f}rið at kaupa, \hld\ at þú þér \alst{F}ręy kveðir &
\ind ȯ·\alst{l}ęiðastan \edtrans{at \alst{l}ifa}{in life}{\Bfootnote{\emph{at lifa} here seems to mean ‘in life/living’ rather than the typical infinitive sense ‘to live’; cf. st. 22 \emph{at dęila} ‘in sharing’ below.  This may be an archaism.}}.“\eva

\bvb Eleven apples have I here, all-golden; \\
\ind those will I to thee, Gird, give \\
to buy thy love, that thou callest Free for thee \\
\ind most unloathsome [lovely] in life.”\evb\evg


\bvg\bva\speakernote{[Gęrðr] kvað:}\mssnote{\Regius~11v/17, \AM~2v/15}%
„\alst{Ę}pli \alst{ę}llifu \hld\ ek þigg \alst{a}ldri⸗gi &
\ind at \alst{m}anns-kis \alst{m}unum, &
né vit \alst{F}ręyr, \hld\ meðan okkart \alst{f}jǫr lifir, &
\ind \alst{b}yggjum \alst{b}ę́ði saman.“\eva

\bvb\speakernoteb{[Gird quoth:]}%
“Eleven apples will I never take \\
\ind to any man’s liking, \\
nor will I and Free—while our life remains— \\
\ind dwell both together.”\evb\evg


\bvg\bva\speakernote{[Skírnir kvað:]}\mssnote{\Regius~11v/19, \AM~2v/17 (ll. 1–2)}%
„\edtrans{\alst{B}aug}{The bigh}{\Bfootnote{While not named, the bigh is clearly Dreepner as known from \Gylfaginning\ 49, which describes Balder’s funeral: \emph{Óðinn lagði á bál’it gull-hring þann, er Draupnir heitir.  Hónum fylgði sú náttúra, at ina níundu hverja nótt drupu af hónum átta gull-hringar jafn-hǫfgir.} ‘Weden laid on the pyre that gold ring which is called Dreepner.  Its nature was such that every ninth night eight even-heavy golden rings dripped from it.’  This passage probably draws on the present stanza.  When \inx[P]{Harmod} later came to \inx[L]{Hell} to try to bring Balder back, Balder told him to bring the bigh back to Weden as a token by which to remember him.}} þér þȧ gef’k, \hld\ þann’s \alst{b}ręndr of vas &
\ind með \alst{u}ngum \alst{Ó}ðins syni; &
\edtext{\alst{á}tta ’ro \alst{ja}fn-hǫfgir, \hld\ es \alst{a}f drjúpa &
\ind hina \alst{n}íundu hvęrja \alst{n}ǫ́tt.“}{\lemma{átta \dots\ nǫ́tt ‘Eight \dots\ night.’}\Bfootnote{In \AM\ these lines and 22:1–2 are missing.  Instead 1–2 here and 22:3–4 are combined into one.}}\eva

\bvb\speakernoteb{[Shirner quoth:]}%
“The \inx[C]{bigh} I then give thee, which was burned \\
\ind with Weden’s young son \ken*{= Balder}. \\
Eight are the even-heavy ones which from it drip \\
\ind every ninth night.”\evb\evg


\bvg\bva\speakernote{[Gęrðr] kvað:}\mssnote{\Regius~11v/21, \AM~2v/18 (ll. 3–4)}%
„\alst{B}aug þikk⸗a’k, \hld\ þó’tt \alst{b}ręndr séi, &
\ind með \alst{u}ngum \alst{Ó}ðins syni; &
es⸗a mér \alst{g}olls vant \hld\ ï \alst{g}ǫrðum \alst{G}ymis &
\ind at dęila \alst{f}é \alst{f}ǫður.“\eva

\bvb\speakernoteb{[Gird quoth:]}%
“The bigh I will not take, though it may have been burned \\
\ind with Weden’s young son. \\
I lack no gold in Gymer’s yards \\
\ind partaking of the money of my father.”\evb\evg


\bvg\bva\speakernote{[Skírnir kvað:]}\mssnote{\Regius~11v/23, \AM~2v/19}%
„Sér þú \edtext{\alst{m}ę́ki, \alst{m}ę́r, \hld\ \alst{m}jóvan, \alst{m}ál-fáan}{\lemma{mę́ki \dots\ mál-fáan ‘sword \dots\ picture-painted’}\Bfootnote{The sword is inlaid with metal (perhaps gold or silver) forming a pattern.  The expression \emph{mę́ki mál-fáan} (acc. sg.) ‘picture-painted sword’ also occurs in \textlink{Brot}[4]/2.}}, &
\ind es \alst{h}ęf’k ï \alst{h}ęndi \alst{h}ér? &
\alst{h}ǫfuð \alst{h}ǫggva \hld\ mun’k þér \alst{h}alsi af, &
\ind nema mér \alst{s}ę́tt \alst{s}ęgir.“\eva

\bvb\speakernoteb{[Shirner quoth:]}%
“Seest thou the sword, maiden—slender, picture-painted— \\
\ind which I have in my hand here? \\
Cut the head will I from thy neck \\
\ind unless thou agree with me.”\evb\evg


\bvg\bva\speakernote{[Gęrðr kvað:]}\mssnote{\Regius~11v/25, \AM~2v/20}%
„\alst{Ȧ}·nauð þola \hld\ vil’k \alst{a}ldri⸗gi &
\ind at \edtrans{\alst{m}anns-kis}{any man’s (lit. ‘no man’s)}{\Afootnote{\emph{manns ęnskis} \AM}} \alst{m}unum, &
þó hins \alst{g}et’k, \hld\ ef it \alst{G}ymir finniðsk &
\alst{v}ígs ȯ·trauðir \hld\ at ykkr \alst{v}ega tíði.“\eva

\bvb\speakernoteb{[Gird quoth:]}%
“Stand coercion I never will \\
\ind to any man’s liking; \\
though I get that if thou and Gymer meet— \\
men unreluctant of conflict—ye two will wish to fight.”\evb\evg


\bvg\bva\speakernote{[Skírnir kvað:]}\mssnote{\Regius~11v/27, \AM~2v/22}%
„Sér þú \alst{m}ę́ki, \alst{m}ę́r, \hld\ \alst{m}jóvan, \alst{m}ál-fáan, &
\ind es \alst{h}ęf’k ï \alst{h}ęndi \alst{h}ér? &
fyr þęssum \alst{ę}ggjum \hld\ hnígr sá hinn \alst{a}ldni jǫtunn, &
\ind verðr þinn \alst{f}ęigr \alst{f}aðir.\eva

\bvb\speakernoteb{[Shirner quoth:]}%
“Seest thou the sword, maiden—slender, picture-painted— \\
\ind which I have in my hand here? \\
By these edges that aged ettin \ken*{= Gymer} sinks; \\
\ind \inx[C]{fey} becomes thy father.\evb\evg


\bvg\bva\mssnote{\Regius~11v/28, \AM~2v/24}%
\Ballnote{With this stanza Shirner’s turns away from threats of violence and begins his long curse (sts. 26–36).  He curses Gird to be loveless, captive, deprived of joy, and the subject of sexual abuse at the hands of loathsome ettins if she does not accede to love Free.}%
\edtrans{\alst{T}ams-vęndi}{taming-wand}{\Bfootnote{Has been interpreted as a sword.  TODO.  The imagery is phallic.}} þik drep’k, \hld\ ęn þik \alst{t}ęmja mun’k, &
\ind \alst{m}ę́r, at mïnum \alst{m}unum, &
þar skalt \alst{g}anga \hld\ es þik \alst{g}umna synir &
\ind \alst{s}íðan ę́va \alst{s}éi.\eva

\bvb With the taming-wand I strike thee—and thee will I tame, \\
\ind O maiden, to my liking! \\
Thou shalt go to where the sons of men \\
\ind never since will see thee!\evb\evg


\bvg\bva\mssnote{\Regius~11v/30, \AM~2v/26}%
\edtrans{\alst{A}ra þúfu \alst{ȧ} \hld\ skalt \alst{á}r sitja}{On an eagle’s perch shalt thou sit for long}{\Afootnote{\emph{ár skalt sitja \hld\ ara þúfu ȧ} ‘for long shalt thou sit on an eagle’s perch’ \AM}\Bfootnote{Namely as a captive of said eagle.}}, &
\ind \edtext{\alst{h}orfa \alst{h}ęimi ór; &
\ind snugga \alst{h}ęljar til}{\lemma{horfa hęimi ór; snugga hęljar til ‘turn out of the world; hanker after Hell’}\Afootnote{\emph{horfa ok snugga hęljar til} ‘turn and hanker after Hell’ \AM}\Bfootnote{i.e. “you will look toward and yearn for the underworld”.}}; &
\alst{m}atr sé þér męir lęiðr \hld\ an \alst{m}anna hvęim &
\ind hinn \alst{f}rȧni ormr með \edtext{\alst{f}irum}{\Bfootnote{This is the last word on fol. 2v of \AM, after which the text cuts off.  Apart from the very last stanza, the rest of the poem is preserved only in \Regius.}}.\eva

\bvb On an eagle’s perch shalt thou sit for long, \\
\ind turn away from the world, \\
\ind hanker after \inx[L]{Hell}! \\
Be thy food more loathsome than to any man \\
\ind the gleaming wyrm \ken*{= the Middenyardswyrm} among folk.\footnoteB{Her food will be more disgusting than the \inx[C]{Middenyardswyrm}, for which cf. \textlink{Hymiskvida} 22.}\evb\evg


\bvg\bva\mssnote{\Regius~11v/32}%
At \alst{u}ndr-sjónum verðir \hld\ es \alst{ú}t of kømr, &
\ind ȧ þik \alst{H}rímnir \alst{h}ari &
\ind ȧ þik \alst{h}ot-vetna stari, &
\alst{v}íð-kunnari \alst{v}erðir \hld\ an \alst{v}ǫrðr með goðum, &
\ind \alst{g}api þú \alst{g}rindum frȧ.\eva

\bvb A wondrous sight mayst thou be when thou comest out; \\
\ind at thee may Rimner ogle; \\
\ind at thee may anyone stare! \\
Mayst thou be more widely known than the watchman with the Gods \ken*{= Homedal}; \\
\ind mayst thou gape from the gates!\evb\evg


\bvg\bva\mssnote{\Regius~12r/2}%
\edtrans{\alst{T}ópi ok ópi, \hld\ \alst{t}jǫsull ok ȯ·þoli}{Toop and woop, tarsle and restlessness}{\Bfootnote{The first three words are magic curse words without clear meaning; I have left them untranslated.  \emph{tjǫsull} may perhaps be related to OE \emph{teors} ‘penis’ and mean ‘little phallus’.}}, &
\ind vaxi þér \alst{t}ǫ́r með \alst{t}rega; &
\alst{s}ętsk þú niðr \hld\ en mun’k \alst{s}ęgja þér &
\ind \alst{s}váran \alst{s}ús-breka, &
\ind ok \alst{t}vinnan \alst{t}rega.\eva

\bvb Toop and woop, tarsle and restlessness— \\
\ind may thy tears grow with grief! \\
Sit thyself down and I will tell thee \\
\ind a heavy roaring-breaker, \\
\ind and a twined grief.\evb\evg


\bvg\bva\mssnote{\Regius~12r/3}%
Tramar \alst{g}nęypa \hld\ þik skulu \alst{g}ęrstan dag &
\ind \alst{jǫ}tna gǫrðum \alst{ï}, &
til \alst{h}rím-þursa \alst{h}allar \hld\ þú skalt \alst{h}vęrjan dag &
\ind \alst{k}ranga \alst{k}osta-laus; &
\ind \alst{k}ranga \alst{k}osta-vǫn; &
\alst{g}rát at \alst{g}amni \hld\ skalt ï \alst{g}ǫgn hafa &
\ind ok lęiða með \alst{t}ǫ́rum \alst{t}rega.\eva

\bvb Fiends shall pine thee on a gloomy day \\
\ind in the yards of the Ettins. \\
To the hall of Rime-Thurses shalt thou every day \\
\ind crawl choice-less; \\
\ind crawl choice-lacking. \\
Weeping shalt thou have in exchange for joy, \\
\ind and nurse grief with tears.\evb\evg


\bvg\bva\mssnote{\Regius~12r/7}%
Með \edtrans{\alst{þ}ursi \alst{þ}rí-hǫfðuðum}{three-headed thurse}{\Bfootnote{Ettins often have an abnormal number of body parts.  For their “manyheadedness” see note to \textlink{Hymiskvida} 8/2.}} \hld\ \alst{þ}ú skalt ę́ nara &
\ind eða \alst{v}er-laus \alst{v}esa; &
\ind þitt \alst{g}ęð \alst{g}rípi, &
\ind þik \alst{m}orn \alst{m}orni; &
\edtrans{ves þú sem \alst{þ}istill}{be thou like the thistle}{\Bfootnote{The thistle was apparently held to be a worthless plant; cf. the English galder against a cattle-thief (Charm IX in margins of CCCC 41. TODO: edit this!) cursing him to be \emph{swá bréðel swa séo þystel} ‘as wretched as the thistle’.}}, \hld\ sá’s \alst{þ}runginn vas &
\ind ï \alst{o}fan-verða \alst{ǫ}nn.\eva

\bvb With a three-headed thurse shalt thou forever live \\
\ind or be husband-less. \\
\ind May thy senses seize; \\
\ind may murrain mourn thee; \\
be thou like the thistle that was pressed \\
\ind during highest harvest!\evb\evg


\bvg\bva\mssnote{\Regius~12r/9}%
Til \alst{h}olts ek gekk \hld\ ok \edtrans{til \alst{h}rás viðar}{to the raw/sappy tree}{\Bfootnote{The wood of a sapling was apparently thought to be the most effective for magic; cf. \textlink{Havamal} 152, which speaks about a runic curse carved on \emph{rótum rás viðar} ‘the roots of a raw/sappy tree’.}} &
\ind \edtrans{\alst{g}amban-tęin}{gombentoe}{\Bfootnote{Perhaps “mighty twig”.  A compound consisting of the very rare word \emph{gamban} ‘magic/curse?’ and \emph{tęinn} ‘twig, branch’ (cf. \emph{mistil-tęinn} ‘mistle-toe’).  This may be the stick on which the runic curse in st. 36 below should be carved, or it is to be identified with the \emph{tams-vǫndr} ‘taming-wand’ of st. 26 above.}} at \alst{g}eta &
\ind \alst{g}amban-tęin ek \alst{g}at.\eva

\bvb To the wood I went and to the raw/sappy tree, \\
\ind the \inx[C]{gombentoe} for to get; \\
\ind the gombentoe I got.\evb\evg


\bvg\bva\mssnote{\Regius~12r/10}%
\alst{R}ęiðr ’s þér Óðinn, \hld\ \alst{r}ęiðr ’s þér Ȧsa-bragr, &
\ind þik skal \alst{F}ręyr \alst{f}íask, &
hin \alst{f}irin-illa mę́r, \hld\ en \alst{f}ingit hęfr &
\ind \alst{g}amban-ręiði \alst{g}oða.\eva

\bvb Wroth with thee is Weden; wroth with thee is Eesebray \name{= Thunder}; \\
\ind thee shall Free come to hate, \\
O most wicked maiden, if thou hast earned \\
\ind the gomben-wrath of the gods.\evb\evg


\bvg\bva\mssnote{\Regius~12r/12}%
\alst{H}ęyri jǫtnar, \hld\ \alst{h}ęyri \alst{h}rím-þursar, &
\alst{s}ynir \alst{S}uttunga, \hld\ \alst{s}jalfir ǫ̇s-liðar, &
hvé \alst{f}yrir býð’k, \hld\ hvé \alst{f}yrir banna’k &
\ind \alst{m}anna glaum \alst{m}ani, &
\ind \alst{m}anna nyt \alst{m}ani.\eva

\bvb Hear may Ettins; hear may Rime-thurses, \\
sons of Sutting, the very Os-troops \ken*{= Eese}— \\
how I forbid, how I forban \\
\ind men’s fellowship from the maid, \\
\ind men’s joy from the maid!\evb\evg


\bvg\bva\mssnote{\Regius~12r/14}%
\alst{H}rím-grímnir hęitir þurs, \hld\ es þik \alst{h}afa skal &
\ind fyr \alst{n}á-grindr \alst{n}eðan, &
þar þér \alst{v}íl-męgir \hld\ ȧ \alst{v}iðar rótum &
\ind \alst{g}ęita-hland \alst{g}efi; &
\alst{ǿ}ðri drykkju \hld\ fȧ þú \alst{a}ldri⸗gi, &
\ind \alst{m}ę́r, af þïnum \alst{m}unum, &
\ind \alst{m}ę́r, at \alst{m}ïnum \alst{m}unum.\eva

\bvb Rimegrimner is called the thurse who shall have thee \\
\ind down beneath Neegrind, \\
where lads of toil \ken{thralls} on the roots of a tree \\
\ind goat-piss will give thee. \\
A finer drink do never get, \\
\ind O maiden, against thy liking, \\
\ind O maiden, to my liking!\evb\evg


\bvg\bva\mssnote{\Regius~12r/16}%
\Ballnote{With the carving of the rune-stick Shirner threatens to fulfill the curse predicted in sts. 26–35, but tells Gird that he will scrape the runes off (thus nullifying the curse) if she will accept his demands.  She promptly does.}%
\edtrans{\alst{Þ}urs}{thurse}{\Bfootnote{Thurse is the name of the \textbf{þ}-rune (ᚦ); it is carved as part of the curse.}} ríst’k \alst{þ}ér \hld\ ok \edtrans{\alst{þ}ría stafi}{three staves}{\Bfootnote{Three runic letters (or phrases) representing the three following words (\emph{ęrgi} ‘queerness, degeneracy’ etc.).  The ritual practice of carving “three staves” is first found on the C7th Gummarp stone: \textbf{h\textsc{a}þuwol\textsc{a}fʀ s\textsc{a}te st\textsc{a}b\textsc{a} þri\textsc{a} fff} ‘Hathwolf placed three staves: fff’, where the \textbf{f}-rune (ᚠ) stands for its name \inx[C]{fee} (i.e. ‘wealth, cattle’) and is thus meant to bring wealth.}}, &
\ind \edtrans{\alst{ę}rgi ok \alst{ǿ}ði ok \alst{ȯ}·þola}{queerness and madness and restlessness}{\Bfootnote{Both \emph{ęrgi} ‘queerness, degeneracy’ and \emph{ȯ·þoli} ‘restlessness’ (here probably from strong lust) are found in the love magic charm on the rune stick B257 from Bryggen (edited below under Galders).  \emph{ęrgi} is also found in the curse-formula on the C7th Proto-Norse runestones from Stentoften and Björketorp.  See further introduction to B257.}}, &
svá ek þat \alst{a}f ríst \hld\ sem ek þat \alst{ȧ} ręist, &
\ind ef gørask \alst{þ}arfar \alst{þ}ęss.“\eva

\bvb \inx[G]{Thurses}[Thurse] I carve thee and three staves: \\
\ind \inx[C]{queerness} and madness and restlessness.— \\
So I will carve it \emph{off} as I carved it \emph{on}, \\
\ind if there be need for that.”\evb\evg


\bvg\bva\speakernote{[Gęrðr kvað:]}\mssnote{\Regius~12r/19}%
„\edtext{\alst{H}ęill ves þú \alst{h}ęldr, svęinn, \hld\ ok tak við \edtrans{\alst{h}rím-kalki}{rime-chalice}{\Bfootnote{Some kind of expensive glazed drinking vessel; the second element \emph{kalkr} ‘chalice’ is a borrowing from Latin \emph{calix} and suggests a Roman origin.  Cf. the \emph{kalkr} in \textlink{Hymiskvida}[28]/4b.}} &
\ind \alst{f}ullum \alst{f}orns mjaðar,}{\lemma{Hęill \dots\ mjaðar ‘Hale \dots\ mead’}\Bfootnote{Formulaic; repeated identically in \textlink{Lokasenna}[53]/1–2.}} &
þó hafða’k \alst{ę́}tlat, \hld\ at mynda’k \alst{a}ldri⸗gi &
\ind unna \edtrans{\alst{v}aningja}{the Waning \ken*{= Free}}{\Bfootnote{The ‘descendant of the \inx[G]{Wanes}’.  A rare word.  Its only other occurence in the Norse corpus is in a \inx[C]{thule} of boar-names.  Boars were sacred to Free, TODO.}} \alst{v}ęl.“\eva

\bvb\speakernoteb{[Gird quoth:]}%
“Hale be thou rather, swain, and receive the rime-chalice, \\
\ind full of ancient mead, \\
though I had intended that I never would \\
\ind love the Waning \ken*{= Free} well.”\evb\evg


\bvg\bva\speakernote{[Skírnir kvað:]}\mssnote{\Regius~12r/21}%
„\alst{Ø}rendi mïn \hld\ vil’k \alst{ǫ}ll vita, &
\ind áðr ríða’k \alst{h}ęim \alst{h}eðan, &
nę́r ȧ \alst{þ}ingi \hld\ munt hinum \alst{þ}roska &
\ind \alst{n}ęnna \alst{N}jarðar syni?“\eva

\bvb\speakernoteb{[Shirner quoth:]}%
“My errands all I wish to know \\
\ind before I ride home hence: \\
when on the \inx[C]{Thing} wilt thou with the virile \\
\ind son of Nearth \ken*{= Free} be joined?”\evb\evg


\bvg\bva\speakernote{[Gęrðr kvað:]}\mssnote{\Regius~12r/23}%
„\alst{B}arri hęitir, \hld\ es vit \alst{b}ę́ði vitum, &
\ind \alst{l}undr \alst{l}ogn-fara, &
en ępt \alst{n}ę́tr \alst{n}íu, \hld\ þar mun \alst{N}jarðar syni &
\ind \alst{G}ęrðr unna \alst{g}amans.“\eva

\bvb\speakernoteb{[Gird quoth:]}%
“Barrey is called—as we both know— \\
\ind a grove of calm breezes, \\
and after nine nights there will to the son of Nearth \\
\ind Gird her pleasure grant.”\evb\evg


\bpg\bpa\mssnote{\Regius~12r/24}%
Þá reið Skírnir heim.  Freyr stóð úti ok kvaddi hann ok spurði tíðenda:\epa

\bpb Then Shirner rode home.  Free stood outside and greeted him and asked for the tidings:\epb\epg


\bvg\bva\mssnote{\Regius~12r/25}%
„\alst{S}ęg mér, Skírnir, \hld\ áðr verpir \alst{s}ǫðli af mar &
\ind \edtrans{ok stígir \alst{f}eti \alst{f}ramarr}{and take a step further}{\Bfootnote{Formulaic; a variant of \emph{feti ganga/gangir framarr} (\textlink{Havamal} 38/2, \textlink{Lokasenna} 1/2).}}, &
hvat \alst{á}rnaðir \hld\ ï \alst{Jǫ}tun-hęima &
\ind þïns eða \alst{m}ïns \alst{m}unar?“\eva

\bvb “Tell me, Shirner, before thou mightst throw the saddle off the steed \\
\ind and take a step further, \\
what thou accomplished in the \inx[L]{Ettinhomes} \\
\ind to thy or my liking?”\evb\evg


\bvg\bva\speakernote{[Skírnir kvað:]}\mssnote{\Regius~12r/27}%
„\alst{B}arri hęitir, \hld\ es vit \alst{b}áðir vitum, &
\ind \alst{l}undr \alst{l}ogn-fara, &
en ępt \alst{n}ę́tr \alst{n}íu, \hld\ þar mun \alst{N}jarðar syni &
\ind \alst{G}ęrðr unna \alst{g}amans.“\eva

\bvb\speakernoteb{[Shirner quoth:]}%
“Barrey is called—as we both know— \\
\ind a grove of calm breezes, \\
and after nine nights there will to the son of Nearth \\
\ind Gird her pleasure grant.”\evb\evg


\bvg\bva\speakernote{[Fręyr kvað:]}\mssnote{\Regius~12r/28, \RegiusProse\Trajectinus\Upsaliensis\Wormianus}%
„\alst{L}ǫng es nǫ́tt, \hld\ \edtrans{\alst{l}angar ’ro tvę́r}{long are two}{\Afootnote{\emph{lǫng es ǫnnur} ‘long is another’ \RegiusProse\Trajectinus\Upsaliensis\Wormianus}}, &
\ind \edtext{hvé of \alst{þ}ręyja’k \alst{þ}ríar?}{\Afootnote{\emph{hvé męga’k þręyja þríar} \RegiusProse\Trajectinus\Upsaliensis\Wormianus}} &
opt \alst{m}ér \alst{m}ȧnaðr \hld\ \alst{m}inni þȯtti &
\ind an sjá \edtrans{\alst{h}ǫlf \alst{h}ý-nǫ́tt}{half wedding-night}{\Bfootnote{The wedding-night is presumably “half” (here meaning “incomplete”) as it is not consumated.}}.“\eva

\bvb\speakernoteb{[Free quoth:]}%
“Long is a night, long are two— \\
\ind how can I yearn for three? \\
Oft a month to me seemed less \\
\ind than this half wedding-night!”\evb\evg

\sectionline
