\bookStart{Thule of Righ}[Rígs þula]
\setBookCode{Rigsthula}

\begin{flushright}%
\textbf{Dating} \parencite{Sapp2022}: early C11th (0.240), late C11th (0.204), late C12th (0.195), C13th (0.280)

\textbf{Meter:} \Fornyrdislag%
\end{flushright}

\section{Introduction}

\subsection{Preservation}

The \textbf{Thule of Righ} (\textlink{Rigsthula}) is only preserved in a single leaf in the 14th century ms. \Wormianus, where it follows the Prose Edda and four grammatical treatises written in the same hand as the poem.  Numerous leaves are unfortunately missing from \Wormianus, and among them is conclusion to \Rigsthula.
%NOTE: Diplomatic edition can be found here https://clarino.uib.no/menota/text/menota/AM-242-fol

\subsection{Contents}

\Rigsthula\ is a powerful and, due to its incompleteness, enigmatic poem.  It takes the form of a folktale that serves as an etiology for the origin of the Wiking Age Norwegian class system and a celebration of the superiority of the warrior-nobility.

In the poem the mysterious Righ (ON \emph{Rígr}) fathers in succession three human sons with three women, Great-grandmother, Grandmother, and Mother.  Their respective sons become Thrall, Churl, and Earl, each representing one of three classes: the slaves, the yeomen or freeholders, and the nobility.  The youngest of the three, Earl, in turn fathers twelve sons, the youngest of whom, Kin the young, becomes the first to hold the title of king.

TODO: Dumezil three-part society, racial caste system, Irish influence. Many interesting things to write here!

\subsubsection{The identity of Righ}

The figure Righ, identified in the prose introduction as Homedall, is rather obscure.  His name has traditionally been seen as a sign of Irish influence on the basis of Old Irish \emph{rí} (oblique sg. stem \emph{ríg-}) ‘king’.  He does not appear in any other poetic source, but his name is found in two related genealogies in \YnglingaSaga\ 17 and in the epitome of the lost \emph{Skjǫldunga saga}.  In \YnglingaSaga\ 17 we read:

\begin{quote}
  \emph{Móðir Dyggva var Drótt, dóttir Danps konungs, sonar Rígs, er fyrstr var konungr kallaðr á danska tungu. [...] Dyggvi var fyrst konungr kallaðr sinna ę́tt-manna. [...] Drótt drótning var systir Dans konungs ins mikil-láta, er Dan-mǫrk er við kend.}

  ‘Due’s mother was Dright, the daughter of king Danp, the son of Righ, who was the first to be called king in [the geographical extent of] the Danish tongue. [...] Due was the first of his kinsmen to be called king. [...] Queen Dright was the sister of king Dene the proud, after whom Denmark is named.’
\end{quote}

In the epitome of the \emph{Skjǫldunga saga} we read:

\begin{quote}
  \emph{TODO}

  ‘TODO.’
\end{quote}

These genealogies, although conflicting, are clearly related to an episode we only get a hint about at the very end of the poem (st.~44), namely Kin the young’s visit to Dene and Danp.  In all cases they serve to justify the origins of monarchy, and this was probably the function of \Rigsthula\ as well.  Having already laid out the origins of the thralls, churls, and earls, it seems inevitable that the remaining sts.~would have dealt with the establishment of monarchy, and probably specifically the Norwegian monarchy, which through Harold Fairhair claimed descent from the House of the Inglings, of which Due is the seventh king listed in \YnglingaSaga.

Righ’s role in the poem is really much more like that of Weden; Righ, like, Weden is an aristocrat, an inciter of war, and a teacher of runes; like Weden he bestows favour on his protegé, who tends to be the youngest son (cf.~\textlink{Grimnismal}[P1]–P2).  The prose may thus be drawing on \textlink{Voluspa}[1]/2, where \emph{męgir Hęim·dalar} ‘sons of Homedall’ uniquely seems to refer to men.  It is of course not impossible that Homedall also had the aforementioned Odinic traits, but we learn of no such thing from other sources.  In conclusion, Righ is more likely to be identified with Weden than Homedall.

\subsubsection{Language and meter}

The poem itself is difficult to date, but comes off as rather late.  Metrically it is unusual for its very high incidence of 3-syllable verses, which are generally not allowed in Norse \Fornyrdislag.  The language is unusually formulaic and repetitive, with many repeated expressions, lines and even stanzas.  These are e.g. \emph{męirr at þat} ‘further after that’, TODO.

\subsubsection{Archeology}

\Rigsthula\ contains several descriptions of dress and lifestyle which may be considered archeologically, particularly in sts. 15–16, 25–26, 28–29, 32.  On this basis \textcite{Nerman1969} estimates the compoqti

\newpage

\section{Text}

\bpg\bpa\mssnote{\Wormianus~78r/1–3}%
Svá segja menn í fornum sǫgum, at einn-hverr af ǫ́sum, sá er Heim·dallr hét, fór ferðar sinnar ok framm með sjóvar-strǫndu nǫkkurri, kom at einum húsa-bǿ ok nefndi⸗st Rígr; eptir þeiri sǫgu er kvę́ði þetta.\epa

\bpb {\huge S}\textsc{o do men say in ancient \inx[C]{saw}[saws]}, that one of the \inx[P]{Eese}—he who was called \inx[P]{Homedall}—went on his journey and, passing forth along a certain lake shore, came upon a lone homestead and called himself Righ.  According to that saw is this poem.\epb\epg


\bvg\bva\mssnote{\Wormianus~78r/4–5}%
\edtrans{Ár}{Of yore}{\Afootnote{emend.; \emph{‘at’} \Wormianus}\Bfootnote{Formulaic; it is very common for poems to begin with \emph{ár} ‘of yore, in the beginning’.  Cf. \textlink{Voluspa}[3]/1, \textlink{Hymiskvida}[1]/1, \textlink{HelgakvidaOne}[1]/1, \textlink{GudrunOne}[1]/1, \textlink{Sigurdskamma}[1]/1.}} kvǫ́ðu \alst{g}anga \hld\ \alst{g}rø̇nar brautir &
\alst{ǫ}flgan ok \alst{a}ldinn \hld\ \alst{ǫ̇}s kunnigan, &
\alst{r}amman ok \alst{r}ǫskvan \hld\ \alst{R}íg stíganda.\eva

\bvb {\huge O}\textsc{f yore, they said}, did walk on green roads \\
a mighty and ancient \inx[P]{Eese}[os], cunning: \\
the strong and brisk Righ, striding.\evb\evg


\bvg\bva\mssnote{\Wormianus~78r/5–6}%
Gekk \alst{m}ęirr at þat \hld\ \alst{m}iðrar brautar, &
kom hann at \alst{h}úsi, \hld\ \alst{h}urð vas ȧ gę̇tti; &
\alst{i}nn nam at ganga, \hld\ \alst{ę}ldr vas ȧ golfi, &
\alst{h}jȯn sǫ́tu þar \hld\ \alst{h}ǫ́r \edtext{at}{\Afootnote{sens. emend.; \emph{af} \Wormianus}} arni, &
\alst{Á}i ok \alst{Ę}dda \hld\ \alst{a}ldin-falda.\eva

\bvb He went further after that in the middle of the road, \\
came to a house—the door was wide open. \\
He took to go inside, fire was on the floor, \\
a couple sat there, hoary by the hearth: \\
Great-Grandpa and Great-Grandma in an old-time shawl.\evb\evg


\bvg\bva\mssnote{\Wormianus~78r/7–8}%
\alst{R}ígr kunni þęim \hld\ \alst{r}ǫ́ð at sęgja; &
\alst{m}ęirr sętti⸗sk hann \hld\ \alst{m}iðra flętja &
ęn ȧ \alst{h}lið \alst{h}vára \hld\ \alst{h}jȯn sal-kynna.\eva

\bvb Righ knew to tell them counsels, \\
further he set himself down on the middle of the bench, \\
and on either side—the couple of the hall.\evb\evg


\bvg\bva\mssnote{\Wormianus~78r/8–10}%
Þȧ tók \alst{Ę}dda \hld\ \alst{ø}kkvinn hlęif, &
\alst{þ}ungan ok \alst{þ}ykkvan, \hld\ \alst{þ}runginn sǫ́ðum, &
bar hǫ̇n \alst{m}ęirr at þat \hld\ \alst{m}iðra skutla, &
\alst{s}oð vas ï bolla \hld\ \alst{s}ętti ȧ bjóð; &
vas \alst{k}alfr soðinn \hld\ \alst{k}rása bętstr; &
\alst{r}ęis hann upp þaðan, \hld\ \alst{r}éð⸗sk at sofna;\eva

\bvb Then Great-Grandma took a lumpy loaf— \\
heavy and thick, stuffed with chaff— \\
carried it further after that in the middle of a trencher; \\
broth was in a bowl—she set it on a platter. \\
A cooked calf was the best dainty; \\
he \ken*{= Righ} rose up thence, resolved to sleep.\evb\evg


\bvg\bva\mssnote{\Wormianus~78r/10–11}%
\alst{R}ígr kunni þęim \hld\ \alst{r}ǫ́ð at sęgja; &
\alst{m}ęirr lagði⸗sk hann \hld\ \alst{m}iðrar rękkju, &
en ȧ \alst{h}lið \alst{h}vára \hld\ \alst{h}jón sal-kynna.\eva

\bvb Righ knew to tell them counsels; \\
further he laid himself down in the middle of the bed, \\
and on either side—the couple of the hall.\evb\evg


\bvg\bva\mssnote{\Wormianus~78r/11–12}%
\alst{Þ}ar vas hann at þat \hld\ \alst{þ}rjár nę́tr saman; &
gekk hann \alst{m}ęirr at þat \hld\ \alst{m}iðrar brautar; &
liðu \alst{m}ęirr at þat \hld\ \alst{m}ǫ̇nuðr níu.\eva

\bvb There he was after that for three nights amidst them; \\
he went further after that in the middle of the road; \\
passed further after that nine months.\evb\evg


\bvg\bva\mssnote{\Wormianus~78r/12}%
\alst{Jó}ð \alst{ó}l \alst{Ę}dda, \hld\ \edtrans{\alst{jó}su vatni}{they sprinkled it with water}{\Bfootnote{A reference to the Heathen naming ceremony wherein water would be poured on a newborn, somewhat resembling the Christian baptism.  See \textlink{Havamal}[156].}} &
\edtrans{\alst{h}ǫrund-svartan}{swarthy of skin}{\Afootnote{emend.; \emph{hǫrfi svartan} ‘swarthy with flax(?)’ \Wormianus}}, \hld\ \alst{h}étu Þrę́l.\eva

\bvb Great-Grandma begot a child—they sprinkled it with water: \\
swarthy of skin, they called it Thrall.\evb\evg


\bvg\bva\mssnote{\Wormianus~78r/12–14}%
Hann nam at \alst{v}axa \hld\ ok \alst{v}ęl dafna; &
vas þar ȧ \alst{h}ǫndum \hld\ \alst{h}rokkit skinn, &
\alst{k}ropnir \alst{k}núar, \hld\ [...] &
\alst{f}ingr digrir, \hld\ \alst{f}úl⸗ligt and-lit, &
\alst{l}otr hryggr, \hld\ \alst{l}angir hę̇lar.\eva

\bvb He took to grow and have it well; \\
there on his hands was wrinkled skin, \\
crooked knuckles, [...], \\
stubby fingers, loathsome face, \\
stooping back, long heels.\evb\evg


\bvg\bva\mssnote{\Wormianus~78r/14–15}%
Nam \alst{m}ęirr at þat \hld\ \alst{m}agns of kosta, &
\alst{b}ast at \alst{b}inda, \hld\ \alst{b}yrðar gørva; &
bar \alst{h}ęim at þat \hld\ \alst{h}rís gęrstan dag.\eva

\bvb He took further after that to try his strength: \\
bast to bind, burdens to make; \\
he carried home after that brushwood on a gloomy day.\evb\evg


\bvg\bva\mssnote{\Wormianus~78r/15–16}%
Þar kom at \alst{g}arði \hld\ \edtrans{\alst{g}ęngil-bęina}{gangle-boned girl}{\Bfootnote{Derogatory, somebody who (due to poverty) only travels by foot.}}, &
\alst{au}rr vas ȧ \alst{i}ljum, \hld\ \alst{a}rmr sól-brunninn, &
\alst{n}iðr-bjúgt es \alst{n}ęf, \hld\ \alst{n}ęfndi⸗sk \edtrans{Þír}{Thew}{\Bfootnote{The name probably means ‘maid-servant’ or ‘female slave’. Unlike Thrall, it is not attested in any prose texts, but probably corresponds to OS \emph{thiwi} ‘maid(-servant)’, being further root-related to \emph{þéa \char`~\ þjá} ‘to enthral’, Proto-Norse \textbf{þewaʀ} ‘servant’, OE \emph{þéow} ‘slave, servant’,.}}.\eva

\bvb There to the farm came a gangle-boned girl: \\
mud was on her footsoles, her arm sunburnt, \\
downturned her nose; she called herself Thew.\evb\evg


\bvg\bva\mssnote{\Wormianus~78r/16–17}%
\edtext{\alst{M}ęirr sętti⸗sk hǫ̇n \hld\ \alst{m}iðra flętja}{\Afootnote{emend. based on other sts.; \emph{miðra flętja \hld\ męirr sętti⸗sk hǫ̇n} \Wormianus}}, &
\alst{s}at hjá hęnni \hld\ \alst{s}onr húss, &
\alst{r}ǿddu ok \alst{r}ẏndu, \hld\ \alst{r}ękkju gørðu &
\alst{Þ}rę́ll ok \alst{Þ}ír \hld\ \alst{þ}rungin dǿgr.\eva

\bvb Further she set herself down on the middle of the bench; \\
by her sat the son of the house \ken*{= Thrall}. \\
They spoke and whispered, made a bed— \\
Thrall and Thew—in hard-pressed nights.\evb\evg


\bvg\bva\mssnote{\Wormianus~78r/17–19}%
\alst{B}ǫrn ólu þau, \hld\ \alst{b}juggu ok unðu; &
\alst{h}ygg’k at \alst{h}éti \hld\ \alst{H}ręimr ok Fjósnir, &
\alst{K}lúrr ok \alst{K}lęggi, \hld\ \alst{K}ęfsir, Fúlnir, &
\alst{D}rumbr, \alst{D}igraldi, \hld\ \alst{D}rǫttr ok Hǫsvir, &
\alst{L}útr ok \alst{L}ęggjaldi; \hld\ \alst{l}ǫgðu garða, &
\alst{a}kra tǫddu, \hld\ \alst{u}nnu at svïnum, &
\alst{g}ęita \alst{g}ę̇ttu, \hld\ \alst{g}rófu torf.\eva

\bvb Children they begot, settled and were content. \\
I think that they were called Rame and Feesner, \\
Clour and Cledge, Chafser, Foulner, \\
Drumber, Digrald, Drant and Hazer, \\
Lout and Leggald. They laid yard-fences, \\
dunged fields, looked after swine, \\
herded goats, dug turf.\evb\evg


\bvg\bva\mssnote{\Wormianus~78r/20–21}%
\alst{D}ǿtr vǫ́ru þę́r \hld\ \alst{D}rumba ok Kumba, &
\alst{Ø}kkvin-kalfa \hld\ ok \alst{A}rin-nęfja, &
\alst{Y}sja ok \alst{A}mbǫ́tt, \hld\ \alst{Ęi}kin-tjasna, &
\alst{T}ǫtrug-hypja \hld\ ok \alst{T}rǫnu-bęina; &
\alst{þ}aðan eru komnar \hld\ \alst{þ}rę́la ę́ttir.\eva

\bvb The daughters were these: Drumb and Cumb, \\
Inkencalf and Arnneb, \\
Eaze and Ambight, Oakentarsen, \\
Tattryhip and Tranebone— \\
thereof are come the lines of thralls.\evb\evg


\bvg\bva\mssnote{\Wormianus~78r/21–22}%
Gekk \alst{R}ígr at þat \hld\ \alst{r}éttar brautir &
kom hann at \edtrans{\emph{\alst{h}ǫllu}}{hall}{\Afootnote{sens. and metr. emend. (cf. st. 37); om. \Wormianus}} \hld\ \alst{h}urð vas ȧ skíði &
\alst{i}nn nam at ganga, \hld\ \alst{ę}ldr vas ȧ golfi &
\alst{h}jón sǫ́tu þar \hld\ \alst{h}eldu ȧ syslu.\eva

\bvb Went Righ after that on straight roads; \\
he came to a hall—the door was hinged. \\
He took to go inside; fire was on the floor. \\
A couple sat there, busy with their chores:\evb\evg


\bvg\bva\mssnote{\Wormianus~78r/23–24}%
\alst{M}aðr tęlgði þar \hld\ \alst{m}ęið til rifjar, &
vas \alst{sk}ęgg \alst{sk}apat, \hld\ \alst{sk}ǫr vas fyr ęnni &
\alst{sk}yrtu þrǫngva \hld\ \alst{sk}okkr vas ȧ golfi.\eva

\bvb A man there carved a stick into a loom-beam. \\
His beard was shapely; locks hung down his forehead; \\
his shirt tight; a toolbox was on the floor.\evb\evg


\bvg\bva\mssnote{\Wormianus~78r/24–25}%
\Ballnote{This st. contains an extremely interesting description of female dress.  \textcite{Nerman1969} deals with this.  TODO.}%
\alst{S}at þar kona, \hld\ \alst{s}vęigði rokk, &
\alst{b}ręiddi faðm, \hld\ \alst{b}jó til váðar; &
\alst{s}vęigr vas ȧ hǫfði, \hld\ \alst{s}mokkr vas ȧ bríngu, &
\alst{d}úkr vas ȧ halsi, \hld\ \edtrans{\alst{d}vergar}{dwarfs}{\Bfootnote{\textcite{Nerman1969} explains this word as referring to an archeological find.  TODO.}} ȧ ǫxlum; &
\alst{A}fi ok \alst{A}mma \hld\ \alst{ǫ́}ttu hús.\eva

\bvb There sat a woman, twirled a distaff, \\
stretched out her arms, readied a cloth. \\
A scarf was on her head, a smock on her breast, \\
a kerchief on her throat, dwarfs on her shoulders— \\
Grandpa and Grandma owned a house.\evb\evg


\bvg\bva\mssnote{\Wormianus~78r/25–27}%
\alst{R}ígr kunni þęim \hld\ \alst{r}ǫ́ð at sęgja, &
\alst{r}ęis frȧ borði \hld\ \alst{r}éð at sofna. &
\alst{M}ęirr lagði⸗sk hann \hld\ \alst{m}iðrar rękkju &
en ȧ \alst{h}lið \alst{h}vára \hld\ \alst{h}jȯn sal-kynna. &
\alst{Þ}ar vas hann at þat \hld\ \alst{þ}rjár nę́tr saman; &
liðu \alst{m}ęirr at þat \hld\ \alst{m}ǫ̇nuðr níu.\eva

\bvb Righ knew to tell them counsels; \\
rose from the table, resolved to sleep. \\
Further he laid himself down in the middle of the bed, \\
and on either side—the couple of the hall. \\
There he was after that for three nights amidst them; \\
passed further after that nine months.\evb\evg


\bvg\bva\mssnote{\Wormianus~78r/27–28}%
\alst{Jó}ð \alst{ó}l \alst{A}mma, \hld\ \alst{jó}su vatni, &
\alst{k}ǫlluðu \alst{K}arl \hld\ \alst{k}ona svęip ripti &
\alst{r}auðan ok \alst{r}jóðan \hld\ \alst{r}iðuðu augu.\eva

\bvb Grandma begot a child, they sprinkled it with water, \\
called it Churl; the woman wrapped him in cloth, \\
red and ruddy; his eyes trembled.\evb\evg


\bvg\bva\mssnote{\Wormianus~78r/28–30}%
Hann nam at \alst{v}axa \hld\ ok \alst{v}ęl dafna, &
\alst{ǫ}xn nam at tęmja \hld\ \alst{a}rðr at gørva &
\alst{h}ús at timbra \hld\ ok \alst{h}lǫður smíða &
\alst{k}arta at gørva \hld\ ok \alst{k}ęyra plóg.\eva

\bvb He took to grow and turn out well; \\
oxen he took to tame, the ard to make, \\
houses to timber and storehouses to forge, \\
carts to make and drive the plough.\evb\evg


\bvg\bva\mssnote{\Wormianus~78r/30–31}%
\alst{H}ęim óku þȧ \hld\ \alst{H}angin-luklu &
\alst{g}ęita-kyrtlu \hld\ \alst{g}iptu Karli. &
\alst{S}nǫr hęitir \alst{s}ú, \hld\ \alst{s}ętti⸗sk und ripti. &
\alst{B}juggu hjȯn, \hld\ \alst{b}auga dęildu, &
\alst{b}ręiddu \alst{b}lę́jur, \hld\ ok \alst{b}ú gørðu.\eva

\bvb Home they then drove with Hangenkey, \\
in a goatskin-skirt, married her to Churl. \\
Daughter-in-law she is called; she sat down beneath a felt. \\
The couple settled, shared their wealth, \\
hung tapestries and made a home.\evb\evg


\bvg\bva\mssnote{\Wormianus~78r/31–78v/1}%
\alst{B}ǫrn ólu þau, \hld\ \alst{b}juggu ok unðu; &
hét \alst{H}alr ok Drengr, \hld\ \alst{H}ǫlðr, Þegn ok Smiðr, &
\alst{B}ręiðr, \alst{B}óndi, \hld\ \alst{B}undin-skęggi, &
\alst{B}úi ok \alst{B}oddi \hld\ \alst{B}ratt-skęggr ok Sęggr.\eva

\bvb Children they begot, settled and were content. \\
They were called Hale and Drang, Health, Thane and Smith, \\
Broad, Bond, Boundenshag, \\
Bower and Bod, Brantshag and Seg.\evb\evg


\bvg\bva\mssnote{\Wormianus~78v/1–2}%
\Ballnote{Most of these terms are mentioned in \Skaldskaparmal\ 84. TODO.}%
\alst{Ę}nn hétu svá \hld\ \alst{ǫ}ðrum nǫfnum &
\alst{S}not, Brúðr, \alst{S}vanni, \hld\ \alst{S}varri, Sprakki, &
\alst{F}ljóð, Sprund, ok Víf, \hld\ \alst{F}ęima, Ristill; &
þaðan eru \alst{k}omnar \hld\ \alst{k}arla ę́ttir.\eva

\bvb Further some were thusly called other names: \\
Snoot, Bride, Swannie, Swarrie, Sprackie, \\
Fleed, Sprund and Wife, Fome, Ristle— \\
therof are come the lines of churls.\evb\evg


\bvg\bva\mssnote{\Wormianus~78v/2–3}%
Gekk \alst{R}ígr þaðan \hld\ \alst{r}éttar brautir &
kom hann at \alst{s}al, \hld\ \alst{s}uðr horfðu dyrr, &
vas \alst{h}urð \alst{h}nigin, \hld\ \alst{h}ringr vas í gę̇tti.\eva

\bvb Went Right thence on straight roads; \\
he came to a hall, south faced the doors; \\
the door was opened, a ring was on the gate.\evb\evg


\bvg\bva\mssnote{\Wormianus~78v/3–4}%
\alst{G}ekk hann inn at þat \hld\ \alst{g}olf vas stráat &
\alst{s}ǫ́tu hjȯn \hld\ \alst{s}ǫ́u⸗sk ï augu &
\alst{f}aðir ok móðir \hld\ \alst{f}ingrum at lęika.\eva

\bvb He walked in after that; the floor was strawed; \\
the couple sat, looked eachother in the eyes, \\
Father and Mother, playing with their fingers.\evb\evg


\bvg\bva\mssnote{\Wormianus~78v/4–5}%
\alst{S}at hús-gumi \hld\ ok \alst{s}nøri stręng &
\alst{a}lm of bęndi \hld\ \alst{ǫ}rvar skępti; &
en \alst{h}ús-kona \hld\ \alst{h}ugði at ǫrmum, &
\alst{st}rauk of ripti \hld\ \alst{st}erti ęrmar.\eva%TODO: sterti or stęrti?

\bvb Sat the man of the house and twisted the bow-string, \\
bent the elmwood, shafted arrows— \\
but the wife of the house minded her arms, \\
smoothened the fabric, tightened the sleeves.\evb\evg


\bvg\bva\mssnote{\Wormianus~78v/6–7}%
\alst{K}ęisti fald, \hld\ \alst{k}inga vas ȧ bringu, &
\alst{s}íðar \alst{s}lǿður, \hld\ \alst{s}ęrk blá-fáan; &
\alst{b}ru̇n \alst{b}jartari, \hld\ \alst{b}rjóst ljósara, &
\alst{h}als \alst{h}vítari \hld\ \alst{h}ręinni mjǫllu.\eva

\bvb The linen hood jutted out, a brooch was on her chest, \\
a trailing gown, a serk dyed blue; \\
her brow was brighter, her chest lighter, \\
her throat whiter than purest snow.\evb\evg


\bvg\bva\mssnote{\Wormianus~78v/7–8}%
\alst{R}ígr kunni þęim \hld\ \alst{r}ǫ́ð at sęgja; &
\alst{m}ęirr sętti⸗sk hann \hld\ \alst{m}iðra flętja &
en ȧ \alst{h}lið \alst{h}vára \hld\ \alst{h}jón sal-kynna.\eva

\bvb Righ knew to tell them counsels, \\
further he set himself down on the middle of the floor-bench, \\
and on either side: the couple of the hall.\evb\evg


\bvg\bva\mssnote{\Wormianus~78v/8–9}%
Þȧ tók \alst{m}óðir \hld\ \alst{m}ęrktan dúk, &
\alst{h}vítan af \alst{h}ǫrvi, \hld\ \alst{h}ulði bjóð; &
\alst{h}ón tók at þat \hld\ \alst{h}lęifa þunna, &
\alst{h}víta af \alst{h}vęiti, \hld\ ok \alst{h}ulði dúk.\eva

\bvb Then Mother took a patterned cloth, \\
white of flax—she covered the platter. \\
She took after that thin loaves, \\
white of wheat—and covered the cloth.\footnoteB{Note the strong parallelism.  The rich household can afford such an excess of expensive fabric and bread that they can cover a plate with an embroidered (\emph{męrktr}) flaxen cloth, and then cover that cloth with loaves of wheat-bread.}\evb\evg


\bvg\bva\mssnote{\Wormianus~78v/9–11}%
Framm sętti hǫ̇n \hld\ skutla fulla &
silfri varða ȧ bjóð &
\edtrans{\alst{f}ȧn}{gizzard}{\Bfootnote{I am convinced by Fritzner (TODO: cite), who sees this word as a variant of \emph{fóarn} ‘gizzard’.}} ok \alst{f}lęski \hld\ ok \alst{f}ugla stęikta; &
\alst{v}ín vas ï kǫnnu, \hld\ \alst{v}arðir kálkar &
\alst{d}rukku ok \alst{d}ǿmðu, \hld\ \alst{d}agr vas ȧ sinnum.\eva

\bvb Forth she set trenchers filled— \\
silver-covered on platters— \\
with gizzard and pork and roasted fowls. \\
Wine was in a flagon; the women from goblets \\
drank and discussed; the day was waning.\evb\evg


\bvg\bva\mssnote{\Wormianus~78v/11}%
\alst{R}ígr kunni þęim \hld\ \alst{r}ǫ́ð at sęgja, &
\alst{r}ęis \alst{R}ígr at þat, \hld\ \alst{r}ękkju gørði.\eva

\bvb Righ knew to tell them counsels; \\
Righ rose after that; he made the bed.\evb\evg


\bvg\bva\mssnote{\Wormianus~78v/12–13}%
\alst{Þ}ar vas hann at þat \hld\ \alst{þ}rjár nę́tr saman; &
gekk hann \alst{m}ęirr at þat \hld\ \alst{m}iðrar brautar; &
liðu \alst{m}ęirr at þat \hld\ \alst{m}ǫ̇nuðr níu.\eva

\bvb There he was after that for three nights amidst them; \\
he went further after that on the middle of the road; \\
passed further after that nine months.\evb\evg


\bvg\bva\mssnote{\Wormianus~78v/13–14}%
\alst{S}vęin ól móðir, \hld\ \alst{s}ilki vafði, &
\alst{jó}su vatni— \hld\ \alst{Ja}rl létu hęita; &
\alst{b}lęikt vas hár, \hld\ \alst{b}jartir vangar, &
\edtrans{\alst{ǫ}tul vǫ́ru \alst{au}gu \hld\ sem \alst{y}rmlingi}{fierce were his eyes like the young serpent’s}{\Bfootnote{It is common throughout Norse texts that people of noble stock distinguish themselves through their appearance, especially a sharp, piercing gaze.  This occurs e.g. in \textlink{Volundarkvida} where the gaze of the king’s son Wayland is like the serpent’s, and at the beginning of \textlink{HelgakvidaTwo}, where Hallow, disguised as a thrall-woman, is almost caught due to his unslavelike eyes, which, like in the present stanza, are said to be \emph{ǫtul} ‘fierce, terrible’.}}.\eva

\bvb Mother begot a swain, swaddled him in silk; \\
they sprinkled him with water, let him be called Earl. \\
Pale was his hair, bright his cheeks; \\
fierce were his eyes like the young serpent’s.\evb\evg


\bvg\bva\mssnote{\Wormianus~78v/14–16}%
\alst{U}pp \alst{ó}x þar \hld\ \alst{Ja}rl ȧ flętjum; &
\alst{l}ind nam at skęlfa, \hld\ \alst{l}ęggja stręngi, &
\alst{a}lm at bęygja, \hld\ \alst{ǫ}rvar skępta, &
\alst{f}lęin at \alst{f}lęyja, \hld\ \alst{f}rǫkkur dýja, &
\alst{h}ęstum ríða, \hld\ \alst{h}undum verpa, &
\alst{s}verðum bregða, \hld\ \alst{s}und at fręmja.\eva

\bvb There Earl grew up on the floor-benches; \\
he took to shake the linden shield, fasten bow-strings, \\
bend elmwood, shaft arrows, \\
throw javelins, hoist Frankish spears, \\
ride horses, sic hounds, \\
brandish swords, practice swimming.\evb\evg


\bvg\bva\mssnote{\Wormianus~78v/16–18}%
\Ballnote{Righ approaches his son, Earl. He reveals himself as his father and initiates him into the warrior aristocracy by teaching him the runes and giving him the noble title Righ; henceforth he will be known as Righ-Earl.  Finally Righ instructs him to set out and win lands for himself.}%
Kom þar ór \alst{r}unni \hld\ \alst{R}ígr gangandi, &
\alst{R}ígr gangandi, \hld\ \alst{r}u̇nar kęnndi; &
\alst{s}itt gaf hęiti, \hld\ \alst{s}on kvęð⸗sk ęiga; &
þann bað hann \alst{ęi}gna⸗sk \hld\ \alst{ó}ðal-vǫllu, &
\alst{ó}ðal-vǫllu, \hld\ \alst{a}ldnar bygðir.\eva

\bvb There from a thicket came Righ, walking: \\
Righ, walking, taught him the runes, \\
gave him his own name, said that he had a son. \\
He bade him possess the ethel-plains: \\
the ethel-plains, the ancient farmsteads.\evb\evg


\bvg\bva\mssnote{\Wormianus~78v/18–20}%
Ręið hann \alst{m}ęirr þaðan \hld\ \alst{m}yrkan við &
\alst{h}élug fjǫll \hld\ und’s at \alst{h}ǫllu kom; &
\alst{sk}apt nam at dýja, \hld\ \alst{sk}ęlfði lind, &
\alst{h}ęsti \alst{h}lęypti, \hld\ ok \alst{h}jǫrvi brá; &
\alst{v}íg nam at \alst{v}ękja, \hld\ \alst{v}ǫll nam at rjóða, &
\alst{v}al nam at fęlla, \hld\ \alst{v}á til landa.\eva

\bvb He \ken*{= Righ-Earl} rode further thence through the mirky wood, \\
through the frosty fells till to a hall he came. \\
The shaft he took to hoist, shook the linden shield, \\
leapt with his horse and brandished his blade. \\
War he took to rouse; the plain he took to redden; \\
men he took to fell—he won the lands.\evb\evg


\bvg\bva\mssnote{\Wormianus~78v/20–21}%
Réð hann \alst{ęi}nn at þat \hld\ \alst{á}t-ján bú\emph{u}m; &
\alst{au}ð nam skipta \hld\ \alst{ǫ}llum vęita &
\alst{m}ęiðmar ok \alst{m}ǫsma, \hld\ \alst{m}ara svang-rifja; &
\edtrans{\alst{h}ringum \alst{h}ręytti}{rings he scattered}{\Bfootnote{Cf. StarkSt Frag 1/2a \emph{hring-hręytanda} ‘ring-scattererer \ken{generous man}’ which contains the same words.}}, \hld\ \alst{h}jó sundr baug.\eva

\bvb He alone ruled after that eighteen homesteads. \\
Wealth he took to hand out; to grant all men \\
gifts and treasures, slender-ribbed steeds; \\
rings he scattered; he struck apart the bigh.\evb\evg


\bvg\bva\mssnote{\Wormianus~78v/21–22}%
\edtext{\alst{Ó}ku*}{\Afootnote{\emph{‘okū’} \Wormianus}} \alst{ę́}rir \hld\ \alst{ú}rgar brautir &
kvǫ́mu at \alst{h}ǫllu \hld\ þar’s \alst{H}ęrsir bjó: &
\edtext{\alst{m}ǿtt\emph{u}}{\Afootnote{\emph{mǿtti} \Wormianus}\Bfootnote{Past singular \emph{mǿtti} is impossible, since the maiden is the one being met.  \emph{mǿta} ‘meet’ takes the dative.}} \edtext{\emph{\alst{m}ęyju}}{\Afootnote{om. \Wormianus}\Bfootnote{A feminine dat. sg. noun meaning ‘maiden, girl’ is required here by the meter and the following adjectives; \emph{męyju} dat. sg. of \emph{mę́r} fits with the alliteration, but is by no means certain.}} \hld\ \edtext{\alst{m}jó-fingrað\emph{r}i}{\Afootnote{\emph{mjó-fingraði} \Wormianus}} &
\alst{h}vítri ok \alst{h}orskri, \hld\ \alst{h}étu Ęrna.\eva

\bvb Messengers drove on drizzling roads, \\
came to the hall where Herser lived, \\
met a maiden slender-fingered, \\
white and wise; they called her Erne.\evb\evg


\bvg\bva\mssnote{\Wormianus~78v/22–24}%
Bǫ́ðu \alst{h}ęnnar \hld\ ok \alst{h}ęim óku, &
\alst{g}iptu Jarli, \hld\ \edtrans{\alst{g}ekk hón und líni}{she went ’neath the linen}{\Bfootnote{She donned the bridal veil; cf. \textlink{Thrymskvida}[27].}}; &
\alst{s}aman bjuggu þau \hld\ ok \alst{s}ér unðu, &
\alst{ę́}ttir \alst{jó}ku \hld\ ok \alst{a}ldrs nutu.\eva

\bvb They asked for her hand and drove home, \\
married her to Earl—she went ’neath the linen. \\
Together they settled and were content, \\
increased their lineage and enjoyed life.\evb\evg


\bvg\bva\mssnote{\Wormianus~78v/24–25}%
\alst{B}urr vas hinn ęldsti, \hld\ en \alst{B}arn annat; &
\alst{Jó}ð ok \alst{A}ðall, \hld\ \alst{A}rfi, Mǫgr, &
\alst{N}iðr ok \alst{N}iðjungr, \hld\ \edtext{(\alst{n}ǫ́mu lęika) &
\alst{S}onr ok \alst{S}vęinn, \hld\ (\alst{s}und ok tafl)}{\lemma{nǫ́mu lęika \dots\ sund ok tafl ‘they learned to partake in swimming and tables’}\Bfootnote{This sentence is embedded in the list of names.  Swimming and board games were stereotypic pasttimes for aristocrats; cf. the two Scaldic stanzas attributed to Earl Rainwald (Rv Lv 1) and King Harold Hardrede (Hharð \emph{Gamv} 4), respectively, where each man recounts his \emph{íþróttir} ‘skills, pursuits’.}} &
\alst{K}undr hét ęinn; \hld\ \alst{K}onr vas hinn yngsti.\eva

\bvb%
Byre was the eldest and Bairn the other; \\
Ede and Athel, Arver and Maw, \\
Nith and Nithing (they learned to partake) \\
Son and Swain (in swimming and tables); \\
Cund was one called; Kin was the youngest.\evb\evg


\bvg\bva\mssnote{\Wormianus~78v/25–27}%
\alst{U}pp \alst{ó}xu þar \hld\ \alst{Ja}rli bornir: &
\alst{h}ęsta tǫmðu, \hld\ \alst{h}lífar bęndu, &
\alst{sk}ęyti \alst{sk}ófu, \hld\ \alst{sk}ęlfðu aska. &
En \edtrans{\alst{K}onr ungr}{Kin the young}{\Bfootnote{A folk etymological pun on \emph{konungr} ‘king’.  It is notable that it is the youngest son who attains the highest rank (for the King is above even the earls); for the motif of a god favouring the youngest son cf. \textlink{Grimnismal}[P1].}} \hld\ \alst{k}unni ru̇nar: &
\alst{ę́}vin-ru̇nar \hld\ ok \alst{a}ldr-ru̇nar.\eva

\bvb There they grew up, Earl’s scions; \\
horses they tamed, shield-rims they bent, \\
shafts they planed, shook ashen spears— \\
but Kin the young knew the runes: \\
ever-runes and life-runes.\evb\evg


\bvg\bva\mssnote{\Wormianus~78v/27–28}%
\alst{M}ęirr kunni hann \hld\ \alst{m}ǫnnum bjarga, &
\alst{ę}ggjar dęyfa, \hld\ \alst{ę́}gi lę́gja; &
\alst{k}lǫk nam fugla, \hld\ \alst{k}yrra ęlda, &
\alst{s}ǿfa ok \alst{s}vęfja, \hld\ \alst{s}orgir lę́gja, &
\alst{a}fl ok \alst{ę}ljun \hld\ \alst{á}tta manna.\eva

\bvb Further he could rescue men, \\
dull blades, lower the sea. \\
He learned the chirping of birds, to calm fires, \\
to lull and put to sleep, to lower sorrows, \\
the strength and zeal of eight men.\evb\evg


\bvg\bva\mssnote{\Wormianus~78v/28–29}%
Hann við \alst{R}íg Jarl \hld\ \alst{r}u̇nar dęildi; &
\alst{b}rǫgðum \alst{b}ęitti \hld\ ok \alst{b}ętr kunni; &
þȧ \alst{ǫ}ðladi⸗sk \hld\ ok þȧ \alst{ęi}ga gat, &
\alst{R}ígr at hęita, \hld\ \alst{r}u̇nar kunna.\eva

\bvb With Righ-Earl he shared runes, \\
employed tricks and knew better. \\
Then he earned and then won the right \\
to be called Righ, to know the runes.\evb\evg


\bvg\bva\mssnote{\Wormianus~78v/30–31}%
Ręið \alst{K}onr ungr \hld\ \alst{k}jǫrr ok skóga; &
\alst{k}olfi flęygði \hld\ \alst{k}yrði fugla; &
þȧ kvað þat \alst{k}ráka \hld\ —sat \alst{k}visti ęin— &
„Hvat skalt, \alst{K}onr ungr, \hld\ \alst{k}yrra fugla? &
\alst{H}ęldr mę́ttið ér \hld\ \alst{h}ęstum ríða &
\edtrans{\emph{\alst{h}ęstum ríða}}{horses to ride}{\Afootnote{emend.; om. (presumably by haplography) \Wormianus}} \hld\ ok \alst{h}ęr fęlla.\eva

\bvb Kin the young rode through brushes and woods, \\
hurled his bolts, stunned the birds. \\
Then quoth a crow—sat on a branch alone— \\
“Why shalt thou, Kin the young, stun the birds? \\
Rather, ye might ride horses, \\
ride horses, fell hosts.”\evb\evg


\bvg\bva\mssnote{\Wormianus~78v/31–32}%
Á \alst{D}anr ok \alst{D}anpr \hld\ \alst{d}ýrar hallir; &
\alst{ǿ}ðra \edtrans{\alst{ó}ðal}{ethel}{\Bfootnote{Ancestral farmland, in this case the eighteen homesteads owned by Earl.}} \hld\ an \edtrans{\alst{é}r}{ye}{\Afootnote{metr. emend.; \emph{þér} ‘id.’ \Wormianus}\Bfootnote{The use of \emph{þér} in the ms., which is simply a younger form of \emph{ér}, shows that the poem has been linguistically modernised.}} hafið; &
þęir \alst{k}unnu vel \hld\ \edtrans{\alst{k}jól at riða}{ride the ship}{\Bfootnote{I.e. to sail.}}, &
\edtrans{\alst{ę}gg at kęnna}{teach the blade}{\Bfootnote{To wage war.  A euphemism; to “teach someone the blade” is to fight (and kill) him.}}, \hld\ \alst{u}ndir rjúfa.“\eva

\bvb Dene and Danp own costly halls, \\
nobler ethel than ye have. \\
They can well ride the ship \\
teach the blade, tear open wounds.”\evb\evg

\begin{center}(At this point fol. 78 of \Wormianus\ ends, and the rest of the poem is lost.  For parallels see the Introduction above.)\end{center}

\sectionline
