\bookStart{Introduction to Eddic Poetry}

The following two sections (Norse Mythic Poetry and Norse Heroic Poetry from the Walsing-Cycle) contain the bulk of preserved Old Norse Eddic poetry, most of which is preserved in only one manuscript, \Regius.  For this reason some notes about Eddic Poetry in general are in order.

\section{Character}

Old Norse poetry is conventionally divided into two categories: the Eddic and the Scaldic.

Eddic poetry is anonymous, composed in one of three meters (\Fornyrdislag, \Ljodahattr, \Malahattr), and free from regular assonance or end-rhyme.  It is easy to understand, uses kennings sparsely, and usually deals with legendary or mythological subjects.

By contrast, Scaldic poetry is usually attributed to named poets, composed in a greater variety of meters (which apart from the three Eddic meters include the exclusively Scaldic \Kviduhattr\ and \Drottkvett), ans employs regular assonance or end-rhym.  It revels in purposeful obscurity by making great use of complex kennings and circumlocutions, and usually deals with historical or biographical subjects.  It typically serves as praise poetry for a living king or other chieftain.

Although this categorization is not explicitly made in any Norse primary source, it is not entirely a philological construct, for it is clearly observed in Snorre’s Edda.  In the first book, \Gylfaginning, we find almost exclusively citations made from Eddic poems (68/70 sts.) while the second book, \Skaldskaparmal\ is the opposite.

\section{Dating}
    Linguistic criteria
    Archeological evidence
    Comparison with known Christian texts (Sólarljóð, Hugsvinnsmál)
    Snorri thought they were old
    Saxo had access to them
    Many of them clearly describe non-Icelandic surroundings
      Especially Hávamál is clearly Norwegian

\section{Manuscripts}

\subsection{Codex Regius (R)}

By far the most important manuscript is GKS 2365 4to (siglum \Regius), the so-called Codex Regius.  It dates to around 1270 and consists of 45 surviving foll. containing 29 poems.  The ms. itself is divided into two parts or sections; the first (on foll. 1–20, containing 11 poems) dealing mostly with mythology, the second (on foll. 20–45, containing 18 poems) dealing with heroic legend from the Walsing cycle.  Scribal characteristics show that these two parts have been copied from separate source manuscripts and they are each introduced with a particularly large initial letter. (TODO: cite Lindblad)

\Regius\ is not a mere anthology of poems, but shows substantial editorial input as well.  Short prose sections tie a group of the mythological poems together into a loose narrative, though it is clear from their meter, style, and language that these poems are separate works composed by various poets over time.  When it comes to the heroic poems long prose segments occur both within and between them, creating a \inx[C]{saw}-like prosimetrical form where the prose sometimes comes to dominate the poetry.  A manuscript closely related to the heroic half of \Regius\ has clearly served as the main source for large swathes of the younger \VolsungaSaga.

A large gap famously occurs in the heroic half; between foll. 32 and 33 one quire has gone missing.  Its contents are mostly unknown, but it would have included the end of \textlink{Sigrdrifumal} and the beginning of the Fragmentary Lay of Siward (TODO).  Some of the stanzas probably contained in it may be restored from the \VolsungaSaga, and these are edited under \Lacuna\ below.  For further literature on \Regius\ see TODO.

\subsection{AM 748 I a 4to (A)}

Second in importance stands AM 748 I a 4to (siglum \AM).  It dates to around 1300 and is in fragmentary state, consisting of just 6 foll.  The beginning and end are absent, and between foll. 2 and 3 there is a lacuna, so that at least 3 (but probably more) foll. are missing.

\AM\ contains seven poems.  On 1r–2v are found in succession the latter half of \textlink{Harbardsljod}, the full \textlink{Baldrsdraumar}, and the first half of \textlink{Skirnismal}.  There is then the lacuna—Finnur Jónsson guesses that just one fol. is missing—and on 3r–6v are found in succession most of \textlink{Vafthrudnismal}, all of \textlink{Grimnismal} and \textlink{Hymiskvida}, and the introductory prose to \textlink{Volundarkvida}.  Among mediæval mss., \textlink{Baldrsdraumar} is only attested in \AM, while the other six poems are also found in the first, mythological, part of \Regius. The order of the poems varies drastically between \AM\ and \Regius.

\AM\ has no trace of \Regius’s frame narrative tying together \textlink{Hymiskvida} and \textlink{Lokasenna} (and indeed the latter poem has left no trace in it), but otherwise \AM\ and \Regius\ do share a substantial amount of prose, a fact which proves that the two, rather than being independent witnesses of oral tradition, stem from a common manuscript archetype.

The edition of \AM\ here consulted is \textcite{Finnur1896}.

\subsection{Manuscripts of Snorre’s Edda}

Snorre’s Edda or the Prose Edda consists of three sections.  The first two—\Gylfaginning\ and \Skaldskaparmal—contain quotations from several Eddic poems.  Snorre reproduces stanzas from the mythological \textlink{Voluspa}, \textlink{Vafthrudnismal}, \textlink{Grimnismal}, \textlink{Skirnismal}, \textlink{Lokasenna}, and \textlink{Hyndluljod} in \Gylfaginning\ and stanzas from \textlink{Alvissmal} in \Skaldskaparmal.  In addition he cites several otherwise unknown stanzas in Eddic meters, of which at least some appear to derive from now-lost mythological poems.  These fragments are all edited at the end of the present section under \textlink{EddicFragments}.

The four main mss. for the Prose Edda are:%TODO: use table like in the Heliand introduction

\begin{enumerate}
  \item Codex Regius of the Prose Edda (GKS 2367 4to, siglum \RegiusProse), dating to 1300-1350.
  \item Codex Trajectinus (Traj 1374, siglum \Trajectinus), a c. 1595 paper copy of a ms. closely related to \RegiusProse.
  \item Codex Wormianus (AM 242 fol., siglum \Wormianus), dating to 1340–70. \Wormianus\ also contains the \textlink{Rigsthula}.
  \item Codex Upsaliensis (DG 11, siglum \Upsaliensis), dating to 1300–25.  Stematically \Upsaliensis\ is the most archaic ms., but unfortunately it has been heavily abbreviated and is filled with errors; this makes it a questionable source, especially for poetry.
\end{enumerate}

%When all four mss. agree on a reading, the abbreviation \GylfMS\ is used synonymously with \RegiusProse\Trajectinus\Wormianus\Upsaliensis.  For discussion on their internal stemmatics and origins I refer to \textcite{Haukur2017}.

\subsection{Other manuscripts}

A few other Eddic-style poems from various sources are also included in the present edition.  \textlink{Rigsthula} is attested in \Wormianus\ and \textlink{Hyndluljod} in \FlatMS.  A younger redaction of \textlink{Voluspa} is found in \Hauksbok.  TODO (Svipdagsmál and \Grougaldr) are found only in post-reformation Icelandic paper mss., namely TODO.  While I have not consulted such paper mss. for poems attested in mediæval mss., I have had to rely on them for these poems.  About these poems in particular it has to be said that late first \emph{attestation} does not necessary imply early \emph{composition}.  A good proof of this is \textlink{Baldrsdraumar}, which is first attested in the fragmentary mediæval \AM, and then (with some interpolated stanzas) in much later paper mss.  We cannot exclude that some of these poems would have existed in other lost mediæval mss., perhaps even on the now-lost pages of \Regius\ or \AM.

\subsection{Abbreviations}

\begin{itemize}%
	\item \AM\ = AM 748 I a 4° (https://handrit.is/manuscript/view/da/AM04-0748-I-a)
	\item \AMb\ = AM 748 I b 4° (https://handrit.is/manuscript/view/is/AM04-0748-Ib)
	\item \EddaBms\ = AM 757 a 4° (https://handrit.is/manuscript/view/is/AM04-0757a)
	\item \FlatMS\ = Flateyjarbók, GKS 1005 fol. (https://handrit.is/manuscript/view/is/GKS02-1005)
%	\item \GylfMS\ = all manuscripts of \Gylfaginning; equivalent to \RegiusProse\Trajectinus\Upsaliensis\Wormianus
	\item \Hauksbok\ = Hauksbók, AM 544 4° (https://handrit.is/manuscript/view/en/AM04-0544)
	\item \VolsungaMS\ = NKS 1824 b 4° (https://onp.ku.dk/onp/onp.php?m9641)
	\item \Regius\ = Codex Regius of the Poetic Edda, GKS 2365 4° (https://eae.ku.dk/q?p=eae/vols/text/1)
	\item \RegiusProse\ = Codex Regius of the Prose Edda, GKS 2367 4° (https://handrit.is/manuscript/view/is/GKS04-2367)
	\item \Trajectinus\ = Codex Trajectinus, Traj 1374ˣ
	\item \Upsaliensis\ = Codex Upsaliensis, DG 11
	\item \Wormianus\ = Codex Wormianus, AM 242 fol. (https://clarino.uib.no/menota/text/menota/AM-242-fol)
\end{itemize}

\bookStart{Introduction to Norse Mythic Poetry}

This section contains all extant narrative poetry concerning the pre-Christian Germanic gods.  This is a genre mentioned already by Tacitus, who speaks of the gods of the ancient Germani celebrated in song (\emph{Germania}, ch. 2).
Yet it is not due to any decision on the part of the editor that this poetry is exclusively in Old Norse, for outside of Iceland there survive no examples of this undoubtedly ancient genre.

That should hardly be taken to mean that no such poetry existed.  On the contrary, as we shall see, many of the stories dealt with in this poetry are attested throughout the Germanic speech-area for a period of many centuries; in addition they show frequent parallels to traditions as distant as Mesopotamia, India, and Egypt.  And how should they have been told and transmitted through unreckoned illiterate ages if not in poetic form?

\section{Table of manuscripts}

The following table shows the manuscript attestation for the included poems.  A + in \Gylfaginning/\Skaldskaparmal\ represents one or more stanzas attested in quotation.

\begin{small}\begin{longtabu} to \textwidth {|l c c c c c c|}
	\hline
	Signum & \Regius & \AM & \Gylfaginning/\Skaldskaparmal & \Wormianus & \Hauksbok & \FlatMS \\ [0.5ex]
	\hline\hline\endhead
	\hline\endfoot
  \textlink{Voluspa}        & 1  & − & + & − & + & − \\
  \textlink{Havamal}        & 2  & − & + & − & − & − \\
  \textlink{Vafthrudnismal} & 3  & 4 & + & − & − & − \\
  \textlink{Grimnismal}     & 4  & 5 & + & − & − & − \\
  \textlink{Skirnismal}     & 5  & 3 & + & − & − & − \\
  \textlink{Harbardsljod}   & 6  & 1 & − & − & − & − \\
  \textlink{Hymiskvida}     & 7  & 6 & − & − & − & − \\
  \textlink{Lokasenna}      & 8  & − & + & − & − & − \\
  \textlink{Thrymskvida}    & 9  & − & − & − & − & − \\
  \textlink{Volundarkvida}  & 10 & 7 & − & − & − & − \\
  \textlink{Alvissmal}      & 11 & − & + & − & − & − \\
  \textlink{Baldrsdraumar}  & −  & 2 & − & − & − & − \\
  \textlink{Rigsthula}      & −  & − & − & + & − & − \\
  \textlink{Hyndluljod}     & −  & − & + & − & − & + \\ [1ex]
  \hline
\end{longtabu}\end{small}

Apart from these 14 poems, 11 additional mythological stanzas or half-stanzas from \Gylfaginning\ and \Skaldskaparmal\ are edited at the end of the present section under the header Fragments from Snorre’s Edda (abbrev. \EddicFragments).
