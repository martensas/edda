\bookStart{Dreams of Balder}[Baldrs draumar]
\setBookCode{Baldrsdraumar}

% Introduction.
\begin{flushright}%
\textbf{Dating} \parencite{Sapp2022}: C10th (0.890)

\textbf{Meter:} \Fornyrdislag%
\end{flushright}

\section{Introduction}

\subsection{Preservation}

The \textbf{Dreams of Balder} (\Baldrsdraumar) is not preserved in \Regius, but rather in the early C14th ms. \AM.  A younger redaction, characterized by a number of post-mediæval inserts, is transmitted in several copies in later paper mss.

\subsection{Content}

The main source for the death and subsequent avenging of the god Balder is \Gylfaginning\ 49–50; for a summary see \textlink{Voluspa}[31] n.  Compared to that narrative \Baldrsdraumar\ corresponds only to the very beginning, namely Balder’s ominous dreams, and does not go into great depth about its subject.  At just 14 sts. the poem is the shortest surviving mythological Eddic poem.  In isolation its purpose seems somewhat unclear, but it may originally have been part of a longer cycle dealing with Balder’s death—that such in fact existed is supported by \textlink{EddicFragments}[F5][5].

\subsection{Summary}

The poem begins \emph{in medias res}; \inx[P]{Balder} has been having nightmares, which the Gods meet at the Thing to discuss (1).  \inx[P]{Weden} rides to \inx[P]{Hell}, where he has an encounter with a bloody hound; he passes it and continues to “the high house of \inx[P]{Hell}” (2–3), from which he rides west, to the grave of a certain \inx[C]{wallow} whom he revives using magic (4). She asks which man has forced her out of the grave (5), and Weden introduces himself as Waytame, before asking for whom the benches of Hell are covered with gold (6). The wallow responds that barrels of mead stand brewed for Balder and that the gods are very anxious (7). Weden asks her who will slay Balder (8), and she responds that it is Hath, carrying a “high fame-beam” (9).  Weden asks who will avenge Balder’s death (10), the wallow responds that \inx[P]{Wrind} will give birth to Weden’s son \inx[P]{Wonnel}, who will slay Hath when only one night old (11).  Weden then asks about some mysterious maidens (12), which apparently betrays his identity.  The wallow announces that she now knows that it is Weden, who in turn retorts that she is not a wallow, but rather the “mother of three thurses” (13). The wallow tells him to ride home and “be famous” and taunts him over his unavoidable death at the \inx[P]{Rakes of the Reins} (14).

\newpage

\section{Text}

\subsection{The Dreams of Balder}

\bvg\bva\mssnote{\AM~1v/18}%
\edtext{Sęnn vǫ́ru \alst{ę̇}sir \hld\ \alst{a}llir ȧ þingi &
ok \alst{ǫ̇}synjur \hld\ \alst{a}llar ȧ máli, &
ok umb þat \alst{r}éðu \hld\ \alst{r}íkir tívar:}{\lemma{Sęnn \dots\ tívar ‘Soon \dots\ Tews’}\Bfootnote{Identically shared with \textlink{Thrymskvida}[14]/1–3.

The ‘Thing’ here is the Thing of the Gods (cf. \textlink{Voluspa}[6]/1–2~n.), but the detail that the male Gods (\emph{ę̇sir} ‘Eese’) are \emph{ȧ þingi} ‘at, upon the Thing’, while the Goddesses (\emph{ǫ̇synjur} ‘Ossens’) are on the sidelines chattering requires elaboration.  Being \emph{ȧ þingi} is probably to be understood as being seated within the ritual enclosure at the legal assembly.  This reflects historical Norse Thing-proceedings where the legal judgments were decided by a selected group of men seated inside a circular holy enclosure (the \emph{vé-bǫnd} ‘wigh-bonds’, i.e. ‘holy cords’) into which all others were forbidden to enter.  So \EgilsSaga\ 56 in a description of the Gole-Thing in Norway: \emph{En þar er dómr’inn var settr, var vǫllr sléttr ok settar niðr hesli-stengr í vǫll’inn í hring, en lǫgð um útan snǿri um·hverfis.  Vǫ́ru þat kǫlluð vé-bǫnd.  En fyrir innan í hringi’num sǫ́tu dómendr [...]; þę́r þrennar tylftir manna skyldu þar dǿma um mál manna.} ‘But where the judgment was set there was a smooth plain and hazelwood-poles were set down into the plain in a ring, and cords tied around them on the outside; those were called the ‘wigh-bonds’.  But inside within the ring sat the judges [...]; those three groups of twelve men each would there judge the cases of men.’  The Norwegian New Law of the Land, I. 3 (\textciteshorttitle[14]{NGL2}) specifies the proceedings of the Gole-Thing and notes that: \emph{Svá er ok mę́lt at engi maðr þeirra er eigi er nefndr skal firir innan vé-bǫnd setjast útan or·lofs.} ‘So it is also said that no man among them who was not named shall be seated inside the wigh-bonds without permission.’}} &
hví vę́ri \alst{B}aldri \hld\ \alst{b}allir draumar?\eva

\bvb%
{\huge S}\textsc{oon were the \inx[P]{Eese}} all at the \inx[C]{Thing}, \\
and the \inx[P]{Ossens} all at speech, \\
and of this counseled the mighty \inx[P]{Tews}: \\
Why did Balder have troubling dreams?\evb\evg


\bvg\bva\mssnote{\AM~1v/19}%
\alst{U}pp ręis \alst{Ó}ðinn, \hld\ \edtext{\alst{a}ldinn}{\Afootnote{emend.; \emph{alda} \AM}} gautr, &
\edtrans{ok hann ȧ \alst{S}lęipni \hld\ \alst{s}ǫðul of lagði}{and he on Slapner the saddle did lay}{\Bfootnote{Possibly formulaic; cf. \textlink{Oddrunargratr}[2]/4.}}, &
ręið \alst{n}iðr þaðan \hld\ \alst{n}ifl-hęljar til; &
mǿtti \edtrans{\alst{h}velpi, \hld\ þęim’s ór \alst{h}ęlju kom}{the whelp that came out of Hell}{\Bfootnote{An otherwise unknown dog, sometimes identified with \inx[P]{Garm}.  The “hellhound” guarding the underworld is well known from world mythology, most famously the Greek \emph{Kérberos}.}}.\eva

\bvb Up rose Weden, the ancient Geat, \\
and he on \inx[P]{Slapner} the saddle did lay; \\
rode down thence to \inx[P]{Nivelhell}; \\
met the whelp that came out of Hell.\evb\evg


\bvg\bva\mssnote{\AM~1v/21}%
Sá vas \alst{b}lóðugr \hld\ of \alst{b}rjóst framan, &
ok \edtrans{\alst{g}aldrs fǫður}{the father of \inx[C]{galder} \ken*{= Weden}}{\Bfootnote{An unparalleled expression for Weden, the master of magic spells and chants (galders).  Cf. \textlink{Havamal}[140], 147–165.}} \hld\ \edtext{\alst{g}ól \emph{of}}{\Afootnote{\emph{‘golv’} (corruption of \emph{*gol ū}) \AM}} lęngi, &
\edtrans{\alst{f}ramm ręið Óðinn, \hld\ \alst{f}old-vegr dunði}{Forth rode Weden—the fold-way \ken{earth} resounded}{\Bfootnote{Cf. the description of \inx[P]{Thunder}’s riding in \Haustlong\ 14: \emph{Ók at ísarn-lęiki \hld\ Jarðar sunr ok dunði \dots\ mȧna vegr und hǫ̇num} ‘The Son of Earth drove to the iron-play \ken{battle} \dots, and the moon’s way \ken{sky/heaven} resounded beneath him’); see further \textlink{Thrymskvida}[21].}}, &
\alst{h}ann kom at \alst{h}ǫ́u \hld\ \alst{H}ęljar ranni.\eva

\bvb It was bloody on the front of its chest, \\
and at the father of \inx[C]{galder} \ken*{= Weden} for a long time bayed. \\
Forth rode Weden—the fold-way \ken{earth} resounded— \\
he came to the high house of Hell.\evb\evg


\bvg\bva\mssnote{\AM~1v/22}%
Þȧ ręið \alst{Ó}ðinn \hld\ fyr \alst{au}stan dyrr, &
þar’s hann \alst{v}issi \hld\ \alst{v}ǫlu lęiði; &
nam hann \alst{v}ittugri \hld\ \edtrans{\alst{v}al-galdr}{slain-galder}{\Bfootnote{A galder to quicken the dead, in this case the wallow.  Cf. \textlink{Havamal}[158] where Weden tells how He can bring hanged men back to life with a galder.}} kveða, &
und’s \alst{n}auðug ręis, \hld\ \alst{n}ás orð of kvað:\eva

\bvb Then rode Weden east from the door \\
whither he knew the wallow’s grave. \\
He began for the cunning woman to sing a slain-\inx[C]{galder}, \\
until forced she rose, a corpse’s words quoth:\evb\evg


\bvg\bva\mssnote{\AM~1v/24}%
„Hvat ’s \alst{m}anna þat \hld\ \alst{m}ér ȯ·kunnra, &
es mér hęfr \alst{au}kit \hld\ \edtrans{\alst{ę}rfitt sinni}{this toilsome journey}{\Bfootnote{i.e. the journey out of the grave.}}? &
\edtext{Vas’k \alst{s}nifin \alst{s}njóvi, \hld\ ok \alst{s}lęgin regni, &
ok \alst{d}rifin \alst{d}ǫggu, \hld\ \alst{d}auð vas’k lęngi.}{\lemma{Vas’k snifin \dots\ lęngi. ‘I was snowed \dots\ long.’}\Bfootnote{Cf. the similar description of a buried person in \textlink{HelgakvidaTwo}[44]–45.}}“\eva

\bvb “What sort of man is this, to me unknown, \\
who has caused for me this toilsome journey? \\
I was snowed by snow and struck by rain, \\
and bespattered with dew—dead was I for long.”\evb\evg


\bvg\bva\mssnote{\AM~1v/25}%
\speakernote{[Óðinn kvað:]}%
„\alst{V}eg·tamr ek hęiti, \hld\ sonr em’k \alst{V}al·tams, &
sęg mér ór \alst{h}ęlju, \hld\ ek man ór \alst{h}ęimi; &
hvęim eru \alst{b}ękkir \hld\ \alst{b}augum sánir, &
\alst{f}lęt \alst{f}agr⸗liga \hld\ \alst{f}lóuð golli?“\eva

\bvb\speakernoteb{[Weden quoth:]}%
“Waytame am I called, I am Waltame’s son; \\
tell me [the news] from Hell—I will [tell those] from the world. \\
For whom are the benches sown with \inx[C]{bigh}[bighs], \\
the floors fairly flooded with gold?”\evb\evg


\bvg\bva\mssnote{\AM~1v/27}%
\speakernote{[Vǫlva kvað:]}%
„Hér stęndr \alst{B}aldri \hld\ of \alst{b}rugginn mjǫðr, &
\alst{sk}írar vęigar, \hld\ liggr \alst{sk}jǫldr yfir, &
ęn \alst{ǫ̇}s-męgir \hld\ ï \alst{o}f-vę̇ni; &
\alst{n}auðug sagða’k, \hld\ \alst{n}ú mun’k þęgja.“\eva

\bvb\speakernoteb{[The wallow quoth:]}%
“Here for Balder mead stands brewed, \\
pure draughts—a shield lies over them; \\
but the os-lads \ken*{= Eese} [stand] in great suspense— \\
forced I spoke, now I will shut up!”\evb\evg


\bvg\bva\mssnote{\AM~1v/29}%
\speakernote{[Óðinn kvað:]}%
„\alst{Þ}ęgj⸗at-tu vǫlva, \hld\ \alst{þ}ik vil’k fregna, &
\alst{u}nd’s \alst{a}l-kunna, \hld\ vil’k \alst{ę}nn vita: &
hvęrr man \alst{B}aldri \hld\ at \alst{b}ana verða, &
ok \alst{Ó}ðins son \hld\ \alst{a}ldri rę̇na?“\eva

\bvb\speakernoteb{[Weden quoth:]}%
“Shut not up, wallow—thee I wish to ask! \\
Until all is known I wish yet to know: \\
Who will become Balder’s bane \\
and rob Weden’s son of life?”\evb\evg


\bvg\bva\mssnote{\AM~2r/1}%
\speakernote{[Vǫlva kvað:]}%
„\alst{H}ǫðr berr \alst{h}ǫ́van \hld\ \edtext{\alst{h}róðr-ba\emph{ð}m}{\Afootnote{emend.; \emph{hróðr-barm} \AM}} þinig, &
hann man \alst{B}aldri \hld\ at \alst{b}ana verða, &
ok \alst{Ó}ðins son \hld\ \alst{a}ldri rę̇na; &
\alst{n}auðug sagða’k, \hld\ \alst{n}ú mun’k þęgja.“\eva

\bvb\speakernoteb{[The wallow quoth:]}%
“\inx[P]{Hath} bears the high glory-beam \ken{mistletoe} thither; \\
he will become Balder’s bane \\
and rob Weden’s son of life— \\
forced I spoke, now I will shut up!”\evb\evg


\bvg\bva\mssnote{\AM~2r/3}%
\speakernote{[Óðinn kvað:]}%
„\alst{Þ}ęgj⸗at-tu vǫlva, \hld\ \alst{þ}ik vil’k fregna, &
\alst{u}nd’s \alst{a}l-kunna, \hld\ vil’k \alst{ę}nn vita, &
hvęrr man \alst{h}ęipt \alst{H}ęði \hld\ \alst{h}ęfnt of vinna, &
eða \alst{B}aldrs \alst{b}ana \hld\ ȧ \alst{b}ál vega?“\eva

\bvb\speakernoteb{[Weden quoth:]}%
“Shut not up, wallow—thee I wish to ask! \\
Until all is known I wish yet to know: \\
Who will avenge that evil on Hath, \\
or cast on the pyre Balder’s bane?”\evb\evg


\bvg\bva\mssnote{\AM~2r/4}%
\speakernote{[Vǫlva kvað:]}%
„\edtrans{\emph{\alst{V}}rindr}{Wrind}{\Afootnote{metr. emend.; \emph{Rindr} \AM}\Bfootnote{As the first nominal in the verse, ms. \emph{Rindr} is expected to carry alliteration with \emph{vestr-sǫlum}.  This is obtained by restoring an archaic initial \emph{vr-} cluster, for which cf. \textlink{Havamal}[26]/2~n.  Unlike other instances of this cluster *\emph{Vrindr} is etymologically obscure.  A likely theory connects the name with Gutnish \emph{rind} ‘ivy, common clubmoss’, in which case Wrind is probably a minor plant goddess \parencite{Lundberg1913}.
\citeauthor{Lundberg1913} further argues that worship of Wrind is indicated by the Old Swedish place name \emph{Vrinda-vi} (modern-day Vrinnevi, supposedly ‘Wrind’s wigh’), although \textcite[78--80]{Sahlgren1924} gives an alternative etymology for the same.
In any case there appears to be some obscure botnanical, probably cultic significance to the fact that it is the son of the Ivy who will avenge the harm done by the Mistletoe.}} berr \edtrans{\emph{\alst{V}ȧla}}{Wonnel}{\Afootnote{required by meter and sense; om. \AM}\Bfootnote{The name must be restored.  Without it the verse is only two syllables long, which is absolutely forbidden in Norse \Fornyrdislag\ meter, and the verb \emph{berr} ‘bears, gives birth to’ is left without an object.}} \hld\ ï \alst{v}estr-sǫlum, &
\edtext{sá man \alst{Ó}ðins sonr \hld\ \alst{ęi}n-nę́ttr vega; &
\alst{h}ǫnd of þvę́r-\edtext{\emph{at}}{\Afootnote{required by meter and sense; om. \AM}} \hld\ né \alst{h}ǫfuð kęmbir, &
áðr ȧ \alst{b}ál of \alst{b}err \hld\ \alst{B}aldrs and·skota;}{\lemma{sá \dots\ and-skota ‘he will \dots\ shooter’}\Bfootnote{These lines are, apart from the tense of the verbs, almost identical to \textlink{Voluspa}[32]/4–33/2.  The direction of influence is unclear. It Is also possible that both are borrowing from a now-lost third poem.}} &
\alst{n}auðug sagða’k, \hld\ \alst{n}ú mun’k þęgja.“\eva

\bvb\speakernoteb{[The wallow quoth:]}%
“Wrind bears \inx[P]{Wonnel} in the western halls: \\
he will, Weden’s son, one night old, fight. \\
He washes not his hand nor combs his head \\
before onto the pyre he bears Balder’s shooter— \\
forced I spoke, now I will shut up.”\evb\evg


\bvg\bva\mssnote{\AM~2r/6}%
\speakernote{[Óðinn kvað:]}%
\Ballnote{According to \Gylfaginning\ 49, Hell promised to give Balder back to the Eese if “all things in the world, living and dead, would cry for him”.  The Eese relayed this message, and “the men and the beasts and the earth and the stones and trees and all metals” cried for Balder.  It may be that these maidens were among the grievers (perhaps they were the Walkirries, and this is what reveals Weden’s identity?) but their identity is otherwise unknown.  They may perhaps be identified with the maidens in \textlink{Vafthrudnismal}[49].}%
„\alst{Þ}ęgj⸗at-tu vǫlva, \hld\ \alst{þ}ik vil’k fregna, &
\alst{u}nd’s \alst{a}l-kunna, \hld\ vil’k \alst{ę}nn vita, &
hvęrjar ’ro \alst{m}ęyjar, \hld\ es at \alst{m}uni gráta &
ok \edtrans{ȧ \alst{h}imin verpa \hld\ \alst{h}alsa-skautum}{onto heaven throw the front-sheets}{\Bfootnote{Unexplained.  Perhaps an astrological allusion.}}?“\eva

\bvb\speakernoteb{[Weden quoth:]}%
“Shut not up, wallow—thee I wish to ask! \\
Until all is known I wish yet to know: \\
Which are the maidens that heartily weep, \\
and onto heaven throw the front-sheets?”\evb\evg


\bvg\bva\mssnote{\AM~2r/8}%
\speakernote{[Vǫlva kvað:]}%
„\alst{E}st⸗at Veg·tamr, \hld\ sęm \alst{e}k hugða, &
hęldr est \alst{Ó}ðinn, \hld\ \alst{a}ldinn gautr!“ &
\speakernote{[Óðinn kvað:]}%
„Est⸗at \alst{v}ǫlva \hld\ né \alst{v}ís kona, &
hęldr est \alst{þ}riggja \hld\ \alst{þ}ursa móðir!“\eva

\bvb\speakernoteb{[The wallow quoth:]}%
“Thou art not Waytame as I thought, \\
rather art thou Weden, the ancient Geat!”— \\
\speakernoteb{[Weden quoth:]}%
“Thou art no \inx[C]{wallow} nor wise woman, \\
rather art thou three \inx[P]{Thurses}’ mother!”\evb\evg


\bvg\bva\mssnote{\AM~2r/9}%
\speakernote{[Vǫlva kvað:]}%
„\alst{H}ęim ríð Óðinn \hld\ \edtrans{ok ves \alst{h}róðigr}{and be renowned}{\Bfootnote{A sarcastic taunt, the sense being: “Your fame, Weden, will not save you!”}}, &
svá komi⸗t \alst{m}anna \hld\ \alst{m}ęirr aptr ȧ vit, &
es \alst{l}auss \alst{L}oki \hld\ \alst{l}íðr ór bǫndum &
ok \alst{r}agna \alst{r}ǫk \hld\ \edtrans{\alst{r}júfęndr}{rippers}{\Bfootnote{Presumably Surt and Lock with his children, as described in \textlink{Voluspa}[40] ff. — The verb \emph{rjúfa} ‘\CV: to break, rip up, break a hole in’ is used in the same context in the formulaic \emph{þȧ’s/und’s (of) rjúfask ręgin} ‘when/until the \inx[P]{Reins} are ripped’ (\textlink{Vafthrudnismal}[52], \textlink{Grimnismal}[4], \textlink{Lokasenna}[41] and \textlink{Sigrdrifumal}[17]).}} koma.“\eva
%NOTE: If printing mythological Eddic poems separately.  One may also compare the similar sounding (but not or only very distantly related) verb \emph{rifna} ‘be riven, rent apart’ used in reference to the destruction of the world in Runic inscription Sö 154 (Skarpåker), and Arn \emph{Hryn} (in \Skp\ 2 pp. 185–6, ll. 3/7–8, see also note there): \emph{meiri verði þinn an þeira \hld\ þrifnuðr allr, und’s himinn rifnar.} ‘greater than theirs may thy whole wealth be, until heaven is riven.’}} koma.“

\bvb\speakernoteb{[The wallow quoth:]}%
“Ride home, Weden, and be renowned! \\
So may no man come again to visit [me] \\
when, loose, Lock slips out of his bonds,\\
and at the \inx[P]{Rakes of the Reins} the rippers come.”\evb\evg


%TODO Late stanzas in paper manuscripts.

\sectionline
