\bookStart{The Speeches of Fathomer}[Fáfnismǫ́l]

\begin{flushright}%
Dating \parencite{Sapp2022}: C10th (0.442), early C11th (0.402), late C11th (0.155)

Meter: \Ljodahattr\ (TODO)%
\end{flushright}

\sectionline

\bvg
\bva „Svęinn ok svęinn! \hld\ Hvęrjum estu svęini of borinn? &
\ind Hvęrra estu manna mǫgr? &
es þú á Fáfni rautt \hld\ þínn hinn frána mę́ki; &
\ind stǫndumk til hjarta hjǫrr!“\eva

\bvb {[Fathomer quoth:]} “Swain and swain! To which swain art thou born; of which men art thou the son? As thou on Fathomer hast reddened thy gleaming blade, the sword stands to my the heart!”\evb
\evg


\bpg\bpa Sigurðr dulði nafns síns fyr því at þat var trúa þeira í forneskju at orð feigs manns mę́tti mikit ef hann bǫlvaði óvin sínum með nafni. Hann kvað:\epa

\bpb Siward concealed his name, because it was their belief in ancient times that the word of a \inx[C]{fey} man could do much if he baled his enemy by his name. He \ken*{= Siward} quoth:\epb\epg


\bvg
\bva „Gǫfugt dýr ek hęiti \hld\ en ek gęngit hef’k &
\ind hinn móðurlausi mǫgr, &
fǫður ek á’kk-a \hld\ sem fira synir, &
\ind gęng ek ęinn saman.“\eva

\bvb “Noble beast I am called, but I have walked as the motherless lad. A father I own not, like the sons of men do; I walk alone.”\evb
\evg


\bvg
\bva „Vęizt, ef fǫður né átt-at \hld\ sem fira synir, &
\ind af hvęrju vastu undri alinn?
[...]“\eva

\bvb {[Fathomer quoth:]} “Knowest thou, if thou haddest not a father like the sons of men, by which wonder thou wast born?”\evb
\evg


\bvg
\bva „Ę́tterni mitt \hld\ kveð’k þér ókunnigt vesa &
\ind ok mik sjalfan hit sama: &
Sigurðr ek hęiti \hld\ Sigmundr hét minn faðir &
\ind es hęf’k þik vápnum vegit.“\eva

\bvb {[Siward quoth:]} “My lineage I declare is unknown to thee, and my self the same.\footnoteB{The meaning is that Fathomer would not recognize Siward’s lineage (i.e. his father) or name, since he is an orphan who up until this point has not won any glory. He is not saying that he is lineage is unknown even to himself, since \emph{sjalfan mik} ‘my self’ is accusative, not dative.} Siward I am called—Syemund was called my father—who with weapons have struck thee.”\evb
\evg


\bvg
\bva „Hvęrr þik hvatti, \hld\ hví hvętjask lézt, &
\ind mínu fjǫrvi at fara? &
Hinn fránęygi svęinn, \hld\ þú áttir fǫður bitran, &
\ind ábornu skjór á skęið.“\eva

\bvb {[Fathomer quoth:]} “Who goaded thee—why didst thou let thyself be goaded—my life for to destroy? Gleaming-eyed swain, thou haddest a sharp father; inborn traits show quickly.\footnoteB{The original is unclear. \emph{á skęið} means roughly ‘rapidly, quickly’, whence the expression \emph{ríða á skęið} ‘\CV: to ride at full speed’, but the other words are uncertain. \textcite{LaFargeGlossary} read ‘your innate qualities show quickly’, suggesting two unattested words: an adjective \emph{*áborinn} ‘innate, inborn’ and a verb \emph{*skjóa} ‘to show’. Yet the lack of i-umlaut in the supposed 3rd sg. pres. ind. \emph{skjór} is difficult. We would expect \emph{**skýr}, as in \emph{skjóta} ‘to shoot,’ with 2nd/3rd sg. pres. ind \emph{skýtr}. A solution here would be reading a 2nd sg. pres. subj. \emph{skjóir}, with a vowel TODO}”\evb
\evg


\bvg
\bva „Hugr mik hvatti, \hld\ hendr mér fulltýðu &
\ind ok minn inn hvassi hjǫrr; &
fár es hvatr \hld\ es hrøðask tękr &
\ind ef í barnǿsku ’s blauðr.“\eva

\bvb {[Siward quoth:]} “My heart goaded me, my hands assisted me, and this my sharp sword—few”\evb
\evg


TODO: More verses...


\bvg
\bva „Hęiptyrði ęin \hld\ tęlr þú þér í hvívętna &
\ind en ek þér satt ęitt sęgi’k: &
It gjalla gull \hld\ ok it glóðrauða fé, &
\ind þér verða þęir baugar at bana!“\eva

\bvb {[Fathomer quoth:]} “With hateful words alone answerest thou anything, but I say to thee truth alone: The resounding gold and the glowing red fee, those bighs will become thy bane!”\evb
\evg


\bvg
\bva „Féi ráða \hld\ skal fyrða hvęrr &
\ind ę́ til ins ęina dags &
því-at ęinu sinni \hld\ skal alda hvęrr &
\ind fara til hęljar heðan.“\eva

\bvb {[Siward quoth:]} “Rule [his] fee shall every man, always, until the one day; for at one time must every man journey hence to Hell.\footnoteB{Siward dismisses the idea of the curse. He must die regardless of whether he takes the gold or not, and he would rather die wealthy and famous than poor and unknown.}”\evb
\evg


\bvg
\bva „Norna dóm \hld\ munt \edtrans{fyr nęsjum}{before the headlands}{\Bfootnote{Formulaic, the sense is that the doom of the norns is close at hand (TODO: How do other scholars explain this?). Cf. the last st. of Sonatorrek (TODO).}} hafa &
\ind ok ósvinns apa; &
í vatni þú drukknar \hld\ ef í vindi rę́r; &
\ind allt es fęigs forað.“\eva

\bvb {[Fathomer quoth:]} “The doom of the Norns shalt thou have before the headlands, and that of an unwise ape. In water [wilt] thou drown if thou row in wind; everything is the pit of the \inx[C]{fey}.\footnoteB{That is, the cursed, death-doomed (fey) man will find sudden death no matter where he turns.}”\evb
\evg


\bvg
\bva „Seg-ðu mér, Fáfnir, \hld\ allz þik fróðan kveða &
\ind ok vęl mart vita: &
Hvęrjar ’ru þę́r nornir \hld\ es nauðgǫnglar ’ru &
\ind ok kjósa mǿðr frá mǫgum?“\eva

\bvb “Say to me, Fathomer, as they call the wise, and knowing well enough: Which are those Norns who are TODO, and choose the mothers from their lads?”\evb
\evg
