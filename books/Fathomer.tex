\bookStart{The Speeches of Fathomer}[Fáfnismǫ́l]

\begin{flushright}%
Dating \parencite{Sapp2022}: C10th (0.442), early C11th (0.402), late C11th (0.155)

Meter: \Ljodahattr\ (TODO)%
\end{flushright}

Titled \emph{Frá dauða Fáfnis} ‘From Fathomer’s death’ in \Regius.

\sectionline

\bvg\bva „Svęinn ok svęinn! \hld\ Hvęrjum est svęini of borinn? &
\ind Hvęrra est manna mǫgr? &
es þú á Fáfni rautt \hld\ þínn hinn frána mę́ki; &
\ind stǫndumk til hjarta hjǫrr!“\eva

\bvb {[Fathomer quoth:]} \\
“O swain and swain! To which swain art thou born; \\
of which men art thou son? \\
As thou on Fathomer hast reddened thy gleaming blade, \\
the sword stands unto my heart!”\evb\evg


\bpg\bpa Sigurðr dulði nafns síns fyr því at þat var trúa þeira í forneskju at orð feigs manns mę́tti mikit ef hann bǫlvaði óvin sínum með nafni. Hann kvað:\epa

\bpb Siward concealed his name, because it was their belief in ancient times that the word of a \inx[C]{fey} man could do much if he baled his enemy by his name. He \ken*{= Siward} quoth:\epb\epg


\bvg\bva „Gǫfugt dýr ek hęiti \hld\ en ek gęngit hef’k &
\ind hinn móður-lausi mǫgr, &
fǫður ek á’kk-a \hld\ sem fira synir, &
\ind gęng ek ęinn saman.“\eva

\bvb “Noble Deer am I called, and I have gone \\
as the motherless lad. \\
A father I have not like the sons of men; \\
I go alone.”\evb\evg


\bvg\bva „Vęitst, ef fǫður né átt-at \hld\ sem fira synir, &
\ind af hvęrju vastu undri alinn?
[...]“\eva

\bvb {[Fathomer quoth:]} \\
“Dost thou know, if thou hast no father, like do the sons of men, \\
by which wonder thou wast begotten?”\evb\evg


\bvg\bva „Ę́tterni mitt \hld\ kveð’k þér ó·kunnigt vesa &
\ind ok mik sjalfan hit sama: &
Sigurðr ek hęiti \hld\ Sigmundr hét minn faðir &
\ind es hęf’k þik vǫ́pnum vegit.“\eva

\bvb {[Siward quoth:]} \\
“My lineage I declare is unknown to thee, \\
and my self the same.\footnoteB{The meaning is that Fathomer would not recognize Siward’s lineage (i.e. his father) or name, since he is an orphan who up until this point has not won any glory. He is not saying that he is lineage is unknown even to himself, since \emph{sjalfan mik} ‘my self’ is accusative, not dative.} \\
Siward am I called—Syemund was called my father— \\
who with weapons have struck thee.”\evb\evg


\bvg\bva „Hvęrr þik hvatti, \hld\ hví hvętjask lést, &
\ind mínu fjǫrvi at fara? &
Hinn frán-ęygi svęinn, \hld\ þú áttir fǫður bitran, &
\ind \edtrans{á-bornu skjór á skęið.}{inborn traits quickly show.}{\Bfootnote{The original is cryptic.  \emph{á skęið} means roughly ‘rapidly, quickly’, whence the expression \emph{ríða á skęið} ‘\CV: to ride at full speed’, but the other words are uncertain.  \textcite{LaFargeGlossary} read ‘your innate qualities show quickly’, suggesting two unattested words: an adjective \emph{*áborinn} ‘innate, inborn’ and a verb \emph{*skjóa} ‘to show’. Yet the lack of i-umlaut in the supposed 3rd sg. pres. ind. \emph{skjór} is difficult. We would expect \emph{**skýr}, as in \emph{skjóta} ‘to shoot,’ with 2nd/3rd sg. pres. ind \emph{skýtr}. A solution here would be reading a 2nd sg. pres. subj. \emph{skjóir}, with a vowel TODO}}“\eva

\bvb {[Fathomer quoth:]} \\
“Who goaded thee—why didst thou let thee be goaded— \\
my life for to destroy? \\
O gleaming-eyed swain, thou haddest a sharp father; \\
inborn traits quickly show!”\evb\evg


\bvg\bva „Hugr mik hvatti, \hld\ hendr mér full-týðu &
\ind ok minn inn hvassi hjǫrr; &
fár es hvatr \hld\ es hrøðask tękr &
\ind ef í barnǿsku ’s blauðr.“\eva

\bvb {[Siward quoth:]} \\
“My heart goaded me, my hands availed me, \\
and this my sharp sword. \\
Few a man is brave when he takes to grow, \\
if in youth he be soft.”\evb\evg


\bvg\bva „Vęit’k, ef þú vaxa nę́ðir \hld\ fyr þinna vina brjósti, &
\ind séi-t maðr þik vręiðan vega; &
nú est haptr \hld\ ok hęr-numinn, &
\ind ę́ kveða bandingja bifask.“\eva

\bvb {[Fathomer quoth:]} \\
“TRANSLATION”\evb\evg


\bvg\bva „Því bregðr þú nú mér, Fáfnir, \hld\ at til fjarri sjá’k &
\ind mínum fęðr-munum, &
ęigi em’k haptr \hld\ þótt vę́ra hęr-numi; &
\ind þú fannt, at ek lauss lifi!“\eva

\bvb {[Siward quoth:]} \\
“TRANSLATION”\evb\evg


\bvg\bva „Hęipt-yrði ęin \hld\ tęlr þú þér í hví-vętna &
\ind en ek þér satt ęitt sęgi’k: &
It gjalla gull \hld\ ok it glóð-rauða fé, &
\ind þér verða þęir baugar at bana!“\eva

\bvb {[Fathomer quoth:]} \\
“With only hateful words dost thou answer anything, \\
but I tell thee truth alone: \\
The resounding gold and the glowing red wealth, \\
those bighs will be thy bane!”\evb\evg


\bvg\bva „Féi ráða \hld\ skal fyrða hvęrr &
\ind ę́ til \edtrans{ins ęina dags}{the one day}{\Bfootnote{i.e. his predetermined time of death.  Siward dismisses the idea of the curse, since he knows that he will die regardless of whether he takes the gold or not; and he would rather die rich and famous than wretched and forgotten.}} &
því-at ęinu sinni \hld\ skal alda hvęrr &
\ind fara til hęljar heðan.“\eva

\bvb {[Siward quoth:]} \\
“Rule [his] wealth shall every man, \\
always, until the one day; \\
for at one time must every man \\
journey hence to Hell.”\evb\evg


\bvg\bva „Norna dóm \hld\ munt \edtrans{fyr nęsjum}{before the headlands}{\Bfootnote{i.e. ‘close at hand, imminent’.  A formulaic expression for imminent death, cf. the last st. of Sonatorrek (TODO).}} hafa &
\ind ok ó·svinns apa; &
í vatni þú drukknar \hld\ ef í vindi rę́r; &
\ind allt es fęigs forað.“\eva

\bvb {[Fathomer quoth:]} \\
“The doom of the Norns shalt thou have before the headlands, \\
and that of an unwise ape. \\
In water wilt thou drown if thou row in wind; \\
everything is the pit of the \inx[C]{fey}.\footnoteB{That is, the cursed, death-doomed (fey) man will find sudden death no matter where he turns.}”\evb\evg


\bvg\bva „Sęg mér, Fáfnir, \hld\ alls þik fróðan kveða &
\ind ok vęl mart vita: &
Hvęrjar ’ru þę́r nornir \hld\ \edtrans{es nauð-gǫnglar ’ru}{that attend in need}{\Bfootnote{lit. ‘who are attendant in need’, i.e. who help ailing mothers during childbirth.  Cf. \Sigrdrifumal\ 8.}} &
\ind ok kjósa mǿðr frá mǫgum?“\eva

\bvb {[Siward quoth:]} \\
“Say to me, Fathomer, as they call thee wise, \\
and knowing well enough: \\
Who are the Norns that attend in need, \\
and choose mothers from their lads?”\evb\evg


\bvg\bva „Sundr-bornar mjǫk \hld\ hygg at nornir sé, &
\ind ęigu-t þę́r ę́tt saman; &
sumar ’ru ás-kunngar, \hld\ sumar alf-kunngar, &
\ind sumar dǿtr Dvalins.“\eva

\bvb {[Fathomer quoth:]} \\
“Of very sundry birth I judge the norns to be; \\
they come not from a common lineage: \\
Some are begotten of the Eese, some begotten of the Elves, \\
some are the daughters of Dwollen \ken{dwarfs}.”\evb\evg


\bvg\bva „Sęg mér þat, Fáfnir, \hld\ alls þik fróðan kveða &
\ind ok vęl margt vita, &
hvé sá holmr hęitir \hld\ es blanda hjǫr-lęgi &
\ind Surtr ok ę́sir saman.“\eva

\bvb {[Siward quoth:]} \\
“Say to me, Fathomer, as they call thee wise, \\
and knowing well enough: \\
What is the islet called, where Surt and the Eese \\
blend sword-water \ken{blood} together?”\evb\evg


\bvg\bva „Ó·skópnir hęitir \hld\ en þar ǫll skulu &
\ind gęirum lęika goð; &
Bil-rǫst brotnar \hld\ es á brott fara &
\ind ok svima í móðu marir.\eva

\bvb {[Fathomer quoth:]} \\
“Unshopner it is called, and there shall all \\
the Gods play with spears; \\
Bilrest shatters when they fare away, \\
and the horses swim in the sea.\evb\evg

\sectionline

Fathomer continues speaking, but there is probably something missing here, since the transition is abrupt. Between its paraphrases of st. 15 and of st. 16, \VolsungaMS\ has \emph{Ok enn mę́lti Fáfnir: „Reginn bróðir minn veldr mínum dauða, ok þat hlę́gir mik, er hann veldr ok þínum dauða, ok ferr þá, sem hann vildi.“} ‘And further spoke Fathomer: “My brother Rein causes my death, and it gladdens me that he also causes thy death, and then it will go like he has willed.”’, which may either be a paraphrase of a lost st., or an addition by the redactor.

\sectionline

\bvg\bva Ǿgis hjalm \hld\ bar’k of alda sonum &
\ind meðan of męnjum lá’k; &
ęinn rammari \hld\ hugðumk ǫllum vesa, &
\ind fann’k-a’k marga mǫgu.“\eva

\bvb A helmet of terror I carried over the sons of men \\
while on the rings I lay; \\
stronger than all I thought myself alone to be; \\
I did not find many men.”\evb\evg


\bvg %NOTE: Heavily formulaic.
\bva „Ǿgis hjalmr \hld\ bergr ęinu-gi &
\ind hvar’s skulu vręiðir vega; &
þá þat finnr \hld\ es með flęirum kømr &
\ind at ęngi es ęinna hvatastr.“\eva

\bvb {[Siward quoth:]} \\
“A helmet of terror saves no man, \\
wherever wroth men should fight; \\
then he finds, when among the many he comes, \\
that none is the boldest of all.”\evb\evg


\bvg\bva „Ęitri ek fnę́sta \hld\ es á arfi lá’k &
\ind miklum míns fǫður.“\eva

\bvb {[Fathomer quoth:]} \\
“Venom I snorted, while I lay on the great \\
inheritance of my father.”\evb\evg


\bvg\bva „Inn rammi ormr, \hld\ þú gørðir frę́s mikla &
\ind ok gatst harðan hug; &
\ind hęipt at męiri \hld\ verðr hǫlða sonum &
\ind at þann hjalm hafi.“\eva

\bvb {[Siward quoth:]} \\
“O mighty wyrm, thou madest a great snort, \\
and didst get a hard heart; \\
TODO.”\evb\evg


\bvg\bva „Rę́ð’k þér nú, Sigurðr, \hld\ en þú ráð nemir &
\ind ok ríð hęim heðan; &
it gjalla gull \hld\ ok it glóð-rauða fé, &
\ind þér verða þęir baugar at bana!“\eva

\bvb {[Fathomer quoth:]} \\
“I counsel thee now, O Siward—and thou oughtst to take the counsel, \\
and ride home, hence! \\
The resounding gold and the glowing red wealth, \\
those bighs will become thy bane!”\evb\evg


\bvg\bva „Ráð ’s þér ráðit \hld\ en ek ríða mun &
\ind til þęss gulls es í lyngvi liggr, &
en þú, Fáfnir, ligg \hld\ í fjǫr-brotum &
\ind \edtrans{þar’s þik Hęl hafi}{where Hell may have thee}{\Bfootnote{Formulaic. TODO.}}!“\eva

\bvb {[Siward quoth:]} \\
“Thy counsel has been counseled—but I will ride, \\
to the gold which in the heather lies; \\
but \emph{thou}, Fathomer, lie in the blood-tracks, \\
where Hell may have thee!”\evb\evg


\bvg% NOTE: Pun.
\bva „Ręginn mik réð, \hld\ hann þik ráða mun, &
\ind hann mun okkr verða bǫ́ðum at bana; &
fjǫr sitt láta \hld\ hygg at Fáfnir myni; &
\ind þitt varð nú męira męgin.“\eva

\bvb {[Fathomer quoth:]} \\
“Rein betrayed \emph{me}, he will betray \emph{thee}; \\
he will become the bane of us both; \\
give his life, I judge that Fathomer will; \\
thy strength was now the greater.”\evb\evg


\bpg
\bpa Reginn var á brott horfinn meðan Sigurðr vá Fáfni ok kom þá aptr er Sigurðr strauk blóð af sverðinu. Reginn kvað:\epa

\bpb Rein had gone away while Siward smote Fathomer, and then came back as Siward wiped the blood off the sword. Rein quoth:\epb
\epg


\bvg\bva „Hęill þú nú, Sigurðr, \hld\ nú hęfir sigr vegit &
\ind ok Fáfni of farit; &
manna þęira \hld\ es mold troða &
\ind þik kveð’k ó·blauðastan alinn.“\eva

\bvb {[SPEAKER quoth:]} \\
“Hail thee now, O Siward—now thou hast won victory \\
and Fathomer destroyed! \\
Of those men who tread on the earth \\
I declare \emph{thee} with least softness begotten.”\evb\evg


\bvg\bva „VERSE“\eva

\bvb {[SPEAKER quoth:]} \\
“TRANSLATION”\evb\evg

\sectionline
