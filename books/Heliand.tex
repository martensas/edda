\bookStart{Heliand}
%TODO. Check and remove all %TODO and %NOTE tags in the text.

Very much a work in progress.

The following is a complete list of relevant manuscript, in chronological order.
  %NOTE: Source: https://escholarship.org/content/qt19k1z5h8/qt19k1z5h8_noSplash_88c7cf5d8c1e26a95c76028adf012df9.pdf?t=mtfq0h p. 11.
  % See p. 66 for parallel texts with C, M, P and L.
  % Manuscript V: https://titus.fkidg1.uni-frankfurt.de/texte/etcs/germ/asachs/heliandv/helia.htm

\begin{itemize}
  \item L. 840–850 (Thomas 4073 (Ms)), which appears to originally have belonged to the same codex as
  \item P. 840–850 (R 56/2537 (PA))
  \item V. 800–850 (Palatini Latini 1447)
  \item S. 850 (cgm. 8840)
  \item M. 850–875 (cgm. 25)
  \item C. 950–1000 (Cotton Caligula A. VII sign. 3-11)
\end{itemize}

Fragments L and P appear to originally belong to the same codex?


Notes on the normalization:
  \begin{itemize}
    \item Long vowels are marked by the acute rather than by the circumflex accent or macron. This is both faithful to the original manuscripts and concordant with my practice in normalising other Germanic languages.
    \item Long vowels \emph{ô} and \emph{ê} resulting from monophthongisation of diphthongs \emph{au} and \emph{ai} are, however, written with the circumflex accent.
    \item \emph{ó} when coming from etymological \emph{a} or \emph{á} is written as \emph{ǫ́}
    \item When attested in all mss., epenthetic (svarabhakti) vowels are marked with an underdot. Otherwise they are deleted.
    \item Long vowels resulting from nasal assimilation are marked with an overdot. \emph{i} is written as \emph{ï}.
    \item ms. \emph{e} and \emph{i}, when occuring between vowels are written as \emph{j}.
    \item ms. \emph{i}, when word-initial or following \emph{g} and corresponding to etymological \emph{j} is written as \emph{j}
    \item ms. \emph{e} as resulting from \emph{i}-mutation is written as \emph{ę}.
    \item ms. \emph{b} or \emph{ƀ}, when representing the voiced bilabial fricative, is written as \emph{v}.
    \item ms. \emph{th} is written as \emph{þ}.
    \item ms. \emph{uu} is written as \emph{w}.
  \end{itemize}

\sectionline

\bvg\bva
\alst{M}anega wáron, \hld\ þe sia iro \alst{m}ód ge·spón, &
þat sia bi·gunnun word godes, &
\alst{r}ękkjan þat gi·\alst{r}úni, \hld\ þat þie \alst{r}íkjo Krist &
undar \alst{m}an-kunnja \hld\ \alst{m}áriða gi·frumida &
mid \alst{w}ordun ęndi mid \alst{w}erkun. \hld\ Þat wolda þó \alst{w}ísara filo &
\alst{l}iudo barno \alst{l}ovon, \hld\ \alst{l}êra Kristes, &
\alst{h}êlag word godas, \hld\ ęndi mid iro \alst{h}andon skrívan &
\alst{b}erẹht-líko an \alst{b}uok, \hld\ hwó sia is gi·\alst{b}od-skip skoldin &
\alst{f}rummjan, \alst{f}iriho barn. \hld\ Þan wárun þoh sia \alst{f}iori te þiu &
under þera \alst{m}ęnigo, \hld\ þia habdon \alst{m}aht godes, &
\alst{h}elpa fan \alst{h}imila, \hld\ \alst{h}êlagna gêst, &
\alst{k}raft fan \alst{K}riste; \hld\ sia wurðun gi·\alst{k}orana te þio, &
þat sie þan \alst{É}wangelium \hld\ \alst{ê}nan skoldun &
an \alst{b}uok skrívan \hld\ endo só manag gi·\alst{b}od godes, &
\alst{h}êlag \alst{h}imilisk word: \hld\ sia ne muosta \alst{h}ęliðo þan mêr, &
\alst{f}iriho barno \alst{f}rummjan, \hld\ newan þat sia \alst{f}iori te þio &
þuru \alst{k}raft godas \hld\ ge·\alst{k}orana wurðun, &
\alst{M}atheus ęndi \alst{M}arkus, \hld\ —só wárun þia \alst{m}an hêtana— &
Lukas ęndi \alst{J}ohannes; \hld\ sia wárun \alst{g}ode lieva, &
\alst{w}irðiga ti þem gi·\alst{w}irkje. \hld\ Habda im \alst{w}aldand god, &
þem \alst{h}ęliðon an iro \alst{h}ertan \hld\ \alst{h}êlagna gêst &
\alst{f}asto bi·\alst{f}olhan \hld\ ęndi \alst{f}erahtan hugi, &
só manag \alst{w}ís-lík \alst{w}ord \hld\ ęndi gi·\alst{w}it mikil, &
þat sea skoldin a·\alst{h}ębbjan \hld\ \alst{h}êlagaro stemnun &
\alst{g}od-spell þat \alst{g}uoda, \hld\ þat ni havit ênigan gi·\alst{g}adon hwęrgin, &
þiu \alst{w}ord an þesaro \alst{w}er-oldi, \hld\ þat io \alst{w}aldand mêr, &
\alst{d}rohtin \alst{d}iurje \hld\ efþo \alst{d}ervi þing, &
\alst{f}irin-werk \alst{f}ęllje \hld\ efþo \alst{f}íundo níð, &
\alst{st}ríd wiðer·\alst{st}ande—, \hld\ hwand hie habda \alst{st}arkan hugi, &
\alst{m}ildjan ęndi guodan, \hld\ þie þe \alst{m}êster was, &
\alst{a}ðal-\alst{o}rd-frumo \hld\ \alst{a}lo-mahtig. &
Þat skoldun sea fiori \hld\ þuo fingron skrívan, &
sęttjan ęndi singan \hld\ ęndi sęggjan forð, &
þat sea fan Kristes \hld\ krafte þem mikilon &
gi·sáhun ęndi gi·hôrdun, \hld\ þes hie selvo gi·sprak, &
gi·wísda ęndi gi·warạhta, \hld\ wundạr-líkas filo, &
só manag mid mannon \hld\ mahtig drohtin, &
all so hie it fan þem an·ginne \hld\ þuru is ênes kraht, &%TODO: Check kraht.
waldand gi·sprak, \hld\ þuo hie êrist þesa wer-old gi·skuop &
ęndi þuo all bi·fieng \hld\ mid ênu wordo, &
himil ęndi erða \hld\ ęndi al þat sea bi·hlidan êgun &
gi·warạhtes ęndi gi·wahsanes: \hld\ þat warð þuo all mid wordon godas &
fasto bi·fangan, \hld\ ęndi gi·frumid after þiu, &
hwi-lik þan liud-skępi \hld\ landes skoldi &
wídost gi·waldan, \hld\ efþo hwar þiu wer-old-aldar &
endon skoldin. \hld\ Ên was iro þuo noh þan &
firiho barnun bi·foran, \hld\ ęndi þiu fïvi wárun a·gangan: &
skolda þuo þat sehsta \hld\ sálig-líko &
kuman þuru kraft godes \hld\  ęndi Kristas gi·burd, &
hêlandero bestan, \hld\ hêlagas gêstes, &
an þesan middil-gard \hld\ managon te helpun, &
firjo barnon ti frumon \hld\ wið fíundo níð, &
wið dęrnero dwalm. \hld\ Þan habda þuo drohtin god &
Rómano-liudjon far·liwan \hld\ ríkjo mêsta, &
habda þem hęri-skipje \hld\ herta gi·sterkid, &
þat sia habdon bi·þwungana \hld\ þiedo gi·hwi-lika, &
habdun fan Rúmu-burg \hld\ ríki gi·wunnan &
helm-gi·trôstjon, \hld\ sáton iro hęri-togon &
an lando gi·hwem, \hld\ habdun liudjo gi·wald, &
allon ęli-þeodon. \hld\ Erodes was &
an Hjerusalem \hld\ over þat Judeono folk &
gi·koran te kuninge, \hld\ só ina þie kêser þarod, &
fon Rúmu-burg \hld\ ríki þiodan &
satta undar þat gi·sïði. \hld\ Hie ni was þoh mid sibbjon bi·lang &
avaron Israheles, \hld\ ęðili-gi·burdi, &
kuman fon iro knuosle, \hld\ newan þat hie þuru þes kêsures þank &
fan Rúmu-burg \hld\ ríki habda, &
þat im wárun só gi·hôriga \hld\ hildi-skalkos, &
avaron Israheles \hld\ ęlljan-ruova: &
swíðo un·wanda wini, \hld\ þan lang hie gi·wald êhta, &
Erodes þes ríkjas \hld\ ęndi rád-burdjon held &
Judeo liudi. \hld\ Þan was þar ên gi·gamalod mann, &
þat was fruod gomo, \hld\ habda ferehtan hugi, &
was fan þem liudjon \hld\ Lewias kunnes, &
Jakobas sunjas, \hld\ guodero þiedo: &
Zakharias was hie hêtan. \hld\ Þat was só sálig man, &
hwand hie simblon gerno \hld\ gode þeonoda, &
warạhta after is willjon; \hld\ deda is wíf só self &
—was iru gi·aldrod idis: \hld\ ni muosta im ęrvi-ward &
an iro juguð-hêdi \hld\ giviðig werðan— &
libdun im far·úter laster, \hld\ waruhtun lof goda, &
wárun só gi·hôriga \hld\ hevan-kuninge, &
diuridon u̇san drohtin: \hld\ ni weldun dęrvjas wiht &
under man-kunnje, \hld\ mênes gi·frummjan, &
ne *saka ne sundja; \hld\ was im þoh an sorgun hugi, &
þat sie ęrvi-ward \hld\ êgan ni móstun, &
ak wárun im barno-lôs. \hld\ Þan skolda he gi·bod godes &
þar an Hjerusalem, \hld\ só oft só is gi·gęngi gi·stód, &
þat ina torht-líko \hld\ tídi gi·manodun, &
só skolda he at þem wíha \hld\ waldandes geld &
hêlag bi·hwervan, \hld\ hevan-kuninges, &
godes jungar-skępi: \hld\ gern was he swíðo, &
þat he it þurh ferhtan hugi \hld\ frummjan mósti. &
Þó warð þiu tíd kuman, \hld\ —þat þar gi·tald habdun &%NOTE: Fitt 2.
wísa man mid wordun,— \hld\ þat skolda þana wíh godes &
Zakharias bi·sehan. \hld\ Þó warð þar gi·samnod filu &
þar te Hjerusalem \hld\ Judeo liudi, &
werodes te þem wíha, \hld\ þar sie waldand god &
swíðo þeo-líko \hld\ þiggjan skoldun, &
hêrron is huldi, \hld\ þat sie hevan-kuning &
lêðes a·léti. \hld\ Þea liudi stódun &
umbi þat hêlaga hús, \hld\ ęndi géng im þe gi·hêrodo man &
an þana wíh innan. \hld\ Þat werod ǫ́ðar bêd &
umbi þana alah útan, \hld\ Ebreo liudi, &
hwan êr þe fródo man \hld\ gi·frumid habdi &
waldandes willjon. \hld\ Só he þó þana wí-rôk dróg, &
ald aftar þem alaha, \hld\ ęndi umbi þana altari géng &
mid is rôk-fatun \hld\ ríkjun þionon, &
—fręmida ferht-líko \hld\ fráon sínes, &
godes jungar-skępi \hld\ gerno swíðo &
mid hluttru hugi, \hld\ *só man hêrren skal &
gerno ful-gangan—, \hld\ grurjos kwámun im, &
ęgison an þem alahe: \hld\ hie gi·sah þar aftar þiu ênna ęngil godes &
an þem wíhe innan, \hld\ hie sprak im mid is wordun tuo, &
hiet þat fruod gumo \hld\ foroht ni wári, &
hiet þat hie im ni and-riede: \hld\ þína dádi sind“, kwat-hie*, &
„waldanda werðe \hld\ ęndi þín word só self, &
þín þionost is im an þanke, \hld\ þat þú su·lika gi·þáht haves &
an is ênes kraft. \hld\ Ik is ęngil bium, &
Gabriel bium ik hêtan, \hld\ þe gio for goda standu, &
and-ward for þem alo-waldon, \hld\ ne sí þat he me an is ârundi hwarod &
sęndjan willja. \hld\ Nu hiet he me an þesan sïð faran, &
hiet þat ik þi þoh gi·ku̇ðdi, \hld\ þat þi kind gi·boran, &
fon þínera alderu idis \hld\ ôdan skoldi &
werðan an þesero wer-oldi, \hld\ wordun spáhi. &
Þat ni skal an is liva gio \hld\ líðes an·bítan, &
wínes an is wer-oldi: \hld\ só haved im wurd-gi·skapu, &
metod gi·markod \hld\ ęndi maht godes. &
Hét þat ik þi þoh sagdi, \hld\ þat it skoldi gi·sïð wesan &
hevan-kuninges, \hld\ hét þat git it heldin wel, &
tuhin þurh trewa, \hld\ kwað þat he im tíras só filu &
an godes ríkja \hld\ for·gevan weldi. &
He kwað þat þe gódo gumo \hld\ Johannes te namon &
hębbjan skoldi, \hld\ gi·bôd þat git it hétin só, &
þat kind, þan it kwámi, \hld\ kwað þat it Kristes gi·sïð &
an þesaro wídun wer-old \hld\ werðan skoldi, &
is selves sunjes, \hld\ ęndi kwað þat sie sliumo herod &
an is bod-skępi \hld\ bêðe kwámin.“ &
Zakharias þó gi·mahalda \hld\ ęndi wið selvan sprak &
drohtines ęngil, \hld\ ęndi im þero dádjo bi·gan, &
wundron þero wordo: \hld\ „hwó mag þat gi·werðan só“, kwað he, &
„aftar an aldre? \hld\ it is unk al te lat &
só te gi·winnanne, \hld\ só þú mid þínun wordun gi·sprikis. &
Hwanda wit habdun aldres \hld\ êr efno twên-tig &
wintro an unkro wer-oldi, \hld\ êr þan kwámi þit wíf te mi; &
þan wárun wit nu at-samna \hld\ ant·sivunta wintro &
gi·bęnkjon ęndi gi·będdjon, \hld\ sïðor ik sie mi te brúdi ge·kôs. &
Só wit þes an unkro juguði \hld\ gi·girnan ni mohtun, &
þat wit ęrvi-ward \hld\ êgan móstin, &
fódjan an unkun flęttja, \hld\ nu wit sus gi·fródod sint &
—havad unk ęldi bi·noman \hld\ ęlljan-dádi, &
þat wit sint an unkro siuni gi·slekit \hld\ ęndi an unkun sídun lat; &
flêsk is unk ant·fallan, \hld\ fel un·skóni, &
is unka lud gi·liðen, \hld\ lík gi·drusnod, &
sind unka and-bári \hld\ ǫ́ðar-líkaron, &
mód ęndi męgin-kraft—, \hld\ só wit giu só managan dag &
wárun an þesero wer-oldi, \hld\ só mi þes wundạr þunkit, &
hwó it só gi·werðan mugi, \hld\ só þú mid þínun wordun gi·sprikis. &
Þó warð þat heven-kuninges bodon \hld\ harm an is móde, &%NOTE: Fitt 3.
þat he is gi·werkes \hld\ só wundron skolda &
ęndi þat ni welda gi·huggjan, \hld\ þat ina mahta hêlag god &
só ala-jungan, \hld\ só he fon êrist was, &
selvo gi·wirkjan, \hld\ of he só weldi. &
Skerida im þó te wítja, \hld\ þat he ni mahte ênig word sprekan, &
gi·mahljen mid is mu̇ðu, \hld\ „êr þan þi magu wirðid, &
fon þínero aldero idis \hld\ erl a·fódit, &
kind-jung gi·boran \hld\ kunnjes gódes, &
wánum te þesero wer-oldi. \hld\ Þan skalt þú eft word sprekan, &
hębbjan þínaro stemna gi·wald; \hld\ ni þarft þú stum wesan &
lęngron hwíla.“ \hld\ Þó warð it sán gi·lêstid só, &
gi·worðan te wáron, \hld\ só þar an þem wíha gi·sprak &
ęngil þes alo-waldon: \hld\ warð ald gumo &
spráka bi·lôsit, \hld\ þoh he spáhan hugi &
bári an is breostun. \hld\ Bidun allan dag &
þat werod for þem wíha \hld\ ęndi wundrodun alla, &
bi·hwí he þar só lango, \hld\ lof-sálig man, &
swíðo fród gumo \hld\ fráon sínun &
þionon þorfti, \hld\ só þar êr ênig þegno ni deda, &
þan sie þar at þem wíha \hld\ waldandes geld &
folmon frumidun. \hld\ Þó kwam fród gumo &
út fon þem alaha. \hld\ Erlos þrungun &
náhor mikilu: \hld\ was im niud mikil, &
hwat he im sǫ́ð-líkes \hld\ sęggjan weldi, &
wísjan te wáron. \hld\ He ni mohta þó ênig word sprekan, &
gi·sęggjan þem gi·sïðja, \hld\ b·útan þat he mid is swíðron hand &
wísda þem weroda, \hld\ þat sie u̇ses waldandes &
lêra lêstin. \hld\ Þea liudi for·stódun, &
þat he þar habda gegnungo \hld\ god-kundes hwat &
for·sehen selvo, \hld\ þoh he is ni mahti gi·sęggjan wiht, &
gi·wísjan te wáron. \hld\ Þó habda he u̇ses waldandes &
geld gi·lêstid, \hld\ al só is gi·gęngi was &
gi·markod mid mannun. \hld\ Þó warð sán aftar þiu maht godes, &
gi·ku̇ðid is kraft mikil: \hld\ warð þiu kwán ôkan, &
idis an ira ęldju: \hld\ skolda im ęrvi-ward, &
swíðo god-kund gumo \hld\ giviðig werðan, &
barn an burgun. \hld\ Bêd aftar þiu &
þat wíf wurdi-gi·skapu. \hld\ Skrêd þe wintạr forð, &
géng þes gę́res gi·tal. \hld\ Johannes kwam &
an liudjo lioht: \hld\ lík was im skóni, &
was im fel fagar, \hld\ fahs ęndi naglos, &
wangun wárun im wlitige. \hld\ Þó fórun þar wíse man, &
snelle te·samne, \hld\ þea swásostun mêst, &
wundrodun þes werkes, \hld\ bi·hwí it gio mahti gi·werðan só, &
þat undar só aldun twêm \hld\ ôdan wurði &
barn an gi·burdjon, \hld\ ni wári þat it gi·bod godes &
selves wári: \hld\ af·suovun sie garo, &
þat it elkor só wán-lík \hld\ werðan ni mahti. &
Þó sprak þar ên gi·fródot man, \hld\ þe só filo konsta &
wísaro wordo, \hld\ habde gi·wit mikil, &
frágode niud-líko, \hld\ hwat is namo skoldi &
wesan an þesaro wer-oldi: \hld\ „mi þunkid an is wísu gi·lík &
iak an is gi·bárja, \hld\ þat he sí bętara þan wi, &
só ik wániu, \hld\ þat ina u̇s gegnungo god fon himila &
selvo sęndi“. \hld\ Þó sprak sán aftar &
þiu módar þes kindes, \hld\ þiu þana magu habda, &
þat barn an ire barme: \hld\ „hér kwam gi·bod godes“, kwað siu, &
„fernun gę́re, \hld\ furmon wordu &
gi·bôd, þat he Johannes \hld\ bi godes lêrun &
hêtan skoldi. \hld\ Þat ik an mínumu hugi ni gi·dar &
węndjan mid wihti, \hld\ of ik is gi·waldan mót“. &
Þó sprak ên gêl-hert man, \hld\ þe ira gaduling was: &
„ne hét êr gio·wiht só“, \hld\ kwað he, „aðal-boranes &
u̇ses kunnjes efþo knósles; \hld\ wita kiasan im ǫ́ðrana &
niud-samna namon: \hld\ he niate of he móti“. &
Þó sprak eft þe fródo man, \hld\ þe þar konsta filo mahljan: &
„ni givu ik þat te ráde“, \hld\ kwað he, „rinko neg·ênun, &
þat he word godes \hld\ węndjan bi·ginna; &
ak wita is þana fader frágon, \hld\ þe þar só gi·fródod sitit, &
wís an is wín-sęli: \hld\ þoh he ni mugi ênig word sprekan, &
þoh mag he bi bók-stavon \hld\ bréf ge·wirkjan, &
namon gi·skrívan“. \hld\ Þó he náhor géng, &
lęgda im êna bók an barm \hld\ ęndi bad gerno &
wrítan wís-líko \hld\ word-gi·merkjun, &
hwat sie þat hêlaga barn \hld\ hêtan skoldin. &
Þó nam he þia bók an hand \hld\ ęndi an is hugi þáhte &
swíðo gerno te gode: \hld\ Johannes namon &
wís-líko gi·wrêt \hld\ ęndi ôk aftar mid is wordu gi·sprak &
swíðo spáh-líko: \hld\ habda im eft is spráka gi·wald, &
gi·wittjas ęndi wísun. \hld\ Þat wíti was þó a·gangan, &
hard harm-skare, \hld\ þe im hêlag god &
mahtig makode, \hld\ þat he an is mód-sevon &
godes ni for·gáti, \hld\ þan he im eft sęndi is jungron tó. &
Þó ni was lang aftar þiu, \hld\ ne it al só gi·lêstid warð, &%NOTE: Fitt 4.
só he man-kunnja \hld\ managa hwíla, &
god alo-mahtig \hld\ for·geven habda, &
þat he is himilisk barn \hld\ herod te wer-oldi, &
si selves sunu \hld\ sęndjan weldi, &
te þiu þat he hér a·lôsdi \hld\ al liud-stamna, &
werod fon wítja. \hld\ Þó warð is wisbodo &
an Galilea-land, \hld\ Gabriel kuman, &
ęngil þes alo-waldon, \hld\ þar he êne idis wisse, &
muni-líka magað: \hld\ Maria was siu hêten, &
was iru þiorna gi·þigan. \hld\ Sea ên þegạn habda, &
Joseph gi·mahlit, \hld\ gódes kunnjes man, &
þea Dawides dohter: \hld\ þat was só diur-lík wíf, &
idis ant·hêti. \hld\ Þar sie þe ęngil godes &
an Nazareth-burg \hld\ bi namon selvo &
grótte gęgin-warde \hld\ ęndi sie fon gode kwędda: &
„Hêl wis þú, Maria“, \hld\ kwað he, „þú bist þínun hêrron liof, &
waldande wirðig, \hld\ hwand þú gi·wit haves, &
idis ęnstjo fol. \hld\ Þu skalt for allun wesan &
wívun gi·wíhit. \hld\ Ne have þú wêkan hugi, &
ne forhti þú þínun ferhe: \hld\ ne kwam ik þi te ênigun frêson herod, &
ne dragu ik ênig drugi·þing. \hld\ Þu skalt u̇ses drohtines wesan &
módar mid mannun \hld\ ęndi skalt þana magu fódjan, &
þes hôhon hevan-kuninges suno. \hld\ Þe skal hêljand te namon &
êgan mid ęldjun. \hld\ Neo ęndi ni kumid, &
þes wídon ríkjas gi·wand, \hld\ þe he gi·waldan skal, &
mári þeodan.“ \hld\ Þó sprak im eft þiu magað an·gęgin, &
wið þana ęngil godes \hld\ idiso skónjost, &
allaro wívo wlitigost: \hld\ „hwó mag þat gi·werðen só“, kwað siu, &
„þat ik magu fódje? \hld\ Ne ik gio mannes ni warð &
wís an mínera wer-oldi.“ \hld\ Þó habde eft is word garu &
ęngil þes alo-waldon \hld\ þero idisiu te·gęgnes: &
„an þi skal hêlag gêst \hld\ fon hevan-wange &
kuman þurh kraft godes. \hld\ Þanan skal þi kind ôdan &
werðan an þesaro wer-oldi; \hld\ waldandes kraft &
skal þi fon þem hôhoston \hld\ hevan-kuninge &
skadowan mid skimon. \hld\ Ni warð skónjera gi·burd, &
ne só mári mid mannun, \hld\ hwand siu kumid þurh maht godes &
an þese wídon wer-old.“ \hld\ Þó warð eft þes wíves hugi &
aftar þem ârundje \hld\ al gi·hworven &
an godes willjon. \hld\ „Þan ik hér garu standu“, kwað siu, &
„te su·likun ambaht-skępi, \hld\ só he mi êgan wili. &
Þiu bium ik þeot-godes. \hld\ Nu ik þeses þinges gi·trúon; &
werðe mi aftar þínun wordun, \hld\ al só is willjo sí, &
hêrron mínes; \hld\ nis mi hugi twífli, &
ne word ne wísa.“ \hld\ Só gi·fragn ik, þat þat wíf ant·féng &
þat godes ârundi \hld\ gerno swíðo &
mid leohtu hugi \hld\ ęndi mid gi·lôvon gódun &
ęndi mid hluttrun trewun; \hld\ warð þe hêlago gêst, &
þat barn an ira bósma; \hld\ ęndi siu ira breostun for·stód &
iak an ire sevon selvo, \hld\ sagda þem siu welda, &
þat sie habde gi·ôkana \hld\ þes alo-waldon kraft &
hêlag fon himile. \hld\ Þó warð hugi Josepes, &
is mód gi·worrid, \hld\ þe im êr þea magað habda, &
þea idis ant·hêttja, \hld\ aðal-knósles wíf &
gi·boht im te brúdju. \hld\ He af·sóf þat siu habda barn undar iru: &
ni wánda þes mid wihti, \hld\ þat iru þat wíf habdi &
gi·wardod só waro-líko: \hld\ ni wisse waldandes þó noh &
blíði gi·bod-skępi. \hld\ Ni welda sia imo te brúdi þó, &
halon imo te híwon, \hld\ ak bi·gan im þó an hugi þęnkjan, &
hwó he sie só for·léti, \hld\ só iru þar nu wurði lêdes wiht, &
ôdan arvides. \hld\ Ni welda sie aftar þiu &
meldon for męnigi: \hld\ antd-réd þat sie manno barn &
lívu bi·námin. \hld\ Só was þan þero liudjo þau &
þurh þen aldon êw, \hld\ Ebreo folkes, &
só hwi-lik só þar an un·reht \hld\ idis gi·híwida, &
þat siu simbla þana bed-skępi \hld\ buggjan skolda, &
frí mid ira ferhu: \hld\ ni was gio þiu fêmja só gód, &
þat siu mid þem liudun lęng \hld\ libbjen mósti, &
wesan undar þem weroda. \hld\ Bi·gan im þe wíso mann, &
swíðo gód gumo, \hld\ Joseph an is móda &
þęnkjan þero þingo, \hld\ hwó he þea þiornun þó &
listjun for·léti. \hld\ Þó ni was lang te þiu, &
þat im þar an drôma \hld\ kwam drohtines ęngil, &
hevan-kuninges bodo, \hld\ ęndi hét sie ina haldan wel, &
minnjon sie an is móde: \hld\ „Ni wis þú“, kwað he, „Mariun wrêð, &
þiornun þínaro; \hld\ siu is gi·þungan wíf; &
ne for·hugi þú sie te hardo; \hld\ þú skalt sie haldan wel, &
wardon ira an þesaro wer-oldi. \hld\ Lêsti þú inka wini-trewa &
forð só þú dádi, \hld\ ęndi hald inkan friund-skępi wel! &
Ne lát þú sie þi þiu lêðaron, \hld\ þoh siu undar ira liðon êgi, &
barn an ira bósma. \hld\ It kumid þurh gi·bod godes, &
hêlages gêstes \hld\ fon hevan-wanga: &
þat is Jésu Krist, \hld\ godes êgan barn, &
waldandes sunu. \hld\ Þu skalt sie wel haldan, &
hêlag-líko. \hld\ Ne lát þú þi þínan hugi twífljen, &
męrrjan þína mód-gi·þáht.“ \hld\ Þó warð eft þes mannes hugi &
gi·węndid aftar þem wordun, \hld\ þat he im te þem wíva genam, &
te þera magað minnja: \hld\ ant·kęnda maht godes, &
waldandes gi·bod; \hld\ was im willjo mikil, &
þat he sia só hêlag-líko \hld\ haldan mósti: &
bi·sorgoda sie an is gi·sïðja, \hld\ ęndi siu só súvro dróg &
al te huldi godes \hld\ hêlagna gêst, &
gód-líkan gumon, \hld\ ant-þat sie godes gi·skapu &
mahtig gi·manodun, \hld\ þat siu ina an manno lioht, &
allaro barno bętst, \hld\ brengjan skolda. &
Þó warð fon Rúmu-burg \hld\ ríkes mannes &%NOTE: Fitt 5.
ovar alla þesa irmin-þiod \hld\ Oktawiánas &
ban ęndi bod-skępi \hld\ ovar þea is brêdon gi·wald &
kuman fon þem kêsure \hld\ kuningo gi·hwi-likun, &
hêm-sittjandjun, \hld\ só wído só is hęri-togon &
ovar al þat land-skępi \hld\ liudjo gi·weldun. &
Hiet man þat alla þea ęli-lęndjun man \hld\ iro óðil sóhtin, &
hęliðos iro hand-mahal \hld\ an·gegen iro hêrron bodon, &
kwámi te þem knósla gi·hwe, \hld\ þanan he kunnjas was, &
gi·boran fon þem burgjun. \hld\ Þat gi·bod warð gi·lêstid &
ovar þesa wídon wer-old; \hld\ werod samnoda &
te allaro burgeo gi·hwem. \hld\ Fórun þea bodon ovar all, &
þea fon þem kêsura \hld\ kumana wá*run, &
bók-spáha weros, \hld\ ęndi an bréf skrivun &
swíðo niud-líko \hld\ namono gi·hwi-likan, &
ia land ia liudi, \hld\ þat im ni mahti a·lęttjan mann &
gumono su·lika gambra, \hld\ só im skolda geldan gi·hwe &
hęliðo fon is hôvda. \hld\ Þó gi·wêt im ôk mid is híwiska &
Joseph þe gódo, \hld\ só it god mahtig, &
waldand welda: \hld\ sóhta im þiu wánamon hêm, &
þea burg an Bethleem, \hld\ þar iro bęiðero was, &%NOTE: ęi is original.
þes hęliðes hand-mahal* \hld\ ęndi ôk þera hêlagun þiornun, &
Mariun þera gódun. \hld\ Þar was þes márjon stól &
an êr-dagun, \hld\ aðal-kuninges, &
Dawides þes gódon, \hld\ þan langa þe he þana druht-skępi þar, &
erl undar Ebreon \hld\ êgan mósta, &
haldan hôh-gi·setu. \hld\ Sie wárun is híwiskas, &
kuman fon is knósla, \hld\ kunnjas gódes, &
bêðju bi gi·burdjun. \hld\ Þar gi·fragn ik, þat sie þiu berhtun gi·skapu, &
Mariun gi·manodun \hld\ *ęndi maht godes, &
þat iru an þem sïða \hld\ sunu ôdan warð, &
gi·boran an Bethleem \hld\ barno strangost, &
allaro kuningo kraftigost: \hld\ kuman warð þe márjo, &
mahtig an manno lioht, \hld\ só is êr managan dag &
biliði wárun \hld\ ęndi bókno filu &
gi·worðen an þesero wer-oldi. \hld\ Þó was it all gi·wárod só, &
só it êr spáha man \hld\ gi·sprokan habdun, &
þurh hwi-lik ôd-módi \hld\ he þit erð-ríki herod &
þurh is selves kraft \hld\ sókjan welda, &
managaro mund-boro. \hld\ Þó ina þiu módar nam, &
bi·wand ina mid wádju \hld\ wívo skónjost, &
fagaron fratahun, \hld\ ęndi ina mid iro folmon twêm &
lęgda liov-líko \hld\ luttilna man, &
þat kind an êna kribbjun, \hld\ þoh he habdi kraft godes, &
manno drohtin. \hld\ Þar sat þiu módar bi·foran, &
wíf wakogjandi, \hld\ war*doda selvo, &
held þat hêlaga barn: \hld\ ni was ira hugi twífli, &
þera magað ira mód-sevo. \hld\ Þó warð þat managun ku̇ð &
ovar þesa wídon wer-old, \hld\ wardos ant·fundun, &
þea þar ehu-skalkos \hld\ úta wárun, &
weros an wahtu, \hld\ wiggjo gômjan, &
fehas aftar fel*da: \hld\ gi·sáhun finistri an twê &
te·látan an lufte, \hld\ ęndi kwam lioht godes &
wánum þurh þiu wolkan \hld\ ęndi þea wardos þar &
bi·féng an þem felda. \hld\ Sie wurðun an forhtun þó, &
þea man an ira móda: \hld\ gi·sáhun þar mahtigna &
godes ęngil kuman, \hld\ þe im te·gęgnes sprak, &
hét þat im þea wardos \hld\ wiht ne antd-rédin &
lêðes fon þem liohta: \hld\ „ik skal eu“, kwað he, „liovara þing, &
swíðo wár-líko \hld\ willjon sęggjan, &
ku̇ðjan kraft mikil: \hld\ nu is Krist ge·boran &
an þeser*o selvun naht, \hld\ sálig barn godes, &
an þera Dawides burg, \hld\ drohtin þe gódo. &
Þat is męndislo \hld\ manno kunnjas, &
allaro firiho fruma. \hld\ Þar gí ina fïðan mugun, &
an Bethlema-burg \hld\ barno ríkjost: &
hębbjad þat te têkna, \hld\ þat ik eu gi·tęlljan mag &
wárun wordun, \hld\ þat he þar bi·wundan ligid, &
þat kind an ênera kribbjun, \hld\ þoh he sí kuning ovar al &
erðun ęndi himiles \hld\ ęndi ovar ęldjo barn, &
wer-oldes waldand“. \hld\ Reht só he þó þat word gi·sprak, &
só warð þar ęngilo te þem ênun \hld\ un·rím kuman, &
hêlag hęri-skępi \hld\ fon hevan-wanga, &
fagar folk godes, \hld\ ęndi filu sprákun, &
lof-word manag \hld\ liudjo hêrron. &
Af·hóvun þó hêlagna sang, \hld\ þó sie eft te hevan-wanga &
wundun þurh þiu wolkan. \hld\ Þea wardos hôrdun, &
hwó þiu ęngilo kraft \hld\ alo-mahtigna god &
swíðo werð-líko \hld\ wordun lovodun: &
„diuriða sí nu“, \hld\ kwáðun sie, „drohtine selvun &
an þem hôhoston \hld\ himilo ríkja &
ęndi friðu an erðu \hld\ firiho barnun, &
gód-willigun gumun, \hld\ þem þe god ant·kęnnjad &
þurh hluttran hugi.“ \hld\ Þea hirdjo for·stódun, &
þat sie mahtig þing \hld\ gi·manod habda, &
blíð-lík bod-skępi: \hld\ gi·witun im te Bethleem þanan &
nahtes sïðon; \hld\ was im niud mikil, &
þat sie selvon Krist \hld\ gi·sehan móstin. &
Habda im þe ęngil godes \hld\ al gi·wísid &%NOTE: Fitt 6.
torhtun têknun, \hld\ þat sie im tó selvun, &
te þem godes barne \hld\ gangan mahtun, &
ęndi fundun sán \hld\ folko drohtin, &
liudjo hêrron. \hld\ Sagdun þó lof goda, &
waldande mid iro wordun \hld\ ęndi wído ku̇ðdun &
ovar þea berhtun burg, \hld\ hwi-lik im þar biliði warð &
fon hevan-wanga \hld\ hêlag gi·tôgit, &
fagar an felde. \hld\ Þat frí al bi·held &
an ira hugi-skęftjun, \hld\ hêlag þiorna, &
þiu magað an ira móde, \hld\ só hwat só siu gi·hôrda þea mann sprekan. &
Fódda ina þó fagaro \hld\ frího skánjosta, &
þiu módar þurh minnja \hld\ managaro drohtin, &
hêlag himilisk barn. \hld\ hęliðos gi·sprákun &
an þem ahtodon daga \hld\ erlos managa, &
swíðo glawa gumon \hld\ mid þera godes þiornun, &
þat he hêljand te namon \hld\ hębbjan skoldi, &
só it þe godes ęngil \hld\ Gabriel gi·sprak &
wáron wordun \hld\ ęndi þem wíve gi·bôd, &
bodo drohtines, \hld\ þó siu êrist þat barn ant·féng &
wánum te þesero wer-oldi; \hld\ was iru willjo mikil, &
þat siu ina só hêlag-líko \hld\ haldan mósti, &
ful-géng im þó só gerno. \hld\ Þat gę́r furðor skrêd &
unt-þat þat friðu-barn godes \hld\ fiar-tig habda &
dago ęndi nahto. \hld\ Þó skoldun sie þar êna dád frummjan, &
þat sie ina te Hjerusalem \hld\ for·gevan skoldun &
waldanda te þem wíha. \hld\ Só was iro wísa þan, &
þero liudjo land-sidu, \hld\ þat þat ni mósta for·látan ne-gên &
idis undar Ebreon, \hld\ ef iru at êrist warð &
sunu a·fódit, \hld\ ne siu ina simbla þarod &
te þem godes wíha \hld\ for·gevan skolda. &
Gi·witun im þó þiu gódun twê, \hld\ Joseph ęndi Maria &
bêðju fon Bethleem: \hld\ habdun þat barn mid im, &
hêlagna Krist, \hld\ sóhtun im hús godes &
an Hjerusalem; \hld\ þar skoldun sie is geld frummjan &
waldanda at þem wíha \hld\ wísa lêstjan &
Judeo folkes. \hld\ Þar fundun sea ênna gódan man &
aldan at þem alaha, \hld\ aðal-boranan, &
þe habda at þem wíha só filu \hld\ wintro ęndi sumaro &
gi·libd an þem liohta: \hld\ oft warhta he þar lof goda &
mid hluttru hugi; \hld\ habda im hêlagna gêst, &
sálig-líkan sevon; \hld\ Simeon was he hêtan. &
Im habda gi·wísid \hld\ waldandas kraft &
langa hwíla, \hld\ þat he ni mósta êr þit lioht a·gevan, &
węndjan af þesero wer-oldi, \hld\ êr þan im þe willjo gi·stódi, &
þat he selvan Krist \hld\ gi·sehan mósti, &
hêlagna hevan-kuning. \hld\ Þó warð im is hugi swíðo &
blíði an is briostun, \hld\ þó he gi·sah þat barn kuman &
an þena wíh innan. \hld\ Þuo sagda hie waldande þank, &
al-mahtigon gode, \hld\ þes he ina mid is ôgun gi·sah. &
Géng im þó te·gęgnes \hld\ ęndi ina gerno ant·féng &
ald mid is armun: \hld\ al ant·kęnde &
bókan ęndi biliði \hld\ ęndi ôk þat barn godes, &
hêlagna hevan-kuning. \hld\ „Nu ik þi, hêrro, skal“, kwað he, &
„gerno biddjan, \hld\ nu ik sus gigamalod bium, &
þat þú þínan holdan skalk \hld\ nu hinan hwervan látas, &
an þína friðu-wára faran, \hld\ þar êr mína forðrun dedun, &
weros fon þesero wer-oldi, \hld\ nu mi þe willjo gi·stód, &
dago liovosto, \hld\ þat ik mínan drohtin gi·sah, &
holdan hêrron, \hld\ só mi gi·hêtan was &
langa hwíla. \hld\ Þu bist lioht mikil &
allun ęli-þiodun, \hld\ þea êr þes alo-waldon &
kraft ne ant·kęndun. \hld\ Þína kumi sindun &
te dóma ęndi te diurðon, \hld\ drohtin frô mín, &
avarun Israhelas, \hld\ êganumu folke, &
þínun liovun *liudjun.“ \hld\ Listjun talde þó &
þe aldo man an þem alaha \hld\ idis þero gódun, &
sagda sǫ́ð-líko, \hld\ hwó iro sunu skolda &
ovar þesan middil-gard \hld\ managun werðan &
sumun te falle, sumun te fróvru \hld\ firiho barnun, &
þem liudjun te leova, \hld\ þe is lêrun gi·hôrdin, &
ęndi þem te harma, \hld\ þe hôrjen ni weldin &
Kristas lêron. \hld\ „Þu skalt noh“, kwað he, „kara þiggjan, &
harm an þínumu herton, \hld\ þan ina hęliðo barn &
wápnun wítnod. \hld\ Þat wirðid þi werk mikil, &
þrim te gi·þolonna.“ \hld\ Þiu þiorna al for·stód &
wísas mannas word. \hld\ Þó kwam þar ôk ên wíf gangan &
ald innan þem alaha: \hld\ Anna was siu hêtan, &
dohtar Fanueles; \hld\ siu habde ira drohtine wel &
gi·þionod te þanka, \hld\ was iru gi·þungan wíf. &
Siu mósta aftar ira magað-hêdi, \hld\ sïðor siu mannes warð, &
erles an êhti \hld\ ęðili þiorne, &
só mósta siu mid ira brúdi-gumon \hld\ bodlo gi·waldan &
sivun wintạr saman. \hld\ Þó gi·fragn ik þat iru þar sorga gi·stód &
þat sie þiu mikila maht \hld\ metodes te·dêlda, &
wrêð wurdi-gi·skapu. \hld\ Þó was siu widowa aftar þiu &
at þem friðu-wíha \hld\ fior ęndi ant·ahtoda &
wintro an iro wer-oldi, \hld\ só siu nia þana wíh ni for·lét, &
ak siu þar ira drohtine wel \hld\ dages ęndi nahtes, &
gode þionode. \hld\ Siu kwam þar ôk gangan tó &
an þea selvun tíd: \hld\ sán ant·kęnde &
þat hêlage barn godes \hld\ ęndi þem hęliðon ku̇ðde, &
þem weroda aftar þem wíha \hld\ wil-spel mikil, &
kwað þat im nęrjandas ginist \hld\ gi·náhid wári, &
helpa heven-kuninges: \hld\ „nu is þe hêlago Krist, &
waldand selvo \hld\ an þesan wíh kuman &
te a·lôsjenne þea liudi, \hld\ þe hér nu lango bidun &
an þesara middil-gard, \hld\ managa hwíla, &
þurftig þioda, \hld\ só nu þes þinges mugun &
męndjan man-kunni.“ \hld\ Manag fagonoda &
werod aftar þem wíha: \hld\ gi·hôrdun wil-spel mikil &
fon gode sęggjan. \hld\ Þat geld habde þó gi·lêstid &
þiu idis an þem alaha, \hld\ al só it im an ira êwa gi·bôd &
ęndi an þera berhtun burg \hld\ bók gi·wísdun, &
hêlagaro hand-gi·werk. \hld\ Gi·witun im þó te hús þanan &
fon Hjerusalem \hld\ Joseph ęndi Maria, &
hêlag híwiski: \hld\ habdun im heven-kuning &
simbla te gi·sïða, \hld\ sunu drohtines, &
managaro mund-boron, \hld\ só it gio mári ni warð &
þan wídor an þesaro wer-oldi, \hld\ b·útan só is willjo géng, &
heven-kuninges hugi. \hld\ Þoh þar þan gi·hwi-lik hêlag man &
Krist ant·kęndi, \hld\ þoh ni warð it gio te þes kuninges hove &
þem mannun gi·márid, \hld\ þea im an iro mód-sevon &
holde ni wárun, \hld\ ak was im só bi·halden forð &
mid wordun ęndi mid werkun, \hld\ ant-þat þar weros ôstan, &
swíðo glawa gumon \hld\ gangan kwámun &
þrea te þero þiodu, \hld\ þegnos snelle, &
an langan weg \hld\ ovar þat land þarod: &
folgodun ênun berhtun bókne \hld\ ęndi sóhtun þat barn godes &
mid hluttru hugi: \hld\ weldun im hnígan tó, &
gehan im te jungrun: \hld\ drivun im godes gi·skapu. &
Þó sie Erodesan þar \hld\ ríkjan fundun &
an is sęli sittjen, \hld\ slíð-wurdjan kuning, &
módagna mid is mannun: \hld\ —simbla was he morðes gern— &
þó kwaddun sie ina kúsko \hld\ an kuning-wísun, &
fagaro an is flęttje, \hld\ ęndi he frágoda sán, &
hwi-lik sie ârundi \hld\ úta gi·bráhti, &
weros an þana wrak-sïð: \hld\ „hweðer lêdjad gi wundan gold &
te gevu hwi-likun gumuno? \hld\ te hwí gi þus an ganga kumad, &
gi·faran an fóðju? \hld\ Hwat, gí n·êt-hwanan ferran sind &
erlos fon ǫ́ðrun þiodun. \hld\ Ik gi·sihu þat gi sind ęðili-gi·burdjun &
kunnjes fon knósle gódun: \hld\ nio hér êr su·lika kumana ni wurðun &
éri fon ǫ́ðrun þiodun, \hld\ sïðor ik mósta þesas erlo folkes, &
gi·waldan þesas wídon ríkjas. \hld\ Gí skulun mi te wárun sęggjan &
for þesun liudjo folke, \hld\ bi·hwí gí sín te þesun lande kumana“. &
Þó sprákun im eft te·gęgnes \hld\ gumon ôstr-onja, &
word-spáhe weros: \hld\ „wí þi te wárun mugun“, kwáðun sie, &
„u̇se ârundi \hld\ óðo gi·tęlljen, &
gi·sęggjan sǫ́ð-líko, \hld\ bi·hwí wí kwámun an þesan sïð herod &
fon ôstan te þesaro erðu. \hld\ Giu wárun þar aðaljes man, &
gód-sprákja gumon, \hld\ þea u̇s gódes só filu, &
helpa gi·hétun \hld\ fon heven-kuninge &
wárum wordun. \hld\ Þan was þar ên gi·wittig man, &
fród ęndi fil-wís \hld\ —forn was þat giu—, &
u̇se aldiro óstar hinan, \hld\ —þar ni warð sïðor ênig man &
sprákono só spáhi—; \hld\ he mahte rekkjen spel godes, &
hwand im habde for·liwan \hld\ liudjo hêrro, &
þat he mahte fon erðu \hld\ up gi·hôrjan &
waldandes word: \hld\ bi·þiu was is gi·wit mikil, &
þes þegnes gi·þáhti. \hld\ Þó he þanan skolda, &
a·geven gardos, \hld\ gadulingo gi·mang, &
for·láten liudjo drôm, \hld\ sókjen lioht ǫ́ðar, &
þó he is jungron hét \hld\ gangan náhor, &
ęrvi-wardos, \hld\ ęndi is erlun þó &
sagde sǫ́ð-líko: \hld\ —þat al sïðor kwam, &
gi·warð* an þesaro wer-oldi—: \hld\ þó sagda he þat hér skoldi kuman ên wís-kuning &
mári ęndi mahtig \hld\ an þesan middil-gard &
þes bętston gi·burdjes; \hld\ kwað þat it skoldi wesan barn godes, &
kwað þat he þesero wer-oldes \hld\ waldan skoldi &
gio te êwan-daga, \hld\ erðun ęndi himiles. &
He kwað þat an þem selvon daga, \hld\ þe ina sáligna &
an þesan middil-gard \hld\ módar gi·drógi, &
só kwað he þat ôstana \hld\ ên skoldi skínan &
himil-tungal hwít, \hld\ su·lik só wí hér ne habdin êr &
undar·twisk erða ęndi himil \hld\ ǫ́ðar hwerigin, &
ne su·lik barn ne su·lik bókan. \hld\ Hét þat þar te bedu fórin &
þrea man fon þero þiodu, \hld\ hét sie þęnkjan wel, &
hwan êr sie gi·sáwin ôstana \hld\ up síðogjan, &%TODO: Check síðogjan
þat godes bókan gangan, \hld\ hét sie garwjan sán, &
hét þat wí im folgodin, \hld\ só it furi wurði, &
westar ovar þesa wer-oldi. \hld\ Nu is it al gi·wárod só, &
kuman þurh kraft godes: \hld\ þe kuning is gi·fódit, &
gi·boran bald ęndi strang: \hld\ wí gi·sáhun is bókan skínan &
hêdro fon himiles tunglun, \hld\ só ik wêt, þat it hêlag drohtin, &
markoda mahtig selvo; \hld\ wí gi·sáhun morgno gi·hwi-likes &
blíkan þana berhton sterron, \hld\ ęndi wí géngun aftar þem bókna herod &
wegas ęndi waldas hwílon. \hld\ Þat wári u̇s allaro willjono mêsta, &
þat wí ina selvon gi·sehan móstin, \hld\ wissin, hwar wí ina sókjan skoldin, &
þana kuning an þesumu kêsur-dóma. \hld\ Saga u̇s, undar hwi-likumu he sí þesaro kunnjo a·fódit.“ &
Þó warð Erodesa \hld\ innan briostun &
harm wið herta, \hld\ bi·gan im is hugi wallan, &
sevo mid sorgun: \hld\ gi·hôrde sęggjan þó, &
þat he þar ovar-hôvdon \hld\ êgan skoldi, &
kraftagoron kuning \hld\ kunnjes gódes, &
sáligoron undar þem gi·sïðja. \hld\ Þó he samnon hét, &
só hwat só an Hjerusalem \hld\ gódaro manno &
allaro spáhoston \hld\ sprákono wárun &
ęndi an iro brioston \hld\ bók-kraftes mêst &
wissun te wárun, \hld\ ęndi he sie mid wordun fragn, &
swíðo niud-líko \hld\ níð-hugdig man, &
kuning þero liudjo, \hld\ hwar Krist gi·boran &
an wer-old-ríkja \hld\ werðan skoldi, &
friðu-gumono bętst. \hld\ Þó sprak im eft þat folk an·gęgin, &
þat werod wár-líko, \hld\ kwáðun þat sie wissin garo, &
þat he skoldi an Bethleem gi·boran werðan: \hld\ „só is an u̇sun bókun gi·skrivan, &
wís-líko gi·writan, \hld\ só it wár-sagon, &
swíðo glawa gumon \hld\ bi godes krafta &
fil-wíse man \hld\ furn gi·sprákun, &
þat skoldi fon Bethleem \hld\ burgo hirdi, &
liof landes ward \hld\ an þit lioht kuman, &
ríki rád-gevo, \hld\ þe rihtjen skal &
Judeono gum-skępi \hld\ ęndi is geva wesan &
mildi ovar middil-gard \hld\ managun þiodun.“ &
Þó gi·fragn ik þat sán aftar þiu \hld\ slíð-mód kuning &
þero wár-sagono word \hld\ þem wrękkjun sagda, &
þea þar an ęli-lęndi \hld\ erlos wárun &
ferran gi·farana, \hld\ ęndi he frágoda aftar þiu, &
hwan sie an óstar-wegun \hld\ êrist gi·sáhin &
þana kuning-sterron kuman, \hld\ kumbal liuhtjen &
hêdro fon himile. \hld\ Sie ni weldun is im þó helen eo·wiht, &
ak sagdun it im sǫ́ð-líko. \hld\ Þó hét he sie an þana sïð faran, &
hét þat sie ira ârundi al \hld\ undar·fundin &
umbi þes kindes kumi, \hld\ ęndi þe kuning selvo gi·bôd &
swíðo hard-liko, \hld\ hêrro Judeono, &
þem wísun mannun, \hld\ êr þan sie fórin westan forð, &
þat sie im eft gi·ku̇ðdin, \hld\ hwar he þana kuning skoldi &
sókjan at is selðon; \hld\ kwað þat he þar weldi mid is gi·sïðun tó, &
bedan te þem barne. \hld\ Þan hogda he im te banon werðan &
wápnes ęggjun. \hld\ Þan eft waldand god &
þáhte wið þem þinga: \hld\ he mahta a·þęngjan mêr, &
gi·lêstjan an þesum liohte: \hld\ þat is noh lango skín, &
gi·ku̇ðid kraft godes. \hld\ Þó géngun eft þiu kumbl forð &
wánum undar wolknun. \hld\ Þó wárun þea wíson man &
fu̇sa te faranne: \hld\ gi·witun im forð þanan &
balda an bod-skępi: \hld\ weldun þat barn godes &
selvon sókjan. \hld\ Sie ni habdun þanan gi·sïðjas mêr, &
b·útan þat sie þrie wárun: \hld\ wissun im þingo gi·skêð, &
wárun im glawe gumon, \hld\ þe þea geva lêddun. &
Þan sáhun sie só wís-líko \hld\ undar þana wolknes skion, &
up te þem hôhon himile, \hld\ hwó fórun þea hwíton sterron &
—ant·kęndun sie þat kumbal godes—, \hld\ þiu wárun þurh Krista herod &
gi·warht te þesero wer-oldi. \hld\ Þea weros aftar géngun, &
folgodun feraht-líko \hld\ —sie frumide þe mahte— &
ant-þat sie gi·sáhun, \hld\ sïð-wórige man, &
berht bókan godes, \hld\ blêk an himile &
stillo gi·standen. \hld\ Þe sterro liohto skên &
hwít ovar þem húse, \hld\ þar þat hêlage barn &
wonode an willjon \hld\ ęndi ina þat wíf bi·held, &
þiu þiorne gi·þiudo. \hld\ Þó warð þero þegno hugi &
blíði an iro briostun: \hld\ bi þem bókna for·stódun, &
þat sie þat friðu-barn godes \hld\ funden habdun, &
hêlagna heven-kuning. \hld\ Þó sie an þat hús innan &
mid iro gevun géngun, \hld\ gumon ôstr-onja, &
sïð-wórige man: \hld\ sán ant·kęndun &
þea weros waldand Krist. \hld\ Þea wrękkjon fellun &
te þem kinde an kneo-beda \hld\ ęndi ina an kuning-wísa &
gódan gróttun \hld\ ęndi im þea geva drógun, &
gold ęndi wíh-rôk \hld\ bi godes têknun &
*ęndi myrra þar mid. \hld\ Þea man stódun garowa, &
holde for iro hêrron, \hld\ þea it mid iro handun sán &
fagaro ant·féngun. \hld\ Þó gi·witun im þea ferahton man, &
sęggi te selðon \hld\ sïð-wórige, &
gumon an gast-sęli. \hld\ Þar im godes ęngil &
slápandjun an naht \hld\ swevan gi·tôgde, &
gi·drog im an drôme, \hld\ al so it drohtin self, &
waldand welde, \hld\ þat im þúhte þat man im mid wordun gi·budi, &
þat sie im* þanan ǫ́ðran weg, \hld\ erlos fórin, &
liðodin sie te lande \hld\ ęndi þana lêðan man, &
Erodesan \hld\ eft ni sóhtin, &
módagna kuning. \hld\ Þó warð morgan kuman &
wánum te þesero wer-oldi. \hld\ Þó bi·gunnun þea wíson man &
sęggjan iro swevanos; \hld\ selvon ant·kęndun &
waldandes word, \hld\ hwand sie gi·wit mikil &
bárun an iro briostun: \hld\ bádun alo-waldon, &
hêron heven-kuning, \hld\ þat sie móstin is huldi forð, &
gi·wirkjan is willjon, \hld\ kwáðun þat sea ti im habdin gi·węndit hugi, &
*iro mód morgan gi·hwem. \hld\ Þó fórun eft þie man þanan, &
erlos ôstr-onje, \hld\ al só im þe ęngil godes &
wordun gi·wísde: \hld\ námun im weg ǫ́ðran, &
ful-géngun godes lêrun: \hld\ ni weldun þemu Judeo kuninge &
umbi þes barnes gi·burd \hld\ bodon ôstr-onje, &
sïð-wórige man \hld\ sęggjan gio·wiht, &
ak wendun im eft an iro willjon. \hld\ Þó warð sán aftar þiu waldandes, &
godes ęngil kumen \hld\ Josepe te sprákun, &
sagde im an swefne \hld\ slápandjum an naht, &
bodo drohtines, \hld\ þat þat barn godes &
slíð-mód kuning \hld\ sókjan welda, &
áhtjan is aldres; \hld\ „nu skaltu ine an Aegypteo &
land ant·lêdjan \hld\ ęndi undar þem liudjun wesan &
mid þiu godes barnu \hld\ ęndi mid þeru gódan þior*nan, &
wunon undar þemu werode, \hld\ unt-þat þi word kume &
hêrron þínes, \hld\ þat þú þat hêlage barn &
eft te þesum land-skępi \hld\ lêdjan mótis, &
drohtin þínen.“ \hld\ Þó fon þem drôma an·sprang &
Joseph an is gęst-sęli, \hld\ ęndi þat godes gi·bod &
sán ant·kęnda: \hld\ gi·wêt im an þana sïð þanen &
þe þegạn mid þeru þiornon, \hld\ sóhta im þiod ǫ́ðra &
ovar brêdan berg: \hld\ welda þat barn godes &
fíundun ant·fórjan. \hld\ *Þó gi·frang aftar þiu &%NOTE: gi·frang [sic]
Erodes þe kuning, \hld\ þar he an is ríkja sat, &
þat wárun þea wíson man \hld\ westan gi·hworvan &
óstar an iro óðil \hld\ ęndi fórun im ǫ́ðran weg: &
wisse þat sie im þat ârundi \hld\ eft ni weldun &
sęggjan an is selðon. \hld\ Þó warð im þes an sorgun hugi, &
mód mornondi, \hld\ kwað þat it im þie man dedin, &
hęliðos* te hônðun. \hld\ Þó he só hriwig sat, &
balg ina an is briostun, \hld\ kwað þat he is mahti bętaron rád, &
ǫ́ðran gi·þęnkjen: \hld\ „nu ik is aldar kan, &
wêt is winter-gi·talu: \hld\ nu ik gi·winnan mag, &
þat he io ovar þesaro erðu \hld\ ald ni wirðit, &
hér undar þesum hęri-skępi.“ \hld\ Þó he só hardo gi·bôd, &
Erodes ovar is riki, \hld\ hét þó is rinkos faran &
kuning þero liudjo, \hld\ hét þat sie kinda só filo &
þurh iro hand-magen \hld\ hôvdu bi·námin, &
só manag barn umbi Bethleem, \hld\ só filo só þar gi·boran wurði, &
an twêm gêrun a·togan. \hld\ Tionon frumidon &
þes kuninges gi·sïðos. \hld\ Þó skolda þar só manag kindisk man &
sweltan sundjono lôs. \hld\ Ni warð síð noh êr &
gjámar-líkara for·gang \hld\ jungaro manno, &
arm-líkara dôð. \hld\ Idisi wiopun, &
módar managa, \hld\ gi·sáhun iro męgi spildjan: &
ni mahte siu im nio gi·formon, \hld\ þoh siu mid iro faðmon twêm &
iro êgan barn \hld\ armun bi·féngi, &
liof ęndi luttil, \hld\ þoh skolda is simbla þat líf gevan, &
þe magu for þeru módar. \hld\ Mênes ni sáhun, &
wítjes þie wam-skaðon: \hld\ wápnes ęggjun &
fręmidun firin-werk mikil. \hld\ Fellun managa &
magu-junge man. \hld\ Þia módar wiopun &
kind-jungaro kwalm; \hld\ kara was an Bethleem, &
hofno hlúdost: \hld\ þoh man im iro herton an twê &
sniði mid swerdu, \hld\ þoh ni mohta im gio sêrara dád &
werðan an þesaro wer-oldi, \hld\ wívun managun, &
brúdjun an Bethleem: \hld\ gi·sáhun iro barn bi·foran, &
kind-junge man, \hld\ kwalmu sweltan &
blódag an iro barmun. \hld\ Þie banon wítnodun &
un·skuldige skole: \hld\ ni bi·skrivun gio·wiht &
þea man umbi mên-werk: \hld\ weldun mahtigna, &
Krist selvon a·kwęlljan. \hld\ Þan habde ina kraftag god &
gi·nęridan wið iro níðe, \hld\ þat inan nahtes þanan &
an Aegypteo land \hld\ erlos ant·lêddun, &
gumon mid Josepe \hld\ an þana grónjon wang, &
an erðono bętstun, \hld\ þar ên aha fliutid, &
Níl-strôm mikil \hld\ norð te sêwa, &
flódo fagorosta. \hld\ Þar þat friðu-barn godes &
wonoda an willjon, \hld\ ant-þat wurd for·nam &
Erodes þana kuning, \hld\ þat he for·lét ęldjo barn, &
módag manno drôm. \hld\ Þó skolda þero marka gi·wald &
êgan is ęrvi-ward: \hld\ þe was Arkheláus &
hêtan, hęri-togo \hld\ helm-berandero: &
þe skolda umbi Hjerusalem \hld\ Judeono folkes, &
werodes gi·waldan. \hld\ Þó warð word kuman &
þar an Egypti \hld\ ęðiljun manne, &
þat he þar te Josepe, \hld\ godes ęngil sprak, &
bodo drohtines, \hld\ hét ina eft þat barn þanan &
lêdjen te lande. \hld\ „nu havað þit lioht af·geven“, kwað he, &
„Erodes þe kuning; \hld\ he welde is áhtjen giu, &
frêson is ferahas. \hld\ Nu maht þú an friðu lêdjen &
þat kind undar ewa kunni, \hld\ nu þe kuning ni livod, &
erl ovar-módig.“ \hld\ Al ant·kęnde &
Josep godes têkạn: \hld\ geriwide ina sniumo &
þe þegạn mit þera þiornun, \hld\ þó sie þanan weldun &
bêðju mid þiu barnu: \hld\ lêstun þiu berhton gi·skapu, &
waldandes willjon, \hld\ al só he im êr mid is wordun gi·bôd. &
Gi·witun im þó eft an Galilea-land \hld\ Joseph ęndi Maria, &
hêlag híwiski \hld\ heven-kuninges, &
wárun im an Nazareth-burg. \hld\ Þar þe nęrjondio Krist &
wóhs undar þem werode, \hld\ warð gi·wittjes ful, &
an was imu anst godes, \hld\ he was allun liof &
módar-mágun: \hld\ he ni was ǫ́ðrun mannun gi·lík, &
þe gumo an sínera gódi. \hld\ Þó he gę́r-talo &
twe-livi habde, \hld\ þó warð þiu tíd kuman, &
þat sie þar te Hjerusalem, \hld\ Juðeo liudi &
iro þiod-gode \hld\ þionon skoldun, &
wirkjan is willjon. \hld\ Þó warð þar an þana wíh innan &
þar te Hjerusalem \hld\ Judeono gi·samnod &
man-kraft mikil. \hld\ Þar Maria was &
self an gi·sïðja \hld\ ęndi iru sunu habda, &
godes êgan barn. \hld\ Þó sie þat geld habdun, &
erlos an þem alaha, \hld\ só it an iro êwa gi·bôd, &
gi·lêstid te iro land-wísun, \hld\ þó fórun im eft þie liudi þanan, &
weros an iro willjon \hld\ ęndi þar an þem wíha af·stód &
mahtig barn godes, \hld\ só ina þiu módar þar &
ni wissa te wáron; \hld\ ak siu wánda þat he mid þem weroda forð, &
fóri mit iro friundun. \hld\ Gi·frang aftar þiu &
eft an ǫ́ðrun daga \hld\ aðal-kunnjes wíf, &
sálig þiorna, \hld\ þat he undar þem gi·sïðia ni was. &
warð Mariun þó \hld\ mód an sorgun, &
hriwig umbi iro herta, \hld\ þó siu þat hêlaga barn &
ni fand undar þem folka: \hld\ filu gornoda &
þiu godes þiorna. \hld\ Gi·witun im þó eft te Hjerusalem &
iro sunu sókjan, \hld\ fundun ina sittjan þar &
an þem wíha innan, \hld\ þar þe wísa man, &
swíðo glauwa gumon \hld\ an godes êwa &
lásun ende línodun, \hld\ hwó sie lof skoldin &
wirkjan mid iro wordun þem, \hld\ þe þesa wer-old gi·skóp. &
Þar sat undar middjun \hld\ mahtig barn godes, &
Krist alo-waldo, \hld\ só is þea ni mahtun ant·kęnnjan wiht, &
þe þes wíhes þar \hld\ wardon skoldun, &
ęndi frágoda sie \hld\ firi-wit-líko &
wísera wordo. \hld\ Sie wundradun alle, &
bu-hwí gio só kindisk man \hld\ su·lika kwidi mahti &
mid is mu̇ðu gi·mênjan. \hld\ Þar ina þiu módar fand &
sittjan under þem gi·sïðja \hld\ ęndi iro sunu grótta, &
wísan undar þem weroda, \hld\ sprak im mid ira wordun tó: &
„hwí weldes þú þínera módar, \hld\ manno liovosto, &
gi·sidon su·lika sorga, \hld\ þat ik þi só sêrag-mód, &
idis arm-hugdig \hld\ êskon skolda &
undar þesun burg-liudjun?“ \hld\ Þó sprak iru eft þat barn an·gęgin &
wísun wordun: \hld\ „hwat, þú wêst garo“, kwað he, &
„þat ik þar gi·rísu, \hld\ þar ik bi rehton skal &
wonon an willjon, \hld\ þar gi·wald havad &
mín mahtig fader.“ \hld\ Þie man ni for·stódun, &
þie weros an þem wíha, \hld\ bi·hwí he só þat word gi·sprak, &
gi·mênda mid is mu̇ðu: \hld\ Maria al bi·held, &
gi·barg an ira breostun, \hld\ só hwat só siu gi·hôrda ira barn sprekan &
wisaro wordo. \hld\ Gi·witun im þó eft þanan &
fon Hjerusalem \hld\ Joseph ęndi Maria, &
habdun im te gi·sïðja \hld\ sunu drohtines, &
allaro barno bętsta, \hld\ þero þe io gi·boran wurði &
magu fon módar: \hld\ habdun im þar minnja tó &
þurh hluttran hugi, \hld\ ęndi he só gi·hôrig was, &
godes êgan barn \hld\ gaduling-mágun &
þurh is ôd-módi, \hld\ aldron sínun: &
ni welda an is kindiski þó noh \hld\ is kraft mikil &
mannun márjan, \hld\ þat he su·lik męgin êhta, &
gi·wald an þesaro wer-oldi, \hld\ ak he im an is willjon bêd &
gi·þiudo undar þero þiodu \hld\ þrí-tig gę́ro, &
êr þan he þar têkạn ênig \hld\ tôgjan weldi, &
sęggjan þem gi·sïðja, \hld\ þat he selvo was &
an þesaro middil-gard \hld\ manno drohtin. &
Habda im só bi·halden \hld\ hêlag barn godes &
word ęndi wís-dóm \hld\ ende allaro gi·wittjo mêst, &
tulgo spáhan hugi: \hld\ ni mahta man is an is sprákun werðan, &
an is wordun gi·war, \hld\ þat he su·lik gi·wit êhta, &
þegạn su·lika gi·þáhti, \hld\ ak he im só gi·þiudo bêd &
torhtaro têkno. \hld\ Ni was noh þan þiu tíd kuman, &
þat he ina ovar þesan \hld\ middil-gard márjan skolda, &
lêrjan þie liudi, \hld\ hwó sie skoldin iro gi·lôvon haldan, &
wirkjan willjon godes; \hld\ wissun þat þoh managa &
liudi aftar þem landa, \hld\ þat he was an þit lioht kuman, &
þoh sie ina ku̇ð-líko \hld\ an·kęnnjan ni mahtin, &
êr þan he ina selvo \hld\ sęggjan welda. &
Þan was im Johannes \hld\ fon is juguð-hêdi &
awahsan an ênero wóstunni; \hld\ þar ni was werodes þan mêr, &
b·útan þat he þar ên-kora \hld\ alo-waldon gode, &
þegạn þionoda: \hld\ for·lét þioda gi·mang, &
manno gi·mênðon. \hld\ Þar warð im mahtig kuman &
an þero wóstunni \hld\ word fon himila, &
gód-lík stemna godes, \hld\ ęndi Johanne gi·bod, &
þat he Kristes kumi \hld\ ęndi is kraft mikil &
ovar þesan middil-gard \hld\ márjan skoldi; &
hét ina wár-líko \hld\ wordun sęggjan, &
þat wári hevan-riki \hld\ hęliðo barnun &
an þem land-skępi, \hld\ liudjun gi·náhid, &
welono wun-samost. \hld\ Im was þó willjo mikil, &
þat he fon su·likun sáldun \hld\ sęggjan mósti. &
Gi·wêt im þó gangan, \hld\ al só Jordan flót, &
watar an willjon, \hld\ ęndi þem weroda allan dag, &
aftar þem land-skępi \hld\ þem liudjun ku̇ðda, &
þat sie mid fastunnju \hld\ firin-werk manag, &
iro selvoro \hld\ sundja bóttin, &
„þat gí werðan hrênja“, \hld\ kwað he. „Hevan-riki is &
gi·náhid manno barnun. \hld\ Nu látad eu an ewan mód-sevon &
ewar selvoro \hld\ sundja hrewan, &
lêdas þat gí an þesun liohta fręmidun, \hld\ ęndi mínun lêrun hôrjad, &
węndjat aftar mínun wordun. \hld\ Ik eu an watara skal &
gi·dôpjan diur-líko, \hld\ þoh ik ewa dádi ne mugi, &
ewar selvaro \hld\ sundja a·látan, &
þat gí þurh mín hand-gi·werk \hld\ hluttra werðan &
lêðaro gi·lêsto: \hld\ ak þe is an þit lioht kuman, &
mahtig te mannun \hld\ ęndi undar eu middjun stéd, &
—þoh gí ina selvun \hld\ gi·sehan ni willjan—, &
þe eu gi·dôpjan skal \hld\ an ewes drohtines namon &
an þana hálagon gêst. \hld\ Þat is hêrro ovar al: &
he mag allaro manno gi·hwena \hld\ mên-gi·þáhtjo, &
sundjono sikoron, \hld\ só hwene só só sálig mót &
werðen an þesaro wer-oldi, \hld\ þat þes willjon havad, &
þat he só gi·lêstja, \hld\ só he þesun liudjun wili, &
gi·bioden barn godes. \hld\ Ik bium an is bod-skępi herod &
an þesa wer-old kumen \hld\ ęndi skal im þana weg rúmjen, &
lêrjan þesa liudi, \hld\ hwó sea skulin iro gi·lôvon haldan &
þurh hluttran hugi, \hld\ ęndi þat sie an hęllja ni þurvin, &
faran an fern þat hêta. \hld\ Þes wirðid só fagan an is móde &
man te só managaro stundu, \hld\ só hwe só þat mên for·látid, &
gerno þes gramon anbusni, \hld\ —só mag im þes gódon gi·wirkjan, &
huldi heven-kuninges,— \hld\ só hwe só havad hluttra trewa &
up te þem alo-mahtigon gode.“ \hld\ Erlos managa &
bi þem lêrun þó, \hld\ liudi wándun, &
weros wár-líko, \hld\ þat þat waldand Krist &
selbo wári, \hld\ hwanda he só filu sǫ́ðes gi·sprak, &
wároro wordo. \hld\ Þó warð þat só wído ku̇ð &
ovar þat for·gevana land \hld\ gumono gi·hwi-likum, &
sęggjun at iro selðun: \hld\ þó kwámun ina sókjan þarod &
fon Hjerusalem \hld\ Judeo liudjo &
bodon fon þeru burgi \hld\ ęndi frágodun, ef he wári þat barn godes, &
„þat hér lango giu“, \hld\ kwaðun sie, „liudi sagdun, &
weros wár-líko, \hld\ þat he skoldi an þesa wer-old kuman“. &
Johannes þó gi·mahalde \hld\ ęndi te·gęgnes sprak &
þem bodun bald-líko: \hld\ „ni bium ik“, kwað he, „þat barn godes, &
wár waldand Krist, \hld\ ak ik skal im þana weg rúmjen, &
hêrron mínumu.“ \hld\ Þea hęliðos frugnun, &
þea þar an þem ârundje \hld\ erlos wárun, &
bodon fon þero burgi: \hld\ „ef þú nu ni bist þat barn godes, &
bist þú þan þoh Elias, \hld\ þe hér an êr-dagun &
was undar þesumu werode? \hld\ He is wis·kumo &
eft an þesan middil-gard. \hld\ Saga u̇s hwat þú manno sís! &
Bist þú ênig þero, \hld\ þe hér êr wári &
wísaro wár-saguno? \hld\ Hwat skulun wí þem werode fon þi &
sęggjan te sǫ́ðon? \hld\ Neo hér êr su·lik ni warð &
an þesun middil-gard \hld\ man ǫ́ðar kuman &
dádjun só mári. \hld\ Bi·hwí þú hér dôpisli &
fręmis undar þesumu folke, \hld\ ef þú þaro fora·sagono &
ên-hwi-lik ni bist?“ \hld\ Þó habde eft garo &
Johannes þe gódo \hld\ glau and-wordi: &
„Ik bium fora·bodo \hld\ fráon mínes, &
lioves hêrron; \hld\ ik skal þit land rekon, &
þit werod aftar is willjon. \hld\ Ik hębbju fon is worde mid mi &
stranga stemna, \hld\ þoh sie hér ni willje for·standan filo &
werodes an þesaro wóstunni. \hld\ Ni bium ik mid wihti gi·lík &
drohtine mínumu: \hld\ he is mid is dádjun só strang, &
só mári ęndi só mahtig \hld\ —þat wirðid managun ku̇ð, &
werun aftar þesaro wer-oldi— \hld\ þat ik þes wirðig ni bium, &
þat ik móti an is gi·skuoha, \hld\ þoh ik sí is skalk êgan, &
an só ríkjumu drohtine, \hld\ þea reomon ant·bindan: &
só mikilu is he bętara þan ik. \hld\ Nis þes bodon gi·mako &
ênig ovar erðu, \hld\ ne nu aftar ni skal &
werðan an þesaro wer-oldi. \hld\ Hębbjad ewan willjon þarod, &
liudi ewan gi·lôvon: \hld\ þan eu lango skal &
wesan ewa hugi hrómag; \hld\ þan gi hęlli-gi·þwing, &
for·látad lêðaro drôm \hld\ ęndi sókjad eu lioht godes, &
up-ôdes hêm, \hld\ êwig ríki, &
hôhan heven-wang. \hld\ Ne látad ewan hugi twífljen!“ &
Só sprak þó jung gumo \hld\ bi godes lêrun &
mannun te márðu. \hld\ Manag samnoda &
þar te Bethania \hld\ barn Israheles; &
kwámun þar te Johannese \hld\ kuningo gi·sïðos, &
liudi te lêrun \hld\ ęndi iro gi·lôvon ant·féngun. &
He dôpte sie dago gi·hwi-likes \hld\ ęndi im iro dádi lóg, &
wrêðaro willjon, \hld\ ęndi lovode im word godes, &
hêrron sínes: \hld\ „heven-ríki wirðid“, kwað he, &
„garu gumono só hwem, \hld\ só ti gode þęnkid &
ęndi an þana hêljand *wili \hld\ hluttro gi·lôvjan, &%NOTE ms. -- wili] P 1r.
lêstjan is lêra“. \hld\ Þó ni was lang te þiu, &
þat im fon Galilea gi·wêt \hld\ godes êgan barn, &
*diur-lík drohtines sunu, \hld\ dôpi suokjan. &
was im þuo an is wastme \hld\ waldandes barn*, &
al só he mid þero þiodu \hld\ þrí-tig habdi &
wintro an is wer-oldi. \hld\ Þó he an is willjon kwam, &
þar Johannes \hld\ an Jordana strôme &
allan langan dag \hld\ liudi manage &
dôpte diur-líko. \hld\ Reht só he þó is drohtin gi·sah, &
holdan hêrron, \hld\ só warð im is hugi blíði, &
þes im þe willjo gi·stód, \hld\ ęndi sprak im þó mid is wordun tó, &
swíðo gód gumo, \hld\ Johannes te Kriste: &
„nu kumis þú te mínero dôpi, \hld\ drohtin frô mín, &
þiod-gumono bętsto: \hld\ só skolde ik te þínero duan, &
hwand þú bist allaro kuningo kraftigost.“ \hld\ Krist selvo gi·bôd, &
waldand wár-líko, \hld\ þat he ni spráki þero wordo þan mêr: &
„wêst þú, þat u̇s só gi·rísid“, \hld\ kwað he, „allaro rehto gi·hwi-lik &
te gi·fulljanne \hld\ forð-wardes nu &
an godes willjon“. \hld\ Johannes stód, &
dôpte allan dag \hld\ druht-folk mikil, &
werod an watere \hld\ ęndi ôk waldand Krist, &
hêran heven-kuning \hld\ handun sínun &
an allaro baðo þem bętston \hld\ ęndi im þar te bedu gi·hnêg &
an kneo kraftag. \hld\ Krist up gi·wêt &
fagar fon þem flóde, \hld\ friðu-barn godes, &
liof liudjo ward. \hld\ Só he þó þat land af·stóp, &
só ant·hlidun þó himiles doru, \hld\ ęndi kwam þe hêlago gêst &
fon þem alo-waldon \hld\ ovane te Kriste: &
—was im an gi·lík-nissje \hld\ lungras fugles, &
diur-líkara dúvun— \hld\ ęndi sat im uppan u̇ses drohtines ahslu, &
wonoda im ovar þem waldandes barne. \hld\ Aftar kwam þar word fon himile, &
hlúd fon þem hôhon radura \hld\ ęndi grótta þane hêljand selvon, &
Krista, allaro kuningo bętston, \hld\ kwað þat he ina gi·korana habdi &
selvo fon sínun ríkja, \hld\ kwað þat im þe sunu líkodi &
bętst allaro gi·boranaro manno, \hld\ kwað þat he im wári allaro barno liovost. &
Þat móste Johannes þó, \hld\ al só it god welde, &
gi·sehan ęndi gi·hôrjan. \hld\ He gi·deda it sán aftar þiu &
mannun mári, \hld\ þat sie þar mahtigna &
hêrron habdun: \hld\ „Þit is“, kwað he, „heven-kuninges sunu, &
ên alo-waldand: \hld\ þesas willjo ik ur·kundjo &
wesan an þesaro wer-oldi, \hld\ hwand it sagda mi word godes, &
drohtines stemne, \hld\ þó he mi dôpjan hét &
weros an watare, \hld\ só hwar só ik gi·sáwi wár-líko &
þana hêlagon gêst \hld\ *fan hevan-wange &
an þesan middil-gard \hld\ ênigan man waron, &
kuman mid kraftu; \hld\ þat kwað, þat skoldi Krist wesan, &
diur-lík drohtines suno. \hld\ Hie dôpjan skal &
an þana hêlagan gêst \hld\ ęndi hêljan managa &%NOTE ms. -- þana] P end.
manno mên-dádi. \hld\ He havad maht fon gode, &
þat he a·látan mag \hld\ liudjo gi·hwi-likun &
saka ęndi sundja. \hld\ Þit is selvo Krist, &
godes êgan barn, \hld\ gumono bętsto, &
friðu wið fíundun. \hld\ Wala þat eu þes mag frâh-mód hugi &
wesan an þesaro wer-oldi, \hld\ þes eu þe willjo gi·stód, &
þat gí só libbjanda \hld\ þana landes ward &
selvon gi·sáhun. \hld\ Ní mót sliumo sundjono lôs &
manag gêst faran \hld\ an godes willjon &
tionon a·tómid, \hld\ þe mid trewon wili &
wið is wini wirkjan \hld\ ęndi an waldand Krist &
fasto gi·lôvjan. \hld\ Þat skal te frumun werðen &
gumono só hwi-likun, \hld\ só þat gerno dót“. &
Só ge·fragn ik þat Johannes \hld\ þó gumono gi·hwi-likun, &
lovoda þem liudjun \hld\ lêra Kristes, &
hêrron sínes, \hld\ ęndi heven-ríki &
te gi·winnanne, \hld\ welono þane mêston, &
sálig sin-líf. \hld\ Þó he im selvo gi·wêt &
aftar þem dôpislja, \hld\ drohtin þe gódo, &
an êna wóstunnja, \hld\ waldandes sunu; &
was im þar an þero ên-ôdi \hld\ erlo drohtin &
lange hwíla; \hld\ ne habda liudjo þan mêr, &
sęggjo te gi·sïðun, \hld\ al só he im selvo gi·kôs: &
welda is þar látan koston \hld\ kraftiga wihti, &
selvon Satanasan, \hld\ þe gio an sundja spenit, &
man an mên-werk: \hld\ he konsta is mód-sevon, &
wrêðan willjon, \hld\ hwó he þesa wer-old êrist, &
an þem an·ginnja \hld\ irmin-þioda &
bi·swêk mit sundjun, \hld\ þó he þiu sinhíun twê, &
Ádaman ęndi Éwan, \hld\ þurh un·trewa &
for·lêdda mid luginun, \hld\ þat liudo barn &
aftar iro hin-fęrdi \hld\ hęllja sóhtun, &
gumono gêstos. \hld\ Þó welda þat god mahtig, &
waldand węndjan \hld\ ęndi welda þesum werode for·geven &
hôh himil-ríki: \hld\ be·þiu he herod hêlagna bodon, &
is sunu sęnda. \hld\ Þat was Satanase &
tulgo harm an is hugi: \hld\ afonsta hevan-ríkjes &
manno kunnje: \hld\ welda þó mahtigna &
mid þem selvon sakun \hld\ sunu drohtines, &
þem he Ádaman \hld\ an êr-dagun &
darnungo bi-dróg, \hld\ þat he warð is drohtine lêð, &
bi·swêk ina mid sundjun \hld\ —só welda he þó selvan dón &
hêlandjan Krist. \hld\ Þan habda he is hugi fasto &
wið þana wam-skaðon, \hld\ waldandes barn, &
herte só gi·hęrdid: \hld\ welda heven-ríki &
liudjun gi·lêstjan. \hld\ Was im þes landes ward &
an fastunnja \hld\ fior-tig nahto, &
manno drohtin, \hld\ só he þar mates ni ant·bêt; &
þan langa ni gi·dorstun \hld\ im dęrnja wihti, &
níð-hugdig fíund, \hld\ náhor gangan, &
grótjan ina gęgin-warðan: \hld\ wánde þat he god ên-fald, &
for·útar man-kunnjes wiht \hld\ mahtig wári, &
hêleg himiles ward. \hld\ Só he ina þó ge·hungrjan lét, &
þat ina bi·gan bi þero męnnisko \hld\ móses lustjan &
aftar þem fiuwar-tig dagun, \hld\ þe fíund náhor géng, &
mirki mên-skaðo: \hld\ wánda þat he man ên-fald &
wári wissungo, \hld\ sprak im þó mid is wordun tó, &
grótta ina þe gêr-fíund: \hld\ „ef þú sís godes sunu“, kwað he, &
„be·hwí ni hêtis þú þan werðan, \hld\ ef þú gi·wald haves, &
allaro barno bętst, \hld\ brôd af þesun stênun? &
Ge·hêli þínna hungar!“ \hld\ Þó sprak eft þe hêlago Krist: &
„ni mugun ęldi-barn“, \hld\ kwað he, „ên-faldes brôdes, &
liudi libbjen, \hld\ ak sie skulun þurh lêra godes &
wesan an þesero wer-oldi \hld\ ęndi skulun þiu werk frummjen, &
þea þar werðad a·hlúdid \hld\ fon þero hêlogun tungun, &
fon þem galme godes: \hld\ þat is gumono líf &
liudjo só hwi-likon, \hld\ só þat lêstjan wili, &
þat fon waldandes \hld\ worde ge·biudid.“ &
Þó bi·gan eft niuson \hld\ ęndi náhor géng &
un·hiuri fíund \hld\ ǫ́ðru sïðu, &
fandoda is frôhan. \hld\ Þat friðu-barn þolode &
wrêðes willjon \hld\ ęndi im gi·wald for·gaf, &
þat he umbi is kraft mikil \hld\ koston mósti, &
lét ina þó lêdjan \hld\ þana liud-skaðon, &
þat he ina an Hjerusalem \hld\ te þem godes wíha, &
alles ovan-wardan, \hld\ up gi·sętta &
an allaro húso hôhost, \hld\ ęndi hosk-wordun sprak, &
þe gramo þurh gelp mikil: \hld\ „ef þú sís godes sunu“, kwað he, &
„skríd þi te erðu hinan. \hld\ Ge·skrivan was it giu lango, &
an bókun ge·writen, \hld\ hwó gi·boden havad &
is ęngilun \hld\ alo-mahtig fader, &
þat sie þi at wege ge·hwem \hld\ wardos sinðun, &
haldad þi undar iro handun. \hld\ Hwat, þú hwargin ni þarft &
mid þínun fótun \hld\ an felis be·spurnan, &
an hardan stên.“ \hld\ Þó sprak eft þe hêlago Krist, &
allaro barno bętst: \hld\ „só is ôk an bókun ge·skrivan“, kwað he, &
„þat þú te hardo ni skalt \hld\ hêrran þínes, &
fandon þínes frôhan: \hld\ þat nis þi allaro frumono neg·ên.“ &
Lét ina þó an þana þriddjan sïð \hld\ þana þiod-skaðon &
gi·brengen uppan ênan berg þen hôhon: \hld\ þar ina þe balo-wíso &
lét al ovar-sehan \hld\ irmin-þiode, &
wonod-saman welon \hld\ ęndi wer-old-ríki &
ęndi all su·lik ôdes, \hld\ só þius erða bi·havad &
fagororo frumono, \hld\ ęndi sprak im þó þe fíund an·gęgin, &
kwað þat he im þat al só gód-lík \hld\ for·geven weldi, &
hôha hęri-dómos, \hld\ „ef þú wilt hnígan te mí, &
fallan te mínun fótun \hld\ ęndi mí for frôhan havas, &
bedos te mínun barma. \hld\ Þan látu ik þí brúkan wel &
alles þes ôd-welon, \hld\ þes ik þí hębbju gi·ôgit hír.“ &
Þó ni welda þes lêðan word \hld\ lęngeron hwíle &
hôrjan þe hêlago Krist, \hld\ ak he ina fon is huldi for·drêf, &
Satanasan for·swêp, \hld\ ęndi sán aftar sprak &
allaro barno bętst, \hld\ kwað þat man bedon skoldi &
up te þem alo-mahtigon gode \hld\ ęndi im ênum þionon &
swíðo þio-liko \hld\ þegnos managa, &
hęliðos aftar is huldi: \hld\ „þar ist þiu helpa ge·lang &
manno ge·hwi-likun.“ \hld\ Þó gi·wêt im þe mên-skaðo, &
swíðo sêrag-mód \hld\ Satanas þanan, &
fíund undar fern-dalu. \hld\ Warð þar folk mikil &
fon þem alo-waldan \hld\ ovana te Kriste &
godes ęngilo kumen, \hld\ þie im sïðor jungar-dóm, &
skoldun ambaht-skępi \hld\ aftar lêstjen, &
þionon þio-líko: \hld\ só skal man þiod-gode, &
hêrron aftar huldi, \hld\ hevan-kuninge. &
Was im an þem sin-weldi \hld\ sálig barn godes &
lange hwíle, \hld\ unt-þat im þó liovora warð, &
þat he is kraft mikil \hld\ ku̇ðjen wolda &
weroda te willjon. \hld\ Þó for·lét he waldes hleo, &%NOTE: Not hlêo.
ên-ôdjes ard \hld\ ęndi sóhte im eft erlo ge·mang, &
mári męgin-þiode \hld\ ęndi manno drôm, &
géng im þó bi Jordanes staðe: \hld\ þar ina Johannes ant·fand, &
þat friðu-barn godes, \hld\ frôhan sínan, &
hêlagana heven-kuning, \hld\ ęndi þem hęliðun sagda, &
Johannes is jungurun, \hld\ þó he ina gangan ge·sah: &
„þit is þat lamb godes, \hld\ þat þar lôsjan skal &
af þesaro wídon wer-old \hld\ wrêða sundja, &
man-kunnjas mên, \hld\ mári drohtin, &
kuningo kraftigost.“ \hld\ Krist im forð gi·wêt &
an Galileo land, \hld\ godes êgan barn, &
fór im te þem friundun, \hld\ þar he a·fódit was, &
tír-líko a·togan, \hld\ ęndi talda mid wordun &
Krist undar is kunnje, \hld\ kuningo ríkjost, &
hwó sie skoldin iro selvoro \hld\ sundja bótjan, &
hét þat sie im iro harm-werk manag \hld\ hrewan létin, &
feldin iro firin-dádi: \hld\ „nu is it all ge·fullot só, &
só hír alde man \hld\ êr hwanna sprákun, &
ge·hétun eu te helpu \hld\ heven-ríki: &
nu is it giu gi·náhid þurh þes nęrjandan kraft: \hld\ þes mótun gí neotan forð, &
só hwe só gerno wili \hld\ gode þeonogjan, &
wirkjan aftar is willjon.“ \hld\ Þó warð þes werodes filu, &
þero liudjo an lustun: \hld\ wurðun im þea lêra Kristes, &
só swótja þem gi·sïðja. \hld\ He bi·gan im samnon þó &
gumono te jungoron, \hld\ gódoro manno, &
word-spáha weros. \hld\ Géng im þó bi ênes watares staðe, &
þat þar habda Jordan \hld\ anevan Galileo land &
ênna sê ge·warhtan. \hld\ Þar he sittjan fand &
Andreas ęndi Petrus \hld\ bi þem aha-strôme, &
bêðja þea ge·bróðar, \hld\ þar sie an brêd watar &
swíðo niud-líko \hld\ nętti þenidun, &
fiskodun im an þem flóde. \hld\ Þar sie þat friðu-barn godes &
bi þes sêes staðe \hld\ selvo grótta, &
hét þat sie im folgodin, \hld\ kwað þat he im só filu woldi &
godes ríkjas for·geven; \hld\ „al só git hír an Jordanes strôme &
fiskos fáhat, \hld\ só skulun git noh firiho barn &
halon te inkun handun, \hld\ þat sie an heven-ríki &
þurh inka lêra \hld\ líðan mótin, &
faran folk manag.“ \hld\ Þó warð frô-mód hugi &
bêðjun þem gi·bróðrun: \hld\ ant·kęndun þat barn godes, &
liovan hêrron: \hld\ for·létun al saman &
Andreas ęndi Petrus, \hld\ só hwat só sie bi þeru ahu habdun, &
ge·wunstes bi þem watare: \hld\ was im willjo mikil, &
þat sie mid þem godes barne \hld\ gangan móstin, &
samad an is gi·sïðja, \hld\ skoldun sálig-líko &
lôn ant·fáhan: \hld\ só dót liudjo so hwi-lik, &
só þes hêrran wili \hld\ huldi gi·þionon, &
ge·wirkjan is willjon. \hld\ Þó sie bi þes watares staðe &
furðor kwámun, \hld\ þó fundun sie þar ênna fródan man &
sittjan bi þem sêwa \hld\ ęndi is suni twêne, &
Jakobus ęndi Johannes: \hld\ wárun im junga man. &
Sátun im þá ge·sun-fader \hld\ an ênumu sande uppen, &
brugdun ęndi bóttun \hld\ bêðjum handun &
þiu nętti niud-líko, \hld\ þea sie habdun nahtes êr &
for·sliten an þem sêwa. \hld\ Þar sprak im selvo tó &
sálig barn godes, \hld\ hét þat sie an þana sïð mid im, &
Jakobus ęndi Johannes, \hld\ géngin bêðje, &
kind-junge man. \hld\ Þó wárun im Kristes word &
só wirðig an þesaro wer-oldi, \hld\ þat sie bi þes watares staðe &
iro aldan fader \hld\ ênna for·létun, &
fródan bi þem flóde, \hld\ ęndi al þat sie þar fehas êhtun, &
nęttju ęndi nęglit-skipu, \hld\ ge·kurun im þana nęrjandan Krist, &
hêlagna te hêrron, \hld\ was im is helpono þarf &
te gi·þiononne: \hld\ só is allaro þegno ge·hwem, &
wero an þesero wer-oldi. \hld\ Þó gi·wêt im þe waldandes sunu &
mid þem fiuwarjun forð, \hld\ ęndi im þó þana fïfton gi·kôs &
Krist an ênero kôp-stędi, \hld\ kuninges jungoron, &
mód-spáhana man: \hld\ Mattheus was hé hêtan, &
was im ambahtjo \hld\ ęðilero manno, &
skolda þar te is hêrron \hld\ handun ant·fáhan &
tins ęndi tolna; \hld\ trewa habda hé góda, &
aðal-and·bári: \hld\ for·lét al saman &
gold ęndi siluvar \hld\ ęndi geva managa, &
diurje mêðmos, \hld\ ęndi warð im u̇ses drohtines man; &
kôs im þe kuninges þegn \hld\ Krist te hêrran, &
milderan mêðom-gevon, \hld\ þan êr is man-drohtin &
wári an þesero wer-oldi: \hld\ féng im wóðera þing, &
lang-samoron rád. \hld\ Þó warð it allun þem liudjun ku̇ð, &
fon allaro burgo gi·hwem, \hld\ hwó þat barn godes &
samnode ge·sïðos \hld\ ęndi selvo ge·sprak &
só manag wís-lík word \hld\ ęndi wáres só filu, &
torhtes gi·tôgde \hld\ ęndi têkạn manag &
ge·warhte an þesero wer-oldi. \hld\ Was þat an is wordun skín &
iak an is dádjun só same, \hld\ þat hé drohtin was, &
himilisk hêrro \hld\ ęndi te helpu kwam &
an þesan middil-gard \hld\ manno barnun, &
liudjun te þesun liohta. \hld\ Oft ge·deda hé þat an þem lande skín, &
þan hé þar torht-líko \hld\ só manag têkạn gi·warhte, &
þar hé hêlde mid is handun \hld\ halte ęndi blinde, &
lôsde af þeru léf-hêdi \hld\ liudi manage, &
af su·likun suhtjun, \hld\ só þan allaro swároston &
an firiho barn \hld\ fíund bi·wurpun, &
tulgo lang-sam legar. \hld\ Þó fórun þar þie liudi tó &
allaro dago ge·hwi-likes, \hld\ þar u̇sa drohtin was &
selvo undar þem gi·sïðje, \hld\ unt-þat þar ge·samnod warð &
męgin-folk mikil \hld\ managero þiodo, &
þoh sie þar alle be ge·líkumu \hld\ ge·lôvon ni kwámin. &
weros þurh ênan willjon: \hld\ sume sóhtun sie þat waldandes barn, &
armoro manno filu \hld\ —was im átes þarf—, &
þat sie im þar at þeru męnigi \hld\ mates ęndi drankes, &
þigidin at þeru þiodu; \hld\ hwand þar was manag þegạn só gód, &
þie ira alamosnje \hld\ armun mannun &
gerno gávun. \hld\ Sume wárun sie im eft Judeono kunnjes, &
fêgni folk-skępi: \hld\ wárun þar ge·farana te þiu, &
þat sie u̇ses drohtines \hld\ dádjo ęndi wordo &
fáron woldun, \hld\ habdun im fêgnjen hugi, &
wrêðen willjon: \hld\ woldun waldand Krist &
a·lêdjen þem liudjun, \hld\ þat sie is lêron ni hôrdin, &
ne węndin aftar is willjon. \hld\ Suma wárun sie im eft só wíse man, &
wárun im glawe gumon \hld\ ęndi gode werðe, &
a·lesane undar þem liudjun, \hld\ kwámun im þarod be þem lêron Kristes, &
þat sie is hêlag word \hld\ hôrjen móstin, &
línon ęndi lêstjen: \hld\ habdun mid iro ge·lôvon te im &
fasto ge·fangen, \hld\ habdun im ferhten hugi, &
wurðun is þegnos te þiu, \hld\ þat he sie an þiod-welon &
aftar iro ên-dagon \hld\ up ge·bráhti, &
an godes ríki. \hld\ He só gerno ant·féng &
man-kunnjes manag \hld\ ęndi mund-burd gi·hét &
te langaru hwílu, \hld\ ęndi mahta só gi·lêstjen wel. &
Þó warð þar męgin só mikil \hld\ umbi þana márjon Krist, &
liudjo ge·samnod: \hld\ þó gi·sah hé fon allun landun kuman, &
fon allun wídun wegun \hld\ werod te·samne &
lungro liudjo: \hld\ is lof was só wído &
managun ge·márid. \hld\ Þó gi·wêt im mahtig self &
an ênna berg uppan, \hld\ barno ríkjost, &
sundar ge·sittjen, \hld\ ęndi im selvo ge·kôs &
twe-livi ge·talda, \hld\ trew-hafta man, &
gódoro gumono, \hld\ þea hé im te jungoron forð &
allaro dago ge·hwi-likes, \hld\ drohtin welda &
an is ge·sïð-skępja \hld\ simblon hębbjan. &
Nęmnida sie þó bi naman \hld\ ęndi hét sie im þó náhor gangan, &
Andreas ęndi Petrus \hld\ êrist sána, &
ge·bróðar twêne, \hld\ ęndi bêðje mid im, &
Jakobus ęndi Johannes: \hld\ sie wárun gode werðe; &
mildi was hé im an is móde; \hld\ sie wárun ênes mannes suni &
bêðje bi ge·burdjun; \hld\ sie kôs þat barn godes &
góde te jungoron \hld\ ęndi gumono filu, &
márjero manno: \hld\ Mattheus ęndi Þomas, &
Judasas twêna \hld\ ęndi Jakob ǫ́ðran, &
is selves swiri: \hld\ sie wárun fon gi·sustruonjon twêm &
knósles kumana, \hld\ Krist ęndi Jakob, &
góde gadulingos. \hld\ Þó habda þero gumono þar &
þe nęrjendo Krist \hld\ niguni ge·talde, &%TODO: check niguni
trew-hafte man: \hld\ þó hét hé ôk þana te·handon gangan &
selvo mid þem gi·sïðun: \hld\ Símon was hé hêtan; &
hét ôk Bartholomeus \hld\ an þana berg uppan &
faran fan þem folke áðrum \hld\ ęndi Philippus mid im, &
trew-hafte man. \hld\ Þó géngun sie twe-livi samad, &
rinkos te þeru rúnu, \hld\ þar þe rádand sat, &
managoro mund-boro, \hld\ þe allumu man-kunnje &
wið hęllje ge·þwing \hld\ helpan welde, &
formon wið þem ferne, \hld\ só hwem só frummjen wili &
só liov-líka lêra, \hld\ só hé þem liudjun þar &
þurh is gi·wit mikil \hld\ wísjan hogda. &
Þó umbi þana nęrjandon Krist \hld\ náhor géngun &%NOTE ms. -- Þó] V 1 (27r)
su·lika ge·sïðos, \hld\ só hé im selvo ge·kôs, &
waldand undar þem werode. \hld\ Stódun wísa man, &
gumon umbi þana godes sunu \hld\ gerno swíðo, &
weros an willjon: \hld\ was im þero wordo niud, &
þáhtun ęndi þagodun, \hld\ hwat im þero þiodo drohtin, &
weldi waldand self \hld\ wordun ku̇ðjan &
þesum liudjun te liove. \hld\ Þan sat im þe landes hirdi &
gęgin-ward for þem gumun, \hld\ godes êgan barn: &
welda mid is sprákun \hld\ spáh-word manag &
lêrjan þea liudi, \hld\ hwó sie lof gode &
an þesum wer-old-ríkja \hld\ wirkjan skoldin. &
Sat im þó ęndi swígoda \hld\ ęndi sah sie an lango, &
was im hold an is hugi \hld\ hêlag drohtin, &
mildi an is móde, \hld\ ęndi þó is mund ant·lôk, &
wísde mid wordun \hld\ waldandes sunu &
manag már-lík þing \hld\ ęndi þem mannum sagde &
spáhun wordun, \hld\ þem þe hé te þeru spráku þarod, &
Krist alo-waldo, \hld\ ge·koran habda, &
hwi-like wárin allaro \hld\ irmin-manno &
gode werðoston \hld\ gumono kunnjes; &
sagde im þó te sǫ́ðan, \hld\ kwað þat þie sáliga wárin, &
man an þesoro middil-gardun, \hld\ þie hér an iro móde wárin &
arme þurh ôd-módi: \hld\ „þem is þat êwana ríki, &
swíðo hêlag-lík \hld\ an hevan-wange &
sin-líf far·geven.“ \hld\ Kwað þat ôk sálige wárin &
máð-mundje man: \hld\ „þie mótun þie márjon erðe, &
of-sittjen þat selve ríki.“ \hld\ Kwað þat ôk sálige wárin, &
þie hír wiopin iro wammun dádi; \hld\ „þie mótun eft willjon ge·bídan, &
frófre an iro fráhon ríkja. \hld\ Sálige sind ôk, þe sie hír frumono gi·lustid, &
rinkos, þat sie rehto a·dómjen. \hld\ Þes mótun sie werðan an þem ríkja drohtines &
gi·fullit þurh iro ferhton dádi: \hld\ su-líkoro mótun sie frumono bi·knégan &
þie rinkos, þie hír rehto a·dómjad, \hld\ ne willjad an rúnun be·swíkan &
man, þar sie at mahle sittjad. \hld\ Sálige sind ôk þem hír mildi wirðit &
hugi an hęliðo briostun: \hld\ þem wirðit þe hêlego drohtin, &
mildi mahtig selvo. \hld\ Sálige sind ôk undar þesaro managon þiodu, &
þie hębbjad iro herta gi·hrênod: \hld\ þie mótun þane hevenes waldand &
sehan an sínum ríkja.“ \hld\ Kwað þat ôk sálige wárin, &
„þie þe friðu-samo undar þesumu folke libbjod \hld\ ęndi ni willjad êniga fehta ge·wirken, &
saka mid iro selvoro dádjun: \hld\ þie mótun wesan suni drohtines ge·nęmnide, &
hwande hé im wil ge·nádig werðen; \hld\ þes mótun sie niotan lango &
selvon þes sínes ríkjes.“ \hld\ Kwað þat ôk sálige wárin &
þie rinkos, þe rehto weldin, \hld\ „ęndi þurh þat þolod ríkjoro manno &
hęti ęndi harm-kwidi: \hld\ þem is ôk an himile eft &
godes wang for·geven \hld\ ęndi gêst-lík líf &
aftar te êwan-dage, \hld\ só is io ęndi ni kumit, &%NOTE ms. -- aftar] V 2 (32v)
welan wun-sames.“ \hld\ Só habde þó waldand Krist &
for þem erlom þar \hld\ ahto ge·talda &
sálda ge·sagda; \hld\ mid þem skal simbla gi·hwe &
himil-ríki ge·halon, \hld\ ef hé it hębbjan wili, &
etþo hé skal te êwan-daga \hld\ aftar þarvon &
welon ęndi willjon, \hld\ sïðor hé þese wer-old a·givid, &
erð-lívi-gi·skapu, \hld\ ęndi sókit im ǫ́ðar lioht &
só liof só lêð, \hld\ só hé mid þesun liudjun hér &
gi·werkod an þesoro wer-oldi, \hld\ al só it þar þó mid is wordun sagde &
Krist alo-waldo, \hld\ kuningo ríkjost &
godes êgan barn \hld\ jungorun sínun: &
„Ge werðat ôk só sálige“, \hld\ kwað he, „þes iu saka biodat &
liudi aftar þeson lande \hld\ ęndi lêð sprekat, &
hębbjad iu te hoska \hld\ ęndi harmes filu &
ge·wirkjad an þesoro wer-oldi \hld\ ęndi wíti ge·frummjad, &
felgjad iu firin-spráka \hld\ ęndi fíund-skępi, &
lágnjad iuwa lêra, \hld\ dót iu lêðes filu, &
harmes þurh iuwan hêrron. \hld\ Þes látad gi iuwan hugi simbla, &
líf an lustun, \hld\ hwand iu þat lôn stęndit &
an godes ríkja garu, \hld\ gódo ge·hwi-likes, &
mikil ęndi manag-fald: \hld\ þat is iu te médu far·gevan, &
hwand gi hér êr bi·foran \hld\ arvid þolodun, &
wíti an þesoro wer-oldi. \hld\ Wirs is þem ǫ́ðrum, &
giviðig grimmora þing, \hld\ þem þe hér gód êgun, &
wídan worold-welon: \hld\ þie for·slítat iro wunnja hér; &
ge·niudot sie ge·nóges, \hld\ skulun eft narowaro þing &
aftar iro hin-fęrdi \hld\ hęliðos þolojan. &
Þan wópjan þar wan-skęfti, \hld\ þie hér êr an wunnjon sín, &
libbjad an allon lustun, \hld\ ne willjad þes far·látan wiht, &
mêni-gi·þáhtjo, \hld\ þes sie an iro mód spenit, &
lêðoro gi·lêstjo. \hld\ Þan im þat lôn kumid, &
uvil arved-sam, \hld\ þan sie is þane ęndi skulun &
sorgondi ge·sehan. \hld\ Þan wirðid im sêr hugi, &
þes sie þesero wer-oldes só filu \hld\ willjan ful-géngun, &%NOTE ms. -- sie] V end.
man an iro mód-sevon. \hld\ Nu skulun gi im þat mên lahan, &
węrjan mid wordun, \hld\ al só ik giu nu ge·wísjan mag, &
sęggjan sǫ́ð-líko, \hld\ ge·sïðos míne, &
wárun wordun, \hld\ þat gi þesoro wer-oldes nu forð &
skulun salt wesan, \hld\ sundigero manno, &
bótjan iro balu-dádi, \hld\ þat sie an bętara þing, &
folk far·fáhan ęndi for·látan \hld\ fíundes gi·werk, &
diuvales ge·dádi, \hld\ ęndi sókjan iro drohtines ríki. &
Só skulun gi mid iuwon lêrun \hld\ liud-folk manag &
węndjan aftar mínon willjon. \hld\ Ef iuwar þan a·wirðid hwi-lik, &
far·látid þea lêra, \hld\ þea hé lêstjan skal, &
þan is im só þem salte, \hld\ þe man bi sêes staðe &
wído te·wirpit: \hld\ þan it te wihti ni dóg, &
ak it firiho barn \hld\ fótun spurnat, &
gumon an greote. \hld\ Só wirðid þem, þe þat godes word skal &
mannum márjan: \hld\ ef hé im þan látid is mód twehon, &
þat hi ne willja mid hluttro hugi \hld\ te heven-ríkja &
spanen mid is spráku \hld\ ęndi sęggjan spel godes, &
ak węnkid þero wordo, \hld\ þan wirðid im waldand gram, &
mahtig módag, \hld\ ęndi só samo manno barn; &
wirðid allun þan \hld\ irmin-þiodun, &
liudjun a·lêðid, \hld\ ef is lêra ni dugun.“ &
So sprak hé þó spáh-líko \hld\ ęndi sagda spel godes, &
lêrde þe landes ward \hld\ liudi síne &
mid hluttru hugi. \hld\ Hęliðos stódun, &
gumon umbi þana godes sunu \hld\ gerno swíðo, &
weros an willjon: \hld\ was im þero wordo niud, &
þáhtun ęndi þagodun, \hld\ gi·hôrdun þero þiodo drohtin &
sęggjan êw godes \hld\ ęldi-barnun; &
gi·hét im heven-ríki \hld\ ęndi te þem hęliðun sprak: &
„ók mag ik iu sęggjan, \hld\ ge·sïðos mína, &
wárun wordun, \hld\ þat gi þesoro wer-oldes nu forð &
skulun lioht wesan \hld\ liudjo barnun, &
fagar mid firihun \hld\ ovar folk manag, &
wlitig ęndi wun-sam: \hld\ ni mugun iuwa werk mikil &
bi·holan werðan, \hld\ mid hwi-liko gi sea hugi ku̇ðjat: &
þan mêr þe þiu burg ni mag, \hld\ þiu an berge stáð, &
hôh holm-klivu, \hld\ bi·holen werðen, &
wrisi-lík gi·werk, \hld\ ni mugun iuwa word þan mêr &
an þesoro middil-gard \hld\ mannum werðen, &
iuwa dádi bi·dęrnit. \hld\ Dót, só ik iu lêrju: &
látad iuwa lioht mikil \hld\ liudjun skínan, &
manno barnun, \hld\ þat sie far·standan iuwan mód-sevon, &
iuwa werk ęndi iuwan willjon, \hld\ ęndi þes waldand god &
mit hluttro hugi, \hld\ himiliskan fader, &
lovon an þesumu liohte, \hld\ þes hé iu su·lika lêra far·gaf. &
Ni skal neoman lioht, þe it havad, \hld\ liudjun dęrnjan, &
te hardo be·hwelvjan, \hld\ ak hé it hôho skal &
an sęli sęttjan, \hld\ þat þea ge·sehan mugin &
alla ge·liko, \hld\ þea þar inna sind, &
hęliðos an hallu. \hld\ Þan hald ni skulun gi iuwa hêlag word &
an þesumu land-skępa \hld\ liudjun dęrnjen, &
hęlið-kunnje far·helan, \hld\ ak ge it hôho skulun &
brêdjan, þat gi·bod godes, \hld\ þat it allaro barno ge·hwi-lik, &
ovar al þit land-skępi \hld\ liudi far·standan &
ęndi só ge·frummjen, \hld\ só it an forn-dagun &
tulgo wíse man \hld\ wordun ge·sprákun, &
þan sie þana aldan êw \hld\ erlos heldun, &
ęndi ôk su·liku swíðor, \hld\ só ik iu nu sęggjan mag, &
alloro gumono ge·hwi-lik \hld\ gode þionojan, &
þan it þar an þem aldom \hld\ êwa ge·beode. &
Ni wánjat gi þes mit wihtju, \hld\ þat ik bi þiu an þesa wer-old kwámi, &
þat ik þana aldan êw \hld\ irrjen willje, &
felljan undar þesumu folke \hld\ efþo þero fora·sagono &
word wiðar-werpen, \hld\ þea hér só gi·wárja man &
bar-líko ge·budun. \hld\ Êr skal bêðju te·faran, &
himil ęndi erðe, \hld\ þiu nu bi·hlidan standat, &
êr þan þero wordo \hld\ wiht bi·líva &
un·lêstid an þesumu liohte, \hld\ þea sie þesum liudjun hér &
wár-líko ge·budun. \hld\ Ni kwam ik an þesa wer-old te þiu, &
þat ik feldi þero fora·sagono word, \hld\ ak ik siu fulljen skal, &
ókjon ęndi nígjan \hld\ ęldi-barnum, &
þesumu folke te frumu. \hld\ Þat was forn ge·skrivan &
an þem aldon êo \hld\ —ge hôrdun it oft sprekan &
word-wíse man—: \hld\ só hwe só þat an þesoro wer-oldi gi·dót, &
þat hé áðrana \hld\ aldru bi·neote, &
lívu bi·lôsje, \hld\ þem skulun liudjo barn &
dôd a·dêljan. \hld\ Þan willjo ik it iu diopor nu, &
furður bi·fáhan: \hld\ só hwe só ina þurh fíund-skępi, &
man wiðar ǫ́ðrana \hld\ an is mód-sevon &
bilgit an is breostun \hld\ —hwand sie alle ge·bróðar sint, &
sálig folk godes, \hld\ sibbjon bi·tengja, &%TODO: Check etymology of bi·tengja.
man mid mág-skępi—, \hld\ þan wirðit þoh hwe ǫ́ðrumu an is móde só gram, &
líbes weldi ina bi·lôsjen, \hld\ of hé mahti gi·lêstjen só: &
þan is hé sán a·féhit \hld\ ęndi is þes ferahas skolo, &
al su·likes ur-dêljes \hld\ só þe ǫ́ðar was, &
þe þurh is hand-męgin \hld\ hôvdo bi·lôsde &
erl ǫ́ðarna. \hld\ Ôk is an þem êo ge·skrivan &
wárun wordun, \hld\ só gi witon alle, &
þan man is náhiston \hld\ niud-líko skal &
\alst{m}innjan an is \alst{m}óde, \hld\ wesen is \alst{m}águn hold, &
\alst{g}adulingun \alst{g}ód, \hld\ wesen is \alst{g}eva mildi, &
\alst{f}ráhon is \alst{f}riunda ge·hwane, \hld\ ęndi skal is \alst{f}íund hatan, &
wiðer·\alst{st}anden þem mid \alst{st}rídu \hld\ ęndi mid \alst{st}arku hugi, &
\alst{w}ęrjan wiðar \alst{w}rêðun. \hld\ Þan sęggjo ik iu te \alst{w}áron nu, &
\alst{f}ul-líkur for þesumu \alst{f}olke, \hld\ þat gí iuwa \alst{f}íund skulun &
\alst{m}innjon an iuwomu \alst{m}óde, \hld\ só samo só gí iuwa \alst{m}ágos dót, &
an \alst{g}odes namon. \hld\ Dót im \alst{g}ódes filu, &
tôgjat im hluttran hugi, \hld\ holda trewa, &
liof wiðar ira lêðe. \hld\ Þat is lang-sam rád &
manno só hwi-likumu, \hld\ só is mód te þiu &
ge·flíhit wiðar is fíunde. \hld\ Þan mótun gí þea fruma êgan, &
þat gí mótun hêten \hld\ heven-kuninges suni, &
is blíði barn. \hld\ Ne mugun gí iu bętaran rád &
ge·winnan an þesoro wer-oldi. \hld\ Þan sęggjo ik iu te wáron ôk, &
barno ge·hwi-likum, \hld\ þat gí ne mugun mid gi·bolgono hugi &
iuwas gódes wiht \hld\ te godes húsun &
waldande far·gevan, \hld\ þat it imu wirðig sí &
te ant·fáhanne, \hld\ só lango só þú fíund-skępjes wiht, &
wiðer ǫ́ðran man \hld\ in·wid hugis. &
Êr skalt þú þi simbla ge·sónjen \hld\ wið þana sak-waldand, &
ge·módi gi·mahljan: \hld\ sïðor maht þú mêðmos þína &
te þem godes altere a·gevan: \hld\ þan sind sie þemu gódan werðe, &
heven-kuninge. \hld\ Mér skulun gi aftar is huldi þionon, &
godes willjon ful-gán, \hld\ þan ǫ́ðra Judeon duon, &
ef gi willjat êgan \hld\ êwan ríki, &
sin-líf sehan. \hld\ Ôk skal ik iu sęggjan noh, &
hwó it þar an þem aldon \hld\ êo ge·biudid, &
þat ênig erl ǫ́ðres \hld\ idis ni bi·swíka, &
wíf mid wammu. \hld\ Þan sęggjo ik iu te wáron ôk, &
þat þar man is siuni mugun \hld\ swíðo far·lêdjan &
an mirki mên, \hld\ ef hi ina látid is mód spanen, &
þat hé be·ginna þero girnjan, \hld\ þiu imu ge·gangan ni skal. &
Þan haved hé an imu selvon sán \hld\ sundja ge·warhta, &
ge·hęftid an is hertan \hld\ hęlli-wíti. &
Ef þan þana man is siun wili \hld\ etþa is swíðare hand &
far·lêdjen is liðo hwi-lik \hld\ an lêðan weg, &
þan is erlo ge·hwem \hld\ ǫ́ðar bętara, &
firiho barno, \hld\ þat hé ina fram werpa &
ęndi þana lið lôsje \hld\ af is lík-hamon &
ęndi ina áno kuma \hld\ up te himile, &
þan hé só mid allun \hld\ te þem Inferne, &
hwerve mid só hêlun \hld\ an hęlli-grund. &
Þan mênid þiu léf-hêd, \hld\ þat ênig liudjo ni skal &
far·folgan is friunde, \hld\ ef hé ina an firina spanit, &
swás man an saka: \hld\ þan ne sí hé imu eo só swíðo an sibbjun bi·lang, &
ne iro mág-skępi só mikil, \hld\ ef hé ina an morð spęnit, &
bédid balu-werko; \hld\ bętera is imu þan ǫ́ðar, &
þat hé þana friund fan imu \hld\ fer far·werpa, &
míðe þes máges \hld\ ęndi ni hębbja þar êniga minnja tó, &
þat hé móti êno \hld\ up ge·stígan &
\edtext{hôh}{\Afootnote{TODO: Critical note (ms. apparently has hô)}} himil-ríki, \hld\ þan sie hęlli-ge·þwing, &
brêd balu-wíti \hld\ bêðja gi·sókjan, &
uvil arvidi. \hld\ Ôk is an þem êo ge·skrivan &
wárun wordun, \hld\ só gí witun alle, &
þat míðe mên-êðos \hld\ man-kunnjes ge·hwi-lik, &
ni for·swęrje ina selvon, \hld\ hwand þat is sundje te mikil, &
far·lêdid liudi \hld\ an lêðan weg. &
Þan willjo ik iu eft sęggjan, \hld\ þan sán ni swęrja neoman &
ênigan êð-staf \hld\ ęldi-barno, &
ne bi himile þemu hôhon, \hld\ hwand þat is þes hêrron stól, &
ne bi erðu þar undar, \hld\ hwand þat is þes alo-waldon &
fagar fót-skamel, \hld\ nek ênig firiho barno &
ne swęrja bi is selves hôvde, \hld\ hwand he ni mag þar ne swart ne hwít &
ênig hár ge·wirkjan, \hld\ b·útan só it þe hêlago god, &
ge·markode mahtig; \hld\ be·þiu skulun míðan filu &
erlos êð-wordo. \hld\ Só hwe só it ofto dót, &
só wirðid is simbla wirsa, \hld\ hwand he imu gi·wardon ni mag. &
Bi·þiu skal ik iu nu te wárun \hld\ wordun gi·beodan, &
þat gi neo ne swęrjen \hld\ swíðoron êðos, &
méron met mannun, \hld\ b·útan só ik iu mid mínun hér &
swíðo wár-liko \hld\ wordun ge·biudu: &
ef man hwemu saka sókja, \hld\ bi·sęggja þat wáre, &
kweðe já, gef it sí, \hld\ geha þes þar wár is, &
kweðe nên, af it nis, \hld\ láta im ge·nóg an þiu; &
só hwat só is mêr ovar þat \hld\ man ge·frummjad, &
só kumid it al fan uvile \hld\ ęldi-barnun, &
þat erl þurh un·trewa \hld\ ǫ́ðres ni wili &
wordo ge·lôvjan. \hld\ Þan sęggjo ik iu te wáron ôk, &
hwó it þar an þem aldon \hld\ êo ge·biudit: &
só hwe só ôgon ge·nimid \hld\ ǫ́ðres mannes, &
lôsid af is lík-haman, \hld\ etþa is liðo hwi-likan, &
þat he it eft mid is selves skal \hld\ sán ant·gelden &
mid ge·líkun liðjon. \hld\ Þan willjo ik iu lêrjan nu, &
þat gí só ni wrekan \hld\ wrêða dádi, &
ak þat gí þurh ôd-módi \hld\ al ge·þologjan &
wítjes ęndi wammes, \hld\ só hwat só man iu an þesoro wer-oldi ge·dóe. &
Dóe alloro erlo ge·hwi-lik \hld\ ǫ́ðrom manne &
frume ęndi ge·fóri, \hld\ só he willje, þat im firiho barn &
gódes an·gęgin dóen. \hld\ Þan wirðit im god mildi, &
liudjo só hwi-likum, \hld\ só þat lêstjen wili. &
Êrod gí arme man, \hld\ dêljad iuwan ôd-welon &
undar þero þurftigon þiodu; \hld\ ne rókjad, hweðar gí is ênigan þank ant·fáhan &
efþo lôn an þesoro léhnjon wer-oldi, \hld\ ak huggjat te iuwomu leovon hêrran &
þero gevono te gelde, \hld\ þat sie iu god lôno, &
mahtig mund-boro, \hld\ só hwat só gi is þurh is minnes gi·dót. &
Ef þú þan gevogjan wili \hld\ gódun mannun &
fagare feho-skattos, \hld\ þar þú eft frumono hugis &
mêr ant·fáhan, \hld\ te hwí havas þú þes êniga méda fon gode &
etþa lôn an þemu is liohte? \hld\ hwand þat is léhni feho. &
Só is þes alles ge·hwat, \hld\ þe þú ǫ́ðrun ge·duos &
liudjon te leove, \hld\ þar þú hugis eft ge·lík neman &
þero wordo ęndi þero werko: \hld\ te hwí wêt þi þes u̇sa waldand þank, &
þes þú þín só bi·filhis \hld\ ęndi ant·fáhis eft þan þú wili? &
iuwan ôð-welon \hld\ gevan gi þem armun mannun, &
þe ina iu an þesoro wer-oldi ne lônon \hld\ ęndi rómot te iuwes waldandes ríkja. &
Te hlúd ni dó þú it, \hld\ þan þú mid þínun handun bi·felhas &
þína alamosna þemu armon manne, \hld\ ak dó im þurh ôd-módjen &
gerno þurh godes þank: \hld\ þan móst þú eft geld niman, &
swíðo liof-lík lôn, \hld\ þar þú is lango bi·þarft, &
fagaroro frumono. \hld\ Só hwat só þú is só þurh ferhtan hugi &
darno ge·dêljas, \hld\ —so is u̇sumu drohtine werð— &
ne galpo þú far þínun gevun te swíðo, \hld\ noh ênig gumono ne skal, &
þat siu im þurh ídale hróm \hld\ eft ni werðe &
lêð-líko far·loren. \hld\ Þanna þú skalt lôn nemen &
fora godes ôgun \hld\ gódero werko. &
Ôk skal ik iu ge·beodan, \hld\ þan gi willjad te bedu hnígan &
ęndi willjad te iuwomu hêrron \hld\ helpono biddjan, &
þat he iu a·láte \hld\ lêðes þinges, &
þero sakono ęndi þero sundjono, \hld\ þea gi iu selvon hír &
wrêða ge·wirkjad, \hld\ þat gi it þan for ǫ́ðrumu werode ni duad: &
ni márjad it far męnigi, \hld\ þat iu þes man ni lovon, &
ni diurjan þero dádjo, \hld\ þat gi iuwes drohtines gi·bed &
þurh þat ídala hróm \hld\ al ne far·leosan. &
Ak þan gi willjan te iuwomo hêrron \hld\ helpono biddjan, &
þiggjan þeo-líko, \hld\ —þes iu is þarf mikil— &
þat iu sigi-drohtin \hld\ sundjono tómja, &
þan dót gi þat só darno: \hld\ þoh wêt it iuwe drohtin self &
hêlag an himile, \hld\ hwand imu nis bi·holan n·eo·wiht &
ne wordo ne werko. \hld\ He látid it þan al ge·werðan só, &
só gi ina þan biddjad, \hld\ þan gi te þero bedo hnígad &
mid hluttru hugi.“ \hld\ Hęliðos stódun, &
gumon umbi þana godes sunu \hld\ gerno swíðo, &
weros an willjon: \hld\ was im þero wordo niud, &
þáhtun ęndi þagodun, \hld\ was im þarf mikil, &
þat sie þat eft ge·hogdin, \hld\ þat im þat hêlaga barn &
an þana forman sïð \hld\ filu mid wordun &
torhtes ge·talde. \hld\ Þó sprak im eft ên þero twe-livjo an·gęgin, &
glauworo gumono, \hld\ te þem godes barne: &
„Hérro þe gódo“, \hld\ kwað he, „u̇s is þínoro huldi þarf, &
te gi·wirkenne þínna willjon, \hld\ ęndi ôk þínoro wordo só self, &
allaro barno bętst, \hld\ þat þú u̇s bedon lêres, &
jungoron þíne, \hld\ só Johannes duot, &
diur-lík dôperi, \hld\ dago ge·hwi-likas &
is werod mid wordun, \hld\ hwí sie waldand skulun, &
gódan grótjan. \hld\ Dó þína jungorun só self: &
ge·rihti u̇s þat ge·rúni.“ \hld\ Þó habda eft þe ríkjo garu &
sán aftar þiu, \hld\ sunu drohtines, &
gód word an·gęgin: \hld\ „Þan gi god willjan“, kwað he, &
„weros mid iuwon wordun \hld\ waldand grótjan, &
allaro kuningo kraftigostan, \hld\ þan kweðad gi, só ik iu lêrju: &
Fadar u̇sa \hld\ firiho barno, &
þú bist an þem hôhon \hld\ himila ríkja, &
ge·wíhid sí þín namo \hld\ wordo ge·hwi-liko. &
Kuma þín \hld\ kraftag ríki. &
Werða þín willjo \hld\ ovar þesa wer-old alla, &
só sama an erðo, \hld\ só þar uppa ist &
an þem hôhon \hld\ himilo ríkja. &
Gef u̇s dago ge·hwi-likes rád, \hld\ drohtin þe gódo, &
þína hêlaga helpa, \hld\ ęndi a·lát u̇s, hevenes ward, &
managoro mên-skuldjo, \hld\ al só we ǫ́ðrum mannum dóan. &
Ne lát u̇s far·lêdjan \hld\ lêða wihti &
só forð an iro willjon, \hld\ só wí wirðige sind, &
ak help u̇s wiðar allun \hld\ uvilon dádjun. &
Só skulun gi biddjan, \hld\ þan gi te bede hnígad &
weros mid iuwom wordun, \hld\ þat iu waldand god &
lêðes a·láte \hld\ an leut-kunnja. &
Ef gi þan willjad a·látan \hld\ liudjo ge·hwi-likun &
þero sakono ęndi þero sundjono, \hld\ þe sie wið iu selvon hír &
wrêða ge·wirkjat, \hld\ þan a·látid iu waldand god, &
fadar ala-mahtig \hld\ firin-werk mikil, &
managoro mên-skuldjo. \hld\ Ef iu þan wirðid iuwa mód te stark, &
þat gi ne wileat ǫ́ðrun \hld\ erlun a·látan, &
weron wam-dádi, \hld\ þan ne wil iu ôk waldand god &
grim-werk far·gevan, \hld\ ak gi skulun is geld niman, &
swíðo lêð-lik lôn \hld\ te languru hwílu, &
alles þes un·rehtes, \hld\ þes gi ǫ́ðrum hír &
gi·lêstjad an þesumu liohte \hld\ ęndi þan wið liudjo barn &
þea saka ni gi·sónjad, \hld\ êr gi an þana sïð faran, &
weros fon þesoro wer-oldi. \hld\ Ok skal ik iu te wárun sęggjan, &
hwó gi lêstjan skulun \hld\ lêra mína: &
þan gi iuwa fastonnja \hld\ frummjan willjan, &
minson iuwa mên-dádi, \hld\ þan ni duad gi þat te managom ku̇ð, &
ak míðad is far ǫ́ðrum mannun: \hld\ þoh wêt mahtig god, &
waldand iuwan willjan, \hld\ þoh iu werod ǫ́ðar, &
liudjo barn ne lovon. \hld\ He gildid is iu lôn aftar þiu, &
iuwa hêlag fadar \hld\ an himil-ríkja, &
þes ge im mid su·likum ôd-módja, \hld\ erlos þeonod, &
só ferht-líko undar þesumu folke. \hld\ Ne willjat feho winnan &
erlos an un·reht, \hld\ ak wirkjad up te gode &
man aftar médu: \hld\ þat is méra þing, &
þan man hír an erðu \hld\ ôdag libbja, &
wer-old-skattes ge·wono. \hld\ Ef gi willjad mínun wordun hôrjan, &
þan ne samnod gi hír sink mikil \hld\ silovres ne goldes &
an þesoro middil-gard, \hld\ mêðom-hordes, &
hwand it rotat hír an roste, \hld\ ęndi ręgin-þeovos far·stelad, &
wurmi a·wardjad, \hld\ wirðid þat gi·wádi far·slitan, &
ti-gangid þe gold-welo. \hld\ Léstjad iuwa gódon werk, &
samnod iu an himile \hld\ hord þat méra, &
fagara feho-skattos: \hld\ þat ni mag iu ênig fíund be·niman, &
ne-wiht an·węndjan, \hld\ hwand þe welo standid &
garu iu te·gęgnes, \hld\ só hwat só gi gódes þarod, &
an þat himil-ríki \hld\ hordes ge·samnod, &
hęliðos þurh iuwa hand-geva, \hld\ ęndi hębbjad þarod iuwan hugi fasto; &
hwand þar ist alloro manno gi·hwes \hld\ mód-ge·þáhti, &
hugi ęndi herta, \hld\ þar is hord ligid, &
sink ge·samnod. \hld\ Nis eo só sálig man, &
þat mugi an þesoro brêdon wer-old \hld\ bêðju ant·hengjan, &
ge þat hi an þesoro erðo \hld\ ôdag libbja, &
an allun wer-old-lustun wesa, \hld\ ge þoh waldand gode &
te þanke ge·þeono: \hld\ ak he skal alloro þingo gi·hwes &
simbla ǫ́ðar-hweðar \hld\ ên far·látan &
etþo lusta þes lík-hamon \hld\ etþo líf êwig. &
Be·þiu ni gornot gi umbi iuwa ge·garuwi, \hld\ ak huggjad te gode fasto, &
ne mornont an iuwomu móde, \hld\ hwat gi eft an morgan skulin &
etan efþo drinkan \hld\ etþo an hębbjan &
weros te ge·wę́dja: \hld\ it wêt al waldand god, &
hwes þea bi·þurvun, \hld\ þea im hír þionod wel, &
folgod iro frôhan willjon. \hld\ Hwat, gi þat bi þesun fuglun mugun &
wár-líko undar·witan, \hld\ þea hír an þesoro wer-oldi sint, &
farad an feðar-hamun: \hld\ sie ni kunnun ênig feho winnan, &
þoh givid im drohtin god \hld\ dago ge·hwi-likes &
helpa wiðar hungre. \hld\ Ôk mugun gi an iuwom hugi markon, &
weros umbi iuwa ge·wádi, \hld\ hwó þie wurti sint &
fagoro ge·fratohot, \hld\ þea hír an felde stád, &
berht-líko ge·blóid: \hld\ ne mahta þe burges ward, &
Salomon þe suning, \hld\ þe habda sink mikil, &
mêðom-hordas mêst, \hld\ þero þe ênig man êhti, &
welono ge·wunnan \hld\ ęndi allaro ge·wádjo kust,— &
þoh ni mohte he an is líve, \hld\ þoh he habdi alles þeses landes ge·wald, &
a·winnan su·lik ge·wádi, \hld\ só þiu wurt havad, &
þiu hír an felde stád \hld\ fagoro ge·gariwit, &
lilli mid só liof-líku blómon: \hld\ ina wádit þe landes waldand &
hér fan hevenes wange. \hld\ Mér is im þoh umbi þit hęliðo kunni, &
liudi sint im liovoron mikilu, \hld\ þea he im an þesumu lande ge·warhte, &
waldand an willjon sínan. \hld\ Be·þiu ne þurvon gi umbi iuwa ge·wádi sorgon, &
ne gornot gi umbi iuwa ge·gariwi te swíðo: \hld\ god wili is alles rádan, &
helpan fan hevenes wange, \hld\ ef gi willjad aftar is huldi þeonon. &
Gerot gi simbla êrist þes godes ríkjas, \hld\ ęndi þan duat aftar þem is gódun werkun, &
rómod gi rehtoro þingo: \hld\ þan wili iu þe ríkjo drohtin &
gevon mid alloro gódu ge·hwi-liku, \hld\ ef gi im þus ful-gangan willjad, &
só ik iu te wárun hír \hld\ wordun sęggjo. &
Ne skulun gi ênigumu manne \hld\ un·rehtes wiht, &
dęrvjes a·dêljan, \hld\ hwand þe dóm eft kumid &
ovar þana selvon man, \hld\ þar it im te sorgon skal, &
werðan þem te wítja, \hld\ þe hír mid is wordun ge·sprikid &
un·reht ǫ́ðrum. \hld\ Neo þat iuwar ênig ne dua &
gumono an þesom gardon \hld\ geldes etþo kôpes, &
þat hi un·reht gi·met \hld\ ǫ́ðrumu manne &
mên-ful mako, \hld\ hwand it simbla mótjan skal &
erlo ge·hwi-likomu, \hld\ su·lik só he it ǫ́ðrumu ge·dód, &
só kumid it im eft te·gęgnes, \hld\ þar he gerno ne wili &
ge·sehan is sundjon. \hld\ Ôk skal ik iu sęggjan noh, &
hwar gi iu wardon skulun \hld\ wítjo mêsta, &
mên-werk manag: \hld\ te hwí skalt þú ênigan man be·sprekan, &
bróðar þínan, \hld\ þat þú undar is bráhon ge·sehas &
halm an is ôgon, \hld\ ęndi ge·huggjan ni wili &
þana swáran balkon, \hld\ þe þú an þínoro siuni havas, &
hard trio ęndi hevig. \hld\ Lát þi þat an þínan hugi fallan, &
hwó þú þana êrist a·lôsjas: \hld\ þan skínid þi lioht be·foran, &
ôgun werðad þi ge·oponot; \hld\ þan maht þú aftar þiu &
swáses mannes gesiun \hld\ sïðor ge·bótjan, &
ge·hêljan an is hôvde. \hld\ Só mag þat an is hugi méra &
an þesoro middil-gard \hld\ manno ge·hwi-likumu, &
wesan an þesoro wer-oldi, \hld\ þat hi hír wammas ge·duot, &
þan hi ahtogja \hld\ ǫ́ðres mannes &
saka ęndi sundja, \hld\ ęndi havad im selvo mêr &
firin-werko ge·frumid. \hld\ Ef he wili is fruma lêstjan, &
þan skal hi ina selvon êr \hld\ sundjono a·tómjan, &
lêð-werko lôson: \hld\ sïðor mag hi mid is lêrun werðan &
hęliðun te helpu, \hld\ sïðor hi ina hluttran wêt, &
sundjono sikoran. \hld\ Ne skulun gi swínum te·foran &
iuwa mere-gríton makon \hld\ etþo mêðmo ge·striuni, &
hêlag hals-męni, \hld\ hwand siu it an horu spurnat, &
sulwjad an sande: \hld\ ne witun súvrjas ge·skêð, &
fagaroro fratoho. \hld\ Su-lik sint hír folk manag, &
þe iuwa hêlag word \hld\ hôrjan ne willjad, &
ful-gangan godes lêrun: \hld\ ne witun gódes ge·skêð, &
ak sind im lári word \hld\ leovoron mikilu, &
umbi·þarvi þing, \hld\ þanna þeot-godes &
werk ęndi willjo. \hld\ Ne sind sie wirðige þan, &
þat sie ge·hôrjan iuwa hêlag word, \hld\ ef sie is ne willjad an iro hugi þęnkjan, &
ne línon ne lêstjan. \hld\ Þem ni sęggjan gi iuworo lêron wiht, &
þat gi þea spráka godes \hld\ ęndi spel managu &
ne far·leosan an þem liudjun, \hld\ þea þar ne willjan gi·lôvjan tó, &
wároro wordo. \hld\ Ôk skulun gí iu wardon filu &
listjun undar þesun liudjun, \hld\ þar gí aftar þesumu lande farad, &
þat iu þea luggjon ne mugin \hld\ lêron be·swíkan &
ni mid wordun ni mid werkun. \hld\ Sie kumad an su·likom ge·wádjon te iu, &
fagoron fratohon: \hld\ þoh hębbjad sie fêknan hugi: &
þea mugun gi sán ant·kęnnjan, \hld\ só gi sie kuman ge·sehad: &
sie sprekad wís-lík word, \hld\ þoh iro werk ne dugin, &
þero þegno ge·þáhti. \hld\ Hwand gi witun, þat eo an þorniun ne skulun &
wín-beri wesan \hld\ efþa welon eo·wiht, &
fagororo fruhtjo, \hld\ nek ôk fígun ne lesad &
hęliðos an hiopon. \hld\ Þat mugun gi undar·huggjan wel, &
þat eo þe uvilo bôm, \hld\ þar he an erðu stád, &
góden wastum ne givid, \hld\ nek it ôk god ni ge·skóp, &
þat þe gódo bôm \hld\ gumono barnun &
bári bittres wiht, \hld\ ak kumid fan alloro bámo ge·hwi-likumu &
su·lik wastom te þesero wer-oldi, \hld\ só im fan is wurtjon ge·dregid, &
etþa berht etþa bittar. \hld\ Þat mênid þoh breost-hugi, &
managoro mód-sevon \hld\ manno kunnjes, &
hwó alloro erlo ge·hwi-lik \hld\ ôgit selvo, &
meldod mid is mu̇ðu, \hld\ hwi-likan he mód havad, &
hugi umbi is herte: \hld\ þes ni mag he far·helan eo·wiht, &
ak kumad fan þem uvilan man \hld\ in·wid-rádos, &
bittara balu-spráka, \hld\ su·lik só hi an is breostun havad &
ge·hęftid umbi is herte: \hld\ simbla is hugi ku̇ðid, &
is willjon mid is wordun, \hld\ ęndi farad is werk aftar þiu. &
Só kumad fan þemu gódan manne \hld\ glau and-wordi, &
wís-lík fan is ge·wittja, \hld\ þat hi simbla mid is wordu ge·sprikid, &
man mid is míðu su·lik, \hld\ só he an is móde havad &
hord umbi is herte. \hld\ Þanan kumad þea hêlagan lêra, &
swíðo wun-sam word, \hld\ ęndi skulun is werk aftar þiu &
þeodu ge·þíhan, \hld\ þegnun managun &
werðan te willjon, \hld\ al só it waldand self &
gódun mannun far·givid, \hld\ god alo-mahtig, &
himilisk hêrro, \hld\ hwand sie áno is helpa ni mugun &
ne mid wordun ne mid werkun \hld\ wiht a·þęngjan &
gódes an þesun gardun. \hld\ Be·þiu skulun gumono barn &
an is ênes kraft \hld\ alle gi·lôvjan. &
Ôk skal ik iu wísjan, \hld\ hwó hír wegos twêna &
liggjad an þesumu liohte, \hld\ þea farad liudjo barn, &
al irmin-þiod. \hld\ Þero is ǫ́ðar sán &
wíd stráta ęndi brêd, \hld\ —farid sie werodes filu, &
man-kunnjes manag, \hld\ hwand sie þarod iro mód spenit, &
wer-old-lusta weros— \hld\ þiu an þea wirson hand &
liudi lêdid, \hld\ þar sie te far·lora werðad, &
hęliðos an hęllju, \hld\ þar is hêt ęndi swart, &
ęgis-lík an innan: \hld\ óði ist þarod te faranne &
eldi-barnun, \hld\ þoh it im at þemu ęndje ni dugi. &
Þan ligid eft ǫ́ðar \hld\ engira mikilu &
weg an þesoro wer-oldi, \hld\ fęrid ina werodes lút, &
fáho folk-skępi: \hld\ ni willjad ina firiho barn &
gerno gangan, \hld\ þoh he te godes ríkja, &
an þat êwiga líf, \hld\ erlos lêdja. &
Þan nimad gi iu þana engjan: \hld\ þoh he só óði ne sí &
firihon te faranne, \hld\ þoh skal hi te frumu werðan &
só hwemu só ina þurh-gęngid, \hld\ só skal is geld niman, &
swíðo lang-sam lôn \hld\ ęndi líf êwig, &
diur-líkan drôm. \hld\ Eo gi þes drohtin skulun, &
waldand biddjen, \hld\ þat gi þana weg mótin &
fan foran ant·fáhan \hld\ ęndi forð þurh gi·gangan &
an þat godes ríki. \hld\ He ist garu simbla &
wiðar þiu te gevanne, \hld\ þe man ina gerno bidid, &
fergot firiho barn. \hld\ Sókjad fadar iuwan &
up te þemu êwinom ríkja: \hld\ þan mótun gi ina aftar þiu &
te iuworu frumu fïðan. \hld\ Ku̇ðjad iuwa fard þarod &
at iuwas drohtines durun: \hld\ þan werðad iu andón aftar þiu, &
himil-portun ant·hlidan, \hld\ þat gi an þat hêlage lioht, &
an þat godes ríki \hld\ gangan mótun, &
sin-líf sehan. \hld\ Ôk skal ik iu sęggjan noh &
far þesumu werode allun \hld\ wár-lík biliði, &
þat alloro liudjo só hwi-lik, \hld\ só þesa mína lêra wili &
ge·haldan an is herton \hld\ ęndi wil iro an is hugi a·þęnkjan, &
lêstjan sea an þesumu lande, \hld\ þe gi·líko duot &
wísumu manne, \hld\ þe gi·wit havad, &
horska hugi-skęfti, \hld\ ęndi hús-stędi kiusid &
an fastoro foldun \hld\ ęndi an felisa uppan &
wégos wirkid, \hld\ þar im wind ni mag, &
ne wág ne watares strôm \hld\ wihtju ge·tiunjan, &
ak mag im þar wið un·gi·widereon \hld\ allun standan &
an þemu felise uppan, \hld\ hwand it só fasto warð &
gi·stellit an þemu stêne: \hld\ anthavad it þiu stędi niðana, &
wreðid wiðar winde, \hld\ þat it wíkan ni mag. &
Só duot eft manno só hwi-lik, \hld\ só þesun mínun ni wili &
lêrun hôrjen ne þero \hld\ lêstjen wiht, &
só duot þe un·wíson \hld\ erla ge·líko, &
un·ge·wittigon were, \hld\ þe im be watares staðe &
an sande wili \hld\ sęli-hús wirkjan, &
þar it westrani wind \hld\ ęndi wágo strôm, &
sêes u̇ðjon te·sláad; \hld\ ne mag im sand ęndi greot &
ge·wreðjen wið þemu winde, \hld\ ak wirðid te·worpan þan, &
te·fallen an þemu flóde, \hld\ hwand it an fastoro nis &
erðu ge·timbrod. \hld\ Só skal allaro erlo ge·hwes &
werk ge·þíhan wiðar þiu, \hld\ þe hi þius mín word frumid, &
haldid hêlag ge·bod.“ \hld\ Þó bi·gunnun an iro hugi wundron &
męgin-folk mikil: \hld\ ge·hôrdun mahtiges godes &
liof-líka lêra; \hld\ ne wárun an þemu lande ge·wuno, &
þat sie eo fan su·likun êr \hld\ sęggjan ge·hôrdin &
wordun etþo werkun. \hld\ Far·stódun wíse man, &
þat he só lêrde, \hld\ liudjo drohtin, &
wárun wordun, \hld\ só he ge·wald habde, &
allun þem un·ge·líko, \hld\ þe þar an êr-dagun &
undar þem liud-skępja \hld\ lêrjon wárun &
a·koran undar þemu kunnje: \hld\ ne habdun þiu Kristes word &
ge·makon mid mannun, \hld\ þe he far þero męnigi sprak, &
ge·bôd uppan þemu berge. \hld\ He im þó bêðju be·falh &
ge te sęggennja \hld\ sínom wordun, &
hwó man himil-ríki \hld\ ge·halon skoldi, &
wíd-brêdan welan, \hld\ gia he im ge·wald far·gaf, &
þat sie móstin hêljan \hld\ halte ęndi blinde, &
liudjo léf-hêdi, \hld\ legar-będ manag, &
swára suhti, \hld\ giak he im selvo ge·bôd, &
þat sie at ênigumu manne \hld\ méde ne námin, &
diurje mêðmos: \hld\ „ge·huggjad gi“, kwað he, —„hwand iu is þiu dád kuman, &
þat ge·wit ęndi þe wís-dóm, \hld\ ęndi iu þea ge·wald far·givid &
alloro firiho fadar, \hld\ só gi sie ni þurvun mid ênigo feho kôpon, &
médjan mid ênigun mêðmun,— \hld\ só wesat gi iro mannun forð &
an iuwon hugi-skęftjun \hld\ helpono mildja, &
lêrjad gi liudjo barn \hld\ lang-samna rád, &
fruma forð-wardes; \hld\ firin-werk lahad, &
swára sundjon. \hld\ Ne látad iu silovar nek gold &
wihti þes wirðig, \hld\ þat it eo an iuwa ge·wald kuma, &
fagara feho-skattos: \hld\ it ni mag iu te ênigoro frumu hwęrgin, &
werðan te ênigumu willjon. \hld\ Ne skulun gi ge·wádjas þan mêr &
erlos êgan, \hld\ b·útan só gi þan an hębbjan, &
gumon te garewea, \hld\ þan gi gangan skulun &
an þat gi·mang innan. \hld\ Neo gi umbi iuwan męti ni sorgot, &
lęng umbi iuwa líf-nare, \hld\ hwand þene lêrjand skulun &
fódjan þat folk-skępi: \hld\ þes sint þea fruma werða, &
leov-líkes lônes, \hld\ þe hi þem liudjun sagad. &
wirðig is þe wurhtjo, \hld\ þat man ina wel fódja, &
þana man mid mósu, \hld\ þe só managoro skal &
seola bi·sorgan \hld\ ęndi an þana sïð spanen, &
gêstos an godes wang. \hld\ Þat is grôtara þing, &
þat man bi·sorgon skal \hld\ seolun managa, &
hwó man þea ge·halde \hld\ te heven-ríkja, &
þan man þene lík-hamon \hld\ liudi-barno &
mósu bi·morna. \hld\ Be·þiu man skulun &
haldan þene hold-líko, \hld\ þe im te heven-ríkja &
þene weg wísit \hld\ ęndi sie wam-skaðun, &
feondun wit-fáhit \hld\ ęndi firin-werk lahid, &
swára sundjon. \hld\ Nu ik iu sęndjan skal &
aftar þesumu land-skępje \hld\ só lamb undar wulvos: &
só skulun gi undar iuwa fíund faren, \hld\ undar filu þeodo, &
undar mis-líke man. \hld\ Hębbjad iuwan mód wiðar þem &
só glawan te·gęgnes, \hld\ só samo só þe gelwo wurm, &
nádra þiu féha, \hld\ þar siu iro níð-skępjes, &
witodes wánit, \hld\ þat man iu undar þemu werode ne mugi &
be·swíkan an þemu sïðe. \hld\ Far þiu gi sorgon skulun, &
þat iu þea man ni mugin \hld\ mód-ge·þáhti, &
willjan a·wardjen. \hld\ Wesat iu so wara wiðar þiu, &
wið iro fêknjon dádjun, \hld\ só man wiðar fíundun skal. &
Þan wesat gi eft an iuwon dádjun \hld\ dúvon ge·líka, &
hębbjad wið erlo ge·hwene \hld\ ên-faldan hugi, &
mildjan mód-sevon, \hld\ þat þar man neg·ên &
þurh iuwa dádi \hld\ be·drogan ne werðe, &
be·swikan þurh iuwa sundja. \hld\ Nu skulun gi an þana sïð faran, &
an þat ârundi: \hld\ þar skulun gi arvidjes só filu &
ge·þolon undar þeru þiod \hld\ ęndi ge·þwing só samo &
manag ęndi mis-lík, \hld\ hwand gi an mínumu namon &
þea liudi lêrjat. \hld\ Be·þiu skulun gi þar lêðes filu &
fora wer-old-kuningun, \hld\ wítjas ant·fáhan. &
Oft skulun gi þar for ríkja \hld\ þurh þius mín rehtun word &
ge·bundane standen \hld\ ęndi bêðju ge·þologjan, &
ge hosk ge harm-kwidi: \hld\ umbi þat ne látad gi iuwan hugi twíflon, &
sevon swíkandjan: \hld\ gi ni þurvun an ênigun sorgun wesan &
an iuwomu hugi hwęrgin, \hld\ þan man iu for þea hêri forð &
an þene gast-sęli \hld\ gangan hêtid, &
hwat gi im þan te·gęgnes skulin \hld\ gódoro wordo, &
spáh-líkoro ge·sprekan, \hld\ hwand iu þiu spód kumid, &
helpe fon himile, \hld\ ęndi sprikid þe hêlogo gêst, &
mahtig fon iuwomu munde. \hld\ Be·þiu ne and-rádad gi iu þero manno níð &
ne forhtjat iro fíund-skępi: \hld\ þoh sie hębbjan iuwas ferahes ge·wald, &
þat sie mugin þene lík-hamon \hld\ lívu be·neotan, &
a·slahan mid swerde, \hld\ þoh sie þeru seolun ne mugun &
wiht a·wardjan. \hld\ Antd-rádad iu waldand god, &
forhtjad fader iuwan, \hld\ frummjad gerno &
is ge·bod-skępi, \hld\ hwand hi havad bêðjes gi·wald, &
liudjo líves \hld\ ęndi ôk iro lík-hamon &
gek þero seolon só self: \hld\ ef gi iuwa an þem sïðe þarod &
far·liosat þurh þesa lêra, \hld\ þan mótun gi sie eft an þemu liohte godes &
be·foran fïðan, \hld\ hwand sie fader iuwa, &
haldid hêlag god \hld\ an himil-ríkja. &
Ne kumat þea alle te himile, \hld\ þea þe hír hrópat te mi &
manno te mund-burd. \hld\ Managa sind þero, &
þea willjad alloro dago ge·hwi-likes \hld\ te drohtine hnígan, &
hrópad þar te helpu \hld\ ęndi huggjad an ǫ́ðar, &
wirkjad wam-dádi: \hld\ ne sind im þan þiu word fruma, &
ak þea mótun hwervan \hld\ an þat himiles lioht, &
gangan an þat godes ríki, \hld\ þea þes gerne sint, &
þat sie hír ge·frummjen \hld\ fader ala-waldan &
werk ęndi willjon. \hld\ Þea ni þurvun mid wordun só fílu &
hrópan te helpu, \hld\ hwanda þe hêlogo god &
wêt alloro manno ge·hwes \hld\ mód-ge·þáhti, &
word ęndi willjon, \hld\ ęndi gildid im is werko lôn. &
Be·þiu skulun gi sorgon, \hld\ þan gi an þene sïð farad, &
hwó gi þat ârundi \hld\ ti ęndja be·brengen. &
Þan gi líðan skulun \hld\ aftar þesumu land-skępja, &
wído aftar þesoro wer-oldi, \hld\ al só iu wegos lêdjad, &
brêd stráta te burg, \hld\ simbla sókjad gi iu þene bętston sán &
man undar þeru męnegi \hld\ ęndi ku̇ðjad imu iuwan móð-sevon &
wárun wordun. \hld\ Ef sie þan þes wirðige sint, &
þat sie iuwa gódun werk \hld\ gerno ge·lêstjen &
mid hluttru hugi, \hld\ þan gi an þemu húse mid im &
wonod an willjon \hld\ ęndi im wel lônod, &
geldad im mid gódu \hld\ ęndi sie te gode selvon &
wordun ge·wíhad \hld\ ęndi sęggjad im wissan friðu, &
hêlaga helpa \hld\ heven-kuninges. &
Ef sie þan só sáliga \hld\ þurh iro selvoro dád &
werðan ni mótun, \hld\ þat sie iuwa werk frummjen, &
lêstjen iuwa lêra, \hld\ þan gi fan þem liudjun sán, &
farad fan þemu folke, \hld\ —þe iuwa friðu hwirvid &
eft an iuworo selvoro sïð,— \hld\ ęndi látad sie mid sundjun forð, &
mid balu-werkun búan \hld\ ęndi sókjad iu burg ǫ́ðra, &
mikil man-werod, \hld\ ęndi ne látad þes melmes wiht &
folgan an iuwom fótun, \hld\ þanan þe man iu ant·fáhan ne wili, &
ak skuddjat it fan iuwon skóhun, \hld\ þat it im eft te skamu werðe, &
þemu werode te ge·wit-skępje, \hld\ þat iro willjo ne dóg. &
Þan sęggjo ik iu te wárun, \hld\ só hwan só þius wer-old ęndjad &
ęndi þe márjo dag \hld\ ovar man farid, &
þat þan Sodomo-burg, \hld\ þiu hír þurh sundjon warð &
an af·grundi \hld\ êldes kraftu, &
fiuru bi·fallen, \hld\ þat þiu þan havad friðu méran, &
mildiran mund-burd, \hld\ þan þea man êgin, &
þe iu hír wiðar-werpat \hld\ ęndi ne willjad iuwa word frummjen. &
Só hwe só iu þan ant·fáhit \hld\ þurh ferhtan hugi, &
þurh mildjan mód, \hld\ só havad mínan forð &
willjon ge·warhten \hld\ ęndi ôk waldand god, &
ant·fangan fader iuwan, \hld\ firiho drohtin, &
ríkjan rád-gevon, \hld\ þene þe al reht bi·kan. &
wêt waldand self, \hld\ ęndi willjan lônot &
gumono ge·hwi-likumu, \hld\ só hwat só hi hír gódes ge·duot, &
þoh hi þurh minnja godes \hld\ manno hwi-likumu &
willjandi far·geve \hld\ watares drinkan, &
þat hi þurftigumu manne \hld\ þurst ge·hêlje, &
kaldes brunnan. \hld\ Þesa kwidi werðad wára, &
þat eo ne bi·lívid, \hld\ ne hi þes lôn skuli, &
fora godes ôgun \hld\ geld ant·fáhan, &
méda manag-falde, \hld\ só hwat só hi is þurh mína minnja ge·duot. &
Só hwe só mín þan far·lógnid \hld\ liudi-barno, &
hęliðo for þesoro hęrju, \hld\ só dóm ik is an himile só self &
þar uppe far þem alo-waldan fader \hld\ ęndi for allumu is ęngilo krafte, &
far þeru mikilon męnigi. \hld\ Só hwi-lik só þan eft manno barno &
an þesoro wer-oldi ne wili \hld\ wordun míðan, &
ak gihit far gum-skępi, \hld\ þat he mín jungoro sí, &
þene willju ek eft ógjan \hld\ far ôgun godes, &
fora alloro firiho fader, \hld\ þar folk manag &
for þene alo-waldon \hld\ alla gangad &
reðinon wið þene ríkjon. \hld\ Þar willju ik imu an reht wesan &
mildi mund-boro, \hld\ só hwemu só mínun hír &
wordun hôrid \hld\ ęndi þiu werk frumid, &
þea ik hír an þesumu berge uppan \hld\ ge·boden hębbju.“ &
Habda þó te wárun \hld\ waldandes sunu &
ge·lêrid þea liudi, \hld\ hwó sie lof gode &
wirkjan skoldin. \hld\ Þó lét hi þat werod þanan &
an alloro halva ge·hwi-lika, \hld\ hęri-skępi manno &
sïðon te selðon. \hld\ Habdun selves word, &
ge·hôrid heven-kuninges \hld\ hêlaga lêra, &
só eo te wer-oldi sint \hld\ wordo ęndi dádjo, &
man-kunnjes manag \hld\ ovar þesan middil-gard &
sprákono þiu spáhiron, \hld\ só hwe só þiu spel ge·frang, &
þea þar an þemu berge ge·sprak \hld\ barno ríkjast. &
Ge·wêt imu þó umbi þrea naht aftar þiu \hld\ þesoro þiodo drohtin &
an Galileo land, \hld\ þar he te ênum gômum warð, &
ge·bedan þat barn godes: \hld\ þar skolda man êna brúd gevan, &
muna-líka magað. \hld\ Þar Maria was, &
mid iro suni selvo, \hld\ sálig þiorna, &
mahtiges móder. \hld\ Managoro drohtin &
géng imu þó mid is jungoron, \hld\ godes êgan barn, &
an þat hôha hús, \hld\ þar þe hęri drank, &
þea Judeon an þemu gast-sęli: \hld\ he im ôk at þem gômun was, &
giak hi þar ge·ku̇ðde, \hld\ þat hi habda kraft godes, &
helpa fan himil-fader, \hld\ hêlagna gêst, &
waldandes wís-dóm. \hld\ Werod blíðode, &
wárun þar an luston \hld\ liudi at-samne, &
gumon glad-módje. \hld\ Géngun ambaht-man, &
skęnkjon mid skálun, \hld\ drógun skírjane wín &
mid orkun ęndi mid alo-fatun; \hld\ was þar erlo drôm &
fagar an flęttja, \hld\ þó þar folk undar im &
an þem bęnkjon só bętst \hld\ blíðsea af·hóvun, &
wárun þar an wunnjun. \hld\ Þó im þes wínes brast, &
þem liudjun þes líðes: \hld\ is ni was far·lêvid wiht &
hwęrgin an þemu húse, \hld\ þat for þene hęri forð &
skęnkjon drógin, \hld\ ak þiu skapu wárun &
líðes a·lárid. \hld\ Þó ni was lang te þiu, &
þat it sán ant·funda \hld\ frío skónjosta, &
Kristes móder: \hld\ géng wið iro kind sprekan, &
wið iro sunu selvon, \hld\ sagda im mid wordun, &
þat þea werdos þó mêr \hld\ wínes ne habdun &
þem gęstjun te gômun. \hld\ Siu þó gerno bad, &
þat is þe hêlogo Krist \hld\ helpa ge·riedi &
þemu werode te willjon. \hld\ Þó habda eft is word garu &
mahtig barn godes \hld\ ęndi wið is móder sprak: &
„hwat ist mi ęndi þi“, \hld\ kwað he, „umbi þesoro manno lið, &
umbi þeses werodes wín? \hld\ Te hwí sprikis þú þes, wíf, só filu, &
manos mi far þesoro męnigi? \hld\ Ne sint mína noh &
tídi kumana.“ \hld\ Þan þoh gi·trúoda siu wel &
an iro hugi-skęftjun, \hld\ hêlag þiorne, &
þat is aftar þem wordun \hld\ waldandes barn, &
hêljandoro bętst \hld\ helpan weldi. &
Hét þó þea ambaht-man \hld\ idiso skónjost, &
skęnkjon ęndi skap-wardos, \hld\ þea þar skoldun þero skolu þionon, &
þat sie þes ne word ne werk \hld\ wiht ne far·létin, &
þes sie þe hêlogo Krist \hld\ hêtan weldi &
lêstjan far þem liudjun. \hld\ Lárja stódun þar &
stên-fatu sehsi. \hld\ Þó só stillo ge·bôd &
mahtig barn godes, \hld\ só it þar manno filu &
ne wissa te wárun, \hld\ hwó he it mid is wordu ge·sprak; &
he hét þea skęnkjon \hld\ þó skírjas watares &
þiu fatu fulljen, \hld\ ęndi hi þar mid is fingrun þó, &
segnade selvo \hld\ sínun handun, &
warhte it te wíne \hld\ ęndi hét is an ên wégi hlaðen, &
skęppjen mid ênoro skálon, \hld\ ęndi þó te þem skęnkjon sprak, &
hét is þero gęstjo, \hld\ þe at þem gômun was &
þemu hêroston \hld\ an hand gevan, &
ful mid folmun, \hld\ þemu þe þes folkes þar &
ge·weld aftar þemu werde. \hld\ Reht só hi þes wínes ge·drank, &
só ni mahte he be·míðan, \hld\ ne hi far þeru męnigi sprak &
te þemu brúdi-gumon, \hld\ kwað þat simbla þat bętste líð &
alloro erlo ge·hwi-lik \hld\ êrist skoldi &
gevan at is gômun: \hld\ „undar þiu wirðid þero gumono hugi &
a·wękid mid wínu, \hld\ þat sie wel blíðod, &
drunkan drômjad. \hld\ Þan mag man þar dragan aftar þiu &
líht-líkora líð: \hld\ só ist þesoro liudjo þau. &
Þan havas þú nu wunder-líko \hld\ werd-skępi þínan &
ge·markod far þesoro męnigi: \hld\ hétis far þit manno folk &
alles þínes wínes \hld\ þat wirsiste &
þíne ambaht-man \hld\ êrist brengjan, &
gevan at þínun gômun. \hld\ Nu sint þína gęsti sade, &
sint þíne druhtingos \hld\ drunkane swíðo, &
is þit folk frô-mód: \hld\ nu hétis þú hír forð dragan &
alloro líðo lof-samost, \hld\ þero þe ik eo an þesumu liohte ge·sah &
hwęrgin hębbjan. \hld\ Mid þius skoldis þú u̇s hin-dag êr &
gevon ęndi gômjan: \hld\ þan it alloro gumono ge·hwi-lik &
ge·þigedi te þanke.“ \hld\ Þó warð þar þegạn manag &
ge·war aftar þem wordun, \hld\ sïðor sie þes wínes ge·drunkun, &
þat þar þe hêlogo Krist \hld\ an þemu húse innan &
têkạn warhte: \hld\ trúodun sie sïðor &
þiu mêr an is mund-burd, \hld\ þat hi habdi maht godes, &
ge·wald an þesoro wer-oldi. \hld\ Þó warð þat só wído ku̇ð &
ovar Galileo land \hld\ Judeo liudjun, &
hwó þar selvo ge·deda \hld\ sunu drohtines &
water te wíne: \hld\ þat warð þar wundro êrist, &
þero þe hi þar an Galilea \hld\ Judeo liudjon, &
têkno ge·tôgdi. \hld\ Ne mag þat ge·tęlljan man, &
ge·sęggjan te sǫ́ðan, \hld\ hwat þar sïðor warð &
wundres undar þemu werode, \hld\ þar waldand Krist &
an godes namon \hld\ Judeo liudjon &
allan langan dag \hld\ lêra sagde, &
gi·hét im heven-ríki \hld\ ęndi hęlljo ge·þwing &
węride mid wordun, \hld\ hét sie wara godes, &
sin-líf sókjan: \hld\ þar is seolono lioht, &
drôm drohtines \hld\ ęndi dag-skímon, &
gód-lík-nissja godes; \hld\ þar gêst manag &
wunod an willjan, \hld\ þe hír wel þęnkid, &
þat he hír bi·halde \hld\ heven-kuninges ge·bod. &
Ge·wêt imu þó mid is jungoron \hld\ fan þem gômun forð &
Kristus te Kapharnaum, \hld\ kuningo ríkjost, &
te þeru márjon burg. \hld\ Megin samnode, &
gumon imu te·gęgnes, \hld\ gódoro manno &
sálig ge·sïði: \hld\ weldun þiu is swótjan word &
hêlag hôrjen. \hld\ Þar im ên hunno kwam, &
ên gód man an·gęgin \hld\ ęndi ina gerno bad &
helpan hêlagne, \hld\ kwað þat hi undar is híwiskja &
ênna lefna lamon \hld\ lango habdi, &
seokan an is selðon: \hld\ „só ina ênig sęggjo ne mag &
handun ge·hêljen. \hld\ Nu is im þínoro helpono þarf, &
frô mín þe gódo.“ \hld\ Þó sprak im eft þat friðu-barn godes &
sán aftar þiu \hld\ selvo te·gęgnes, &
kwað þat he þar kwámi \hld\ ęndi þat kind weldi &
nęrjan af þeru nôdi. \hld\ Þó im náhor géng &
þe man far þeru męnigi \hld\ wið só mahtigna &
wordun wehslan: \hld\ „ik þes wirðig ne bium,“ kwað he, &
„hêrro þe gódo, \hld\ þat þú an mín hús kumes, &
sókjas mína seliða, \hld\ hwand ik bium só sundig man &
mid wordun ęndi mid werkun. \hld\ Ik ge·lôvju þat þú ge·wald havas, &
þat þú ina hinana maht \hld\ hêlan ge·wirkjan, &
waldand frô mín: \hld\ ef þú it mid þínun wordun ge·sprikis, &
þan is sán þiu léf-hêd lôsot \hld\ ęndi wirðid is lík-hamo &
hêl ęndi hrêni, \hld\ ef þú im þína helpa far·givis. &
Ik bium mi ambaht-man, \hld\ hębbju mi ôdes ge·nóg, &
welono ge·wunnen: \hld\ þoh ik undar ge·weldi sí &
aðal-kuninges, \hld\ þoh hębbju ik erlo ge·trôst, &
holde hęri-rinkos, \hld\ þea mi só ge·hôriga sint, &
þat sie þes ne word ne werk \hld\ wiht ne far·látad, &
þes ik sie an þesumu land-skępje \hld\ lêstjan héte, &
ak sie farad ęndi frummjad \hld\ ęndi eft te iro frôhan kumad, &
holde te iro hêrron. \hld\ Þoh ik at mínumu hús êgi &
wíd-brêdene welon \hld\ ęndi werodes ge·nóg, &
hęliðos hugi-dęrvje, \hld\ þoh ni gi·dar ik þi só hêlagna &
biddjen, barn godes, \hld\ þat þú an mín bú gangas, &
sókjas mína seliða, \hld\ hwand ik só sundig bium, &
wêt mína far·wurhti.“ \hld\ Þó sprak eft waldand Krist, &
þe gumo wið is jungoron, \hld\ kwað þat hi an Judeon hwęrgin &
undar Israheles \hld\ avoron ne fundi &
ge·makon þes mannes, \hld\ þe io mêr te gode &
an þemu land-skępi \hld\ ge·lôvon habdi, &
þan hluttron te himile: \hld\ „nu látu ik iu þar hôrjen tó, &
þar ik it iu te wárun hír \hld\ wordun sęggjo, &
þat noh skulun ęli-þeoda \hld\ ôstane ęndi westane, &
man-kunnjes kuman \hld\ manag te·samne, &
hêlag folk godes \hld\ an heven-ríki: &
þea motun þar an Abrahames \hld\ ęndi an Isaakes só self &
ęndi ôk an Jakobes, \hld\ gódoro manno, &
barmun restjen \hld\ ęndi bêðju ge·þologjan, &
welon ęndi willjon \hld\ ęndi wonod-sam líf, &
gód lioht mid gode. \hld\ Þan skal Judeono filu, &
þeses ríkjas suni \hld\ be·róvode werðen, &
be·dêlide su·likoro diurðo, \hld\ ęndi skulun an dalun þiustron &
an þemu alloro ferristan \hld\ ferne liggen. &
Þar mag man ge·hôrjen \hld\ hęliðos kwíðjan, &
þar sie iro torn manag \hld\ tandon bítad; &
þar ist grist-grimmo \hld\ ęndi grádag fiur, &
hard hęlljo ge·þwing, \hld\ hêt ęndi þiustri, &
swart sin-nahti \hld\ sundja te lône, &
wrêðoro ge·wurhtjo, \hld\ só hwemu só þes willjon ne havad, &
þat he ina a·lôsje, \hld\ êr hi þit lioht a·geve, &
węndje fan þesoro wer-oldi. \hld\ Nu maht þú þi an þínan willjon forð &
sïðon te selðun; \hld\ þan findis þú ge·sundan at hús &
mago-jungan man: \hld\ mód is imu an luston, &
þat barn is ge·hêlid, \hld\ só þú bédi te mi: &
it wirðid al só ge·lêstid, \hld\ só þú ge·lôvon havas &
an þínumu hugi hardo.“ \hld\ Þó sagde heven-kuninge, &
þe ambaht-man \hld\ alo-waldon gode &
þank for þero þiodo, \hld\ þes he imu at su·likun þarvun halp. &
Habda þo gi·ârundid, \hld\ al só he welde, &
sálig-líko: \hld\ gi·wêt imu an þana sïð þanan, &
wende an is willjan, \hld\ þar he welon êhte, &
bú ęndi bodlos: \hld\ fand þat barn ge·sund, &
kind-jungan man. \hld\ Kristes wárun þó &
word ge·fullot: \hld\ hi ge·wald habda &
te tôgjanna têkạn, \hld\ só þat ni mag gi·tęlljen man, &
ge·ahton ovar þesoro erðu, \hld\ hwat he þurh is ênes kraft &
an þesaro middil-gard \hld\ máriða ge·frumide, &
wundres ge·warhte, \hld\ hwand al an is ge·weldi stád, &
himil ęndi erðe. \hld\ Þó ge·wêt imu þe hêlogo Krist &
forð-wardes faren, \hld\ fręmide alo-mahtig &
alloro dago ge·hwi-likes, \hld\ drohtin þe gódo, &
liudjo barnum leof, \hld\ lêrde mid wordun &
godes willjon gumun, \hld\ habda imu jungorono filu &
simbla te gi·sïðun, \hld\ sálig folk godes, &
manno męgin-kraft, \hld\ managoro þeodo, &
hêlag hęri-skępi, \hld\ was is helpono gód, &
mannun mildi. \hld\ Þó hi mid þeru męnigi kwam, &
mid þiu brahtmu þat barn godes \hld\ te burg þeru hôhon, &
þe nęrjendo te Naim: \hld\ þar skolde is namo werðen &
mannun ge·márid. \hld\ Þó géng mahtig tó &
nęrjendo Krist, \hld\ antat he gi·náhid was, &
hêljandero bętst: \hld\ þó sáhun sie þar ên hrêo dragan, &
ênan líf-lôsan lík-hamon \hld\ þea liudi fórjen, &
beran an ênaru báru \hld\ út at þera burges dore, &
magu-jungan man. \hld\ Þiu móder aftar géng &
an iro hugi hriwig \hld\ ęndi handun slóg, &
karode ęndi kúmde \hld\ iro kindes dôð, &
idis arm-skapan; \hld\ it was ira ênag barn: &
siu was iru widowa, \hld\ ne habda wunnja þan mêr, &
bi·úten te þemu ênagun \hld\ sunje al geláten &
wunnja ęndi willjan, \hld\ ant-tat ina iru wurd be·nam, &
mári metodo-ge·skapu. \hld\ Megin folgode, &
burg-liudjo ge·brak, \hld\ þar man ina an báru dróg, &
jungan man te grave. \hld\ Þar warð imu þe godes sunu, &
mahtig mildi \hld\ ęndi te þeru móder sprak, &
hét þat þiu widowa \hld\ wóp far·léti, &
kara aftar þemu kinde: \hld\ „þú skalt hír kraft sehan, &
waldandes gi·werk: \hld\ þi skal hír willjo ge·standen, &
frófra far þesumu folke: \hld\ ne þarft þú ferah karon &
barnes þínes.“ \hld\ *Þuo hie ti þero báron géng &
iak hie ina selvo ant·hrên, \hld\ suno drohtines, &
hêlagon handon, \hld\ ęndi ti þem hęliðe sprak, &
hiet ina só ala-jungan \hld\ up a·standan, &
a·rísan fan þeru restun. \hld\ Þie rink up a·sat, &
þat barn an þero bárun: \hld\ warð im eft an is briost kuman &
þie gêst þuru godes kraft, \hld\ ęndi hie te·gęgnes sprak, &
þe man wið is mágos. \hld\ Þuo ina eft þero muoder bi·falah &
hêlandi Krist an hand: \hld\ hugi warð iro te frovra, &
þes wíves an wunnjon, \hld\ hwand iro þar su·lik willjo gi·stuod. &
Fell siu þó te fuotun Kristes \hld\ ęndi þena folko drohtin &
lovoda for þero liudjo męnigi, \hld\ hwand hie iro at só liobes ferahe &
mundoda wiðer metodi-gi·skęftje: \hld\ far·stuod siu þat hie was þie mahtigo drohtin, &
þie hêlago, þie himiles gi·waldid, \hld\ ęndi þat hie mahti gi·helpan managon, &
allon irmin-þiedon. \hld\ Þuo bi·gunnun þat ahton managa, &
þat wunder, þat under þem weroda gi·burida, \hld\ kwáðun þat waldand selvo, &
mahtig kwámi þarod is męnigi wíson, \hld\ ęndi þat hie im só márjan sandi &
wár-sagon an þero wer-oldes ríki, \hld\ þie im þar su·likan willjon frumidi. &
warð þar þuo erl manag \hld\ ęgison bi·fangan, &
þat folk warð an forohton: \hld\ gi·sáhun þena is ferah êgan, &
dages lioht sehan, \hld\ þena þe êr dôð for·nam, &
an suht-będdjon swalt: \hld\ þuo was im eft gi·sund after þiu, &
kind-jung a·kwikot. \hld\ Þuo warð þat ku̇ð obar all &
avaron Israheles. \hld\ Reht só þuo ávand kwam, &
só warð þar all gi·samnod \hld\ seokora manno, &
haltaro ęndi hávaro, \hld\ só hwat só þar hwęrgin was, &
þia lévun under þem liudjon, \hld\ ęndi wurðun þar gi·lêdit tuo, &
kumana te Kriste, \hld\ þar hie im þuru is kraft mikil &
halp ęndi sie hêlda, \hld\ ęndi liet sia eft gi·haldana þanan &
wendan an iro willjon. \hld\ Be·þiu skal man is werk lovon, &
diuran is dádi, \hld\ hwand hie is drohtin self, &
mahtig mund-boro \hld\ manno kunnje, &
liudjo só hwi-likon, \hld\ só þar gi·lôbit tuo &
an is word ęndi an is werk. \hld\ Þuo was þar werodes só filo &
allaro ęli-þiodo \hld\ kuman te þem êron Kristes, &
te só mahtiges mund-burd. \hld\ Þuo welda hie þar êna męri líðan, &
þie godes suno mid is jungron \hld\ anevan Galilea-land, &
waldand ênna wágo strôm. \hld\ Þuo hiet hie þat werod ǫ́ðar &
forð-werdes faran, \hld\ ęndi hie gi·wêt im fahora sum &
an ênna nakon innan, \hld\ nęrjendi Krist, &
slápan sïð-wórig. \hld\ Segel up dádun &
weder-wísa weros, \hld\ lietun wind after &
manon ovar þena męri-strôm, \hld\ unþat hie te middjan kwam, &
waldand mid is werodu. \hld\ Þuo bi·gan þes wedares kraft, &
u̇st up stígan, \hld\ u̇ðjun wahsan; &
swang gi·swerk an gi·mang: \hld\ þie sêw warð an hruoru, &
wan wind ęndi water; \hld\ weros sorogodun, &
þiu męri warð só muodag, \hld\ ni wánda þero manno nig·ên &
lęngron líves. \hld\ Þuo sia landes ward &
wękidun mid iro wordon \hld\ ęndi sagdun im þes wedares kraft, &
bádun þat im gi·náðig \hld\ nęrjendi Krist &
wurði wið þem watare: \hld\ „efþa wí skulun hier te wunder-kwálu &
sweltan an þeson sêwe.“ \hld\ Self up a·rês &
þie guodo godes suno \hld\ ęndi te is jungron sprak, &
hiet þat sia im wedares gi·win \hld\ wiht ni and-rédin: &
„te hwí sind gi só forhta?“ \hld\ kwat-hie. „Nis iu noh fast hugi, &
gi·lôvo is iu te luttil. \hld\ Nis nu lang te þiu, &
þat þia strômos skulun \hld\ stilrun werðan &
gi þit *wedar wun-sam.“ \hld\ Þo hi te þem winde sprak &
ge te þemu sêwa só self \hld\ ęndi sie smultro hét &
bêðja ge·bárjan. \hld\ Sie gi·bod lêstun, &
waldandes word: \hld\ weder stillodun, &
fagar warð an flóde. \hld\ Þó bi·gan þat folk undar im, &
werod wundrajan, \hld\ ęndi suma mid iro wordun sprákun, &
hwi-lik þat só mahtigoro \hld\ manno wári, &
þat imu só þe wind ęndi þe wág \hld\ wordu hôrdin, &
bêðja is gi·bod-skępjes. \hld\ Þó habda sie þat barn godes &
gi·nęrid fan þeru nôdi: \hld\ þe nako furðor skręid, &%NOTE: ęi is original.
hôh-hurnid skip; \hld\ hęliðos kwámun, &
liudi te lande, \hld\ sagdun lof gode, &
máridun is męgin-kraft. \hld\ Kwam þar manno filu &
an·gęgin þemu godes sunje; \hld\ he sie gerno ant·féng, &
só hwene só þar mid hluttru hugi \hld\ helpa sóhte; &
lêrde sie iro gi·lôvon \hld\ ęndi iro lík-hamon &
handun hêlde: \hld\ nio þe man só hardo ni was &
gi·sêrit mid suhtjun: \hld\ þoh ina Satanases &
fêknja jungoron \hld\ fíundes kraftu &
habdin undar handun \hld\ ęndi is hugi-skęfti, &
gi·wit a·wardid, \hld\ þat he wódjendi &
fóri undar þemu folke, \hld\ þoh im simbla ferh far·gaf &
hêlandjo Krist, \hld\ ef he te is handun kwam, &
drêf þea diuvlas þanan \hld\ drohtines kraftu, &
wárun wordun, \hld\ ęndi im is ge·wit far·gaf, &
lét ina þan hêlan \hld\ wiðer hettjandun, &
gaf im wið þie fíund friðu, \hld\ ęndi im forð gi·wêt &
an só hwi-lik þero lando, \hld\ só im þan leovost was. &
Só deda þe drohtines sunu \hld\ dago ge·hwi-likes &
gód werk mid is jungeron, \hld\ só neo Judeon umbi þat &
an þea is mikilun kraft \hld\ þiu mêr ne ge·lôvdun, &
þat he alo-waldo \hld\ alles wári, &
landes ęndi liudjo: \hld\ þes sie noh lôn nimat, &
wídana wrak-sïð, \hld\ þes sie þar þat ge·win drivun &
wið selvan þene sunu drohtines. \hld\ Þó he im mid is ge·sïðon gi·wêt &
eft an Galilaeo land, \hld\ godes êgan barn, &
fór im te þem friundun, \hld\ þar he a·fódid was &
ęndi al undar is kunnje \hld\ kind-jung a·wóhs, &
þe hêlago hêljand. \hld\ Umbi ina hęri-skępi, &
þeoda þrungun; \hld\ þar was þegạn manag &
só sálig undar þem ge·sïðe. \hld\ Þar drógun ênna seokan man &
erlos an iro armun: \hld\ weldun ina for ôgun Kristes, &
brengjan for þat barn godes \hld\ —was im bótono þarf, &
þat ina ge·hêldi \hld\ hevenes waldand, &
manno mund-boro—, \hld\ þe was êr só managan dag &
liðu-wastmon bi·lamod, \hld\ ni mahte is lík-hamon &
wiht ge·waldan. \hld\ Þan was þar werodes só filu, &
þat sie ina fora þat barn godes \hld\ brengjan ni mahtun, &
ge·þringan þurh þea þioda, \hld\ þat sie só þurftiges &
sunnja ge·sagdin. \hld\ Þó gi·wêt imu an ênna sęli innan &
hêljando Krist; \hld\ hwarf warð þar umbi, &
męgin-þeodo ge·mang. \hld\ Þó bi·gunnun þea man spreken, &
þe þene léfna lamon \hld\ lango fórdun, &
bárun mid is będdju, \hld\ hwó sie ina ge·drógin fora þat barn godes, &
an þat werod innan, \hld\ þar ina waldand Krist &
selvo gi·sáwi. \hld\ Þó géngun þea ge·sïðos tó, &
hóvun ina mid iro handun \hld\ ęndi uppan þat hús stigun, &
slitun þene sęli ovana \hld\ ęndi ina mid sélun létun &
an þene rakud innan, \hld\ þar þe ríkjo was, &
kuningo kraftigost. \hld\ Reht só he ina þó kuman gi·sah &
þurh þes húses hróst, \hld\ só he þó an iro hugi far·stód, &
an þero manno mód-sevon, \hld\ þat sie mikilana te imu &
ge·lôvon habdun, \hld\ þó he for þen liudjun sprak, &
kwað þat he þene siakon man \hld\ sundjono tómjan &
látan weldi. \hld\ Þó sprákun im eft þea liudi an·gęgin, &
gram-harde Judeon, \hld\ þea þes godes barnes &
word aftar-warodun, \hld\ kwáðun þat þat ni mahti gi·werðen só, &
grim-werk far·geven, \hld\ bi·útan god êno, &
waldand þesaro wer-oldes. \hld\ Þó habda eft is word garu &
mahtig barn godes: \hld\ „ik gi·dón þat“, kwað he, „an þesumu manne skín, &
þe hír só siak ligid \hld\ an þesumu sęli innan, &
te wundron gi·wêgid, \hld\ þat ik ge·wald hębbju &
sundja te far·gevanne \hld\ ęndi ôk seokan man &
te ge·hêljanne, \hld\ só ik ina hrínan ni þarf.“ &
Manoda ina þó \hld\ þe márjo drohtin, &
liggjandjan lamon, \hld\ hét ina far þem liudjun a·standan &
up alo-hêlan \hld\ ęndi hét ina an is ahslun niman, &
is bed-gi·wádi te baka; \hld\ he þat gi·bod lêste &
sniumo for þemu gi·sïðja \hld\ ęndi géng imu eft ge·sund þanan, &
hêl fan þemu húse. \hld\ Þó þes só manag hêðin man, &
weros wundradun, \hld\ kwáðun þat imu waldand self, &
god alo-mahtig \hld\ far·gevan habdi &
méron mahti \hld\ þan elkor ênigumu mannes sunje, &
kraft ęndi kústi; \hld\ sie ni weldun ant·kęnnjan þoh, &
Judeo liudi, \hld\ þat he god wári, &
ne ge·lôvdun is lêran, \hld\ ak habdun im lêðan stríd, &
wunnun wiðar is wordun: \hld\ þes sie werk hlutun, &
lêð-lík lôn-geld, \hld\ ęndi só noh lango skulun, &
þes sie ni weldun hôrjen \hld\ heven-kuninges, &
Kristes lêrun, \hld\ þea he ku̇ðde ovar al, &
wído aftar þesaro wer-oldi, \hld\ ęndi lét sie is werk sehan &
allaro dago ge·hwi-likes, \hld\ is dádi skawon, &
hôrjen is hêlag word, \hld\ þe he te helpu ge·sprak &
manno barnun, \hld\ ęndi só manag mahtig-lík &
têkạn ge·tôgda, \hld\ þat sie gi·trúodin þiu bet, &
gi·lôvdin an is lêra. \hld\ He só managan lík-hamon &
balu-suhtjo ant·band \hld\ ęndi bóta ge·skęride, &
far·gaf fêgjun ferah, \hld\ þem þe fu̇sid was &
hęlið an hęl-sïð: \hld\ þan gi·deda ina þe hêland self, &
Krist þurh is kraft mikil \hld\ kwikan aftar dôða, &
lét ina an þesaro wer-oldi forð \hld\ wunnjono neotan. &
Só hêlde he þea haltun man \hld\ ęndi þea hávon só self, &
bótta, þem þar blinde wárun, \hld\ lét sie þat berhte lioht, &
sin-skóni sehan, \hld\ sundja lôsda, &
gumono grim-werk. \hld\ Ni was gio Judeono be·þiu, &
lêðes liud-skępjes \hld\ gi·lôvo þiu bętara &
an þene hêlagon Krist, \hld\ ak habdun im hardene mód, &
swíðo starkan stríd, \hld\ far·standan ni weldun, &
þat sie habdun for·fangan \hld\ fíundun an willjan, &
liudi mid iro ge·lôvun. \hld\ Ni was gio þiu latoro be·þiu &
sunu drohtines, \hld\ ak he sagde mid wordun, &
hwó sie skoldin ge·halon \hld\ himiles ríki, &
lêrde aftar þemu lande, \hld\ habde imu þero liudjo só filu &
gi·wenid mid is wordun, \hld\ þat im werod mikil, &
folk folgoda, \hld\ ęndi he im filu sagda, &
be biliðjun þat barn godes, \hld\ þes sie ni mahtun an iro breostun far·standan, &
undar·huggjan an iro herton, \hld\ êr it im þe hêlago Krist &
ovar þat erlo folk \hld\ oponun wordun &
þurh is selves kraft \hld\ sęggjan welda, &
márjan hwat he mênde. \hld\ Þar ina męgin umbi, &
þioda þrungun: \hld\ was im þarf mikil &
te gi·hôrjenne \hld\ heven-kuninges &
wár-fastun word. \hld\ He stód imu þó bi ênes watares staðe, &
ni welde þó bi þemu ge·þringe \hld\ ovar þat þegno folk &
an þemu lande uppan \hld\ þea lêra ku̇ðjan, &
ak géng imu þó þe gódo \hld\ ęndi is jungaron mid imu, &
friðu-barn godes, \hld\ þemu flóde náhor &
an ên skip innan, \hld\ ęndi it skalden hét &
lande rúmur, \hld\ þat ina þea liudi só filu, &
þioda ni þrungi. \hld\ Stód þegạn manag, &
werod bi þemu watare, \hld\ þar waldand Krist &
ovar þat liudjo folk \hld\ lêra sagde: &
„hwat, ik iu sęggjan mag“, \hld\ kwað he, „ge·sïðos míne, &
hwó imu ên erl bi·gan \hld\ an erðu sájan &
hrên-korni mid is handun. \hld\ Sum it an hardan stên &
ovan-wardan fel, \hld\ erðon ni habda, &
þat it þar mahti wahsan \hld\ efþa wurtjo gi·fáhan, &
kínan efþa bi·klíven, \hld\ ak warð þat korn far·loren, &
þat þar an þeru léian gi·lag. \hld\ Sum it eft an land bi·fel, &
an erðun aðal-kunnjes: \hld\ bi·gan imu aftar þiu &
wahsen wán-líko \hld\ ęndi wurtjo fáhan, &
lód an lustun: \hld\ was þat land só gód, &
fránisko gi·fehod. \hld\ Sum it eft bi·fallen warð &
an êna starka strátun, \hld\ þar stópon géngun, &
hrosso hóf-slaga \hld\ ęndi hęliðo tráda; &
warð imu þar an erðu \hld\ ęndi eft up gi·géng, &
bi·gan imu an þemu wege wahsen; \hld\ þó it eft þes werodes far·nam, &
þes folkes fard mikil \hld\ ęndi fuglos a·lásun, &
þat is þemu éksan wiht \hld\ aftar ni móste &
werðan te willjan, \hld\ þes þar an þene weg bi·fel. &
Sum warð it þan bi·fallen, \hld\ þar só filu stódun &
þikkero þorno \hld\ an þemu dage; &
warð imu þar an erðu \hld\ ęndi eft up gi·géng, &
kén imu þar ęndi klivode. \hld\ Þó slógun þar eft krúd an gi·mang, &
węridun imu þene wastom: \hld\ habda it þes waldes hlea &
forana ovar-fangan, \hld\ þat it ni mahte te ênigaro frumu werðen, &
ef it þea þornos \hld\ só þringan móstun.“ &
Þó sátun ęndi swígodun \hld\ ge·sïðos Kristes, &
word-spáha weros: \hld\ was im wundạr mikil, &
be hwi-likun biliðjun \hld\ þat barn godes &
su·lik sǫ́ð-lík spel \hld\ sęggjan bi·gunni. &
Þó bi·gan is þero erlo \hld\ ên frágojan &
holdan hêrron, \hld\ hnêg imu te·gęgnes &
tulgo werð-liko: \hld\ „hwat, þú ge·wald havas“, kwað he, &
„ia an himile ia an erðu, \hld\ hêlag drohtin, &
uppa ęndi niðara, \hld\ bist þú alo-waldo &
gumono gêsto, \hld\ ęndi wí þíne jungaron sind, &
an u̇sumu hugi holde. \hld\ Hérro þe gódo, &
ef it þín willjo sí, \hld\ lát u̇s þínaro wordo þar &
ęndi gi·hôrjen, \hld\ þat wí it aftar þi &
ovar al Kristin-folk \hld\ ku̇ðjan mótin. &
wí witun þat þínun wordun \hld\ wár-lík biliði &
forð folgojad, \hld\ ęndi u̇s is firinun þarf, &
þat wí þín word ęndi þín werk, \hld\ —hwand it fan su·likumu ge·wittja kumid— &
þat wí it an þesumu lande \hld\ at þi línon mótin.“ &
Þó im eft te·gęgnes \hld\ gumono bętsta &
and-wordi ge·sprak: \hld\ „ni mênde ik elkor wiht“, kwað he, &
„te bi·dęrnjenne \hld\ dádjo mínaro, &
wordo efþa werko; \hld\ þit skulun gi witan alle, &
jungaron míne, \hld\ hwand iu far·geven havad &
waldand þesaro wer-oldes, \hld\ þat gi witan mótun &
an iuwom hugi-skęftjun \hld\ himilisk ge·rúni; &
þem ǫ́ðrun skal man be biliðjun \hld\ þat gi·bod godes &
wordun wísjen. \hld\ Nu willju ik iu te wárun hier &
márjen, hwat ik mênde, \hld\ þat gi mína þiu bet &
ovar al þit land-skępi \hld\ lêra far·standan. &
Þat sád, þat ik iu sagda, \hld\ þat is selves word, &
þiu hêlaga lêra \hld\ heven-kuninges, &
hwó man þea márjen skal \hld\ ovar þene middil-gard, &
wído aftar þesaro wer-oldi. \hld\ Weros sind im gi·hugide, &
man mis-líko: \hld\ sum su·likan mód dregid, &
harda hugi-skęfti \hld\ ęndi hrêan sevon, &
þat ina ni ge·werðod, \hld\ þat he it be iuwon wordun due, &
þat he þesa mína lêra forð \hld\ lêstjen willje, &
ak werðad þar só far·lorana \hld\ lêra mína, &
godes ambusni \hld\ ęndi iuwaro gumono word &
an þemu uvilon manne, \hld\ só ik iu êr sagda, &
þat þat korn far·warð, \hld\ þat þar mid kíðun ni mahte &
an þemu stêne uppan \hld\ stędi-haft werðan. &
Só wirðid al far·loran \hld\ ęðilero spráka, &
ârundi godes, \hld\ só hwat só man þemu uvilon manne &
wordun ge·wísid, \hld\ ęndi he an þea wirson hand, &
undar fíundo folk \hld\ fard ge·kiusid, &
an godes un·wiljan \hld\ ęndi an gramono hróm &
ęndi an fiures farm. \hld\ Forð skal he hêtjan &
mid is breost-hugi \hld\ brêda logna. &
Nio gi an þesumu lande þiu lés \hld\ lêra mína &
wordun ni wísjad: \hld\ is þeses werodes só filu, &
erlo aftar þesaro erðun: \hld\ bi·stéd þar ǫ́ðar man, &
þe is imu jung ęndi glau, \hld\ —ęndi havad imu gódan mód—, &
sprákono spáhi \hld\ ęndi wêt iuwaro spello gi·skêð, &
hugid is þan an is herton \hld\ ęndi hôrid þar mid is ôrun tó &
swíðo niud-líko \hld\ ęndi náhor stéd, &
an is breost hlędid \hld\ þat gi·bod godes, &
línod ęndi lêstid: \hld\ is is gi·lôvo só gód, &
talod imu, \hld\ hwó he ǫ́ðrana eft gi·hwervje &
mên-dádigan man, \hld\ þat is mód draga &
hluttra trewa \hld\ te heven-kuninge. &
Þan brêdid an þes breostun \hld\ þat gi·bod godes, &
þie luvigo gi·lôbo, \hld\ só an þemu lande duod &
þat korn mid kíðun, \hld\ þar it gi·kund havad &
ęndi imu þiu wurð bi·hagod \hld\ ęndi wederes gang, &
ręgin ęndi sunne, \hld\ þat it is reht havad. &
Só duod þiu godes lêra \hld\ an þemu gódun manne &
dages ęndi nahtes, \hld\ ęndi gangid imu diuval fer, &
wrêða wihti \hld\ ęndi þe ward godes &
náhor mikilu \hld\ nahtes ęndi dages, &
ant-tat sie ina brengjad, \hld\ þat þar bêðju wirðid &
ia þiu lêra te frumu \hld\ liudjo barnun, &
þe fan is mu̇ðe kumid, \hld\ iak wirðid þe man gode; &
havad só gi·wehslod \hld\ te þesaro wer-old-stundu &
mid is hugi-skęftjun \hld\ himil-ríkjas gi·dêl, &
welono þene mêstan: \hld\ farid imu an gi·wald godes, &
tionuno tómig. \hld\ Trewa sind só góda &
gumono ge·hwi-likumu, \hld\ só nis goldes hord &
ge·lík su·likumu gi·lôvon. \hld\ Wesad iuwaro lêrono forð &
man-kunnje mildje; \hld\ sie sind só mis-líka, &
hęliðos ge·hugda: \hld\ sum havad iro hardan stríd, &
wrêðan willjan, \hld\ wankolna hugi, &
is imu fêknes ful \hld\ ęndi firin-werko. &
Þan bi·ginnid imu þunkjan, \hld\ þan he undar þeru þiodu stád &
ęndi þar gi·hôrid \hld\ ovar hlust mikil &
þea godes lêra, \hld\ þan þunkid imu, þat he sie gerno forð &
lêstjen willje; \hld\ þan bi·ginnid imu þiu lêra godes &
an is hugi hafton, \hld\ ant-tat imu þan eft an hand kumid &
feho te gi·fórja \hld\ ęndi fręmiði skat. &
Þan far·lêdjad ina \hld\ lêða wihti, &
þan he imu far·fáhid \hld\ an feho-giri, &
a·lęskid þene gi·lôbon: \hld\ þan was imu þat luttil fruma, &
þat he it gio an is hertan ge·hugda, \hld\ ef he it halden ne wili. &
Þat is só þe wastom, \hld\ þe an þemu wege be·gan, &
liodan an þemu lande: \hld\ þó far·nam ina eft þero liudjo fard. &
Só duot þea męgin-sundjon \hld\ an þes mannes hugi &
þea godes lêra, \hld\ ef he is ni gômid wel; &
elkor bi·fęlljad sia ina \hld\ ferne te boðme, &
an þene hêtan hęl, \hld\ þar he heven-kuninge &
ni wirðid furður te frumu, \hld\ ak ina fíund skulun &
wítju gi·waragjan. \hld\ Simla gí mid wordun forð &
lêrjad an þesumu lande: \hld\ *ik kan þesaro liudjo hugi, &
só mis-líkan muod-sevon \hld\ manno kunnjes, &
só wanda wísa \hld\ {[...]} &
Sum havit all te þiu is muod gi·látan \hld\ ęndi mêr sorogot, &
hwó hie þat hord bi·halde, \hld\ þan hwó hie hevan-kuninges &
willjon gi·wirkje. \hld\ Be·þiu þar wahsan ni mag &
þat hêlaga gi·bod godes, \hld\ þoh it þar a·hafton mugi, &
wurtjon bi·werpan, \hld\ hwand it þie welo þringit. &
Só samo só þat krúd ęndi þie þorn \hld\ þat korn ant·fáhat, &
węrjat im þena wastom, \hld\ só duot þie welo manne: &
gi·heftid is herta, \hld\ þat hie it gi·huggjan ni muot, &
þie man an is muode, \hld\ þes hie mêst bi·þarf, &
hwó hie þat gi·wirkje, \hld\ þan lang þie hie an þesaro wer-oldi sí, &
þat hie ti êwon-dage \hld\ after muoti &
hębbjan þuru is hêrren þank \hld\ himiles ríki, &
só ęndi-lôsan welon, \hld\ só þat ni mag ênig man &
witan an þesaro wer-oldi. \hld\ Nio hie só wído ni kan &
te gi·þęnkjanne, \hld\ þegạn an is muode, &
þat it bi·haldan mugi \hld\ herta þes mannes, &
þat hie þat ti wáron witi, \hld\ hwat waldand god havit &
guodes gi·gerewid, \hld\ þat all gęgin-werd stéð &
manno só hwi-likon, \hld\ só ina hier minnjot wel &
ęndi selvo te þiu \hld\ is seola gi·haldit, &
þat hie an lioht godes \hld\ líðan muoti.“ &
Só wísda hie þuo mid wordon, \hld\ stuod werod mikil &
umbi þat barn godes, \hld\ ge·hôrdun ina bi biliðon filo &
umbi þesaro wer-oldes gi·wand \hld\ wordon tęlljan; &
kwað þat im ôk ên aðales man \hld\ an is akker sáidi &
hluttar hrên-korni \hld\ handon sínon: &
wolda im þar só wun-sames \hld\ wastmes tiljan, &
fagares fruhtes. \hld\ Þuo géng þar is fíond aftar &
þuru dęrnjan hugi, \hld\ ęndi it all mid durðu ovar-séu, &%TODO: séu unclear.
mid weodo wirsiston. \hld\ Þuo wóhsun sia bêðju, &
ge þat korn ge þat krúd. \hld\ Só kwámun gangan &
is haga-stoldos te hús, \hld\ iro hêrren sagdun, &
þegnos iro þiodne \hld\ þrístjon wordon: &
„hwat, þú sáidos hluttar korn, \hld\ hêrro þie guodo, &
ên-fald an þínon akkar: \hld\ nu ni gi·sihit ênig erlo þan mêr &
weodes wahsan. \hld\ Hwí mohta þat gi·werðan só?“ &
Þuo sprak eft þie aðales man \hld\ þem erlon te·gęgnes, &
þiodan wið is þegnos, \hld\ kwað þat hie it mahti undar·þęnkjan wel, &
þat im þar un·hold man \hld\ aftar sáida, &
fíond fêkni krúd: \hld\ „ne gionsta mi þero fruhtjo wel, &
a·werda mi þena wastom.“ \hld\ Þuo þar eft wini sprákun, &
is jungron te·gęgnes, \hld\ kwáðun þat sia þar weldin gangan tuo, &
kuman mid kraftu \hld\ ęndi lôsjan þat krúd þanan, &
halon it mid iro handon. \hld\ Þuo sprak im eft iro hêrro an·gęgin: &
„ne welleo ik, þat gi it wiodon“, \hld\ kwat-hie, „hwand gi bi·wardon ni mugun, &
gi·gômjan an iuwon gange, \hld\ þoh gí it gerno ni duan, &
ni gí þes kornes te filo, \hld\ kíðo a·węrdjat, &
fęlljat under iuwa fuoti. \hld\ Láte man sia forð hinan &
bêðju wahsan, \hld\ und êr bewod kume &
ęndi an þem felde sind \hld\ fruhti rípja, &
aroa an þem akkare: \hld\ þan faran wí þar alla tuo, &
halon it mid u̇ssan handon \hld\ ęndi þat hrên-kurni lesan &
súvro te·samne \hld\ ęndi it an mínon sęli duojan, &
hębbjan it þar gi·haldan, \hld\ þat it hwęrgin ni mugi &
wiht a·węrdjan, \hld\ ęndi þat wiod niman, &
bindan it te burðinnjon \hld\ ęndi werpan it an bittar fiur, &
láton it þar halojan \hld\ hêta lógna, &
éld un·fuodi.“ \hld\ Þuo stuod erl manag, &
þegnos þagjandi, \hld\ hwat þiod-gomo, &
*mári mahtig Krist \hld\ mênjan weldi, &
bóknjen mid þiu biliðju \hld\ barno ríkjost. &
Bádun þó só gerno \hld\ gódan drohtin &
ant·lúkan þea lêra, \hld\ þat sia móstin þea liudi forð, &
hêlaga hôrjan. \hld\ Þó sprak im eft iro hêrro an·gęgin, &
mári mahtig Krist: \hld\ „þat is“, kwað he, „mannes sunu: &
ik selvo bium, þat þar sáiu, \hld\ ęndi sind þesa sáliga man &
þat hluttra hrên-korni, \hld\ þea mí hér hôrjad wel, &
wirkjad mínan willjan; \hld\ þius wer-old is þe akkar, &
þit brêda bú-land \hld\ barno man-kunnjes; &
Satanas selvo is, \hld\ þat þar sáid aftar &
só lêð-líka lêra: \hld\ havad þesaro liudjo só filu, &
werodes a·wardid, \hld\ þat sie wam frummjad, &
wirkjad aftar is willjon; \hld\ þoh skulun sie hér wahsen forð, &
þea for·griponon gumon, \hld\ só samo só þea gódun man, &
ant-tat Múdspelles męgin \hld\ ovar man fęrid, &
ęndi þesaro wer-oldes. \hld\ Þan is allaro akkaro ge·hwi-lik &
ge·rípod an þesumu ríkja: \hld\ skulun iro regan-gi·skapu &
frummjen firiho barn. \hld\ Þan te·farid erða: &
þat is allaro bewo brêdost; \hld\ þan kumid þe berhto drohtin &
ovana mid is ęngilo kraftu, \hld\ ęndi kumad alle te·samne &
liudi, þe io þit lioht gi·sáun, \hld\ ęndi skulun þan lôn ant·fáhan &
uviles ęndi gódes. \hld\ Þan gangad ęngilos godes, &
hêlage heven-wardos, \hld\ ęndi lesat þea hluttron man &
sundor te·samne, \hld\ ęndi duat sie an sin-skóni, &
hôh himiles lioht, \hld\ ęndi þea ǫ́ðra an hęllja grund, &
werpad þea far·warhton \hld\ an wallandi fiur; &
þar skulun sie gi·bundene \hld\ bittra logna, &
þrá-werk þolon, \hld\ ęndi þea ǫ́ðra þiod-welon &
an heven-ríkja, \hld\ hwítaro sunnon &
liohtjan ge·líko. \hld\ Su-lik lôn nimad &
weros wal-dádjo. \hld\ Só hwe só gi·wit êgi, &
ge·hugdi an is hertan, \hld\ etþa gi·hôrjen mugi, &
erl mid is ôrun, \hld\ só láta imu þit an innan sorga, &
an is mód-sevon, \hld\ hwó he skal an þemu márjon dage &
wið þene ríkjon god \hld\ an reðju standen &%TODO: check reðju
wordo ęndi werko allaro, \hld\ þe he an þesaro wer-oldi gi·duod. &
Þat is ęgis-líkost \hld\ allaro þingo, &
forht-líkost firiho barnun, \hld\ þat sie skulun wið iro fráhon mahljen, &
gumon wið þene gódan drohtin: \hld\ þan weldi gerno ge·hwe wesan, &
allaro manno ge·hwi-lik \hld\ mênes tómig, &
slíðero sakono. \hld\ Aftar þiu skal sorgon êr &
allaro liudjo ge·hwi-lik, \hld\ êr he þit lioht af·geve, &
þe þan êgan wili \hld\ alungan tír, &
hôh heven-ríki \hld\ ęndi huldi godes.“ &
Só gi·fragn ik þat þó selvo \hld\ sunu drohtines, &
allaro barno bętst \hld\ biliðjo sagda, &
hwi-lik þero wári \hld\ an wer-old-ríkja &
undar hęlið-kunnje \hld\ himil-ríkje ge·lík; &
kwað þat oft luttiles hwat \hld\ liohtora wurði, &
só hôho af·huovi, \hld\ „so duot himil-ríki: &
þat is simla méra, \hld\ þan is man ênig &
wánje an þesaro wer-oldi. \hld\ Ôk is imu þat werk ge·lík, &
þat man an sêo innan \hld\ segina wirpit, &
fisk-nęt an flód \hld\ ęndi fáhit bêðju, &
uvile ęndi góde, \hld\ tiuhid up te staðe, &
liðod sie te lande, \hld\ lisit aftar þiu &
þea gódun an greote \hld\ ęndi látid þea ǫ́ðra eft an grund faran, &
an wídan wág. \hld\ Só duod waldand god &
an þemu márjon dage \hld\ męnniskono barn: &
brengid irmin-þiod, \hld\ alle te·samne, &
lisit imu þan þea hluttron \hld\ an heven-ríki, &
látid þea far·griponon \hld\ an grund faren &
hęllje fiures. \hld\ Ni wêt hęliðo man &
þes wítjes wiðar-lága, \hld\ þes þar weros þiggjat, &
an þemu Inferne \hld\ irmin-þioda. &
Þan hald ni mag þera médan man \hld\ gi·makon fïðen, &
ni þes welon ni þes willjon, \hld\ þes þar waldand skerid, &%NOTE: skerid uncertain; skeran or skęrjan?
gildid god selvo \hld\ gumono só hwi-likumu, &
só ina hér gi·haldid, \hld\ þat he an heven-ríki, &
an þat lang-same lioht \hld\ líðan móti.“ &
Só lêrda he þó mid listjun. \hld\ Þan fórun þar þea liudi tó &
ovar al Galilaeo land \hld\ þat godes barn sehan: &
dádun it bi þemu wundre, \hld\ hwanen imu mahti su·lik word kumen, &
só spáh-líko gi·sprokan, \hld\ þat he spel godes &
gio só sǫ́ð-líko \hld\ sęggjan konsti, &
só kraftig-líko gi·kweðen: \hld\ „he is þeses kunnjes hinen“, kwáðun sie, &
„þe man þurh mág-skępi: \hld\ hér is is móder mid u̇s, &
wíf undar þesumu werode. \hld\ Hwat, wí þe hér witun alle, &
só ku̇ð is u̇s is kuni-burd \hld\ ęndi is knósles ge·hwat; &
a·wóhs al undar þesumu werode: \hld\ hwanen skoldi imu su·lik ge·wit kuman, &
méron mahti, \hld\ þan hér ǫ́ðra man êgin?“ &
Só far·munste ina þat manno folk \hld\ ęndi sprákun im gi·méd-lik word, &
far·hogdun ina só hêlagna, \hld\ hôrjen ni weldun &
is gi·bod-skępjes. \hld\ Ni he þar ôk biliðjo filu &
þurh iro un·gi·lôvon \hld\ ógjan ni welde, &
torhtero têkno, \hld\ hwand he wisse iro twífljan hugi, &
iro wrêðan willjan, \hld\ þat ni wárun weros ǫ́ðra &
só grimme under Judeon, \hld\ só wárun umbi Galilaeo land, &
só hardo ge·hugide: \hld\ só þar was þe hêlago Krist, &
gi·boren þat barn godes, \hld\ si ni weldun is gi·bod-skępi þoh &
ant·fáhan ferht-líko, \hld\ ak bi·gan þat folk undar im, &
rinkos rádan, \hld\ hwó sie þene ríkjon Krist &
wêgdin te wundron. \hld\ Hétun þó iro werod kumen, &
ge·sïði te·samne: \hld\ sundja weldun &
an þene godes sunu \hld\ gerno gi·tęlljen &
wrêðes willjon; \hld\ ni was im is wordo niud, &
spáharo spello, \hld\ ak sie bi·gunnun sprekan undar im, &
hwó sie ina só kraftagne \hld\ fan ênumu klive wurpin, &
ovar ênna berges wal: \hld\ weldun þat barn godes &
livu bi·lôsjen. \hld\ Þó he imu mid þem liudjun samad &
frô-líko fór: \hld\ ni was imu foraht hugi, &
—wisse þat imu ni mahtun \hld\ męnniskono barn, &
bi þeru god-kundi \hld\ Judeo liudi &
êr is tídjun wiht \hld\ teonon gi·frummjen, &
lêðaro gi·lêsto—, \hld\ ak he imu mid þem liudjun samad &
stêg uppen þene stên-holm, \hld\ ant-þat sie te þeru stędi kwámun, &
þar sie ine fan þemu walle niðer \hld\ werpen hugdun, &
fęlljen te foldu, \hld\ þat he wurði is ferhes lôs, &
is aldres at ęndje. \hld\ Þó warð þero erlo hugi, &
an þemu berge uppen \hld\ bittra gi·þáhti &
Juðeono te·gangen, \hld\ þat iro ênig ni habde só grimmon sevon &
ni só wrêðen willjon, \hld\ þat sie mahtin þene waldandes sunu, &
Krist ant·kęnnjen; \hld\ he ni was iro ku̇ð ênigumu, &
þat sie ina þó undar·wissin. \hld\ Só mahte he undar ira werode standen &
ęndi an iro gi·mange \hld\ middjumu gangen, &
faren undar iro folke. \hld\ He dede imu þene friðu selvo, &
mund-burd wið þeru męnegi \hld\ ęndi gi·wêt imu þurh middi þanan &
þes fíundo folkes, \hld\ fór imu þó, þar he welde, &
an êne wóstunnje \hld\ waldandes sunu, &
kuningo kraftigost: \hld\ habde þero kustes gi·wald, &
hwar imu an þemu lande \hld\ leovost wári &
te wesanne an þesaru wer-oldi. \hld\ Þan fór imu an weg ǫ́ðran &
Johannes mid is jungarun, \hld\ godes ambaht-man, &
lêrde þea liudi \hld\ lang-samane rád, &
hét þat sie frume fręmidin, \hld\ firina far·létin, &
mên ęndi morð-werk. \hld\ He was þar managumu liof &
gódaro gumono. \hld\ He sóhte imu þó þene Judeono kuning, &
þene hęri-togon at hús, \hld\ þe hêten was &
Erodes aftar is ęldiron, \hld\ ovar-módig man: &
búide imu be þeru brúdi, \hld\ þiu êr sínes bróðer was, &
idis an êhti, \hld\ ant-tat he elljor skók, &
wer-old weslode. \hld\ Þó imu þat wíf gi·nam &
þe kuning te kwenun; \hld\ êr wárun iro kind ôdan, &
barn be is bróðer. \hld\ Þó bi·gan imu þea brúd lahan &
Johannes þe gódo, \hld\ kwað þat it gode wári, &
waldande wiðer-mód, \hld\ þat it ênig wero frumidi, &
þat bróðer brúd \hld\ an is bed námi, &
hębbje sie imu te híwun. \hld\ „Ef þú mi hôrjen wili, &
gi·lôvjen mínun lêrun, \hld\ ni skalt þú sie lęng êgan, &
ak míð ire an þínumu móde: \hld\ ni hava þar su·lika minnja tó, &
ni sundjo þi te swíðo.“ \hld\ Þó warð an sorgun hugi &
þes wíves aftar þem wordun; \hld\ and-réd þat he þene wer-old-kuning &
sprákono ge·spóni \hld\ ęndi spáhun wordun, &
þat he sie far·léti. \hld\ Be·gan siu imu þó lêðes filu &
ráden an rúnon, \hld\ ęndi ine rinkos hét, &
un·sundigane \hld\ erlos fáhan &
ęndi ine an ênumu karkerja \hld\ klústar-bęndjun, &
liðo-kospun bi·lúkan: \hld\ be þem liudjun ne gi·dorstun &
ine ferahu bi·lôsjen, \hld\ hwand sie wárun imu friund alle, &
wissun ine só góden \hld\ ęndi gode werðen, &
habdun ina for wár-sagon, \hld\ só sia wela mahtun. &
Þó wurðun an þemu gę́r-tale \hld\ Judeo kuninges &
tídi kumana, \hld\ só þar gi·tald habdun &
fróde folk-weros, \hld\ þó he gi·fódid was, &
an lioht kuman. \hld\ Só was þero liudjo þau, &
þat þat erlo ge·hwi-lik \hld\ óvjan skolde, &
Judeono mid gômun. \hld\ Þó warð þar an þene gast-sęli &
męgin-kraft mikil \hld\ manno ge·samnod, &
hęri-togono an þat hús, \hld\ þar iro hêrro was &
an is kuning-stóle. \hld\ Kwámun managa &
Judeon an þene gast-sęli; \hld\ warð im þar glad-mód hugi, &
blíði an iro breostun: \hld\ gi·sáhun iro bág-gevon &
wesen an wunnjon. \hld\ Dróg man wín an flęt &
skíri mid skálun, \hld\ skęnkjon hwurvun, &
géngun mid gold-fatun: \hld\ gaman was þar inne &
hlúd an þero hallu, \hld\ hęliðos drunkun. &
was þes an lustun \hld\ landes hirdi, &
hwat he þemu werode mêst \hld\ te wunnjun gi·fręmidi. &
Hét he þó gangen forð \hld\ gêla þiornun, &
is bróder barn, \hld\ þar he an is bęnki sat &
wínu gi·wlęnkid, \hld\ ęndi þó te þemu wíve sprak; &
grótte sie fora þemu gum-skępje \hld\ ęndi gerno bad, &
þat siu þar fora þem gastjun \hld\ gaman af·hóvi &
fagar an flęttje: \hld\ „lát þit folk sehan, &
hwó þú ge·línod havas \hld\ liudjo męnegi &
te blíðsjanne an bęnkjun; \hld\ ef þú mi þera bede tugiðos, &
mín word for þesumu werode, \hld\ þan willju ik it hér te wárun ge·kweðen, &
liahto fora þesun liudjun \hld\ ęndi ôk gi·lêstjen só, &
þat ik þi þan aftar þiu \hld\ êron willju, &
só hwes só þú mi bidis \hld\ for þesun mínun bág-winjun: &
þoh þú mi þesaro hęri-dómo \hld\ halvaro fergos, &
ríkjas mínes, \hld\ þoh gi·dón ik, þat it ênig rinko ni mag &
wordun gi·węndjen, \hld\ ęndi it skal gi·werðen só.“ &
Þó warð þera magað aftar þiu \hld\ mód gi·hworven, &
hugi aftar iro hêrron, \hld\ þat siu an þemu húse innen, &
an þemu gast-sęli \hld\ gamen up a·huof, &
al só þero liudjo \hld\ land-wíse gi·dróg, &
þero þiodo þau. \hld\ Þiu þiorne spilode &
hrór aftar þemu húse: \hld\ hugi was an lustun, &%NOTE: hrór checked.
managaro mód-sevo. \hld\ Þó þiu magað habda &
gi·þionod te þanke \hld\ þiod-kuninge &
ęndi allumu þemu erl-skępje, \hld\ þe þar inne was &
gódaro gumono, \hld\ siu welde þó ira geva êgan, &
þiu magað for þeru męnegi: \hld\ géng þó wið iro módar sprekan &
ęndi frágode sie \hld\ firi-wit-líko, &
hwes siu þene burges ward \hld\ biddjen skoldi. &
Þó wísde siu aftar iro willjon, \hld\ hét þat siu wihtes þan êr &
ni gerodi for þemu gum-skępje, \hld\ bi·útan þat man iru Johannes &
an þeru hallu innan \hld\ hôvid gávi &
a·lôsid af is lík-hamon. \hld\ Þat was allun þem liudjun harm, &
þem mannun an iro móde, \hld\ þó sie þat gi·hôrdun þea magað sprekan; &
só was it ôk þemu kuninge: \hld\ he ni mahte is kwidi liagan, &
is word węndjen: \hld\ hét þó is wę́pan-berand &
gangen fan þemu gast-sęli \hld\ ęndi hét þene godes man &
lívu bi·lôsjen. \hld\ Þó ni was lang te þiu, &
þat man an þea halla \hld\ hôvid bráhte &
þes þiod-gumon, \hld\ ęndi it þar þeru þiornun far·gaf, &
magað for þeru męnegi: \hld\ siu dróg it þeru móder forð. &
Þó was ên-dago \hld\ allaro manno &
þes wísoston, \hld\ þero þe gio an þesa wer-old kwámi, &
þero þe kwene ênig \hld\ kind gi·bári, &
idis fan erle, \hld\ lét man simla þen ênon bi·foran, &
þe þiu þiorne gi·dróg, \hld\ þe gio þegnes ni warð &
wís an iro wer-oldi, \hld\ bi·útan só ine waldand god &
fan heven-wange \hld\ hêlages gêstes &
gi·markode mahtig: \hld\ þe ni habde ênigan gi·makon hwęrgin &
êr nek aftar. \hld\ Erlos hwurvun, &%NOTE: nek checked.
gumon umbi Johannen, \hld\ is jungaron managa, &
sálig ge·sïði, \hld\ ęndi ine an sande bi·gróvun, &
leoves lík-hamon: \hld\ wissun þat he lioht godes, &
diur-líkan drôm \hld\ mid is drohtine samad, &
up-ôdas hêm \hld\ êgan móste, &
sálig sókjan. \hld\ Þó ge·witun im þea ge·sïðos þanen, &
Johannes gjungaron \hld\ gjámer-móde, &
hêlag-feraha: \hld\ was im iro hêrron dôð &
swíðo an sorgun. \hld\ Ge·witun im sókjan þó &
an þeru wóstunni \hld\ waldandes sunu, &
kraftigana Krist \hld\ ęndi imu ku̇ð gi·dedun &
gódes mannes for·gang, \hld\ hwó habde þe Judeono kuning &
manno þene márjostan \hld\ mákjas ęggjun &
hôvdu bi·hauwan: \hld\ he ni welde is ênigen harm spreken, &
sunu drohtines; \hld\ he wisse þat þiu seole was &
hêlag gi·halden \hld\ wiðer hettjandjon, &
an friðe wiðer fíundun. \hld\ Þó só gi·frági warð &
aftar þem land-skępjun \hld\ lêrjandero bętst &
an þeru wóstunni: \hld\ werod samnode, &
fór folkun tó: \hld\ was im firi-wit mikil &
wísaro wordo; \hld\ imu was ôk willjo só samo, &
sunje drohtines, \hld\ þat he su·lik ge·sïðo folk &
an þat lioht godes \hld\ laðojan mósti, &
węnnjen mid willjon. \hld\ Waldand lêrde &
allan langan dag \hld\ liudi managa, &
ęli-þeodige man, \hld\ ant-tat an ávand sêg &
sunne te sedle. \hld\ Þó géngun is ge·sïðos twe-livi, &
gumon te þemu godes barne \hld\ ęndi sagdun iro gódumu hêrron, &
mid hwi-liku arvedju þar þea erlos livdin, \hld\ kwáðun þat sie is êra bi·þorftin, &
weros an þemu wóstjon lande: \hld\ „sie ni mugun sie hér mid wihti ant·hębbjen, &
hęliðos bi hungres ge·þwinge. \hld\ Nu lát þú sie, hêrro þe gódo, &
sïðon, þar sie sęliða fïðen. \hld\ Náh sind hér ge·setana burgi &
managa mid męgin-þiodun: \hld\ þar fïðad sie męti te kôpe, &
weros aftar þem wíkjon.“ \hld\ Þó sprak eft waldand Krist, &
þioda drohtin, \hld\ kwað þat þes êniga þurufti ni wárin, &
„þat sie þurh męti-lôsi \hld\ mína far·látan &
leov-líka lêra. \hld\ Gevad gi þesun liudjun gi·nóg, &
węnnjad sie hér mid willjon.“ \hld\ Þó habde eft is word garu &
Philippus fród gumo, \hld\ kwað þat þar só filu wári &
manno męnigi: \hld\ „þoh wí hér te męti habdin &
garu im te gevanne, \hld\ só wí mahtin far·gelden mêst, &
ef wí hér gi·saldin \hld\ siluver-skatto &
twê hund samad, \hld\ tweho wári is noh þan, &
þat iro ênig þar \hld\ ênes gi·námi: &
só luttik wári þat þesun liudjun.“ \hld\ Þó sprak eft þe landes ward &%NOTE: luttik checked.
ęndi frágode sie \hld\ firi-wit-líko, &
manno drohtin, \hld\ hwat sie þar te męti habdin &
wistes ge·wunnin. \hld\ Þó sprak imu eft mid is wordun an·gęgin &
Andreas fora þem erlun \hld\ ęndi þemu alo-waldon &
selvumu sagde, \hld\ þat sie an iro gi·sïðje þan mêr &
garowes ni habdin, \hld\ „bi·útan girstin brôd &
fïvi an u̇saru fęrdi \hld\ ęndi fiskos twêne. &
Hwat mag þat þoh þesaru męnigi?“ \hld\ Þó sprak imu eft mahtig Krist, &
þe gódo godes sunu, \hld\ ęndi hét þat gumono folk &
skęrjen ęndi skêðen \hld\ ęndi hét þea skola sęttjen, &
erlos aftar þeru erðu, \hld\ irmin-þioda &
an grase gruonimu, \hld\ ęndi þó te is jungarun sprak, &
allaro barno bętst, \hld\ hét imu þiu brôd halon &
ęndi þea fiskos forð. \hld\ Þat folk stillo bêd, &
sat ge·sïði mikil; \hld\ undar þiu he þurh is selves kraft, &
manno drohtin, \hld\ þene męti wíhide, &
hêlag heven-kuning, \hld\ ęndi mid is handun brak, &
gaf it is jungarun forð, \hld\ ęndi it sie undar þemu gum-skępje hét &
dragan ęndi dêljen. \hld\ Sie lêstun iro drohtines word, &
is geva gerno drógun \hld\ gumono gi·hwemu, &
hêlaga helpa. \hld\ It undar iro handun wóhs, &
męti manno gi·hwemu: \hld\ þeru męgin-þiodu warð &
líf an lustun, \hld\ þea liudi wurðun alle, &
sade sálig folk, \hld\ só hwat só þar gi·samnod was &
fan allun wídun wegun. \hld\ Þó hét waldand Krist &
gangen is jungaron \hld\ ęndi hét sie gômjen wel, &
þat þiu léva þar \hld\ far·loren ni wurði; &
hét sie þó samnon, \hld\ þó þar sade wárun &
man-kunnjes manag. \hld\ Þar móses warð, &
brôdes te lévu, \hld\ þat man birilos gi·las &
twe-livi fulle: \hld\ þat was têkạn mikil, &
grôt kraft godes, \hld\ hwand þar was gumono gi·tald &
áno wíf ęndi kind, \hld\ werodes at-samme &
fïf þúsundig. \hld\ Þat folk al far·stód, &
þea man an iro móde, \hld\ þat sie þar mahtigna &
hêrron habdun. \hld\ Þó sie heven-kuning, &
þea liudi lovodun, \hld\ kwáðun þat gio ni wurði an þit lioht kuman &
wísaro wár-sago, \hld\ efþa þat he gi·wald mid gode &
an þesaru middil-gard \hld\ méron habdi, &
ên-faldaran hugi. \hld\ Alle gi·sprákun, &
þat he wári wirðig \hld\ welono ge·hwi-likes, &
þat he erð-ríki \hld\ êgan mósti, &
wídene wer-old-stól, \hld\ „nu he su·lik ge·wit havad, &
só grôte kraft mid gode.“ \hld\ Þea gumon alle gi·warð, &
þat sie ine gi·hóvin \hld\ te hêrosten, &
gi·kurin ine te kuninge: \hld\ þat Kriste ni was &
wihtes wirðig, \hld\ hwand he þit wer-old-ríki, &
erðe ęndi up-himil \hld\ þurh is ênes kraft &
selvo gi·warhte \hld\ ęndi sïðor gi·held, &
land ęndi liud-skępi, \hld\ —þoh þes ênigan gi·lôvon ni dedin &
wrêðe wiðer-sakon— \hld\ þat al an is gi·walde stád, &
kuning-ríkjo kraft \hld\ ęndi kêsur-dómes, &
męgin-þiodo mahal. \hld\ Be·þiu ni welde he þurh þero manno spráka &
hębbjan ênigan hêr-dóm, \hld\ hêlag drohtin, &
wer-old-kuninges namon; \hld\ ni he þó mid wordun stríd &
ni af·hóf wið þat folk furður, \hld\ ak fór imu þó, þar he welde, &
an ên ge·birgi uppan: \hld\ flóh þat barn godes &
gêlaro gelp-kwidi \hld\ ęndi is jungaron hét &
ovar ênne sêo sïðon \hld\ ęndi im selvo gi·bôd, &
hwar sie im eft te·gęgnes \hld\ gangen skoldin. &
Þó te·\alst{l}ét þat \alst{l}iud-werod \hld\ aftar þemu \alst{l}ande allumu, &
te·\alst{f}ór \alst{f}olk mikil, \hld\ sïðor iro \alst{f}ráho gi·wêt &
an þat ge·\alst{b}irgi uppan, \hld\ \alst{b}arno ríkjost, &
\alst{w}aldand an is \alst{w}illjon. \hld\ Þó te þes \alst{w}atares staðe &
\alst{s}amnodun þea ge·\alst{s}ïðos Kristes, \hld\ þe he imu habde \alst{s}elvo gi·korane, &
sie \alst{t}welivi þurh iro \alst{t}rewa góda: \hld\ ni was im \alst{t}weho nigijan, &
nevu sie an þat \alst{g}odes þionost \hld\ \alst{g}erno weldin &
ovar þene \alst{s}êo \alst{s}ïðon. \hld\ Þó létun sie \alst{s}wíðjan strôm, &
\alst{h}ôh \alst{h}urnid-skip \hld\ \alst{h}luttron u̇ðjon, &
\alst{sk}êðan \alst{sk}ír water. \hld\ \alst{Sk}rêd lioht dages, &
\alst{s}unne warð an \alst{s}edle; \hld\ þe \alst{s}êo-líðandjan &
\alst{n}aht \alst{n}evulo bi·warp; \hld\ \alst{n}áðidun erlos &
\alst{f}orð-wardes an \alst{f}lód; \hld\ warð þiu \alst{f}iorðe tíd &
þera \alst{n}ahtes kuman \hld\ —\alst{n}ęrjendo Krist &
\alst{w}arode þea \alst{w}ág-líðand—: \hld\ þó warð \alst{w}ind mikil, &
\alst{h}ôh weder af·\alst{h}aven: \hld\ \alst{h}lamodun u̇ðjon, &
\alst{st}rôm an \alst{st}amne; \hld\ \alst{st}rídjun fęridun &
þea \alst{w}eros wiðer \alst{w}inde, \hld\ was im \alst{w}rêð hugi, &
\alst{s}evo \alst{s}orgono ful: \hld\ \alst{s}elvon ni wándun &
\alst{l}agu-\alst{l}íðandja \hld\ an \alst{l}and kumen &
þurh þes \alst{w}ederes ge·\alst{w}in. \hld\ Þó gi·sáhun sie \alst{w}aldand Krist &
an þemu \alst{s}êe uppan \hld\ \alst{s}elvun gangan, &
\alst{f}aran an \alst{f}áðjon: \hld\ ni mahte an þene \alst{f}lód innan, &
an þene \alst{s}êo \alst{s}inkan, \hld\ hwand ine is \alst{s}elves kraft &
\alst{h}êlag ant·\alst{h}abde. \hld\ \alst{H}ugi warð an forhtun, &
þero \alst{m}anno \alst{m}ód-sevo: \hld\ and-rédun þat it im \alst{m}ahtig fíund &
te gi·\alst{d}roge \alst{d}ádi. \hld\ Þó sprak im iro \alst{d}rohtin tó, &
\alst{h}êlag \alst{h}even-kuning, \hld\ ęndi sagde im þat he iro \alst{h}êrro was &
\alst{m}ári ęndi \alst{m}ahtig: \hld\ „nu gí \alst{m}ódes skulun &
\alst{f}astes \alst{f}áhen; \hld\ ne sí iu \alst{f}orht hugi, &
gi·\alst{b}árjad gi \alst{b}ald-líko: \hld\ ik \alst{b}ium þat barn godes, &
is \alst{s}elves \alst{s}unu, \hld\ þe iu wið þesumu \alst{s}êe skal, &
\alst{m}undon wið þesan \alst{m}ęri-strôm.“ \hld\ Þó sprak imu ên þero \alst{m}anno an·gęgin &
ovar \alst{b}ord skipes, \hld\ \alst{b}ar-wirðig gumo, &
\alst{P}etrus þe gódo \hld\ —ni welde \alst{p}íne þolon, &
\alst{w}atares \alst{w}íti—: \hld\ „ef þú it \alst{w}aldand sís“, kwað he, &
„\alst{h}êrro þe gódo, \hld\ só mi an mínumu \alst{h}ugi þunkit, &
hét mi þan þarod \alst{g}angan te þi \hld\ ovar þesen \alst{g}evenes strôm, &
\alst{d}rokno ovar \alst{d}iap water, \hld\ ef þú mín \alst{d}rohtin sís, &
\alst{m}anagoro \alst{m}und-boro.“ \hld\ Þó hét ine \alst{m}ahtig Krist &
\alst{g}angan imu te·\alst{g}ęgnes. \hld\ He warð \alst{g}aru sáno, &
\alst{st}óp af þemu \alst{st}amne \hld\ ęndi \alst{st}rídjun géng &
\alst{f}orð te is \alst{f}rôjan. \hld\ Þiu \alst{f}lód ant·habde &
þene \alst{m}an þurh \alst{m}aht godes, \hld\ antat he imu an is \alst{m}óde bi·gan &
and-ráden \alst{d}iap water, \hld\ þó he \alst{d}ríven gi·sah &
þene \alst{w}ég mid \alst{w}indu: \hld\ \alst{w}undun ina u̇ðjon, &
\alst{h}ôh strôm umbi·\alst{h}ring. \hld\ Reht só he þó an is \alst{h}ugi twehode, &
só \alst{w}êk imu þat \alst{w}ater under, \hld\ ęndi he an þene \alst{w}ág innan, &
\alst{s}ank an þene \alst{s}êo-strôm, \hld\ ęndi he hriop \alst{s}án aftar þiu &
\alst{g}áhon te þemu \alst{g}odes sunje \hld\ ęndi \alst{g}erno bad, &
þat he ine þó ge·\alst{n}ęridi, \hld\ þó he an \alst{n}ôdjun was, &
\alst{þ}egạn an ge·\alst{þ}winge. \hld\ \alst{Þ}iodo drohtin &
ant·\alst{f}eng ine mid is \alst{f}aðmun \hld\ ęndi \alst{f}rágode sána, &
te hwí he þó ge·\alst{t}wehodi: \hld\ „hwat, þú mahtes ge·\alst{t}rúojan wel, &
\alst{w}iten þat te \alst{w}árun, \hld\ þat þi \alst{w}atares kraft &
an þemu \alst{s}êe innen \hld\ þínes \alst{s}ïðes ni mahte, &
\alst{l}agu-strôm gi·\alst{l}ęttjen, \hld\ só lango só þú habdes ge·\alst{l}ôvon te mi &
an þínumu \alst{h}ugi \alst{h}ardo. \hld\ Nu willju ik þi an \alst{h}elpun wesen, &
\alst{n}ęrjen þi an þesaru \alst{n}ôdi“. \hld\ Þó \alst{n}am ine alo-mahtig, &
\alst{h}êlag bi \alst{h}andun: \hld\ þó warð imu eft \alst{h}lutter water &
\alst{f}ast under \alst{f}ótun, \hld\ ęndi sie an \alst{f}áði samad &
\alst{b}êðja géngun, \hld\ antat sie ovar \alst{b}ord skipes &
\alst{st}ópun fan þemu strôme, \hld\ ęndi an þemu \alst{st}amne ge·sat &
allaro \alst{b}arno bętst. \hld\ Þó warð \alst{b}rêd water, &
\alst{st}rômos ge·stillid, \hld\ ęndi sie te \alst{st}aðe kwámun, &
\alst{l}agu-\alst{l}íðandja \hld\ an \alst{l}and samen &
þurh þes \alst{w}ateres ge·\alst{w}in, \hld\ sagdun þo \alst{w}aldande þank, &
\alst{d}iurden iro \alst{d}rohtin \hld\ \alst{d}ádjun ęndi wordun, &
\alst{f}ellun imu te \alst{f}ótun \hld\ ęndi \alst{f}ilu sprákun &
\alst{w}ísaro \alst{w}ordo, \hld\ kwáðun þat sie \alst{w}issin garo, &
þat he wári \alst{s}elvo \hld\ \alst{s}unu drohtines &
wár an þesaru wer-oldi \hld\ ęndi ge·wald habdi &
ovar middil-gard, \hld\ ęndi þat he mahti allaro manno gi·hwes &
ferahe gi·formon, \hld\ al só he im an þemu flóde dede &
wið þes watares ge·win. \hld\ Þó gi·wêt imu waldand Krist &
sïðon fan þemu sêe, \hld\ sunu drohtines, &
ênag barn godes. \hld\ Eli-þioda kwam imu, &
gumon te·gęgnes: \hld\ wárun is gódun werk &
ferran ge·frági, \hld\ þat he só filu sagde &
wároro wordo: \hld\ imu was willjo mikil, &
þat he su·lik folk-skępi \hld\ frummjen mósti, &
þat sie simla gerno \hld\ gode þionodin, &
wárin ge·hôrige \hld\ heven-kuninge &
man-kunnjes manag. \hld\ Þó gi·wêt he imu over þea marka Judeono, &
sóhte imu Sidono burg, \hld\ habde ge·sïðos mid imu, &
góde jungaron. \hld\ Þar imu te·gęgnes kwam &
ên idis fan áðrom þiodun; \hld\ siu was iru aðali-ge·burdjo, &
kunnjes fan Kananeo lande; \hld\ siu bad þene kraftagan drohtin, &
hêlagna, þat he iru helpe ge·rédi, \hld\ kwað þat iru wári harm gi·standen, &
soroga at iru selvaru dohter, \hld\ kwað þat siu wári mid suhtjun bi·fangen: &
„be·drogan habbjad sie dęrnja wihti. \hld\ Nu is iro dôd at hęndi, &
þea wrêðon habbjad sie ge·wittju be·numane. \hld\ Nu biddju ik þi, waldand frô min, &
selvo sunu Dawides, \hld\ þat sie af su·likum suhtjun a·tómjes, &
þat þú sie só arma \hld\ ê-gróht-fullo &
wam-skaðon bi·weri.“ \hld\ Ni gaf iru þó noh waldand Krist &
ênig and-wordi; \hld\ siu imu aftar géng, &
folgode fruokno, \hld\ antat siu te is fótun kwam, &
grótte ina greatandi. \hld\ Gjungaron Kristes &
bádun iro hêrron, \hld\ þat he an is hugja mildi &
wurði þemu wíve. \hld\ Þó habde eft is word garu &
sunu drohtines \hld\ ęndi te is ge·sïðun sprak: &
„êrist skal ik Israheles \hld\ avoron werðen, &
folk-skępi te frumu, \hld\ þat sie ferhtan hugi &
hębbjan te iro hêrron: \hld\ im is helpono þarf, &
þea liudi sind far·lorane, \hld\ far·láten habbjad &
waldandes word, \hld\ þat werod is ge·twíflid, &
drívad im dęrnjan hugi, \hld\ ne willjad iro drohtine hôrjen &
Israhelo erl-skępi, \hld\ un·gi·lôviga sind &
hęliðos iro hêrron: \hld\ þoh skal þanen helpe kumen &
allun ęli-þiodun.“ \hld\ Agalêto bad &
þat wíf mid iro wordun, \hld\ þat iru waldand Krist &
an is mód-sevon \hld\ mildi wurði, &
þat siu iro barnes forð \hld\ brúkan mósti, &
hębbjan sie hêle. \hld\ Þó sprak iru hêrro an·gęgin, &
mári ęndi mahtig: \hld\ „nis þat“, kwað he, „mannes reht, &
gumono nig·ênum \hld\ gód te gi·frummjenne &
þat he is barnun \hld\ brôdes af·tíhe, &
węrnje im ovar willjon, \hld\ láte sie wíti þoljan, &
hungar hęti-grimmen, \hld\ ęndi fódje is hundos mid þiu.“ &
„wár is þat, waldand“, \hld\ kwað siu, „þat þú mid þínun wordun sprikis, &
sǫ́ð-líko sagis: \hld\ hwat, þoh oft an sęli innen &
undar iro hêrron diske \hld\ hwelpos hwervad &
brosmono fulle \hld\ þero fan þemu biode niðer &
ant·fallat iro frôjan.“ \hld\ Þó gi·hôrde þat friðu-barn godes &
willjan þes wíves \hld\ ęndi sprak iru mid is wordun tó: &
„wela þat þú wíf haves \hld\ willjan góden! &
Mikil is þín gi·lôvo \hld\ an þea maht godes, &
an þene liudjo drohtin. \hld\ Al wirðid gi·lêstid só &
umbi þínes barnes líf, \hld\ só þú bádi te mi.“ &
Þó warð siu sán gi·hêlid, \hld\ só it þe hêlago ge·sprak &
wordun wár-fastun: \hld\ þat wíf fagonode, &
þes siu iro barnes forð \hld\ brúkan móste; &
habde iru gi·holpen \hld\ hêljando Krist, &
habde sie far·fangane \hld\ fíundo kraftu, &
wam-skaðun bi·węrid. \hld\ Þó gi·wêt imu waldand forð, &
barno þat bętste, \hld\ sóhte imu burg ǫ́ðre, &
þiu só þikko was \hld\ mid þeru þiodu Judeono, &
mid \alst{s}u̇ðar-liudjun gi·\alst{s}eten. \hld\ Þar gi·fragn ik þat he is ge·\alst{s}ïðos grótte, &
þe \alst{j}ungaron þe he imu habde be is \alst{g}óde gi·korane, \hld\ þat sie mid imu \alst{g}erno ge·wunodun, &
\alst{w}eros þurh is \alst{w}íson spráka: \hld\ „alle skal ik iu“, kwað he, „mid \alst{w}ordun frágon, &
\alst{j}ungaron míne: \hld\ hwat kweðat þese \alst{J}udeo liudi, &
\alst{m}ári \alst{m}ęgin-þioda, \hld\ hwat ik \alst{m}anno sí?“ &
Imu and-wordidun \alst{f}rô-líko \hld\ is \alst{f}riund an·gęgin, &
\alst{j}ungaron síne: \hld\ „nis þit \alst{J}udeono folk, &
\alst{e}rlos \alst{ê}n-wordje: \hld\ sum sagad þat þú \alst{E}lias sís, &
\alst{w}ís \alst{w}ár-sago, \hld\ þe hér giu \alst{w}as lango, &
\alst{g}ód undar þesumu \alst{g}um-skępje, \hld\ sum sagad þat þú \alst{J}ohannes sís, &
\alst{d}iur-lík \alst{d}rohtines bodo, \hld\ þe hér \alst{d}ôpte iu &
\alst{w}erod an \alst{w}atere; \hld\ alle sie mid \alst{w}ordun sprekad, &
þat þú \alst{ê}n-hwi-lik sís \hld\ \alst{ę}ðilero manno, &
þero \alst{w}ár-sagono, \hld\ þe hér mid \alst{w}ordun giu &
\alst{l}êrdun þese \alst{l}iudi, \hld\ ęndi þat þú sís eft an þit \alst{l}ioht kumen &
te \alst{w}ísjanne þesumu \alst{w}erode.“ \hld\ Þó sprak eft \alst{w}aldand Krist: &
„hwe kweðad \alst{g}i, þat ik sí“, \hld\ kwað he, „\alst{j}ungaron míne, &
\alst{l}iovon \alst{l}iud-weros?“ \hld\ Þó te \alst{l}at ni warð &
\alst{S}ímon Petrus: \hld\ sprak \alst{s}án an·gęgin &
\alst{ê}no for im \alst{a}llun \hld\ —habde imu \alst{ę}lljen gód, &
\alst{þ}rístja gi·\alst{þ}áhti, \hld\ was is \alst{þ}eodone hold—: &
„þú bist þe \alst{w}áro \hld\ \alst{w}aldandes sunu, &
\alst{l}ibbjendes godes, \hld\ þe þit \alst{l}ioht gi·skóp, &
\alst{K}rist \alst{k}uning êwig: \hld\ só willjad wí \alst{k}weðen alle, &
\alst{j}ungaron þíne, \hld\ þat þú sís \alst{g}od selvo, &
\alst{h}êljandero bętst.“ \hld\ Þó sprak imu eft is \alst{h}êrro an·gęgin: &
„\alst{s}álig bist þú \alst{S}ímon“, kwað he, „\alst{s}unu Jonases; \hld\ ni mahtes þú þat \alst{s}elvo ge·huggjan, &
gi·\alst{m}arkon an þínun \alst{m}ód-gi·þáhtjun, \hld\ ne it ni mahte þi \alst{m}annes tunge &
\alst{w}ordun ge·\alst{w}ísjen, \hld\ ak dede it þi \alst{w}aldand selvo, &
\alst{f}ader allaro \alst{f}iriho barno, \hld\ þat þú só \alst{f}orð gi·spráki, &
só \alst{d}iapo bi \alst{d}rohtin þínen. \hld\ \alst{D}iur-líko skalt þú þes lôn ant·fáhen, &
\alst{h}luttro havas þú an þínan \alst{h}êrron gi·lôvon, \hld\ \alst{h}ugi-skęfti sind þíne stêne ge·líka, &
só \alst{f}ast bist þú só \alst{f}elis þe hardo; \hld\ hêten skulun þi \alst{f}iriho barn &
\alst{s}ankte Péter: \hld\ ovar þemu stêne skal man mínen \alst{s}ęli wirkjan, &
\alst{h}êlag \alst{h}ús godes; \hld\ þar skal is \alst{h}íwiski tó &
\alst{s}álig \alst{s}amnon: \hld\ ni mugun wið þem þínun \alst{s}wíðjun krafte &
an·þebbjen \alst{h}ęllje portun. \hld\ Ik far·givu þi \alst{h}imil-ríkjas slutilas, &%TODO: Etymology an·þebbjen
þat þú móst \alst{a}ftar mi \hld\ \alst{a}llun gi·waldan &
\alst{k}ristinum folke; \hld\ \alst{k}umad alle te þi &
\alst{g}umono \alst{g}êstos; \hld\ þú have \alst{g}rôte gi·wald, &
hwene þú hér an \alst{e}rðu \hld\ \alst{ę}ldi-barno &
ge·\alst{b}inden willjes: \hld\ þemu is \alst{b}êðju gi·duan, &
\alst{h}imil-ríki bi·loken, \hld\ ęndi \alst{h}ęllje sind imu opana, &
\alst{b}rinnandi fiur; \hld\ só hwene só þú eft ant·\alst{b}inden wili, &
an-þeftjen is \alst{h}ęndi, \hld\ þemu is \alst{h}imil-ríki, &
ant·\alst{l}oken \alst{l}iohto mêst \hld\ ęndi \alst{l}íf êwig, &
\alst{g}róni \alst{g}odes wang. \hld\ Mid su·likaru ik þi \alst{g}evu willju &
\alst{l}ônon þínen gi·\alst{l}ôvon. \hld\ Ni willju ik, þat gí þesun \alst{l}iudjun noh, &
\alst{m}árjen þesaru \alst{m}ęnigi, \hld\ þat ik bium \alst{m}ahtig Krist, &
\alst{g}odes êgan barn. \hld\ Mi skulun \alst{J}udeon noh, &
\alst{u}n·skuldigna \hld\ \alst{e}rlos binden, &
\alst{w}êgjan mi te \alst{w}undrun \hld\ —dót mi \alst{w}ítjes filo— &
innan \alst{Hj}erusalem \hld\ \alst{g}êres ordun, &
áhtjen mínes aldres \hld\ ęggjun skarpun, &
bi·lôsjen mi lívu. \hld\ Ik an þesumu liohte skal &
þurh u̇ses drohtines kraft \hld\ fan dôde a·standen &
an þriddjumu dage“. \hld\ Þó warð þegno bętst &
swíðo an sorgun, \hld\ Símon Petrus, &
warð imu hugi hriwig, \hld\ ęndi te is hêrron sprak &
rink an rúnun: \hld\ „ni skal þat ríki god“, kwað he, &
„waldand willjen, \hld\ þat þú eo su·lik wíti mikil &
gi·þolos undar þesaru þiod: \hld\ nis þes þarf nigijan, &
hêlag drohtin.“ \hld\ Þó sprak imu eft is hêrro an·gęgin, &
mári mahtig Krist \hld\ —was imu an is móde hold—: &
„hwat, þú nu wiðer-ward bist“, \hld\ kwað he, „willjon mínes, &
þegno bętsto! \hld\ Hwat, þú þesaro þiodo kanst &
męnniskan sidu: \hld\ þú ni wêst þe maht godes, &
þe ik gi·frummjen skal. \hld\ Ik mag þi filu sęggjan &
wárun wordun, \hld\ þar hér undar þesumu werode standad &
ge·sïðos míne, \hld\ þea ni mótun swelten êr, &
hwerven an hinen-fard \hld\ êr sie himiles lioht, &
godes ríki sehat.“ \hld\ Kôs imu jungarono þó &
sán aftar þiu \hld\ Símon Petrus, &
Jakob ęndi Johannes, \hld\ ea gumon twêne, &
bêðja þea gi·bróðer, \hld\ ęndi imu þó uppen þene berg gi·wêt &
sunder mid þem ge·sïðun, \hld\ sálig barn godes, &
mid þem þegnun þrim, \hld\ þiodo drohtin, &
waldand þesaro wer-oldes: \hld\ welde im þar wundres filu, &
têkno tôgjan, \hld\ þat sie gi·trúodin þiu bet, &
þat he selvo was \hld\ sunu drohtines, &
hêlag heven-kuning. \hld\ Þó sie an hôhan wall &
stigun stên ęndi berg, \hld\ antat sie te þeru stędi kwámun, &
weros wiðer wolkan, \hld\ þar waldand Krist, &
kuningo kraftigost \hld\ gi·koren habde, &
þat he is god-kundi \hld\ jungarun sínun &
þurh is ênes kraft \hld\ ógjan welde, &
berht-lík biliði. \hld\ Þó imu þar te bedu gi·hnêg, &
þó warð imu þar uppe \hld\ ǫ́ðar-líkora &
wliti ęndi gi·wádi: \hld\ wurðun imu is wangun liohte, &
blíkandi só þiu berhte sunne: \hld\ só skên þat barn godes, &
liuhte is lík-hamo: \hld\ liomon stódun &
wánamo fan þemu waldandes barne; \hld\ warð is ge·wádi só hwít &
só snêw te sehanne. \hld\ Þó warð þar seld-lík þing &
gi·ôgid aftar þiu: \hld\ Elias ęndi Moyses &
kwámun þar te Kriste \hld\ wið só kraftagne &
wordun wehsljan. \hld\ Þar warð só wun-sam spráka, &
só gód word undar gumun, \hld\ þar þe godes sunu &
wið þea márjan man \hld\ mahljen welde, &
só blíði warð uppan þemu berge: \hld\ skên þat berhte lioht, &
was þar gard gód-lík \hld\ ęndi gróni wang, &
Paradise ge·lík. \hld\ Petrus þó gi·mahalde, &
hęlið hard-módig \hld\ ęndi te is hêrron sprak, &
grótte þene godes sunu: \hld\ „gód is it hér te wesanne, &
ef þú it gi·kiosan wili, \hld\ Krist alo-waldo, &
þat man þi hér an þesaru hôhe \hld\ ên hús ge·wirkja, &
már-líko ge·mako \hld\ ęndi Moysese ǫ́ðer &
ęndi Eliase þriddja: \hld\ þit is ôdas hêm, &
welono wun-samost.“ \hld\ Reht só he þó þat word ge·sprak, &
só ti-lét þiu luft an twê: \hld\ lioht wolkan skên, &
glítandi glímo, \hld\ ęndi þea gódun man &
wliti-skóni be·warp. \hld\ Þó fan þemu wolkne kwam &
hêlag stemne godes, \hld\ ęndi þem hęliðun þar &
selvo sagde, \hld\ þat þat is sunu wári, &
libbjendero liovost: \hld\ „an þemu mi líkod wel &
an mínun hugi-skęftjun. \hld\ Þemu gí hôrjen skulun, &
ful-gangad imu gerno.“ \hld\ Þó ni mahtun þea jungaron Kristes &
þes wolknes wliti \hld\ ęndi word godes, &
þea is mikilon maht \hld\ þea man ant·standen, &
ak sie bi·fellun þó forð-wardes: \hld\ ferhes ni wándun, &
lęngiron líves. \hld\ Þó géng im tó þe landes ward, &
be·hrên sie mid is handun \hld\ hêljandero bętst, &
hét þat sie im ni and-rédin: \hld\ „ni skal iu hér derjen eo·wiht, &
þes gi hér seld-líkes \hld\ gi·sehen habbjad, &
mérjaro þingo.“ \hld\ Þó eft þem mannun warð &
hugi at iro herton \hld\ ęndi gi·hêlid mód, &
gi·bade an iro breostun: \hld\ gi·sáhun þat barn godes &
ênna standen, \hld\ was þat oðer þó, &
be·hliden himiles lioht. \hld\ Þó gi·wêt imu þe hêlago Krist &
fan þemu berge niðer; \hld\ gi·bôd aftar þiu &
jungarun sínun, \hld\ þat sie ovar Judeono folk &
ni sagdin þea gi·sioni: \hld\ „er þan ik selvo hér &
swíðo diur-líko \hld\ fan dôðe a·stande, &
a·ríse fan þeru restu: \hld\ sïðor mugun gi it rekkjen forð, &
márjen ovar middil-gard \hld\ managun þiodun &
wído aftar þesaru wer-oldi.“ \hld\ Þó gi·wêt imu waldand Krist &
eft an Galileo land, \hld\ sóhte is gadulingos, &
mahtig is mágo hêm, \hld\ sagde þar manages hwat &
berhtero biliðjo, \hld\ ęndi þat barn godes &
þem is sáligun ge·sïðun \hld\ sorg-spell ni for·hal, &
ak he im open-líko \hld\ allun sagde, &
þem is gódun jungarun, \hld\ hwó ine skolde þat Judeono folk &
wêgjan te wundrun. \hld\ Þes wurðun þar wíse man &
swíðo an sorgun, \hld\ warð im sêr hugi, &
hriwig umbi iro herte: \hld\ gi·hôrdun iro hêrron þó, &
waldandes sunu \hld\ wordun tęlljen, &
hwat he undar þeru þiodu \hld\ þolojan skolde, &
willjendi undar þemu werode. \hld\ Þó gi·wêt imu waldand Krist, &
gumo fan Galilea, \hld\ sóhte imu Judeono burg, &
kwámun im te Kafarnaum. \hld\ Þar fundun sie ênan kuninges þegạn &
wlankan undar þemu werode: \hld\ kwað þat he wári gi·weldig bodo &
aðal-kêsures; \hld\ he grótte aftar þiu &
Símon Petrusen, \hld\ kwað þat he wári gi·sęndid þarod, &
þat he þar gi·manodi \hld\ manno ge·hwi-liken &
þero hôvid-skatto, \hld\ þe sie te þemu hove skoldin &
tinsi gelden: \hld\ „nis þes tweho ênig &
gumono ni-giênumu, \hld\ ne sie ina far·gelden sán &
mêðmo kustjon, \hld\ bi·úten iuwe mêster êno &
havad it far·láten. \hld\ Ni skal þat líkon wel &
mínumu hêrron, \hld\ só man it imu at is hove ku̇ðid, &
aðal-kêsure.“ \hld\ Þó géng aftar þiu &
Símon Petrus, \hld\ welde it sęggjan þó &
hêrron sínumu: \hld\ he was is an is hugi iu þan, &%TODO: Check sínumu.
gi·waro waldand Krist: \hld\ —imu ni mahte word ênig &
bi·holen werðen, \hld\ he wisse hugi-skęfti &
manno ge·hwi-likes—: \hld\ hét þó þene is márjan þegạn, &
Símon Petrus \hld\ an þene sêo innen &
angul werpen: \hld\ „su·liken só þú þar êrist mugis &
fisk gi·fáhen“, \hld\ kwað he, „só teoh þú þene fan þemu flóde te þi, &
ant·klemmi imu þea kinni: \hld\ þar maht þú undar þem kaflon nimen &
guldine skattos, \hld\ þat þú far·gelden maht &
þemu manne te gi·módja \hld\ mínen ęndi þínen &
tinseo só hwi-likan, \hld\ só he u̇s tó sókid.“ &
He ni þorfte imu þó aftar þiu \hld\ ǫ́ðaru wordu &
furður gi·bioden: \hld\ géng fiskari gód, &
Símon Petrus, \hld\ warp an þene sêo innen &
angul an u̇ðjon \hld\ ęndi up gi·tóh &
fisk an flóde \hld\ mid is folmun twêm, &
te·klóf imu þea kinni \hld\ ęndi undar þem kaflun nam &
guldine skattos: \hld\ dede al, só imu þe godes sunu &
wordun ge·wísde. \hld\ Þar was þó waldandes &
męgin-kraft gi·márid, \hld\ hwó skal allaro manno ge·hwi-lik &
swíðo willjendi \hld\ is wer-old-hêrron &
skuldi ęndi skattos, \hld\ þea imu gi·skęride sind, &
gerno gelden: \hld\ ni skal ine far·gúmon eo·wiht, &
ni far·muni ine an is móde, \hld\ ak wese imu mildi an is hugi, &
þiono imu þio-líko: \hld\ an þiu mag he þiodgodes &
willjan ge·wirkjan \hld\ ęndi ôk is wer-old-hêrron &
huldi habbjen. \hld\ Só lêrde þe hêlago Krist &
þea is gódon jungaron: \hld\ „ef ênig gumono wið iu“, kwað he, &
„sundja ge·wirkja, \hld\ þan nim þú ina sundar te þi, &
þene rink an rúna \hld\ ęndi imu is rád saga, &
wísi imu mid wordun. \hld\ Ef imu þan þes werð ne sí, &
þat he þi gi·hôrje, \hld\ hala þi þar ǫ́ðara tó &
gódaro gumono, \hld\ ęndi lah imu is grimmun werk, &
sak ina sǫ́ð-wordun. \hld\ Ef imu þan is sundja aftar þiu, &
lôs-werk ni lêðon, \hld\ gi·duo it ǫ́ðrun liudjun ku̇ð, &
mári it þan for męnegi \hld\ ęndi lát manno filu &
witen is far·wurhti: \hld\ óðo be·ginnad imu þan is werk tregan, &
an is hugi hrewen, \hld\ þan he it gi·hôrid hęliðo filu, &
ahton ęldi-barn \hld\ ęndi imu is uvilon dád &
węrjad mid wordun. \hld\ Ef he þan ôk węndjen ne wili, &
ak far·módat su·lika męnegi, \hld\ þan lát þú þene man faren, &
hava ina þan far hêðinen \hld\ ęndi lát ina þi an þínumu hugi lêðen, &
míð is an þínumu móde, \hld\ ne sí þat imu eft mildi god, &
hêr heven-kuning \hld\ helpe far·líhe, &
fader allaro firiho barno.“ \hld\ Þó frágode Petrus, &
allaro þegno bętst \hld\ þeodan sínan: &
„hwó oft skal ik þem mannun, \hld\ þe wið mi habbjad &
lêð-werk gi·duan, \hld\ leovo drohtin, &
skal ik im sivun sïðun \hld\ iro sundja a·láten, &
wrêðaro werko, \hld\ êr þan ik is êniga wréka frummje, &%TODO: Check wréka.
lêðes te lône?“ \hld\ Þó sprak eft þe landes ward, &
an·gęgin þe godes sunu \hld\ gódumu þegne: &
„ni sęggju ik þi fan sivunjun, \hld\ só þú selvo sprikis, &
mahlis mid þínu mu̇ðu, \hld\ ik duom þi méra þar tó: &
sivun sïðun sivun-tig \hld\ só skalt þú sundja ge·hwemu, &
lêðes a·láten: \hld\ só willju ik þi te lêrun geven &
wordun wár-fastun. \hld\ Nu ik þí su·lika gi·wald far·gaf, &
þat þú mínes híwiskes \hld\ hêrost wáris, &
manages mann-kunnjes, \hld\ nu skalt þú im mildi wesen, &
liudjun líði.“ \hld\ Þó þar te þemu lêrjande kwam &
ên jung man an·gęgin \hld\ ęndi frágode Jesu Krist: &
„mêster þe gódo“, \hld\ kwað he, „hwat skal ik manages duan, &
an þiu þe ik heven-ríki \hld\ ge·halan móti?“ &
Habde imu ôd-welon \hld\ allen ge·wunnen, &
mêðom-hord manag, \hld\ þoh he mildjan hugi &
bári an is breostun. \hld\ Þó sprak imu þat barn godes: &
„hwat kwiðis þú umbi gódon? \hld\ nis þat gumono ênig &
bi·útan þe êno, \hld\ þe þar al ge·skóp, &
wer-old ęndi wunnja. \hld\ Ef þú is willjan havas, &
þat þú an lioht godes \hld\ líðan mótis, &
þan skalt þú bi·halden \hld\ þea hêlagon lêra, &
þe þar an þemu aldon \hld\ êwa ge·biudid, &
þat þú man ni slah, \hld\ ni þú mênes ni sweri, &
far·legar-nessi far·lát \hld\ ęndi luggi ge·wit-skępi, &
stríd ęndi stulina; \hld\ ne wis þú te stark an hugi, &
ne níðin ne hatul, \hld\ ni nôd-róf ni fręmi; &
av·unst alla far·lát; \hld\ wis þínun ęldirun gód, &
fader ęndi móder, \hld\ ęndi þínun friundun hold, &
þem náhistun gi·náðig. \hld\ Þan þú þi gi·niodon móst &
himilo ríkjas, \hld\ ef þú it bi·halden wili, &
ful-gangan godes lêrun.“ \hld\ Þó sprak eft þe jungo man &
„al hębbju ik só gi·lêstid“, \hld\ kwað he, „só þú mi lêris nu, &
wordun wísis, \hld\ só ik is eo wiht ni far·lét &
fan mínero kindiski.“ \hld\ Þó bi·gan ina Krist sehan &
an mid is ôgun: \hld\ „ên is þar noh nu“, kwað he, &
„wan þero werko: \hld\ ef þú is willjon havas, &
þat þú þurh-fręmid \hld\ þionon mótis &
hêrron þínumu, \hld\ þan skalt þú þat þín hord nimen, &
skalt þínan ôd-welon \hld\ allan far·kôpjen, &
diurje mêðmos, \hld\ ęndi dêljen hét &
armun mannun: \hld\ þan havas þú aftar þiu &
hord an himile; \hld\ kum þi þan gi·halden te mi, &
folgo þi mínaro fęrdi: \hld\ þan havas þú friðu sïður.“ &
Þó wurðun Kristes word \hld\ kind-jungumu manne &
swíðo an sorgun, \hld\ was imu sêr hugi, &
mód umbi herte: \hld\ habde mêðmo filu, &
welono ge·wunnen; \hld\ wende imu eft þanen, &
was imu unóðo \hld\ innan breostun, &
an is sevon swáro. \hld\ Sah imu aftar þó &
Krist alo-waldo, \hld\ kwað it þó, þar he welde, &
te þem is jungarun gęgin-wardun, \hld\ þat wári an godes ríki &
un·óði ôdagumu manne \hld\ up te kumanne: &
„óður mag man olvundjon, \hld\ þoh he sí un·met grôt, &
þurh náðlan gat, \hld\ þoh it sí naru swíðo, &
sáftur þurh-slópjen, \hld\ þan mugi kuman þiu siole te himile &
þes ôdagan mannes, \hld\ þe hér al havad &
gi·węndid an þene wer-old-skat \hld\ willjon sínen, &
mód-gi·þáhti, \hld\ ęndi ni hugid umbi þie maht godes.“ &
Imu and-wordjade \hld\ êr-þungan gumo, &
Símon Petrus, \hld\ ęndi sęggjan bad &
leovan hêrron: \hld\ „hwat skulun wí þes te lône nimen“, kwað he, &
„gódes te gelde, \hld\ þes wí þurh þín jungar-dóm &
êgan ęndi ęrvi \hld\ al far·létun &
hovos ęndi híwiski \hld\ ęndi þi te hêrron gi·kurun, &
folgodun þínaru fęrdi: \hld\ hwat skal u̇s þes te frumu werðen, &
langes te lône?“ \hld\ liudjo drohtin &
sagde im þó selvo: \hld\ „þan ik sittjen kumu“, kwað he, &
„an þie mikilan maht \hld\ an þemu márjan dage, &
þar ik allun skal \hld\ irmin-þiodun &
dómos a·dêljen, \hld\ þan mótun gi mid iuwomu drohtine þar &
selvon sittjen \hld\ ęndi mótun þera saka waldan: &
mótun gi Israhelo \hld\ ęðili-folkun &
a·dêljen aftar iro dádjun: \hld\ só mótun gi þar gi·diuride wesen. &
Þan sęggju ik iu te wáran: \hld\ só hwe só þat an þesaru wer-oldi gi·duot, &
þat he þurh mína minnja \hld\ mágo ge·sidli &
liof far·létid, \hld\ þes skal hi hér lôn niman &
tehan sïðun tehin-fald, \hld\ ef he it mid trewon duot, &
mid hluttru hugi. \hld\ Ovar þat havad he ôk himiles lioht, &
open êwig líf.“ \hld\ Bi·gan imu þó aftar þiu &
allaro barno bętst \hld\ ên biliði sęggjan, &
kwað þat þar ên ôdag man \hld\ an êr-dagun &
wári undar þemu werode: \hld\ „þe habde welono ge·nóg, &
sinkas gi·samnod \hld\ ęndi imu simlun was &
garu mid goldu \hld\ ęndi mid godo-wębbju, &
fagarun fratahun \hld\ ęndi imu so filu habde &
gódes an is gardun \hld\ ęndi imu at gômun sat &
allaro dago ge·hwi-likes: \hld\ habde imu diur-lík líf, &
blíðsea an is bęnkjun. \hld\ Þan was þar eft ên biddjendi man, &
gi·lévod an is lík-hamon, \hld\ Lazarus was he hêten, &
lag imu dago ge·hwi-likes \hld\ at þem durun foren, &
þar he þene ôdagan man \hld\ inne wisse &
an is gęst-sęli \hld\ gôme þiggjan, &
sittjen at sumble, \hld\ ęndi he simlun bêd &
gi·armod þar úte: \hld\ ni móste þar in kuman, &
ne he ni mahte ge·biddjen, \hld\ þat man imu þes brôdes þarod &
gi·dragan weldi, \hld\ þes þar fan þemu diske niðer &
ant·fel undar iro fóti: \hld\ ni mahte imu þar ênig fruma werðen &
fan þemu hêroston, þe þes húses gi·weld, \hld\ bi·útan þat þar géngun is hundos tó, &
likkodun is lík-wundon, \hld\ þar he liggjandi &
hungar þolode; \hld\ ni kwam imu þar te helpu wiht &
fan þemu ríkjon manne. \hld\ Þó gi·fragn ik þat ina is regano-gi·skapu, &
þene armon man \hld\ is ên-dago &
gi·manoda mahtjun swíð, \hld\ þat he manno drôm &
a·geven skolde. \hld\ Godes ęngilos &
ant·féngun is ferh \hld\ ęndi lêddun ine forð þanen, &
þat sie an Abrahames barm \hld\ þes armon mannes &
siole gi·sęttun: \hld\ þar móste he simlun forð &
wesen an wunnjun. \hld\ Þó kwámun ôk wurde-gi·skapu, &
þemu ôdagan man \hld\ or·lag-hwíle, &
þat he þit lioht far·lét: \hld\ lêða wihti &
be·sinkodun is siole \hld\ an þene swarton hęl, &
an þat fern innen \hld\ fíundun te willjan, &
be·gróvun ine an gramono hêm. \hld\ Þanen mahte he þene gódan skawon, &
Abraham ge·sehen, \hld\ þar he uppe was &
líves an lustun, \hld\ ęndi Lazarus sat &
blíði an is barme, \hld\ berht lôn ant·féng &
allaro is arm-ódjo, \hld\ ęndi lag þe ôdago man &
hêto an þeru hęllju, \hld\ hriop up þanen: &
„fader Abraham“, \hld\ kwað he, „mi is firinun þarf, &
þat þú mi an þínumu mód-sevon \hld\ mildi werðes, &
líði an þesaru lognu: \hld\ sęndi mi Lazarus herod, &
þat he mí ge·fórja \hld\ an þit fern innan &
kaldes wateres. \hld\ Ik hér kwik brinnu &
hêto an þesaru hęllju: \hld\ nu is mi þínaro helpono þarf, &
þat he mi a·lęskje \hld\ mid is luttikon fingru &
tungon míne, \hld\ nu siu têkạn havad, &
uvil arvedi. \hld\ Inwid-rádo, &
lêðaro spráka, \hld\ alles is mi nu þes lôn kumen.“ &
Imu and-wordjade þó Abraham \hld\ —þat was ald-fader—: &
„ge·hugi þú an þínumu herton“, \hld\ kwað he, „hwat þú habdes iu &
welono an wer-oldi. \hld\ Hwat, þú þar alle þíne wunnja far·sliti, &
gódes an gardun, \hld\ só hwat só þi giviðig forð &
werðen skolde. \hld\ Wíti þolode &
Lazarus an þemu liohte, \hld\ habde þar lêðes filu, &
wítjas an wer-oldi. \hld\ Be·þiu skal he nu welon êgan, &
libbjen an lustun: \hld\ þú skalt þea logna þolan, &
brinnendi fiur: \hld\ ni mag is þi ênig bóte kumen &
hinana te hęllju: \hld\ it havad þe hêlago god &
só gi·fastnod mid is faðmun: \hld\ ni mag þar faren ênig &
þegno þurh þat þiustri: \hld\ it is hér só þikki undar u̇s.“ &
Þó sprak eft Abrahame \hld\ þe erl te·gęgnes &
fan þeru hêtan hęll \hld\ ęndi helpono bad, &
þat he Lazarus \hld\ an liudjo drôm &
selvon sandi: \hld\ „þat he ge·sęggja þar &
bróðarun mínun, \hld\ hwó ik hér brinnendi &
þrá-werk þolon; \hld\ si þar undar þeru þiodu sind, &
si fïvi undar þemu folke: \hld\ ik an forhtun bium, &
þat sie im þar far·wirkjen, \hld\ þat sie skulin ôk an þit wíti te mi, &
an só grádag fiur.“ \hld\ Þó imu eft te·gęgnes sprak &
Abraham ald-fader, \hld\ kwað þat sie þar êo godes &
an þemu land-skępi, \hld\ liudi habdin, &
Moyseses gi·bôd \hld\ ęndi þar managaro tó &
wár-saguno word: \hld\ „ef sie is willige sind, &
þat sie þat bi·halden, \hld\ þan ni þurvun sie an þea hęll innen, &
an þat fern faren, \hld\ ef sie ge·frummjad só, &
só þea ge·biodad, \hld\ þe þea bók lesat &
þem liudjun te lêrun. \hld\ Ef sie þes þan ni willjad lêstjen wiht, &
þanne ni hôrjad sie ôk \hld\ þemu þe hinan a·stád, &
man fan dôðe. \hld\ Láte man sie an iro mód-sevon &
selvon keosen, \hld\ hweðer im swótjera þunkje &
te gi·winnanne, \hld\ só lango só sie an þesaru wer-oldi sind, &
þat sie eft uvil etþa gód \hld\ aftar habbjen.“ &
Só lêrde he þó þea liudi \hld\ liohton wordon, &
allaro barno bętst, \hld\ ęndi biliði sagde &
manag man-kunnje \hld\ mahtig drohtin, &
kwað þat imu ên sálig gumo \hld\ samnon bi·gunni &
man an morgen, \hld\ „ęndi im méda gi·hét, &
þe hêrosto þes híwiskjas, \hld\ swíðo *hold-lík lôn“, &
kwað þat hie iro allaro gi·hwem \hld\ ênna gávi &
silovrinna skat. \hld\ „Þuo samnodun managa &
weros an is wín-gardon, \hld\ —ęndi hie im werk bi·falah— &
ádro an úhtan. \hld\ Sum kwam þar ôk an undorn tuo, &
sum kwam þar an middjan dag, \hld\ man te þem werke, &
sum kwam þar te nónu, \hld\ þuo was þiu niguða tíd &
sumar-langes dages; \hld\ sum þar ôk sïðor kwam &
an þia elliftun tíd. \hld\ Þuo géng þar ávand tuo, &
sunna ti sedle. \hld\ Þuo hie selvo gi·bôd &
is ambahtjon, \hld\ erlo drohtin, &
þat man þero manno gi·hwem \hld\ is meoda for·guldi, &
þem erlon arvid-lôn; \hld\ hiet þiem at êrist gevan. &
þia þar at letst wárun, \hld\ liudi kumana, &
weros te þem werke, \hld\ ęndi mid is wordon gi·bôd, &
þat man þem mannon iro \hld\ mieda for·guldi &
alles at aftan, \hld\ þem þar kwámun at êrist tuo &
willendi te þem werke. \hld\ Wándun sia swíðo, &
þat man im méra lôn \hld\ gi·makod habdi &
wið iro aravedje: \hld\ þan man im allon gaf, &
þem liudjon gi·líko. \hld\ Lêð was þat swíðo, &
allon þem ando, \hld\ þem þar kwámun at êrist tuo: &
„wí kwámun hier an moragan“, \hld\ kwáðun sia, „ęndi þolodun hier manag te dage &
aravid-werko, \hld\ hwílon un·met hét, &
skínandja sunna: \hld\ nu ni givis þú u̇s skattes þan mêr, &
þie þú þem ǫ́ðron duos, \hld\ þia hier êna hwíla &
wáron an þínon werke.“ \hld\ Þuo habda eft is word garo &
þie hêrosto þes híwiskes, \hld\ kwað þat hie im ni habdi gi·hêtan þan mêr &
werðes wið iro werke: \hld\ „hwat, ik gi·wald hębbju“, kwat-hie, &
„þat ik iu allon gi·líko \hld\ muot lôn for·geldan, &
iuwes werkes werð.“ \hld\ Þan waldandi Krist &
mênda im þoh méra þing, \hld\ þoh hie ovar þat manno folk &
fan þem wín-gardon só \hld\ wordon spráki, &
hwó þar un·efno \hld\ erlos kwámun, &
weros te þem werke. \hld\ Só skulun fan þero wer-oldi duon &
mann-kunnjes barn \hld\ an þat márjo lioht, &
gumon an godes wang: \hld\ sum bi·ginnit ina giriwan sán &
an is kindiski, \hld\ havit im gi·koranan muod, &
willjon guodan, \hld\ wer-old-saka míðit, &
far·látit is lusta; \hld\ ni mag ina is lík-hamo &
an un·spuod for·spanan: \hld\ spáhiða línot, &
godes êw, \hld\ gramono for·látit, &
wrêðaro willjon, \hld\ duot im só te is wer-oldi forð, &
lêstit só an þeson liohte, \hld\ ant-þat im is líves kumit, &
aldres ávand; \hld\ gi·wítit im þan up-wegos: &
þar wirðit im is aravedi \hld\ all gi·lônot, &
far·goldan mid guodu \hld\ an godes ríkje. &
Þat mêndun þia wuruhtjon, \hld\ þia an þem wín-gardon &
ádro an úhta \hld\ arvid-líko &
werk bi·gunnun \hld\ ęndi þuru-wonodun forð, &
erlos unt ávand. \hld\ Sum þar ôk an undern kwam, &
habda þuo far·męrrid, \hld\ þia moragan-stunda &
þes dag-werkes for·duolon; \hld\ só duot doloro filo, &
gi·médaro manno: \hld\ drívit im mis-lík þing &
gerno an is juguði, \hld\ —havit im gelp-kwidi &
lêða gi·línot \hld\ ęndi lôs-word manag—, &
ant-þat is kindiski \hld\ far·kuman wirðit, &
þat ina after is juguði \hld\ godes anst manot &
blíði an is brioston; \hld\ fáhit im te bęteron þan &
wordon ęndi werkon, \hld\ lêdit im is wer-old mid þiu, &
is aldar ant þena ęndi: \hld\ kumit im alles lôn &
an godes ríkje, \hld\ gódaro werko. &
Sum mann þan mid-firi \hld\ mên far·látid, &
swára sundjun, \hld\ fáhit im an sálig þing, &
bi·ginnit im þuru godes kraft \hld\ guodaro werko, &
buotit balo-spráka, \hld\ látit im is bittrun dád &
an is hugje hrewan; \hld\ kumit im þiu helpa fon gode, &
þat im gi·lêstid þie gi·lôvo, \hld\ só lango só im is líf warod; &
farit im forð mid þiu, \hld\ ant·fáhit is mieda, &
guod lôn at gode; \hld\ ni sindun êniga geva bęteran. &
Sum bi·ginnit þan ôk furðor, \hld\ þan hie ist fruodot mêr, &
is aldares af·heldit, \hld\ —þan bi·ginnat im is uvilon werk &
lêðon an þeson liohte, \hld\ þan ina lêra godes &
gi·manod an is muode: \hld\ wirðit im mildera hugi, &
þuru-gęngit im mid guodu \hld\ ęndi geld nimit, &
hôh himil-ríki, \hld\ þan hie hinan węndit, &
wirðit im is mieda só sama, \hld\ só þem man *nun warð, &
þea þar te nónu dages, \hld\ an þea nigunda tíd, &
an þene wín-gardon \hld\ wirkjan kwámun. &
Sum wirðid þan só swíðo ge·fródot, \hld\ só he ni wili is sundja bótjen, &
ak he ôkid sie mid uvilu ge·hwi-liku, \hld\ antat imu is ávand náhid, &
is wer-old ęndi is wunnja far·slítid; \hld\ þan be·ginnid he imu wíti and-réden, &
is sundjon werðad imu sorga an móde: \hld\ ge·hugid hwat he selvo ge·frumide &
grimmes þan lango, þe he móste is juguðjo neoten; \hld\ ni mag þan mid ǫ́ðru gódu gi·bótjen &
þea dádi, þea he só dęrvja ge·frumide, \hld\ ak he slęhit allaro dago ge·hwi-likes &
an is breost mid bêðjun handun \hld\ ęndi wópit sie mid bittrun trahnun, &
hlúdo he sie mid hofnu kúmid, \hld\ bidid þene hêlagon drohtin &
mahtigne, þat he imu mildi werðe: \hld\ ni látid imu sïðor is mód gi·twífljen; &
só ê-gróht-ful is, þe þar alles ge·weldid: \hld\ he ni wili ênigumu irmin-manne &
far·węrnjen willjan sínes; \hld\ far·givid imu waldand selvo &
hêlag himil-ríki: \hld\ þan is imu gi·holpen sïður. &
Alle skulun sie þar êra ant·fáhen, \hld\ þoh sie þarod te ênaru tídi &
ni kumen, þat kunni manno, \hld\ þoh wili imu þe kraftigo drohtin, &
gi·lônon allaro liudjo só hwi-likumu, \hld\ só hér is gi·lôvon ant·fáhit: &
ên himil-ríki \hld\ givid he allun þeodun, &
mannun te médu. \hld\ Þat mênde mahtig Krist, &
barno þat bętste, \hld\ þó he þat biliði sprak, &
hwó þar te þem wín-gardun \hld\ wurhtjon kwámin, &
man mis-líko: \hld\ þoh nam is méde ge·hwe &
fulle te is frôjan. \hld\ Só skulun firiho barn &
at gode selvumu \hld\ geld ant·fáhen, &
swíðo leov-lík lôn, \hld\ þoh sie sume só late werðan. &
Hét imu þó þea is gódan \hld\ jungaron náhor &
twe-livi gangan \hld\ —þea wárun imu triuwiston &
man ovar erðu—, \hld\ sagde im mahtig selvo &
ǫ́ðer-sïðu, \hld\ hwi-lik imu þar arvedi &
tó·ward wárun: \hld\ „þes ni mag ênig tweho werðen“, kwað he; &
kwað þat sie þó te Hjerusalem \hld\ an þat Judeono folk &
líðan skoldin: \hld\ „þar wirðid all gi·lêstid só, &
ge·frumid undar þemu folke, \hld\ só it an furn-dagun &
wíse man be mi \hld\ wordun ge·sprákun. &
Þar skulun mi far·kôpon \hld\ undar þea kraftigon þiod, &
hęliðos te þeru hêri; \hld\ þar werðat mína hęndi ge·bundana, &
faðmos werðad mi þar gefastnod; \hld\ filu skal ik þar gi·þolojan, &
hoskes gi·hôrjen \hld\ ęndi harm-kwidi, &
bi·smer-spráka \hld\ ęndi bi·hêt-word manag; &
sie wêgjat mi te wundron \hld\ wápnes ęggjun, &
bi·lôsjad mi lívu: \hld\ ik te þesumu liohte skal &
þurh drohtines kraft \hld\ fan dôðe a·standen &
an þriddjon dage. \hld\ Ni kwam ik undar þesa þeoda herod &
te þiu, þat mín ęldi-barn \hld\ arved habdin, &
þat mi þionodi þius þiod: \hld\ ni willju ik is sie þiggjen nu, &
fergon þit folk-skępi, \hld\ ak ik skal imu te frumu werðen, &
þeonon imu þeo-líko \hld\ ęndi for alla þesa þeoda geven &
seole míne. \hld\ Ik willju sie selvo nu &
lôsjen mid mínu lívu, \hld\ þea hér lango bidun, &
man-kunnjes manag, \hld\ mínara helpa.“ &
Fór imu þó forð-wardes \hld\ —habde imu fasten hugi, &
blíðjan an is breostun \hld\ barn drohtines— &
welda im te Hjerusalem \hld\ Judeo folkes &
willjon wísan: \hld\ he konste þes werodes só garo &
hęti-grimmen hugi \hld\ ęndi hardan stríd, &
wrêðan willjon. \hld\ Werod sïðode &
furi Hjerikho-burg; \hld\ was þe godes sunu, &
mahtig undar þero męnigi. \hld\ Þar sátun twênje man bi wege, &
blinde wárun sie bêðje: \hld\ was im bótono þarf, &
þat sie ge·hêldi \hld\ hevenes waldand, &
hwand sie só lango \hld\ liohtes þolodun, &
managa hwíla. \hld\ Sie gi·hôrdun þó þat męgin faren &
ęndi frágodun sán \hld\ firi-wit-líko &
ręgini-blindun, \hld\ hwi-lik þar ríki man &
undar þemu folk-skępi \hld\ furista wári, &
hêrost an hôvid. \hld\ Þó sprak im ên hęlið an·gęgin, &
kwað þat þar Hjesu Krist \hld\ fan Galilea-lande, &
hêljandero bętst \hld\ hêrost wári, &
fóri mid is folku. \hld\ Þó warð fráh-mód hugi &
bêðjun þem blindun mannun, \hld\ þó sie þat barn godes &
wissun under þemu werode: \hld\ hreopun im þó mid iro wordun tó, &
hlúdo te þemu hêlagon Kriste, \hld\ bádun þat he im helpe ge·rédi: &
„drohtin Dawides sunu: \hld\ wis u̇s mid þínun dádjun mildi, &
nęri u̇s af þesaru nôdi, \hld\ só þú gi·nóge dós &
manno kunnjes: \hld\ þú bist managun gód, &
hilpis ęndi hêlis.“ \hld\ Þo bi·gan im þat hęliðo folk &
węrjen mid wordun, \hld\ þat sie an waldand Krist &
só hlúdo ni hriopin. \hld\ Si ni weldun im hôrjen te þiu, &
ak sie simla mêr ęndi mêr \hld\ ovar þat manno folk &
hlúdo hreopun. \hld\ Héljand ge·stód, &
allaro barno bętst, \hld\ hét sie þó brengjen te imu, &
lêdjen þurh þea liudi, \hld\ sprak im listjun tó &
mild-líko for þeru męnegi: \hld\ „hwat willjad git mínaro hér“, kwað he, &
„helpono habbjen?“ \hld\ Sie bádun ina hêlagna, &
þat he im ira ôgon \hld\ opana gi·dádi, &
far·liwi þeses liohtes, \hld\ þat sie liudjo drôm, &
swigle sunnun skín \hld\ gi·sehen móstin, &
wliti-skónje wer-old. \hld\ Waldand frumide, &
hrên sie þó mid is handun, \hld\ dede is helpe þar tó, &
þat þem blindun þó \hld\ bêðjum wurðun &
ôgon gi·oponod, \hld\ þat sie erðe ęndi himil &
þurh kraft godes \hld\ ant·kiennjen mahtun, &
lioht ęndi liudi. \hld\ Þó sagdun sie lof gode, &
diurdun u̇san drohtin, \hld\ þes sie dages liohtes &
brúkan móstun: \hld\ ge·witun im bêðje mid imu, &
folgodun is fęrdi: \hld\ was im þiu fruma giviðig, &
ęndi ôk waldandes werk \hld\ wído ge·ku̇ðid, &
managun gi·márid. \hld\ Þar was só mahtig-lík &
biliði gi·bóknid, \hld\ þar þe blindon man &
bi þemu wege sátun, \hld\ wíti þolodun, &
liohtes lôse: \hld\ þat mênid þoh liudjo barn, &
al man-kunni, \hld\ hwó sie mahtig god &
an þemu ana·ginne \hld\ þurh is ênes kraft &
sin-híun twê \hld\ selvo gi·warhte, &
Ádam ęndi Éwan: \hld\ far·gaf im up-wegos, &
himilo ríki; \hld\ ak þó warð im þe hatola te náh, &
fíund mid fêknu \hld\ ęndi mid firin-werkun, &
bi·swêk sie mid sundjun, \hld\ þat sie sin-skóni, &
lioht far·létun: \hld\ wurðun an lêðaron stędi, &
an þesen middil-gard \hld\ man far·worpen, &
þolodun hér an þiustrju \hld\ þiod-arvedi, &
wunnun wrak-sïðos, \hld\ welon þarvodun: &
far·gátun godes ríkjes, \hld\ gramon þeonodun, &
fíundo barnun; \hld\ sie guldun is im mid fiuru lôn &
an þeru hêton hęllju. \hld\ Be·þiu wárun siu an iro hugi blinda &
an þesaru middil-gard, \hld\ męnniskono barn, &
hwand siu ine ni ant·kiendun, \hld\ kraftagne god, &
himilisken hêrron, \hld\ þene þe sie mid is handun gi·skóp, &
gi·warhte an is willjon. \hld\ Þius wer-old was þó só far·hwervid, &
bi·þwungen an þiustrje, \hld\ an þiod-arvidi, &
an dôðes dalu: \hld\ sátun im þó bi þeru drohtines strátun &
jámar-móde, \hld\ godes helpe bidun: &
siu ni mahte im þó êr werðen, \hld\ êr þan waldand god &
an þesan middil-gard, \hld\ mahtig drohtin, &
is selves sunu \hld\ sęndjen weldi &
þat he lioht ant·luki \hld\ liudjo barnun, &
oponodi im êwig líf, \hld\ þat sie þene alo-waldon &
mahtin ant·kęnnjen wel, \hld\ kraftagna god. &
Ôk mag ik giu gi·tęlljen, \hld\ of gí þar tó willjad &
huggjen ęndi hôrjen, \hld\ þat gí þes hêljandes mugun &
kraft ant·kęnnjen, \hld\ hwó is kumi wurðun &
an þesaru middil-gard \hld\ managun te helpu, &
ia hwat he mid þem dádjun \hld\ drohtin selvo &
manages mênde, \hld\ ia be·hwiu þiu márje burg &
Hjerikho hêtid, \hld\ þiu þar an Judeon stád &
gi·makod mid múrun: \hld\ þiu is aftar þemu mánen gi·nęmnid, &
aftar þemu torhten tungle: \hld\ he ni mag is tídi be·míðen, &
ak he dago ge·hwi-likes \hld\ duod ǫ́ðer-hweðer, &
wanod ohþo wahsid. \hld\ Só dód an þesaro wer-oldi hér, &
an þesaru middil-gard \hld\ męnniskono barn: &
farad ęndi folgod, \hld\ fróde stervad, &
werðad eft junga \hld\ aftar kumane, &
weros a·wahsane, \hld\ unt-tat sie eft wurd far·nimid. &
Þat mênde þat barn godes, \hld\ þó he fon þeru burgi fór, &
þe gódo fan Hjerikho, \hld\ þat ni mahte êr werðen gumono barnun &
þiu blindja gi·bótid, \hld\ þat sie þat berhte lioht, &
gi·sáhin sin-skóni, \hld\ êr þan he selvo hér &
an þesaru middil-gard \hld\ męnniski ant·féng, &
flêsk ęndi lík-hamon. \hld\ Þó wurðun þes firiho barn &
gi·war an þesaru wer-oldi, \hld\ þe hér an wítje êr, &
sátun an sundjun \hld\ gi·siunjes lôse, &
þolodun an þiustrje, \hld\ —sie af·sóvun þat was þesaru þiod kuman &
hêljand te helpu \hld\ fan heven-ríkje, &
Krist allaro kuningo best; \hld\ sie mahtun is ant·kęnnjen sán, &
gi·fóljen is fardjo. \hld\ Þó sie só filu hriopun, &
þe man te þemu mahtigon gode, \hld\ þat im mildi aftar þiu &
waldand wurði. \hld\ Þan węridun im swíðo &
þia swárun sundjon, \hld\ þe sie im êr selvon gi·dádun, &
lettun sie þes gi·lôbon. \hld\ Sie ni mahtun þem liudjun þoh &
bi·węrjen iro willjon, \hld\ ak sie an waldand god &
hlúdo hriopun, \hld\ antat he im iro hêli far·gaf, &
þat sie sin-líf \hld\ gi·sehen móstin, &
open êwig lioht \hld\ ęndi an faren &
an þiu berhtun bú. \hld\ Þat mêndun þea blindun man, &
þe þar bi Hjerikho-burg \hld\ te þemu godes barne &
hlúdo hriopun, \hld\ þat he im iro hêli far·lihi, &
liohtes an þesumu líve: \hld\ þan im þea liudi só filu &
węridun mid wordun, \hld\ þea þar an þemu wege fórun &
bi·foren ęndi bi·hinden: \hld\ só dót þea firin-sundjon &
an þesaru middil-gard \hld\ man-kunnje. &
hôrjad nu hwó þie blindun, \hld\ sïður im gi·bótid warð, &
þat sie sunnun lioht \hld\ ge·sehen móstun, &
hwó si þó dádun: \hld\ ge·witun im mid iro drohtine samad, &
folgodun is fęrdi, \hld\ sprákun filu wordo &
þemu landes hirdje te love: \hld\ só dód im noh liudjo barn &
wído aftar þesaru wer-oldi, \hld\ sïður im waldand Krist &
ge·liuhte mid is lêrun \hld\ ęndi im líf êwig, &
godes ríki far·gaf \hld\ gódun mannun, &
hôh himiles lioht \hld\ ęndi is helpe þar tó, &
só hwemu só þat gi·werkod, \hld\ þat he móti þemu is wege folgon. &
Þó náhide \hld\ nęrjendo Krist, &
þe gódo te Hjerusalem. \hld\ Kwam imu þar te·gęgnes filu &
werodes an willjon \hld\ wel huggendjes, &
ant·féngun ina fagaro \hld\ ęndi imu bi·foren stręidun &%NOTE: ęi is original.
þene weg mid iro gi·wádjun \hld\ ęndi mid wurtjun só same, &
mid berhtun blómun \hld\ ęndi mid bômo tógun, &
þat feld mid fagaron palmun, \hld\ al só is fard ge·buride, &
þat þe godes sunu \hld\ gangan welde &
te þeru márjan burg. \hld\ Hwarf ina męgin umbi &
liudjo an lustun, \hld\ ęndi lof-sang a·hóf &
þat werod an willjon: \hld\ sagdun waldande þank, &
þes þar selvo kwam \hld\ sunu Dawides &
wíson þes werodes. \hld\ Þó ge·sah waldand Krist &
þe gódo te Hjerusalem, \hld\ gumono bętsta, &
blíkan þene burges wal \hld\ ęndi bú Judeono, &
hôha horn-sęli \hld\ ęndi ôk þat hús godes, &
allaro wího wun-samost. \hld\ Þó wel imu an innen &
hugi wið is herte: \hld\ þó ni mahte þat hêlage barn &
wópu a·wísjen, \hld\ sprak þó wordo filu &
hriwig-líko \hld\ —was imu is hugi sêreg—: &
„wê warð þi, Hjerusalem“, \hld\ kwað he, „þes þú te wárun ni wêst &
þea wurde-gi·skęfti, \hld\ þe þi noh gi·werðen skulun, &
hwó þú noh wirðis be·habd \hld\ hęrjes kraftu &
ęndi þi bi·sittjad \hld\ slíð-móde man, &
fíund mid folkun. \hld\ Þan ni havas þú friðu hwęrgin, &
mund-burd mid mannun: \hld\ lêdjad þi hér manage tó &
ordos ęndi ęggja, \hld\ or·legas word, &
far·fioþ þín folk-skępi \hld\ fiures liomon, &
þese wíki a·wóstjad, \hld\ wallos hôha &
fęlljad te foldun: \hld\ ni af·stád is felis nigijan, &
stên ovar ǫ́ðrumu, \hld\ ak werðad þesa stędi wóstja &
umbi Hjerusalem \hld\ Judeo liudjo, &
hwand sie ni ant·kęnnjad, \hld\ þat im kumana sind &
iro tídi tó-wardes, \hld\ ak sie habbjad im twífljen hugi, &
ni witun þat iro wísad \hld\ waldandes kraft.“ &
Gi·wêt imu þó mid þeru męnegi \hld\ manno drohtin &
an þea berhton burg. \hld\ Só þó þat barn godes &
innan Hjerusalem \hld\ mid þiu gumono folku, &
sêg mid þiu ge·sïðu, \hld\ þó warð þar allaro sango mêst, &
hlúd stemnje af·haven \hld\ hêlagun wordun, &
lovodun þene landes ward \hld\ liudjo męnegi, &
barno þat bętste; \hld\ þiu burg warð an hróru, &
þat folk warð an forhtun \hld\ ęndi frágodun sán, &
hwe þat wári, \hld\ þat þar mid þiu werodu kwam, &
mid þeru mikilon męnegi. \hld\ Þó sprak im ên man an·gęgin, &
kwað þat þar Hjesu Krist \hld\ fan Galileo lande, &
fan Nazareth-burg \hld\ nęrjand kwámi, &
witig wár-sago \hld\ þemu werode te helpu. &
Þó was þem Judiun, \hld\ þe imu êr grame wárun, &
un·holde an hugi, \hld\ harm an móde, &
þat imu þea liudi só filu \hld\ lof-sang warhtun, &
diurdun iro drohtin. \hld\ Þó géngun dol-móde, &
þat sie wið waldand Krist \hld\ wordun sprákun, &
bádun þat he þat ge·sïði \hld\ swígon héti, &
letti þea liudi, \hld\ þat sie imu lof só filu &
wordun ni warhtin: \hld\ „it is þesumu werode lêð“, kwáðun sie, &
„þesun burg-liudjun.“ \hld\ Þó sprak eft þat barn godes: &
„ef gi sie a·męrrjad“, \hld\ kwað he, „þat hér ni mótin manno barn &
waldandes kraft \hld\ wordun diurjen, &
þan skulun it hrópen þoh \hld\ harde stênos &
for þesumu folk-skępi, \hld\ felisos starka, &
êr þan it eo be·líve, \hld\ nevo man is lof spreke &
wído aftar þesaru wer-oldi.“ \hld\ Þó he an þene wíh innen, &
géng an þat godes hús: \hld\ fand þar Judeono filu, &
mis-líke man, \hld\ manage at-samne, &
þea im þar kôp-stędi \hld\ gi·koran habdun, &
mangodun im þar mid manages hwí: \hld\ muniterjas sátun &
an þemu wíhe innan, \hld\ habdun iro wesl gi·dago &
garu te gevanne. \hld\ Þat was þemu godes barne &
al an andun: \hld\ drêf sie ut þanen &
rúmo fan þemu rakude, \hld\ kwað þat wári rehtara dád, &
þat þar te bedu fórin \hld\ barn Israheles &
„ęndi an þesumu mínumu húse \hld\ helpono biddjan, &
þat sia sigi-drohtin \hld\ sundjono tuomje, &
þan hér þeovas \hld\ an þing-stędi halden, &
þea far·warhton weros \hld\ wehsal drívan, &
un·reht ên-fald. \hld\ Ne gi êniga êra ni witun &
þeses godes húses, \hld\ Judeo liudi.“ &
Só rúmde he þó ęndi rekode, \hld\ ríki drohtin, &
þat hêlaga hús \hld\ ęndi an helpun was &
managumu man-kunnje, \hld\ þem þe is mikilon kraft &
ferrene ge·frugnun \hld\ ęndi þar gi·faran kwámun &
ovar langan weg. \hld\ Warð þar léf so manag, &
halt gi·hêlid \hld\ ęndi háf só same, &
blindun gi·bótid. \hld\ Só dede þat barn godes &
willjendi þemu werode, \hld\ hwand al an is gi·weldi stéd &
umbi þesaro liudjo líf \hld\ ęndi ôk umbi þit land só same. &
Stód imu þó fora þemu wíhe \hld\ waldandeo Krist, &
liof landes ward, \hld\ ęndi imu þero liudjo hugi, &
iro willjon aftar-warode: \hld\ gi·sah werod mikil &
an þat márje hús \hld\ mêðmos fórjen, &
gevon mid goldu \hld\ ęndi mid godu-wębbju, &
diurjun fratahun. \hld\ Þat al drohtin Krist &
warode wís-líko. \hld\ Þó kwam þar ôk ên widowa tó, &
idis arm-skapen, \hld\ ęndi te þemu alaha géng &
ęndi siu an þat tresur-hús \hld\ twêne lęgde &
êríne skattos: \hld\ was iru ên-fald hugi, &
willjan gódes. \hld\ Þó sprak waldand Krist, &
þe gumo wið is gjungaron, \hld\ kwað þat siu þar geva bráhti &
méron mikilu þan elkor \hld\ ênig mannes sunu: &
„ef hér ôdaga man“, \hld\ kwað he, „êra bráhtun, &
mêðom-hord manag, \hld\ sie létun im mêr at hús &
welona ge·wunnen. \hld\ Ni dede þius widowa só, &
ak siu te þesumu alahe gaf \hld\ al þat siu habde &
welono ge·wunnen, \hld\ só siu iru wiht ni far·lét &
gódes an iro gardun. \hld\ Be·þiu sind ira geva méron, &
waldande werða, \hld\ hwand siu it mid su·likumu willjon dede &
te þesumu godes húse. \hld\ Þes skal siu geld niman, &
swíðo lang-sam lôn, \hld\ þes siu su·likan gi·lôvon havad.“ &
Só gi·fragn ik þat þar an þemu wíhe \hld\ waldandeo Krist &
allaro dago ge·hwi-likes, \hld\ drohtin manno, &
wísde mid wordun. \hld\ Stód ine werod umbi, &
grôt folk Judeono, \hld\ gi·hôrdun is gódan word, &
swótja sęggjan. \hld\ Sum só sálig warð &
manno undar þeru męnegi, \hld\ þat it bi·gan an is mód hladen; &
línodun im þea lêra, \hld\ þe þe landes ward &
al be biliðjun sprak, \hld\ barn drohtines. &
Sumun wárun eft so lêða \hld\ lêra Kristes, &
waldandes word: \hld\ was im wiðer-mód hugi &
allun þem, þe an þemu hęri-skępi \hld\ hêrost wárun, &
furiston an þemu folke: \hld\ fáres hugdun &
wrêða mid iro wordun \hld\ —habdun im wiðer-sakon &
gi·haloden te helpu, \hld\ þes hêroston man, &
Erodeses þegạn, \hld\ þe þar and-ward stód &
wrêðes willjan, \hld\ þat he iro word ovar-hôrdi— &
ef sie ina for·féngin, \hld\ þat sie ina þan feteros an, &
þea liudi liðo-bęndi \hld\ lęggjen móstin, &
sundja lôsan. \hld\ Þó géngun im þea ge·sïðos tó &
bittra gi·hugde, \hld\ þat sie wið þat barn godes, &
wrêða wiðer-sakon \hld\ wordun sprákun: &
„hwat, þú bist êo-sago“, \hld\ kwáðun sie, „allun þiodun, &
wísis wáres só filu: \hld\ nis þi werð eo·wiht &
te bi·míðanne \hld\ manno ni-ênumu &
umbi is ríki-dóm, \hld\ nevo þú simlun þat reht sprikis &
ęndi an þene godes weg \hld\ gumono ge·sïði &
lêdis mid þinun lêrun: \hld\ ni mag þi laster man &
fïðan undar þesumu folke. \hld\ Nu wí þi frágon skulun. &
ríki þiodan, \hld\ hwi-lik reht havad &
þe kêsur fan Rúmu, \hld\ þe imu te þesumu kunnje herod &
tinsi sókid \hld\ ęndi gi·tald havad, &
hwat wí imu gelden skulin \hld\ gę́ro ge·hwi-likes &
hôvid-skatto. \hld\ Saga hwat þi þes an þínumu hugi þunkja: &
is it reht þe nis? \hld\ Rád for þínun &
land-mégun wel: \hld\ u̇s is þínaro lêrono þarf.“ &
Sie weldun þat he it ant·kwáði: \hld\ þan mahte he þoh ant·kęnnjen wel &
iro wrêðon willjon: \hld\ „te hwí gi wár-logon“, kwað he, &
„fandot mín só frókno? \hld\ Ni skal iu þat te frumu werðen, &
þat gi dreogerjas \hld\ darnungo nu &
willjad mi far·fáhen.“ \hld\ Hét he þó forð dragan &
te skawonne þe skattos, \hld\ „þe gi skuldige sind &
an þat geld geven.“ \hld\ Judeon drógun &
ênna siluvrinna forð: \hld\ sáhun manage tó, &
hwó he was ge·munitod: \hld\ was an middjen skín &
þes kêsures biliði \hld\ —þat mahtun sie ant·kęnnjen wel—, &
iro hêrron hôvid-mál. \hld\ Þó frágode sie þe hêlago Krist, &
aftar hwemu þiu ge·lík-nessi \hld\ gi·legid wári. &
Sie kwáðun þat it wári \hld\ wer-old-kêsures &
fan Rúmu-burg, \hld\ „þes þe alles þeses ríkes havad &
ge·wald an þesaru wer-oldi.“ \hld\ „Þan willju ik iu te wárun hér“, kwað he, &
„selvo sęggjan, \hld\ þat gi imu sín gevad, &
wer-old-hêrron is ge·wunst, \hld\ ęndi waldand gode &
sęlljad, þat þar sín ist: \hld\ þat skulun iuwa seolon wesen, &
gumono gêstos.“ \hld\ Þó warð þero Judeono hugi &
ge·minsod an þemu mahle: \hld\ ni mahtun þe mên-skaðon &
wordun ge·winnen, \hld\ só iro willjo géng, &
þat sie ina far·féngin, \hld\ hwand imu þat friðu-barn godes &
wardode wið þe wrêðon \hld\ ęndi im wár an·gęgin, &
sǫ́ð-spel sagde, \hld\ þoh sie ni wárin só sálige te þiu, &
þat sie it só far·féngin, \hld\ só it iro fruma wári. &
Sie ni weldun it þoh far·láten, \hld\ ak hétun þar lêdjen forð &
ên wíf for þemu werode, \hld\ þiu habde wam ge·frumid, &
un·reht ên-fald: \hld\ þiu idis was bi·fangen &
an far·legar-nessi, \hld\ was iro líves skolo, &
þat sie firiho barn \hld\ ferahu bi·námin, &
êhtin iro aldres: \hld\ só was an iro êw ge·skriven. &
Sie bi·gunnun ina þó frágon, \hld\ fruokne liudi, &
wrêða mid iro wordun, \hld\ hwat sie skoldin þemu wíve duan, &
hweðer sie sie kwęlidin, \hld\ þe sie sie kwika létin, &
þe hwat he umbi su·lika dádi \hld\ a·dêljen weldi: &
„þú wêst, hwó þesaru męnegi“, \hld\ kwáðun sie, „Moyses gi·bôd &
wárun wordun, \hld\ þat allaro wívo ge·hwi-lik &
an far·legar-nessi \hld\ líves far·warhti &
ęndi þat sie þan a·wurpin \hld\ weros mid handun, &
starkun stênun: \hld\ nu maht þú sie sehan standen hér &
an sundjun bi·fangan: \hld\ saga hwat þú is willjes.“ &
weldun ine þea wiðer-sakon \hld\ wordun far·fáhen, &
ef he þat gi·kwáði, \hld\ þat sie sie kwika létin, &
friðodi ira ferahe, \hld\ þan weldi þat folk Judeono &
kweðen, þat he iro aldiron \hld\ êo wiðer-sagdi, &
þero liudjo land-reht; \hld\ ef he sie þan héti lívu bi·nimen, &
þea magað fur þeru męnegi, \hld\ þan weldin sie kweðen, þat he só mildjene hugi &
ni bári an is breostun, \hld\ só skoldi habbjen barn godes: &
weldun sie só hweðeres \hld\ hêlagne Krist &
þero wordo ge·wítnon, \hld\ só he þar for þemu werode ge·spráki, &
a·dêldi te dóme. \hld\ Þan wisse drohtin Krist &
þero manno só garo \hld\ mód-gi·þáhti, &
iro wrêðon willjon; \hld\ þó he te þemu werode sprak, &
te allun þem erlun: \hld\ „só hwi-lik só iuwar áno sí“, kwað he, &
„slíðja sundjon, \hld\ só ganga iru selvo tó &
ęndi sie at êrist \hld\ erl mid is handun &
stên ana werpe.“ \hld\ Só stódun Judeon, &
þáhtun ęndi þagodun: \hld\ ni mahte þegạn nigijan &
wið þem word-kwidi \hld\ wiðer-saka finden: &
ge·hugde manno ge·hwi-lik \hld\ mên-gi·þáhti, &
is selves sundja: \hld\ ni was iro só sikur ênig, &
þat he bi þemu worde \hld\ þemu wíve ge·dorsti &
stên an werpen, \hld\ ak létun sie standen þar &
ênan þar inne \hld\ ęndi im út þanen &
géngun gram-harde \hld\ Judeo liudi, &
ên aftar ǫ́ðrumu, \hld\ antat iro þar ênig ni was &
þes fíundo folkes, \hld\ þe iro ferhes þó, &
þeru idis aldar-lago \hld\ áhtjen weldi. &
Þó gi·fragn ik þat sie frágode \hld\ friðu-barn godes, &
allaro gumono bętst: \hld\ „hwar kwámun þit Judeono folk“, kwað he, &
„þine wiðer-sakon, \hld\ þea þi hér wrógdun te mi? &
Ne sie þi hiudu wiht \hld\ harmes ne gi·dádun, &
þea liudi lêðes, \hld\ þe þi weldun lívu be·niman, &
wêgjan te wundrun?“ \hld\ Þó sprak imu eft þat wíf an·gęgin, &
kwað þat iru þar nio·man \hld\ þurh þes nęrjandan &
hêlaga helpa \hld\ harm ne gi·frumidi &
wammes te lône. \hld\ Þó sprak eft waldand Krist, &
drohtin manno: \hld\ „ne ik þi geþ ni derju n·eo·wiht“, kwað he, &
„ak gang þi hêl hinen, \hld\ lát þi an þínumu hugi sorga, &
þat þú nio sïð aftar þius \hld\ sundig ni werðes.“ &
Habde iru þó gi·holpen \hld\ hêlag barn godes, &
ge·friðot iro ferahe. \hld\ Þan stód þat folk Judeono &
uviles an·mód \hld\ só fan êristan, &
wrêðes willjan, \hld\ hwó sie word-hęti &
wið þat friðu-barn godes \hld\ frummjen móstin. &
Habdun þea liudi an twê \hld\ mid iro gi·lôvon gi·fangan: &
was þiu smale þioda \hld\ sínes willjan &
gernora mikilu, \hld\ þes godes barnes word &
te ge·frummjenne, \hld\ só im iro fráho gi·bôd: &
rómodun te rehta \hld\ bet þan þie ríkjon man, &
habdun ina far iro hêrron \hld\ ia far heven-kuning, &
ful-géngun imu gerno. \hld\ Þó gi·wêt imu þe godes sunu &
an þene wíh innan: \hld\ hwarf ina werod umbi, &
męgin-þiodo gi·mang. \hld\ He an middjen stód, &
lêrde þea liudi \hld\ liohtun wordun, &
hlúdero stemnun: \hld\ was hlust mikil, &
þagode þegạn manag, \hld\ ęndi he þeru þiod gi·bôd, &
só hwe só þar mid þurstu \hld\ bi·þwungan wári, &
„só ganga imu herod drinkan te mi“, \hld\ kwað he, „dago ge·hwi-likes &
swótjes brunnan. \hld\ Ik mag sęggjan iu, &
só hwe só hér gi·lôvid te mi \hld\ liudjo barno &
fasto undar þesumu folke, \hld\ þat imu þan flioten skulun &
fan is lík-hamon \hld\ libbjendi flód, &
irnandi water, \hld\ aho-spring mikil, &
kumad þanen kwika brunnon. \hld\ Þesa kwidi werðad wára, &
liudjun gi·lêstid, \hld\ só hwemu só hér gi·lôvid te mi.“ &
Þan mênde mid þiu wataru \hld\ waldandeo Krist, &
hêr heven-kuning \hld\ hêlagna gêst, &
hwó þene firiho barn \hld\ ant·fáhen skoldin, &
lioht ęndi listi \hld\ ęndi líf êwig, &
hôh heven-ríki \hld\ ęndi huldi godes. &
wurðun þó þea liudi \hld\ umbi þea lêra Kristes, &
umbi þiu word an ge·winne: \hld\ stódun wlanka man, &
gêl-móde Judeon, \hld\ sprákun gelp mikil, &
habdun it im te hoska, \hld\ kwaðun þat sie mahtin gi·hôrjen wel, &
þat imu mahlidin fram \hld\ módaga wihti, &
un·holde út: \hld\ „nu he an avu lêrid“, kwáðun sie, &
„wordu ge·hwi-liku.“ \hld\ Þó sprak eft þat werod ǫ́ðar: &
„ni þurvun gi þene lêrjand lahan“, \hld\ kwáðun sie: „kumad líves word &
mahtig fan is múde; \hld\ he wirkid manages hwat, &
wundres an þesaru wer-oldi: \hld\ nis þat wrêðaro dád, &
fíundo kraftes: \hld\ nio it þan te su·likaru frumu ni wurði, &
ak it gegnungo \hld\ fan gode alo-waldon, &
kumid fan is krafte. \hld\ Þat mugun gi ant·kęnnjen wel &
an þem is wárun wordun, \hld\ þat he gi·wald havad &
alles ovar erðu.“ \hld\ Þó weldun ina þe andsakon þar &
an stędi fáhen \hld\ efþa stên ana werpen, &
ef sie im þero manno \hld\ męnigi ni and-rédin, &
ni forhtodin þat folk-skępi. \hld\ Þó sprak þat friðu-barn godes: &
„ik tôgju iu gódes só filu“, \hld\ kwað he, „fan gode selvumu, &
wordo ęndi werko: \hld\ nu willjad gi mi wítnon hér &
þurh iuwan starkan hugi, \hld\ stên ana werpen, &
bi·lôsjen mi lívu.“ \hld\ Þó sprákun imu eft þea liudi an·gęgin, &
wrêða wiðer-sakon: \hld\ „ne wí it be þínun werkun ni duat“, kwáðun sia, &
„þat wí þi aldres \hld\ tó áhtjen willjad, &
ak wí duat it be þínun wordun, \hld\ hwand þú su·lik wáh sprikis, &
*hwand þú þik só máris \hld\ ęndi su·lik mên sagis, &
gihis for þeson Judeon, \hld\ þat þú sís god selvo, &
mahtig drohtin, \hld\ ęndi bist þi þoh man só wi, &
kuman fan þeson kunnje.“ \hld\ Krist alo-waldo &
ne wolda þero Judeono þuo lęng \hld\ gelpes hôrjan, &
wrêðaro willjon, \hld\ ak hie im af þem wíhe fuor &
ovar Jordanes strôm; \hld\ habda jungron mid im, &
þia is sáligun gi·sïðos, \hld\ þia im simlon mid im &
willjon wonodun: \hld\ suohta werod ǫ́ðer, &
deda þar só hie gi·wonoda, \hld\ drohtin selvo, &
lêrda þia liudi: \hld\ gi·lôvda þie wolda &
an is hêlagun word. \hld\ Þat skolda sinnon wel &
manno só hwi-likon, \hld\ só þat an is muod gi·nam. &
Þuo gi·frang ik þat þar te Kriste \hld\ kumana wurðun &%NOTE: gi·frang] Checked according to C.
bodon fan Bethaniu \hld\ ęndi sagdun þem barne godes, &
þat sia an þat ârundi þarod \hld\ idisi sęndin, &
Maria ęndi Marþa, \hld\ magað frí-líka, &
swíðo wun-sama wíf; \hld\ þia wissa hie bêðja, &
wárun im gi·swester twá, \hld\ þia hie selvo êr &
minnjoda an is muode \hld\ þuru iro mildjan hugi, &
þiu wíf þuru iro willjon guodan. \hld\ Sia im te wáron þuo &
an-budun fon Bethaniu, \hld\ þat iro bruoðer was &
Lazarus legar-fast \hld\ ęndi þat sia is líves ni wándun; &
bádun þat þarod kwámi \hld\ Krist alo-waldo &
hêlag te helpu. \hld\ Reht só hie sia gi·hôrda þuo &
sęggjan fan só siekon, \hld\ só sprak hie sán an·gęgin, &
kwað þat Lazaruses \hld\ legar ni wári &
gi·duan im te dôðe, \hld\ „ak þar skal drohtines lof“, kwat-hie, &
„gi·frumid werðan: \hld\ nis it im te ǫ́ðron frêson gi·duan.“ &
was im þar þuo selvo \hld\ suno drohtines &
twá naht ęndi dagas. \hld\ Þiu tíd was þuo ge·náhit, &
þat hie eft te Hjerusalem \hld\ Judeo liudjo &
wíson welda, \hld\ só hie gi·wald habda. &
Sagda þuo is gi·sïðon \hld\ suno drohtines, &
þat hie eft ovar Jordan \hld\ Judeo liudi &
suokjan welda. \hld\ Þuo sprákun im sán an·gęgin &
jungron sína: \hld\ „te hwí bist þú só gern þarod“, kwaðun sia, &
„frô mín, te faranne? \hld\ Ni þat nu furn ni was, &
þat sia þik þínero wordo \hld\ wítnon hogdun, &
weldun þi mid stênon starkan a·werpan? \hld\ nu þú eft undar þia strídigun þioda &
fundos te faranne, \hld\ þar ist fíondo ginuog, &
erlos ovar-muoda?“ \hld\ Þuo ên þero twe-livjo, &
Þuomas gi·málda \hld\ —was im gi·þungan mann, &
diur-lík drohtines þegạn—: \hld\ „ne skulun wí im þia dád lahan“, kwat-hie, &
„ni węrnjan wí im þes willjen, \hld\ ak wita im wonjan mid, &
þuolojan mid u̇sson þiodne: \hld\ þat ist þegnes kust, &
þat hie mid is fráhon samad \hld\ fasto gi·stande, &
dôje mid im þar an duome. \hld\ Duan u̇s alla só, &
folgon im te þero fęrdi: \hld\ ni látan u̇se ferah wið þiu &
wihtes wirðig, \hld\ neva wí an þem werode mid im, &
dôjan mid u̇son drohtine. \hld\ Þan lêvot u̇s þoh duom after, &
guod word for gumon.“ \hld\ Só wurðun þuo jungron Kristes, &
erlos aðal-borana \hld\ an ên-falden hugje, &
hêrren te willjen. \hld\ Þuo sagda hêlag Krist &
selvo is gi·sïðon \hld\ þat a·slápan was &
Lazarus fan þem legare, \hld\ „havit þit lioht a·gevan, &
an-swevit ist an selmon. \hld\ Nu wí an þena sïð faran &
ęndi ina a·wękkjan, \hld\ þat hie muoti eft þesa wer-old sehan, &
libbjandi lioht: \hld\ þan wirðit iuwa gi·lôvo after þiu &
forð-werd gi·fęstid.“ \hld\ Þuo gi·wêt hie im ovar þia fluod þanan, &
þie guodo godes suno, \hld\ anþat hie mid is jungron kwam &
þar te Bithaniu, \hld\ barn drohtines &
selvo mid is gi·sïðon, \hld\ þar þia gi·swester twá, &
Maria ęndi Marþa \hld\ an muod-karon &
sêraga sátun. \hld\ Was þar gi·samnot filo &
fan Hjerusalem \hld\ Judeo liudo, &
þia þiu *wíf weldun \hld\ wordun fruovrjan, &
þat sie só ni karodin \hld\ kind-jungas dôð, &
Lazaruses far·lust. \hld\ Só þó þe landes ward &
géng an þiu gardos, \hld\ só wurðun þes godes barnes &
kumi þar gi·ku̇ðid, \hld\ þat he só kraftig was &
bi þeru burg úten. \hld\ Þó im bêðjun was, &
þem wívun su·lik willjo, \hld\ þat sie im waldand tó, &
þat friðu-barn godes, \hld\ farandjen wissun. &
Þó þem wívun was \hld\ willjono mêsta &
kumi drohtines \hld\ ęndi Kristes word &
te gi·hôrjenne. \hld\ Heovandi géng &
Martha mód-karag \hld\ wið só mahtigne &
wordun wehslan \hld\ ęndi wið waldand sprak &
an iro hugi hriwig: \hld\ „þar þú mi, hêrro mín“, kwað siu, &
„nęrjendero bętst, \hld\ náhor wáris, &
hêljand þe gódo, \hld\ þan ni þorfti ik nu su·lik harm þolon, &
bittra breost-kara, \hld\ þan ni wári nu mín bróðer dôd, &
Lazarus fan þesumu liohte, \hld\ ak he imu mahti libbjen forð &
ferahes ge·fullid. \hld\ Ik þoh, frô mín, te þi &
liohto gi·lôvju, \hld\ lêrjandero bętst, &
só hwes só þú biddjen wili \hld\ berhton drohtin, &
þat he it þi sán far·givid, \hld\ god alo-mahtig, &
gi·werðot þínan willjan.“ \hld\ Þó sprak eft waldand Krist &
þeru idis and-wordi: \hld\ „ni lát þú þi an innan þes“, kwað he, &
„þínan sevon swerkan: \hld\ ik þi sęggjan mag &
wárun wordun, \hld\ þat þes nis gi·wand ênig, &
nevu þín bróðer skal \hld\ þurh gi·bod godes, &
þurh drohtines kraft \hld\ fan dôðe a·standen &
an is lík-hamon.“ \hld\ „All hębbju ik gi·lôvon só“, kwað siu, &
„þat it só gi·werðen skal, \hld\ só hwan só þius wer-old ęndjod &
ęndi þe márjo dag \hld\ ovar man fęrid, &
þat he þan fan erðu skal \hld\ up a·standen &
an þemu dómes daga, \hld\ þan werðad fan dôðe kwika &
þurh maht godes \hld\ man-kunnjes ge·hwi-lik, &
a·rísad fan restu.“ \hld\ Þó sagde ríkjo Krist &
þeru idis alo-mahtig \hld\ oponun wordun, &
þat he selvo was \hld\ sunu drohtines, &
bêðju ia líf ia lioht \hld\ liudjo barnon &
te a·standanne: \hld\ „nio þe sterven ni skal, &
líf far·liosen, \hld\ þe hér gi·lôvid te mi: &
þoh ina ęldi-barn \hld\ erðu bi·þękkjen, &
diapo bi·delven, \hld\ nis he dôd þiu mêr: &
þat flêsk is bi·folhen, \hld\ þat ferah is gi·halden, &
is þiu siola gi·sund.“ \hld\ Þó sprak imu eft sán an·gęgin &
þat wíf mid iro wordun: \hld\ „ik gi·lôvju þat þú þe wáro bist“, kwað siu, &
„Krist godes sunu: \hld\ þat mag man ant·kęnnjen wel, &
witen an þínun wordun, \hld\ þat þú gi·wald haves &
þurh þiu hêlagon gi·skapu \hld\ himiles ęndi erðun.“ &
Þó ge·fragn ik þat þar þero idisio kwam \hld\ ǫ́ðar gangan &
Maria mód-karag: \hld\ géngun iro managa aftar &
Judeo liudi. \hld\ Þó siu þemu godes barne &
sagde sêrag-mód, \hld\ hwat iru te sorgun gi·stód &
an iro hugi harmes: \hld\ hofnu kúmde &
Lazaruses far·lust, \hld\ liaves mannes, &
griat gornundi, \hld\ antat þemu godes barne &
hugi warð gi·hrórid: \hld\ hête trahni &
wópu a·wellun, \hld\ ęndi þó te þem wívun sprak, &
hét ina þó lêdjen, \hld\ þar Lazarus was &
foldu bi·folhen. \hld\ Lag þar ên felis bi·ovan, &
hard stên be·hliden. \hld\ Þó hét þe hêlago Krist &
ant·lúkan þea léia, \hld\ þat he mósti þat lík sehan, &
hrêo skawojen. \hld\ Þó ni mahte an iro hugi míðan &
Marþa for þeru męnegi, \hld\ wið mahtigne sprak: &
„frô mín þe gódo“, \hld\ kwað siu, „ef man þene felis nimid, &
þene stên ant·lúkid, \hld\ þan wániu ik þat þanen stank kume, &
un·swóti swek, \hld\ hwand ik þi sęggjan mag &
wárun wordun, \hld\ þat þes nis gi·wand ênig, &
þat he þar nu bi·folhen was \hld\ fiuwar naht ęndi dagos &
an þemu erð-grave.“ \hld\ And-wordi gaf &
waldand þemu wíve: \hld\ „hwat, ni sagde ik þi te wárun êr“, kwað he, &
„ef þú gi·lôvjen wili, \hld\ þan nis nu lang te þiu, &
þat þú hér ant·kęnnjen skalt \hld\ kraft drohtines, &
þe mikilon maht godes?“ \hld\ Þó géngun manage tó, &
af·hóvun harden stên. \hld\ Þó sah þe hêlago Krist &
up mid is ôgun, \hld\ ǫ́·lát sagde &%NOTE: ǫ́·lát = álát
þemu þe þese wer-old gi·skóp, \hld\ „þes þú mín word gi·hôris“, kwað he, &
„sigi-drohtin selvo; \hld\ ik wêt þat þú só simlun duos, &
ak ik duom it be þesumu grôton \hld\ Judeono folke, &
þat sie þat te wárun witin, \hld\ þat þú mi an þese wer-old sęndes &
þesun liudjun te lêrun.“ \hld\ Þó he te Lazaruse hriop &
starkaru stemnju \hld\ ęndi hét ina standen up &
ia fan þemu grave gangan. \hld\ Þó warð þe gêst kumen &
an þene lík-hamon: \hld\ he bi·gan is liði hrórjen, &
ant·warp undar þemu gi·wę́dje: \hld\ was imo só be·wunden þó noh, &
an hrêo-będdjon bi·helid. \hld\ Hét imu helpen þó &
waldandeo Krist. \hld\ Weros géngun tó, &
ant·wundun þat ge·wádi. \hld\ Wánum up a·rês &
Lazarus te þesumu liohte: \hld\ was imu is líf far·geven, &
þat he is aldar-lagu \hld\ êgan mósti, &
friðu forð-wardes. \hld\ Þó fagonadun bêðja, &
Maria ęndi Marþa: \hld\ ni mag þat man ǫ́ðrumu &
gi·sęggjan te sǫ́ðe, \hld\ hwó þea ge·swester twó &
męndjodun an iro móde. \hld\ Maneg wundrode &
Judeo liudjo, \hld\ þó sie ina fan þemu grave sáhun &
sïðon ge·sunden, \hld\ þene þe êr suht far·nam &
ęndi sie bi·dulvun \hld\ diapo undar erðu &
líves lôsen: \hld\ þó móste imu libbjen forð &
hêl an hêmun. \hld\ Só mag heven-kuninges, &
þiu mikile maht godes \hld\ manno ge·hwi-likes &
ferahe gi·formon \hld\ ęndi wið fíundo níð &
hêlag helpen, \hld\ só hwemu só he is huldi far·givid &
Þó warð þar só managumu manne \hld\ mód aftar Kriste, &
gi·hworven hugi-skęfti, \hld\ sïðor sie is hêlagon werk &
selvon gi·sáhun, \hld\ hwand eo êr su·lik ni warð &
wunder an wer-oldi. \hld\ Þan was eft þes werodes só filu, &
só mód-starke man: \hld\ ni weldon þe maht godes &
ant·kęnnjen ku̇ð-líko, \hld\ ak sie wið is kraft mikil &
wunnun mid iro wordun: \hld\ wárun im waldandes &
lêra so lêða: \hld\ sóhtun im liudi ǫ́ðra &
an Hjerusalem, \hld\ þar Judeono was &
hêri hand-mahal \hld\ ęndi hôvid-stędi, &
grôt gum-skępi \hld\ grimmaro þioda. &
Sie ku̇ðdun im þó Kristes werk, \hld\ kwáðun þat sie kwikan sáhin &
þene erl mid iro ôgun, \hld\ þe an erðu was, &
foldu bi·folhen \hld\ fiuwar naht ęndi dagos, &
dôd bi·dolven, \hld\ antat he ina mid is dádjun selvo, &
mid is wordun a·wękide, \hld\ þat he mósti þese wer-old sehan. &
Þó was þat só wiðer-ward \hld\ wlankun mannun, &
Judeo liudjun: \hld\ hétun iro gum-skępi þó, &
werod samnojan \hld\ ęndi warvos fáhen, &
męgin-þioda gi·mang, \hld\ an mahtigna Krist &
riedun an rúnun: \hld\ „nis þat rád ênig“, kwáðun sie, &
„þat wí þat gi·þolojan: \hld\ wili þesaro þioda te filu &
gi·lôvjen aftar is lêrun. \hld\ Þan u̇s liudi farad, &
an eo-rid-folk, \hld\ werðat u̇sa ovar-hôvdun &
rinkos fan Rúmu. \hld\ Þan wí þeses ríkjes skulun &
lôse libbjen \hld\ efþa wí skulun u̇ses líves þolon, &
hęliðos u̇saro hôvdo.“ \hld\ Þó sprak þar ên gi·hêrod man &
ovar warf wero, \hld\ þe was þes werodes þó &
an þeru burg innan \hld\ biskop þero liudjo &
—Kaiphas was he hêten; \hld\ habdun ina gi·koranen te þiu &
an þeru gę́r-talu \hld\ Judeo liudi, &
þat he þes godes húses \hld\ gômjen skoldi, &
wardon þes wíhes—: \hld\ „mi þunkid wunder mikil“, kwað he, &
„mári þioda, \hld\ —gí kunnun manages gi·skêð— &
hwí gí þat te wárun ni witin, \hld\ werod Judeono, &
þat hér is bętera rád \hld\ barno ge·hwi-likumu, &
þat man hér ênne man \hld\ aldru bi·lôsje &
ęndi þat he þurh iuwa dádi \hld\ drôreg sterve, &
for þesumu folk-skępi \hld\ ferah far·láte, &
þan al þit liud-werod \hld\ far·loren werðe.“ &
Ni was it þoh is willjan, \hld\ þat he só wár ge·sprak, &
só forð for þemu folke, \hld\ frume man-kunnjes &
gi·mênde for þeru męnegi, \hld\ ak it kwam imu fan þeru maht godes &
þurh is hêlagan hêd, \hld\ hwand he þat hús godes &
þar an Hjerusalem \hld\ bi·gangan skolde, &
wardon þes wíhes: \hld\ be·þiu he só wár gi·sprak, &
biskop þero liudjo, \hld\ hwó skoldi þat barn godes &
alla irmin-þiod \hld\ mid is ênes ferhe, &
mid is lívu a·lôsjen: \hld\ þat was allaro þesaro liudjo rád, &
hwand he gi·halode \hld\ mid þiu hêðina liudi, &
weros an is willjon \hld\ waldandio Krist. &
Þó wurðun ên-wordje \hld\ ovar-módje man, &
werod Judeono, \hld\ ęndi an iro warve gi·sprákun, &
mári þioda, \hld\ þat sie im ni létin iro mód twehon: &
só hwe só ina undar þemu folke \hld\ finden mahti, &
þat ina sán gi·féngi \hld\ ęndi forð bráhti &
an þero þiodo þing; \hld\ kwáðun þat sie ni mahtin gi·þolojan lęng, &
þat sie þe êno man \hld\ só alla weldi, &
werod far·winnen. \hld\ Þan wisse waldand Krist &
þero manno só garo \hld\ mód-gi·þáhti, &
hęti-grimmon hugi, \hld\ hwand imu ni was bi·holen eo·wiht &
an þesaru middil-gard: \hld\ he ni welde þó an þie męnigi innen &
sïður open-líko, \hld\ under þat erlo folk, &
gangan under þea Judeon: \hld\ bêd þe godes sunu &
þero torohtjon tíd, \hld\ þe imu tó·ward was, &
þat he far þesa þioda \hld\ þolojan welde, &
far þit werod wíti: \hld\ wisse imu selvo &
þat dag-þingi garo. \hld\ Þó gi·wêt imu u̇se drohtin forð &
ęndi imu þó an Effrem \hld\ alo-waldo Krist &
an þeru hôhon burg \hld\ hêlag drohtin &
wunode mid is werodu, \hld\ antat he an is willjan hwarf &
eft te Bethania \hld\ brahtmu þiu mikilun, &
mid þiu is gódum gum-skępi. \hld\ Judeon bi·sprákun þat &
wordu ge·hwi-liku, \hld\ þó sie imu su·lik werod mikil &
folgon gi·sáhun: \hld\ „nis frume ênig“, kwáðun sie, &
„u̇ses ríkjes gi·rádi, \hld\ þoh wí reht sprekan, &
ni þíhit u̇ses þinges wiht: \hld\ þius þiod wili &
węndjen after is willjan; \hld\ imu all þius wer-old folgot, &
liudi bi þem is lêrun, \hld\ þat wí imu lêðes wiht &
for þesumu folk-skępi \hld\ gi·frummjen ni mótun.“ &
Gi·wêt imu þó þat barn godes \hld\ innan Bethania &
sehs nahtun êr, \hld\ þan þiu samnunga &
þar an Hjerusalem \hld\ Judeo liudjo &
an þem wíh-dagun \hld\ werðen skolde, &
þat sie skoldun haldan \hld\ þea hêlagon tídi, &
Judeono paskha. \hld\ Béd þe godes sunu, &
mahtig under þeru męnegi: \hld\ was þar manno kraft, &
werodes bi þem is wordun. \hld\ Þar géngun ina twê wíf umbi, &
Maria ęndi Marþa, \hld\ mid mildju hugi, &
þionodun imu þeo-líko. \hld\ Þiodo drohtin &
gaf im lang-sam lôn: \hld\ lét sea lêðes gi·hwes, &
sundjono sikora, \hld\ ęndi selvo gi·bôd, &
þat sea an friðe fórin \hld\ wiðer fíundo níð, &
þea idisa mid is orlovu gódu: \hld\ habdun iro ambaht-skępi &
bi·węndid an is willjon. \hld\ Þó gi·wêt imu waldand Krist &
forð mid þiu folku, \hld\ firiho drohtin, &
innan Hjerusalem, \hld\ þar Judeono was &
hete-lík hard-buri, \hld\ þar sie þea hêlagon tíd &
warodun at þemu wíhe; \hld\ was þar werodes só filu, &
kraftigaro kunnjo, \hld\ þie ni weldun Kristes word &
gerno hôrjen \hld\ ni te þemu godes barne &
an iro mód-sevon \hld\ minnje ni habdun, &
ak wárun im só wrêða \hld\ wlanka þioda, &
módeg man-kunni, \hld\ habdun im morð-hugi, &
in·wid an innan: \hld\ an avuh far·féngun &
Kristes lêre, \hld\ weldun ina kraftigna &
wítnon þero wordo; \hld\ ak was þar werodes só filu, &
umbi erl-skępi \hld\ ant·langana dag, &
habde ine þiu smale þiod \hld\ þurh is swótjun word &
werodu bi·worpen, \hld\ þat ine þie wiðer-sakon &
under þemu folk-skępi \hld\ fáhen ne gi·dorstun, &
ak miðun is bi þeru męnegi. \hld\ Þan stód mahtig Krist &
an þemu wíhe innan, \hld\ sagde word manag &
firiho barnun te frumu. \hld\ Was þar folk umbi &
allan langan dag, \hld\ antat þiu liohte gi·wêt &
sunne te sedle. \hld\ Þó te seliðun fór &
man-kunnjes manag. \hld\ Þan was þar ên mári berg &
bi þeru burg úten, \hld\ þe was brêd ęndi hôh, &
gróni ęndi skóni: \hld\ hétun ina Judeo liudi &
Oliueti bi namon. \hld\ Þar imu up gi·wêt &
nęrjendjo Krist, \hld\ só ina þiu naht bi·féng, &
was imu þar mid is jungarun, \hld\ só ine þar Judeono ênig &
ni wisse ti wárun, \hld\ hwand he an þemu wíhe stód, &
liudjo drohtin, \hld\ só lioht óstene kwam, &
ant·féng þat folk-skępi \hld\ ęndi im filu sagde &
wároro wordo, \hld\ só nis an þesaru wer-oldi ênig, &
an þesaru middil-gard \hld\ manno só spáhi, &
liudjo barno nig·ên, \hld\ þat þero lêrono mugi &
ęndi gi·tęlljen, \hld\ þe he þar an þemu alahe gi·sprak, &
waldand an þemu wíhe, \hld\ ęndi simlun mid is wordun gi·bôd, &
þat sie sie gerewidin \hld\ te godes ríkje, &
allaro manno ge·hwi-lik, \hld\ þat sie móstin an þemu márjon daga &
iro drohtines \hld\ diuriða ant·fáhen. &
Sagde im hwat sie it sundjun frumidun \hld\ ęndi simlun gi·bôd, &
þat sie þea a·lęskidin; \hld\ hét sie lioht godes &
minnjon an iro móde, \hld\ mên far·láten, &
avoha ovar-hugdi, \hld\ ôd-módi niman, &
hlaðen þat an iro hertan; \hld\ kwað þat im þan wári heven-ríki, &
garu gódo mêst. \hld\ Þó warð þar gumono só filu &
gi·węndid aftar is willjon, \hld\ sïður sie þat word godes &
hêlag gi·hôrdun, \hld\ heven-kuninges, &
ant·kęndun kraft mikil, \hld\ kumi drohtines, &
hêrron helpe, \hld\ ia þat heven-ríki was, &
nęrjendi gi·náhid \hld\ ęndi náða godes &
manno barnun. \hld\ Sum só módeg was &
Judeo folkes, \hld\ habdun grimman hugi, &
slíð-móden sevon \hld\ {[...]}, &
ni weldun is worde gi·lôvjen, \hld\ ak habdun im ge·win mikil &
wið þea Kristes kraft: \hld\ kumen ni móstun &
þea liudi þurh lêðen stríd, \hld\ þat sie gi·lôvon te imu &
fasto gi·féngin; \hld\ ni was im þiu frume giviðig, &
þat sie heven-ríki \hld\ habbjen móstin. &
Géng imu þó þe godes sunu \hld\ ęndi is jungaron mid imu, &
waldand fan þemu wíhe, \hld\ all só is willjo géng, &
iak imu uppen þene berg gi·stêg \hld\ barn drohtines: &
sat imu þar mid is ge·sïðun \hld\ ęndi im sagde filu &
wároro wordo. \hld\ Sí bi·gunnun im þó umbi þene wíh sprekan, &
þie gumon umbi þat godes hús, \hld\ kwáðun þat ni wári gód-líkora &
alah ovar erðu \hld\ þurh erlo hand, &
þurh mannes gi·werk \hld\ mid męgin-kraftu &
rakud a·rihtid. \hld\ Þó þe ríkjo sprak, &
hêr heven-kuning \hld\ —hôrdun þe ǫ́ðra—: &
„ik mag iu gi·tęlljen“, \hld\ kwað he, „þat noh wirðid þiu tíd kumen, &
þat is af·standen ni skal \hld\ stên ovar ǫ́ðrumu, &
ak it fallid ti foldu \hld\ ęndi fiur nimid, &
grádag logna, \hld\ þoh it nu só gód-lík sí, &
só wís-líko gi·warht, \hld\ ęndi só dód all þesaro wer-oldes gi·skapu, &
te·glídid gróni wang.“ \hld\ Þó géngun imu is jungaron tó, &
frágodun ina só stillo: \hld\ „hwó lango skal standen noh“, kwáðun sie, &
„þius wer-old an wunnjun, \hld\ êr þan þat gi·wand kume, &
þat þe lasto dag \hld\ liohtes skíne &
þurh wolkan-skion, \hld\ efþo hwan is þín eft wán kumen &
an þene middil-gard, \hld\ manno kunnje &
te a·dêljenne, \hld\ dôdun ęndi kwikun? &
frô mín þe gódo, \hld\ u̇s is þes firi-wit mikil, &
waldandeo Krist, \hld\ hwan þat gi·werðen skuli.“ &
Þó im and-wordi \hld\ alo-waldo Krist &
gód-lík far·gaf \hld\ þem gumun selvo: &
„þat havad só bi·dęrnid“, \hld\ kwað he, „drohtin þe gódo, &
iak só hardo far·holen \hld\ himil-ríkjes fader, &
waldand þesaro wer-oldes, \hld\ só þat witen ni mag &
ênig mannisk barn, \hld\ hwan þiu márje tíd &
gi·wirðid an þesaru wer-oldi, \hld\ ne it ôk te wáran ni kunnun &
godes ęngilos, \hld\ þie for imu gęgin-warde &
simlun sindun: \hld\ sie it ôk gi·sęggjan ni mugun &
te wáran mid iro wordun, \hld\ hwan þat gi·werðen skuli, &
þat he willje an þesan middil-gard, \hld\ mahtig drohtin, &
firiho fandon. \hld\ Fader wêt it êno &
hêlag fan himile: \hld\ elkur is it bi·holen allun, &
kwikun ęndi dôdun, \hld\ hwan is kumi werðad, &
Ik mag iu þoh gi·tęlljen, \hld\ hwi-lik hér têkạn bi·foran &
gi·werðad wunder-lík, \hld\ êr þan he an þese wer-old kume &
an þemu márjon daga: \hld\ þat wirðid hér êr an þemu mánon skín &
iak an þeru sunnon só same; \hld\ gi·swerkad siu bêðju, &
mid finistre werðad bi·fangan; \hld\ fallad sterron, &
hwít heven-tungal, \hld\ ęndi hrisid erðe, &
bivod þius brêde wer-old \hld\ —wirðid su·likaro bókno filu—: &
grimmid þe grôto sêo, \hld\ wirkid þie gevenes strôm &
ęgison mid is u̇ðjun \hld\ erð-búandjun. &
Þan þorrot þiu þiod \hld\ þurh þat ge·þwing mikil, &
folk þurh þea forhta: \hld\ þan nis friðu hwęrgin, &
ak wirðid wíg só maneg \hld\ ovar þese wer-old alla &
hete-lík af·haben, \hld\ ęndi hęri lêdid &
kunni ovar ǫ́ðar: \hld\ wirðid kuningo gi·win, &
męgin-fard mikil: \hld\ wirðid managoro kwalm, &
open ur-lagi \hld\ —þat is ęgis-lík þing, &
þat io su·lik morð \hld\ skulun man af·hębbjen—, &
wirðid wól só mikil \hld\ ovar þese wer-old alle, &
man-stervono mêst, \hld\ þero þe gio an þesaru middil-gard &
swulti þurh suhti: \hld\ liggjad seoka man, &
driosat ęndi dôjat \hld\ ęndi iro dag ęndjad, &
fulljad mid iro ferahu; \hld\ fęrid un·met grôt &
hungar hęti-grim \hld\ ovar hęliðo barn, &
męti-gêdjono mêst: \hld\ nis þat minniste &
þero wítjo an þesaru wer-oldi, \hld\ þe hér gi·werðen skulun &
êr dómes dage. \hld\ Só hwan só gi þea dádi gi·sehan &
gi·werðen an þesaru wer-oldi, \hld\ só mugun gi þan te wáran far·standen, &
þat þan þe latsto dag \hld\ liudjun náhid &
mári te mannun \hld\ ęndi maht godes, &
himil-kraftes hróri \hld\ ęndi þes hêlagon kumi, &
drohtines mid is diuriðun. \hld\ Hwat, gi þesaro dádjo mugun &
bi þesun bômun \hld\ biliði ant·kęnnjen: &
þan sie brustjad ęndi blójat \hld\ ęndi bladu tôgjat, &
lóf ant·lúkad, \hld\ þan witun liudjo barn, &
þat þan is sán after þiu \hld\ sumer gi·náhid &
warm ęndi wun-sam \hld\ ęndi weder skóni. &
Só witin gi ôk bi þesun têknun, \hld\ þe ik iu talde hér, &
hwan þe latsto dag \hld\ liudjun náhid. &
Þan sęggjo ik iu te wáran, \hld\ þat êr þit werod ni mót, &
te·faran þit folk-skępi, \hld\ êr þan werðe ge·fullid só, &
mínu word gi·wárod. \hld\ Noh gi·wand kumid &
himiles ęndi erðun, \hld\ ęndi stéid mín hêlag word &
fast forð-wardes \hld\ ęndi wirðid al ge·fullod só, &
gi·lêstid an þesumu liohte, \hld\ só ik for þesun liudjun ge·spriku. &
wakot gi war-líko: \hld\ iu is wis-kumo &
duom-dag þe márjo \hld\ ęndi iuwes drohtines kraft, &
þiu mikilo męgin-strengi \hld\ ęndi þiu márje tíd, &
gi·wand þesaro wer-oldes. \hld\ Fora þiu gi wardon skulun, &
þat he iu slápandje \hld\ an swef-restu &
fárungo ni bi·fáhe \hld\ an firin-werkun, &
mênes fulle. \hld\ Mútspelli kumit &
an þiustrja naht, \hld\ al só þiof fęrid &
darno mid is dádjun, \hld\ só kumid þe dag mannun, &
þe latsto þeses liohtes, \hld\ só it êr þese liudi ni witun, &
só samo só þiu flód deda \hld\ an furn-dagun, &
þe þar mid lagu-strômun \hld\ liudi far·tęride &
bi Nóeas tídjun, \hld\ bi·útan þat ina nęride god &
mid is híwiskja, \hld\ hêlag drohtin, &
wið þes flódes farm: \hld\ só warð ôk þat fiur kuman &
hêt fan himile, \hld\ þat þea hôhon burgi &
umbi Sodomo land \hld\ swart logna bi·féng &
grim ęndi grádag, \hld\ þat þar n·ênig gumono ni gi·nas &
bi·útan Loth êno: \hld\ ina ant·lêddun þanen &
drohtines ęngilos \hld\ ęndi is dohter twá &
an ênan berg uppen: \hld\ þat ǫ́ðar al brinnandi fiur, &
ia land ia liudi \hld\ logna far·tęride: &
só fárungo warð þat fiur kumen, \hld\ só warð êr þe flód só samo: &
só wirðid þe latsto dag. \hld\ For þiu skal allaro liudjo ge·hwi-lik &
þęnkjan fora þemu þinge; \hld\ þes is þarf mikil &
manno ge·hwi-likumu: \hld\ be·þiu látad iu an iuwan mód sorga. &
Hwand só hwan só þat ge·wirðid, \hld\ þat waldand Krist, &
mári mannes sunu \hld\ mid þeru maht godes, &
kumit mid þiu kraftu \hld\ kuningo ríkjost &
sittjan an is selves maht \hld\ ęndi samod mid imu &
alle þea ęngilos, \hld\ þe þar uppa sind &
hêlaga an himile, \hld\ þan skulun þarod hęliðo barn, &
ęli-þeoda kuman \hld\ alla te·samne &
libbjandero liudjo, \hld\ só hwat só io an þesumu liohte warð &
firiho a·fódid. \hld\ Þar he þemu folke skal, &
allumu man-kunnje \hld\ mári drohtin &
a·dêljen aftar iro dádjun. \hld\ Þan skêðid he þea far·duanan man, &
þea far·warhton weros \hld\ an þea winistron hand: &
só duot he ôk þea sáligon \hld\ an þea swíðeron half; &
grótid he þan þea gódun \hld\ ęndi im te·gęgnes sprikid: &
„kumad gí“, kwiðid he, „þea þar gi·korene sindun, \hld\ ęndi ant·fáhad þit kraftiga ríki, &
þat góde, þat þar gi·gerewid stęndid, \hld\ þat þar warð gumono barnun &
gi·warht fan þesaro wer-oldes ęndje: \hld\ iu havad ge·wíhid selvo &
fader allaro firiho barno: \hld\ gí mótun þesaro frumono neotan, &
ge·waldon þeses wídon ríkjas, \hld\ hwand gí oft mínan willjon frumidun, &
ful-géngun mí gerno \hld\ ęndi wárun mí iuwaro gevo mildje, &
þan ik bi·þwungan was \hld\ þurstu ęndi hungru, &
frostu bi·fangan \hld\ efþo an feteron lag, &
bi·klemmid an karkare: \hld\ oft wurðun mí kumana þarod &
helpa fan iuwun handun: \hld\ gí wárun mí an iuwomu hugi mildje, &
wísodun mín werð-liko.“ \hld\ Þan sprikid imu eft þat werod an·gęgin: &
„frô mín þe gódo“, \hld\ kweðat sie, „hwan wári þú bi·fangan só, &
be·þwungan an su·likun þarạvun, \hld\ só þú fora þesaru þiod tęlis, &
mahtig mênis? \hld\ Hwan gi·sah þí man ênig &
be·þwungen an su·likun þarạvun? \hld\ Hwat, þú haves allaro þiodo gi·wald &
iak só samo þero mêðmo, \hld\ þero þe io manno barn &
ge·wunnun an þesaro wer-oldi.“ \hld\ Þan sprikid im eft waldand god: &
„só hwat só gí dádun“, \hld\ kwiðit he, „an iuwes drohtines namon, &
gódes far·gávun \hld\ an godes êra &
þem mannun, þe hér minniston sindun, \hld\ þero nu undar þesaru męnegi standad &
ęndi þurh ôd-módi \hld\ arme wárun &
weros, hwand sie mínan willjon fręmidun \hld\ —só hwat só gí im iuwaro welono far·gávun, &
gi·dádun þurh diuriða, \hld\ þat ant·féng iuwa drohtin selvo, &
þiu helpe kwam te heven-kuninge. \hld\ Be·þiu wili iu þe hêlago drohtin &
lônon iuwan gi·lôvon: \hld\ givid iu líf êwig.“ &
Węndid ina þan waldand \hld\ an þea winistron hand, &
drohtin te þem far·duanun mannun, \hld\ sagad im þat sie skulin þea dád ant·gelden, &
þea man iro mên-gi·werk: \hld\ „nu gí fan mí skulun“, kwiðit he, &
„faran só for·flókane \hld\ an þat fiur êwig, &
þat þar gi·garewid warð \hld\ godes and-sakun, &
fíundo folke \hld\ be firin-werkun, &
hwand gí mí ni hulpun, \hld\ þan mí hunger ęndi þurst &
wêgde te wundrun \hld\ efþa ik ge·wádjes lôs &
géng jámer-mód, \hld\ was mí grôtun þarf, &
þan ni habde ik þar ênige helpe, \hld\ þan ik ge·hęftid was, &
an liðo-kospun bi·lokan, \hld\ efþa mi legar bi·féng, &
swára suhti: \hld\ þan ni weldun gí mín siokes þar &
wíson mid wihti: \hld\ ni was iu werð eo·wiht, &
þat gí mín ge·hugdin. \hld\ Be·þiu gí an hęllje skulun &
þolon an þiustre.“ \hld\ Þan sprikid imu eft þiu þiod an·gęgin: &
„wola waldand god“, \hld\ kweðad sie, „hwí wilt þú só wið þit werod sprekan, &
mahljen wið þese męnegi? \hld\ Hwan was þi io manno þarf, &
gumono gódes? \hld\ Hwat, sie it al be þínun gevun êgun, &
welon an þesaro wer-oldi“. \hld\ Þan sprikid eft waldand god: &
„þan gí þea armostun“, \hld\ kwiðid he, „ęldi-barno, &
manno þea minniston \hld\ an iuwomu mód-sevon &
hęliðos far·hugdun, \hld\ létun sea iu an iuwomu hugi lêðe, &
be·dêldun sie iuwaro diurða, \hld\ þan dádun gí iuwana drohtin só sama, &
gi·węrnidun imu iuwaro welono: \hld\ be·þiu ni wili iu waldand god, &
ant·fáhen fader iuwa, \hld\ ak gí an þat fiur skulun, &
an þene diopun dôð, \hld\ diuvlun þionon, &
wrêðun wiðer-sakun, \hld\ hwand gí só warhtun bi·foran.“ &
Þan aftar þem wordun skêðit \hld\ þat werod an twê, &
þea gódun ęndi þea uvilon: \hld\ farad þea far·griponon man &
an þea hêtan hęl \hld\ hriwig-móde, &
þea far·warhton weros, \hld\ wíti ant·fáhat, &
uvil ęndi-lôs. \hld\ Lêdid up þanen &
hêr heven-kuning \hld\ þea hluttaron þeoda &
an þat lang-same lioht: \hld\ þar is líf êwig, &
gi·garewid godes ríki \hld\ gódaro þiado.“ &
Só ge·fragn ik þat þem rinkun þó \hld\ ríki drohtin &
umbi þesaro wer-oldes gi·wand \hld\ wordun talde, &
hwó þiu forð fęrid, \hld\ þan lango þe sie firiho barn &
ardon mótun, \hld\ ia hwó siu an þemu ęndje skal &
te·glíden ęndi te·gangen. \hld\ He sagde ôk is jungarun þar &
wárun wordun: \hld\ „hwat, gí witun alle“, kwað he, &
„þat nu ovar twá naht \hld\ sind tídi kumana, &
Gjudeono paskha, \hld\ þat sie skulun iro gode þionon, &
weros an þemu wíhe. \hld\ Þes nis ge·wand ênig, &
þat þar wirðid mannes sunu \hld\ te þeru męgin-þiodu &
kraftag far·kôpot \hld\ ęndi an krúke a·slagan, &
þolod þiad-kwála.“ \hld\ Þó warð þar þegạn manag &
slíð-mód gi·samnod, \hld\ su̇ðar-liudjo, &
Judeono gum-skępi, \hld\ þar sie skoldun iro gode þionon. &
wurðun êo-sagon \hld\ alle kumane, &
an warf weros, \hld\ þe sie þó wísostun &
undar þeru męnegi \hld\ manno taldun, &
kraftag kuni-burd. \hld\ Þar Kaiphas was, &
biskop þero liudjo. \hld\ Sie rédun þó an þat barn godes, &
hwó sie ina a·sluogin \hld\ sundja lôsan, &
kwáðun þat sie ina an þemu hêlagon daga \hld\ hrínen ni skoldin &
undar þero manno męnegi, \hld\ „þat ni werðe þius męgin-þioda, &
hęliðos an hróru, \hld\ hwand ina þit hęri-skępi wili &
far·standen mid strídu. \hld\ Wí só stillo skulun &
frêson is ferahes, \hld\ þat þit folk Judeono &
an þesun wíh-dagun \hld\ wróht ni af·hębbjen.“ &
Þó géng imu þar Júdas forð, \hld\ jungaro Kristes, &
ên þero twe-livjo, \hld\ þar þat aðali sat, &
Judeono gum-skępi; \hld\ kwað þat he is im gódan rád &
sęggjan mahti: \hld\ „hwat willjad gí mí sęlljen hér“, kwað he, &
„mêðmo te médu, \hld\ ef ik iu þene man givu &
áno wíg ęndi áno wróht?“ \hld\ Þó warð þes werodes hugi, &
þero liudjo an lustun: \hld\ „ef þú wili gi·lêstjen só“, kwáðun sie, &
„þín word gi·wáron, \hld\ þan þú gi·wald haves, &
hwat þú at þesaru þiodu \hld\ þiggjan willjes &
gódaro mêðmo.“ \hld\ Þó gi·hét imu þat gum-skępi þar &
an is selves dóm \hld\ siluvar-skatto &
þrí-tig at-samne, \hld\ ęndi he te þeru þiodu gi·sprak &
dereveun wordun, \hld\ þat he gávi is drohtin wið þiu. &
wende ina þó fan þemu werode: \hld\ was im wrêð hugi, &
talode im só treu-lôs, \hld\ hwan êr wurði imu þiu tíd kuman, &
þat he ina mahti far·wísjen \hld\ wrêðaro þiodo, &
fíundo folke. \hld\ Þan wisse þat friðu-barn godes, &
wár waldand Krist, \hld\ þat he þese wer-old skolde, &
a·geven þese gardos \hld\ ęndi sókjen imu godes ríki, &
gi·faren is fader-oðil. \hld\ Þó ni gi·sah ênig firiho barno &
méron minnje, \hld\ þan he þó te þem mannun gi·nam, &
te þem is gódun jungaron: \hld\ gôme warhte, &
sętte sie swás-líko \hld\ ęndi im sagde filu &
wároro wordo. \hld\ Skrêd wester dag, &
sunne te sedle. \hld\ Þó he selvo gi·bôd, &
waldand mid is wordun, \hld\ hét im water dragan &
hluttar te handun, \hld\ ęndi rês þó þe hêlago Krist, &
þe gódo at þem gômun \hld\ ęndi þar is jungarono þwóg &
fóti mid is folmun \hld\ ęndi swarf sie mid is fanon aftar, &
druknide sie diur-líka. \hld\ Þó wið is drohtin sprak &
Símon Petrus: \hld\ „ni þunkid mi þit sómi þing“, kwað he, &
„frô mín þe gódo, \hld\ þat þú míne fóti þwahes &
mid þem þínun hêlagun handun.“ \hld\ Þó sprak imu eft is hêrro an·gęgin, &
waldand mid is wordun: \hld\ „ef þú is willjan ni haves“, kwað he, &
„te ant·fáhanne, \hld\ þat ik þíne fóti þwahe &
þurh su·lika minnja, \hld\ só ik þesun ǫ́ðrun mannun hér &
dóm þurh diurða, \hld\ þan ni haves þú ênigan dêl mid mi &
an heven-ríkja.“ \hld\ Hugi warð þó gi·węndid &
Símon Petruse: \hld\ „þú hava þi selvo gi·wald“, kwað he, &
„frô mín þe gódo, \hld\ fóto ęndi hando &
ęndi mínes hôvdes só sama, \hld\ handun þínun, &
þiadan, te þwahanne, \hld\ te þiu þak ik móti þína forð &
huldi hębbjan \hld\ ęndi heven-ríkjes &
su·lik gi·dêli, \hld\ só þú mi, drohtin, wili &
far·geven þurh þína gódi.“ \hld\ Jungaron Kristes, &
þene ambaht-skępi \hld\ erlos þolodun, &
þegnos mid gi·þuldjon, \hld\ só hwat só im iro þiodan dede, &
mahtig þurh þea minnja, \hld\ ęndi mênde imu al méra þing &
firihon te gi·frummjenne. \hld\ friðu-barn godes &
géng imu þó eft gi·sittjen \hld\ under þat ge·sïðo folk &
ęndi im sagda filu lang-samna rád. \hld\ Warð eft lioht kuman, &
morgen te mannun. \hld\ Mahtigne Krist &
gróttun is jungaron ęndi frágodun, \hld\ hwar sie is gôma þó &
an þemu wíh-dage \hld\ wirkjen skoldin, &
hwar he weldi halden \hld\ þea hêlagon tídi &
selvo mid is ge·sïðun. \hld\ Þó he sie sókjen hét, &
þea gumon Hjerusalem: \hld\ „só gí þan gangan kumad“, kwað he, &
„an þea burg innan \hld\ —þar is braht mikil, &
męgin-þiodo gi·mang—, \hld\ þar mugun gí ênan man sehan &
an is handun dragen \hld\ hluttres watares &
ful mid folmun. \hld\ Þemu gí folgon skulun &
an só hwi-like gardos, \hld\ só gí ina gangan gi·sehat, &
ia gí þan þemu hêrron, \hld\ þe þie hovos êgi, &
selvon sęggjad, \hld\ þat ik iu sęnde þarod &
te gi·garuwenne mína gôma. \hld\ Þan tôgid he iu ên gód-lík hús, &
hôhan soleri, \hld\ þe is bi·hangen al &
fagarun fratahun. \hld\ Þar gí frummjen skulun &
werd-skępi mínan. \hld\ Þar bium ik wis·kumo &
selvo mid mínun ge·sïðun.“ \hld\ Þó wurðun sán aftar þiu &
þar te Hjerusalem \hld\ jungaron Kristes &
forð-ward an fęrdi, \hld\ fundun all só he sprak &
word-têkạn wár: \hld\ ni was þes gi·wand ênig. &
Þar gerewidun sie þea gôma. \hld\ Warð þe godes sunu, &
hêlag drohtin \hld\ an þat hús kuman, &
þar sie þe land-wíse \hld\ lêstjen skoldun, &
ful-gangan godes gi·bode, \hld\ al só Judeono was &
êo ęndi ald-sidu \hld\ an êr-dagun. &
Gi·wêt imu þó an þemu ávande \hld\ alo-waldand Krist &
an þene sęli sittjen; \hld\ hét þar is ge·sïðos te imu &
twe-livi gangan, \hld\ þea im gi·triwiston &
an iro mód-sevon \hld\ manno wárun &
bi wordun ęndi bi wísun: \hld\ wisse imu selvo &
iro hugi-skęfti \hld\ hêlag drohtin. &
Grótte sie þó ovar þem gômun: \hld\ „gern bium ik swíðo“, kwað he, &
„þat ik samad mid iu \hld\ sittjen móti, &
gômono neoten, \hld\ Judeono paskha &
dêljen mid iu só diurjun. \hld\ Nu ik iu iuwes drohtines skal &
willjon sęggjan, \hld\ þat ik an þesaro wer-oldi ni mót &
mid mannun mêr \hld\ móses an·bíten &
furður mid firihun, \hld\ êr þan gi·fullod wirðid &
himilo ríki. \hld\ Mi is an handun nu &
wíti ęndi wunder-kwále, \hld\ þea ik for þesumu werode skal, &
þolon for þesaru þiodu.“ \hld\ Só he þó só te þem þegnun sprak, &
hêlag drohtin, \hld\ só warð imu is hugi dróvi, &
warð imu gi·sworken sevo, \hld\ ęndi eft te þem ge·sïðun sprak, &
þe gódo te þem is jungarun: \hld\ „hwat, ik iu godes ríki“, kwað he, &
„gi·hét himiles lioht, \hld\ ęndi gí mí hold-líko &
iuwan þegạn-skępi. \hld\ Nu ni willjat gí a·þęngjan só, &
ak węnkjat þero wordo. \hld\ Nu sęggju ik iu te wáran hér, &
þat wili iuwar twe-livjo ên \hld\ trewana swíkan, &
wili mi far·kôpon \hld\ undar þit kunni Judeono, &
gi·sęlljen wiðer siluvre, \hld\ ęndi wili imu þar sink niman, &
diurje mêðmos, \hld\ ęndi geven is drohtin wið þiu, &
holdan hêrran. \hld\ Þat imu þoh te harme skal, &
werðan te wítje; \hld\ be þat he þea wurdi far·sihit &
ęndi he þes arvedjes \hld\ ęndi skawot, &
þan wêt he þat te wáran, \hld\ þat imu wári wóðjera þing, &
bętera mikilu, \hld\ þat he gio gi·boran ni wurði &
libbjendi te þesumu liohte, \hld\ þan he þat lôn nimid, &
uvil arvedi \hld\ in·wid-rádo.“ &
Þó bi·gan þero erlo ge·hwi-lik \hld\ te ǫ́ðrumu skawon, &
sorgondi sehan; \hld\ was im sêr hugi, &
hriwig umbi iro herta: \hld\ gi·hôrdun iro hêrron þó &
gorn-word sprekan. \hld\ Þea gumon sorgodun, &
hwi-likan he þero twe-livjo \hld\ te þiu tęlljen weldi, &
skuldigna skaðon, \hld\ þat he habdi þea skattos þar &
ge·þingod at þeru þiod. \hld\ Ni was þero þegno ênigumu &
su·likes in·widdjes \hld\ óði te gehanne, &
mên-gi·þáhtjo \hld\ —ant·suok þero manno ge·hwi-lik—, &
wurðun alle an forhtun, \hld\ frágon ne gi·dorstun, &
êr þan þó ge·bóknide \hld\ bar-wirðig gumo, &
Símon Petrus \hld\ —ne gi·dorste it selvo sprekan— &
te Johanne þemu gódon: \hld\ he was þemu godes barne &
an þem dagun \hld\ þegno liovost, &
mêst an minnjun \hld\ ęndi móste þar þó an þes mahtiges Kristes &
barme restjen \hld\ ęndi an is breostun lag, &
hlinode mid is hôvdu: \hld\ þar nam he só manag hêlag ge·rúni, &
diapa gi·þáhti, \hld\ ęndi þó te is drohtine sprak, &
be·gan ina þó frágon: \hld\ „hwe skal þat, frô mín, wesen“, kwað he, &
„þat þi far·kôpon wili, \hld\ kuningo ríkjost, &
undar þínaro fíundo folk? \hld\ Ús wári þes firi-wit mikil, &
waldand, te witanne.“ \hld\ Þó habde eft is word garu &
hêljando Krist: \hld\ „seh þi, hwemu ik hér an hand geve &
mínes móses for þesun mannun: \hld\ þe haved mên-gi·þáht, &
birid bittran hugi; \hld\ þe skal mi an banono ge·wald, &
fíundun bi·felhen, \hld\ þar man mínes ferhes skal, &
aldres áhtjen.“ \hld\ Nam he þó aftar þiu &
þes móses for þem mannun \hld\ ęndi gaf is þemu mên-skaðen, &
Judase an hand \hld\ ęndi imu te·gęgnes sprak &
selvo for þem is ge·sïðun \hld\ ęndi ina sniumo hét &
faran fan þemu is folke: \hld\ „frumi só þú þęnkis“, kwað he, &
„dó þat þú duan skalt: \hld\ þú ni maht bi·dęrnjen lęng &
willjon þínan. \hld\ Þiu wurd is at handun, &
þea tídi sind nu gi·náhid.“ \hld\ Só þó þe treu-logo &
þat mós ant·féng \hld\ ęndi mid is mu̇ðu an·bêt, &
só af·gaf ina þó þiu godes kraft, \hld\ gramon in ge·witun &
an þene lík-hamon, \hld\ lêða wihti, &
warð imu Satanas \hld\ sêro bi·tengi, &%TODO: Check etymology of teng-.
hardo umbi is herte, \hld\ sïður ine þiu helpe godes &
far·lét an þesumu liohte. \hld\ Só is þena liudjo wê, &
þe só undar þesumu himile skal \hld\ hêrron wehslon. &
Gi·wêt imu þó út þanen \hld\ in·widjas gern &
Judas gangan: \hld\ habde imu grimmen hugi &
þegạn wið is þiodan. \hld\ Was þó iu þiustri naht, &
swíðo gi·sworken. \hld\ Sunu drohtines &
was ima at þem gômun forð \hld\ ęndi is jungarun þar &
waldand wín ęndi brôd \hld\ wíhide bêðju, &
hêlagode heven-kuning, \hld\ mid is handun brak, &
gaf it undar þem is jungarun \hld\ ęndi gode þankode, &
sagde þem ǫ́·lát, \hld\ þe þar al gi·skóp, &
wer-old ęndi wunnja, \hld\ ęndi sprak word manag: &
„gi·lôvjot gí þes liohto“, \hld\ kwað he, „þat þit is mín lík-hamo &
ęndi mín blód só same: \hld\ givu ik iu hér bêðju samad &
etan ęndi drinkan. \hld\ Þit ik an erðu skal &
gevan ęndi geotan \hld\ ęndi iu te godes ríkje &
lôsjen mid mínu lík-hamen \hld\ an líf êwig, &
an þat himiles lioht. \hld\ Gi·huggjat gí simlun, &
þat gí þiu ful-gangan, \hld\ þiu ik an þesun gômun dón; &
márjad þit for męnegi: \hld\ þit is mahtig þing, &
mid þius skulun gí iuwomu drohtine \hld\ diuriða frummjen, &
habbjad þit mín te gi·hugdjun, \hld\ hêlag biliði, &
þat it ęldi-barn \hld\ aftar lêstjen, &
waron an þesaru wer-oldi, \hld\ þat þat witin alle, &
man ovar þesan middil-gard, \hld\ þat it is þurh mína minnja gi·duan &
hêrron te huldi. \hld\ Ge·huggjad gí simlun, &
hweo ik iu hér ge·biudu, \hld\ þat gí iuwan bróðer-skępi &
fasto frummjad: \hld\ habbjad ferhtan hugi, &
minnjod iu an iuwomu móde, \hld\ þat þat manno barn &
ovar irmin-þiod \hld\ alle far·standen, &
þat gí sind gegnungo \hld\ jungaron míne. &
Ôk skal ik iu ku̇ðjen, \hld\ hwó hér wili kraftag fíund, &
hettjand heru-grim, \hld\ umbi iuwan hugi niusjen, &
Satanas selvo: \hld\ he kumid iuwaro seolono herod &
frókno frêson. \hld\ Simlun gí fasto te gode &
berad iuwa breost-gi·þáht: \hld\ ik skal an iuwaru bedu standen, &
þat iu ni mugi þe mên-skaðo \hld\ mód ge·twífljan; &
ik ful-lêstju iu wiðer þemu fíunde. \hld\ Ôk kwam he herod giu frêson mín, &
þoh imu is willjon hér \hld\ wiht ne gi·stódi, &
lioves an þemu mínumu lík-hamon. \hld\ Nu ni willju ik iu lęng helen, &
hwat iu hér nu sniumo skal \hld\ te sorgu gi·standen: &
gi skulun mi ge·swíkan, \hld\ ge·sïðos míne, &
iuwes þegạn-skępjes, \hld\ êr þan þius þiustrje naht &
liudi far·líða \hld\ ęndi eft lioht kume, &
morgan te mannun.“ \hld\ Þó warð mód gumon &
swíðo gi·sworken \hld\ ęndi sêr hugi, &
hriwig umbi iro herte \hld\ ęndi iro hêrron word &
swíðo an sorgun. \hld\ Símon Petrus þó, &
þegạn wið is þiodan \hld\ þríst-wordun sprak &
bi huldi *wið is hêrron: \hld\ „þoh þi all þit hęliðo folk“, kwat-hie, &
„gi·swíkan þína gi·sïðos, \hld\ þoh ik sinnon mid þi &
at allon þarạvon \hld\ þolojan willju. &
Ik biun garo sinnon, \hld\ ef mi god látið, &
þat ik an þínon ful-lêstje \hld\ fasto gi·stande; &
þoh sia þi an karkarjes \hld\ klústron hardo, &
þesa liudi bi·lúkan, \hld\ þoh ist mi luttil tweho, &
ne ik an þem bęndjon mid þi \hld\ bídan willje, &
liggjan mid þi só lieven; \hld\ ef sia þínes líves þan &
þuru ęggja níð \hld\ áhtjan willjad, &
frô mín þie guodo, \hld\ ik givu mín ferah furi þik &
an wápno spil: \hld\ nis mi werð iowiht &
te bi·míðanne, \hld\ só lango só mi mín warod &
hugi ęndi hand-kraft.“ \hld\ Þuo sprak im eft is hêrro an·gęgin: &
„hwat, þú þik bi·wánis“, \hld\ kwat-hie, „wissaro trewono, &
þrístero þingo: \hld\ þú havis þegnes hugi, &
willjon guodan. \hld\ Ik mag þi sęggjan, hwó it þoh gi·werðan skal, &
þat þú wirðis só wêk-muod, \hld\ þoh þú nu ni wánjes só, &
þat þú þínes þiadnes te naht \hld\ þríwo far·lógnis &
êr hano-krádi ęndi kwiðis, \hld\ þak ik þín hêrro ni sí, &
ak þú far·manst mína mund-burd.“ \hld\ Þuo sprak eft þie man an·gęgin: &
„ef it gio an wer-oldi“, \hld\ kwat-hie, „gi·werðan muosti, &
þat ik samad midi þi \hld\ sweltan muosti, &
dôjan diur-líko, \hld\ þan ne wurði gio þie dag kuman, &
þat ik þín far·lógnidi, \hld\ lievo drohtin, &
gerno for þeson Juðeon.“ \hld\ Þuo kwáðun alla þia jungron só, &
þat sia þar an þem þingon mid im \hld\ þoljan weldin &
Þuo im eft mid is wordon gi·bôd \hld\ waldand selvo, &
hêr hevan-kuning, \hld\ þat sia im ni lietin iro hugi twífljan, &
hiet þat sia ni weldin{[...]} \hld\ diopa gi·þáhti: &
„ne druovie iuwa herta \hld\ þuru iuwes drohtines word, &
ne forohtjat te filo: \hld\ ik skal fader u̇san &
selvan suokjan \hld\ ęndi iu sęndjan skal &
fan hevan-ríkje \hld\ hêlagna gêst: &
þie skal iu eft gi·fruofrjan \hld\ ęndi te frumu werðan, &
manon iu þero mahlo, \hld\ þie ik iu manag hębbju &
wordon gi·wísid. \hld\ Hie givit iu gi·wit an briost, &
lust-sama lêra, \hld\ þat gi lêstjan forð &
þiu word ęndi þiu werk, \hld\ þia ik iu an þesaro wer-oldi gi·bôd.“ &
A·rês im þuo þe ríkjo \hld\ an þemo rakode innan, &
nęrjendo Krist \hld\ ęndi gi·wêt im nahtes þanan &
selvo mid is gi·sïðon: \hld\ sêrago géngun &
swíðo gornondja \hld\ jungron Kristes, &
hriwig-muoda. \hld\ Þuo hie im an þena hôhan gi·wêt &
Oliueti-berg: \hld\ þar was hie up gi·wuno &
gangan mid is jungron. \hld\ Þat wissa Judas wel, &
balo-hugdig man, \hld\ hwand hie was oft an þem berege mid im. &
Þar gruotta þie godes suno \hld\ iúgron sína: &%NOTE: iúgron checked, might still correct?
„gi sind nu só druovja“, \hld\ kwat-hie, „nu gi mínan dôð witun; &
nu gornonð gi ęndi griotand, \hld\ ęndi þesa Juðeon sind an luston, &
męndit þius męnigi, \hld\ sindun an iro muode fráha, &
þius wer-old ist an wunnjon. \hld\ Þes wirðit þoh gi·wand kuman &
sniumo tulgo: \hld\ þan wirðit im sêr hugi, &
þan mornjat sia an iro móde, \hld\ ęndi gi męndjan skulun &
after te êwon-dage, \hld\ hwand gio ęndi ni kumið, &
iuwes wellíves gi·wand: \hld\ be·þiu ne þurvun iu þius werk tregan, &
hrewan mín hin-fard, \hld\ hwand þanan skal þiu helpa kuman &
gumono barnon.“ \hld\ Þuo hiet hie is jungron þar &
bídan uppan þemo berge, \hld\ kwað þat hie ti bedu weldi &
an þiu holm-klivu \hld\ hôhor stígan; &
hiet þuo þria mid im \hld\ þegnos gangan, &
Jakobe ęndi Johannese \hld\ ęndi þena guodan Petruse, &
þríst-muodjan þegạn. \hld\ Þuo sia mid iro þiedne samad &
gerno géngun. \hld\ Þuo hiet sia þie godes suno &
an berge uppan \hld\ te bedu hnígan, &
hiet sia god gruotjan, \hld\ *gerno biddjan, &
þat he im þero kostondero \hld\ kraft far·stódi, &
wrêðaro willjon, \hld\ þat im þe wiðer-sako, &
ni mahti þe mên-skaðo \hld\ mód gi·twífljan, &
iak imu þó selvo gi·hnêg \hld\ sunu drohtines &
kraftag an knio-beda, \hld\ kuningo ríkjost, &
forð-ward te foldu: \hld\ fader alo-þiado &
gódan grótte, \hld\ gorn-wordun sprak &
hriwig-líko: \hld\ was imu is hugi dróvi, &
bi þeru męnniski \hld\ mód gi·hrórid, &
is flêsk was an forhtun: \hld\ fellun imo trahni, &
dróp is diur-lík swêt, \hld\ al só drôr kumid &
wallan fan wundun. \hld\ Was an ge·winne þó &
an þemu godes barne \hld\ þe gêst ęndi þe lík-hamo: &
ǫ́ðar was fu̇sid \hld\ an forð-wegos, &
þe gêst an godes ríki, \hld\ ǫ́ðar gjámar stód, &
lík-hamo Kristes: \hld\ ni welde þit lioht a·geven, &
ak drovde for þemu dôðe. \hld\ Simla he hreop te drohtine forð &
þiu mêr aftar þiu \hld\ mahtigna grótte, &
hôhan himil-fader, \hld\ hêlagna god, &
waldand mid is wordun: \hld\ „ef nu werðen ni mag“, kwað he, &
„man-kunni ge·nęrid, \hld\ ne sí þat ik mínan geve &
liovan lík-hamon \hld\ for liudjo barn &
te wêgjanne te wundrun, \hld\ it sí þan þín willjo só, &
ik willju is þan gi·koston: \hld\ ik nimu þene kelik an hand, &
drinku ina þi te diurðu, \hld\ drohtin frô mín, &
mahtig mund-boro. \hld\ Ni seh þú mínes hér &
flêskes gi·fórjes. \hld\ Ik fullon skal &
willjon þínen: \hld\ þú haves ge·wald ovar al.“ &
Gi·wêt imu þó gangen, \hld\ þar he êr is jungaron lét &
bídan uppan þemu berge; \hld\ fand sie þat barn godes &
slápen sorgandje: \hld\ was im sêr hugi, &
þes sie fan iro drohtine \hld\ dêljen skoldun. &
Só sind þat mód-þraka \hld\ manno ge·hwi-likumu, &
þat he far·láten skal \hld\ liavane hêrron, &
af·geven þene só gódene. \hld\ Þó he te is jungarun sprak, &
wahte sie waldand \hld\ ęndi wordun grótte: &
„hwí willjad gi só slápen?“ \hld\ kwað he; „ni mugun samad mid mi &
wakon êne tíd? \hld\ Þiu wurd is at handun, &
þat it só gi·gangen skal, \hld\ só it god fader &
gi·markode mahtig. \hld\ Mi nis an mínumu móde tweho: &
mín gêst is garu \hld\ an godes willjan, &
fu̇s te faranne: \hld\ mín flêsk is an sorgun, &
letid mik mín lík-hamo: \hld\ lêð is imu swíðo &
wíti te þolonne. \hld\ Ik þoh willjan skal &
mínes fader ge·frummjen; \hld\ hębbjad gi fasten hugi.“ &
Gi·wêt imu þó eft þanan \hld\ ǫ́ðer-sïðu &
an þene berg uppen \hld\ te bedu gangan, &
mári drohtin, \hld\ ęndi þar só manag gi·sprak &
gódoro wordo. \hld\ Godes ęngil kwam &
hêlag fan himile, \hld\ is hugi fastnode, &
beldide te þem bęndjun. \hld\ He was an þeru bedu simla &
forð an flíte \hld\ ęndi is fader grótte, &
waldand mid is wordun: \hld\ „ef it nu wesen ni mag“, kwað he, &
„mári drohtin, \hld\ nevu ik for þit manno folk &
þiod-kwále þoloie, \hld\ ik an þínan skal &
willjan wonjan.“ \hld\ Gi·wêt imu þó eft þanen &
sókjan is ge·sïðos: \hld\ fand sie slápandje, &
grótte sie gáhun. \hld\ Géng imu eft þanen &
þriddjon sïðu te bedu \hld\ ęndi sprak þiod-kuning &
al þiu selvon word, \hld\ sunu drohtines, &
te þemu alo-waldon fader, \hld\ só he êr dede, &
manode mahtigna \hld\ manno frumana &
swíðo niud-líko \hld\ nęrjando Krist, &
géng imu þó eft te þem is jungarun, \hld\ grótte sie sáno: &
„slápad gi ęndi restjad“, \hld\ kwað he. „Nu wirðid sniumo herod &
kuman mid kraftu, \hld\ þe mi far·kôpot havad, &
sundja lôsan gi·sald.“ \hld\ Ge·sïðos Kristes &
wakodun þó aftar þem wordun \hld\ ęndi gi·sáhun þó þat werod kuman &
an þene berg uppen \hld\ brahtmu þiu mikilon, &
wrêða wápan-berand. \hld\ Wísde im Judas, &
gram-hugdig man; \hld\ Judeon aftar sigun, &
fíundo folk-skępi; \hld\ dróg man fiur an gi·mang, &
logna an lioht-fatun, \hld\ lêdde man faklon &
brinnandja fan burg, \hld\ þar sie an þene berg uppan &
stigun mid strídu. \hld\ Þea stędi wisse Judas wel, &
hwar he þea liudi \hld\ tó lêdjan skolde. &
Sagde imu þó te têkne, \hld\ þó sie þar tó fórun &
þemu folke bi·foran, \hld\ te þiu þat sie ni far·féngin þar, &
erlos ǫ́ðren man: \hld\ „ik gangu imu at êrist tó“, kwað he, &
„kussju ine ęndi kwaddju: \hld\ þat is Krist selvo. &
Þene gi fáhen skulun \hld\ folko kraftu, &
binden ina uppan þemu berge \hld\ ęndi ina te burg hinan &
lêdjen undar þea liudi: \hld\ he is líves havad &
mid is wordun far·werkod.“ \hld\ Werod sïðode þó, &
antat sie te Kriste \hld\ kumane wurðun, &
grim folk Judeono, \hld\ þar he mid is jungarun stód, &
mári drohtin: \hld\ bêd metodo-gi·skapu, &
torhtero tídjo. \hld\ Þó géng imu treu-lôs man, &
Judas te·gęgnes \hld\ ęndi te þemu godes barne &
hnêg mid is hôvdu \hld\ ęndi is hêrron kwędde, &
kuste ina kraftagne \hld\ ęndi is kwidi lêste, &
wísde ina þemu werode, \hld\ al só he êr mid wordun ge·hét. &
Þat þolode al mid gi·þuldjun \hld\ þiodo drohtin, &
waldand þesara wer-oldes \hld\ ęndi sprak imu mid is wordun tó, &
frágode ine frókno: \hld\ „be·hwí kumis þú só mid þius folku te mi, &
be·hwí lêdis þú mi só þese liudi tó \hld\ ęndi mi te þesare lêðan þiode sprekan, &
far·kôpos mid þínu kussu \hld\ under þit kunni Judeono, &
meldos mi te þesaru męnegi?“ \hld\ Géng imu þó wið þea man &
wið þat werod ǫ́ðar \hld\ ęndi sie mid is wordun fragn, &
hwene sie mid þiu ge·sïðju \hld\ sókjan kwámin &
só niud-liko an naht, \hld\ „so gi willjan nôd frummjen &
manno hwi-likumu.“ \hld\ Þó sprak imu eft þiu męnegi an·gęgin, &
kwáðun þat im hêljand \hld\ þar an þemu holme uppan &
ge·wísid wári, \hld\ „þe þit gi·wer frumid &
Judeo liudjun \hld\ ęndi ina godes sunu &
selvon hêtid. \hld\ Ina kwámun wí sókjan herod, &
weldin ina gerno bi·geten: \hld\ he is fan Galileo lande, &
fan Nazareth-burg.“ \hld\ Só im þó þe nęrjendjo Krist &
sagde te sǫ́ðan, \hld\ þat he it selvo was, &
só wurðun þó an forhtun \hld\ folk Judeono, &
wurðun under·badode, \hld\ þat sie under bak fellun &
alle efno sán, \hld\ erðe gi·sóhtun, &
wiðer-wardes þat werod: \hld\ ni mahte þat word godes, &
þie stemnje ant·standan: \hld\ wárun þoh só strídige man, &
a·hliopun eft up an þemu holme, \hld\ hugi fastnodun, &
bundun briost-gi·þáht, \hld\ gi·bolgane géngun &
náhor mid níðu, \hld\ ant-tat sie þene nęrjendjon Krist &
werodo bi·wurpun. \hld\ Stódun wíse man, &
swíðo gornundje \hld\ gjungaron Kristes &
bi·foran þeru derevjon dádi \hld\ ęndi te iro drohtine sprákun: &
„wári it nu þín willjo“, \hld\ kwáðun sie, „waldand frô mín, &
þat sie u̇s hér an speres ordun \hld\ spildjen móstin &
wápnun wunde, \hld\ þan ni wári u̇s wiht só gód, &
só þat wí hér for u̇sumu drohtine \hld\ dóan móstin &
beniðjun blêka“. \hld\ Þó gi·bolgan warð &
snel swerd-þegạn, \hld\ Símon Petrus, &
well imu innan hugi, \hld\ þat he ni mahte ênig word sprekan: &
só harm warð imu an is hertan, \hld\ þat man is hêrron þar &
binden welde. \hld\ Þó he gi·bolgan géng, &
swíðo þríst-mód þegạn \hld\ for is þiodan standen, &
hard for is hêrron: \hld\ ni was imu is hugi twífli, &
blóð an is breostun, \hld\ ak he is bil a·tóh, &
swerd bi sídu, \hld\ slóg imu te·gęgnes &
an þene furiston fíund \hld\ folmo krafto, &
þat þó Malkhus warð \hld\ mákjas ęggjun, &
an þea swíðaron half \hld\ swerdu gi·málod: &
þiu hlust warð imu far·hawan, \hld\ he warð an þat hôvid wund, &
þat imu heru-drôrag \hld\ hlear ęndi ôre &
bęni-wundun brast: \hld\ blód aftar sprang, &
well fan wundun. \hld\ Þó was an is wangun skard &
þe furisto þero fíundo. \hld\ Þó stód þat folk an rúm: &
and-rédun im þes billes biti. \hld\ Þó sprak þat barn godes &
selvo te Símon Petruse, \hld\ hét þat he is swerd dedi &
skarp an skêðja: \hld\ „ef ik wið þesa skola weldi“, kwað he, &
„wið þeses werodes ge·win \hld\ wíg-saka frummjen, &
þan manodi ik þene márjon \hld\ mahtigne god, &
hêlagne fader \hld\ an himil-ríkja, &
þat he mi só managan ęngil herod \hld\ ovana sandi &
wíges só wísen, \hld\ só ni mahtin iro wápan-þręki &
man a·dógen: \hld\ iro ni stódi gio su·lik męgin samad, &
folkes gi·fastnod, \hld\ þat im iro ferh aftar þiu &
werðen mahti. \hld\ Ak it havad waldand god, &
alo-mahtig fader \hld\ an ǫ́ðar gi·markot, &
þat wí gi·þolojan skulun, \hld\ só hwat só u̇s þius þioda tó &
bittres brengit: \hld\ ni skulun u̇s belgan wiht, &
wrêðjan wið iro ge·winne; \hld\ hwand só hwe só wápno níð, &
grimman gêr-hęti wili \hld\ gerno frummjen, &
he swiltit imu \hld\ eft swerdes ęggjun, &
dóit im bi·drôregan: \hld\ wí mid u̇sun dádjun ni skulun &
wiht a·węrdjan.“ \hld\ Géng he þó te þemu wundon manne, &
lęgde mid listjun \hld\ lík te·samne, &
hôvid-wundon, \hld\ þat siu sán gi·hêlid warð, &
þes billes biti, \hld\ ęndi sprak þat barn godes &
wið þat wrêðe werod: \hld\ „mi þunkid wunder mikil“, kwað he, &
„ef gi mi lêðes wiht \hld\ lêstjen weldun, &
hwí gi mi þó ni fengun, \hld\ þan ik undar iuwomu folke stód, &
an þemu wíhe innan \hld\ ęndi þar word manag &
sǫ́ð-lík sagde. \hld\ Þan was sunnon skín, &
diur-lik dages lioht, \hld\ þan ni weldun gi mi dóan eo·wiht &
lêðes an þesumu liohte, \hld\ ęndi nu lêdjad mi iuwa liudi tó &
an þiustrje naht, \hld\ al só man þiove dót, &
þan man þene fáhan wili \hld\ ęndi he is ferhes havad &
far·werkot, wam-skaðo.“ \hld\ werod Judeono &
gripun þó an þene godes sunu, \hld\ grimma þioda, &
hatandjero hóp, \hld\ hwurvun ina umbi &
módag manno folk \hld\ —mênes ni sáhun—, &
heftun heru-bęndjun \hld\ handi te·samne, &
faðmos mid fitereun. \hld\ Im ni was su·likaro firin-kwála &
þarf te gi·þolonne, \hld\ þiod-arvedjes, &
te winnanne su·lik wíti, \hld\ ak he it þurh þit werod deda, &
hwand he liudjo barn \hld\ lôsjen welda, &
halon fan hęllju \hld\ an himil-ríki, &
an þene wídon welon: \hld\ be·þiu he þes wiht ne bi·sprak, &
þes sie imu þurh in·wid-níð \hld\ ógjan weldun. &
Þó wurðun þes só malske \hld\ módag folk Judeono, &
þiu hêri warð þes só hrómeg, \hld\ þes sie þena hêlagon Krist &
an liðo-bęndjon \hld\ lêdjan muostun, &
fórjan an fiterjun. \hld\ Þie fíund eft ge·witun &
fan þemu berge te burg. \hld\ Géng þat barn godes &
undar þemu hęri-skępi \hld\ handun ge·bunden, &
drúvondi te dale. \hld\ Wárun imu þea is diurjon þó &
ge·sïðos ge·swikane, \hld\ al só he im êr selvo gi·sprak: &
ni was it þoh be ênigaru blóði, \hld\ þat sie þat barn godes, &
lioven far·létun, \hld\ ak it was só lango bi·foren &
wár-sagono word, \hld\ þat it skoldi gi·werðen só: &
be·þiu ni mahtun sie is be·míðan. \hld\ Þan aftar þeru męnegi géngun &
Johannes ęndi Petrus, \hld\ þie gumon twêne, &
folgodun ferrane: \hld\ was im firi-wit mikil, &
hwat þea grimmon Judeon \hld\ þemu godes barne, &
weldin iro drohtine dóen. \hld\ Þó sie te dale kwámun &
fan þemu berge te burg, \hld\ þar iro biskop was, &
iro wíhes ward, \hld\ þar lêddun ina wlanke man, &
erlos undar ederos. \hld\ Þar was êld mikil, &
fiur an fríd-hove \hld\ þemu folke te·gęgnes, &
ge·warht for þemu werode: \hld\ þar géngun sie im węrmjen tó, &
Judeo liudi, \hld\ létun þene godes sunu &
bídon an bęndjun. \hld\ Was þar braht mikil, &
gêl-módigaro galm. \hld\ Johannes was êr &
þemu hêroston ku̇ð: \hld\ be·þiu móste he an þene hof innan &
þringan mid þeru þioda. \hld\ Stód allaro þegno bętsto, &
Petrus þar úte: \hld\ ni lét ina þe portun ward &
folgon is frôen, \hld\ êr it at is friunde a·bad, &
Johannes at ênumu Judeon, \hld\ þat man ina gangan lét &
forð an þene fríd-hof. \hld\ Þar kwam im ên fêkni wíf &
gangan te·gęgnes, \hld\ þiu ênas Judeon was, &
iro þeodanes þiw, \hld\ ęndi þó te þemu þegne sprak &
magað un·wán-lík: \hld\ „hwat, þú mahtis man wesan“, kwað siu, &
„gjungaro fan Galilea, \hld\ þes þe þar genower stéd &
faðmun gi·fastnod.“ \hld\ Þó an forhtun warð &
Símon Petrus sán, \hld\ slak an is móde, &
kwað þat he þes wíves \hld\ word ni bi·konsti &
ni þes þeodanes \hld\ þegạn ni wári: &
méð is þó for þeru męnegi, \hld\ kwað þat he þena man ni ant·kęndi: &
„ni sind mí þíne kwidi ku̇ðe“, \hld\ kwað he; was imu þiu kraft godes, &
þe hęrdislo fan þemu hertan. \hld\ Hwarạvondi géng &
forð undar þemu folke, \hld\ antat he te þemu fiure kwam; &
gi·wêt ina þó warmjen. \hld\ Þar im ôk ên wíf bi·gan &
felgjan firin-spráka: \hld\ „hér mugun gi“, kwað siu, „an iuwan fíund sehan: &
þit is gegnungo \hld\ gjungaro Kristes, &
is selves ge·sïð.“ \hld\ Þó géngun imu sán aftar þiu &
náhor níð-hwata \hld\ ęndi ina niud-líko &
frágodun fíundo barn, \hld\ hwi-likes he folkes wári: &
“ni bist þú þesoro burg-liudjo“, \hld\ kwáðun sie; „þat mugun wí an þínumu gi·bárje gi·sehan, &
an þínun wordun ęndi an þínaru wíson, \hld\ þat þú þeses werodes ni bist, &
ak þú bist galiléisk man.“ \hld\ He ni welda þes þó gehan eo·wiht, &
ak stód þó ęndi strídda \hld\ ęndi starkan êð &
swíð-líko ge·swór, \hld\ þat he þes ge·sïðes ni wári. &
Ni habda is wordo ge·wald: \hld\ it skolde gi·werðen só, &
só it þe ge·markode, \hld\ þe man-kunnjes &
far·wardot an þesaru wer-oldi. \hld\ Þó kwam imu ôk an þemu warve tó &
þes mannes mág-wini, \hld\ þe he êr mid is mákjo giheu, &%NOTE: giheu checked. Might still correct?
swerdu þiu skarpon, \hld\ kwað þat he ina sáhi þar &
an þemu berge uppan, \hld\ „þar wí an þemu bôm-gardon &
hêrron þínumu \hld\ hęndi bundun, &
fastnodun is folmos.“ \hld\ He þó þurh forhtan hugi &
for·lógnide þes is lioves hêrron, \hld\ kwað þat he weldi wesan þes líves skolo, &
ef it mahti ênig þar \hld\ irmin-manno &
gi·sęggjan te sǫ́ðan, \hld\ þat he þes ge·sïðes wári, &
folgodi þeru fęrdi. \hld\ Þó warð an þena formon sïð &
hano-krád af·haven. \hld\ Þó sah þe hêlago Krist, &
barno þat bętste, \hld\ þar he ge·bunden stóð, &
selvo te Símon Petruse, \hld\ sunu drohtines &
te þemu erle ovar is ahsla. \hld\ Þó warð imu an innan sán, &
Símon Petruse \hld\ sêr an is móde, &
harm an is hertan \hld\ ęndi is hugi dróvi, &
swíðo warð imu an sorgun, \hld\ þat he êr selvo ge·sprak: &
gi·hugde þero wordo þó, \hld\ þe imu êr waldand Krist &
selvo sagda, \hld\ þat he an þeru swartan naht &
êr hano-krádi \hld\ is hêrron skoldi &
þríwo far·lógnjen. \hld\ Þes þram imu an innan mód &
bittro an is breostun, \hld\ ęndi géng imu þó gi·bolgan þanen &
þe man fan þeru męnigi \hld\ an mód-karu, &
swíðo an sorgun, \hld\ ęndi is selves word, &
wam-skęfti weop, \hld\ antat imu wallan kwámun &
þurh þea hert-kara \hld\ hête trahni, &
blódage fan is breostun. \hld\ He ni wánde þat he is mahti gi·bótjen wiht, &
firin-werko furður \hld\ efþa te is fráhon kuman, &
hêrron huldi: \hld\ nis ênig hęliðo só ald, &
þat io mannes sunu \hld\ mêr gi·sáhi &
is selves word \hld\ sêrur hrewan, &%NOTE: sêrur hrewan checked.
karon efþa kúmjen: \hld\ „wola krafteg god“, kwað he, &
þat ik hębbju mi só for·werkot, \hld\ só ik mínaro wer-oldes ni þarf &
ǫ́·lát sęggjan. \hld\ Ef ik nu te aldre skal &
huldjo þínaro \hld\ ęndi heven-ríkjas, &
þeoden, þolojan, \hld\ þan ni þarf mi þes ênig þank wesan, &
liovo drohtin, \hld\ þat ik io te þesumu liohte kwam. &
Ni bium ik nu þes wirðig, \hld\ waldand frô mín, &
þat ik under þíne jungaron \hld\ gangan móti, &
þus sundig under þíne ge·sïðos: \hld\ ik iro selvo skal &
míðan an mínumu móde, \hld\ nu ik mi su·lik mên ge·sprak.“ &
Só gornode \hld\ gumono bętsta, &
hrau im só hardo, \hld\ þat he habde is hêrren þó &
leoves far·lógnid. \hld\ Þan ni þurvun þes liudjo barn, &
weros wundrojan, \hld\ be·hwí it weldi god, &
þat só lioven man \hld\ lêð gi·stódi, &
þat he só hôn-líko \hld\ hêrron sínes &
þurh þera þiwun word, \hld\ þegno snellost, &
far·lógnide só lioves: \hld\ it was al bi þesun liudjun gi·duan, &
firiho barnun te frumu. \hld\ He welde ina te furiston dóan, &
hêrost ovar is híwiski, \hld\ hêlag drohtin: &
lét ina ge·kunnon, \hld\ hwi-like kraft havet &
þe męnniska mód \hld\ áno þe maht godes; &
lét ina ge·sundjon, \hld\ þat he sïðor þiu bet &
liudjun gi·lôvdi, \hld\ hwó liof is þar &
manno gi·hwi-likumu, \hld\ þan he mên ge·frumit, &
þat man ina a·láte \hld\ lêðes þinges, &
sakono ęndi sundjono, \hld\ só im þó selvo dede &
heven-ríki god \hld\ harm-ge·wurhti. &
Be þiu nis mannes bág \hld\ mikilun bi·þervi, &
hagu-staldes hróm: \hld\ ef imu þiu helpe godes &
ge·swíkid þurh is sundjon, \hld\ þan is imu sán aftar þiu &
breost-hugi blóðora, \hld\ þoh he êr bi·hêt spreka, &
hrómje fan is hildi \hld\ ęndi fan is hand-krafti, &
þe man fan is męgine. \hld\ Þat warð þar an þemu márjon skín, &
þegno bętston, \hld\ þó imu is þiodanes gi·swêk &
hêlag helpe. \hld\ Be·þiu ni skoldi hrómjen man &
te swíðo fan imu selvon, \hld\ hwand imu þar swíkid oft &
wán ęndi willjo, \hld\ ef imu waldand god, &
hêr heven-kuning \hld\ herte ni sterkit. &
Þan bêd allaro barno bętst, \hld\ bęndi þolode &
þurh man-kunni. \hld\ Hwurvun ina managa umbi &
Judeono liudi, \hld\ sprákun gelp mikil, &
habdun ina te hoska, \hld\ þar he gi·heftid stód, &
þolode mid ge·þuldjun, \hld\ só hwat só imu þiu þiod deda, &
liudi lêðes. \hld\ Þó warð eft lioht kuman, &
morgan te mannun. \hld\ Manag samnoda &
hęri Judeono: \hld\ habdun im hugi wulvo, &
in·wid an innan. \hld\ Warð þar êo-sago &
an morgan-tíd \hld\ manag gi·samnod &
irri ęndi ên-hard, \hld\ in·widjas gern, &
wrêðes willjan. \hld\ Géngun im an warf samad &
rinkos an rúna, \hld\ bi·gunnun im rádan þó, &
hwó sie ge·wísadin \hld\ mid wár-lôsun, &
mannun mên-ge·witun \hld\ an mahtigna Krist &
te gi·sęggjanne sundja \hld\ þurh is selves word, &
þat sie ina þan te wunder-kwálu \hld\ wêgjan móstin, &
a·dêljen te dôðe. \hld\ Sie ni mahtun an þemu dage finden &
só wrêð ge·wit-skępi, \hld\ þat sie imu wíti be·þiu &
a·dêljen gi·dorstin \hld\ efþa dôð frummjen, &
lívu bi·lôsjen. \hld\ Þó kwámun þar at latstan forð &
an þena warf wero \hld\ wár-lôse man &
twêne gangan \hld\ ęndi bi·gunnun im tęlljen an, &
kwáðun þat sie ina selvon \hld\ sęggjan gi·hôrdin, &
þat he mahti te·werpen \hld\ þena wíh godes, &
allaro húso hôhost \hld\ ęndi þurh is hand-męgin, &
þurh is ênes kraft \hld\ up a·rihtjen &
an þriddjon daga, \hld\ só is elkor ni þorfti be·þíhan man. &
He þagoda ęndi þoloda: \hld\ ni sprak imu io þiu þiod só filu, &
þea liudi mid luginun, \hld\ þat he it mid lêðun an·gęgin &
wordun wráki. \hld\ Þó þar undar þemu werode a·rês &
balu-hugdig man, \hld\ biskop þero liudjo, &
þe furisto þes folkes \hld\ ęndi frágode Krist &
iak ina be imu selvon bi·swór \hld\ swíðon êðun, &
grótte ina an godes namon \hld\ ęndi gerno bad, &
þat he im þat gi·sagdi, \hld\ ef he sunu wári &
þes libbjendjes godes: \hld\ „þes þit lioht ge·skóp, &
Krist kuning êwig. \hld\ Wí ni mugun is ant·kiennjen wiht &
ne an þínun wordun ni an þínun werkun.“ \hld\ Þó sprak imu eft þe wáro an·gęgin, &
þe gódo godes sunu: \hld\ „þú kwiðis it for þesun Judeon nu, &
sǫ́ð-líko sęgis, \hld\ þat ik it selvo bium. &
Þes ni gi·lôvjad mi þese liudi: \hld\ ni willjad mi for·látan be·þiu; &
ni sind im mín word wirðig. \hld\ Nu sęggju ik iu te wárun þoh, &
þat gi noh skulun sittjen gi·sehan \hld\ an þe swíðaron half godes &
márjan mannes sunu, \hld\ an męgin-krafte &
þes alo-walden fader, \hld\ ęndi þanan eft kuman &
an himil-wolknun herod \hld\ ęndi allumu hęliðo kunnje &
mid is wordun a·dêljen, \hld\ al só iro ge·wurhti sind.“ &
Þo balg ina þe biskop, \hld\ habde bittren hugi, &
wrêðida wið þemu worde \hld\ ęndi is gi·wádi slêt, &
brak for is breostun: \hld\ „nu ni þurvun gi bídan lęng“, kwað he, &
„þit werod ge·wit-skępjes, \hld\ nu im su·lik word farad, &
mên-spráka fan is mu̇ðe. \hld\ Þat gi·hôrid hér nu manno filu, &
rinko an þesumu rakude, \hld\ þat he ina só ríkjan telit, &
gihid þat he god sí. \hld\ Hwat willjad gi Judeon þes &
a·dêljen te dóme? \hld\ Is he dôðes nu &
wirðig be su·likun wordun?“ \hld\ Þat werod al ge·sprak, &
folk Judeono, \hld\ þat he wári þes ferhes skolo, &
wítjes só wirðig. \hld\ Ni was it þoh be is ge·wurhtjun gi·dóen, &
þat ine þar an Hjerusalem \hld\ Judeo liudi, &
sunu drohtines \hld\ sundja lôsen &
a·dêldun te dôðe. \hld\ Þó was þero dádjo hróm &
Judeo liudjun, \hld\ hwat sie þemu godes barne mahtin &
só haftemu mêst, \hld\ harmes ge·frummjen. &
Be·wurpun ina þó mid werodu \hld\ ęndi ina an is wangon slógun, &
an is hleor mid iro handun \hld\ —al was imu þat te hoske gi·dóen—, &
felgidun imu firin-word \hld\ fíundo męnegi, &
bi·smer-spráka. \hld\ Stód þat barn godes &
fast under fíundun: \hld\ wárun imu is faðmos ge·bundene, &
þolode mid gi·þuldjun, \hld\ só hwat só imu þiu þioda tó &
bittres bráhte: \hld\ ni balg ina n·eo·wiht &
wið þes werodes ge·win. \hld\ Þó námon ina wrêðe man &
só gi·bundanan, \hld\ þat barn godes, &
ęndi ina þó lêddun, \hld\ þar þero liudjo was, &
þere þiade þing-hús. \hld\ Þar þegạn manag &
hwurvun umbi iro hęri-togon. \hld\ Þar was iro hêrron bodo &
fan Rúmu-burg, \hld\ þes þe þó þes ríkjas gi·weld: &
kumen was he fan þemu kêsure, \hld\ gi·sęndid was he undar þat kunni Judeono &
te rihtjenne þat ríki, \hld\ was þar rád-gevo: &
Pilatus was he hêten; \hld\ he was fan Ponteo lande &
knósles kęnnit. \hld\ Habde imu kraft mikil, &
an þemu þing-húse \hld\ þiod gi·samnod, &
an warf weros; \hld\ wár-lôse man &
a·gávun þó þena godes sunu, \hld\ Judeo liudi, &
under fíundo folk, \hld\ kwáðun þat he wári þes ferhes skolo, &
þat man ina wítnodi \hld\ wápnes ęggjun, &
skarpun skúrun. \hld\ Ni welde þiu skole Judeono &
þringan an þat þing-hús, \hld\ ak þiu þiod úte stód, &
mahlidun þanen wið þea męnegi: \hld\ ni weldun an þat gi·mang faren, &
an ęli-landige man, \hld\ þat sie þar un·reht word, &
an þemu dage dęrvjes wiht \hld\ a·dêljan ne gi·hôrdin, &
ak kwáðun þat sie im só hluttro \hld\ hêlaga tídi, &
weldin iro paskha halden. \hld\ Pilatus ant·féng &
at þem wam-skaðun \hld\ waldandes barn, &
sundja lôsen. \hld\ Þó an sorgun warð &
Judases hugi, \hld\ þó he a·gevan gi·sah &
is drohtin te dôðe, \hld\ þó bi·gan imu þiu dád aftar þiu &
an is hugja hrewan, \hld\ þat he habde is hêrron êr &
sundja lôsen gi·sald. \hld\ Nam imu þó þat siluvar an hand, &
þrí-tig skatto, \hld\ þat man imu êr wið is þiodane gaf, &
géng imu þó te þem Judiun \hld\ ęndi im is grimmon dád, &
sundjon sagde, \hld\ ęndi im þat siluvar bôd &
gerno te a·gevanne: \hld\ „ik hębbju it só grio-líko“, kwað he, &
„mínes drohtines \hld\ drôru gi·kôpot, &
só ik wêt þat it mi ni þíhit.“ \hld\ Þiod Judeono &
ni weldun it þó ant·fáhan, \hld\ ak hétun ina forð aftar þiu &
umbi su·lika sundja \hld\ selvon ahton, &
hwat he wið is fráhon \hld\ ge·frumid habdi: &
„þú sáhi þi selvo þes“, \hld\ kwaðun sie; „hwat wili þú þes nu sóken te u̇s? &
Ne wít þú þat þesumu werode!“ \hld\ Þó gi·wêt imu eft þanan &
Judas gangan \hld\ te þemu godes wíhe &
swíðo an sorgun \hld\ ęndi þat siluvar warp &
an þena alah innan, \hld\ ne gi·dorste it êgan lęng; &
fór imu þó só an forhtun, \hld\ só ina fíundo barn &
módage manodun: \hld\ habdun þes mannes hugi &
gramon under·gripanen, \hld\ was imu god a·bolgan, &
þat he imu selvon þó \hld\ símon warhte, &
hnêg þó an heru-sél \hld\ an hinginna, &
warag an wurgil \hld\ ęndi wíti ge·kôs, &
hard hęllje ge·þwing, \hld\ hêt ęndi þiustri, &
diap dôðes dalu, \hld\ hwand he êr umbi is drohtin swêk. &
Þan bêd þat barn godes \hld\ —bęndi þolode &
an þemu þing-húse—, \hld\ hwan êr þiu þiod under im, &
erlos ên-wordje \hld\ alle wurðin, &
hwat sie imu þan te ferah-kwálu \hld\ frummjan weldin. &
Þó þar an þem bęnkjun a·rês \hld\ bodo kêsures &
fan Rúmu-burg \hld\ ęndi géng imu wið þat ríki Judeono &
módag mahljen, \hld\ þar þiu męnigi stód &
aftar þemu hove hwarvon: \hld\ ni weldun an þat hús kuman &
an þemu paskha-dage. \hld\ Pilatus bi·gan &
frókno frágon \hld\ ovar þat folk Judeono, &
mid hwiu þe man habdi \hld\ morðes gi·skuldit, &
wítjes gi·werkot: \hld\ „be hwí gi imu só wrêðe sind, &
an iuwomu hugja hótje?“ \hld\ Sie kwáðun þat he im habdi harmes só filu, &
lêðes gi·lêstid: \hld\ „ni gávin ina þesa liudi þi, &
þar sie ina êr bi·foran \hld\ uvilan ni wissin, &
wordun far·warhten. \hld\ He havat þeses werodes só filu &
far·lêdid mid is lêrun \hld\ —ęndi þesa liudi męrrid, &
dóit im iro hugi twífljen—, \hld\ þat wí ni mótun te þemu hove kêsures &
tinsi gelden; \hld\ þat mugun wí ina gi·tęlljen an &
mid wáru ge·wit-skępi. \hld\ He sprikid ôk word mikil, &
kwiðit þat he Krist sí, \hld\ kuning ovar þit ríki, &
be·gihit ina só grôtes.“ \hld\ Þó im eft te·gęgnes sprak &
bodo kêsures: \hld\ „ef he só bar-líko“, kwað he, &
„under þesaru męnigi \hld\ mên-werk frumid, &
ant·fáhad ina þan eft under iuwe folk-skępi, \hld\ ef he sí is ferhes skolo, &
ęndi imu só a·dêljad, \hld\ ef he sí dôðes werð, &
só it an iuwaro aldrono \hld\ êo ge·biode.“ &
Sie kwáðun þó, þat sie ni móstin \hld\ manno nig·ênumu &
an þea hêlagon tíd \hld\ te hand-banon, &
werðen mid wápnun \hld\ an þemu wíh-dage. &
Þó wende ina fan þemu werode \hld\ wrêð-hugdig man, &
þegạn kêsures, \hld\ þe ovar þea þioda was &
bodo fan Rúmu-burg—: \hld\ hét imu þó þat barn godes &
náhor gangan \hld\ ęndi ina niud-líko, &
frágoda frókno, \hld\ ef he ovar þat folk kuning &
þes werodes wári. \hld\ Þó habde eft is word garu &
sunu drohtines: \hld\ „hweðer þú þat fan þi selvumu sprikis“, kwað he, &
„þe it þi ǫ́ðre hér \hld\ erlos sagdun, &
kwáðun umbi mínan kuning-duom?“ \hld\ Þó sprak eft þe kêsures bodo &
wlank ęndi wrêð-mód, \hld\ þar he wið waldand Krist &
reðjode an þem rakude: \hld\ „ni bium ik þeses ríkjes hinan“, kwað he, &
„Gjudeo liudjo, \hld\ ni gadoling þín, &
þesaro manno mág-wini, \hld\ ak mi þi þius męnigi bi·falah, &
a·gávun þi þína gadulingos mi, \hld\ Judeo liudi, &
haftan te handun. \hld\ Hwat havas þú harmes gi·duan, &
þat þú só bittro skalt \hld\ bęndi þolojan, &
kwalm undar þínumu kunnje?“ \hld\ Þó sprak imu eft Krist an·gęgin, &
hêlendero bętst, \hld\ þar he gi·heftid stód &
an þemu rakude innan: \hld\ „nis mín ríki hinan“, kwað he, &
„fan þesaru wer-old-stundu. \hld\ Ef it þoh wári só, &
þan wárin só stark-móde \hld\ wiðer stríd-hugi, &
wiðer grama þioda \hld\ jungaron míne, &
só man mi ni gávi \hld\ Judeo liudjun, &
hettendjun an hand \hld\ an heru-bęndjun &
te wêgjanne te wundrun. \hld\ Te þiu warð ik an þesaru wer-oldi gi·boran, &
þat ik ge·wit-skępi giu \hld\ wáres þinges &
mid mínun kumiun ku̇ðdi. \hld\ Þat mugun ant·kęnnjen wel &
þe weros, þe sind fan wáre kumane: \hld\ þe mugun mín word far·standen, &
gi·lôvjen mínun lêrun.“ \hld\ Þó ni mahte lasteres wiht &
an þem barne godes \hld\ bodo kêsures, &
findan fêknja word, \hld\ þat he is ferhes be·þiu &
skuldig wári. \hld\ Þó géng he im eft wið þea skola Judeono &
módag mahljen \hld\ ęndi þeru męnigi sagde &
ovar hlust mikil, \hld\ þat he an þemu hafton manne &
su·lika firin-spráka \hld\ finden ni mahti &
for þem folk-skipje, \hld\ só he wári is ferhes skolo, &
dôðes wirðig. \hld\ Þan stódun dol-móde &
Judeo liudi \hld\ ęndi þane godes sunu &
wordun wrógdun: \hld\ kwáðun þat he gi·wer êrist &
be·gunni an Galileo lande, \hld\ „ęndi ovar Judeon fór &
herod-wardes þanan, \hld\ hugi twíflode, &
manno mód-sevon, \hld\ só he is morðes werð, &
þat man ina wítnoje \hld\ wápnes ęggjun, &
ef eo man mid su·likun dádjun mag \hld\ dôðes ge·skuldjen.“ &
Só wrógdun ina mid wordun \hld\ werod Judeono &
þurh hótjan hugi. \hld\ Þó þe hęri-togo, &
slíð-módig man \hld\ sęggjan gi·hôrde, &
fan hwi-likumu kunnje was \hld\ Krist a·fódid, &
manno þe bętsto: \hld\ he was fan þeru márjan þiadu, &
þe gódo fan Galilea-lande; \hld\ þar was gum-skępi &
ęðiljero manno; \hld\ Erodes bi·held þar &
kraftagne kuning-dóm, \hld\ só ina imu þe kêsur far·gaf, &
þe ríkjo fan Rúmu, \hld\ þat he þar rehto ge·hwi-lik &
ge·frumidi undar þemu folke \hld\ ęndi friðu lêsti, &
dómos a·dêldi. \hld\ He was ôk an þemu dage selvo &
an Hjerusalem \hld\ mid is gum-skępi, &
mid is werode at þemu wíhe: \hld\ só was iro wíse þan, &
þat sie þar þia hêlagun tíd \hld\ haldan skoldun, &
paskha Judeono. \hld\ Pilatus gi·bôd þó, &
þat þena hafton man \hld\ hęliðos námin &
só gi·bundanan, \hld\ þat barn godes, &
hét þat sie ina Erodese, \hld\ erlos bráhtin &
haften te handun, \hld\ hwand he fan is hęri-skępi was, &
fan is werodes ge·wald. \hld\ Wígand frumidun &
iro hêrron word: \hld\ hêlagne Krist &
fórdun an fiterjun \hld\ for þena folk-togun, &
allaro barno bętst, \hld\ þero þe io gi·boren wurði &
an liudjo lioht; \hld\ an liðu-bęndjun géng, &
antat sie ina bráhtun, \hld\ þar he an is bęnkja sat, &
kuning Erodes: \hld\ umbi·hwarf ina kraft wero, &
wlanke wígandos: \hld\ was im willjo mikil, &
þat sie þar selvon Krist \hld\ gi·sehan móstin: &
wándun þat he im sum têkạn \hld\ þar tôgjan skoldi, &
mári ęndi mahtig, \hld\ só he managun dede &
þurh is god-kundi \hld\ Judeo *liudjon. &
Frágoda ina þuo þie folk-kuning \hld\ firi-wit-líko &
managon wordon, \hld\ wolda is muod-sevon &
forð undar·findan, \hld\ hwat hie te frumu mohti &
mannon gi·markon. \hld\ Þan stuod mahtig Krist, &
þagoda ęndi þoloda: \hld\ ne wolda þem þied-kuninge, &
Erodese ne is erlon \hld\ ant·swór gevan &
wordo nig·ênon. \hld\ Þan stuod þiu wrêða þiod, &
Judeo liudi \hld\ ęndi þena godes suno &
wurrun ęndi wruogdun, \hld\ anþat im warð þie wer-old-kuning &
an is huge huoti \hld\ ęndi all is hęri-skipi, &
far·muonstun ina an iro muode: \hld\ ne ant·kęndun maht godes, &
himiliskan hêrron, \hld\ ak was im iro hugi þiustri, &
baluwes gi·blandan. \hld\ Barn drohtines &
iro wrêðun werk, \hld\ word ęndi dádi &
þuru ôd-muodi \hld\ all gi·þoloda, &
só hwat só sia im tionono þuo \hld\ tuogjan woldun. &
Sia hietun im þuo te hoske \hld\ hwít gi·wádi &
umbi is liði lęggjan, \hld\ þiu mêr hie wurði þem liudjon þar, &
jungron te gamne. \hld\ Judeon faganodun, &
þuo sia ina te hoske \hld\ hębbjan gi·sáhun, &
erlos ovar-muoda. \hld\ Þuo sęnda ina eft þanan &
Erodes se kuning \hld\ an þat ǫ́ðer folk; &
a·lêdjan hiet ina lungra mann, \hld\ ęndi lastar sprákun, &
felgidun im firin-word, \hld\ þar hie an feteron géng &
bi·hlagan mid hosku: \hld\ ni was im hugi twífli, &
neva hie it þuru ôd-muodi \hld\ all gi·þoloda; &
ne welda iro uvilun word \hld\ idug-lônon, &
hosk ęndi harm-kwidi. \hld\ Þuo bráhtun sia ina eft an þat hús innan, &
an þia palenkja uppan, \hld\ þar Pilatus was &
an þero þing-stędi. \hld\ Þegnos a·gávun &
barno þat besta \hld\ banon te handon &
sundi-lôsjan, \hld\ só hie selvo gi·kôs: &
welda manno barn \hld\ morðes a·tuomjan, &
nęrjan af nôdi. \hld\ Stuodun níð-hwata, &
Judeon far þem gast-sęlje: \hld\ habdun sia gramono barn, &
þia skola far·skundid, \hld\ þat sia ne be·skrivun iowiht &
grimmera dádjo. \hld\ Þuo gi·wêt im gangan þarod &
þegạn kêsures \hld\ wið þia þiod sprekan, &
hard hęri-togo: \hld\ „hwat, gi mi þesan haftan mann“, kwat-hie, &
„an þesan sęli sęndun \hld\ ęndi selvon an·budun, &
þat hie iuwes werodes só filo \hld\ a·werdit habdi, &
far·lêdid mid is lêron. \hld\ Nu ik mid þeson liudon ni mag, &
findan mid þius folku, \hld\ þat hie is ferahes sí &
furi þesaro skolu skuldig. \hld\ Skín was þat hiudu: &
Erodes mohta, \hld\ þie iuwan êo bi·kan, &
iuwaro liudo land-reht, \hld\ hie ni mahta is líves gi·frêson, &
þat hie hier þuru êniga sundja te dage \hld\ sweltan skoldi, &
líf far·látan. \hld\ Nu willju ik ina for þeson liudjon hier &
gi·þróon mid þingon, \hld\ þrístjon wordun, &
buotjan im is briost-hugi, \hld\ látan ina brúkan forð &
ferahes mid firjon.“ \hld\ Folk Judeono &
hreopun þuo alla samad \hld\ hlúdero stemnu, &
hietun flít-líko \hld\ ferahes áhtjan &
Krist mid kwalmu \hld\ ęndi an krúki slahan, &
wêgjan te wundron: \hld\ „hie mid is wordon havit &
dôðes gi·skuldid: \hld\ sagit þat hie drohtin sí, &
gegnungo godes suno. \hld\ Þat hie a·geldan skal, &
in·wid-spráka, \hld\ só is an u̇son êwe gi·skrivan, &
þat man su·lika firin-kwidi \hld\ ferahu kôpo.“ &
Þuo warð þie an forahton, \hld\ þie þes folkes gi·weld, &
mikilon an is muode, \hld\ þuo hie gi·hôrda þia man sprekan, &
þat sia ina selvon \hld\ sęggjan gi·hôrdin, &
gehan fur þem gum-skipe, \hld\ þat hie wári godes suno. &
Þuo hwarf im eft þie hęri-togo \hld\ an þat hús innan &
te þero þing-stędi, \hld\ þrístjon wordon &
gruotta þena godes suno \hld\ ęndi frágoda, hwat hie gumono wári: &
„hwat bist þú manno?“ \hld\ kwat-hie. „Te hwí þú mí só þínan muod hilis, &
dęrnis diop-gi·þáht? \hld\ Wêst þú þat it all an mínon duome stéd &%TODO: Check stéd.
umbi þínes líves gi·lagu? \hld\ Mí þi hębbjat þesa liudi far·gevan, &
werod Judeono, \hld\ þat ik gi·waldan muot &
só þik te spildjanne \hld\ an speres orde, &
só ti kwęlljanne an krúkjum, \hld\ só kwikan látan, &
só hweðer sí mi selvon \hld\ suotera þunkit &
te gi·frummjanne mid mínu folku.“ \hld\ Þuo sprak eft þat friðu-barn godes: &
„wêst þú þat te wáron“, \hld\ kwat-hie, „þat þú gi·wald ovar mik &
hębbjan ni mohtis, \hld\ ne wári þat it þi hêlag god &
selvo far·gávi? \hld\ Ôk hębbjat þia sundjono mêr, &
þia mik þi bi·fulhun \hld\ þuru fíond-skipi, &
gi·saldun an símon haftan.“ \hld\ Þuo welda ina sïð after þiu &
gram-hugdig man \hld\ gerno far·látan, &
þegạn kêsures, \hld\ þar hie is havdi for þero þioda gi·wald; &
ak sia węridun im þena willjon \hld\ wordu gi·hwi-liku, &
kunni Judeono: \hld\ „ne bist þú“, kwáðun sia, „þes kêsures friund, &
þínon hêrren hold, \hld\ ef þú ina hinan látis &
sïðon gi·sundon: \hld\ þat þi noh te soragan mag, &
werðan te wíte, \hld\ hwand só hwe só su·lik word sprikit, &
a·havið ina só hôho, \hld\ kwiðit þat hie hębbjan mugi &
kuning-duomes namon, \hld\ ne sí þat ina im þie kêsur geve, &
hie wirrid im is weruld-ríki \hld\ ęndi is word far·hugid, &
far·man ina an is muode. \hld\ Be·þiu skalt þú su·lik mên wrekan, &
hosk-word manag, \hld\ ef þú umbi þínes hêrren ruokis, &
umbi þínes frôhon friund-skipi, \hld\ þan skalt þú ina þiu ferhu be·niman.“ &
Þuo gi·hôrda þie hęri-togo \hld\ þia hêri Juðeono &
þrêgjan fan is þiodne; \hld\ þuo hie far þero þing-stędi géng &
selvo gi·sittjan, \hld\ þar gi·samnod was &
só mikil warf werodes, \hld\ hiet waldand Krist &
lêdjan for þia liudi. \hld\ Langoda Judeon, &
hwan êr sia þat hêlaga barn \hld\ hangon gi·sáwin, &
kwęlan an krúkje; \hld\ sia kwáðun þat sia kuning ǫ́ðran &
ne havdin undar iro hęri-skipje, \hld\ nevan þena hêran kêsar &
fan Rúmu-burg: \hld\ „þie havit hier ríki over u̇s. &
Be·þiu ni skalt þú þesan far·látan; \hld\ hie havit u̇s só filo lêðes gi·sprokan, &
far·duan havit hie im mid is dádjon. \hld\ Hie skal dôð þolon, &
wíti ęndi wundạr-kwála.“ \hld\ Werod Judeono &
só manag mis-lík þing \hld\ an mahtigna Krist &
sagdun te sundjun. \hld\ Hie swígondi stuod &
þuru ôð-muodi, \hld\ ne ant·wordida n·io·wiht &
wið iro wrêðun word: \hld\ wolda þesa wer-old alla &
lôsjan mid is lívu: \hld\ bi·þiu liet hie ina þia lêðun þiod &
wêgjan te wundron, \hld\ all só iro willjo géng: &
ni wolda im opan-líko \hld\ allon ku̇ðjan &
Judeo liudjon, \hld\ þat hie was god selvo; &
hwand wissin sia þat te wáron, \hld\ þat hie su·lika gi·wald havdi &
ovar þeson middil-gard, \hld\ þan wurði im iro muod-sevo &
gi·blôðit an iro brioston: \hld\ þan ne gi·dorstin sia þat barn godes &
handon ant·hrínan: \hld\ þan ni wurði hevan-ríki, &
ant·lokan liohto mêst \hld\ liudjo barnon. &
Be·þiu méð hie is só an is muode, \hld\ ne lét þat manno folk &
witan, hwat sia warạhtun. \hld\ Þiu wurd náhida þuo, &
mári maht godes \hld\ ęndi middi dag, &
þat sia þia ferah-kwála \hld\ frummjan skoldun. &
Þan lag þar ôk an bęndjon \hld\ an þero burg innan &
ên ruof ręgin-skaðo, \hld\ þie habda under þem ríke só filo &
morðes gi·rádan \hld\ ęndi man-slahta gi·frumid, &
was mári męgin-þiof: \hld\ ni was þar is gi·mako hwęrgin; &
was þar ôk bi sínon \hld\ sundjon gi·hęftid, &
Barrabas was hie hêtan; \hld\ hie after þem burgjon was &
þuru is mên-dádi \hld\ manogon gi·ku̇ðid. &
Þan was land-wísa \hld\ liudjo Judeono, &
þat sia járo gi·hwen \hld\ an godes minnja &
an þem hêlagon dage \hld\ ênna haftan mann &
a·biddjan skoldun, \hld\ þat im iro burges ward, &
iro folk-togo \hld\ ferah far·gávi. &
Þuo bi·gan þie hęri-togo \hld\ þia hêri Judeono, &
þat folk frágojan, \hld\ þar sia im fora stuodun, &
hweðeron sia þero twejo \hld\ tuomjan weldin, &
ferahes biddjan: \hld\ „þia hier an feteron sind &
haft undar þeson hęri-skipje?“ \hld\ Þiu hêri Judeono &
habdun þuo þia aramun man \hld\ alla gi·spanana, &
þat sia þemo land-skaðen \hld\ líf a·bádin, &
gi·þingodin þem þiove, \hld\ þie oft an þiustrja naht &
wam gi·warạhta, \hld\ ęndi waldand Krist &
kwęlidin an krúkje. \hld\ Þuo warð þat ku̇ð ovar all, &
hwó þiu þiod havda duomos a·dêlid. \hld\ Þuo skoldun sia þia dád frummjan, &
háhan þat hêlaga barn. \hld\ Þat warð þem hęri-togen &
sïðor te sorgon, \hld\ þat hie þia saka wissa, &
þat sia þuru níð-skipi \hld\ nęrjendon Krist, &
hatoda þiu hêri, \hld\ ęndi hie im hôrda te þiu, &
warạhta iro willjon: \hld\ þes hie wíti ant·féng, &
lôn an þeson liohte \hld\ ęndi lang after, &
wói sïðor wann, \hld\ sïðor hie þesa wer-old a·gaf. &%NOTE: wói checked.
Þuo warð þas þie wrêðo gi·waro, \hld\ wam-skaðono mêst, &
Satanas selvo, \hld\ þuo þiu seola kwam &
Judases an grund \hld\ grimmaro hęlljun— &
þuo wissa hie te wáren, \hld\ þat þat was waldand Krist, &
barn drohtines, \hld\ þat þar gi·bundan stuod; &
wissa þuo te wáron, \hld\ þat hie welda þesa wer-old alla &
mid is henginnja \hld\ hęllja gi·þwinges, &
liudi a·lôsjan \hld\ an lioht godes. &
Þat was Satanase \hld\ sêr an muode, &
tulgo harm an is hugje: \hld\ welda is helpan þuo, &
þat im liudjo barn \hld\ líf ne bi·námin, &
ne kwęlidin an krúkje, \hld\ ak hie welda, þat hie kwik livdi, &
te þiu þat firiho barn \hld\ fernes ne wurðin, &
sundjono sikura. \hld\ Satanas gi·wêt im þuo, &
þar þes hęri-togen \hld\ híwiski was &
an þero burg innan. \hld\ Hie þero is brúdi bi·gann, &
þera idis opan-líko \hld\ un·hiuri fíond &
wunder tôgjan, \hld\ þat sia an word-helpon &
Kriste wári, \hld\ þat hie muosti kwik libbjan, &
drohtin manno \hld\ —hie was iu þan te dôðe gi·skęrid— &
wissa þat te wáron, \hld\ þat hie im skoldi þia gi·wald bi·niman, &
þat hie sia ovar þesan middil-gard \hld\ só mikila ni havdi, &
ovar wída wer-old. \hld\ Þat wíf warð þuo an forahton, &
swíðo an sorogon, \hld\ þuo iru þiu gi·siuni kwámun &
þuru þes dęrnjen dád \hld\ an dages liohte, &
an hęlið-helme bi·helid. \hld\ Þuo siu te iru hêrren an·bôd, &
þat wíf mid iro wordon \hld\ ęndi im te wáren hiet &
selvon sęggjan, \hld\ hwat iro þar te gi·siunjon kwam &
þuru þena hêlagan mann, \hld\ ęndi im helpan bad, &
formon is ferhe: \hld\ „ik hębbju hier só filo þuru ina &
seld-líkes gi·sewan, \hld\ só ik wêt, þat þia sundjun skulun &
allaro erlo gi·hwem \hld\ uvilo gi·þíhan, &
só im fruokno tuo \hld\ ferahes áhtið.“ &
Þie sęgg warð þuo an sïðe, \hld\ antat hie sittjan fand &
þena hęri-togon \hld\ an hwarạve innan &
an þem stên-wege, \hld\ þar þiu stráta was &
felison gi·fuogid. \hld\ Þar hie te is frôhon géng, &
sagda im þes wíves word. \hld\ Þuo warð im wrêð hugi, &
þem hęri-togen, \hld\ —hwarạvoda an innan—, &
gi·blôðit briost-gi·þáht: \hld\ was im bêðjes wê, &
gie þat sea ina sluogin \hld\ sundja lôsan, &
gie it bi þem liudjon þuo \hld\ for·látan ne gi·dorsta &
þuru þes werodes word. \hld\ Warð im gi·węndid þuo &
hugi an herten \hld\ after þero hêri Judeono, &
te werkjanne iro willjon: \hld\ ne wardoda im nie-wiht &
þia swárun sundjun, \hld\ þia hie im þar þuo selvo gi·deda. &
Hiet im þuo te is handon dragan \hld\ hluttran brunnjon, &
watar an wégje, \hld\ þar hie furi þem werode sat, &
þwóg ina þar for þero þioda \hld\ þegạn kêsures, &
hard hęri-togo \hld\ ęndi þuo fur þero hêri sprak, &
kwað þat hie ina þero sundjono þar \hld\ sikoran dádi, &
wrêðero werko: \hld\ „ne willju ik þes wihtes plegan“, kwat-hie, &
„umbi þesan hêlagan mann, \hld\ ak hleotad gi þes alles, &
gie wordo gie werko, \hld\ þes gi im hér te wítje gi·duan.“ &
Þuo hreop all saman \hld\ hęri-skipi Judeono, &
þiu mikila męnigi, \hld\ kwáðun þat sia weldin umbi þena man plegan &
deravoro dádjo: \hld\ „fare is drôr ovar u̇s, &
is bluod ęndi is baneði \hld\ ęndi ovar u̇sa barn só samo, &
ovar u̇sa avaron þar after \hld\ —wí willjat is alles plegan“, kwaðun sia, &
„umbi þena slegi selvon,— \hld\ ef wí þar êniga sundja gi·duan!“ &
A·gevan warð þar þuo furi þem Judeon \hld\ allaro gumono besta &
hettendjon an hand, \hld\ an heru-bęndjon &
narawo gi·nôdid, \hld\ þar ina níð-hwata, &
fíond ant·féngun: \hld\ folk ina umbi·hwarf, &
mên-skaðono męgin. \hld\ Mahtig drohtin &
þoloda gi·þuldjon, \hld\ só hwat só im þiu þioda deda. &
Sia hietun ina þuo filljan, \hld\ êr þan sia im ferahes tuo, &
aldres áhtin, \hld\ ęndi im undar is ôgun spiwun, &
dedun im þat te hoske, \hld\ þat sia mid iro handon slógun, &
weros an is wangun \hld\ ęndi im is gi·wádi bi·námun, &
róvodun ina þia ręgin-skaðon, \hld\ rôdes lakanes &
dedun im eft ǫ́ðer an \hld\ þuru un·huldi; &
hietun þuo hôvid-band \hld\ hardaro þorno &
wundron windan \hld\ ęndi an waldand Krist &
selvon sęttjan, \hld\ ęndi géngun im þia gi·sïðos tuo, &
kwęddun ina an kuning-wísu \hld\ ęndi þar an knio fellun, &
hnigun im mid iro hôvdu: \hld\ all was im þat te hoske gi·duan, &
þoh hie it all gi·þolodi, \hld\ þiodo drohtin, &
mahtig þuru þia minnja \hld\ manno kunnjes. &
Hietun sia þuo wirkjan \hld\ wápnes ęggjon &
hęliðos mid iro handon \hld\ hardes bômes &
kraftiga krúki \hld\ ęndi hietun sia Kristan þuo, &
sálig barn godes \hld\ selvon fuorjan, &
dragan hietun sia u̇san drohtin, \hld\ þar hie be·drôragad skolda &
sweltan sundjono lôs. \hld\ Síðodun Judeon, &
weros an willon, \hld\ lêddun waldand Krist, &
drohtin te dôðe. \hld\ Þar mohta man þuo derevi þing &
harm-lík gi·hôrjan: \hld\ hiovandi þar after &
géngun wíf mid wópu, \hld\ weros gnornodun, &
þia fan Galilea mid im \hld\ gangan kwámun, &
folgodun ovar ferr-wegos: \hld\ was im iro frôhon dôð &
swíðo an soragan. \hld\ Þuo hie selvo sprak, &
barno þat besta \hld\ ęndi under bak be·sah, &
hiet þat sia ni wépin: \hld\ „ni þarf iu wiht tregan“, kwat-hie, &
„mínero hin-fęrdjo, \hld\ ak gi mid hofnu mugun &
iuwa wrêðan werk \hld\ wópu kúmjan, &
tornon trahnon. \hld\ Noh wirðið þiu tíd kuman, &
þat þia muoder þes \hld\ męndendja sind, &
brúdi Judeono, \hld\ þem gio barn ni warð &
ôdan an aldre. \hld\ Þan gi iuwa in·wid skulun &
grimmo an·geldan; \hld\ þan gi só gerna sind, &
þat iu hier bi·hlídan \hld\ hôha bergos, &
diopo be·delvan; \hld\ dôð wári iu þan allon &
liovera an þeson lande \hld\ þan su·lik liudjo kwalm &
te gi·þoljanne, \hld\ só hier þan þesaro þioda kumid.“ &
Þuo sia þar an griete \hld\ galgon rihtun, &
an þem felde uppan \hld\ folk Judeono, &
bôm an berege, \hld\ ęndi þar an þat barn godes &
kwęlidun an krúkje: \hld\ slógun kald ísarn, &
niwa naglos \hld\ níðon skarpa &
hardo mid hamuron \hld\ þuru is hęndi ęndi þuru is fuoti, &
bittra bęndi: \hld\ is blód ran an erða, &
drôr fan u̇son drohtine. \hld\ Hie ni welda þoh þia dád wrekan &
grimma an þem Judeon, \hld\ ak hie þes god fader &
mahtigna bad, \hld\ þat hie ni wári þem manno folke, &
þem werode þiu wrêðra: \hld\ „hwand sia ni witun, hwat sia duot“, kwat-hie. &
Þuo þia wígandos \hld\ gi·wádi Kristes, &
drohtines dêldun, \hld\ derevja mann, &
þes ríken gi·rôbi. \hld\ Þia rinkos ni mahtun &
umbi þena selvon {[...]} \hld\ sam-wurdi gi·sprekan, &
êr sia an iro hwarạve \hld\ hlôtos wurpun, &
hwi-lik iro skoldi hębbjan \hld\ þia hêlagun pêda, &
allaro gi·wádjo wun-samost. \hld\ Þes werodes hirdi &
hiet þuo, þe hęri-togo, \hld\ ovar þem hôvde selves &
Kristes an krúke skrívan, \hld\ þat þat wári kuning Judeono, &
Jesus fan Nazareth-burh, \hld\ þie þar nęglid stuod &
an niwon galgon þuru \hld\ níð-skipi, &
an bômin treo. \hld\ Þuo bádun þia liudi &
þat word węndjan, \hld\ kwáðun þat hie im só an is willjon spráki, &
selvo sagdi, \hld\ þat hie habdi þes gi·sïðes gi·wald, &
kuning wári ovar Judeon. \hld\ Þuo sprak eft þie kêsures bodo, &
hard hęri-togo: \hld\ „it ist iu só ovar is hôvde gi·skrivan, &
wís-líko gi·writan, \hld\ só ik it nu węndjan ni mag.“ &
Dádun þuo þar te wítje \hld\ werod Judeono &
twêna far·talda man \hld\ an twá halva &
Kristes an krúki: \hld\ lietun sia kwalm þolon &
an þem warạg-trewe \hld\ werko te lône, &
lêðaro dádjo. \hld\ Þia liudi sprákun &
hosk-word manag \hld\ hêlagon Kriste, &
grottun ina mid gelpu: \hld\ sáwun allaro gumono þen beston &
kwęlan an þemo krúkje: \hld\ „ef þú sís kuning ovar all“, kwáðun sia, &
„suno drohtines, \hld\ só þú havis selvo gi·sprokan, &
nęri þik fan þero nôdi \hld\ ęndi níðes a·tuomi, &
gang þi hêl herod; \hld\ þan welljat an þik hęliðo barn, &
þesa liudi gi·lôvjan.“ \hld\ Sum imo ôk lastar sprak &
swíðo gêl-hert Judeo, \hld\ þar hie fur þem galgon stuod: &
„wah warð þesaro wer-oldi“, \hld\ kwat-hie, „ef þú iro skoldis gi·wald êgan. &
Þu sagdas þat þú mahtis an ênon dage \hld\ all te·werpan &
þat hôha hús \hld\ hevan-kuninges, &
stên-werko mêst \hld\ ęndi eft standan gi·duon &
an þriddjon dage, \hld\ só is elkor ni þorfti bi·þíhan mann &
þeses folkes furðor. \hld\ Sínu hwó þú nu gi·fastnod stés, &
swíðo gi·sêrid: \hld\ ni maht þi selvon wiht &
balowes gi·buotjan.“ \hld\ Þuo þar ôk an þem bęndjon sprak &
þero þeovo ǫ́ðer, \hld\ all só hie þia þioda gi·hôrda, &
wrêðon wordon \hld\ —ne was is willjo guod, &
þes þegnes gi·þáht—: \hld\ „ef þú sís þiod-kuning“, kwat-hie, &
„Krist, godes suno, \hld\ gang þi þan fan þem krúke niðer, &
slópi þi fan þem símon \hld\ ęndi u̇s samad allon &
hilp ęndi hêli. \hld\ Ef þú sís hevan-kuning, &
waldand þesaro wer-oldes, \hld\ gi·duo it þan an þínon werkon skín, &
mári þik fur þesaro męnigi.“ \hld\ Þuo sprak þero manno ǫ́ðer &
an þero hęnginna, \hld\ þar hie gi·hęftid stuod, &
wan wunder-kwála: \hld\ „be·hwí wilt þú su·lik word sprekan, &
gruotis ina mid gelpu? \hld\ stés þi hier an galgen haft, &
gi·brókan an bôme. \hld\ Wit hier bêðja þolod &
sêr þuru unka sundjun: \hld\ is unk unkero selvero dád &
worðan te wítje. \hld\ Hie stéd hier wammes lôs, &
allaro sundjono sikur, \hld\ só hie selvo gio &
firina ni gi·frumida, \hld\ botan þat hie þuru þeses folkes nið &
willendi an þesaro weruldi \hld\ wíti ant·fáhid. &
Ik willju þar gi·lôvjan tuo“, \hld\ kwat-hie, „ęndi willju þena landes ward, &
þena godes suno \hld\ gerno biddjan, &
þat þú mín gi·huggjes \hld\ ęndi an helpun sís, &
rádendero best, \hld\ þan þú an þín ríki kumis: &
wes mi þan gi·náðig.“ \hld\ Þuo sprak im eft nęrjendo Krist &
wordon te·gęgnes: \hld\ „ik sęggju þi te wáron hier“, kwat-hie, &
„þat þú noh hiu-du móst \hld\ an himil-ríke &
mid mí samad \hld\ sehan lioht godes, &
an þemo Paradyse, \hld\ þoh þú nu an su·likoro pínu sís.“ &
Þan stuod þar ôk Maria, \hld\ muoder Kristes, &
blêk under þem bôme, \hld\ gi·sah iro barn þolon, &
winnan wunder-kwála. \hld\ Ôk wárun þar wíf mid iro &
an só mahtiges \hld\ minnja kumana— &
þan stuod þar ôk Johannes, \hld\ jungro Kristes, &
hriwi undar is hêrren, \hld\ was im is hugi sêrag— &
drúvodun fur þem dôðe. \hld\ Þar sprak drohtin Krist &
mahtig te þero muoder: \hld\ „nu ik þí hier mínemo skal &
jungron be·felhan, \hld\ þem þí hier gęgin-ward stéd: &
wis þí an is gi·sïðje samad: \hld\ þú skalt ina furi suno hębbjan.“ &
Grótta hie þuo Johannes, \hld\ hiet þat hie iru ful-géngi wel, &
minnjodi sia só mildo, \hld\ só man is muoder skal, &
idis un·wamma. \hld\ Þuo hie sia an is êra ant·féng &
þuru hluttran hugi, \hld\ só im is hêrro gi·bôd. &
Þuo warð þar an middjan dag \hld\ mahtig têkạn, &
wundạr-lík gi·warạht \hld\ ovar þesan wer-old allan, &
þuo man þena godes suno \hld\ an þena galgon huof, &
Krist an þat krúki: \hld\ þuo warð it ku̇ð ovar all, &
hwó þiu sunna warð gi·sworkan: \hld\ ni mahta swigli lioht &
skóni gi·skínan, \hld\ ak sia skado far·féng, &
þimm ęndi þiustri \hld\ ęndi só gi·þrusmod neval. &
Warð allaro dago druovost, \hld\ dunkar swíðo &
ovar þesan wídun weruld, \hld\ só lango só waldand Krist &
kwal an þemo krúkje, \hld\ kuningo ríkost, &
ant nuon dages. \hld\ Þuo þie neval ti-skrêd, &
þat gi·swerk warð þuo te·swungan, \hld\ bi·gan sunnun lioht &
hêdron an himile. \hld\ Þuo hreop up te gode &
allaro kuningo kraftigost, \hld\ þuo hie an þemo krúkje stuod &
faðmon gi·fastnot: \hld\ „fader alo-mahtig“, kwat-hie, &
„te hwí þú mik só far·lieti, \hld\ lievo drohtin, &
hêlag hevan-kuning, \hld\ ęndi þína helpa dedos, &
fullisti só ferr? \hld\ Ik standu under þeson fíondon hier &
wundron gi·wêgid.“ \hld\ Werod Judeono &
hlógun is im þuo te hoske: \hld\ gi·hôrdun þena hêlagun Krist, &
drohtin furi þem dôðe \hld\ drinkan biddjan, &
kwað þat ina þurstidi. \hld\ Þiu þioda ne latta, &
wrêða wiðar-sakon: \hld\ was im willjo mikil, &
hwat sia im bittres tuo \hld\ bringan mahtin. &
Habdun im un·swóti \hld\ ekid ęndi galla &
gi·mengid þia mên-hwaton; \hld\ stuod ên mann garo, &
swíðo skuldig skaðo, \hld\ þena habdun sia gi·skęrid te þiu, &
far·spanan mid sprákon, \hld\ þat hie sia en êna spunsia nam, &
líðo þes lêðosten, \hld\ druog it an ênon langan skafte, &
gi·bundan an ênon bôme \hld\ ęndi deda it þem barne godes, &
mahtigon te mu̇ðe. \hld\ Hie an·kęnda iro mirkjun dádi, &
gi·fuolda iro fégnes: \hld\ furðor ni welda &
is só bittres an·bítan, \hld\ ak hreop þat barn godes &
hlúdo te þem himiliskon fader: \hld\ „ik an þina hęndi be·filhu“, kwat-hie, &
„mínon gêst an godes willjon; \hld\ hie ist nu garo te þiu, &
fu̇s te faranne.“ \hld\ Firiho drohtin &
gi·hnêgida þuo is hôvid, \hld\ hêlagon áðom &
liet fan þemo lík-hamen. \hld\ Só þuo þie landes ward &
swalt an þem símon, \hld\ só warð sán after þiu &
wundạr-têkạn gi·warạht, \hld\ þat þar waldandes dôð &
un·kweðandes só filo \hld\ ant·kęnnjan skolda, &
þiadnes ên-dagon: \hld\ erða bivoda, &
hrisidun þia hôhun bergos, \hld\ harda stênos kluvun, &
felisos after þem felde, \hld\ ęndi þat fêha lakan te·brast &
an middjon an twê, \hld\ þat êr managan dag &
an þemo wíhe innan \hld\ wundron gi·striunid &
hêl hangoda \hld\ —ni muostun hęliðo barn, &
þia liudi skawon, \hld\ hwat under þemo lakane was &
hêlages be·hangan: \hld\ þuo mohtun an þat horð sehan &
Judeo liudi— \hld\ gravu wurðun gi·opanod &
dôdero manno, \hld\ ęndi sia þuru drohtines kraft &
an iro lík-hamon \hld\ libbjandi a·stuodun &
up fan erðu \hld\ ęndi wurðun gi·ôgida þar &
mannon te márðu. \hld\ Þat was só mahtig þing, &
þat þar Kristes dôð \hld\ ant·kęnnjan skoldun, &
só filo þes gi·fuoljan, \hld\ þie gio mid firihon ne sprak &
word an þesaro wer-oldi. \hld\ Werod Judeono &
sáwun seld-lík þing, \hld\ ak was im iro slíði hugi &
só far·hardod an iro herten, \hld\ þat þar io só hêlag ni warð &
têkạn gi·tôgid, \hld\ þat sia trúodin þiu bat &
an þia Kristes kraft, \hld\ þat hie kuning ovar all, &
þes werodes wári. \hld\ Suma sia þar mid iro wordon gi·sprákun, &
þia þes hrêwes þar \hld\ huodjan skoldun, &
þat þat wári te wáren \hld\ waldandes suno, &
godes gegnungo, \hld\ þat þar an þem galgon swalt, &
barno þat besta. \hld\ Slógun an iro briost filo &
wópjandero wívo: \hld\ was im þiu wunder-kwála &
harm an iro herten \hld\ ęndi iro hêrren dôð &
swíðo an sorogon. \hld\ Þan was sido Judeono, &
þat sia þia haftun þuru þena hêlagon dag \hld\ hangon ni lietin &
lęngerun hwíla, \hld\ þan im þat líf skriði, &
þiu seola be·sunki: \hld\ slíð-muoda mann &
géngun im mid níð-skipju náhor, \hld\ þar só be·nęglida stuodun &
þeovos twêna, \hld\ þolodun bêðja &
kwála bi Kriste: \hld\ wárun im kwika noh þan, &
unt-þat sia þia grimmun \hld\ Judeo liudi &
bênon be·brákon, \hld\ þat sia bêðja samad &
líf far·lietun, \hld\ suohtun im lioht ǫ́ðer. &
Sia ni þorftun \alst{d}rohtin Krist \hld\ \alst{d}ôðes bêdjan &
\alst{f}urðor mid ênigon \alst{f}irinon: \hld\ fundun ina gi·\alst{f}aranan þuo iu: &
is \alst{s}eola was gi·\alst{s}ęndid \hld\ an \alst{s}uoðan weg, &
an \alst{l}ang-sam \alst{l}ioht, \hld\ is \alst{l}iði kuolodun, &
þat \alst{f}erah was af þem \alst{f}lêske. \hld\ Þuo géng im ên þero \alst{f}íondo tuo &
an \alst{n}íð-hugi, \hld\ druog \alst{n}ęgilid sper &
\alst{h}ard an is \alst{h}andon, \hld\ mid \alst{h}eru-þrummjon stak, &
liet \alst{w}ápnes ord \hld\ \alst{w}undum sníðan, &
þat an \alst{s}elves warð \hld\ \alst{s}ídu Kristes &
ant·\alst{l}okan is \alst{l}ík-hamo. \hld\ Þia \alst{l}iudi gi·sáwun, &
þat þanan \alst{b}luod ęndi water \hld\ \alst{b}êðju sprungun, &
\alst{w}ellun fan þero \alst{w}undun, \hld\ all só is \alst{w}illjo géng &
ęndi hie habda gi·\alst{m}arkod êr \hld\ \alst{m}anno kunnje, &
\alst{f}iriho barnon te \alst{f}rumu: \hld\ þuo was it all gi·\alst{f}ullid só. &
Só þuo gi·ségid warð \hld\ seðle náhor &
hêdra sunna \hld\ mid hevan-tunglon &
an þem druoven dage, \hld\ þuo géng im u̇ses drohtines þegạn &
—was im glau gumo, \hld\ jungro Kristes &
managa hwíla, \hld\ só it þar manno filo &
ne wissa te wáron, \hld\ hwand hie it mid is wordon hal &
Juðeono gum-skipje: \hld\ Joseph was hie hêtan, &
darnungo was hie u̇ses drohtines jungro: \hld\ hie ni welda þero far·duanun þiod &
folgon te ênigon firin-werkon, \hld\ ak hie bêd im under þem folke Judeono, &
hêlag himilo ríkjes— \hld\ hie géng im þuo wið þena hęri-togon mahljan, &
þingon wið þena þegạn kêsures, \hld\ þigida ina gerno, &
þat hie muosti a·lôsjan \hld\ þena lík-hamon &
Kristes fan þemo krúkje, \hld\ þie þar gi·kwelmid stuod, &
þes guoden fan þem galgen \hld\ ęndi an graf lęggjan, &
foldu bi·felahan. \hld\ Im ni welda þie folk-togo þuo &
węrnjan þes willjen, \hld\ ak im gi·wald far·gaf, &
þat hie só muosti gi·frummjan. \hld\ Hie gi·wêt im þuo forð þanan &
gangan te þem galgon, \hld\ þar hie wissa þat godes barn, &
hrêo hangondi \hld\ hêrren sínes, &
nam ina þuo an þero niwun ruodun \hld\ ęndi ina fan naglon a·tuomda, &
ant·féng ina mid is faðmon, \hld\ só man is frôhon skal, &
lioves lík-hamon, \hld\ ęndi ina an líne bi·wand, &
druog ina diur-líko \hld\ —só was þie drohtin werð—, &
þar sia þia stędi havdun \hld\ an ênon stêne innan &
handon gi·hauwan, \hld\ þar gio hęliðo barn &
gumon ne bi·gruovon. \hld\ Þar sia þat godes barn &
te iro land-wísu, \hld\ líko hêlgost &
foldu bi·fulhun \hld\ ęndi mid ênu felisu be·lukun &
allaro gravo guod-líkost. \hld\ Griotandi sátun &
idisi arm-skapana, \hld\ þia þat all for·sáwun, &
þes gumen grimman dôð. \hld\ Gi·witun im þuo gangan þanan &
wópjandi wíf \hld\ ęndi wara námun, &
hwó sia eft te þem grave \hld\ gangan mahtin: &
havdun im far·sewana \hld\ soroga gi·nuogja, &
mikila muod-kara: \hld\ Maria wárun sia hêtana, &
idisi arm-skapana. \hld\ Þuo warð ávand kuman, &
naht mid neflu. \hld\ Niðfolk Judeono &
warð an moragan eft, \hld\ męnigi gi·samnod, &
rekidun an rúnon: \hld\ „hwat, þú wêst, hwó þit ríki was &
þuru þesan ênan man \hld\ all gi·twíflid, &
werod gi·worran: \hld\ nu ligid hie wundon siok, &
diopa bi·dolvan. \hld\ Hie sagda simnen, þat hie skoldi fan dôðe a·standan &
an þriddjan dage. \hld\ Þius þiod gi·lôvit te filo, &
þit werod after is wordon. \hld\ Nu þú hier wardon hét, &
ovar þem grave gômjan, \hld\ þat ina is jungron þar &
ne far·stelan an þemo stêne \hld\ ęndi sęggjan þan, þat hie a·standan sí, &
ríki fan raston: \hld\ þan wirðit þit rinko folk &
mêr gi·męrrid, \hld\ ef sia it bi·ginnat márjan hier.“ &
Þuo wurðun þar gi·skęrida \hld\ fan þero skolu Judeono &
weros te þero wahtu: \hld\ gi·witun im mid iro gi·wápnjon þarod &
te þem grave gangan, \hld\ þar sia skoldun þes godes barnes &
hrêwes huodjan. \hld\ Warð þie hêlago dag &
Judeono far·gangan. \hld\ Sia ovar þemo grave sátun, &
weros an þero wahtun \hld\ wannom nahton, &
bidun undar iro bordon, \hld\ hwan êr þie berẹhto dag &
ovar middil-gard \hld\ mannon kwámi, &
liudon te liohte. \hld\ Þuo ni was lang te þiu, &
þat þar warð þie gêst kuman \hld\ be godes krafte, &
hálag áðom \hld\ undar þena hardon stên &
an þena lík-hamon. \hld\ Lioht was þuo gi·opanod &
firiho barnon te frumu: \hld\ was ferkal manag &
ant·hęftid fan hęll-doron \hld\ ęndi te himile weg &
gi·warạht fan þesaro wer-oldi. \hld\ Wánom up a·stuod &
friðu-barn godes, \hld\ fuor im þuo þar hie welda, &
só þia wardos þes \hld\ wiht ni af·swovun, &
dęrvja liudi, \hld\ hwan hie fan þem dôðe a·stuod, &
a·rês fan þero rastun. \hld\ Rinkos sátun &
umbi þat graf útan, \hld\ Judeo liudi, &
skola mid iro skildjon. \hld\ Skrêd forð-wardes &
swigli sunnun lioht. \hld\ Síðodun idisi &
te þem grave gangan, \hld\ gum-kunnjes wíf, &
Mariun muni-líka: \hld\ habdun mêðmo filo &
gi·sald wiðer salvum, \hld\ siluvres ęndi goldes, &
werðes wiðer wurtjon, \hld\ só sia mahtun a·winnan mêst, &
þat sia þena lík-hamon \hld\ lioves hêrren, &
suno drohtines, \hld\ salvon muostin, &
wundun writanan. \hld\ Þiu wíf soragodun &
an iro sevon swíðo, \hld\ ęndi suma sprákun, &
hwie im þena grôtan stên \hld\ fan þemo grave skoldi &
gi·hwerevjan an halva, \hld\ þe sia ovar þat hrêo sáwun &
þia liudi lęggjan, \hld\ þuo sia þena lík-hamon þar &
be·fulhun an þemo felise. \hld\ Só þiu frí havdun &
ge·gangan te þem gardon, \hld\ þat sia te þem grave mahtun &
gi·sehan selvon, \hld\ þuo þar swógan kwam &
ęngil þes alo-waldon \hld\ ovana fan radure, &
faran an feðer-hamon, \hld\ þat all þiu folda an skian, &
þiu erða dunida \hld\ ęndi þia erlos wurðun &
an wêkan hugje, \hld\ wardos Juðeono, &
bi·fellun bi þem forahton: \hld\ ne wándun ira ferah êgan, &
líf langerun hwíl. \hld\ Lágun þa wardos, &
þia gi·sïðos sám-kwika: \hld\ sán up a·hléd &%TODO: lemma of hléd unclear.
þie grôto stên fan þem grave, \hld\ só ina þie godes ęngil &
gi·hwerivida an halva, \hld\ ęndi im uppan þem hlêwe gi·sat &
diur-lík drohtines bodo. \hld\ Hie was an is dádjon ge·lík, &
an is an·siunjon, \hld\ só hwem só ina muosta undar is ôgon skawon, &
só berẹht ęndi só blíði \hld\ all só bliksmun lioht; &
was im is gi·wádi \hld\ wintạr-kaldon &
snêwe gi·líkost. \hld\ Þuo sáwun sia ina sittjan þar, &
þiu wíf uppan þem gi·węndidan stêne, \hld\ ęndi im fan þem wlitje kwámun, &
þem idison su·lika ęgison te·gęgnes: \hld\ all wurðun fan þem grurje &
þiu frí an forahton mikilon, \hld\ furðor ne gi·dorstun &
te þemo grave gangan, \hld\ êr sia þie godes ęngil, &
waldandes bodo \hld\ wordon gruotta, &
kwað þat hie iro ârundi \hld\ all bi·kunsti, &
werk ęndi willjon \hld\ ęndi þero wívo hugi, &
hiet þat sia im ne and-rédin: \hld\ „ik wêt þat gi iuwan drohtin suokat, &
nęrjendon Krist \hld\ fan Nazareth-burg, &
þena þi hier kwęlidun \hld\ ęndi an krúki slógun &
Judeo liudi \hld\ ęndi an graf lagdun &
sundi-lôsjan. \hld\ Nu nist hie selvo hier, &
ak hie ist a·standan iu, \hld\ ęndi sind þesa stędi lárja, &%NOTE ms. -- a·standan] L 1r.
þit graf an þeson griote. \hld\ Nu mugun gi gangan herod &
náhor mikilu \hld\ —ik wêt þat is iu ist niud sehan &
an þeson stêne innan—: \hld\ hier sind noh þia stędi skína, &
þar is lík-hamo lag.“ \hld\ Lungra fengun &
gi·bada an iro brioston \hld\ blêka idisi, &
wliti-skóni wíf: \hld\ was im wil-spell mikil &
te gi·hôrjanne, \hld\ þat im fan iro hêrren sagda &
ęngil þes alo-walden. \hld\ Hiet sia eft þanan &
fan þem grave gangan ęndi faran \hld\ te þem jungron Kristes, &
sęggjan þem is gi·sïðon \hld\ suoðon wordon, &
þat iro drohtin was \hld\ fan dôðe a·standan. &
Hiet ôk an sundron \hld\ Símon Petruse &
will-spell mikil \hld\ wordon ku̇ðjan, &
kumi drohtines, \hld\ gie þat Krist selvo &
was an Galileo land, \hld\ „þar ina eft is jungron skulun, &
gi·sehan is gi·sïðos, \hld\ só hie im êr selvo gi·sprak &
wárom wordon.“ \hld\ Reht só þuo þiu wíf þanan &
gangan weldun, \hld\ só stuodun im te·gęgnes þar &
ęngilos twêna \hld\ an ala-hwíton &
wánamon gi·wádjom \hld\ ęndi sprákun im mid iro wordon tuo &
hêlag-líko: \hld\ hugi warð gi·blôðid &
þen idison an ęgison: \hld\ ne mahtun an þia ęngilos godes &
bi þemo wlite skawon: \hld\ was im þiu wánami te strang, &%NOTE ms. -- strang] L 1v.
te swíði te sehanne. \hld\ Þuo sprákun \edtext{im sán}{\Afootnote{so C; om. L}} an·gęgin &
waldandes bodun \hld\ ęndi þiu wíf frágodun, &
te hwí sia Kristan þarod \hld\ kwikan mid dôdon, &
suno drohtines \hld\ suokjan kwámin &
ferahes fullan; \hld\ „nu gi ina ni findat hier &
an þeson stên-grave, \hld\ ak hie ist a·standan nu &
an is lík-hamon: \hld\ þes gi gi·lôvjan skulun &
ęndi gi·huggjan þero wordo, \hld\ þe hie iu te wáron oft &
selvo sagda, \hld\ þan hie an iuwon ge·sïðja was &
an Galilea-lande, \hld\ hwó hie skoldi gi·gevan werðan, &
gi·sald selvo \hld\ an sundigaro manno, &
hettjandero hand, \hld\ hêlag drohtin, &
þat sea ina kwęlidin \hld\ ęndi an krúki slógin, &
dôdan gi·dádin \hld\ ęndi þat hie skoldi þuruh drohtines kraft &
an þriddjon dage \hld\ þioda te willjan &
libbjandi a·standan. \hld\ Nu havat hie all gi·lêstid só, &
ge·frumid mid firihon: \hld\ íljat gi nu forð hinan, &
gangat gáh-líko \hld\ ęndi duot it þem is jungron ku̇ð. &
Hie havat sia iu fur·farana \hld\ ęndi ist im forð hinan &
an Galileo land, \hld\ þar ina eft is jungron skulun, &
gi·sehan is ge·sïðos.“ \hld\ Þuo warð \edtext{sán}{\Afootnote{so L; om. C}} after þiu &
þem wívon an willjon, \hld\ þat sia gi·hôrdun su·lik word sprekan, &
ku̇ðjan þia kraft godes \hld\ —wárun im só a·kumana þuo noh &
gie só forahta ge·frumida—: \hld\ gi·witun im forð þanan &%NOTE ms. -- forahta] L end.
fan þem grave gangan \hld\ ęndi sagdun þem jungron Kristes &
seld-lík gi·siuni, \hld\ þar sia sorogondi &
bidun su·likero buota. \hld\ Þuo wurðun ôk an þia burg kumana &
Judeono wardos, \hld\ þia ovar þemo grave sátun &
alla langa naht \hld\ ęndi þes lík-hamen þar, &
huodun þes hrêwes. \hld\ Sia sagdun þero hêri Judeono, &
hwi-lika im þar and-warda \hld\ ęgison kwámun, &
seld-lík gi·siuni, \hld\ sagdun mid wordon, &
al só it gi·duan was \hld\ an þero drohtines kraft, &
ni miðun an iro muode. \hld\ Þuo budun im mêðmo filo &
Judeo liudi, \hld\ gold ęndi siluvar, &
saldun im sink manag, \hld\ te þiu þat sia it ni sagdin forð, &
ne máridin þero męnigi: \hld\ „ak kweðat þat iu móði hugi &
an·swevidi mid slápu \hld\ ęndi þat þar kwámin is gi·sïðos tuo, &
far·stálin ina an þem stêne. \hld\ Simnen wesat gi an stríde mid þiu, &
forð an flíte: \hld\ ef it wirðit þem folk-togen ku̇ð, &
wí gi·helpat iu wið þena hêrosten, \hld\ þat hie iu harmes wiht, &
lêðes ni gi·lêstid.“ \hld\ Þuo námun sia an þem liudon filo &
diurero mêðmo, \hld\ dádun all só sia bi·gunnun &
—ne gi·weldun iro willjon— \hld\ dádun só wído ku̇ð &
þem liudon after þem lande, \hld\ þat sia su·lika lugina woldun &
a·hębbjan be þan hêlagan drohtin. \hld\ Þan was eft gi·hêlid hugi &
jungron Kristes, \hld\ þuo sia gi·hôrdun þiu guodun wíf &
márjan þia maht godes; \hld\ þuo wárun sia an iro muode fráha, &
gie im te þem grave bêðja, \hld\ Johannes ęndi Petrus &
runnun ovast-líko: \hld\ warð êr kuman &
Johannes þie guodo, \hld\ ęndi im ovar þem grave gi·stuod, &
antat þar sán after kwam \hld\ Símon Petrus, &
erl ęllan-ruof \hld\ ęndi im þar in gi·wêt &
an þat graf gangan: \hld\ gi·sah þar þes godes barnes, &
hrêo-gi·wádi \hld\ hêrren sínes &
línin liggjan, \hld\ mid þiu was êr þie lík-hamo &
fagaro bi·fangan; \hld\ lag þie fano sundar, &
mit þem was þat hôvid bi·helid \hld\ hêlages Kristes, &
ríkjes drohtines, \hld\ þan hie an þesaro rastu was. &
Þuo géng im ôk Johannes \hld\ an þat graf innan &
sehan seld-lík þing; \hld\ warð im sán after þiu &
ant·lokan is gi·lôvo, \hld\ þat hie wissa, þat skolda eft an þit lioht kuman &
is drohtin diur-líko, \hld\ fan dôðe a·standan &
up fan erðu. \hld\ Þuo gi·witun im eft þanan &
Johannes ęndi Petrus, \hld\ ęndi kwámun þia jungron Kristes, &
þia gi·sïðos te·samne. \hld\ Þan stuod sêrag-muod &
ên þera idiso \hld\ ǫ́ðer-sïðu &
griotandi ovar þem grave, \hld\ was iro jámar muod— &
Maria was þat Magdalena—, \hld\ was iro muod-gi·þáht, &
sevo mit sorogon gi·blandan, \hld\ ne wissa hwarod siu sókjan skolda &
þena hêrron, þar iro wárun at þia helpa gi·langa. \hld\ Siu ni mohta þuo hofnu a·wísan, &
þat wíf ni mahta wóp for·látan: \hld\ ne wissa hwarod siu sia węndjan skolda; &
gi·męrrid wárun iro þes muod-gi·þáhti. \hld\ Þuo gi·sah siu þena mahtigan þar &
Kriste standan, \hld\ þuoh siu ina ku̇ð-líko &
ant·kęnnjan ni mohti, \hld\ êr þan hie ina ku̇ðjan welda, &
sęggjan þat hie it selvo wári. \hld\ Hie frágoda hwat siu só sêro bi·wiepi, &
só harmo mid hêton trahnin. \hld\ Siu kwað, þat siu umbi iro hêrron ni wissi &
te wáren, hwarod hie werðan skoldi: \hld\ „ef þú ina mí gi·wísan mohtis, &
frô mín, ef ik þik frágon gi·dorsti, \hld\ ef þú ina hier an þeson felise gi·námis, &
wísi ina mí mid wordon þínon: \hld\ þan wári mí allaro willjono mêsta, &
þat ik ina selvo gi·sáhi.“ \hld\ Sia ni wissa, þat sia þie suno drohtines &
gruotta mid gódaro sprákun: \hld\ siu wánda þat it þie gardari wári, &
hof-ward hêrren sínes. \hld\ Þuo gruotta sia þie hêlago drohtin, &
bi namen nęrjendero best: \hld\ siu géng im þuo náhor sniumo, &
þat wíf mid willjon guodan, \hld\ ant·kęnda iro waldand selvan, &
míðan siu is þuru þia minnja ni wissa: \hld\ welda ina mid iro mundon grípan, &
þiu féhmja an þena folko drohtin, \hld\ novan þat iro friðu-barn godes &
węrida mid wordon sínon, \hld\ kwað þat siu ina mid wihti ni mósti &
handon ant·hrínan: \hld\ „ik ni stêg noh“, kwat-hie, „te þem himiliskon fader; &
ak íli þú nu ofst-líko \hld\ ęndi þem erlon ku̇ði, &
bruoðron mínon, \hld\ þat ik u̇ser bêðero fader &
ala-waldan, \hld\ iuwan ęndi mínan &
suoð-fastan god \hld\ suokjan willju.“ &
Þat wíf warð þuo an wunnon, \hld\ þat siu muosta su·likan willjon ku̇ðjan, &
sęggjan fan im gi·sundon: \hld\ warð sán garo &
þiu idis an þat ârundi \hld\ ęndi þem erlon bráhta, &
will-spel weron, \hld\ þat siu waldand Krist &
gi·sundan gi·sáwi, \hld\ ęndi sagda hwó he iru selvo gi·bôd &
torohtero têkno. \hld\ Sia ni weldun gi·trúojan þuo noh &
þes wíves wordon, \hld\ þat siu su·lik will-spel bráhte &
gegnungo fan þemo godes suno, \hld\ ak sia sátun im jámor-muoda, &
hęliðos hriwonda. \hld\ Þuo warð þie hêlago Krist &
eft opan-líko \hld\ ǫ́ðer-sïðu, &
drohtin gi·tôgid, \hld\ sïðor hie fan dôðe a·stuod, &
þan wívon an willjon, \hld\ þat hie im þar an wege muotta. &
kwędda sia ku̇ð-líko, \hld\ ęndi sia te is kneohon hnigun, &
fellun im tó fuoton. \hld\ Hie hét þat sia forahtan hugi &
ne bárin an iro brioston: \hld\ „ak gi mínon bruoðron skulun &
þesa kwidi ku̇ðjan, \hld\ þat sia kuman after mi &
an Galileo land; \hld\ þar ik im eft te·gęgnes biun.“ &
Þan fuorun im ôk fan Hjerusalem \hld\ þero jungrono twêna &
an þem selvon daga \hld\ sán an morgan, &
erlos an iro ârundi: \hld\ weldun im te Emaus &
þat kastel suokan. \hld\ Þuo bi·gunnun im kwidi managa &
under þem weron wahsan, \hld\ þar sia after þem wege fuorun, &
þem hęliðon umbi iro hêrron. \hld\ Þuo kwam im þar þie hêlago tuo &
gangandi godes suno. \hld\ Sia ni mahtun ina garo-l!ko &
ant·kęnnan kraftigna: \hld\ hie ni welda ina þuo noh ku̇ðjan te im; &
was im þoh an iro gi·sïðje samad \hld\ ęndi frágoda, umbi hwi-lika sia saka sprákin: &
„hwí gangat gi só gornondja?“ \hld\ kwat-hie; „Ist ink jámer hugi, &
sevo soragono full.“ \hld\ Sia sprákun im sán an·gęgin, &
þia erlos and·wurdi: \hld\ „te hwí þú þes êskos só“, kwáðun sia; &
„bist þi fan Hjerusalem \hld\ Judeono folkas &
hêlagumu gêste \hld\ fan heven-wange, &
mid þem grôtun godes kraft.“ \hld\ Nam is jungaron þó, &
erlos góde, \hld\ lêdda sie út þanan, &
antat he sie bráhte \hld\ an Bethania; &
þar hóf he is hęndi up \hld\ ęndi hêlegoda sie alle, &
wíhida sie mid is wordun. \hld\ Gi·wêt imo up þanan, &
sóhta imo þat hôha himilo ríki \hld\ ęndi þena is hêlagon stól: &
sitit imo þar \hld\ an þea swíðron half godes, &
alo-mahtiges fader \hld\ ęndi þanan all ge·sihit &
waldandjo Krist, \hld\ só hwat só þius wer-old be·havet. &
Þó an þeru selvon stędi \hld\ ge·sïðos góde &
te bedu fellun \hld\ ęndi im eft te burg þanan &
þar te Hjerusalem \hld\ jungaron Kristes &
fórun faganondi: \hld\ was im fráh-mód hugi, &
wárun im þar at þemu wíhe. \hld\ Waldandes kraft &
{[...]}\eva

\bvb IGNORE.\evb\evg
