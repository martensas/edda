\bookStart{Heliand}

Manega wáron, \hld\ þe sia iro mód ge-spón,
[...] \hld\ þat sia bi—gunnun word godes,
rękkjan þat gi—rúni, \hld\ þat þie ríkjo Krist
undar man-kunnja \hld\ máriða gi-frumida
mid wordun endi mid werkun. \hld\ Þat wolda þó wísara filo
liudo barno loƀon, \hld\ léra Kristes,
hélag word godas, \hld\ endi mid iro handon skríƀan
bereht-líko an buok, \hld\ hwó sia is gi-bod-skip skoldin
frummjan, firiho barn. \hld\ Þan wárun þoh sia fiori te þiu
under þera menigo, \hld\ þia habdon maht godes,
helpa fan himila, \hld\ hélagna gést,
kraft fan Kriste; \hld\ sia wurðun gi-korana te þio,
þat sie þan Éwangelium \hld\ énan skoldun
an buok skríƀan \hld\ endo só manag gi-bod godes,
hélag himilisk word: \hld\ sia ne muosta heliðo þan mér,
firiho barno frummjan, \hld\ neuan þat sia fiori te þio
þuru kraft godas \hld\ ge-korana wurðun,
Matheus endi Markus, \hld\ —só wárun þia man hétana—
Lukas endi Johannes; \hld\ sia wárun gode lieƀa,
wirðiga ti þem gi-wirkje. \hld\ Habda im waldand god,
þem heliðon an iro hertan \hld\ hélagna gést
fasto bi-folhan \hld\ endi ferahtan hugi,
só manag wíslík word \hld\ endi giwit mikil,
þat sea skoldin a-hębbjan \hld\ hélagaro stemnun
god-spell þat guoda, \hld\ þat ni haƀit énigan gi-gadon hwergin,
þiu word an þesaro weroldi, \hld\ þat io waldand mér,
drohtin diurie \hld\ efþo derƀi þing,
firin-werk fellie \hld\ efþo fíundo níð,
stríd wiðer-stande—, \hld\ hwand hie habda starkan hugi,
mildjan endi guodan, \hld\ þie þe méster was,
aðal-ord-frumo \hld\ alo-mahtig.
Þat skoldun sea fiori \hld\ þuo fingron skríƀan,
settian endi singan \hld\ endi sęggjan forð,
þat sea fan Kristes \hld\ krafte þem mikilon
gi-sáhun endi gi-hórdun, \hld\ þes hie selƀo gisprak,
gi-wísda endi gi-warahta, \hld\ wundar-líkas filo,
só manag mid mannon \hld\ mahtig drohtin,
all so hie it fan þem anginne \hld\ þuru is énes kraht,
waldand gisprak, \hld\ þuo hie érist þesa werold giskuop
endi þuo all bifieng \hld\ mid énu wordo,
himil endi erða \hld\ endi al þat sea bihlidan égun
giwarahtes endi giwahsanes: \hld\ þat warð þuo all mid wordon godas
fasto bifangan, \hld\ endi gifrumid after þiu,
hwilik þan liud-skępi \hld\ landes skoldi
wídost giwaldan, \hld\ efþo hwar þiu weroldaldar
endon skoldin. \hld\ Én was iro þuo noh þan
firiho barnun biforan, \hld\ endi þiu fíƀi wárun agangan:
skolda þuo þat sehsta \hld\ sáliglíko
kuman þuru kraft godes \hld\  endi Kristas giburd,
hélandero bestan, \hld\ hélagas géstes,
an þesan middil-gard \hld\ managon te helpun,
firio barnon ti frumon \hld\ wið fíundo níð,
wið dernero dwalm. \hld\ Þan habda þuo drohtin god
Rómano-liudjon farliwan \hld\ ríkjo mésta,
habda þem heri-skipje \hld\ herta gi-sterkid,
þat sia habdon bi-þwungana \hld\ þiedo gi-hwilika,
habdun fan Rúmu-burg \hld\ ríki gi-wunnan
helm-gitrósteon, \hld\ sáton iro hęri-togon
an lando gi-hwem, \hld\ habdun liudjo giwald,
allon eli-þeodon. \hld\ Erodes was
an Hierusalem \hld\ oƀer þat Judeono folk
gi-koran te kuninge, \hld\ só ina þie késer þarod,
fon Rúmu-burg \hld\ ríki þiodan
satta undar þat gi-síði. \hld\ Hie ni was þoh mid sibbeon bilang
aƀaron Israheles, \hld\ eðili-gi-burdi,
kuman fon iro knuosle, \hld\ neuan þat hie þuru þes késures þank
fan Rúmu-burg \hld\ ríki habda,
þat im wárun só gi-hóriga \hld\ hildi-skalkos,
aƀaron Israheles \hld\ ęlljan-ruoƀa:
swíðo un-wanda wini, \hld\ þan lang hie giwald éhta,
Erodes þes ríkjas \hld\ endi rád-burdeon held
Judeo liudi. \hld\ Þan was þar én gi-gamalod mann,
þat was fruod gomo, \hld\ habda ferehtan hugi,
was fan þem liudjon \hld\ Lewias kunnes,
Jakobas suneas, \hld\ guodero þiedo:
Zakharias was hie hétan. \hld\ Þat was só sálig man,
hwand hie simblon gerno \hld\ gode þeonoda,
warahta after is willjon; \hld\ deda is wíf só self
— was iru gi-aldrod idis: \hld\ ni muosta im erƀi-ward
an iro iuguð-hédi \hld\ gi-ƀiðig werðan —
libdun im farúter laster, \hld\ waruhtun lof goda,
wárun só gi-hóriga \hld\ heƀan-kuninge,
diuridon úsan drohtin: \hld\ ni weldun derƀeas wiht
under man-kunnje, \hld\ ménes gi-frummjan,
ne *saka ne sundja; \hld\ was im þoh an sorgun hugi,
þat sie erƀi-ward \hld\ égan ni móstun,
ak wárun im barno-lós. \hld\ Þan skolda he gi-bod godes
þar an Hierusalem, \hld\ só oft só is gi-gengi gi-stód,
þat ina torht-líko \hld\ tídi gi-manodun,
só skolda he at þem wíha \hld\ waldandes geld
hélag bi-hwerƀan, \hld\ heƀan-kuninges,
godes jungar-skępi: \hld\ gern was he swíðo,
þat he it þurh ferhtan hugi \hld\ frummjan mósti.

FITT 2.

Þó warð þiu tíd kuman, \hld\ — þat þar gitald habdun
wísa man mid wordun, \hld\ — þat skolda þana wíh godes
Zakharias bi-sehan. \hld\ Þó warð þar gi-samnod filu
þar te Hierusalem \hld\ Judeo liudi,
werodes te þem wíha, \hld\ þar sie waldand god
swíðo þeolíko \hld\ þiggean skoldun,
hérron is huldi, \hld\ þat sie heƀan-kuning
léðes aléti. \hld\ Þea liudi stódun
umbi þat hélaga hús, \hld\ endi geng im þe gi-hérodo man
an þana wíh innan. \hld\ Þat werod óðar béd
umbi þana alah útan, \hld\ Ebreo liudi,
hwan ér þe fródo man \hld\ gi-frumid habdi
waldandes willjon. \hld\ Só he þó þana wírók dróg,
ald aftar þem alaha, \hld\ endi umbi þana altari geng
mid is rókfatun \hld\ ríkiun þionon,
—fremida ferht-líko \hld\ fráon sínes,
godes jungar-skępi \hld\ gerno swíðo
mid hluttru hugi, *só man hérren skal
gerno ful-gangan—, \hld\ grurios kwámun im,
egison an þem alahe: \hld\ hie gisah þar aftar þiu énna ęngil godes
an þem wíhe innan, \hld\ hie sprak im mid is wordun tuo,
hiet þat fruod gumo \hld\ foroht ni wári,
hiet þat hie im ni andriede: \hld\ þína dádi sindʼ, kwaþie*,
ʽwaldanda werðe \hld\ endi þín word só self,
þín þionost is im an þanke, \hld\ þat þu sulika giþáht haƀes
an is énes kraft. \hld\ Ik is ęngil bium,
Gabriel bium ik hétan, \hld\ þe gio for goda standu,
and-ward for þem alo-waldon, \hld\ ne sí þat he me an is árundi hwarod
sęndjan willja. \hld\ Nu hiet he me an þesan síð faran,
hiet þat ik þi þoh gi-kúðdi, \hld\ þat þi kind gi-boran,
fon þínera alderu idis \hld\ ódan skoldi
werðan an þesero weroldi, \hld\ wordun spáhi.
Þat ni skal an is liƀa gio \hld\ líðes an-bítan,
wínes an is weroldi: \hld\ só haƀed im wurd-gi-skapu,
metod gi-markod \hld\ endi maht godes.
Hét þat ik þi þoh sagdi, \hld\ þat it skoldi gi-síð wesan
heƀan-kuninges, \hld\ hét þat git it heldin wel,
tuhin þurh trewa, \hld\ kwað þat he im tíras só filu
an godes ríkja \hld\ for-geƀan weldi.
He kwað þat þe gódo gumo \hld\ Johannes te namon
hębbjan skoldi, \hld\ gi-bód þat git it hétin só,
þat kind, þan it kwámi, \hld\ kwað þat it Kristes gi-síð
an þesaro wídun werold \hld\ werðan skoldi,
is selƀes sunjes, \hld\ endi kwað þat sie sliumo herod
an is bod-skępi \hld\ béðe kwámin.ʼ
Zakharias þó gi-mahalda \hld\ endi wið selƀan sprak
drohtines ęngil, \hld\ endi im þero dádjo bi-gan,
wundron þero wordo: \hld\ ʽhwó mag þat gi-werðan sóʼ, kwað he,
ʽaftar an aldre? \hld\ it is unk al te lat
só te gi-winnanne, \hld\ só þu mid þínun wordun gi-sprikis.
Hwanda wit habdun aldres \hld\ ér efno twén-tig
wintro an unkro weroldi, \hld\ ér þan kwámi þit wíf te mi;
þan wárun wit nu at-samna \hld\ ant-siƀunta wintro
gi-bęnkjon endi gi-będdjon, \hld\ siðor ik sie mi te brúdi ge-kós.
Só wit þes an unkro iuguði \hld\ gi-girnan ni mohtun,
þat wit erƀi-ward \hld\ égan móstin,
fódean an unkun flettea, \hld\ nu wit sus gi-fródod sint
—haƀad unk eldi bi-noman \hld\ ęlljan-dádi,
þat wit sint an unkro siuni gi-slekit \hld\ endi an unkun sídun lat;
flésk is unk ant-fallan, \hld\ fel un-skóni,
is unka lud gi-liðen, \hld\ lík gi-drusnod,
sind unka and-bári \hld\ óðar-líkaron,
mód endi megin-kraft—, \hld\ só wit giu só managan dag
wárun an þesero weroldi, \hld\ só mi þes wundar þunkit,
hwó it só gi-werðan mugi, \hld\ só þu mid þínun wordun gi-sprikis.

FITT 3.

Þó warð þat heƀen-kuninges bodon \hld\ harm an is móde,
þat he is gi-werkes \hld\ só wundron skolda
endi þat ni welda gi-huggjan, \hld\ þat ina mahta hélag god
só alaiungan, \hld\ só he fon érist was,
selƀo gi-wirkjan, \hld\ of he só weldi.
Skerida im þó te wítea, \hld\ þat he ni mahte énig word sprekan,
gimahlien mid is múðu, \hld\ ʽér þan þi magu wirðid,
fon þínero aldero idis \hld\ erl afódit,
kindiung giboran \hld\ kunnjes gódes,
wánum te þesero weroldi. \hld\ Þan skalt þu eft word sprekan,
hębbjan þínaro stemna giwald; \hld\ ni þarft þu stum wesan
lengron hwíla.ʼ \hld\ Þó warð it sán giléstid só,
giworðan te wáron, \hld\ só þar an þem wíha gisprak
ęngil þes alo-waldon: \hld\ warð ald gumo
spráka bilósit, \hld\ þoh he spáhan hugi
bári an is breostun. \hld\ Bidun allan dag
þat werod for þem wíha \hld\ endi wundrodun alla,
bi-hwí he þar só lango, \hld\ lof-sálig man,
swíðo fród gumo \hld\ fráon sínun
þionon þorfti, \hld\ só þar ér énig þegno ni deda,
þan sie þar at þem wíha \hld\ waldandes geld
folmon frumidun. \hld\ Þó kwam fród gumo
út fon þem alaha. \hld\ Erlos þrungun
náhor mikilu: \hld\ was im niud mikil,
hwat he im sóðlíkes \hld\ sęggjan weldi,
wísean te wáron. \hld\ He ni mohta þó énig word sprekan,
gisęggjan þem gisiðea, \hld\ bútan þat he mid is swíðron hand
wísda þem weroda, \hld\ þat sie úses waldandes
léra léstin. \hld\ Þea liudi forstódun,
þat he þar habda gegnungo \hld\ god-kundes hwat
forsehen selƀo, \hld\ þoh he is ni mahti gisęggjan wiht,
giwísean te wáron. \hld\ Þó habda he úses waldandes
geld giléstid, \hld\ al só is gigengi was
gimarkod mid mannun. \hld\ Þó warð sán aftar þiu maht godes,
gikúðid is kraft mikil: \hld\ warð þiu kwán ókan,
idis an ira eldiu: \hld\ skolda im erƀi-ward,
swíðo god-kund gumo \hld\ gi-ƀiðig werðan,
barn an burgun. \hl\ Béd aftar þiu
þat wíf wurdi-gi-skapu. \hld\ Skréd þe wintar forð,
geng þes géres gital. \hld\ Johannes kwam
an liudjo lioht: \hld\ lík was im skóni,
was im fel fagar, \hld\ fahs endi naglos,
wangun wárun im wlitige. \hld\ Þó fórun þar wíse man,
snelle tesamne, \hld\ þea swásostun mést,
wundrodun þes werkes, \hld\ bi-hwí it gio mahti giwerðan só,
þat undar só aldun twém \hld\ ódan wurði
barn an giburdeon, \hld\ ni wári þat it gi-bod godes
selƀes wári: \hld\ afsuoƀun sie garo,
þat it elkor só wánlík \hld\ werðan ni mahti.
Þó sprak þar én gifródot man, \hld\ þe só filo konsta
wísaro wordo, \hld\ habde giwit mikil,
frágode niudlíko, \hld\ hwat is namo skoldi
wesan an þesaro weroldi: \hld\ ʽmi þunkid an is wísu gilík
iak an is gibárea, \hld\ þat he sí betara þan wi,
só ik wániu, \hld\ þat ina ús gegnungo god fon himila
selƀo sendiʼ. \hld\ Þó sprak sán aftar
þiu módar þes kindes, \hld\ þiu þana magu habda,
þat barn an ire barme: \hld\ ʽhér kwam gi-bod godesʼ, kwað siu,
ʽfernun gére, \hld\ furmon wordu
gibód, þat he Johannes \hld\ bi godes lérun
hétan skoldi. \hld\ Þat ik an mínumu hugi ni gidar
wendean mid wihti, \hld\ of ik is giwaldan mótʼ.
Þó sprak én gélhert man, \hld\ þe ira gaduling was:
ʽne hét ér giowiht sóʼ, \hld\ kwað he, ʽaðal-boranes
úses kunnjes efþo knósles; \hld\ wita kiasan im óðrana
niudsamna namon: \hld\ he niate of he mótiʼ.
Þó sprak eft þe fródo man, \hld\ þe þar konsta filo mahlian:
ʽni giƀu ik þat te rádeʼ, \hld\ kwað he, ʽrinko negénun,
þat he word godes \hld\ wendean biginna;
ak wita is þana fader frágon, \hld\ þe þar só gifródod sitit,
wís an is wínseli: \hld\ þoh he ni mugi énig word sprekan,
þoh mag he bi bókstaƀon \hld\ bréf gewirkjan,
namon giskríƀanʼ. \hld\ Þó he náhor geng,
legda im éna bók an barm \hld\ endi bad gerno
wrítan wíslíko \hld\ wordgimerkiun,
hwat sie þat hélaga barn \hld\ hétan skoldin.
Þó nam he þia bók an hand \hld\ endi an is hugi þáhte
swíðo gerno te gode: \hld\ Johannes namon
wís-líko gi-wrét \hld\ endi ók aftar mid is wordu gisprak
swíðo spáh-líko: \hld\ habda im eft is spráka giwald,
giwitteas endi wísun. \hld\ Þat wíti was þó agangan,
hard harm-skare, \hld\ þe im hélag god
mahtig makode, \hld\ þat he an is mód-seƀon
godes ni forgáti, \hld\ þan he im eft sendi is jungron tó.

FITT 4.

Þó ni was lang aftar þiu, \hld\ ne it al só giléstid warð,
só he man-kunnja \hld\ managa hwíla,
god alo-mahtig \hld\ for-geƀen habda,
þat he is himilisk barn \hld\ herod te weroldi,
si selƀes sunu \hld\ sęndjan weldi,
te þiu þat he hér a-lósdi \hld\ al liud-stamna,
werod fon wítea. \hld\ Þó warð is wisbodo
an Galilea-land, \hld\ Gabriel kuman,
ęngil þes alo-waldon, \hld\ þar he éne idis wisse,
munilíka magað: \hld\ Maria was siu héten,
was iru þiorna giþigan. \hld\ Sea én þegan habda,
Joseph gimahlit, \hld\ gódes kunnjes man,
þea Dawides dohter: \hld\ þat was só diurlík wíf,
idis ant-héti. \hld\ Þar sie þe ęngil godes
an Nazareth-burg \hld\ bi namon selƀo
grótte geginwarde \hld\ endi sie fon gode qhwedda:
ʽHél wis þu, Mariaʼ, \hld\ kwað he, ʽþu bist þínun hérron liof,
waldande wirðig, \hld\ hwand þu giwit haƀes,
idis enstio fol. \hld\ Þu skalt for allun wesan
wíƀun giwíhit. \hld\ Ne haƀe þu wékan hugi,
ne forhti þu þínun ferhe: \hld\ ne kwam ik þi te énigun fréson herod,
ne dragu ik énig drugiþing. \hld\ Þu skalt úses drohtines wesan
módar mid mannun \hld\ endi skalt þana magu fódean,
þes hóhon heƀan-kuninges suno. \hld\ Þe skal héljand te namon
égan mid eldiun. \hld\ Neo endi ni kumid,
þes wídon ríkjas giwand, \hld\ þe he giwaldan skal,
mári þeodan.ʼ \hld\ Þó sprak im eft þiu magað angegin,
wið þana ęngil godes \hld\ idiso skóniost,
allaro wíƀo wlitigost: \hld\ ʽhwó mag þat giwerðen sóʼ, kwað siu,
ʽþat ik magu fódie? \hld\ Ne ik gio mannes ni warð
wís an mínera weroldi.ʼ \hld\ Þó habde eft is word garu
ęngil þes alo-waldon \hld\ þero idisiu tegegnes:
ʽan þi skal hélag gést \hld\ fon heƀan-wange
kuman þurh kraft godes. \hld\ Þanan skal þi kind ódan
werðan an þesaro weroldi; \hld\ waldandes kraft
skal þi fon þem hóhoston \hld\ heƀan-kuninge
skadowan mid skimon. \hld\ Ni warð skóniera giburd,
ne só mári mid mannun, \hld\ hwand siu kumid þurh maht godes
an þese wídon werold.ʼ \hld\ Þó warð eft þes wíbes hugi
aftar þem árundie \hld\ al gihworƀen
an godes willjon. \hld\ ʽÞan ik hér garu standuʼ, kwað siu,
ʽte sulikun ambaht-skępi, \hld\ só he mi égan wili.
Þiu bium ik þeot-godes. \hld\ Nu ik þeses þinges gitrúon;
werðe mi aftar þínun wordun, \hld\ al só is willjo sí,
hérron mínes; \hld\ nis mi hugi twífli,
ne word ne wísa.ʼ \hld\ Só gi-fragn ik, þat þat wíf ant-feng
þat godes árundi \hld\ gerno swíðo
mid leohtu hugi \hld\ endi mid gilóƀon gódun
endi mid hluttrun trewun; \hld\ warð þe hélago gést,
þat barn an ira bósma; \hld\ endi siu ira breostun forstód
iak an ire seƀon selƀo, \hld\ sagda þem siu welda,
þat sie habde giókana \hld\ þes alo-waldon kraft
hélag fon himile. \hld\ Þó warð hugi Josepes,
is mód giworrid, \hld\ þe im ér þea magað habda,
þea idis ant-héttea, \hld\ aðal-knósles wíf
giboht im te brúdiu. \hld\ He afsóf þat siu habda barn undar iru:
ni wánda þes mid wihti, \hld\ þat iru þat wíf habdi
giwardod só warolíko: \hld\ ni wisse waldandes þó noh
blíði gi-bod-skępi. \hld\ Ni welda sia imo te brúdi þó,
halon imo te híwon, \hld\ ak bi-gan im þó an hugi þęnkjan,
hwó he sie só for-léti, \hld\ só iru þar nu wurði lédes wiht,
ódan arƀides. \hld\ Ni welda sie aftar þiu
meldon for menigi: \hld\ antdréd þat sie manno barn
líƀu binámin. \hld\ Só was þan þero liudjo þau
þurh þen aldon éu, \hld\ Ebreo folkes,
só hwilik só þar an unreht \hld\ idis gihíwida,
þat siu simbla þana bedskepi \hld\ buggean skolda,
frí mid ira ferhu: \hld\ ni was gio þiu fémea só gód,
þat siu mid þem liudun leng \hld\ libbien mósti,
wesan undar þem weroda. \hld\ Bigan im þe wíso mann,
swíðo gód gumo, \hld\ Joseph an is móda
þęnkjan þero þingo, \hld\ hwó he þea þiornun þó
listiun for-léti. \hld\ Þó ni was lang te þiu,
þat im þar an dróma \hld\ kwam drohtines ęngil,
heƀan-kuninges bodo, \hld\ endi hét sie ina haldan wel,
minnjon sie an is móde: \hld\ ʽNi wis þuʼ, kwað he, ʽMariun wréð,
þiornun þínaro; \hld\ siu is giþungan wíf;
ne forhugi þu sie te hardo; \hld\ þu skalt sie haldan wel,
wardon ira an þesaro weroldi. \hld\ Lésti þu inka wini-trewa
forð só þu dádi, \hld\ endi hald inkan friund-skępi wel!
Ne lát þu sie þi þiu léðaron, \hld\ þoh siu undar ira liðon égi,
barn an ira bósma. \hld\ It kumid þurh gi-bod godes,
hélages géstes \hld\ fon heƀan-wanga:
þat is Iésu Krist, \hld\ godes égan barn,
waldandes sunu. \hld\ Þu skalt sie wel haldan,
hélag-líko. \hld\ Ne lát þu þi þínan hugi twíflien,
merrean þína mód-gi-þáht.ʼ \hld\ Þó warð eft þes mannes hugi
giwendid aftar þem wordun, \hld\ þat he im te þem wíƀa genam,
te þera magað minnja: \hld\ ant-kenda maht godes,
waldandes gi-bod; \hld\ was im willjo mikil,
þat he sia só hélaglíko \hld\ haldan mósti:
bi-sorgoda sie an is gi-síðea, \hld\ endi siu só súƀro dróg
al te huldi godes \hld\ hélagna gést,
gód-líkan gumon, \hld\ antþat sie godes gi-skapu
mahtig gimanodun, \hld\ þat siu ina an manno lioht,
allaro barno bezt, \hld\ brengean skolda.

FITT 5.

Þó warð fon Rúmu-burg \hld\ ríkes mannes
oƀar alla þesa irmin-þiod \hld\ Oktauiánas
ban endi bod-skępi \hld\ oƀar þea is brédon giwald
kuman fon þem késure \hld\ kuningo gihwilikun,
hém-sittjandiun, \hld\ só wído só is hęri-togon
oƀar al þat land-skępi \hld\ liudjo gi-weldun.
Hiet man þat alla þea eli-lendiun man \hld\ iro óðil sóhtin,
heliðos iro hand-mahal \hld\ angegen iro hérron bodon,
kwámi te þem knósla gihwe, \hld\ þanan he kunnjas was,
gi-boran fon þem burgiun. \hld\ Þat gi-bod warð giléstid
oƀar þesa wídon werold; \hld\ werod samnoda
te allaro burgeo gihwem. \hld\ Fórun þea bodon oƀar all,
þea fon þem késura \hld\ kumana wá*run,
bók-spáha weros, \hld\ endi an bréf skriƀun
swíðo niud-líko \hld\ namono gi-hwilikan,
ia land ia liudi, \hld\ þat im ni mahti a-lettjan mann
gumono sulika gambra, \hld\ só im skolda geldan gihwe
heliðo fon is hóƀda. \hld\ Þó gi-wét im ók mid is híwiska
Joseph þe gódo, \hld\ só it god mahtig,
waldand welda: \hld\ sóhta im þiu wánamon hém,
þea burg an Bethleem, \hld\ þar iro beiðero was,
þes heliðes hand-mahal* \hld\ endi ók þera hélagun þiornun,
Mariun þera gódun. \hld\ Þar was þes márjon stól
an ér-dagun, \hld\ aðalkuninges,
Dawides þes gódon, \hld\ þan langa þe he þana druht-skępi þar,
erl undar Ebreon \hld\ égan mósta,
haldan hóhgisetu. \hld\ Sie wárun is híwiskas,
kuman fon is knósla, \hld\ kunnjas gódes,
béðiu bi giburdiun. \hld\ Þar gi-fragn ik, þat sie þiu berhtun gi-skapu,
Mariun gimanodun \hld\ *endi maht godes,
þat iru an þem síða \hld\ sunu ódan warð,
gi-boran an Bethleem \hld\ barno strangost,
allaro kuningo kraftigost: \hld\ kuman warð þe márjo,
mahtig an manno lioht, \hld\ só is ér managan dag
biliði wárun \hld\ endi bókno filu
gi-worðen an þesero weroldi. \hld\ Þó was it all giwárod só,
só it ér spáha man \hld\ gi-sprokan habdun,
þurh hwilik ód-módi \hld\ he þit erð-ríki herod
þurh is selƀes kraft \hld\ sókjan welda,
managaro mund-boro. \hld\ Þó ina þiu módar nam,
bi-wand ina mid wádju \hld\ wíƀo skóniost,
fagaron fratahun, \hld\ endi ina mid iro folmon twém
legda lioƀ-líko \hld\ luttilna man,
þat kind an éna kribbiun, \hld\ þoh he habdi kraft godes,
manno drohtin. \hld\ Þar sat þiu módar bi-foran,
wíf wakogeandi, \hld\ war*doda selƀo,
held þat hélaga barn: \hld\ ni was ira hugi twífli,
þera magað ira mód-seƀo. \hld\ Þó warð þat managun kúð
oƀar þesa wídon werold, \hld\ wardos ant-fundun,
þea þar ehu-skalkos \hld\ úta wárun,
weros an wahtu, \hld\ wiggeo gómean,
fehas aftar fel*da: \hld\ gi-sáhun finistri an twé
telátan an lufte, \hld\ endi kwam lioht godes
wánum þurh þiu wolkan \hld\ endi þea wardos þar
bifeng an þem felda. \hld\ Sie wurðun an forhtun þó,
þea man an ira móda: \hld\ gisáhun þar mahtigna
godes ęngil kuman, \hld\ þe im tegegnes sprak,
hét þat im þea wardos \hld\ wiht ne antdrédin
léðes fon þem liohta: \hld\ ʽik skal euʼ, kwað he, ʽlioƀara þing,
swíðo wár-líko \hld\ willjon sęggjan,
kúðean kraft mikil: \hld\ nu is Krist geboran
an þeser*o selƀun naht, \hld\ sálig barn godes,
an þera Dawides burg, \hld\ drohtin þe gódo.
Þat is mendislo \hld\ manno kunnjas,
allaro firiho fruma. \hld\ Þar gi ina fíðan mugun,
an Bethlemaburg \hld\ barno ríkjost:
hębbjad þat te tékna, \hld\ þat ik eu gi-tęlljan mag
wárun wordun, \hld\ þat he þar biwundan ligid,
þat kind an énera kribbiun, \hld\ þoh he sí kuning oƀar al
erðun endi himiles \hld\ endi oƀar eldeo barn,
weroldes waldandʼ. \hld\ Reht só he þó þat word gisprak,
só warð þar ęngilo te þem énun \hld\ unrím kuman,
hélag hęri-skępi \hld\ fon heƀan-wanga,
fagar folk godes, \hld\ endi filu sprákun,
lof-word manag \hld\ liudjo hérron.
Af-hóƀun þó hélagna sang, \hld\ þó sie eft te heƀan-wanga
wundun þurh þiu wolkan. \hld\ Þea wardos hórdun,
hwó þiu ęngilo kraft \hld\ alo-mahtigna god
swíðo werð-líko \hld\ wordun loƀodun:
ʽdiuriða sí nuʼ, \hld\ kwáðun sie, ʽdrohtine selƀun
an þem hóhoston \hld\ himilo ríkja
ęndi friðu an erðu \hld\ firiho barnun,
gód-willigun gumun, \hld\ þem þe god ant-kęnnjad
þurh hluttran hugi.ʼ \hld\ Þea hirdjo for-stódun,
þat sie mahtig þing \hld\ gi-manod habda,
blíð-lík bod-skępi: \hld\ gi-witun im te Bethleem þanan
nahtes síðon; \hld\ was im niud mikil,
þat sie selƀon Krist \hld\ gi-sehan móstin.

FITT 6.

Habda im þe ęngil godes \hld\ al gi-wísid
torhtun téknun, \hld\ þat sie im tó selƀun,
te þem godes barne \hld\ gangan mahtun,
endi fundun sán \hld\ folko drohtin,
liudjo hérron. \hld\ Sagdun þó lof goda,
waldande mid iro wordun \hld\ endi wído kúðdun
oƀar þea berhtun burg, \hld\ hwilik im þar biliði warð
fon heƀan-wanga \hld\ hélag gi-tógit,
fagar an felde. \hld\ Þat frí al biheld
an ira hugi-skęftjun, \hld\ hélag þiorna,
þiu magað an ira móde, \hld\ só hwat só siu gi-hórda þea mann sprekan.
Fódda ina þó fagaro \hld\ frího skánjosta,
þiu módar þurh minnja \hld\ managaro drohtin,
hélag himilisk barn. \hld\ Heliðos gisprákun
an þem ahtodon daga \hld\ erlos managa,
swíðo glawa gumon \hld\ mid þera godes þiornun,
þat he héljand te namon \hld\ hębbjan skoldi,
só it þe godes ęngil \hld\ Gabriel gisprak
wáron wordun \hld\ endi þem wíƀe gibód,
bodo drohtines, \hld\ þó siu érist þat barn ant-feng
wánum te þesero weroldi; \hld\ was iru willjo mikil,
þat siu ina só hélaglíko \hld\ haldan mósti,
fulgeng im þó só gerno. \hld\ Þat gér furðor skréd
untþat þat friðu-barn godes \hld\ fiartig habda
dago endi nahto. \hld\ Þó skoldun sie þar éna dád frummjan,
þat sie ina te Hierusalem \hld\ forgeƀan skoldun
waldanda te þem wíha. \hld\ Só was iro wísa þan,
þero liudjo land-sidu, \hld\ þat þat ni mósta for-látan negén
idis undar Ebreon, \hld\ ef iru at érist warð
sunu afódit, \hld\ ne siu ina simbla þarod
te þem godes wíha \hld\ forgeƀan skolda.
Giwitun im þó þiu gódun twé, \hld\ Joseph endi Maria
béðiu fon Bethleem: \hld\ habdun þat barn mid im,
hélagna Krist, \hld\ sóhtun im hús godes
an Hierusalem; \hld\ þar skoldun sie is geld frummjan
waldanda at þem wíha \hld\ wísa léstean
Judeo folkes. \hld\ Þar fundun sea énna gódan man
aldan at þem alaha, \hld\ aðal-boranan,
þe habda at þem wíha só filu \hld\ wintro endi sumaro
gilibd an þem liohta: \hld\ oft warhta he þar lof goda
mid hluttru hugi; \hld\ habda im hélagna gést,
sáliglíkan seƀon; \hld\ Simeon was he hétan.
Im habda giwísid \hld\ waldandas kraft
langa hwíla, \hld\ þat he ni mósta ér þit lioht ageƀan,
wendean af þesero weroldi, \hld\ ér þan im þe willjo gistódi,
þat he selƀan Krist \hld\ gisehan mósti,
hélagna heƀan-kuning. \hld\ Þó warð im is hugi swíðo
blíði an is briostun, \hld\ þó he gisah þat barn kuman
an þena wíh innan. \hld\ Þuo sagda hie waldande þank,
al-mahtigon gode, \hld\ þes he ina mid is ógun gisah.
Geng im þó tegegnes \hld\ endi ina gerno ant-feng
ald mid is armun: \hld\ al ant-kende
bókan endi biliði \hld\ endi ók þat barn godes,
hélagna heƀan-kuning. \hld\ ʽNu ik þi, hérro, skalʼ, kwað he,
ʽgerno biddjan, \hld\ nu ik sus gigamalod bium,
þat þu þínan holdan skalk \hld\ nu hinan hwerƀan látas,
an þína friðu-wára faran, \hld\ þar ér mína forðrun dedun,
weros fon þesero weroldi, \hld\ nu mi þe willjo gistód,
dago lioƀosto, \hld\ þat ik mínan drohtin gisah,
holdan hérron, \hld\ só mi gihétan was
langa hwíla. \hld\ Þu bist lioht mikil
allun eli-þiodun, \hld\ þea ér þes alo-waldon
kraft ne ant-kendun. \hld\ Þína kumi sindun
te dóma endi te diurðon, \hld\ drohtin fró mín,
aƀarun Israhelas, \hld\ éganumu folke,
þínun lioƀun *liudjun.ʼ \hld\ Listiun talde þó
þe aldo man an þem alaha \hld\ idis þero gódun,
sagda sóð-líko, \hld\ hwó iro sunu skolda
oƀar þesan middil-gard \hld\ managun werðan
sumun te falle, sumun te fróƀru \hld\ firiho barnun,
þem liudjun te leoƀa, \hld\ þe is lérun gi-hórdin,
endi þem te harma, \hld\ þe hórjen ni weldin
Kristas léron. \hld\ ʽÞu skalt nohʼ, kwað he, ʽkara þiggean,
harm an þínumu herton, \hld\ þan ina heliðo barn
wápnun wítnod. \hld\ Þat wirðid þi werk mikil,
þrim te gi-þolonna.ʼ \hld\ Þiu þiorna al forstód
wísas mannas word. \hld\ Þó kwam þar ók én wíf gangan
ald innan þem alaha: \hld\ Anna was siu hétan,
dohtar Fanueles; \hld\ siu habde ira drohtine wel
giþionod te þanka, \hld\ was iru giþungan wíf.
Siu mósta aftar ira magað-hédi, \hld\ síðor siu mannes warð,
erles an éhti \hld\ eðili þiorne,
só mósta siu mid ira brúdi-gumon \hld\ bódlo giwaldan
siƀun wintar saman. \hld\ Þó gi-fragn ik þat iru þar sorga gistód
þat sie þiu mikila maht \hld\ metodes tedélda,
wréð wurdi-gi-skapu. \hld\ Þó was siu widowa aftar þiu
at þem friðu-wíha \hld\ fior endi antahtoda
wintro an iro weroldi, \hld\ só siu nia þana wíh ni for-lét,
ak siu þar ira drohtine wel \hld\ dages endi nahtes,
gode þionode. \hld\ Siu kwam þar ók gangan tó
an þea selƀun tíd: \hld\ sán ant-kende
þat hélage barn godes \hld\ endi þem heliðon kúðde,
þem weroda aftar þem wíha \hld\ wil-spel mikil,
kwað þat im neriandas ginist \hld\ gináhid wári,
helpa heƀen-kuninges: \hld\ ʽnu is þe hélago Krist,
waldand selƀo \hld\ an þesan wíh kuman
te a-lósjenne þea liudi, \hld\ þe hér nu lango bidun
an þesara middil-gard, \hld\ managa hwíla,
þurftig þioda, \hld\ só nu þes þinges mugun
mendian man-kunni.ʼ \hld\ Manag fagonoda
werod aftar þem wíha: \hld\ gi-hórdun wilspel mikil
fon gode sęggjan. \hld\ Þat geld habde þó giléstid
þiu idis an þem alaha, \hld\ al só it im an ira éwa gibód
endi an þera berhtun burg \hld\ bók giwísdun,
hélagaro hand-gi-werk. \hld\ Giwitun im þó te hús þanan
fon Hierusalem \hld\ Joseph endi Maria,
hélag híwiski: \hld\ habdun im heƀen-kuning
simbla te gi-síða, sunu drohtines,
managaro mund-boron, \hld\ só it gio mári ni warð
þan wídor an þesaro weroldi, \hld\ bútan só is willjo geng,
heƀen-kuninges hugi. \hld\ Þoh þar þan gihwilik hélag man
Krist ant-kendi, \hld\ þoh ni warð it gio te þes kuninges hoƀe
þem mannun gimárid, \hld\ þea im an iro mód-seƀon
holde ni wárun, \hld\ ak was im só bihalden forð
mid wordun endi mid werkun, \hld\ antþat þar weros óstan,
swíðo glawa gumon \hld\ gangan kwámun
þrea te þero þiodu, \hld\ þegnos snelle,
an langan weg \hld\ oƀar þat land þarod:
folgodun énun berhtun bókne \hld\ endi sóhtun þat barn godes
mid hluttru hugi: \hld\ weldun im hnígan tó,
gehan im te jungrun: \hld\ driƀun im godes gi-skapu.
Þó sie Erodesan þar \hld\ ríkjan fundun
an is seli sittjen, \hld\ slíð-wurdean kuning,
módagna mid is mannun: \hld\ —simbla was he morðes gern—
þó kwaddun sie ina kúsko \hld\ an kuning-wísun,
fagaro an is flęttje, \hld\ endi he frágoda sán,
hwilik sie árundi \hld\ úta gibráhti,
weros an þana wrak-síð: \hld\ ʽhweðer lédiad gi wundan gold
te geƀu hwilikun gumuno? \hld\ te hwí gi þus an ganga kumad,
gi-faran an fóðiu? \hld\ Huat, gi néþwanan ferran sind
erlos fon óðrun þiodun. \hld\ Ik gisihu þat gi sind eðili-gi-burdiun
kunnjes fon knósle gódun: \hld\ nio hér ér sulika kumana ni wurðun
éri fon óðrun þiodun, \hld\ síðor ik mósta þesas erlo folkes,
giwaldan þesas wídon ríkjas. \hld\ Gi skulun mi te wárun sęggjan
for þesun liudjo folke, \hld\ bi-hwí gi sín te þesun lande kumanaʼ.
Þó sprákun im eft tegegnes \hld\ gumon óstr-onja,
wordspáhe weros: \hld\ ʽwi þi te wárun mugunʼ, kwáðun sie,
ʽúse árundi \hld\ óðo gitęlljen,
gisęggjan sóðlíko, \hld\ bi-hwí wi kwámun an þesan sið herod
fon óstan te þesaro erðu. \hld\ Giu wárun þar aðalies man,
gódsprákja gumon, \hld\ þea ús gódes só filu,
helpa gihétun \hld\ fon heƀen-kuninge
wárum wordun. \hld\ Þan was þar én giwittig man,
fród endi fil-wís \hld\ —forn was þat giu—,
úse aldiro óstar hinan, —þar ni warð síðor énig man
sprákono só spáhi—; \hld\ he mahte rekkien spel godes,
hwand im habde for-liwan \hld\ liudjo hérro,
þat he mahte fon erðu \hld\ up gi-hórjan
waldandes word: \hld\ biþiu was is giwit mikil,
þes þegnes giþáhti. \hld\ Þó he þanan skolda,
ageƀen gardos, \hld\ gadulingo gimang,
for-láten liudjo dróm, \hld\ sókien lioht óðar,
þó he is jungron \hld\ hét gangan náhor,
erƀi-wardos, \hld\ endi is erlun þó
sagde sóðlíko: \hld\ —þat al síðor kwam,
giwarð* an þesaro weroldi—: \hld\ þó sagda he þat hér skoldi kuman én wís-kuning
mári endi mahtig \hld\ an þesan middil-gard
þes bezton giburdies; \hld\ kwað þat it skoldi wesan barn godes,
kwað þat he þesero weroldes \hld\ waldan skoldi
gio te éwan-daga, \hld\ erðun endi himiles.
He kwað þat an þem selƀon daga, \hld\ þe ina sáligna
an þesan middil-gard \hld\ módar gidrógi,
só kwað he þat óstana \hld\ én skoldi skínan
himil-tungal hwít, \hld\ sulik só wi hér ne habdin ér
undartuisk erða endi himil \hld\ óðar hwerigin,
ne sulik barn ne sulik bókan. \hld\ Hét þat þar te bedu fórin
þrea man fon þero þiodu, \hld\ hét sie þęnkjan wel,
hwan ér sie gi-sáwin óstana \hld\ up síðogean,
þat godes bókan gangan, \hld\ hét sie garwian sán,
hét þat wi im folgodin, \hld\ só it furi wurði,
westar oƀar þesa weroldi. \hld\ Nu is it al giwárod só,
kuman þurh kraft godes: \hld\ þe kuning is gifódit,
gi-boran bald endi strang: \hld\ wi gisáhun is bókan skínan
hédro fon himiles tunglun, \hld\ só ik wét, þat it hélag drohtin,
markoda mahtig selƀo; \hld\ wi gisáhun morgno gihwilikes
blíkan þana berhton sterron, \hld\ endi wi gengun aftar þem bókna herod
wegas endi waldas hwílon. \hld\ Þat wári ús allaro willjono mésta,
þat wi ina selƀon gisehan móstin, \hld\ wissin, hwar wi ina sókjan skoldin,
þana kuning an þesumu késurdóma. \hld\ Saga ús, undar hwilikumu he sí þesaro kunneo afódit.ʼ
Þó warð Erodesa \hld\ innan briostun
harm wið herta, \hld\ bigan im is hugi wallan,
seƀo mid sorgun: \hld\ gi-hórde sęggjan þó,
þat he þar oƀar-hóƀdon \hld\ égan skoldi,
kraftagoron kuning \hld\ kunnjes gódes,
sáligoron undar þem gisíðja. \hld\ Þó he samnon hét,
só hwat só an Hierusalem \hld\ gódaro manno
allaro spáhoston \hld\ sprákono wárun
endi an iro brioston \hld\ bók-kraftes mést
wissun te wárun, \hld\ endi he sie mid wordun fragn,
swíðo niud-líko \hld\ níð-hugdig man,
kuning þero liudjo, \hld\ hwar Krist gi-boran
an werold-ríkja \hld\ werðan skoldi,
friðu-gumono bezt. \hld\ Þó sprak im eft þat folk angegin,
þat werod wár-líko, \hld\ kwáðun þat sie wissin garo,
þat he skoldi an Bethleem gi-boran werðan: \hld\ ʽsó is an úsun bókun giskriƀan,
wís-líko giwritan, \hld\ só it wár-sagon,
swíðo glawa gumon \hld\ bi godes krafta
fil-wíse man \hld\ furn gisprákun,
þat skoldi fon Bethleem \hld\ burgo hirdi,
liof landes ward \hld\ an þit lioht kuman,
ríki rád-geƀo, \hld\ þe rihtien skal
Judeono gum-skępi \hld\ endi is geƀa wesan
mildi oƀar middil-gard \hld\ managun þiodun.ʼ
Þó gi-fragn ik þat sán aftar þiu \hld\ slíð-mód kuning
þero wár-sagono word \hld\ þem wrekkiun sagda,
þea þar an eli-lendi \hld\ erlos wárun
ferran gifarana, \hld\ endi he frágoda aftar þiu,
hwan sie an óstar-wegun \hld\ érist gisáhin
þana kuningsterron kuman, \hld\ kumbal luihtien
hédro fon himile. \hld\ Sie ni weldun is im þó helen eowiht,
ak sagdun it im sóð-líko. \hld\ Þó hét he sie an þana síð faran,
hét þat sie ira árundi al \hld\ undar-fundin
umbi þes kindes kumi, \hld\ endi þe kuning selƀo gibód
swíðo hard-liko, \hld\ hérro Judeono,
þem wísun mannun, \hld\ ér þan sie fórin westan forð,
þat sie im eft gikúðdin, \hld\ hwar he þana kuning skoldi
sókjan at is selðon; \hld\ kwað þat he þar weldi mid is gisíðun tó,
bedan te þem barne. \hld\ Þan hogda he im te banon werðan
wápnes ęggjun. \hld\ Þan eft waldand god
þáhte wið þem þinga: \hld\ he mahta aþengean mér,
giléstean an þesum liohte: \hld\ þat is noh lango skín,
gikúðid kraft godes. \hld\ Þó gengun eft þiu kumbl forð
wánum undar wolknun. \hld\ Þó wárun þea wíson man
fúsa te faranne: \hld\ giwitun im forð þanan
balda an bod-skępi: \hld\ weldun þat barn godes
selƀon sókjan. \hld\ Sie ni habdun þanan gisíðjas mér,
bútan þat sie þrie wárun: \hld\ wissun im þingo giskéð,
wárun im glawe gumon, \hld\ þe þea geƀa léddun.
Þan sáhun sie só wís-líko \hld\ undar þana wolknes skion,
up te þem hóhon himile, \hld\ hwó fórun þea hwíton sterron
—ant-kendun sie þat kumbal godes—, \hld\ þiu wárun þurh Krista herod
giwarht te þesero weroldi. \hld\ Þea weros aftar gengun,
folgodun feraht-líko \hld\ —sie frumide þe mahte—
antþat sie gisáhun, \hld\ síð-wórige man,
berht bókan godes, \hld\ blék an himile
stillo gistanden. \hld\ Þe sterro liohto skén
hwít oƀar þem húse, \hld\ þar þat hélage barn
wonode an willjon \hld\ endi ina þat wíf biheld,
þiu þiorne giþiudo. \hld\ Þó warð þero þegno hugi
blíði an iro briostun: \hld\ bi þem bókna for-stódun,
þat sie þat friðu-barn godes \hld\ funden habdun,
hélagna heƀen-kuning. \hld\ Þó sie an þat hús innan
mid iro geƀun gengun, \hld\ gumon óstr-onja,
síð-wórige man: \hld\ sán ant-kendun
þea weros waldand Krist. \hld\ Þea wrekkion fellun
te þem kinde an kneobeda \hld\ endi ina an kuning-wísa
gódan gróttun \hld\ endi im þea geƀa drógun,
gold endi wíh-rók \hld\ bi godes téknun
*endi myrra þar mid. \hld\ Þea man stódun garowa,
holde for iro hérron, \hld\ þea it mid iro handun sán
fagaro ant-fengun. \hld\ Þó gi-witun im þea ferahton man,
seggi te selðon \hld\ síð-wórige,
gumon an gast-sęli. \hld\ Þar im godes ęngil
slápandiun an naht \hld\ sweƀan gitógde,
gi-drog im an dróme, \hld\ al so it drohtin self,
waldand welde, \hld\ þat im þúhte þat man im mid wordun gibudi,
þat sie im* þanan óðran weg, \hld\ erlos fórin,
liðodin sie te lande \hld\ endi þana léðan man,
Erodesan \hld\ eft ni sóhtin,
módagna kuning. \hld\ Þó warð morgan kuman
wánum te þesero weroldi. \hld\ Þó bi-gunnun þea wíson man
sęggjan iro sweƀanos; \hld\ selƀon ant-kendun
waldandes word, \hld\ hwand sie giwit mikil
bárun an iro briostun: \hld\ bádun alo-waldon,
héron heƀen-kuning, \hld\ þat sie móstin is huldi forð,
gi-wirkjan is willjon, \hld\ kwáðun þat sea ti im habdin gi-wendit hugi,
*iro mód morgan gihwem. \hld\ Þó fórun eft þie man þanan,
erlos óstr-onje, \hld\ al só im þe ęngil godes
wordun gi-wísde: \hld\ námun im weg óðran,
ful-gengun godes lérun: \hld\ ni weldun þemu Judeo kuninge
umbi þes barnes giburd \hld\ bodon óstr-onje,
síð-wórige man \hld\ sęggjan giowiht,
ak wendun im eft an iro willjon. \hld\ Þó warð sán aftar þiu waldandes,
godes ęngil kumen \hld\ Josepe te sprákun,
sagde im an swefne \hld\ slápandium an naht,
bodo drohtines, \hld\ þat þat barn godes
slíð-mód kuning \hld\ sókjan welda,
áhtean is aldres; \hld\ ʽnu skaltu ine an Aegypteo
land ant-lédean \hld\ endi undar þem liudjun wesan
mid þiu godes barnu \hld\ endi mid þeru gódan þior*nan,
wunon undar þemu werode, \hld\ untþat þi word kume
hérron þínes, \hld\ þat þu þat hélage barn
eft te þesum land-skępi \hld\ lédian mótis,
drohtin þínen.ʼ \hld\ Þó fon þem dróma ansprang
Joseph an is gest-sęli, \hld\ endi þat godes gi-bod
sán ant-kenda: \hld\ gi-wét im an þana síð þanen
þe þegan mid þeru þiornon, \hld\ sóhta im þiod óðra
oƀar brédan berg: \hld\ welda þat barn godes
fíundun ant-fórjan. \hld\ *Þó gi-frang aftar þiu % NOTE: gi-frang [sic]
Erodes þe kuning, \hld\ þar he an is ríkja sat,
þat wárun þea wíson man \hld\ westan gi-hworƀan
óstar an iro óðil \hld\ endi fórun im óðran weg:
wisse þat sie im þat árundi \hld\ eft ni weldun
sęggjan an is selðon. \hld\ Þó warð im þes an sorgun hugi,
mód mornondi, \hld\ kwað þat it im þie man dedin,
heliðos* te hónðun. \hld\ Þó he só hriwig sat,
balg ina an is briostun, \hld\ kwað þat he is mahti betaron rád,
óðran gi-þęnkjen: \hld\ ʽnu ik is aldar kan,
wét is winter-gi-talu: \hld\ nu ik gi-winnan mag,
þat he io oƀar þesaro erðu \hld\ ald ni wirðit,
hér undar þesum hęri-skępi.ʼ \hld\ Þó he só hardo gibód,
Erodes oƀar is riki, \hld\ hét þó is rinkos faran
kuning þero liudjo, \hld\ hét þat sie kinda só filo
þurh iro hand-magen \hld\ hóƀdu bi-námin,
só manag barn umbi Bethleem, \hld\ só filo só þar gi-boran wurði,
an twém gérun atogan. \hld\ Tionon frumidon
þes kuninges gi-síðos. \hld\ Þó skolda þar só manag kindisk man
sweltan sundjono lós. \hld\ Ni warð sið noh ér
giámar-líkara for-gang \hld\ jungaro manno,
arm-líkara dóð. \hld\ Idisi wiopun,
módar managa, \hld\ gi-sáhun iro megi spildian:
ni mahte siu im nio gi-formon, \hld\ þoh siu mid iro faðmon twém
iro égan barn \hld\ armun bi-fengi,
liof endi luttil, \hld\ þoh skolda is simbla þat líf geƀan,
þe magu for þeru módar. \hld\ Ménes ni sáhun,
wítjes þie wam-skaðon: \hld\ wápnes ęggjun
fremidun firin-werk mikil. \hld\ Fellun managa
magu-junge man. \hld\ Þia módar wiopun
kind-jungaro kwalm; \hld\ kara was an Bethleem,
hofno hlúdost: \hld\ þoh man im iro herton an twé
sniði mid swerdu, \hld\ þoh ni mohta im gio sérara dád
werðan an þesaro weroldi, \hld\ wíƀun managun,
brúdiun an Bethleem: \hld\ gi-sáhun iro barn biforan,
kind-junge man, \hld\ kwalmu sweltan
blódag an iro barmun. \hld\ Þie banon wítnodun
un-skuldige skole: \hld\ ni bi-skriƀun giowiht
þea man umbi mén-werk: \hld\ weldun mahtigna,
Krist selƀon a-kwęlljan. \hld\ Þan habde ina kraftag god
gineridan wið iro níðe, \hld\ þat inan nahtes þanan
an Aegypteo land \hld\ erlos ant-léddun,
gumon mid Josepe \hld\ an þana grónjon wang,
an erðono beztun, \hld\ þar én aha fliutid,
Níl-stróm mikil \hld\ norð te séwa,
flódo fagorosta. \hld\ Þar þat friðu-barn godes
wonoda an willjon, \hld\ antþat wurd fornam
Erodes þana kuning, \hld\ þat he for-lét eldeo barn,
módag manno dróm. \hld\ Þó skolda þero marka giwald
égan is erƀi-ward: \hld\ þe was Arkheláus
hétan, hęri-togo \hld\ helm-berandero:
þe skolda umbi Hierusalem \hld\ Judeono folkes,
werodes gi-waldan. \hld\ Þó warð word kuman
þar an Egypti \hld\ eðiliun manne,
þat he þar te Josepe, \hld\ godes ęngil sprak,
bodo drohtines, \hld\ hét ina eft þat barn þanan
lédjen te lande. \hld\ ʽnu haƀað þit lioht afgeƀenʼ, kwað he,
ʽErodes þe kuning; \hld\ he welde is áhtien giu,
fréson is ferahas. \hld\ Nu maht þu an friðu lédjen
þat kind undar ewa kunni, \hld\ nu þe kuning ni liƀod,
erl oƀar-módig.ʼ \hld\ Al ant-kende
Josep godes tékan: \hld\ geriwide ina sniumo
þe þegan mit þera þiornun, \hld\ þó sie þanan weldun
béðiu mid þiu barnu: \hld\ léstun þiu berhton gi-skapu,
waldandes willjon, \hld\ al só he im ér mid is wordun gibód.
Gi-witun im þó eft an Galilea-land \hld\ Joseph endi Maria,
hélag híwiski \hld\ heƀen-kuninges,
wárun im an Nazareth-burg. \hld\ Þar þe nęrjondio Krist
wóhs undar þem werode, \hld\ warð gi-witties ful,
an was imu anst godes, \hld\ he was allun liof
módar-mágun: \hld\ he ni was óðrun mannun gilík,
þe gumo an sínera gódi. \hld\ Þó he gér-talo
tweliƀi habde, \hld\ þó warð þiu tíd kuman,
þat sie þar te Hierusalem, \hld\ Juðeo liudi
iro þiod-gode \hld\ þionon skoldun,
wirkjan is willjon. \hld\ Þó warð þar an þana wíh innan
þar te Hierusalem \hld\ Judeono gi-samnod
man-kraft mikil. \hld\ Þar Maria was
self an gi-síðja \hld\ endi iru sunu habda,
godes égan barn. \hld\ Þó sie þat geld habdun,
erlos an þem alaha, \hld\ só it an iro éwa gibód,
gi-léstid te iro land-wísun, \hld\ þó fórun im eft þie liudi þanan,
weros an iro willjon \hld\ endi þar an þem wíha afstód
mahtig barn godes, \hld\ só ina þiu módar þar
ni wissa te wáron; \hld\ ak siu wánda þat he mid þem weroda forð,
fóri mit iro friundun. \hld\ Gifrang aftar þiu
eft an óðrun daga \hld\ aðal-kunnjes wíf,
sálig þiorna, \hld\ þat he undar þem gisíðia ni was.
warð Mariun þó \hld\ mód an sorgun,
hriwig umbi iro herta, \hld\ þó siu þat hélaga barn
ni fand undar þem folka: \hld\ filu gornoda
þiu godes þiorna. \hld\ Giwitun im þó eft te Hierusalem
iro sunu sókjan, \hld\ fundun ina sittjan þar
an þem wíha innan, \hld\ þar þe wísa man,
swíðo glawua gumon \hld\ an godes éwa
lásun ende línodun, \hld\ hwó sie lof skoldin
wirkjan mid iro wordun þem, \hld\ þe þesa werold gi-skóp.
Þar sat undar middjun \hld\ mahtig barn godes,
Krist alo-waldo, \hld\ só is þea ni mahtun ant-kęnnian wiht,
þe þes wíhes þar \hld\ wardon skoldun,
endi frágoda sie \hld\ firi-wit-líko
wísera wordo. \hld\ Sie wundradun alle,
buhwí gio só kindisk man \hld\ sulika kwidi mahti
mid is múðu gi-ménean. \hld\ Þar ina þiu módar fand
sittjan under þem gisíðja \hld\ endi iro sunu grótta,
wísan undar þem weroda, \hld\ sprak im mid ira wordun tó:
ʽhwí weldes þu þínera módar, \hld\ manno lioƀosto,
gi-sidon sulika sorga, \hld\ þat ik þi só sérag-mód,
idis arm-hugdig \hld\ éskon skolda
undar þesun burg-liudjun?ʼ \hld\ Þó sprak iru eft þat barn an-gegin
wísun wordun: \hld\ ʽhwat, þu wést garoʼ, kwað he,
ʽþat ik þar gi-rísu, \hld\ þar ik bi rehton skal
wonon an willjon, \hld\ þar gi-wald haƀad
mín mahtig fader.ʼ \hld\ Þie man ni for-stódun,
þie weros an þem wíha, \hld\ bi-hwí he só þat word gi-sprak,
gi-ménda mid is múðu: \hld\ Maria al bi-held,
gi-barg an ira breostun, \hld\ só hwat só siu gi-hórda ira barn sprekan
wisaro wordo. \hld\ Gi-witun im þó eft þanan
fon Hierusalem \hld\ Joseph endi Maria,
habdun im te gi-síðja \hld\ sunu drohtines,
allaro barno bezta, \hld\ þero þe io gi-boran wurði
magu fon módar: \hld\ habdun im þar minnja tó
þurh hluttran hugi, \hld\ endi he só gi-hórig was,
godes égan barn \hld\ gaduling-mágun
þurh is ód-módi, \hld\ aldron sínun:
ni welda an is kindiski þó noh \hld\ is kraft mikil
mannun márjan, \hld\ þat he sulik megin éhta,
giwald an þesaro weroldi, \hld\ ak he im an is willjon béd
gi-þiudo undar þero þiodu \hld\ þrítig géro,
ér þan he þar tékan énig \hld\ tógean weldi,
sęggjan þem gi-síðja, \hld\ þat he selƀo was
an þesaro middil-gard \hld\ manno drohtin.
Habda im só bi-halden \hld\ hélag barn godes
word endi wís-dóm \hld\ ende allaro gi-witteo mést,
tulgo spáhan hugi: \hld\ ni mahta man is an is sprákun werðan,
an is wordun giwar, \hld\ þat he sulik giwit éhta,
þegan sulika gi-þáhti, \hld\ ak he im só giþiudo béd
torhtaro tékno. \hld\ Ni was noh þan þiu tíd kuman,
þat he ina oƀar þesan \hld\ middil-gard márjan skolda,
lérjan þie liudi, \hld\ hwó sie skoldin iro gilóƀon haldan,
wirkjan willjon godes; \hld\ wissun þat þoh managa
liudi aftar þem landa, \hld\ þat he was an þit lioht kuman,
þoh sie ina kúð-líko \hld\ an-kennian ni mahtin,
ér þan he ina selƀo \hld\ sęggjan welda.
Þan was im Johannes \hld\ fon is iuguð-hédi
awahsan an énero wóstunni; \hld\ þar ni was werodes þan mér,
bútan þat he þar énkora \hld\ alo-waldon gode,
þegan þionoda: \hld\ for-lét þioda gimang,
manno giménðon. \hld\ Þar warð im mahtig kuman
an þero wóstunni \hld\ word fon himila,
gód-lík stemna godes, \hld\ endi Johanne gi-bod,
þat he Kristes kumi \hld\ endi is kraft mikil
oƀar þesan middil-gard \hld\ márjan skoldi;
hét ina wár-líko \hld\ wordun sęggjan,
þat wári heƀan-riki \hld\ heliðo barnun
an þem land-skępi, \hld\ liudjun gináhid,
welono wun-samost. \hld\ Im was þó willjo mikil,
þat he fon sulikun sáldun \hld\ sęggjan mósti.
Gi-wét im þó gangan, \hld\ al só Jordan flót,
watar an willjon, \hld\ endi þem weroda allan dag,
aftar þem land-skępi \hld\ þem liudjun kúðda,
þat sie mid fastunniu \hld\ firin-werk manag,
iro selƀoro \hld\ sundja bóttin,
ʽþat gi werðan hréneaʼ, \hld\ kwað he. ʽHeƀan-riki is
gi-náhid manno barnun. \hld\ Nu látad eu an ewan mód-seƀon
ewar selƀoro \hld\ sundja hrewan,
lédas þat gi an þesun liohta fremidun, \hld\ endi mínun lérun hórjad,
wendeat aftar mínun wordun. \hld\ Ik eu an watara skal
gi-dópjan diur-líko, \hld\ þoh ik ewa dádi ne mugi,
ewar selƀaro \hld\ sundja alátan,
þat gi þurh mín hand-gi-werk \hld\ hluttra werðan
léðaro gi-lésto: \hld\ ak þe is an þit lioht kuman,
mahtig te mannun \hld\ endi undar eu middjun stéd,
—þoh gi ina selƀun \hld\ gi-sehan ni willjan—,
þe eu gi-dópjan skal \hld\ an ewes drohtines namon
an þana hálagon gést. \hld\ Þat is hérro oƀar al:
he mag allaro manno gi-hwena \hld\ mén-gi-þáhteo,
sundjono sikoron, \hld\ só hwene só só sálig mót
werðen an þesaro weroldi, \hld\ þat þes willjon haƀad,
þat he só giléstea, \hld\ só he þesun liudjun wili,
gi-bioden barn godes. \hld\ Ik bium an is bod-skępi herod
an þesa werold kumen \hld\ endi skal im þana weg rúmien,
lérean þesa liudi, \hld\ hwó sea skulin iro gilóƀon haldan
þurh hluttran hugi, \hld\ endi þat sie an hęllja ni þurƀin,
faran an fern þat héta. \hld\ Þes wirðid só fagan an is móde
man te só managaro stundu, \hld\ só hwe só þat mén for-látid,
gerno þes gramon anbusni, \hld\ —só mag im þes gódon giwirkjan,
huldi heƀen-kuninges,— \hld\ só hwe só haƀad hluttra trewa
up te þem alo-mahtigon gode.ʼ \hld\ Erlos managa
bi þem lérun þó, \hld\ liudi wándun,
weros wár-líko, \hld\ þat þat waldand Krist
selbo wári, \hld\ hwanda he só filu sóðes gisprak,
wároro wordo. \hld\ Þó warð þat só wído kúð
oƀar þat for-geƀana land \hld\ gumono gi-hwilikum,
sęggjun at iro selðun: \hld\ þó kwámun ina sókjan þarod
fon Hierusalem \hld\ Judeo liudjo
bodon fon þeru burgi \hld\ endi frágodun, ef he wári þat barn godes,
ʽþat hér lango giuʼ, \hld\ kwaðun sie, ʽliudi sagdun,
weros wár-líko, \hld\ þat he skoldi an þesa werold kumanʼ.
Johannes þó gimahalde \hld\ endi te-gegnes sprak
þem bodun bald-líko: \hld\ ʽni bium ikʼ, kwað he, ʽþat barn godes,
wár waldand Krist, \hld\ ak ik skal im þana weg rúmien,
hérron mínumu.ʼ \hld\ Þea heliðos frugnun,
þea þar an þem árundie \hld\ erlos wárun,
bodon fon þero burgi: \hld\ ʽef þu nu ni bist þat barn godes,
bist þu þan þoh Elias, \hld\ þe hér an ér-dagun
was undar þesumu werode? \hld\ He is wiskumo
eft an þesan middil-gard. \hld\ Saga ús hwat þu manno sís!
Bist þu énig þero, \hld\ þe hér ér wári
wísaro wár-saguno? \hld\ Hwat skulun wi þem werode fon þi
sęggjan te sóðon? \hld\ Neo hér ér sulik ni warð
an þesun middil-gard \hld\ man óðar kuman
dádjun só mári. \hld\ Bi-hwí þu hér dópisli
fremis undar þesumu folke, \hld\ ef þu þaro fora-sagono
én-hwilik ni bist?ʼ \hld\ Þó habde eft garo
Johannes þe gódo \hld\ glau and-wordi:
ʽIk bium fora-bodo \hld\ fráon mínes,
lioƀes hérron; \hld\ ik skal þit land rekon,
þit werod aftar is willjon. \hld\ Ik hębbju fon is worde mid mi
stranga stemna, \hld\ þoh sie hér ni willje for-standan filo
werodes an þesaro wóstunni. \hld\ Ni bium ik mid wihti gilík
drohtine mínumu: \hld\ he is mid is dádjun só strang,
só mári endi só mahtig \hld\ —þat wirðid managun kúð,
werun aftar þesaro weroldi— \hld\ þat ik þes wirðig ni bium,
þat ik móti an is gi-skuoha, \hld\ þoh ik sí is skalk égan,
an só ríkiumu drohtine, \hld\ þea reomon ant-bindan:
só mikilu is he betara þan ik. \hld\ Nis þes bodon gi-mako
énig oƀar erðu, \hld\ ne nu aftar ni skal
werðan an þesaro weroldi. \hld\ Hębbjad ewan willjon þarod,
liudi ewan gi-lóƀon: \hld\ þan eu lango skal
wesan ewa hugi hrómag; \hld\ þan gi hęlli-gi-þwing,
for-látad léðaro dróm \hld\ endi sókjad eu lioht godes,
up-ódes hém, \hld\ éwig ríki,
hóhan heƀen-wang. \hld\ Ne látad ewan hugi twíflien!ʼ
Só sprak þó jung gumo \hld\ bi godes lérun
mannun te márðu. \hld\ Manag samnoda
þar te Bethania \hld\ barn Israheles;
kwámun þar te Johannese \hld\ kuningo gi-síðos,
liudi te lérun \hld\ endi iro gi-lóƀon ant-fengun.
He dópte sie dago gihwilikes endi im iro dádi lóg,
wréðaro willjon, endi loƀode im word godes,
hérron sínes: ʽheƀen-ríki wirðidʼ, kwað he,
ʽgaru gumono só hwem, só ti gode þenkid
endi an þana héljand *wili hluttro gilóƀjan,
léstean is léraʼ. Þó ni was lang te þiu,
þat im fon Galilea giwét godes égan barn,
*diurlík drohtines sunu, dópi suokjan.
was im þuo an is wastme waldandes barn*,
al só he mid þero þiodu þrítig habdi
wintro an is weroldi. Þó he an is willjon kwam,
þar Johannes an Jordana stróme
allan langan dag liudi manage
dópte diurlíko. Reht só he þó is drohtin gisah,
holden hérron, só warð im is hugi blíði,
þes im þe willjo gistód, endi sprak im þó mid is wordun tó,
swíðo gód gumo, Johannes te Kriste:
ʽnu kumis þu te mínero dópi, drohtin fró mín,
þiod-gumono bezto: \hld\ só skolde ik te þínero duan,
hwand þu bist allaro kuningo kraftigost.ʼ \hld\ Krist selƀo gi-bód,
waldand wár-líko, \hld\ þat he ni spráki þero wordo þan mér:
ʽwést þu, þat ús só girísidʼ, \hld\ kwað he, ʽallaro rehto gi-hwilik
te gi-fulleanne \hld\ forð-wardes nu
an godes willjonʼ. \hld\ Johannes stód,
dópte allan dag \hld\ druht-folk mikil,
werod an watere \hld\ endi ók waldand Krist,
héran heƀen-kuning \hld\ handun sínun
an allaro baðo þem bezton \hld\ endi im þar te bedu gi-hnég
an kneo kraftag. \hld\ Krist up gi-wét
fagar fon þem flóde, \hld\ friðu-barn godes,
liof liudjo ward. \hld\ Só he þó þat land af-stóp,
só ant-hlidun þó himiles doru, \hld\ endi kwam þe hélago gést
fon þem alo-waldon oƀane te Kriste:
— was im an gilíknissie lungras fugles,
diurlíkara dúƀun — endi sat im uppan úses drohtines ahslu,
wonoda im oƀar þem waldandes barne. Aftar kwam þar word fon himile,
hlúd fon þem hóhon radura en grótta þane héljand selƀon,
Krista, allaro kuningo bezton, kwað þat he ina gikorana habdi
selƀo fon sínun ríkja, kwað þat im þe sunu líkodi
bezt allaro gi-boranaro manno, kwað þat he im wári allaro barno lioƀost.
Þat móste Johannes þó, al só it god welde,
gisehan endi gi-hórjan. He gideda it sán aftar þiu
mannun mári, þat sie þar mahtigna
hérron habdun: ʽþit isʼ, kwað he, ʽheƀen-kuninges sunu,
én alo-waldand: þesas willjo ik ur-kundjo
wesan an þesaro weroldi, hwand it sagda mi word godes,
drohtines stemne, þó he mi dópjan hét
weros an watare, só hwar só ik gisáhi wár-líko
þana hélagon gést *fan heƀan-wange
an þesan middil-gard énigan man waron,
kuman mid kraftu; þat kwað, þat skoldi Krist wesan,
diurlík drohtines suno. Hie dópjan skal
an þana* hélagan gést* endi hélean managa
manno méndádi. He haƀad maht fon gode,
þat he alátan mag liudjo gihwilikun
saka endi sundja. Þit is selƀo Krist,
godes égan barn, gumono bezto,
friðu wið fíundun. Wala þat eu þes mag fráhmód hugi
wesan an þesaro weroldi, þes eu þe willjo gistód,
þat gi só libbjanda þana landes ward
selƀon gisáhun. Nu mót sliumo sundjono lós
manag gést faran an godes willjon
tionon atómid, þe mid trewon wili
wið is wini wirkjan endi an waldand Krist
fasto gilóƀjan. Þat skal te frumun werðen
gumono só hwilikun, só þat gerno dótʼ.
Só gefragn ik þat Johannes \hld\ þó gumono gi-hwilikun,
loƀoda þem liudjun \hld\ léra Kristes,
hérron sines, \hld\ endi heƀen-ríki
te giwinnanne, \hld\ welono þane méston,
sálig sin-líf. \hld\ Þó he im selƀo giwét
aftar þem dópislea, \hld\ drohtin þe gódo,
an éna wóstunnja, \hld\ waldandes sunu;
was im þar an þero én-ódi \hld\ erlo drohtin
lange hwíla; \hld\ ne habda liudjo þan mér,
seggjo te gi-síðun, \hld\ al só he im selƀo gi-kós:
welda is þar látan koston \hld\ kraftiga wihti,
selƀon Satanasan, \hld\ þe gio an sundja spenit,
man an mén-werk: \hld\ he konsta is mód-seƀon,
wréðan willjon, \hld\ hwó he þesa werold érist,
an þem anginnja \hld\ irmin-þioda
bi-swék mit sundjun, \hld\ þó he þiu sinhíun twé,
Ádaman endi Éuan, \hld\ þurh un-trewa
for-lédda mid luginun, \hld\ þat liudo barn
aftar iro hin-ferdi \hld\ hęllja sóhtun,
gumono géstos. \hld\ Þó welda þat god mahtig,
waldand wendean \hld\ endi welda þesum werode forgeƀen
hóh himil-ríki: \hld\ beþiu he herod hélagna bodon,
is sunu senda. \hld\ Þat was Satanase
tulgo harm an is hugi: \hld\ afonsta heƀan-ríkjes
manno kunnje: \hld\ welda þó mahtigna
mid þem selƀon sakun \hld\ sunu drohtines,
þem he Ádaman \hld\ an ér-dagun
darnungo bidróg, \hld\ þat he warð is drohtine léð,
bi-swék ina mid sundjun — só welda he þó selƀan dón
hélandean Krist. Þan habda he is hugi fasto
wið þana wam-skaðon, waldandes barn,
herte só giherdid: welda heƀen-ríki
liudjun giléstean. Was im þes landes ward
an fastunnea fiortig nahto,
manno drohtin, só he þar mates ni antbét;
þan langa ni gidorstun im dernea wihti,
níðhugdig fíund, náhor gangan,
grótean ina geginwarðan: wánde þat he god énfald,
forútar man-kunnjes wiht mahtig wári,
héleg himiles ward. Só he ina þó gehungrean lét,
þat ina bigan bi þero mennisko móses lustean
aftar þem fiwartig dagun, þe fíund náhor geng,
mirki ménskaðo: wánda þat he man énfald
wári wissungo, sprak im þó mid is wordun tó,
grótta ina þe gérfíund: ʽef þu sís godes sunuʼ, kwað he,
ʽbe-hwí ni hétis þu þan werðan, ef þu giwald haƀes,
allaro barno bezt, bród af þesun sténun?
Gehéli þínna hungar.ʼ Þó sprak eft þe hélago Krist:
ʽni mugun eldi-barnʼ, kwað he, ʽénfaldes bródes,
liudi libbien, ak sie skulun þruh léra godes
wesan an þesero weroldi endi skulun þiu werk frummjen,
þea þar werðad ahlúdid fon þero hélogun tungun,
fon þem galme godes: þat is gumono líf
liudjo só hwilikon, só þat léstean wili,
þat fon waldandes worde gebiudid.ʼ
Þó bigan eft niuson endi náhor geng
unhiuri fíund óðru síðu,
fandoda is fróhan. Þat friðu-barn þolode
wréðes willjon endi im giwald forgaf,
þat he umbi is kraft mikil koston mósti,
lét ina þó lédean þana liud-skaðon,
þat he ina an Hierusalem te þem godes wíha,
alles oƀanwardan, up gisetta
an allaro húso hóhost, endi hoskwordun sprak,
þe gramo þurh gelp mikil: ʽef þu sís godes sunuʼ, kwað he,
ʽskríd þi te erðu hinan. Geskriƀan was it giu lango,
an bókun gewriten, hwó gi-boden haƀad
is ęngilun alo-mahtig fader,
þat sie þi at wege gehwem wardos sinðun,
haldad þi undar iro handun. Hwat, þu hwargin ni þarft
mid þínun fótun an felis bespurnan,
an hardan stén.ʼ Þó sprak eft þe hélago Krist,
allaro barno bezt: ʽsó is ók an bókun geskriƀanʼ, kwað he,
ʽþat þu te hardo ni skalt hérran þínes,
fandon þínes fróhan: þat nis þi allaro frumono negén.ʼ
Lét ina þó an þana þriddjan síð þana þiod-skaðon
gibrengen uppan énan berg þen hóhon: þar ina þe balo-wíso
lét al oƀar-sehan irmin-þiode,
wonod-saman welon endi werold-ríki
endi all sulik ódes, só þius erða bi-haƀad
fagororo frumono, endi sprak im þó þe fíund angegin,
kwað þat he im þat al só gód-lík forgeƀen weldi,
hóha heri-dómos, ʽef þu wilt hnígan te mi,
fallan te mínun fótun endi mi for fróhan haƀas,
bedos te mínun barma. Þan látu ik þi brúkan wel
alles þes ódwelon, þes ik þi hębbju giógit hír.ʼ
Þó ni welda þes léðan word lengeron hwíle
hórjan þe hélago Krist, ak he ina fon is huldi fordréf,
Satanasan forswép, endi sán aftar sprak
allaro barno bezt, kwað þat man bedon skoldi
up te þem alo-mahtigon gode endi im énum þionon
swíðo þioliko þegnos managa,
heliðos aftar is huldi: ʽþar ist þiu helpa gelang
manno gehwilikun.ʼ Þó giwét im þe ménskaðo,
swíðo sérag-mód Satanas þanan,
fíund undar ferndalu. Warð þar folk mikil
fon þem alo-waldan oƀana te Kriste
godes ęngilo kumen, þie im síðor jungardóm,
skoldun ambahtskepi aftar léstien,
þionon þiolíko: só skal man þiodgode,
hérron aftar huldi, heƀan-kuninge.
was im an þem sinweldi sálig barn godes
lange hwíle, untþat im þó lioƀora warð,
þat he is kraft mikil kúðien wolda
weroda te willjon. Þó for-lét he waldes hléo,
énódies ard endi sóhte im eft erlo gemang,
mári megin-þiode endi manno dróm,
geng im þó bi Jordanes staðe: þar ina Johannes antfand,
þat friðu-barn godes, fróhan sínan,
hélagana heƀen-kuning, endi þem heliðun sagda,
Johannes is jungurun, þó he ina gangan gesah:
ʽþit is þat lamb godes, þat þar lósean skal
af þesaro wídon werold wréða sundja,
man-kunnjas mén, mári drohtin,
kuningo kraftigost.ʼ Krist im forð giwét
an Galileo land, godes égan barn,
fór im te þem friundun, þar he afódit was,
tírlíko atogan, endi talda mid wordun
Krist undar is kunnje, kuningo ríkjost,
hwó sie skoldin iro selƀoro sundja bótean,
hét þat sie im iro harmwerk manag hrewan létin,
feldin iro firindádi: ʽnu is it all gefullot só,
só hír alde man ér hwanna sprákun,
gehétun eu te helpu heƀen-ríki:
nu is it giu gináhid þurh þes neriandan kraft: þes mótun gi neotan forð,
só hwe só gerno wili gode þeonogean,
wirkjan aftar is willjon.ʼ Þó warð þes werodes filu,
þero liudjo an lustun: wurðun im þea léra Kristes,
só swótja þem gisíðja. He bigan im samnon þó
gumono te iungoron, gódoro manno,
wordspáha weros. Geng im þó bi énes watares staðe,
þat þar habda Jordan aneƀan Galileo land
énna sé gewarhtan. Þar he sittjan fand
Andreas endi Petrus bi þem ahastróme,
béðea þea gebróðar, þar sie an bréd watar
swíðo niudlíko netti þenidun,
fiskodun im an þem flóde. Þar sie þat friðu-barn godes
bi þes sées staðe selƀo grótta,
hét þat sie im folgodin, kwað þat he im só filu woldi
godes ríkjas forgeƀen; ʽal só git hír an Jordanes stróme
fiskos fáhat, só skulun git noh firiho barn
halon te inkun handun, þat sie an heƀen-ríki
þurh inka léra líðan mótin,
faran folk manag.ʼ Þó warð frómód hugi
béðiun þem gibróðrun: ant-kendun þat barn godes,
lioƀan hérron: for-létun al saman
Andreas endi Petrus, só hwat só sie bi þeru ahu habdun,
ge-wunstes bi þem watare: was im willjo mikil,
þat sie mid þem godes barne gangan móstin,
samad an is gisíðja, skoldun sáliglíko
lón antfáhan: só dót liudjo so hwilik,
só þes hérran wili huldi giþionon,
gewirkjan is willjon. Þó sie bi þes watares staðe
furðor kwámun, þó fundun sie þar énna fródan man
sittjan bi þem séwa endi is suni twéne,
Jakobus endi Johannes: wárun im iunga man.
Sátun im þá gesunfader an énumu sande uppen,
brugdun endi bóttun béðium handun
þiu netti niudlíko, þea sie habdun nahtes ér
forsliten an þem séwa. Þar sprak im selƀo tó
sálig barn godes, hét þat sie an þana síð mid im,
Jakobus endi Johannes, gengin béðie,
kindiunge man. Þó wárun im Kristes word
só wirðig an þesaro weroldi, þat sie bi þes watares staðe
iro aldan fader énna for-létun,
fródan bi þem flóde, endi al þat sie þar fehas éhtun,
nettiu endi neglitskipu, gekurun im þana neriandan Krist,
hélagna te hérron, was im is helpono þarf
te giþiononne: só is allaro þegno gehwem,
wero an þesero weroldi. Þó giwét im þe waldandes sunu
mid þem fiwariun forð, endi im þó þana fífton gikós
Krist an énero kópstedi, kuninges iungoron,
módspáhana man: Mattheus was he hétan,
was im ambahtjo eðilero manno,
skolda þar te is hérron handun antfáhan
tins endi tolna; trewa habda he góda,
áðalandbári: for-lét al saman
gold endi siluƀar endi geƀa managa,
diurie méðmos, endi warð im úses drohtines man;
kós im þe kuninges þegn Krist te hérran,
milderan méðom-geƀon, þan ér is man-drohtin
wári an þesero weroldi: feng im wóðera þing,
lang-samoron rád. Þó warð it allun þem liudjun kúð,
fon allaro burgo gihwem, hwó þat barn godes
samnode ge-síðos endi selƀo gesprak
só manag wíslík word endi wáres só filu,
torhtes gitógde endi tékan manag
gewarhte an þesero weroldi. \hld\ Was þat an is wordun skín
iak an is dádjun só same, \hld\ þat he drohtin was,
himilisk hérro \hld\ endi te helpu kwam
an þesan middil-gard \hld\ manno barnun,
liudjun te þesun liohta. \hld\ Oft ge-deda he þat an þem lande skín,
þan he þar torht-líko \hld\ só manag tékan giwarhte,
þar he hélde mid is handun \hld\ halte endi blinde,
lósde af þeru léf-hédi \hld\ liudi manage,
af sulikun suhtiun, \hld\ só þan allaro swároston
an firiho barn \hld\ fíund bi-wurpun,
tulgo lang-sam legar. \hld\ Þó fórun þar þie liudi tó
allaro dago ge-hwilikes, \hld\ þar úsa drohtin was
selƀo undar þem gi-síðje, \hld\ untþat þar ge-samnod warð
megin-folk mikil \hld\ managero þiodo,
þoh sie þar alle be ge-líkumu \hld\ ge-lóƀon ni kwámin.
weros þurh énan willjon: \hld\ sume sóhtun sie þat waldandes barn,
armoro manno filu \hld\ —was im átes þarf—,
þat sie im þar at þeru menigi \hld\ mates endi drankes,
þigidin at þeru þiodu; \hld\ hwand þar was manag þegan só gód,
þie ira alamosnie \hld\ armun mannun
gerno gáƀun. \hld\ Sume wárun sie im eft Judeono kunnjes,
fégni folk-skępi: \hld\ wárun þar ge-farana te þiu,
þat sie úses drohtines \hld\ dádjo endi wordo
fáron woldun, \hld\ habdun im fégnien hugi,
wréðen willjon: \hld\ woldun waldand Krist
alédjen þem liudjun, \hld\ þat sie is léron ni hórdin,
ne wendin aftar is willjon. \hld\ Suma wárun sie im eft só wíse man,
wárun im glawe gumon \hld\ endi gode werðe,
alesane undar þem liudjun, \hld\ kwámun im þarod be þem léron Kristes,
þat sie is hélag word \hld\ hórjen móstin,
línon endi léstien: \hld\ habdun mid iro ge-lóƀon te im
fasto gefangen, \hld\ habdun im ferhten hugi,
wurðun is þegnos te þiu, \hld\ þat he sie an þiod-welon
aftar iro én-dagon \hld\ up ge-bráhti,
an godes ríki. \hld\ He só gerno ant-feng
man-kunnjes manag \hld\ endi mund-burd gihét
te langaru hwílu, \hld\ endi mahta só gi-léstjen wel.
Þó warð þar megin só mikil \hld\ umbi þana márjon Krist,
liudjo ge-samnod: \hld\ þó gisah he fon allun landun kuman,
fon allun wídun wegun \hld\ werod te-samne
lungro liudjo: is lof was só wído
managun gemárid. Þó giwét im mahtig self
an énna berg uppan, barno ríkjost,
sundar gesittjen, endi im selƀo gekós
tweliƀi getalda, trew-hafta man,
gódoro gumono, þea he im te iungoron forð
allaro dago gehwilikes, drohtin welda
an is ge-síðskępja \hld\ simblon hębbjan.
Nemnida sie þó bi naman \hld\ endi hét sie im þó náhor gangan,
Andreas endi Petrus \hld\ érist sána,
ge-bróðar twéne, \hld\ endi béðie mid im,
Jakobus endi Johannes: \hld\ sie wárun gode werðe;
mildi was he im an is móde; \hld\ sie wárun énes mannes suni
béðie bi ge-burdjun; \hld\ sie kós þat barn godes
góde te jungoron \hld\ endi gumono filu,
márjero manno: \hld\ Mattheus endi Þomas,
Judasas twéna \hld\ endi Jakob óðran,
is selƀes swiri: \hld\ sie wárun fon gisustruonion twém
knósles kumana, \hld\ Krist endi Jakob,
góde gadulingos. \hld\ Þó habda þero gumono þar
þe nęrjendo Krist \hld\ niguni getalde,
trew-hafte man: \hld\ þó hét he ók þana tehandon gangan
selƀo mid þem gisíðun: \hld\ Símon was he hétan;
hét ók Bartholomeus \hld\ an þana berg uppan
faran fan þem folke áðrum \hld\ endi Philippus mid im,
trew-hafte man. \hld\ Þó gengun sie tweliƀi samad,
rinkos te þeru rúnu, \hld\ þar þe rádand sat,
managoro mund-boro, \hld\ þe allumu man-kunnje
wið hęllje geþwing \hld\ helpan welde,
formon wið þem ferne, \hld\ só hwem só frummjen wili
só lioƀ-líka léra, \hld\ só he þem liudjun þar
þurh is gi-wit mikil \hld\ wísean hogda.
*Þó umbi þana nęrjendon Krist \hld\ náhor gengun
sulike ge-síðos, só he im selƀo ge-kós,
waldand undar þem werode. Stódun wísa man,
gumon umbi þana godes sunu gerno swíðo,
weros an willjon: was im þero wordo niud,
þáhtun endi þagodun, hwat im þero þiodo drohtin,
weldi waldand self wordun kúðien
þesum liudjun te lioƀe. Þan sat im þe landes hirdi
geginward for þem gumun, godes égan barn:
welda mid is sprákun spáhword manag
lérean þea liudi, hwó sie lof gode
an þesum werold-ríkja wirkjan skoldin.
Sat im þó endi swígoda endi sah sie an lango,
was im hold an is hugi hélag drohtin,
mildi an is móde, endi þó is mund antlók,
wísde mid wordun waldandes sunu
manag márlík þing endi þem mannum sagde
spáhun wordun, þem þe he te þeru spráku þarod,
Krist alo-waldo, gekoran habda,
hwilike wárin allaro irminmanno
gode werðoston gumono kunnjes;
sagde im þó te sóðan, kwað þat þie sáliga wárin,
man an þesoro middil-gardun, þie hér an iro móde wárin
arme þurh ód-módi: ʽþem is þat éwana ríki,
swíðo hélaglík an heƀan-wange
sinlíf fargeƀen.ʼ kwað þat ók sálige wárin
máðmundie man: ʽþie mótun þie márjon erðe,
ofsittjen þat selƀe ríki.ʼ kwað þat ók sálige wárin,
þie hír wiopin iro wammun dádi; ʽþie mótun eft willjon gebídan,
frófre an iro fráhon ríkja. Sálige sind ók, þe sie hír frumono gilustid,
rinkos, þat sie rehto adómien. Þes mótun sie werðan an þem ríkja drohtines
gifullit þurh iro ferhton dádi: sulíkoro mótun sie frumono biknégan
þie rinkos, þie hír rehto adómiad, ne willjad an rúnun beswíkan
man, þar sie at mahle sittjad. Sálige sind ók þem hír mildi wirðit
hugi an heliðo briostun: þem wirðit þe hélego drohtin,
mildi mahtig selƀo. Sálige sind ók undar þesaro managon þiodu,
þie hębbjad iro herta gihrénod: þie mótun þane heƀenes waldand
sehan an sínum ríkja.ʼ kwað þat ók sálige wárin,
ʽþie þe friðusamo undar þesumu folke libbiod endi ni willjad éniga fehta gewirken,
saka mid iro selƀoro dádjun: þie mótun wesan suni drohtines genemnide,
hwande he im wil genádig werðen; þes mótun sie niotan lango
selƀon þes sínes ríkjes.ʼ kwað þat ók sálige wárin
þie rinkos, þe rehto weldin, ʽendi þurh þat þolod ríkjoro manno
heti endi harm-kwidi: þem is ók an himile eft
godes wang for-geƀen endi gést-lík líf
aftar te éwan-dage, só is io endi ni kumit,
welan wun-sames.ʼ \hld\ Só habde þó waldand Krist
for þem erlon þar \hld\ ahto getalda
sálda gesagda; \hld\ mid þem skal simbla gihwe
himil-rík gehalon, \hld\ ef he it hębbjen wili,
etþo he skal te éwan-daga aftar þarƀon
welon endi willjon, síðor he þese werold agiƀid,
erðlíƀigi-skapu, endi sókit im óðar lioht
só liof só léð, só he mid þesun liudjun hér
gi-werkod an þesoro weroldi, al só it þar þó mid is wordun sagde
Krist alo-waldo, kuningo ríkjost
godes égen barn iungorun sínun:
ʽGe werðat ók só sáligeʼ, kwað he, ʽþes iu saka biodat
liudi aftar þeson lande endi léð sprekat,
hębbjad iu te hoska endi harmes filu
gewirkiad an þesoro weroldi endi wíti gefrummjad,
felgiad iu firinspráka endi fíundskepi,
lágniad iuwa léra, dót iu léðes filu,
harmes þurh iuwen hérron. Þes látad gi ewan hugi simbla,
líf an lustun, hwand iu þat lón stendit
an godes ríkja garu, gódo gehwilikes,
mikil endi managfald: þat is iu te médu fargeƀen,
hwand gi hér ér biforan arƀid þolodun,
wíti an þesoro weroldi. \hld\ Wirs is þem óðrun,
giƀiðig grimmora þing, \hld\ þem þe hér gód égun,
wídan woroldwelon: \hld\ þie for-slítat iro wunnja hér;
geniudot sie genóges, \hld\ skulun eft narowaro þing
aftar iro hin-ferdi \hld\ heliðos þolojan.
Þan wópjan þar wan-skęfti, \hld\ þie hér ér an wunnjon sín,
libbjad an allon lustun, ne willjad þes farlátan wiht,
ménigiþáhtio, þes sie an iro mód spenit,
léðoro giléstio. Þan im þat lón kumid,
uƀil arƀetsam, þan sie is þane endi skulun
sorgondi gesehan. Þan wirðid im sér hugi,
þes sie* þesero weroldes só filu willjan fulgengun,
man an iro mód-seƀon. Nu skulun gi im þat mén lahan,
werean mid wordun, al só ik giu nu gewísean mag,
sęggjan sóðlíko, ge-síðos míne,
wárun wordun, þat gi þesoro weroldes nu forð
skulun salt wesan, sundigero manno,
bótjan iro baludádi, þat sie an betara þing,
folk farfáhan endi for-látan fíundes giwerk,
diuƀales gedádi, endi sókjan iro drohtines ríki.
Só skulun gi mid iuwon lérun liudfolk manag
wendean aftar mínon willjon. Ef iuwar þan awirðid hwilik,
farlátid þea léra, þea he léstean skal,
þan is im só þem salte, þe man bi sées staðe
wído tewirpit: þan it te wihti ni dóg,
ak it firiho barn fótun spurnat,
gumon an greote. Só wirðid þem, þe þat godes word skal
mannum márjan: ef he im þan látid is mód twehon,
þat hi ne willja mid hluttro hugi te heƀen-ríkja
spanen mid is spráku endi sęggjan spel godes,
ak wenkid þero wordo, þan wirðid im waldand gram,
mahtig módag, endi só samo manno barn;
wirðid allun þan irmin-þiodun,
liudjun aléðid, ef is léra ni dugun.ʼ
So sprak he þó spáhlíko endi sagda spel godes,
lérde þe landes ward liudi síne
mid hluttru hugi. Heliðos stódun,
gumon umbi þana godes sunu gerno swíðo,
weros an willjon: was im þero wordo niud,
þáhtun endi þagodun, gi-hórdun þero þiodo drohtin
sęggjan éu godes eldi-barnun;
gihét im heƀen-ríki endi te þem heliðun sprak:
ʽók mag ik iu sęggjan, ge-síðos mína,
wárun wordun, þat gi þesoro weroldes nu forð
skulun lioht wesan liudjo barnun,
fagar mid firihun oƀar folk manag,
wlitig endi wun-sam: ni mugun iuwa werk mikil
biholan werðan, mid hwiliko gi sea hugi kúðeat:
þan mér þe þiu burg ni mag, þiu an berge stáð,
hóh holmkliƀu, biholen werðen,
wrisi-lík giwerk, ni mugun iuwa word þan mér
an þesoro middil-gard mannum werðen,
iuwa dádi bidernit. Dót, só ik iu lériu:
látad iuwa lioht mikil liudjun skínan,
manno barnun, þat sie farstandan iuwan mód-seƀon,
iuwa werk endi iuwan willjon, \hld\ endi þes waldand god
mit hluttro hugi, \hld\ himiliskan fader,
loƀon an þesumu liohte, \hld\ þes he iu sulika léra far-gaf.
Ni skal neoman lioht, þe it haƀad, \hld\ liudjun dernean,
te hardo be-hwelƀean, \hld\ ak he it hóho skal
an seli sęttjan, \hld\ þat þea ge-sehan mugin
alla ge-liko, \hld\ þea þar inna sind,
heliðos an hallu. \hld\ Þan hald ni skulun gi iuwa hélag word
an þesumu land-skępa \hld\ liudjun dernien,
helið-kunnje farhelan, \hld\ ak ge it hóho skulun
brédean, þat gi-bod godes, \hld\ þat it allaro barno gehwilik,
oƀar al þit land-skępi \hld\ liudi far-standan
endi só ge-frummjen, \hld\ só it an forn-dagun
tulgo wíse man \hld\ wordun ge-sprákun,
þan sie þana aldan éw \hld\ erlos heldun,
endi ók suliku swíðor, \hld\ só ik iu nu sęggjan mag,
alloro gumono ge-hwilik \hld\ gode þionojan,
þan it þar an þem aldom \hld\ éwa ge-beode.
Ni wánjat gi þes mit wihtju, \hld\ þat ik bi þiu an þesa werold kwámi,
þat ik þana aldan éu \hld\ irrjen willje,
fellean undar þesumu folke efþo þero fora-sagono
word wiðar-werpen, þea hér só gi-wárea man
bar-líko gebudun. Ér skal béðiu tefaran,
himil endi erðe, þiu nu bi-hlidan standat,
ér þan þero wordo wiht bilíƀa
un-léstid an þesumu liohte, þea sie þesum liudjun hér
wár-líko gebudun. Ni kwam ik an þesa werold te þiu,
þat ik feldi þero forasagono word, ak ik siu fullien skal,
ókion endi nígean eldi-barnum,
þesumu folke te frumu. Þat was forn geskriƀan
an þem aldon éo — ge hórdun it oft sprekan
wordwíse man —: só hwe só þat an þesoro weroldi gidót,
þat he áðrana aldru bineote,
líƀu bilósje, þem skulun liudjo barn
dód adélean. Þan willjo ik it iu diopor nu,
furður bifáhan: só hwe só ina þurh fíundskepi,
man wiðar óðrana an is mód-seƀon
bilgit an is breostun — hwand sie alle gebróðar sint,
sálig folk godes, sibbjon bitengea,
man mid mágskepi —, þan wirðit þoh hwe óðrumu an is móde só gram,
líbes weldi ina bilósjen, of he mahti giléstien só:
þan is he sán aféhit endi is þes ferahas skolo,
al sulikes urdélies só þe óðar was,
þe þurh is handmegin hóƀdo bilósde
erl óðarna. Ôk is an þem éo geskriƀan
wárun wordun, só gi witon alle,
þan man is náhiston niudlíko skal
minnjan an is móde, wesen is mágun hold,
gadulingun gód, wesen is geƀa mildi,
fráhon is friunda gehwane, endi skal is fíund hatan,
wiðer-standen þem mid strídu endi mid starku hugi,
werean wiðar wréðun. Þan seggjo ik iu te wáron nu,
fullíkur for þesumu folke, þat gi iuwa fíund skulun
minnjon an iuwomu móde, só samo só gi iuwa mágos dót,
an godes namon. Dót im gódes filu,
tógeat im hluttran hugi, holda trewa,
liof wiðar ira léðe. Þat is lang-sam rád
manno só hwilikumu, só is mód te þiu
geflíhit wiðar is fíunde. Þan mótun gi þea fruma égan,
þat gi mótun héten heƀen-kuninges suni,
is blíði barn. Ne mugun gi iu betaran rád
gewinnan an þesoro weroldi. Þan sęggjo ik iu te wáron ók,
barno gehwilikum, þat gi ne mugun mid gibolgono hugi
iuwas gódes wiht te godes húsun
waldande fargeƀan, þat it imu wirðig sí
te antfáhanne, só lango só þu fíund-skępjes wiht,
wiðer óðran man inwid hugis.
Ér skalt þu þi simbla gesónien wið þana sakwaldand,
gemódi gimahlean: síðor maht þu méðmos þína
te þem godes altere ageƀan: þan sind sie þemu gódan werðe,
heƀen-kuninge. Mér skulun gi aftar is huldi þionon,
godes willjon fulgán, þan óðra Judeon duon,
ef gi willjat égan éwan ríki,
sinlíf sehan. Ôk skal ik iu sęggjan noh,
hwó it þar an þem aldon éo gebiudid,
þat énig erl óðres idis ni biswíka,
wíf mid wammu. Þan sęggjo ik iu te wáron ók,
þat þar man is siuni mugun swíðo farlédean
an mirki mén, ef hi ina látid is mód spanen,
þat he beginna þero girnean, þiu imu gegangan ni skal.
Þan haƀed he an imu selƀon sán sundja gewarhta,
geheftid an is hertan helliwíti.
Ef þan þana man is siun wili etþa is swíðare hand
farlédjen is liðo hwilik an léðan weg,
þan is erlo gehwem óðar betara,
firiho barno, þat he ina fram werpa
endi þana lið lósje af is lík-hamon
endi ina áno kuma up te himile,
þan he só mid allun te þem inferne,
hwerƀe mid só hélun an helligrund.
Þan ménid þiu léf-héd, þat énig liudjo ni skal
farfolgan is friunde, ef he ina an firina spanit,
swás man an saka: þan ne sí he imu eo só swíðo an sibbiun bilang,
ne iro mágskepi só mikil, ef he ina an morð spenit,
bédid balu-werko; betera is imu þan óðar,
þat he þana friund fan imu fer farwerpa,
míðe þes máges endi ni hębbja þar éniga minnja tó,
þat he móti éno up gestígan
hó himil-ríki, þan sie helligeþwing,
bréd bal-wíti \hld\ béðea gisókjan,
uƀil arƀidi. \hld\ Ôk is an þem éo geskriƀan
wárun wordun, \hld\ só gi witun alle,
þat míðe mén-éðos \hld\ man-kunnjes gehwilik,
ni for-swerie ina selƀon, \hld\ hwand þat is sundie te mikil,
far-lédid liudi \hld\ an léðan weg.
Þan willjo ik iu eft sęggjan, \hld\ þan sán ni swerea neoman
énigan éð-staf \hld\ eldi-barno,
ne bi himile þemu hóhon, \hld\ hwand þat is þes hérron stól,
ne bi erðu þar undar, \hld\ hwand þat is þes alo-waldon
fagar fót-skamel, \hld\ nek énig firiho barno
ne swerea bi is selƀes hóƀde, \hld\ hwand he ni mag þar ne swart ne hwít
énig hár ge-wirkjan, \hld\ bútan só it þe hélago god,
ge-markode mahtig; \hld\ beþiu skulun míðan filu
erlos éð-wordo. \hld\ Só hwe só it ofto dót,
só wirðid is simbla wirsa, \hld\ hwand he imu giwardon ni mag.
Biþiu skal ik iu nu te wárun \hld\ wordun gibeodan,
þat gi neo ne swerien \hld\ swíðoron éðos,
méron met mannun, \hld\ bútan só ik iu mid mínun hér
swíðo wár-liko \hld\ wordun gebiudu:
ef man hwemu saka sókja, \hld\ bisęggja þat wáre,
kweðe iá, gef it sí, \hld\ geha þes þar wár is,
kweðe nén, af it nis, \hld\ láta im genóg an þiu;
só hwat só is mér oƀar þat man gefrummjad,
só kumid it al fan uƀile \hld\ eldi-barnun,
þat erl þurh un-trewa \hld\ óðres ni wili
wordo gelóƀian. Þan sęggjo ik iu te wáron ók,
hwó it þar an þem aldon éo gebiudit:
só hwe só ógon genimid óðres mannes,
lósid af is lík-haman, etþa is liðo hwilikan,
þat he it eft mid is selƀes skal sán ant-gelden
mid gelíkun liðion. Þan willjo ik iu lérjan nu,
þat gi só ni wrekan wréða dádi,
ak þat gi þurh ód-módi al geþologian
wítjes endi wammes, só hwat só man iu an þesoro weroldi gedóe.
Dóe alloro erlo gehwilik óðrom manne
frume endi gefóri, só he willje, þat im firiho barn
gódes angegin dóen. Þan wirðit im god mildi,
liudjo só hwilikum, só þat léstien wili.
Érod gi arme man, déliad iwan ódwelon
undar þero þurftigon þiodu; ne rókjad, hweðar gi is énigan þank antfáhan
efþo lón an þesoro léhneon weroldi, ak huggjat te iuwomu leoƀon hérran
þero geƀono te gelde, þat sie iu god lóno,
mahtig mund-boro, só hwat só gi is þurh is minnes gidót.
Ef þu þan geƀogean wili gódun mannun
fagare fehoskattos, þar þu eft frumono hugis
mér antfáhan, te hwí haƀas þu þes éniga méda fon gode
etþa lón an þemu is liohte? hwand þat is léhni feho.
Só is þes alles gehwat, þe þu óðrun geduos
liudjon te leoƀe, þar þu hugis eft gelík neman
þero wordo endi þero werko: te hwí wét þi þes úsa waldand þank,
þes þu þín só bifilhis endi antfáhis eft þan þu wili?
iuwan óðwelon geƀan gi þem armun mannun,
þe ina iu an þesoro weroldi ne lónon endi rómot te iuwes waldandes ríkja.
Te hlúd ni dó þu it, þan þu mid þínun handun bifelhas
þína alamosna þemu armon manne, ak dó im þurh ód-módien
gerno þurh godes þank: þan móst þu eft geld niman,
swíðo lioflík lón, þar þu is lango biþarft,
fagaroro frumono. Só hwat só þu is só þurh ferhtan hugi
darno gedéleas, — so is úsumu drohtine werð —
ne galpo þu far þínun geƀun te swíðo, noh énig gumono ne skal,
þat siu im þurh ídale hróm eft ni werðe
léðlíko farloren. Þanna þu skalt lón nemen
fora godes ógun gódero werko.
Ôk skal ik iu gebeodan, þan gi willjad te bedu hnígan
endi willjad te iuwomu hérron helpono biddjan,
þat he iu aláte léðes þinges,
þero sakono endi þero sundjono, þea gi iu selƀon hír
wréða gewirkjad, þat gi it þan for óðrumu werode ni duad:
ni márjad it far menigi, þat iu þes man ni loƀon,
ni diurean þero dádjo, þat gi iuwes drohtines gibed
þurh þat ídala hróm al ne farleosan.
Ak þan gi willjan te iuwomo hérron helpono biddjan,
þiggean þeolíko, — þes iu is þarf mikil —
þat iu sigidrohtin sundjono tómea,
þan dót gi þat só darno: þoh wét it iuwe drohtin self
hélag an himile, hwand imu nis biholan neo-wiht
ne wordo ne werko. He látid it þan al gewerðan só,
só gi ina þan biddjad, þan gi te þero bedo hnígad
mid hluttru hugi.ʼ Heliðos stódun,
gumon umbi þana godes sunu gerno swíðo,
weros an willjon: was im þero wordo niud,
þáhtun endi þagodun, was im þarf mikil,
þat sie þat eft gehogdin, þat im þat hélaga barn
an þana forman sið filu mid wordun
torhtes getalde. Þó sprak im eft én þero tweliƀio angegin,
glawuoro gumono, te þem godes barne:
ʽHérro þe gódoʼ, kwað he, ʽús is þínoro huldi þarf,
te giwirkenne þínna willjon, endi ók þínoro wordo só self,
allaro barno bezt, þat þu ús bedon léres,
iungoron þíne, só Johannes duot,
diurlík dóperi, dago gehwilikas
is werod mid wordun, hwí sie waldand skulun,
gódan grótean. Dó þína iungorun só self:
gerihti ús þat gerúni.ʼ Þó habda eft þe ríkjo garu
sán aftar þiu, sunu drohtines,
gód word angegin: ʽÞan gi god willjanʼ, kwað he,
ʽweros mid iuwon wordun waldand grótean,
allaro kuningo kraftigostan, þan kweðad gi, só ik iu lériu:
Fadar úsa firiho barno,
þu bist an þem hóhon himila ríkja,
gewíhid sí þín namo wordo gehwiliko.
kuma þín kraftag ríki.
werða þín willjo oƀar þesa werold alla,
só sama an erðo, só þar uppa ist
an þem hóhon himilo ríkja.
Gef ús dago gehwilikes rád, drohtin þe gódo,
þína hélaga helpa, endi alát ús, heƀenes ward,
managoro ménskuldio, al só we óðrum mannum dóan.
Ne lát ús farlédean léða wihti
só forð an iro willjon, só wi wirðige sind,
ak help ús wiðar allun uƀilon dádjun.
Só skulun gi biddjan, þan gi te bede hnígad
weros mid iuwom wordun, þat iu waldand god
léðes aláte an leutkunnja.
Ef gi þan willjad alátan liudjo gehwilikun
þero sakono endi þero sundjono, þe sie wið iu selƀon hír
wréða gewirkjat, þan alátid iu waldand god,
fadar alamahtig firin-werk mikil,
managoro ménskuldeo. Ef iu þan wirðid iuwa mód te stark,
þat gi ne wileat óðrun erlun alátan,
weron wam-dádi, þan ne wil iu ók waldand god
grim-werk far-geƀan, ak gi skulun is geld niman,
swíðo léð-lik lón \hld\ te languru hwílu,
alles þes un-rehtes, \hld\ þes gi óðrum hír
gi-léstjad an þesumu liohte \hld\ endi þan wið liudjo barn
þea saka ni gisónjad, \hld\ ér gi an þana síð faran,
weros fon þesoro weroldi. \hld\ Ok skal ik iu te wárun sęggjan,
hwó gi léstjan skulun \hld\ léra mína:
þan gi iuwa fastonnja \hld\ frummjan willjan,
minson iuwa mén-dádi, þan ni duad gi þat te managom kúð,
ak míðad is far óðrum mannun: þoh wét mahtig god,
waldand iuwan willjan, þoh iu werod óðar,
liudjo barn ne loƀon. He gildid is iu lón aftar þiu,
iuwa hélag fadar an himil-ríkja,
þes ge im mid sulikum ód-módea, erlos þeonod,
só ferhtlíko undar þesumu folke. Ne willjat feho winnan
erlos an unreht, ak wirkjad up te gode
man aftar médu: þat is méra þing,
þan man hír an erðu ódag libbja,
weroldskattes gewono. Ef gi willjad mínun wordun hórjan,
þan ne samnod gi hír sink mikil siloƀres ne goldes
an þesoro middil-gard, méðomhordes,
hwand it rotat hír an roste, endi reginþeoƀos farstelad,
wurmi awardiad, wirðid þat giwádi farslitan,
tigangid þe goldwelo. Léstead iuwa gódon werk,
samnod iu an himile hord þat méra,
fagara fehoskattos: þat ni mag iu énig fíund beniman,
newiht anwendean, hwand þe welo standid
garu iu tegegnes, só hwat só gi gódes þarod,
an þat himil-ríki hordes gesamnod,
heliðos þurh iuwa handgeƀa, endi hębbjad þarod iuwan hugi fasto;
hwand þar ist alloro manno gihwes módgeþáhti,
hugi endi herta, þar is hord ligid,
sink gesamnod. Nis eo só sálig man,
þat mugi an þesoro brédon werold béðiu ant-hengean,
ge þat hi an þesoro erðo ódag libbja,
an allun weroldlustun wesa, ge þoh waldand gode
te þanke geþeono: ak he skal alloro þingo gihwes
simbla óðarhweðar én farlátan
etþo lusta þes lík-hamon etþo líf éwig.
Beþiu ni gornot gi umbi iuwa gegarwui, ak huggjad te gode fasto,
ne mornont an iuwomu móde, hwat gi eft an morgan skulin
etan efþo drinkan etþo an hębbjan
weros te gewédea: it wét al waldand god,
hwes þea biþurƀun, þea im hír þionod wel,
folgod iro fróhan willjon. Hwat, gi þat bi þesun fuglun mugun
wár-líko undarwitan, þea hír an þesoro weroldi sint,
farad an feðarhamun: sie ni kunnun énig feho winnan,
þoh giƀid im drohtin god dago gehwilikes
helpa wiðar hungre. Ôk mugun gi an iuwom hugi markon,
weros umbi iuwa gewádi, hwó þie wurti sint
fagoro gefratohot, þea hír an felde stád,
berhtlíko geblóid: ne mahta þe burges ward,
Salomon þe suning, þe habda sink mikil,
méðomhordas mést, þero þe énig man éhti,
welono ge-wunnan endi allaro gewádjo kust, —
þoh ni mohte he an is líƀe, þoh he habdi alles þeses landes gewald,
awinnan sulik gewádi, só þiu wurt haƀad,
þiu hír an felde stád fagoro gegariwit,
lilli mid só liof-líku blómon: ina wádit þe landes waldand
hér fan heƀenes wange. Mér is im þoh umbi þit heliðo kunni,
liudi sint im lioƀoron mikilu, þea he im an þesumu lande gewarhte,
waldand an willjon sínan. Beþiu ne þurƀon gi umbi iuwa gewádi sorgon,
ne gornot gi umbi iuwa gegariwi te swíðo: god wili is alles rádan,
helpan fan heƀenes wange, ef gi willjad aftar is huldi þeonon.
Gerot gi simbla érist þes godes ríkjas, endi þan duat aftar þem is gódun werkun,
rómod gi rehtoro þingo: þan wili iu þe ríkjo drohtin
geƀon mid alloro gódu gehwiliku, ef gi im þus fulgangan willjad,
só ik iu te wárun hír wordun seggjo.
Ne skulun gi énigumu manne unrehtes wiht,
derƀies adélean, hwand þe dóm eft kumid
oƀar þana selƀon man, þar it im te sorgon skal,
werðan þem te wítea, þe hír mid is wordun gesprikid
unreht óðrum. Neo þat iuwar énig ne dua
gumono an þesom gardon geldes etþo kópes,
þat hi unreht gimet óðrumu manne
ménful mako, hwand it simbla mótean skal
erlo gehwilikomu, sulik só he it óðrumu gedód,
só kumid it im eft tegegnes, þar he gerno ne wili
gesehan is sundjon. Ôk skal ik iu sęggjan noh,
hwar gi iu wardon skulun wíteo mésta,
ménwerk manag: te hwí skalt þu énigan man besprekan,
bróðar þínan, þat þu undar is bráhon gesehas
halm an is ógon, endi gehuggjan ni wili
þana swáran balkon, þe þu an þínoro siuni haƀas,
hard trio endi heƀig. Lát þi þat an þínan hugi fallan,
hwó þu þana érist a-lósjas: þan skínid þi lioht beforan,
ógun werðad þi geoponot; þan maht þu aftar þiu
swáses mannes gesiun síðor gebótean,
gehélean an is hóƀde. Só mag þat an is hugi méra
an þesoro middil-gard manno gehwilikumu,
wesan an þesoro weroldi, þat hi hír wammas geduot,
þan hi ahtogea óðres mannes
saka endi sundja, endi haƀad im selƀo mér
firin-werko gefrumid. Ef he wili is fruma léstean,
þan skal hi ina selƀon ér sundjono atómean,
léðwerko lóson: síðor mag hi mid is lérun werðan
heliðun te helpu, síðor hi ina hluttran wét,
sundjono sikoran. Ne skulun gi swínum teforan
iuwa meregríton makon etþo méðmo gestriuni,
hélag halsmeni, hwand siu it an horu spurnat,
sulwiad an sande: ne witun súƀreas geskéð,
fagaroro fratoho. Sulik sint hír folk manag,
þe iuwa hélag word hórjan ne willjad,
fulgangan godes lérun: ne witun gódes geskéð,
ak sind im lári word leoƀoron mikilu,
umbiþarƀi þing, þanna þeotgodes
werk endi willjo. Ne sind sie wirðige þan,
þat sie gehórjan iuwa hélag word, ef sie is ne willjad an iro hugi þęnkjan,
ne línon ne léstean. Þem ni sęggjan gi iuworo léron wiht,
þat gi þea spráka godes endi spel managu
ne farleosan an þem liudjun, þea þar ne willjan gilóƀjan tó,
wároro wordo. Ôk skulun gi iu wardon filu
listiun undar þesun liudjun, þar gi aftar þesumu lande farad,
þat iu þea luggjon ne mugin léron beswíkan
ni mid wordun ni mid werkun. Sie kumad an sulikom gewádjon te iu,
fagoron fratohon: þoh hębbjad sie féknan hugi:
þea mugun gi sán ant-kęnnjan, só gi sie kuman gesehad:
sie sprekad wíslík word, þoh iro werk ne dugin,
þero þegno geþáhti. Hwand gi witun, þat eo an þorniun ne skulun
wínberi wesan efþa welon eowiht,
fagororo fruhteo, nek ók fígun ne lesad
heliðos an hiopon. Þat mugun gi undarhuggjan wel,
þat eo þe uƀilo bóm, þar he an erðu stád,
góden wastum ne giƀid, nek it ók god ni geskóp,
þat þe gódo bóm gumono barnun
bári bittres wiht, ak kumid fan alloro bámo gehwilikumu
sulik wastom te þesero weroldi, só im fan is wurtjon gedregid,
etþa berht etþa bittar. Þat ménid þoh breosþugi,
managoro mód-seƀon manno kunnjes,
hwó alloro erlo gehwilik ógit selƀo,
meldod mid is múðu, hwilikan he mód haƀad,
hugi umbi is herte: þes ni mag he farhelan eowiht,
ak kumad fan þem uƀilan man Inwid-rádos,
bittara baluspráka, sulik só hi an is breostun haƀad
geheftid umbi is herte: simbla is hugi kúðid,
is willjon mid is wordun, endi farad is werk aftar þiu.
Só kumad fan þemu gódan manne \hld\ glau and-wordi,
wíslík fan is gewittea, \hld\ þat hi simbla mid is wordu gesprikid,
man mid is míðu sulik, \hld\ só he an is móde haƀad
hord umbi is herte. \hld\ Þanan kumad þea hélagan léra,
swíðo wun-sam word, \hld\ endi skulun is werk aftar þiu
þeodu geþíhan, \hld\ þegnun managun
werðan te willjon, \hld\ al só it waldand self
gódun mannun far-giƀid, \hld\ god alo-mahtig,
himilisk hérro, \hld\ hwand sie áno is helpa ni mugun
ne mid wordun ne mid werkun \hld\ wiht aþengean
gódes an þesun gardun. \hld\ Beþiu skulun gumono barn
an is énes kraft \hld\ alle gi-lóƀjan.
Ôk skal ik iu wísean, \hld\ hwó hír wegos twéna
liggjad an þesumu liohte, \hld\ þea farad liudjo barn,
al irmin-þiod. \hld\ Þero is óðar sán
wíd stráta endi bréd, \hld\ —farid sie werodes filu,
man-kunnjes manag, hwand sie þarod iro mód spenit,
weroldlusta weros— \hld\ þiu an þea wirson hand
liudi lédid, \hld\ þar sie te farlora werðad,
heliðos an hęllju, \hld\ þar is hét endi swart,
egis-lík an innan: \hld\ óði ist þarod te faranne
eldi-barnun, \hld\ þoh it im at þemu endie ni dugi.
Þan ligid eft óðar \hld\ engira mikilu
weg an þesoro weroldi, \hld\ ferid ina werodes lút,
fáho folk-skępi: \hld\ ni willjad ina firiho barn
gerno gangan, \hld\ þoh he te godes ríkja,
an þat éwiga líf, \hld\ erlos lédja.
Þan nimad gi iu þana engean: \hld\ þoh he só óði ne sí
firihon te faranne, þoh skal hi te frumu werðan
só hwemu só ina þurh-gengid, \hld\ só skal is geld niman,
swíðo lang-sam lón \hld\ endi líf éwig,
diurlíkan dróm. \hld\ Eo gi þes drohtin skulun,
waldand biddjen, \hld\ þat gi þana weg mótin
fan foran ant-fáhan \hld\ endi forð þurh gigangan
an þat godes ríki. \hld\ He ist garu simbla
wiðar þiu te geƀanne, \hld\ þe man ina gerno bidid,
fergot firiho barn. \hld\ Sókjad fadar iuwan
up te þemu éwinom ríkja: \hld\ þan mótun gi ina aftar þiu
te iuworu frumu fíðan. \hld\ Kúðead iuwa fard þarod
at iuwas drohtines durun: \hld\ þan werðad iu andón aftar þiu,
himil-portun ant-hlidan, \hld\ þat gi an þat hélage lioht,
an þat godes ríki \hld\ gangan mótun,
sin-líf sehan. \hld\ Ôk skal ik iu sęggjan noh
far þesumu werode allun \hld\ wár-lík biliði,
þat alloro liudjo só hwilik, \hld\ só þesa mína léra wili
gehaldan an is herton \hld\ endi wil iro an is hugi aþęnkjan,
léstean sea an þesumu lande, \hld\ þe gi-líko duot
wísumu manne, \hld\ þe giwit haƀad,
horska hugi-skęfti, \hld\ endi hús-stędi kiusid
an fastoro foldun \hld\ endi an felisa uppan
wégos wirkid, \hld\ þar im wind ni mag,
ne wág ne watares stróm \hld\ wihtiu ge-tiunean,
ak mag im þar wið un-gi-widereon \hld\ allun standan
an þemu felise uppan, \hld\ hwand it só fasto warð
gistellit an þemu sténe: \hld\ anthaƀad it þiu stedi niðana,
wreðid wiðar winde, þat it wíkan ni mag.
Só duot eft manno só hwilik, só þesun mínun ni wili
lérun hórjen ne þero léstien wiht,
só duot þe unwíson erla gelíko,
ungewittigon were, þe im be watares staðe
an sande wili selihús wirkjan,
þar it westrani wind endi wágo stróm,
sées úðjon tesláad; ne mag im sand endi greot
gewreðien wið þemu winde, ak wirðid teworpan þan,
tefallen an þemu flóde, hwand it an fastoro nis
erðu getimbrod. Só skal allaro erlo gehwes
werk geþíhan wiðar þiu, þe hi þius mín word frumid,
haldid hélag gebod.ʼ Þó bigunnun an iro hugi wundron
meginfolk mikil: gehórdun mahtiges godes
lioflíka léra; ne wárun an þemu lande ge-wuno,
þat sie eo fan sulikun ér sęggjan gehórdin
wordun etþo werkun. Farstódun wíse man,
þat he só lérde, liudjo drohtin,
wárun wordun, só he gewald habde,
allun þem ungelíko, þe þar an ér-dagun
undar þem liud-skępja \hld\ lérjon wárun
akoran undar þemu kunnje: \hld\ ne habdun þiu Kristes word
gemakon mid mannun, \hld\ þe he far þero menigi sprak,
ge-bód uppan þemu berge. \hld\ He im þó béðiu be-falh
ge te seggennea \hld\ sínom wordun,
hwó man himil-ríki \hld\ ge-halon skoldi,
wíd-brédan welan, \hld\ gia he im gewald far-gaf,
þat sie móstin hélean \hld\ halte endi blinde,
liudjo léf-hédi, \hld\ legar-będ manag,
swára suhti, \hld\ giak he im selƀo gebód,
þat sie at énigumu manne méde ne námin,
diurie méðmos: ʽgehuggjad giʼ, kwað he, — ʽhwand iu is þiu dád kuman,
þat gewit endi þe wísdóm, endi iu þea gewald far-giƀid
alloro firiho fadar, só gi sie ni þurƀun mid énigo feho kópon,
médean mid énigun méðmun, — só wesat gi iro mannun forð
an iuwon hugi-skęftjun helpono mildja,
léread gi liudjo barn lang-samna rád,
fruma forð-wardes; firin-werk lahad,
swára sundjon. Ne látad iu siloƀar nek gold
wihti þes wirðig, þat it eo an iuwa gewald kuma,
fagara fehoskattos: it ni mag iu te énigoro frumu hwergin,
werðan te énigumu willjon. Ne skulun gi gewádeas þan mér
erlos égan, bútan só gi þan an hębbjan,
gumon te garewea, þan gi gangan skulun
an þat gimang innan. Neo gi umbi iuwan meti ni sorgot,
leng umbi iuwa lífnare, hwand þene léreand skulun
fódean þat folk-skępi: þes sint þea fruma werða,
leoƀlíkes lónes, þe hi þem liudjun sagad.
wirðig is þe wurhtjo, þat man ina wel fódea,
þana man mid mósu, þe só managoro skal
seola bisorgan endi an þana síð spanen,
géstos an godes wang. Þat is grótara þing,
þat man bisorgon skal seolun managa,
hwó man þea gehalde te heƀen-ríkja,
þan man þene lík-hamon liudibarno
mósu bimorna. Beþiu man skulun
haldan þene holdlíko, þe im te heƀen-ríkja
þene weg wísit endi sie wam-skaðun,
feondun witfáhit endi firin-werk lahid,
swára sundjon. Nu ik iu sęndjan skal
aftar þesumu land-skępje só lamb undar wlƀos:
só skulun gi undar iuwa fíund faren, undar filu þeodo,
undar mislíke man. Hębbjad iuwan mód wiðar þem
só glawan tegegnes, só samo só þe gelwo wurm,
nádra þiu féha, þar siu iro níð-skępjes,
witodes wánit, þat man iu undar þemu werode ne mugi
beswíkan an þemu síðe. Far þiu gi sorgon skulun,
þat iu þea man ni mugin módgeþáhti,
willjan awardien. Wesat iu so wara wiðar þiu,
wið iro fékneon dádjun, só man wiðar fíundun skal.
Þan wesat gi eft an iuwon dádjun dúƀon gelíka,
hębbjad wið erlo gehwene énfaldan hugi,
mildjan mód-seƀon, þat þar man negén
þurh iuwa dádi bedrogan ne werðe,
beswikan þurh iuwa sundja. Nu skulun gi an þana síð faran,
an þat árundi: þar skulun gi arƀidies só filu
geþolon undar þeru þiod endi geþwing só samo
manag endi mislík, hwand gi an mínumu namon
þea liudi léreat. Beþiu skulun gi þar léðes filu
fora werold-kuningun, wíteas antfáhan.
Oft skulun gi þar for ríkja þurh þius mín rehtun word
gebundane standen endi béðiu geþologean,
ge hosk ge harmkwidi: umbi þat ne látad gi iuwan hugi twíflon,
seƀon swíkandean: gi ni þurƀun an énigun sorgun wesan
an iuwomu hugi hwergin, þan man iu for þea héri forð
an þene gast-sęli gangan hétid,
hwat gi im þan tegegnes skulin gódoro wordo,
spáhlíkoro gesprekan, hwand iu þiu spód kumid,
helpe fon himile, endi sprikid þe hélogo gést,
mahtig fon iuwomu munde. Beþiu ne andrádad gi iu þero manno níð
ne forhteat iro fíundskepi: þoh sie hębbjan iuwas ferahes gewald,
þat sie mugin þene lík-hamon líƀu beneotan,
aslahan mid swerde, þoh sie þeru seolun ne mugun
wiht awardean. Antdrádad iu waldand god,
forhtead fader iuwan, frummjad gerno
is gebod-skępi, hwand hi haƀad béðies giwald,
liudjo líƀes endi ók iro lík-hamon
gek þero seolon só self: ef gi iuwa an þem síðe þarod
farliosat þurh þesa léra, þan mótun gi sie eft an þemu liohte godes
beforan fíðan, hwand sie fader iuwa,
haldid hélag god an himil-ríkja.
Ne kumat þea alle te himile, þea þe hír hrópat te mi
manno te mundburd. Managa sind þero,
þea willjad alloro dago gehwilikes te drohtine hnígan,
hrópad þar te helpu endi huggjad an óðar,
wirkjad wamdádi: ne sind im þan þiu word fruma,
ak þea mótun hwerƀan an þat himiles lioht,
gangan an þat godes ríki, þea þes gerne sint,
þat sie hír gefrummjen fader alawaldan
werk endi willjon. Þea ni þurƀun mid wordun só fílu
hrópan te helpu, hwanda þe hélogo god
wét alloro manno gehwes módgeþáhti,
word endi willjon, endi gildid im is werko lón.
Beþiu skulun gi sorgon, þan gi an þene síð farad,
hwó gi þat árundi ti endea bebrengen.
Þan gi líðan skulun aftar þesumu land-skępea,
wído aftar þesoro weroldi, al só iu wegos lédiad,
bréd stráta te burg, simbla sókiad gi iu þene bezton sán
man undar þeru menegi endi kúðead imu iuwan móðseƀon
wárun wordun. Ef sie þan þes wirðige sint,
þat sie iuwa gódun werk gerno geléstien
mid hluttru hugi, þan gi an þemu húse mid im
wonod an willjon endi im wel lónod,
geldad im mid gódu endi sie te gode selƀon
wordun gewíhad endi sęggjad im wissan friðu,
hélaga helpa heƀen-kuninges.
Ef sie þan só sáliga þurh iro selƀoro dád
werðan ni mótun, þat sie iuwa werk frummjen,
léstien iuwa léra, þan gi fan þem liudjun sán,
farad fan þemu folke, — þe iuwa friðu hwirƀid
eft an iuworo selƀoro síð, — endi látad sie mid sundjun forð,
mid balu-werkun búan endi sókiad iu burg óðra,
mikil manwerod, endi ne látad þes melmes wiht
folgan an iuwom fótun, þanan þe man iu antfáhan ne wili,
ak skuddiat it fan iuwon skóhun, þat it im eft te skamu werðe,
þemu werode te ge-wit-skępje, \hld\ þat iro willjo ne dóg.
Þan sęggjo ik iu te wárun, só hwan só þius werold endiad
endi þe márjo dag oƀar man farid,
þat þan Sodomo-burg, þiu hír þurh sundjon warð
an af-grundi éldes kraftu,
fiuru bifallen, þat þiu þan haƀad friðu méran,
mildiran mund-burd, þan þea man égin,
þe iu hír wiðar-werpat endi ne willjad iuwa word frummjen.
Só hwe só iu þan ant-fáhit þurh ferhtan hugi,
þurh mildjan mód, só haƀad mínan forð
willjon ge-warhten endi ók waldand god,
ant-fangan fader iuwan, firiho drohtin,
ríkjan rád-geƀon, þene þe al reht bikan.
wét waldand self, endi willjan lónot
gumono ge-hwilikumu, só hwat só hi hír gódes geduot,
þoh hi þurh minnja godes manno hwilikumu
willjandi far-geƀe watares drinkan,
þat hi þurftigumu manne þurst ge-hélje,
kaldes brunnan. Þesa kwidi werðad wára,
þat eo ne bi-líƀid, ne hi þes lón skuli,
fora godes ógun geld ant-fáhan,
méda manag-falde, só hwat só hi is þurh mína minnja geduot.
Só hwe só mín þan far-lógnid liudi-barno,
heliðo for þesoro hęrju, só dóm ik is an himile só self
þar uppe far þem alo-waldan fader endi for allumu is ęngilo krafte,
far þeru mikilon menigi. Só hwilik só þan eft manno barno
an þesoro weroldi ne wili wordun míðan,
ak gihit far gum-skępi, þat he mín iungoro sí,
þene willju ek eft ógean far ógun godes,
fora alloro firiho fader, þar folk manag
for þene alo-waldon alla gangad
reðinon wið þene ríkjon. Þar willju ik imu an reht wesan
mildi mund-boro, só hwemu só mínun hír
wordun hórid endi þiu werk frumid,
þea ik hír an þesumu berge uppan geboden hębbju.ʼ
Habda þó te wárun waldandes sunu
gelérid þea liudi, hwó sie lof gode
wirkjan skoldin. Þó lét hi þat werod þanan
an alloro halƀa gehwilika, hęri-skępi manno
síðon te selðon. Habdun selƀes word,
gehórid heƀen-kuninges hélaga léra,
só eo te weroldi sint wordo endi dádjo,
man-kunnjes manag oƀar þesan middil-gard
sprákono þiu spáhiron, só hwe só þiu spel gefrang,
þea þar an þemu berge gesprak barno ríkjast.
Gewét imu þó umbi þrea naht aftar þiu þesoro þiodo drohtin
an Galileo land, þar he te énum gómum warð,
gebedan þat barn godes: þar skolda man éna brúd geƀan,
munalíka magað. Þar Maria was,
mid iro suni selƀo, sálig þiorna,
mahtiges móder. Managoro drohtin
geng imu þó mid is iungoron, godes égan barn,
an þat hóha hús, þar þe heri drank,
þea Judeon an þemu gast-sęli: he im ók at þem gómun was,
giak hi þar gekúðde, þat hi habda kraft godes,
helpa fan himilfader, hélagna gést,
waldandes wísdóm. Werod blíðode,
wárun þar an luston liudi atsamne,
gumon gladmódie. Gengun ambahtman,
skenkeon mid skálun, drógun skíriane wín
mid orkun endi mid alofatun; was þar erlo dróm
fagar an flettea, þó þar folk undar im
an þem bęnkjon só bezt blíðsea af-hóƀun,
wárun þar an wunnjun. Þó im þes wínes brast,
þem liudjun þes líðes: is ni was farléƀid wiht
hwergin an þemu húse, þat for þene heri forð
skenkeon drógin, ak þiu skapu wárun
líðes alárid. Þó ni was lang te þiu,
þat it sán antfunda frío skóniosta,
Kristes móder: geng wið iro kind sprekan,
wið iro sunu selƀon, sagda im mid wordun,
þat þea werdos þó mér wínes ne habdun
þem gestiun te gómun. Siu þó gerno bad,
þat is þe hélogo Krist helpa geriedi
þemu werode te willjon. Þó habda eft is word garu
mahtig barn godes endi wið is móder sprak:
ʽhwat ist mi endi þiʼ, kwað he, ʽumbi þesoro manno lið,
umbi þeses werodes wín? Te hwí sprikis þu þes, wíf, só filu,
manos mi far þesoro menigi? Ne sint mína noh
tídi kumana.ʼ Þan þoh gitrúoda siu wel
an iro hugi-skęftjun, hélag þiorne,
þat is aftar þem wordun waldandes barn,
héljandoro bezt helpan weldi.
Hét þó þea ambahtman idiso skóniost,
skenkeon endi skapwardos, þea þar skoldun þero skolu þionon,
þat sie þes ne word ne werk wiht ne farlétin,
þes sie þe hélogo Krist hétan weldi
léstean far þem liudjun. Lárea stódun þar
sténfatu sehsi. Þó só stillo gebód
mahtig barn godes, só it þar manno filu
ne wissa te wárun, hwó he it mid is wordu gesprak;
he hét þea skenkeon þó skíreas watares
þiu fatu fullien, endi hi þar mid is fingrun þó,
segnade selƀo sínun handun,
warhte it te wíne endi hét is an én wégi hlaðen,
skęppjen mid énoro skálon, endi þó te þem skenkeon sprak,
hét is þero gestjo, þe at þem gómun was
þemu héroston an hand geƀan,
ful mid folmun, þemu þe þes folkes þar
ge-weld aftar þemu werde. Reht só hi þes wínes ge-drank,
só ni mahte he bemíðan, ne hi far þeru menigi sprak
te þemu brúdi-gumon, kwað þat simbla þat bezte líð
alloro erlo ge-hwilik érist skoldi
geƀan at is gómun: ʽundar þiu wirðid þero gumono hugi
a-wękid mid wínu, þat sie wel blíðod,
drunkan drómjad. Þan mag man þar dragan aftar þiu
líhtlíkora líð: só ist þesoro liudjo þau.
Þan haƀas þu nu wunder-líko werd-skępi þínan
ge-markod far þesoro menigi: hétis far þit manno folk
alles þínes wínes þat wirsiste
þíne ambahtman érist brengjan,
geƀan at þínun gómun. Nu sint þína gesti sade,
sint þíne druhtingos drunkane swíðo,
is þit folk frómód: nu hétis þu hír forð dragan
alloro líðo lof-samost, þero þe ik eo an þesumu liohte gesah
hwergin hębbjan. Mid þius skoldis þu ús hindag ér
geƀon endi gómean: þan it alloro gumono gehwilik
geþigedi te þanke.ʼ Þó warð þar þegan manag
gewar aftar þem wordun, síðor sie þes wínes gedrunkun,
þat þar þe hélogo Krist an þemu húse innan
tékan warhte: trúodun sie síðor
þiu mér an is mundburd, þat hi habdi maht godes,
gewald an þesoro weroldi. Þó warð þat só wído kúð
oƀar Galileo land Judeo liudjun,
hwó þar selƀo gededa sunu drohtines
water te wíne: þat warð þar wundro érist,
þero þe hi þar an Galilea Judeo liudjon,
tékno getógdi. Ne mag þat ge-tęlljan man,
gesęggjan te sóðan, hwat þar síðor warð
wundres undar þemu werode, þar waldand Krist
an godes namon Judeo liudjon
allan langan dag léra sagde,
gihét im heƀen-ríki \hld\ endi hęlljo geþwing
weride mid wordun, \hld\ hét sie wara godes,
sinlíf sókjan: \hld\ þar is seolono lioht,
dróm drohtines \hld\ endi dag-skímon,
gód-líknissea godes; \hld\ þar gést manag
wunod an willjan, \hld\ þe hír wel þenkid,
þat he hír bihalde \hld\ heƀen-kuninges gebod.
Gewét imu þó mid is iungoron \hld\ fan þem gómun forð
Kristus te kapharnaum, \hld\ kuningo ríkjost,
te þeru márjon burg. \hld\ Megin samnode,
gumon imu tegegnes, \hld\ gódoro manno
sálig ge-síði: weldun þiu is swótjan word
hélag hórjen. Þar im én hunno kwam,
én gód man angegin endi ina gerno bad
helpan hélagne, kwað þat hi undar is híwiskja
énna lefna lamon lango habdi,
seokan an is selðon: ʽsó ina énig seggjo ne mag
handun gehélien. Nu is im þínoro helpono þarf,
fró mín þe gódo.ʼ Þó sprak im eft þat friðu-barn godes
sán aftar þiu selƀo tegegnes,
kwað þat he þar kwámi endi þat kind weldi
nerean af þeru nódi. Þó im náhor geng
þe man far þeru menigi wið só mahtigna
wordun wehslan: ʽik þes wirðig ne bium,ʼ kwað he,
ʽhérro þe gódo, þat þu an mín hús kumes,
sókjas mína seliða, hwand ik bium só sundig man
mid wordun endi mid werkun. Ik gelóƀju þat þu gewald haƀas,
þat þu ina hinana maht hélan gewirkjan,
waldand fró mín: ef þu it mid þínun wordun gesprikis,
þan is sán þiu léf-héd lósot endi wirðid is lík-hamo
hél endi hréni, ef þu im þína helpa far-giƀis.
Ik bium mi ambahtman, hębbju mi ódes genóg,
welono ge-wunnen: þoh ik undar geweldi sí
aðalkuninges, þoh hębbju ik erlo getróst,
holde heririnkos, þea mi só gehóriga sint,
þat sie þes ne word ne werk wiht ne farlátad,
þes ik sie an þesumu land-skępje léstean héte,
ak sie farad endi frummjad endi eft te iro fróhan kumad,
holde te iro hérron. Þoh ik at mínumu hús égi
wídbrédene welon endi werodes genóg,
heliðos hugiderƀie, þoh ni gidar ik þi só hélagna
biddjen, barn godes, þat þu an mín bú gangas,
sókjas mína seliða, hwand ik só sundig bium,
wét mína far-wurhti.ʼ Þó sprak eft waldand Krist,
þe gumo wið is iungoron, kwað þat hi an Judeon hwergin
undar Israheles aƀoron ne fundi
gemakon þes mannes, þe io mér te gode
an þemu land-skępi gelóƀon habdi,
þan hluttron te himile: ʽnu látu ik iu þar hórjen tó,
þar ik it iu te wárun hír wordun seggjo,
þat noh skulun eli-þeoda \hld\ óstane endi uestane,
man-kunnjes kuman \hld\ manag tesamne,
hélag folk godes \hld\ an heƀen-ríki:
þea motun þar an Abrahames \hld\ endi an Isaakes só self
endi ók an Jakobes, \hld\ gódoro manno,
barmun restjen \hld\ endi béðiu geþologean,
welon endi willjon \hld\ endi wonod-sam líf,
gód lioht mid gode. \hld\ Þan skal Judeono filu,
þeses ríkjas suni \hld\ beróƀode werðen,
bedélide sulikoro diurðo, \hld\ endi skulun an dalun þiustron
an þemu alloro ferristan \hld\ ferne liggen.
Þar mag man gehórjen \hld\ heliðos kwíðjan,
þar sie iro torn manag \hld\ tandon bítad;
þar ist gristgrimmo \hld\ endi grádag fiur,
hard hęlljo ge-þwing, \hld\ hét endi þiustri,
swart sinnahti \hld\ sundja te lóne,
wréðoro ge-wurhtjo, \hld\ só hwemu só þes willjon ne haƀad,
þat he ina a-lósje, \hld\ ér hi þit lioht ageƀe,
wendie fan þesoro weroldi. \hld\ Nu maht þu þi an þínan willjon forð
síðon te selðun; \hld\ þan findis þu ge-sundan at hús
mago-jungan man: \hld\ mód is imu an luston,
þat barn is ge-hélid, \hld\ só þu bédi te mi:
it wirðid al só ge-léstid, \hld\ só þu gelóƀon haƀas
an þínumu hugi hardo.ʼ \hld\ Þó sagde heƀen-kuninge,
þe ambahtman \hld\ alo-waldon gode
þank for þero þiodo, \hld\ þes he imu at sulikun þarƀun halp.
Habda þo gi-árundid, \hld\ al só he welde,
sálig-líko: \hld\ gi-wét imu an þana síð þanan,
wende an is willjan, \hld\ þar he welon éhte,
bú endi bód-los: \hld\ fand þat barn ge-sund,
kind-jungan man. \hld\ Kristes wárun þó
word ge-fullot: \hld\ hi gewald habda
te tógeanna tékan, \hld\ só þat ni mag gi-tęlljen man,
ge-ahton oƀar þesoro erðu, \hld\ hwat he þurh is énes kraft
an þesaro middil-gard \hld\ máriða ge-frumide,
wundres ge-warhte, \hld\ hwand al an is ge-weldi stád,
himil endi erðe. \hld\ Þó gewét imu þe hélogo Krist
forð-wardes faren, \hld\ fremide alo-mahtig
alloro dago ge-hwilikes, \hld\ drohtin þe gódo,
liudjo barnum leof, \hld\ lérde mid wordun
godes willjon gumun, \hld\ habda imu jungorono filu
simbla te gi-síðun, \hld\ sálig folk godes,
manno megin-kraft, \hld\ managoro þeodo,
hélag hęri-skępi, \hld\ was is helpono gód,
mannun mildi. \hld\ Þó hi mid þeru menigi kwam,
mid þiu brahtmu þat barn godes \hld\ te burg þeru hóhon,
þe nęrjendo te Naim: \hld\ þar skolde is namo werðen
mannun gemárid. \hld\ Þó geng mahtig tó
nęrjendo Krist, \hld\ antat he gináhid was,
héljandero bezt: \hld\ þó sáhun sie þar én hréo dragan,
énan líflósan lík-hamon \hld\ þea liudi fórjen,
beran an énaru báru \hld\ út at þera burges dore,
maguiungan man. \hld\ Þiu móder aftar geng
an iro hugi hriwig \hld\ endi handun slóg,
karode endi kúmde \hld\ iro kindes dóð,
idis armskapan; \hld\ it was ira énag barn:
siu was iru widowa, \hld\ ne habda wunnja þan mér,
biúten te þemu énagun \hld\ sunje al geláten
wunnja endi willjan, \hld\ anttat ina iru wurd benam,
mári metodogeskapu. \hld\ Megin folgode,
burgliudjo gebrak, \hld\ þar man ina an báru dróg,
iungan man te graƀe. \hld\ Þar warð imu þe godes sunu,
mahtig mildi \hld\ endi te þeru móder sprak,
hét þat þiu widowa \hld\ wóp farléti,
kara aftar þemu kinde: \hld\ ʽþu skalt hír kraft sehan,
waldandes giwerk: \hld\ þi skal hír willjo gestanden,
frófra far þesumu folke: \hld\ ne þarft þu ferah karon
barnes þínes.ʼ \hld\ *Þuo hie ti þero báron geng
iak hie ina selƀo ant-hrén, \hld\ suno drohtines,
hélagon handon, \hld\ endi ti þem heliðe sprak,
hiet ina só alaiungan \hld\ up astandan,
arísan fan þeru restun. \hld\ Þie rink up asat,
þat barn an þero bárun: \hld\ warð im eft an is briost kuman
þie gést þuru godes kraft, \hld\ endi hie tegegnes sprak,
þe man wið is mágos. \hld\ Þuo ina eft þero muoder bifalah
hélandi Krist an hand: \hld\ hugi warð iro te froƀra,
þes wíƀes an wunnjon, \hld\ hwand iro þar sulik willjo gistuod.
Fell siu þó te fuotun Kristes \hld\ endi þena folko drohtin
loƀoda for þero liudjo menigi, \hld\ hwand hie iro at só liobes ferahe
mundoda wiðer metodi-gi-skęftje: \hld\ far-stuod siu þat hie was þie mahtigo drohtin,
þie hélago, þie himiles giwaldid, \hld\ endi þat hie mahti gihelpan managon,
allon irmin-þiedon. \hld\ Þuo bigunnun þat ahton managa,
þat wunder, þat under þem weroda giburida, \hld\ kwáðun þat waldand selƀo,
mahtig kwámi þarod is menigi wíson, \hld\ endi þat hie im só márjan sandi
wár-sagon an þero weroldes ríki, \hld\ þie im þar sulikan willjon frumidi.
warð þar þuo erl manag \hld\ egison bifangan,
þat folk warð an forohton: \hld\ gisáhun þena is ferah égan,
dages lioht sehan, \hld\ þena þe ér dóð fornam,
an suht-będdjon swalt: \hld\ þuo was im eft gisund after þiu,
kindiung akwikot. \hld\ Þuo warð þat kúð obar all
aƀaron Israheles. \hld\ Reht só þuo áƀand kwam,
só warð þar all gisamnod \hld\ seokora manno,
haltaro endi háƀaro, \hld\ só hwat só þar hwergin was,
þia léƀun under þem liudjon, \hld\ endi wurðun þar gilédit tuo,
kumana te Kriste, \hld\ þar hie im þuru is kraft mikil
halp endi sie hélda, \hld\ endi liet sia eft gihaldana þanan
wendan an iro willjon. \hld\ Beþiu skal man is werk loƀon,
diuran is dádi, \hld\ hwand hie is drohtin self,
mahtig mund-boro \hld\ manno kunnje,
liudjo só hwilikon, \hld\ só þar gilóbit tuo
an is word endi an is werk. \hld\ Þuo was þar werodes só filo
allaro eli-þiodo \hld\ kuman te þem éron Kristes,
te só mahtiges mund-burd. \hld\ Þuo welda hie þar éna meri líðan,
þie godes suno mid is jungron \hld\ aneƀan Galilea-land,
waldand énna wágo stróm. \hld\ Þuo hiet hie þat werod óðar
forð-werdes faran, \hld\ endi hie giwét im fahora sum
an énna nakon innan, \hld\ nęrjendi Krist,
slápan síð-wórig. \hld\ Segel up dádun
weder-wísa weros, \hld\ lietun wind after
manon oƀar þena meri-stróm, \hld\ unþat hie te middjan kwam,
waldand mid is werodu. \hld\ Þuo bigan þes wedares kraft,
úst up stígan, \hld\ úðjun wahsan;
swang gi-swerk an gi-mang: \hld\ þie séu warð an hruoru,
wan wind endi water; \hld\ weros sorogodun,
þiu meri warð só muodag, \hld\ ni wánda þero manno nigén
lengron líƀes. \hld\ Þuo sia landes ward
wekidun mid iro wordon \hld\ endi sagdun im þes wedares kraft,
bádun þat im gináðig \hld\ nęrjendi Krist
wurði wið þem watare: \hld\ ʽefþa wi skulun hier te wunder-kwálu
sweltan an þeson séwe.ʼ \hld\ Self up a-rés
þie guodo godes suno \hld\ endi te is jungron sprak,
hiet þat sia im wedares giwin \hld\ wiht ni and-rédin:
ʽte hwí sind gi só forhta?ʼ \hld\ kwaþie. ʽNis iu noh fast hugi,
gi-lóƀo is iu te luttil. \hld\ Nis nu lang te þiu,
þat þia strómos skulun \hld\ stilrun werðan
gi þit *wedar wun-sam.ʼ \hld\ Þo hi te þem winde sprak
ge te þemu séwa só self \hld\ endi sie smultro hét
béðea gebárean. \hld\ Sie gi-bod léstun,
waldandes word: \hld\ weder stillodun,
fagar warð an flóde. \hld\ Þó bigan þat folk undar im,
werod wundraian, \hld\ endi suma mid iro wordun sprákun,
hwilik þat só mahtigoro \hld\ manno wári,
þat imu só þe wind endi þe wág \hld\ wordu hórdin,
béðea is gi-bod-skępjes. \hld\ Þó habda sie þat barn godes
ginerid fan þeru nódi: \hld\ þe nako furðor skreid,
hóh hurnid-skip; \hld\ heliðos kwámun,
liudi te lande, \hld\ sagdun lof gode,
máridun is megin-kraft. \hld\ Kwam þar manno filu
angegin þemu godes sunje; \hld\ he sie gerno ant-feng,
só hwene só þar mid hluttru hugi \hld\ helpa sóhte;
lérde sie iro gilóƀon \hld\ endi iro lík-hamon
handun hélde: \hld\ nio þe man só hardo ni was
gisérit mid suhtiun: \hld\ þoh ina Satanases
féknea iungoron \hld\ fíundes kraftu
habdin undar handun \hld\ endi is hugi-skęfti,
giwit awardid, þ\hld\ at he wódjendi
fóri undar þemu folke, \hld\ þoh im simbla ferh far-gaf
hélandeo Krist, ef he te is handun kwam,
dréf þea diuƀlas þanan drohtines kraftu,
wárun wordun, endi im is gewit far-gaf,
lét ina þan hélan wiðer hetteandun,
gaf im wið þie fíund friðu, endi im forð giwét
an só hwilik þero lando, só im þan leoƀost was.
Só deda þe drohtines sunu dago gehwilikes
gód werk mid is iungeron, só neo Judeon umbi þat
an þea is mikilun kraft þiu mér ne gelóƀdun,
þat he alo-waldo alles wári,
landes endi liudjo: þes sie noh lón nimat,
wídana wrak-síð, þes sie þar þat gewin driƀun
wið selƀan þene sunu drohtines. Þó he im mid is ge-síðon giwét
eft an Galilaeo land, godes égan barn,
fór im te þem friundun, þar he afódid was
endi al undar is kunnje kindiung awóhs,
þe hélago héljand. Umbi ina hęri-skępi,
þeoda þrungun; þar was þegan manag
só sálig undar þem ge-síðe. Þar drógun énna seokan man
erlos an iro armun: weldun ina for ógun Kristes,
brengean for þat barn godes — was im bótono þarf,
þat ina gehéldi heƀenes waldand,
manno mund-boro —, þe was ér só managan dag
liðu-wastmon bilamod, ni mahte is lík-hamon
wiht gewaldan. Þan was þar werodes só filu,
þat sie ina fora þat barn godes brengean ni mahtun,
geþringan þurh þea þioda, þat sie só þurftiges
sunnea gesagdin. Þó giwét imu an énna seli innan
héljando Krist; hwarf warð þar umbi,
megin-þeodo gemang. Þó bigunnun þea man spreken,
þe þene léfna lamon lango fórdun,
bárun mid is beddiu, hwó sie ina gedrógin fora þat barn godes,
an þat werod innan, þar ina waldand Krist
selƀo gi-sáwi. Þó gengun þea ge-síðos tó,
hóƀun ina mid iro handun endi uppan þat hús stigun,
slitun þene seli oƀana endi ina mid sélun létun
an þene rakud innan, þar þe ríkjo was,
kuningo kraftigost. Reht só he ina þó kuman gisah
þurh þes húses hróst, só he þó an iro hugi farstód,
an þero manno mód-seƀon, þat sie mikilana te imu
gelóƀon habdun, þó he for þen liudjun sprak,
kwað þat he þene siakon man sundjono tómean
látan weldi. Þó sprákun im eft þea liudi angegin,
gramharde Judeon, þea þes godes barnes
word aftarwarodun, kwáðun þat þat ni mahti giwerðen só,
grimwerk fargeƀen, biútan god éno,
waldand þesaro weroldes. Þó habda eft is word garu
mahtig barn godes: ʽik gidón þatʼ, kwað he, ʽan þesumu manne skín,
þe hír só siak ligid an þesumu seli innan,
te wundron giwégid, þat ik gewald hębbju
sundja te fargeƀanne endi ók seokan man
te gehéleanne, só ik ina hrínan ni þarf.ʼ
Manoda ina þó þe márjo drohtin,
liggjandean lamon, hét ina far þem liudjun astandan
up alohélan endi hét ina an is ahslun niman,
is bedgiwádi te baka; he þat gi-bod léste
sniumo for þemu gisíðja endi geng imu eft gesund þanan,
hél fan þemu húse. Þó þes só manag héðin man,
weros wundradun, kwáðun þat imu waldand self,
god alo-mahtig fargeƀan habdi
méron mahti þan elkor énigumu mannes sunje,
kraft endi kústi; sie ni weldun ant-kęnnjan þoh,
Judeo liudi, þat he god wári,
ne gelóƀdun is léran, ak habdun im léðan stríd,
wunnun wiðar is wordun: þes sie werk hlutun,
léðlík lóngeld, endi só noh lango skulun,
þes sie ni weldun hórjen heƀen-kuninges,
Kristes lérun, þea he kúðde oƀar al,
wído aftar þesaro weroldi, endi lét sie is werk sehan
allaro dago gehwilikes, is dádi skawon,
hórjen is hélag word, þe he te helpu gesprak
manno barnun, endi só manag mahtiglík
tékan getógda, þat sie gitrúodin þiu bet,
gilóƀdin an is léra. He só managan lík-hamon
balusuhteo antband endi bóta geskeride,
far-gaf fégiun ferah, þem þe fúsid was
helið an helsíð: þan gideda ina þe héland self,
Krist þurh is kraft mikil kwikan aftar dóða,
lét ina an þesaro weroldi forð wunnjono neotan.
Só hélde he þea haltun man endi þea háƀon só self,
bótta, þem þar blinde wárun, lét sie þat berhte lioht,
sin-skóni sehan, sundja lósda,
gumono grimwerk. Ni was gio Judeono beþiu,
léðes liud-skępjes gilóƀo þiu betara
an þene hélagon Krist, ak habdun im hardene mód,
swíðo starkan stríd, farstandan ni weldun,
þat sie habdun forfangan fíundun an willjan,
liudi mid iro gelóƀun. Ni was gio þiu latoro beþiu
sunu drohtines, ak he sagde mid wordun,
hwó sie skoldin gehalon himiles ríki,
lérde aftar þemu lande, habde imu þero liudjo só filu
giwenid mid is wordun, þat im werod mikil,
folk folgoda, endi he im filu sagda,
be biliðiun þat barn godes, þes sie ni mahtun an iro breostun farstandan,
undarhuggjan an iro herton, ér it im þe hélago Krist
oƀar þat erlo folk oponun wordun
þurh is selƀes kraft sęggjan welda,
márjan hwat he ménde. Þar ina megin umbi,
þioda þrungun: was im þarf mikil
te gi-hórjenne heƀen-kuninges
wár-fastun word. He stód imu þó bi énes watares staðe,
ni welde þó bi þemu geþringe oƀar þat þegno folk
an þemu lande uppan þea léra kúðean,
ak geng imu þó þe gódo endi is jungaron mid imu,
friðu-barn godes, þemu flóde náhor
an én skip innan, endi it skalden hét
lande rúmur, þat ina þea liudi só filu,
þioda ni þrungi. Stód þegan manag,
werod bi þemu watare, þar waldand Krist
oƀar þat liudjo folk léra sagde:
ʽhwat, ik iu sęggjan magʼ, kwað he, ʽge-síðos míne,
hwó imu én erl bigan an erðu sáian
hrénkorni mid is handun. Sum it an hardan stēn
oƀanwardan fel, erðon ni habda,
þat it þar mahti wahsan efþa wurtjo gifáhan,
kínan efþa biklíƀen, ak warð þat korn farloren,
þat þar an þeru léian gilag. Sum it eft an land bifel,
an erðun aðalkunnjes: bigan imu aftar þiu
wahsen wánlíko endi wurtjo fáhan,
lód an lustun: was þat land só gód,
fránisko gifehod. Sum it eft bifallen warð
an éna starka strátun, þar stópon gengun,
hrosso hófslaga endi heliðo tráda;
warð imu þar an erðu endi eft up gigeng,
bigan imu an þemu wege wahsen; þó it eft þes werodes farnam,
þes folkes fard mikil endi fuglos alásun,
þat is þemu éksan wiht aftar ni móste
werðan te willjan, þes þar an þene weg bifel.
Sum warð it þan bifallen, þar só filu stódun
þikkero þorno an þemu dage;
warð imu þar an erðu endi eft up gigeng,
kén imu þar endi kliƀode. Þó slógun þar eft krúd an gimang,
weridun imu þene wastom: habda it þes waldes hlea
forana oƀar-fangan, þat it ni mahte te énigaro frumu werðen,
ef it þea þornos só þringan móstun.ʼ
Þó sátun endi swígodun ge-síðos Kristes,
wordspáha weros: was im wundar mikil,
be hwilikun biliðiun þat barn godes
sulik sóðlík spel sęggjan bigunni.
Þó bigan is þero erlo én frágojan
holdan hérron, hnég imu tegegnes
tulgo werðliko: ʽhwat, þu gewald haƀasʼ, kwað he,
ʽia an himile ia an erðu, hélag drohtin,
uppa endi niðara, bist þu alo-waldo
gumono gésto, endi wi þíne jungaron sind,
an úsumu hugi holde. Hérro þe gódo,
ef it þín willjo sí, lát ús þínaro wordo þar
endi gi-hórjen, þat wi it aftar þi
oƀar al Kristinfolk kúðean mótin.
wi witun þat þínun wordun wár-lík biliði
forð folgojad, endi ús is firinun þarf,
þat wi þín word endi þín werk, — hwand it fan sulikumu gewittea kumid —
þat wi it an þesumu lande at þi línon mótin.ʼ
Þó im eft tegegnes gumono bezta
and-wordi gesprak: ʽni ménde ik elkor wihtʼ, kwað he,
ʽte bidernienne dádjo mínaro,
wordo efþa werko; þit skulun gi witan alle,
jungaron míne, hwand iu fargeƀen haƀad
waldand þesaro weroldes, þat gi witan mótun
an iuwom hugi-skęftjun himilisk gerúni;
þem óðrun skal man be biliðiun þat gi-bod godes
wordun wísien. Nu willju ik iu te wárun hier
márjen, hwat ik ménde, þat gi mína þiu bet
oƀar al þit land-skępi léra farstandan.
Þat sád, þat ik iu sagda, þat is selƀes word,
þiu hélaga léra heƀen-kuninges,
hwó man þea márjen skal oƀar þene middil-gard,
wído aftar þesaro weroldi. Weros sind im gihugide,
man mislíko: sum sulikan mód dregid,
harda hugi-skęfti endi hréan seƀon,
þat ina ni gewerðod, þat he it be iuwon wordun due,
þat he þesa mína léra forð léstien willje,
ak werðad þar só farlorana léra mína,
godes ambusni endi iuwaro gumono word
an þemu uƀilon manne, só ik iu ér sagda,
þat þat korn farwarð, þat þar mid kíðun ni mahte
an þemu sténe uppan stedihaft werðan.
Só wirðid al farloran eðilero spráka,
árundi godes, só hwat só man þemu uƀilon manne
wordun gewísid, endi he an þea wirson hand,
undar fíundo folk fard gekiusid,
an godes unwilean endi an gramono hróm
endi an fiures farm. Forð skal he hétean
mid is breosþugi bréda logna.
Nio gi an þesumu lande þiu lés léra mína
wordun ni wísiad: is þeses werodes só filu,
erlo aftar þesaro erðun: bistéd þar óðar man,
þe is imu iung endi glau, — endi haƀad imu gódan mód —,
sprákono spáhi endi wét iuwaro spello giskéð,
hugid is þan an is herton endi hórid þar mid is órun tó
swíðo niudlíko endi náhor stéd,
an is breost hledid þat gi-bod godes,
línod endi léstid: is is gilóƀo só gód,
talod imu, hwó he óðrana eft gihwerƀie
mén-dádigan man, þat is mód draga
hluttra trewa te heƀen-kuninge.
Þan brédid an þes breostun þat gi-bod godes,
þie luƀigo gilóbo, só an þemu lande duod
þat korn mid kíðun, þar it gikund haƀad
endi imu þiu wurð bihagod endi wederes gang,
regin endi sunne, þat it is reht haƀad.
Só duod þiu godes léra an þemu gódun manne
dages endi nahtes, endi gangid imu diuƀal fer,
wréða wihti endi þe ward godes
náhor mikilu nahtes endi dages,
anttat sie ina brengead, þat þar béðiu wirðid
ia þiu léra te frumu liudjo barnun,
þe fan is múðe kumid, iak wirðid þe man gode;
haƀad só giwehslod te þesaro weroldstundu
mid is hugi-skęftjun himil-ríkjas gidél,
welono þene méstan: farid imu an giwald godes,
tionuno tómig. Trewa sind só góda
gumono gehwilikumu, só nis goldes hord
gelík sulikumu gilóƀon. Wesad iuwaro lérono forð
man-kunnje mildje; sie sind só mislíka,
heliðos gehugda: sum haƀad iro hardan stríd,
wréðan willjan, wankolna hugi,
is imu féknes ful endi firin-werko.
Þan biginnid imu þunkjan, þan he undar þeru þiodu stád
endi þar gi-hórid oƀar hlust mikil
þea godes léra, þan þunkid imu, þat he sie gerno forð
léstien willje; þan biginnid imu þiu léra godes
an is hugi hafton, anttat imu þan eft an hand kumid
feho te gifórja endi fremiði skat.
Þan farlédead ina léða wihti,
þan he imu farfáhid an fehogiri,
aleskid þene gilóbon: þan was imu þat luttil fruma,
þat he it gio an is hertan gehugda, ef he it halden ne wili.
Þat is só þe wastom, þe an þemu wege began,
liodan an þemu lande: þó farnam ina eft þero liudjo fard.
Só duot þea megin-sundjon an þes mannes hugi
þea godes léra, ef he is ni gómid wel;
elkor bifelliad sia ina ferne te boðme,
an þene hétan hel, þar he heƀen-kuninge
ni wirðid furður te frumu, ak ina fíund skulun
wítiu giwaragean. Simla gi mid wordun forð
léread an þesumu lande: *ik kan þesaro liudjo hugi,
só mis-líkan muod-seƀon \hld\ manno kunnjes,
só wanda wísa \hld\ [...]
Sum haƀit all te þiu is muod gilátan endi mér sorogot,
hwó hie þat hord bihalde, þan hwó hie heƀan-kuninges
willjon giwirkje. Beþiu þar wahsan ni mag
þat hélaga gi-bod godes, þoh it þar ahafton mugi,
wurtjon biwerpan, hwand it þie welo þringit.
Só samo só þat krúd endi þie þorn þat korn antfáhat,
weriat im þena wastom, só duot þie welo manne:
giheftid is herta, þat hie it gihuggian ni muot,
þie man an is muode, þes hie mést biþarf,
hwó hie þat giwirkje, þan lang þie hie an þesaro weroldi sí,
þat hie ti éwondage after muoti
hębbjan þuru is hérren þank himiles ríki,
só ęndi-lósan welon, só þat ni mag énig man
witan an þesaro weroldi. Nio hie só wído ni kan
te giþęnkjanne, þegan an is muode,
þat it bihaldan mugi herta þes mannes,
þat hie þat ti wáron witi, hwat waldand god haƀit
guodes gigerewid, þat all geginwerd stéð
manno só hwilikon, só ina hier minnjot wel
endi selƀo te þiu is seola gihaldit,
þat hie an lioht godes líðan muoti.ʼ
Só wísda hie þuo mid wordon, stuod werod mikil
umbi þat barn godes, gehórdun ina bi biliðon filo
umbi þesaro weroldes giwand wordon tęlljan;
kwat þat im ók én aðales man an is akker sáidi
hluttar hrénkorni handon sínon:
wolda im þar só wun-sames wastmes tilian,
fagares fruhtes. Þuo geng þar is fíond aftar
þuru dernian hugi, endi it all mid durðu oƀarséu,
mid weodo wirsiston. Þuo wóhsun sia béðiu,
ge þat korn ge þat krúd. Só kwámun gangan
is hagastoldos te hús, iro hérren sagdun,
þegnos iro þiodne þrístion wordon:
ʽhwat, þu sáidos hluttar korn, hérro þie guodo,
énfald an þínon akkar: nu ni gisihit énig erlo þan mér
weodes wahsan. hwí mohta þat giwerðan só?ʼ
Þuo sprak eft þie aðales man þem erlon tegegnes,
þiodan wið is þegnos, kwat þat hie it mahti undar-þenkian wel,
þat im þar unhold man aftar sáida,
fíond fékni krúd: ʽne gionsta mi þero fruhtio wel,
awerda mi þena wastom.ʼ Þuo þar eft wini sprákun,
is jungron tegegnes, kwáðun þat sia þar weldin gangan tuo,
kuman mid kraftu endi lósjan þat krúd þanan,
halon it mid iro handon. Þuo sprak im eft iro hérro angegin:
ʽne welleo ik, þat gi it wiodonʼ, kwaþie, ʽhwand gi biwardon ni mugun,
gigómean an iuwon gange, þoh gi it gerno ni duan,
ni gi þes kornes te filo, kíðo awerdiat,
felliat under iuwa fuoti. Láte man sia forð hinan
béðiu wahsan, und ér bewod kume
endi an þem felde sind fruhti rípia,
aroa an þem akkare: þan faran wi þar alla tuo,
halon it mid ússan handon endi þat hrénkurni lesan
súƀro tesamne endi it an mínon seli duojan,
hębbjan it þar gihaldan, þat it hwergin ni mugi
wiht awerdian, endi þat wiod niman,
bindan it te burðinnion endi werpan it an bittar fiur,
láton it þar halojan héta lógna,
éld unfuodi.ʼ Þuo stuod erl manag,
þegnos þagiandi, hwat þiod-gomo,
*mári mahtig Krist ménean weldi,
bóknien mid þiu biliðiu barno ríkjost.
Bádun þó só gerno gódan drohtin
antlúkan þea léra, þat sia móstin þea liudi forð,
hélaga hórjan. Þó sprak im eft iro hérro angegin,
mári mahtig Krist: ʽþat isʼ, kwað he, ʽmannes sunu:
ik selƀo bium, þat þar sáiu, endi sind þesa sáliga man
þat hluttra hrénkorni, þea mi hér hórjad wel,
wirkiad mínan willjan; þius werold is þe akkar,
þit bréda búland barno man-kunnjes;
Satanas selƀo is, þat þar sáid aftar
só léðlíka léra: haƀad þesaro liudjo só filu,
werodes awardid, þat sie wam frummjad,
wirkjad aftar is willjon; þoh skulun sie hér wahsen forð,
þea forgriponon gumon, só samo só þea gódun man,
anttat múdspelles megin oƀar man ferid,
endi þesaro weroldes. Þan is allaro akkaro gehwilik
gerípod an þesumu ríkja: skulun iro regan-gi-skapu
frummjen firiho barn. Þan tefarid erða:
þat is allaro bewo brédost; þan kumid þe berhto drohtin
oƀana mid is ęngilo kraftu, endi kumad alle tesamne
liudi, þe io þit lioht gisáun, endi skulun þan lón antfáhan
uƀiles endi gódes. Þan gangad ęngilos godes,
hélage heƀenwardos, endi lesat þea hluttron man
sundor tesamne, endi duat sie an sin-skóni,
hóh himiles lioht, endi þea óðra an hellia grund,
werpad þea farwarhton an wallandi fiur;
þar skulun sie gibundene bittra logna,
þráwerk þolon, endi þea óðra þiodwelon
an heƀen-ríkja, hwítaro sunnon
liohtean gelíko. Sulik lón nimad
weros waldádjo. Só hwe só giwit égi,
gehugdi an is hertan, etþa gi-hórjen mugi,
erl mid is órun, só láta imu þit an innan sorga,
an is mód-seƀon, hwó he skal an þemu márjon dage
wið þene ríkjon god an reðiu standen
wordo endi werko allaro, þe he an þesaro weroldi giduod.
Þat is egislíkost allaro þingo,
forhtlíkost firiho barnun, þat sie skulun wið iro fráhon mahlien,
gumon wið þene gódan drohtin: þan weldi gerno gehwe wesan,
allaro manno gehwilik ménes tómig,
slíðero sakono. Aftar þiu skal sorgon ér
allaro liudjo gehwilik, ér he þit lioht afgeƀe,
þe þan égan wili alungan tír,
hóh heƀen-ríki endi huldi godes.ʼ
Só gi-fragn ik þat þó selƀo sunu drohtines,
allaro barno bezt biliðeo sagda,
hwilik þero wári an werold-ríkja
undar heliðkunnje \hld\ himil-ríkje gelík;
kwað þat oft luttiles hwat \hld\ liohtora wurði,
só hóho af-huoƀi, \hld\ ʽso duot himil-ríki:
þat is simla méra, \hld\ þan is man énig
wánje an þesaro weroldi. \hld\ Ôk is imu þat werk ge-lík,
þat man an séo innan \hld\ segina wirpit,
fisk-net an flód \hld\ endi fáhit béðiu,
uƀile endi góde, \hld\ tiuhid up te staðe,
liðod sie te lande, \hld\ lisit aftar þiu
þea gódun an greote \hld\ endi látid þea óðra eft an grund faran,
an wídan wág. \hld\ Só duod waldand god
an þemu márjon dage \hld\ menniskono barn:
brengid irmin-þiod, \hld\ alle tesamne,
lisit imu þan þea hluttron \hld\ an heƀen-ríki,
látid þea far-griponon \hld\ an grund faren
hellje fiures. \hld\ Ni wét heliðo man
þes wítjes wiðarlága, \hld\ þes þar weros þiggjat,
an þemu inferne \hld\ irmin-þioda.
Þan hald ni mag þera médan man \hld\ gi-makon fíðen,
ni þes welon ni þes willjon, \hld\ þes þar waldand skerid,
gildid god selƀo \hld\ gumono só hwilikumu,
só ina hér gi-haldid, \hld\ þat he an heƀen-ríki,
an þat lang-same lioht \hld\ líðan móti.ʼ
Só lérda he þó mid listiun. \hld\ Þan fórun þar þea liudi tó
oƀar al Galilaeo land \hld\ þat godes barn sehan:
dádun it bi þemu wundre, \hld\ hwanen imu mahti sulik word kumen,
só spáh-líko gi-sprokan, \hld\ þat he spel godes
gio só sóð-líko \hld\ sęggjan konsti,
só kraftig-líko gi-kweðen: \hld\ ʽhe is þeses kunnjes hinenʼ, kwáðun sie,
ʽþe man þurh mág-skępi: \hld\ hér is is móder mid ús,
wíf undar þesumu werode. \hld\ Hwat, wi þe hér witun alle,
só kúð is ús is kuni-burd \hld\ endi is knósles ge-hwat;
a-wóhs al undar þesumu werode: \hld\ hwanen skoldi imu sulik ge-wit kuman,
méron mahti, \hld\ þan hér óðra man égin?ʼ
Só farmunste ina þat manno folk \hld\ endi sprákun im gi-méd-lik word,
far-hogdun ina só hélagna, \hld\ hórjen ni weldun
is gi-bod-skępjes. \hld\ Ni he þar ók biliðeo filu
þurh iro un-gi-lóƀon \hld\ ógjan ni welde,
torhtero tékno, \hld\ hwand he wisse iro twíflean hugi,
iro wréðan willjan, \hld\ þat ni wárun weros óðra
só grimme under Judeon, \hld\ só wárun umbi Galilaeo land,
só hardo ge-hugide: \hld\ só þar was þe hélago Krist,
gi-boren þat barn godes, \hld\ si ni weldun is gi-bod-skępi þoh
ant-fáhan ferht-líko, \hld\ ak bigan þat folk undar im,
rinkos rádan, \hld\ hwó sie þene ríkjon Krist
wégdin te wundron. \hld\ Hétun þó iro werod kumen,
ge-síði tesamne: \hld\ sundja weldun
an þene godes sunu \hld\ gerno gi-tęlljen
wréðes willjon; \hld\ ni was im is wordo niud,
spáharo spello, \hld\ ak sie bi-gunnun sprekan undar im,
hwó sie ina só kraftagne \hld\ fan énumu kliƀe wurpin,
oƀar énna berges wal: \hld\ weldun þat barn godes
liƀu bi-lósjen. \hld\ Þó he imu mid þem liudjun samad
fró-líko fór: \hld\ ni was imu foraht hugi,
—wisse þat imu ni mahtun \hld\ menniskono barn,
bi þeru god-kundi \hld\ Judeo liudi
ér is tídiun wiht \hld\ teonon gifrummjen,
léðaro gilésto—, \hld\ ak he imu mid þem liudjun samad
stég uppen þene stén-holm, \hld\ antþat sie te þeru stedi kwámun,
þar sie ine fan þemu walle niðer \hld\ werpen hugdun,
fellien te foldu, \hld\ þat he wurði is ferhes lós,
is aldres at endie. \hld\ Þó warð þero erlo hugi,
an þemu berge uppen \hld\ bittra gi-þáhti
Juðeono te-gangen, þat iro énig ni habde só grimmon seƀon
ni só wréðen willjon, þat sie mahtin þene waldandes sunu,
Krist ant-kęnnjen; he ni was iro kúð énigumu,
þat sie ina þó undarwissin. \hld\ Só mahte he undar ira werode standen
endi an iro gimange \hld\ middjumu gangen,
faren undar iro folke. \hld\ He dede imu þene friðu selƀo,
mundburd wið þeru menegi \hld\ endi giwét imu þurh middi þanan
þes fíundo folkes, \hld\ fór imu þó, þar he welde,
an éne wóstunnie \hld\ waldandes sunu,
kuningo kraftigost: \hld\ habde þero kustes giwald,
hwar imu an þemu lande \hld\ leoƀost wári
te wesanne an þesaru weroldi. \hld\ Þan fór imu an weg óðran
Johannes mid is jungarun, \hld\ godes ambahtman,
lérde þea liudi \hld\ lang-samane rád,
hét þat sie frume fremidin, \hld\ firina farlétin,
mén endi morðwerk. \hld\ He was þar managumu liof
gódaro gumono. \hld\ He sóhte imu þó þene Judeono kuning,
þene hęri-togon at hús, \hld\ þe héten úuas
Erodes aftar is eldiron, \hld\ oƀar-módig man:
búide imu be þeru brúdi, \hld\ þiu ér sínes bróðer was,
idis an éhti, \hld\ anttat he ellior skók,
werold weslode. \hld\ Þó imu þat wíf ginam
þe kuning te kwenun; \hld\ ér wárun iro kind ódan,
barn be is bróðer. \hld\ Þó bigan imu þea brúd lahan
Johannes þe gódo, \hld\ kwað þat it gode wári,
waldande wiðer-mód, \hld\ þat it énig wero frumidi,
þat bróðer brúd \hld\ an is bed námi,
hębbje sie imu te híwun. \hld\ ʽEf þu mi hórjen wili,
gilóƀien mínun lérun, \hld\ ni skalt þu sie leng égan,
ak míð ire an þínumu móde: \hld\ ni haƀa þar sulika minnja tó,
ni sundjo þi te swíðo.ʼ \hld\ Þó warð an sorgun hugi
þes wíƀes aftar þem wordun; \hld\ andréd þat he þene werold-kuning
sprákono ge-spóni \hld\ endi spáhun wordun,
þat he sie far-léti. \hld\ Be-gan siu imu þó léðes filu
ráden an rúnon, \hld\ endi ine rinkos hét,
un-sundigane \hld\ erlos fáhan
endi ine an énumu karkerea \hld\ klústar-bendiun,
liðo-kospun bi-lúkan: \hld\ be þem liudjun ne gi-dorstun
ine ferahu bi-lósjen, \hld\ hwand sie wárun imu friund alle,
wissun ine só góden \hld\ endi gode werðen,
habdun ina for wár-sagon, \hld\ só sia wela mahtun.
Þó wurðun an þemu gér-tale Judeo kuninges
tídi kumana, \hld\ só þar gi-tald habdun
fróde folk-weros, \hld\ þó he gi-fódid was,
an lioht kuman. \hld\ Só was þero liudjo þau,
þat þat erlo gehwilik \hld\ óƀean skolde,
Judeono mid gómun. \hld\ Þó warð þar an þene gast-sęli
megin-kraft mikil \hld\ manno gesamnod,
hęri-togono an þat hús, \hld\ þar iro hérro was
an is kuningstóle. \hld\ kwámun managa
Judeon an þene gast-sęli; \hld\ warð im þar gladmód hugi,
blíði an iro breostun: \hld\ gisáhun iro bággeƀon
wesen an wunnjon. \hld\ Dróg man wín an flet
skíri mid skálun, \hld\ skenkeon hwurƀun,
gengun mid goldfatun: \hld\ gaman was þar inne
hlúd an þero hallu, \hld\ heliðos drunkun.
was þes an lustun \hld\ landes hirdi,
hwat he þemu werode mést \hld\ te wunnjun gifremidi.
Hét he þó gangen forð \hld\ géla þiornun,
is bróder barn, \hld\ þar he an is bęnki sat
wínu giwlenkid, \hld\ endi þó te þemu wíbe sprak;
grótte sie fora þemu gum-skępje \hld\ endi gerno bad,
þat siu þar fora þem gastjun \hld\ gaman af-hóƀi
fagar an flęttje: \hld\ ʽlát þit folk sehan,
hwó þu gelínod haƀas \hld\ liudjo menegi
te blíðseanne an bęnkjun; \hld\ ef þu mi þera bede tugiðos,
mín word for þesumu werode, þan willju ik it hér te wárun gekweðen,
liahto fora þesun liudjun endi ók giléstien só,
þat ik þi þan aftar þiu éron willju,
só hwes só þu mi bidis for þesun mínun bágwiniun:
þoh þu mi þesaro heridómo halƀaro fergos,
ríkjas mínes, þoh gidón ik, þat it énig rinko ni mag
wordun giwendien, endi it skal giwerðen só.ʼ
Þó warð þera magað aftar þiu mód gihworƀen,
hugi aftar iro hérron, þat siu an þemu húse innen,
an þemu gast-sęli gamen up ahuof,
al só þero liudjo landwíse gidróg,
þero þiodo þau. Þiu þiorne spilode
hrór aftar þemu húse: hugi was an lustun,
managaro mód-seƀo. Þó þiu magað habda
giþionod te þanke þiod-kuninge
endi allumu þemu erl-skępje, þe þar inne was
gódaro gumono, siu welde þó ira geƀa égan,
þiu magað for þeru menegi: geng þó wið iro módar sprekan
endi frágode sie firiwitlíko,
hwes siu þene burges ward biddjen skoldi.
Þó wísde siu aftar iro willjon, hét þat siu wihtes þan ér
ni gerodi for þemu gum-skępje, biútan þat man iru Johannes
an þeru hallu innan hóƀid gáƀi
a-lósid af is lík-hamon. Þat was allun þem liudjun harm,
þem mannun an iro móde, þó sie þat gi-hórdun þea magað sprekan;
só was it ók þemu kuninge: he ni mahte is kwidi liagan,
is word wendien: hét þó is wépanberand
gangen fan þemu gast-sęli endi hét þene godes man
líƀu bilósjen. \hld\ Þó ni was lang te þiu,
þat man an þea halla \hld\ hóƀid bráhte
þes þiod-gumon, endi it þar þeru þiornun far-gaf,
magað for þeru menegi: siu dróg it þeru móder forð.
Þó was éndago allaro manno
þes wísoston, þero þe gio an þesa werold kwámi,
þero þe kwene énig kind gibári,
idis fan erle, lét man simla þen énon biforan,
þe þiu þiorne gidróg, þe gio þegnes ni warð
wís an iro weroldi, biútan só ine waldand god
fan heƀen-wange hélages géstes
gimarkode mahtig: þe ni habde énigan gimakon hwergin
ér nek aftar. Erlos hwurƀun,
gumon umbi Johannen, \hld\ is jungaron managa,
sálig ge-síði, \hld\ endi ine an sande bigróƀun,
leoƀes lík-hamon: \hld\ wissun þat he lioht godes,
diurlíkan dróm \hld\ mid is drohtine samad,
upódas hém \hld\ égan móste,
sálig sókjan. \hld\ Þó gewitun im þea ge-síðos þanen,
Johannes giungaron \hld\ giámer-móde,
hélagferaha: was im iro hérron dóð
swíðo an sorgun. Gewitun im sókjan þó
an þeru wóstunni waldandes sunu,
kraftigana Krist endi imu kúð gidedun
gódes mannes forgang, hwó habde þe Judeono kuning
manno þene márjostan mákjas ęggjun
hóƀdu bihawuan: he ni welde is énigen harm spreken,
sunu drohtines; he wisse þat þiu seole was
hélag gihalden wiðer hettiandeon,
an friðe wiðer fíundun. Þó só gifrági warð
aftar þem land-skępiun léreandero bezt
an þeru wóstunni: werod samnode,
fór folkun tó: was im firiwit mikil
wísaro wordo; imu was ók willjo só samo,
sunje drohtines, þat he sulik ge-síðo folk
an þat lioht godes laðojan mósti,
wennien mid willjon. Waldand lérde
allan langan dag liudi managa,
eli-þeodige man, anttat an áƀand ség
sunne te sedle. Þó gengun is ge-síðos tweliƀi,
gumon te þemu godes barne endi sagdun iro gódumu hérron,
mid hwiliku arƀediu þar þea erlos liƀdin, kwáðun þat sie is éra biþorftin,
weros an þemu wósteon lande: ʽsie ni mugun sie hér mid wihti ant-hębbjen,
heliðos bi hungres geþwinge. Nu lát þu sie, hérro þe gódo,
síðon, þar sie seliða fíðen. Náh sind hér gesetana burgi
managa mid megin-þiodun: þar fíðad sie meti te kópe,
weros aftar þem wíkeon.ʼ Þó sprak eft waldand Krist,
þioda drohtin, kwað þat þes éniga þurufti ni wárin,
ʽþat sie þurh metilósi mína farlátan
leoƀlíka léra. Geƀad gi þesun liudjun ginóg,
wenniad sie hér mid willjon.ʼ Þó habde eft is word garu
Philippus fród gumo, kwað þat þar só filu wári
manno menigi: ʽþoh wi hér te meti habdin
garu im te geƀanne, só wi mahtin fargelden mést,
ef wi hér gisaldin siluƀerskatto
twé hund samad, tueho wári is noh þan,
þat iro énig þar énes ginámi:
só luttik wári þat þesun liudjun.ʼ Þó sprak eft þe landes ward
endi frágode sie firiwitlíko,
manno drohtin, hwat sie þar te meti habdin
wistes ge-wunnin. Þó sprak imu eft mid is wordun angegin
Andreas fora þem erlun endi þemu alo-waldon
selƀumu sagde, þat sie an iro gisíðje þan mér
garowes ni habdin, ʽbi-útan girstin bród
fíƀi an úsaru ferdi endi fiskos twéne.
Huat mag þat þoh þesaru menigi?ʼ Þó sprak imu eft mahtig Krist,
þe gódo godes sunu, endi hét þat gumono folk
skerien endi skéðen endi hét þea skola settien,
erlos aftar þeru erðu, irmin-þioda
an grase gruonimu, endi þó te is jungarun sprak,
allaro barno bezt, hét imu þiu bród halon
endi þea fiskos forð. Þat folk stillo béd,
sat ge-síði mikil; undar þiu he þurh is selƀes kraft,
manno drohtin, þene meti wíhide,
hélag heƀen-kuning, endi mid is handun brak,
gaf it is jungarun forð, endi it sie undar þemu gum-skępje hét
dragan endi délien. Sie léstun iro drohtines word,
is geƀa gerno drógun gumono gihwemu,
hélaga helpa. It undar iro handun wóhs,
meti manno gihwemu: þeru megin-þiodu warð
líf an lustun, þea liudi wurðun alle,
sade sálig folk, só hwat só þar gisamnod was
fan allun wídun wegun. Þó hét waldand Krist
gangen is jungaron endi hét sie gómien wel,
þat þiu léƀa þar farloren ni wurði;
hét sie þó samnon, þó þar sade wárun
man-kunnjes manag. Þar móses warð,
bródes te léƀu, þat man birilos gilas
tweliƀi fulle: þat was tékan mikil,
grót kraft godes, hwand þar was gumono gitald
áno wíf endi kind, werodes atsamme
fíf þúsundig. Þat folk al farstód,
þea man an iro móde, þat sie þar mahtigna
hérron habdun. Þó sie heƀen-kuning,
þea liudi loƀodun, kwáðun þat gio ni wurði an þit lioht kuman
wísaro wár-sago, efþa þat he giwald mid gode
an þesaru middil-gard méron habdi,
énfaldaran hugi. Alle gisprákun,
þat he wári wirðig welono gehwilikes,
þat he erðríki égan mósti,
wídene weroldstól, ʽnu he sulik gewit haƀad,
só gróte kraft mid gode.ʼ Þea gumon alle giwarð,
þat sie ine gihóƀin te hérosten,
gikurin ine te kuninge: þat Kriste ni was
wihtes wirðig, hwand he þit werold-ríki,
erðe endi uphimil þurh is énes kraft
selƀo giwarhte endi síðor giheld,
land endi liud-skępi, — þoh þes énigan gilóƀon ni dedin
wréðe wiðer-sakon — þat al an is giwalde stád,
kuning-ríkjo kraft endi késurdómes,
megin-þiodo mahal. Beþiu ni welde he þurh þero manno spráka
hębbjan énigan hérdóm, hélag drohtin,
werold-kuninges namon; ni he þó mid wordun stríd
ni af-hóf wið þat folk furður, ak fór imu þó, þar he welde,
an én gebirgi uppan: flóh þat barn godes
gélaro gelp-kwidi endi is jungaron hét
oƀar énne séo síðon endi im selƀo gibód,
hwar sie im eft tegegnes gangen skoldin.
Þó telét þat liud-werod aftar þemu lande allumu,
tefór folk mikil, síðor iro fráho giwét
an þat gebirgi uppan, barno ríkjost,
waldand an is willjon. Þó te þes watares staðe
samnodun þea ge-síðos Kristes, þe he imu habde selƀo gikorane,
sie tweliƀi þurh iro trewa góda: ni was im tueho nigiean,
neƀu sie an þat godes þionost gerno weldin
oƀar þene séo síðon. \hld\ Þó létun sie swíðean stróm,
hóh hurnid-skip \hld\ hluttron úðeon,
skéðan skír water. \hld\ Skréd lioht dages,
sunne warð an sedle; \hld\ þe séo-líðandean
naht neƀulo biwarp; \hld\ náðidun erlos
forð-wardes an flód; \hld\ warð þiu fiorðe tid
þera nahtes kuman \hld\ —nęrjendo Krist
warode þea wág-líðand—: \hld\ þó warð wind mikil,
hóh weder af-haƀen: \hld\ hlamodun úðeon,
stróm an stamne; \hld\ strídiun feridun
þea weros wiðer winde, \hld\ was im wréð hugi,
seƀo sorgono ful: \hld\ selƀon ni wándun
lagu-líðandea \hld\ an land kumen
þurh þes wederes gewin. \hld\ Þó gisáhun sie waldand Krist
an þemu sée uppan \hld\ selƀun gangan,
faran an fáðion: \hld\ ni mahte an þene flód innan,
an þene séo sinkan, \hld\ hwand ine is selƀes kraft
hélag ant-habde. \hld\ Hugi warð an forhtun,
þero manno mód-seƀo: \hld\ and-rédun þat it im mahtig fíund
te gi-droge dádi. \hld\ Þó sprak im iro drohtin tó,
hélag heƀen-kuning, \hld\ endi sagde im þat he iro hérro was
mári endi mahtig: \hld\ ʽnu gi módes skulun
fastes fáhen; \hld\ ne sí iu forht hugi,
gi-báriad gi bald-líko: \hld\ ik bium þat barn godes,
is selƀes sunu, \hld\ þe iu wið þesumu sée skal,
mundon wið þesan meri-stróm.ʼ \hld\ Þó sprak imu én þero manno an-gegin
oƀar bord skipes, \hld\ bar-wirðig gumo,
Petrus þe gódo \hld\ —ni welde píne þolon,
watares wíti—: \hld\ ʽef þu it waldand sísʼ, kwað he,
ʽhérro þe gódo, \hld\ só mi an mínumu hugi þunkit,
hét mi þan þarod gangan te þi \hld\ oƀar þesen geƀenes stróm,
drokno oƀar diap water, \hld\ ef þu mín drohtin sís,
managoro mund-boro.ʼ \hld\ Þó hét ine mahtig Krist
gangan imu te-gegnes. \hld\ He warð garu sáno,
stóp af þemu stamne \hld\ endi strídiun geng
forð te is frójan. \hld\ Þiu flód ant-habde
þene man þurh maht godes, \hld\ antat he imu an is móde bigan
andráden diap water, \hld\ þó he dríƀen gisah
þene wég mid windu: \hld\ wundun ina úðeon,
hóh stróm umbihring. \hld\ Reht só he þó an is hugi tuehode,
só wék imu þat water under, \hld\ endi he an þene wág innan,
sank an þene séostróm, \hld\ endi he hriop sán aftar þiu
gáhon te þemu godes sunje \hld\ endi gerno bad,
þat he ine þó ge-neridi, \hld\ þó he an nódjun was,
þegan an ge-þwinge. \hld\ Þiodo drohtin
ant-feng ine mid is faðmun \hld\ endi frágode sána,
te hwí he þó ge-tuehodi: \hld\ ʽhwat, þu mahtes getrúojan wel,
witen þat te wárun, \hld\ þat þi watares kraft
an þemu sée innen \hld\ þínes síðes ni mahte,
lagu-stróm gi-lęttjen, \hld\ só lango só þu habdes ge-lóƀon te mi
an þínumu hugi hardo. \hld\ Nu willju ik þi an helpun wesen,
nęrjen þi an þesaru nódiʼ. \hld\ Þó nam ine alo-mahtig,
hélag bi handun: \hld\ þó warð imu eft hlutter water
fast under fótun, \hld\ endi sie an fáði samad
béðea gengun, \hld\ antat sie oƀar bord skipes
stópun fan þemu stróme, \hld\ endi an þemu stamne gesat
allaro barno bezt. \hld\ Þó warð bréd water,
strómos ge-stillid, \hld\ endi sie te staðe kwámun,
lagu-líðandea \hld\ an land samen
þurh þes wateres ge-win, \hld\ sagdun þo waldande þank,
diurden iro drohtin \hld\ dádjun endi wordun,
fellun imu te fótun \hld\ endi filu sprákun
wísaro wordo, \hld\ kwáðun þat sie wissin garo,
þat he wári selƀo \hld\ sunu drohtines
wár an þesaru weroldi \hld\ endi ge-wald habdi
oƀar middil-gard, \hld\ endi þat he mahti allaro manno gihwes
ferahe giformon, \hld\ al só he im an þemu flóde dede
wið þes watares gewin. \hld\ Þó giwét imu waldand Krist
síðon fan þemu sée, \hld\ sunu drohtines,
énag barn godes. \hld\ Eli-þioda kwam imu,
gumon tegegnes: \hld\ wárun is gódun werk
ferran gefrági, \hld\ þat he só filu sagde
wároro wordo: \hld\ imu was willjo mikil,
þat he sulik folk-skępi \hld\ frummjen mósti,
þat sie simla gerno \hld\ gode þionodin,
wárin gehórige \hld\ heƀen-kuninge
man-kunnjes manag. \hld\ Þó giwét he imu oƀer þea marka Judeono,
sóhte imu Sidono burg, \hld\ habde ge-síðos mid imu,
góde jungaron. \hld\ Þar imu te-gegnes kwam
én idis fan áðrom þiodun; \hld\ siu was iru aðali-ge-burdjo,
kunnjes fan Kananeo lande; \hld\ siu bad þene kraftagan drohtin,
hélagna, þat he iru helpe ge-rédi, \hld\ kwað þat iru wári harm gi-standen,
soroga at iru selƀaru dohter, \hld\ kwað þat siu wári mid suhtiun bi-fangen:
ʽbe-drogan habbjad sie dernea wihti. \hld\ Nu is iro dód at hendi,
þea wréðon habbjad sie ge-wittju be-numane. \hld\ Nu biddju ik þi, waldand fró min,
selƀo sunu Dawides, \hld\ þat sie af sulikum suhtiun atómies,
þat þu sie só arma \hld\ égroht-fullo
wam-skaðon bi-weri.ʼ \hld\ Ni gaf iru þó noh waldand Krist
énig and-wordi; \hld\ siu imu aftar geng,
folgode fruokno, \hld\ antat siu te is fótun kwam,
grótte ina greatandi. \hld\ Giungaron Kristes
bádun iro hérron, \hld\ þat he an is hugja mildi
wurði þemu wíƀe. \hld\ Þó habde eft is word garu
sunu drohtines \hld\ endi te is ge-síðun sprak:
ʽérist skal ik Israheles \hld\ aƀoron werðen,
folk-skępi te frumu, \hld\ þat sie ferhtan hugi
hębbjan te iro hérron: \hld\ im is helpono þarf,
þea liudi sind far-lorane, \hld\ far-láten habbiad
waldandes word, \hld\ þat werod is ge-twíflid,
dríƀad im dernean hugi, \hld\ ne willjad iro drohtine hórjen
Israhelo erl-skępi, \hld\ un-gi-lóƀiga sind
heliðos iro hérron: \hld\ þoh skal þanen helpe kumen
allun eli-þiodun.ʼ \hld\ Agaléto bad
þat wíf mid iro wordun, \hld\ þat iru waldand Krist
an is mód-seƀon \hld\ mildi wurði,
þat siu iro barnes forð \hld\ brúkan mósti,
hębbjan sie héle. \hld\ Þó sprak iru hérro angegin,
mári endi mahtig: \hld\ ʽnis þatʼ, kwað he, ʽmannes reht,
gumono nigénum \hld\ gód te gi-frummjenne
þat he is barnun \hld\ bródes af-tíhe,
wernie im oƀar willjon, \hld\ láte sie wíti þolean,
hungar heti-grimmen, \hld\ endi fódie is hundos mid þiu.ʼ
ʽwár is þat, waldandʼ, \hld\ kwað siu, ʽþat þu mid þínun wordun sprikis,
sóð-líko sagis: \hld\ hwat, þoh oft an seli innen
undar iro hérron diske \hld\ hwelpos hwerƀad
brosmono fulle \hld\ þero fan þemu biode niðer
ant-fallat iro frójan.ʼ \hld\ Þó gi-hórde þat friðu-barn godes
willjan þes wíƀes \hld\ endi sprak iru mid is wordun tó:
ʽwela þat þu wíf haƀes \hld\ willjan góden!
Mikil is þín gi-lóƀo \hld\ an þea maht godes,
an þene liudjo drohtin. \hld\ Al wirðid gi-léstid só
umbi þínes barnes líf, \hld\ só þu bádi te mi.ʼ
Þó warð siu sán gi-hélid, \hld\ só it þe hélago gesprak
wordun wár-fastun: \hld\ þat wíf fagonode,
þes siu iro barnes forð \hld\ brúkan móste;
habde iru gi-holpen \hld\ héljando Krist,
habde sie far-fangane \hld\ fíundo kraftu,
wam-skaðun bi-werid. \hld\ Þó giwét imu waldand forð,
barno þat bezte, \hld\ sóhte imu burg óðre,
þiu só þikko was \hld\ mid þeru þiodu Judeono,
mid súðar-liudjun gi-seten. \hld\ Þar gi-fragn ik þat he is ge-síðos grótte,
þe jungaron þe he imu habde be is góde gi-korane, \hld\ þat sie mid imu gerno ge-wunodun,
weros þurh is wíson spráka: \hld\ ʽalle skal ik iuʼ, kwað he, ʽmid wordun frágon,
jungaron míne: \hld\ hwat kweðat þese Judeo liudi,
mári megin-þioda, \hld\ hwat ik manno sí?ʼ
Imu and-wordidun frólíko \hld\ is friund angegin,
jungaron síne: \hld\ ʽnis þit Judeono folk,
erlos én-wordje: \hld\ sum sagad þat þu Elias sís,
wís wár-sago, \hld\ þe hér giu was lango,
gód undar þesumu gum-skępje, \hld\ sum sagad þat þu Johannes sís,
diur-lík drohtines bodo, \hld\ þe hér dópte iu
werod an watere; \hld\ alle sie mid wordun sprekad,
þat þu én-hwilik sís \hld\ eðilero manno,
þero wár-sagono, \hld\ þe hér mid wordun giu
lérdun þese liudi, \hld\ endi þat þu sís eft an þit lioht kumen
te wíseanne þesumu werode.ʼ \hld\ Þó sprak eft waldand Krist:
ʽhwe kweðad gi, þat ik síʼ, \hld\ kwað he, ʽjungaron míne,
lioƀon liud-weros?ʼ \hld\ Þó te lat ni warð
Símon Petrus: \hld\ sprak sán an-gegin
éno for im allun \hld\ —habde imu ęlljen gód,
þrístea gi-þáhti, \hld\ was is þeodone hold—:
ʽþu bist þe wáro \hld\ waldandes sunu,
libbjendes godes, \hld\ þe þit lioht gi-skóp,
Krist kuning éwig: \hld\ só willjad wi kweðen alle,
jungaron þíne, \hld\ þat þu sís god selƀo,
héljandero bezt.ʼ \hld\ Þó sprak imu eft is hérro an-gegin:
ʽsálig bist þu Símonʼ, \hld\ kwað he, ʽsunu Jonases; ni mahtes þu þat selƀo ge-huggjan,
gi-markon an þínun mód-gi-þáhtiun, \hld\ ne it ni mahte þi mannes tunge
wordun ge-wísjen, \hld\ ak dede it þi waldand selƀo,
fader allaro firiho barno, \hld\ þat þu só forð gi-spráki,
só diapo bi drohtin þínen. \hld\ Diur-líko skalt þu þes lón ant-fáhen,
hluttro haƀas þu an þínan hérron gilóƀon, \hld\ hugi-skęfti sind þíne sténe ge-líka,
só fast bist þu só felis þe hardo; \hld\ héten skulun þi firiho barn
sankte Péter: \hld\ oƀar þemu sténe skal man mínen seli wirkjan,
hélag hús godes; \hld\ þar skal is híwiski tó
sálig samnon: \hld\ ni mugun wið þem þínun swíðeun krafte
an-þebbien hęllje portun. \hld\ Ik far-giƀu þi himil-ríkjas slutilas,
þat þu móst aftar mi \hld\ allun gi-waldan
kristinum folke; \hld\ kumad alle te þi
gumono géstos; \hld\ þu haƀe gróte gi-wald,
hwene þu hér an erðu \hld\ eldi-barno
ge-binden willjes: \hld\ þemu is béðiu giduan,
himil-ríki biloken, \hld\ endi hęllje sind imu opana,
brinnandi fiur; \hld\ só hwene só þu eft ant-binden wili,
an-þeftien is hendi, \hld\ þemu is himil-ríki,
ant-loken liohto mést \hld\ endi líf éwig,
gróni godes wang. \hld\ Mid sulikaru ik þi geƀu willju
lónon þínen gi-lóƀon. \hld\ Ni willju ik, þat gi þesun liudjun noh,
márjen þesaru menigi, \hld\ þat ik bium mahtig Krist,
godes égan barn. \hld\ Mi skulun Judeon noh,
un-skuldigna \hld\ erlos binden,
wégean mi te wundrun \hld\ —dót mi wítjes filo—
innan Hierusalem \hld\ géres ordun,
áhtien mínes aldres \hld\ ęggjun skarpun,
bi-lósjen mi líƀu. \hld\ Ik an þesumu liohte skal
þurh úses drohtines kraft \hld\ fan dóde a-standen
an þriddjumu dageʼ. \hld\ Þó warð þegno bezt
swíðo an sorgun, \hld\ Símon Petrus,
warð imu hugi hriwig, \hld\ endi te is hérron sprak
rink an rúnun: \hld\ ʽni skal þat ríki godʼ, kwað he,
ʽwaldand willjen, \hld\ þat þu eo sulik wíti mikil
gi-þolos undar þesaru þiod: \hld\ nis þes þarf nigiean,
hélag drohtin.ʼ \hld\ Þó sprak imu eft is hérro an-gegin,
mári mahtig Krist \hld\ —was imu an is móde hold—:
ʽhwat, þu nu wiðer-ward bistʼ, \hld\ kwað he, ʽwilljon mínes,
þegno bezto! \hld\ Huat, þu þesaro þiodo kanst
menniskan sidu: \hld\ þu ni wést þe maht godes,
þe ik gi-frummjen skal. \hld\ Ik mag þi filu sęggjan
wárun wordun, \hld\ þar hér undar þesumu werode standad
ge-síðos míne, \hld\ þea ni mótun swelten ér,
hwerƀen an hinen-fard \hld\ ér sie himiles lioht,
godes ríki sehat.ʼ \hld\ Kós imu jungarono þó
sán aftar þiu \hld\ Símon Petrus,
Jakob endi Johannes, \hld\ ea gumon twéne,
béðea þea gi-bróðer, \hld\ endi imu þó uppen þene berg giwét
sunder mid þem ge-síðun, \hld\ sálig barn godes,
mid þem þegnun þrim, \hld\ þiodo drohtin,
waldand þesaro weroldes: \hld\ welde im þar wundres filu,
tékno tógean, \hld\ þat sie gi-trúodin þiu bet,
þat he selƀo was \hld\ sunu drohtines,
hélag heƀen-kuning. \hld\ Þó sie an hóhan wall
stigun stén endi berg, \hld\ antat sie te þeru stędi kwámun,
weros wiðer wolkan, \hld\ þar waldand Krist,
kuningo kraftigost \hld\ gikoren habde,
þat he is god-kundi \hld\ jungarun sínun
þurh is énes kraft \hld\ ógean welde,
berht-lík biliði. \hld\ Þó imu þar te bedu gi-hnég,
þó warð imu þar uppe \hld\ óðar-líkora
wliti endi gi-wádi: \hld\ wurðun imu is wangun liohte,
blíkandi só þiu berhte sunne: \hld\ só skén þat barn godes,
liuhte is lík-hamo: \hld\ liomon stódun
wánamo fan þemu waldandes barne; \hld\ warð is ge-wádi só hwít
só snéu te sehanne. \hld\ Þó warð þar seld-lík þing
giógid aftar þiu: \hld\ Elias endi Moyses
kwámun þar te Kriste \hld\ wið só kraftagne
wordun wehsljan. \hld\ Þar warð só wun-sam spráka,
só gód word undar gumun, \hld\ þar þe godes sunu
wið þea márjan man \hld\ mahlien welde,
só blíði warð uppan þemu berge: \hld\ skén þat berhte lioht,
was þar gard gód-lík \hld\ endi gróni wang,
paradise gelík. \hld\ Petrus þó gimahalde,
helið hard-módig \hld\ endi te is hérron sprak,
grótte þene godes sunu: \hld\ ʽgód is it hér te wesanne,
ef þu it gikiosan wili, \hld\ Krist alo-waldo,
þat man þi hér an þesaru hóhe \hld\ én hús gewirkja,
már-líko ge-mako \hld\ endi Moysese óðer
endi Eliase þriddja: \hld\ þit is ódas hém,
welono wun-samost.ʼ \hld\ Reht só he þó þat word gesprak,
só tilét þiu luft an twé: \hld\ lioht wolkan skén,
glítandi glímo, \hld\ endi þea gódun man
wliti-skóni bewarp. \hld\ Þó fan þemu wolkne kwam
hélag stemne godes, \hld\ endi þem heliðun þar
selƀo sagde, \hld\ þat þat is sunu wári,
libbiendero lioƀost: \hld\ ʽan þemu mi líkod wel
an mínun hugi-skęftjun. \hld\ Þemu gi hórjen skulun,
fulgangad imu gerno.ʼ \hld\ Þó ni mahtun þea jungaron Kristes
þes wolknes wliti endi word godes,
þea is mikilon maht þea man ant-standen,
ak sie bifellun þó forð-wardes: ferhes ni wándun,
lengiron líƀes. Þó geng im tó þe landes ward,
behrén sie mid is handun héljandero bezt,
hét þat sie im ni andrédin: ʽni skal iu hér derien eowiht,
þes gi hér seldlíkes gisehen habbiad,
mériaro þingo.ʼ Þó eft þem mannun warð
hugi at iro herton endi gihélid mód,
gibade an iro breostun: gisáhun þat barn godes
énna standen, was þat oðer þó,
behliden himiles lioht. Þó giwét imu þe hélago Krist
fan þemu berge niðer; gibód aftar þiu
jungarun sínun, þat sie oƀar Judeono folk
ni sagdin þea gisioni: ʽer þan ik selƀo hér
swíðo diurlíko fan dóðe astande,
aríse fan þeru restu: síðor mugun gi it rekkien forð,
márjen oƀar middil-gard managun þiodun
wído aftar þesaru weroldi.ʼ \hld\ Þó giwét imu waldand Krist
eft an Galileo land, sóhte is gadulingos,
mahtig is mágo hém, sagde þar manages hwat
berhtero biliðeo, endi þat barn godes
þem is sáligun ge-síðun sorgspell ni forhal,
ak he im openlíko allun sagde,
þem is gódun jungarun, hwó ine skolde þat Judeono folk
wégean te wundrun. Þes wurðun þar wíse man
swíðo an sorgun, warð im sér hugi,
hriwig umbi iro herte: gi-hórdun iro hérron þó,
waldandes sunu wordun tęlljen,
hwat he undar þeru þiodu þolojan skolde,
willjendi undar þemu werode. Þó giwét imu waldand Krist,
gumo fan Galilea, sóhte imu Judeono burg,
kwámun im te kafarnaum. Þar fundun sie énan kuninges þegan
wlankan undar þemu werode: kwað þat he wári gi-weldig bodo
aðalkésures; he grótte aftar þiu
Símon Petrusen, kwað þat he wári gi-sęndid þarod,
þat he þar gimanodi manno gehwiliken
þero hóƀid-skatto, þe sie te þemu hoƀe skoldin
tinsi gelden: ʽnis þes tueho énig
gumono nigiénumu, ne sie ina fargelden sán
méðmo kusteon, biúten iuwe méster éno
haƀad it farláten. Ni skal þat líkon wel
mínumu hérron, só man it imu at is hoƀe kúðid,
aðalkésure.ʼ Þó geng aftar þiu
Símon Petrus, welde it sęggjan þó
hérron sínumu: he was is an is hugi iu þan,
giwaro waldand Krist: — imu ni mahte word énig
biholen werðen, he wisse hugi-skęfti
manno gehwilikes —: hét þó þene is márjan þegan,
Símon Petrus an þene séo innen
angul werpen: ʽsuliken só þu þar érist mugis
fisk gifáhenʼ, kwað he, ʽsó teoh þu þene fan þemu flóde te þi,
ant-klemmi imu þea kinni: þar maht þu undar þem kaflon nimen
guldine skattos, þat þu fargelden maht
þemu manne te gimódea mínen endi þínen
tinseo só hwilikan, \hld\ só he ús tó sókid.ʼ
He ni þorfte imu þó aftar þiu \hld\ óðaru wordu
furður gibioden: \hld\ geng fiskari gód,
Símon Petrus, \hld\ warp an þene séo innen
angul an úðeon \hld\ endi up gitóh
fisk an flóde \hld\ mid is folmun twém,
teklóf imu þea kinni \hld\ endi undar þem kaflun nam
guldine skattos: \hld\ dede al, só imu þe godes sunu
wordun gewísde. \hld\ Þar was þó waldandes
megin-kraft gimárid, \hld\ hwó skal allaro manno gehwilik
swíðo willjendi \hld\ is werold-hérron
skuldi endi skattos, \hld\ þea imu giskeride sind,
gerno gelden: \hld\ ni skal ine fargúmon eowiht,
ni farmuni ine an is móde, \hld\ ak wese imu mildi an is hugi,
þiono imu þiolíko: \hld\ an þiu mag he þiodgodes
willjan gewirkjan \hld\ endi ók is werold-hérron
huldi habbjen. \hld\ Só lérde þe hélago Krist
þea is gódon jungaron: \hld\ ʽef énig gumono wið iuʼ, kwað he,
ʽsundja gewirkja, \hld\ þan nim þu ina sundar te þi,
þene rink an rúna \hld\ endi imu is rád saga,
wísi imu mid wordun. \hld\ Ef imu þan þes werð ne sí,
þat he þi gi-hórje, \hld\ hala þi þar óðara tó
gódaro gumono, \hld\ endi lah imu is grimmun werk,
sak ina sóð-wordun. Ef imu þan is sundja aftar þiu,
lóswerk ni léðon, giduo it óðrun liudjun kúð,
mári it þan for menegi endi lát manno filu
witen is farwurhti: óðo beginnad imu þan is werk tregan,
an is hugi hrewen, þan he it gi-hórid heliðo filu,
ahton eldi-barn endi imu is uƀilon dád
weread mid wordun. Ef he þan ók wendien ne wili,
ak farmódat sulika menegi, þan lát þu þene man faren,
haƀa ina þan far héðinen endi lát ina þi an þínumu hugi léðen,
míð is an þínumu móde, ne sí þat imu eft mildi god,
hér heƀen-kuning helpe farlíhe,
fader allaro firiho barno.ʼ Þó frágode Petrus,
allaro þegno bezt þeodan sínan:
ʽhwó oft skal ik þem mannun, þe wið mi habbiad
léðwerk giduan, leoƀo drohtin,
skal ik im siƀun síðun iro sundja aláten,
wréðaro werko, \hld\ ér þan ik is éniga wréka frummje,
léðes te lóne?ʼ \hld\ Þó sprak eft þe landes ward,
an-gegin þe godes sunu \hld\ gódumu þegne:
ʽni sęggju ik þi fan siƀunjun, \hld\ só þu selƀo sprikis,
mahlis mid þínu múðu, \hld\ ik duom þi méra þar tó:
siƀun síðun siƀuntig \hld\ só skalt þu sundja gehwemu,
léðes aláten: \hld\ só willju ik þi te lérun geƀen
wordun wár-fastun. \hld\ Nu ik þi sulika giwald far-gaf,
þat þu mínes híwiskes \hld\ hérost wáris,
manages mann-kunnjes, \hld\ nu skalt þu im mildi wesen,
liudjun líði.ʼ Þó þar te þemu léreande kwam
én iung man angegin endi frágode Jesu Krist:
ʽméster þe gódoʼ, kwað he, ʽhwat skal ik manages duan,
an þiu þe ik heƀen-ríki gehalan móti?ʼ
Habde imu ódwelon allen ge-wunnen,
méðomhord manag, þoh he mildjan hugi
bári an is breostun. Þó sprak imu þat barn godes:
ʽhwat kwiðis þu umbi gódon? nis þat gumono énig
biútan þe éno, þe þar al geskóp,
werold endi wunnja. Ef þu is willjan haƀas,
þat þu an lioht godes líðan mótis,
þan skalt þu bihalden þea hélagon léra,
þe þar an þemu aldon éwa gebiudid,
þat þu man ni slah, ni þu ménes ni sweri,
far-legar-nessi farlát endi luggi gewitskepi,
stríd endi stulina; ne wis þu te stark an hugi,
ne níðin ne hatul, ni nódróf ni fremi;
aƀunst alla farlát; wis þínun eldirun gód,
fader endi móder, endi þínun friundun hold,
þem náhistun gináðig. Þan þu þi giniodon móst
himilo ríkjas, ef þu it bihalden wili,
fulgangan godes lérun.ʼ Þó sprak eft þe iungo man
ʽal hębbju ik só giléstidʼ, kwað he, ʽsó þu mi léris nu,
wordun wísis, só ik is eo wiht ni farlét
fan mínero kindiski.ʼ Þó bigan ina Krist sehan
an mid is ógun: ʽén is þar noh nuʼ, kwað he,
ʽwan þero werko: ef þu is willjon haƀas,
þat þu þurhfremid þionon mótis
hérron þínumu, þan skalt þu þat þín hord nimen,
skalt þínan ódwelon allan farkópien,
diurie méðmos, endi délien hét
armun mannun: þan haƀas þu aftar þiu
hord an himile; kum þi þan gihalden te mi,
folgo þi mínaro ferdi: þan haƀas þu friðu síður.ʼ
Þó wurðun Kristes word kindiungumu manne
swíðo an sorgun, was imu sér hugi,
mód umbi herte: habde méðmo filu,
welono ge-wunnen; wende imu eft þanen,
was imu unóðo innan breostun,
an is seƀon swáro. Sah imu aftar þó
Krist alo-waldo, kwað it þó, þar he welde,
te þem is jungarun geginwardun, þat wári an godes ríki
unóði ódagumu manne up te kumanne:
ʽóður mag man olbundeon, þoh he sí unmet grót,
þurh náðlan gat, þoh it sí naru swíðo,
sáftur þurhslópien, þan mugi kuman þiu siole te himile
þes ódagan mannes, þe hér al haƀad
giwendid an þene weroldskat willjon sínen,
mód-gi-þáhti, endi ni hugid umbi þie maht godes.ʼ
Imu and-wordiade érþungan gumo,
Símon Petrus, endi sęggjan bad
leoƀan hérron: ʽhwat skulun wi þes te lóne nimenʼ, kwað he,
ʽgódes te gelde, þes wi þurh þín jungardóm
égan endi erƀi al farlétun
hoƀos endi híwiski endi þi te hérron gikurun,
folgodun þínaru ferdi: hwat skal ús þes te frumu werðen,
langes te lóne?ʼ liudjo drohtin
sagde im þó selƀo: ʽþan ik sittjen kumuʼ, kwað he,
ʽan þie mikilan maht an þemu márjan dage,
þar ik allun skal irmin-þiodun
dómos adélien, þan mótun gi mid iuwomu drohtine þar
selƀon sittjen endi mótun þera saka waldan:
mótun gi Israhelo eðilifolkun
adélien aftar iro dádjun: só mótun gi þar gidiuride wesen.
Þan sęggju ik iu te wáran: só hwe só þat an þesaru weroldi giduot,
þat he þurh mína minnja mágo gesidli
liof farlétid, þes skal hi hér lón niman
tehan síðun tehinfald, ef he it mid trewon duot,
mid hluttru hugi. Oƀar þat haƀad he ók himiles lioht,
open éwig líf.ʼ Bigan imu þó aftar þiu
allaro barno bezt én biliði sęggjan,
kwað þat þar én ódag man an ér-dagun
wári undar þemu werode: ʽþe habde welono genóg,
sinkas gisamnod endi imu simlun was
garu mid goldu endi mid godowebbiu,
fagarun fratahun endi imu so filu habde
gódes an is gardun endi imu at gómun sat
allaro dago gehwilikes: habde imu diurlík líf,
blíðsea an is bęnkjun. Þan was þar eft én biddjendi man,
giléƀod an is lík-hamon, Lazarus was he héten,
lag imu dago gehwilikes at þem durun foren,
þar he þene ódagan man inne wisse
an is gestseli góme þiggean,
sittjen at sumble, endi he simlun béd
giarmod þar úte: ni móste þar in kuman,
ne he ni mahte gebiddjen, þat man imu þes bródes þarod
gidragan weldi, þes þar fan þemu diske niðer
antfel undar iro fóti: ni mahte imu þar énig fruma werðen
fan þemu héroston, þe þes húses gi-weld, \hld\ biútan þat þar gengun is hundos tó,
likkodun is lík-wundon, \hld\ þar he liggiandi
hungar þolode; \hld\ ni kwam imu þar te helpu wiht
fan þemu ríkjon manne. \hld\ Þó gi-fragn ik þat ina is regano-gi-skapu,
þene armon man \hld\ is én-dago
gi-manoda mahtiun swíð, \hld\ þat he manno dróm
a-geƀen skolde. \hld\ Godes ęngilos
ant-fengun is ferh \hld\ endi léddun ine forð þanen,
þat sie an Abrahames barm \hld\ þes armon mannes
siole gi-settun: \hld\ þar móste he simlun forð
wesen an wunnjun. \hld\ Þó kwámun ók wurde-gi-skapu,
þemu ódagan man \hld\ or-lag-hwíle,
þat he þit lioht farlét: \hld\ léða wihti
be-sinkodun is siole \hld\ an þene swarton hel,
an þat fern innen fíundun te willjan,
begróƀun ine an gramono hém. Þanen mahte he þene gódan skawon,
Abraham gesehen, þar he uppe was
líƀes an lustun, endi Lazarus sat
blíði an is barme, berht lón ant-feng
allaro is armódio, endi lag þe ódago man
héto an þeru hęllju, hriop up þanen:
ʽfader Abrahamʼ, kwað he, ʽmi is firinun þarf,
þat þu mi an þínumu mód-seƀon mildi werðes,
líði an þesaru lognu: sendi mi Lazarus herod,
þat he mí ge-fórja an þit fern innan
kaldes wateres. Ik hér kwik brinnu
héto an þesaru hęllju: nu is mi þínaro helpono þarf,
þat he mi aleskie mid is luttikon fingru
tungon míne, nu siu tékan haƀad,
uƀil arƀedi. Inwid-rádo,
léðaro spráka, alles is mi nu þes lón kumen.ʼ
Imu and-wordiade þó Abraham —þat was aldfader—:
ʽgehugi þu an þínumu hertonʼ, kwað he, ʽhwat þu habdes iu
welono an weroldi. Huat, þu þar alle þíne wunnja farsliti,
gódes an gardun, só hwat só þi giƀiðig forð
werðen skolde. Wíti þolode
Lazarus an þemu liohte, habde þar léðes filu,
wíteas an weroldi. Beþiu skal he nu welon égan,
libbien an lustun: þu skalt þea logna þolan,
brinnendi fiur: ni mag is þi énig bóte kumen
hinana te hęllju: it haƀad þe hélago god
só gi-fastnod mid is faðmun: ni mag þar faren énig
þegno þurh þat þiustri: it is hér só þikki undar ús.ʼ
Þó sprak eft Abrahame þe erl tegegnes
fan þeru hétan hell endi helpono bad,
þat he Lazarus an liudjo dróm
selƀon sandi: ʽþat he gesęggja þar
bróðarun mínun, hwó ik hér brinnendi
þráwerk þolon; si þar undar þeru þiodu sind,
si fíƀi undar þemu folke: ik an forhtun bium,
þat sie im þar farwirkjen, þat sie skulin ók an þit wíti te mi,
an só grádag fiur.ʼ Þó imu eft tegegnes sprak
Abraham aldfader, kwað þat sie þar éo godes
an þemu land-skępi, liudi habdin,
Moyseses gibód endi þar managaro tó
wár-saguno word: ʽef sie is willige sind,
þat sie þat bihalden, þan ni þurƀun sie an þea hell innen,
an þat fern faren, ef sie gefrummjad só,
só þea gebiodad, þe þea bók lesat
þem liudjun te lérun. Ef sie þes þan ni willjad léstien wiht,
þanne ni hórjad sie ók þemu þe hinan astád,
man fan dóðe. Láte man sie an iro mód-seƀon
selƀon keosen, hweðer im swótiera þunkie
te giwinnanne, só lango só sie an þesaru weroldi sind,
þat sie eft uƀil etþa gód aftar habbjen.ʼʼ
Só lérde he þó þea liudi liohton wordon,
allaro barno bezt, endi biliði sagde
manag man-kunnje mahtig drohtin,
kwað þat imu én sálig gumo samnon bigunni
man an morgen, ʽendi im méda gihét,
þe hérosto þes híwiskjas, swíðo *holdlík lónʼ,
kwat þat hie iro allaro gihwem énna gáƀi
siloƀrinna skat. ʽÞuo samnodun managa
weros an is wín-gardon, — endi hie im werk bifalah —
ádro an úhtan. Sum kwam þar ók an undorn tuo,
sum kwam þar an middjan dag, man te þem werke,
sum kwam þar te nónu, þuo was þiu niguða tíd
sumarlanges dages; sum þar ók síðor kwam
an þia elliftun tíd. Þuo geng þar áƀand tuo,
sunna ti sedle. Þuo hie selƀo gibód
is ambahtion, erlo drohtin,
þat man þero manno gihwem is meoda forguldi,
þem erlon arƀidlón; hiet þiem at érist geƀan.
þia þar at lezt wárun, liudi kumana,
weros te þem werke, endi mid is wordon gibód,
þat man þem mannon iro mieda forguldi
alles at aftan, þem þar kwámun at érist tuo
willendi te þem werke. Wándun sia swíðo,
þat man im méra lón gimakod habdi
wið iro araƀedie: þan man im allon gaf,
þem liudjon gilíko. Léð was þat swíðo,
allon þem ando, þem þar kwámun at érist tuo:
ʽwi kwámun hier an moraganʼ, kwáðun sia, ʽendi þolodun hier manag te dage
araƀidwerko, hwílon unmet hét,
skínandia sunna: nu ni giƀis þu ús skattes þan mér,
þie þu þem óðron duos, þia hier éna hwíla
wáron an þínon werke.ʼ Þuo habda eft is word garo
þie hérosto þes híwiskes, kwat þat hie im ni habdi gihétan þan mér
werðes wið iro werke: ʽhwat, ik giwald hębbjuʼ, kwaþie,
ʽþat ik iu allon gilíko muot lón forgeldan,
iwes werkes werð.ʼ Þan waldandi Krist
ménda im þoh méra þing, þoh hie oƀar þat manno folk
fan þem wín-gardon só wordon spráki,
hwó þar unefno \hld\ erlos kwámun,
weros te þem werke. \hld\ Só skulun fan þero weroldi duon
mann-kunnjes barn an þat márjo lioht,
gumon an godes wang: sum biginnit ina giriwan sán
an is kindiski, haƀit im gikoranan muod,
willjon guodan, weroldsaka míðit,
farlátit is lusta; ni mag ina is lík-hamo
an unspuod forspanan: spáhiða línot,
godes éu, gramono for-látit,
wréðaro willjon, duot im só te is weroldi forð,
léstit só an þeson liohte, antþat im is líƀes kumit,
aldres áƀand; giwítit im þan up-wegos:
þar wirðit im is araƀedi all gilónot,
fargoldan mid guodu an godes ríkje.
Þat méndun þia wuruhteon, þia an þem wín-gardon
ádro an úhta arƀidlíko
werk bigunnun endi þurwuonodun forð,
erlos unt áƀand. Sum þar ók an undern kwam,
habda þuo farmerrid, þia moraganstunda
þes dagwerkes forduolon; só duot doloro filo,
gimédaro manno: dríƀit im mislík þing
gerno an is iuguði, — haƀit im gelpkwidi
léða gilínot endi lósword manag —,
antþat is kindiski farkuman wirðit,
þat ina after is iuguði godes anst manot
blíði an is brioston; fáhit im te beteron þan
wordon endi werkon, lédit im is werold mid þiu,
is aldar ant þena endi: kumit im alles lón
an godes ríkje, gódaro werko.
Sum mann þan midfiri mén farlátid,
swára sundjun, fáhit im an sálig þing,
biginnit im þuru godes kraft guodaro werko,
buotit balospráka, látit im is bittrun dád
an is hugje hrewan; kumit im þiu helpa fon gode,
þat im giléstid þie gilóƀo, só lango só im is líf warod;
farit im forð mid þiu, antfáhit is mieda,
guod lón at gode; ni sindun éniga geƀa beteran.
Sum biginnit þan ók furðor, þan hie ist fruodot mér,
is aldares af-heldit, — þan biginnat im is uƀilon werk
léðon an þeson liohte, þan ina léra godes
gimanod an is muode: wirðit im mildera hugi,
þurugengit im mid guodu endi geld nimit,
hóh himil-ríki, þan hie hinan wendit,
wirðit im is mieda só sama, só þem man *nun warð,
þea þar te nónu dages, an þea nigunda tíd,
an þene wín-gardon wirkjan kwámun.
Sum wirðid þan só swíðo gefródot, só he ni wili is sundja bótjen,
ak he ókid sie mid uƀilu gehwiliku, antat imu is áƀand náhid,
is werold endi is wunnja farslítid; þan beginnid he imu wíti andréden,
is sundjon werðad imu sorga an móde: gehugid hwat he selƀo gefrumide
grimmes þan lango, þe he móste is iuguðeo neoten; ni mag þan mid óðru gódu gibótjen
þea dádi, þea he só derƀea gefrumide, ak he slehit allaro dago gehwilikes
an is breost mid béðiun handun endi wópit sie mid bittrun trahnun,
hlúdo he sie mid hofnu kúmid, bidid þene hélagon drohtin
mahtigne, þat he imu mildi werðe: ni látid imu síðor is mód gitwíflien;
só égrohtful is, þe þar alles geweldid: he ni wili énigumu irminmanne
farwernien willjan sínes; far-giƀid imu waldand selƀo
hélag himil-ríki: þan is imu giholpen síður.
Alle skulun sie þar éra antfáhen, þoh sie þarod te énaru tídi
ni kumen, þat kunni manno, þoh wili imu þe kraftigo drohtin,
gilónon allaro liudjo só hwilikumu, só hér is gilóƀon antfáhit:
én himil-ríki giƀid he allun þeodun,
mannun te médu. Þat ménde mahtig Krist,
barno þat bezte, þó he þat biliði sprak,
hwó þar te þem wín-gardun wurhtjon kwámin,
man mis-líko: \hld\ þoh nam is méde gehwe
fulle te is frójan. \hld\ Só skulun firiho barn
at gode selƀumu \hld\ geld ant-fáhen,
swíðo leoƀ-lík lón, \hld\ þoh sie sume só late werðan.
Hét imu þó þea is gódan \hld\ jungaron náhor
tweliƀi gangan —þea wárun imu triuwiston
man oƀar erðu—, \hld\ sagde im mahtig selƀo
óðer-síðu, \hld\ hwilik imu þar arƀedi
tóward wárun: \hld\ ʽþes ni mag énig tueho werðenʼ, kwað he;
kwað þat sie þó te Hierusalem \hld\ an þat Judeono folk
líðan skoldin: \hld\ ʽþar wirðid all giléstid só,
gefrumid undar þemu folke, só it an furn-dagun
wíse man be mi \hld\ wordun gesprákun.
Þar skulun mi far-kópon \hld\ undar þea kraftigon þiod,
heliðos te þeru héri; \hld\ þar werðat mína hendi ge-bundana,
faðmos werðad mi þar gefastnod; \hld\ filu skal ik þar gi-þolojan,
hoskes gi-hórjen \hld\ endi harm-kwidi,
bismerspráka \hld\ endi bi-hét-word manag;
sie wégeat mi te wundron \hld\ wápnes ęggjun,
bilósjad mi líƀu: \hld\ ik te þesumu liohte skal
þurh drohtines kraft \hld\ fan dóðe astanden
an þriddjon dage. \hld\ Ni kwam ik undar þesa þeoda herod
te þiu, þat mín eldi-barn \hld\ arƀed habdin,
þat mi þionodi þius þiod: \hld\ ni willju ik is sie þiggjen nu,
fergon þit folk-skępi, \hld\ ak ik skal imu te frumu werðen,
þeonon imu þeo-líko \hld\ endi for alla þesa þeoda geƀen
seole míne. \hld\ Ik willju sie selƀo nu
lósjen mid mínu líƀu, þea hér lango bidun,
man-kunnjes manag, mínara helpa.ʼ
Fór imu þó forð-wardes — habde imu fasten hugi,
blíðean an is breostun barn drohtines —
welda im te Hierusalem Judeo folkes
willjon wísan: he konste þes werodes só garo
hetigrimmen hugi endi hardan stríd,
wréðan willjon. Werod síðode
furi Hierikhoburg; was þe godes sunu,
mahtig undar þero menigi. Þar sátun twénie man bi wege,
blinde wárun sie béðie: was im bótono þarf,
þat sie gehéldi heƀenes waldand,
hwand sie só lango liohtes þolodun,
managa hwíla. Sie gi-hórdun þó þat megin faren
endi frágodun sán firiwitlíko
reginiblindun, hwilik þar ríki man
undar þemu folk-skępi furista wári,
hérost an hóƀid. Þó sprak im én helið angegin,
kwað þat þar Hiesu Krist fan Galilea-lande,
héljandero bezt hérost wári,
fóri mid is folku. Þó warð fráhmód hugi
béðiun þem blindun mannun, þó sie þat barn godes
wissun under þemu werode: hreopun im þó mid iro wordun tó,
hlúdo te þemu hélagon Kriste, bádun þat he im helpe gerédi:
ʽdrohtin Dawides sunu: wis ús mid þínun dádjun mildi,
neri ús af þesaru nódi, só þu ginóge dós
manno kunnjes: þu bist managun gód,
hilpis endi hélis.ʼ Þo bigan im þat heliðo folk
werien mid wordun, þat sie an waldand Krist
só hlúdo ni hriopin. Si ni weldun im hórjen te þiu,
ak sie simla mér endi mér oƀar þat manno folk
hlúdo hreopun. héljand gestód,
allaro barno bezt, hét sie þó brengien te imu,
lédjen þurh þea liudi, sprak im listiun tó
mildlíko for þeru menegi: ʽhwat willjad git mínaro hérʼ, kwað he,
ʽhelpono habbjen?ʼ Sie bádun ina hélagna,
þat he im ira ógon opana gidádi,
farliwi þeses liohtes, þat sie liudjo dróm,
suikle sunnun skín gisehen móstin,
wlitiskónie werold. Waldand frumide,
hrén sie þó mid is handun, dede is helpe þar tó,
þat þem blindun þó béðium wurðun
ógon gioponod, þat sie erðe endi himil
þurh kraft godes \hld\ ant-kiennien mahtun,
lioht endi liudi. \hld\ Þó sagdun sie lof gode,
diurdun úsan drohtin, \hld\ þes sie dages liohtes
brúkan móstun: \hld\ gewitun im béðie mid imu,
folgodun is ferdi: \hld\ was im þiu fruma giƀiðig,
endi ók waldandes werk \hld\ wído gekúðid,
managun gimárid. \hld\ Þar was só mahtiglík
biliði gibóknid, \hld\ þar þe blindon man
bi þemu wege sátun, \hld\ wíti þolodun,
liohtes lóse: \hld\ þat ménid þoh liudjo barn,
al man-kunni, \hld\ hwó sie mahtig god
an þemu ana-ginne \hld\ þurh is énes kraft
sinhíun twé \hld\ selƀo giwarhte,
Ádam endi Éwan: \hld\ far-gaf im up-wegos,
himilo ríki; \hld\ ak þó warð im þe hatola te náh,
fíund mid féknu \hld\ endi mid firin-werkun,
bi-swék sie mid sundjun, \hld\ þat sie sin-skóni,
lioht farlétun: \hld\ wurðun an léðaron stedi,
an þesen middil-gard man farworpen,
þolodun hér an þiustriu þiodarƀedi,
wunnun wrak-síðos, welon þarƀodun:
fargátun godes ríkjes, gramon þeonodun,
fíundo barnun; sie guldun is im mid fiuru lón
an þeru héton hęllju. Beþiu wárun siu an iro hugi blinda
an þesaru middil-gard, menniskono barn,
hwand siu ine ni ant-kiendun, kraftagne god,
himilisken hérron, þene þe sie mid is handun giskóp,
giwarhte an is willjon. Þius werold was þó só far-hwerƀid,
bi-þwungen an þiustrie, an þiodarƀidi,
an dóðes dalu: sátun im þó bi þeru drohtines strátun
iámar-móde, \hld\ godes helpe bidun:
siu ni mahte im þó ér werðen, \hld\ ér þan waldand god
an þesan middil-gard, \hld\ mahtig drohtin,
is selƀes sunu \hld\ sęndjen weldi
þat he lioht ant-luki \hld\ liudjo barnun,
oponodi im éwig líf, þat sie þene alo-waldon
mahtin ant-kęnnjen wel, kraftagna god.
Ôk mag ik giu gitęlljen, of gi þar tó willjad
huggien endi hórjen, þat gi þes héljandes mugun
kraft ant-kęnnjen, hwó is kumi wurðun
an þesaru middil-gard managun te helpu,
ia hwat he mid þem dádjun drohtin selƀo
manages ménde, ia behwiu þiu márje burg
Hierikho hétid, þiu þar an Judeon stád
gimakod mid múrun: þiu is aftar þemu mánen ginemnid,
aftar þemu torhten tungle: he ni mag is tídi bemíðen,
ak he dago gehwilikes duod óðerhweðer,
wanod ohþo wahsid. Só dód an þesaro weroldi hér,
an þesaru middil-gard menniskono barn:
farad endi folgod, fróde sterƀad,
werðad eft iunga aftar kumane,
weros awahsane, unttat sie eft wurd farnimid.
Þat ménde þat barn godes, þó he fon þeru burgi fór,
þe gódo fan Hierikho, þat ni mahte ér werðen gumono barnun
þiu blindia gibótid, þat sie þat berhte lioht,
gi-sáhin sin-skóni, \hld\ ér þan he selƀo hér
an þesaru middil-gard \hld\ menniski ant-feng,
flésk endi lík-hamon. \hld\ Þó wurðun þes firiho barn
giwar an þesaru weroldi, \hld\ þe hér an wítje ér,
sátun an sundjun \hld\ gisiunies lóse,
þolodun an þiustrie, \hld\ —sie af-sóƀun þat was þesaru þiod kuman
héljand te helpu \hld\ fan heƀen-ríkje,
Krist allaro kuningo best; \hld\ sie mahtun is ant-kęnnjen sán,
gifólien is fardio. \hld\ Þó sie só filu hriopun,
þe man te þemu mahtigon gode, \hld\ þat im mildi aftar þiu
waldand wurði. \hld\ Þan weridun im swíðo
þia swárun sundjon, \hld\ þe sie im ér selƀon gidádun,
lettun sie þes gilóbon. \hld\ Sie ni mahtun þem liudjun þoh
biwerien iro willjon, \hld\ ak sie an waldand god
hlúdo hriopun, \hld\ antat he im iro héli far-gaf,
þat sie sin-líf \hld\ gi-sehen móstin,
open éwig lioht \hld\ endi an faren
an þiu berhtun bú. \hld\ Þat méndun þea blindun man,
þe þar bi Hierikho-burg \hld\ te þemu godes barne
hlúdo hriopun, \hld\ þat he im iro héli farlihi,
liohtes an þesumu líƀe: \hld\ þan im þea liudi só filu
weridun mid wordun, \hld\ þea þar an þemu wege fórun
bi-foren endi bi-hinden: \hld\ só dót þea firin-sundjon
an þesaru middil-gard man\hld\ -kunnje.
hórjad nu hwó þie blindun, \hld\ síður im gibótid warð,
þat sie sunnun lioht \hld\ ge-sehen móstun,
hwó si þó dádun: \hld\ ge-witun im mid iro drohtine samad,
folgodun is ferdi, \hld\ sprákun filu wordo
þemu landes hirdie te loƀe: \hld\ só dód im noh liudjo barn
wído aftar þesaru weroldi, \hld\ síður im waldand Krist
ge-liuhte mid is lérun \hld\ endi im líf éwig,
godes ríki far-gaf \hld\ gódun mannun,
hóh himiles lioht \hld\ endi is helpe þar tó,
só hwemu só þat giwerkod, \hld\ þat he móti þemu is wege folgon.
Þó náhide \hld\ nęrjendo Krist,
þe gódo te Hierusalem. \hld\ Kwam imu þar te-gegnes filu
werodes an willjon \hld\ wel huggendies,
ant-fengun ina fagaro \hld\ endi imu bi-foren streidun
þene weg mid iro gi-wádjun \hld\ endi mid wurtjun só same,
mid berhtun blómun \hld\ endi mid bómo tógun,
þat feld mid fagaron palmun, \hld\ al só is fard ge-buride,
þat þe godes sunu \hld\ gangan welde
te þeru márjan burg. \hld\ Hwarf ina megin umbi
liudjo an lustun, \hld\ endi lof-sang ahóf
þat werod an willjon: \hld\ sagdun waldande þank,
þes þar selƀo kwam \hld\ sunu Dawides
wíson þes werodes. \hld\ Þó gesah waldand Krist
þe gódo te Hierusalem, \hld\ gumono bezta,
blíkan þene burges wal \hld\ endi bú Judeono,
hóha horn-sęli \hld\ endi ók þat hús godes,
allaro wího wun-samost. \hld\ Þó wel imu an innen
hugi wið is herte: \hld\ þó ni mahte þat hélage barn
wópu awísien, \hld\ sprak þó wordo filu
hriwig-líko \hld\ —was imu is hugi séreg—:
ʽwé warð þi, Hierusalemʼ, \hld\ kwað he, ʽþes þu te wárun ni wést
þea wurde-gi-skęfti, \hld\ þe þi noh gi-werðen skulun,
hwó þu noh wirðis behabd \hld\ hęrjes kraftu
endi þi bi-sittjad \hld\ slíð-móde man,
fíund mid folkun. \hld\ Þan ni haƀas þu friðu hwergin,
mundburd mid mannun: \hld\ lédiad þi hér manage tó
ordos endi ęggja, \hld\ orlegas word,
farfioþ þín folk-skępi \hld\ fiures liomon,
þese wíki awóstiad, \hld\ wallos hóha
felliad te foldun: \hld\ ni afstád is felis nigiean,
stén oƀar óðrumu, \hld\ ak werðad þesa stedi wóstia
umbi Hierusalem \hld\ Judeo liudjo,
hwand sie ni ant-kęnniad, \hld\ þat im kumana sind
iro tídi tówardes, \hld\ ak sie habbiad im twíflien hugi,
ni witun þat iro wísad \hld\ waldandes kraft.ʼ
Gi-wét imu þó mid þeru menegi \hld\ manno drohtin
an þea berhton burg. \hld\ Só þó þat barn godes
innan Hierusalem \hld\ mid þiu gumono folku,
ség mid þiu ge-síðu, \hld\ þó warð þar allaro sango mést,
hlúd stemnie af-haƀen \hld\ hélagun wordun,
loƀodun þene landes ward \hld\ liudjo menegi,
barno þat bezte; \hld\ þiu burg warð an hróru,
þat folk warð an forhtun \hld\ endi frágodun sán,
hwe þat wári, \hld\ þat þar mid þiu werodu kwam,
mid þeru mikilon menegi. \hld\ Þó sprak im én man angegin,
kwað þat þar Hiesu Krist \hld\ fan Galileo lande,
fan Nazareth-burg \hld\ neriand kwámi,
witig wár-sago þemu werode te helpu.
Þó was þem Judiun, þe imu ér grame wárun,
unholde an hugi, harm an móde,
þat imu þea liudi só filu \hld\ lof-sang warhtun,
diurdun iro drohtin. \hld\ Þó gengun dolmóde,
þat sie wið waldand Krist \hld\ wordun sprákun,
bádun þat he þat ge-síði swígon héti,
letti þea liudi, \hld\ þat sie imu lof só filu
wordun ni warhtin: \hld\ ʽit is þesumu werode léðʼ, kwáðun sie,
ʽþesun burgliudjun.ʼ \hld\ Þó sprak eft þat barn godes:
ʽef gi sie amerriadʼ, \hld\ kwað he, ʽþat hér ni mótin manno barn
waldandes kraft \hld\ wordun diurien,
þan skulun it hrópen þoh \hld\ harde sténos
for þesumu folk-skępi, \hld\ felisos starka,
ér þan it eo belíƀe, \hld\ neƀo man is lof spreke
wído aftar þesaru weroldi.ʼ \hld\ Þó he an þene wíh innen,
geng an þat godes hús: \hld\ fand þar Judeono filu,
mislíke man, \hld\ manage atsamne,
þea im þar kóp-stędi \hld\ gi-koran habdun,
mangodun im þar mid manages hwí: \hld\ muniterias sátun
an þemu wíhe innan, \hld\ habdun iro wesl gidago
garu te geƀanne. \hld\ Þat was þemu godes barne
al an andun: \hld\ dréf sie ut þanen
rúmo fan þemu rakude, \hld\ kwað þat wári rehtara dád,
þat þar te bedu fórin \hld\ barn Israheles
ʽendi an þesumu mínumu húse \hld\ helpono biddjan,
þat sia sigidrohtin \hld\ sundjono tuomie,
þan hér þeoƀas an þing-stędi halden,
þea farwarhton weros wehsal dríƀan,
unreht énfald. Ne gi éniga éra ni witun
þeses godes húses, Judeo liudi.ʼ
Só rúmde he þó endi rekode, ríki drohtin,
þat hélaga hús endi an helpun was
managumu man-kunnje, þem þe is mikilon kraft
ferrene gefrugnun endi þar gifaran kwámun
oƀar langan weg. Warð þar léf so manag,
halt gihélid endi háf só same,
blindun gibótid. \hld\ Só dede þat barn godes
willjendi þemu werode, \hld\ hwand al an is gi-weldi stéd
umbi þesaro liudjo líf \hld\ endi ók umbi þit land só same.
Stód imu þó fora þemu wíhe \hld\ waldandeo Krist,
liof landes ward, \hld\ endi imu þero liudjo hugi,
iro willjon aftar-warode: \hld\ gi-sah werod mikil
an þat márje hús \hld\ méðmos fórjen,
geƀon mid goldu \hld\ endi mid godu-wębbju,
diuriun fratahun. \hld\ Þat al drohtin Krist
warode wís-líko. \hld\ Þó kwam þar ók én widowa tó,
idis arm-skapen, \hld\ endi te þemu alaha geng
endi siu an þat tresur-hús \hld\ twéne legde
éríne skattos: \hld\ was iru énfald hugi,
willjan gódes. \hld\ Þó sprak waldand Krist,
þe gumo wið is giungaron, \hld\ kwað þat siu þar geƀa bráhti
méron mikilu þan elkor \hld\ énig mannes sunu:
ʽef hér ódaga manʼ, \hld\ kwað he, ʽéra bráhtun,
méðom-hord manag, \hld\ sie létun im mér at hús
welona ge-wunnen. \hld\ Ni dede þius widowa só,
ak siu te þesumu alahe gaf \hld\ al þat siu habde
welono ge-wunnen, \hld\ só siu iru wiht ni farlét
gódes an iro gardun. Beþiu sind ira geƀa méron,
waldande werða, hwand siu it mid sulikumu willjon dede
te þesumu godes húse. Þes skal siu geld niman,
swíðo lang-sam lón, þes siu sulikan gilóƀon haƀad.ʼ
Só gi-fragn ik þat þar an þemu wíhe waldandeo Krist
allaro dago gehwilikes, drohtin manno,
wísde mid wordun. Stód ine werod umbi,
grót folk Judeono, gi-hórdun is gódan word,
swótja sęggjan. Sum só sálig warð
manno undar þeru menegi, þat it bigan an is mód hladen;
línodun im þea léra, þe þe landes ward
al be biliðiun sprak, barn drohtines.
Sumun wárun eft so léða léra Kristes,
waldandes word: was im wiðer-mód hugi
allun þem, þe an þemu hęri-skępi hérost wárun,
furiston an þemu folke: fáres hugdun
wréða mid iro wordun — habdun im wiðer-sakon
gihaloden te helpu, þes héroston man,
Erodeses þegan, þe þar andward stód
wréðes willjan, þat he iro word oƀarhórdi —
ef sie ina forfengin, þat sie ina þan feteros an,
þea liudi liðobendi leggien móstin,
sundja lósan. Þó gengun im þea ge-síðos tó
bittra gihugde, þat sie wið þat barn godes,
wréða wiðer-sakon wordun sprákun:
ʽhwat, þu bist éosagoʼ, kwáðun sie, ʽallun þiodun,
wísis wáres só filu: nis þi werðeowiht
te bimíðanne manno niénumu
umbi is ríkidóm, neƀo þu simlun þat reht sprikis
endi an þene godes weg gumono ge-síði
lédis mid þinun lérun: ni mag þi laster man
fíðan undar þesumu folke. Nu wi þi frágon skulun.
ríki þiodan, hwilik reht haƀad
þe késur fan Rúmu, þe imu te þesumu kunnje herod
tinsi sókid endi gitald haƀad,
hwat wi imu gelden skulin géro ge-hwilikes
hóƀid-skatto. Saga hwat þi þes an þínumu hugi þunkja:
is it reht þe nis? Rád for þínun
landmégun wel: ús is þínaro lérono þarf.ʼ
Sie weldun þat he it ant-kwáði: þan mahte he þoh ant-kęnnjen wel
iro wréðon willjon: ʽte hwí gi wár-logonʼ, kwað he,
ʽfandot mín só frókno? Ni skal iu þat te frumu werðen,
þat gi dreogerias darnungo nu
willjad mi farfáhen.ʼ \hld\ Hét he þó forð dragan
te skawonne þe skattos, ʽþe gi skuldige sind
an þat geld geƀen.ʼ \hld\ Judeon drógun
énna siluƀrinna forð: \hld\ sáhun manage tó,
hwó he was gemunitod: \hld\ was an middien skín
þes késures biliði \hld\ —þat mahtun sie ant-kęnnjen wel—,
iro hérron hóƀid-mál. \hld\ Þó frágode sie þe hélago Krist,
aftar hwemu þiu ge-líknessi \hld\ gi-legid wári.
Sie kwáðun þat it wári \hld\ werold-késures
fan Rúmu-burg, \hld\ ʽþes þe alles þeses ríkes haƀad
gewald an þesaru weroldi.ʼ \hld\ ʽÞan willju ik iu te wárun hérʼ, kwað he,
ʽselƀo sęggjan, \hld\ þat gi imu sín geƀad,
werold-hérron is ge-wunst, \hld\ endi waldand gode
sęlljad, þat þar sín ist: \hld\ þat skulun iuwa seolon wesen,
gumono géstos.ʼ \hld\ Þó warð þero Judeono hugi
geminsod an þemu mahle: \hld\ ni mahtun þe mén-skaðon
wordun ge-winnen, \hld\ só iro willjo geng,
þat sie ina far-fengin, \hld\ hwand imu þat friðu-barn godes
wardode wið þe wréðon \hld\ endi im wár an-gegin,
sóð-spel sagde, \hld\ þoh sie ni wárin só sálige te þiu,
þat sie it só far-fengin, \hld\ só it iro fruma wári.
Sie ni weldun it þoh far-láten, \hld\ ak hétun þar lédjen forð
én wíf for þemu werode, \hld\ þiu habde wam ge-frumid,
un-reht én-fald: \hld\ þiu idis was bi-fangen
an far-legar-nessi, \hld\ was iro líƀes skolo,
þat sie firiho barn \hld\ ferahu bi-námin,
éhtin iro aldres: \hld\ só was an iro éu ge-skriƀen.
Sie bi-gunnun ina þó frágon, \hld\ fruokne liudi,
wréða mid iro wordun, \hld\ hwat sie skoldin þemu wíƀe duan,
hweðer sie sie kwelidin, \hld\ þe sie sie kwika létin,
þe hwat he umbi sulika dádi \hld\ adéljen weldi:
ʽþu wést, hwó þesaru menegiʼ, \hld\ kwáðun sie, ʽMoyses gibód
wárun wordun, \hld\ þat allaro wíƀo ge-hwilik
an far-legar-nessi \hld\ líƀes far-warhti
endi þat sie þan a-wurpin \hld\ weros mid handun,
starkun sténun: \hld\ nu maht þu sie sehan standen hér
an sundjun bi-fangan: \hld\ saga hwat þu is willjes.ʼ
weldun ine þea wiðer-sakon wordun farfáhen,
ef he þat gikwáði, þat sie sie kwika létin,
friðodi ira ferahe, þan weldi þat folk Judeono
kweðen, þat he iro aldiron éo wiðer-sagdi,
þero liudjo land-reht; ef he sie þan héti líƀu bi-nimen,
þea magað fur þeru menegi, þan weldin sie kweðen, þat he só mildjene hugi
ni bári an is breostun, \hld\ só skoldi habbjen barn godes:
weldun sie só hweðeres \hld\ hélagne Krist
þero wordo ge-wítnon, \hld\ só he þar for þemu werode ge-spráki,
a-déldi te dóme. \hld\ Þan wisse drohtin Krist
þero manno só garo \hld\ mód-gi-þáhti,
iro wréðon willjon; \hld\ þó he te þemu werode sprak,
te allun þem erlun: \hld\ ʽsó hwilik só iuwar áno síʼ, kwað he,
ʽslíðja sundjon, \hld\ só ganga iru selƀo tó
endi sie at érist \hld\ erl mid is handun
stén ana werpe.ʼ \hld\ Só stódun Judeon,
þáhtun endi þagodun: ni mahte þegan nigiean
wið þem wordkwidi wiðer-saka finden:
gehugde manno gehwilik mén-giþáhti,
is selƀes sundja: ni was iro só sikur énig,
þat he bi þemu worde þemu wíƀe gedorsti
stén an werpen, ak létun sie standen þar
énan þar inne endi im út þanen
gengun gramharde Judeo liudi,
én aftar óðrumu, antat iro þar énig ni was
þes fíundo folkes, þe iro ferhes þó,
þeru idis aldar-lago áhtien weldi.
Þó gi-fragn ik þat sie frágode friðu-barn godes,
allaro gumono bezt: ʽhwar kwámun þit Judeono folkʼ, kwað he,
ʽþine wiðer-sakon, þea þi hér wrógdun te mi?
Ne sie þi hiudu wiht harmes ne gidádun,
þea liudi léðes, þe þi weldun líƀu beniman,
wégean te wundrun?ʼ Þó sprak imu eft þat wíf angegin,
kwað þat iru þar nioman þurh þes neriandan
hélaga helpa harm ne gifrumidi
wammes te lóne. Þó sprak eft waldand Krist,
drohtin manno: ʽne ik þi geþ ni deriu neowihtʼ, kwað he,
ʽak gang þi hél hinen, lát þi an þínumu hugi sorga,
þat þu nio síð aftar þius sundig ni werðes.ʼ
Habde iru þó giholpen hélag barn godes,
gefriðot iro ferahe. Þan stód þat folk Judeono
uƀiles anmód só fan éristan,
wréðes willjan, hwó sie wordheti
wið þat friðu-barn godes frummjen móstin.
Habdun þea liudi an twé mid iro gilóƀon gifangan:
was þiu smale þioda sínes willjan
gernora mikilu, þes godes barnes word
te gefrummjenne, só im iro fráho gibód:
rómodun te rehta bet þan þie ríkjon man,
habdun ina far iro hérron ia far heƀen-kuning,
fulgengun imu gerno. Þó giwét imu þe godes sunu
an þene wíh innan: hwarf ina werod umbi,
megin-þiodo gimang. He an middien stód,
lérde þea liudi liohtun wordun,
hlúdero stemnun: was hlust mikil,
þagode þegan manag, endi he þeru þiod gibód,
só hwe só þar mid þurstu biþwungan wári,
ʽsó ganga imu herod drinkan te miʼ, kwað he, ʽdago gehwilikes
swóties brunnan. Ik mag sęggjan iu,
só hwe só hér gilóƀid te mi liudjo barno
fasto undar þesumu folke, þat imu þan flioten skulun
fan is lík-hamon libbiendi flód,
irnandi water, ahospring mikil,
kumad þanen kwika brunnon. Þesa kwidi werðad wára,
liudjun giléstid, só hwemu só hér gilóƀid te mi.ʼ
Þan ménde mid þiu wataru waldandeo Krist,
hér heƀen-kuning hélagna gést,
hwó þene firiho barn antfáhen skoldin,
lioht endi listi endi líf éwig,
hóh heƀen-ríki endi huldi godes.
wurðun þó þea liudi umbi þea léra Kristes,
umbi þiu word an gewinne: stódun wlanka man,
gélmóde Judeon, sprákun gelp mikil,
habdun it im te hoska, kwaðun þat sie mahtin gi-hórjen wel,
þat imu mahlidin fram módaga wihti,
unholde út: ʽnu he an aƀu léridʼ, kwáðun sie,
ʽwordu gehwiliku.ʼ Þó sprak eft þat werod óðar:
ʽni þurƀun gi þene lérjand lahanʼ, kwáðun sie: ʽkumad líƀes word
mahtig fan is múde; he wirkid manages hwat,
wundres an þesaru weroldi: nis þat wréðaro dád,
fíundo kraftes: nio it þan te sulikaru frumu ni wurði,
ak it gegnungo fan gode alo-waldon,
kumid fan is krafte. Þat mugun gi ant-kęnnjen wel
an þem is wárun wordun, þat he giwald haƀad
alles oƀar erðu.ʼ Þó weldun ina þe andsakon þar
an stedi fáhen efþa stén ana werpen,
ef sie im þero manno menigi ni andrédin,
ni forhtodin þat folk-skępi. Þó sprak þat friðu-barn godes:
ʽik tógiu iu gódes só filuʼ, kwað he, ʽfan gode selƀumu,
wordo endi werko: nu willjad gi mi wítnon hér
þurh iuwan starkan hugi, stén ana werpen,
bilósjen mi líƀu.ʼ Þó sprákun imu eft þea liudi angegin,
wréða wiðersakon: ʽne wi it be þínun werkun ni duatʼ, kwáðun sia,
ʽþat wi þi aldres tó áhtien willjad,
ak wi duat it be þínun wordun, hwand þu sulik wáh sprikis,
*hwand þu þik só máris endi sulik mén sagis,
gihis for þeson Judeon, þat þu sís god selƀo,
mahtig drohtin, endi bist þi þoh man só wi,
kuman fan þeson kunnje.ʼ Krist alo-waldo
ne wolda þero Judeono þuo leng gelpes hórjan,
wréðaro willjon, ak hie im af þem wíhe fuor
oƀar Jordanes stróm; habda jungron mid im,
þia is sáligun gisíðos, þia im simlon mid im
willjon wonodun: suohta werod óðer,
deda þar só hie giwonoda, drohtin selƀo,
lérda þia liudi: gilóƀda þie wolda
an is hélagun word. Þat skolda sinnon wel
manno só hwilikon, só þat an is muod ginam.
Þuo gifrang ik þat þar te Kriste kumana wurðun
bodon fan Bethaniu endi sagdun þem barne godes,
þat sia an þat árundi þarod idisi sendin,
Maria endi Marþa, magað frílíka,
swíðo wun-sama wíf; þia wissa hie béðia,
wárun im giswester twá, þia hie selƀo ér
minnjoda an is muode þuru iro mildjan hugi,
þiu wíf þuru iro willjon guodan. Sia im te wáron þuo
anbudun fon Bethaniu, þat iro bruoðer was
Lazarus legar-fast endi þat sia is líƀes ni wándun;
bádun þat þarod kwámi Krist alo-waldo
hélag te helpu. Reht só hie sia gi-hórda þuo
sęggjan fan só siekon, só sprak hie sán angegin,
kwað þat Lazaruses legar ni wári
giduan im te dóðe, ʽak þar skal drohtines lofʼ, kwaþie,
ʽgifrumid werðan: nis it im te óðron fréson giduan.ʼ
was im þar þuo selƀo suno drohtines
twá naht endi dagas. Þiu tíd was þuo genáhit,
þat hie eft te Hierusalem Judeo liudjo
wíson welda, só hie giwald habda.
Sagda þuo is gisíðon suno drohtines,
þat hie eft oƀar Jordan Judeo liudi
suokjan welda. Þuo sprákun im sán angegin
jungron sína: ʽte hwí bist þu só gern þarodʼ, kwaðun sia,
ʽfro mín, te faranne? \hld\ Ni þat nu furn ni was,
þat sia þik þínero wordo \hld\ wítnon hogdun,
weldun þi mid sténon starkan awerpan? \hld\ nu þu eft undar þia strídigun þioda
fundos te faranne, \hld\ þar ist fíondo ginuog,
erlos oƀar-muoda?ʼ \hld\ Þuo én þero tweliƀio,
Þuomas gimálda \hld\ —was im gi-þungan mann,
diur-lík drohtines þegan—: \hld\ ʽne skulun wi im þia dád lahanʼ, kwaþie,
ʽni wernian wi im þes willjen, \hld\ ak wita im wonian mid,
þuolojan mid ússon þiodne: \hld\ þat ist þegnes kust,
þat hie mid is fráhon samad \hld\ fasto gi-stande,
dóje mid im þar an duome. \hld\ Duan ús alla só,
folgon im te þero ferdi: \hld\ ni látan úse ferah wið þiu
wihtes wirðig, \hld\ neƀa wi an þem werode mid im,
dójan mid úson drohtine. \hld\ Þan léƀot ús þoh duom after,
guod word for gumon.ʼ \hld\ Só wurðun þuo jungron Kristes,
erlos aðal-borana \hld\ an én-falden hugje,
hérren te willjen. \hld\ Þuo sagda hélag Krist
selƀo is gi-síðon \hld\ þat a-slápan was
Lazarus fan þem legare, \hld\ ʽhaƀit þit lioht a-geƀan,
an-sweƀit ist an selmon. \hld\ Nu wi an þena síð faran
endi ina a-wękkjan, \hld\ þat hie muoti eft þesa werold sehan,
libbjandi lioht: \hld\ þan wirðit iuwa gi-lóƀo after þiu
forð-werd gi-fęstid.ʼ \hld\ Þuo giwét hie im oƀar þia fluod þanan,
þie guodo godes suno, \hld\ anþat hie mid is jungron kwam
þar te Bithaniu, \hld\ barn drohtines
selƀo mid is gisíðon, \hld\ þar þia giswester twá,
Maria endi Marþa an muod-karon
séraga sátun. Was þar gisamnot filo
fan Hierusalem Judeo liudo,
þia þiu *wíf weldun wordun fruoƀrean,
þat sie só ni karodin kindiungas dóð,
Lazaruses farlust. Só þó þe landes ward
geng an þiu gardos, só wurðun þes godes barnes
kumi þar gikúðid, þat he só kraftig was
bi þeru burg úten. Þó im béðiun was,
þem wíƀun sulik willjo, þat sie im waldand tó,
þat friðu-barn godes, farandien wissun.
Þó þem wíbun was willjono mésta
kumi drohtines endi Kristes word
te gi-hórjenne. Heoƀandi geng
Marþa módkarag wið só mahtigne
wordun wehslan endi wið waldand sprak
an iro hugi hriwig: ʽþar þu mi, hérro mínʼ, kwað siu,
ʽnęrjendero bezt, náhor wáris,
héljand þe gódo, þan ni þorfti ik nu sulik harm þolon,
bittra breostkara, þan ni wári nu mín bróðer dód,
Lazarus fan þesumu liohte, ak he imu mahti libbien forð
ferahes gefullid. Ik þoh, fró mín, te þi
liohto gilóƀju, lérjandero bezt,
só hwes só þu biddjen wili berhton drohtin,
þat he it þi sán far-giƀid, god alo-mahtig,
giwerðot þínan willjan.ʼ Þó sprak eft waldand Krist
þeru idis and-wordi: ʽni lát þu þi an innan þesʼ, kwað he,
ʽþínan seƀon swerkan: ik þi sęggjan mag
wárun wordun, þat þes nis giwand énig,
neƀu þín bróðer skal þurh gi-bod godes,
þurh drohtines kraft fan dóðe astanden
an is lík-hamon.ʼ ʽAll hębbju ik gilóƀon sóʼ, kwað siu,
ʽþat it só giwerðen skal, só hwan só þius werold endiod
endi þe márjo dag oƀar man ferid,
þat he þan fan erðu skal up astanden
an þemu dómes daga, þan werðad fan dóðe kwika
þurh maht godes man-kunnjes gehwilik,
arísad fan restu.ʼ Þó sagde ríkjo Krist
þeru idis alo-mahtig oponun wordun,
þat he selƀo was sunu drohtines,
béðiu ia líf ia lioht liudjo barnon
te astandanne: \hld\ ʽnio þe sterƀen ni skal,
líf farliosen, \hld\ þe hér gilóƀid te mi:
þoh ina eldi-barn \hld\ erðu biþekkien,
diapo bidelƀen, \hld\ nis he dód þiu mér:
þat flésk is bifolhen, \hld\ þat ferah is gihalden,
is þiu siola gisund.ʼ \hld\ Þó sprak imu eft sán angegin
þat wíf mid iro wordun: \hld\ ʽik gilóƀju þat þu þe wáro bistʼ, kwað siu,
ʽKrist godes sunu: \hld\ þat mag man ant-kęnnjen wel,
witen an þínun wordun, \hld\ þat þu giwald haƀes
þurh þiu hélagon gi-skapu \hld\ himiles endi erðun.ʼ
Þó gefragn ik þat þar þero idisio kwam \hld\ óðar gangan
Maria mód-karag: \hld\ gengun iro managa aftar
Judeo liudi. \hld\ Þó siu þemu godes barne
sagde sérag-mód, \hld\ hwat iru te sorgun gistód
an iro hugi harmes: hofnu kúmde
Lazaruses farlust, liaƀes mannes,
griat gornundi, antat þemu godes barne
hugi warð gihrórid: héte trahni
wópu awellun, endi þó te þem wíƀun sprak,
hét ina þó lédjen, þar Lazarus was
foldu bifolhen. Lag þar én felis bioƀan,
hard stén behliden. Þó hét þe hélago Krist
antlúkan þea léia, þat he mósti þat lík sehan,
hréo skawoien. Þó ni mahte an iro hugi míðan
Marþa for þeru menegi, wið mahtigne sprak:
ʽfró mín þe gódoʼ, kwað siu, ʽef man þene felis nimid,
þene stén antlúkid, þan wániu ik þat þanen stank kume,
unswóti suek, hwand ik þi sęggjan mag
wárun wordun, þat þes nis giwand énig,
þat he þar nu bifolhen was fiuwar naht endi dagos
an þemu erðgraƀe.ʼ and-wordi gaf
waldand þemu wíbe: ʽhwat, ni sagde ik þi te wárun érʼ, kwað he,
ʽef þu gilóƀien wili, þan nis nu lang te þiu,
þat þu hér ant-kęnnjen skalt kraft drohtines,
þe mikilon maht godes?ʼ Þó gengun manage tó,
af-hóƀun harden stén. Þó sah þe hélago Krist
up mid is ógun, ólat sagde
þemu þe þese werold giskóp, ʽþes þu mín word gi-hórisʼ, kwað he,
ʽsigidrohtin selƀo; ik wét þat þu só simlun duos,
ak ik duom it be þesumu gróton Judeono folke,
þat sie þat te wárun witin, þat þu mi an þese werold sendes
þesun liudjun te lérun.ʼ Þó he te Lazaruse hriop
starkaru stemniu endi hét ina standen up
ia fan þemu graƀe gangan. Þó warð þe gést kumen
an þene lík-hamon: he bigan is liði hrórien,
antwarp undar þemu giwédie: was imo só bewunden þó noh,
an hréobeddion bihelid. Hét imu helpen þó
waldandeo Krist. Weros gengun tó,
antwundun þat gewádi. Wánum up arés
Lazarus te þesumu liohte: was imu is líf fargeƀen,
þat he is aldar-lagu égan mósti,
friðu forð-wardes. Þó fagonadun béðea,
Maria endi Marþa: ni mag þat man óðrumu
gisęggjan te sóðe, hwó þea gesuester tuó
mendiodun an iro móde. Maneg wundrode
Judeo liudjo, þó sie ina fan þemu graƀe sáhun
síðon gesunden, þene þe ér suht farnam
endi sie bidulƀun diapo undar erðu
líƀes lósen: þó móste imu libbien forð
hél an hémun. Só mag heƀen-kuninges,
þiu mikile maht godes manno gehwilikes
ferahe giformon endi wið fíundo níð
hélag helpen, só hwemu só he is huldi far-giƀid
Þó warð þar só managumu manne mód aftar Kriste,
gihworƀen hugi-skęfti, síðor sie is hélagon werk
selƀon gisáhun, hwand eo ér sulik ni warð
wunder an weroldi. Þan was eft þes werodes só filu,
só módstarke man: ni weldon þe maht godes
ant-kęnnjen kúðlíko, ak sie wið is kraft mikil
wunnun mid iro wordun: wárun im waldandes
léra so léða: sóhtun im liudi óðra
an Hierusalem, þar Judeono was
héri hand-mahal endi hóƀid-stedi,
grót gum-skępi grimmaro þioda.
Sie kúðdun im þó Kristes werk, kwáðun þat sie kwikan sáhin
þene erl mid iro ógun, þe an erðu was,
foldu bifolhen fiuwar naht endi dagos,
dód bidolƀen, antat he ina mid is dádjun selƀo,
mid is wordun awekide, þat he mósti þese werold sehan.
Þó was þat só wiðerward wlankun mannun,
Judeo liudjun: hétun iro gum-skępi þó,
werod samnojan endi warƀos fáhen,
megin-þioda gimang, an mahtigna Krist
riedun an runun: ʽnis þat rád énigʼ, kwáðun sie,
ʽþat wi þat giþolojan: wili þesaro þioda te filu
gilóƀien aftar is lérun. Þan ús liudi farad,
an eorid-folk, werðat úsa oƀar-hóƀdun
rinkos fan Rúmu. Þan wi þeses ríkjes skulun
lóse libbien efþa wi skulun úses líƀes þolon,
heliðos úsaro hóƀdo.ʼ Þó sprak þar én gihérod man
oƀar warf wero, þe was þes werodes þó
an þeru burg innan biskop þero liudjo
— Kaiphas was he héten; habdun ina gikoranen te þiu
an þeru gértalu Judeo liudi,
þat he þes godes húses gómien skoldi,
wardon þes wíhes —: ʽmi þunkid wunder mikilʼ, kwað he,
ʽmári þioda, — gi kunnun manages giskéð —
hwí gi þat te wárun ni witin, werod Judeono,
þat hér is betera rád barno gehwilikumu,
þat man hér énne man aldru bilósje
endi þat he þurh iuwa dádi dróreg sterƀe,
for þesumu folk-skępi ferah farláte,
þan al þit liud-werod farloren werðe.ʼ
Ni was it þoh is willjan, þat he só wár gesprak,
só forð for þemu folke, frume man-kunnjes
giménde for þeru menegi, ak it kwam imu fan þeru maht godes
þurh is hélagan héd, hwand he þat hús godes
þar an Hierusalem bigangan skolde,
wardon þes wíhes: beþiu he só wár gisprak,
biskop þero liudjo, hwó skoldi þat barn godes
alla irmin-þiod mid is énes ferhe,
mid is líƀu a-lósjen: þat was allaro þesaro liudjo rád,
hwand he gihalode mid þiu héðina liudi,
weros an is willjon waldandio Krist.
Þó wurðun énwordie oƀar-módie man,
werod Judeono, endi an iro warƀe gisprákun,
mári þioda, þat sie im ni létin iro mód tuehon:
só hwe só ina undar þemu folke finden mahti,
þat ina sán gifengi endi forð bráhti
an þero þiodo þing; kwáðun þat sie ni mahtin giþolojan leng,
þat sie þe éno man só alla weldi,
werod farwinnen. Þan wisse waldand Krist
þero manno só garo mód-gi-þáhti,
hetigrimmon hugi, hwand imu ni was biholen eowiht
an þesaru middil-gard: he ni welde þó an þie menigi innen
síður openlíko, under þat erlo folk,
gangan under þea Judeon: béd þe godes sunu
þero torohteon tíd, þe imu tóward was,
þat he far þesa þioda þolojan welde,
far þit werod wíti: wisse imu selƀo
þat dagþingi garo. Þó giwét imu úse drohtin forð
endi imu þó an Effrem alo-waldo Krist
an þeru hóhon burg hélag drohtin
wunode mid is werodu, antat he an is willjan hwarf
eft te Bethania brahtmu þiu mikilun,
mid þiu is gódum gum-skępi. \hld\ Judeon bisprákun þat
wordu gehwiliku, \hld\ þó sie imu sulik werod mikil
folgon gisáhun: \hld\ ʽnis frume énigʼ, kwáðun sie,
ʽúses ríkjes gi-rádi, \hld\ þoh wi reht sprekan,
ni þíhit úses þinges wiht: \hld\ þius þiod wili
wendien after is willjan; \hld\ imu all þius werold folgot,
liudi bi þem is lérun, \hld\ þat wi imu léðes wiht
for þesumu folk-skępi \hld\ gi-frummjen ni mótun.ʼ
Giwét imu þó þat barn godes \hld\ innan Bethania
sehs nahtun ér, \hld\ þan þiu samnunga
þar an Hierusalem \hld\ Judeo liudjo
an þem wíh-dagun \hld\ werðen skolde,
þat sie skoldun haldan \hld\ þea hélagon tídi,
Judeono paskha. \hld\ Béd þe godes sunu,
mahtig under þeru menegi: \hld\ was þar manno kraft,
werodes bi þem is wordun. \hld\ Þar gengun ina twé wíf umbi,
Maria endi Marþa, \hld\ mid mildju hugi,
þionodun imu þeo-líko. \hld\ Þiodo drohtin
gaf im lang-sam lón: \hld\ lét sea léðes gihwes,
sundjono sikora, \hld\ endi selƀo gibód,
þat sea an friðe fórin \hld\ wiðer fíundo níð,
þea idisa mid is orloƀu gódu: \hld\ habdun iro ambaht-skępi
biwendid an is willjon. \hld\ Þó giwét imu waldand Krist
forð mid þiu folku, \hld\ firiho drohtin,
innan Hierusalem, \hld\ þar Judeono was
hete-lík hard-buri, \hld\ þar sie þea hélagon tíd
warodun at þemu wíhe; \hld\ was þar werodes só filu,
kraftigaro kunnjo, \hld\ þie ni weldun Kristes word
gerno hórjen \hld\ ni te þemu godes barne
an iro mód-seƀon \hld\ minnje ni habdun,
ak wárun im só wréða \hld\ wlanka þioda,
módeg man-kunni, \hld\ habdun im morð-hugi,
inwid an innan: \hld\ an aƀuh farfengun
Kristes lére, \hld\ weldun ina kraftigna
wítnon þero wordo; \hld\ ak was þar werodes só filu,
umbi erl-skępi \hld\ ant-langana dag,
habde ine þiu smale þiod \hld\ þurh is swótiun word
werodu biworpen, þat ine þie wiðer-sakon
under þemu folk-skępi fáhen ne gidorstun,
ak miðun is bi þeru menegi. Þan stód mahtig Krist
an þemu wíhe innan, sagde word manag
firiho barnun te frumu. Was þar folk umbi
allan langan dag, antat þiu liohte giwét
sunne te sedle. Þó te seliðun fór
man-kunnjes manag. Þan was þar én mári berg
bi þeru burg úten, þe was bréd endi hóh,
gróni endi skóni: hétun ina Judeo liudi
Oliueti bi namon. Þar imu up giwét
nęrjendeo Krist, só ina þiu naht bifeng,
was imu þar mid is jungarun, só ine þar Judeono énig
ni wisse ti wárun, hwand he an þemu wíhe stód,
liudjo drohtin, só lioht óstene kwam,
ant-feng þat folk-skępi endi im filu sagde
wároro wordo, só nis an þesaru weroldi énig,
an þesaru middil-gard manno só spáhi,
liudjo barno nigén, þat þero lérono mugi
endi gitęlljen, þe he þar an þemu alahe gisprak,
waldand an þemu wíhe, endi simlun mid is wordun gibód,
þat sie sie gerewidin te godes ríkje,
allaro manno gehwilik, þat sie móstin an þemu márjon daga
iro drohtines diuriða antfáhen.
Sagde im hwat sie it sundjun frumidun endi simlun gibód,
þat sie þea aleskidin; hét sie lioht godes
minnjon an iro móde, mén farláten,
aƀoha oƀarhugdi, ód-módi niman,
hlaðen þat an iro hertan; kwað þat im þan wári heƀen-ríki,
garu gódo mést. Þó warð þar gumono só filu
giwendid aftar is willjon, síður sie þat word godes
hélag gi-hórdun, heƀen-kuninges,
ant-kendun kraft mikil, kumi drohtines,
hérron helpe, ia þat heƀen-ríki was,
nęrjendi gináhid endi náða godes
manno barnun. Sum só módeg was
Judeo folkes, habdun grimman hugi,
slíð-móden seƀon \hld\ [...],
ni weldun is worde gilóƀien, ak habdun im gewin mikil
wið þea Kristes kraft: kumen ni móstun
þea liudi þurh léðen stríd, þat sie gilóƀon te imu
fasto gifengin; ni was im þiu frume giƀiðig,
þat sie heƀen-ríki habbjen móstin.
Geng imu þó þe godes sunu endi is jungaron mid imu,
waldand fan þemu wíhe, all só is willjo geng,
iak imu uppen þene berg gistég barn drohtines:
sat imu þar mid is ge-síðun endi im sagde filu
wároro wordo. Sí bigunnun im þó umbi þene wíh sprekan,
þie gumon umbi þat godes hús, kwáðun þat ni wári gód-líkora
alah oƀar erðu þurh erlo hand,
þurh mannes giwerk mid megin-kraftu
rakud arihtid. Þó þe ríkjo sprak,
hér heƀen-kuning — hórdun þe óðra —:
ʽik mag iu gitęlljenʼ, kwað he, ʽþat noh wirðid þiu tíd kumen,
þat is afstanden ni skal stén oƀar óðrumu,
ak it fallid ti foldu endi fiur nimid,
grádag logna, þoh it nu só gód-lík sí,
só wíslíko giwarht, endi só dód all þesaro weroldes gi-skapu,
teglídid gróni wang.ʼ Þó gengun imu is jungaron tó,
frágodun ina só stillo: ʽhwó lango skal standen nohʼ, kwáðun sie,
ʽþius werold an wunnjun, ér þan þat giwand kume,
þat þe lasto dag liohtes skíne
þurh wolkanskion, efþo hwan is þín eft wán kumen
an þene middil-gard, manno kunnje
te adélienne, dódun endi kwikun?
fró mín þe gódo, ús is þes firiwit mikil,
waldandeo Krist, hwan þat giwerðen skuli.ʼ
Þó im and-wordi alo-waldo Krist
gód-lík far-gaf þem gumun selƀo:
ʽþat haƀad só bidernidʼ, kwað he, ʽdrohtin þe gódo
iak só hardo farholen himil-ríkjes fader,
waldand þesaro weroldes, só þat witen ni mag
énig mannisk barn, hwan þiu márje tíd
giwirðid an þesaru weroldi, ne it ók te wáran ni kunnun
godes ęngilos, þie for imu geginwarde
simlun sindun: sie it ók gisęggjan ni mugun
te wáran mid iro wordun, hwan þat giwerðen skuli,
þat he willje an þesan middil-gard, mahtig drohtin,
firiho fandon. Fader wét it éno
hélag fan himile: elkur is it biholen allun,
kwikun endi dódun, hwan is kumi werðad,
Ik mag iu þoh gi-tęlljen, hwilik hér tékan biforan
gi-werðad wunder-lík, ér þan he an þese werold kume
an þemu márjon daga: þat wirðid hér ér an þemu mánon skín
iak an þeru sunnon só same; giswerkad siu béðiu,
mid finistre werðad bifangan; fallad sterron,
hwít heƀen-tungal, endi hrisid erðe,
biƀod þius bréde werold — wirðid sulikaro bókno filu —:
grimmid þe gróto séo, wirkid þie geƀenes stróm
egison mid is úðiun erðbúandiun.
Þan þorrot þiu þiod þurh þat geþwing mikil,
folk þurh þea forhta: þan nis friðu hwergin,
ak wirðid wíg só maneg oƀar þese werold alla
hetelík af-haben, endi heri lédid
kunni oƀar óðar: wirðid kuningo giwin,
meginfard mikil: wirðid managoro kwalm,
open urlagi — þat is egislík þing,
þat io sulik morð skulun man af-hębbjen —,
wirðid wól só mikil oƀar þese werold alle,
mansterƀono mést, þero þe gio an þesaru middil-gard
swlti þurh suhti: liggiad seoka man,
driosat endi dójat endi iro dag endiad,
fulliad mid iro ferahu; ferid unmet grót
hungar hetigrim oƀar heliðo barn,
metigédeono mést: nis þat minniste
þero wíteo an þesaru weroldi, þe hér giwerðen skulun
ér dómes dage. Só hwan só gi þea dádi gisehan
giwerðen an þesaru weroldi, só mugun gi þan te wáran farstanden,
þat þan þe lazto dag liudjun náhid
mári te mannun endi maht godes,
himilkraftes hróri endi þes hélagon kumi,
drohtines mid is diuriðun. Huat, gi þesaro dádjo mugun
bi þesun bómun biliði ant-kęnnjen:
þan sie brustiad endi blóiat endi bladu tógeat,
lóf antlúkad, þan witun liudjo barn,
þat þan is sán after þiu sumer gináhid
warm endi wun-sam endi weder skóni.
Só witin gi ók bi þesun téknun, þe ik iu talde hér,
hwan þe lazto dag liudjun náhid.
Þan sęggjo ik iu te wáran, þat ér þit werod ni mót,
tefaran þit folk-skępi, ér þan werðe gefullid só,
mínu word giwárod. Noh giwand kumid
himiles endi erðun, endi steid mín hélag word
fast forð-wardes endi wirðid al gefullod só,
giléstid an þesumu liohte, só ik for þesun liudjun gespriku.
wakot gi warlíko: iu is wiskumo
duomdag þe márjo endi iuwes drohtines kraft,
þiu mikilo meginstrengi endi þiu márje tíd,
giwand þesaro weroldes. Fora þiu gi wardon skulun,
þat he iu slápandie an suefrestu
fárungo ni bifáhe an firin-werkun,
ménes fulle. Mútspelli kumit
an þiustrea naht, al só þiof ferid
darno mid is dádjun, só kumid þe dag mannun,
þe lazto þeses liohtes, só it ér þese liudi ni witun,
só samo só þiu flód deda an furn-dagun,
þe þar mid lagustrómun liudi farteride
bi Nóeas tídiun, biútan þat ina neride god
mid is híwiskja, hélag drohtin,
wið þes flódes farm: só warð ók þat fiur kuman
hét fan himile, þat þea hóhon burgi
umbi Sodomo land swart logna bifeng
grim endi grádag, þat þar nénig gumono ni ginas
biútan Loth éno: \hld\ ina ant-léddun þanen
drohtines ęngilos \hld\ endi is dohter twá
an énan berg uppen: \hld\ þat óðar al brinnandi fiur,
ia land ia liudi \hld\ logna farteride:
só fárungo warð þat fiur kumen, \hld\ só warð ér þe flód só samo:
só wirðid þe lazto dag. \hld\ For þiu skal allaro liudjo gehwilik
þęnkjan fora þemu þinge; þes is þarf mikil
manno gehwilikumu: beþiu látad iu an iuwan mód sorga.
Huand só hwan só þat gewirðid, þat waldand Krist,
mári mannes sunu mid þeru maht godes,
kumit mid þiu kraftu kuningo ríkjost
sittjan an is selƀes maht endi samod mid imu
alle þea ęngilos, þe þar uppa sind
hélaga an himile, þan skulun þarod heliðo barn,
eli-þeoda kuman alla tesamne
libbjandero liudjo, só hwat só io an þesumu liohte warð
firiho afódid. Þar he þemu folke skal,
allumu man-kunnje mári drohtin
adélien aftar iro dádjun. Þan skéðid he þea farduanan man,
þea farwarhton weros an þea winistron hand:
só duot he ók þea sáligon an þea swíðeron half;
grótid he þan þea gódun endi im tegegnes sprikid:
ʽkumad giʼ, kwiðid he, ʽþea þar gikorene sindun, endi antfáhad þit kraftiga ríki,
þat góde, þat þar gigerewid stendid, þat þar warð gumono barnun
giwarht fan þesaro weroldes endie: iu haƀad gewíhid selƀo
fader allaro firiho barno: gi mótun þesaro frumono neotan,
gewaldon þeses wídon ríkjas, hwand gi oft mínan willjon frumidun,
fulgengun mi gerno endi wárun mi iuwaro geƀo mildje,
þan ik biþwungan was þurstu endi hungru,
frostu bifangan efþo an feteron lag,
biklemmid an karkare: oft wurðun mi kumana þarod
helpa fan iuwn handun: gi wárun mi an iuwomu hugi mildje,
wísodun mín werðliko.ʼ Þan sprikid imu eft þat werod angegin:
ʽfró mín þe gódoʼ, kweðat sie, ʽhwan wári þu bifangan só,
beþwungan an sulikun þaraƀun, só þu fora þesaru þiod telis,
mahtig ménis? Huan gisah þi man énig
beþwungen an sulikun þaraƀun? Huat, þu haƀes allaro þiodo giwald
iak só samo þero méðmo, þero þe io manno barn
ge-wunnun an þesaro weroldi.ʼ Þan sprikid im eft waldand god:
ʽsó hwat só gi dádunʼ, kwiðit he, ʽan iuwes drohtines namon,
gódes far-gáƀun an godes éra
þem mannun, þe hér minniston sindun, þero nu undar þesaru menegi standad
endi þurh ód-módi arme wárun
weros, hwand sie mínan willjon fremidun — só hwat só gi im iuwaro welono fargáƀun,
gidádun þurh diuriða, þat ant-feng iuwa drohtin selƀo,
þiu helpe kwam te heƀen-kuninge. Beþiu wili iu þe hélago drohtin
lónon iuwan gilóƀon: giƀid iu líf éwig.ʼ
wendid ina þan waldand an þea winistron hand,
drohtin te þem farduanun mannun, sagad im þat sie skulin þea dád antgelden,
þea man iro mén-gi-werk: ʽnu gi fan mi skulunʼ, kwiðit he.
ʽfaran só for-flókane \hld\ an þat fiur éwig,
þat þar gi-garewid warð \hld\ godes and-sakun,
fíundo folke \hld\ be firin-werkun,
hwand gi mi ni hulpun, \hld\ þan mi hunger endi þurst
wégde te wundrun \hld\ efþa ik ge-wádjes lós
geng jámer-mód, was mi grótun þarf,
þan ni habde ik þar énige helpe, þan ik geheftid was,
an liðokospun bilokan, efþa mi legar bifeng,
swára suhti: þan ni weldun gi mín siokes þar
wíson mid wihti: ni was iu werð eowiht,
þat gi mín gehugdin. Beþiu gi an hęllje skulun
þolon an þiustre.ʼ Þan sprikid imu eft þiu þiod angegin:
ʽwola waldand godʼ, kweðad sie, ʽhwí wilt þu só wið þit werod sprekan,
mahlien wið þese menegi? Huan was þi io manno þarf,
gumono gódes? Huat, sie it al be þínun geƀun égun,
welon an þesaro weroldiʼ. Þan sprikid eft waldand god:
ʽþan gi þea armostunʼ, kwiðid he, ʽeldi-barno,
manno þea minniston an iuwomu mód-seƀon
heliðos far-hugdun, létun sea iu an iuwomu hugi léðe,
be-déldun sie iuwaro diurða, þan dádun gi iuwana drohtin só sama,
gi-wernidun imu iuwaro welono: beþiu ni wili iu waldand god,
ant-fáhen fader iuwa, ak gi an þat fiur skulun,
an þene diopun dóð, diuƀlun þionon,
wréðun wiðer-sakun, hwand gi só warhtun biforan.ʼ
Þan aftar þem wordun skéðit þat werod an twé,
þea gódun endi þea uƀilon: farad þea far-griponon man
an þea hétan hel hriwig-móde,
þea far-warhton weros, wíti antfáhat,
uƀil ęndi-lós. \hld\ Lédid up þanen
hér heƀen-kuning \hld\ þea hluttaron þeoda
an þat lang-same lioht: \hld\ þar is líf éwig,
gi-garewid godes ríki \hld\ gódaro þiado.ʼ
Só gefragn ik þat þem rinkun þo \hld\ ríki drohtin
umbi þesaro weroldes giwand \hld\ wordun talde,
hwó þiu forð ferid, \hld\ þan lango þe sie firiho barn
ardon mótun, \hld\ ia hwó siu an þemu endie skal
te-glíden endi te-gangen. \hld\ He sagde ók is jungarun þar
wárun wordun: \hld\ ʽhwat, gi witun alleʼ, kwað he,
ʽþat nu oƀar twá naht \hld\ sind tídi kumana,
Giudeono paskha, \hld\ þat sie skulun iro gode þionon,
weros an þemu wíhe. \hld\ Þes nis gewand énig,
þat þar wirðid mannes sunu \hld\ te þeru megin-þiodu
kraftag far-kópot \hld\ endi an krúke aslagan,
þolod þiad-kwála.ʼ \hld\ Þó warð þar þegan manag
slíð-mód gisamnod, \hld\ súðar-liudjo,
Judeono gum-skępi, \hld\ þar sie skoldun iro gode þionon.
wurðun éosagon \hld\ alle kumane,
an warf weros, \hld\ þe sie þó wísostun
undar þeru menegi \hld\ manno taldun,
kraftag kuni-burd. \hld\ Þar Kaiphas was,
biskop þero liudjo. \hld\ Sie rédun þó an þat barn godes,
hwó sie ina a-sluogin \hld\ sundja lósan,
kwáðun þat sie ina an þemu hélagon daga \hld\ hrínen ni skoldin
undar þero manno menegi, \hld\ ʽþat ni werðe þius megin-þioda,
heliðos an hróru, hwand ina þit hęri-skępi wili
farstanden mid strídu. Wi só stillo skulun
fréson is ferahes, þat þit folk Judeono
an þesun wíhdagun wróht ni af-hębbjen.ʼ
Þó geng imu þar Iúdas forð, jungaro Kristes,
én þero tweliƀio, þar þat aðali sat,
Judeono gum-skępi; kwað þat he is im gódan rád
sęggjan mahti: ʽhwat willjad gi mi sęlljen hérʼ, kwað he,
ʽméðmo te médu, ef ik iu þene man giƀu
áno wíg endi áno wróht?ʼ Þó warð þes werodes hugi,
þero liudjo an lustun: ʽef þu wili giléstien sóʼ, kwáðun sie,
ʽþín word giwáron, þan þu giwald haƀes,
hwat þu at þesaru þiodu þiggean willjes
gódaro méðmo.ʼ Þó gihét imu þat gum-skępi þar
an is selƀes dóm siluƀarskatto
þrítig atsamne, endi he te þeru þiodu gisprak
dereƀeun wordun, þat he gáƀi is drohtin wið þiu.
wende ina þó fan þemu werode: was im wréð hugi,
talode im só treulós, hwan ér wurði imu þiu tíd kuman,
þat he ina mahti farwísien wréðaro þiodo,
fíundo folke. Þan wisse þat friðu-barn godes,
wár waldand Krist, þat he þese werold skolde,
ageƀen þese gardos endi sókien imu godes ríki,
gifaren is faderoðil. Þó ni gisah énig firiho barno
méron minnje, þan he þó te þem mannun ginam,
te þem is gódun jungaron: góme warhte,
sette sie swáslíko endi im sagde filu
wároro wordo. Skréd wester dag,
sunne te sedle. Þó he selƀo gibód,
waldand mid is wordun, hét im water dragan
hluttar te handun, endi rés þó þe hélago Krist,
þe gódo at þem gómun endi þar is jungarono þuóg
fóti mid is folmun endi suarf sie mid is fanon aftar,
druknide sie diurlíka. Þó wið is drohtin sprak
Símon Petrus: ʽni þunkid mi þit sómi þingʼ, kwað he,
ʽfró mín þe gódo, þat þu míne fóti þwahes
mid þem þínun hélagun handun.ʼ Þó sprak imu eft is hérro angegin,
waldand mid is wordun: ʽef þu is willjan ni haƀesʼ, kwað he,
ʽte antfáhanne, þat ik þíne fóti þwahe
þurh sulika minnja, só ik þesun óðrun mannun hér
dóm þurh diurða, þan ni haƀes þu énigan dél mid mi
an heƀen-ríkja.ʼ Hugi warð þó giwendid
Símon Petruse: ʽþu haƀa þi selƀo giwaldʼ, kwað he,
ʽfro mín þe gódo, fóto endi hando
b endi mínes hóƀdes só sama, handun þínun,
þiadan, te þwahanne, te þiu þak ik móti þína forð
huldi hębbjan endi heƀen-ríkjes
sulik gidéli, só þu mi, drohtin, wili
fargeƀen þurh þína gódi.ʼ Jungaron Kristes,
þene ambahtskepi erlos þolodun,
þegnos mid giþuldeon, só hwat só im iro þiodan dede,
mahtig þurh þea minnja, endi ménde imu al méra þing
firihon te gifrummjenne. \hld\ friðu-barn godes
geng imu þó eft gisittjen under þat ge-síðo folk
endi im sagda filu lang-samna rád. Warð eft lioht kuman,
morgen te mannun. Mahtigne Krist
gróttun is jungaron endi frágodun, hwar sie is góma þó
an þemu wíhdage wirkjen skoldin,
hwar he weldi halden þea hélagon tídi
selƀo mid is ge-síðun. Þó he sie sókien hét,
þea gumon Hierusalem: ʽsó gi þan gangan kumadʼ, kwað he,
ʽan þea burg innan \hld\ —þar is braht mikil,
megin-þiodo gimang—, \hld\ þar mugun gi énan man sehan
an is handun dragen \hld\ hluttres watares
ful mid folmun. \hld\ Þemu gi folgon skulun
an só hwilike gardos, \hld\ só gi ina gangan gisehat,
ia gi þan þemu hérron, \hld\ þe þie hoƀos égi,
selƀon sęggjad, \hld\ þat ik iu sende þarod
te gi-garuwenne mína góma. \hld\ Þan tógid he iu én gód-lík hús,
hóhan soleri, \hld\ þe is bihangen al
fagarun fratahun. \hld\ Þar gi frummjen skulun
werd-skępi mínan. \hld\ Þar bium ik wiskumo
selƀo mid mínun ge-síðun.ʼ \hld\ Þó wurðun sán aftar þiu
þar te Hierusalem \hld\ jungaron Kristes
forð-ward an ferdi, \hld\ fundun all só he sprak
word-tékan wár: \hld\ ni was þes giwand énig.
Þar gerewidun sie þea góma. \hld\ Warð þe godes sunu,
hélag drohtin \hld\ an þat hús kuman,
þar sie þe land-wíse \hld\ léstien skoldun,
ful-gangan godes gi-bode, \hld\ al só Judeono was
éo endi ald-sidu \hld\ an ér-dagun.
Gi-wét imu þó an þemu áƀande \hld\ alo-waldand Krist
an þene sęli sittjen; \hld\ hét þar is ge-síðos te imu
tweliƀi gangan, \hld\ þea im gi-triwiston
an iro mód-seƀon \hld\ manno wárun
bi wordun endi bi wísun: \hld\ wisse imu selƀo
iro hugi-skęfti \hld\ hélag drohtin.
Grótte sie þó oƀar þem gómun: \hld\ ʽgern bium ik swíðoʼ, kwað he,
ʽþat ik samad mid iu \hld\ sittjen móti,
gómono neoten, \hld\ Judeono paskha
déljen mid iu só diurjun. \hld\ Nu ik iu iuwes drohtines skal
willjon sęggjan, \hld\ þat ik an þesaro weroldi ni mót
mid mannun mér \hld\ móses an-bíten
furður mid firihun, \hld\ ér þan gifullod wirðid
himilo ríki. \hld\ Mi is an handun nu
wíti endi wunder-kwále, \hld\ þea ik for þesumu werode skal,
þolon for þesaru þiodu.ʼ \hld\ Só he þó só te þem þegnun sprak,
hélag drohtin, \hld\ só warð imu is hugi dróƀi,
warð imu gi-sworken seƀo, \hld\ endi eft te þem ge-síðun sprak,
þe gódo te þem is jungarun: \hld\ ʽhwat, ik iu godes ríkiʼ, kwað he,
ʽgi-hét himiles lioht, \hld\ endi gi mi hold-líko
iuwan þegan-skępi. \hld\ Nu ni willjat gi a-þengean só,
ak wenkjat þero wordo. \hld\ Nu sęggju ik iu te wáran hér,
þat wili iuwar tweliƀio én \hld\ trewana swíkan,
wili mi far-kópon undar þit kunni Judeono,
gi-sęlljen wiðer siluƀre, endi wili imu þar sink niman,
diurie méðmos, endi geƀen is drohtin wið þiu,
holdan hérran. Þat imu þoh te harme skal,
werðan te wítje; be þat he þea wurdi farsihit
endi he þes arƀedies endi skawot,
þan wét he þat te wáran, þat imu wári wóðiera þing,
betera mikilu, þat he gio gi-boran ni wurði
libbiendi te þesumu liohte, þan he þat lón nimid,
uƀil arƀedi Inwid-rádo.ʼ
Þó bigan þero erlo gehwilik te óðrumu skawon,
sorgondi sehan; was im sér hugi,
hriwig umbi iro herta: gi-hórdun iro hérron þó
gornword sprekan. Þea gumon sorgodun,
hwilikan he þero tweliƀio \hld\ te þiu tęlljen weldi,
skuldigna skaðon, \hld\ þat he habdi þea skattos þar
geþingod at þeru þiod. \hld\ Ni was þero þegno énigumu
sulikes in-widdies \hld\ óði te gehanne,
mén-giþáhtio — antsuok þero manno gehwilik —,
wurðun alle an forhtun, frágon ne gidorstun,
ér þan þó gebóknide barwirðig gumo,
Símon Petrus — ne gidorste it selƀo sprekan —
te Johanne þemu gódon: he was þemu godes barne
an þem dagun þegno lioƀost,
mést an minnjun endi móste þar þó an þes mahtiges Kristes
barme restjen endi an is breostun lag,
hlinode mid is hóƀdu: þar nam he só manag hélag gerúni,
diapa giþáhti, endi þó te is drohtine sprak,
began ina þó frágon: ʽhwe skal þat, fró mín, wesenʼ, kwað he,
ʽþat þi farkópon wili, kuningo ríkjost,
undar þínaro fíundo folk? ús wári þes firiwit mikil,
waldand, te witanne.ʼ Þó habde eft is word garu
héljando Krist: ʽseh þi, hwemu ik hér an hand geƀe
mínes móses for þesun mannun: þe haƀed mén-gi-þáht,
birid bittran hugi; þe skal mi an banono gewald,
fíundun bifelhen, þar man mínes ferhes skal,
aldres áhtien.ʼ Nam he þó aftar þiu
þes móses for þem mannun endi gaf is þemu ménskaðen,
Judase an hand endi imu tegegnes sprak
selƀo for þem is ge-síðun endi ina sniumo hét
faran fan þemu is folke: ʽfrumi só þu þenkisʼ, kwað he,
ʽdó þat þu duan skalt: þu ni maht bidernien leng
willjon þínan. Þiu wurd is at handun,
þea tídi sind nu gináhid.ʼ Só þó þe treulogo
þat mós ant-feng endi mid is múðu anbét,
só afgaf ina þó þiu godes kraft, gramon in gewitun
an þene lík-hamon, léða wihti,
warð imu Satanas séro bitengi,
hardo umbi is herte, síður ine þiu helpe godes
farlét an þesumu liohte. Só is þena liudjo wé,
þe só undar þesumu himile skal hérron wehslon.
Giwét imu þó út þanen inwideas gern
Judas gangan: habde imu grimmen hugi
þegan wið is þiodan. Was þó iu þiustri naht,
swíðo gisuorken. Sunu drohtines
was ima at þem gómun forð endi is jungarun þar
waldand wín endi bród wíhide béðiu,
hélagode heƀen-kuning, mid is handun brak,
gaf it undar þem is jungarun endi gode þankode,
sagde þem ólat, þe þar al giskóp,
werold endi wunnja, endi sprak word manag:
ʽgilóƀiot gi þes liohtoʼ, kwað he, ʽþat þit is mín lík-hamo
endi mín blód só same: giƀu ik iu hér béðiu samad
etan endi drinkan. Þit ik an erðu skal
geƀan endi geotan endi iu te godes ríkje
lósjen mid mínu lík-hamen an líf éwig,
an þat himiles lioht. Gihuggjat gi simlun,
þat gi þiu fulgangan, þiu ik an þesun gómun dón;
márjad þit for menegi: þit is mahtig þing,
mid þius skulun gi iuwomu drohtine diuriða frummjen,
habbiad þit mín te gihugdiun, hélag biliði,
þat it eldi-barn aftar léstien,
waron an þesaru weroldi, þat þat witin alle,
man oƀar þesan middil-gard, þat it is þurh mína minnja giduan
hérron te huldi. Gehuggiad gi simlun,
hweo ik iu hér gebiudu, þat gi iuwan bróðerskepi
fasto frummjad: habbiad ferhtan hugi,
minnjod iu an iuwomu móde, þat þat manno barn
oƀar irmin-þiod alle farstanden,
þat gi sind gegnungo jungaron míne.
Ôk skal ik iu kúðien, hwó hér wili kraftag fíund,
hetteand herugrim, umbi iuwan hugi niusien,
Satanas selƀo: he kumid iuwaro seolono herod
frókno fréson. Simlun gi fasto te gode
berad iuwa breostgiþáht: ik skal an iuwaru bedu standen,
þat iu ni mugi þe ménskaðo mód getwíflean;
ik fulléstiu iu wiðer þemu fíunde. Ôk kwam he herod giu fréson mín,
þoh imu is willjon hér wiht ne gistódi,
lioƀes an þemu mínumu lík-hamon. Nu ni willju ik iu leng helen,
hwat iu hér nu sniumo skal te sorgu gistanden:
gi skulun mi geswíkan, ge-síðos míne,
iuwes þeganskępjes, ér þan þius þiustrie naht
liudi farlíða endi eft lioht kume,
morgan te mannun.ʼ Þó warð mód gumon
swíðo gisuorken endi sér hugi,
hriwig umbi iro herte endi iro hérron word
swíðo an sorgun. Símon Petrus þó,
þegan wið is þiodan þrístwordun sprak
bi huldi *wið is hérron: ʽþoh þi all þit heliðo folkʼ, kwaþie,
ʽgiswíkan þína gisíðos, þoh ik sinnon mid þi
at allon þaraƀon þolojan willju.
Ik biun garo sinnon, ef mi god látið,
þat ik an þínon fulléstie fasto gistande;
þoh sia þi an karkaries klústron hardo,
þesa liudi bilúkan, þoh ist mi luttil tueho,
ne ik an þem bendion mid þi bídan willje,
liggian mid þi só lieƀen; ef sia þínes líƀes þan
þuru ęggja níð áhtian willjad,
fró mín þie guodo, ik giƀu mín ferah furi þik
an wápno spil: nis mi werð iowiht
te bimíðanne, só lango só mi mín warod
hugi endi handkraft.ʼ Þuo sprak im eft is hérro angegin:
ʽhwat, þu þik biwánisʼ, kwaþie, ʽwissaro trewono,
þrístero þingo: þu haƀis þegnes hugi,
willjon guodan. Ik mag þi sęggjan, hwó it þoh gi-werðan skal,
þat þu wirðis só wék-muod, þoh þu nu ni wánjes só,
þat þu þínes þiadnes te naht þríwo far-lógnis
ér hanokrádi endi kwiðis, þak ik þín hérro ni sí,
ak þu far-manst mína mund-burd.ʼ Þuo sprak eft þie man an-gegin:
ʽef it gio an weroldiʼ, kwaþie, ʽgiwerðan muosti,
þat ik samad midi þi sweltan muosti,
dójan diur-líko, þan ne wurði gio þie dag kuman,
þat ik þín far-lógnidi, \hld\ lieƀo drohtin,
gerno for þeson Juðeon.ʼ \hld\ Þuo kwáðun alla þia jungron só,
þat sia þar an þem þingon mid im \hld\ þolian weldin
Þuo im eft mid is wordon gibód \hld\ waldand selƀo,
hér heƀan-kuning, \hld\ þat sia im ni lietin iro hugi twíflian,
hiet þat sia ni weldin[...] \hld\ diopa giþáhti:
ʽne druoƀie iuwa herta \hld\ þuru iuwes drohtines word,
ne forohteat te filo: \hld\ ik skal fader úsan
selƀan suokjan \hld\ endi iu sendian skal
fan heƀan-ríkje \hld\ hélagna gést:
þie skal iu eft gi-fruofrean \hld\ endi te frumu werðan,
manon iu þero mahlo, \hld\ þie ik iu manag hębbju
wordon gi-wísid. \hld\ Hie giƀit iu giwit an briost,
lust-sama léra, \hld\ þat gi léstian forð
þiu word endi þiu werk, þia ik iu an þesaro weroldi gibód.ʼ
Arés im þuo þe ríkjo an þemo rakode innan,
nęrjendo Krist endi giwét im nahtes þanan
selƀo mid is gi-síðon: sérago gengun
swíðo gornondia jungron Kristes,
hriwig-muoda. Þuo hie im an þena hóhan giwét
Oliueti-berg: þar was hie up giwuno
gangan mid is jungron. Þat wissa Judas wel,
balo-húgdig man, hwand hie was oft an þem berege mid im.
Þar gruotta þie godes suno iúgron sína:
ʽgi sind nu só druoƀiaʼ, kwaþie, ʽnu gi mínan dóð witun;
nu gornonð gi endi griotand, endi þesa Juðeon sind an luston,
mendit þius menigi, sindun an iro muode fráha,
þius werold ist an wunnjon. Þes wirðit þoh giwand kuman
sniumo tulgo: þan wirðit im sér hugi,
þan morniat sia an iro móde, endi gi mendian skulun
after te éwondage, hwand gio endi ni kumið,
iuwes wellíƀes giwand: beþiu ne þurƀun iu þius werk tregan,
hrewan mín hinfard, hwand þanan skal þiu helpa kuman
gumono barnon.ʼ Þuo hiet hie is jungron þar
bídan uppan þemo berge, kwað þat hie ti bedu weldi
an þiu holmkliƀu hóhor stígan;
hiet þuo þria mid im þegnos gangan,
Jakobe endi Johannese endi þena guodan Petruse,
þrístmuodian þegan. Þuo sia mid iro þiedne samad
gerno gengun. Þuo hiet sia þie godes suno
an berge uppan te bedu hnígan,
hiet sia god gruotian, *gerno biddjan,
þat he im þero kostondero kraft farstódi,
wréðaro willjon, þat im þe wiðer-sako,
ni mahti þe ménskaðo mód gitwíflean,
iak imu þó selƀo gihnég sunu drohtines
kraftag an kniobeda, kuningo ríkjost,
forð-ward te foldu: fader aloþiado
gódan grótte, gornwordun sprak
hriwiglíko: was imu is hugi dróƀi,
bi þeru menniski mód gihrórid,
is flésk was an forhtun: fellun imo trahni,
dróp is diurlík suét, al só drór kumid
wallan fan wundun. Was an gewinne þó
an þemu godes barne þe gést endi þe lík-hamo:
óðar was fúsid an forðwegos,
þe gést an godes ríki, óðar giámar stód,
lík-hamo Kristes: ni welde þit lioht ageƀen,
ak droƀde for þemu dóðe. Simla he hreop te drohtine forð
þiu mér aftar þiu mahtigna grótte,
hóhan himilfader, hélagna god,
waldand mid is wordun: ʽef nu werðen ni magʼ, kwað he,
ʽman-kunni generid, ne sí þat ik mínan geƀe
lioƀan lík-hamon for liudjo barn
te wégeanne te wundrun, it sí þan þín willjo só,
ik willju is þan gikoston: ik nimu þene kelik an hand,
drinku ina þi te diurðu, drohtin fró mín,
mahtig mund-boro. Ni seh þu mínes hér
fléskes gifórjes. Ik fullon skal
willjon þínen: þu haƀes gewald oƀar al.ʼ
Giwét imu þó gangen, þar he ér is jungaron lét
bídan uppan þemu berge; fand sie þat barn godes
slápen sorgandie: was im sér hugi,
þes sie fan iro drohtine délien skoldun.
Só sind þat módþraka manno gehwilikumu,
þat he farláten skal liaƀane hérron,
afgeƀen þene só gódene. Þó he te is jungarun sprak,
wahte sie waldand endi wordun grótte:
ʽhwí willjad gi só slápen?ʼ kwað he; ʽni mugun samad mid mi
wakon éne tíd? Þiu wurd is at handun,
þat it só gigangen skal, só it god fader
gimarkode mahtig. Mi nis an mínumu móde tueho:
mín gést is garu an godes willjan,
fús te faranne: mín flésk is an sorgun,
letid mik mín lík-hamo: léð is imu swíðo
wíti te þolonne. Ik þoh willjan skal
mínes fader gefrummjen. hębbjad gi fasten hugi.ʼ
Giwét imu þó eft þanan óðersíðu
an þene berg uppen te bedu gangan,
mári drohtin, endi þar só manag gisprak
gódoro wordo. Godes ęngil kwam
hélag fan himile, is hugi fastnode,
beldide te þem bendiun. He was an þeru bedu simla
forð an flíte endi is fader grótte,
waldand mid is wordun: ʽef it nu wesen ni magʼ, kwað he,
ʽmári drohtin, neƀu ik for þit manno folk
þiodkwále þoloie, ik an þínan skal
willjan wonian.ʼ Giwét imu þó eft þanen
sókjan is ge-síðos: fand sie slápandie,
grótte sie gáhun. Geng imu eft þanen
þriddjon síðu te bedu endi sprak þiod-kuning
al þiu selƀon word, sunu drohtines,
te þemu alo-waldon fader, só he ér dede,
manode mahtigna manno frumana
swíðo niudlíko neriando Krist,
geng imu þó eft te þem is jungarun, grótte sie sáno:
ʽslápad gi endi restiadʼ, kwað he. ʽNu wirðid sniumo herod
kuman mid kraftu, þe mi farkópot haƀad,
sundja lósan gisald.ʼ ge-síðos Kristes
wakodun þó aftar þem wordun endi gisáhun þó þat werod kuman
an þene berg uppen brahtmu þiu mikilon,
wréða wápanberand. \hld\ Wísde im Judas,
gramhugdig man; Judeon aftar sigun,
fíundo folk-skępi; dróg man fiur an gimang,
logna an liohtfatun, lédde man faklon
brinnandea fan burg, þar sie an þene berg uppan
stigun mid strídu. Þea stedi wisse Judas wel,
hwar he þea liudi tó lédean skolde.
Sagde imu þó te tékne, þó sie þar tó fórun
þemu folke biforan, te þiu þat sie ni farfengin þar,
erlos óðren man: ʽik gangu imu at érist tóʼ, kwað he,
ʽkussiu ine endi kwaddiu: þat is Krist selƀo.
Þene gi fáhen skulun folko kraftu,
binden ina uppan þemu berge endi ina te burg hinan
lédjen undar þea liudi: he is líƀes haƀad
mid is wordun farwerkod.ʼ werod síðode þó,
antat sie te Kriste kumane wurðun,
grim folk Judeono, þar he mid is jungarun stód,
mári drohtin: béd metodo-gi-skapu,
torhtero tídeo. Þó geng imu treulós man,
Judas tegegnes endi te þemu godes barne
hnég mid is hóƀdu endi is hérron kwedde,
kuste ina kraftagne endi is kwidi léste,
wísde ina þemu werode, al só he ér mid wordun gehét.
Þat þolode al mid giþuldiun þiodo drohtin,
waldand þesara weroldes endi sprak imu mid is wordun tó,
frágode ine frókno: \hld\ ʽbe-hwí kumis þu só mid þius folku te mi,
be-hwí lédis þu mi só þese liudi tó \hld\ endi mi te þesare léðan þiode sprekan,
far-kópos mid þínu kussu \hld\ under þit kunni Judeono,
meldos mi te þesaru menegi?ʼ \hld\ Geng imu þó wið þea man
wið þat werod óðar \hld\ endi sie mid is wordun fragn,
hwene sie mid þiu ge-síðju \hld\ sókjan kwámin
só niudliko an naht, \hld\ ʽso gi willjan nód frummjen
manno hwilikumu.ʼ \hld\ Þó sprak imu eft þiu menegi angegin,
kwáðun þat im héljand \hld\ þar an þemu holme uppan
ge-wísid wári, \hld\ ʽþe þit giwer frumid
Judeo liudjun \hld\ endi ina godes sunu
selƀon hétid. \hld\ Ina kwámun wi sókjan herod,
weldin ina gerno bi-geten: \hld\ he is fan Galileo lande,
fan Nazareth-burg.ʼ \hld\ Só im þó þe nęrjendio Krist
sagde te sóðan, \hld\ þat he it selƀo was,
só wurðun þó an forhtun \hld\ folk Judeono,
wurðun underbadode, \hld\ þat sie under bak fellun
alle efno sán, \hld\ erðe gisóhtun,
wiðerwardes þat werod: \hld\ ni mahte þat word godes,
þie stemnie ant-standan: \hld\ wárun þoh só strídige man,
ahliopun eft up an þemu holme, \hld\ hugi fastnodun,
bundun briost-gi-þáht, \hld\ gi-bolgane gengun
náhor mid níðu, \hld\ anttat sie þene nęrjendion Krist
werodo biwurpun. \hld\ Stódun wíse man,
swíðo gornundie \hld\ giungaron Kristes
biforan þeru dereƀeon dádi \hld\ endi te iro drohtine sprákun:
ʽwári it nu þín willjoʼ, \hld\ kwáðun sie, ʽwaldand fró mín,
þat sie ús hér an speres ordun \hld\ spildien móstin
wápnun wunde, \hld\ þan ni wári ús wiht só gód,
só þat wi hér for úsumu drohtine \hld\ dóan móstin
beniðiun blékaʼ. \hld\ Þó gibolgan warð
snel swerd-þegan, \hld\ Símon Petrus,
well imu innan hugi, \hld\ þat he ni mahte énig word sprekan:
só harm warð imu an is hertan, \hld\ þat man is hérron þar
binden welde. \hld\ Þó he gibolgan geng,
swíðo þríst-mód þegan \hld\ for is þiodan standen,
hard for is hérron: \hld\ ni was imu is hugi twífli,
blóð an is breostun, \hld\ ak he is bil atóh,
swerd bi sídu, \hld\ slóg imu tegegnes
an þene furiston fíund \hld\ folmo krafto,
þat þó Malkhus warð \hld\ mákjas ęggjun,
an þea swíðaron half \hld\ swerdu gimálod:
þiu hlust warð imu far-hawan, \hld\ he warð an þat hóƀid wund,
þat imu heru-drórag \hld\ hlear endi óre
bęni-wundun brast: \hld\ blód aftar sprang,
well fan wundun. \hld\ Þó was an is wangun skard
þe furisto þero fíundo. \hld\ Þó stód þat folk an rúm:
andrédun im þes billes biti. \hld\ Þó sprak þat barn godes
selƀo te Símon Petruse, \hld\ hét þat he is swerd dedi
skarp an skéðia: \hld\ ʽef ik wið þesa skola weldiʼ, kwað he,
ʽwið þeses werodes ge-win \hld\ wíg-saka frummjen,
þan manodi ik þene márjon \hld\ mahtigne god,
hélagne fader \hld\ an himil-ríkja,
þat he mi só managan ęngil herod \hld\ oƀana sandi
wíges só wísen, \hld\ só ni mahtin iro wápan-þręki
man adógen: \hld\ iro ni stódi gio sulik megin samad,
folkes gi-fastnod, þat im iro ferh aftar þiu
werðen mahti. Ak it haƀad waldand god,
alo-mahtig fader an óðar gimarkot,
þat wi giþolojan skulun, só hwat só ús þius þioda tó
bittres brengit: ni skulun ús belgan wiht,
wréðean wið iro gewinne; hwand só hwe só wápno níð,
grimman gérheti wili gerno frummjen,
he suiltit imu eft swerdes ęggjun,
dóit im bidróregan: wi mid úsun dádjun ni skulun
wiht awerdian.ʼ Geng he þó te þemu wundon manne,
legde mid listiun lík tesamne,
hóƀidwundon, þat siu sán gihélid warð,
þes billes biti, endi sprak þat barn godes
wið þat wréðe werod: ʽmi þunkid wunder mikilʼ, kwað he,
ʽef gi mi léðes wiht léstien weldun,
hwí gi mi þó ni fengun, þan ik undar iuwomu folke stód,
an þemu wíhe innan endi þar word manag
sóðlík sagde. Þan was sunnon skín,
diurlik dages lioht, þan ni weldun gi mi dóan eowiht
léðes an þesumu liohte, endi nu lédiad mi iuwa liudi tó
an þiustrie naht, al só man þioƀe dót,
þan man þene fáhan wili endi he is ferhes haƀad
farwerkot, wam-skaðo.ʼ werod Judeono
gripun þó an þene godes sunu, grimma þioda,
hatandiero hóp, hwurƀun ina umbi
módag manno folk — ménes ni sáhun —,
heftun herubendiun handi tesamne,
faðmos mid fitereun. Im ni was sulikaro firinkwála
þarf te giþolonne, þiodarƀedies,
te winnanne sulik wíti, ak he it þurh þit werod deda,
hwand he liudjo barn lósjen welda,
halon fan hęllju an himil-ríki,
an þene wídon welon: beþiu he þes wiht ne bisprak,
þes sie imu þurh inwidníð ógean weldun.
Þó wurðun þes só malske módag folk Judeono,
þiu héri warð þes só hrómeg, þes sie þena hélagon Krist
an liðobendion lédian muostun,
fórjan an fitereun. Þie fíund eft gewitun
fan þemu berge te burg. Geng þat barn godes
undar þemu hęri-skępi handun gebunden,
drúƀondi te dale. Wárun imu þea is diurion þó
ge-síðos gesuikane, al só he im ér selƀo gisprak:
ni was it þoh be énigaru blóði, þat sie þat barn godes,
lioƀen farlétun, ak it was só lango biforen
wár-sagono word, þat it skoldi giwerðen só:
beþiu ni mahtun sie is bemíðan. Þan aftar þeru menegi gengun
Johannes endi Petrus, þie gumon twéne,
folgodun ferrane: was im firiwit mikil,
hwat þea grimmon Judeon þemu godes barne,
weldin iro drohtine dóen. Þó sie te dale kwámun
fan þemu berge te burg, þar iro biskop was,
iro wíhes ward, þar léddun ina wlanke man,
erlos undar ederos. Þar was éld mikil,
fiur an frídhoƀe þemu folke tegegnes,
gewarht for þemu werode: þar gengun sie im wermien tó,
Judeo liudi, létun þene godes sunu
bídon an bendiun. Was þar braht mikil,
gélmódigaro galm. Johannes was ér
þemu héroston kúð: beþiu móste he an þene hof innan
þringan mid þeru þioda. Stód allaro þegno bezto,
Petrus þar úte: ni lét ina þe portun ward
folgon is fróen, ér it at is friunde abad,
Johannes at énumu Judeon, þat man ina gangan lét
forð an þene frídhof. Þar kwam im én fékni wíf
gangan tegegnes, þiu énas Judeon was,
iro þeodanes þiw, endi þó te þemu þegne sprak
magað unwánlík: ʽhwat, þu mahtis man wesanʼ, kwað siu,
ʽgiungaro fan Galilea, þes þe þar genower stéd
faðmun gi-fastnod.ʼ Þó an forhtun warð
Símon Petrus sán, slak an is móde,
kwað þat he þes wíƀes word ni bikonsti
ni þes þeodanes þegan ni wári:
méð is þó for þeru menegi, kwað þat he þena man ni ant-kendi:
ʽni sind mi þíne kwidi kúðeʼ, kwað he; was imu þiu kraft godes,
þe herdislo fan þemu hertan. Huaraƀondi geng
forð undar þemu folke, antat he te þemu fiure kwam;
giwét ina þó warmien. Þar im ók én wíf bigan
felgian firinspráka: ʽhér mugun giʼ, kwað siu, ʽan iuwan fíund sehan:
þit is gegnungo giungaro Kristes,
is selƀes ge-síð.ʼ Þó gengun imu sán aftar þiu
náhor níð-hwata endi ina niudlíko
frágodun fíundo barn, hwilikes he folkes wári:
ʼni bist þu þesoro burg-liudjoʼ, kwáðun sie; ʽþat mugun wi an þínumu gibárie gisehan,
an þínun wordun endi an þínaru wíson, þat þu þeses werodes ni bist,
ak þu bist galiléisk man.ʼ He ni welda þes þó gehan eowiht,
ak stód þó endi strídda endi starkan éð
swíð-líko ge-swór, þat he þes ge-síðes ni wári.
Ni habda is wordo ge-wald: it skolde gi-werðen só,
só it þe ge-markode, þe man-kunnjes
farwardot an þesaru weroldi. Þó kwam imu ók an þemu warƀe tó
þes mannes mágwini, þe he ér mid is mákeo giheu,
swerdu þiu skarpon, kwað þat he ina sáhi þar
an þemu berge uppan, ʽþar wi an þemu bómgardon
hérron þínumu hendi bundun,
fastnodun is folmos.ʼ He þó þurh forhtan hugi
for-lógnide þes is lioƀes hérron, kwað þat he weldi wesan þes líƀes skolo,
ef it mahti énig þar irminmanno
gisęggjan te sóðan, þat he þes ge-síðes wári,
folgodi þeru ferdi. Þó warð an þena formon síð
hanokrád af-haƀen. Þó sah þe hélago Krist,
barno þat bezte, þar he gebunden stóð,
selƀo te Símon Petruse, sunu drohtines
te þemu erle oƀar is ahsla. Þó warð imu an innan sán,
Símon Petruse sér an is móde,
harm an is hertan endi is hugi dróƀi,
swíðo warð imu an sorgun, þat he ér selƀo gesprak:
gihugde þero wordo þó, þe imu ér waldand Krist
selƀo sagda, þat he an þeru swartan naht
ér hanokrádi is hérron skoldi
þríwo farlógnien. Þes þram imu an innan mód
bittro an is breostun, endi geng imu þó gibolgan þanen
þe man fan þeru menigi an módkaru,
swíðo an sorgun, endi is selƀes word,
wam-skęfti weop, \hld\ antat imu wallan kwámun
þurh þea hert-kara \hld\ héte trahni,
blódage fan is breostun. \hld\ He ni wánde þat he is mahti gibótjen wiht,
firin-werko furður efþa te is fráhon kuman,
hérron huldi: nis énig heliðo só ald,
þat io mannes sunu mér gisáhi
is selƀes word sérur hrewan,
karon efþa kúmien: ʽwola krafteg godʼ, kwað he,
þat ik hębbju mi só forwerkot, só ik mínaro weroldes ni þarf
ólat sęggjan. Ef ik nu te aldre skal
huldeo þínaro endi heƀen-ríkjas,
þeoden, þolojan, þan ni þarf mi þes énig þank wesan,
lioƀo drohtin, þat ik io te þesumu liohte kwam.
Ni bium ik nu þes wirðig, waldand fró mín,
þat ik under þíne jungaron gangan móti,
þus sundig under þíne ge-síðos: ik iro selƀo skal
míðan an mínumu móde, nu ik mi sulik mén gesprak.ʼ
Só gornode gumono bezta,
hrau im só hardo, þat he habde is hérren þó
leoƀes farlógnid. Þan ni þurƀun þes liudjo barn,
weros wundrojan, be-hwí it weldi god,
þat só lioƀen man léð gistódi,
þat he só hónlíko hérron sínes
þurh þera þiwun word, þegno snellost,
far-lógnide só lioƀes: it was al bi þesun liudjun giduan,
firiho barnun te frumu. He welde ina te furiston dóan,
hérost oƀar is híwiski, hélag drohtin:
lét ina gekunnon, hwilike kraft haƀet
þe menniska mód áno þe maht godes;
lét ina gesundjon, þat he síðor þiu bet
liudjun gilóƀdi, hwó liof is þar
manno gihwilikumu, þan he mén gefrumit,
þat man ina aláte léðes þinges,
sakono endi sundjono, só im þó selƀo dede
heƀen-ríki god harm-ge-wurhti.
Be þiu nis mannes bág mikilun biþerƀi,
hagustaldes hróm: ef imu þiu helpe godes
geswíkid þurh is sundjon, þan is imu sán aftar þiu
breosþugi blóðora, þoh he ér bihét spreka,
hrómie fan is hildi endi fan is handkrafti,
þe man fan is megine. Þat warð þar an þemu márjon skín,
þegno bezton, þó imu is þiodanes gi-swék
hélag helpe. Beþiu ni skoldi hrómien man
te swíðo fan imu selƀon, hwand imu þar swíkid oft
wán endi willjo, ef imu waldand god,
hér heƀen-kuning herte ni sterkit.
Þan béd allaro barno bezt, bendi þolode
þurh man-kunni. Hwurƀun ina managa umbi
Judeono liudi, sprákun gelp mikil,
habdun ina te hoska, þar he giheftid stód,
þolode mid geþuldiun, só hwat só imu þiu þiod deda,
liudi léðes. Þó warð eft lioht kuman,
morgan te mannun. Manag samnoda
heri Judeono: habdun im hugi wlƀo,
inwid an innan. Warð þar éosago
an morgantíd manag gisamnod
irri endi énhard, inwideas gern,
wréðes willjan. Gengun im an warf samad
rinkos an rúna, bigunnun im rádan þó,
hwó sie gewísadin mid wár-lósun,
mannun méngewitun an mahtigna Krist
te gisęggjanne sundja þurh is selƀes word,
þat sie ina þan te wunder-kwálu wégean móstin,
adélien te dóðe. Sie ni mahtun an þemu dage finden
só wréð gewitskepi, þat sie imu wíti beþiu
adélien gidorstin efþa dóð frummjen,
líƀu bilósjen. Þó kwámun þar at laztan forð
an þena warf wero wár-lóse man
twéne gangan endi bigunnun im tęlljen an,
kwáðun þat sie ina selƀon sęggjan gi-hórdin,
þat he mahti tewerpen þena wíh godes,
allaro húso hóhost endi þurh is handmegin,
þurh is énes kraft up arihtien
an þriddion daga, só is elkor ni þorfti beþíhan man.
He þagoda endi þoloda: ni sprak imu io þiu þiod só filu,
þea liudi mid luginun, þat he it mid léðun angegin
wordun wráki. Þó þar undar þemu werode arés
baluhugdig man, biskop þero liudjo,
þe furisto þes folkes endi frágode Krist
iak ina be imu selƀon biswór swíðon éðun,
grótte ina an godes namon endi gerno bad,
þat he im þat gisagdi, ef he sunu wári
þes libbiendies godes: ʽþes þit lioht geskóp,
Krist kuning éwig. Wi ni mugun is ant-kiennien wiht
ne an þínun wordun ni an þínun werkun.ʼ Þó sprak imu eft þe wáro angegin,
þe gódo godes sunu: ʽþu kwiðis it for þesun Judeon nu,
sóðlíko segis, þat ik it selƀo bium.
Þes ni gilóƀiad mi þese liudi: ni willjad mi for-látan beþiu;
ni sind im mín word wirðig. Nu sęggju ik iu te wárun þoh,
þat gi noh skulun sittjen gisehan an þe swíðaron half godes
márjan mannes sunu, an megin-krafte
þes alo-walden fader, endi þanan eft kuman
an himilwolknun herod endi allumu heliðo kunnje
mid is wordun adélien, al só iro gewurhti sind.ʼ
Þo balg ina þe biskop, habde bittren hugi,
wréðida wið þemu worde endi is giwádi slét,
brak for is breostun: ʽnu ni þurƀun gi bídan lengʼ, kwað he,
ʽþit werod gewitskępjes, nu im sulik word farad,
ménspráka fan is múðe. Þat gi-hórid hér nu manno filu,
rinko an þesumu rakude, þat he ina só ríkjan telit,
gihid þat he god sí. Huat willjad gi Judeon þes
adélien te dóme? Is he dóðes nu
wirðig be sulikun wordun?ʼ \hld\ Þat werod al gesprak,
folk Judeono, þat he wári þes ferhes skolo,
wítjes só wirðig. Ni was it þoh be is gewurhtiun gidóen,
þat ine þar an Hierusalem Judeo liudi,
sunu drohtines sundja lósen
adéldun te dóðe. Þó was þero dádjo hróm
Judeo liudjun, hwat sie þemu godes barne mahtin
só haftemu mést, harmes gefrummjen.
Bewurpun ina þó mid werodu endi ina an is wangon slógun,
an is hleor mid iro handun — al was imu þat te hoske gidóen —,
felgidun imu firin-word \hld\ fíundo menegi,
bismer-spráka. \hld\ Stód þat barn godes
fast under fíundun: \hld\ wárun imu is faðmos ge-bundene,
þolode mid gi-þuldiun, \hld\ só hwat só imu þiu þioda tó
bittres bráhte: \hld\ ni balg ina neowiht
wið þes werodes gewin. \hld\ Þó námon ina wréðe man
só gi-bundanan, \hld\ þat barn godes,
endi ina þó léddun, \hld\ þar þero liudjo was,
þere þiade þing-hús. \hld\ Þar þegan manag
hwurƀun umbi iro hęri-togon. \hld\ Þar was iro hérron bodo
fan Rúmu-burg, \hld\ þes þe þó þes ríkjas gi-weld:
kumen was he fan þemu késure, \hld\ gisendid was he undar þat kunni Judeono
te rihtjenne þat ríki, \hld\ was þar rád-geƀo:
Pilatus was he héten; \hld\ he was fan Ponteo lande
knósles kennit. \hld\ Habde imu kraft mikil,
an þemu þing-húse \hld\ þiod gi-samnod,
an warf weros; \hld\ wár-lóse man
a-gáƀun þó þena godes sunu, \hld\ Judeo liudi,
under fíundo folk, \hld\ kwáðun þat he wári þes ferhes skolo,
þat man ina wítnodi \hld\ wápnes ęggjun,
skarpun skúrun. Ni welde þiu skole Judeono
þringan an þat þing-hús, ak þiu þiod úte stód,
mahlidun þanen wið þea menegi: ni weldun an þat gimang faren,
an eli-landige man, þat sie þar unreht word,
an þemu dage derƀies wiht adélian ne gi-hórdin,
ak kwáðun þat sie im só hluttro hélaga tídi,
weldin iro paskha halden. \hld\ Pilatus ant-feng
at þem wam-skaðun \hld\ waldandes barn,
sundja lósen. \hld\ Þó an sorgun warð
Judases hugi, \hld\ þó he a-geƀan gisah
is drohtin te dóðe, \hld\ þó bigan imu þiu dád aftar þiu
an is hugja hrewan, \hld\ þat he habde is hérron ér
sundja lósen gisald. \hld\ Nam imu þó þat siluƀar an hand,
þrítig skatto, \hld\ þat man imu ér wið is þiodane gaf,
geng imu þó te þem Judiun \hld\ endi im is grimmon dád,
sundjon sagde, \hld\ endi im þat siluƀar bód
gerno te ageƀanne: \hld\ ʽik hębbju it só griolíkoʼ, kwað he,
ʽmines drohtines dróru gikópot,
só ik wét þat it mi ni þíhit.ʼ Þiod Judeono
ni weldun it þó antfáhan, ak hétun ina forð aftar þiu
umbi sulika sundja selƀon ahton,
hwat he wið is fráhon gefrumid habdi:
ʽþu sáhi þi selƀo þesʼ, kwaðun sie; ʽhwat wili þu þes nu sóken te ús?
Ne wít þu þat þesumu werode!ʼ Þó giwét imu eft þanan
Judas gangan te þemu godes wíhe
swíðo an sorgun endi þat siluƀar warp
an þena alah innan, ne gidorste it égan leng;
fór imu þó só an forhtun, só ina fíundo barn
módage manodun: habdun þes mannes hugi
gramon undergripanen, was imu god abolgan,
þat he imu selƀon þó símon warhte,
hnég þó an herusél an hinginna,
warag an wurgil endi wíti gekós,
hard hęllje geþwing, hét endi þiustri,
diap dóðes dalu, hwand he ér umbi is drohtin swék.
Þan béd þat barn godes — bendi þolode
an þemu þing-húse —, hwan ér þiu þiod under im,
erlos énwordie alle wurðin,
hwat sie imu þan te ferahkwálu frummjan weldin.
Þó þar an þem bęnkjun arés bodo késures
fan Rúmu-burg endi geng imu wið þat ríki Judeono
módag mahlien, þar þiu menigi stód
aftar þemu hoƀe hwarƀon: ni weldun an þat hús kuman
an þemu paskhadage. Pilatus bigan
frókno frágon oƀar þat folk Judeono,
mid hwiu þe man habdi morðes giskuldit,
wítjes giwerkot: ʽbe hwí gi imu só wréðe sind,
an iuwomu hugja hótie?ʼ Sie kwáðun þat he im habdi harmes só filu,
léðes giléstid: ʽni gáƀin ina þesa liudi þi,
þar sie ina ér biforan uƀilan ni wissin,
wordun farwarhten. He haƀat þeses werodes só filu
farlédid mid is lérun — endi þesa liudi merrid,
dóit im iro hugi twíflien —, þat wi ni mótun te þemu hoƀe késures
tinsi gelden; þat mugun wi ina gitęlljen an
mid wáru gewitskepi. He sprikid ók word mikil,
kwiðit þat he Krist sí, kuning oƀar þit ríki,
begihit ina só grótes.ʼ Þó im eft tegegnes sprak
bodo késures: ʽef he só barlíkoʼ, kwað he,
ʽunder þesaru menigi ménwerk frumid,
antfáhad ina þan eft under iuwe folk-skępi, ef he sí is ferhes skolo,
endi imu só adéliad, ef he sí dóðes werð,
só it an iuwaro aldrono éo gebiode.ʼ
Sie kwáðun þó, þat sie ni móstin manno nigénumu
an þea hélagon tíd te handbanon,
werðen mid wápnun an þemu wíhdage.
Þó wende ina fan þemu werode wréðhugdig man,
þegan késures, þe oƀar þea þioda was
bodo fan Rúmu-burg —: hét imu þó þat barn godes
náhor gangan endi ina niudlíko,
frágoda frókno, ef he oƀar þat folk kuning
þes werodes wári. Þó habde eft is word garu
sunu drohtines: ʽhweðer þu þat fan þi selƀumu sprikisʼ, kwað he,
ʽþe it þi óðre hér erlos sagdun,
kwáðun umbi mínan kuningduom?ʼ Þó sprak eft þe késures bodo
wlank endi wréðmód, þar he wið waldand Krist
reðiode an þem rakude: ʽni bium ik þeses ríkjes hinanʼ, kwað he,
ʽGiudeo liudjo, ni gadoling þín,
þesaro manno mágwini, ak mi þi þius menigi bifalah,
agáƀun þi þína gadulingos mi, Judeo liudi,
haftan te handun. Huat haƀas þu harmes giduan,
þat þu só bittro skalt bendi þolojan,
kwalm undar þínumu kunnje?ʼ Þó sprak imu eft Krist angegin,
hélendero bezt, þar he giheftid stód
an þemu rakude innan: ʽnis mín ríki hinanʼ, kwað he,
ʽfan þesaru weroldstundu. Ef it þoh wári só,
þan wárin só starkmóde wiðer strídhugi,
wiðer grama þioda jungaron míne,
só man mi ni gáƀi Judeo liudjun,
hettendiun an hand an herubendiun
te wégeanne te wundrun. Te þiu warð ik an þesaru weroldi gi-boran,
þat ik gewitskepi giu wáres þinges
mid mínun kumiun kúðdi. Þat mugun ant-kęnnjen wel
þe weros, þe sind fan wáre kumane: þe mugun mín word farstanden,
gilóƀien mínun lérun.ʼ Þó ni mahte lasteres wiht
an þem barne godes bodo késures,
findan féknea word, þat he is ferhes beþiu
skuldig wári. Þó geng he im eft wið þea skola Judeono
módag mahlien endi þeru menigi sagde
oƀar hlust mikil, þat he an þemu hafton manne
sulika firinspráka finden ni mahti
for þem folkskipie, só he wári is ferhes skolo,
dóðes wirðig. Þan stódun dolmóde
Judeo liudi endi þane godes sunu
wordun wrógdun: kwáðun þat he giwer érist
begunni an Galileo lande, ʽendi oƀar Judeon fór
herodwardes þanan, hugi twíflode,
manno mód-seƀon, só he is morðes werð,
þat man ina wítnoie wápnes ęggjun,
ef eo man mid sulikun dádjun mag dóðes geskuldien.ʼ
Só wrógdun ina mid wordun werod Judeono
þurh hótean hugi. Þó þe hęri-togo,
slíð-módig man sęggjan gi-hórde,
fan hwilikumu kunnje was Krist afódid,
manno þe bezto: he was fan þeru márjan þiadu,
þe gódo fan Galilea-lande; þar was gum-skępi
eðiliero manno; Erodes biheld þar
kraftagne kuningdóm, só ina imu þe késur far-gaf,
þe ríkjo fan Rúmu, þat he þar rehto gehwilik
gefrumidi undar þemu folke endi friðu lésti,
dómos adéldi. He was ók an þemu dage selƀo
an Hierusalem mid is gum-skępi,
mid is werode at þemu wíhe: só was iro wíse þan,
þat sie þar þia hélagun tíd haldan skoldun,
paskha Judeono. Pilatus gibód þó,
þat þena hafton man heliðos námin
só gibundanan, þat barn godes,
hét þat sie ina Erodese, erlos bráhtin
haften te handun, hwand he fan is hęri-skępi was,
fan is werodes gewald. Wígand frumidun
iro hérron word: hélagne Krist
fórdun an fiteriun for þena folktogun,
allaro barno bezt, þero þe io gi-boren wurði
an liudjo lioht; an liðubendiun geng,
antat sie ina bráhtun, þar he an is bęnkja sat,
kuning Erodes: umbihwarf ina kraft wero,
wlanke wígandos: was im willjo mikil,
þat sie þar selƀon Krist gisehan móstin:
wándun þat he im sum tékan þar tógean skoldi,
mári endi mahtig, só he managun dede
þurh is god-kundi Judeo *liudjon.
Frágoda ina þuo þie folk-kuning firiwitlíko
managon wordon, wolda is muod-seƀon
forð undar-findan, hwat hie te frumu mohti
mannon gimarkon. Þan stuod mahtig Krist,
þagoda endi þoloda: ne wolda þem þied-kuninge,
Erodese ne is erlon ant-swór geƀan
wordo nigénon. Þan stuod þiu wréða þiod,
Judeo liudi endi þena godes suno
wurrun endi wruogdun, anþat im warð þie werold-kuning
an is huge huoti endi all is hęri-skipi,
farmuonstun ina an iro muode: \hld\ ne ant-kendun maht godes,
himiliskan hérron, \hld\ ak was im iro hugi þiustri,
baluwes gi-blandan. \hld\ Barn drohtines
iro wréðun werk, \hld\ word endi dádi
þuru ód-muodi \hld\ all giþoloda,
só hwat só sia im tionono þuo \hld\ tuogjan woldun.
Sia hietun im þuo te hoske \hld\ hwít gi-wádi
umbi is liði lęggjan, \hld\ þiu mér hie wurði þem liudjon þar,
jungron te gamne. \hld\ Judeon faganodun,
þuo sia ina te hoske \hld\ hębbjan gisáhun,
erlos oƀar-muoda. \hld\ Þuo senda ina eft þanan
Erodes se kuning an þat óðer folk;
alédian hiet ina lungra mann, endi lastar sprákun,
felgidun im firin-word, \hld\ þar hie an feteron geng
bi-hlagan mid hosku: \hld\ ni was im hugi twífli,
neƀa hie it þuru ód-muodi \hld\ all gi-þoloda;
ne welda iro uƀilun word \hld\ idug-lónon,
hosk endi harmkwidi. \hld\ Þuo bráhtun sia ina eft an þat hús innan,
an þia palenkja uppan, \hld\ þar Pilatus was
an þero þing-stędi. \hld\ Þegnos a-gáƀun
barno þat besta \hld\ banon te handon
sundi-lósjan, \hld\ só hie selƀo gi-kós:
welda manno barn \hld\ morðes atuomian,
nęrjan af nódi. \hld\ Stuodun níð-hwata,
Judeon far þem gast-sęlje: \hld\ habdun sia gramono barn,
þia skola far-skundid, \hld\ þat sia ne be-skriƀun iowiht
grimmera dádjo. \hld\ Þuo giwét im gangan þarod
þegan késures wið þia þiod sprekan,
hard hęri-togo: ʽhwat, gi mi þesan haftan mannʼ, kwaþie,
ʽan þesan seli sendun endi selƀon anbudun,
þat hie iuwes werodes só filo awerdit habdi,
farlédid mid is léron. Nu ik mid þeson liudon ni mag,
findan mid þius folku, þat hie is ferahes sí
furi þesaro skolu skuldig. Skín was þat hiudu:
Erodes mohta, þie iuwan éo bikan,
iuwaro liudo landreht, hie ni mahta is líƀes gifréson,
þat hie hier þuru éniga sundja te dage sweltan skoldi,
líf farlátan. Nu willju ik ina for þeson liudjon hier
gi-þróon mid þingon, þrístion wordun,
buotjan im is briost-hugi, látan ina brúkan forð
ferahes mid firjon.ʼ Folk Judeono
hreopun þuo alla samad hlúdero stemnu,
hietun flít-líko ferahes áhtjan
Krist mid kwalmu endi an krúki slahan,
wégian te wundron: ʽhie mid is wordon haƀit
dóðes giskuldid: sagit þat hie drohtin sí,
gegnungo godes suno. Þat hie ageldan skal,
inwid-spráka, só is an úson éwe giskriƀan,
þat man sulika firin-kwidi \hld\ ferahu kópo.ʼ
Þuo warð þie an forahton, þie þes folkes gi-weld,
mikilon an is muode, þuo hie gi-hórda þia man sprekan,
þat sia ina selƀon sęggjan gi-hórdin,
gehan fur þem gum-skipe, þat hie wári godes suno.
Þuo hwarf im eft þie hęri-togo an þat hús innan
te þero þing-stędi, þrístion wordon
gruotta þena godes suno endi frágoda, hwat hie gumono wári:
ʽhwat bist þu manno?ʼ kwaþie. ʽTe hwí þu mi só þínan muod hilis,
dernis diopgiþáht? wést þu þat it all an mínon duome stéd
umbi þínes líƀes gilagu? Mi þi hębbjat þesa liudi fargeƀan,
werod Judeono, þak ik giwaldan muot
só þik te spildianne an speres orde,
só ti kwęlljanne an krúkium, só kwikan látan,
só hweðer só mi selƀon suotera þunkit
te gifrummjanne mid mínu folku.ʼ Þuo sprak eft þat friðu-barn godes:
ʽwést þu þat te wáronʼ, kwaþie, ʽþat þu giwald oƀar mik
hębbjan ni mohtis, ne wári þat it þi hélag god
selƀo fargáƀi? Ôk hębbjat þia sundjono mér,
þia mik þi bifulhun þuru fíondskipi,
gisaldun an símon haftan.ʼ Þuo welda ina síð after þiu
gramhugdig man gerno farlátan,
þegan késures, þar hie is haƀdi for þero þioda giwald;
ak sia weridun im þena willjon wordu gihwiliku,
kunni Judeono: ʽne bist þuʼ, kwáðun sia, ʽþes késures friund,
þínon hérren hold, ef þu ina hinan látis
síðon gisundon: þat þi noh te soragan mag,
werðan te wíte, hwand só hwe só sulik word sprikit,
ahaƀið ina só hóho, kwiðit þat hie hębbjan mugi
kuningduomes namon, ne sí þat ina im þie késur geƀe,
hie wirrid im is weruld-ríki \hld\ endi is word far-hugid,
farman ina an is muode. \hld\ Beþiu skalt þu sulik mén wrekan,
hoskword manag, \hld\ ef þu umbi þínes hérren ruokis,
umbi þínes fróhon friund-skipi, \hld\ þan skalt þu ina þiu ferhu beniman.ʼ
Þuo gi-hórda þie hęri-togo \hld\ þia héri Juðeono
þrégian fan is þiodne; \hld\ þuo hie far þero þing-stędi geng
selƀo gi-sittjan, \hld\ þar gi-samnod was
só mikil warf werodes, \hld\ hiet waldand Krist
lédian for þia liudi. \hld\ Langoda Judeon,
hwan ér sia þat hélaga barn \hld\ hangon gi-sáwin,
kwelan an krúkje; \hld\ sia kwáðun þat sia kuning óðran
ne haƀdin undar iro hęri-skipje, \hld\ neƀan þena héran késar
fan Rúmu-burg: \hld\ ʽþie haƀit hier ríki oƀer ús.
Beþiu ni skalt þu þesan far-látan; \hld\ hie haƀit ús só filo léðes gisprokan,
farduan haƀit hie im mid is dádjon. \hld\ Hie skal dóð þolon,
wíti endi wundar-kwála.ʼ \hld\ Werod Judeono
só manag mislík þing \hld\ an mahtigna Krist
sagdun te sundjun. \hld\ Hie swígondi stuod
þuru óð-muodi, \hld\ ne ant-wordida niowiht
wið iro wréðun word: \hld\ wolda þesa werold alla
lósjan mid is líƀu: \hld\ biþiu liet hie ina þia léðun þiod
wégian te wundron, \hld\ all só iro willjo geng:
ni wolda im opan-líko \hld\ allon kúðjan
Judeo liudjon, þat hie was god selƀo;
hwand wissin sia þat te wáron, þat hie sulika giwald haƀdi
oƀar þeson middil-gard, þan wurði im iro muod-seƀo
gi-blóðit an iro brioston: þan ne gidorstin sia þat barn godes
handon ant-hrínan: þan ni wurði heƀan-ríki,
antlokan liohto mést liudjo barnon.
Beþiu méð hie is só an is muode, ne lét þat manno folk
witan, hwat sia warahtun. Þiu wurd náhida þuo,
mári maht godes endi middi dag,
þat sia þia ferahkwála frummjan skoldun.
Þan lag þar ók an bendion an þero burg innan
én ruof reginskaðo, þie habda under þem ríke só filo
morðes girádan endi manslahta gifrumid,
was mári megin-þiof: ni was þar is gimako hwergin;
was þar ók bi sínon sundjon giheftid,
Barrabas was hie hétan; hie after þem burgion was
þuru is méndádi manogon gikúðid.
Þan was landwísa liudjo Judeono,
þat sia iáro gihwen an godes minnja
an þem hélagon dage énna haftan mann
abiddjan skoldun, þat im iro burges ward,
iro folktogo ferah fargáƀi.
Þuo bigan þie hęri-togo þia héri Judeono,
þat folk frágojan, þar sia im fora stuodun,
hweðeron sia þero tueio tuomian weldin,
ferahes biddjan: ʽþia hier an feteron sind
haft undar þeson hęri-skipje?ʼ Þiu héri Judeono
habdun þuo þia aramun man alla gispanana,
þat sia þemo landskaðen líf abádin,
giþingodin þem þioƀe, þie oft an þiustria naht
wam giwarahta, endi waldand Krist
kwelidin an krúkje. Þuo warð þat kúð oƀar all,
hwó þiu þiod haƀda duomos adélid. Þuo skoldun sia þia dád frummjan,
háhan þat hélaga barn. Þat warð þem hęri-togen
síðor te sorgon, þat hie þia saka wissa,
þat sia þuru níð-skipi nęrjendon Krist,
hatoda þiu héri, endi hie im hórda te þiu,
warahta iro willjon: þes hie wíti ant-feng,
lón an þeson liohte endi lang after,
wói síðor wann, síðor hie þesa werold agaf.
Þuo warð þas þie wréðo giwaro, wam-skaðono mést,
Satanas selƀo, þuo þiu seola kwam
Judases an grund grimmaro hęlljun —
þuo wissa hie te wáren, þat þat was waldand Krist,
barn drohtines, þat þar gibundan stuod;
wissa þuo te wáron, þat hie welda þesa werold alla
mid is henginnia hellia giþwinges,
liudi a-lósjan an lioht godes.
Þat was Satanase sér an muode,
tulgo harm an is hugje: welda is helpan þuo,
þat im liudjo barn \hld\ líf ne binámin,
ne kwelidin an krúkje, ak hie welda, þat hie kwik liƀdi,
te þiu þat firiho barn fernes ne wurðin,
sundjono sikura. Satanas giwét im þuo,
þar þes hęri-togen híwiski was
an þero burg innan. Hie þero is brúdi bigann,
þera idis opanlíko unhiuri fíond
wunder tógian, þat sia an wordhelpon
Kriste wári, þat hie muosti kwik libbjan,
drohtin manno — hie was iu þan te dóðe giskerid —
wissa þat te wáron, þat hie im skoldi þia giwald biniman,
þat hie sia oƀar þesan middil-gard só mikila ni haƀdi,
oƀar wída werold. Þat wíf warð þuo an forahton,
swíðo an sorogon, þuo iru þiu gisiuni kwámun
þuru þes dernien dád an dages liohte,
an heliðhelme bihelid. Þuo siu te iru hérren anbód,
þat wíf mid iro wordon endi im te wáren hiet
selƀon sęggjan, hwat iro þar te gisiunion kwam
þuru þena hélagan mann, endi im helpan bad,
formon is ferhe: ʽik hębbju hier só filo þuru ina
seldlíkes gisewan, só ik wét, þat þia sundjun skulun
allaro erlo gihwem uƀilo giþíhan,
só im fruokno tuo ferahes áhtið.ʼ
Þie segg warð þuo an síðe, antat hie sittjan fand
þena hęri-togon an hwaraƀe innan
an þem sténwege, þar þiu stráta was
felison gifuogid. Þar hie te is fróhon geng,
sagda im þes wíƀes word. Þuo warð im wréð hugi,
þem hęri-togen, — hwaraƀoda an innan —,
giblóðit briost-gi-þáht: was im béðies wé,
gie þat sea ina sluogin sundja lósan,
gie it bi þem liudjon þuo for-látan ne gidorsta
þuru þes werodes word. Warð im giwendid þuo
hugi an herten after þero héri Judeono,
te werkjanne iro willjon: ne wardoda im niewiht
þia swárun sundjun, þia hie im þar þuo selƀo gideda.
Hiet im þuo te is handon dragan hluttran brunnion,
watar an wégie, þar hie furi þem werode sat,
þuóg ina þar for þero þioda þegan késures,
hard hęri-togo endi þuo fur þero héri sprak,
kwað þat hie ina þero sundjono þar sikoran dádi,
wréðero werko: ʽne willju ik þes wihtes pleganʼ, kwaþie,
ʽumbi þesan hélagan mann, ak hleotad gi þes alles,
gie wordo gie werko, þes gi im hér te wítje giduan.ʼ
Þuo hreop all saman hęri-skipi Judeono,
þiu mikila menigi, kwáðun þat sia weldin umbi þena man plegan
deraƀoro dádjo: ʽfare is drór oƀar ús,
is bluod endi is baneði endi oƀar úsa barn só samo,
oƀar úsa aƀaron þar after — wi willjat is alles pleganʼ, kwaðun sia,
ʽumbi þena slegi selƀon, — ef wi þar éniga sundja giduan!ʼ
Ageƀan warð þar þuo furi þem Judeon allaro gumono besta
hettendion an hand, an herubendion
narawo ginódid, þar ina níð-hwata,
fíond ant-fengun: folk ina umbihwarf,
ménskaðono megin. Mahtig drohtin
þoloda giþuldion, só hwat só im þiu þioda deda.
Sia hietun ina þuo fillian, ér þan sia im ferahes tuo,
aldres áhtin, endi im undar is ógun spiwun,
dedun im þat te hoske, þat sia mid iro handon slógun,
weros an is wangun endi im is giwádi binámun,
róƀodun ina þia reginskaðon, ródes lakanes
dedun im eft óðer an þuru unhuldi;
hietun þuo hóƀidband hardaro þorno
wundron windan endi an waldand Krist
selƀon sęttjan, endi gengun im þia gisíðos tuo,
kweddun ina an kuning-wísu endi þar an knio fellun,
hnigun im mid iro hóƀdu: all was im þat te hoske giduan,
þoh hie it all giþolodi, þiodo drohtin,
mahtig þuru þia minnja manno kunnjes.
Hietun sia þuo wirkian wápnes ęggjon
heliðos mid iro handon hardes bómes
kraftiga krúki endi hietun sia Kristan þuo,
sálig barn godes selƀon fuorian,
dragan hietun sia úsan drohtin, þar hie bedróragad skolda
sweltan sundjono lós. Síðodun Judeon,
weros an willon, léddun waldand Krist,
drohtin te dóðe. Þar mohta man þuo dereƀi þing
harmlík gi-hórjan: hioƀandi þar after
gengun wíf mid wópu, weros gnornodun,
þia fan Galilea mid im gangan kwámun,
folgodun oƀar ferrwegos: was im iro fróhon dóð
swíðo an soragan. Þuo hie selƀo sprak,
barno þat besta endi under bak besah,
hiet þat sia ni wépin: ʽni þarf iu wiht treganʼ, kwaþie,
ʽmínero hinferdio, ak gi mid hofnu mugun
iuwa wréðan werk wópu kúmian,
tornon trahnon. Noh wirðið þiu tíd kuman,
þat þia muoder þes mendendia sind,
brúdi Judeono, þem gio barn ni warð
ódan an aldre. Þan gi iuwa inwid skulun
grimmo angeldan; þan gi só gerna sind,
þat iu hier bihlídan hóha bergos,
diopo bedelƀan; dóð wári iu þan allon
lioƀera an þeson lande þan sulik liudjo kwalm
te giþolianne, só hier þan þesaro þioda kumid.ʼ
Þuo sia þar an griete galgon rihtun,
an þem felde uppan folk Judeono,
bóm an berege, endi þar an þat barn godes
kwelidun an krúkje: \hld\ slógun kald ísarn,
niwa naglos \hld\ níðon skarpa
hardo mid hamuron \hld\ þuru is hendi endi þuru is fuoti,
bittra bendi: \hld\ is blód ran an erða,
drór fan úson drohtine. \hld\ Hie ni welda þoh þia dád wrekan
grimma an þem Judeon, \hld\ ak hie þes god fader
mahtigna bad, \hld\ þat hie ni wári þem manno folke,
þem werode þiu wréðra: \hld\ ʽhwand sia ni witun, hwat sia duotʼ, kwaþie.
Þuo þia wígandos \hld\ giwádi Kristes,
drohtines déldun, \hld\ dereƀia mann,
þes ríken giróbi. \hld\ Þia rinkos ni mahtun
umbi þena selƀon {[...]} \hld\ sam-wurdi gisprekan,
ér sia an iro hwaraƀe \hld\ hlótos wurpun,
hwilik iro skoldi hębbjan \hld\ þia hélagun péda,
allaro gi-wádjo wun-samost. \hld\ Þes werodes hirdi
hiet þuo, þe hęri-togo, \hld\ oƀar þem hóƀde selƀes
Kristes an krúke skríƀan, \hld\ þat þat wári kuning Judeono,
Jesus fan Nazareth-burh, \hld\ þie þar neglid stuod
an niwon galgon þuru \hld\ níð-skipi,
an bómin treo. \hld\ Þuo bádun þia liudi
þat word wendian, \hld\ kwáðun þat hie im só an is willjon spráki,
selƀo sagdi, \hld\ þat hie habdi þes gisíðes giwald,
kuning wári oƀar Judeon. \hld\ Þuo sprak eft þie késures bodo,
hard hęri-togo: \hld\ ʽit ist iu só oƀar is hóƀde giskriƀan,
wíslíko giwritan, \hld\ só ik it nu wendian ni mag.ʼ
Dádun þuo þar te wítje \hld\ werod Judeono
twéna far-talda man \hld\ an twá halƀa
Kristes an krúki: lietun sia kwalm þolon
an þem waragtrewe werko te lóne,
léðaro dádjo. Þia liudi sprákun
hoskword manag hélagon Kriste,
grottun ina mid gelpu: sáwun allaro gumono þen beston
kwelan an þemo krúkje: ʽef þu sís kuning oƀar allʼ, kwáðun sia,
ʽsuno drohtines, só þu haƀis selƀo gisprokan,
neri þik fan þero nódi endi níðes atuomi,
gang þi hél herod; þan welliat an þik heliðo barn,
þesa liudi gilóƀian.ʼ Sum imo ók lastar sprak
swíðo gélhert Judeo, þar hie fur þem galgon stuod:
ʽwah warð þesaro weroldiʼ, kwaþie, ʽef þu iro skoldis giwald égan.
Þu sagdas þat þu mahtis an énon dage all tewerpan
þat hóha hús heƀan-kuninges,
sténwerko mést endi eft standan giduon
an þriddion dage, só is elkor ni þorfti biþíhan mann
þeses folkes furðor. Sínu hwó þu nu gi-fastnod stés,
swíðo gi-sérid: ni maht þi selƀon wiht
balowes gi-buotjan.ʼ \hld\ Þuo þar ók an þem bendion sprak
þero þeoƀo óðer, \hld\ all só hie þia þioda gi-hórda,
wréðon wordon \hld\ —ne was is willjo guod,
þes þegnes gi-þáht—: \hld\ ʽef þu sís þiod-kuningʼ, kwaþie,
ʽKrist, godes suno, \hld\ gang þi þan fan þem krúke niðer,
slópi þi fan þem símon \hld\ endi ús samad allon
hilp endi héli. \hld\ Ef þu sís heƀan-kuning,
waldand þesaro weroldes, \hld\ gi-duo it þan an þínon werkon skín,
mári þik fur þesaro menigi.ʼ \hld\ Þuo sprak þero manno óðer
an þero hęnginna, \hld\ þar hie gi-hęftid stuod,
wan wunder-kwála: \hld\ ʽbe-hwí wilt þu sulik word sprekan,
gruotis ina mid gelpu? \hld\ stés þi hier an galgen haft,
gi-brókan an bóme. \hld\ Wit hier béðia þolod
sér þuru unka sundjun: \hld\ is unk unkero selƀero dád
worðan te wítje. \hld\ Hie stéd hier wammes lós,
allaro sundjono sikur, \hld\ só hie selƀo gio
firina ni gi-frumida, \hld\ botan þat hie þuru þeses folkes nið
willendi an þesaro weruldi \hld\ wíti ant-fáhid.
Ik willju þar gi-lóƀjan tuoʼ, \hld\ kwaþie, ʽendi willju þena landes ward,
þena godes suno \hld\ gerno biddjan,
þat þu mín gi-huggjes \hld\ endi an helpun sís,
rádendero best, \hld\ þan þu an þín ríki kumis:
wes mi þan gi-náðig.ʼ \hld\ Þuo sprak im eft nęrjendo Krist
wordon te-gegnes: \hld\ ʽik sęggju þi te wáron hierʼ, kwaþie,
ʽþat þu noh hiudu móst \hld\ an himil-ríke
mid mi samad \hld\ sehan lioht godes,
an þemo paradyse, \hld\ þoh þu nu an sulikoro pínu sís.ʼ
Þan stuod þar ók Maria, \hld\ muoder Kristes,
blék under þem bóme, \hld\ gi-sah iro barn þolon,
winnan wunder-kwála. \hld\ Ôk wárun þar wíf mid iro
an só mahtiges \hld\ minnja kumana—
þan stuod þar ók Johannes, \hld\ jungro Kristes,
hriwi undar is hérren, \hld\ was im is hugi sérag—
drúƀodun fur þem dóðe. \hld\ Þar sprak drohtin Krist
mahtig te þero muoder: \hld\ ʽnu ik þi hier mínemo skal
jungron be-felhan, \hld\ þem þi hier gegin-ward stéd:
wis þi an is gi-síðje samad: \hld\ þu skalt ina furi suno hębbjan.ʼ
Grótta hie þuo Johannes, \hld\ hiet þat hie iru fulgengi wel,
minnjodi sia só mildo, \hld\ só man is muoder skal,
idis un-wamma. \hld\ Þuo hie sia an is éra ant-feng
þuru hluttran hugi, \hld\ só im is hérro gibód.
Þuo warð þar an middjan dag \hld\ mahtig tékan,
wundar-lík gi-waraht \hld\ oƀar þesan werold allan,
þuo man þena godes suno \hld\ an þena galgon huof,
Krist an þat krúki: \hld\ þuo warð it kúð oƀar all,
hwó þiu sunna warð gi-sworkan: \hld\ ni mahta swigli lioht
skóni gi-skínan, \hld\ ak sia skado far-feng,
þimm endi þiustri \hld\ endi só gi-þrusmod neƀal.
Warð allaro dago druoƀost, \hld\ dunkar swíðo
oƀar þesan wídun weruld, \hld\ só lango só waldand Krist
kwal an þemo krúkje, \hld\ kuningo ríkost,
ant nuon dages. \hld\ Þuo þie neƀal tiskréd,
þat gi-swerk warð þuo te-swungan, \hld\ bi-gan sunnun lioht
hédron an himile. \hld\ Þuo hreop up te gode
allaro kuningo kraftigost, \hld\ þuo hie an þemo krúkje stuod
faðmon gi-fastnot: \hld\ ʽfader alo-mahtigʼ, kwaþie,
ʽte hwí þu mik só far-lieti, \hld\ lieƀo drohtin,
hélag heƀan-kuning, endi þína helpa dedos,
fullisti só ferr? \hld\ Ik standu under þeson fíondon hier
wundron gi-wégid.ʼ \hld\ Werod Judeono
hlógun is im þuo te hoske: \hld\ gi-hórdun þena hélagun Krist,
drohtin furi þem dóðe \hld\ drinkan biddjan,
kwað þat ina þurstidi. \hld\ Þiu þioda ne latta,
wréða wiðar-sakon: \hld\ was im willjo mikil,
hwat sia im bittres tuo \hld\ bringan mahtin.
Habdun im un-swóti \hld\ ekid endi galla
gi-mengid þia mén-hwaton; \hld\ stuod én mann garo,
swíðo skuldig skaðo, \hld\ þena habdun sia gi-skerid te þiu,
far-spanan mid sprákon, \hld\ þat hie sia en éna spunsia nam,
líðo þes léðosten, \hld\ druog it an énon langan skafte,
gi-bundan an énon bóme \hld\ endi deda it þem barne godes,
mahtigon te múðe. \hld\ Hie an-kenda iro mirkiun dádi,
gi-fuolda iro fégnes: \hld\ furðor ni welda
is só bittres an-bítan, \hld\ ak hreop þat barn godes
hlúdo te þem himiliskon fader: \hld\ ʽik an þina hendi be-filhuʼ, kwaþie,
ʽmínon gést an godes willjon; \hld\ hie ist nu garo te þiu,
fús te faranne.ʼ \hld\ Firiho drohtin
gi-hnégida þuo is hóƀid, \hld\ hélagon áðom
liet fan þemo lík-hamen. \hld\ Só þuo þie landes ward
swalt an þem símon, \hld\ só warð sán after þiu
wundar-tékan giwaraht, \hld\ þat þar waldandes dóð
un-kweðandes só filo \hld\ ant-kęnnian skolda,
þiadnes éndagon: \hld\ erða biƀoda,
hrisidun þia hóhun bergos, \hld\ harda sténos kluƀun,
felisos after þem felde, \hld\ endi þat féha lakan tebrast
an middjon an twé, \hld\ þat ér managan dag
an þemo wíhe innan \hld\ wundron gi-striunid
hél hangoda \hld\ —ni muostun heliðo barn,
þia liudi skawon, \hld\ hwat under þemo lakane was
hélages be-hangan: \hld\ þuo mohtun an þat horð sehan
Judeo liudi— \hld\ graƀu wurðun gi-opanod
dódero manno, \hld\ endi sia þuru drohtines kraft
an iro lík-hamon \hld\ libbjandi a-stuodun
up fan erðu \hld\ endi wurðun gi-ógida þar
mannon te márðu. \hld\ Þat was só mahtig þing,
þat þar Kristes dóð \hld\ ant-kęnnian skoldun,
só filo þes gi-fuolian, \hld\ þie gio mid firihon ne sprak
word an þesaro weroldi. \hld\ Werod Judeono
sáwun seld-lík þing, \hld\ ak was im iro slíði hugi
só far-hardod an iro herten, \hld\ þat þar io só hélag ni warð
tékan gi-tógid, \hld\ þat sia trúodin þiu bat
an þia Kristes kraft, \hld\ þat hie kuning oƀar all,
þes werodes wári. \hld\ Suma sia þar mid iro wordon gi-sprákun,
þia þes hréwes þar \hld\ huodian skoldun,
þat þat wári te wáren \hld\ waldandes suno,
godes gegnungo, \hld\ þat þar an þem galgon swalt,
barno þat besta. \hld\ Slógun an iro briost filo
wópjandero wíƀo: \hld\ was im þiu wunder-kwála
harm an iro herten \hld\ endi iro hérren dóð
swíðo an sorogon. \hld\ Þan was sido Judeono,
þat sia þia haftun þuru þena hélagon dag \hld\ hangon ni lietin
lengerun hwíla, \hld\ þan im þat líf skriði,
þiu seola besunki: \hld\ slíð-muoda mann
gengun im mid níð-skipiu náhor, \hld\ þar só beneglida stuodun
þeoƀos twéna, \hld\ þolodun béðia
kwála bi Kriste: \hld\ wárun im kwika noh þan,
untþat sia þia grimmun \hld\ Judeo liudi
bénon be-brákon, \hld\ þat sia béðia samad
líf far-lietun, \hld\ suohtun im lioht óðer.
Sia ni þorftun drohtin Krist \hld\ dóðes bédjan
furðor mid énigon firinon: \hld\ fundun ina gi-faranan þuo iu:
is seola was gi-sendid \hld\ an suoðan weg,
an lang-sam lioht, \hld\ is liði kuolodun,
þat ferah was af þem fléske. \hld\ Þuo geng im én þero fíondo tuo
an níð-hugi, \hld\ druog negilid sper
hard an is handon, \hld\ mid heru-þrummjon stak,
liet wápnes ord \hld\ wundum sníðan,
þat an selƀes warð \hld\ sídu Kristes
ant-lokan is lík-hamo. \hld\ Þia liudi gi-sáwun,
þat þanan bluod endi water \hld\ béðiu sprungun,
wellun fan þero wundun, \hld\ all só is willjo geng
endi hie habda gi-markod ér \hld\ manno kunnje,
firiho barnon te frumu: \hld\ þuo was it all gi-fullid só.
Só þuo gi-ségid warð \hld\ seðle náhor
hédra sunna \hld\ mid heƀan-tunglon
an þem druoƀen dage, \hld\ þuo geng im úses drohtines þegan
—was im glau gumo, \hld\ jungro Kristes
managa hwíla, \hld\ só it þar manno filo
ne wissa te wáron, \hld\ hwand hie it mid is wordon hal
Juðeono gum-skipje: \hld\ Joseph was hie hétan,
darnungo was hie úses drohtines jungro: \hld\ hie ni welda þero far-duanun þiod
folgon te énigon firin-werkon, \hld\ ak hie béd im under þem folke Judeono,
hélag himilo ríkjes— \hld\ hie geng im þuo wið þena hęri-togon mahlian,
þingon wið þena þegan késures, \hld\ þigida ina gerno,
þat hie muosti a-lósjan \hld\ þena lík-hamon
Kristes fan þemo krúkje, þie þar gikwelmid stuod,
þes guoden fan þem galgen endi an graf lęggjan,
foldu bifelahan. Im ni welda þie folktogo þuo
wernian þes willjen, ak im giwald far-gaf,
þat hie só muosti gifrummjan. Hie giwét im þuo forð þanan
gangan te þem galgon, þar hie wissa þat godes barn,
hréo hangondi hérren sínes,
nam ina þuo an þero niwun ruodun endi ina fan naglon atuomda,
ant-feng ina mid is faðmon, só man is fróhon skal,
lioƀes lík-hamon, endi ina an líne biwand,
druog ina diurlíko — só was þie drohtin werð —,
þar sia þia stedi haƀdun an énon sténe innan
handon gihawuan, þar gio heliðo barn
gumon ne bigruoƀon. Þar sia þat godes barn
te iro landwísu, líko hélgost
foldu bifulhun endi mid énu felisu belukun
allaro graƀo guodlíkost. Griotandi sátun
idisi armskapana, þia þat all forsáwun,
þes gumen grimman dóð. Giwitun im þuo gangan þanan
wópjandi wíf endi wara námun,
hwó sia eft te þem graƀe gangan mahtin:
haƀdun im far-sewana soroga ginuogia,
mikila muod-kara: Maria wárun sia hétana,
idisi armskapana. Þuo warð áƀand kuman,
naht mid neflu. Niðfolk Judeono
warð an moragan eft, menigi gisamnod,
rekidun an rúnon: ʽhwat, þu wést, hwó þit ríki was
þuru þesan énan man all gitwíflid,
werod giworran: nu ligid hie wundon siok,
diopa bidolƀan. Hie sagda simnen, þat hie skoldi fan dóðe astandan
an þriddian dage. Þius þiod gilóƀit te filo,
þit werod after is wordon. Nu þu hier wardon hét,
oƀar þem graƀe gómian, þat ina is jungron þar
ne farstelan an þemo sténe endi sęggjan þan, þat hie astandan sí,
ríki fan raston: þan wirðit þit rinko folk
mér gimerrid, ef sia it biginnat márjan hier.ʼ
Þuo wurðun þar giskerida fan þero skolu Judeono
weros te þero wahtu: giwitun im mid iro giwápnion þarod
te þem graƀe gangan, þar sia skoldun þes godes barnes
hréwes huodian. Warð þie hélago dag
Judeono far-gangan. Sia oƀar þemo graƀe sátun,
weros an þero wahtun wannom nahton,
bidun undar iro bordon, hwan ér þie berehto dag
oƀar middil-gard mannon kwámi,
liudon te liohte. Þuo ni was lang te þiu,
þat þar warð þie gést kuman be godes krafte,
hálag áðom undar þena hardon stén
an þena lík-hamon. Lioht was þuo giopanod
firiho barnon te frumu: was ferkal manag
ant-hęftid fan helldoron endi te himile weg
giwaraht fan þesaro weroldi. Wánom up astuod
friðu-barn godes, fuor im þuo þar hie welda,
só þia wardos þes wiht ni afsuoƀun,
derƀia liudi, hwan hie fan þem dóðe astuod,
arés fan þero rastun. Rinkos sátun
umbi þat graf útan, Judeo liudi,
skola mid iro skildion. Skréd forð-wardes
suigli sunnun lioht. Síðodun idisi
te þem graƀe gangan, gumkunnjes wíf,
Mariun munilíka: habdun méðmo filo
gisald wiðer salƀum, siluƀres endi goldes,
werðes wiðer wurtjon, só sia mahtun awinnan mést,
þat sia þena lík-hamon lioƀes hérren,
suno drohtines, salƀon muostin,
wundun writanan. Þiu wíf soragodun
an iro seƀon swíðo, endi suma sprákun,
hwie im þena grótan stén fan þemo graƀe skoldi
gihwereƀian an halƀa, þe sia oƀar þat hréo sáwun
þia liudi lęggjan, þuo sia þena lík-hamon þar
befulhun an þemo felise. Só þiu frí haƀdun
gegangan te þem gardon, þat sia te þem graƀe mahtun
gisehan selƀon, þuo þar suógan kwam
ęngil þes alo-waldon oƀana fan radure,
faran an feðerhamon, þat all þiu folda an skian,
þiu erða dunida endi þia erlos wurðun
an wékan hugje, wardos Juðeono,
bifellun bi þem forahton: ne wándun ira ferah égan,
líf langerun hwíl. \hld\ Lágun þa wardos,
þia gisíðos sámkwika: sán up ahléd
þie gróto stén fan þem graƀe, só ina þie godes ęngil
gihweriƀida an halƀa, endi im uppan þem hléwe gisat
diurlík drohtines bodo. Hie was an is dádjon gelík,
an is ansiunion, só hwem só ina muosta undar is ógon skawon,
só bereht endi só blíði all só bliksmun lioht;
was im is giwádi wintarkaldon
snéwe gilíkost. Þuo sáwun sia ina sittjan þar,
þiu wíf uppan þem giwendidan sténe, endi im fan þem wlitie kwámun,
þem idison sulika egison tegegnes: all wurðun fan þem grurie
þiu frí an forahton mikilon, furðor ne gidorstun
te þemo graƀe gangan, ér sia þie godes ęngil,
waldandes bodo wordon gruotta,
kwað þat hie iro árundi all bikunsti,
werk endi willjon endi þero wíƀo hugi,
hiet þat sia im ne andrédin: ʽik wét þat gi iuwan drohtin suokat,
nęrjendon Krist fan Nazareth-burg,
þena þi hier kwelidun endi an krúki slógun
Judeo liudi endi an graf lagdun
sundi-lósjan. Nu nist hie selƀo hier,
ak hie ist astandan iu, endi sind þesa stedi lárea,
þit graf an þeson griote. Nu mugun gi gangan herod
náhor mikilu — ik wét þat is iu ist niud sehan
an þeson sténe innan —: hier sind noh þia stedi skína,
þar is lík-hamo lag.ʼ Lungra fengun
gibada an iro brioston bléka idisi,
wlitiskóni wíf: was im wilspell mikil
te gi-hórjanne, þat im fan iro hérren sagda
ęngil þes alo-walden. Hiet sia eft þanan
fan þem graƀe gangan endi faran te þem jungron Kristes,
sęggjan þem is gisíðon suoðon wordon,
þat iro drohtin was fan dóðe astandan.
Hiet ók an sundron Símon Petruse
will-spell mikil wordon kúðjan,
kumi drohtines, gie þat Krist selƀo
was an Galileo land, ʽþar ina eft is jungron skulun,
gisehan is gisíðos, só hie im ér selƀo gisprak
wáron wordon.ʼ Reht só þuo þiu wíf þanan
gangan weldun, só stuodun im tegegnes þar
ęngilos twéna an alahwíton
wánamon giwádjon endi sprákun im mid iro wordon tuo
hélaglíko: hugi warð giblóðid
þen idison an egison: ne mahtun an þia ęngilos godes
bi þemo wlite skawon: was im þiu wánami te strang,
te swíði te sehanne. Þuo sprákun im sán angegin
waldandes bodun endi þiu wíf frágodun,
te hwí sia Kristan þarod kwikan mid dódon,
suno drohtines suokian kwámin
ferahes fullan; ʽnu gi ina ni findat hier
an þeson sténgraƀe, ak hie ist astandan nu
an is lík-hamen: þes gi gilóƀian skulun
endi gihuggian þero wordo, þe hie iu te wáron oft
selƀo sagda, þan hie an iuwon gisíðe was
an Galilea-lande, hwó hie skoldi gigeƀan werðan,
gisald selƀo an sundigaro manno,
hettiandero hand, hélag drohtin,
þat sia ina kwelidin endi an krúki slógin,
dódan gidádin endi þat hie skoldi þuru drohtines kraft
an þriddion dage þioda te willjon
libbjandi astandan. Nu haƀit hie all giléstid só,
gifrumid mid firihon: íliat gi nu forð hinan,
gangat gáhlíko endi duot it þem is jungron kúð.
Hie haƀit sia iu furfarana endi ist im forð hinan
an Galileo land, þar ina eft is jungron skulun,
gisehan is gisíðos.ʼ Þuo warð sán after þiu
þem wiƀon an willon, þat sia gi-hórdun sulik word sprekan,
kúðjan þia kraft godes — wárun im só akumana þuo noh
gie só forahta gefrumida —: giwitun im forð þanan
fan þem graƀe gangan endi sagdun þem jungron Kristes
seldlík gisiuni, þar sia sorogondi
bidun sulikero buota. Þuo wurðun ók an þia burg kumana
Judeono wardos, þia oƀar þemo graƀe sátun
alla langa naht endi þes lík-hamen þar,
huodun þes hréwes. Sia sagdun þero héri Judeono,
hwilika im þar andwarda egison kwámun,
seldlík gisiuni, sagdun mid wordon,
al só it giduan was an þero drohtines kraft,
ni miðun an iro muode. Þuo budun im méðmo filo
Judeo liudi, gold endi siluƀar,
saldun im sink manag, te þiu þat sia it ni sagdin forð,
ne máridin þero menigi: ʽak kweðat þat iu móði hugi
answeƀidi mid slápu endi þat þar kwámin is gisíðos tuo,
farstálin ina an þem sténe. Simnen wesat gi an stríde mid þiu,
forð an flíte: ef it wirðit þem folktogen kúð,
wi gihelpat iu wið þena hérosten, þat hie iu harmes wiht,
léðes ni giléstid.ʼ Þuo námun sia an þem liudon filo
diurero méðmo, dádun all só sia bigunnun
— ne gi-weldun iro willjon — dádun só wído kúð
þem liudon after þem lande, þat sia sulika lugina woldun
ahębbjan be þan hélagan drohtin. Þan was eft gihélid hugi
jungron Kristes, þuo sia gi-hórdun þiu guodun wíf
márjan þia maht godes; þuo wárun sia an iro muode fráha,
gie im te þem graƀe béðia, Johannes endi Petrus
runnun oƀastlíko: warð ér kuman
Johannes þie guodo, endi im oƀar þem graƀe gistuod,
antat þar sán after kwam Símon Petrus,
erl ellanruof endi im þar in giwét
an þat graf gangan: gisah þar þes godes barnes,
hréogiwádi hérren sínes
línin liggian, mid þiu was ér þie lík-hamo
fagaro bifangan; lag þie fano sundar,
mit þem was þat hóƀid bihelid hélages Kristes,
ríkjes drohtines, þan hie an þesaro rastu was.
Þuo geng im ók Johannes an þat graf innan
sehan seldlík þing; warð im sán after þiu
antlokan is gilóƀo, þat hie wissa, þat skolda eft an þit lioht kuman
is drohtin diurlíko, fan dóðe astandan
up fan erðu. Þuo giwitun im eft þanan
Johannes endi Petrus, \hld\ endi kwámun þia jungron Kristes,
þia gisíðos tesamne. \hld\ Þan stuod sérag-muod
én þera idiso \hld\ óðer-síðu
griotandi oƀar þem graƀe, \hld\ was iro iámar muod—
Maria was þat Magdalena—, \hld\ was iro muod-giþáht,
seƀo mit sorogon giblandan, \hld\ ne wissa hwarod siu sókian skolda
þena hérron, þar iro wárun at þia helpa gilanga. \hld\ Siu ni mohta þuo hofnu awísan,
þat wíf ni mahta wóp for-látan: \hld\ ne wissa hwarod siu sia wendian skolda;
gimerrid wárun iro þes muod-giþáhti. \hld\ Þuo gisah siu þena mahtigan þar
Kriste standan, \hld\ þuoh siu ina kúðlíko
ant-kęnnian ni mohti, \hld\ ér þan hie ina kúðjan welda,
sęggjan þat hie it selƀo wári. \hld\ Hie frágoda hwat siu só séro biwiepi,
só harmo mid héton trahnin. Siu kwað, þat siu umbi iro hérron ni wissi
te wáren, hwarod hie werðan skoldi: ʽef þu ina mi giwísan mohtis,
fró mín, ef ik þik frágon gidorsti, ef þu ina hier an þeson felise ginámis,
wísi ina mi mid wordon þínon: þan wári mi allaro willjono mésta,
þat ik ina selƀo gisáhi.ʼ Sia ni wissa, þat sia þie suno drohtines
gruotta mid gódaro sprákun: siu wánda þat it þie gardari wári,
hofward hérren sínes. Þuo gruotta sia þie hélago drohtin,
bi namen nęrjendero best: siu geng im þuo náhor sniumo,
þat wíf mid willjon guodan, ant-kenda iro waldand selƀan,
míðan siu is þuru þia minnja ni wissa: welda ina mid iro mundon grípan,
þiu féhmia an þena folko drohtin, noƀan þat iro friðu-barn godes
werida mid wordon sínon, kwað þat siu ina mid wihti ni mósti
handon ant-hrínan: ʽik ni stég nohʼ, kwaþie, ʽte þem himiliskon fader;
ak íli þu nu ofstlíko endi þem erlon kúði,
bruoðron mínon, \hld\ þat ik úser béðero fader
ala-waldan, \hld\ iuwan endi mínan
suoðfastan god \hld\ suokjan willju.ʼ
Þat wíf warð þuo an wunnon, \hld\ þat siu muosta sulikan willjon kúðjan,
sęggjan fan im gi-sundon: \hld\ warð sán garo
þiu idis an þat árundi \hld\ endi þem erlon bráhta,
will-spel weron, \hld\ þat siu waldand Krist
gisundan gi-sáwi, endi sagda hwó he iru selƀo gibód
torohtero tékno. Sia ni weldun gitrúojan þuo noh
þes wíbes wordon, þat siu sulik will-spel bráhte
gegnungo fan þemo godes suno, ak sia sátun im iámor-muoda,
heliðos hriwonda. Þuo warð þie hélago Krist
eft opanlíko óðersíðu,
drohtin gitógid, síðor hie fan dóðe astuod,
þan wíƀon an willjon, þat hie im þar an wege muotta.
kwedda sia kúðlíko, endi sia te is kneohon hnigun,
fellun im tó fuoton. Hie hét þat sia forahtan hugi
ne bárin an iro brioston: ʽak gi mínon bruoðron skulun
þesa kwidi kúðjan, þat sia kuman after mi
an Galileo land; þar ik im eft tegegnes biun.ʼ
Þan fuorun im ók fan Hierusalem þero jungrono twéna
an þem selƀon daga sán an morgan,
erlos an iro árundi: weldun im te Emaus
þat kastel suokan. Þuo bigunnun im kwidi managa
under þem weron wahsan, þar sia after þem wege fuorun,
þem heliðon umbi iro hérron. Þuo kwam im þar þie hélago tuo
gangandi godes suno. Sia ni mahtun ina garolíko
ant-kęnnan kraftigna: hie ni welda ina þuo noh kúðjan te im;
was im þoh an iro gisíðje samad endi frágoda, umbi hwilika sia saka sprákin:
ʽhwí gangat gi só gornondia?ʼ \hld\ kwaþie; ʽIst ink jámer hugi,
seƀo soragono full.ʼ \hld\ Sia sprákun im sán an-gegin,
þia erlos and-wurdi: \hld\ ʽte hwí þu þes éskos sóʼ, kwáðun sia;
ʽbist þi fan Hierusalem \hld\ Judeono folkas
hélagumu géste \hld\ fan heƀen-wange,
mid þem grótun godes kraft.ʼ \hld\ Nam is jungaron þó,
erlos góde, \hld\ lédda sie út þanan,
antat he sie bráhte \hld\ an Beþania;
þar hóf he is hendi up \hld\ endi hélegoda sie alle,
wíhida sie mid is wordun. \hld\ Giwét imo up þanan,
sóhta imo þat hóha himilo ríki \hld\ endi þena is hélagon stól:
sitit imo þar an þea \hld\ swíðron half godes,
alo-mahtiges fader \hld\ endi þanan all gesihit
waldandeo Krist, \hld\ só hwat só þius werold behaƀet.
Þó an þeru selƀon stedi \hld\ ge-síðos góde
te bedu fellun \hld\ endi im eft te burg þanan
þar te Hierusalem \hld\ jungaron Kristes
fórun faganondi: \hld\ was im fráhmod hugi,
wárun im þar at þemu wíhe; \hld\ waldandes kraft
[...]
