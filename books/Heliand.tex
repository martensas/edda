\bookStart{Heliand}

Manuscripts in chronological order:
  Source: https://escholarship.org/content/qt19k1z5h8/qt19k1z5h8_noSplash_88c7cf5d8c1e26a95c76028adf012df9.pdf?t=mtfq0h p. 11.
  See p. 66 for parallel texts with C, M, P and L.


  L 840–850 (Thomas 4073 (Ms))
  P 840–850 (R 56/2537 (PA))
  V 800–850 (Palatini Latini 1447)
  S 850 (cgm. 8840)
  M 850–875 (cgm. 25)
  C 950–1000 (Cotton Caligula A. VII sign. 3-11)

Fragments L and P appear to originally belong to the same codex?


Notes on the normalization:
  \begin{itemize}
    \item Long vowels are marked by the acute rather than by the circumflex. This is both faithful to the original manuscripts and concordant with my practice in normalising other Germanic languages.
    \item Long vowels \emph{ò} and \emph{è} resulting from monophthongisation of \emph{au} and \emph{ai} are, however, written with the grave.
    \item When attested in all mss., epenthetic (svarabhakti) vowels are written with a dot beneath them. Otherwise they are deleted.
    \item ms. \emph{e} and \emph{i}, when occuring, between vowels are written as \emph{j}.
    \item ms. \emph{e} as resulting from \emph{i}-mutation is written as \emph{ę}.
    \item ms. \emph{b} or \emph{ƀ}, when representing the voiced bilabial fricative, is written as \emph{v}.
    \item ms. \emph{th} is written as \emph{þ}.
    \item ms. \emph{uu} is written as \emph{w}.
  \end{itemize}

\sectionline

Manega wáron, \hld\ þe sia iro mód ge-spón,
þat sia bi—gunnun word godes,
rękkjan þat gi—rúni, \hld\ þat þie ríkjo Krist
undar man-kunnja \hld\ máriða gi-frumida
mid wordun endi mid werkun. \hld\ Þat wolda þó wísara filo
liudo barno lovon, \hld\ léra Kristes,
hèlag word godas, \hld\ endi mid iro handon skrívan
berẹht-líko an buok, \hld\ hwó sia is gi-bod-skip skoldin
frummjan, firiho barn. \hld\ Þan wárun þoh sia fiori te þiu
under þera menigo, \hld\ þia habdon maht godes,
helpa fan himila, \hld\ hèlagna gést,
kraft fan Kriste; \hld\ sia wurðun gi-korana te þio,
þat sie þan Éwangelium \hld\ énan skoldun
an buok skrívan \hld\ endo só manag gi-bod godes,
hèlag himilisk word: \hld\ sia ne muosta hęliðo þan mér,
firiho barno frummjan, \hld\ neuan þat sia fiori te þio
þuru kraft godas \hld\ ge-korana wurðun,
Matheus endi Markus, \hld\ —só wárun þia man hétana—
Lukas endi Johannes; \hld\ sia wárun gode lieva,
wirðiga ti þem gi-wirkje. \hld\ Habda im waldand god,
þem hęliðon an iro hertan \hld\ hèlagna gést
fasto bi-folhan \hld\ endi ferahtan hugi,
só manag wíslík word \hld\ endi gi-wit mikil,
þat sea skoldin a-hębbjan \hld\ hèlagaro stemnun
god-spell þat guoda, \hld\ þat ni havit énigan gi-gadon hwergin,
þiu word an þesaro weroldi, \hld\ þat io waldand mér,
drohtin diurje \hld\ efþo dervi þing,
firin-werk fellie \hld\ efþo fíundo níð,
stríd wiðer-stande—, \hld\ hwand hie habda starkan hugi,
mildjan endi guodan, \hld\ þie þe méster was,
aðal-ord-frumo \hld\ alo-mahtig.
Þat skoldun sea fiori \hld\ þuo fingron skrívan,
settian endi singan \hld\ endi sęggjan forð,
þat sea fan Kristes \hld\ krafte þem mikilon
gi-sáhun endi gi-hórdun, \hld\ þes hie selvo gi-sprak,
gi-wísda endi gi-warahta, \hld\ wundạr-líkas filo,
só manag mid mannon \hld\ mahtig drohtin,
all so hie it fan þem an-ginne \hld\ þuru is énes kraht,
waldand gi-sprak, \hld\ þuo hie érist þesa werold gi-skuop
endi þuo all bi-fieng \hld\ mid énu wordo,
himil endi erða \hld\ endi al þat sea bi-hlidan égun
gi-warahtes endi gi-wahsanes: \hld\ þat warð þuo all mid wordon godas
fasto bi-fangan, \hld\ endi gi-frumid after þiu,
hwi-lik þan liud-skępi \hld\ landes skoldi
wídost gi-waldan, \hld\ efþo hwar þiu werold-aldar
endon skoldin. \hld\ Én was iro þuo noh þan
firiho barnun bi-foran, \hld\ endi þiu fívi wárun agangan:
skolda þuo þat sehsta \hld\ sáliglíko
kuman þuru kraft godes \hld\  endi Kristas gi-burd,
hélandero bestan, \hld\ hèlagas géstes,
an þesan middil-gard \hld\ managon te helpun,
firio barnon ti frumon \hld\ wið fíundo níð,
wið dernero dwalm. \hld\ Þan habda þuo drohtin god
Rómano-liudjon far-liwan \hld\ ríkjo mésta,
habda þem heri-skipje \hld\ herta gi-sterkid,
þat sia habdon bi-þwungana \hld\ þiedo gi-hwi-lika,
habdun fan Rúmu-burg \hld\ ríki gi-wunnan
helm-gi-trósteon, \hld\ sáton iro hęri-togon
an lando gi-hwem, \hld\ habdun liudjo gi-wald,
allon eli-þeodon. \hld\ Erodes was
an Hierusalem \hld\ over þat Judeono folk
gi-koran te kuninge, \hld\ só ina þie késer þarod,
fon Rúmu-burg \hld\ ríki þiodan
satta undar þat gi-síði. \hld\ Hie ni was þoh mid sibbeon bilang
avaron Israheles, \hld\ eðili-gi-burdi,
kuman fon iro knuosle, \hld\ neuan þat hie þuru þes késures þank
fan Rúmu-burg \hld\ ríki habda,
þat im wárun só gi-hóriga \hld\ hildi-skalkos,
avaron Israheles \hld\ ęlljan-ruova:
swíðo un-wanda wini, \hld\ þan lang hie gi-wald éhta,
Erodes þes ríkjas \hld\ endi rád-burdjon held
Judeo liudi. \hld\ Þan was þar én gi-gamalod mann,
þat was fruod gomo, \hld\ habda ferehtan hugi,
was fan þem liudjon \hld\ Lewias kunnes,
Jakobas suneas, \hld\ guodero þiedo:
Zakharias was hie hétan. \hld\ Þat was só sálig man,
hwand hie simblon gerno \hld\ gode þeonoda,
warahta after is willjon; \hld\ deda is wíf só self
—was iru gi-aldrod idis: \hld\ ni muosta im ervi-ward
an iro iuguð-hédi \hld\ giviðig werðan—
libdun im far-úter laster, \hld\ waruhtun lof goda,
wárun só gi-hóriga \hld\ hevan-kuninge,
diuridon úsan drohtin: \hld\ ni weldun derveas wiht
under man-kunnje, \hld\ ménes gi-frummjan,
ne *saka ne sundja; \hld\ was im þoh an sorgun hugi,
þat sie ervi-ward \hld\ égan ni móstun,
ak wárun im barno-lòs. \hld\ Þan skolda he gi-bod godes
þar an Hierusalem, \hld\ só oft só is gi-gengi gi-stód,
þat ina torht-líko \hld\ tídi gi-manodun,
só skolda he at þem wíha \hld\ waldandes geld
hèlag bi-hwervan, \hld\ hevan-kuninges,
godes jungar-skępi: \hld\ gern was he swíðo,
þat he it þurh ferhtan hugi \hld\ frummjan mósti.

FITT 2.

Þó warð þiu tíd kuman, \hld\ — þat þar gi-tald habdun
wísa man mid wordun, \hld\ — þat skolda þana wíh godes
Zakharias bi-sehan. \hld\ Þó warð þar gi-samnod filu
þar te Hierusalem \hld\ Judeo liudi,
werodes te þem wíha, \hld\ þar sie waldand god
swíðo þeolíko \hld\ þiggjan skoldun,
hérron is huldi, \hld\ þat sie hevan-kuning
léðes aléti. \hld\ Þea liudi stódun
umbi þat hèlaga hús, \hld\ endi geng im þe gi-hérodo man
an þana wíh innan. \hld\ Þat werod óðar béd
umbi þana alah útan, \hld\ Ebreo liudi,
hwan ér þe fródo man \hld\ gi-frumid habdi
waldandes willjon. \hld\ Só he þó þana wírók dróg,
ald aftar þem alaha, \hld\ endi umbi þana altari geng
mid is rókfatun \hld\ ríkiun þionon,
—fremida ferht-líko \hld\ fráon sínes,
godes jungar-skępi \hld\ gerno swíðo
mid hluttru hugi, \hld\ *só man hérren skal
gerno ful-gangan—, \hld\ grurios kwámun im,
egison an þem alahe: \hld\ hie gi-sah þar aftar þiu énna ęngil godes
an þem wíhe innan, \hld\ hie sprak im mid is wordun tuo,
hiet þat fruod gumo \hld\ foroht ni wári,
hiet þat hie im ni and-riede: \hld\ þína dádi sindʼ, kwaþie*,
ʽwaldanda werðe \hld\ endi þín word só self,
þín þionost is im an þanke, \hld\ þat þu su-lika gi-þáht haves
an is énes kraft. \hld\ Ik is ęngil bium,
Gabriel bium ik hétan, \hld\ þe gio for goda standu,
and-ward for þem alo-waldon, \hld\ ne sí þat he me an is árundi hwarod
sęndjan willja. \hld\ Nu hiet he me an þesan síð faran,
hiet þat ik þi þoh gi-kúðdi, \hld\ þat þi kind gi-boran,
fon þínera alderu idis \hld\ ódan skoldi
werðan an þesero weroldi, \hld\ wordun spáhi.
Þat ni skal an is liva gio \hld\ líðes an-bítan,
wínes an is weroldi: \hld\ só haved im wurd-gi-skapu,
metod gi-markod \hld\ endi maht godes.
Hét þat ik þi þoh sagdi, \hld\ þat it skoldi gi-síð wesan
hevan-kuninges, \hld\ hét þat git it heldin wel,
tuhin þurh trewa, \hld\ kwað þat he im tíras só filu
an godes ríkja \hld\ for-gevan weldi.
He kwað þat þe gódo gumo \hld\ Johannes te namon
hębbjan skoldi, \hld\ gi-bód þat git it hétin só,
þat kind, þan it kwámi, \hld\ kwað þat it Kristes gi-síð
an þesaro wídun werold \hld\ werðan skoldi,
is selves sunjes, \hld\ endi kwað þat sie sliumo herod
an is bod-skępi \hld\ béðe kwámin.ʼ
Zakharias þó gi-mahalda \hld\ endi wið selvan sprak
drohtines ęngil, \hld\ endi im þero dádjo bi-gan,
wundron þero wordo: \hld\ ʽhwó mag þat gi-werðan sóʼ, kwað he,
ʽaftar an aldre? \hld\ it is unk al te lat
só te gi-winnanne, \hld\ só þu mid þínun wordun gi-sprikis.
Hwanda wit habdun aldres \hld\ ér efno twén-tig
wintro an unkro weroldi, \hld\ ér þan kwámi þit wíf te mi;
þan wárun wit nu at-samna \hld\ ant-sivunta wintro
gi-bęnkjon endi gi-będdjon, \hld\ siðor ik sie mi te brúdi ge-kòs.
Só wit þes an unkro iuguði \hld\ gi-girnan ni mohtun,
þat wit ervi-ward \hld\ égan móstin,
fódjan an unkun flęttja, \hld\ nu wit sus gi-fródod sint
—havad unk eldi bi-noman \hld\ ęlljan-dádi,
þat wit sint an unkro siuni gi-slekit \hld\ endi an unkun sídun lat;
flésk is unk ant-fallan, \hld\ fel un-skóni,
is unka lud gi-liðen, \hld\ lík gi-drusnod,
sind unka and-bári \hld\ óðar-líkaron,
mód endi męgin-kraft—, \hld\ só wit giu só managan dag
wárun an þesero weroldi, \hld\ só mi þes wundạr þunkit,
hwó it só gi-werðan mugi, \hld\ só þu mid þínun wordun gi-sprikis.

FITT 3.

Þó warð þat heven-kuninges bodon \hld\ harm an is móde,
þat he is gi-werkes \hld\ só wundron skolda
endi þat ni welda gi-huggjan, \hld\ þat ina mahta hèlag god
só ala-jungan, \hld\ só he fon érist was,
selvo gi-wirkjan, \hld\ of he só weldi.
Skerida im þó te wítea, \hld\ þat he ni mahte énig word sprekan,
gi-mahlien mid is múðu, \hld\ ʽér þan þi magu wirðid,
fon þínero aldero idis \hld\ erl afódit,
kind-jung gi-boran \hld\ kunnjes gódes,
wánum te þesero weroldi. \hld\ Þan skalt þu eft word sprekan,
hębbjan þínaro stemna gi-wald; \hld\ ni þarft þu stum wesan
lengron hwíla.ʼ \hld\ Þó warð it sán gi-léstid só,
gi-worðan te wáron, \hld\ só þar an þem wíha gi-sprak
ęngil þes alo-waldon: \hld\ warð ald gumo
spráka bi-lósit, \hld\ þoh he spáhan hugi
bári an is breostun. \hld\ Bidun allan dag
þat werod for þem wíha \hld\ endi wundrodun alla,
bi-hwí he þar só lango, \hld\ lof-sálig man,
swíðo fród gumo \hld\ fráon sínun
þionon þorfti, \hld\ só þar ér énig þegno ni deda,
þan sie þar at þem wíha \hld\ waldandes geld
folmon frumidun. \hld\ Þó kwam fród gumo
út fon þem alaha. \hld\ Erlos þrungun
náhor mikilu: \hld\ was im niud mikil,
hwat he im sóð-líkes \hld\ sęggjan weldi,
wísjan te wáron. \hld\ He ni mohta þó énig word sprekan,
gi-sęggjan þem gi-siðea, \hld\ bútan þat he mid is swíðron hand
wísda þem weroda, \hld\ þat sie úses waldandes
léra léstin. \hld\ Þea liudi for-stódun,
þat he þar habda gegnungo \hld\ god-kundes hwat
for-sehen selvo, \hld\ þoh he is ni mahti gi-sęggjan wiht,
gi-wísjan te wáron. \hld\ Þó habda he úses waldandes
geld gi-léstid, \hld\ al só is gi-gengi was
gi-markod mid mannun. \hld\ Þó warð sán aftar þiu maht godes,
gi-kúðid is kraft mikil: \hld\ warð þiu kwán ókan,
idis an ira eldiu: \hld\ skolda im ervi-ward,
swíðo god-kund gumo \hld\ giviðig werðan,
barn an burgun. \hld\ Béd aftar þiu
þat wíf wurdi-gi-skapu. \hld\ Skréd þe wintạr forð,
geng þes géres gi-tal. \hld\ Johannes kwam
an liudjo lioht: \hld\ lík was im skóni,
was im fel fagar, \hld\ fahs endi naglos,
wangun wárun im wlitige. \hld\ Þó fórun þar wíse man,
snelle te-samne, \hld\ þea swásostun mést,
wundrodun þes werkes, \hld\ bi-hwí it gio mahti gi-werðan só,
þat undar só aldun twém \hld\ ódan wurði
barn an gi-burdjon, \hld\ ni wári þat it gi-bod godes
selves wári: \hld\ af-suovun sie garo,
þat it elkor só wán-lík \hld\ werðan ni mahti.
Þó sprak þar én gi-fródot man, \hld\ þe só filo konsta
wísaro wordo, \hld\ habde gi-wit mikil,
frágode niud-líko, \hld\ hwat is namo skoldi
wesan an þesaro weroldi: \hld\ ʽmi þunkid an is wísu gi-lík
iak an is gi-bárea, \hld\ þat he sí bętara þan wi,
só ik wániu, \hld\ þat ina ús gegnungo god fon himila
selvo sendiʼ. \hld\ Þó sprak sán aftar
þiu módar þes kindes, \hld\ þiu þana magu habda,
þat barn an ire barme: \hld\ ʽhér kwam gi-bod godesʼ, kwað siu,
ʽfernun gére, \hld\ furmon wordu
gi-bód, þat he Johannes \hld\ bi godes lérun
hétan skoldi. \hld\ Þat ik an mínumu hugi ni gidar
węndjan mid wihti, \hld\ of ik is gi-waldan mótʼ.
Þó sprak én gélhert man, \hld\ þe ira gaduling was:
ʽne hét ér giowiht sóʼ, \hld\ kwað he, ʽaðal-boranes
úses kunnjes efþo knósles; \hld\ wita kiasan im óðrana
niudsamna namon: \hld\ he niate of he mótiʼ.
Þó sprak eft þe fródo man, \hld\ þe þar konsta filo mahlian:
ʽni givu ik þat te rádeʼ, \hld\ kwað he, ʽrinko negénun,
þat he word godes \hld\ węndjan biginna;
ak wita is þana fader frágon, \hld\ þe þar só gi-fródod sitit,
wís an is wín-sęli: \hld\ þoh he ni mugi énig word sprekan,
þoh mag he bi bók-stavon \hld\ bréf ge-wirkjan,
namon gi-skrívanʼ. \hld\ Þó he náhor geng,
legda im éna bók an barm \hld\ endi bad gerno
wrítan wíslíko \hld\ wordgimerkiun,
hwat sie þat hèlaga barn \hld\ hétan skoldin.
Þó nam he þia bók an hand \hld\ endi an is hugi þáhte
swíðo gerno te gode: \hld\ Johannes namon
wís-líko gi-wrét \hld\ endi ók aftar mid is wordu gi-sprak
swíðo spáh-líko: \hld\ habda im eft is spráka gi-wald,
gi-witteas endi wísun. \hld\ Þat wíti was þó agangan,
hard harm-skare, \hld\ þe im hèlag god
mahtig makode, \hld\ þat he an is mód-sevon
godes ni for-gáti, \hld\ þan he im eft sendi is jungron tó.

FITT 4.

Þó ni was lang aftar þiu, \hld\ ne it al só gi-léstid warð,
só he man-kunnja \hld\ managa hwíla,
god alo-mahtig \hld\ for-geven habda,
þat he is himilisk barn \hld\ herod te weroldi,
si selves sunu \hld\ sęndjan weldi,
te þiu þat he hér a-lósdi \hld\ al liud-stamna,
werod fon wítea. \hld\ Þó warð is wisbodo
an Galilea-land, \hld\ Gabriel kuman,
ęngil þes alo-waldon, \hld\ þar he éne idis wisse,
muni-líka magað: \hld\ Maria was siu héten,
was iru þiorna gi-þigan. \hld\ Sea én þegan habda,
Joseph gi-mahlit, \hld\ gódes kunnjes man,
þea Dawides dohter: \hld\ þat was só diur-lík wíf,
idis ant-héti. \hld\ Þar sie þe ęngil godes
an Nazareth-burg \hld\ bi namon selvo
grótte geginwarde \hld\ endi sie fon gode qhwedda:
ʽHél wis þu, Mariaʼ, \hld\ kwað he, ʽþu bist þínun hérron liof,
waldande wirðig, \hld\ hwand þu gi-wit haves,
idis enstio fol. \hld\ Þu skalt for allun wesan
wívun gi-wíhit. \hld\ Ne have þu wékan hugi,
ne forhti þu þínun ferhe: \hld\ ne kwam ik þi te énigun fréson herod,
ne dragu ik énig drugi-þing. \hld\ Þu skalt úses drohtines wesan
módar mid mannun \hld\ endi skalt þana magu fódjan,
þes hòhon hevan-kuninges suno. \hld\ Þe skal héljand te namon
égan mid eldiun. \hld\ Neo endi ni kumid,
þes wídon ríkjas gi-wand, \hld\ þe he gi-waldan skal,
mári þeodan.ʼ \hld\ Þó sprak im eft þiu magað an-gegin,
wið þana ęngil godes \hld\ idiso skónjost,
allaro wívo wlitigost: \hld\ ʽhwó mag þat gi-werðen sóʼ, kwað siu,
ʽþat ik magu fódie? \hld\ Ne ik gio mannes ni warð
wís an mínera weroldi.ʼ \hld\ Þó habde eft is word garu
ęngil þes alo-waldon \hld\ þero idisiu te-gegnes:
ʽan þi skal hèlag gést \hld\ fon hevan-wange
kuman þurh kraft godes. \hld\ Þanan skal þi kind ódan
werðan an þesaro weroldi; \hld\ waldandes kraft
skal þi fon þem hòhoston \hld\ hevan-kuninge
skadowan mid skimon. \hld\ Ni warð skónjera gi-burd,
ne só mári mid mannun, \hld\ hwand siu kumid þurh maht godes
an þese wídon werold.ʼ \hld\ Þó warð eft þes wíbes hugi
aftar þem árundie \hld\ al gi-hworven
an godes willjon. \hld\ ʽÞan ik hér garu standuʼ, kwað siu,
ʽte su-likun ambaht-skępi, \hld\ só he mi égan wili.
Þiu bium ik þeot-godes. \hld\ Nu ik þeses þinges gi-trúon;
werðe mi aftar þínun wordun, \hld\ al só is willjo sí,
hérron mínes; \hld\ nis mi hugi twífli,
ne word ne wísa.ʼ \hld\ Só gi-fragn ik, þat þat wíf ant-feng
þat godes árundi \hld\ gerno swíðo
mid leohtu hugi \hld\ endi mid gi-lóvon gódun
endi mid hluttrun trewun; \hld\ warð þe hèlago gést,
þat barn an ira bósma; \hld\ endi siu ira breostun for-stód
iak an ire sevon selvo, \hld\ sagda þem siu welda,
þat sie habde giókana \hld\ þes alo-waldon kraft
hèlag fon himile. \hld\ Þó warð hugi Josepes,
is mód gi-worrid, \hld\ þe im ér þea magað habda,
þea idis ant-héttea, \hld\ aðal-knósles wíf
gi-boht im te brúdiu. \hld\ He afsóf þat siu habda barn undar iru:
ni wánda þes mid wihti, \hld\ þat iru þat wíf habdi
gi-wardod só waro-líko: \hld\ ni wisse waldandes þó noh
blíði gi-bod-skępi. \hld\ Ni welda sia imo te brúdi þó,
halon imo te híwon, \hld\ ak bi-gan im þó an hugi þęnkjan,
hwó he sie só for-léti, \hld\ só iru þar nu wurði lédes wiht,
ódan arvides. \hld\ Ni welda sie aftar þiu
meldon for menigi: \hld\ antdréd þat sie manno barn
lívu binámin. \hld\ Só was þan þero liudjo þau
þurh þen aldon éu, \hld\ Ebreo folkes,
só hwi-lik só þar an un-reht \hld\ idis gi-híwida,
þat siu simbla þana bed-skępi \hld\ buggean skolda,
frí mid ira ferhu: \hld\ ni was gio þiu fémea só gód,
þat siu mid þem liudun leng \hld\ libbien mósti,
wesan undar þem weroda. \hld\ Bigan im þe wíso mann,
swíðo gód gumo, \hld\ Joseph an is móda
þęnkjan þero þingo, \hld\ hwó he þea þiornun þó
listjun for-léti. \hld\ Þó ni was lang te þiu,
þat im þar an dròma \hld\ kwam drohtines ęngil,
hevan-kuninges bodo, \hld\ endi hét sie ina haldan wel,
minnjon sie an is móde: \hld\ ʽNi wis þuʼ, kwað he, ʽMariun wréð,
þiornun þínaro; \hld\ siu is gi-þungan wíf;
ne forhugi þu sie te hardo; \hld\ þu skalt sie haldan wel,
wardon ira an þesaro weroldi. \hld\ Lésti þu inka wini-trewa
forð só þu dádi, \hld\ endi hald inkan friund-skępi wel!
Ne lát þu sie þi þiu léðaron, \hld\ þoh siu undar ira liðon égi,
barn an ira bósma. \hld\ It kumid þurh gi-bod godes,
hèlages géstes \hld\ fon hevan-wanga:
þat is Iésu Krist, \hld\ godes égan barn,
waldandes sunu. \hld\ Þu skalt sie wel haldan,
hèlag-líko. \hld\ Ne lát þu þi þínan hugi twíflien,
merrean þína mód-gi-þáht.ʼ \hld\ Þó warð eft þes mannes hugi
gi-wendid aftar þem wordun, \hld\ þat he im te þem wíva genam,
te þera magað minnja: \hld\ ant-kenda maht godes,
waldandes gi-bod; \hld\ was im willjo mikil,
þat he sia só hèlag-líko \hld\ haldan mósti:
bi-sorgoda sie an is gi-síðea, \hld\ endi siu só súvro dróg
al te huldi godes \hld\ hèlagna gést,
gód-líkan gumon, \hld\ antþat sie godes gi-skapu
mahtig gi-manodun, \hld\ þat siu ina an manno lioht,
allaro barno bętst, \hld\ brengean skolda.

FITT 5.

Þó warð fon Rúmu-burg \hld\ ríkes mannes
ovar alla þesa irmin-þiod \hld\ Oktawiánas
ban endi bod-skępi \hld\ ovar þea is brédon gi-wald
kuman fon þem késure \hld\ kuningo gi-hwi-likun,
hém-sittjandiun, \hld\ só wído só is hęri-togon
ovar al þat land-skępi \hld\ liudjo gi-weldun.
Hiet man þat alla þea eli-lendiun man \hld\ iro óðil sóhtin,
hęliðos iro hand-mahal \hld\ an-gegen iro hérron bodon,
kwámi te þem knósla gi-hwe, \hld\ þanan he kunnjas was,
gi-boran fon þem burgiun. \hld\ Þat gi-bod warð gi-léstid
ovar þesa wídon werold; \hld\ werod samnoda
te allaro burgeo gi-hwem. \hld\ Fórun þea bodon ovar all,
þea fon þem késura \hld\ kumana wá*run,
bók-spáha weros, \hld\ endi an bréf skrivun
swíðo niud-líko \hld\ namono gi-hwi-likan,
ia land ia liudi, \hld\ þat im ni mahti a-lettjan mann
gumono su-lika gambra, \hld\ só im skolda geldan gi-hwe
hęliðo fon is hóvda. \hld\ Þó gi-wét im ók mid is híwiska
Joseph þe gódo, \hld\ só it god mahtig,
waldand welda: \hld\ sóhta im þiu wánamon hém,
þea burg an Bethleem, \hld\ þar iro beiðero was,
þes hęliðes hand-mahal* \hld\ endi ók þera hèlagun þiornun,
Mariun þera gódun. \hld\ Þar was þes márjon stól
an ér-dagun, \hld\ aðalkuninges,
Dawides þes gódon, \hld\ þan langa þe he þana druht-skępi þar,
erl undar Ebreon \hld\ égan mósta,
haldan hòh-gi-setu. \hld\ Sie wárun is híwiskas,
kuman fon is knósla, \hld\ kunnjas gódes,
béðju bi gi-burdiun. \hld\ Þar gi-fragn ik, þat sie þiu berhtun gi-skapu,
Mariun gi-manodun \hld\ *endi maht godes,
þat iru an þem síða \hld\ sunu ódan warð,
gi-boran an Bethleem \hld\ barno strangost,
allaro kuningo kraftigost: \hld\ kuman warð þe márjo,
mahtig an manno lioht, \hld\ só is ér managan dag
biliði wárun \hld\ endi bókno filu
gi-worðen an þesero weroldi. \hld\ Þó was it all gi-wárod só,
só it ér spáha man \hld\ gi-sprokan habdun,
þurh hwi-lik ód-módi \hld\ he þit erð-ríki herod
þurh is selves kraft \hld\ sókjan welda,
managaro mund-boro. \hld\ Þó ina þiu módar nam,
bi-wand ina mid wádju \hld\ wívo skónjost,
fagaron fratahun, \hld\ endi ina mid iro folmon twém
legda liov-líko \hld\ luttilna man,
þat kind an éna kribbiun, \hld\ þoh he habdi kraft godes,
manno drohtin. \hld\ Þar sat þiu módar bi-foran,
wíf wakogeandi, \hld\ war*doda selvo,
held þat hèlaga barn: \hld\ ni was ira hugi twífli,
þera magað ira mód-sevo. \hld\ Þó warð þat managun kúð
ovar þesa wídon werold, \hld\ wardos ant-fundun,
þea þar ehu-skalkos \hld\ úta wárun,
weros an wahtu, \hld\ wiggeo gómjan,
fehas aftar fel*da: \hld\ gi-sáhun finistri an twé
telátan an lufte, \hld\ endi kwam lioht godes
wánum þurh þiu wolkan \hld\ endi þea wardos þar
bi-feng an þem felda. \hld\ Sie wurðun an forhtun þó,
þea man an ira móda: \hld\ gi-sáhun þar mahtigna
godes ęngil kuman, \hld\ þe im te-gegnes sprak,
hét þat im þea wardos \hld\ wiht ne antdrédin
léðes fon þem liohta: \hld\ ʽik skal euʼ, kwað he, ʽliovara þing,
swíðo wár-líko \hld\ willjon sęggjan,
kúðjan kraft mikil: \hld\ nu is Krist ge-boran
an þeser*o selvun naht, \hld\ sálig barn godes,
an þera Dawides burg, \hld\ drohtin þe gódo.
Þat is mendislo \hld\ manno kunnjas,
allaro firiho fruma. \hld\ Þar gi ina fíðan mugun,
an Bethlemaburg \hld\ barno ríkjost:
hębbjad þat te tékna, \hld\ þat ik eu gi-tęlljan mag
wárun wordun, \hld\ þat he þar bi-wundan ligid,
þat kind an énera kribbiun, \hld\ þoh he sí kuning ovar al
erðun endi himiles \hld\ endi ovar eldeo barn,
weroldes waldandʼ. \hld\ Reht só he þó þat word gi-sprak,
só warð þar ęngilo te þem énun \hld\ unrím kuman,
hèlag hęri-skępi \hld\ fon hevan-wanga,
fagar folk godes, \hld\ endi filu sprákun,
lof-word manag \hld\ liudjo hérron.
Af-hóvun þó hèlagna sang, \hld\ þó sie eft te hevan-wanga
wundun þurh þiu wolkan. \hld\ Þea wardos hórdun,
hwó þiu ęngilo kraft \hld\ alo-mahtigna god
swíðo werð-líko \hld\ wordun lovodun:
ʽdiuriða sí nuʼ, \hld\ kwáðun sie, ʽdrohtine selvun
an þem hòhoston \hld\ himilo ríkja
ęndi friðu an erðu \hld\ firiho barnun,
gód-willigun gumun, \hld\ þem þe god ant-kęnnjad
þurh hluttran hugi.ʼ \hld\ Þea hirdjo for-stódun,
þat sie mahtig þing \hld\ gi-manod habda,
blíð-lík bod-skępi: \hld\ gi-witun im te Bethleem þanan
nahtes síðon; \hld\ was im niud mikil,
þat sie selvon Krist \hld\ gi-sehan móstin.

FITT 6.

Habda im þe ęngil godes \hld\ al gi-wísid
torhtun téknun, \hld\ þat sie im tó selvun,
te þem godes barne \hld\ gangan mahtun,
endi fundun sán \hld\ folko drohtin,
liudjo hérron. \hld\ Sagdun þó lof goda,
waldande mid iro wordun \hld\ endi wído kúðdun
ovar þea berhtun burg, \hld\ hwi-lik im þar biliði warð
fon hevan-wanga \hld\ hèlag gi-tógit,
fagar an felde. \hld\ Þat frí al bi-held
an ira hugi-skęftjun, \hld\ hèlag þiorna,
þiu magað an ira móde, \hld\ só hwat só siu gi-hórda þea mann sprekan.
Fódda ina þó fagaro \hld\ frího skánjosta,
þiu módar þurh minnja \hld\ managaro drohtin,
hèlag himilisk barn. \hld\ hęliðos gi-sprákun
an þem ahtodon daga \hld\ erlos managa,
swíðo glawa gumon \hld\ mid þera godes þiornun,
þat he héljand te namon \hld\ hębbjan skoldi,
só it þe godes ęngil \hld\ Gabriel gi-sprak
wáron wordun \hld\ endi þem wíve gi-bód,
bodo drohtines, \hld\ þó siu érist þat barn ant-feng
wánum te þesero weroldi; \hld\ was iru willjo mikil,
þat siu ina só hèlag-líko \hld\ haldan mósti,
ful-geng im þó só gerno. \hld\ Þat gér furðor skréd
untþat þat friðu-barn godes \hld\ fiartig habda
dago endi nahto. \hld\ Þó skoldun sie þar éna dád frummjan,
þat sie ina te Hierusalem \hld\ for-gevan skoldun
waldanda te þem wíha. \hld\ Só was iro wísa þan,
þero liudjo land-sidu, \hld\ þat þat ni mósta for-látan negén
idis undar Ebreon, \hld\ ef iru at érist warð
sunu afódit, \hld\ ne siu ina simbla þarod
te þem godes wíha \hld\ for-gevan skolda.
Gi-witun im þó þiu gódun twé, \hld\ Joseph endi Maria
béðju fon Bethleem: \hld\ habdun þat barn mid im,
hèlagna Krist, \hld\ sóhtun im hús godes
an Hierusalem; \hld\ þar skoldun sie is geld frummjan
waldanda at þem wíha \hld\ wísa léstjan
Judeo folkes. \hld\ Þar fundun sea énna gódan man
aldan at þem alaha, \hld\ aðal-boranan,
þe habda at þem wíha só filu \hld\ wintro endi sumaro
gilibd an þem liohta: \hld\ oft warhta he þar lof goda
mid hluttru hugi; \hld\ habda im hèlagna gést,
sáliglíkan sevon; \hld\ Simeon was he hétan.
Im habda gi-wísid \hld\ waldandas kraft
langa hwíla, \hld\ þat he ni mósta ér þit lioht agevan,
węndjan af þesero weroldi, \hld\ ér þan im þe willjo gi-stódi,
þat he selvan Krist \hld\ gi-sehan mósti,
hèlagna hevan-kuning. \hld\ Þó warð im is hugi swíðo
blíði an is briostun, \hld\ þó he gi-sah þat barn kuman
an þena wíh innan. \hld\ Þuo sagda hie waldande þank,
al-mahtigon gode, \hld\ þes he ina mid is ógun gi-sah.
Geng im þó te-gegnes \hld\ endi ina gerno ant-feng
ald mid is armun: \hld\ al ant-kende
bókan endi biliði \hld\ endi ók þat barn godes,
hèlagna hevan-kuning. \hld\ ʽNu ik þi, hérro, skalʼ, kwað he,
ʽgerno biddjan, \hld\ nu ik sus gigamalod bium,
þat þu þínan holdan skalk \hld\ nu hinan hwervan látas,
an þína friðu-wára faran, \hld\ þar ér mína forðrun dedun,
weros fon þesero weroldi, \hld\ nu mi þe willjo gi-stód,
dago liovosto, \hld\ þat ik mínan drohtin gi-sah,
holdan hérron, \hld\ só mi gi-hétan was
langa hwíla. \hld\ Þu bist lioht mikil
allun eli-þiodun, \hld\ þea ér þes alo-waldon
kraft ne ant-kendun. \hld\ Þína kumi sindun
te dóma endi te diurðon, \hld\ drohtin fró mín,
avarun Israhelas, \hld\ éganumu folke,
þínun liovun *liudjun.ʼ \hld\ Listiun talde þó
þe aldo man an þem alaha \hld\ idis þero gódun,
sagda sóð-líko, \hld\ hwó iro sunu skolda
ovar þesan middil-gard \hld\ managun werðan
sumun te falle, sumun te fróvru \hld\ firiho barnun,
þem liudjun te leova, \hld\ þe is lérun gi-hórdin,
endi þem te harma, \hld\ þe hórjen ni weldin
Kristas léron. \hld\ ʽÞu skalt nohʼ, kwað he, ʽkara þiggjan,
harm an þínumu herton, \hld\ þan ina hęliðo barn
wápnun wítnod. \hld\ Þat wirðid þi werk mikil,
þrim te gi-þolonna.ʼ \hld\ Þiu þiorna al for-stód
wísas mannas word. \hld\ Þó kwam þar ók én wíf gangan
ald innan þem alaha: \hld\ Anna was siu hétan,
dohtar Fanueles; \hld\ siu habde ira drohtine wel
gi-þionod te þanka, \hld\ was iru gi-þungan wíf.
Siu mósta aftar ira magað-hédi, \hld\ síðor siu mannes warð,
erles an éhti \hld\ eðili þiorne,
só mósta siu mid ira brúdi-gumon \hld\ bódlo gi-waldan
sivun wintạr saman. \hld\ Þó gi-fragn ik þat iru þar sorga gi-stód
þat sie þiu mikila maht \hld\ metodes tedélda,
wréð wurdi-gi-skapu. \hld\ Þó was siu widowa aftar þiu
at þem friðu-wíha \hld\ fior endi antahtoda
wintro an iro weroldi, \hld\ só siu nia þana wíh ni for-lét,
ak siu þar ira drohtine wel \hld\ dages endi nahtes,
gode þionode. \hld\ Siu kwam þar ók gangan tó
an þea selvun tíd: \hld\ sán ant-kende
þat hèlage barn godes \hld\ endi þem hęliðon kúðde,
þem weroda aftar þem wíha \hld\ wil-spel mikil,
kwað þat im nęrjandas ginist \hld\ gi-náhid wári,
helpa heven-kuninges: \hld\ ʽnu is þe hèlago Krist,
waldand selvo \hld\ an þesan wíh kuman
te a-lósjenne þea liudi, \hld\ þe hér nu lango bidun
an þesara middil-gard, \hld\ managa hwíla,
þurftig þioda, \hld\ só nu þes þinges mugun
mendjan man-kunni.ʼ \hld\ Manag fagonoda
werod aftar þem wíha: \hld\ gi-hórdun wilspel mikil
fon gode sęggjan. \hld\ Þat geld habde þó gi-léstid
þiu idis an þem alaha, \hld\ al só it im an ira éwa gi-bód
endi an þera berhtun burg \hld\ bók gi-wísdun,
hèlagaro hand-gi-werk. \hld\ Gi-witun im þó te hús þanan
fon Hierusalem \hld\ Joseph endi Maria,
hèlag híwiski: \hld\ habdun im heven-kuning
simbla te gi-síða, \hld\ sunu drohtines,
managaro mund-boron, \hld\ só it gio mári ni warð
þan wídor an þesaro weroldi, \hld\ bútan só is willjo geng,
heven-kuninges hugi. \hld\ Þoh þar þan gi-hwi-lik hèlag man
Krist ant-kendi, \hld\ þoh ni warð it gio te þes kuninges hove
þem mannun gi-márid, \hld\ þea im an iro mód-sevon
holde ni wárun, \hld\ ak was im só bi-halden forð
mid wordun endi mid werkun, \hld\ antþat þar weros óstan,
swíðo glawa gumon \hld\ gangan kwámun
þrea te þero þiodu, \hld\ þegnos snelle,
an langan weg \hld\ ovar þat land þarod:
folgodun énun berhtun bókne \hld\ endi sóhtun þat barn godes
mid hluttru hugi: \hld\ weldun im hnígan tó,
gehan im te jungrun: \hld\ drivun im godes gi-skapu.
Þó sie Erodesan þar \hld\ ríkjan fundun
an is sęli sittjen, \hld\ slíð-wurdean kuning,
módagna mid is mannun: \hld\ —simbla was he morðes gern—
þó kwaddun sie ina kúsko \hld\ an kuning-wísun,
fagaro an is flęttje, \hld\ endi he frágoda sán,
hwi-lik sie árundi \hld\ úta gi-bráhti,
weros an þana wrak-síð: \hld\ ʽhweðer lédjad gi wundan gold
te gevu hwi-likun gumuno? \hld\ te hwí gi þus an ganga kumad,
gi-faran an fóðiu? \hld\ Hwat, gi néþwanan ferran sind
erlos fon óðrun þiodun. \hld\ Ik gi-sihu þat gi sind eðili-gi-burdiun
kunnjes fon knósle gódun: \hld\ nio hér ér su-lika kumana ni wurðun
éri fon óðrun þiodun, \hld\ síðor ik mósta þesas erlo folkes,
gi-waldan þesas wídon ríkjas. \hld\ Gi skulun mi te wárun sęggjan
for þesun liudjo folke, \hld\ bi-hwí gi sín te þesun lande kumanaʼ.
Þó sprákun im eft te-gegnes \hld\ gumon óstr-onja,
word-spáhe weros: \hld\ ʽwi þi te wárun mugunʼ, kwáðun sie,
ʽúse árundi \hld\ óðo gi-tęlljen,
gi-sęggjan sóð-líko, \hld\ bi-hwí wi kwámun an þesan sið herod
fon óstan te þesaro erðu. \hld\ Giu wárun þar aðalies man,
gódsprákja gumon, \hld\ þea ús gódes só filu,
helpa gi-hétun \hld\ fon heven-kuninge
wárum wordun. \hld\ Þan was þar én gi-wittig man,
fród endi fil-wís \hld\ —forn was þat giu—,
úse aldiro óstar hinan, \hld\ —þar ni warð síðor énig man
sprákono só spáhi—; \hld\ he mahte rekkien spel godes,
hwand im habde for-liwan \hld\ liudjo hérro,
þat he mahte fon erðu \hld\ up gi-hórjan
waldandes word: \hld\ bi-þiu was is gi-wit mikil,
þes þegnes gi-þáhti. \hld\ Þó he þanan skolda,
ageven gardos, \hld\ gadulingo gi-mang,
for-láten liudjo dròm, \hld\ sókjen lioht óðar,
þó he is jungron \hld\ hét gangan náhor,
ervi-wardos, \hld\ endi is erlun þó
sagde sóð-líko: \hld\ —þat al síðor kwam,
gi-warð* an þesaro weroldi—: \hld\ þó sagda he þat hér skoldi kuman én wís-kuning
mári endi mahtig \hld\ an þesan middil-gard
þes bętston gi-burdies; \hld\ kwað þat it skoldi wesan barn godes,
kwað þat he þesero weroldes \hld\ waldan skoldi
gio te éwan-daga, \hld\ erðun endi himiles.
He kwað þat an þem selvon daga, \hld\ þe ina sáligna
an þesan middil-gard \hld\ módar gi-drógi,
só kwað he þat óstana \hld\ én skoldi skínan
himil-tungal hwít, \hld\ su-lik só wi hér ne habdin ér
undar-twisk erða endi himil \hld\ óðar hwerigin,
ne su-lik barn ne su-lik bókan. \hld\ Hét þat þar te bedu fórin
þrea man fon þero þiodu, \hld\ hét sie þęnkjan wel,
hwan ér sie gi-sáwin óstana \hld\ up síðogean,
þat godes bókan gangan, \hld\ hét sie garwian sán,
hét þat wi im folgodin, \hld\ só it furi wurði,
westar ovar þesa weroldi. \hld\ Nu is it al gi-wárod só,
kuman þurh kraft godes: \hld\ þe kuning is gi-fódit,
gi-boran bald endi strang: \hld\ wi gi-sáhun is bókan skínan
hédro fon himiles tunglun, \hld\ só ik wét, þat it hèlag drohtin,
markoda mahtig selvo; \hld\ wi gi-sáhun morgno gi-hwi-likes
blíkan þana berhton sterron, \hld\ endi wi gengun aftar þem bókna herod
wegas endi waldas hwílon. \hld\ Þat wári ús allaro willjono mésta,
þat wi ina selvon gi-sehan móstin, \hld\ wissin, hwar wi ina sókjan skoldin,
þana kuning an þesumu késur-dóma. \hld\ Saga ús, undar hwi-likumu he sí þesaro kunneo afódit.ʼ
Þó warð Erodesa \hld\ innan briostun
harm wið herta, \hld\ bi-gan im is hugi wallan,
sevo mid sorgun: \hld\ gi-hórde sęggjan þó,
þat he þar ovar-hóvdon \hld\ égan skoldi,
kraftagoron kuning \hld\ kunnjes gódes,
sáligoron undar þem gi-síðja. \hld\ Þó he samnon hét,
só hwat só an Hierusalem \hld\ gódaro manno
allaro spáhoston \hld\ sprákono wárun
endi an iro brioston \hld\ bók-kraftes mést
wissun te wárun, \hld\ endi he sie mid wordun fragn,
swíðo niud-líko \hld\ níð-hugdig man,
kuning þero liudjo, \hld\ hwar Krist gi-boran
an werold-ríkja \hld\ werðan skoldi,
friðu-gumono bętst. \hld\ Þó sprak im eft þat folk an-gegin,
þat werod wár-líko, \hld\ kwáðun þat sie wissin garo,
þat he skoldi an Bethleem gi-boran werðan: \hld\ ʽsó is an úsun bókun gi-skrivan,
wís-líko gi-writan, \hld\ só it wár-sagon,
swíðo glawa gumon \hld\ bi godes krafta
fil-wíse man \hld\ furn gi-sprákun,
þat skoldi fon Bethleem \hld\ burgo hirdi,
liof landes ward \hld\ an þit lioht kuman,
ríki rád-gevo, \hld\ þe rihtjen skal
Judeono gum-skępi \hld\ endi is geva wesan
mildi ovar middil-gard \hld\ managun þiodun.ʼ
Þó gi-fragn ik þat sán aftar þiu \hld\ slíð-mód kuning
þero wár-sagono word \hld\ þem wrekkiun sagda,
þea þar an eli-lendi \hld\ erlos wárun
ferran gi-farana, \hld\ endi he frágoda aftar þiu,
hwan sie an óstar-wegun \hld\ érist gi-sáhin
þana kuning-sterron kuman, \hld\ kumbal liuhtjen
hédro fon himile. \hld\ Sie ni weldun is im þó helen eowiht,
ak sagdun it im sóð-líko. \hld\ Þó hét he sie an þana síð faran,
hét þat sie ira árundi al \hld\ undar-fundin
umbi þes kindes kumi, \hld\ endi þe kuning selvo gi-bód
swíðo hard-liko, \hld\ hérro Judeono,
þem wísun mannun, \hld\ ér þan sie fórin westan forð,
þat sie im eft gi-kúðdin, \hld\ hwar he þana kuning skoldi
sókjan at is selðon; \hld\ kwað þat he þar weldi mid is gi-síðun tó,
bedan te þem barne. \hld\ Þan hogda he im te banon werðan
wápnes ęggjun. \hld\ Þan eft waldand god
þáhte wið þem þinga: \hld\ he mahta aþengean mér,
gi-léstjan an þesum liohte: \hld\ þat is noh lango skín,
gi-kúðid kraft godes. \hld\ Þó gengun eft þiu kumbl forð
wánum undar wolknun. \hld\ Þó wárun þea wíson man
fúsa te faranne: \hld\ gi-witun im forð þanan
balda an bod-skępi: \hld\ weldun þat barn godes
selvon sókjan. \hld\ Sie ni habdun þanan gi-síðjas mér,
bútan þat sie þrie wárun: \hld\ wissun im þingo gi-skéð,
wárun im glawe gumon, \hld\ þe þea geva léddun.
Þan sáhun sie só wís-líko \hld\ undar þana wolknes skion,
up te þem hòhon himile, \hld\ hwó fórun þea hwíton sterron
—ant-kendun sie þat kumbal godes—, \hld\ þiu wárun þurh Krista herod
gi-warht te þesero weroldi. \hld\ Þea weros aftar gengun,
folgodun feraht-líko \hld\ —sie frumide þe mahte—
antþat sie gi-sáhun, \hld\ síð-wórige man,
berht bókan godes, \hld\ blék an himile
stillo gi-standen. \hld\ Þe sterro liohto skén
hwít ovar þem húse, \hld\ þar þat hèlage barn
wonode an willjon \hld\ endi ina þat wíf bi-held,
þiu þiorne gi-þiudo. \hld\ Þó warð þero þegno hugi
blíði an iro briostun: \hld\ bi þem bókna for-stódun,
þat sie þat friðu-barn godes \hld\ funden habdun,
hèlagna heven-kuning. \hld\ Þó sie an þat hús innan
mid iro gevun gengun, \hld\ gumon óstr-onja,
síð-wórige man: \hld\ sán ant-kendun
þea weros waldand Krist. \hld\ Þea wrekkion fellun
te þem kinde an kneobeda \hld\ endi ina an kuning-wísa
gódan gróttun \hld\ endi im þea geva drógun,
gold endi wíh-rók \hld\ bi godes téknun
*endi myrra þar mid. \hld\ Þea man stódun garowa,
holde for iro hérron, \hld\ þea it mid iro handun sán
fagaro ant-fengun. \hld\ Þó gi-witun im þea ferahton man,
seggi te selðon \hld\ síð-wórige,
gumon an gast-sęli. \hld\ Þar im godes ęngil
slápandjun an naht \hld\ swevan gi-tógde,
gi-drog im an dròme, \hld\ al so it drohtin self,
waldand welde, \hld\ þat im þúhte þat man im mid wordun gi-budi,
þat sie im* þanan óðran weg, \hld\ erlos fórin,
liðodin sie te lande \hld\ endi þana léðan man,
Erodesan \hld\ eft ni sóhtin,
módagna kuning. \hld\ Þó warð morgan kuman
wánum te þesero weroldi. \hld\ Þó bi-gunnun þea wíson man
sęggjan iro swevanos; \hld\ selvon ant-kendun
waldandes word, \hld\ hwand sie gi-wit mikil
bárun an iro briostun: \hld\ bádun alo-waldon,
héron heven-kuning, \hld\ þat sie móstin is huldi forð,
gi-wirkjan is willjon, \hld\ kwáðun þat sea ti im habdin gi-wendit hugi,
*iro mód morgan gi-hwem. \hld\ Þó fórun eft þie man þanan,
erlos óstr-onje, \hld\ al só im þe ęngil godes
wordun gi-wísde: \hld\ námun im weg óðran,
ful-gengun godes lérun: \hld\ ni weldun þemu Judeo kuninge
umbi þes barnes gi-burd \hld\ bodon óstr-onje,
síð-wórige man \hld\ sęggjan giowiht,
ak wendun im eft an iro willjon. \hld\ Þó warð sán aftar þiu waldandes,
godes ęngil kumen \hld\ Josepe te sprákun,
sagde im an swefne \hld\ slápandium an naht,
bodo drohtines, \hld\ þat þat barn godes
slíð-mód kuning \hld\ sókjan welda,
áhtean is aldres; \hld\ ʽnu skaltu ine an Aegypteo
land ant-lédean \hld\ endi undar þem liudjun wesan
mid þiu godes barnu \hld\ endi mid þeru gódan þior*nan,
wunon undar þemu werode, \hld\ untþat þi word kume
hérron þínes, \hld\ þat þu þat hèlage barn
eft te þesum land-skępi \hld\ lédjan mótis,
drohtin þínen.ʼ \hld\ Þó fon þem dròma ansprang
Joseph an is gęst-sęli, \hld\ endi þat godes gi-bod
sán ant-kenda: \hld\ gi-wét im an þana síð þanen
þe þegan mid þeru þiornon, \hld\ sóhta im þiod óðra
ovar brédan berg: \hld\ welda þat barn godes
fíundun ant-fórjan. \hld\ *Þó gi-frang aftar þiu % NOTE: gi-frang [sic]
Erodes þe kuning, \hld\ þar he an is ríkja sat,
þat wárun þea wíson man \hld\ westan gi-hworvan
óstar an iro óðil \hld\ endi fórun im óðran weg:
wisse þat sie im þat árundi \hld\ eft ni weldun
sęggjan an is selðon. \hld\ Þó warð im þes an sorgun hugi,
mód mornondi, \hld\ kwað þat it im þie man dedin,
hęliðos* te hónðun. \hld\ Þó he só hriwig sat,
balg ina an is briostun, \hld\ kwað þat he is mahti bętaron rád,
óðran gi-þęnkjen: \hld\ ʽnu ik is aldar kan,
wét is winter-gi-talu: \hld\ nu ik gi-winnan mag,
þat he io ovar þesaro erðu \hld\ ald ni wirðit,
hér undar þesum hęri-skępi.ʼ \hld\ Þó he só hardo gi-bód,
Erodes ovar is riki, \hld\ hét þó is rinkos faran
kuning þero liudjo, \hld\ hét þat sie kinda só filo
þurh iro hand-magen \hld\ hóvdu bi-námin,
só manag barn umbi Bethleem, \hld\ só filo só þar gi-boran wurði,
an twém gérun atogan. \hld\ Tionon frumidon
þes kuninges gi-síðos. \hld\ Þó skolda þar só manag kindisk man
sweltan sundjono lòs. \hld\ Ni warð sið noh ér
giámar-líkara for-gang \hld\ jungaro manno,
arm-líkara dóð. \hld\ Idisi wiopun,
módar managa, \hld\ gi-sáhun iro megi spildjan:
ni mahte siu im nio gi-formon, \hld\ þoh siu mid iro faðmon twém
iro égan barn \hld\ armun bi-fengi,
liof endi luttil, \hld\ þoh skolda is simbla þat líf gevan,
þe magu for þeru módar. \hld\ Ménes ni sáhun,
wítjes þie wam-skaðon: \hld\ wápnes ęggjun
fremidun firin-werk mikil. \hld\ Fellun managa
magu-junge man. \hld\ Þia módar wiopun
kind-jungaro kwalm; \hld\ kara was an Bethleem,
hofno hlúdost: \hld\ þoh man im iro herton an twé
sniði mid swerdu, \hld\ þoh ni mohta im gio sérara dád
werðan an þesaro weroldi, \hld\ wívun managun,
brúdiun an Bethleem: \hld\ gi-sáhun iro barn bi-foran,
kind-junge man, \hld\ kwalmu sweltan
blódag an iro barmun. \hld\ Þie banon wítnodun
un-skuldige skole: \hld\ ni bi-skrivun giowiht
þea man umbi mén-werk: \hld\ weldun mahtigna,
Krist selvon a-kwęlljan. \hld\ Þan habde ina kraftag god
gineridan wið iro níðe, \hld\ þat inan nahtes þanan
an Aegypteo land \hld\ erlos ant-léddun,
gumon mid Josepe \hld\ an þana grónjon wang,
an erðono bętstun, \hld\ þar én aha fliutid,
Níl-stròm mikil \hld\ norð te séwa,
flódo fagorosta. \hld\ Þar þat friðu-barn godes
wonoda an willjon, \hld\ antþat wurd fornam
Erodes þana kuning, \hld\ þat he for-lét eldeo barn,
módag manno dròm. \hld\ Þó skolda þero marka gi-wald
égan is ervi-ward: \hld\ þe was Arkheláus
hétan, hęri-togo \hld\ helm-berandero:
þe skolda umbi Hierusalem \hld\ Judeono folkes,
werodes gi-waldan. \hld\ Þó warð word kuman
þar an Egypti \hld\ eðiliun manne,
þat he þar te Josepe, \hld\ godes ęngil sprak,
bodo drohtines, \hld\ hét ina eft þat barn þanan
lédjen te lande. \hld\ ʽnu havað þit lioht afgevenʼ, kwað he,
ʽErodes þe kuning; \hld\ he welde is áhtien giu,
fréson is ferahas. \hld\ Nu maht þu an friðu lédjen
þat kind undar ewa kunni, \hld\ nu þe kuning ni livod,
erl ovar-módig.ʼ \hld\ Al ant-kende
Josep godes tékạn: \hld\ geriwide ina sniumo
þe þegan mit þera þiornun, \hld\ þó sie þanan weldun
béðju mid þiu barnu: \hld\ léstun þiu berhton gi-skapu,
waldandes willjon, \hld\ al só he im ér mid is wordun gi-bód.
Gi-witun im þó eft an Galilea-land \hld\ Joseph endi Maria,
hèlag híwiski \hld\ heven-kuninges,
wárun im an Nazareth-burg. \hld\ Þar þe nęrjondio Krist
wóhs undar þem werode, \hld\ warð gi-witties ful,
an was imu anst godes, \hld\ he was allun liof
módar-mágun: \hld\ he ni was óðrun mannun gi-lík,
þe gumo an sínera gódi. \hld\ Þó he gér-talo
twelivi habde, \hld\ þó warð þiu tíd kuman,
þat sie þar te Hierusalem, \hld\ Juðeo liudi
iro þiod-gode \hld\ þionon skoldun,
wirkjan is willjon. \hld\ Þó warð þar an þana wíh innan
þar te Hierusalem \hld\ Judeono gi-samnod
man-kraft mikil. \hld\ Þar Maria was
self an gi-síðja \hld\ endi iru sunu habda,
godes égan barn. \hld\ Þó sie þat geld habdun,
erlos an þem alaha, \hld\ só it an iro éwa gi-bód,
gi-léstid te iro land-wísun, \hld\ þó fórun im eft þie liudi þanan,
weros an iro willjon \hld\ endi þar an þem wíha afstód
mahtig barn godes, \hld\ só ina þiu módar þar
ni wissa te wáron; \hld\ ak siu wánda þat he mid þem weroda forð,
fóri mit iro friundun. \hld\ Gifrang aftar þiu
eft an óðrun daga \hld\ aðal-kunnjes wíf,
sálig þiorna, \hld\ þat he undar þem gi-síðia ni was.
warð Mariun þó \hld\ mód an sorgun,
hriwig umbi iro herta, \hld\ þó siu þat hèlaga barn
ni fand undar þem folka: \hld\ filu gornoda
þiu godes þiorna. \hld\ Gi-witun im þó eft te Hierusalem
iro sunu sókjan, \hld\ fundun ina sittjan þar
an þem wíha innan, \hld\ þar þe wísa man,
swíðo glauwa gumon \hld\ an godes éwa
lásun ende línodun, \hld\ hwó sie lof skoldin
wirkjan mid iro wordun þem, \hld\ þe þesa werold gi-skóp.
Þar sat undar middjun \hld\ mahtig barn godes,
Krist alo-waldo, \hld\ só is þea ni mahtun ant-kęnnjan wiht,
þe þes wíhes þar \hld\ wardon skoldun,
endi frágoda sie \hld\ firi-wit-líko
wísera wordo. \hld\ Sie wundradun alle,
bu-hwí gio só kindisk man \hld\ su-lika kwidi mahti
mid is múðu gi-ménean. \hld\ Þar ina þiu módar fand
sittjan under þem gi-síðja \hld\ endi iro sunu grótta,
wísan undar þem weroda, \hld\ sprak im mid ira wordun tó:
ʽhwí weldes þu þínera módar, \hld\ manno liovosto,
gi-sidon su-lika sorga, \hld\ þat ik þi só sèrag-mód,
idis arm-hugdig \hld\ éskon skolda
undar þesun burg-liudjun?ʼ \hld\ Þó sprak iru eft þat barn an-gegin
wísun wordun: \hld\ ʽhwat, þu wést garoʼ, kwað he,
ʽþat ik þar gi-rísu, \hld\ þar ik bi rehton skal
wonon an willjon, \hld\ þar gi-wald havad
mín mahtig fader.ʼ \hld\ Þie man ni for-stódun,
þie weros an þem wíha, \hld\ bi-hwí he só þat word gi-sprak,
gi-ménda mid is múðu: \hld\ Maria al bi-held,
gi-barg an ira breostun, \hld\ só hwat só siu gi-hórda ira barn sprekan
wisaro wordo. \hld\ Gi-witun im þó eft þanan
fon Hierusalem \hld\ Joseph endi Maria,
habdun im te gi-síðja \hld\ sunu drohtines,
allaro barno bętsta, \hld\ þero þe io gi-boran wurði
magu fon módar: \hld\ habdun im þar minnja tó
þurh hluttran hugi, \hld\ endi he só gi-hórig was,
godes égan barn \hld\ gaduling-mágun
þurh is ód-módi, \hld\ aldron sínun:
ni welda an is kindiski þó noh \hld\ is kraft mikil
mannun márjan, \hld\ þat he su-lik męgin éhta,
gi-wald an þesaro weroldi, \hld\ ak he im an is willjon béd
gi-þiudo undar þero þiodu \hld\ þrítig géro,
ér þan he þar tékạn énig \hld\ tógjan weldi,
sęggjan þem gi-síðja, \hld\ þat he selvo was
an þesaro middil-gard \hld\ manno drohtin.
Habda im só bi-halden \hld\ hèlag barn godes
word endi wís-dóm \hld\ ende allaro gi-witteo mést,
tulgo spáhan hugi: \hld\ ni mahta man is an is sprákun werðan,
an is wordun gi-war, \hld\ þat he su-lik gi-wit éhta,
þegan su-lika gi-þáhti, \hld\ ak he im só gi-þiudo béd
torhtaro tékno. \hld\ Ni was noh þan þiu tíd kuman,
þat he ina ovar þesan \hld\ middil-gard márjan skolda,
lérjan þie liudi, \hld\ hwó sie skoldin iro gi-lóvon haldan,
wirkjan willjon godes; \hld\ wissun þat þoh managa
liudi aftar þem landa, \hld\ þat he was an þit lioht kuman,
þoh sie ina kúð-líko \hld\ an-kennjan ni mahtin,
ér þan he ina selvo \hld\ sęggjan welda.
Þan was im Johannes \hld\ fon is iuguð-hédi
awahsan an énero wóstunni; \hld\ þar ni was werodes þan mér,
bútan þat he þar énkora \hld\ alo-waldon gode,
þegan þionoda: \hld\ for-lét þioda gi-mang,
manno giménðon. \hld\ Þar warð im mahtig kuman
an þero wóstunni \hld\ word fon himila,
gód-lík stemna godes, \hld\ endi Johanne gi-bod,
þat he Kristes kumi \hld\ endi is kraft mikil
ovar þesan middil-gard \hld\ márjan skoldi;
hét ina wár-líko \hld\ wordun sęggjan,
þat wári hevan-riki \hld\ hęliðo barnun
an þem land-skępi, \hld\ liudjun gi-náhid,
welono wun-samost. \hld\ Im was þó willjo mikil,
þat he fon su-likun sáldun \hld\ sęggjan mósti.
Gi-wét im þó gangan, \hld\ al só Jordan flót,
watar an willjon, \hld\ endi þem weroda allan dag,
aftar þem land-skępi \hld\ þem liudjun kúðda,
þat sie mid fastunniu \hld\ firin-werk manag,
iro selvoro \hld\ sundja bóttin,
ʽþat gi werðan hréneaʼ, \hld\ kwað he. ʽHevan-riki is
gi-náhid manno barnun. \hld\ Nu látad eu an ewan mód-sevon
ewar selvoro \hld\ sundja hrewan,
lédas þat gi an þesun liohta fremidun, \hld\ endi mínun lérun hórjad,
węndjat aftar mínun wordun. \hld\ Ik eu an watara skal
gi-dópjan diur-líko, \hld\ þoh ik ewa dádi ne mugi,
ewar selvaro \hld\ sundja a-látan,
þat gi þurh mín hand-gi-werk \hld\ hluttra werðan
léðaro gi-lésto: \hld\ ak þe is an þit lioht kuman,
mahtig te mannun \hld\ endi undar eu middjun stéd,
—þoh gi ina selvun \hld\ gi-sehan ni willjan—,
þe eu gi-dópjan skal \hld\ an ewes drohtines namon
an þana hálagon gést. \hld\ Þat is hérro ovar al:
he mag allaro manno gi-hwena \hld\ mén-gi-þáhteo,
sundjono sikoron, \hld\ só hwene só só sálig mót
werðen an þesaro weroldi, \hld\ þat þes willjon havad,
þat he só gi-léstja, \hld\ só he þesun liudjun wili,
gi-bioden barn godes. \hld\ Ik bium an is bod-skępi herod
an þesa werold kumen \hld\ endi skal im þana weg rúmien,
lérjan þesa liudi, \hld\ hwó sea skulin iro gi-lóvon haldan
þurh hluttran hugi, \hld\ endi þat sie an hęllja ni þurvin,
faran an fern þat héta. \hld\ Þes wirðid só fagan an is móde
man te só managaro stundu, \hld\ só hwe só þat mén for-látid,
gerno þes gramon anbusni, \hld\ —só mag im þes gódon gi-wirkjan,
huldi heven-kuninges,— \hld\ só hwe só havad hluttra trewa
up te þem alo-mahtigon gode.ʼ \hld\ Erlos managa
bi þem lérun þó, \hld\ liudi wándun,
weros wár-líko, \hld\ þat þat waldand Krist
selbo wári, \hld\ hwanda he só filu sóðes gi-sprak,
wároro wordo. \hld\ Þó warð þat só wído kúð
ovar þat for-gevana land \hld\ gumono gi-hwi-likum,
sęggjun at iro selðun: \hld\ þó kwámun ina sókjan þarod
fon Hierusalem \hld\ Judeo liudjo
bodon fon þeru burgi \hld\ endi frágodun, ef he wári þat barn godes,
ʽþat hér lango giuʼ, \hld\ kwaðun sie, ʽliudi sagdun,
weros wár-líko, \hld\ þat he skoldi an þesa werold kumanʼ.
Johannes þó gi-mahalde \hld\ endi te-gegnes sprak
þem bodun bald-líko: \hld\ ʽni bium ikʼ, kwað he, ʽþat barn godes,
wár waldand Krist, \hld\ ak ik skal im þana weg rúmien,
hérron mínumu.ʼ \hld\ Þea hęliðos frugnun,
þea þar an þem árundie \hld\ erlos wárun,
bodon fon þero burgi: \hld\ ʽef þu nu ni bist þat barn godes,
bist þu þan þoh Elias, \hld\ þe hér an ér-dagun
was undar þesumu werode? \hld\ He is wiskumo
eft an þesan middil-gard. \hld\ Saga ús hwat þu manno sís!
Bist þu énig þero, \hld\ þe hér ér wári
wísaro wár-saguno? \hld\ Hwat skulun wi þem werode fon þi
sęggjan te sóðon? \hld\ Neo hér ér su-lik ni warð
an þesun middil-gard \hld\ man óðar kuman
dádjun só mári. \hld\ Bi-hwí þu hér dópisli
fremis undar þesumu folke, \hld\ ef þu þaro fora-sagono
én-hwi-lik ni bist?ʼ \hld\ Þó habde eft garo
Johannes þe gódo \hld\ glau and-wordi:
ʽIk bium fora-bodo \hld\ fráon mínes,
lioves hérron; \hld\ ik skal þit land rekon,
þit werod aftar is willjon. \hld\ Ik hębbju fon is worde mid mi
stranga stemna, \hld\ þoh sie hér ni willje for-standan filo
werodes an þesaro wóstunni. \hld\ Ni bium ik mid wihti gi-lík
drohtine mínumu: \hld\ he is mid is dádjun só strang,
só mári endi só mahtig \hld\ —þat wirðid managun kúð,
werun aftar þesaro weroldi— \hld\ þat ik þes wirðig ni bium,
þat ik móti an is gi-skuoha, \hld\ þoh ik sí is skalk égan,
an só ríkiumu drohtine, \hld\ þea reomon ant-bindan:
só mikilu is he bętara þan ik. \hld\ Nis þes bodon gi-mako
énig ovar erðu, \hld\ ne nu aftar ni skal
werðan an þesaro weroldi. \hld\ Hębbjad ewan willjon þarod,
liudi ewan gi-lóvon: \hld\ þan eu lango skal
wesan ewa hugi hrómag; \hld\ þan gi hęlli-gi-þwing,
for-látad léðaro dròm \hld\ endi sókjad eu lioht godes,
up-ódes hém, \hld\ éwig ríki,
hòhan heven-wang. \hld\ Ne látad ewan hugi twíflien!ʼ
Só sprak þó jung gumo \hld\ bi godes lérun
mannun te márðu. \hld\ Manag samnoda
þar te Bethania \hld\ barn Israheles;
kwámun þar te Johannese \hld\ kuningo gi-síðos,
liudi te lérun \hld\ endi iro gi-lóvon ant-fengun.
He dópte sie dago gi-hwi-likes \hld\ endi im iro dádi lóg,
wréðaro willjon, \hld\ endi lovode im word godes,
hérron sínes: \hld\ ʽheven-ríki wirðidʼ, kwað he,
ʽgaru gumono só hwem, \hld\ só ti gode þenkid
endi an þana héljand *wili \hld\ hluttro gi-lóvjan, %NOTE: wili] P 1r.
léstjan is léraʼ. \hld\ Þó ni was lang te þiu,
þat im fon Galilea gi-wét \hld\ godes égan barn,
*diur-lík drohtines sunu, \hld\ dópi suokjan.
was im þuo an is wastme \hld\ waldandes barn*,
al só he mid þero þiodu \hld\ þrítig habdi
wintro an is weroldi. \hld\ Þó he an is willjon kwam,
þar Johannes \hld\ an Jordana stròme
allan langan dag \hld\ liudi manage
dópte diur-líko. \hld\ Reht só he þó is drohtin gi-sah,
holdan hérron, \hld\ só warð im is hugi blíði,
þes im þe willjo gi-stód, \hld\ endi sprak im þó mid is wordun tó,
swíðo gód gumo, \hld\ Johannes te Kriste:
ʽnu kumis þu te mínero dópi, \hld\ drohtin fró mín,
þiod-gumono bętsto: \hld\ só skolde ik te þínero duan,
hwand þu bist allaro kuningo kraftigost.ʼ \hld\ Krist selvo gi-bód,
waldand wár-líko, \hld\ þat he ni spráki þero wordo þan mér:
ʽwést þu, þat ús só girísidʼ, \hld\ kwað he, ʽallaro rehto gi-hwi-lik
te gi-fulleanne \hld\ forð-wardes nu
an godes willjonʼ. \hld\ Johannes stód,
dópte allan dag \hld\ druht-folk mikil,
werod an watere \hld\ endi ók waldand Krist,
héran heven-kuning \hld\ handun sínun
an allaro baðo þem bętston \hld\ endi im þar te bedu gi-hnég
an kneo kraftag. \hld\ Krist up gi-wét
fagar fon þem flóde, \hld\ friðu-barn godes,
liof liudjo ward. \hld\ Só he þó þat land af-stóp,
só ant-hlidun þó himiles doru, \hld\ endi kwam þe hèlago gést
fon þem alo-waldon \hld\ ovane te Kriste:
—was im an gi-líknissie \hld\ lungras fugles,
diur-líkara dúvun— \hld\ endi sat im uppan úses drohtines ahslu,
wonoda im ovar þem waldandes barne. \hld\ Aftar kwam þar word fon himile,
hlúd fon þem hòhon radura \hld\ endi grótta þane héljand selvon,
Krista, allaro kuningo bętston, \hld\ kwað þat he ina gi-korana habdi
selvo fon sínun ríkja, \hld\ kwað þat im þe sunu líkodi
bętst allaro gi-boranaro manno, \hld\ kwað þat he im wári allaro barno liovost.
Þat móste Johannes þó, \hld\ al só it god welde,
gi-sehan endi gi-hórjan. \hld\ He gi-deda it sán aftar þiu
mannun mári, \hld\ þat sie þar mahtigna
hérron habdun: \hld\ ʽÞit isʼ, kwað he, ʽheven-kuninges sunu,
én alo-waldand: \hld\ þesas willjo ik ur-kundjo
wesan an þesaro weroldi, \hld\ hwand it sagda mi word godes,
drohtines stemne, \hld\ þó he mi dópjan hét
weros an watare, \hld\ só hwar só ik gi-sáwi wár-líko
þana hèlagon gést \hld\ *fan hevan-wange
an þesan middil-gard \hld\ énigan man waron,
kuman mid kraftu; \hld\ þat kwað, þat skoldi Krist wesan,
diur-lík drohtines suno. \hld\ Hie dópjan skal
an þana hèlagan gést \hld\ endi hélean managa %NOTE: þana] P end.
manno mén-dádi. \hld\ He havad maht fon gode,
þat he a-látan mag \hld\ liudjo gi-hwi-likun
saka endi sundja. \hld\ Þit is selvo Krist,
godes égan barn, \hld\ gumono bętsto,
friðu wið fíundun. \hld\ Wala þat eu þes mag fráh-mód hugi
wesan an þesaro weroldi, \hld\ þes eu þe willjo gi-stód,
þat gi só libbjanda \hld\ þana landes ward
selvon gi-sáhun. \hld\ Nu mót sliumo sundjono lòs
manag gést faran \hld\ an godes willjon
tionon a-tómid, \hld\ þe mid trewon wili
wið is wini wirkjan \hld\ endi an waldand Krist
fasto gi-lóvjan. \hld\ Þat skal te frumun werðen
gumono só hwi-likun, \hld\ só þat gerno dótʼ.
Só ge-fragn ik þat Johannes \hld\ þó gumono gi-hwi-likun,
lovoda þem liudjun \hld\ léra Kristes,
hérron sines, \hld\ endi heven-ríki
te gi-winnanne, \hld\ welono þane méston,
sálig sin-líf. \hld\ Þó he im selvo gi-wét
aftar þem dópislea, \hld\ drohtin þe gódo,
an éna wóstunnja, \hld\ waldandes sunu;
was im þar an þero én-ódi \hld\ erlo drohtin
lange hwíla; \hld\ ne habda liudjo þan mér,
seggjo te gi-síðun, \hld\ al só he im selvo gi-kòs:
welda is þar látan koston \hld\ kraftiga wihti,
selvon Satanasan, \hld\ þe gio an sundja spenit,
man an mén-werk: \hld\ he konsta is mód-sevon,
wréðan willjon, \hld\ hwó he þesa werold érist,
an þem an-ginnja \hld\ irmin-þioda
bi-swék mit sundjun, \hld\ þó he þiu sinhíun twé,
Ádaman endi Éuan, \hld\ þurh un-trewa
for-lédda mid luginun, \hld\ þat liudo barn
aftar iro hin-ferdi \hld\ hęllja sóhtun,
gumono géstos. \hld\ Þó welda þat god mahtig,
waldand węndjan \hld\ endi welda þesum werode for-geven
hòh himil-ríki: \hld\ be-þiu he herod hèlagna bodon,
is sunu senda. \hld\ Þat was Satanase
tulgo harm an is hugi: \hld\ afonsta hevan-ríkjes
manno kunnje: \hld\ welda þó mahtigna
mid þem selvon sakun \hld\ sunu drohtines,
þem he Ádaman \hld\ an ér-dagun
darnungo bidróg, \hld\ þat he warð is drohtine léð,
bi-swék ina mid sundjun \hld\ —só welda he þó selvan dón
hélandean Krist. \hld\ Þan habda he is hugi fasto
wið þana wam-skaðon, \hld\ waldandes barn,
herte só gi-herdid: \hld\ welda heven-ríki
liudjun gi-léstjan. \hld\ Was im þes landes ward
an fastunnea \hld\ fior-tig nahto,
manno drohtin, \hld\ só he þar mates ni antbét;
þan langa ni gi-dorstun \hld\ im dernea wihti,
níð-hugdig fíund, \hld\ náhor gangan,
grótjan ina gegin-warðan: \hld\ wánde þat he god énfald,
for-útar man-kunnjes wiht \hld\ mahtig wári,
héleg himiles ward. \hld\ Só he ina þó ge-hungrean lét,
þat ina bi-gan bi þero męnnisko \hld\ móses lustean
aftar þem fiwar-tig dagun, \hld\ þe fíund náhor geng,
mirki mén-skaðo: \hld\ wánda þat he man énfald
wári wissungo, \hld\ sprak im þó mid is wordun tó,
grótta ina þe gér-fíund: \hld\ ʽef þu sís godes sunuʼ, kwað he,
ʽbe-hwí ni hétis þu þan werðan, \hld\ ef þu gi-wald haves,
allaro barno bętst, \hld\ bród af þesun sténun?
Gehéli þínna hungar.ʼ \hld\ Þó sprak eft þe hèlago Krist:
ʽni mugun eldi-barnʼ, \hld\ kwað he, ʽénfaldes bródes,
liudi libbjen, \hld\ ak sie skulun þruh léra godes %TODO: Verify þruh
wesan an þesero weroldi \hld\ endi skulun þiu werk frummjen,
þea þar werðad a-hlúdid \hld\ fon þero hélogun tungun,
fon þem galme godes: \hld\ þat is gumono líf
liudjo só hwi-likon, \hld\ só þat léstjan wili,
þat fon waldandes \hld\ worde ge-biudid.ʼ
Þó bi-gan eft niuson \hld\ endi náhor geng
un-hiuri fíund \hld\ óðru síðu,
fandoda is fróhan. \hld\ Þat friðu-barn þolode
wréðes willjon \hld\ endi im gi-wald for-gaf,
þat he umbi is kraft mikil \hld\ koston mósti,
lét ina þó lédean \hld\ þana liud-skaðon,
þat he ina an Hierusalem \hld\ te þem godes wíha,
alles ovan-wardan, \hld\ up gi-setta
an allaro húso hòhost, \hld\ endi hosk-wordun sprak,
þe gramo þurh gelp mikil: \hld\ ʽef þu sís godes sunuʼ, kwað he,
ʽskríd þi te erðu hinan. \hld\ Ge-skrivan was it giu lango,
an bókun ge-writen, \hld\ hwó gi-boden havad
is ęngilun \hld\ alo-mahtig fader,
þat sie þi at wege ge-hwem \hld\ wardos sinðun,
haldad þi undar iro handun. \hld\ Hwat, þu hwargin ni þarft
mid þínun fótun \hld\ an felis be-spurnan,
an hardan stén.ʼ \hld\ Þó sprak eft þe hèlago Krist,
allaro barno bętst: \hld\ ʽsó is ók an bókun ge-skrivanʼ, kwað he,
ʽþat þu te hardo ni skalt \hld\ hérran þínes,
fandon þínes fróhan: \hld\ þat nis þi allaro frumono negén.ʼ
Lét ina þó an þana þriddjan síð \hld\ þana þiod-skaðon
gi-brengen uppan énan berg þen hòhon: \hld\ þar ina þe balo-wíso
lét al ovar-sehan \hld\ irmin-þiode,
wonod-saman welon \hld\ endi werold-ríki
endi all su-lik ódes, \hld\ só þius erða bi-havad
fagororo frumono, \hld\ endi sprak im þó þe fíund an-gegin,
kwað þat he im þat al só gód-lík \hld\ for-geven weldi,
hòha heri-dómos, \hld\ ʽef þu wilt hnígan te mi,
fallan te mínun fótun \hld\ endi mi for fróhan havas,
bedos te mínun barma. \hld\ Þan látu ik þi brúkan wel
alles þes òd-welon, \hld\ þes ik þi hębbju gi-ógit hír.ʼ
Þó ni welda þes léðan word \hld\ lengeron hwíle
hórjan þe hèlago Krist, \hld\ ak he ina fon is huldi for-dréf,
Satanasan for-swép, \hld\ endi sán aftar sprak
allaro barno bętst, \hld\ kwað þat man bedon skoldi
up te þem alo-mahtigon gode \hld\ endi im énum þionon
swíðo þio-liko \hld\ þegnos managa,
hęliðos aftar is huldi: \hld\ ʽþar ist þiu helpa gelang
manno ge-hwi-likun.ʼ \hld\ Þó gi-wét im þe mén-skaðo,
swíðo sèrag-mód \hld\ Satanas þanan,
fíund undar fern-dalu. \hld\ Warð þar folk mikil
fon þem alo-waldan \hld\ ovana te Kriste
godes ęngilo kumen, \hld\ þie im síðor jungar-dóm,
skoldun ambaht-skępi \hld\ aftar léstjen,
þionon þiolíko: \hld\ só skal man þiod-gode,
hérron aftar huldi, \hld\ hevan-kuninge.
was im an þem sin-weldi \hld\ sálig barn godes
lange hwíle, \hld\ untþat im þó liovora warð,
þat he is kraft mikil \hld\ kúðjen wolda
weroda te willjon. \hld\ Þó for-lét he waldes hléo,
én-ódjes ard \hld\ endi sóhte im eft erlo gemang,
mári męgin-þiode \hld\ endi manno dròm,
geng im þó bi Jordanes staðe: \hld\ þar ina Johannes ant-fand,
þat friðu-barn godes, \hld\ fróhan sínan,
hèlagana heven-kuning, \hld\ endi þem hęliðun sagda,
Johannes is jungurun, \hld\ þó he ina gangan ge-sah:
ʽþit is þat lamb godes, \hld\ þat þar lósjan skal
af þesaro wídon werold \hld\ wréða sundja,
man-kunnjas mén, \hld\ mári drohtin,
kuningo kraftigost.ʼ \hld\ Krist im forð gi-wét
an Galileo land, \hld\ godes égan barn,
fór im te þem friundun, \hld\ þar he a-fódit was,
tír-líko atogan, \hld\ endi talda mid wordun
Krist undar is kunnje, \hld\ kuningo ríkjost,
hwó sie skoldin iro selvoro \hld\ sundja bótjan,
hét þat sie im iro harmwerk manag \hld\ hrewan létin,
feldin iro firin-dádi: \hld\ ʽnu is it all ge-fullot só,
só hír alde man \hld\ ér hwanna sprákun,
ge-hétun eu te helpu \hld\ heven-ríki:
nu is it giu gi-náhid þurh þes nęrjandan kraft: \hld\ þes mótun gi neotan forð,
só hwe só gerno wili \hld\ gode þeonogean,
wirkjan aftar is willjon.ʼ \hld\ Þó warð þes werodes filu,
þero liudjo an lustun: \hld\ wurðun im þea léra Kristes,
só swótja þem gi-síðja. \hld\ He bi-gan im samnon þó
gumono te jungoron, \hld\ gódoro manno,
word-spáha weros. \hld\ Geng im þó bi énes watares staðe,
þat þar habda Jordan \hld\ anevan Galileo land
énna sé ge-warhtan. \hld\ Þar he sittjan fand
Andreas endi Petrus \hld\ bi þem aha-stròme,
béðja þea ge-bróðar, \hld\ þar sie an bréd watar
swíðo niud-líko \hld\ netti þenidun,
fiskodun im an þem flóde. \hld\ Þar sie þat friðu-barn godes
bi þes sées staðe \hld\ selvo grótta,
hét þat sie im folgodin, \hld\ kwað þat he im só filu woldi
godes ríkjas for-geven; \hld\ ʽal só git hír an Jordanes stròme
fiskos fáhat, \hld\ só skulun git noh firiho barn
halon te inkun handun, \hld\ þat sie an heven-ríki
þurh inka léra \hld\ líðan mótin,
faran folk manag.ʼ \hld\ Þó warð fró-mód hugi
béðjun þem gi-bróðrun: \hld\ ant-kendun þat barn godes,
liovan hérron: \hld\ for-létun al saman
Andreas endi Petrus, \hld\ só hwat só sie bi þeru ahu habdun,
ge-wunstes bi þem watare: \hld\ was im willjo mikil,
þat sie mid þem godes barne \hld\ gangan móstin,
samad an is gi-síðja, \hld\ skoldun sálig-líko
lòn ant-fáhan: \hld\ só dót liudjo so hwi-lik,
só þes hérran wili \hld\ huldi gi-þionon,
ge-wirkjan is willjon. \hld\ Þó sie bi þes watares staðe
furðor kwámun, \hld\ þó fundun sie þar énna fródan man
sittjan bi þem séwa \hld\ endi is suni twéne,
Jakobus endi Johannes: \hld\ wárun im junga man.
Sátun im þá ge-sun-fader \hld\ an énumu sande uppen,
brugdun endi bóttun \hld\ béðjum handun
þiu netti niud-líko, \hld\ þea sie habdun nahtes ér
for-sliten an þem séwa. \hld\ Þar sprak im selvo tó
sálig barn godes, \hld\ hét þat sie an þana síð mid im,
Jakobus endi Johannes, \hld\ gengin béðje,
kind-junge man. \hld\ Þó wárun im Kristes word
só wirðig an þesaro weroldi, \hld\ þat sie bi þes watares staðe
iro aldan fader \hld\ énna for-létun,
fródan bi þem flóde, \hld\ endi al þat sie þar fehas éhtun,
nęttju endi neglit-skipu, \hld\ ge-kurun im þana nęrjandan Krist,
hèlagna te hérron, \hld\ was im is helpono þarf
te gi-þiononne: \hld\ só is allaro þegno ge-hwem,
wero an þesero weroldi. \hld\ Þó gi-wét im þe waldandes sunu
mid þem fiwariun forð, \hld\ endi im þó þana fífton gi-kòs
Krist an énero kóp-stędi, \hld\ kuninges jungoron,
mód-spáhana man: Mattheus was he hétan,
was im ambahtjo eðilero manno,
skolda þar te is hérron handun ant-fáhan
tins endi tolna; trewa habda he góda,
áðalandbári: for-lét al saman
gold endi siluvar endi geva managa,
diurje méðmos, endi warð im úses drohtines man;
kòs im þe kuninges þegn Krist te hérran,
milderan mèðom-gevon, þan ér is man-drohtin
wári an þesero weroldi: feng im wóðera þing,
lang-samoron rád. Þó warð it allun þem liudjun kúð,
fon allaro burgo gi-hwem, hwó þat barn godes
samnode ge-síðos endi selvo ge-sprak
só manag wíslík word endi wáres só filu,
torhtes gi-tógde \hld\ endi tékạn manag
ge-warhte an þesero weroldi. \hld\ Was þat an is wordun skín
iak an is dádjun só same, \hld\ þat he drohtin was,
himilisk hérro \hld\ endi te helpu kwam
an þesan middil-gard \hld\ manno barnun,
liudjun te þesun liohta. \hld\ Oft ge-deda he þat an þem lande skín,
þan he þar torht-líko \hld\ só manag tékạn gi-warhte,
þar he hélde mid is handun \hld\ halte endi blinde,
lósde af þeru léf-hédi \hld\ liudi manage,
af su-likun suhtiun, \hld\ só þan allaro swároston
an firiho barn \hld\ fíund bi-wurpun,
tulgo lang-sam legar. \hld\ Þó fórun þar þie liudi tó
allaro dago ge-hwi-likes, \hld\ þar úsa drohtin was
selvo undar þem gi-síðje, \hld\ untþat þar ge-samnod warð
męgin-folk mikil \hld\ managero þiodo,
þoh sie þar alle be ge-líkumu \hld\ ge-lóvon ni kwámin.
weros þurh énan willjon: \hld\ sume sóhtun sie þat waldandes barn,
armoro manno filu \hld\ —was im átes þarf—,
þat sie im þar at þeru menigi \hld\ mates endi drankes,
þigidin at þeru þiodu; \hld\ hwand þar was manag þegan só gód,
þie ira alamosnje \hld\ armun mannun
gerno gávun. \hld\ Sume wárun sie im eft Judeono kunnjes,
fégni folk-skępi: \hld\ wárun þar ge-farana te þiu,
þat sie úses drohtines \hld\ dádjo endi wordo
fáron woldun, \hld\ habdun im fégnjen hugi,
wréðen willjon: \hld\ woldun waldand Krist
alédjen þem liudjun, \hld\ þat sie is léron ni hórdin,
ne wendin aftar is willjon. \hld\ Suma wárun sie im eft só wíse man,
wárun im glawe gumon \hld\ endi gode werðe,
alesane undar þem liudjun, \hld\ kwámun im þarod be þem léron Kristes,
þat sie is hèlag word \hld\ hórjen móstin,
línon endi léstjen: \hld\ habdun mid iro ge-lóvon te im
fasto gefangen, \hld\ habdun im ferhten hugi,
wurðun is þegnos te þiu, \hld\ þat he sie an þiod-welon
aftar iro én-dagon \hld\ up ge-bráhti,
an godes ríki. \hld\ He só gerno ant-feng
man-kunnjes manag \hld\ endi mund-burd gi-hét
te langaru hwílu, \hld\ endi mahta só gi-léstjen wel.
Þó warð þar męgin só mikil \hld\ umbi þana márjon Krist,
liudjo ge-samnod: \hld\ þó gi-sah he fon allun landun kuman,
fon allun wídun wegun \hld\ werod te-samne
lungro liudjo: \hld\ is lof was só wído
managun ge-márid. \hld\ Þó gi-wét im mahtig self
an énna berg uppan, \hld\ barno ríkjost,
sundar ge-sittjen, \hld\ endi im selvo ge-kòs
twelivi ge-talda, \hld\ trew-hafta man,
gódoro gumono, \hld\ þea he im te jungoron forð
allaro dago ge-hwi-likes, \hld\ drohtin welda
an is ge-síð-skępja \hld\ simblon hębbjan.
Nemnida sie þó bi naman \hld\ endi hét sie im þó náhor gangan,
Andreas endi Petrus \hld\ érist sána,
ge-bróðar twéne, \hld\ endi béðje mid im,
Jakobus endi Johannes: \hld\ sie wárun gode werðe;
mildi was he im an is móde; \hld\ sie wárun énes mannes suni
béðje bi ge-burdjun; \hld\ sie kòs þat barn godes
góde te jungoron \hld\ endi gumono filu,
márjero manno: \hld\ Mattheus endi Þomas,
Judasas twéna \hld\ endi Jakob óðran,
is selves swiri: \hld\ sie wárun fon gi-sustruonion twém
knósles kumana, \hld\ Krist endi Jakob,
góde gadulingos. \hld\ Þó habda þero gumono þar
þe nęrjendo Krist \hld\ niguni ge-talde, %TODO: check niguni
trew-hafte man: \hld\ þó hét he ók þana te-handon gangan
selvo mid þem gi-síðun: \hld\ Símon was he hétan;
hét ók Bartholomeus \hld\ an þana berg uppan
faran fan þem folke áðrum \hld\ endi Philippus mid im,
trew-hafte man. \hld\ Þó gengun sie twelivi samad,
rinkos te þeru rúnu, \hld\ þar þe rádand sat,
managoro mund-boro, \hld\ þe allumu man-kunnje
wið hęllje ge-þwing \hld\ helpan welde,
formon wið þem ferne, \hld\ só hwem só frummjen wili
só liov-líka léra, \hld\ só he þem liudjun þar
þurh is gi-wit mikil \hld\ wísjan hogda.
*Þó umbi þana nęrjendon Krist \hld\ náhor gengun
su-like ge-síðos, \hld\ só he im selvo ge-kòs,
waldand undar þem werode. \hld\ Stódun wísa man,
gumon umbi þana godes sunu \hld\ gerno swíðo,
weros an willjon: \hld\ was im þero wordo niud,
þáhtun endi þagodun, \hld\ hwat im þero þiodo drohtin,
weldi waldand self \hld\ wordun kúðjen
þesum liudjun te liove. \hld\ Þan sat im þe landes hirdi
gegin-ward for þem gumun, \hld\ godes égan barn:
welda mid is sprákun \hld\ spáh-word manag
lérjan þea liudi, \hld\ hwó sie lof gode
an þesum werold-ríkja \hld\ wirkjan skoldin.
Sat im þó endi swígoda \hld\ endi sah sie an lango,
was im hold an is hugi \hld\ hèlag drohtin,
mildi an is móde, \hld\ endi þó is mund ant-lók,
wísde mid wordun \hld\ waldandes sunu
manag már-lík þing \hld\ endi þem mannum sagde
spáhun wordun, \hld\ þem þe he te þeru spráku þarod,
Krist alo-waldo, \hld\ ge-koran habda,
hwi-like wárin allaro \hld\ irmin-manno
gode werðoston \hld\ gumono kunnjes;
sagde im þó te sóðan, \hld\ kwað þat þie sáliga wárin,
man an þesoro middil-gardun, \hld\ þie hér an iro móde wárin
arme þurh ód-módi: \hld\ ʽþem is þat éwana ríki,
swíðo hèlag-lík \hld\ an hevan-wange
sin-líf far-geven.ʼ \hld\ Kwað þat ók sálige wárin
máð-mundje man: \hld\ ʽþie mótun þie márjon erðe,
of-sittjen þat selve ríki.ʼ \hld\ Kwað þat ók sálige wárin,
þie hír wiopin iro wammun dádi; \hld\ ʽþie mótun eft willjon ge-bídan,
frófre an iro fráhon ríkja. \hld\ Sálige sind ók, þe sie hír frumono gi-lustid,
rinkos, þat sie rehto a-dómien. \hld\ Þes mótun sie werðan an þem ríkja drohtines
gi-fullit þurh iro ferhton dádi: \hld\ su-líkoro mótun sie frumono bi-knégan
þie rinkos, þie hír rehto a-dómjad, \hld\ ne willjad an rúnun be-swíkan
man, þar sie at mahle sittjad. \hld\ Sálige sind ók þem hír mildi wirðit
hugi an hęliðo briostun: \hld\ þem wirðit þe hélego drohtin,
mildi mahtig selvo. \hld\ Sálige sind ók undar þesaro managon þiodu,
þie hębbjad iro herta gi-hrénod: \hld\ þie mótun þane hevenes waldand
sehan an sínum ríkja.ʼ \hld\ Kwað þat ók sálige wárin,
ʽþie þe friðu-samo undar þesumu folke libbiod \hld\ endi ni willjad éniga fehta ge-wirken,
saka mid iro selvoro dádjun: \hld\ þie mótun wesan suni drohtines ge-nemnide,
hwande he im wil ge-nádig werðen; \hld\ þes mótun sie niotan lango
selvon þes sínes ríkjes.ʼ \hld\ Kwað þat ók sálige wárin
þie rinkos, þe rehto weldin, \hld\ ʽendi þurh þat þolod ríkjoro manno
heti endi harm-kwidi: \hld\ þem is ók an himile eft
godes wang for-geven \hld\ endi gést-lík líf
aftar te éwan-dage, \hld\ só is io endi ni kumit,
welan wun-sames.ʼ \hld\ Só habde þó waldand Krist
for þem erlon þar \hld\ ahto ge-talda
sálda ge-sagda; \hld\ mid þem skal simbla gi-hwe
himil-rík ge-halon, \hld\ ef he it hębbjen wili,
etþo he skal te éwan-daga \hld\ aftar þarvon
welon endi willjon, \hld\ síðor he þese werold agivid,
erð-lívi-gi-skapu, \hld\ endi sókit im óðar lioht
só liof só léð, \hld\ só he mid þesun liudjun hér
gi-werkod an þesoro weroldi, \hld\ al só it þar þó mid is wordun sagde
Krist alo-waldo, \hld\ kuningo ríkjost
godes égen barn \hld\ jungorun sínun:
ʽGe werðat ók só sáligeʼ, \hld\ kwað he, ʽþes iu saka biodat
liudi aftar þeson lande \hld\ endi léð sprekat,
hębbjad iu te hoska \hld\ endi harmes filu
ge-wirkiad an þesoro weroldi \hld\ endi wíti ge-frummjad,
felgiad iu firin-spráka \hld\ endi fíund-skępi,
lágnjad iuwa léra, \hld\ dót iu léðes filu,
harmes þurh iuwen hérron. \hld\ Þes látad gi ewan hugi simbla,
líf an lustun, \hld\ hwand iu þat lòn stendit
an godes ríkja garu, \hld\ gódo ge-hwi-likes,
mikil endi manag-fald: \hld\ þat is iu te médu far-geven,
hwand gi hér ér bi-foran \hld\ arvid þolodun,
wíti an þesoro weroldi. \hld\ Wirs is þem óðrun,
giviðig grimmora þing, \hld\ þem þe hér gód égun,
wídan worold-welon: \hld\ þie for-slítat iro wunnja hér;
ge-niudot sie ge-nóges, \hld\ skulun eft narowaro þing
aftar iro hin-ferdi \hld\ hęliðos þolojan.
Þan wópjan þar wan-skęfti, \hld\ þie hér ér an wunnjon sín,
libbjad an allon lustun, \hld\ ne willjad þes far-látan wiht,
méni-gi-þáhtio, \hld\ þes sie an iro mód spenit,
léðoro gi-léstio. \hld\ Þan im þat lòn kumid,
uvil arvet-sam, \hld\ þan sie is þane endi skulun
sorgondi ge-sehan. \hld\ Þan wirðid im sér hugi,
þes sie* þesero weroldes só filu \hld\ willjan ful-gengun,
man an iro mód-sevon. \hld\ Nu skulun gi im þat mén lahan,
węrjan mid wordun, \hld\ al só ik giu nu ge-wísjan mag,
sęggjan sóð-líko, \hld\ ge-síðos míne,
wárun wordun, \hld\ þat gi þesoro weroldes nu forð
skulun salt wesan, \hld\ sundigero manno,
bótjan iro balu-dádi, \hld\ þat sie an bętara þing,
folk far-fáhan endi for-látan \hld\ fíundes gi-werk,
diuvales ge-dádi, \hld\ endi sókjan iro drohtines ríki.
Só skulun gi mid iuwon lérun \hld\ liud-folk manag
węndjan aftar mínon willjon. \hld\ Ef iuwar þan a-wirðid hwi-lik,
far-látid þea léra, \hld\ þea he léstjan skal,
þan is im só þem salte, \hld\ þe man bi sées staðe
wído te-wirpit: \hld\ þan it te wihti ni dóg,
ak it firiho barn \hld\ fótun spurnat,
gumon an greote. \hld\ Só wirðid þem, þe þat godes word skal
mannum márjan: \hld\ ef he im þan látid is mód twehon,
þat hi ne willja mid hluttro hugi \hld\ te heven-ríkja
spanen mid is spráku \hld\ endi sęggjan spel godes,
ak wenkid þero wordo, \hld\ þan wirðid im waldand gram,
mahtig módag, \hld\ endi só samo manno barn;
wirðid allun þan \hld\ irmin-þiodun,
liudjun a-léðid, \hld\ ef is léra ni dugun.ʼ
So sprak he þó spáh-líko \hld\ endi sagda spel godes,
lérde þe landes ward \hld\ liudi síne
mid hluttru hugi. \hld\ Hęliðos stódun,
gumon umbi þana godes sunu \hld\ gerno swíðo,
weros an willjon: \hld\ was im þero wordo niud,
þáhtun endi þagodun, \hld\ gi-hórdun þero þiodo drohtin
sęggjan éu godes \hld\ eldi-barnun;
gi-hét im heven-ríki \hld\ endi te þem hęliðun sprak:
ʽók mag ik iu sęggjan, \hld\ ge-síðos mína,
wárun wordun, \hld\ þat gi þesoro weroldes nu forð
skulun lioht wesan \hld\ liudjo barnun,
fagar mid firihun \hld\ ovar folk manag,
wlitig endi wun-sam: \hld\ ni mugun iuwa werk mikil
bi-holan werðan, \hld\ mid hwi-liko gi sea hugi kúðjat:
þan mér þe þiu burg ni mag, \hld\ þiu an berge stáð,
hòh holm-klivu, \hld\ bi-holen werðen,
wrisi-lík gi-werk, \hld\ ni mugun iuwa word þan mér
an þesoro middil-gard \hld\ mannum werðen,
iuwa dádi bi-dernit. \hld\ Dót, só ik iu lériu:
látad iuwa lioht mikil \hld\ liudjun skínan,
manno barnun, \hld\ þat sie far-standan iuwan mód-sevon,
iuwa werk endi iuwan willjon, \hld\ endi þes waldand god
mit hluttro hugi, \hld\ himiliskan fader,
lovon an þesumu liohte, \hld\ þes he iu su-lika léra far-gaf.
Ni skal neoman lioht, þe it havad, \hld\ liudjun dernean,
te hardo be-hwelvean, \hld\ ak he it hòho skal
an sęli sęttjan, \hld\ þat þea ge-sehan mugin
alla ge-liko, \hld\ þea þar inna sind,
hęliðos an hallu. \hld\ Þan hald ni skulun gi iuwa hèlag word
an þesumu land-skępa \hld\ liudjun dernjen,
hęlið-kunnje farhelan, \hld\ ak ge it hòho skulun
brédean, þat gi-bod godes, \hld\ þat it allaro barno ge-hwi-lik,
ovar al þit land-skępi \hld\ liudi far-standan
endi só ge-frummjen, \hld\ só it an forn-dagun
tulgo wíse man \hld\ wordun ge-sprákun,
þan sie þana aldan éw \hld\ erlos heldun,
endi ók su-liku swíðor, \hld\ só ik iu nu sęggjan mag,
alloro gumono ge-hwi-lik \hld\ gode þionojan,
þan it þar an þem aldom \hld\ éwa ge-beode.
Ni wánjat gi þes mit wihtju, \hld\ þat ik bi þiu an þesa werold kwámi,
þat ik þana aldan éu \hld\ irrjen willje,
fellean undar þesumu folke \hld\ efþo þero fora-sagono
word wiðar-werpen, \hld\ þea hér só gi-wárea man
bar-líko ge-budun. \hld\ Ér skal béðju te-faran,
himil endi erðe, \hld\ þiu nu bi-hlidan standat,
ér þan þero wordo \hld\ wiht bi-líva
un-léstid an þesumu liohte, \hld\ þea sie þesum liudjun hér
wár-líko ge-budun. \hld\ Ni kwam ik an þesa werold te þiu,
þat ik feldi þero fora-sagono word, \hld\ ak ik siu fulljen skal,
ókjon endi nígean \hld\ eldi-barnum,
þesumu folke te frumu. \hld\ Þat was forn ge-skrivan
an þem aldon éo \hld\ —ge hórdun it oft sprekan
word-wíse man—: \hld\ só hwe só þat an þesoro weroldi gidót,
þat he áðrana \hld\ aldru bi-neote,
lívu bi-lósje, \hld\ þem skulun liudjo barn
dód a-déljan. \hld\ Þan willjo ik it iu diopor nu,
furður bi-fáhan: \hld\ só hwe só ina þurh fíund-skępi,
man wiðar óðrana \hld\ an is mód-sevon
bilgit an is breostun \hld\ —hwand sie alle ge-bróðar sint,
sálig folk godes, \hld\ sibbjon bi-tengja,
man mid mág-skępi—, \hld\ þan wirðit þoh hwe óðrumu an is móde só gram,
líbes weldi ina bi-lósjen, \hld\ of he mahti gi-léstjen só:
þan is he sán a-féhit \hld\ endi is þes ferahas skolo,
al su-likes ur-déljes \hld\ só þe óðar was,
þe þurh is hand-męgin \hld\ hóvdo bi-lósde
erl óðarna. \hld\ Ók is an þem éo ge-skrivan
wárun wordun, \hld\ só gi witon alle,
þan man is náhiston \hld\ niud-líko skal
minnjan an is móde, \hld\ wesen is mágun hold,
gadulingun gód, \hld\ wesen is geva mildi,
fráhon is friunda ge-hwane, \hld\ endi skal is fíund hatan,
wiðer-standen þem mid strídu \hld\ endi mid starku hugi,
węrjan wiðar wréðun. \hld\ Þan sęggjo ik iu te wáron nu,
fullíkur for þesumu folke, \hld\ þat gi iuwa fíund skulun
minnjon an iuwomu móde, \hld\ só samo só gi iuwa mágos dót,
an godes namon. \hld\ Dót im gódes filu,
tógjat im hluttran hugi, \hld\ holda trewa,
liof wiðar ira léðe. \hld\ Þat is lang-sam rád
manno só hwi-likumu, \hld\ só is mód te þiu
ge-flíhit wiðar is fíunde. \hld\ Þan mótun gi þea fruma égan,
þat gi mótun héten \hld\ heven-kuninges suni,
is blíði barn. \hld\ Ne mugun gi iu bętaran rád
ge-winnan an þesoro weroldi. \hld\ Þan sęggjo ik iu te wáron ók,
barno ge-hwi-likum, \hld\ þat gi ne mugun mid gi-bolgono hugi
iuwas gódes wiht \hld\ te godes húsun
waldande far-gevan, \hld\ þat it imu wirðig sí
te ant-fáhanne, \hld\ só lango só þu fíund-skępjes wiht,
wiðer óðran man \hld\ inwid hugis.
Ér skalt þu þi simbla ge-sónjen \hld\ wið þana sak-waldand,
ge-módi gi-mahlean: \hld\ síðor maht þu méðmos þína
te þem godes altere a-gevan: \hld\ þan sind sie þemu gódan werðe,
heven-kuninge. \hld\ Mér skulun gi aftar is huldi þionon,
godes willjon ful-gán, \hld\ þan óðra Judeon duon,
ef gi willjat égan \hld\ éwan ríki,
sin-líf sehan. \hld\ Ók skal ik iu sęggjan noh,
hwó it þar an þem aldon \hld\ éo ge-biudid,
þat énig erl óðres \hld\ idis ni bi-swíka,
wíf mid wammu. \hld\ Þan sęggjo ik iu te wáron ók,
þat þar man is siuni mugun \hld\ swíðo far-lédean
an mirki mén, \hld\ ef hi ina látid is mód spanen,
þat he be-ginna þero girnean, \hld\ þiu imu ge-gangan ni skal.
Þan haved he an imu selvon sán \hld\ sundja ge-warhta,
ge-hęftid an is hertan \hld\ hęlli-wíti.
Ef þan þana man is siun wili \hld\ etþa is swíðare hand
far-lédjen is liðo hwi-lik \hld\ an léðan weg,
þan is erlo ge-hwem \hld\ óðar bętara,
firiho barno, \hld\ þat he ina fram werpa
endi þana lið lósje \hld\ af is lík-hamon
endi ina áno kuma \hld\ up te himile,
þan he só mid allun \hld\ te þem inferne,
hwerve mid só hélun \hld\ an hęlli-grund.
Þan ménid þiu léf-héd, \hld\ þat énig liudjo ni skal
far-folgan is friunde, \hld\ ef he ina an firina spanit,
swás man an saka: \hld\ þan ne sí he imu eo só swíðo an sibbiun bilang,
ne iro mág-skępi só mikil, \hld\ ef he ina an morð spenit,
bédid balu-werko; \hld\ bętera is imu þan óðar,
þat he þana friund fan imu \hld\ fer far-werpa,
míðe þes máges \hld\ endi ni hębbja þar éniga minnja tó,
þat he móti éno \hld\ up ge-stígan
hó himil-ríki, \hld\ þan sie hęlli-ge-þwing,
bréd bal-wíti \hld\ béðja gi-sókjan,
uvil arvidi. \hld\ Ók is an þem éo ge-skrivan
wárun wordun, \hld\ só gi witun alle,
þat míðe mén-éðos \hld\ man-kunnjes ge-hwi-lik,
ni for-swęrje ina selvon, \hld\ hwand þat is sundje te mikil,
far-lédid liudi \hld\ an léðan weg.
Þan willjo ik iu eft sęggjan, \hld\ þan sán ni swęrja neoman
énigan éð-staf \hld\ eldi-barno,
ne bi himile þemu hòhon, \hld\ hwand þat is þes hérron stól,
ne bi erðu þar undar, \hld\ hwand þat is þes alo-waldon
fagar fót-skamel, \hld\ nek énig firiho barno
ne swęrja bi is selves hóvde, \hld\ hwand he ni mag þar ne swart ne hwít
énig hár ge-wirkjan, \hld\ bútan só it þe hèlago god,
ge-markode mahtig; \hld\ be-þiu skulun míðan filu
erlos éð-wordo. \hld\ Só hwe só it ofto dót,
só wirðid is simbla wirsa, \hld\ hwand he imu gi-wardon ni mag.
Biþiu skal ik iu nu te wárun \hld\ wordun gi-beodan,
þat gi neo ne swęrjen \hld\ swíðoron éðos,
méron met mannun, \hld\ bútan só ik iu mid mínun hér
swíðo wár-liko \hld\ wordun ge-biudu:
ef man hwemu saka sókja, \hld\ bisęggja þat wáre,
kweðe iá, gef it sí, \hld\ geha þes þar wár is,
kweðe nén, af it nis, \hld\ láta im ge-nóg an þiu;
só hwat só is mér ovar þat \hld\ man ge-frummjad,
só kumid it al fan uvile \hld\ eldi-barnun,
þat erl þurh un-trewa \hld\ óðres ni wili
wordo ge-lóvjan. \hld\ Þan sęggjo ik iu te wáron ók,
hwó it þar an þem aldon \hld\ éo ge-biudit:
só hwe só ógon ge-nimid \hld\ óðres mannes,
lósid af is lík-haman, \hld\ etþa is liðo hwi-likan,
þat he it eft mid is selves skal \hld\ sán ant-gelden
mid ge-líkun liðion. \hld\ Þan willjo ik iu lérjan nu,
þat gi só ni wrekan \hld\ wréða dádi,
ak þat gi þurh ód-módi \hld\ al ge-þologian
wítjes endi wammes, \hld\ só hwat só man iu an þesoro weroldi gedóe.
Dóe alloro erlo ge-hwi-lik \hld\ óðrom manne
frume endi ge-fóri, \hld\ só he willje, þat im firiho barn
gódes an-gegin dóen. \hld\ Þan wirðit im god mildi,
liudjo só hwi-likum, \hld\ só þat léstjen wili.
Érod gi arme man, \hld\ déljad iwan òd-welon
undar þero þurftigon þiodu; \hld\ ne rókjad, hweðar gi is énigan þank ant-fáhan
efþo lòn an þesoro léhneon weroldi, \hld\ ak huggjat te iuwomu leovon hérran
þero gevono te gelde, \hld\ þat sie iu god lòno,
mahtig mund-boro, \hld\ só hwat só gi is þurh is minnes gidót.
Ef þu þan gevogean wili \hld\ gódun mannun
fagare feho-skattos, \hld\ þar þu eft frumono hugis
mér ant-fáhan, \hld\ te hwí havas þu þes éniga méda fon gode
etþa lòn an þemu is liohte? \hld\ hwand þat is léhni feho.
Só is þes alles ge-hwat, \hld\ þe þu óðrun ge-duos
liudjon te leove, \hld\ þar þu hugis eft ge-lík neman
þero wordo endi þero werko: \hld\ te hwí wét þi þes úsa waldand þank,
þes þu þín só bi-filhis \hld\ endi ant-fáhis eft þan þu wili?
iuwan óð-welon \hld\ gevan gi þem armun mannun,
þe ina iu an þesoro weroldi ne lònon \hld\ endi rómot te iuwes waldandes ríkja.
Te hlúd ni dó þu it, \hld\ þan þu mid þínun handun bi-felhas
þína alamosna þemu armon manne, \hld\ ak dó im þurh ód-módjen
gerno þurh godes þank: \hld\ þan móst þu eft geld niman,
swíðo liof-lík lòn, \hld\ þar þu is lango bi-þarft,
fagaroro frumono. \hld\ Só hwat só þu is só þurh ferhtan hugi
darno ge-déljas, \hld\ —so is úsumu drohtine werð—
ne galpo þu far þínun gevun te swíðo, \hld\ noh énig gumono ne skal,
þat siu im þurh ídale hróm \hld\ eft ni werðe
léð-líko far-loren. \hld\ Þanna þu skalt lòn nemen
fora godes ógun \hld\ gódero werko.
Ók skal ik iu ge-beodan, \hld\ þan gi willjad te bedu hnígan
endi willjad te iuwomu hérron \hld\ helpono biddjan,
þat he iu a-láte \hld\ léðes þinges,
þero sakono endi þero sundjono, \hld\ þea gi iu selvon hír
wréða ge-wirkjad, \hld\ þat gi it þan for óðrumu werode ni duad:
ni márjad it far menigi, \hld\ þat iu þes man ni lovon,
ni diurjan þero dádjo, \hld\ þat gi iuwes drohtines gi-bed
þurh þat ídala hróm \hld\ al ne far-leosan.
Ak þan gi willjan te iuwomo hérron \hld\ helpono biddjan,
þiggjan þeo-líko, \hld\ —þes iu is þarf mikil—
þat iu sigi-drohtin \hld\ sundjono tómja,
þan dót gi þat só darno: \hld\ þoh wét it iuwe drohtin self
hèlag an himile, \hld\ hwand imu nis bi-holan neo-wiht
ne wordo ne werko. \hld\ He látid it þan al ge-werðan só,
só gi ina þan biddjad, \hld\ þan gi te þero bedo hnígad
mid hluttru hugi.ʼ \hld\ Hęliðos stódun,
gumon umbi þana godes sunu \hld\ gerno swíðo,
weros an willjon: \hld\ was im þero wordo niud,
þáhtun endi þagodun, \hld\ was im þarf mikil,
þat sie þat eft ge-hogdin, \hld\ þat im þat hèlaga barn
an þana forman sið \hld\ filu mid wordun
torhtes ge-talde. \hld\ Þó sprak im eft én þero twelivjo an-gegin,
glauworo gumono, \hld\ te þem godes barne:
ʽHérro þe gódoʼ, \hld\ kwað he, ʽús is þínoro huldi þarf,
te gi-wirkenne þínna willjon, \hld\ endi ók þínoro wordo só self,
allaro barno bętst, \hld\ þat þu ús bedon léres,
jungoron þíne, \hld\ só Johannes duot,
diur-lík dóperi, \hld\ dago ge-hwi-likas
is werod mid wordun, \hld\ hwí sie waldand skulun,
gódan grótjan. \hld\ Dó þína jungorun só self:
ge-rihti ús þat ge-rúni.ʼ \hld\ Þó habda eft þe ríkjo garu
sán aftar þiu, \hld\ sunu drohtines,
gód word an-gegin: \hld\ ʽÞan gi god willjanʼ, kwað he,
ʽweros mid iuwon wordun \hld\ waldand grótjan,
allaro kuningo kraftigostan, \hld\ þan kweðad gi, só ik iu lériu:
Fadar úsa \hld\ firiho barno,
þu bist an þem hòhon \hld\ himila ríkja,
ge-wíhid sí þín namo \hld\ wordo ge-hwi-liko.
Kuma þín \hld\ kraftag ríki.
Werða þín willjo \hld\ ovar þesa werold alla,
só sama an erðo, \hld\ só þar uppa ist
an þem hòhon \hld\ himilo ríkja.
Gef ús dago ge-hwi-likes rád, \hld\ drohtin þe gódo,
þína hèlaga helpa, \hld\ endi a-lát ús, hevenes ward,
managoro mén-skuldjo, \hld\ al só we óðrum mannum dóan.
Ne lát ús far-lédjan \hld\ léða wihti
só forð an iro willjon, \hld\ só wi wirðige sind,
ak help ús wiðar allun \hld\ uvilon dádjun.
Só skulun gi biddjan, \hld\ þan gi te bede hnígad
weros mid iuwom wordun, \hld\ þat iu waldand god
léðes a-láte \hld\ an leut-kunnja.
Ef gi þan willjad a-látan \hld\ liudjo ge-hwi-likun
þero sakono endi þero sundjono, \hld\ þe sie wið iu selvon hír
wréða ge-wirkjat, \hld\ þan a-látid iu waldand god,
fadar ala-mahtig \hld\ firin-werk mikil,
managoro mén-skuldjo. \hld\ Ef iu þan wirðid iuwa mód te stark,
þat gi ne wileat óðrun \hld\ erlun a-látan,
weron wam-dádi, \hld\ þan ne wil iu ók waldand god
grim-werk far-gevan, \hld\ ak gi skulun is geld niman,
swíðo léð-lik lòn \hld\ te languru hwílu,
alles þes un-rehtes, \hld\ þes gi óðrum hír
gi-léstjad an þesumu liohte \hld\ endi þan wið liudjo barn
þea saka ni gi-sónjad, \hld\ ér gi an þana síð faran,
weros fon þesoro weroldi. \hld\ Ok skal ik iu te wárun sęggjan,
hwó gi léstjan skulun \hld\ léra mína:
þan gi iuwa fastonnja \hld\ frummjan willjan,
minson iuwa mén-dádi, \hld\ þan ni duad gi þat te managom kúð,
ak míðad is far óðrum mannun: \hld\ þoh wét mahtig god,
waldand iuwan willjan, \hld\ þoh iu werod óðar,
liudjo barn ne lovon. \hld\ He gildid is iu lòn aftar þiu,
iuwa hèlag fadar \hld\ an himil-ríkja,
þes ge im mid su-likum ód-módja, \hld\ erlos þeonod,
só ferhtlíko undar þesumu folke. \hld\ Ne willjat feho winnan
erlos an un-reht, \hld\ ak wirkjad up te gode
man aftar médu: \hld\ þat is méra þing,
þan man hír an erðu \hld\ ódag libbja,
werold-skattes ge-wono. \hld\ Ef gi willjad mínun wordun hórjan,
þan ne samnod gi hír sink mikil \hld\ silovres ne goldes
an þesoro middil-gard, \hld\ mèðom-hordes,
hwand it rotat hír an roste, \hld\ endi ręgin-þeovos far-stelad,
wurmi a-wardjad, \hld\ wirðid þat gi-wádi far-slitan,
ti-gangid þe gold-welo. \hld\ Léstjad iuwa gódon werk,
samnod iu an himile \hld\ hord þat méra,
fagara feho-skattos: \hld\ þat ni mag iu énig fíund beniman,
ne-wiht an-węndjan, \hld\ hwand þe welo standid
garu iu te-gegnes, \hld\ só hwat só gi gódes þarod,
an þat himil-ríki \hld\ hordes ge-samnod,
hęliðos þurh iuwa hand-geva, \hld\ endi hębbjad þarod iuwan hugi fasto;
hwand þar ist alloro manno gi-hwes \hld\ mód-ge-þáhti,
hugi endi herta, \hld\ þar is hord ligid,
sink ge-samnod. \hld\ Nis eo só sálig man,
þat mugi an þesoro brédon werold \hld\ béðju ant-hengean,
ge þat hi an þesoro erðo \hld\ ódag libbja,
an allun werold-lustun wesa, \hld\ ge þoh waldand gode
te þanke ge-þeono: \hld\ ak he skal alloro þingo gi-hwes
simbla óðar-hweðar \hld\ én far-látan
etþo lusta þes lík-hamon \hld\ etþo líf éwig.
Be-þiu ni gornot gi umbi iuwa ge-garuwi, \hld\ ak huggjad te gode fasto,
ne mornont an iuwomu móde, \hld\ hwat gi eft an morgan skulin
etan efþo drinkan \hld\ etþo an hębbjan
weros te ge-wédja: \hld\ it wét al waldand god,
hwes þea bi-þurvun, \hld\ þea im hír þionod wel,
folgod iro fróhan willjon. \hld\ Hwat, gi þat bi þesun fuglun mugun
wár-líko undar-witan, \hld\ þea hír an þesoro weroldi sint,
farad an feðar-hamun: \hld\ sie ni kunnun énig feho winnan,
þoh givid im drohtin god \hld\ dago ge-hwi-likes
helpa wiðar hungre. \hld\ Ók mugun gi an iuwom hugi markon,
weros umbi iuwa ge-wádi, \hld\ hwó þie wurti sint
fagoro ge-fratohot, \hld\ þea hír an felde stád,
berht-líko ge-blóid: \hld\ ne mahta þe burges ward,
Salomon þe suning, \hld\ þe habda sink mikil,
mèðom-hordas mést, \hld\ þero þe énig man éhti,
welono ge-wunnan \hld\ endi allaro ge-wádjo kust, —
þoh ni mohte he an is líve, \hld\ þoh he habdi alles þeses landes ge-wald,
a-winnan su-lik ge-wádi, \hld\ só þiu wurt havad,
þiu hír an felde stád \hld\ fagoro ge-gariwit,
lilli mid só liof-líku blómon: \hld\ ina wádit þe landes waldand
hér fan hevenes wange. \hld\ Mér is im þoh umbi þit hęliðo kunni,
liudi sint im liovoron mikilu, \hld\ þea he im an þesumu lande ge-warhte,
waldand an willjon sínan. \hld\ Be-þiu ne þurvon gi umbi iuwa ge-wádi sorgon,
ne gornot gi umbi iuwa ge-gariwi te swíðo: \hld\ god wili is alles rádan,
helpan fan hevenes wange, \hld\ ef gi willjad aftar is huldi þeonon.
Gerot gi simbla érist þes godes ríkjas, \hld\ endi þan duat aftar þem is gódun werkun,
rómod gi rehtoro þingo: \hld\ þan wili iu þe ríkjo drohtin
gevon mid alloro gódu ge-hwi-liku, \hld\ ef gi im þus ful-gangan willjad,
só ik iu te wárun hír \hld\ wordun seggjo.
Ne skulun gi énigumu manne \hld\ un-rehtes wiht,
dervjes a-déljan, \hld\ hwand þe dóm eft kumid
ovar þana selvon man, \hld\ þar it im te sorgon skal,
werðan þem te wítea, \hld\ þe hír mid is wordun ge-sprikid
un-reht óðrum. \hld\ Neo þat iuwar énig ne dua
gumono an þesom gardon \hld\ geldes etþo kópes,
þat hi un-reht gi-met \hld\ óðrumu manne
ménful mako, \hld\ hwand it simbla mótean skal
erlo ge-hwi-likomu, \hld\ su-lik só he it óðrumu gedód,
só kumid it im eft te-gegnes, \hld\ þar he gerno ne wili
ge-sehan is sundjon. \hld\ Ók skal ik iu sęggjan noh,
hwar gi iu wardon skulun \hld\ wíteo mésta,
mén-werk manag: \hld\ te hwí skalt þu énigan man besprekan,
bróðar þínan, \hld\ þat þu undar is bráhon ge-sehas
halm an is ógon, \hld\ endi ge-huggjan ni wili
þana swáran balkon, \hld\ þe þu an þínoro siuni havas,
hard trio endi hevig. \hld\ Lát þi þat an þínan hugi fallan,
hwó þu þana érist a-lósjas: \hld\ þan skínid þi lioht beforan,
ógun werðad þi ge-oponot; \hld\ þan maht þu aftar þiu
swáses mannes gesiun \hld\ síðor ge-bótjan,
ge-hélean an is hóvde. \hld\ Só mag þat an is hugi méra
an þesoro middil-gard \hld\ manno ge-hwi-likumu,
wesan an þesoro weroldi, \hld\ þat hi hír wammas ge-duot,
þan hi ahtogea \hld\ óðres mannes
saka endi sundja, \hld\ endi havad im selvo mér
firin-werko ge-frumid. \hld\ Ef he wili is fruma léstjan,
þan skal hi ina selvon ér \hld\ sundjono a-tómjan,
léð-werko lóson: \hld\ síðor mag hi mid is lérun werðan
hęliðun te helpu, \hld\ síðor hi ina hluttran wét,
sundjono sikoran. \hld\ Ne skulun gi swínum teforan
iuwa mere-gríton makon \hld\ etþo méðmo ge-striuni,
hèlag hals-męni, \hld\ hwand siu it an horu spurnat,
sulwiad an sande: \hld\ ne witun súvreas ge-skéð,
fagaroro fratoho. \hld\ Sulik sint hír folk manag,
þe iuwa hèlag word \hld\ hórjan ne willjad,
ful-gangan godes lérun: \hld\ ne witun gódes ge-skéð,
ak sind im lári word \hld\ leovoron mikilu,
umbi-þarvi þing, \hld\ þanna þeot-godes
werk endi willjo. \hld\ Ne sind sie wirðige þan,
þat sie ge-hórjan iuwa hèlag word, \hld\ ef sie is ne willjad an iro hugi þęnkjan,
ne línon ne léstjan. \hld\ Þem ni sęggjan gi iuworo léron wiht,
þat gi þea spráka godes \hld\ endi spel managu
ne far-leosan an þem liudjun, \hld\ þea þar ne willjan gi-lóvjan tó,
wároro wordo. \hld\ Ók skulun gi iu wardon filu
listjun undar þesun liudjun, \hld\ þar gi aftar þesumu lande farad,
þat iu þea luggjon ne mugin \hld\ léron be-swíkan
ni mid wordun ni mid werkun. \hld\ Sie kumad an su-likom ge-wádjon te iu,
fagoron fratohon: \hld\ þoh hębbjad sie féknan hugi:
þea mugun gi sán ant-kęnnjan, \hld\ só gi sie kuman ge-sehad:
sie sprekad wíslík word, \hld\ þoh iro werk ne dugin,
þero þegno ge-þáhti. \hld\ Hwand gi witun, þat eo an þorniun ne skulun
wín-beri wesan \hld\ efþa welon eowiht,
fagororo fruhteo, \hld\ nek ók fígun ne lesad
hęliðos an hiopon. \hld\ Þat mugun gi undar-huggjan wel,
þat eo þe uvilo bóm, \hld\ þar he an erðu stád,
góden wastum ne givid, \hld\ nek it ók god ni ge-skóp,
þat þe gódo bóm \hld\ gumono barnun
bári bittres wiht, \hld\ ak kumid fan alloro bámo ge-hwi-likumu
su-lik wastom te þesero weroldi, \hld\ só im fan is wurtjon ge-dregid,
etþa berht etþa bittar. \hld\ Þat ménid þoh breost-hugi,
managoro mód-sevon \hld\ manno kunnjes,
hwó alloro erlo ge-hwi-lik \hld\ ógit selvo,
meldod mid is múðu, \hld\ hwi-likan he mód havad,
hugi umbi is herte: \hld\ þes ni mag he farhelan eowiht,
ak kumad fan þem uvilan man \hld\ inwid-rádos,
bittara balu-spráka, \hld\ su-lik só hi an is breostun havad
ge-hęftid umbi is herte: \hld\ simbla is hugi kúðid,
is willjon mid is wordun, \hld\ endi farad is werk aftar þiu.
Só kumad fan þemu gódan manne \hld\ glau and-wordi,
wíslík fan is ge-wittea, \hld\ þat hi simbla mid is wordu ge-sprikid,
man mid is míðu su-lik, \hld\ só he an is móde havad
hord umbi is herte. \hld\ Þanan kumad þea hèlagan léra,
swíðo wun-sam word, \hld\ endi skulun is werk aftar þiu
þeodu ge-þíhan, \hld\ þegnun managun
werðan te willjon, \hld\ al só it waldand self
gódun mannun far-givid, \hld\ god alo-mahtig,
himilisk hérro, \hld\ hwand sie áno is helpa ni mugun
ne mid wordun ne mid werkun \hld\ wiht aþengean
gódes an þesun gardun. \hld\ Be-þiu skulun gumono barn
an is énes kraft \hld\ alle gi-lóvjan.
Ók skal ik iu wísjan, \hld\ hwó hír wegos twéna
liggjad an þesumu liohte, \hld\ þea farad liudjo barn,
al irmin-þiod. \hld\ Þero is óðar sán
wíd stráta endi bréd, \hld\ —farid sie werodes filu,
man-kunnjes manag, \hld\ hwand sie þarod iro mód spenit,
weroldlusta weros— \hld\ þiu an þea wirson hand
liudi lédid, \hld\ þar sie te far-lora werðad,
hęliðos an hęllju, \hld\ þar is hét endi swart,
egis-lík an innan: \hld\ óði ist þarod te faranne
eldi-barnun, \hld\ þoh it im at þemu endie ni dugi.
Þan ligid eft óðar \hld\ engira mikilu
weg an þesoro weroldi, \hld\ ferid ina werodes lút,
fáho folk-skępi: \hld\ ni willjad ina firiho barn
gerno gangan, \hld\ þoh he te godes ríkja,
an þat éwiga líf, \hld\ erlos lédja.
Þan nimad gi iu þana engean: \hld\ þoh he só óði ne sí
firihon te faranne, \hld\ þoh skal hi te frumu werðan
só hwemu só ina þurh-gengid, \hld\ só skal is geld niman,
swíðo lang-sam lòn \hld\ endi líf éwig,
diur-líkan dròm. \hld\ Eo gi þes drohtin skulun,
waldand biddjen, \hld\ þat gi þana weg mótin
fan foran ant-fáhan \hld\ endi forð þurh gigangan
an þat godes ríki. \hld\ He ist garu simbla
wiðar þiu te gevanne, \hld\ þe man ina gerno bidid,
fergot firiho barn. \hld\ Sókjad fadar iuwan
up te þemu éwinom ríkja: \hld\ þan mótun gi ina aftar þiu
te iuworu frumu fíðan. \hld\ Kúðead iuwa fard þarod
at iuwas drohtines durun: \hld\ þan werðad iu andón aftar þiu,
himil-portun ant-hlidan, \hld\ þat gi an þat hèlage lioht,
an þat godes ríki \hld\ gangan mótun,
sin-líf sehan. \hld\ Ók skal ik iu sęggjan noh
far þesumu werode allun \hld\ wár-lík biliði,
þat alloro liudjo só hwi-lik, \hld\ só þesa mína léra wili
ge-haldan an is herton \hld\ endi wil iro an is hugi aþęnkjan,
léstjan sea an þesumu lande, \hld\ þe gi-líko duot
wísumu manne, \hld\ þe gi-wit havad,
horska hugi-skęfti, \hld\ endi hús-stędi kiusid
an fastoro foldun \hld\ endi an felisa uppan
wégos wirkid, \hld\ þar im wind ni mag,
ne wág ne watares stròm \hld\ wihtiu ge-tiunean,
ak mag im þar wið un-gi-widereon \hld\ allun standan
an þemu felise uppan, \hld\ hwand it só fasto warð
gi-stellit an þemu sténe: \hld\ anthavad it þiu stędi niðana,
wreðid wiðar winde, \hld\ þat it wíkan ni mag.
Só duot eft manno só hwi-lik, \hld\ só þesun mínun ni wili
lérun hórjen ne þero \hld\ léstjen wiht,
só duot þe un-wíson \hld\ erla ge-líko,
un-ge-wittigon were, \hld\ þe im be watares staðe
an sande wili \hld\ sęli-hús wirkjan,
þar it westrani wind \hld\ endi wágo stròm,
sées úðjon te-sláad; \hld\ ne mag im sand endi greot
ge-wreðjen wið þemu winde, \hld\ ak wirðid te-worpan þan,
te-fallen an þemu flóde, \hld\ hwand it an fastoro nis
erðu ge-timbrod. \hld\ Só skal allaro erlo ge-hwes
werk ge-þíhan wiðar þiu, \hld\ þe hi þius mín word frumid,
haldid hèlag ge-bod.ʼ \hld\ Þó bi-gunnun an iro hugi wundron
męgin-folk mikil: \hld\ ge-hórdun mahtiges godes
liof-líka léra; \hld\ ne wárun an þemu lande ge-wuno,
þat sie eo fan su-likun ér \hld\ sęggjan ge-hórdin
wordun etþo werkun. \hld\ Far-stódun wíse man,
þat he só lérde, \hld\ liudjo drohtin,
wárun wordun, \hld\ só he ge-wald habde,
allun þem un-ge-líko, \hld\ þe þar an ér-dagun
undar þem liud-skępja \hld\ lérjon wárun
akoran undar þemu kunnje: \hld\ ne habdun þiu Kristes word
gemakon mid mannun, \hld\ þe he far þero menigi sprak,
ge-bód uppan þemu berge. \hld\ He im þó béðju be-falh
ge te seggennea \hld\ sínom wordun,
hwó man himil-ríki \hld\ ge-halon skoldi,
wíd-brédan welan, \hld\ gia he im ge-wald far-gaf,
þat sie móstin hélean \hld\ halte endi blinde,
liudjo léf-hédi, \hld\ legar-będ manag,
swára suhti, \hld\ giak he im selvo ge-bód,
þat sie at énigumu manne \hld\ méde ne námin,
diurje méðmos: \hld\ ʽge-huggjad giʼ, kwað he, —ʽhwand iu is þiu dád kuman,
þat ge-wit endi þe wís-dóm, \hld\ endi iu þea ge-wald far-givid
alloro firiho fadar, \hld\ só gi sie ni þurvun mid énigo feho kópon,
médean mid énigun méðmun,— \hld\ só wesat gi iro mannun forð
an iuwon hugi-skęftjun \hld\ helpono mildja,
lérjad gi liudjo barn \hld\ lang-samna rád,
fruma forð-wardes; \hld\ firin-werk lahad,
swára sundjon. \hld\ Ne látad iu silovar nek gold
wihti þes wirðig, \hld\ þat it eo an iuwa ge-wald kuma,
fagara feho-skattos: \hld\ it ni mag iu te énigoro frumu hwergin,
werðan te énigumu willjon. \hld\ Ne skulun gi ge-wádjas þan mér
erlos égan, \hld\ bútan só gi þan an hębbjan,
gumon te garewea, \hld\ þan gi gangan skulun
an þat gi-mang innan. \hld\ Neo gi umbi iuwan meti ni sorgot,
leng umbi iuwa lífnare, \hld\ hwand þene lérjand skulun
fódjan þat folk-skępi: \hld\ þes sint þea fruma werða,
leov-líkes lònes, \hld\ þe hi þem liudjun sagad.
wirðig is þe wurhtjo, \hld\ þat man ina wel fódja,
þana man mid mósu, \hld\ þe só managoro skal
seola bi-sorgan \hld\ endi an þana síð spanen,
géstos an godes wang. \hld\ Þat is grótara þing,
þat man bi-sorgon skal \hld\ seolun managa,
hwó man þea ge-halde \hld\ te heven-ríkja,
þan man þene lík-hamon \hld\ liudi-barno
mósu bi-morna. \hld\ Be-þiu man skulun
haldan þene hold-líko, \hld\ þe im te heven-ríkja
þene weg wísit \hld\ endi sie wam-skaðun,
feondun wit-fáhit \hld\ endi firin-werk lahid,
swára sundjon. \hld\ Nu ik iu sęndjan skal
aftar þesumu land-skępje \hld\ só lamb undar wulvos:
só skulun gi undar iuwa fíund faren, \hld\ undar filu þeodo,
undar mislíke man. \hld\ Hębbjad iuwan mód wiðar þem
só glawan te-gegnes, \hld\ só samo só þe gelwo wurm,
nádra þiu féha, \hld\ þar siu iro níð-skępjes,
witodes wánit, \hld\ þat man iu undar þemu werode ne mugi
be-swíkan an þemu síðe. \hld\ Far þiu gi sorgon skulun,
þat iu þea man ni mugin \hld\ mód-ge-þáhti,
willjan a-wardjen. \hld\ Wesat iu so wara wiðar þiu,
wið iro fékneon dádjun, \hld\ só man wiðar fíundun skal.
Þan wesat gi eft an iuwon dádjun \hld\ dúvon ge-líka,
hębbjad wið erlo ge-hwene \hld\ én-faldan hugi,
mildjan mód-sevon, \hld\ þat þar man negén
þurh iuwa dádi \hld\ be-drogan ne werðe,
be-swikan þurh iuwa sundja. \hld\ Nu skulun gi an þana síð faran,
an þat árundi: \hld\ þar skulun gi arvidjes só filu
ge-þolon undar þeru þiod \hld\ endi ge-þwing só samo
manag endi mis-lík, \hld\ hwand gi an mínumu namon
þea liudi lérjat. \hld\ Be-þiu skulun gi þar léðes filu
fora werold-kuningun, \hld\ wíteas ant-fáhan.
Oft skulun gi þar for ríkja \hld\ þurh þius mín rehtun word
ge-bundane standen \hld\ endi béðju ge-þologean,
ge hosk ge harm-kwidi: \hld\ umbi þat ne látad gi iuwan hugi twíflon,
sevon swíkandjan: \hld\ gi ni þurvun an énigun sorgun wesan
an iuwomu hugi hwergin, \hld\ þan man iu for þea héri forð
an þene gast-sęli \hld\ gangan hétid,
hwat gi im þan te-gegnes skulin \hld\ gódoro wordo,
spáh-líkoro ge-sprekan, \hld\ hwand iu þiu spód kumid,
helpe fon himile, \hld\ endi sprikid þe hélogo gést,
mahtig fon iuwomu munde. \hld\ Be-þiu ne and-rádad gi iu þero manno níð
ne forhteat iro fíund-skępi: \hld\ þoh sie hębbjan iuwas ferahes ge-wald,
þat sie mugin þene lík-hamon \hld\ lívu beneotan,
a-slahan mid swerde, \hld\ þoh sie þeru seolun ne mugun
wiht a-wardean. \hld\ Antdrádad iu waldand god,
forhtead fader iuwan, \hld\ frummjad gerno
is ge-bod-skępi, \hld\ hwand hi havad béðjes gi-wald,
liudjo líves \hld\ endi ók iro lík-hamon
gek þero seolon só self: \hld\ ef gi iuwa an þem síðe þarod
far-liosat þurh þesa léra, \hld\ þan mótun gi sie eft an þemu liohte godes
beforan fíðan, \hld\ hwand sie fader iuwa,
haldid hèlag god \hld\ an himil-ríkja.
Ne kumat þea alle te himile, \hld\ þea þe hír hrópat te mi
manno te mundburd. \hld\ Managa sind þero,
þea willjad alloro dago ge-hwi-likes \hld\ te drohtine hnígan,
hrópad þar te helpu \hld\ endi huggjad an óðar,
wirkjad wam-dádi: \hld\ ne sind im þan þiu word fruma,
ak þea mótun hwervan \hld\ an þat himiles lioht,
gangan an þat godes ríki, \hld\ þea þes gerne sint,
þat sie hír ge-frummjen \hld\ fader ala-waldan
werk endi willjon. \hld\ Þea ni þurvun mid wordun só fílu
hrópan te helpu, \hld\ hwanda þe hélogo god
wét alloro manno ge-hwes \hld\ mód-ge-þáhti,
word endi willjon, \hld\ endi gildid im is werko lòn.
Be-þiu skulun gi sorgon, \hld\ þan gi an þene síð farad,
hwó gi þat árundi \hld\ ti ęndja be-brengen.
Þan gi líðan skulun \hld\ aftar þesumu land-skępja,
wído aftar þesoro weroldi, \hld\ al só iu wegos lédjad,
bréd stráta te burg, \hld\ simbla sókjad gi iu þene bętston sán
man undar þeru menegi \hld\ endi kúðjad imu iuwan móð-sevon
wárun wordun. \hld\ Ef sie þan þes wirðige sint,
þat sie iuwa gódun werk \hld\ gerno ge-léstjen
mid hluttru hugi, \hld\ þan gi an þemu húse mid im
wonod an willjon \hld\ endi im wel lònod,
geldad im mid gódu \hld\ endi sie te gode selvon
wordun ge-wíhad \hld\ endi sęggjad im wissan friðu,
hèlaga helpa \hld\ heven-kuninges.
Ef sie þan só sáliga \hld\ þurh iro selvoro dád
werðan ni mótun, \hld\ þat sie iuwa werk frummjen,
léstjen iuwa léra, \hld\ þan gi fan þem liudjun sán,
farad fan þemu folke, \hld\ —þe iuwa friðu hwirvid
eft an iuworo selvoro síð,— \hld\ endi látad sie mid sundjun forð,
mid balu-werkun búan \hld\ endi sókjad iu burg óðra,
mikil man-werod, \hld\ endi ne látad þes melmes wiht
folgan an iuwom fótun, \hld\ þanan þe man iu ant-fáhan ne wili,
ak skuddjat it fan iuwon skóhun, \hld\ þat it im eft te skamu werðe,
þemu werode te ge-wit-skępje, \hld\ þat iro willjo ne dóg.
Þan sęggjo ik iu te wárun, \hld\ só hwan só þius werold endjad
endi þe márjo dag \hld\ ovar man farid,
þat þan Sodomo-burg, \hld\ þiu hír þurh sundjon warð
an af-grundi \hld\ éldes kraftu,
fiuru bi-fallen, \hld\ þat þiu þan havad friðu méran,
mildiran mund-burd, \hld\ þan þea man égin,
þe iu hír wiðar-werpat \hld\ endi ne willjad iuwa word frummjen.
Só hwe só iu þan ant-fáhit \hld\ þurh ferhtan hugi,
þurh mildjan mód, \hld\ só havad mínan forð
willjon ge-warhten \hld\ endi ók waldand god,
ant-fangan fader iuwan, \hld\ firiho drohtin,
ríkjan rád-gevon, \hld\ þene þe al reht bikan.
wét waldand self, \hld\ endi willjan lònot
gumono ge-hwi-likumu, \hld\ só hwat só hi hír gódes geduot,
þoh hi þurh minnja godes \hld\ manno hwi-likumu
willjandi far-geve \hld\ watares drinkan,
þat hi þurftigumu manne \hld\ þurst ge-hélje,
kaldes brunnan. \hld\ Þesa kwidi werðad wára,
þat eo ne bi-lívid, \hld\ ne hi þes lòn skuli,
fora godes ógun \hld\ geld ant-fáhan,
méda manag-falde, \hld\ só hwat só hi is þurh mína minnja geduot.
Só hwe só mín þan far-lógnid \hld\ liudi-barno,
hęliðo for þesoro hęrju, \hld\ só dóm ik is an himile só self
þar uppe far þem alo-waldan fader \hld\ endi for allumu is ęngilo krafte,
far þeru mikilon menigi. \hld\ Só hwi-lik só þan eft manno barno
an þesoro weroldi ne wili \hld\ wordun míðan,
ak gihit far gum-skępi, \hld\ þat he mín jungoro sí,
þene willju ek eft ógjan \hld\ far ógun godes,
fora alloro firiho fader, \hld\ þar folk manag
for þene alo-waldon \hld\ alla gangad
reðinon wið þene ríkjon. \hld\ Þar willju ik imu an reht wesan
mildi mund-boro, \hld\ só hwemu só mínun hír
wordun hórid \hld\ endi þiu werk frumid,
þea ik hír an þesumu berge uppan \hld\ ge-boden hębbju.ʼ
Habda þó te wárun \hld\ waldandes sunu
ge-lérid þea liudi, \hld\ hwó sie lof gode
wirkjan skoldin. \hld\ Þó lét hi þat werod þanan
an alloro halva ge-hwi-lika, \hld\ hęri-skępi manno
síðon te selðon. \hld\ Habdun selves word,
ge-hórid heven-kuninges \hld\ hèlaga léra,
só eo te weroldi sint \hld\ wordo endi dádjo,
man-kunnjes manag \hld\ ovar þesan middil-gard
sprákono þiu spáhiron, \hld\ só hwe só þiu spel ge-frang,
þea þar an þemu berge ge-sprak \hld\ barno ríkjast.
Ge-wét imu þó umbi þrea naht aftar þiu \hld\ þesoro þiodo drohtin
an Galileo land, \hld\ þar he te énum gómum warð,
ge-bedan þat barn godes: \hld\ þar skolda man éna brúd gevan,
muna-líka magað. \hld\ Þar Maria was,
mid iro suni selvo, \hld\ sálig þiorna,
mahtiges móder. \hld\ Managoro drohtin
geng imu þó mid is jungoron, \hld\ godes égan barn,
an þat hòha hús, \hld\ þar þe hęri drank,
þea Judeon an þemu gast-sęli: \hld\ he im ók at þem gómun was,
giak hi þar ge-kúðde, \hld\ þat hi habda kraft godes,
helpa fan himil-fader, \hld\ hèlagna gést,
waldandes wís-dóm. \hld\ Werod blíðode,
wárun þar an luston \hld\ liudi at-samne,
gumon glad-módje. \hld\ Gengun ambaht-man,
skęnkjon mid skálun, \hld\ drógun skírjane wín
mid orkun endi mid alo-fatun; \hld\ was þar erlo dròm
fagar an flęttja, \hld\ þó þar folk undar im
an þem bęnkjon só bętst \hld\ blíðsea af-hóvun,
wárun þar an wunnjun. \hld\ Þó im þes wínes brast,
þem liudjun þes líðes: \hld\ is ni was far-lévid wiht
hwergin an þemu húse, \hld\ þat for þene hęri forð
skęnkjon drógin, \hld\ ak þiu skapu wárun
líðes a-lárid. \hld\ Þó ni was lang te þiu,
þat it sán ant-funda \hld\ frío skónjosta,
Kristes móder: \hld\ geng wið iro kind sprekan,
wið iro sunu selvon, \hld\ sagda im mid wordun,
þat þea werdos þó mér \hld\ wínes ne habdun
þem gęstjun te gómun. \hld\ Siu þó gerno bad,
þat is þe hélogo Krist \hld\ helpa ge-riedi
þemu werode te willjon. \hld\ Þó habda eft is word garu
mahtig barn godes \hld\ endi wið is móder sprak:
ʽhwat ist mi endi þiʼ, \hld\ kwað he, ʽumbi þesoro manno lið,
umbi þeses werodes wín? \hld\ Te hwí sprikis þu þes, wíf, só filu,
manos mi far þesoro menigi? \hld\ Ne sint mína noh
tídi kumana.ʼ \hld\ Þan þoh gi-trúoda siu wel
an iro hugi-skęftjun, \hld\ hèlag þiorne,
þat is aftar þem wordun \hld\ waldandes barn,
héljandoro bętst \hld\ helpan weldi.
Hét þó þea ambaht-man \hld\ idiso skónjost,
skęnkjon endi skap-wardos, \hld\ þea þar skoldun þero skolu þionon,
þat sie þes ne word ne werk \hld\ wiht ne far-létin,
þes sie þe hélogo Krist \hld\ hétan weldi
léstjan far þem liudjun. \hld\ Lárea stódun þar
stén-fatu sehsi. \hld\ Þó só stillo ge-bód
mahtig barn godes, \hld\ só it þar manno filu
ne wissa te wárun, \hld\ hwó he it mid is wordu ge-sprak;
he hét þea skęnkjon \hld\ þó skírjas watares
þiu fatu fulljen, \hld\ endi hi þar mid is fingrun þó,
segnade selvo \hld\ sínun handun,
warhte it te wíne \hld\ endi hét is an én wégi hlaðen,
skęppjen mid énoro skálon, \hld\ endi þó te þem skęnkjon sprak,
hét is þero gęstjo, \hld\ þe at þem gómun was
þemu héroston \hld\ an hand gevan,
ful mid folmun, \hld\ þemu þe þes folkes þar
ge-weld aftar þemu werde. \hld\ Reht só hi þes wínes ge-drank,
só ni mahte he be-míðan, \hld\ ne hi far þeru menigi sprak
te þemu brúdi-gumon, \hld\ kwað þat simbla þat bętste líð
alloro erlo ge-hwi-lik \hld\ érist skoldi
gevan at is gómun: \hld\ ʽundar þiu wirðid þero gumono hugi
a-wękid mid wínu, \hld\ þat sie wel blíðod,
drunkan dròmjad. \hld\ Þan mag man þar dragan aftar þiu
líht-líkora líð: \hld\ só ist þesoro liudjo þau.
Þan havas þu nu wunder-líko \hld\ werd-skępi þínan
ge-markod far þesoro menigi: \hld\ hétis far þit manno folk
alles þínes wínes \hld\ þat wirsiste
þíne ambaht-man \hld\ érist brengjan,
gevan at þínun gómun. \hld\ Nu sint þína gęsti sade,
sint þíne druhtingos \hld\ drunkane swíðo,
is þit folk fró-mód: \hld\ nu hétis þu hír forð dragan
alloro líðo lof-samost, \hld\ þero þe ik eo an þesumu liohte gesah
hwergin hębbjan. \hld\ Mid þius skoldis þu ús hindag ér
gevon endi gómjan: \hld\ þan it alloro gumono ge-hwi-lik
ge-þigedi te þanke.ʼ \hld\ Þó warð þar þegan manag
ge-war aftar þem wordun, \hld\ síðor sie þes wínes ge-drunkun,
þat þar þe hélogo Krist \hld\ an þemu húse innan
tékạn warhte: \hld\ trúodun sie síðor
þiu mér an is mund-burd, \hld\ þat hi habdi maht godes,
ge-wald an þesoro weroldi. \hld\ Þó warð þat só wído kúð
ovar Galileo land \hld\ Judeo liudjun,
hwó þar selvo ge-deda \hld\ sunu drohtines
water te wíne: \hld\ þat warð þar wundro érist,
þero þe hi þar an Galilea \hld\ Judeo liudjon,
tékno ge-tógdi. \hld\ Ne mag þat ge-tęlljan man,
ge-sęggjan te sóðan, \hld\ hwat þar síðor warð
wundres undar þemu werode, \hld\ þar waldand Krist
an godes namon \hld\ Judeo liudjon
allan langan dag \hld\ léra sagde,
gi-hét im heven-ríki \hld\ endi hęlljo ge-þwing
weride mid wordun, \hld\ hét sie wara godes,
sin-líf sókjan: \hld\ þar is seolono lioht,
dròm drohtines \hld\ endi dag-skímon,
gód-líknissea godes; \hld\ þar gést manag
wunod an willjan, \hld\ þe hír wel þenkid,
þat he hír bi-halde \hld\ heven-kuninges ge-bod.
Ge-wét imu þó mid is jungoron \hld\ fan þem gómun forð
Kristus te kapharnaum, \hld\ kuningo ríkjost,
te þeru márjon burg. \hld\ Megin samnode,
gumon imu te-gegnes, \hld\ gódoro manno
sálig ge-síði: \hld\ weldun þiu is swótjan word
hèlag hórjen. \hld\ Þar im én hunno kwam,
én gód man an-gegin \hld\ endi ina gerno bad
helpan hèlagne, \hld\ kwað þat hi undar is híwiskja
énna lefna lamon \hld\ lango habdi,
seokan an is selðon: \hld\ ʽsó ina énig seggjo ne mag
handun ge-hélien. \hld\ Nu is im þínoro helpono þarf,
fró mín þe gódo.ʼ \hld\ Þó sprak im eft þat friðu-barn godes
sán aftar þiu \hld\ selvo te-gegnes,
kwað þat he þar kwámi \hld\ endi þat kind weldi
nerean af þeru nòdi. \hld\ Þó im náhor geng
þe man far þeru menigi \hld\ wið só mahtigna
wordun wehslan: \hld\ ʽik þes wirðig ne bium,ʼ kwað he,
ʽhérro þe gódo, \hld\ þat þu an mín hús kumes,
sókjas mína seliða, \hld\ hwand ik bium só sundig man
mid wordun endi mid werkun. \hld\ Ik ge-lóvju þat þu ge-wald havas,
þat þu ina hinana maht \hld\ hélan ge-wirkjan,
waldand fró mín: \hld\ ef þu it mid þínun wordun ge-sprikis,
þan is sán þiu léf-héd lósot \hld\ endi wirðid is lík-hamo
hél endi hréni, \hld\ ef þu im þína helpa far-givis.
Ik bium mi ambaht-man, \hld\ hębbju mi ódes ge-nóg,
welono ge-wunnen: \hld\ þoh ik undar ge-weldi sí
aðal-kuninges, \hld\ þoh hębbju ik erlo ge-tróst,
holde hęri-rinkos, \hld\ þea mi só ge-hóriga sint,
þat sie þes ne word ne werk \hld\ wiht ne far-látad,
þes ik sie an þesumu land-skępje \hld\ léstjan héte,
ak sie farad endi frummjad \hld\ endi eft te iro fróhan kumad,
holde te iro hérron. \hld\ Þoh ik at mínumu hús égi
wíd-brédene welon \hld\ endi werodes ge-nóg,
hęliðos hugi-dervje, \hld\ þoh ni gidar ik þi só hèlagna
biddjen, barn godes, \hld\ þat þu an mín bú gangas,
sókjas mína seliða, \hld\ hwand ik só sundig bium,
wét mína far-wurhti.ʼ \hld\ Þó sprak eft waldand Krist,
þe gumo wið is jungoron, \hld\ kwað þat hi an Judeon hwergin
undar Israheles \hld\ avoron ne fundi
ge-makon þes mannes, \hld\ þe io mér te gode
an þemu land-skępi \hld\ ge-lóvon habdi,
þan hluttron te himile: \hld\ ʽnu látu ik iu þar hórjen tó,
þar ik it iu te wárun hír \hld\ wordun seggjo,
þat noh skulun eli-þeoda \hld\ óstane endi uestane,
man-kunnjes kuman \hld\ manag te-samne,
hèlag folk godes \hld\ an heven-ríki:
þea motun þar an Abrahames \hld\ endi an Isaakes só self
endi ók an Jakobes, \hld\ gódoro manno,
barmun restjen \hld\ endi béðju ge-þologean,
welon endi willjon \hld\ endi wonod-sam líf,
gód lioht mid gode. \hld\ Þan skal Judeono filu,
þeses ríkjas suni \hld\ beróvode werðen,
be-délide su-likoro diurðo, \hld\ endi skulun an dalun þiustron
an þemu alloro ferristan \hld\ ferne liggen.
Þar mag man ge-hórjen \hld\ hęliðos kwíðjan,
þar sie iro torn manag \hld\ tandon bítad;
þar ist grist-grimmo \hld\ endi grádag fiur,
hard hęlljo ge-þwing, \hld\ hét endi þiustri,
swart sinnahti \hld\ sundja te lòne,
wréðoro ge-wurhtjo, \hld\ só hwemu só þes willjon ne havad,
þat he ina a-lósje, \hld\ ér hi þit lioht ageve,
węndje fan þesoro weroldi. \hld\ Nu maht þu þi an þínan willjon forð
síðon te selðun; \hld\ þan findis þu ge-sundan at hús
mago-jungan man: \hld\ mód is imu an luston,
þat barn is ge-hélid, \hld\ só þu bédi te mi:
it wirðid al só ge-léstid, \hld\ só þu ge-lóvon havas
an þínumu hugi hardo.ʼ \hld\ Þó sagde heven-kuninge,
þe ambaht-man \hld\ alo-waldon gode
þank for þero þiodo, \hld\ þes he imu at su-likun þarvun halp.
Habda þo gi-árundid, \hld\ al só he welde,
sálig-líko: \hld\ gi-wét imu an þana síð þanan,
wende an is willjan, \hld\ þar he welon éhte,
bú endi bód-los: \hld\ fand þat barn ge-sund,
kind-jungan man. \hld\ Kristes wárun þó
word ge-fullot: \hld\ hi ge-wald habda
te tógjanna tékạn, \hld\ só þat ni mag gi-tęlljen man,
ge-ahton ovar þesoro erðu, \hld\ hwat he þurh is énes kraft
an þesaro middil-gard \hld\ máriða ge-frumide,
wundres ge-warhte, \hld\ hwand al an is ge-weldi stád,
himil endi erðe. \hld\ Þó ge-wét imu þe hélogo Krist
forð-wardes faren, \hld\ fremide alo-mahtig
alloro dago ge-hwi-likes, \hld\ drohtin þe gódo,
liudjo barnum leof, \hld\ lérde mid wordun
godes willjon gumun, \hld\ habda imu jungorono filu
simbla te gi-síðun, \hld\ sálig folk godes,
manno męgin-kraft, \hld\ managoro þeodo,
hèlag hęri-skępi, \hld\ was is helpono gód,
mannun mildi. \hld\ Þó hi mid þeru menigi kwam,
mid þiu brahtmu þat barn godes \hld\ te burg þeru hòhon,
þe nęrjendo te Naim: \hld\ þar skolde is namo werðen
mannun ge-márid. \hld\ Þó geng mahtig tó
nęrjendo Krist, \hld\ antat he gi-náhid was,
héljandero bętst: \hld\ þó sáhun sie þar én hréo dragan,
énan líf-lósan lík-hamon \hld\ þea liudi fórjen,
beran an énaru báru \hld\ út at þera burges dore,
magu-jungan man. \hld\ Þiu móder aftar geng
an iro hugi hriwig \hld\ endi handun slóg,
karode endi kúmde \hld\ iro kindes dóð,
idis arm-skapan; \hld\ it was ira énag barn:
siu was iru widowa, \hld\ ne habda wunnja þan mér,
bi-úten te þemu énagun \hld\ sunje al geláten
wunnja endi willjan, \hld\ anttat ina iru wurd benam,
mári metodo-ge-skapu. \hld\ Megin folgode,
burg-liudjo ge-brak, \hld\ þar man ina an báru dróg,
jungan man te grave. \hld\ Þar warð imu þe godes sunu,
mahtig mildi \hld\ endi te þeru móder sprak,
hét þat þiu widowa \hld\ wóp far-léti,
kara aftar þemu kinde: \hld\ ʽþu skalt hír kraft sehan,
waldandes gi-werk: \hld\ þi skal hír willjo ge-standen,
frófra far þesumu folke: \hld\ ne þarft þu ferah karon
barnes þínes.ʼ \hld\ *Þuo hie ti þero báron geng
iak hie ina selvo ant-hrén, \hld\ suno drohtines,
hèlagon handon, \hld\ endi ti þem hęliðe sprak,
hiet ina só ala-jungan \hld\ up a-standan,
arísan fan þeru restun. \hld\ Þie rink up asat,
þat barn an þero bárun: \hld\ warð im eft an is briost kuman
þie gést þuru godes kraft, \hld\ endi hie te-gegnes sprak,
þe man wið is mágos. \hld\ Þuo ina eft þero muoder bi-falah
hélandi Krist an hand: \hld\ hugi warð iro te frovra,
þes wíves an wunnjon, \hld\ hwand iro þar su-lik willjo gi-stuod.
Fell siu þó te fuotun Kristes \hld\ endi þena folko drohtin
lovoda for þero liudjo menigi, \hld\ hwand hie iro at só liobes ferahe
mundoda wiðer metodi-gi-skęftje: \hld\ far-stuod siu þat hie was þie mahtigo drohtin,
þie hèlago, þie himiles gi-waldid, \hld\ endi þat hie mahti gi-helpan managon,
allon irmin-þiedon. \hld\ Þuo bi-gunnun þat ahton managa,
þat wunder, þat under þem weroda gi-burida, \hld\ kwáðun þat waldand selvo,
mahtig kwámi þarod is menigi wíson, \hld\ endi þat hie im só márjan sandi
wár-sagon an þero weroldes ríki, \hld\ þie im þar su-likan willjon frumidi.
warð þar þuo erl manag \hld\ egison bi-fangan,
þat folk warð an forohton: \hld\ gi-sáhun þena is ferah égan,
dages lioht sehan, \hld\ þena þe ér dóð fornam,
an suht-będdjon swalt: \hld\ þuo was im eft gi-sund after þiu,
kind-jung a-kwikot. \hld\ Þuo warð þat kúð obar all
avaron Israheles. \hld\ Reht só þuo ávand kwam,
só warð þar all gi-samnod \hld\ seokora manno,
haltaro endi hávaro, \hld\ só hwat só þar hwergin was,
þia lévun under þem liudjon, \hld\ endi wurðun þar gi-lédit tuo,
kumana te Kriste, \hld\ þar hie im þuru is kraft mikil
halp endi sie hélda, \hld\ endi liet sia eft gi-haldana þanan
wendan an iro willjon. \hld\ Be-þiu skal man is werk lovon,
diuran is dádi, \hld\ hwand hie is drohtin self,
mahtig mund-boro \hld\ manno kunnje,
liudjo só hwi-likon, \hld\ só þar gi-lóbit tuo
an is word endi an is werk. \hld\ Þuo was þar werodes só filo
allaro eli-þiodo \hld\ kuman te þem éron Kristes,
te só mahtiges mund-burd. \hld\ Þuo welda hie þar éna męri líðan,
þie godes suno mid is jungron \hld\ anevan Galilea-land,
waldand énna wágo stròm. \hld\ Þuo hiet hie þat werod óðar
forð-werdes faran, \hld\ endi hie gi-wét im fahora sum
an énna nakon innan, \hld\ nęrjendi Krist,
slápan síð-wórig. \hld\ Segel up dádun
weder-wísa weros, \hld\ lietun wind after
manon ovar þena męri-stròm, \hld\ unþat hie te middjan kwam,
waldand mid is werodu. \hld\ Þuo bi-gan þes wedares kraft,
úst up stígan, \hld\ úðjun wahsan;
swang gi-swerk an gi-mang: \hld\ þie séu warð an hruoru,
wan wind endi water; \hld\ weros sorogodun,
þiu męri warð só muodag, \hld\ ni wánda þero manno ni-gén
lęngron líves. \hld\ Þuo sia landes ward
wękidun mid iro wordon \hld\ endi sagdun im þes wedares kraft,
bádun þat im gi-náðig \hld\ nęrjendi Krist
wurði wið þem watare: \hld\ ʽefþa wi skulun hier te wunder-kwálu
sweltan an þeson séwe.ʼ \hld\ Self up a-rés
þie guodo godes suno \hld\ endi te is jungron sprak,
hiet þat sia im wedares gi-win \hld\ wiht ni and-rédin:
ʽte hwí sind gi só forhta?ʼ \hld\ kwaþie. ʽNis iu noh fast hugi,
gi-lóvo is iu te luttil. \hld\ Nis nu lang te þiu,
þat þia stròmos skulun \hld\ stilrun werðan
gi þit *wedar wun-sam.ʼ \hld\ Þo hi te þem winde sprak
ge te þemu séwa só self \hld\ endi sie smultro hét
béðja ge-bárean. \hld\ Sie gi-bod léstun,
waldandes word: \hld\ weder stillodun,
fagar warð an flóde. \hld\ Þó bi-gan þat folk undar im,
werod wundraian, \hld\ endi suma mid iro wordun sprákun,
hwi-lik þat só mahtigoro \hld\ manno wári,
þat imu só þe wind endi þe wág \hld\ wordu hórdin,
béðja is gi-bod-skępjes. \hld\ Þó habda sie þat barn godes
ginerid fan þeru nòdi: \hld\ þe nako furðor skreid,
hòh hurnid-skip; \hld\ hęliðos kwámun,
liudi te lande, \hld\ sagdun lof gode,
máridun is męgin-kraft. \hld\ Kwam þar manno filu
an-gegin þemu godes sunje; \hld\ he sie gerno ant-feng,
só hwene só þar mid hluttru hugi \hld\ helpa sóhte;
lérde sie iro gi-lóvon \hld\ endi iro lík-hamon
handun hélde: \hld\ nio þe man só hardo ni was
gi-sérit mid suhtiun: \hld\ þoh ina Satanases
féknea jungoron \hld\ fíundes kraftu
habdin undar handun \hld\ endi is hugi-skęfti,
gi-wit a-wardid, \hld\ þat he wódjendi
fóri undar þemu folke, \hld\ þoh im simbla ferh far-gaf
hélandjo Krist, \hld\ ef he te is handun kwam,
dréf þea diuvlas þanan \hld\ drohtines kraftu,
wárun wordun, \hld\ endi im is ge-wit far-gaf,
lét ina þan hélan \hld\ wiðer hetteandun,
gaf im wið þie fíund friðu, \hld\ endi im forð gi-wét
an só hwi-lik þero lando, \hld\ só im þan leovost was.
Só deda þe drohtines sunu \hld\ dago ge-hwi-likes
gód werk mid is jungeron, \hld\ só neo Judeon umbi þat
an þea is mikilun kraft \hld\ þiu mér ne ge-lóvdun,
þat he alo-waldo \hld\ alles wári,
landes endi liudjo: \hld\ þes sie noh lòn nimat,
wídana wrak-síð, \hld\ þes sie þar þat ge-win drivun
wið selvan þene sunu drohtines. \hld\ Þó he im mid is ge-síðon gi-wét
eft an Galilaeo land, \hld\ godes égan barn,
fór im te þem friundun, \hld\ þar he a-fódid was
endi al undar is kunnje \hld\ kind-jung awóhs,
þe hèlago héljand. \hld\ Umbi ina hęri-skępi,
þeoda þrungun; \hld\ þar was þegan manag
só sálig undar þem ge-síðe. \hld\ Þar drógun énna seokan man
erlos an iro armun: \hld\ weldun ina for ógun Kristes,
brengean for þat barn godes \hld\ —was im bótono þarf,
þat ina ge-héldi \hld\ hevenes waldand,
manno mund-boro—, \hld\ þe was ér só managan dag
liðu-wastmon bi-lamod, \hld\ ni mahte is lík-hamon
wiht ge-waldan. \hld\ Þan was þar werodes só filu,
þat sie ina fora þat barn godes \hld\ brengjan ni mahtun,
ge-þringan þurh þea þioda, \hld\ þat sie só þurftiges
sunnea ge-sagdin. \hld\ Þó gi-wét imu an énna sęli innan
héljando Krist; \hld\ hwarf warð þar umbi,
męgin-þeodo gemang. \hld\ Þó bi-gunnun þea man spreken,
þe þene léfna lamon \hld\ lango fórdun,
bárun mid is będdju, \hld\ hwó sie ina ge-drógin fora þat barn godes,
an þat werod innan, \hld\ þar ina waldand Krist
selvo gi-sáwi. \hld\ Þó gengun þea ge-síðos tó,
hóvun ina mid iro handun \hld\ endi uppan þat hús stigun,
slitun þene sęli ovana \hld\ endi ina mid sélun létun
an þene rakud innan, \hld\ þar þe ríkjo was,
kuningo kraftigost. \hld\ Reht só he ina þó kuman gi-sah
þurh þes húses hróst, \hld\ só he þó an iro hugi far-stód,
an þero manno mód-sevon, \hld\ þat sie mikilana te imu
ge-lóvon habdun, \hld\ þó he for þen liudjun sprak,
kwað þat he þene siakon man \hld\ sundjono tómjan
látan weldi. \hld\ Þó sprákun im eft þea liudi an-gegin,
gram-harde Judeon, \hld\ þea þes godes barnes
word aftar-warodun, \hld\ kwáðun þat þat ni mahti gi-werðen só,
grim-werk far-geven, \hld\ bi-útan god éno,
waldand þesaro weroldes. \hld\ Þó habda eft is word garu
mahtig barn godes: \hld\ ʽik gidón þatʼ, kwað he, ʽan þesumu manne skín,
þe hír só siak ligid \hld\ an þesumu sęli innan,
te wundron gi-wégid, \hld\ þat ik ge-wald hębbju
sundja te far-gevanne \hld\ endi ók seokan man
te ge-héljanne, \hld\ só ik ina hrínan ni þarf.ʼ
Manoda ina þó \hld\ þe márjo drohtin,
liggjandjan lamon, \hld\ hét ina far þem liudjun a-standan
up alo-hélan \hld\ endi hét ina an is ahslun niman,
is bed-gi-wádi te baka; \hld\ he þat gi-bod léste
sniumo for þemu gi-síðja \hld\ endi geng imu eft ge-sund þanan,
hél fan þemu húse. \hld\ Þó þes só manag héðin man,
weros wundradun, \hld\ kwáðun þat imu waldand self,
god alo-mahtig \hld\ far-gevan habdi
méron mahti \hld\ þan elkor énigumu mannes sunje,
kraft endi kústi; \hld\ sie ni weldun ant-kęnnjan þoh,
Judeo liudi, \hld\ þat he god wári,
ne ge-lóvdun is léran, \hld\ ak habdun im léðan stríd,
wunnun wiðar is wordun: \hld\ þes sie werk hlutun,
léð-lík lòn-geld, \hld\ endi só noh lango skulun,
þes sie ni weldun hórjen \hld\ heven-kuninges,
Kristes lérun, \hld\ þea he kúðde ovar al,
wído aftar þesaro weroldi, \hld\ endi lét sie is werk sehan
allaro dago ge-hwi-likes, \hld\ is dádi skawon,
hórjen is hèlag word, \hld\ þe he te helpu ge-sprak
manno barnun, \hld\ endi só manag mahtig-lík
tékạn ge-tógda, \hld\ þat sie gi-trúodin þiu bet,
gi-lóvdin an is léra. \hld\ He só managan lík-hamon
balu-suhteo ant-band \hld\ endi bóta ge-skeride,
far-gaf fégiun ferah, \hld\ þem þe fúsid was
hęlið an hel-síð: \hld\ þan gi-deda ina þe héland self,
Krist þurh is kraft mikil \hld\ kwikan aftar dóða,
lét ina an þesaro weroldi forð \hld\ wunnjono neotan.
Só hélde he þea haltun man \hld\ endi þea hávon só self,
bótta, þem þar blinde wárun, \hld\ lét sie þat berhte lioht,
sin-skóni sehan, \hld\ sundja lósda,
gumono grim-werk. \hld\ Ni was gio Judeono be-þiu,
léðes liud-skępjes \hld\ gi-lóvo þiu bętara
an þene hèlagon Krist, \hld\ ak habdun im hardene mód,
swíðo starkan stríd, \hld\ far-standan ni weldun,
þat sie habdun for-fangan \hld\ fíundun an willjan,
liudi mid iro ge-lóvun. \hld\ Ni was gio þiu latoro be-þiu
sunu drohtines, \hld\ ak he sagde mid wordun,
hwó sie skoldin ge-halon \hld\ himiles ríki,
lérde aftar þemu lande, \hld\ habde imu þero liudjo só filu
gi-wenid mid is wordun, \hld\ þat im werod mikil,
folk folgoda, \hld\ endi he im filu sagda,
be biliðjun þat barn godes, \hld\ þes sie ni mahtun an iro breostun far-standan,
undar-huggjan an iro herton, \hld\ ér it im þe hèlago Krist
ovar þat erlo folk \hld\ oponun wordun
þurh is selves kraft \hld\ sęggjan welda,
márjan hwat he ménde. \hld\ Þar ina męgin umbi,
þioda þrungun: \hld\ was im þarf mikil
te gi-hórjenne \hld\ heven-kuninges
wár-fastun word. \hld\ He stód imu þó bi énes watares staðe,
ni welde þó bi þemu ge-þringe \hld\ ovar þat þegno folk
an þemu lande uppan \hld\ þea léra kúðjan,
ak geng imu þó þe gódo \hld\ endi is jungaron mid imu,
friðu-barn godes, \hld\ þemu flóde náhor
an én skip innan, \hld\ endi it skalden hét
lande rúmur, \hld\ þat ina þea liudi só filu,
þioda ni þrungi. \hld\ Stód þegan manag,
werod bi þemu watare, \hld\ þar waldand Krist
ovar þat liudjo folk \hld\ léra sagde:
ʽhwat, ik iu sęggjan magʼ, \hld\ kwað he, ʽge-síðos míne,
hwó imu én erl bi-gan \hld\ an erðu sájan
hrén-korni mid is handun. \hld\ Sum it an hardan stēn
ovan-wardan fel, \hld\ erðon ni habda,
þat it þar mahti wahsan \hld\ efþa wurtjo gi-fáhan,
kínan efþa bi-klíven, \hld\ ak warð þat korn far-loren,
þat þar an þeru léian gi-lag. \hld\ Sum it eft an land bi-fel,
an erðun aðal-kunnjes: \hld\ bi-gan imu aftar þiu
wahsen wán-líko \hld\ endi wurtjo fáhan,
lód an lustun: \hld\ was þat land só gód,
fránisko gi-fehod. \hld\ Sum it eft bi-fallen warð
an éna starka strátun, \hld\ þar stópon gengun,
hrosso hóf-slaga \hld\ endi hęliðo tráda;
warð imu þar an erðu \hld\ endi eft up gi-geng,
bi-gan imu an þemu wege wahsen; \hld\ þó it eft þes werodes farnam,
þes folkes fard mikil \hld\ endi fuglos a-lásun,
þat is þemu éksan wiht \hld\ aftar ni móste
werðan te willjan, \hld\ þes þar an þene weg bi-fel.
Sum warð it þan bi-fallen, \hld\ þar só filu stódun
þikkero þorno \hld\ an þemu dage;
warð imu þar an erðu \hld\ endi eft up gi-geng,
kén imu þar endi klivode. \hld\ Þó slógun þar eft krúd an gi-mang,
weridun imu þene wastom: \hld\ habda it þes waldes hlea
forana ovar-fangan, \hld\ þat it ni mahte te énigaro frumu werðen,
ef it þea þornos \hld\ só þringan móstun.ʼ
Þó sátun endi swígodun \hld\ ge-síðos Kristes,
word-spáha weros: \hld\ was im wundạr mikil,
be hwi-likun biliðjun \hld\ þat barn godes
su-lik sóð-lík spel \hld\ sęggjan bi-gunni.
Þó bi-gan is þero erlo \hld\ én frágojan
holdan hérron, \hld\ hnég imu te-gegnes
tulgo werð-liko: \hld\ ʽhwat, þu ge-wald havasʼ, kwað he,
ʽia an himile ia an erðu, \hld\ hèlag drohtin,
uppa endi niðara, \hld\ bist þu alo-waldo
gumono gésto, \hld\ endi wi þíne jungaron sind,
an úsumu hugi holde. \hld\ Hérro þe gódo,
ef it þín willjo sí, \hld\ lát ús þínaro wordo þar
endi gi-hórjen, \hld\ þat wi it aftar þi
ovar al Kristin-folk \hld\ kúðjan mótin.
wi witun þat þínun wordun \hld\ wár-lík biliði
forð folgojad, \hld\ endi ús is firinun þarf,
þat wi þín word endi þín werk, \hld\ —hwand it fan su-likumu ge-wittja kumid—
þat wi it an þesumu lande \hld\ at þi línon mótin.ʼ
Þó im eft te-gegnes \hld\ gumono bętsta
and-wordi ge-sprak: \hld\ ʽni ménde ik elkor wihtʼ, kwað he,
ʽte bi-dernjenne \hld\ dádjo mínaro,
wordo efþa werko; \hld\ þit skulun gi witan alle,
jungaron míne, \hld\ hwand iu far-geven havad
waldand þesaro weroldes, \hld\ þat gi witan mótun
an iuwom hugi-skęftjun \hld\ himilisk ge-rúni;
þem óðrun skal man be biliðjun \hld\ þat gi-bod godes
wordun wísjen. \hld\ Nu willju ik iu te wárun hier
márjen, hwat ik ménde, \hld\ þat gi mína þiu bet
ovar al þit land-skępi \hld\ léra far-standan.
Þat sád, þat ik iu sagda, \hld\ þat is selves word,
þiu hèlaga léra \hld\ heven-kuninges,
hwó man þea márjen skal \hld\ ovar þene middil-gard,
wído aftar þesaro weroldi. \hld\ Weros sind im gi-hugide,
man mislíko: \hld\ sum su-likan mód dregid,
harda hugi-skęfti \hld\ endi hréan sevon,
þat ina ni ge-werðod, \hld\ þat he it be iuwon wordun due,
þat he þesa mína léra forð \hld\ léstjen willje,
ak werðad þar só far-lorana \hld\ léra mína,
godes ambusni \hld\ endi iuwaro gumono word
an þemu uvilon manne, \hld\ só ik iu ér sagda,
þat þat korn far-warð, \hld\ þat þar mid kíðun ni mahte
an þemu sténe uppan \hld\ stędi-haft werðan.
Só wirðid al far-loran \hld\ eðilero spráka,
árundi godes, \hld\ só hwat só man þemu uvilon manne
wordun ge-wísid, \hld\ endi he an þea wirson hand,
undar fíundo folk \hld\ fard ge-kiusid,
an godes un-wiljan \hld\ endi an gramono hróm
endi an fiures farm. \hld\ Forð skal he hétean
mid is breost-hugi \hld\ bréda logna.
Nio gi an þesumu lande þiu lés \hld\ léra mína
wordun ni wísjad: \hld\ is þeses werodes só filu,
erlo aftar þesaro erðun: \hld\ bi-stéd þar óðar man,
þe is imu jung endi glau, \hld\ —endi havad imu gódan mód—,
sprákono spáhi \hld\ endi wét iuwaro spello gi-skéð,
hugid is þan an is herton \hld\ endi hórid þar mid is órun tó
swíðo niud-líko \hld\ endi náhor stéd,
an is breost hledid \hld\ þat gi-bod godes,
línod endi léstid: \hld\ is is gi-lóvo só gód,
talod imu, \hld\ hwó he óðrana eft gi-hwervje
mén-dádigan man, \hld\ þat is mód draga
hluttra trewa \hld\ te heven-kuninge.
Þan brédid an þes breostun \hld\ þat gi-bod godes,
þie luvigo gi-lóbo, \hld\ só an þemu lande duod
þat korn mid kíðun, \hld\ þar it gi-kund havad
endi imu þiu wurð bi-hagod \hld\ endi wederes gang,
ręgin endi sunne, \hld\ þat it is reht havad.
Só duod þiu godes léra \hld\ an þemu gódun manne
dages endi nahtes, \hld\ endi gangid imu diuval fer,
wréða wihti \hld\ endi þe ward godes
náhor mikilu \hld\ nahtes endi dages,
anttat sie ina brengead, \hld\ þat þar béðju wirðid
ia þiu léra te frumu \hld\ liudjo barnun,
þe fan is múðe kumid, \hld\ iak wirðid þe man gode;
havad só gi-wehslod \hld\ te þesaro werold-stundu
mid is hugi-skęftjun \hld\ himil-ríkjas gi-dél,
welono þene méstan: \hld\ farid imu an gi-wald godes,
tionuno tómig. \hld\ Trewa sind só góda
gumono ge-hwi-likumu, só nis goldes hord
ge-lík su-likumu gi-lóvon. Wesad iuwaro lérono forð
man-kunnje mildje; sie sind só mislíka,
hęliðos ge-hugda: sum havad iro hardan stríd,
wréðan willjan, wankolna hugi,
is imu féknes ful endi firin-werko.
Þan biginnid imu þunkjan, þan he undar þeru þiodu stád
endi þar gi-hórid ovar hlust mikil
þea godes léra, þan þunkid imu, þat he sie gerno forð
léstjen willje; þan biginnid imu þiu léra godes
an is hugi hafton, anttat imu þan eft an hand kumid
feho te gi-fórja endi fremiði skat.
Þan far-lédead ina léða wihti,
þan he imu far-fáhid \hld\ an feho-giri,
a-leskid þene gi-lóbon: \hld\ þan was imu þat luttil fruma,
þat he it gio an is hertan ge-hugda, \hld\ ef he it halden ne wili.
Þat is só þe wastom, \hld\ þe an þemu wege began,
liodan an þemu lande: \hld\ þó farnam ina eft þero liudjo fard.
Só duot þea męgin-sundjon \hld\ an þes mannes hugi
þea godes léra, \hld\ ef he is ni gómid wel;
elkor bi-felljad sia ina \hld\ ferne te boðme,
an þene hétan hel, \hld\ þar he heven-kuninge
ni wirðid furður te frumu, \hld\ ak ina fíund skulun
wítiu gi-waragean. \hld\ Simla gi mid wordun forð
lérjad an þesumu lande: \hld\ *ik kan þesaro liudjo hugi,
só mis-líkan muod-sevon \hld\ manno kunnjes,
só wanda wísa \hld\ [...]
Sum havit all te þiu is muod gi-látan \hld\ endi mér sorogot,
hwó hie þat hord bi-halde, \hld\ þan hwó hie hevan-kuninges
willjon gi-wirkje. \hld\ Be-þiu þar wahsan ni mag
þat hèlaga gi-bod godes, \hld\ þoh it þar a-hafton mugi,
wurtjon bi-werpan, \hld\ hwand it þie welo þringit.
Só samo só þat krúd endi þie þorn \hld\ þat korn ant-fáhat,
weriat im þena wastom, \hld\ só duot þie welo manne:
gi-heftid is herta, þat hie it gi-huggjan ni muot,
þie man an is muode, þes hie mést bi-þarf,
hwó hie þat gi-wirkje, þan lang þie hie an þesaro weroldi sí,
þat hie ti éwon-dage after muoti
hębbjan þuru is hérren þank himiles ríki,
só ęndi-lósan welon, só þat ni mag énig man
witan an þesaro weroldi. Nio hie só wído ni kan
te gi-þęnkjanne, þegan an is muode,
þat it bi-haldan mugi herta þes mannes,
þat hie þat ti wáron witi, hwat waldand god havit
guodes gi-gerewid, þat all gegin-werd stéð
manno só hwi-likon, só ina hier minnjot wel
endi selvo te þiu is seola gi-haldit,
þat hie an lioht godes líðan muoti.ʼ
Só wísda hie þuo mid wordon, stuod werod mikil
umbi þat barn godes, ge-hórdun ina bi biliðon filo
umbi þesaro weroldes gi-wand wordon tęlljan;
kwat þat im ók én aðales man an is akker sáidi
hluttar hrénkorni handon sínon:
wolda im þar só wun-sames wastmes tilian,
fagares fruhtes. Þuo geng þar is fíond aftar
þuru dernian hugi, endi it all mid durðu ovarséu,
mid weodo wirsiston. Þuo wóhsun sia béðju,
ge þat korn ge þat krúd. Só kwámun gangan
is hagastoldos te hús, iro hérren sagdun,
þegnos iro þiodne þrístion wordon:
ʽhwat, þu sáidos hluttar korn, hérro þie guodo,
énfald an þínon akkar: nu ni gi-sihit énig erlo þan mér
weodes wahsan. hwí mohta þat gi-werðan só?ʼ
Þuo sprak eft þie aðales man þem erlon te-gegnes,
þiodan wið is þegnos, kwat þat hie it mahti undar-þenkian wel,
þat im þar un-hold man aftar sáida,
fíond fékni krúd: ʽne gionsta mi þero fruhtio wel,
awerda mi þena wastom.ʼ Þuo þar eft wini sprákun,
is jungron te-gegnes, kwáðun þat sia þar weldin gangan tuo,
kuman mid kraftu endi lósjan þat krúd þanan,
halon it mid iro handon. Þuo sprak im eft iro hérro an-gegin:
ʽne welleo ik, þat gi it wiodonʼ, kwaþie, ʽhwand gi bi-wardon ni mugun,
gi-gómjan an iuwon gange, þoh gi it gerno ni duan,
ni gi þes kornes te filo, kíðo a-werdiat,
felliat under iuwa fuoti. Láte man sia forð hinan
béðju wahsan, und ér bewod kume
endi an þem felde sind fruhti rípia,
aroa an þem akkare: þan faran wi þar alla tuo,
halon it mid ússan handon endi þat hrénkurni lesan
súvro te-samne endi it an mínon sęli duojan,
hębbjan it þar gi-haldan, þat it hwergin ni mugi
wiht awerdian, endi þat wiod niman,
bindan it te burðinnion endi werpan it an bittar fiur,
láton it þar halojan héta lógna,
éld unfuodi.ʼ Þuo stuod erl manag,
þegnos þagiandi, hwat þiod-gomo,
*mári mahtig Krist ménean weldi,
bóknien mid þiu biliðju barno ríkjost.
Bádun þó só gerno gódan drohtin
ant-lúkan þea léra, þat sia móstin þea liudi forð,
hèlaga hórjan. Þó sprak im eft iro hérro an-gegin,
mári mahtig Krist: ʽþat isʼ, kwað he, ʽmannes sunu:
ik selvo bium, þat þar sáiu, endi sind þesa sáliga man
þat hluttra hrénkorni, þea mi hér hórjad wel,
wirkiad mínan willjan; þius werold is þe akkar,
þit bréda búland barno man-kunnjes;
Satanas selvo is, þat þar sáid aftar
só léð-líka léra: havad þesaro liudjo só filu,
werodes awardid, þat sie wam frummjad,
wirkjad aftar is willjon; þoh skulun sie hér wahsen forð,
þea for-griponon gumon, só samo só þea gódun man,
anttat múdspelles męgin ovar man ferid,
endi þesaro weroldes. Þan is allaro akkaro ge-hwi-lik
gerípod an þesumu ríkja: skulun iro regan-gi-skapu
frummjen firiho barn. Þan te-farid erða:
þat is allaro bewo brédost; þan kumid þe berhto drohtin
ovana mid is ęngilo kraftu, endi kumad alle te-samne
liudi, þe io þit lioht gi-sáun, endi skulun þan lòn ant-fáhan
uviles endi gódes. Þan gangad ęngilos godes,
hèlage hevenwardos, endi lesat þea hluttron man
sundor te-samne, endi duat sie an sin-skóni,
hòh himiles lioht, endi þea óðra an hęllja grund,
werpad þea far-warhton \hld\ an wallandi fiur;
þar skulun sie gi-bundene \hld\ bittra logna,
þrá-werk þolon, \hld\ endi þea óðra þiod-welon
an heven-ríkja, \hld\ hwítaro sunnon
liohtean ge-líko. \hld\ Sulik lòn nimad
weros wal-dádjo. \hld\ Só hwe só gi-wit égi,
ge-hugdi an is hertan, \hld\ etþa gi-hórjen mugi,
erl mid is órun, \hld\ só láta imu þit an innan sorga,
an is mód-sevon, \hld\ hwó he skal an þemu márjon dage
wið þene ríkjon god \hld\ an reðiu standen
wordo endi werko allaro, \hld\ þe he an þesaro weroldi gi-duod.
Þat is egis-líkost \hld\ allaro þingo,
forhtlíkost firiho barnun, \hld\ þat sie skulun wið iro fráhon mahlien,
gumon wið þene gódan drohtin: \hld\ þan weldi gerno ge-hwe wesan,
allaro manno ge-hwi-lik \hld\ ménes tómig,
slíðero sakono. \hld\ Aftar þiu skal sorgon ér
allaro liudjo ge-hwi-lik, \hld\ ér he þit lioht afgeve,
þe þan égan wili \hld\ alungan tír,
hòh heven-ríki \hld\ endi huldi godes.ʼ
Só gi-fragn ik þat þó selvo \hld\ sunu drohtines,
allaro barno bętst \hld\ biliðeo sagda,
hwi-lik þero wári \hld\ an werold-ríkja
undar hęlið-kunnje \hld\ himil-ríkje ge-lík;
kwað þat oft luttiles hwat \hld\ liohtora wurði,
só hòho af-huovi, \hld\ ʽso duot himil-ríki:
þat is simla méra, \hld\ þan is man énig
wánje an þesaro weroldi. \hld\ Ók is imu þat werk ge-lík,
þat man an séo innan \hld\ segina wirpit,
fisk-net an flód \hld\ endi fáhit béðju,
uvile endi góde, \hld\ tiuhid up te staðe,
liðod sie te lande, \hld\ lisit aftar þiu
þea gódun an greote \hld\ endi látid þea óðra eft an grund faran,
an wídan wág. \hld\ Só duod waldand god
an þemu márjon dage \hld\ męnniskono barn:
brengid irmin-þiod, \hld\ alle te-samne,
lisit imu þan þea hluttron \hld\ an heven-ríki,
látid þea far-griponon \hld\ an grund faren
hęllje fiures. \hld\ Ni wét hęliðo man
þes wítjes wiðarlága, \hld\ þes þar weros þiggjat,
an þemu inferne \hld\ irmin-þioda.
Þan hald ni mag þera médan man \hld\ gi-makon fíðen,
ni þes welon ni þes willjon, \hld\ þes þar waldand skerid,
gildid god selvo \hld\ gumono só hwi-likumu,
só ina hér gi-haldid, \hld\ þat he an heven-ríki,
an þat lang-same lioht \hld\ líðan móti.ʼ
Só lérda he þó mid listjun. \hld\ Þan fórun þar þea liudi tó
ovar al Galilaeo land \hld\ þat godes barn sehan:
dádun it bi þemu wundre, \hld\ hwanen imu mahti su-lik word kumen,
só spáh-líko gi-sprokan, \hld\ þat he spel godes
gio só sóð-líko \hld\ sęggjan konsti,
só kraftig-líko gi-kweðen: \hld\ ʽhe is þeses kunnjes hinenʼ, kwáðun sie,
ʽþe man þurh mág-skępi: \hld\ hér is is móder mid ús,
wíf undar þesumu werode. \hld\ Hwat, wi þe hér witun alle,
só kúð is ús is kuni-burd \hld\ endi is knósles ge-hwat;
a-wóhs al undar þesumu werode: \hld\ hwanen skoldi imu su-lik ge-wit kuman,
méron mahti, \hld\ þan hér óðra man égin?ʼ
Só far-munste ina þat manno folk \hld\ endi sprákun im gi-méd-lik word,
far-hogdun ina só hèlagna, \hld\ hórjen ni weldun
is gi-bod-skępjes. \hld\ Ni he þar ók biliðeo filu
þurh iro un-gi-lóvon \hld\ ógjan ni welde,
torhtero tékno, \hld\ hwand he wisse iro twíflean hugi,
iro wréðan willjan, \hld\ þat ni wárun weros óðra
só grimme under Judeon, \hld\ só wárun umbi Galilaeo land,
só hardo ge-hugide: \hld\ só þar was þe hèlago Krist,
gi-boren þat barn godes, \hld\ si ni weldun is gi-bod-skępi þoh
ant-fáhan ferht-líko, \hld\ ak bi-gan þat folk undar im,
rinkos rádan, \hld\ hwó sie þene ríkjon Krist
wégdin te wundron. \hld\ Hétun þó iro werod kumen,
ge-síði te-samne: \hld\ sundja weldun
an þene godes sunu \hld\ gerno gi-tęlljen
wréðes willjon; \hld\ ni was im is wordo niud,
spáharo spello, \hld\ ak sie bi-gunnun sprekan undar im,
hwó sie ina só kraftagne \hld\ fan énumu klive wurpin,
ovar énna berges wal: \hld\ weldun þat barn godes
livu bi-lósjen. \hld\ Þó he imu mid þem liudjun samad
fró-líko fór: \hld\ ni was imu foraht hugi,
—wisse þat imu ni mahtun \hld\ męnniskono barn,
bi þeru god-kundi \hld\ Judeo liudi
ér is tídiun wiht \hld\ teonon gi-frummjen,
léðaro gi-lésto—, \hld\ ak he imu mid þem liudjun samad
stég uppen þene stén-holm, \hld\ antþat sie te þeru stędi kwámun,
þar sie ine fan þemu walle niðer \hld\ werpen hugdun,
fellien te foldu, \hld\ þat he wurði is ferhes lós,
is aldres at endie. \hld\ Þó warð þero erlo hugi,
an þemu berge uppen \hld\ bittra gi-þáhti
Juðeono te-gangen, \hld\ þat iro énig ni habde só grimmon sevon
ni só wréðen willjon, \hld\ þat sie mahtin þene waldandes sunu,
Krist ant-kęnnjen; \hld\ he ni was iro kúð énigumu,
þat sie ina þó undar-wissin. \hld\ Só mahte he undar ira werode standen
endi an iro gi-mange \hld\ middjumu gangen,
faren undar iro folke. \hld\ He dede imu þene friðu selvo,
mundburd wið þeru menegi \hld\ endi gi-wét imu þurh middi þanan
þes fíundo folkes, \hld\ fór imu þó, þar he welde,
an éne wóstunnie \hld\ waldandes sunu,
kuningo kraftigost: \hld\ habde þero kustes gi-wald,
hwar imu an þemu lande \hld\ leovost wári
te wesanne an þesaru weroldi. \hld\ Þan fór imu an weg óðran
Johannes mid is jungarun, \hld\ godes ambaht-man,
lérde þea liudi \hld\ lang-samane rád,
hét þat sie frume fremidin, \hld\ firina far-létin,
mén endi morð-werk. \hld\ He was þar managumu liof
gódaro gumono. \hld\ He sóhte imu þó þene Judeono kuning,
þene hęri-togon at hús, \hld\ þe héten úuas
Erodes aftar is eldiron, \hld\ ovar-módig man:
búide imu be þeru brúdi, \hld\ þiu ér sínes bróðer was,
idis an éhti, \hld\ anttat he ellior skók,
werold weslode. \hld\ Þó imu þat wíf ginam
þe kuning te kwenun; \hld\ ér wárun iro kind ódan,
barn be is bróðer. \hld\ Þó bi-gan imu þea brúd lahan
Johannes þe gódo, \hld\ kwað þat it gode wári,
waldande wiðer-mód, \hld\ þat it énig wero frumidi,
þat bróðer brúd \hld\ an is bed námi,
hębbje sie imu te híwun. \hld\ ʽEf þu mi hórjen wili,
gi-lóvjen mínun lérun, \hld\ ni skalt þu sie leng égan,
ak míð ire an þínumu móde: \hld\ ni hava þar su-lika minnja tó,
ni sundjo þi te swíðo.ʼ \hld\ Þó warð an sorgun hugi
þes wíves aftar þem wordun; \hld\ and-réd þat he þene werold-kuning
sprákono ge-spóni \hld\ endi spáhun wordun,
þat he sie far-léti. \hld\ Be-gan siu imu þó léðes filu
ráden an rúnon, \hld\ endi ine rinkos hét,
un-sundigane \hld\ erlos fáhan
endi ine an énumu karkerea \hld\ klústar-bęndjun,
liðo-kospun bi-lúkan: \hld\ be þem liudjun ne gi-dorstun
ine ferahu bi-lósjen, \hld\ hwand sie wárun imu friund alle,
wissun ine só góden \hld\ endi gode werðen,
habdun ina for wár-sagon, \hld\ só sia wela mahtun.
Þó wurðun an þemu gér-tale \hld\ Judeo kuninges
tídi kumana, \hld\ só þar gi-tald habdun
fróde folk-weros, \hld\ þó he gi-fódid was,
an lioht kuman. \hld\ Só was þero liudjo þau,
þat þat erlo ge-hwi-lik \hld\ óvean skolde,
Judeono mid gómun. \hld\ Þó warð þar an þene gast-sęli
męgin-kraft mikil \hld\ manno ge-samnod,
hęri-togono an þat hús, \hld\ þar iro hérro was
an is kuning-stóle. \hld\ Kwámun managa
Judeon an þene gast-sęli; \hld\ warð im þar glad-mód hugi,
blíði an iro breostun: \hld\ gi-sáhun iro bág-gevon
wesen an wunnjon. \hld\ Dróg man wín an flęt
skíri mid skálun, \hld\ skęnkjon hwurvun,
gengun mid gold-fatun: \hld\ gaman was þar inne
hlúd an þero hallu, \hld\ hęliðos drunkun.
was þes an lustun \hld\ landes hirdi,
hwat he þemu werode mést \hld\ te wunnjun gi-fręmidi.
Hét he þó gangen forð \hld\ géla þiornun,
is bróder barn, \hld\ þar he an is bęnki sat
wínu gi-wlenkid, \hld\ endi þó te þemu wíbe sprak;
grótte sie fora þemu gum-skępje \hld\ endi gerno bad,
þat siu þar fora þem gastjun \hld\ gaman af-hóvi
fagar an flęttje: \hld\ ʽlát þit folk sehan,
hwó þu gelínod havas \hld\ liudjo menegi
te blíðseanne an bęnkjun; \hld\ ef þu mi þera bede tugiðos,
mín word for þesumu werode, \hld\ þan willju ik it hér te wárun ge-kweðen,
liahto fora þesun liudjun \hld\ endi ók gi-léstjen só,
þat ik þi þan aftar þiu \hld\ éron willju,
só hwes só þu mi bidis \hld\ for þesun mínun bág-winjun:
þoh þu mi þesaro hęri-dómo \hld\ halvaro fergos,
ríkjas mínes, \hld\ þoh gi-dón ik, þat it énig rinko ni mag
wordun gi-węndjen, \hld\ endi it skal gi-werðen só.ʼ
Þó warð þera magað aftar þiu \hld\ mód gi-hworven,
hugi aftar iro hérron, \hld\ þat siu an þemu húse innen,
an þemu gast-sęli \hld\ gamen up a-huof,
al só þero liudjo \hld\ land-wíse gi-dróg,
þero þiodo þau. \hld\ Þiu þiorne spilode
hrór aftar þemu húse: \hld\ hugi was an lustun,
managaro mód-sevo. \hld\ Þó þiu magað habda
gi-þionod te þanke \hld\ þiod-kuninge
endi allumu þemu erl-skępje, \hld\ þe þar inne was
gódaro gumono, \hld\ siu welde þó ira geva égan,
þiu magað for þeru menegi: \hld\ geng þó wið iro módar sprekan
endi frágode sie \hld\ firi-wit-líko,
hwes siu þene burges ward \hld\ biddjen skoldi.
Þó wísde siu aftar iro willjon, \hld\ hét þat siu wihtes þan ér
ni gerodi for þemu gum-skępje, \hld\ bi-útan þat man iru Johannes
an þeru hallu innan \hld\ hòvid gávi
a-lósid af is lík-hamon. \hld\ Þat was allun þem liudjun harm,
þem mannun an iro móde, \hld\ þó sie þat gi-hórdun þea magað sprekan;
só was it ók þemu kuninge: \hld\ he ni mahte is kwidi liagan,
is word węndjen: \hld\ hét þó is wépan-berand
gangen fan þemu gast-sęli \hld\ endi hét þene godes man
lívu bi-lósjen. \hld\ Þó ni was lang te þiu,
þat man an þea halla \hld\ hòvid bráhte
þes þiod-gumon, \hld\ endi it þar þeru þiornun far-gaf,
magað for þeru menegi: \hld\ siu dróg it þeru móder forð.
Þó was én-dago \hld\ allaro manno
þes wísoston, \hld\ þero þe gio an þesa werold kwámi,
þero þe kwene énig \hld\ kind gi-bári,
idis fan erle, \hld\ lét man simla þen énon bi-foran,
þe þiu þiorne gi-dróg, \hld\ þe gio þegnes ni warð
wís an iro weroldi, \hld\ bi-útan só ine waldand god
fan heven-wange \hld\ hèlages géstes
gi-markode mahtig: \hld\ þe ni habde énigan gi-makon hwergin
ér nek aftar. \hld\ Erlos hwurvun,
gumon umbi Johannen, \hld\ is jungaron managa,
sálig ge-síði, \hld\ endi ine an sande bi-gróvun,
leoves lík-hamon: \hld\ wissun þat he lioht godes,
diur-líkan dròm \hld\ mid is drohtine samad,
upódas hém \hld\ égan móste,
sálig sókjan. \hld\ Þó ge-witun im þea ge-síðos þanen,
Johannes giungaron \hld\ giámer-móde,
hèlag-feraha: \hld\ was im iro hérron dóð
swíðo an sorgun. \hld\ Ge-witun im sókjan þó
an þeru wóstunni \hld\ waldandes sunu,
kraftigana Krist \hld\ endi imu kúð gi-dedun
gódes mannes for-gang, \hld\ hwó habde þe Judeono kuning
manno þene márjostan \hld\ mákjas ęggjun
hóvdu bi-hauwan: \hld\ he ni welde is énigen harm spreken,
sunu drohtines; \hld\ he wisse þat þiu seole was
hèlag gi-halden \hld\ wiðer hettiandeon,
an friðe wiðer fíundun. \hld\ Þó só gi-frági warð
aftar þem land-skępjun \hld\ lérjandero bętst
an þeru wóstunni: \hld\ werod samnode,
fór folkun tó: \hld\ was im firi-wit mikil
wísaro wordo; \hld\ imu was ók willjo só samo,
sunje drohtines, \hld\ þat he su-lik ge-síðo folk
an þat lioht godes \hld\ laðojan mósti,
wennien mid willjon. \hld\ Waldand lérde
allan langan dag \hld\ liudi managa,
eli-þeodige man, \hld\ anttat an ávand ség
sunne te sedle. \hld\ Þó gengun is ge-síðos twelivi,
gumon te þemu godes barne \hld\ endi sagdun iro gódumu hérron,
mid hwi-liku arvediu þar þea erlos livdin, \hld\ kwáðun þat sie is éra bi-þorftin,
weros an þemu wósteon lande: \hld\ ʽsie ni mugun sie hér mid wihti ant-hębbjen,
hęliðos bi hungres ge-þwinge. \hld\ Nu lát þu sie, hérro þe gódo,
síðon, þar sie sęliða fíðen. \hld\ Náh sind hér ge-setana burgi
managa mid męgin-þiodun: \hld\ þar fíðad sie meti te kópe,
weros aftar þem wíkeon.ʼ \hld\ Þó sprak eft waldand Krist,
þioda drohtin, \hld\ kwað þat þes éniga þurufti ni wárin,
ʽþat sie þurh meti-lósi \hld\ mína far-látan
leov-líka léra. \hld\ Gevad gi þesun liudjun gi-nóg,
wennjad sie hér mid willjon.ʼ \hld\ Þó habde eft is word garu
Philippus fród gumo, \hld\ kwað þat þar só filu wári
manno menigi: \hld\ ʽþoh wi hér te meti habdin
garu im te gevanne, \hld\ só wi mahtin far-gelden mést,
ef wi hér gi-saldin \hld\ siluver-skatto
twé hund samad, \hld\ tweho wári is noh þan,
þat iro énig þar \hld\ énes gi-námi:
só luttik wári þat þesun liudjun.ʼ \hld\ Þó sprak eft þe landes ward
endi frágode sie \hld\ firi-wit-líko,
manno drohtin, \hld\ hwat sie þar te meti habdin
wistes ge-wunnin. \hld\ Þó sprak imu eft mid is wordun an-gegin
Andreas fora þem erlun \hld\ endi þemu alo-waldon
selvumu sagde, \hld\ þat sie an iro gi-síðje þan mér
garowes ni habdin, \hld\ ʽbi-útan girstin bród
fívi an úsaru ferdi \hld\ endi fiskos twéne.
Hwat mag þat þoh þesaru menigi?ʼ \hld\ Þó sprak imu eft mahtig Krist,
þe gódo godes sunu, \hld\ endi hét þat gumono folk
skerien endi skéðen \hld\ endi hét þea skola settien,
erlos aftar þeru erðu, \hld\ irmin-þioda
an grase gruonimu, \hld\ endi þó te is jungarun sprak,
allaro barno bętst, \hld\ hét imu þiu bród halon
endi þea fiskos forð. \hld\ Þat folk stillo béd,
sat ge-síði mikil; \hld\ undar þiu he þurh is selves kraft,
manno drohtin, \hld\ þene meti wíhide,
hèlag heven-kuning, \hld\ endi mid is handun brak,
gaf it is jungarun forð, \hld\ endi it sie undar þemu gum-skępje hét
dragan endi déljen. \hld\ Sie léstun iro drohtines word,
is geva gerno drógun \hld\ gumono gi-hwemu,
hèlaga helpa. \hld\ It undar iro handun wóhs,
meti manno gi-hwemu: \hld\ þeru męgin-þiodu warð
líf an lustun, \hld\ þea liudi wurðun alle,
sade sálig folk, \hld\ só hwat só þar gi-samnod was
fan allun wídun wegun. \hld\ Þó hét waldand Krist
gangen is jungaron \hld\ endi hét sie gómjen wel,
þat þiu léva þar \hld\ far-loren ni wurði;
hét sie þó samnon, \hld\ þó þar sade wárun
man-kunnjes manag. \hld\ Þar móses warð,
bródes te lévu, \hld\ þat man birilos gi-las
twelivi fulle: \hld\ þat was tékạn mikil,
grót kraft godes, \hld\ hwand þar was gumono gi-tald
áno wíf endi kind, \hld\ werodes at-samme
fíf þúsundig. \hld\ Þat folk al far-stód,
þea man an iro móde, \hld\ þat sie þar mahtigna
hérron habdun. \hld\ Þó sie heven-kuning,
þea liudi lovodun, \hld\ kwáðun þat gio ni wurði an þit lioht kuman
wísaro wár-sago, \hld\ efþa þat he gi-wald mid gode
an þesaru middil-gard \hld\ méron habdi,
én-faldaran hugi. \hld\ Alle gi-sprákun,
þat he wári wirðig \hld\ welono ge-hwi-likes,
þat he erð-ríki \hld\ égan mósti,
wídene werold-stól, \hld\ ʽnu he su-lik ge-wit havad,
só gróte kraft mid gode.ʼ \hld\ Þea gumon alle gi-warð,
þat sie ine gi-hóvin \hld\ te hérosten,
gi-kurin ine te kuninge: \hld\ þat Kriste ni was
wihtes wirðig, \hld\ hwand he þit werold-ríki,
erðe endi up-himil \hld\ þurh is énes kraft
selvo gi-warhte \hld\ endi síðor gi-held,
land endi liud-skępi, \hld\ —þoh þes énigan gi-lóvon ni dedin
wréðe wiðer-sakon— \hld\ þat al an is gi-walde stád,
kuning-ríkjo kraft \hld\ endi késur-dómes,
męgin-þiodo mahal. \hld\ Be-þiu ni welde he þurh þero manno spráka
hębbjan énigan hér-dóm, \hld\ hèlag drohtin,
werold-kuninges namon; \hld\ ni he þó mid wordun stríd
ni af-hóf wið þat folk furður, \hld\ ak fór imu þó, þar he welde,
an én ge-birgi uppan: \hld\ flóh þat barn godes
gélaro gelp-kwidi \hld\ endi is jungaron hét
ovar énne séo síðon \hld\ endi im selvo gi-bód,
hwar sie im eft te-gegnes \hld\ gangen skoldin.
Þó te-lét þat liud-werod \hld\ aftar þemu lande allumu,
te-fór folk mikil, \hld\ síðor iro fráho gi-wét
an þat ge-birgi uppan, \hld\ barno ríkjost,
waldand an is willjon. \hld\ Þó te þes watares staðe
samnodun þea ge-síðos Kristes, \hld\ þe he imu habde selvo gi-korane,
sie twelivi þurh iro trewa góda: \hld\ ni was im tweho nigiean,
nevu sie an þat godes þionost \hld\ gerno weldin
ovar þene séo síðon. \hld\ Þó létun sie swíðean stròm,
hòh hurnid-skip \hld\ hluttron úðjon,
skéðan skír water. \hld\ Skréd lioht dages,
sunne warð an sedle; \hld\ þe séo-líðandean
naht nevulo bi-warp; \hld\ náðidun erlos
forð-wardes an flód; \hld\ warð þiu fiorðe tid
þera nahtes kuman \hld\ —nęrjendo Krist
warode þea wág-líðand—: \hld\ þó warð wind mikil,
hòh weder af-haven: \hld\ hlamodun úðjon,
stròm an stamne; \hld\ strídiun feridun
þea weros wiðer winde, \hld\ was im wréð hugi,
sevo sorgono ful: \hld\ selvon ni wándun
lagu-líðandea \hld\ an land kumen
þurh þes wederes ge-win. \hld\ Þó gi-sáhun sie waldand Krist
an þemu sée uppan \hld\ selvun gangan,
faran an fáðion: \hld\ ni mahte an þene flód innan,
an þene séo sinkan, \hld\ hwand ine is selves kraft
hèlag ant-habde. \hld\ Hugi warð an forhtun,
þero manno mód-sevo: \hld\ and-rédun þat it im mahtig fíund
te gi-droge dádi. \hld\ Þó sprak im iro drohtin tó,
hèlag heven-kuning, \hld\ endi sagde im þat he iro hérro was
mári endi mahtig: \hld\ ʽnu gi módes skulun
fastes fáhen; \hld\ ne sí iu forht hugi,
gi-báriad gi bald-líko: \hld\ ik bium þat barn godes,
is selves sunu, \hld\ þe iu wið þesumu sée skal,
mundon wið þesan męri-stròm.ʼ \hld\ Þó sprak imu én þero manno an-gegin
ovar bord skipes, \hld\ bar-wirðig gumo,
Petrus þe gódo \hld\ —ni welde píne þolon,
watares wíti—: \hld\ ʽef þu it waldand sísʼ, kwað he,
ʽhérro þe gódo, \hld\ só mi an mínumu hugi þunkit,
hét mi þan þarod gangan te þi \hld\ ovar þesen gevenes stròm,
drokno ovar diap water, \hld\ ef þu mín drohtin sís,
managoro mund-boro.ʼ \hld\ Þó hét ine mahtig Krist
gangan imu te-gegnes. \hld\ He warð garu sáno,
stóp af þemu stamne \hld\ endi strídiun geng
forð te is frójan. \hld\ Þiu flód ant-habde
þene man þurh maht godes, \hld\ antat he imu an is móde bi-gan
and-ráden diap water, \hld\ þó he dríven gi-sah
þene wég mid windu: \hld\ wundun ina úðjon,
hòh stròm umbi-hring. \hld\ Reht só he þó an is hugi twehode,
só wék imu þat water under, \hld\ endi he an þene wág innan,
sank an þene séo-stròm, \hld\ endi he hriop sán aftar þiu
gáhon te þemu godes sunje \hld\ endi gerno bad,
þat he ine þó ge-neridi, \hld\ þó he an nòdjun was,
þegan an ge-þwinge. \hld\ Þiodo drohtin
ant-feng ine mid is faðmun \hld\ endi frágode sána,
te hwí he þó ge-twehodi: \hld\ ʽhwat, þu mahtes ge-trúojan wel,
witen þat te wárun, \hld\ þat þi watares kraft
an þemu sée innen \hld\ þínes síðes ni mahte,
lagu-stròm gi-lęttjen, \hld\ só lango só þu habdes ge-lóvon te mi
an þínumu hugi hardo. \hld\ Nu willju ik þi an helpun wesen,
nęrjen þi an þesaru nòdiʼ. \hld\ Þó nam ine alo-mahtig,
hèlag bi handun: \hld\ þó warð imu eft hlutter water
fast under fótun, \hld\ endi sie an fáði samad
béðja gengun, \hld\ antat sie ovar bord skipes
stópun fan þemu stròme, \hld\ endi an þemu stamne gesat
allaro barno bętst. \hld\ Þó warð bréd water,
stròmos ge-stillid, \hld\ endi sie te staðe kwámun,
lagu-líðandea \hld\ an land samen
þurh þes wateres ge-win, \hld\ sagdun þo waldande þank,
diurden iro drohtin \hld\ dádjun endi wordun,
fellun imu te fótun \hld\ endi filu sprákun
wísaro wordo, \hld\ kwáðun þat sie wissin garo,
þat he wári selvo \hld\ sunu drohtines
wár an þesaru weroldi \hld\ endi ge-wald habdi
ovar middil-gard, \hld\ endi þat he mahti allaro manno gi-hwes
ferahe gi-formon, \hld\ al só he im an þemu flóde dede
wið þes watares ge-win. \hld\ Þó gi-wét imu waldand Krist
síðon fan þemu sée, \hld\ sunu drohtines,
énag barn godes. \hld\ Eli-þioda kwam imu,
gumon te-gegnes: \hld\ wárun is gódun werk
ferran ge-frági, \hld\ þat he só filu sagde
wároro wordo: \hld\ imu was willjo mikil,
þat he su-lik folk-skępi \hld\ frummjen mósti,
þat sie simla gerno \hld\ gode þionodin,
wárin ge-hórige \hld\ heven-kuninge
man-kunnjes manag. \hld\ Þó gi-wét he imu over þea marka Judeono,
sóhte imu Sidono burg, \hld\ habde ge-síðos mid imu,
góde jungaron. \hld\ Þar imu te-gegnes kwam
én idis fan áðrom þiodun; \hld\ siu was iru aðali-ge-burdjo,
kunnjes fan Kananeo lande; \hld\ siu bad þene kraftagan drohtin,
hèlagna, þat he iru helpe ge-rédi, \hld\ kwað þat iru wári harm gi-standen,
soroga at iru selvaru dohter, \hld\ kwað þat siu wári mid suhtiun bi-fangen:
ʽbe-drogan habbjad sie dernea wihti. \hld\ Nu is iro dód at hendi,
þea wréðon habbjad sie ge-wittju be-numane. \hld\ Nu biddju ik þi, waldand fró min,
selvo sunu Dawides, \hld\ þat sie af su-likum suhtiun a-tómjes,
þat þu sie só arma \hld\ égroht-fullo
wam-skaðon bi-weri.ʼ \hld\ Ni gaf iru þó noh waldand Krist
énig and-wordi; \hld\ siu imu aftar geng,
folgode fruokno, \hld\ antat siu te is fótun kwam,
grótte ina greatandi. \hld\ Giungaron Kristes
bádun iro hérron, \hld\ þat he an is hugja mildi
wurði þemu wíve. \hld\ Þó habde eft is word garu
sunu drohtines \hld\ endi te is ge-síðun sprak:
ʽérist skal ik Israheles \hld\ avoron werðen,
folk-skępi te frumu, \hld\ þat sie ferhtan hugi
hębbjan te iro hérron: \hld\ im is helpono þarf,
þea liudi sind far-lorane, \hld\ far-láten habbjad
waldandes word, \hld\ þat werod is ge-twíflid,
drívad im dernean hugi, \hld\ ne willjad iro drohtine hórjen
Israhelo erl-skępi, \hld\ un-gi-lóviga sind
hęliðos iro hérron: \hld\ þoh skal þanen helpe kumen
allun eli-þiodun.ʼ \hld\ Agaléto bad
þat wíf mid iro wordun, \hld\ þat iru waldand Krist
an is mód-sevon \hld\ mildi wurði,
þat siu iro barnes forð \hld\ brúkan mósti,
hębbjan sie héle. \hld\ Þó sprak iru hérro an-gegin,
mári endi mahtig: \hld\ ʽnis þatʼ, kwað he, ʽmannes reht,
gumono nigénum \hld\ gód te gi-frummjenne
þat he is barnun \hld\ bródes af-tíhe,
wernie im ovar willjon, \hld\ láte sie wíti þolean,
hungar heti-grimmen, \hld\ endi fódie is hundos mid þiu.ʼ
ʽwár is þat, waldandʼ, \hld\ kwað siu, ʽþat þu mid þínun wordun sprikis,
sóð-líko sagis: \hld\ hwat, þoh oft an sęli innen
undar iro hérron diske \hld\ hwelpos hwervad
brosmono fulle \hld\ þero fan þemu biode niðer
ant-fallat iro frójan.ʼ \hld\ Þó gi-hórde þat friðu-barn godes
willjan þes wíves \hld\ endi sprak iru mid is wordun tó:
ʽwela þat þu wíf haves \hld\ willjan góden!
Mikil is þín gi-lóvo \hld\ an þea maht godes,
an þene liudjo drohtin. \hld\ Al wirðid gi-léstid só
umbi þínes barnes líf, \hld\ só þu bádi te mi.ʼ
Þó warð siu sán gi-hélid, \hld\ só it þe hèlago ge-sprak
wordun wár-fastun: \hld\ þat wíf fagonode,
þes siu iro barnes forð \hld\ brúkan móste;
habde iru gi-holpen \hld\ héljando Krist,
habde sie far-fangane \hld\ fíundo kraftu,
wam-skaðun bi-werid. \hld\ Þó gi-wét imu waldand forð,
barno þat bętste, \hld\ sóhte imu burg óðre,
þiu só þikko was \hld\ mid þeru þiodu Judeono,
mid súðar-liudjun gi-seten. \hld\ Þar gi-fragn ik þat he is ge-síðos grótte,
þe jungaron þe he imu habde be is góde gi-korane, \hld\ þat sie mid imu gerno ge-wunodun,
weros þurh is wíson spráka: \hld\ ʽalle skal ik iuʼ, kwað he, ʽmid wordun frágon,
jungaron míne: \hld\ hwat kweðat þese Judeo liudi,
mári męgin-þioda, \hld\ hwat ik manno sí?ʼ
Imu and-wordidun fró-líko \hld\ is friund an-gegin,
jungaron síne: \hld\ ʽnis þit Judeono folk,
erlos én-wordje: \hld\ sum sagad þat þu Elias sís,
wís wár-sago, \hld\ þe hér giu was lango,
gód undar þesumu gum-skępje, \hld\ sum sagad þat þu Johannes sís,
diur-lík drohtines bodo, \hld\ þe hér dópte iu
werod an watere; \hld\ alle sie mid wordun sprekad,
þat þu én-hwi-lik sís \hld\ eðilero manno,
þero wár-sagono, \hld\ þe hér mid wordun giu
lérdun þese liudi, \hld\ endi þat þu sís eft an þit lioht kumen
te wísjanne þesumu werode.ʼ \hld\ Þó sprak eft waldand Krist:
ʽhwe kweðad gi, þat ik síʼ, \hld\ kwað he, ʽjungaron míne,
liovon liud-weros?ʼ \hld\ Þó te lat ni warð
Símon Petrus: \hld\ sprak sán an-gegin
éno for im allun \hld\ —habde imu ęlljen gód,
þrístea gi-þáhti, \hld\ was is þeodone hold—:
ʽþu bist þe wáro \hld\ waldandes sunu,
libbjendes godes, \hld\ þe þit lioht gi-skóp,
Krist kuning éwig: \hld\ só willjad wi kweðen alle,
jungaron þíne, \hld\ þat þu sís god selvo,
héljandero bętst.ʼ \hld\ Þó sprak imu eft is hérro an-gegin:
ʽsálig bist þu Símonʼ, \hld\ kwað he, ʽsunu Jonases; ni mahtes þu þat selvo ge-huggjan,
gi-markon an þínun mód-gi-þáhtiun, \hld\ ne it ni mahte þi mannes tunge
wordun ge-wísjen, \hld\ ak dede it þi waldand selvo,
fader allaro firiho barno, \hld\ þat þu só forð gi-spráki,
só diapo bi drohtin þínen. \hld\ Diur-líko skalt þu þes lòn ant-fáhen,
hluttro havas þu an þínan hérron gi-lóvon, \hld\ hugi-skęfti sind þíne sténe ge-líka,
só fast bist þu só felis þe hardo; \hld\ héten skulun þi firiho barn
sankte Péter: \hld\ ovar þemu sténe skal man mínen sęli wirkjan,
hèlag hús godes; \hld\ þar skal is híwiski tó
sálig samnon: \hld\ ni mugun wið þem þínun swíðeun krafte
an-þebbien hęllje portun. \hld\ Ik far-givu þi himil-ríkjas slutilas,
þat þu móst aftar mi \hld\ allun gi-waldan
kristinum folke; \hld\ kumad alle te þi
gumono géstos; \hld\ þu have gróte gi-wald,
hwene þu hér an erðu \hld\ eldi-barno
ge-binden willjes: \hld\ þemu is béðju gi-duan,
himil-ríki biloken, \hld\ endi hęllje sind imu opana,
brinnandi fiur; \hld\ só hwene só þu eft ant-binden wili,
an-þeftien is hendi, \hld\ þemu is himil-ríki,
ant-loken liohto mést \hld\ endi líf éwig,
gróni godes wang. \hld\ Mid su-likaru ik þi gevu willju
lònon þínen gi-lóvon. \hld\ Ni willju ik, þat gi þesun liudjun noh,
márjen þesaru menigi, \hld\ þat ik bium mahtig Krist,
godes égan barn. \hld\ Mi skulun Judeon noh,
un-skuldigna \hld\ erlos binden,
wégjan mi te wundrun \hld\ —dót mi wítjes filo—
innan Hierusalem \hld\ géres ordun,
áhtien mínes aldres \hld\ ęggjun skarpun,
bi-lósjen mi lívu. \hld\ Ik an þesumu liohte skal
þurh úses drohtines kraft \hld\ fan dóde a-standen
an þriddjumu dageʼ. \hld\ Þó warð þegno bętst
swíðo an sorgun, \hld\ Símon Petrus,
warð imu hugi hriwig, \hld\ endi te is hérron sprak
rink an rúnun: \hld\ ʽni skal þat ríki godʼ, kwað he,
ʽwaldand willjen, \hld\ þat þu eo su-lik wíti mikil
gi-þolos undar þesaru þiod: \hld\ nis þes þarf nigiean,
hèlag drohtin.ʼ \hld\ Þó sprak imu eft is hérro an-gegin,
mári mahtig Krist \hld\ —was imu an is móde hold—:
ʽhwat, þu nu wiðer-ward bistʼ, \hld\ kwað he, ʽwilljon mínes,
þegno bętsto! \hld\ Hwat, þu þesaro þiodo kanst
męnniskan sidu: \hld\ þu ni wést þe maht godes,
þe ik gi-frummjen skal. \hld\ Ik mag þi filu sęggjan
wárun wordun, \hld\ þar hér undar þesumu werode standad
ge-síðos míne, \hld\ þea ni mótun swelten ér,
hwerven an hinen-fard \hld\ ér sie himiles lioht,
godes ríki sehat.ʼ \hld\ Kòs imu jungarono þó
sán aftar þiu \hld\ Símon Petrus,
Jakob endi Johannes, \hld\ ea gumon twéne,
béðja þea gi-bróðer, \hld\ endi imu þó uppen þene berg gi-wét
sunder mid þem ge-síðun, \hld\ sálig barn godes,
mid þem þegnun þrim, \hld\ þiodo drohtin,
waldand þesaro weroldes: \hld\ welde im þar wundres filu,
tékno tógjan, \hld\ þat sie gi-trúodin þiu bet,
þat he selvo was \hld\ sunu drohtines,
hèlag heven-kuning. \hld\ Þó sie an hòhan wall
stigun stén endi berg, \hld\ antat sie te þeru stędi kwámun,
weros wiðer wolkan, \hld\ þar waldand Krist,
kuningo kraftigost \hld\ gi-koren habde,
þat he is god-kundi \hld\ jungarun sínun
þurh is énes kraft \hld\ ógean welde,
berht-lík biliði. \hld\ Þó imu þar te bedu gi-hnég,
þó warð imu þar uppe \hld\ óðar-líkora
wliti endi gi-wádi: \hld\ wurðun imu is wangun liohte,
blíkandi só þiu berhte sunne: \hld\ só skén þat barn godes,
liuhte is lík-hamo: \hld\ liomon stódun
wánamo fan þemu waldandes barne; \hld\ warð is ge-wádi só hwít
só snéu te sehanne. \hld\ Þó warð þar seld-lík þing
giógid aftar þiu: \hld\ Elias endi Moyses
kwámun þar te Kriste \hld\ wið só kraftagne
wordun wehsljan. \hld\ Þar warð só wun-sam spráka,
só gód word undar gumun, \hld\ þar þe godes sunu
wið þea márjan man \hld\ mahlien welde,
só blíði warð uppan þemu berge: \hld\ skén þat berhte lioht,
was þar gard gód-lík \hld\ endi gróni wang,
paradise ge-lík. \hld\ Petrus þó gi-mahalde,
hęlið hard-módig \hld\ endi te is hérron sprak,
grótte þene godes sunu: \hld\ ʽgód is it hér te wesanne,
ef þu it gi-kiosan wili, \hld\ Krist alo-waldo,
þat man þi hér an þesaru hòhe \hld\ én hús ge-wirkja,
már-líko ge-mako \hld\ endi Moysese óðer
endi Eliase þriddja: \hld\ þit is ódas hém,
welono wun-samost.ʼ \hld\ Reht só he þó þat word ge-sprak,
só tilét þiu luft an twé: \hld\ lioht wolkan skén,
glítandi glímo, \hld\ endi þea gódun man
wliti-skóni bewarp. \hld\ Þó fan þemu wolkne kwam
hèlag stemne godes, \hld\ endi þem hęliðun þar
selvo sagde, \hld\ þat þat is sunu wári,
libbiendero liovost: \hld\ ʽan þemu mi líkod wel
an mínun hugi-skęftjun. \hld\ Þemu gi hórjen skulun,
ful-gangad imu gerno.ʼ \hld\ Þó ni mahtun þea jungaron Kristes
þes wolknes wliti endi word godes,
þea is mikilon maht þea man ant-standen,
ak sie bi-fellun þó forð-wardes: ferhes ni wándun,
lengiron líves. Þó geng im tó þe landes ward,
be-hrén sie mid is handun héljandero bętst,
hét þat sie im ni and-rédin: ʽni skal iu hér derien eowiht,
þes gi hér seld-líkes gi-sehen habbjad,
mériaro þingo.ʼ Þó eft þem mannun warð
hugi at iro herton endi gi-hélid mód,
gi-bade an iro breostun: gi-sáhun þat barn godes
énna standen, was þat oðer þó,
be-hliden himiles lioht. Þó gi-wét imu þe hèlago Krist
fan þemu berge niðer; gi-bód aftar þiu
jungarun sínun, þat sie ovar Judeono folk
ni sagdin þea gi-sioni: ʽer þan ik selvo hér
swíðo diur-líko fan dóðe a-stande,
aríse fan þeru restu: síðor mugun gi it rekkien forð,
márjen ovar middil-gard managun þiodun
wído aftar þesaru weroldi.ʼ \hld\ Þó gi-wét imu waldand Krist
eft an Galileo land, sóhte is gadulingos,
mahtig is mágo hém, sagde þar manages hwat
berhtero biliðeo, endi þat barn godes
þem is sáligun ge-síðun sorgspell ni forhal,
ak he im openlíko allun sagde,
þem is gódun jungarun, hwó ine skolde þat Judeono folk
wégjan te wundrun. Þes wurðun þar wíse man
swíðo an sorgun, warð im sér hugi,
hriwig umbi iro herte: gi-hórdun iro hérron þó,
waldandes sunu wordun tęlljen,
hwat he undar þeru þiodu þolojan skolde,
willjendi undar þemu werode. Þó gi-wét imu waldand Krist,
gumo fan Galilea, sóhte imu Judeono burg,
kwámun im te kafarnaum. Þar fundun sie énan kuninges þegan
wlankan undar þemu werode: kwað þat he wári gi-weldig bodo
aðalkésures; he grótte aftar þiu
Símon Petrusen, kwað þat he wári gi-sęndid þarod,
þat he þar gi-manodi manno ge-hwi-liken
þero hòvid-skatto, þe sie te þemu hove skoldin
tinsi gelden: ʽnis þes tweho énig
gumono nigiénumu, ne sie ina far-gelden sán
méðmo kusteon, bi-úten iuwe méster éno
havad it far-láten. Ni skal þat líkon wel
mínumu hérron, só man it imu at is hove kúðid,
aðalkésure.ʼ Þó geng aftar þiu
Símon Petrus, welde it sęggjan þó
hérron sínumu: he was is an is hugi iu þan,
gi-waro waldand Krist: \hld\ —imu ni mahte word énig
bi-holen werðen, he wisse hugi-skęfti
manno ge-hwi-likes—: hét þó þene is márjan þegan,
Símon Petrus an þene séo innen
angul werpen: ʽsu-liken só þu þar érist mugis
fisk gi-fáhenʼ, kwað he, ʽsó teoh þu þene fan þemu flóde te þi,
ant-klemmi imu þea kinni: þar maht þu undar þem kaflon nimen
guldine skattos, þat þu far-gelden maht
þemu manne te gi-módja mínen endi þínen
tinseo só hwi-likan, \hld\ só he ús tó sókid.ʼ
He ni þorfte imu þó aftar þiu \hld\ óðaru wordu
furður gi-bioden: \hld\ geng fiskari gód,
Símon Petrus, \hld\ warp an þene séo innen
angul an úðjon \hld\ endi up gi-tóh
fisk an flóde \hld\ mid is folmun twém,
te-klóf imu þea kinni \hld\ endi undar þem kaflun nam
guldine skattos: \hld\ dede al, só imu þe godes sunu
wordun ge-wísde. \hld\ Þar was þó waldandes
męgin-kraft gi-márid, \hld\ hwó skal allaro manno ge-hwi-lik
swíðo willjendi \hld\ is werold-hérron
skuldi endi skattos, \hld\ þea imu gi-skeride sind,
gerno gelden: \hld\ ni skal ine far-gúmon eowiht,
ni far-muni ine an is móde, \hld\ ak wese imu mildi an is hugi,
þiono imu þio-líko: \hld\ an þiu mag he þiodgodes
willjan ge-wirkjan \hld\ endi ók is werold-hérron
huldi habbjen. \hld\ Só lérde þe hèlago Krist
þea is gódon jungaron: \hld\ ʽef énig gumono wið iuʼ, kwað he,
ʽsundja ge-wirkja, \hld\ þan nim þu ina sundar te þi,
þene rink an rúna \hld\ endi imu is rád saga,
wísi imu mid wordun. \hld\ Ef imu þan þes werð ne sí,
þat he þi gi-hórje, \hld\ hala þi þar óðara tó
gódaro gumono, \hld\ endi lah imu is grimmun werk,
sak ina sóð-wordun. \hld\ Ef imu þan is sundja aftar þiu,
lós-werk ni léðon, \hld\ gi-duo it óðrun liudjun kúð,
mári it þan for menegi \hld\ endi lát manno filu
witen is far-wurhti: \hld\ óðo be-ginnad imu þan is werk tregan,
an is hugi hrewen, \hld\ þan he it gi-hórid hęliðo filu,
ahton eldi-barn \hld\ endi imu is uvilon dád
węrjad mid wordun. \hld\ Ef he þan ók węndjen ne wili,
ak far-módat su-lika menegi, \hld\ þan lát þu þene man faren,
hava ina þan far héðinen \hld\ endi lát ina þi an þínumu hugi léðen,
míð is an þínumu móde, \hld\ ne sí þat imu eft mildi god,
hér heven-kuning \hld\ helpe far-líhe,
fader allaro firiho barno.ʼ \hld\ Þó frágode Petrus,
allaro þegno bętst \hld\ þeodan sínan:
ʽhwó oft skal ik þem mannun, \hld\ þe wið mi habbjad
léð-werk gi-duan, \hld\ leovo drohtin,
skal ik im sivun síðun \hld\ iro sundja a-láten,
wréðaro werko, \hld\ ér þan ik is éniga wréka frummje,
léðes te lòne?ʼ \hld\ Þó sprak eft þe landes ward,
an-gegin þe godes sunu \hld\ gódumu þegne:
ʽni sęggju ik þi fan sivunjun, \hld\ só þu selvo sprikis,
mahlis mid þínu múðu, \hld\ ik duom þi méra þar tó:
sivun síðun sivuntig \hld\ só skalt þu sundja ge-hwemu,
léðes a-láten: \hld\ só willju ik þi te lérun geven
wordun wár-fastun. \hld\ Nu ik þi su-lika gi-wald far-gaf,
þat þu mínes híwiskes \hld\ hérost wáris,
manages mann-kunnjes, \hld\ nu skalt þu im mildi wesen,
liudjun líði.ʼ \hld\ Þó þar te þemu lérjande kwam
én jung man an-gegin \hld\ endi frágode Jesu Krist:
ʽméster þe gódoʼ, \hld\ kwað he, ʽhwat skal ik manages duan,
an þiu þe ik heven-ríki \hld\ ge-halan móti?ʼ
Habde imu òd-welon \hld\ allen ge-wunnen,
mèðom-hord manag, \hld\ þoh he mildjan hugi
bári an is breostun. \hld\ Þó sprak imu þat barn godes:
ʽhwat kwiðis þu umbi gódon? \hld\ nis þat gumono énig
bi-útan þe éno, \hld\ þe þar al ge-skóp,
werold endi wunnja. \hld\ Ef þu is willjan havas,
þat þu an lioht godes \hld\ líðan mótis,
þan skalt þu bi-halden \hld\ þea hèlagon léra,
þe þar an þemu aldon \hld\ éwa ge-biudid,
þat þu man ni slah, \hld\ ni þu ménes ni sweri,
far-legar-nessi far-lát \hld\ endi luggi ge-wit-skępi,
stríd endi stulina; \hld\ ne wis þu te stark an hugi,
ne níðin ne hatul, \hld\ ni nòd-róf ni fremi;
avunst alla far-lát; \hld\ wis þínun eldirun gód,
fader endi móder, \hld\ endi þínun friundun hold,
þem náhistun gi-náðig. \hld\ Þan þu þi gi-niodon móst
himilo ríkjas, \hld\ ef þu it bi-halden wili,
ful-gangan godes lérun.ʼ \hld\ Þó sprak eft þe jungo man
ʽal hębbju ik só gi-léstidʼ, \hld\ kwað he, ʽsó þu mi léris nu,
wordun wísis, \hld\ só ik is eo wiht ni far-lét
fan mínero kindiski.ʼ \hld\ Þó bi-gan ina Krist sehan
an mid is ógun: \hld\ ʽén is þar noh nuʼ, kwað he,
ʽwan þero werko: \hld\ ef þu is willjon havas,
þat þu þurh-fremid \hld\ þionon mótis
hérron þínumu, \hld\ þan skalt þu þat þín hord nimen,
skalt þínan òd-welon \hld\ allan far-kópjen,
diurje méðmos, \hld\ endi déljen hét
armun mannun: \hld\ þan havas þu aftar þiu
hord an himile; \hld\ kum þi þan gi-halden te mi,
folgo þi mínaro ferdi: \hld\ þan havas þu friðu síður.ʼ
Þó wurðun Kristes word \hld\ kind-jungumu manne
swíðo an sorgun, \hld\ was imu sér hugi,
mód umbi herte: \hld\ habde méðmo filu,
welono ge-wunnen; \hld\ wende imu eft þanen,
was imu unóðo \hld\ innan breostun,
an is sevon swáro. \hld\ Sah imu aftar þó
Krist alo-waldo, \hld\ kwað it þó, þar he welde,
te þem is jungarun gegin-wardun, \hld\ þat wári an godes ríki
un-óði ódagumu manne \hld\ up te kumanne:
ʽóður mag man olbundeon, \hld\ þoh he sí un-met grót,
þurh náðlan gat, \hld\ þoh it sí naru swíðo,
sáftur þurh-slópien, \hld\ þan mugi kuman þiu siole te himile
þes ódagan mannes, \hld\ þe hér al havad
gi-wendid an þene werold-skat \hld\ willjon sínen,
mód-gi-þáhti, \hld\ endi ni hugid umbi þie maht godes.ʼ
Imu and-wordiade \hld\ ér-þungan gumo,
Símon Petrus, \hld\ endi sęggjan bad
leovan hérron: \hld\ ʽhwat skulun wi þes te lòne nimenʼ, kwað he,
ʽgódes te gelde, \hld\ þes wi þurh þín jungar-dóm
égan endi ervi \hld\ al far-létun
hovos endi híwiski \hld\ endi þi te hérron gi-kurun,
folgodun þínaru ferdi: \hld\ hwat skal ús þes te frumu werðen,
langes te lòne?ʼ \hld\ liudjo drohtin
sagde im þó selvo: \hld\ ʽþan ik sittjen kumuʼ, kwað he,
ʽan þie mikilan maht \hld\ an þemu márjan dage,
þar ik allun skal \hld\ irmin-þiodun
dómos a-déljen, \hld\ þan mótun gi mid iuwomu drohtine þar
selvon sittjen \hld\ endi mótun þera saka waldan:
mótun gi Israhelo \hld\ eðili-folkun
a-déljen aftar iro dádjun: \hld\ só mótun gi þar gi-diuride wesen.
Þan sęggju ik iu te wáran: \hld\ só hwe só þat an þesaru weroldi gi-duot,
þat he þurh mína minnja \hld\ mágo ge-sidli
liof far-létid, \hld\ þes skal hi hér lòn niman
tehan síðun tehin-fald, \hld\ ef he it mid trewon duot,
mid hluttru hugi. \hld\ Ovar þat havad he ók himiles lioht,
open éwig líf.ʼ \hld\ Bigan imu þó aftar þiu
allaro barno bętst \hld\ én biliði sęggjan,
kwað þat þar én ódag man \hld\ an ér-dagun
wári undar þemu werode: \hld\ ʽþe habde welono ge-nóg,
sinkas gi-samnod \hld\ endi imu simlun was
garu mid goldu \hld\ endi mid godo-wębbju,
fagarun fratahun \hld\ endi imu so filu habde
gódes an is gardun \hld\ endi imu at gómun sat
allaro dago ge-hwi-likes: \hld\ habde imu diur-lík líf,
blíðsea an is bęnkjun. \hld\ Þan was þar eft én biddjendi man,
gi-lévod an is lík-hamon, \hld\ Lazarus was he héten,
lag imu dago ge-hwi-likes \hld\ at þem durun foren,
þar he þene ódagan man \hld\ inne wisse
an is gęst-sęli \hld\ góme þiggjan,
sittjen at sumble, \hld\ endi he simlun béd
giarmod þar úte: \hld\ ni móste þar in kuman,
ne he ni mahte ge-biddjen, \hld\ þat man imu þes bródes þarod
gidragan weldi, \hld\ þes þar fan þemu diske niðer
ant-fel undar iro fóti: \hld\ ni mahte imu þar énig fruma werðen
fan þemu héroston, þe þes húses gi-weld, \hld\ bi-útan þat þar gengun is hundos tó,
likkodun is lík-wundon, \hld\ þar he liggiandi
hungar þolode; \hld\ ni kwam imu þar te helpu wiht
fan þemu ríkjon manne. \hld\ Þó gi-fragn ik þat ina is regano-gi-skapu,
þene armon man \hld\ is én-dago
gi-manoda mahtiun swíð, \hld\ þat he manno dròm
a-geven skolde. \hld\ Godes ęngilos
ant-fengun is ferh \hld\ endi léddun ine forð þanen,
þat sie an Abrahames barm \hld\ þes armon mannes
siole gi-settun: \hld\ þar móste he simlun forð
wesen an wunnjun. \hld\ Þó kwámun ók wurde-gi-skapu,
þemu ódagan man \hld\ or-lag-hwíle,
þat he þit lioht far-lét: \hld\ léða wihti
be-sinkodun is siole \hld\ an þene swarton hel,
an þat fern innen \hld\ fíundun te willjan,
be-gróvun ine an gramono hém. \hld\ Þanen mahte he þene gódan skawon,
Abraham ge-sehen, \hld\ þar he uppe was
líves an lustun, \hld\ endi Lazarus sat
blíði an is barme, \hld\ berht lòn ant-feng
allaro is armódio, \hld\ endi lag þe ódago man
héto an þeru hęllju, hriop up þanen:
ʽfader Abrahamʼ, kwað he, ʽmi is firinun þarf,
þat þu mi an þínumu mód-sevon mildi werðes,
líði an þesaru lognu: sendi mi Lazarus herod,
þat he mí ge-fórja an þit fern innan
kaldes wateres. Ik hér kwik brinnu
héto an þesaru hęllju: nu is mi þínaro helpono þarf,
þat he mi aleskie mid is luttikon fingru
tungon míne, nu siu tékạn havad,
uvil arvedi. Inwid-rádo,
léðaro spráka, alles is mi nu þes lòn kumen.ʼ
Imu and-wordiade þó Abraham —þat was aldfader—:
ʽge-hugi þu an þínumu hertonʼ, kwað he, ʽhwat þu habdes iu
welono an weroldi. Hwat, þu þar alle þíne wunnja farsliti,
gódes an gardun, só hwat só þi giviðig forð
werðen skolde. Wíti þolode
Lazarus an þemu liohte, \hld\ habde þar léðes filu,
wíteas an weroldi. \hld\ Be-þiu skal he nu welon égan,
libbien an lustun: \hld\ þu skalt þea logna þolan,
brinnendi fiur: \hld\ ni mag is þi énig bóte kumen
hinana te hęllju: \hld\ it havad þe hèlago god
só gi-fastnod mid is faðmun: \hld\ ni mag þar faren énig
þegno þurh þat þiustri: \hld\ it is hér só þikki undar ús.ʼ
Þó sprak eft Abrahame \hld\ þe erl te-gegnes
fan þeru hétan hęll \hld\ endi helpono bad,
þat he Lazarus \hld\ an liudjo dròm
selvon sandi: \hld\ ʽþat he ge-sęggja þar
bróðarun mínun, \hld\ hwó ik hér brinnendi
þrá-werk þolon; \hld\ si þar undar þeru þiodu sind,
si fívi undar þemu folke: ik an forhtun bium,
þat sie im þar far-wirkjen, þat sie skulin ók an þit wíti te mi,
an só grádag fiur.ʼ Þó imu eft te-gegnes sprak
Abraham aldfader, kwað þat sie þar éo godes
an þemu land-skępi, liudi habdin,
Moyseses gi-bód endi þar managaro tó
wár-saguno word: ʽef sie is willige sind,
þat sie þat bi-halden, þan ni þurvun sie an þea hęll innen,
an þat fern faren, ef sie ge-frummjad só,
só þea ge-biodad, þe þea bók lesat
þem liudjun te lérun. Ef sie þes þan ni willjad léstjen wiht,
þanne ni hórjad sie ók þemu þe hinan astád,
man fan dóðe. Láte man sie an iro mód-sevon
selvon keosen, hweðer im swótiera þunkie
te gi-winnanne, só lango só sie an þesaru weroldi sind,
þat sie eft uvil etþa gód aftar habbjen.ʼʼ
Só lérde he þó þea liudi liohton wordon,
allaro barno bętst, endi biliði sagde
manag man-kunnje mahtig drohtin,
kwað þat imu én sálig gumo samnon bi-gunni
man an morgen, ʽendi im méda gi-hét,
þe hérosto þes híwiskjas, swíðo *hold-lík lònʼ,
kwat þat hie iro allaro gi-hwem énna gávi
silovrinna skat. ʽÞuo samnodun managa
weros an is wín-gardon, \hld\ —endi hie im werk bi-falah—
ádro an úhtan. Sum kwam þar ók an undorn tuo,
sum kwam þar an middjan dag, man te þem werke,
sum kwam þar te nónu, þuo was þiu niguða tíd
sumar-langes dages; sum þar ók síðor kwam
an þia elliftun tíd. Þuo geng þar ávand tuo,
sunna ti sedle. Þuo hie selvo gi-bód
is ambahtion, erlo drohtin,
þat man þero manno gi-hwem \hld\ is meoda for-guldi,
þem erlon arvid-lòn; hiet þiem at érist gevan.
þia þar at letst wárun, \hld\ liudi kumana,
weros te þem werke, \hld\ endi mid is wordon gi-bód,
þat man þem mannon iro \hld\ mieda for-guldi
alles at aftan, \hld\ þem þar kwámun at érist tuo
willendi te þem werke. \hld\ Wándun sia swíðo,
þat man im méra lòn \hld\ gi-makod habdi
wið iro aravedie: \hld\ þan man im allon gaf,
þem liudjon gi-líko. \hld\ Léð was þat swíðo,
allon þem ando, \hld\ þem þar kwámun at érist tuo:
ʽwi kwámun hier an moraganʼ, \hld\ kwáðun sia, ʽendi þolodun hier manag te dage
aravid-werko, \hld\ hwílon unmet hét,
skínandia sunna: \hld\ nu ni givis þu ús skattes þan mér,
þie þu þem óðron duos, \hld\ þia hier éna hwíla
wáron an þínon werke.ʼ \hld\ Þuo habda eft is word garo
þie hérosto þes híwiskes, \hld\ kwat þat hie im ni habdi gi-hétan þan mér
werðes wið iro werke: \hld\ ʽhwat, ik gi-wald hębbjuʼ, kwaþie,
ʽþat ik iu allon gi-líko \hld\ muot lòn for-geldan,
iwes werkes werð.ʼ \hld\ Þan waldandi Krist
ménda im þoh méra þing, \hld\ þoh hie ovar þat manno folk
fan þem wín-gardon só \hld\ wordon spráki,
hwó þar unefno \hld\ erlos kwámun,
weros te þem werke. \hld\ Só skulun fan þero weroldi duon
mann-kunnjes barn \hld\ an þat márjo lioht,
gumon an godes wang: \hld\ sum bi-ginnit ina giriwan sán
an is kindiski, \hld\ havit im gi-koranan muod,
willjon guodan, \hld\ werold-saka míðit,
far-látit is lusta; \hld\ ni mag ina is lík-hamo
an un-spuod for-spanan: \hld\ spáhiða línot,
godes éu, \hld\ gramono for-látit,
wréðaro willjon, \hld\ duot im só te is weroldi forð,
léstit só an þeson liohte, \hld\ antþat im is líves kumit,
aldres ávand; \hld\ gi-wítit im þan up-wegos:
þar wirðit im is aravedi \hld\ all gi-lònot,
far-goldan mid guodu \hld\ an godes ríkje.
Þat méndun þia wuruhtjon, \hld\ þia an þem wín-gardon
ádro an úhta \hld\ arvid-líko
werk bi-gunnun \hld\ endi þuru-wonodun forð,
erlos unt ávand. \hld\ Sum þar ók an undern kwam,
habda þuo far-merrid, \hld\ þia moragan-stunda
þes dag-werkes for-duolon; \hld\ só duot doloro filo,
gi-médaro manno: \hld\ drívit im mis-lík þing
gerno an is iuguði, \hld\ —havit im gelp-kwidi
léða gi-línot \hld\ endi lós-word manag—,
antþat is kindiski \hld\ far-kuman wirðit,
þat ina after is iuguði \hld\ godes anst manot
blíði an is brioston; \hld\ fáhit im te bęteron þan
wordon endi werkon, \hld\ lédit im is werold mid þiu,
is aldar ant þena endi: \hld\ kumit im alles lòn
an godes ríkje, \hld\ gódaro werko.
Sum mann þan mid-firi \hld\ mén far-látid,
swára sundjun, \hld\ fáhit im an sálig þing,
biginnit im þuru godes kraft \hld\ guodaro werko,
buotit balo-spráka, \hld\ látit im is bittrun dád
an is hugje hrewan; \hld\ kumit im þiu helpa fon gode,
þat im gi-léstid þie gi-lóvo, \hld\ só lango só im is líf warod;
farit im forð mid þiu, \hld\ ant-fáhit is mieda,
guod lòn at gode; \hld\ ni sindun éniga geva bęteran.
Sum biginnit þan ók furðor, \hld\ þan hie ist fruodot mér,
is aldares af-heldit, \hld\ —þan bi-ginnat im is uvilon werk
léðon an þeson liohte, \hld\ þan ina léra godes
gi-manod an is muode: \hld\ wirðit im mildera hugi,
þuru-gengit im mid guodu \hld\ endi geld nimit,
hòh himil-ríki, \hld\ þan hie hinan wendit,
wirðit im is mieda só sama, \hld\ só þem man *nun warð,
þea þar te nónu dages, \hld\ an þea nigunda tíd,
an þene wín-gardon \hld\ wirkjan kwámun.
Sum wirðid þan só swíðo ge-fródot, \hld\ só he ni wili is sundja bótjen,
ak he ókid sie mid uvilu ge-hwi-liku, \hld\ antat imu is ávand náhid,
is werold endi is wunnja far-slítid; \hld\ þan be-ginnid he imu wíti and-réden,
is sundjon werðad imu sorga an móde: \hld\ ge-hugid hwat he selvo ge-frumide
grimmes þan lango, þe he móste is iuguðeo neoten; \hld\ ni mag þan mid óðru gódu gi-bótjen
þea dádi, þea he só dervea ge-frumide, \hld\ ak he slehit allaro dago ge-hwi-likes
an is breost mid béðjun handun \hld\ endi wópit sie mid bittrun trahnun,
hlúdo he sie mid hofnu kúmid, \hld\ bidid þene hèlagon drohtin
mahtigne, þat he imu mildi werðe: \hld\ ni látid imu síðor is mód gi-twíflien;
só é-groht-ful is, þe þar alles ge-weldid: \hld\ he ni wili énigumu irmin-manne
farwernien willjan sínes; \hld\ far-givid imu waldand selvo
hèlag himil-ríki: \hld\ þan is imu gi-holpen síður.
Alle skulun sie þar éra ant-fáhen, \hld\ þoh sie þarod te énaru tídi
ni kumen, þat kunni manno, \hld\ þoh wili imu þe kraftigo drohtin,
gi-lònon allaro liudjo só hwi-likumu, \hld\ só hér is gi-lóvon ant-fáhit:
én himil-ríki \hld\ givid he allun þeodun,
mannun te médu. \hld\ Þat ménde mahtig Krist,
barno þat bętste, \hld\ þó he þat biliði sprak,
hwó þar te þem wín-gardun \hld\ wurhtjon kwámin,
man mis-líko: \hld\ þoh nam is méde ge-hwe
fulle te is frójan. \hld\ Só skulun firiho barn
at gode selvumu \hld\ geld ant-fáhen,
swíðo leov-lík lòn, \hld\ þoh sie sume só late werðan.
Hét imu þó þea is gódan \hld\ jungaron náhor
twelivi gangan \hld\ —þea wárun imu triuwiston
man ovar erðu—, \hld\ sagde im mahtig selvo
óðer-síðu, \hld\ hwi-lik imu þar arvedi
tóward wárun: \hld\ ʽþes ni mag énig tweho werðenʼ, kwað he;
kwað þat sie þó te Hierusalem \hld\ an þat Judeono folk
líðan skoldin: \hld\ ʽþar wirðid all gi-léstid só,
ge-frumid undar þemu folke, \hld\ só it an furn-dagun
wíse man be mi \hld\ wordun ge-sprákun.
Þar skulun mi far-kópon \hld\ undar þea kraftigon þiod,
hęliðos te þeru héri; \hld\ þar werðat mína hendi ge-bundana,
faðmos werðad mi þar gefastnod; \hld\ filu skal ik þar gi-þolojan,
hoskes gi-hórjen \hld\ endi harm-kwidi,
bismerspráka \hld\ endi bi-hét-word manag;
sie wégjat mi te wundron \hld\ wápnes ęggjun,
bi-lósjad mi lívu: \hld\ ik te þesumu liohte skal
þurh drohtines kraft \hld\ fan dóðe a-standen
an þriddjon dage. \hld\ Ni kwam ik undar þesa þeoda herod
te þiu, þat mín eldi-barn \hld\ arved habdin,
þat mi þionodi þius þiod: \hld\ ni willju ik is sie þiggjen nu,
fergon þit folk-skępi, \hld\ ak ik skal imu te frumu werðen,
þeonon imu þeo-líko \hld\ endi for alla þesa þeoda geven
seole míne. \hld\ Ik willju sie selvo nu
lósjen mid mínu lívu, \hld\ þea hér lango bidun,
man-kunnjes manag, \hld\ mínara helpa.ʼ
Fór imu þó forð-wardes \hld\ —habde imu fasten hugi,
blíðean an is breostun \hld\ barn drohtines—
welda im te Hierusalem \hld\ Judeo folkes
willjon wísan: \hld\ he konste þes werodes só garo
heti-grimmen hugi \hld\ endi hardan stríd,
wréðan willjon. \hld\ Werod síðode
furi Hierikho-burg; \hld\ was þe godes sunu,
mahtig undar þero menigi. \hld\ Þar sátun twénje man bi wege,
blinde wárun sie béðje: \hld\ was im bótono þarf,
þat sie ge-héldi \hld\ hevenes waldand,
hwand sie só lango \hld\ liohtes þolodun,
managa hwíla. \hld\ Sie gi-hórdun þó þat męgin faren
endi frágodun sán \hld\ firi-wit-líko
ręgini-blindun, \hld\ hwi-lik þar ríki man
undar þemu folk-skępi \hld\ furista wári,
hérost an hòvid. \hld\ Þó sprak im én hęlið an-gegin,
kwað þat þar Hiesu Krist \hld\ fan Galilea-lande,
héljandero bętst \hld\ hérost wári,
fóri mid is folku. \hld\ Þó warð fráh-mód hugi
béðjun þem blindun mannun, \hld\ þó sie þat barn godes
wissun under þemu werode: \hld\ hreopun im þó mid iro wordun tó,
hlúdo te þemu hèlagon Kriste, \hld\ bádun þat he im helpe gerédi:
ʽdrohtin Dawides sunu: \hld\ wis ús mid þínun dádjun mildi,
neri ús af þesaru nòdi, \hld\ só þu gi-nóge dós
manno kunnjes: \hld\ þu bist managun gód,
hilpis endi hélis.ʼ \hld\ Þo bi-gan im þat hęliðo folk
werien mid wordun, \hld\ þat sie an waldand Krist
só hlúdo ni hriopin. \hld\ Si ni weldun im hórjen te þiu,
ak sie simla mér endi mér \hld\ ovar þat manno folk
hlúdo hreopun. \hld\ Héljand ge-stód,
allaro barno bętst, \hld\ hét sie þó brengien te imu,
lédjen þurh þea liudi, \hld\ sprak im listjun tó
mild-líko for þeru menegi: \hld\ ʽhwat willjad git mínaro hérʼ, kwað he,
ʽhelpono habbjen?ʼ \hld\ Sie bádun ina hèlagna,
þat he im ira ógon \hld\ opana gi-dádi,
far-liwi þeses liohtes, \hld\ þat sie liudjo dròm,
swigle sunnun skín \hld\ gi-sehen móstin,
wliti-skónje werold. \hld\ Waldand frumide,
hrén sie þó mid is handun, \hld\ dede is helpe þar tó,
þat þem blindun þó \hld\ béðjum wurðun
ógon gi-oponod, \hld\ þat sie erðe endi himil
þurh kraft godes \hld\ ant-kiennien mahtun,
lioht endi liudi. \hld\ Þó sagdun sie lof gode,
diurdun úsan drohtin, \hld\ þes sie dages liohtes
brúkan móstun: \hld\ ge-witun im béðje mid imu,
folgodun is ferdi: \hld\ was im þiu fruma giviðig,
endi ók waldandes werk \hld\ wído ge-kúðid,
managun gi-márid. \hld\ Þar was só mahtiglík
biliði gi-bóknid, \hld\ þar þe blindon man
bi þemu wege sátun, \hld\ wíti þolodun,
liohtes lóse: \hld\ þat ménid þoh liudjo barn,
al man-kunni, \hld\ hwó sie mahtig god
an þemu ana-ginne \hld\ þurh is énes kraft
sinhíun twé \hld\ selvo gi-warhte,
Ádam endi Éwan: \hld\ far-gaf im up-wegos,
himilo ríki; \hld\ ak þó warð im þe hatola te náh,
fíund mid féknu \hld\ endi mid firin-werkun,
bi-swék sie mid sundjun, \hld\ þat sie sin-skóni,
lioht far-létun: \hld\ wurðun an léðaron stędi,
an þesen middil-gard \hld\ man far-worpen,
þolodun hér an þiustriu \hld\ þiod-arvedi,
wunnun wrak-síðos, \hld\ welon þarvodun:
far-gátun godes ríkjes, \hld\ gramon þeonodun,
fíundo barnun; \hld\ sie guldun is im mid fiuru lòn
an þeru héton hęllju. \hld\ Be-þiu wárun siu an iro hugi blinda
an þesaru middil-gard, \hld\ męnniskono barn,
hwand siu ine ni ant-kiendun, \hld\ kraftagne god,
himilisken hérron, \hld\ þene þe sie mid is handun gi-skóp,
gi-warhte an is willjon. \hld\ Þius werold was þó só far-hwervid,
bi-þwungen an þiustrje, \hld\ an þiod-arvidi,
an dóðes dalu: \hld\ sátun im þó bi þeru drohtines strátun
iámar-móde, \hld\ godes helpe bidun:
siu ni mahte im þó ér werðen, \hld\ ér þan waldand god
an þesan middil-gard, \hld\ mahtig drohtin,
is selves sunu \hld\ sęndjen weldi
þat he lioht ant-luki \hld\ liudjo barnun,
oponodi im éwig líf, \hld\ þat sie þene alo-waldon
mahtin ant-kęnnjen wel, \hld\ kraftagna god.
Ók mag ik giu gi-tęlljen, \hld\ of gi þar tó willjad
huggjen endi hórjen, \hld\ þat gi þes héljandes mugun
kraft ant-kęnnjen, \hld\ hwó is kumi wurðun
an þesaru middil-gard \hld\ managun te helpu,
ia hwat he mid þem dádjun \hld\ drohtin selvo
manages ménde, \hld\ ia be-hwiu þiu márje burg
Hierikho hétid, \hld\ þiu þar an Judeon stád
gi-makod mid múrun: \hld\ þiu is aftar þemu mánen gi-nemnid,
aftar þemu torhten tungle: \hld\ he ni mag is tídi be-míðen,
ak he dago ge-hwi-likes \hld\ duod óðer-hweðer,
wanod ohþo wahsid. \hld\ Só dód an þesaro weroldi hér,
an þesaru middil-gard \hld\ męnniskono barn:
farad endi folgod, \hld\ fróde stervad,
werðad eft junga \hld\ aftar kumane,
weros a-wahsane, \hld\ unttat sie eft wurd far-nimid.
Þat ménde þat barn godes, \hld\ þó he fon þeru burgi fór,
þe gódo fan Hierikho, \hld\ þat ni mahte ér werðen gumono barnun
þiu blindja gi-bótid, \hld\ þat sie þat berhte lioht,
gi-sáhin sin-skóni, \hld\ ér þan he selvo hér
an þesaru middil-gard \hld\ męnniski ant-feng,
flésk endi lík-hamon. \hld\ Þó wurðun þes firiho barn
gi-war an þesaru weroldi, \hld\ þe hér an wítje ér,
sátun an sundjun \hld\ gi-siunies lóse,
þolodun an þiustrie, \hld\ —sie af-sóvun þat was þesaru þiod kuman
héljand te helpu \hld\ fan heven-ríkje,
Krist allaro kuningo best; \hld\ sie mahtun is ant-kęnnjen sán,
gi-fólien is fardio. \hld\ Þó sie só filu hriopun,
þe man te þemu mahtigon gode, \hld\ þat im mildi aftar þiu
waldand wurði. \hld\ Þan weridun im swíðo
þia swárun sundjon, \hld\ þe sie im ér selvon gi-dádun,
lettun sie þes gi-lóbon. \hld\ Sie ni mahtun þem liudjun þoh
bi-werjen iro willjon, \hld\ ak sie an waldand god
hlúdo hriopun, \hld\ antat he im iro héli far-gaf,
þat sie sin-líf \hld\ gi-sehen móstin,
open éwig lioht \hld\ endi an faren
an þiu berhtun bú. \hld\ Þat méndun þea blindun man,
þe þar bi Hierikho-burg \hld\ te þemu godes barne
hlúdo hriopun, \hld\ þat he im iro héli far-lihi,
liohtes an þesumu líve: \hld\ þan im þea liudi só filu
weridun mid wordun, \hld\ þea þar an þemu wege fórun
bi-foren endi bi-hinden: \hld\ só dót þea firin-sundjon
an þesaru middil-gard man\hld\ -kunnje.
hórjad nu hwó þie blindun, \hld\ síður im gi-bótid warð,
þat sie sunnun lioht \hld\ ge-sehen móstun,
hwó si þó dádun: \hld\ ge-witun im mid iro drohtine samad,
folgodun is ferdi, \hld\ sprákun filu wordo
þemu landes hirdie te love: \hld\ só dód im noh liudjo barn
wído aftar þesaru weroldi, \hld\ síður im waldand Krist
ge-liuhte mid is lérun \hld\ endi im líf éwig,
godes ríki far-gaf \hld\ gódun mannun,
hòh himiles lioht \hld\ endi is helpe þar tó,
só hwemu só þat gi-werkod, \hld\ þat he móti þemu is wege folgon.
Þó náhide \hld\ nęrjendo Krist,
þe gódo te Hierusalem. \hld\ Kwam imu þar te-gegnes filu
werodes an willjon \hld\ wel huggendies,
ant-fengun ina fagaro \hld\ endi imu bi-foren streidun
þene weg mid iro gi-wádjun \hld\ endi mid wurtjun só same,
mid berhtun blómun \hld\ endi mid bómo tógun,
þat feld mid fagaron palmun, \hld\ al só is fard ge-buride,
þat þe godes sunu \hld\ gangan welde
te þeru márjan burg. \hld\ Hwarf ina męgin umbi
liudjo an lustun, \hld\ endi lof-sang a-hóf
þat werod an willjon: \hld\ sagdun waldande þank,
þes þar selvo kwam \hld\ sunu Dawides
wíson þes werodes. \hld\ Þó gesah waldand Krist
þe gódo te Hierusalem, \hld\ gumono bętsta,
blíkan þene burges wal \hld\ endi bú Judeono,
hòha horn-sęli \hld\ endi ók þat hús godes,
allaro wího wun-samost. \hld\ Þó wel imu an innen
hugi wið is herte: \hld\ þó ni mahte þat hèlage barn
wópu awísien, \hld\ sprak þó wordo filu
hriwig-líko \hld\ —was imu is hugi séreg—:
ʽwé warð þi, Hierusalemʼ, \hld\ kwað he, ʽþes þu te wárun ni wést
þea wurde-gi-skęfti, \hld\ þe þi noh gi-werðen skulun,
hwó þu noh wirðis be-habd \hld\ hęrjes kraftu
endi þi bi-sittjad \hld\ slíð-móde man,
fíund mid folkun. \hld\ Þan ni havas þu friðu hwergin,
mundburd mid mannun: \hld\ lédjad þi hér manage tó
ordos endi ęggja, \hld\ orlegas word,
farfioþ þín folk-skępi \hld\ fiures liomon,
þese wíki awóstiad, \hld\ wallos hòha
felliad te foldun: \hld\ ni afstád is felis nigiean,
stén ovar óðrumu, \hld\ ak werðad þesa stędi wóstia
umbi Hierusalem \hld\ Judeo liudjo,
hwand sie ni ant-kęnnjad, \hld\ þat im kumana sind
iro tídi tówardes, \hld\ ak sie habbjad im twíflien hugi,
ni witun þat iro wísad \hld\ waldandes kraft.ʼ
Gi-wét imu þó mid þeru menegi \hld\ manno drohtin
an þea berhton burg. \hld\ Só þó þat barn godes
innan Hierusalem \hld\ mid þiu gumono folku,
ség mid þiu ge-síðu, \hld\ þó warð þar allaro sango mést,
hlúd stemnie af-haven \hld\ hèlagun wordun,
lovodun þene landes ward \hld\ liudjo menegi,
barno þat bętste; \hld\ þiu burg warð an hróru,
þat folk warð an forhtun \hld\ endi frágodun sán,
hwe þat wári, \hld\ þat þar mid þiu werodu kwam,
mid þeru mikilon menegi. \hld\ Þó sprak im én man an-gegin,
kwað þat þar Hiesu Krist \hld\ fan Galileo lande,
fan Nazareth-burg \hld\ nęrjand kwámi,
witig wár-sago \hld\ þemu werode te helpu.
Þó was þem Judiun, \hld\ þe imu ér grame wárun,
un-holde an hugi, \hld\ harm an móde,
þat imu þea liudi só filu \hld\ lof-sang warhtun,
diurdun iro drohtin. \hld\ Þó gengun dolmóde,
þat sie wið waldand Krist \hld\ wordun sprákun,
bádun þat he þat ge-síði \hld\ swígon héti,
letti þea liudi, \hld\ þat sie imu lof só filu
wordun ni warhtin: \hld\ ʽit is þesumu werode léðʼ, kwáðun sie,
ʽþesun burgliudjun.ʼ \hld\ Þó sprak eft þat barn godes:
ʽef gi sie amerriadʼ, \hld\ kwað he, ʽþat hér ni mótin manno barn
waldandes kraft \hld\ wordun diurjen,
þan skulun it hrópen þoh \hld\ harde sténos
for þesumu folk-skępi, \hld\ felisos starka,
ér þan it eo belíve, \hld\ nevo man is lof spreke
wído aftar þesaru weroldi.ʼ \hld\ Þó he an þene wíh innen,
geng an þat godes hús: \hld\ fand þar Judeono filu,
mislíke man, \hld\ manage at-samne,
þea im þar kóp-stędi \hld\ gi-koran habdun,
mangodun im þar mid manages hwí: \hld\ muniterias sátun
an þemu wíhe innan, \hld\ habdun iro wesl gidago
garu te gevanne. \hld\ Þat was þemu godes barne
al an andun: \hld\ dréf sie ut þanen
rúmo fan þemu rakude, \hld\ kwað þat wári rehtara dád,
þat þar te bedu fórin \hld\ barn Israheles
ʽendi an þesumu mínumu húse \hld\ helpono biddjan,
þat sia sigi-drohtin \hld\ sundjono tuomie,
þan hér þeovas \hld\ an þing-stędi halden,
þea far-warhton weros \hld\ wehsal drívan,
un-reht énfald. \hld\ Ne gi éniga éra ni witun
þeses godes húses, \hld\ Judeo liudi.ʼ
Só rúmde he þó endi rekode, \hld\ ríki drohtin,
þat hèlaga hús \hld\ endi an helpun was
managumu man-kunnje, \hld\ þem þe is mikilon kraft
ferrene ge-frugnun \hld\ endi þar gi-faran kwámun
ovar langan weg. \hld\ Warð þar léf so manag,
halt gi-hélid \hld\ endi háf só same,
blindun gi-bótid. \hld\ Só dede þat barn godes
willjendi þemu werode, \hld\ hwand al an is gi-weldi stéd
umbi þesaro liudjo líf \hld\ endi ók umbi þit land só same.
Stód imu þó fora þemu wíhe \hld\ waldandeo Krist,
liof landes ward, \hld\ endi imu þero liudjo hugi,
iro willjon aftar-warode: \hld\ gi-sah werod mikil
an þat márje hús \hld\ méðmos fórjen,
gevon mid goldu \hld\ endi mid godu-wębbju,
diuriun fratahun. \hld\ Þat al drohtin Krist
warode wís-líko. \hld\ Þó kwam þar ók én widowa tó,
idis arm-skapen, \hld\ endi te þemu alaha geng
endi siu an þat tresur-hús \hld\ twéne legde
éríne skattos: \hld\ was iru énfald hugi,
willjan gódes. \hld\ Þó sprak waldand Krist,
þe gumo wið is giungaron, \hld\ kwað þat siu þar geva bráhti
méron mikilu þan elkor \hld\ énig mannes sunu:
ʽef hér ódaga manʼ, \hld\ kwað he, ʽéra bráhtun,
mèðom-hord manag, \hld\ sie létun im mér at hús
welona ge-wunnen. \hld\ Ni dede þius widowa só,
ak siu te þesumu alahe gaf \hld\ al þat siu habde
welono ge-wunnen, \hld\ só siu iru wiht ni far-lét
gódes an iro gardun. \hld\ Be-þiu sind ira geva méron,
waldande werða, \hld\ hwand siu it mid su-likumu willjon dede
te þesumu godes húse. \hld\ Þes skal siu geld niman,
swíðo lang-sam lòn, \hld\ þes siu su-likan gi-lóvon havad.ʼ
Só gi-fragn ik þat þar an þemu wíhe \hld\ waldandeo Krist
allaro dago ge-hwi-likes, \hld\ drohtin manno,
wísde mid wordun. \hld\ Stód ine werod umbi,
grót folk Judeono, \hld\ gi-hórdun is gódan word,
swótja sęggjan. \hld\ Sum só sálig warð
manno undar þeru menegi, \hld\ þat it bi-gan an is mód hladen;
línodun im þea léra, \hld\ þe þe landes ward
al be biliðjun sprak, \hld\ barn drohtines.
Sumun wárun eft so léða \hld\ léra Kristes,
waldandes word: \hld\ was im wiðer-mód hugi
allun þem, þe an þemu hęri-skępi \hld\ hérost wárun,
furiston an þemu folke: \hld\ fáres hugdun
wréða mid iro wordun \hld\ —habdun im wiðer-sakon
gi-haloden te helpu, \hld\ þes héroston man,
Erodeses þegan, \hld\ þe þar and-ward stód
wréðes willjan, \hld\ þat he iro word ovar-hórdi—
ef sie ina for-fengin, \hld\ þat sie ina þan feteros an,
þea liudi liðo-bendi \hld\ leggjen móstin,
sundja lósan. \hld\ Þó gengun im þea ge-síðos tó
bittra gi-hugde, \hld\ þat sie wið þat barn godes,
wréða wiðer-sakon \hld\ wordun sprákun:
ʽhwat, þu bist éosagoʼ, \hld\ kwáðun sie, ʽallun þiodun,
wísis wáres só filu: \hld\ nis þi werðeowiht
te bi-míðanne \hld\ manno niénumu
umbi is ríkidóm, \hld\ nevo þu simlun þat reht sprikis
endi an þene godes weg \hld\ gumono ge-síði
lédis mid þinun lérun: \hld\ ni mag þi laster man
fíðan undar þesumu folke. \hld\ Nu wi þi frágon skulun.
ríki þiodan, \hld\ hwi-lik reht havad
þe késur fan Rúmu, \hld\ þe imu te þesumu kunnje herod
tinsi sókid \hld\ endi gi-tald havad,
hwat wi imu gelden skulin \hld\ géro ge-hwi-likes
hòvid-skatto. \hld\ Saga hwat þi þes an þínumu hugi þunkja:
is it reht þe nis? \hld\ Rád for þínun
land-mégun wel: \hld\ ús is þínaro lérono þarf.ʼ
Sie weldun þat he it ant-kwáði: \hld\ þan mahte he þoh ant-kęnnjen wel
iro wréðon willjon: \hld\ ʽte hwí gi wár-logonʼ, kwað he,
ʽfandot mín só frókno? \hld\ Ni skal iu þat te frumu werðen,
þat gi dreogerias \hld\ darnungo nu
willjad mi farfáhen.ʼ \hld\ Hét he þó forð dragan
te skawonne þe skattos, \hld\ ʽþe gi skuldige sind
an þat geld geven.ʼ \hld\ Judeon drógun
énna siluvrinna forð: \hld\ sáhun manage tó,
hwó he was gemunitod: \hld\ was an middien skín
þes késures biliði \hld\ —þat mahtun sie ant-kęnnjen wel—,
iro hérron hòvid-mál. \hld\ Þó frágode sie þe hèlago Krist,
aftar hwemu þiu ge-líknessi \hld\ gi-legid wári.
Sie kwáðun þat it wári \hld\ werold-késures
fan Rúmu-burg, \hld\ ʽþes þe alles þeses ríkes havad
ge-wald an þesaru weroldi.ʼ \hld\ ʽÞan willju ik iu te wárun hérʼ, kwað he,
ʽselvo sęggjan, \hld\ þat gi imu sín gevad,
werold-hérron is ge-wunst, \hld\ endi waldand gode
sęlljad, þat þar sín ist: \hld\ þat skulun iuwa seolon wesen,
gumono géstos.ʼ \hld\ Þó warð þero Judeono hugi
geminsod an þemu mahle: \hld\ ni mahtun þe mén-skaðon
wordun ge-winnen, \hld\ só iro willjo geng,
þat sie ina far-fengin, \hld\ hwand imu þat friðu-barn godes
wardode wið þe wréðon \hld\ endi im wár an-gegin,
sóð-spel sagde, \hld\ þoh sie ni wárin só sálige te þiu,
þat sie it só far-fengin, \hld\ só it iro fruma wári.
Sie ni weldun it þoh far-láten, \hld\ ak hétun þar lédjen forð
én wíf for þemu werode, \hld\ þiu habde wam ge-frumid,
un-reht én-fald: \hld\ þiu idis was bi-fangen
an far-legar-nessi, \hld\ was iro líves skolo,
þat sie firiho barn \hld\ ferahu bi-námin,
éhtin iro aldres: \hld\ só was an iro éu ge-skriven.
Sie bi-gunnun ina þó frágon, \hld\ fruokne liudi,
wréða mid iro wordun, \hld\ hwat sie skoldin þemu wíve duan,
hweðer sie sie kwęlidin, \hld\ þe sie sie kwika létin,
þe hwat he umbi su-lika dádi \hld\ a-déljen weldi:
ʽþu wést, hwó þesaru menegiʼ, \hld\ kwáðun sie, ʽMoyses gi-bód
wárun wordun, \hld\ þat allaro wívo ge-hwi-lik
an far-legar-nessi \hld\ líves far-warhti
endi þat sie þan a-wurpin \hld\ weros mid handun,
starkun sténun: \hld\ nu maht þu sie sehan standen hér
an sundjun bi-fangan: \hld\ saga hwat þu is willjes.ʼ
weldun ine þea wiðer-sakon \hld\ wordun far-fáhen,
ef he þat gi-kwáði, \hld\ þat sie sie kwika létin,
friðodi ira ferahe, \hld\ þan weldi þat folk Judeono
kweðen, þat he iro aldiron \hld\ éo wiðer-sagdi,
þero liudjo land-reht; \hld\ ef he sie þan héti lívu bi-nimen,
þea magað fur þeru menegi, \hld\ þan weldin sie kweðen, þat he só mildjene hugi
ni bári an is breostun, \hld\ só skoldi habbjen barn godes:
weldun sie só hweðeres \hld\ hèlagne Krist
þero wordo ge-wítnon, \hld\ só he þar for þemu werode ge-spráki,
a-déldi te dóme. \hld\ Þan wisse drohtin Krist
þero manno só garo \hld\ mód-gi-þáhti,
iro wréðon willjon; \hld\ þó he te þemu werode sprak,
te allun þem erlun: \hld\ ʽsó hwi-lik só iuwar áno síʼ, kwað he,
ʽslíðja sundjon, \hld\ só ganga iru selvo tó
endi sie at érist \hld\ erl mid is handun
stén ana werpe.ʼ \hld\ Só stódun Judeon,
þáhtun endi þagodun: \hld\ ni mahte þegan nigiean
wið þem word-kwidi \hld\ wiðer-saka finden:
ge-hugde manno ge-hwi-lik \hld\ mén-gi-þáhti,
is selves sundja: \hld\ ni was iro só sikur énig,
þat he bi þemu worde \hld\ þemu wíve ge-dorsti
stén an werpen, \hld\ ak létun sie standen þar
énan þar inne \hld\ endi im út þanen
gengun gram-harde \hld\ Judeo liudi,
én aftar óðrumu, \hld\ antat iro þar énig ni was
þes fíundo folkes, \hld\ þe iro ferhes þó,
þeru idis aldar-lago \hld\ áhtien weldi.
Þó gi-fragn ik þat sie frágode \hld\ friðu-barn godes,
allaro gumono bętst: \hld\ ʽhwar kwámun þit Judeono folkʼ, kwað he,
ʽþine wiðer-sakon, \hld\ þea þi hér wrógdun te mi?
Ne sie þi hiudu wiht \hld\ harmes ne gi-dádun,
þea liudi léðes, \hld\ þe þi weldun lívu beniman,
wégjan te wundrun?ʼ \hld\ Þó sprak imu eft þat wíf an-gegin,
kwað þat iru þar nioman \hld\ þurh þes nęrjandan
hèlaga helpa \hld\ harm ne gi-frumidi
wammes te lòne. \hld\ Þó sprak eft waldand Krist,
drohtin manno: \hld\ ʽne ik þi geþ ni deriu neowihtʼ, kwað he,
ʽak gang þi hél hinen, \hld\ lát þi an þínumu hugi sorga,
þat þu nio síð aftar þius \hld\ sundig ni werðes.ʼ
Habde iru þó gi-holpen \hld\ hèlag barn godes,
ge-friðot iro ferahe. \hld\ Þan stód þat folk Judeono
uviles an-mód \hld\ só fan éristan,
wréðes willjan, \hld\ hwó sie word-heti
wið þat friðu-barn godes \hld\ frummjen móstin.
Habdun þea liudi an twé \hld\ mid iro gi-lóvon gi-fangan:
was þiu smale þioda \hld\ sínes willjan
gernora mikilu, \hld\ þes godes barnes word
te ge-frummjenne, \hld\ só im iro fráho gi-bód:
rómodun te rehta \hld\ bet þan þie ríkjon man,
habdun ina far iro hérron \hld\ ia far heven-kuning,
ful-gengun imu gerno. \hld\ Þó gi-wét imu þe godes sunu
an þene wíh innan: \hld\ hwarf ina werod umbi,
męgin-þiodo gi-mang. \hld\ He an middien stód,
lérde þea liudi \hld\ liohtun wordun,
hlúdero stemnun: \hld\ was hlust mikil,
þagode þegan manag, endi he þeru þiod gi-bód,
só hwe só þar mid þurstu bi-þwungan wári,
ʽsó ganga imu herod drinkan te miʼ, kwað he, ʽdago ge-hwi-likes
swóties brunnan. Ik mag sęggjan iu,
só hwe só hér gi-lóvid te mi liudjo barno
fasto undar þesumu folke, þat imu þan flioten skulun
fan is lík-hamon libbiendi flód,
irnandi water, ahospring mikil,
kumad þanen kwika brunnon. Þesa kwidi werðad wára,
liudjun gi-léstid, só hwemu só hér gi-lóvid te mi.ʼ
Þan ménde mid þiu wataru waldandeo Krist,
hér heven-kuning hèlagna gést,
hwó þene firiho barn ant-fáhen skoldin,
lioht endi listi endi líf éwig,
hòh heven-ríki endi huldi godes.
wurðun þó þea liudi umbi þea léra Kristes,
umbi þiu word an ge-winne: \hld\ stódun wlanka man,
gél-móde Judeon, \hld\ sprákun gelp mikil,
habdun it im te hoska, \hld\ kwaðun þat sie mahtin gi-hórjen wel,
þat imu mahlidin fram \hld\ módaga wihti,
un-holde út: \hld\ ʽnu he an avu léridʼ, kwáðun sie,
ʽwordu ge-hwi-liku.ʼ \hld\ Þó sprak eft þat werod óðar:
ʽni þurvun gi þene lérjand lahanʼ, \hld\ kwáðun sie: ʽkumad líves word
mahtig fan is múde; \hld\ he wirkid manages hwat,
wundres an þesaru weroldi: \hld\ nis þat wréðaro dád,
fíundo kraftes: \hld\ nio it þan te su-likaru frumu ni wurði,
ak it gegnungo \hld\ fan gode alo-waldon,
kumid fan is krafte. \hld\ Þat mugun gi ant-kęnnjen wel
an þem is wárun wordun, \hld\ þat he gi-wald havad
alles ovar erðu.ʼ \hld\ Þó weldun ina þe andsakon þar
an stędi fáhen \hld\ efþa stén ana werpen,
ef sie im þero manno \hld\ menigi ni and-rédin,
ni forhtodin þat folk-skępi. \hld\ Þó sprak þat friðu-barn godes:
ʽik tógju iu gódes só filuʼ, \hld\ kwað he, ʽfan gode selvumu,
wordo endi werko: \hld\ nu willjad gi mi wítnon hér
þurh iuwan starkan hugi, \hld\ stén ana werpen,
bi-lósjen mi lívu.ʼ \hld\ Þó sprákun imu eft þea liudi an-gegin,
wréða wiðersakon: \hld\ ʽne wi it be þínun werkun ni duatʼ, kwáðun sia,
ʽþat wi þi aldres \hld\ tó áhtien willjad,
ak wi duat it be þínun wordun, \hld\ hwand þu su-lik wáh sprikis,
*hwand þu þik só máris \hld\ endi su-lik mén sagis,
gihis for þeson Judeon, \hld\ þat þu sís god selvo,
mahtig drohtin, \hld\ endi bist þi þoh man só wi,
kuman fan þeson kunnje.ʼ \hld\ Krist alo-waldo
ne wolda þero Judeono þuo leng gelpes hórjan,
wréðaro willjon, ak hie im af þem wíhe fuor
ovar Jordanes stròm; habda jungron mid im,
þia is sáligun gi-síðos, þia im simlon mid im
willjon wonodun: suohta werod óðer,
deda þar só hie gi-wonoda, drohtin selvo,
lérda þia liudi: gi-lóvda þie wolda
an is hèlagun word. Þat skolda sinnon wel
manno só hwi-likon, só þat an is muod ginam.
Þuo gi-frang ik þat þar te Kriste \hld\ kumana wurðun %NOTE: gi-frang] Checked according to C.
bodon fan Bethaniu \hld\ endi sagdun þem barne godes,
þat sia an þat árundi þarod \hld\ idisi sendin,
Maria endi Marþa, \hld\ magað frílíka,
swíðo wun-sama wíf; \hld\ þia wissa hie béðia,
wárun im gi-swester twá, \hld\ þia hie selvo ér
minnjoda an is muode \hld\ þuru iro mildjan hugi,
þiu wíf þuru iro willjon guodan. \hld\ Sia im te wáron þuo
an-budun fon Bethaniu, \hld\ þat iro bruoðer was
Lazarus legar-fast \hld\ endi þat sia is líves ni wándun;
bádun þat þarod kwámi \hld\ Krist alo-waldo
hèlag te helpu. \hld\ Reht só hie sia gi-hórda þuo
sęggjan fan só siekon, \hld\ só sprak hie sán an-gegin,
kwað þat Lazaruses \hld\ legar ni wári
gi-duan im te dóðe, \hld\ ʽak þar skal drohtines lofʼ, kwaþie,
ʽgi-frumid werðan: \hld\ nis it im te óðron fréson gi-duan.ʼ
was im þar þuo selvo \hld\ suno drohtines
twá naht endi dagas. \hld\ Þiu tíd was þuo ge-náhit,
þat hie eft te Hierusalem \hld\ Judeo liudjo
wíson welda, \hld\ só hie gi-wald habda.
Sagda þuo is gi-síðon \hld\ suno drohtines,
þat hie eft ovar Jordan \hld\ Judeo liudi
suokjan welda. \hld\ Þuo sprákun im sán an-gegin
jungron sína: \hld\ ʽte hwí bist þu só gern þarodʼ, kwaðun sia,
ʽfro mín, te faranne? \hld\ Ni þat nu furn ni was,
þat sia þik þínero wordo \hld\ wítnon hogdun,
weldun þi mid sténon starkan awerpan? \hld\ nu þu eft undar þia strídigun þioda
fundos te faranne, \hld\ þar ist fíondo ginuog,
erlos ovar-muoda?ʼ \hld\ Þuo én þero twelivjo,
Þuomas gimálda \hld\ —was im gi-þungan mann,
diur-lík drohtines þegan—: \hld\ ʽne skulun wi im þia dád lahanʼ, kwaþie,
ʽni wernjan wi im þes willjen, \hld\ ak wita im wonjan mid,
þuolojan mid ússon þiodne: \hld\ þat ist þegnes kust,
þat hie mid is fráhon samad \hld\ fasto gi-stande,
dóje mid im þar an duome. \hld\ Duan ús alla só,
folgon im te þero ferdi: \hld\ ni látan úse ferah wið þiu
wihtes wirðig, \hld\ neva wi an þem werode mid im,
dójan mid úson drohtine. \hld\ Þan lévot ús þoh duom after,
guod word for gumon.ʼ \hld\ Só wurðun þuo jungron Kristes,
erlos aðal-borana \hld\ an én-falden hugje,
hérren te willjen. \hld\ Þuo sagda hèlag Krist
selvo is gi-síðon \hld\ þat a-slápan was
Lazarus fan þem legare, \hld\ ʽhavit þit lioht a-gevan,
an-swevit ist an selmon. \hld\ Nu wi an þena síð faran
endi ina a-wękkjan, \hld\ þat hie muoti eft þesa werold sehan,
libbjandi lioht: \hld\ þan wirðit iuwa gi-lóvo after þiu
forð-werd gi-fęstid.ʼ \hld\ Þuo gi-wét hie im ovar þia fluod þanan,
þie guodo godes suno, \hld\ anþat hie mid is jungron kwam
þar te Bithaniu, \hld\ barn drohtines
selvo mid is gi-síðon, \hld\ þar þia gi-swester twá,
Maria endi Marþa \hld\ an muod-karon
sèraga sátun. \hld\ Was þar gi-samnot filo
fan Hierusalem \hld\ Judeo liudo,
þia þiu *wíf weldun \hld\ wordun fruovrean,
þat sie só ni karodin \hld\ kind-jungas dóð,
Lazaruses far-lust. \hld\ Só þó þe landes ward
geng an þiu gardos, \hld\ só wurðun þes godes barnes
kumi þar gi-kúðid, \hld\ þat he só kraftig was
bi þeru burg úten. \hld\ Þó im béðjun was,
þem wívun su-lik willjo, \hld\ þat sie im waldand tó,
þat friðu-barn godes, \hld\ farandien wissun.
Þó þem wíbun was \hld\ willjono mésta
kumi drohtines \hld\ endi Kristes word
te gi-hórjenne. \hld\ Heovandi geng
Martha mód-karag \hld\ wið só mahtigne
wordun wehslan \hld\ endi wið waldand sprak
an iro hugi hriwig: \hld\ ʽþar þu mi, hérro mínʼ, kwað siu,
ʽnęrjendero bętst, \hld\ náhor wáris,
héljand þe gódo, \hld\ þan ni þorfti ik nu su-lik harm þolon,
bittra breost-kara, \hld\ þan ni wári nu mín bróðer dód,
Lazarus fan þesumu liohte, \hld\ ak he imu mahti libbien forð
ferahes ge-fullid. \hld\ Ik þoh, fró mín, te þi
liohto gi-lóvju, lérjandero bętst,
só hwes só þu biddjen wili berhton drohtin,
þat he it þi sán far-givid, god alo-mahtig,
gi-werðot þínan willjan.ʼ Þó sprak eft waldand Krist
þeru idis and-wordi: ʽni lát þu þi an innan þesʼ, kwað he,
ʽþínan sevon swerkan: ik þi sęggjan mag
wárun wordun, þat þes nis gi-wand énig,
nevu þín bróðer skal þurh gi-bod godes,
þurh drohtines kraft fan dóðe a-standen
an is lík-hamon.ʼ ʽAll hębbju ik gi-lóvon sóʼ, kwað siu,
ʽþat it só gi-werðen skal, só hwan só þius werold endjod
endi þe márjo dag ovar man ferid,
þat he þan fan erðu skal up a-standen
an þemu dómes daga, þan werðad fan dóðe kwika
þurh maht godes man-kunnjes ge-hwi-lik,
arísad fan restu.ʼ Þó sagde ríkjo Krist
þeru idis alo-mahtig oponun wordun,
þat he selvo was sunu drohtines,
béðju ia líf ia lioht liudjo barnon
te a-standanne: \hld\ ʽnio þe sterven ni skal,
líf far-liosen, \hld\ þe hér gi-lóvid te mi:
þoh ina eldi-barn \hld\ erðu bi-þekkien,
diapo bidelven, \hld\ nis he dód þiu mér:
þat flésk is bi-folhen, \hld\ þat ferah is gi-halden,
is þiu siola gi-sund.ʼ \hld\ Þó sprak imu eft sán an-gegin
þat wíf mid iro wordun: \hld\ ʽik gi-lóvju þat þu þe wáro bistʼ, kwað siu,
ʽKrist godes sunu: \hld\ þat mag man ant-kęnnjen wel,
witen an þínun wordun, \hld\ þat þu gi-wald haves
þurh þiu hèlagon gi-skapu \hld\ himiles endi erðun.ʼ
Þó ge-fragn ik þat þar þero idisio kwam \hld\ óðar gangan
Maria mód-karag: \hld\ gengun iro managa aftar
Judeo liudi. \hld\ Þó siu þemu godes barne
sagde sèrag-mód, \hld\ hwat iru te sorgun gi-stód
an iro hugi harmes: hofnu kúmde
Lazaruses far-lust, liaves mannes,
griat gornundi, antat þemu godes barne
hugi warð gi-hrórid: héte trahni
wópu awellun, endi þó te þem wívun sprak,
hét ina þó lédjen, þar Lazarus was
foldu bi-folhen. Lag þar én felis biovan,
hard stén be-hliden. Þó hét þe hèlago Krist
ant-lúkan þea léia, þat he mósti þat lík sehan,
hréo skawojen. Þó ni mahte an iro hugi míðan
Marþa for þeru menegi, wið mahtigne sprak:
ʽfró mín þe gódoʼ, kwað siu, ʽef man þene felis nimid,
þene stén ant-lúkid, þan wániu ik þat þanen stank kume,
unswóti suek, hwand ik þi sęggjan mag
wárun wordun, þat þes nis gi-wand énig,
þat he þar nu bi-folhen was fiuwar naht endi dagos
an þemu erð-grave.ʼ and-wordi gaf
waldand þemu wíbe: ʽhwat, ni sagde ik þi te wárun érʼ, kwað he,
ʽef þu gi-lóvjen wili, þan nis nu lang te þiu,
þat þu hér ant-kęnnjen skalt kraft drohtines,
þe mikilon maht godes?ʼ Þó gengun manage tó,
af-hóvun harden stén. Þó sah þe hèlago Krist
up mid is ógun, ólat sagde
þemu þe þese werold gi-skóp, ʽþes þu mín word gi-hórisʼ, kwað he,
ʽsigi-drohtin selvo; ik wét þat þu só simlun duos,
ak ik duom it be þesumu gróton Judeono folke,
þat sie þat te wárun witin, þat þu mi an þese werold sendes
þesun liudjun te lérun.ʼ Þó he te Lazaruse hriop
starkaru stemniu endi hét ina standen up
ia fan þemu grave gangan. Þó warð þe gést kumen
an þene lík-hamon: he bi-gan is liði hrórien,
antwarp undar þemu gi-wédie: was imo só be-wunden þó noh,
an hréo-będdjon bi-helid. Hét imu helpen þó
waldandeo Krist. Weros gengun tó,
ant-wundun þat ge-wádi. Wánum up a-rés
Lazarus te þesumu liohte: was imu is líf far-geven,
þat he is aldar-lagu égan mósti,
friðu forð-wardes. Þó fagonadun béðja,
Maria endi Marþa: ni mag þat man óðrumu
gi-sęggjan te sóðe, hwó þea gesuester tuó
mendjodun an iro móde. Maneg wundrode
Judeo liudjo, þó sie ina fan þemu grave sáhun
síðon ge-sunden, þene þe ér suht farnam
endi sie bidulvun diapo undar erðu
líves lósen: þó móste imu libbien forð
hél an hémun. Só mag heven-kuninges,
þiu mikile maht godes manno ge-hwi-likes
ferahe gi-formon endi wið fíundo níð
hèlag helpen, só hwemu só he is huldi far-givid
Þó warð þar só managumu manne mód aftar Kriste,
gi-hworven hugi-skęfti, síðor sie is hèlagon werk
selvon gi-sáhun, hwand eo ér su-lik ni warð
wunder an weroldi. Þan was eft þes werodes só filu,
só módstarke man: ni weldon þe maht godes
ant-kęnnjen kúðlíko, ak sie wið is kraft mikil
wunnun mid iro wordun: wárun im waldandes
léra so léða: sóhtun im liudi óðra
an Hierusalem, þar Judeono was
héri hand-mahal endi hòvid-stędi,
grót gum-skępi grimmaro þioda.
Sie kúðdun im þó Kristes werk, kwáðun þat sie kwikan sáhin
þene erl mid iro ógun, þe an erðu was,
foldu bi-folhen fiuwar naht endi dagos,
dód bidolven, antat he ina mid is dádjun selvo,
mid is wordun awekide, þat he mósti þese werold sehan.
Þó was þat só wiðerward wlankun mannun,
Judeo liudjun: hétun iro gum-skępi þó,
werod samnojan endi warvos fáhen,
męgin-þioda gi-mang, an mahtigna Krist
riedun an runun: ʽnis þat rád énigʼ, kwáðun sie,
ʽþat wi þat gi-þolojan: wili þesaro þioda te filu
gi-lóvjen aftar is lérun. Þan ús liudi farad,
an eorid-folk, werðat úsa ovar-hóvdun
rinkos fan Rúmu. Þan wi þeses ríkjes skulun
lóse libbien efþa wi skulun úses líves þolon,
hęliðos úsaro hóvdo.ʼ Þó sprak þar én gi-hérod man
ovar warf wero, þe was þes werodes þó
an þeru burg innan biskop þero liudjo
—Kaiphas was he héten; \hld\ habdun ina gi-koranen te þiu
an þeru gér-talu \hld\ Judeo liudi,
þat he þes godes húses \hld\ gómjen skoldi,
wardon þes wíhes—: \hld\ ʽmi þunkid wunder mikilʼ, kwað he,
ʽmári þioda, \hld\ —gi kunnun manages gi-skéð—
hwí gi þat te wárun ni witin, \hld\ werod Judeono,
þat hér is bętera rád \hld\ barno ge-hwi-likumu,
þat man hér énne man \hld\ aldru bi-lósje
endi þat he þurh iuwa dádi \hld\ dròreg sterve,
for þesumu folk-skępi \hld\ ferah far-láte,
þan al þit liud-werod \hld\ far-loren werðe.ʼ
Ni was it þoh is willjan, \hld\ þat he só wár ge-sprak,
só forð for þemu folke, frume man-kunnjes
giménde for þeru menegi, ak it kwam imu fan þeru maht godes
þurh is hèlagan héd, hwand he þat hús godes
þar an Hierusalem bi-gangan skolde,
wardon þes wíhes: be-þiu he só wár gi-sprak,
biskop þero liudjo, hwó skoldi þat barn godes
alla irmin-þiod mid is énes ferhe,
mid is lívu a-lósjen: þat was allaro þesaro liudjo rád,
hwand he gi-halode mid þiu héðina liudi,
weros an is willjon waldandio Krist.
Þó wurðun én-wordje ovar-módje man,
werod Judeono, endi an iro warve gi-sprákun,
mári þioda, þat sie im ni létin iro mód twehon:
só hwe só ina undar þemu folke finden mahti,
þat ina sán gi-fengi endi forð bráhti
an þero þiodo þing; kwáðun þat sie ni mahtin gi-þolojan leng,
þat sie þe éno man só alla weldi,
werod far-winnen. \hld\ Þan wisse waldand Krist
þero manno só garo \hld\ mód-gi-þáhti,
heti-grimmon hugi, \hld\ hwand imu ni was bi-holen eowiht
an þesaru middil-gard: \hld\ he ni welde þó an þie menigi innen
síður open-líko, \hld\ under þat erlo folk,
gangan under þea Judeon: \hld\ béd þe godes sunu
þero torohteon tíd, \hld\ þe imu tóward was,
þat he far þesa þioda \hld\ þolojan welde,
far þit werod wíti: \hld\ wisse imu selvo
þat dag-þingi garo. \hld\ Þó gi-wét imu úse drohtin forð
endi imu þó an Effrem \hld\ alo-waldo Krist
an þeru hòhon burg \hld\ hèlag drohtin
wunode mid is werodu, \hld\ antat he an is willjan hwarf
eft te Bethania \hld\ brahtmu þiu mikilun,
mid þiu is gódum gum-skępi. \hld\ Judeon bisprákun þat
wordu ge-hwi-liku, \hld\ þó sie imu su-lik werod mikil
folgon gi-sáhun: \hld\ ʽnis frume énigʼ, kwáðun sie,
ʽúses ríkjes gi-rádi, \hld\ þoh wi reht sprekan,
ni þíhit úses þinges wiht: \hld\ þius þiod wili
węndjen after is willjan; \hld\ imu all þius werold folgot,
liudi bi þem is lérun, \hld\ þat wi imu léðes wiht
for þesumu folk-skępi \hld\ gi-frummjen ni mótun.ʼ
Gi-wét imu þó þat barn godes \hld\ innan Bethania
sehs nahtun ér, \hld\ þan þiu samnunga
þar an Hierusalem \hld\ Judeo liudjo
an þem wíh-dagun \hld\ werðen skolde,
þat sie skoldun haldan \hld\ þea hèlagon tídi,
Judeono paskha. \hld\ Béd þe godes sunu,
mahtig under þeru menegi: \hld\ was þar manno kraft,
werodes bi þem is wordun. \hld\ Þar gengun ina twé wíf umbi,
Maria endi Marþa, \hld\ mid mildju hugi,
þionodun imu þeo-líko. \hld\ Þiodo drohtin
gaf im lang-sam lòn: \hld\ lét sea léðes gi-hwes,
sundjono sikora, \hld\ endi selvo gi-bód,
þat sea an friðe fórin \hld\ wiðer fíundo níð,
þea idisa mid is orlovu gódu: \hld\ habdun iro ambaht-skępi
bi-wendid an is willjon. \hld\ Þó gi-wét imu waldand Krist
forð mid þiu folku, \hld\ firiho drohtin,
innan Hierusalem, \hld\ þar Judeono was
hete-lík hard-buri, \hld\ þar sie þea hèlagon tíd
warodun at þemu wíhe; \hld\ was þar werodes só filu,
kraftigaro kunnjo, \hld\ þie ni weldun Kristes word
gerno hórjen \hld\ ni te þemu godes barne
an iro mód-sevon \hld\ minnje ni habdun,
ak wárun im só wréða \hld\ wlanka þioda,
módeg man-kunni, \hld\ habdun im morð-hugi,
inwid an innan: \hld\ an avuh farfengun
Kristes lére, \hld\ weldun ina kraftigna
wítnon þero wordo; \hld\ ak was þar werodes só filu,
umbi erl-skępi \hld\ ant-langana dag,
habde ine þiu smale þiod \hld\ þurh is swótiun word
werodu bi-worpen, þat ine þie wiðer-sakon
under þemu folk-skępi fáhen ne gi-dorstun,
ak miðun is bi þeru menegi. Þan stód mahtig Krist
an þemu wíhe innan, sagde word manag
firiho barnun te frumu. Was þar folk umbi
allan langan dag, antat þiu liohte gi-wét
sunne te sedle. Þó te seliðun fór
man-kunnjes manag. Þan was þar én mári berg
bi þeru burg úten, þe was bréd endi hòh,
gróni endi skóni: hétun ina Judeo liudi
Oliueti bi namon. Þar imu up gi-wét
nęrjendjo Krist, só ina þiu naht bi-feng,
was imu þar mid is jungarun, só ine þar Judeono énig
ni wisse ti wárun, hwand he an þemu wíhe stód,
liudjo drohtin, só lioht óstene kwam,
ant-feng þat folk-skępi endi im filu sagde
wároro wordo, só nis an þesaru weroldi énig,
an þesaru middil-gard manno só spáhi,
liudjo barno nigén, þat þero lérono mugi
endi gi-tęlljen, þe he þar an þemu alahe gi-sprak,
waldand an þemu wíhe, endi simlun mid is wordun gi-bód,
þat sie sie gerewidin te godes ríkje,
allaro manno ge-hwi-lik, þat sie móstin an þemu márjon daga
iro drohtines diuriða ant-fáhen.
Sagde im hwat sie it sundjun frumidun endi simlun gi-bód,
þat sie þea aleskidin; hét sie lioht godes
minnjon an iro móde, mén far-láten,
avoha ovarhugdi, ód-módi niman,
hlaðen þat an iro hertan; kwað þat im þan wári heven-ríki,
garu gódo mést. Þó warð þar gumono só filu
gi-wendid aftar is willjon, síður sie þat word godes
hèlag gi-hórdun, heven-kuninges,
ant-kendun kraft mikil, kumi drohtines,
hérron helpe, ia þat heven-ríki was,
nęrjendi gi-náhid endi náða godes
manno barnun. Sum só módeg was
Judeo folkes, habdun grimman hugi,
slíð-móden sevon \hld\ [...],
ni weldun is worde gi-lóvjen, ak habdun im ge-win mikil
wið þea Kristes kraft: kumen ni móstun
þea liudi þurh léðen stríd, þat sie gi-lóvon te imu
fasto gi-fengin; ni was im þiu frume giviðig,
þat sie heven-ríki habbjen móstin.
Geng imu þó þe godes sunu endi is jungaron mid imu,
waldand fan þemu wíhe, all só is willjo geng,
iak imu uppen þene berg gi-stég barn drohtines:
sat imu þar mid is ge-síðun endi im sagde filu
wároro wordo. Sí bi-gunnun im þó umbi þene wíh sprekan,
þie gumon umbi þat godes hús, kwáðun þat ni wári gód-líkora
alah ovar erðu þurh erlo hand,
þurh mannes gi-werk mid męgin-kraftu
rakud arihtid. Þó þe ríkjo sprak,
hér heven-kuning \hld\ —hórdun þe óðra—:
ʽik mag iu gi-tęlljenʼ, \hld\ kwað he, ʽþat noh wirðid þiu tíd kumen,
þat is af-standen ni skal stén ovar óðrumu,
ak it fallid ti foldu endi fiur nimid,
grádag logna, þoh it nu só gód-lík sí,
só wís-líko gi-warht, endi só dód all þesaro weroldes gi-skapu,
te-glídid gróni wang.ʼ Þó gengun imu is jungaron tó,
frágodun ina só stillo: ʽhwó lango skal standen nohʼ, kwáðun sie,
ʽþius werold an wunnjun, ér þan þat gi-wand kume,
þat þe lasto dag liohtes skíne
þurh wolkanskion, efþo hwan is þín eft wán kumen
an þene middil-gard, manno kunnje
te a-déljenne, dódun endi kwikun?
fró mín þe gódo, ús is þes firi-wit mikil,
waldandeo Krist, hwan þat gi-werðen skuli.ʼ
Þó im and-wordi alo-waldo Krist
gód-lík far-gaf þem gumun selvo:
ʽþat havad só bi-dernidʼ, \hld\ kwað he, ʽdrohtin þe gódo
iak só hardo far-holen \hld\ himil-ríkjes fader,
waldand þesaro weroldes, \hld\ só þat witen ni mag
énig mannisk barn, \hld\ hwan þiu márje tíd
gi-wirðid an þesaru weroldi, \hld\ ne it ók te wáran ni kunnun
godes ęngilos, \hld\ þie for imu gegin-warde
simlun sindun: \hld\ sie it ók gi-sęggjan ni mugun
te wáran mid iro wordun, \hld\ hwan þat gi-werðen skuli,
þat he willje an þesan middil-gard, \hld\ mahtig drohtin,
firiho fandon. \hld\ Fader wét it éno
hèlag fan himile: \hld\ elkur is it bi-holen allun,
kwikun endi dódun, \hld\ hwan is kumi werðad,
Ik mag iu þoh gi-tęlljen, \hld\ hwi-lik hér tékạn bi-foran
gi-werðad wunder-lík, \hld\ ér þan he an þese werold kume
an þemu márjon daga: \hld\ þat wirðid hér ér an þemu mánon skín
iak an þeru sunnon só same; \hld\ gi-swerkad siu béðju,
mid finistre werðad bi-fangan; \hld\ fallad sterron,
hwít heven-tungal, \hld\ endi hrisid erðe,
bivod þius bréde werold \hld\ —wirðid su-likaro bókno filu—:
grimmid þe gróto séo, \hld\ wirkid þie gevenes stròm
egison mid is úðiun \hld\ erð-búandiun.
Þan þorrot þiu þiod \hld\ þurh þat ge-þwing mikil,
folk þurh þea forhta: \hld\ þan nis friðu hwergin,
ak wirðid wíg só maneg \hld\ ovar þese werold alla
hete-lík af-haben, \hld\ endi heri lédid
kunni ovar óðar: \hld\ wirðid kuningo gi-win,
męgin-fard mikil: \hld\ wirðid managoro kwalm,
open ur-lagi \hld\ —þat is egis-lík þing,
þat io su-lik morð \hld\ skulun man af-hębbjen—,
wirðid wól só mikil \hld\ ovar þese werold alle,
man-stervono mést, \hld\ þero þe gio an þesaru middil-gard
swulti þurh suhti: \hld\ liggjad seoka man,
driosat endi dójat \hld\ endi iro dag endjad,
fulljad mid iro ferahu; \hld\ ferid un-met grót
hungar heti-grim \hld\ ovar hęliðo barn,
meti-gé-deono mést: \hld\ nis þat minniste
þero wíteo an þesaru weroldi, \hld\ þe hér gi-werðen skulun
ér dómes dage. \hld\ Só hwan só gi þea dádi gi-sehan
gi-werðen an þesaru weroldi, \hld\ só mugun gi þan te wáran far-standen,
þat þan þe latsto dag \hld\ liudjun náhid
mári te mannun \hld\ endi maht godes,
himilkraftes hróri \hld\ endi þes hèlagon kumi,
drohtines mid is diuriðun. \hld\ Hwat, gi þesaro dádjo mugun
bi þesun bómun \hld\ biliði ant-kęnnjen:
þan sie brustjad endi blójat \hld\ endi bladu tógjat,
lóf ant-lúkad, \hld\ þan witun liudjo barn,
þat þan is sán after þiu \hld\ sumer gi-náhid
warm endi wun-sam \hld\ endi weder skóni.
Só witin gi ók bi þesun téknun, \hld\ þe ik iu talde hér,
hwan þe latsto dag \hld\ liudjun náhid.
Þan sęggjo ik iu te wáran, \hld\ þat ér þit werod ni mót,
te-faran þit folk-skępi, \hld\ ér þan werðe ge-fullid só,
mínu word gi-wárod. \hld\ Noh gi-wand kumid
himiles endi erðun, endi steid mín hèlag word
fast forð-wardes endi wirðid al ge-fullod só,
gi-léstid an þesumu liohte, só ik for þesun liudjun ge-spriku.
wakot gi warlíko: iu is wis-kumo
duom-dag þe márjo endi iuwes drohtines kraft,
þiu mikilo męgin-strengi endi þiu márje tíd,
gi-wand þesaro weroldes. Fora þiu gi wardon skulun,
þat he iu slápandje an swef-restu
fárungo ni bi-fáhe an firin-werkun,
ménes fulle. Mútspelli kumit
an þiustrea naht, al só þiof ferid
darno mid is dádjun, só kumid þe dag mannun,
þe latsto þeses liohtes, só it ér þese liudi ni witun,
só samo só þiu flód deda an furn-dagun,
þe þar mid lagu-stròmun liudi far-teride
bi Nóeas tídjun, bi-útan þat ina neride god
mid is híwiskja, hèlag drohtin,
wið þes flódes farm: só warð ók þat fiur kuman
hét fan himile, þat þea hòhon burgi
umbi Sodomo land swart logna bi-feng
grim endi grádag, þat þar nénig gumono ni ginas
bi-útan Loth éno: \hld\ ina ant-léddun þanen
drohtines ęngilos \hld\ endi is dohter twá
an énan berg uppen: \hld\ þat óðar al brinnandi fiur,
ia land ia liudi \hld\ logna farteride:
só fárungo warð þat fiur kumen, \hld\ só warð ér þe flód só samo:
só wirðid þe latsto dag. \hld\ For þiu skal allaro liudjo ge-hwi-lik
þęnkjan fora þemu þinge; þes is þarf mikil
manno ge-hwi-likumu: be-þiu látad iu an iuwan mód sorga.
Hwand só hwan só þat ge-wirðid, þat waldand Krist,
mári mannes sunu mid þeru maht godes,
kumit mid þiu kraftu kuningo ríkjost
sittjan an is selves maht endi samod mid imu
alle þea ęngilos, þe þar uppa sind
hèlaga an himile, þan skulun þarod hęliðo barn,
eli-þeoda kuman \hld\ alla te-samne
libbjandero liudjo, só hwat só io an þesumu liohte warð
firiho afódid. Þar he þemu folke skal,
allumu man-kunnje mári drohtin
a-déljen aftar iro dádjun. Þan skéðid he þea far-duanan man,
þea far-warhton weros an þea winistron hand:
só duot he ók þea sáligon an þea swíðeron half;
grótid he þan þea gódun endi im te-gegnes sprikid:
ʽkumad giʼ, kwiðid he, ʽþea þar gi-korene sindun, endi ant-fáhad þit kraftiga ríki,
þat góde, þat þar gi-gerewid stendid, \hld\ þat þar warð gumono barnun
gi-warht fan þesaro weroldes endie: iu havad ge-wíhid selvo
fader allaro firiho barno: gi mótun þesaro frumono neotan,
ge-waldon þeses wídon ríkjas, hwand gi oft mínan willjon frumidun,
ful-gengun mi gerno endi wárun mi iuwaro gevo mildje,
þan ik bi-þwungan was þurstu endi hungru,
frostu bi-fangan efþo an feteron lag,
biklemmid an karkare: oft wurðun mi kumana þarod
helpa fan iuwn handun: gi wárun mi an iuwomu hugi mildje,
wísodun mín werðliko.ʼ Þan sprikid imu eft þat werod an-gegin:
ʽfró mín þe gódoʼ, kweðat sie, ʽhwan wári þu bi-fangan só,
be-þwungan an su-likun þaravun, só þu fora þesaru þiod telis,
mahtig ménis? Hwan gi-sah þi man énig
be-þwungen an su-likun þaravun? Hwat, þu haves allaro þiodo gi-wald
iak só samo þero méðmo, þero þe io manno barn
ge-wunnun an þesaro weroldi.ʼ Þan sprikid im eft waldand god:
ʽsó hwat só gi dádunʼ, kwiðit he, ʽan iuwes drohtines namon,
gódes far-gávun \hld\ an godes éra
þem mannun, þe hér minniston sindun, \hld\ þero nu undar þesaru menegi standad
endi þurh ód-módi \hld\ arme wárun
weros, hwand sie mínan willjon fremidun \hld\ —só hwat só gi im iuwaro welono far-gávun,
gi-dádun þurh diuriða, \hld\ þat ant-feng iuwa drohtin selvo,
þiu helpe kwam te heven-kuninge. \hld\ Be-þiu wili iu þe hèlago drohtin
lònon iuwan gi-lóvon: \hld\ givid iu líf éwig.ʼ
wendid ina þan waldand \hld\ an þea winistron hand,
drohtin te þem farduanun mannun, \hld\ sagad im þat sie skulin þea dád antgelden,
þea man iro mén-gi-werk: \hld\ ʽnu gi fan mi skulunʼ, kwiðit he.
ʽfaran só for-flókane \hld\ an þat fiur éwig,
þat þar gi-garewid warð \hld\ godes and-sakun,
fíundo folke \hld\ be firin-werkun,
hwand gi mi ni hulpun, \hld\ þan mi hunger endi þurst
wégde te wundrun \hld\ efþa ik ge-wádjes lós
geng jámer-mód, \hld\ was mi grótun þarf,
þan ni habde ik þar énige helpe, \hld\ þan ik ge-hęftid was,
an liðokospun bi-lokan, \hld\ efþa mi legar bi-feng,
swára suhti: \hld\ þan ni weldun gi mín siokes þar
wíson mid wihti: \hld\ ni was iu werð eowiht,
þat gi mín ge-hugdin. \hld\ Be-þiu gi an hęllje skulun
þolon an þiustre.ʼ \hld\ Þan sprikid imu eft þiu þiod an-gegin:
ʽwola waldand godʼ, \hld\ kweðad sie, ʽhwí wilt þu só wið þit werod sprekan,
mahlien wið þese menegi? \hld\ Hwan was þi io manno þarf,
gumono gódes? \hld\ Hwat, sie it al be þínun gevun égun,
welon an þesaro weroldiʼ. \hld\ Þan sprikid eft waldand god:
ʽþan gi þea armostunʼ, \hld\ kwiðid he, ʽeldi-barno,
manno þea minniston \hld\ an iuwomu mód-sevon
hęliðos far-hugdun, \hld\ létun sea iu an iuwomu hugi léðe,
be-déldun sie iuwaro diurða, \hld\ þan dádun gi iuwana drohtin só sama,
gi-wernidun imu iuwaro welono: \hld\ be-þiu ni wili iu waldand god,
ant-fáhen fader iuwa, \hld\ ak gi an þat fiur skulun,
an þene diopun dóð, \hld\ diuvlun þionon,
wréðun wiðer-sakun, \hld\ hwand gi só warhtun bi-foran.ʼ
Þan aftar þem wordun skéðit \hld\ þat werod an twé,
þea gódun endi þea uvilon: \hld\ farad þea far-griponon man
an þea hétan hęl \hld\ hriwig-móde,
þea far-warhton weros, \hld\ wíti ant-fáhat,
uvil ęndi-lós. \hld\ Lédid up þanen
hér heven-kuning \hld\ þea hluttaron þeoda
an þat lang-same lioht: \hld\ þar is líf éwig,
gi-garewid godes ríki \hld\ gódaro þiado.ʼ
Só ge-fragn ik þat þem rinkun þo \hld\ ríki drohtin
umbi þesaro weroldes gi-wand \hld\ wordun talde,
hwó þiu forð ferid, \hld\ þan lango þe sie firiho barn
ardon mótun, \hld\ ia hwó siu an þemu endie skal
te-glíden endi te-gangen. \hld\ He sagde ók is jungarun þar
wárun wordun: \hld\ ʽhwat, gi witun alleʼ, kwað he,
ʽþat nu ovar twá naht \hld\ sind tídi kumana,
Giudeono paskha, \hld\ þat sie skulun iro gode þionon,
weros an þemu wíhe. \hld\ Þes nis ge-wand énig,
þat þar wirðid mannes sunu \hld\ te þeru męgin-þiodu
kraftag far-kópot \hld\ endi an krúke a-slagan,
þolod þiad-kwála.ʼ \hld\ Þó warð þar þegan manag
slíð-mód gi-samnod, \hld\ súðar-liudjo,
Judeono gum-skępi, \hld\ þar sie skoldun iro gode þionon.
wurðun éosagon \hld\ alle kumane,
an warf weros, \hld\ þe sie þó wísostun
undar þeru menegi \hld\ manno taldun,
kraftag kuni-burd. \hld\ Þar Kaiphas was,
biskop þero liudjo. \hld\ Sie rédun þó an þat barn godes,
hwó sie ina a-sluogin \hld\ sundja lósan,
kwáðun þat sie ina an þemu hèlagon daga \hld\ hrínen ni skoldin
undar þero manno menegi, \hld\ ʽþat ni werðe þius męgin-þioda,
hęliðos an hróru, \hld\ hwand ina þit hęri-skępi wili
far-standen mid strídu. \hld\ Wi só stillo skulun
fréson is ferahes, \hld\ þat þit folk Judeono
an þesun wíh-dagun \hld\ wróht ni af-hębbjen.ʼ
Þó geng imu þar Iúdas forð, \hld\ jungaro Kristes,
én þero twelivjo, \hld\ þar þat aðali sat,
Judeono gum-skępi; \hld\ kwað þat he is im gódan rád
sęggjan mahti: \hld\ ʽhwat willjad gi mi sęlljen hérʼ, kwað he,
ʽméðmo te médu, \hld\ ef ik iu þene man givu
áno wíg endi áno wróht?ʼ \hld\ Þó warð þes werodes hugi,
þero liudjo an lustun: \hld\ ʽef þu wili gi-léstjen sóʼ, kwáðun sie,
ʽþín word gi-wáron, \hld\ þan þu gi-wald haves,
hwat þu at þesaru þiodu \hld\ þiggjan willjes
gódaro méðmo.ʼ \hld\ Þó gi-hét imu þat gum-skępi þar
an is selves dóm \hld\ siluvar-skatto
þrítig at-samne, \hld\ endi he te þeru þiodu gi-sprak
dereveun wordun, \hld\ þat he gávi is drohtin wið þiu.
wende ina þó fan þemu werode: \hld\ was im wréð hugi,
talode im só treulós, \hld\ hwan ér wurði imu þiu tíd kuman,
þat he ina mahti far-wísjen \hld\ wréðaro þiodo,
fíundo folke. \hld\ Þan wisse þat friðu-barn godes,
wár waldand Krist, \hld\ þat he þese werold skolde,
ageven þese gardos \hld\ endi sókjen imu godes ríki,
gi-faren is fader-oðil. \hld\ Þó ni gi-sah énig firiho barno
méron minnje, \hld\ þan he þó te þem mannun ginam,
te þem is gódun jungaron: \hld\ góme warhte,
sette sie swáslíko \hld\ endi im sagde filu
wároro wordo. \hld\ Skréd wester dag,
sunne te sedle. \hld\ Þó he selvo gi-bód,
waldand mid is wordun, \hld\ hét im water dragan
hluttar te handun, \hld\ endi rés þó þe hèlago Krist,
þe gódo at þem gómun \hld\ endi þar is jungarono þuóg
fóti mid is folmun \hld\ endi suarf sie mid is fanon aftar,
druknide sie diur-líka. \hld\ Þó wið is drohtin sprak
Símon Petrus: \hld\ ʽni þunkid mi þit sómi þingʼ, kwað he,
ʽfró mín þe gódo, \hld\ þat þu míne fóti þwahes
mid þem þínun hèlagun handun.ʼ \hld\ Þó sprak imu eft is hérro an-gegin,
waldand mid is wordun: \hld\ ʽef þu is willjan ni havesʼ, kwað he,
ʽte ant-fáhanne, \hld\ þat ik þíne fóti þwahe
þurh su-lika minnja, \hld\ só ik þesun óðrun mannun hér
dóm þurh diurða, \hld\ þan ni haves þu énigan dél mid mi
an heven-ríkja.ʼ \hld\ Hugi warð þó gi-wendid
Símon Petruse: \hld\ ʽþu hava þi selvo gi-waldʼ, kwað he,
ʽfro mín þe gódo, \hld\ fóto endi hando
b endi mínes hóvdes só sama, \hld\ handun þínun,
þiadan, te þwahanne, \hld\ te þiu þak ik móti þína forð
huldi hębbjan \hld\ endi heven-ríkjes
su-lik gi-déli, \hld\ só þu mi, drohtin, wili
far-geven þurh þína gódi.ʼ \hld\ Jungaron Kristes,
þene ambaht-skępi \hld\ erlos þolodun,
þegnos mid gi-þuldeon, \hld\ só hwat só im iro þiodan dede,
mahtig þurh þea minnja, \hld\ endi ménde imu al méra þing
firihon te gi-frummjenne. \hld\ friðu-barn godes
geng imu þó eft gi-sittjen \hld\ under þat ge-síðo folk
endi im sagda filu lang-samna rád. \hld\ Warð eft lioht kuman,
morgen te mannun. \hld\ Mahtigne Krist
gróttun is jungaron endi frágodun, \hld\ hwar sie is góma þó
an þemu wíh-dage \hld\ wirkjen skoldin,
hwar he weldi halden \hld\ þea hèlagon tídi
selvo mid is ge-síðun. \hld\ Þó he sie sókjen hét,
þea gumon Hierusalem: \hld\ ʽsó gi þan gangan kumadʼ, kwað he,
ʽan þea burg innan \hld\ —þar is braht mikil,
męgin-þiodo gi-mang—, \hld\ þar mugun gi énan man sehan
an is handun dragen \hld\ hluttres watares
ful mid folmun. \hld\ Þemu gi folgon skulun
an só hwi-like gardos, \hld\ só gi ina gangan gi-sehat,
ia gi þan þemu hérron, \hld\ þe þie hovos égi,
selvon sęggjad, \hld\ þat ik iu sende þarod
te gi-garuwenne mína góma. \hld\ Þan tógid he iu én gód-lík hús,
hòhan soleri, \hld\ þe is bi-hangen al
fagarun fratahun. \hld\ Þar gi frummjen skulun
werd-skępi mínan. \hld\ Þar bium ik wiskumo
selvo mid mínun ge-síðun.ʼ \hld\ Þó wurðun sán aftar þiu
þar te Hierusalem \hld\ jungaron Kristes
forð-ward an ferdi, \hld\ fundun all só he sprak
word-tékạn wár: \hld\ ni was þes gi-wand énig.
Þar gerewidun sie þea góma. \hld\ Warð þe godes sunu,
hèlag drohtin \hld\ an þat hús kuman,
þar sie þe land-wíse \hld\ léstjen skoldun,
ful-gangan godes gi-bode, \hld\ al só Judeono was
éo endi ald-sidu \hld\ an ér-dagun.
Gi-wét imu þó an þemu ávande \hld\ alo-waldand Krist
an þene sęli sittjen; \hld\ hét þar is ge-síðos te imu
twelivi gangan, \hld\ þea im gi-triwiston
an iro mód-sevon \hld\ manno wárun
bi wordun endi bi wísun: \hld\ wisse imu selvo
iro hugi-skęfti \hld\ hèlag drohtin.
Grótte sie þó ovar þem gómun: \hld\ ʽgern bium ik swíðoʼ, kwað he,
ʽþat ik samad mid iu \hld\ sittjen móti,
gómono neoten, \hld\ Judeono paskha
déljen mid iu só diurjun. \hld\ Nu ik iu iuwes drohtines skal
willjon sęggjan, \hld\ þat ik an þesaro weroldi ni mót
mid mannun mér \hld\ móses an-bíten
furður mid firihun, \hld\ ér þan gi-fullod wirðid
himilo ríki. \hld\ Mi is an handun nu
wíti endi wunder-kwále, \hld\ þea ik for þesumu werode skal,
þolon for þesaru þiodu.ʼ \hld\ Só he þó só te þem þegnun sprak,
hèlag drohtin, \hld\ só warð imu is hugi dróvi,
warð imu gi-sworken sevo, \hld\ endi eft te þem ge-síðun sprak,
þe gódo te þem is jungarun: \hld\ ʽhwat, ik iu godes ríkiʼ, kwað he,
ʽgi-hét himiles lioht, \hld\ endi gi mi hold-líko
iuwan þegan-skępi. \hld\ Nu ni willjat gi a-þengean só,
ak wenkjat þero wordo. \hld\ Nu sęggju ik iu te wáran hér,
þat wili iuwar twelivjo én \hld\ trewana swíkan,
wili mi far-kópon undar þit kunni Judeono,
gi-sęlljen wiðer siluvre, endi wili imu þar sink niman,
diurje méðmos, endi geven is drohtin wið þiu,
holdan hérran. Þat imu þoh te harme skal,
werðan te wítje; be þat he þea wurdi farsihit
endi he þes arvedies endi skawot,
þan wét he þat te wáran, þat imu wári wóðiera þing,
bętera mikilu, þat he gio gi-boran ni wurði
libbiendi te þesumu liohte, þan he þat lòn nimid,
uvil arvedi Inwid-rádo.ʼ
Þó bi-gan þero erlo ge-hwi-lik te óðrumu skawon,
sorgondi sehan; was im sér hugi,
hriwig umbi iro herta: gi-hórdun iro hérron þó
gornword sprekan. Þea gumon sorgodun,
hwi-likan he þero twelivjo \hld\ te þiu tęlljen weldi,
skuldigna skaðon, \hld\ þat he habdi þea skattos þar
ge-þingod at þeru þiod. \hld\ Ni was þero þegno énigumu
su-likes in-widdies \hld\ óði te gehanne,
mén-gi-þáhtio \hld\ —ant-suok þero manno ge-hwi-lik—,
wurðun alle an forhtun, \hld\ frágon ne gi-dorstun,
ér þan þó ge-bóknide \hld\ bar-wirðig gumo,
Símon Petrus \hld\ —ne gi-dorste it selvo sprekan—
te Johanne þemu gódon: \hld\ he was þemu godes barne
an þem dagun \hld\ þegno liovost,
mést an minnjun \hld\ endi móste þar þó an þes mahtiges Kristes
barme restjen \hld\ endi an is breostun lag,
hlinode mid is hóvdu: \hld\ þar nam he só manag hèlag ge-rúni,
diapa gi-þáhti, \hld\ endi þó te is drohtine sprak,
be-gan ina þó frágon: \hld\ ʽhwe skal þat, fró mín, wesenʼ, kwað he,
ʽþat þi far-kópon wili, \hld\ kuningo ríkjost,
undar þínaro fíundo folk? \hld\ Ús wári þes firi-wit mikil,
waldand, te witanne.ʼ \hld\ Þó habde eft is word garu
héljando Krist: \hld\ ʽseh þi, hwemu ik hér an hand geve
mínes móses for þesun mannun: \hld\ þe haved mén-gi-þáht,
birid bittran hugi; \hld\ þe skal mi an banono ge-wald,
fíundun bi-felhen, \hld\ þar man mínes ferhes skal,
aldres áhtien.ʼ \hld\ Nam he þó aftar þiu
þes móses for þem mannun \hld\ endi gaf is þemu mén-skaðen,
Judase an hand \hld\ endi imu te-gegnes sprak
selvo for þem is ge-síðun \hld\ endi ina sniumo hét
faran fan þemu is folke: \hld\ ʽfrumi só þu þenkisʼ, kwað he,
ʽdó þat þu duan skalt: \hld\ þu ni maht bi-dernjen leng
willjon þínan. \hld\ Þiu wurd is at handun,
þea tídi sind nu gi-náhid.ʼ \hld\ Só þó þe treu-logo
þat mós ant-feng \hld\ endi mid is múðu anbét,
só afgaf ina þó þiu godes kraft, \hld\ gramon in ge-witun
an þene lík-hamon, \hld\ léða wihti,
warð imu Satanas \hld\ séro bitengi,
hardo umbi is herte, \hld\ síður ine þiu helpe godes
far-lét an þesumu liohte. \hld\ Só is þena liudjo wé,
þe só undar þesumu himile skal \hld\ hérron wehslon.
Gi-wét imu þó út þanen \hld\ in-widjas gern
Judas gangan: \hld\ habde imu grimmen hugi
þegan wið is þiodan. \hld\ Was þó iu þiustri naht,
swíðo gi-sworken. \hld\ Sunu drohtines
was ima at þem gómun forð \hld\ endi is jungarun þar
waldand wín endi bród \hld\ wíhide béðju,
hèlagode heven-kuning, \hld\ mid is handun brak,
gaf it undar þem is jungarun \hld\ endi gode þankode,
sagde þem ólat, \hld\ þe þar al gi-skóp,
werold endi wunnja, \hld\ endi sprak word manag:
ʽgi-lóvjot gi þes liohtoʼ, \hld\ kwað he, ʽþat þit is mín lík-hamo
endi mín blód só same: \hld\ givu ik iu hér béðju samad
etan endi drinkan. \hld\ Þit ik an erðu skal
gevan endi geotan \hld\ endi iu te godes ríkje
lósjen mid mínu lík-hamen \hld\ an líf éwig,
an þat himiles lioht. \hld\ Gi-huggjat gi simlun,
þat gi þiu ful-gangan, \hld\ þiu ik an þesun gómun dón;
márjad þit for menegi: \hld\ þit is mahtig þing,
mid þius skulun gi iuwomu drohtine \hld\ diuriða frummjen,
habbjad þit mín te gi-hugdjun, \hld\ hèlag biliði,
þat it eldi-barn \hld\ aftar léstjen,
waron an þesaru weroldi, \hld\ þat þat witin alle,
man ovar þesan middil-gard, \hld\ þat it is þurh mína minnja gi-duan
hérron te huldi. \hld\ Ge-huggjad gi simlun,
hweo ik iu hér ge-biudu, \hld\ þat gi iuwan bróðer-skępi
fasto frummjad: \hld\ habbjad ferhtan hugi,
minnjod iu an iuwomu móde, \hld\ þat þat manno barn
ovar irmin-þiod \hld\ alle far-standen,
þat gi sind gegnungo \hld\ jungaron míne.
Ók skal ik iu kúðjen, \hld\ hwó hér wili kraftag fíund,
hetteand heru-grim, \hld\ umbi iuwan hugi niusien,
Satanas selvo: \hld\ he kumid iuwaro seolono herod
frókno fréson. \hld\ Simlun gi fasto te gode
berad iuwa breost-gi-þáht: \hld\ ik skal an iuwaru bedu standen,
þat iu ni mugi þe mén-skaðo \hld\ mód ge-twíflean;
ik fulléstiu iu wiðer þemu fíunde. \hld\ Ók kwam he herod giu fréson mín,
þoh imu is willjon hér \hld\ wiht ne gi-stódi,
lioves an þemu mínumu lík-hamon. \hld\ Nu ni willju ik iu leng helen,
hwat iu hér nu sniumo skal \hld\ te sorgu gi-standen:
gi skulun mi ge-swíkan, \hld\ ge-síðos míne,
iuwes þegan-skępjes, \hld\ ér þan þius þiustrie naht
liudi far-líða \hld\ endi eft lioht kume,
morgan te mannun.ʼ \hld\ Þó warð mód gumon
swíðo gi-sworken \hld\ endi sér hugi,
hriwig umbi iro herte \hld\ endi iro hérron word
swíðo an sorgun. \hld\ Símon Petrus þó,
þegan wið is þiodan \hld\ þríst-wordun sprak
bi huldi *wið is hérron: \hld\ ʽþoh þi all þit hęliðo folkʼ, kwaþie,
ʽgi-swíkan þína gi-síðos, \hld\ þoh ik sinnon mid þi
at allon þaravon \hld\ þolojan willju.
Ik biun garo sinnon, \hld\ ef mi god látið,
þat ik an þínon fulléstje \hld\ fasto gi-stande;
þoh sia þi an karkaries \hld\ klústron hardo,
þesa liudi bilúkan, \hld\ þoh ist mi luttil tweho,
ne ik an þem bęndjon mid þi \hld\ bídan willje,
liggian mid þi só lieven; \hld\ ef sia þínes líves þan
þuru ęggja níð \hld\ áhtian willjad,
fró mín þie guodo, \hld\ ik givu mín ferah furi þik
an wápno spil: \hld\ nis mi werð iowiht
te bi-míðanne, \hld\ só lango só mi mín warod
hugi endi hand-kraft.ʼ \hld\ Þuo sprak im eft is hérro an-gegin:
ʽhwat, þu þik bi-wánisʼ, \hld\ kwaþie, ʽwissaro trewono,
þrístero þingo: \hld\ þu havis þegnes hugi,
willjon guodan. \hld\ Ik mag þi sęggjan, hwó it þoh gi-werðan skal,
þat þu wirðis só wék-muod, \hld\ þoh þu nu ni wánjes só,
þat þu þínes þiadnes te naht \hld\ þríwo far-lógnis
ér hano-krádi endi kwiðis, \hld\ þak ik þín hérro ni sí,
ak þu far-manst mína mund-burd.ʼ \hld\ Þuo sprak eft þie man an-gegin:
ʽef it gio an weroldiʼ, \hld\ kwaþie, ʽgi-werðan muosti,
þat ik samad midi þi \hld\ sweltan muosti,
dójan diur-líko, \hld\ þan ne wurði gio þie dag kuman,
þat ik þín far-lógnidi, \hld\ lievo drohtin,
gerno for þeson Juðeon.ʼ \hld\ Þuo kwáðun alla þia jungron só,
þat sia þar an þem þingon mid im \hld\ þoljan weldin
Þuo im eft mid is wordon gi-bód \hld\ waldand selvo,
hér hevan-kuning, \hld\ þat sia im ni lietin iro hugi twíflian,
hiet þat sia ni weldin[...] \hld\ diopa gi-þáhti:
ʽne druovie iuwa herta \hld\ þuru iuwes drohtines word,
ne forohteat te filo: \hld\ ik skal fader úsan
selvan suokjan \hld\ endi iu sęndjan skal
fan hevan-ríkje \hld\ hèlagna gést:
þie skal iu eft gi-fruofrean \hld\ endi te frumu werðan,
manon iu þero mahlo, \hld\ þie ik iu manag hębbju
wordon gi-wísid. \hld\ Hie givit iu gi-wit an briost,
lust-sama léra, \hld\ þat gi léstian forð
þiu word endi þiu werk, þia ik iu an þesaro weroldi gi-bód.ʼ
Arés im þuo þe ríkjo an þemo rakode innan,
nęrjendo Krist endi gi-wét im nahtes þanan
selvo mid is gi-síðon: sèrago gengun
swíðo gornondia jungron Kristes,
hriwig-muoda. Þuo hie im an þena hòhan gi-wét
Oliueti-berg: þar was hie up gi-wuno
gangan mid is jungron. Þat wissa Judas wel,
balo-húgdig man, hwand hie was oft an þem berege mid im.
Þar gruotta þie godes suno iúgron sína:
ʽgi sind nu só druovjaʼ, kwaþie, ʽnu gi mínan dóð witun;
nu gornonð gi endi griotand, \hld\ endi þesa Juðeon sind an luston,
mendit þius menigi, \hld\ sindun an iro muode fráha,
þius werold ist an wunnjon. \hld\ Þes wirðit þoh gi-wand kuman
sniumo tulgo: \hld\ þan wirðit im sér hugi,
þan mornjat sia an iro móde, \hld\ endi gi mendjan skulun
after te éwon-dage, \hld\ hwand gio endi ni kumið,
iuwes wellíves gi-wand: \hld\ be-þiu ne þurvun iu þius werk tregan,
hrewan mín hin-fard, \hld\ hwand þanan skal þiu helpa kuman
gumono barnon.ʼ \hld\ Þuo hiet hie is jungron þar
bídan uppan þemo berge, \hld\ kwað þat hie ti bedu weldi
an þiu holm-klivu \hld\ hòhor stígan;
hiet þuo þria mid im þegnos gangan,
Jakobe endi Johannese endi þena guodan Petruse,
þríst-muodian þegan. Þuo sia mid iro þiedne samad
gerno gengun. Þuo hiet sia þie godes suno
an berge uppan te bedu hnígan,
hiet sia god gruotian, *gerno biddjan,
þat he im þero kostondero kraft far-stódi,
wréðaro willjon, þat im þe wiðer-sako,
ni mahti þe mén-skaðo mód gi-twíflean,
iak imu þó selvo gi-hnég sunu drohtines
kraftag an kniobeda, kuningo ríkjost,
forð-ward te foldu: fader aloþiado
gódan grótte, gorn-wordun sprak
hriwig-líko: was imu is hugi dróvi,
bi þeru męnniski mód gi-hrórid,
is flésk was an forhtun: fellun imo trahni,
dróp is diur-lík suét, al só dròr kumid
wallan fan wundun. Was an ge-winne þó
an þemu godes barne þe gést endi þe lík-hamo:
óðar was fúsid an forðwegos,
þe gést an godes ríki, óðar giámar stód,
lík-hamo Kristes: ni welde þit lioht ageven,
ak drovde for þemu dóðe. Simla he hreop te drohtine forð
þiu mér aftar þiu mahtigna grótte,
hòhan himil-fader, hèlagna god,
waldand mid is wordun: ʽef nu werðen ni magʼ, kwað he,
ʽman-kunni generid, ne sí þat ik mínan geve
liovan lík-hamon for liudjo barn
te wégjanne te wundrun, it sí þan þín willjo só,
ik willju is þan gi-koston: ik nimu þene kelik an hand,
drinku ina þi te diurðu, drohtin fró mín,
mahtig mund-boro. Ni seh þu mínes hér
fléskes gi-fórjes. Ik fullon skal
willjon þínen: þu haves ge-wald ovar al.ʼ
Gi-wét imu þó gangen, þar he ér is jungaron lét
bídan uppan þemu berge; fand sie þat barn godes
slápen sorgandie: was im sér hugi,
þes sie fan iro drohtine déljen skoldun.
Só sind þat módþraka manno ge-hwi-likumu,
þat he far-láten skal liavane hérron,
afgeven þene só gódene. Þó he te is jungarun sprak,
wahte sie waldand endi wordun grótte:
ʽhwí willjad gi só slápen?ʼ kwað he; ʽni mugun samad mid mi
wakon éne tíd? Þiu wurd is at handun,
þat it só gigangen skal, só it god fader
gi-markode mahtig. Mi nis an mínumu móde tweho:
mín gést is garu an godes willjan,
fús te faranne: mín flésk is an sorgun,
letid mik mín lík-hamo: léð is imu swíðo
wíti te þolonne. Ik þoh willjan skal
mínes fader ge-frummjen. hębbjad gi fasten hugi.ʼ
Gi-wét imu þó eft þanan óðer-síðu
an þene berg uppen te bedu gangan,
mári drohtin, endi þar só manag gi-sprak
gódoro wordo. Godes ęngil kwam
hèlag fan himile, is hugi fastnode,
beldide te þem bęndjun. He was an þeru bedu simla
forð an flíte endi is fader grótte,
waldand mid is wordun: ʽef it nu wesen ni magʼ, kwað he,
ʽmári drohtin, nevu ik for þit manno folk
þiodkwále þoloie, ik an þínan skal
willjan wonjan.ʼ Gi-wét imu þó eft þanen
sókjan is ge-síðos: fand sie slápandje,
grótte sie gáhun. Geng imu eft þanen
þriddjon síðu te bedu endi sprak þiod-kuning
al þiu selvon word, sunu drohtines,
te þemu alo-waldon fader, só he ér dede,
manode mahtigna manno frumana
swíðo niud-líko nęrjando Krist,
geng imu þó eft te þem is jungarun, grótte sie sáno:
ʽslápad gi endi restiadʼ, kwað he. ʽNu wirðid sniumo herod
kuman mid kraftu, þe mi far-kópot havad,
sundja lósan gi-sald.ʼ ge-síðos Kristes
wakodun þó aftar þem wordun endi gi-sáhun þó þat werod kuman
an þene berg uppen brahtmu þiu mikilon,
wréða wápan-berand. \hld\ Wísde im Judas,
gram-hugdig man; Judeon aftar sigun,
fíundo folk-skępi; dróg man fiur an gi-mang,
logna an lioht-fatun, lédde man faklon
brinnandja fan burg, þar sie an þene berg uppan
stigun mid strídu. Þea stędi wisse Judas wel,
hwar he þea liudi tó lédean skolde.
Sagde imu þó te tékne, þó sie þar tó fórun
þemu folke bi-foran, te þiu þat sie ni farfengin þar,
erlos óðren man: ʽik gangu imu at érist tóʼ, kwað he,
ʽkussiu ine endi kwaddiu: þat is Krist selvo.
Þene gi fáhen skulun folko kraftu,
binden ina uppan þemu berge endi ina te burg hinan
lédjen undar þea liudi: he is líves havad
mid is wordun farwerkod.ʼ werod síðode þó,
antat sie te Kriste kumane wurðun,
grim folk Judeono, þar he mid is jungarun stód,
mári drohtin: béd metodo-gi-skapu,
torhtero tídeo. Þó geng imu treulós man,
Judas te-gegnes endi te þemu godes barne
hnég mid is hóvdu endi is hérron kwedde,
kuste ina kraftagne endi is kwidi léste,
wísde ina þemu werode, al só he ér mid wordun ge-hét.
Þat þolode al mid gi-þuldiun þiodo drohtin,
waldand þesara weroldes endi sprak imu mid is wordun tó,
frágode ine frókno: \hld\ ʽbe-hwí kumis þu só mid þius folku te mi,
be-hwí lédis þu mi só þese liudi tó \hld\ endi mi te þesare léðan þiode sprekan,
far-kópos mid þínu kussu \hld\ under þit kunni Judeono,
meldos mi te þesaru menegi?ʼ \hld\ Geng imu þó wið þea man
wið þat werod óðar \hld\ endi sie mid is wordun fragn,
hwene sie mid þiu ge-síðju \hld\ sókjan kwámin
só niud-liko an naht, \hld\ ʽso gi willjan nòd frummjen
manno hwi-likumu.ʼ \hld\ Þó sprak imu eft þiu menegi an-gegin,
kwáðun þat im héljand \hld\ þar an þemu holme uppan
ge-wísid wári, \hld\ ʽþe þit gi-wer frumid
Judeo liudjun \hld\ endi ina godes sunu
selvon hétid. \hld\ Ina kwámun wi sókjan herod,
weldin ina gerno bi-geten: \hld\ he is fan Galileo lande,
fan Nazareth-burg.ʼ \hld\ Só im þó þe nęrjendjo Krist
sagde te sóðan, \hld\ þat he it selvo was,
só wurðun þó an forhtun \hld\ folk Judeono,
wurðun under-badode, \hld\ þat sie under bak fellun
alle efno sán, \hld\ erðe gi-sóhtun,
wiðer-wardes þat werod: \hld\ ni mahte þat word godes,
þie stemnie ant-standan: \hld\ wárun þoh só strídige man,
a-hliopun eft up an þemu holme, \hld\ hugi fastnodun,
bundun briost-gi-þáht, \hld\ gi-bolgane gengun
náhor mid níðu, \hld\ anttat sie þene nęrjendjon Krist
werodo bi-wurpun. \hld\ Stódun wíse man,
swíðo gornundie \hld\ giungaron Kristes
bi-foran þeru dereveon dádi \hld\ endi te iro drohtine sprákun:
ʽwári it nu þín willjoʼ, \hld\ kwáðun sie, ʽwaldand fró mín,
þat sie ús hér an speres ordun \hld\ spildien móstin
wápnun wunde, \hld\ þan ni wári ús wiht só gód,
só þat wi hér for úsumu drohtine \hld\ dóan móstin
beniðjun blékaʼ. \hld\ Þó gi-bolgan warð
snel swerd-þegan, \hld\ Símon Petrus,
well imu innan hugi, \hld\ þat he ni mahte énig word sprekan:
só harm warð imu an is hertan, \hld\ þat man is hérron þar
binden welde. \hld\ Þó he gi-bolgan geng,
swíðo þríst-mód þegan \hld\ for is þiodan standen,
hard for is hérron: \hld\ ni was imu is hugi twífli,
blóð an is breostun, \hld\ ak he is bil atóh,
swerd bi sídu, \hld\ slóg imu te-gegnes
an þene furiston fíund \hld\ folmo krafto,
þat þó Malkhus warð \hld\ mákjas ęggjun,
an þea swíðaron half \hld\ swerdu gimálod:
þiu hlust warð imu far-hawan, \hld\ he warð an þat hòvid wund,
þat imu heru-dròrag \hld\ hlear endi óre
bęni-wundun brast: \hld\ blód aftar sprang,
well fan wundun. \hld\ Þó was an is wangun skard
þe furisto þero fíundo. \hld\ Þó stód þat folk an rúm:
and-rédun im þes billes biti. \hld\ Þó sprak þat barn godes
selvo te Símon Petruse, \hld\ hét þat he is swerd dedi
skarp an skéðia: \hld\ ʽef ik wið þesa skola weldiʼ, kwað he,
ʽwið þeses werodes ge-win \hld\ wíg-saka frummjen,
þan manodi ik þene márjon \hld\ mahtigne god,
hèlagne fader \hld\ an himil-ríkja,
þat he mi só managan ęngil herod \hld\ ovana sandi
wíges só wísen, \hld\ só ni mahtin iro wápan-þręki
man adógen: \hld\ iro ni stódi gio su-lik męgin samad,
folkes gi-fastnod, þat im iro ferh aftar þiu
werðen mahti. Ak it havad waldand god,
alo-mahtig fader an óðar gi-markot,
þat wi gi-þolojan skulun, só hwat só ús þius þioda tó
bittres brengit: ni skulun ús belgan wiht,
wréðean wið iro ge-winne; hwand só hwe só wápno níð,
grimman gérheti wili gerno frummjen,
he swiltit imu \hld\ eft swerdes ęggjun,
dóit im bi-dròregan: \hld\ wi mid úsun dádjun ni skulun
wiht a-werdjan.ʼ \hld\ Geng he þó te þemu wundon manne,
legde mid listjun \hld\ lík te-samne,
hòvid-wundon, \hld\ þat siu sán gi-hélid warð,
þes billes biti, \hld\ endi sprak þat barn godes
wið þat wréðe werod: \hld\ ʽmi þunkid wunder mikilʼ, kwað he,
ʽef gi mi léðes wiht \hld\ léstjen weldun,
hwí gi mi þó ni fengun, \hld\ þan ik undar iuwomu folke stód,
an þemu wíhe innan \hld\ endi þar word manag
sóð-lík sagde. \hld\ Þan was sunnon skín,
diur-lik dages lioht, \hld\ þan ni weldun gi mi dóan eowiht
léðes an þesumu liohte, \hld\ endi nu lédjad mi iuwa liudi tó
an þiustrie naht, \hld\ al só man þiove dót,
þan man þene fáhan wili \hld\ endi he is ferhes havad
far-werkot, wam-skaðo.ʼ \hld\ werod Judeono
gripun þó an þene godes sunu, grimma þioda,
hatandiero hóp, hwurvun ina umbi
módag manno folk — ménes ni sáhun —,
heftun heru-bęndjun handi te-samne,
faðmos mid fitereun. Im ni was su-likaro firin-kwála
þarf te gi-þolonne, \hld\ þiod-arvedjes,
te winnanne su-lik wíti, ak he it þurh þit werod deda,
hwand he liudjo barn lósjen welda,
halon fan hęllju an himil-ríki,
an þene wídon welon: be-þiu he þes wiht ne bisprak,
þes sie imu þurh inwidníð ógean weldun.
Þó wurðun þes só malske módag folk Judeono,
þiu héri warð þes só hrómeg, þes sie þena hèlagon Krist
an liðo-bęndjon lédjan muostun,
fórjan an fiterjun. Þie fíund eft ge-witun
fan þemu berge te burg. Geng þat barn godes
undar þemu hęri-skępi \hld\ handun ge-bunden,
drúvondi te dale. \hld\ Wárun imu þea is diurion þó
ge-síðos ge-swikane, al só he im ér selvo gi-sprak:
ni was it þoh be énigaru blóði, þat sie þat barn godes,
lioven far-létun, ak it was só lango bi-foren
wár-sagono word, þat it skoldi gi-werðen só:
be-þiu ni mahtun sie is bemíðan. Þan aftar þeru menegi gengun
Johannes endi Petrus, þie gumon twéne,
folgodun ferrane: was im firi-wit mikil,
hwat þea grimmon Judeon þemu godes barne,
weldin iro drohtine dóen. Þó sie te dale kwámun
fan þemu berge te burg, þar iro biskop was,
iro wíhes ward, þar léddun ina wlanke man,
erlos undar ederos. Þar was éld mikil,
fiur an fríd-hove þemu folke te-gegnes,
ge-warht for þemu werode: þar gengun sie im wermien tó,
Judeo liudi, \hld\ létun þene godes sunu
bídon an bęndjun. \hld\ Was þar braht mikil,
gélmódigaro galm. \hld\ Johannes was ér
þemu héroston kúð: \hld\ be-þiu móste he an þene hof innan
þringan mid þeru þioda. \hld\ Stód allaro þegno bętsto,
Petrus þar úte: \hld\ ni lét ina þe portun ward
folgon is fróen, \hld\ ér it at is friunde abad,
Johannes at énumu Judeon, \hld\ þat man ina gangan lét
forð an þene fríd-hof. \hld\ Þar kwam im én fékni wíf
gangan te-gegnes, \hld\ þiu énas Judeon was,
iro þeodanes þiw, \hld\ endi þó te þemu þegne sprak
magað un-wán-lík: \hld\ ʽhwat, þu mahtis man wesanʼ, kwað siu,
ʽgiungaro fan Galilea, \hld\ þes þe þar genower stéd
faðmun gi-fastnod.ʼ \hld\ Þó an forhtun warð
Símon Petrus sán, \hld\ slak an is móde,
kwað þat he þes wíves \hld\ word ni bi-konsti
ni þes þeodanes \hld\ þegan ni wári:
méð is þó for þeru menegi, \hld\ kwað þat he þena man ni ant-kendi:
ʽni sind mi þíne kwidi kúðeʼ, \hld\ kwað he; was imu þiu kraft godes,
þe herdislo fan þemu hertan. \hld\ Huaravondi geng
forð undar þemu folke, \hld\ antat he te þemu fiure kwam;
gi-wét ina þó warmien. \hld\ Þar im ók én wíf bi-gan
felgian firin-spráka: ʽhér mugun giʼ, kwað siu, ʽan iuwan fíund sehan:
þit is gegnungo giungaro Kristes,
is selves ge-síð.ʼ Þó gengun imu sán aftar þiu
náhor níð-hwata endi ina niud-líko
frágodun fíundo barn, hwi-likes he folkes wári:
ʼni bist þu þesoro burg-liudjoʼ, kwáðun sie; ʽþat mugun wi an þínumu gi-bárie gi-sehan,
an þínun wordun endi an þínaru wíson, þat þu þeses werodes ni bist,
ak þu bist galiléisk man.ʼ He ni welda þes þó gehan eowiht,
ak stód þó endi strídda endi starkan éð
swíð-líko ge-swór, þat he þes ge-síðes ni wári.
Ni habda is wordo ge-wald: it skolde gi-werðen só,
só it þe ge-markode, þe man-kunnjes
far-wardot an þesaru weroldi. Þó kwam imu ók an þemu warve tó
þes mannes mágwini, þe he ér mid is mákeo giheu,
swerdu þiu skarpon, kwað þat he ina sáhi þar
an þemu berge uppan, ʽþar wi an þemu bómgardon
hérron þínumu hendi bundun,
fastnodun is folmos.ʼ He þó þurh forhtan hugi
for-lógnide þes is lioves hérron, kwað þat he weldi wesan þes líves skolo,
ef it mahti énig þar irminmanno
gi-sęggjan te sóðan, þat he þes ge-síðes wári,
folgodi þeru ferdi. Þó warð an þena formon síð
hano-krád af-haven. Þó sah þe hèlago Krist,
barno þat bętste, \hld\ þar he ge-bunden stóð,
selvo te Símon Petruse, \hld\ sunu drohtines
te þemu erle ovar is ahsla. \hld\ Þó warð imu an innan sán,
Símon Petruse \hld\ sér an is móde,
harm an is hertan \hld\ endi is hugi dróvi,
swíðo warð imu an sorgun, \hld\ þat he ér selvo ge-sprak:
gi-hugde þero wordo þó, \hld\ þe imu ér waldand Krist
selvo sagda, \hld\ þat he an þeru swartan naht
ér hano-krádi \hld\ is hérron skoldi
þríwo far-lógnien. \hld\ Þes þram imu an innan mód
bittro an is breostun, \hld\ endi geng imu þó gi-bolgan þanen
þe man fan þeru menigi \hld\ an mód-karu,
swíðo an sorgun, \hld\ endi is selves word,
wam-skęfti weop, \hld\ antat imu wallan kwámun
þurh þea hert-kara \hld\ héte trahni,
blódage fan is breostun. \hld\ He ni wánde þat he is mahti gi-bótjen wiht,
firin-werko furður efþa te is fráhon kuman,
hérron huldi: nis énig hęliðo só ald,
þat io mannes sunu mér gi-sáhi
is selves word sérur hrewan,
karon efþa kúmien: ʽwola krafteg godʼ, kwað he,
þat ik hębbju mi só forwerkot, só ik mínaro weroldes ni þarf
ólat sęggjan. Ef ik nu te aldre skal
huldeo þínaro endi heven-ríkjas,
þeoden, þolojan, þan ni þarf mi þes énig þank wesan,
liovo drohtin, þat ik io te þesumu liohte kwam.
Ni bium ik nu þes wirðig, \hld\ waldand fró mín,
þat ik under þíne jungaron \hld\ gangan móti,
þus sundig under þíne ge-síðos: \hld\ ik iro selvo skal
míðan an mínumu móde, \hld\ nu ik mi su-lik mén ge-sprak.ʼ
Só gornode \hld\ gumono bętsta,
hrau im só hardo, \hld\ þat he habde is hérren þó
leoves far-lógnid. \hld\ Þan ni þurvun þes liudjo barn,
weros wundrojan, \hld\ be-hwí it weldi god,
þat só lioven man léð gi-stódi,
þat he só hónlíko hérron sínes
þurh þera þiwun word, þegno snellost,
far-lógnide só lioves: it was al bi þesun liudjun gi-duan,
firiho barnun te frumu. He welde ina te furiston dóan,
hérost ovar is híwiski, hèlag drohtin:
lét ina ge-kunnon, hwi-like kraft havet
þe męnniska mód áno þe maht godes;
lét ina ge-sundjon, þat he síðor þiu bet
liudjun gi-lóvdi, hwó liof is þar
manno gi-hwi-likumu, \hld\ þan he mén ge-frumit,
þat man ina a-láte \hld\ léðes þinges,
sakono endi sundjono, \hld\ só im þó selvo dede
heven-ríki god \hld\ harm-ge-wurhti.
Be þiu nis mannes bág \hld\ mikilun bi-þervi,
hagu-staldes hróm: \hld\ ef imu þiu helpe godes
ge-swíkid þurh is sundjon, \hld\ þan is imu sán aftar þiu
breost-hugi blóðora, \hld\ þoh he ér bi-hét spreka,
hrómje fan is hildi \hld\ endi fan is hand-krafti,
þe man fan is męgine. \hld\ Þat warð þar an þemu márjon skín,
þegno bętston, \hld\ þó imu is þiodanes gi-swék
hèlag helpe. \hld\ Be-þiu ni skoldi hrómjen man
te swíðo fan imu selvon, \hld\ hwand imu þar swíkid oft
wán endi willjo, \hld\ ef imu waldand god,
hér heven-kuning \hld\ herte ni sterkit.
Þan béd allaro barno bętst, \hld\ bendi þolode
þurh man-kunni. \hld\ Hwurvun ina managa umbi
Judeono liudi, \hld\ sprákun gelp mikil,
habdun ina te hoska, \hld\ þar he gi-heftid stód,
þolode mid ge-þuldiun, \hld\ só hwat só imu þiu þiod deda,
liudi léðes. \hld\ Þó warð eft lioht kuman,
morgan te mannun. \hld\ Manag samnoda
heri Judeono: \hld\ habdun im hugi wulvo,
in-wid an innan. \hld\ Warð þar éosago
an morgan-tíd \hld\ manag gi-samnod
irri endi én-hard, \hld\ in-widjas gern,
wréðes willjan. \hld\ Gengun im an warf samad
rinkos an rúna, \hld\ bi-gunnun im rádan þó,
hwó sie ge-wísadin \hld\ mid wár-lósun,
mannun mén-ge-witun \hld\ an mahtigna Krist
te gi-sęggjanne sundja \hld\ þurh is selves word,
þat sie ina þan te wunder-kwálu \hld\ wégjan móstin,
a-déljen te dóðe. \hld\ Sie ni mahtun an þemu dage finden
só wréð ge-wit-skępi, \hld\ þat sie imu wíti be-þiu
a-déljen gi-dorstin \hld\ efþa dóð frummjen,
lívu bi-lósjen. \hld\ Þó kwámun þar at latstan forð
an þena warf wero \hld\ wár-lóse man
twéne gangan \hld\ endi bi-gunnun im tęlljen an,
kwáðun þat sie ina selvon \hld\ sęggjan gi-hórdin,
þat he mahti te-werpen \hld\ þena wíh godes,
allaro húso hòhost \hld\ endi þurh is hand-męgin,
þurh is énes kraft \hld\ up a-rihtjen
an þriddjon daga, só is elkor ni þorfti beþíhan man.
He þagoda endi þoloda: ni sprak imu io þiu þiod só filu,
þea liudi mid luginun, þat he it mid léðun an-gegin
wordun wráki. Þó þar undar þemu werode a-rés
baluhugdig man, biskop þero liudjo,
þe furisto þes folkes endi frágode Krist
iak ina be imu selvon biswór swíðon éðun,
grótte ina an godes namon endi gerno bad,
þat he im þat gi-sagdi, ef he sunu wári
þes libbjendjes godes: \hld\ ʽþes þit lioht ge-skóp,
Krist kuning éwig. \hld\ Wi ni mugun is ant-kiennjen wiht
ne an þínun wordun ni an þínun werkun.ʼ \hld\ Þó sprak imu eft þe wáro an-gegin,
þe gódo godes sunu: \hld\ ʽþu kwiðis it for þesun Judeon nu,
sóð-líko segis, \hld\ þat ik it selvo bium.
Þes ni gi-lóvjad mi þese liudi: \hld\ ni willjad mi for-látan be-þiu;
ni sind im mín word wirðig. \hld\ Nu sęggju ik iu te wárun þoh,
þat gi noh skulun sittjen gi-sehan \hld\ an þe swíðaron half godes
márjan mannes sunu, \hld\ an męgin-krafte
þes alo-walden fader, \hld\ endi þanan eft kuman
an himil-wolknun herod \hld\ endi allumu hęliðo kunnje
mid is wordun a-déljen, \hld\ al só iro ge-wurhti sind.ʼ
Þo balg ina þe biskop, \hld\ habde bittren hugi,
wréðida wið þemu worde \hld\ endi is gi-wádi slét,
brak for is breostun: \hld\ ʽnu ni þurvun gi bídan lengʼ, kwað he,
ʽþit werod ge-wit-skępjes, \hld\ nu im su-lik word farad,
ménspráka fan is múðe. \hld\ Þat gi-hórid hér nu manno filu,
rinko an þesumu rakude, \hld\ þat he ina só ríkjan telit,
gihid þat he god sí. \hld\ Hwat willjad gi Judeon þes
a-déljen te dóme? \hld\ Is he dóðes nu
wirðig be su-likun wordun?ʼ \hld\ Þat werod al ge-sprak,
folk Judeono, \hld\ þat he wári þes ferhes skolo,
wítjes só wirðig. \hld\ Ni was it þoh be is ge-wurhtiun gidóen,
þat ine þar an Hierusalem \hld\ Judeo liudi,
sunu drohtines \hld\ sundja lósen
a-déldun te dóðe. \hld\ Þó was þero dádjo hróm
Judeo liudjun, \hld\ hwat sie þemu godes barne mahtin
só haftemu mést, \hld\ harmes ge-frummjen.
Be-wurpun ina þó mid werodu \hld\ endi ina an is wangon slógun,
an is hleor mid iro handun \hld\ —al was imu þat te hoske gi-dóen—,
felgidun imu firin-word \hld\ fíundo menegi,
bismer-spráka. \hld\ Stód þat barn godes
fast under fíundun: \hld\ wárun imu is faðmos ge-bundene,
þolode mid gi-þuldiun, \hld\ só hwat só imu þiu þioda tó
bittres bráhte: \hld\ ni balg ina neowiht
wið þes werodes ge-win. \hld\ Þó námon ina wréðe man
só gi-bundanan, \hld\ þat barn godes,
endi ina þó léddun, \hld\ þar þero liudjo was,
þere þiade þing-hús. \hld\ Þar þegan manag
hwurvun umbi iro hęri-togon. \hld\ Þar was iro hérron bodo
fan Rúmu-burg, \hld\ þes þe þó þes ríkjas gi-weld:
kumen was he fan þemu késure, \hld\ gi-sendid was he undar þat kunni Judeono
te rihtjenne þat ríki, \hld\ was þar rád-gevo:
Pilatus was he héten; \hld\ he was fan Ponteo lande
knósles kennit. \hld\ Habde imu kraft mikil,
an þemu þing-húse \hld\ þiod gi-samnod,
an warf weros; \hld\ wár-lóse man
a-gávun þó þena godes sunu, \hld\ Judeo liudi,
under fíundo folk, \hld\ kwáðun þat he wári þes ferhes skolo,
þat man ina wítnodi \hld\ wápnes ęggjun,
skarpun skúrun. \hld\ Ni welde þiu skole Judeono
þringan an þat þing-hús, \hld\ ak þiu þiod úte stód,
mahlidun þanen wið þea menegi: \hld\ ni weldun an þat gi-mang faren,
an eli-landige man, \hld\ þat sie þar un-reht word,
an þemu dage dervjes wiht \hld\ a-déljan ne gi-hórdin,
ak kwáðun þat sie im só hluttro \hld\ hèlaga tídi,
weldin iro paskha halden. \hld\ Pilatus ant-feng
at þem wam-skaðun \hld\ waldandes barn,
sundja lósen. \hld\ Þó an sorgun warð
Judases hugi, \hld\ þó he a-gevan gi-sah
is drohtin te dóðe, \hld\ þó bi-gan imu þiu dád aftar þiu
an is hugja hrewan, \hld\ þat he habde is hérron ér
sundja lósen gi-sald. \hld\ Nam imu þó þat siluvar an hand,
þrítig skatto, \hld\ þat man imu ér wið is þiodane gaf,
geng imu þó te þem Judiun \hld\ endi im is grimmon dád,
sundjon sagde, \hld\ endi im þat siluvar bód
gerno te agevanne: \hld\ ʽik hębbju it só griolíkoʼ, kwað he,
ʽmines drohtines dròru gi-kópot,
só ik wét þat it mi ni þíhit.ʼ Þiod Judeono
ni weldun it þó ant-fáhan, ak hétun ina forð aftar þiu
umbi su-lika sundja selvon ahton,
hwat he wið is fráhon ge-frumid habdi:
ʽþu sáhi þi selvo þesʼ, kwaðun sie; ʽhwat wili þu þes nu sóken te ús?
Ne wít þu þat þesumu werode!ʼ Þó gi-wét imu eft þanan
Judas gangan te þemu godes wíhe
swíðo an sorgun endi þat siluvar warp
an þena alah innan, ne gi-dorste it égan leng;
fór imu þó só an forhtun, só ina fíundo barn
módage manodun: habdun þes mannes hugi
gramon under-gripanen, was imu god abolgan,
þat he imu selvon þó símon warhte,
hnég þó an herusél an hinginna,
warag an wurgil endi wíti ge-kòs,
hard hęllje ge-þwing, hét endi þiustri,
diap dóðes dalu, hwand he ér umbi is drohtin swék.
Þan béd þat barn godes — bendi þolode
an þemu þing-húse —, hwan ér þiu þiod under im,
erlos én-wordje alle wurðin,
hwat sie imu þan te ferah-kwálu frummjan weldin.
Þó þar an þem bęnkjun a-rés \hld\ bodo késures
fan Rúmu-burg endi geng imu wið þat ríki Judeono
módag mahlien, þar þiu menigi stód
aftar þemu hove hwarvon: ni weldun an þat hús kuman
an þemu paskha-dage. Pilatus bi-gan
frókno frágon ovar þat folk Judeono,
mid hwiu þe man habdi morðes gi-skuldit,
wítjes gi-werkot: ʽbe hwí gi imu só wréðe sind,
an iuwomu hugja hótie?ʼ Sie kwáðun þat he im habdi harmes só filu,
léðes gi-léstid: ʽni gávin ina þesa liudi þi,
þar sie ina ér bi-foran \hld\ uvilan ni wissin,
wordun far-warhten. \hld\ He havat þeses werodes só filu
far-lédid mid is lérun \hld\ —endi þesa liudi merrid,
dóit im iro hugi twíflien—, \hld\ þat wi ni mótun te þemu hove késures
tinsi gelden; \hld\ þat mugun wi ina gi-tęlljen an
mid wáru ge-wit-skępi. \hld\ He sprikid ók word mikil,
kwiðit þat he Krist sí, \hld\ kuning ovar þit ríki,
be-gihit ina só grótes.ʼ \hld\ Þó im eft te-gegnes sprak
bodo késures: \hld\ ʽef he só bar-líkoʼ, kwað he,
ʽunder þesaru menigi \hld\ mén-werk frumid,
ant-fáhad ina þan eft under iuwe folk-skępi, ef he sí is ferhes skolo,
endi imu só a-déljad, ef he sí dóðes werð,
só it an iuwaro aldrono éo ge-biode.ʼ
Sie kwáðun þó, þat sie ni móstin manno nigénumu
an þea hèlagon tíd te hand-banon,
werðen mid wápnun an þemu wíh-dage.
Þó wende ina fan þemu werode wréðhugdig man,
þegan késures, þe ovar þea þioda was
bodo fan Rúmu-burg —: hét imu þó þat barn godes
náhor gangan endi ina niud-líko,
frágoda frókno, ef he ovar þat folk kuning
þes werodes wári. Þó habde eft is word garu
sunu drohtines: ʽhweðer þu þat fan þi selvumu sprikisʼ, kwað he,
ʽþe it þi óðre hér erlos sagdun,
kwáðun umbi mínan kuning-duom?ʼ Þó sprak eft þe késures bodo
wlank endi wréð-mód, þar he wið waldand Krist
reðiode an þem rakude: ʽni bium ik þeses ríkjes hinanʼ, kwað he,
ʽGiudeo liudjo, ni gadoling þín,
þesaro manno mágwini, ak mi þi þius menigi bi-falah,
a-gávun þi þína gadulingos mi, Judeo liudi,
haftan te handun. Hwat havas þu harmes gi-duan,
þat þu só bittro skalt bendi þolojan,
kwalm undar þínumu kunnje?ʼ Þó sprak imu eft Krist an-gegin,
hélendero bętst, þar he gi-heftid stód
an þemu rakude innan: ʽnis mín ríki hinanʼ, kwað he,
ʽfan þesaru werold-stundu. Ef it þoh wári só,
þan wárin só stark-móde wiðer stríd-hugi,
wiðer grama þioda jungaron míne,
só man mi ni gávi Judeo liudjun,
hettendiun an hand an heru-bęndjun
te wégjanne te wundrun. Te þiu warð ik an þesaru weroldi gi-boran,
þat ik ge-wit-skępi giu wáres þinges
mid mínun kumiun kúðdi. Þat mugun ant-kęnnjen wel
þe weros, þe sind fan wáre kumane: þe mugun mín word far-standen,
gi-lóvjen mínun lérun.ʼ Þó ni mahte lasteres wiht
an þem barne godes bodo késures,
findan féknea word, þat he is ferhes be-þiu
skuldig wári. Þó geng he im eft wið þea skola Judeono
módag mahlien endi þeru menigi sagde
ovar hlust mikil, þat he an þemu hafton manne
su-lika firin-spráka finden ni mahti
for þem folk-skipje, só he wári is ferhes skolo,
dóðes wirðig. Þan stódun dolmóde
Judeo liudi endi þane godes sunu
wordun wrógdun: kwáðun þat he gi-wer érist
begunni an Galileo lande, ʽendi ovar Judeon fór
herod-wardes þanan, hugi twíflode,
manno mód-sevon, só he is morðes werð,
þat man ina wítnoie wápnes ęggjun,
ef eo man mid su-likun dádjun mag dóðes ge-skuldjen.ʼ
Só wrógdun ina mid wordun werod Judeono
þurh hótean hugi. Þó þe hęri-togo,
slíð-módig man sęggjan gi-hórde,
fan hwi-likumu kunnje was Krist afódid,
manno þe bętsto: he was fan þeru márjan þiadu,
þe gódo fan Galilea-lande; þar was gum-skępi
eðiliero manno; Erodes bi-held þar
kraftagne kuning-dóm, só ina imu þe késur far-gaf,
þe ríkjo fan Rúmu, þat he þar rehto ge-hwi-lik
ge-frumidi undar þemu folke endi friðu lésti,
dómos a-déldi. He was ók an þemu dage selvo
an Hierusalem mid is gum-skępi,
mid is werode at þemu wíhe: só was iro wíse þan,
þat sie þar þia hèlagun tíd haldan skoldun,
paskha Judeono. Pilatus gi-bód þó,
þat þena hafton man hęliðos námin
só gi-bundanan, þat barn godes,
hét þat sie ina Erodese, erlos bráhtin
haften te handun, hwand he fan is hęri-skępi was,
fan is werodes ge-wald. \hld\ Wígand frumidun
iro hérron word: \hld\ hèlagne Krist
fórdun an fiteriun \hld\ for þena folk-togun,
allaro barno bętst, \hld\ þero þe io gi-boren wurði
an liudjo lioht; \hld\ an liðu-bęndjun geng,
antat sie ina bráhtun, \hld\ þar he an is bęnkja sat,
kuning Erodes: \hld\ umbi-hwarf ina kraft wero,
wlanke wígandos: \hld\ was im willjo mikil,
þat sie þar selvon Krist \hld\ gi-sehan móstin:
wándun þat he im sum tékạn \hld\ þar tógjan skoldi,
mári endi mahtig, \hld\ só he managun dede
þurh is god-kundi \hld\ Judeo *liudjon.
Frágoda ina þuo þie folk-kuning \hld\ firi-wit-líko
managon wordon, \hld\ wolda is muod-sevon
forð undar-findan, \hld\ hwat hie te frumu mohti
mannon gi-markon. \hld\ Þan stuod mahtig Krist,
þagoda endi þoloda: \hld\ ne wolda þem þied-kuninge,
Erodese ne is erlon \hld\ ant-swór gevan
wordo nigénon. \hld\ Þan stuod þiu wréða þiod,
Judeo liudi \hld\ endi þena godes suno
wurrun endi wruogdun, \hld\ anþat im warð þie werold-kuning
an is huge huoti \hld\ endi all is hęri-skipi,
far-muonstun ina an iro muode: \hld\ ne ant-kendun maht godes,
himiliskan hérron, \hld\ ak was im iro hugi þiustri,
baluwes gi-blandan. \hld\ Barn drohtines
iro wréðun werk, \hld\ word endi dádi
þuru ód-muodi \hld\ all gi-þoloda,
só hwat só sia im tionono þuo \hld\ tuogjan woldun.
Sia hietun im þuo te hoske \hld\ hwít gi-wádi
umbi is liði lęggjan, \hld\ þiu mér hie wurði þem liudjon þar,
jungron te gamne. \hld\ Judeon faganodun,
þuo sia ina te hoske \hld\ hębbjan gi-sáhun,
erlos ovar-muoda. \hld\ Þuo senda ina eft þanan
Erodes se kuning \hld\ an þat óðer folk;
a-lédjan hiet ina lungra mann, \hld\ endi lastar sprákun,
felgidun im firin-word, \hld\ þar hie an feteron geng
bi-hlagan mid hosku: \hld\ ni was im hugi twífli,
neva hie it þuru ód-muodi \hld\ all gi-þoloda;
ne welda iro uvilun word \hld\ idug-lònon,
hosk endi harmkwidi. \hld\ Þuo bráhtun sia ina eft an þat hús innan,
an þia palenkja uppan, \hld\ þar Pilatus was
an þero þing-stędi. \hld\ Þegnos a-gávun
barno þat besta \hld\ banon te handon
sundi-lósjan, \hld\ só hie selvo gi-kòs:
welda manno barn \hld\ morðes a-tuomian,
nęrjan af nòdi. \hld\ Stuodun níð-hwata,
Judeon far þem gast-sęlje: \hld\ habdun sia gramono barn,
þia skola far-skundid, \hld\ þat sia ne be-skrivun iowiht
grimmera dádjo. \hld\ Þuo gi-wét im gangan þarod
þegan késures \hld\ wið þia þiod sprekan,
hard hęri-togo: \hld\ ʽhwat, gi mi þesan haftan mannʼ, kwaþie,
ʽan þesan sęli sendun \hld\ endi selvon an-budun,
þat hie iuwes werodes só filo \hld\ a-werdit habdi,
far-lédid mid is léron. \hld\ Nu ik mid þeson liudon ni mag,
findan mid þius folku, \hld\ þat hie is ferahes sí
furi þesaro skolu skuldig. \hld\ Skín was þat hiudu:
Erodes mohta, \hld\ þie iuwan éo bikan,
iuwaro liudo land-reht, \hld\ hie ni mahta is líves gi-fréson,
þat hie hier þuru éniga sundja te dage \hld\ sweltan skoldi,
líf far-látan. \hld\ Nu willju ik ina for þeson liudjon hier
gi-þróon mid þingon, \hld\ þrístion wordun,
buotjan im is briost-hugi, \hld\ látan ina brúkan forð
ferahes mid firjon.ʼ \hld\ Folk Judeono
hreopun þuo alla samad \hld\ hlúdero stemnu,
hietun flít-líko \hld\ ferahes áhtjan
Krist mid kwalmu \hld\ endi an krúki slahan,
wégian te wundron: \hld\ ʽhie mid is wordon havit
dóðes gi-skuldid: \hld\ sagit þat hie drohtin sí,
gegnungo godes suno. \hld\ Þat hie a-geldan skal,
inwid-spráka, \hld\ só is an úson éwe gi-skrivan,
þat man su-lika firin-kwidi \hld\ ferahu kópo.ʼ
Þuo warð þie an forahton, þie þes folkes gi-weld,
mikilon an is muode, þuo hie gi-hórda þia man sprekan,
þat sia ina selvon sęggjan gi-hórdin,
gehan fur þem gum-skipe, þat hie wári godes suno.
Þuo hwarf im eft þie hęri-togo an þat hús innan
te þero þing-stędi, þrístion wordon
gruotta þena godes suno endi frágoda, hwat hie gumono wári:
ʽhwat bist þu manno?ʼ kwaþie. ʽTe hwí þu mi só þínan muod hilis,
dernis diop-gi-þáht? wést þu þat it all an mínon duome stéd
umbi þínes líves gi-lagu? Mi þi hębbjat þesa liudi far-gevan,
werod Judeono, þak ik gi-waldan muot
só þik te spildjanne \hld\ an speres orde,
só ti kwęlljanne an krúkium, \hld\ só kwikan látan,
só hweðer só mi selvon \hld\ suotera þunkit
te gi-frummjanne mid mínu folku.ʼ Þuo sprak eft þat friðu-barn godes:
ʽwést þu þat te wáronʼ, kwaþie, ʽþat þu gi-wald ovar mik
hębbjan ni mohtis, ne wári þat it þi hèlag god
selvo far-gávi? \hld\ Ók hębbjat þia sundjono mér,
þia mik þi bi-fulhun \hld\ þuru fíond-skipi,
gi-saldun an símon haftan.ʼ \hld\ Þuo welda ina síð after þiu
gram-hugdig man \hld\ gerno far-látan,
þegan késures, \hld\ þar hie is havdi for þero þioda gi-wald;
ak sia weridun im þena willjon \hld\ wordu gi-hwi-liku,
kunni Judeono: \hld\ ʽne bist þuʼ, kwáðun sia, ʽþes késures friund,
þínon hérren hold, \hld\ ef þu ina hinan látis
síðon gi-sundon: \hld\ þat þi noh te soragan mag,
werðan te wíte, \hld\ hwand só hwe só su-lik word sprikit,
a-havið ina só hòho, \hld\ kwiðit þat hie hębbjan mugi
kuning-duomes namon, \hld\ ne sí þat ina im þie késur geve,
hie wirrid im is weruld-ríki \hld\ endi is word far-hugid,
far-man ina an is muode. \hld\ Be-þiu skalt þu su-lik mén wrekan,
hosk-word manag, \hld\ ef þu umbi þínes hérren ruokis,
umbi þínes fróhon friund-skipi, \hld\ þan skalt þu ina þiu ferhu beniman.ʼ
Þuo gi-hórda þie hęri-togo \hld\ þia héri Juðeono
þrégian fan is þiodne; \hld\ þuo hie far þero þing-stędi geng
selvo gi-sittjan, \hld\ þar gi-samnod was
só mikil warf werodes, \hld\ hiet waldand Krist
lédjan for þia liudi. \hld\ Langoda Judeon,
hwan ér sia þat hèlaga barn \hld\ hangon gi-sáwin,
kwęlan an krúkje; \hld\ sia kwáðun þat sia kuning óðran
ne havdin undar iro hęri-skipje, \hld\ nevan þena héran késar
fan Rúmu-burg: \hld\ ʽþie havit hier ríki over ús.
Be-þiu ni skalt þu þesan far-látan; \hld\ hie havit ús só filo léðes gi-sprokan,
farduan havit hie im mid is dádjon. \hld\ Hie skal dóð þolon,
wíti endi wundạr-kwála.ʼ \hld\ Werod Judeono
só manag mislík þing \hld\ an mahtigna Krist
sagdun te sundjun. \hld\ Hie swígondi stuod
þuru óð-muodi, \hld\ ne ant-wordida niowiht
wið iro wréðun word: \hld\ wolda þesa werold alla
lósjan mid is lívu: \hld\ bi-þiu liet hie ina þia léðun þiod
wégian te wundron, \hld\ all só iro willjo geng:
ni wolda im opan-líko \hld\ allon kúðjan
Judeo liudjon, \hld\ þat hie was god selvo;
hwand wissin sia þat te wáron, \hld\ þat hie su-lika gi-wald havdi
ovar þeson middil-gard, \hld\ þan wurði im iro muod-sevo
gi-blóðit an iro brioston: \hld\ þan ne gi-dorstin sia þat barn godes
handon ant-hrínan: \hld\ þan ni wurði hevan-ríki,
ant-lokan liohto mést \hld\ liudjo barnon.
Be-þiu méð hie is só an is muode, \hld\ ne lét þat manno folk
witan, hwat sia warahtun. \hld\ Þiu wurd náhida þuo,
mári maht godes \hld\ endi middi dag,
þat sia þia ferah-kwála \hld\ frummjan skoldun.
Þan lag þar ók an bęndjon \hld\ an þero burg innan
én ruof ręgin-skaðo, \hld\ þie habda under þem ríke só filo
morðes gi-rádan \hld\ endi man-slahta gi-frumid,
was mári męgin-þiof: \hld\ ni was þar is gi-mako hwergin;
was þar ók bi sínon \hld\ sundjon gi-hęftid,
Barrabas was hie hétan; \hld\ hie after þem burgion was
þuru is mén-dádi \hld\ manogon gi-kúðid.
Þan was land-wísa \hld\ liudjo Judeono,
þat sia iáro gi-hwen \hld\ an godes minnja
an þem hèlagon dage \hld\ énna haftan mann
a-biddjan skoldun, \hld\ þat im iro burges ward,
iro folk-togo \hld\ ferah far-gávi.
Þuo bi-gan þie hęri-togo \hld\ þia héri Judeono,
þat folk frágojan, \hld\ þar sia im fora stuodun,
hweðeron sia þero twejo \hld\ tuomjan weldin,
ferahes biddjan: \hld\ ʽþia hier an feteron sind
haft undar þeson hęri-skipje?ʼ \hld\ Þiu héri Judeono
habdun þuo þia aramun man \hld\ alla gi-spanana,
þat sia þemo land-skaðen \hld\ líf abádin,
gi-þingodin þem þiove, \hld\ þie oft an þiustria naht
wam gi-warahta, \hld\ endi waldand Krist
kwęlidin an krúkje. \hld\ Þuo warð þat kúð ovar all,
hwó þiu þiod havda duomos a-délid. \hld\ Þuo skoldun sia þia dád frummjan,
háhan þat hèlaga barn. \hld\ Þat warð þem hęri-togen
síðor te sorgon, \hld\ þat hie þia saka wissa,
þat sia þuru níð-skipi \hld\ nęrjendon Krist,
hatoda þiu héri, \hld\ endi hie im hórda te þiu,
warahta iro willjon: \hld\ þes hie wíti ant-feng,
lòn an þeson liohte \hld\ endi lang after,
wói síðor wann, \hld\ síðor hie þesa werold a-gaf. %NOTE: wói [sic.]
Þuo warð þas þie wréðo gi-waro, \hld\ wam-skaðono mést,
Satanas selvo, \hld\ þuo þiu seola kwam
Judases an grund \hld\ grimmaro hęlljun—
þuo wissa hie te wáren, \hld\ þat þat was waldand Krist,
barn drohtines, \hld\ þat þar gi-bundan stuod;
wissa þuo te wáron, \hld\ þat hie welda þesa werold alla
mid is henginnja \hld\ hęllja gi-þwinges,
liudi a-lósjan \hld\ an lioht godes.
Þat was Satanase \hld\ sér an muode,
tulgo harm an is hugje: \hld\ welda is helpan þuo,
þat im liudjo barn \hld\ líf ne binámin,
ne kwęlidin an krúkje, \hld\ ak hie welda, þat hie kwik livdi,
te þiu þat firiho barn \hld\ fernes ne wurðin,
sundjono sikura. \hld\ Satanas gi-wét im þuo,
þar þes hęri-togen \hld\ híwiski was
an þero burg innan. \hld\ Hie þero is brúdi bi-gann,
þera idis opan-líko \hld\ un-hiuri fíond
wunder tógian, \hld\ þat sia an word-helpon
Kriste wári, \hld\ þat hie muosti kwik libbjan,
drohtin manno \hld\ —hie was iu þan te dóðe gi-skerid—
wissa þat te wáron, \hld\ þat hie im skoldi þia gi-wald biniman,
þat hie sia ovar þesan middil-gard \hld\ só mikila ni havdi,
ovar wída werold. \hld\ Þat wíf warð þuo an forahton,
swíðo an sorogon, \hld\ þuo iru þiu gi-siuni kwámun
þuru þes dernjen dád \hld\ an dages liohte,
an hęlið-helme bi-helid. \hld\ Þuo siu te iru hérren anbód,
þat wíf mid iro wordon \hld\ endi im te wáren hiet
selvon sęggjan, \hld\ hwat iro þar te gi-siunion kwam
þuru þena hèlagan mann, \hld\ endi im helpan bad,
formon is ferhe: \hld\ ʽik hębbju hier só filo þuru ina
seld-líkes gi-sewan, \hld\ só ik wét, þat þia sundjun skulun
allaro erlo gi-hwem \hld\ uvilo gi-þíhan,
só im fruokno tuo \hld\ ferahes áhtið.ʼ
Þie segg warð þuo an síðe, \hld\ antat hie sittjan fand
þena hęri-togon \hld\ an hwarave innan
an þem stén-wege, \hld\ þar þiu stráta was
felison gi-fuogid. \hld\ Þar hie te is fróhon geng,
sagda im þes wíves word. \hld\ Þuo warð im wréð hugi,
þem hęri-togen, \hld\ —hwaravoda an innan—,
gi-blóðit briost-gi-þáht: was im béðjes wé,
gie þat sea ina sluogin sundja lósan,
gie it bi þem liudjon þuo for-látan ne gi-dorsta
þuru þes werodes word. Warð im gi-wendid þuo
hugi an herten after þero héri Judeono,
te werkjanne iro willjon: ne wardoda im niewiht
þia swárun sundjun, þia hie im þar þuo selvo gideda.
Hiet im þuo te is handon dragan hluttran brunnion,
watar an wégie, þar hie furi þem werode sat,
þuóg ina þar for þero þioda þegan késures,
hard hęri-togo endi þuo fur þero héri sprak,
kwað þat hie ina þero sundjono þar sikoran dádi,
wréðero werko: ʽne willju ik þes wihtes pleganʼ, kwaþie,
ʽumbi þesan hèlagan mann, ak hleotad gi þes alles,
gie wordo gie werko, þes gi im hér te wítje gi-duan.ʼ
Þuo hreop all saman hęri-skipi Judeono,
þiu mikila menigi, kwáðun þat sia weldin umbi þena man plegan
deravoro dádjo: ʽfare is dròr ovar ús,
is bluod endi is baneði endi ovar úsa barn só samo,
ovar úsa avaron þar after — wi willjat is alles pleganʼ, kwaðun sia,
ʽumbi þena slegi selvon, — ef wi þar éniga sundja gi-duan!ʼ
Agevan warð þar þuo furi þem Judeon allaro gumono besta
hettendjon an hand, an heru-bęndjon
narawo gi-nòdid, þar ina níð-hwata,
fíond ant-fengun: folk ina umbi-hwarf,
mén-skaðono męgin. Mahtig drohtin
þoloda gi-þuldion, só hwat só im þiu þioda deda.
Sia hietun ina þuo fillian, ér þan sia im ferahes tuo,
aldres áhtin, endi im undar is ógun spiwun,
dedun im þat te hoske, þat sia mid iro handon slógun,
weros an is wangun endi im is gi-wádi binámun,
róvodun ina þia ręgin-skaðon, \hld\ ródes lakanes
dedun im eft óðer an \hld\ þuru un-huldi;
hietun þuo hòvid-band \hld\ hardaro þorno
wundron windan \hld\ endi an waldand Krist
selvon sęttjan, \hld\ endi gengun im þia gi-síðos tuo,
kweddun ina an kuning-wísu \hld\ endi þar an knio fellun,
hnigun im mid iro hóvdu: \hld\ all was im þat te hoske gi-duan,
þoh hie it all gi-þolodi, þiodo drohtin,
mahtig þuru þia minnja manno kunnjes.
Hietun sia þuo wirkian wápnes ęggjon
hęliðos mid iro handon hardes bómes
kraftiga krúki endi hietun sia Kristan þuo,
sálig barn godes selvon fuorian,
dragan hietun sia úsan drohtin, þar hie be-dròragad skolda
sweltan sundjono lós. Síðodun Judeon,
weros an willon, léddun waldand Krist,
drohtin te dóðe. Þar mohta man þuo derevi þing
harmlík gi-hórjan: hiovandi þar after
gengun wíf mid wópu, weros gnornodun,
þia fan Galilea mid im gangan kwámun,
folgodun ovar ferrwegos: was im iro fróhon dóð
swíðo an soragan. Þuo hie selvo sprak,
barno þat besta endi under bak besah,
hiet þat sia ni wépin: ʽni þarf iu wiht treganʼ, kwaþie,
ʽmínero hinferdio, ak gi mid hofnu mugun
iuwa wréðan werk wópu kúmian,
tornon trahnon. Noh wirðið þiu tíd kuman,
þat þia muoder þes mendendja sind,
brúdi Judeono, þem gio barn ni warð
ódan an aldre. Þan gi iuwa inwid skulun
grimmo angeldan; þan gi só gerna sind,
þat iu hier bi-hlídan hòha bergos,
diopo bedelvan; dóð wári iu þan allon
liovera an þeson lande þan su-lik liudjo kwalm
te gi-þoljanne, só hier þan þesaro þioda kumid.ʼ
Þuo sia þar an griete galgon rihtun,
an þem felde uppan folk Judeono,
bóm an berege, endi þar an þat barn godes
kwęlidun an krúkje: \hld\ slógun kald ísarn,
niwa naglos \hld\ níðon skarpa
hardo mid hamuron \hld\ þuru is hendi endi þuru is fuoti,
bittra bendi: \hld\ is blód ran an erða,
dròr fan úson drohtine. \hld\ Hie ni welda þoh þia dád wrekan
grimma an þem Judeon, \hld\ ak hie þes god fader
mahtigna bad, \hld\ þat hie ni wári þem manno folke,
þem werode þiu wréðra: \hld\ ʽhwand sia ni witun, hwat sia duotʼ, kwaþie.
Þuo þia wígandos \hld\ gi-wádi Kristes,
drohtines déldun, \hld\ derevia mann,
þes ríken giróbi. \hld\ Þia rinkos ni mahtun
umbi þena selvon {[...]} \hld\ sam-wurdi gi-sprekan,
ér sia an iro hwarave \hld\ hlótos wurpun,
hwi-lik iro skoldi hębbjan \hld\ þia hèlagun péda,
allaro gi-wádjo wun-samost. \hld\ Þes werodes hirdi
hiet þuo, þe hęri-togo, \hld\ ovar þem hóvde selves
Kristes an krúke skrívan, \hld\ þat þat wári kuning Judeono,
Jesus fan Nazareth-burh, \hld\ þie þar neglid stuod
an niwon galgon þuru \hld\ níð-skipi,
an bómin treo. \hld\ Þuo bádun þia liudi
þat word węndjan, \hld\ kwáðun þat hie im só an is willjon spráki,
selvo sagdi, \hld\ þat hie habdi þes gi-síðes gi-wald,
kuning wári ovar Judeon. \hld\ Þuo sprak eft þie késures bodo,
hard hęri-togo: \hld\ ʽit ist iu só ovar is hóvde gi-skrivan,
wíslíko gi-writan, \hld\ só ik it nu węndjan ni mag.ʼ
Dádun þuo þar te wítje \hld\ werod Judeono
twéna far-talda man \hld\ an twá halva
Kristes an krúki: \hld\ lietun sia kwalm þolon
an þem warạg-trewe \hld\ werko te lòne,
léðaro dádjo. \hld\ Þia liudi sprákun
hosk-word manag \hld\ hèlagon Kriste,
grottun ina mid gelpu: \hld\ sáwun allaro gumono þen beston
kwęlan an þemo krúkje: \hld\ ʽef þu sís kuning ovar allʼ, kwáðun sia,
ʽsuno drohtines, \hld\ só þu havis selvo gi-sprokan,
neri þik fan þero nòdi \hld\ endi níðes a-tuomi,
gang þi hél herod; \hld\ þan welliat an þik hęliðo barn,
þesa liudi gi-lóvjan.ʼ \hld\ Sum imo ók lastar sprak
swíðo gél-hert Judeo, \hld\ þar hie fur þem galgon stuod:
ʽwah warð þesaro weroldiʼ, \hld\ kwaþie, ʽef þu iro skoldis gi-wald égan.
Þu sagdas þat þu mahtis an énon dage \hld\ all te-werpan
þat hòha hús \hld\ hevan-kuninges,
stén-werko mést \hld\ endi eft standan gi-duon
an þriddjon dage, \hld\ só is elkor ni þorfti bi-þíhan mann
þeses folkes furðor. \hld\ Sínu hwó þu nu gi-fastnod stés,
swíðo gi-sérid: \hld\ ni maht þi selvon wiht
balowes gi-buotjan.ʼ \hld\ Þuo þar ók an þem bęndjon sprak
þero þeovo óðer, \hld\ all só hie þia þioda gi-hórda,
wréðon wordon \hld\ —ne was is willjo guod,
þes þegnes gi-þáht—: \hld\ ʽef þu sís þiod-kuningʼ, kwaþie,
ʽKrist, godes suno, \hld\ gang þi þan fan þem krúke niðer,
slópi þi fan þem símon \hld\ endi ús samad allon
hilp endi héli. \hld\ Ef þu sís hevan-kuning,
waldand þesaro weroldes, \hld\ gi-duo it þan an þínon werkon skín,
mári þik fur þesaro menigi.ʼ \hld\ Þuo sprak þero manno óðer
an þero hęnginna, \hld\ þar hie gi-hęftid stuod,
wan wunder-kwála: \hld\ ʽbe-hwí wilt þu su-lik word sprekan,
gruotis ina mid gelpu? \hld\ stés þi hier an galgen haft,
gi-brókan an bóme. \hld\ Wit hier béðia þolod
sér þuru unka sundjun: \hld\ is unk unkero selvero dád
worðan te wítje. \hld\ Hie stéd hier wammes lós,
allaro sundjono sikur, \hld\ só hie selvo gio
firina ni gi-frumida, \hld\ botan þat hie þuru þeses folkes nið
willendi an þesaro weruldi \hld\ wíti ant-fáhid.
Ik willju þar gi-lóvjan tuoʼ, \hld\ kwaþie, ʽendi willju þena landes ward,
þena godes suno \hld\ gerno biddjan,
þat þu mín gi-huggjes \hld\ endi an helpun sís,
rádendero best, \hld\ þan þu an þín ríki kumis:
wes mi þan gi-náðig.ʼ \hld\ Þuo sprak im eft nęrjendo Krist
wordon te-gegnes: \hld\ ʽik sęggju þi te wáron hierʼ, kwaþie,
ʽþat þu noh hiudu móst \hld\ an himil-ríke
mid mi samad \hld\ sehan lioht godes,
an þemo paradyse, \hld\ þoh þu nu an su-likoro pínu sís.ʼ
Þan stuod þar ók Maria, \hld\ muoder Kristes,
blék under þem bóme, \hld\ gi-sah iro barn þolon,
winnan wunder-kwála. \hld\ Ók wárun þar wíf mid iro
an só mahtiges \hld\ minnja kumana—
þan stuod þar ók Johannes, \hld\ jungro Kristes,
hriwi undar is hérren, \hld\ was im is hugi sèrag—
drúvodun fur þem dóðe. \hld\ Þar sprak drohtin Krist
mahtig te þero muoder: \hld\ ʽnu ik þi hier mínemo skal
jungron be-felhan, \hld\ þem þi hier gegin-ward stéd:
wis þi an is gi-síðje samad: \hld\ þu skalt ina furi suno hębbjan.ʼ
Grótta hie þuo Johannes, \hld\ hiet þat hie iru ful-gengi wel,
minnjodi sia só mildo, \hld\ só man is muoder skal,
idis un-wamma. \hld\ Þuo hie sia an is éra ant-feng
þuru hluttran hugi, \hld\ só im is hérro gi-bód.
Þuo warð þar an middjan dag \hld\ mahtig tékạn,
wundạr-lík gi-waraht \hld\ ovar þesan werold allan,
þuo man þena godes suno \hld\ an þena galgon huof,
Krist an þat krúki: \hld\ þuo warð it kúð ovar all,
hwó þiu sunna warð gi-sworkan: \hld\ ni mahta swigli lioht
skóni gi-skínan, \hld\ ak sia skado far-feng,
þimm endi þiustri \hld\ endi só gi-þrusmod neval.
Warð allaro dago druovost, \hld\ dunkar swíðo
ovar þesan wídun weruld, \hld\ só lango só waldand Krist
kwal an þemo krúkje, \hld\ kuningo ríkost,
ant nuon dages. \hld\ Þuo þie neval tiskréd,
þat gi-swerk warð þuo te-swungan, \hld\ bi-gan sunnun lioht
hédron an himile. \hld\ Þuo hreop up te gode
allaro kuningo kraftigost, \hld\ þuo hie an þemo krúkje stuod
faðmon gi-fastnot: \hld\ ʽfader alo-mahtigʼ, kwaþie,
ʽte hwí þu mik só far-lieti, \hld\ lievo drohtin,
hèlag hevan-kuning, \hld\ endi þína helpa dedos,
fullisti só ferr? \hld\ Ik standu under þeson fíondon hier
wundron gi-wégid.ʼ \hld\ Werod Judeono
hlógun is im þuo te hoske: \hld\ gi-hórdun þena hèlagun Krist,
drohtin furi þem dóðe \hld\ drinkan biddjan,
kwað þat ina þurstidi. \hld\ Þiu þioda ne latta,
wréða wiðar-sakon: \hld\ was im willjo mikil,
hwat sia im bittres tuo \hld\ bringan mahtin.
Habdun im un-swóti \hld\ ekid endi galla
gi-mengid þia mén-hwaton; \hld\ stuod én mann garo,
swíðo skuldig skaðo, \hld\ þena habdun sia gi-skerid te þiu,
far-spanan mid sprákon, \hld\ þat hie sia en éna spunsia nam,
líðo þes léðosten, \hld\ druog it an énon langan skafte,
gi-bundan an énon bóme \hld\ endi deda it þem barne godes,
mahtigon te múðe. \hld\ Hie an-kenda iro mirkiun dádi,
gi-fuolda iro fégnes: \hld\ furðor ni welda
is só bittres an-bítan, \hld\ ak hreop þat barn godes
hlúdo te þem himiliskon fader: \hld\ ʽik an þina hendi be-filhuʼ, kwaþie,
ʽmínon gést an godes willjon; \hld\ hie ist nu garo te þiu,
fús te faranne.ʼ \hld\ Firiho drohtin
gi-hnégida þuo is hòvid, \hld\ hèlagon áðom
liet fan þemo lík-hamen. \hld\ Só þuo þie landes ward
swalt an þem símon, \hld\ só warð sán after þiu
wundạr-tékạn gi-waraht, \hld\ þat þar waldandes dóð
un-kweðandes só filo \hld\ ant-kęnnjan skolda,
þiadnes éndagon: \hld\ erða bivoda,
hrisidun þia hòhun bergos, \hld\ harda sténos kluvun,
felisos after þem felde, \hld\ endi þat féha lakan tebrast
an middjon an twé, \hld\ þat ér managan dag
an þemo wíhe innan \hld\ wundron gi-striunid
hél hangoda \hld\ —ni muostun hęliðo barn,
þia liudi skawon, \hld\ hwat under þemo lakane was
hèlages be-hangan: \hld\ þuo mohtun an þat horð sehan
Judeo liudi— \hld\ gravu wurðun gi-opanod
dódero manno, \hld\ endi sia þuru drohtines kraft
an iro lík-hamon \hld\ libbjandi a-stuodun
up fan erðu \hld\ endi wurðun gi-ógida þar
mannon te márðu. \hld\ Þat was só mahtig þing,
þat þar Kristes dóð \hld\ ant-kęnnjan skoldun,
só filo þes gi-fuolian, \hld\ þie gio mid firihon ne sprak
word an þesaro weroldi. \hld\ Werod Judeono
sáwun seld-lík þing, \hld\ ak was im iro slíði hugi
só far-hardod an iro herten, \hld\ þat þar io só hèlag ni warð
tékạn gi-tógid, \hld\ þat sia trúodin þiu bat
an þia Kristes kraft, \hld\ þat hie kuning ovar all,
þes werodes wári. \hld\ Suma sia þar mid iro wordon gi-sprákun,
þia þes hréwes þar \hld\ huodian skoldun,
þat þat wári te wáren \hld\ waldandes suno,
godes gegnungo, \hld\ þat þar an þem galgon swalt,
barno þat besta. \hld\ Slógun an iro briost filo
wópjandero wívo: \hld\ was im þiu wunder-kwála
harm an iro herten \hld\ endi iro hérren dóð
swíðo an sorogon. \hld\ Þan was sido Judeono,
þat sia þia haftun þuru þena hèlagon dag \hld\ hangon ni lietin
lengerun hwíla, \hld\ þan im þat líf skriði,
þiu seola besunki: \hld\ slíð-muoda mann
gengun im mid níð-skipiu náhor, \hld\ þar só beneglida stuodun
þeovos twéna, \hld\ þolodun béðia
kwála bi Kriste: \hld\ wárun im kwika noh þan,
untþat sia þia grimmun \hld\ Judeo liudi
bénon be-brákon, \hld\ þat sia béðia samad
líf far-lietun, \hld\ suohtun im lioht óðer.
Sia ni þorftun drohtin Krist \hld\ dóðes bédjan
furðor mid énigon firinon: \hld\ fundun ina gi-faranan þuo iu:
is seola was gi-sendid \hld\ an suoðan weg,
an lang-sam lioht, \hld\ is liði kuolodun,
þat ferah was af þem fléske. \hld\ Þuo geng im én þero fíondo tuo
an níð-hugi, \hld\ druog negilid sper
hard an is handon, \hld\ mid heru-þrummjon stak,
liet wápnes ord \hld\ wundum sníðan,
þat an selves warð \hld\ sídu Kristes
ant-lokan is lík-hamo. \hld\ Þia liudi gi-sáwun,
þat þanan bluod endi water \hld\ béðju sprungun,
wellun fan þero wundun, \hld\ all só is willjo geng
endi hie habda gi-markod ér \hld\ manno kunnje,
firiho barnon te frumu: \hld\ þuo was it all gi-fullid só.
Só þuo gi-ségid warð \hld\ seðle náhor
hédra sunna \hld\ mid hevan-tunglon
an þem druoven dage, \hld\ þuo geng im úses drohtines þegan
—was im glau gumo, \hld\ jungro Kristes
managa hwíla, \hld\ só it þar manno filo
ne wissa te wáron, \hld\ hwand hie it mid is wordon hal
Juðeono gum-skipje: \hld\ Joseph was hie hétan,
darnungo was hie úses drohtines jungro: \hld\ hie ni welda þero far-duanun þiod
folgon te énigon firin-werkon, \hld\ ak hie béd im under þem folke Judeono,
hèlag himilo ríkjes— \hld\ hie geng im þuo wið þena hęri-togon mahlian,
þingon wið þena þegan késures, \hld\ þigida ina gerno,
þat hie muosti a-lósjan \hld\ þena lík-hamon
Kristes fan þemo krúkje, þie þar gi-kwelmid stuod,
þes guoden fan þem galgen endi an graf lęggjan,
foldu bi-felahan. Im ni welda þie folk-togo þuo
wernjan þes willjen, ak im gi-wald far-gaf,
þat hie só muosti gi-frummjan. Hie gi-wét im þuo forð þanan
gangan te þem galgon, þar hie wissa þat godes barn,
hréo hangondi hérren sínes,
nam ina þuo an þero niwun ruodun endi ina fan naglon a-tuomda,
ant-feng ina mid is faðmon, só man is fróhon skal,
lioves lík-hamon, endi ina an líne bi-wand,
druog ina diur-líko —só was þie drohtin werð—,
þar sia þia stędi havdun an énon sténe innan
handon gi-hauwan, þar gio hęliðo barn
gumon ne bi-gruovon. Þar sia þat godes barn
te iro land-wísu, líko hélgost
foldu bi-fulhun endi mid énu felisu belukun
allaro gravo guod-líkost. Griotandi sátun
idisi arm-skapana, þia þat all for-sáwun,
þes gumen grimman dóð. \hld\ Gi-witun im þuo gangan þanan
wópjandi wíf \hld\ endi wara námun,
hwó sia eft te þem grave \hld\ gangan mahtin:
havdun im far-sewana \hld\ soroga gi-nuogia,
mikila muod-kara: \hld\ Maria wárun sia hétana,
idisi arm-skapana. \hld\ Þuo warð ávand kuman,
naht mid neflu. \hld\ Niðfolk Judeono
warð an moragan eft, \hld\ menigi gi-samnod,
rekidun an rúnon: \hld\ ʽhwat, þu wést, hwó þit ríki was
þuru þesan énan man \hld\ all gi-twíflid,
werod gi-worran: \hld\ nu ligid hie wundon siok,
diopa bi-dolvan. \hld\ Hie sagda simnen, þat hie skoldi fan dóðe a-standan
an þriddjan dage. \hld\ Þius þiod gi-lóvit te filo,
þit werod after is wordon. \hld\ Nu þu hier wardon hét,
ovar þem grave gómjan, \hld\ þat ina is jungron þar
ne far-stelan an þemo sténe \hld\ endi sęggjan þan, þat hie a-standan sí,
ríki fan raston: \hld\ þan wirðit þit rinko folk
mér gi-merrid, \hld\ ef sia it biginnat márjan hier.ʼ
Þuo wurðun þar gi-skerida \hld\ fan þero skolu Judeono
weros te þero wahtu: \hld\ gi-witun im mid iro gi-wápnion þarod
te þem grave gangan, \hld\ þar sia skoldun þes godes barnes
hréwes huodian. \hld\ Warð þie hèlago dag
Judeono far-gangan. \hld\ Sia ovar þemo grave sátun,
weros an þero wahtun \hld\ wannom nahton,
bidun undar iro bordon, \hld\ hwan ér þie berẹhto dag
ovar middil-gard \hld\ mannon kwámi,
liudon te liohte. \hld\ Þuo ni was lang te þiu,
þat þar warð þie gést kuman \hld\ be godes krafte,
hálag áðom \hld\ undar þena hardon stén
an þena lík-hamon. \hld\ Lioht was þuo gi-opanod
firiho barnon te frumu: \hld\ was ferkal manag
ant-hęftid fan hęll-doron \hld\ endi te himile weg
gi-waraht fan þesaro weroldi. \hld\ Wánom up astuod
friðu-barn godes, \hld\ fuor im þuo þar hie welda,
só þia wardos þes \hld\ wiht ni af-swovun,
dervia liudi, \hld\ hwan hie fan þem dóðe astuod,
a-rés fan þero rastun. \hld\ Rinkos sátun
umbi þat graf útan, \hld\ Judeo liudi,
skola mid iro skildion. \hld\ Skréd forð-wardes
swigli sunnun lioht. \hld\ Síðodun idisi
te þem grave gangan, \hld\ gum-kunnjes wíf,
Mariun muni-líka: \hld\ habdun méðmo filo
gi-sald wiðer salvum, \hld\ siluvres endi goldes,
werðes wiðer wurtjon, \hld\ só sia mahtun a-winnan mést,
þat sia þena lík-hamon \hld\ lioves hérren,
suno drohtines, \hld\ salvon muostin,
wundun writanan. \hld\ Þiu wíf soragodun
an iro sevon swíðo, endi suma sprákun,
hwie im þena grótan stén fan þemo grave skoldi
gi-hwerevian an halva, þe sia ovar þat hréo sáwun
þia liudi lęggjan, þuo sia þena lík-hamon þar
be-fulhun an þemo felise. Só þiu frí havdun
ge-gangan te þem gardon, þat sia te þem grave mahtun
gi-sehan selvon, þuo þar suógan kwam
ęngil þes alo-waldon ovana fan radure,
faran an feðer-hamon, þat all þiu folda an skian,
þiu erða dunida endi þia erlos wurðun
an wékan hugje, wardos Juðeono,
bi-fellun bi þem forahton: ne wándun ira ferah égan,
líf langerun hwíl. \hld\ Lágun þa wardos,
þia gi-síðos sámkwika: sán up a-hléd
þie gróto stén fan þem grave, só ina þie godes ęngil
gi-hwerivida an halva, endi im uppan þem hléwe gi-sat
diur-lík drohtines bodo. Hie was an is dádjon ge-lík,
an is an-siunion, só hwem só ina muosta undar is ógon skawon,
só berẹht endi só blíði all só bliksmun lioht;
was im is gi-wádi wintạr-kaldon
snéwe gi-líkost. Þuo sáwun sia ina sittjan þar,
þiu wíf uppan þem gi-wendidan sténe, \hld\ endi im fan þem wlitje kwámun,
þem idison su-lika egison te-gegnes: \hld\ all wurðun fan þem grurie
þiu frí an forahton mikilon, furðor ne gi-dorstun
te þemo grave gangan, ér sia þie godes ęngil,
waldandes bodo wordon gruotta,
kwað þat hie iro árundi all bi-kunsti,
werk endi willjon endi þero wívo hugi,
hiet þat sia im ne and-rédin: ʽik wét þat gi iuwan drohtin suokat,
nęrjendon Krist fan Nazareth-burg,
þena þi hier kwęlidun endi an krúki slógun
Judeo liudi endi an graf lagdun
sundi-lósjan. Nu nist hie selvo hier,
ak hie ist a-standan iu, \hld\ endi sind þesa stędi lárja, %NOTE: a-sandan] L 1r.
þit graf an þeson griote. \hld\ Nu mugun gi gangan herod
náhor mikilu \hld\ —ik wét þat is iu ist niud sehan
an þeson sténe innan—: \hld\ hier sind noh þia stędi skína,
þar is lík-hamo lag.ʼ \hld\ Lungra fengun
gi-bada an iro brioston \hld\ bléka idisi,
wliti-skóni wíf: \hld\ was im wil-spell mikil
te gi-hórjanne, \hld\ þat im fan iro hérren sagda
ęngil þes alo-walden. \hld\ Hiet sia eft þanan
fan þem grave gangan endi faran \hld\ te þem jungron Kristes,
sęggjan þem is gi-síðon \hld\ suoðon wordon,
þat iro drohtin was \hld\ fan dóðe a-standan.
Hiet ók an sundron \hld\ Símon Petruse
will-spell mikil \hld\ wordon kúðjan,
kumi drohtines, \hld\ gie þat Krist selvo
was an Galileo land, \hld\ ʽþar ina eft is jungron skulun,
gi-sehan is gi-síðos, \hld\ só hie im ér selvo gi-sprak
wárom wordon.ʼ \hld\ Reht só þuo þiu wíf þanan
gangan weldun, \hld\ só stuodun im te-gegnes þar
ęngilos twéna \hld\ an ala-hwíton
wánamon gi-wádjom \hld\ endi sprákun im mid iro wordon tuo
hèlag-líko: \hld\ hugi warð gi-blóðid
þen idison an egison: \hld\ ne mahtun an þia ęngilos godes
bi þemo wlite skawon: \hld\ was im þiu wánami te strang, %NOTE: strang] L 1v.
te swíði te sehanne. \hld\ Þuo sprákun \edtext{im sán}{\Afootnote{so C; om. L}} an-gegin
waldandes bodun \hld\ endi þiu wíf frágodun,
te hwí sia Kristan þarod \hld\ kwikan mid dódon,
suno drohtines \hld\ suokjan kwámin
ferahes fullan; \hld\ ʽnu gi ina ni findat hier
an þeson stén-grave, \hld\ ak hie ist a-standan nu
an is lík-hamon: \hld\ þes gi gi-lóvjan skulun
endi gi-huggjan þero wordo, \hld\ þe hie iu te wáron oft
selvo sagda, \hld\ þan hie an iuwon ge-síðja was
an Galilea-lande, \hld\ hwó hie skoldi gi-gevan werðan,
gi-sald selvo \hld\ an sundigaro manno,
hettjandero hand, \hld\ hèlag drohtin,
þat sea ina kwęlidin \hld\ endi an krúki slógin,
dódan gi-dádin \hld\ endi þat hie skoldi þuruh drohtines kraft
an þriddjon dage \hld\ þioda te willjan
libbjandi a-standan. \hld\ Nu havat hie all gi-léstid só,
ge-frumid mid firihon: \hld\ íljat gi nu forð hinan,
gangat gáh-líko \hld\ endi duot it þem is jungron kúð.
Hie havat sia iu fur-farana \hld\ endi ist im forð hinan
an Galileo land, \hld\ þar ina eft is jungron skulun,
gi-sehan is ge-síðos.ʼ \hld\ Þuo warð \edtext{sán}{\Afootnote{so L; om. C}} after þiu
þem wívon an willjon, \hld\ þat sia gi-hórdun su-lik word sprekan,
kúðjan þia kraft godes \hld\ —wárun im só a-kumana þuo noh
gie só forahta ge-frumida—: \hld\ gi-witun im forð þanan %NOTE: forahta] L end.
fan þem grave gangan \hld\ endi sagdun þem jungron Kristes
seld-lík gi-siuni, \hld\ þar sia sorogondi
bidun su-likero buota. \hld\ Þuo wurðun ók an þia burg kumana
Judeono wardos, \hld\ þia ovar þemo grave sátun
alla langa naht \hld\ endi þes lík-hamen þar,
huodun þes hréwes. \hld\ Sia sagdun þero héri Judeono,
hwi-lika im þar and-warda \hld\ egison kwámun,
seld-lík gi-siuni, \hld\ sagdun mid wordon,
al só it gi-duan was \hld\ an þero drohtines kraft,
ni miðun an iro muode. \hld\ Þuo budun im méðmo filo
Judeo liudi, \hld\ gold endi siluvar,
saldun im sink manag, \hld\ te þiu þat sia it ni sagdin forð,
ne máridin þero menigi: \hld\ ʽak kweðat þat iu móði hugi
an-swevidi mid slápu \hld\ endi þat þar kwámin is gi-síðos tuo,
far-stálin ina an þem sténe. \hld\ Simnen wesat gi an stríde mid þiu,
forð an flíte: \hld\ ef it wirðit þem folk-togen kúð,
wi gi-helpat iu wið þena hérosten, \hld\ þat hie iu harmes wiht,
léðes ni gi-léstid.ʼ \hld\ Þuo námun sia an þem liudon filo
diurero méðmo, \hld\ dádun all só sia bi-gunnun
—ne gi-weldun iro willjon— \hld\ dádun só wído kúð
þem liudon after þem lande, \hld\ þat sia su-lika lugina woldun
a-hębbjan be þan hèlagan drohtin. \hld\ Þan was eft gi-hélid hugi
jungron Kristes, \hld\ þuo sia gi-hórdun þiu guodun wíf
márjan þia maht godes; \hld\ þuo wárun sia an iro muode fráha,
gie im te þem grave béðia, \hld\ Johannes endi Petrus
runnun ovast-líko: \hld\ warð ér kuman
Johannes þie guodo, \hld\ endi im ovar þem grave gi-stuod,
antat þar sán after kwam \hld\ Símon Petrus,
erl ellan-ruof \hld\ endi im þar in gi-wét
an þat graf gangan: \hld\ gi-sah þar þes godes barnes,
hréo-gi-wádi \hld\ hérren sínes
línin liggjan, \hld\ mid þiu was ér þie lík-hamo
fagaro bi-fangan; \hld\ lag þie fano sundar,
mit þem was þat hòvid bi-helid \hld\ hèlages Kristes,
ríkjes drohtines, \hld\ þan hie an þesaro rastu was.
Þuo geng im ók Johannes \hld\ an þat graf innan
sehan seld-lík þing; \hld\ warð im sán after þiu
ant-lokan is gi-lóvo, \hld\ þat hie wissa, þat skolda eft an þit lioht kuman
is drohtin diur-líko, \hld\ fan dóðe a-standan
up fan erðu. \hld\ Þuo gi-witun im eft þanan
Johannes endi Petrus, \hld\ endi kwámun þia jungron Kristes,
þia gi-síðos te-samne. \hld\ Þan stuod sèrag-muod
én þera idiso \hld\ óðer-síðu
griotandi ovar þem grave, \hld\ was iro iámar muod—
Maria was þat Magdalena—, \hld\ was iro muod-gi-þáht,
sevo mit sorogon gi-blandan, \hld\ ne wissa hwarod siu sókjan skolda
þena hérron, þar iro wárun at þia helpa gilanga. \hld\ Siu ni mohta þuo hofnu awísan,
þat wíf ni mahta wóp for-látan: \hld\ ne wissa hwarod siu sia węndjan skolda;
gi-merrid wárun iro þes muod-gi-þáhti. \hld\ Þuo gi-sah siu þena mahtigan þar
Kriste standan, \hld\ þuoh siu ina kúðlíko
ant-kęnnjan ni mohti, \hld\ ér þan hie ina kúðjan welda,
sęggjan þat hie it selvo wári. \hld\ Hie frágoda hwat siu só séro bi-wiepi,
só harmo mid héton trahnin. \hld\ Siu kwað, þat siu umbi iro hérron ni wissi
te wáren, hwarod hie werðan skoldi: ʽ\hld\ ef þu ina mi gi-wísan mohtis,
fró mín, ef ik þik frágon gi-dorsti, \hld\ ef þu ina hier an þeson felise gi-námis,
wísi ina mi mid wordon þínon: \hld\ þan wári mi allaro willjono mésta,
þat ik ina selvo gi-sáhi.ʼ \hld\ Sia ni wissa, þat sia þie suno drohtines
gruotta mid gódaro sprákun: \hld\ siu wánda þat it þie gardari wári,
hof-ward hérren sínes. \hld\ Þuo gruotta sia þie hèlago drohtin,
bi namen nęrjendero best: \hld\ siu geng im þuo náhor sniumo,
þat wíf mid willjon guodan, \hld\ ant-kenda iro waldand selvan,
míðan siu is þuru þia minnja ni wissa: \hld\ welda ina mid iro mundon grípan,
þiu féhmia an þena folko drohtin, \hld\ novan þat iro friðu-barn godes
werida mid wordon sínon, \hld\ kwað þat siu ina mid wihti ni mósti
handon ant-hrínan: \hld\ ʽik ni stég nohʼ, kwaþie, ʽte þem himiliskon fader;
ak íli þu nu ofst-líko \hld\ endi þem erlon kúði,
bruoðron mínon, \hld\ þat ik úser béðero fader
ala-waldan, \hld\ iuwan endi mínan
suoð-fastan god \hld\ suokjan willju.ʼ
Þat wíf warð þuo an wunnon, \hld\ þat siu muosta su-likan willjon kúðjan,
sęggjan fan im gi-sundon: \hld\ warð sán garo
þiu idis an þat árundi \hld\ endi þem erlon bráhta,
will-spel weron, \hld\ þat siu waldand Krist
gi-sundan gi-sáwi, \hld\ endi sagda hwó he iru selvo gi-bód
torohtero tékno. \hld\ Sia ni weldun gi-trúojan þuo noh
þes wíbes wordon, \hld\ þat siu su-lik will-spel bráhte
gegnungo fan þemo godes suno, \hld\ ak sia sátun im iámor-muoda,
hęliðos hriwonda. \hld\ Þuo warð þie hèlago Krist
eft opan-líko \hld\ óðer-síðu,
drohtin gi-tógid, \hld\ síðor hie fan dóðe a-stuod,
þan wívon an willjon, \hld\ þat hie im þar an wege muotta.
kwedda sia kúð-líko, \hld\ endi sia te is kneohon hnigun,
fellun im tó fuoton. \hld\ Hie hét þat sia forahtan hugi
ne bárin an iro brioston: \hld\ ʽak gi mínon bruoðron skulun
þesa kwidi kúðjan, \hld\ þat sia kuman after mi
an Galileo land; \hld\ þar ik im eft te-gegnes biun.ʼ
Þan fuorun im ók fan Hierusalem \hld\ þero jungrono twéna
an þem selvon daga \hld\ sán an morgan,
erlos an iro árundi: \hld\ weldun im te Emaus
þat kastel suokan. \hld\ Þuo bi-gunnun im kwidi managa
under þem weron wahsan, \hld\ þar sia after þem wege fuorun,
þem hęliðon umbi iro hérron. \hld\ Þuo kwam im þar þie hèlago tuo
gangandi godes suno. \hld\ Sia ni mahtun ina garo-líko
ant-kęnnan kraftigna: \hld\ hie ni welda ina þuo noh kúðjan te im;
was im þoh an iro gi-síðje samad endi frágoda, \hld\ umbi hwi-lika sia saka sprákin:
ʽhwí gangat gi só gornondia?ʼ \hld\ kwaþie; ʽIst ink jámer hugi,
sevo soragono full.ʼ \hld\ Sia sprákun im sán an-gegin,
þia erlos and-wurdi: \hld\ ʽte hwí þu þes éskos sóʼ, kwáðun sia;
ʽbist þi fan Hierusalem \hld\ Judeono folkas
hèlagumu géste \hld\ fan heven-wange,
mid þem grótun godes kraft.ʼ \hld\ Nam is jungaron þó,
erlos góde, \hld\ lédda sie út þanan,
antat he sie bráhte \hld\ an Bethania;
þar hóf he is hendi up \hld\ endi hélegoda sie alle,
wíhida sie mid is wordun. \hld\ Gi-wét imo up þanan,
sóhta imo þat hòha himilo ríki \hld\ endi þena is hèlagon stól:
sitit imo þar an þea \hld\ swíðron half godes,
alo-mahtiges fader \hld\ endi þanan all gesihit
waldandeo Krist, \hld\ só hwat só þius werold be-havet.
Þó an þeru selvon stędi \hld\ ge-síðos góde
te bedu fellun \hld\ endi im eft te burg þanan
þar te Hierusalem \hld\ jungaron Kristes
fórun faganondi: \hld\ was im fráh-mod hugi,
wárun im þar at þemu wíhe; \hld\ waldandes kraft
[...]
