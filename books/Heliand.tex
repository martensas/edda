\bookStart{Heliand}% TODO. Check and remove all TODO and NOTE tags in the text.
\def\thisBookCode{Heliand}

\begin{flushright}%
\textbf{Dating:} 830s

\textbf{Meter:} \Fornyrdislag%para
\end{flushright}%

\section{Introduction}

The \textbf{Heliand} (\Heliand; OS \emph{Hêljand} ‘Saviour’, cf. OE \emph{Hę̂lend}, German \emph{Heiland}) is an Old Saxon epic poem that narrates the life of Jesus.  Although apparently based on the Old High German translation of Tatian’s C2nd gospel harmony, the \emph{Diatessaron}, it is still an original work in the epic tradition that betrays an original creative spirit not afraid to dialogue with earlier, now-lost, pagan poetry.  It is by far the most important source of Old Saxon literature.

\subsection{Historical context}

We are fortunate to have a Latin preface preserved independently of \Heliand\ itself, which can offer some external historical information about the poem.  The original of this fragment is now lost, but it was fortunately printed by the Croatian reformer Flacius Illyricus in 1562.  It consists of two titled parts.

The first is in prose and entitled \emph{Praefatio ad librum antiquum in lingua Saxonica conscriptum} ‘Preface to an ancient book written in the Saxon language’.  This short text in turn appears to consist of two separate paragraphs.  According to the first, \Heliand\ was composed at the behest of emperor Ludwig (\emph{Ludowicus Augustus}, probably Ludwig “the Pious” 778–840, son of Charlemagne), who commanded a Saxon man, \emph{qui apud suos non ignobilis vates habebatur} ‘who was regarded among his own as a not undistinguished poet’ to render the entirety of the Old and New Testaments into Saxon verse.  Thus, the poet, \emph{a mundi creatione initium capiens, iuxta historiae veritatem quaeque excellentiora summatim decerpens, interdum quaedam ubi commodum duxit, mystico sensu depingens, ad finem totius Veteris ac Novi Testamenti interpretando more poetico satis faceta eloquentia perduxit,} ‘beginning with the creation of the world, and summarizing according to the truth of history the most significant events, at times depicting certain events with a mystical sense where he saw fit, led the interpretation, according to poetic custom and with rather witty eloquence, through to the end of the entire Old and New Testaments;’ further, \emph{iuxta morem vero illius poematis omne opus per vitteas distinxit, quas nos lectiones vel sententias possumus appellare} ‘according to the manner of that poem, he distinguished every work by \emph{fitts}, which we can call lessons or sentences.’

There is no reason to doubt the general truth of this account, although it is hard to believe that our unnamed poet should have rendered the entirety of the Old and New Testaments, even the prophets and epistles, into alliterative verse.  The antiquity of this paragraph of the preface is in any case certified by the use of the Germanic technical word \emph{vitteas} ‘fitts’, which, as pointed out already by Sievers (TODO), could not possibly have been known by a 16th century scholar.  The rendering of the Old Testament is probably to be identified with \SaxonGenesis, while the New Testament is what we have before us in \Heliand.

This first paragraph of the \emph{Praefatio} is then followed by a second, where we hear (in part) that, “they say that this same poet, while he was still entirely ignorant of this art, was warned in a dream to adapt the precepts of the Sacred Law into song, with a fitting melody in his own language.” (\emph{ferunt eundem Vatem dum adhuc artis huius penitus esset ignarus, in somnis esse admonitum, ut Sacrae Legis praecepta ad cantilenam propriae linguae congrua modulatione coaptaret.})   This narrative is clearly closely related to that which Bede (TODO) tells us about the illiterate Anglo-Saxon poet Cadman (see Cadman’s Hymn below); in fact its Latin wording is so close to that of Bede that it must have been plagiarised thence.  Finally, the same narrative is then told in Latin verse under the title \emph{Versus de poeta et interprete huius codicis} ‘Verses about the poet and interpreter of this codex’.

Whatever the truth of Cadman’s story, it can scarcely be the case that the poet(s) behind \Heliand\ and \SaxonGenesis\ were ignorant of the poetic art.  Both poems are wrought in an intricate style, and their composer must doubtless have been trained in the traditional craft, having first mastered the art of secular (or pagan) heroic poetry before he was commissioned to versify the Biblical texts; the first paragraph of the \emph{praefatio} itself tells us as much when it says that he “was regarded among his own as a not undistinguished poet”, and the idea that Emperor Ludwig would have commissioned a man entirely without poetic experience is obviously absurd.  This strongly suggests that the second paragraph of the \emph{praefatio} and the \emph{versus} are both later interpolations, and not of historical weight.

\subsection{Style and content}

It was for good reason that the poet was esteemed among his own, for he displays considerable mastery in such “Beowulfian” type scenes as the feast in the great mead-hall (2005–12, 2736–42), the stormy sea-voyage (2233–68, 2906–65), or the host asking for the identity of noble strangers come to his land (551–561); a mastery which reveals his training in traditional vernacular Saxon poetry dealing with heroic matters.  In fact, it is precisely in these passages that his poetry is most fluent, for it is here he can make the most use of his inherited stock of oral-formulaic expressions, synonyms, and kennings.  When our poet, by contrast has to deal with exclusively Christian matters, he is treading new ground, and it is it is apparent that his work suffers as a result.  This is in part due to the lack of traditional formulae for the new religion, and although he invents some (e.g. for Christ \emph{allaro barno bętst} ‘best of all babes’ and \emph{friðu-barn godes} ‘peace-child of God’), they quickly end up stale from overuse.  Another hinder is, as will be discussed shortly, his frequent moralising, which is entirely foreign to the genuine Germanic poetry.

Another notable traditional element found throughout the poem is the relationship between Christ and his Disciples, who are consistently described using the vocabulary of the Germanic warband (as found in earlier heroic poetry like \Beowulf\ and \Hildebrandslied).  Thus, the Disciples are brave “thanes” who express their undying loyalty towards their lord Jesus Christ through long heroic speeches, exclaiming their wish to win ever-lasting fame and glory by dying alongside him in the “dance of weapons” (e.g. Thomas at 3994–4002, Simon Peter at 4674–4689).  In conjunction with this there is an emphasis on the noble ancestry and high social status of the Disciples (e.g. 4003a) and especially Jesus and his family (e.g. 361b–367a), something which gives us an idea of the intended audience—these were members of the Saxon social elite, no mere commoners, and it was important for them that the heroes of the Gospel-story were of similarly high birth.

Still, we should not interpret such traditional elements as evidence for \Heliand\ reflecting a syncretist Germanic “warrior Christianity”, as some more romantic scholars have done.  It would not have been possible for the poet to excise the traditional heroic language—after all, he was hired to write an alliterative poem, and those elements were built into the very essence of the alliterative genre, and were necessary for the poetry to function in the social setting of courtly performance, and for it to work as poetry at all, for the alliteration itself required the existence of a large number of poetic synonyms and formulaic expressions.  It is thus within these confines that the poet relates the New Testament message, but that message is still one of pacifism and humility.  The New Testament is not a warlike text, and neither is \Heliand; regardless of its aesthetics, its \emph{ethics} are thoroughly Christian.

Although \Heliand\ generally adheres closely to Germanic poetic tradition in its language, we find important divergences in its content.  Here the heroic poetic tradition is turned against itself, and the Germanic warrior ideology comes under direct attack by means of its own specialised vocabulary, which is condemned not just in the speeches of Jesus Christ, but in the poet’s own, sermonising voice.  This is perhaps best seen in the episode of the Denial of Peter.  At the Last Supper Peter first makes a solemn speech (4674–4689), declaring in formal heroic language that he will not betray Jesus Christ, his lord, but stay with him until the end and give his life in battle; he swears upon his heart (\emph{hugi}) and strength of hand (\emph{hand-kraft}).  Jesus first praises Peter’s courage, and says that he indeed has a “thane’s heart” (\emph{þegnes hugi}), but then predicts that he will betray him thrice before cockcrow anyway.  Peter does just that, and upon hearing the cock repents by a lamenting speech (5012–5021).  The poet himself then delivers a short sermon on the events (5022–50)—if not even Peter, “the best of men” and “most valiant of thanes” could keep his promise without God’s help, what is its worth?  Man’s solemn vow (\emph{bi-hêt}, = OE \emph{béot} which is used positively in \Beowulf), pride (\emph{hróm}; cf. \Hildebrandslied\ 60), bravery (\emph{mód}), and strength of hand (\emph{hand-kraft}, by which Peter vowed) are all to no avail if the God’s grace should fail him due to his lack of faith.  Naturally, moral exegesis of this kind is totally foreign to the older pagan tradition.

In this context it is of value to talk about the language of war; although the New Testament is not a warlike text, the poet takes the opportunity to break out some traditional formulae when he can, e.g. at the arrest of Jesus (4866–4885).  Still, he is generally very restrained, and tries hard to avoid the active \emph{celebration} of war, probably because of its association with the warlike pre-Christian cult of Weden and his Walkirries and Oneharriers.  Where warlike sentiments are expressed by the Disciples (e.g. at the arrest, or in Peter’s vow at the Last Supper; see above) they are swiftly reproached by Jesus and ultimately proven flawed and misguided.  Traditional motifs like the greedy beasts of battle are entirely expunged, and the old feminine poetic synonyms \emph{*gu̇ðja} and \emph{hildi}, found in \Hildebrandslied\ and commonplace in Norse and English poetry, have not fared much better.  \emph{*gu̇ðja}, found in early OS female names and the non-\Heliand\ compound \emph{gu̇þ-fano} ‘field standard’, is entirely absent, and \emph{hildi} is only used twice (ll. 68, 5044)—in both cases disparagingly.  In their stead we find neuter-gender synonyms like \emph{stríd, ur-lagi, wíg}, and \emph{gi·winn}.  It is probably significant that \emph{Gunnr} and \emph{Hildr} are known as walkirries in the Norse tradition, and in the C9th were still actively worshipped in pagan Denmark, just to the north of Saxony.

It was described above how the depiction of the Disciples in their relationship as servants of Christ makes use of the language of the Germanic war-band, and that is the case when it comes to singular words, but although the Disciples are described as loyal thanes (\emph{þegnos}), heroes (\emph{hęliðos}, cf. \Hildebrandslied\ 6), and earls (\emph{erlos})—words perfectly fit for a Germanic war-band in a poem like \Beowulf—they are not a \emph{war}-band and are never described by explicitly warlike terms like \emph{hildi-skalkos} ‘war-servants’, \emph{wépạn-berandos} ‘weapon-bearers’, or \emph{helm-berandos} ‘helmet-bearers’.  Those terms—which in \Beowulf\ or Norse poetry could describe any group of warriors, including the protagonists—are instead given a derogatory sense, and for the most part refer only to the wicked Jews under their kings (68b, 765b, 2779b, 4811a).

There are, of course, other ways in which \Heliand\ departs from Germanic heroic tradition.  One that deserves mention is the treatment of hostile fate, which often plays a key role in driving the narrative in the old pagan legends (e.g. in \Hildebrandslied\ or the Walsing Cycle).  Although \Heliand\ refers to fated events by what are almost certainly originally pagan expressions like \emph{regano gi·skapu} ‘Shapes of the Reins’ and \emph{wurdi-gi·skapu} ‘Shapes of Weird’, fated events can also be called \emph{godes gi·skapu} ‘God’s Shapes’, for in the Christian worldview it is God that wields the destinies of Men—not the ambivalent Norns.

\subsection{Orthography}

Notes on the normalization:
  \begin{itemize}
    \item Long vowels are marked by the acute rather than by the circumflex accent or macron. This is both faithful to the original manuscripts and concordant with my practice in normalising other Germanic languages.
    \item Long vowels \emph{ê} and \emph{ô} resulting from monophthongisation of diphthongs \emph{ai} and \emph{au} are, however, written with the circumflex accent. That these were in fact articulated separately is seen by the following circumstance: in the mss. etymological \emph{é} and \emph{ó} are frequently written as \emph{ie} and \emph{uo}, but this is never done for \emph{ê} and \emph{ô}.
    \item If attested in all mss., epenthetic (\emph{svara-bʰaktí}) vowels are marked with an underdot. Otherwise they are deleted.
    \item Unstressed \emph{a}-vowels reduced to \emph{e} in \textbf{C} are reverted back to \emph{a}
    \item Long vowels resulting from nasal assimilation are marked with an overdot. \emph{i} is written as \emph{ï}.
    \item ms. \emph{e} and \emph{i}, when occuring between vowels are written as \emph{j}.
    \item ms. \emph{i}, when word-initial or following \emph{g} and corresponding to etymological \emph{j} is written as \emph{j}
    \item ms. \emph{e} as resulting from \emph{i}-mutation is written as \emph{ę}.
    \item ms. \emph{b} or \emph{ƀ}, when representing the voiced bilabial fricative, is written as \emph{v}.
    \item ms. \emph{th} is written as \emph{þ}.
    \item ms. \emph{uu} is written as \emph{w}.
  \end{itemize}

\subsection{Preservation}
The following is an exhaustive list of source mss. in chronological order.

\begin{small}\begin{longtabu} to \textwidth {|c l c c|}
	\hline
	Siglum & Date & Lines & Full name \\
	\hline\hline\endhead
  \textbf{L} & 840–850 & 5824b–5871a & Thomas 4073 \\
  \textbf{P} & 840–850 & 958–1006a & Berlin DHM R 56/2537 \\
  \textbf{V} & 800–850 & 1279–1358a & Palatini Latini 1447 \\
  \textbf{S} & 850 & \begin{tabular}{@{}c@{}}351b–360a, 368b–384, 393–400a, \\ 492–582a, 675–683a, 693–706, \\ 716b–722a\end{tabular} & BSB Cgm 8840 \\
  \textbf{M} & 850–875 & TODO & BSB Cgm 25 \\
  \textbf{C} & 950–1000 & 1–5970 & Cotton Caligula A VII \\
	\hline
\end{longtabu}\end{small}

The two main mss. are \textbf{M} and \textbf{C}.  Fragments \textbf{L} and \textbf{P} are identical in terms of handwriting and page layout and appear to have originally belonged to the same codex.  \textbf{V} also attests \SaxonGenesis, which suggests a close relation between that text and \Heliand.

\subsection{NOTE!}

The following edition is very much a work in progress.  The radically normalized orthography has been implemented, as has the marking of alliteration, but the original text has not been thoroughly critically edited, nor is there any English translation.

\section{Heliand}

\bvg\bva%
\alst{M}anega wáron, \hld\ þe sia iro \alst{m}ód ge·spón, &
þat sia bi·gunnun word godes, &
\alst{r}ękkjan þat gi·\alst{r}úni, \hld\ þat þie \alst{r}íkjo Krist &
undar \alst{m}an-kunnja \hld\ \alst{m}áriða gi·frumida &
mid \alst{w}ordun ęndi mid \alst{w}erkun. \hld\ Þat wolda þó \alst{w}ísara filo &
\alst{l}iudo barno \alst{l}ovon, \hld\ \alst{l}êra Kristes, &
\alst{h}êlag word godas, \hld\ ęndi mid iro \alst{h}andon skrívan &
\alst{b}erẹht-líko an \alst{b}uok, \hld\ hwó sia is gi·\alst{b}od-skip skoldin &
\alst{f}rummjan, \alst{f}iriho barn. \hld\ Þan wárun þoh sia \alst{f}iori te þiu &
under þera \alst{m}ęnigo, \hld\ þia habdon \alst{m}aht godes, &
\alst{h}elpa fan \alst{h}imila, \hld\ \alst{h}êlagna gêst, &
\alst{k}raft fan \alst{K}riste; \hld\ sia wurðun gi·\alst{k}orana te þio, &
þat sie þan \alst{É}wangelium \hld\ \alst{ê}nan skoldun &
an \alst{b}uok skrívan \hld\ endo só manag gi·\alst{b}od godes, &
\alst{h}êlag \alst{h}imilisk word: \hld\ sia ne muosta \alst{h}ęliðo þan mêr, &
\alst{f}iriho barno \alst{f}rummjan, \hld\ newan þat sia \alst{f}iori te þio &
þuru \alst{k}raft godas \hld\ ge·\alst{k}orana wurðun, &
\alst{M}atheus ęndi \alst{M}arkus, \hld\ —só wárun þia \alst{m}an hêtana— &
Lukas ęndi \alst{J}ohannes; \hld\ sia wárun \alst{g}ode lieva, &
\alst{w}irðiga ti þem gi·\alst{w}irkje. \hld\ Habda im \alst{w}aldand god, &
þem \alst{h}ęliðon an iro \alst{h}ertan \hld\ \alst{h}êlagna gêst &
\alst{f}asto bi·\alst{f}olhan \hld\ ęndi \alst{f}erạhtan hugi, &
só manag \alst{w}ís-lík \alst{w}ord \hld\ ęndi gi·\alst{w}it mikil, &
þat sea skoldin a·\alst{h}ębbjan \hld\ \alst{h}êlagaro stemnun &
\alst{g}od-spell þat \alst{g}uoda, \hld\ þat ni havit ênigan gi·\alst{g}adon hwęrgin, &
þiu \alst{w}ord an þesaro \alst{w}er-oldi, \hld\ þat io \alst{w}aldand mêr, &
\alst{d}rohtin \alst{d}iurje \hld\ efþo \alst{d}ervi þing, &
\alst{f}irin-werk \alst{f}ęllje \hld\ efþo \alst{f}íundo níð, &
\alst{st}ríd wiðẹr·\alst{st}ande—, \hld\ hwand hie habda \alst{st}arkan hugi, &
\alst{m}ildjan ęndi guodan, \hld\ þie þe \alst{m}êster was, &
\alst{a}ðal-\alst{o}rd-frumo \hld\ \alst{a}lo-mahtig. &
Þat skoldun sea \alst{f}iori \hld\ þuo \alst{f}ingron skrívan, &
\alst{s}ęttjan ęndi \alst{s}ingan \hld\ ęndi \alst{s}ęggjan forð, &
þat sea fan \alst{K}ristes \hld\ \alst{k}rafte þem mikilon &
gi·\alst{s}áhun ęndi gi·hôrdun, \hld\ þes hie \alst{s}elvo gi·sprak, &
gi·\alst{w}ísda ęndi gi·\alst{w}arạhta, \hld\ \alst{w}undạr-líkas filo, &
só \alst{m}anag mid \alst{m}annon \hld\ \alst{m}ahtig drohtin, &
all so hie it fan þem \alst{a}n-ginne \hld\ þuru is \alst{ê}nes kraht, &%NOTE: kraht checked.
\alst{w}aldand gi·sprak, \hld\ þuo hie êrist þesa \alst{w}er-old gi·skuop &
ęndi þuo \alst{a}ll bi·fieng \hld\ mid \alst{ê}nu wordo, &
\alst{h}imil ęndi erða \hld\ ęndi al þat sea bi·\alst{h}lidan êgun &
gi·\alst{w}arạhtes ęndi gi·\alst{w}ahsanes: \hld\ þat warð þuo all mid \alst{w}ordon godas &
\alst{f}asto bi·\alst{f}angan, \hld\ ęndi gi·\alst{f}rumid after þiu, &
hwi-lik þan \alst{l}iud-skępi \hld\ \alst{l}andes skoldi &
\alst{w}ídost gi·\alst{w}aldan, \hld\ efþo hwár þiu \alst{w}er-old-aldạr &
\alst{ę}ndon skoldin. \hld\ \alst{Ê}n was iro þuo noh þan &
\alst{f}iriho barnun bi·\alst{f}oran, \hld\ ęndi þiu \alst{f}ïvi wárun a·gangan: &
skolda þuo þat \alst{s}ehsta \hld\ \alst{s}álig-líko &
\alst{k}uman þuru \alst{k}raft godes \hld\ ęndi \alst{K}ristas gi·burd, &
\alst{h}êlandero bęstan, \hld\ \alst{h}êlagas gêstes, &
an þesan \alst{m}iddil-gard \hld\ \alst{m}anagon te helpun, &
\alst{f}irjo barnon ti \alst{f}rumon \hld\ wið \alst{f}íundo níð, &
wið \alst{d}ęrnero \alst{d}walm. \hld\ Þan habda þuo \alst{d}rohtin god &
\alst{R}ómano-liudjon far·liwan \hld\ \alst{r}íkjo mêsta, &
\alst{h}abda þem \alst{h}ęri-skipje \hld\ \alst{h}erta gi·stęrkid, &
þat sia habdon bi·\alst{þ}wungana \hld\ \alst{þ}iedo gi·hwi-lika, &
habdun fan \alst{R}úmu-burg \hld\ \alst{r}íki gi·wunnan &
\alst{h}elm-gi·trôstjon, \hld\ sáton iro \alst{h}ęri-togon &
an \alst{l}ando gi·hwem, \hld\ habdun \alst{l}iudjo gi·wald, &
\alst{a}llon \alst{ę}li-þeodon. \hld\ \alst{E}rodes was &
an \alst{J}erusalem \hld\ over þat \alst{J}udeono folk &
gi·\alst{k}oran te \alst{k}uninge, \hld\ só ina þie \alst{k}êser þarod, &
fon \alst{R}úmu-burg \hld\ \alst{r}íki þiodan &
\alst{s}atta undar þat gi·\alst{s}ïði. \hld\ Hie ni was þoh mid \alst{s}ibbjon bi·lang &
\alst{a}varon \alst{I}sraheles, \hld\ \alst{ę}ðili-gi·burdi, &
\alst{k}uman fon iro \alst{k}nuosle, \hld\ newan þat hie þuru þes \alst{k}êsures þank &
fan \alst{R}úmu-burg \hld\ \alst{r}íki habda, &
þat im wárun só gi·\alst{h}ôriga \hld\ \alst{h}ildi-skalkos, &
\alst{a}varon \alst{I}sraheles \hld\ \alst{ę}lljan-ruova: &
swíðo un·\alst{w}anda \alst{w}ini, \hld\ þan lang hie gi·\alst{w}ald êhta, &
\edtext{E\alst{r}ódes}{\Bfootnote{The name \emph{Erodes} can alliterate either with a vowel (following the Germanic root stress pattern: / x x) or with the consonant \emph{r} (following the Latin penultimate stress: x / x).  Out of 17 total appearances of the name in \Heliand, 12 alliterate with a vowel; 4 with \emph{r}; and 1 has no alliteration.}} þes \alst{r}íkjas \hld\ ęndi \alst{r}ád-burdjon held &
\alst{J}udeo liudi. \hld\ Þan was þár ên gi·\alst{g}amalod mann, &
þat was \alst{f}ruod gomo, \hld\ habda \alst{f}erẹhtan hugi, &
was fan þem \alst{l}iudjon \hld\ \alst{L}ewias kunnes, &
\alst{J}akobas sunjas, \hld\ \alst{g}uodero þiedo: &
\alst{Z}akharias was hie hêtan. \hld\ Þat was só \alst{s}álig man, &
hwand hie simblon \alst{g}erno \hld\ \alst{g}ode þeonoda, &
\alst{w}arạhta after is \alst{w}illjon; \hld\ deda is \alst{w}íf só self &
—was iru gi·\alst{a}ldrod \alst{i}dis: \hld\ ni muosta im \alst{ę}rvi-ward &
an iro \alst{j}uguð-hêdi \hld\ \alst{g}iviðig werðan— &
\alst{l}ibdun im far·úter \alst{l}aster, \hld\ warụhtun \alst{l}of goda, &
wárun só gi·\alst{h}ôriga \hld\ \alst{h}evan-kuninge, &
\alst{d}iuridon u̇san \alst{d}rohtin: \hld\ ni weldun \alst{d}ęrvjas wiht &
under \alst{m}an-kunnje, \hld\ \alst{m}ênes gi·frummjan, &
ne \edtext{\alst{s}aka}{\Afootnote{With this word \textbf{M} begins.  Above it seven lines have been erased.}} ne \alst{s}undja; \hld\ was im þoh an \alst{s}orgun hugi, &
þat sie \alst{ę}rvi-ward \hld\ \alst{ê}gan ni móstun, &
ak wárun im \alst{b}arno-lôs. \hld\ Þan skolda hé gi·\alst{b}od godes &
þár an \alst{J}erusalem, \hld\ só oft só is gi·\alst{g}ęngi gi·stód, &
þat ina \alst{t}orht-líko \hld\ \alst{t}ídi gi·manodun, &
só skolda hé at þem \alst{w}íha \hld\ \alst{w}aldandes geld &
\alst{h}êlag bi·\alst{h}wervan, \hld\ \alst{h}evan-kuninges, &
\alst{g}odes \alst{j}ungar-skępi: \hld\ \alst{g}ern was hé swíðo, &
þat hé it þurh \alst{f}erhtan hugi \hld\ \alst{f}rummjan mósti.\eva

\bvb TODO.\evb\evg

\bvg\bva[2][94]%
Þó warð þiu \alst{t}íd kuman, \hld\ —þat þár gi·\alst{t}ald habdun &
\alst{w}ísa man mid \alst{w}ordun,— \hld\ þat skolda þana \alst{w}íh godes &
\alst{Z}akharias bi·\alst{s}ehan. \hld\ Þó warð þár gi·\alst{s}amnod filu &
þár te \alst{J}erusalem \hld\ \alst{J}udeo liudi, &
\alst{w}erodes te þem \alst{w}íha, \hld\ þár sie \alst{w}aldand god &
swíðo \alst{þ}eo-líko \hld\ \alst{þ}iggjan skoldun, &
\alst{h}êrron is \alst{h}uldi, \hld\ þat sie \alst{h}evan-kuning &
\alst{l}êðes a·\alst{l}éti. \hld\ Þea \alst{l}iudi stódun &
umbi þat \alst{h}êlaga \alst{h}ús, \hld\ ęndi géng im þe gi·\alst{h}êrodo man &
an þana \alst{w}íh innan. \hld\ Þat \alst{w}erod ȯðar bêd &
umbi þana \alst{a}lạh \alst{ú}tan, \hld\ \alst{E}breo liudi, &
hwan êr þe \alst{f}ródo man \hld\ gi·\alst{f}rumid habdi &
\alst{w}aldandes \alst{w}illjon. \hld\ Só hé þó þana \alst{w}í-rôk dróg, &
\alst{a}ld aftar þem \alst{a}lạha, \hld\ ęndi umbi þana \alst{a}ltari géng &
mid is \alst{r}ôk-fatun \hld\ \alst{r}íkjun þionon, &
—\alst{f}ręmida \alst{f}erht-líko \hld\ \alst{f}râon sínes, &
\alst{g}odes \alst{j}ungar-skępi \hld\ \alst{g}erno swíðo &
mid \alst{h}luttru \alst{h}ugi, \hld\ *só man \alst{h}êrren skal &
\alst{g}erno ful-\alst{g}angan—, \hld\ \alst{g}rurjos kwámun im, &
\alst{ę}gison an þem \alst{a}lạhe: \hld\ hie gi·sah þár aftar þiu ênna \alst{ę}ngil godes &
an þem \alst{w}íhe innan, \hld\ hie sprak im mid is \alst{w}ordun tuo, &
hiet þat \alst{f}ruod gumo \hld\ \alst{f}orọht ni wári, &
hiet þat hie im ni \edtext{an·\alst{d}riede}{\Bfootnote{The original segmenting of \emph{an·drádan} is \emph{and-} + \emph{rádan}, but already by the time of \Heliand\ it had clearly been reanalyzed as \emph{an(t)·drádan}, as seen by the alliteration in the present line and by the variant spelling \emph{antdrádan} seen throughout the poem.  Cf. English \emph{dread}, from OE \emph{drǽdan}, from earlier OE \emph{on·drǽdan}.}}: \hld\ „þína \alst{d}ádi sind“, kwaþ-hie*, &%
„\alst{w}aldanda \alst{w}erðe \hld\ ęndi þín \alst{w}ord só self, &
þín \alst{þ}ionost is im an \alst{þ}anke, \hld\ þat þú su·lika gi·\alst{þ}ȧht haves &
an is \alst{ê}nes kraft. \hld\ Ik is \alst{ę}ngil bium, &
\alst{G}abriel bium ik hêtan, \hld\ þe gio for \alst{g}oda standu, &
\alst{a}nd-ward for þem \alst{a}lo-waldon, \hld\ ne sí þat hé mé an is \alst{â}rundi hwárod &
\alst{s}ęndjan willja. \hld\ Nú hiet hé mé an þesan \alst{s}ïð faran, &
hiet þat ik þi þoh gi·\alst{k}u̇ðdi, \hld\ þat þi \alst{k}ind gi·boran, &
fon þínera \alst{a}lderu \alst{i}dis \hld\ \alst{ô}dan skoldi &
\alst{w}erðan an þesero \alst{w}er-oldi, \hld\ \alst{w}ordun spáhi. &
Þat ni skal an is \alst{l}iva gio \hld\ \alst{l}íðes an·bítan, &
\alst{w}ínes an is \alst{w}er-oldi: \hld\ só haved im \alst{w}urd-gi·skapu, &
\alst{m}etod gi·\alst{m}arkod \hld\ ęndi \alst{m}aht godes. &
Hét þat ik þi þoh \alst{s}agdi, \hld\ þat it skoldi gi·\alst{s}ïð wesan &
\alst{h}evan-kuninges, \hld\ hét þat git it \alst{h}eldin wel, &
\alst{t}uhin þurh \alst{t}reuwa, \hld\ kwað þat hé im \alst{t}íras só filu &
an \alst{g}odes ríkja \hld\ for·\alst{g}evan weldi. &
Hé kwað þat þe \alst{g}ódo \alst{g}umo \hld\ \alst{J}ohannes te namon &
\alst{h}ębbjan skoldi, \hld\ gi·bôd þat git it \alst{h}étin só, &
þat \alst{k}ind, þan it \alst{k}wámi, \hld\ kwað þat it \alst{K}ristes gi·sïð &
an þesaro \alst{w}ídun \alst{w}er-old \hld\ \alst{w}erðan skoldi, &
is \alst{s}elves \alst{s}unjes, \hld\ ęndi kwað þat sie \alst{s}liumo herod &
an is \alst{b}od-skępi \hld\ \alst{b}êðe kwámin.“ &
\alst{Z}akharias þó gi·mahạlda \hld\ ęndi wið \alst{s}elvan sprak &
\alst{d}rohtines ęngil, \hld\ ęndi im þero \alst{d}ádjo bi·gan, &
\alst{w}undron þero \alst{w}ordo: \hld\ „hwó mag þat gi·\alst{w}erðan só“, kwað hé, &
„\alst{a}ftar an \alst{a}ldre? \hld\ it is unk \alst{a}l te lat &
só te gi·\alst{w}innanne, \hld\ só þú mid þínun \alst{w}ordun gi·sprikis. &
Hwanda wit habdun \alst{a}ldres \hld\ êr \alst{e}fno twên-tig &
\alst{w}intro an unkro \alst{w}er-oldi, \hld\ êr þan kwámi þit \alst{w}íf te mí; &
þan wárun wit nú at·\alst{s}amna \hld\ ant·\alst{s}ivunta wintro &
gi·\alst{b}ęnkjon ęndi gi·\alst{b}ęddjon, \hld\ sïðor ik sie mí te \alst{b}rúdi ge·kôs. &
Só wit þes an unkro \alst{j}uguði \hld\ gi·\alst{g}irnan ni mohtun, &
þat wit \alst{ę}rvi-ward \hld\ \alst{ê}gan móstin, &
\alst{f}ódjan an unkun \alst{f}lęttja, \hld\ nú wit sus gi·\alst{f}ródod sint &
—havad unk \alst{ę}ldi bi·noman \hld\ \alst{ę}lljan-dádi, &
þat wit sint an unkro \alst{s}iuni gi·\alst{s}lekit \hld\ ęndi an unkun \alst{s}ídun lat; &
\alst{f}lêsk is unk ant·\alst{f}allan, \hld\ \alst{f}el un·skôni, &
is unka \alst{l}ud gi·\alst{l}iðen, \hld\ \alst{l}ík gi·drusnod, &
sind \alst{u}nka \alst{a}nd-bári \hld\ \alst{ȯ}ðar-líkaron, &
\alst{m}ód ęndi \alst{m}ęgin-kraft—, \hld\ só wit giu só \alst{m}anagan dag &
\alst{w}árun an þesero \alst{w}er-oldi, \hld\ só mí þes \alst{w}undạr þunkit, &
hwó it só gi·\alst{w}erðan mugi, \hld\ só þú mid þínun \alst{w}ordun gi·sprikis.\eva

\bvb TODO.\evb\evg

\bvg\bva[3][159]%
Þó warð þat \alst{h}evan-kuninges bodon \hld\ \alst{h}arm an is móde, &
þat hé is gi·\alst{w}erkes \hld\ só \alst{w}undron skolda &
ęndi þat ni welda gi·\alst{h}uggjan, \hld\ þat ina mahta \alst{h}êlag god &
só \alst{a}la-jungan, \hld\ só hé fon \alst{ê}rist was, &
\alst{s}elvo gi·wirkjan, \hld\ of hé \alst{s}ó weldi. &
Skęrida im þó te \alst{w}ítja, \hld\ þat hé ni mahte ênig \alst{w}ord sprekan, &
gi·\alst{m}ahljen mid is \alst{m}u̇ðu, \hld\ „êr þan þi \alst{m}agu wirðid, &
fon þínero \alst{a}ldero \alst{i}dis \hld\ \alst{e}rl a·fódit, &
\alst{k}ind-jung gi·boran \hld\ \alst{k}unnjes gódes, &
\alst{w}ánum te þesero \alst{w}er-oldi. \hld\ Þan skalt þú eft \alst{w}ord sprekan, &
hębbjan þínaro \alst{st}emna gi·wald; \hld\ ni þarft þú \alst{st}um wesan &
\alst{l}ęngron hwíla.“ \hld\ Þó warð it sán gi·\alst{l}êstid só, &
gi·\alst{w}orðan te \alst{w}áron, \hld\ só þár an þem \alst{w}íha gi·sprak &
\alst{ę}ngil þes \alst{a}lo-waldon: \hld\ warð \alst{a}ld gumo &
\alst{sp}ráka bi·lôsit, \hld\ þoh hé \alst{sp}áhan hugi &
\alst{b}ári an is \alst{b}reostun. \hld\ \alst{B}idun allan dag &
þat \alst{w}erod for þem \alst{w}íha \hld\ ęndi \alst{w}undrodun alla, &
bi·hwí hé þár só \alst{l}ango, \hld\ \alst{l}of-sálig man, &
swíðo \alst{f}ród gumo \hld\ \alst{f}râon sínun &
\alst{þ}ionon \alst{þ}orfti, \hld\ só þár êr ênig \alst{þ}egno ni deda, &
þan sie þár at þem \alst{w}íha \hld\ \alst{w}aldandes geld &
\alst{f}olmon \alst{f}rumidun. \hld\ Þó kwam \alst{f}ród gumo &
\alst{ú}t fon þem \alst{a}lạha. \hld\ \alst{E}rlos þrungun &
\alst{n}áhor mikilu: \hld\ was im \alst{n}iud mikil, &
hwat hé im \alst{s}ȯð-líkes \hld\ \alst{s}ęggjan weldi, &
\alst{w}ísjan te \alst{w}áron. \hld\ Hé ni mohta þó ênig \alst{w}ord sprekan, &
gi·\alst{s}ęggjan þem gi·\alst{s}ïðja, \hld\ b·útan þat hé mid is \alst{s}wíðron hand &
\alst{w}ísda þem \alst{w}eroda, \hld\ þat sie u̇ses \alst{w}aldandes &
\alst{l}êra \alst{l}êstin. \hld\ Þea \alst{l}iudi for·stódun, &
þat hé þár habda \alst{g}egnungo \hld\ \alst{g}od-kundes hwat &
for·\alst{s}ehen \alst{s}elvo, \hld\ þoh hé is ni mahti gi·\alst{s}ęggjan wiht, &
gi·\alst{w}ísjan te \alst{w}áron. \hld\ Þó habda hé u̇ses \alst{w}aldandes &
\alst{g}eld gi·lêstid, \hld\ al só is gi·\alst{g}ęngi was &
gi·\alst{m}arkod mid \alst{m}annun. \hld\ Þó warð sán aftar þiu \alst{m}aht godes, &
gi·\alst{k}u̇ðid is \alst{k}raft mikil: \hld\ warð þiu \alst{k}wán ôkan, &
\alst{i}dis an ira \alst{ę}ldju: \hld\ skolda im \alst{ę}rvi-ward, &
swíðo \alst{g}od-kund \alst{g}umo \hld\ \alst{g}iviðig werðan, &
\alst{b}arn an \alst{b}urgun. \hld\ \alst{B}êd aftar þiu &
þat \alst{w}íf \alst{w}urdi-gi·skapu. \hld\ Skrêd þe \alst{w}intạr forð, &
\alst{g}éng þes \alst{g}ę́res gi·tal. \hld\ \alst{J}ohannes kwam &
an \alst{l}iudjo \alst{l}ioht: \hld\ \alst{l}ík was im skôni, &
was im \alst{f}el \alst{f}agạr, \hld\ \alst{f}ahs ęndi naglos, &
\alst{w}angun wárun im \alst{w}litige. \hld\ Þó fórun þár \alst{w}íse man, &
\alst{s}nelle te·\alst{s}amne, \hld\ þea \alst{s}wásostun mêst, &
\alst{w}undrodun þes \alst{w}erkes, \hld\ bi·hwí it gio mahti gi·\alst{w}erðan só, &
þat undar só \alst{a}ldun twêm \hld\ \alst{ô}dan wurði &
\alst{b}arn an gi·\alst{b}urdjon, \hld\ ni wári þat it gi·\alst{b}od godes &
\alst{s}elves wári: \hld\ af·\alst{s}uovun sie garo, &
þat it elkor só \alst{w}án-lík \hld\ \alst{w}erðan ni mahti. &
Þó sprak þár ên gi·\alst{f}ródot man, \hld\ þe só \alst{f}ilo konsta &
\alst{w}ísaro \alst{w}ordo, \hld\ habde gi·\alst{w}it mikil, &
frágode \alst{n}iud-líko, \hld\ hwat is \alst{n}amo skoldi &
\alst{w}esan an þesaro \alst{w}er-oldi: \hld\ „mí þunkid an is \alst{w}ísu gi·lík &
iak an is gi·\alst{b}árja, \hld\ þat hé sí \alst{b}ętara þan wi, &
só ik wániu, þat ina u̇s \alst{g}egnungo \hld\ \alst{g}od fon himila &
\alst{s}elvo \alst{s}ęndi“. \hld\ Þó sprak \alst{s}án aftar &
þiu \alst{m}ódar þes kindes, \hld\ þiu þana \alst{m}agu habda, &
þat \alst{b}arn an ire \alst{b}arme: \hld\ „hér kwam gi·\alst{b}od godes“, kwað siu, &
„\alst{f}ernun gę́re, \hld\ \alst{f}urmon wordu &
gi·bôd, þat hé \alst{J}ohannes \hld\ bi \alst{g}odes lêrun &
\alst{h}êtan skoldi. \hld\ Þat ik an mínumu \alst{h}ugi ni gi·dar &
\alst{w}ęndjan mid \alst{w}ihti, \hld\ of ik is gi·\alst{w}aldan mót“. &
Þó sprak ên \alst{g}êl-hert man, \hld\ þe ira \alst{g}aduling was: &
„ne hét êr \alst{io}·wiht só“, \hld[kwað hé,] „\alst{a}ðal-boranes &
u̇ses \alst{k}unnjes efþo \alst{k}nósles; \hld\ wita \alst{k}iasan im ȯðrana &
\alst{n}iud-samna \alst{n}amon: \hld\ hé \alst{n}iate of hé móti“. &
Þó sprak eft þe \alst{f}ródo man, \hld\ þe þár konsta \alst{f}ilo mahljan: &
„ni givu ik þat te \alst{r}áde“, \hld[kwað hé,] „\alst{r}inko neg·ênun, &
þat hé \alst{w}ord godes \hld\ \alst{w}ęndjan bi·ginna; &
ak wita is þana \alst{f}ader \alst{f}rágon, \hld\ þe þár só gi·\alst{f}ródod sitit, &
\alst{w}ís an is \alst{w}ín-sęli: \hld\ þoh hé ni mugi ênig \alst{w}ord sprekan, &
þoh mag hé bi \alst{b}ók-stavon \hld\ \alst{b}réf ge·wirkjan, &
\alst{n}amon gi·skrívan“. \hld\ Þó hé \alst{n}áhor géng, &
lęgda im êna \alst{b}ók an \alst{b}arm \hld\ ęndi \alst{b}ad gerno &
\alst{w}rítan \alst{w}ís-líko \hld\ \alst{w}ord-gi·merkjun, &
hwat sie þat \alst{h}êlaga barn \hld\ \alst{h}êtan skoldin. &
Þó nam hé þia bók an \alst{h}and \hld\ ęndi an is \alst{h}ugi þȧhte &
swíðo \alst{g}erno te \alst{g}ode: \hld\ \alst{J}ohannes namon &
\alst{w}ís-líko gi·\alst{w}rêt \hld\ ęndi ôk aftar mid is \alst{w}ordu gi·sprak &
swíðo \alst{sp}áh-líko: \hld\ habda im eft is \alst{sp}ráka gi·wald, &
gi·\alst{w}ittjas ęndi \alst{w}ísun. \hld\ Þat \alst{w}íti was þó a·gangan, &
\alst{h}ard \alst{h}arm-skare, \hld\ þe im \alst{h}êlag god &
\alst{m}ahtig \alst{m}akode, \hld\ þat hé an is \alst{m}ód-sevon &
\alst{g}odes ni for·\alst{g}áti, \hld\ þan hé im eft sęndi is \alst{j}ungron tó.\eva

\bvb TODO.\evb\evg

\bvg\bva[4][243]%
Þó ni was \alst{l}ang aftar þiu, \hld\ ne it al só gi·\alst{l}êstid warð, &
só hé \alst{m}an-kunnja \hld\ \alst{m}anaga hwíla, &
\alst{g}od alo-mahtig \hld\ for·\alst{g}even habda, &
þat hé is \alst{h}imilisk barn \hld\ \alst{h}erod te wer-oldi, &
sí \alst{s}elves \alst{s}unu \hld\ \alst{s}ęndjan weldi, &
te þiu þat hé hér a·\alst{l}ôsdi \hld\ al \alst{l}iud-stamna, &
\alst{w}erod fon \alst{w}ítja. \hld\ Þó warð is \alst{w}is-bodo &
an \alst{G}alilea-land, \hld\ \alst{G}abriel kuman, &
\alst{ę}ngil þes \alst{a}lo-waldon, \hld\ þár hé êne \alst{i}dis wisse, &
\alst{m}uni-líka \alst{m}agað: \hld\ \alst{M}aría was siu hêten, &
was iru \alst{þ}iorna gi·\alst{þ}igan. \hld\ Sea ên \alst{þ}egạn habda, &
\alst{J}oseph gi·mahlit, \hld\ \alst{g}ódes kunnjes man, &
þea \alst{D}awides \alst{d}ohter: \hld\ þat was só \alst{d}iur-lík wíf, &
\alst{i}dis \alst{a}nt-hêti. \hld\ Þár sie þe \alst{ę}ngil godes &
an \alst{N}azareth-burg \hld\ bi \alst{n}amon selvo &
\alst{g}rótte \alst{g}ęgin-warde \hld\ ęndi sie fon \alst{g}ode kwędda: &
„\alst{H}êl wis þú, Maria“, \hld[kwað hé,] „þú bist þínun \alst{h}êrron liof, &
\alst{w}aldande \alst{w}irðig, \hld\ hwand þú gi·\alst{w}it haves, &
\alst{i}dis \alst{ę}nstjo fol. \hld\ Þú skalt for \alst{a}llun wesan &
\alst{w}ívun gi·\alst{w}íhit. \hld\ Ne have þú \alst{w}êkan hugi, &
ne \alst{f}orhti þú þínun \alst{f}erhe: \hld\ ne kwam ik þi te ênigun \alst{f}rêson herod, &
ne \alst{d}ragu ik ênig \alst{d}rugi-þing. \hld\ Þú skalt u̇ses \alst{d}rohtines wesan &
\alst{m}ódar mid \alst{m}annun \hld\ ęndi skalt þana \alst{m}agu fódjan, &
þes \alst{h}ôhon \edtext{\alst{h}evan-kuninges}{\Afootnote{so \textbf{M}; \emph{himilcuninges} \textbf{C}}} suno. \hld\ Þe skal \alst{h}êljand te namon &
\alst{ê}gan mid \alst{ę}ldjun. \hld\ Neo \alst{ę}ndi ni kumid, &
þes \alst{w}ídon ríkjas gi·\alst{w}and, \hld\ þe hé gi·\alst{w}aldan skal, &
\alst{m}ári þeodan.“ \hld\ Þó sprak im eft þiu \alst{m}agað an·gęgin, &
wið þana \alst{ę}ngil godes \hld\ \alst{i}diso skônjost, &
allaro \alst{w}ívo \alst{w}litigost: \hld\ „hwó mag þat gi·\alst{w}erðen só“, kwað siu, &
„þat ik \alst{m}agu fódje? \hld\ Ne ik gio \alst{m}annes ni warð &
\alst{w}ís an mínera \alst{w}er-oldi.“ \hld\ Þó habde eft is \alst{w}ord garu &
\alst{ę}ngil þes \alst{a}lo-waldon \hld\ þero \alst{i}disiu te·gęgnes: &
„an þí skal \alst{h}êlag gêst \hld\ fon \alst{h}evan-wange &
\alst{k}uman þurh \alst{k}raft godes. \hld\ Þanan skal þi \alst{k}ind ôdan &
\alst{w}erðan an þesaro \alst{w}er-oldi; \hld\ \alst{w}aldandes kraft &
skal þi fon þem \alst{h}ôhoston \hld\ \alst{h}evan-kuninge &
\alst{sk}adowan mid \alst{sk}imon. \hld\ Ni warð \alst{sk}ônjera gi·burd, &
ne só \alst{m}ári mid \alst{m}annun, \hld\ hwand siu kumid þurh \alst{m}aht godes &
an þese \alst{w}ídon \alst{w}er-old.“ \hld\ Þó warð eft þes \alst{w}íves hugi &
\alst{a}ftar þem \alst{â}rundje \hld\ \alst{a}l gi·hworven &
an \alst{g}odes willjon. \hld\ „Þan ik hér \alst{g}aru standu“, kwað siu, &
„te su·likun \alst{a}mbaht-skępi, \hld\ só hé mí \alst{ê}gan wili. &
Þiu bium ik \alst{þ}eot-godes. \hld\ Nú ik þeses \alst{þ}inges gi·trúon; &
\alst{w}erðe mí aftar þínun \alst{w}ordun, \hld\ al só is \alst{w}illjo sí, &
\alst{h}êrron mínes; \hld\ nis mí \alst{h}ugi twífli, &
ne \alst{w}ord ne \alst{w}ísa.“ \hld\ Só gi·fragn ik, þat þat \alst{w}íf ant·féng &
þat \alst{g}odes ârundi \hld\ \alst{g}erno swíðo &
mid \alst{l}eohtu hugi \hld\ ęndi mid gi·\alst{l}ôvon gódun &
ęndi mid \alst{h}luttrun treuwun; \hld\ warð þe \alst{h}êlago gêst, &
þat \alst{b}arn an ira \alst{b}ósma; \hld\ ęndi siu ira \alst{b}reostun for·stód &
iak an ire \alst{s}evon \alst{s}elvo, \hld\ \alst{s}agda þem siu welda, &
þat sie habde gi·\alst{ô}kana \hld\ þes \alst{a}lo-waldon kraft &
\alst{h}êlag fon \alst{h}imile. \hld\ Þó warð \alst{h}ugi Josepes, &
is \alst{m}ód gi·worrid, \hld\ þe im êr þea \alst{m}agað habda, &
þea \alst{i}dis \alst{a}nt-hêttja, \hld\ \alst{a}ðal-knósles wíf &
gi·\alst{b}oht im te \alst{b}rúdju. \hld\ Hé af·sóf þat siu habda \alst{b}arn undar iru: &
ni \alst{w}ánda þes mid \alst{w}ihti, \hld\ þat iru þat \alst{w}íf habdi &
gi·\alst{w}ardod só \alst{w}aro-líko: \hld\ ni wisse \alst{w}aldandes þó noh &
\alst{b}líði gi·\alst{b}od-skępi. \hld\ Ni welda sia imo te \alst{b}rúdi þó, &
\alst{h}alon imo te \alst{h}íwon, \hld\ ak bi·gan im þó an \alst{h}ugi þęnkjan, &
hwó hé sie só for·\alst{l}éti, \hld\ só iru þár nú wurði \alst{l}êdes wiht, &
\alst{ô}dan \alst{a}rvides. \hld\ Ni welda sie \alst{a}ftar þiu &
\alst{m}eldon for \alst{m}ęnigi: \hld\ ant·dréd þat sie \alst{m}anno barn &
\alst{l}ívu bi·námin. \hld\ Só was þan þero \alst{l}iudjo þau &
þurh þen \alst{a}ldon \alst{ê}w, \hld\ \alst{E}breo folkes, &
só hwi-lik só þár an \alst{u}n-reht \hld\ \alst{i}dis gi·híwida, &
þat siu simbla þana \alst{b}ed-skępi \hld\ \alst{b}uggjan skolda, &
\alst{f}rí mid ira \alst{f}erhu: \hld\ ni was gio þiu \alst{f}êmja só gód, &
þat siu mid þem \alst{l}iudun \alst{l}ęng \hld\ \alst{l}ibbjen mósti, &
\alst{w}esan undar þem \alst{w}eroda. \hld\ Bi·gan im þe \alst{w}íso mann, &
swíðo \alst{g}ód \alst{g}umo, \hld\ \alst{J}oseph an is móda &
\alst{þ}ęnkjan þero \alst{þ}ingo, \hld\ hwó hé þea \alst{þ}iornun þó &
\alst{l}istjun for·\alst{l}éti. \hld\ Þó ni was \alst{l}ang te þiu, &
þat im þár an \alst{d}rôma \hld\ kwam \alst{d}rohtines ęngil, &
\alst{h}evan-kuninges bodo, \hld\ ęndi hét sie ina \alst{h}aldan wel, &
\alst{m}innjon sie an is \alst{m}óde: \hld\ „Ni wis þú“, kwað hé, „\alst{M}ariun wrêð, &
\alst{þ}iornun \alst{þ}ínaro; \hld\ siu is gi·\alst{þ}ungan wíf; &
ne for·\alst{h}ugi þú sie te \alst{h}ardo; \hld\ þú skalt sie \alst{h}aldan wel, &
\alst{w}ardon ira an þesaro \alst{w}er-oldi. \hld\ Lêsti þú inka \alst{w}ini-treuwa &
\alst{f}orð só þú dádi, \hld\ ęndi hald inkan \alst{f}riund-skępi wel! &
Ne lát þú sie þi þiu \alst{l}êðaron, \hld\ þoh siu undar ira \alst{l}iðon êgi, &
\alst{b}arn an ira \alst{b}ósma. \hld\ It kumid þurh gi·\alst{b}od godes, &
\alst{h}êlages gêstes \hld\ fon \alst{h}evan-wanga: &
þat is \alst{J}ésu Krist, \hld\ \alst{g}odes êgan barn, &
\alst{w}aldandes sunu. \hld\ Þú skalt sie \alst{w}el haldan, &
\alst{h}êlag-líko. \hld\ Ne lát þú þi þínan \alst{h}ugi twífljen, &
\alst{m}ęrrjan þína \alst{m}ód-gi·þȧht.“ \hld\ Þó warð eft þes \alst{m}annes hugi &
gi·\alst{w}ęndid aftar þem \alst{w}ordun, \hld\ þat hé im te þem \alst{w}íva ge·nam, &
te þera \alst{m}agað \alst{m}innja: \hld\ ant·kęnda \alst{m}aht godes, &
\alst{w}aldandes gi·bod; \hld\ was im \alst{w}illjo mikil, &
þat hé sia só \alst{h}êlag-líko \hld\ \alst{h}aldan mósti: &
bi·\alst{s}orgoda sie an is gi·\alst{s}ïðja, \hld\ ęndi siu só \alst{s}úvro dróg &%TODO: dróg or drôg.
al te \alst{h}uldi godes \hld\ \alst{h}êlagna gêst, &
\alst{g}ód-líkan \alst{g}umon, \hld\ ant-þat sie \edtrans{\alst{g}odes gi·skapu}{God’s shapes}{\Bfootnote{TODO: some note about this.}} &
\alst{m}ahtig gi·\alst{m}anodun, \hld\ þat siu ina an \alst{m}anno lioht, &
allaro \alst{b}arno \alst{b}ętst, \hld\ \alst{b}rengjan skolda.\eva

\bvb TODO.\evb\evg

\bvg\bva[5][339]%
Þó warð fon \alst{R}úmu-burg \hld\ \alst{r}íkes mannes &
ovar \alst{a}lla þesa \alst{i}rmin-þiod \hld\ \alst{O}ktawiánas &
\alst{b}an ęndi \alst{b}od-skępi \hld\ ovar þea is \alst{b}rêdon gi·wald &
\alst{k}uman fon þem \alst{k}êsure \hld\ \alst{k}uningo gi·hwi-likun, &
\alst{h}êm-sittjandjun, \hld\ só wído só is \alst{h}ęri-togon &
ovar al þat \alst{l}and-skępi \hld\ \alst{l}iudjo gi·weldun. &
Hiet man þat \alst{a}lla þea \alst{ę}li-lęndjun man \hld\ iro \alst{ó}ðil sóhtin, &
\alst{h}ęliðos iro \alst{h}and-mahạl \hld\ an·gegen iro \alst{h}êrron bodon, &
\alst{k}wámi te þem \alst{k}nósla gi·hwe, \hld\ þanan hé \alst{k}unnjas was, &
gi·\alst{b}oran fon þem \alst{b}urgjun. \hld\ Þat gi·\alst{b}od warð gi·lêstid &
ovar þesa \alst{w}ídon \alst{w}er-old; \hld\ \alst{w}erod samnoda &
te allaro \alst{b}urgjo gi·hwem. \hld\ Fórun þea \alst{b}odon ovar all, &
þea fon þem \alst{k}êsura \hld\ \alst{k}umana wárun, &%NOTE: run start S
\alst{b}ók-spáha weros, \hld\ ęndi an \alst{b}réf skrivun &
swíðo \alst{n}iud-líko \hld\ \alst{n}amono gi·hwi-likan, &
ia \alst{l}and ia \alst{l}iudi, \hld\ þat im ni mahti a·\alst{l}ęttjan mann &
\alst{g}umono su·lika \alst{g}ambra, \hld\ só im skolda \alst{g}eldan gi·hwe &
\alst{h}ęliðo fon is \alst{h}ôvda. \hld\ Þó gi·wêt im ôk mid is \alst{h}íwiska &
\alst{J}oseph þe \alst{g}ódo, \hld\ só it \alst{g}od mahtig, &
\alst{w}aldand \alst{w}elda: \hld\ sóhta im þiu \alst{w}ánamon hêm, &
þea \alst{b}urg an \alst{B}ethleem, \hld\ þár iro \edtext{\alst{b}ęiðero}{\Afootnote{so \textbf{M} (\emph{‘beidero’}) \textbf{S} (\emph{‘beiðera’}); \emph{‘bethero’} \textbf{C}}\Bfootnote{This very rare occurrence of the original diphthong, which almost everywhere else has been contracted to \emph{ê}, is found in 2/3 witness mss.  It also occurs at lines 2265 and 3674.}} was, &
þes \alst{h}ęliðes \alst{h}and-mahạl* \hld\ ęndi ôk þera \alst{h}êlagun þiornun, &
\alst{M}ariun þera gódun. \hld\ Þár was þes \alst{m}árjon stól &
an \alst{ê}r-dagun, \hld\ \alst{a}ðal-kuninges, &
\alst{D}awides þes gódon, \hld\ þan langa þe hé þana \alst{d}ruht-skępi þár, &
\alst{e}rl undar \alst{E}breon \hld\ \alst{ê}gan mósta, &
\alst{h}aldan \alst{h}ôh-gi·setu. \hld\ Sie wárun is \alst{h}íwiskas, &
\alst{k}uman fon is \alst{k}nósla, \hld\ \alst{k}unnjas gódes, &
\alst{b}êðju bi gi·\alst{b}urdjun. \hld\ Þár gi·fragn ik, þat sie þiu \alst{b}erhtun gi·skapu, &
\alst{M}ariun gi·\alst{m}anodun \hld\ *ęndi \alst{m}aht godes, &
þat iru an þem \alst{s}ïða \hld\ \alst{s}unu ôdan warð, &
gi·\alst{b}oran an \alst{B}ethleem \hld\ \alst{b}arno strangost, &
allaro \alst{k}uningo \alst{k}raftigost: \hld\ \alst{k}uman warð þe márjo, &
\alst{m}ahtig an \alst{m}anno lioht, \hld\ só is êr \alst{m}anagan dag &
\alst{b}iliði wárun \hld\ ęndi \alst{b}ôkno filu &
gi·\alst{w}orðen an þesero \alst{w}er-oldi. \hld\ Þó was it all gi·\alst{w}árod só, &
só it êr \alst{sp}áha man \hld\ gi·\alst{sp}rokan habdun, &
þurh hwi-lik \alst{ô}d-módi \hld\ hé þit \alst{e}rð-ríki herod &
þurh is \alst{s}elves kraft \hld\ \alst{s}ókjan welda, &
\alst{m}anagaro \alst{m}und-boro. \hld\ Þó ina þiu \alst{m}ódar nam, &
bi·\alst{w}and ina mid \alst{w}ádju \hld\ \alst{w}ívo skônjost, &
\alst{f}agạron \alst{f}ratahun, \hld\ ęndi ina mid iro \alst{f}olmon twêm &
\alst{l}ęgda \alst{l}iov-líko \hld\ \alst{l}uttilna man, &
þat \alst{k}ind an êna \alst{k}ribbjun, \hld\ þoh hé habdi \alst{k}raft godes, &
\alst{m}anno drohtin. \hld\ Þár sat þiu \alst{m}ódar bi·foran, &
\alst{w}íf \alst{w}akogjandi, \hld\ \alst{w}ar*doda selvo, &
\alst{h}eld þat \alst{h}êlaga barn: \hld\ ni was ira \alst{h}ugi twífli, &
þera \alst{m}agað ira \alst{m}ód-sevo. \hld\ Þó warð þat \alst{m}anagun ku̇ð &
ovar þesa \alst{w}ídon \alst{w}er-old, \hld\ \alst{w}ardos ant·fundun, &
þea þár \alst{e}hu-skalkos \hld\ \alst{ú}ta wárun, &
\alst{w}eros an \alst{w}ahtu, \hld\ \alst{w}iggjo gômjan, &
\alst{f}ehas aftar \alst{f}el*da: \hld\ gi·sáhun \alst{f}inistri an twê &
te·\alst{l}átan an \alst{l}ufte, \hld\ ęndi kwam \alst{l}ioht godes &
\alst{w}ánum þurh þiu \alst{w}olkạn \hld\ ęndi þea \alst{w}ardos þár &
bi·\alst{f}éng an þem \alst{f}elda. \hld\ Sie wurðun an \alst{f}orhtun þó, &
þea \alst{m}an an ira \alst{m}óda: \hld\ gi·sáhun þár \alst{m}ahtigna &
\alst{g}odes ęngil kuman, \hld\ þe im te·\alst{g}ęgnes sprak, &
hét þat im þea \alst{w}ardos \hld\ \alst{w}iht ne ant·drédin &
\alst{l}êðes fon þem \alst{l}iohta: \hld\ „ik skal eu“, kwað hé, „\alst{l}iovara þing, &
swíðo \alst{w}ár-líko \hld\ \alst{w}illjon sęggjan, &
\alst{k}u̇ðjan \alst{k}raft mikil: \hld\ nú is \alst{K}rist ge·boran &
an þeser*o \alst{s}elvun naht, \hld\ \alst{s}álig barn godes, &
an þera \alst{D}awides burg, \hld\ \alst{d}rohtin þe gódo. &
Þat is \alst{m}ęndislo \hld\ \alst{m}anno kunnjas, &
allaro \alst{f}iriho \alst{f}ruma. \hld\ Þár gí ina \alst{f}ïðan mugun, &
an \alst{B}ethlema-burg \hld\ \alst{b}arno ríkjost: &
hębbjad þat te \alst{t}êkna, \hld\ þat ik eu gi·\alst{t}ęlljan mag &
\alst{w}árun \alst{w}ordun, \hld\ þat hé þár bi·\alst{w}undan ligid, &
þat \alst{k}ind an ênera \alst{k}ribbjun, \hld\ þoh hé sí \alst{k}uning ovar al &
\alst{e}rðun ęndi himiles \hld\ ęndi ovar \alst{ę}ldjo barn, &
\alst{w}er-oldes \alst{w}aldand“. \hld\ Reht só hé þó þat \alst{w}ord gi·sprak, &
só warð þár \alst{ę}ngilo te þem \alst{ê}nun \hld\ \alst{u}n-rím kuman, &
\alst{h}êlag \alst{h}ęri-skępi \hld\ fon \alst{h}evan-wanga, &
\alst{f}agạr \alst{f}olk godes, \hld\ ęndi \alst{f}ilu sprákun, &
\alst{l}of-word manag \hld\ \alst{l}iudjo hêrron. &
Af·\alst{h}óvun þó \alst{h}êlagna sang, \hld\ þó sie eft te \alst{h}evan-wanga &
\alst{w}undun þurh þiu \alst{w}olkạn. \hld\ Þea \alst{w}ardos hôrdun, &
hwó þiu \alst{ę}ngilo kraft \hld\ \alst{a}lo-mahtigna god &
swíðo \alst{w}erð-líko \hld\ \alst{w}ordun lovodun: &
„\alst{d}iuriða sí nú“, \hld[kwáðun sie,] „\alst{d}rohtine selvun &
an þem \alst{h}ôhoston \hld\ \alst{h}imilo ríkja &
ęndi \alst{f}riðu an erðu \hld\ \alst{f}iriho barnun, &
\alst{g}ód-willigun \alst{g}umun, \hld\ þem þe \alst{g}od ant·kęnnjad &
þurh \alst{h}luttran \alst{h}ugi.“ \hld\ Þea \alst{h}irdjo for·stódun, &
þat sie \alst{m}ahtig þing \hld\ gi·\alst{m}anod habda, &
\alst{b}líð-lík \alst{b}od-skępi: \hld\ gi·witun im te \alst{B}ethleem þanan &
\alst{n}ahtes sïðon; \hld\ was im \alst{n}iud mikil, &
þat sie \alst{s}elvon Krist \hld\ gi·\alst{s}ehan móstin.\eva

\bvb TODO.\evb\evg

\bvg\bva[6][427]%
Habda im þe \alst{ę}ngil godes \hld\ \alst{a}l gi·wísid &
\alst{t}orhtun \alst{t}êknun, \hld\ þat sie im \alst{t}ó selvun, &
te þem \alst{g}odes barne \hld\ \alst{g}angan mahtun, &
ęndi \alst{f}undun sán \hld\ \alst{f}olko drohtin, &
\alst{l}iudjo hêrron. \hld\ Sagdun þó \alst{l}of goda, &
\alst{w}aldande mid iro \alst{w}ordun \hld\ ęndi \alst{w}ído ku̇ðdun &
ovar þea \alst{b}erhtun \alst{b}urg, \hld\ hwi-lik im þár \alst{b}iliði warð &
fon \alst{h}evan-wanga \hld\ \alst{h}êlag gi·tôgit, &
\alst{f}agạr an \alst{f}elde. \hld\ Þat \alst{f}rí al bi·held &
an ira \alst{h}ugi-skęftjun, \hld\ \alst{h}êlag þiorna, &
þiu \alst{m}agað an ira \alst{m}óde, \hld\ só hwat só siu gi·hôrda þea \alst{m}ann sprekan. &
\alst{F}ódda ina þó \alst{f}agạro \hld\ \alst{f}rího skânjosta, &
þiu \alst{m}ódar þurh \alst{m}innja \hld\ \alst{m}anagaro drohtin, &
\alst{h}êlag \alst{h}imilisk barn. \hld\ \alst{H}ęliðos gi·sprákun &
an þem \alst{a}htodon daga \hld\ \alst{e}rlos managa, &
swíðo \alst{g}lawa \alst{g}umon \hld\ mid þera \alst{g}odes þiornun, &
þat hé \alst{h}êljand te namon \hld\ \alst{h}ębbjan skoldi, &
só it þe \alst{g}odes ęngil \hld\ \alst{G}abriel gi·sprak &
\alst{w}áron \alst{w}ordun \hld\ ęndi þem \alst{w}íve gi·bôd, &
\alst{b}odo drohtines, \hld\ þó siu êrist þat \alst{b}arn ant·féng &
\alst{w}ánum te þesero \alst{w}er-oldi; \hld\ was iru \alst{w}illjo mikil, &
þat siu ina só \alst{h}êlag-líko \hld\ \alst{h}aldan mósti, &
ful-\alst{g}éng im þó só \alst{g}erno. \hld\ Þat \alst{g}ę́r furðor skrêd &
unt-þat þat \alst{f}riðu-barn godes \hld\ \alst{f}iar-tig habda &
\alst{d}ago ęndi nahto. \hld\ Þó skoldun sie þár êna \alst{d}ád frummjan, &
þat sie ina te \alst{J}erusalem \hld\ for·\alst{g}evan skoldun &
\alst{w}aldanda te þem \alst{w}íha. \hld\ Só was iro \alst{w}ísa þan, &
þero \alst{l}iudjo \alst{l}and-sidu, \hld\ þat þat ni mósta for·\alst{l}átan ne-gên &
\alst{i}dis undar \alst{E}breon, \hld\ ef iru at \alst{ê}rist warð &
\alst{s}unu a·fódit, \hld\ ne siu ina \alst{s}imbla þarod &
te þem \alst{g}odes wíha \hld\ for·\alst{g}evan skolda. &
Gi·witun im þó þiu \alst{g}ódun twê, \hld\ \alst{J}oseph ęndi Maria &
\alst{b}êðju fon Bethleem: \hld\ habdun þat \alst{b}arn mid im, &
\alst{h}êlagna Krist, \hld\ sóhtun im \alst{h}ús godes &
an \alst{J}erusalem; \hld\ þár skoldun sie is \alst{g}eld frummjan &
\alst{w}aldanda at þem \alst{w}íha \hld\ \alst{w}ísa lêstjan &
\alst{J}udeo folkes. \hld\ Þár fundun sea ênna \alst{g}ódan man &
\alst{a}ldan at þem \alst{a}lạha, \hld\ \alst{a}ðal-boranan, &
þe habda at þem \alst{w}íha só filu \hld\ \alst{w}intro ęndi sumaro &
gi·\alst{l}ibd an þem \alst{l}iohta: \hld\ oft warhta hé þár \alst{l}of goda &
mid \alst{h}luttru \alst{h}ugi; \hld\ habda im \alst{h}êlagna gêst, &
\alst{s}álig-líkan \alst{s}evon; \hld\ \alst{S}imeon was hé hêtan. &
Im habda gi·\alst{w}ísid \hld\ \alst{w}aldandas kraft &
\alst{l}anga hwíla, \hld\ þat hé ni mósta êr þit \alst{l}ioht a·gevan, &
\alst{w}ęndjan af þesero \alst{w}er-oldi, \hld\ êr þan im þe \alst{w}illjo gi·stódi, &
þat hé \alst{s}elvan Krist \hld\ gi·\alst{s}ehan mósti, &
\alst{h}êlagna \alst{h}evan-kuning. \hld\ Þó warð im is \alst{h}ugi swíðo &
\alst{b}líði an is \alst{b}riostun, \hld\ þó hé gi·sah þat \alst{b}arn kuman &
an þena \alst{w}íh innan. \hld\ Þuo sagda hie \alst{w}aldande þank, &
\alst{a}l-mahtigon gode, \hld\ þes hé ina mid is \alst{ô}gun gi·sah. &
\alst{G}éng im þó te·\alst{g}ęgnes \hld\ ęndi ina \alst{g}erno ant·féng &
\alst{a}ld mid is \alst{a}rmun: \hld\ \alst{a}l ant·kęnde &
\alst{b}ôkan ęndi \alst{b}iliði \hld\ ęndi ôk þat \alst{b}arn godes, &
\alst{h}êlagna \alst{h}evan-kuning. \hld\ „Nú ik þi, \alst{h}êrro, skal“, kwað hé, &
„\alst{g}erno biddjan, \hld\ nú ik sus gi·\alst{g}amalod bium, &
þat þú þínan \alst{h}oldan skalk \hld\ nú hinan \alst{h}wervan látas, &
an þína \alst{f}riðu-wára \alst{f}aran, \hld\ þár êr mína \alst{f}orðrun dedun, &
\alst{w}eros fon þesero \alst{w}er-oldi, \hld\ nú mí þe \alst{w}illjo gi·stód, &
\alst{d}ago liovosto, \hld\ þat ik mínan \alst{d}rohtin gi·sah, &
\alst{h}oldan \alst{h}êrron, \hld\ só mí gi·\alst{h}êtan was &
\alst{l}anga hwíla. \hld\ Þú bist \alst{l}ioht mikil &
\alst{a}llun \alst{ę}li-þiodun, \hld\ þea êr þes \alst{a}lo-waldon &
\alst{k}raft ne ant·\alst{k}ęndun. \hld\ Þína \alst{k}umi sindun &
te \alst{d}óma ęndi te \alst{d}iurðon, \hld\ \alst{d}rohtin frô mín, &
\alst{a}varun \alst{I}srahelas, \hld\ \alst{ê}ganumu folke, &
þínun \alst{l}iovun *\alst{l}iudjun.“ \hld\ \alst{L}istjun talde þó &
þe \alst{a}ldo man an þem \alst{a}lạha \hld\ \alst{i}dis þero gódun, &
\alst{s}agda \alst{s}ȯð-líko, \hld\ hwó iro \alst{s}unu skolda &
ovar þesan \alst{m}iddil-gard \hld\ \alst{m}anagun werðan &
sumun te \alst{f}alle, sumun te \alst{f}róvru \hld\ \alst{f}iriho barnun, &
þem \alst{l}iudjun te \alst{l}eova, \hld\ þe is \alst{l}êrun gi·hôrdin, &
ęndi þem te \alst{h}arma, \hld\ þe \alst{h}ôrjen ni weldin &
\alst{K}ristas lêron. \hld\ „Þu skalt noh“, kwað hé, „\alst{k}ara þiggjan, &
\alst{h}arm an þínumu \alst{h}erton, \hld\ þan ina \alst{h}ęliðo barn &
\alst{w}ápnun \alst{w}ítnod. \hld\ Þat wirðid þi \alst{w}erk mikil, &
\alst{þ}rim te gi·\alst{þ}olonna.“ \hld\ Þiu \alst{þ}iorna al for·stód &
\alst{w}ísas mannas \alst{w}ord. \hld\ Þó kwam þár ôk ên \alst{w}íf gangan &
\alst{a}ld innan þem \alst{a}lạha: \hld\ \alst{A}nna was siu hêtan, &
\alst{d}ohtar Fanueles; \hld\ siu habde ira \alst{d}rohtine wel &
gi·\alst{þ}ionod te \alst{þ}anka, \hld\ was iru gi·\alst{þ}ungan wíf. &
Siu \alst{m}ósta aftar ira \alst{m}agað-hêdi, \hld\ sïðor siu \alst{m}annes warð, &
\alst{e}rles an \alst{ê}hti \hld\ \alst{ę}ðili þiorne, &
só mósta siu mid ira \alst{b}rúdi-gumon \hld\ \alst{b}odlo gi·waldan &
\alst{s}ivun wintạr \alst{s}aman. \hld\ Þó gi·fragn ik þat iru þár \alst{s}orga gi·stód &
þat sie þiu \alst{m}ikila \alst{m}aht \hld\ \alst{m}etodes te·dêlda, &
\alst{w}rêð \alst{w}urdi-gi·skapu. \hld\ Þó was siu \alst{w}idowa aftar þiu &
at þem \alst{f}riðu-wíha \hld\ \alst{f}ior ęndi ant·ahtoda &
\alst{w}intro an iro \alst{w}er-oldi, \hld\ só siu nia þana \alst{w}íh ni for·lét, &
ak siu þár ira \alst{d}rohtine wel \hld\ \alst{d}ages ęndi nahtes, &
\alst{g}ode þionode. \hld\ Siu kwam þár ôk \alst{g}angan tó &
an þea \alst{s}elvun tíd: \hld\ \alst{s}án ant·kęnde &
þat \alst{h}êlage barn godes \hld\ ęndi þem \alst{h}ęliðon ku̇ðde, &
þem \alst{w}eroda aftar þem \alst{w}íha \hld\ \alst{w}il-spel mikil, &
kwað þat im \alst{n}ęrjandas gi·\alst{n}ist \hld\ gi·\alst{n}áhid wári, &
\alst{h}elpa \alst{h}evan-kuninges: \hld\ „nú is þe \alst{h}êlago Krist, &
\alst{w}aldand selvo \hld\ an þesan \alst{w}íh kuman &
te a·\alst{l}ôsjenne þea \alst{l}iudi, \hld\ þe hér nú \alst{l}ango bidun &
an þesara \alst{m}iddil-gard, \hld\ \alst{m}anaga hwíla, &
\alst{þ}urftig \alst{þ}ioda, \hld\ só nú þes \alst{þ}inges mugun &
\alst{m}ęndjan \alst{m}an-kunni.“ \hld\ \alst{M}anag fagonoda &
\alst{w}erod aftar þem \alst{w}íha: \hld\ gi·hôrdun \alst{w}il-spel mikil &
fon \alst{g}ode sęggjan. \hld\ Þat \alst{g}eld habde þó gi·lêstid &
þiu \alst{i}dis an þem \alst{a}lạha, \hld\ al só it im an ira \alst{ê}wa gi·bôd &
ęndi an þera \alst{b}erhtun \alst{b}urg \hld\ \alst{b}ók gi·wísdun, &
\alst{h}êlagaro \alst{h}and-gi·werk. \hld\ Gi·witun im þó te \alst{h}ús þanan &
fon \alst{J}erusalem \hld\ \alst{J}oseph ęndi Maria, &
\alst{h}êlag \alst{h}íwiski: \hld\ habdun im \alst{h}evan-kuning &
\alst{s}imbla te gi·\alst{s}ïða, \hld\ \alst{s}unu drohtines, &
\alst{m}anagaro \alst{m}und-boron, \hld\ só it gio \alst{m}ári ni warð &
þan \alst{w}ídor an þesaro \alst{w}er-oldi, \hld\ b·útan só is \alst{w}illjo géng, &
\alst{h}evan-kuninges \alst{h}ugi.\eva

\bvb TODO.\evb\evg

\bvg\bva[7][537]%
\hspace*{100pt} Þoh þár þan gi·hwi-lik \alst{h}êlag man &%NOTE: in cæsura
\alst{K}rist ant·\alst{k}ęndi, \hld\ þoh ni warð it gio te þes \alst{k}uninges hove &
þem \alst{m}annun gi·\alst{m}árid, \hld\ þea im an iro \alst{m}ód-sevon &
\alst{h}olde ni wárun, \hld\ ak was im só bi·\alst{h}alden forð &
mid \alst{w}ordun ęndi mid \alst{w}erkun, \hld\ ant-þat þár \alst{w}eros ôstan, &
swíðo \alst{g}lawa \alst{g}umon \hld\ \alst{g}angan kwámun &
\alst{þ}rea te þero \alst{þ}iodu, \hld\ \alst{þ}egnos snelle, &
an \alst{l}angan weg \hld\ ovar þat \alst{l}and þarod: &
folgodun ênun \alst{b}erhtun \alst{b}ôkne \hld\ ęndi sóhtun þat \alst{b}arn godes &
mid \alst{h}luttru \alst{h}ugi: \hld\ weldun im \alst{h}nígan tó, &
\alst{g}ehan im te \alst{j}ungrun: \hld\ drivun im \alst{g}odes gi·skapu. &
Þó sie E\alst{r}ódesan þár \hld\ \alst{r}íkjan fundun &
an is \alst{s}ęli \alst{s}ittjen, \hld\ \alst{s}líð-wurdjan kuning, &
\alst{m}ódagna mid is \alst{m}annun: \hld\ —simbla was hé \alst{m}orðes gern— &
þó \alst{k}waddun sie ina \alst{k}u̇sko \hld\ an \alst{k}uning-wísun, &
\alst{f}agạro an is \alst{f}lęttje, \hld\ ęndi hé \alst{f}rágoda sán, &
hwi-lik sie \alst{â}rundi \hld\ \alst{ú}ta gi·brȧhti, &
\alst{w}eros an þana \alst{w}rak-sïð: \hld\ „hweðer lêdjad gí \alst{w}undan gold &
te \alst{g}evu hwi-likun \alst{g}umuno? \hld\ te hwí gí þus an \alst{g}anga kumad, &
gi·\alst{f}aran an \alst{f}óðju? \hld\ Hwat gí n·êt-hwanan \alst{f}erran sind &
\alst{e}rlos fon \alst{ȯ}ðrun þiodun. \hld\ Ik gi·sihu þat gí sind \alst{ę}ðili-gi·burdjun &
\alst{k}unnjes fon \alst{k}nósle gódun: \hld\ nio hér êr su·lika \alst{k}umana ni wurðun &
\alst{ê}ri fon \alst{ȯ}ðrun þiodun, \hld\ sïðor ik mósta þesas \alst{e}rlo folkes, &
gi·\alst{w}aldan þesas \alst{w}ídon ríkjas. \hld\ Gí skulun mí te \alst{w}árun sęggjan &
for þesun \alst{l}iudjo folke, \hld\ bi·hwí gí sín te þesun \alst{l}ande kumana“. &
Þó sprákun im eft te·\alst{g}ęgnes \hld\ \alst{g}umon ôstr-onja, &
\alst{w}ord-spáhe \alst{w}eros: \hld\ „wí þí te \alst{w}árun mugun“, kwáðun sie, &
„\alst{u̇}se \alst{â}rundi \hld\ \alst{ô}ðo gi·tęlljen, &
gi·\alst{s}ęggjan \alst{s}ȯð-líko, \hld\ bi·hwí wí kwámun an þesan \alst{s}ïð herod &
fon \alst{ô}stan te þesaro \alst{e}rðu. \hld\ Giu wárun þár \alst{a}ðaljes man, &
\alst{g}ód-sprákja \alst{g}umon, \hld\ þea u̇s \alst{g}ódes só filu, &
\alst{h}elpa gi·\alst{h}étun \hld\ fon \alst{h}evan-kuninge &
\alst{w}árum \alst{w}ordun. \hld\ Þan was þár ên gi·\alst{w}ittig man, &
\alst{f}ród ęndi \alst{f}il-wís \hld\ —\alst{f}orn was þat giu—, &
\alst{u̇}se \alst{a}ldiro \alst{ô}star hinan, \hld\ —þár ni warð sïðor \alst{ê}nig man &
\alst{sp}rákono só \alst{sp}áhi—; \hld\ hé mahte rekkjen \alst{sp}el godes, &
hwand im habde for·\alst{l}iwan \hld\ \alst{l}iudjo hêrro, &
þat hé mahte fon \alst{e}rðu \hld\ \alst{u}p gi·hôrjan &
\alst{w}aldandes \alst{w}ord: \hld\ bi·þiu was is gi·\alst{w}it mikil, &
þes \alst{þ}egnes gi·\alst{þ}ȧhti. \hld\ Þó hé \alst{þ}anan skolda, &
a·\alst{g}even \alst{g}ardos, \hld\ \alst{g}adulingo gi·mang, &
for·\alst{l}áten \alst{l}iudjo drôm, \hld\ sókjen \alst{l}ioht ȯðar, &
þó hé is \alst{j}ungron hét \hld\ \alst{g}angan náhor, &
\alst{ę}rvi-wardos, \hld\ ęndi is \alst{e}rlun þó &
\alst{s}agde \alst{s}ȯð-líko: \hld\ —þat al \alst{s}ïðor kwam, &
gi·\alst{w}arð* an þesaro \alst{w}er-oldi—: \hld\ þó sagda hé þat hér skoldi kuman ên \alst{w}ís-kuning &
\alst{m}ári ęndi \alst{m}ahtig \hld\ an þesan \alst{m}iddil-gard &
þes \alst{b}ętston gi·\alst{b}urdjes; \hld\ kwað þat it skoldi wesan \alst{b}arn godes, &
kwað þat hé þesero \alst{w}er-oldes \hld\ \alst{w}aldan skoldi &
gio te \alst{ê}wan-daga, \hld\ \alst{e}rðun ęndi himiles. &
Hé kwað þat an þem \alst{s}elvon daga, \hld\ þe ina \alst{s}áligna &
an þesan \alst{m}iddil-gard \hld\ \alst{m}ódar gi·drógi, &
só kwað hé þat \alst{ô}stana \hld\ \alst{ê}n skoldi skínan &
\alst{h}imil-tungạl \alst{h}wít, \hld\ su·lik só wí hér ne \alst{h}abdin êr &
undar·twisk \alst{e}rða ęndi himil \hld\ \alst{ȯ}ðar hwęrigin, &
ne su·lik \alst{b}arn ne su·lik \alst{b}ôkan. \hld\ Hét þat þár te \alst{b}edu fórin &
\alst{þ}rea man fon þero \alst{þ}iodu, \hld\ hét sie \alst{þ}ęnkjan wel, &
hwan êr sie gi·\alst{s}áwin ôstana \hld\ up \alst{s}ïðojan, &
þat \alst{g}odes bôkan \alst{g}angan, \hld\ hét sie \alst{g}arwjan sán, &
hét þat wí im \alst{f}olgodin, \hld\ só it \alst{f}uri wurði, &
\alst{w}estạr ovar þesa \alst{w}er-oldi. \hld\ Nú is it al gi·\alst{w}árod só, &
\alst{k}uman þurh \alst{k}raft godes: \hld\ þe \alst{k}uning is gi·fódit, &
gi·\alst{b}oran \alst{b}ald ęndi strang: \hld\ wí gi·sáhun is \alst{b}ôkan skínan &
\alst{h}êdro fon \alst{h}imiles tunglun, \hld\ só ik wêt, þat it \alst{h}êlag drohtin, &
\alst{m}arkoda \alst{m}ahtig selvo; \hld\ wí gi·sáhun \alst{m}orgno gi·hwi-likes &
\alst{b}líkan þana \alst{b}erhton sterron, \hld\ ęndi wí géngun aftar þem \alst{b}ôkna herod &
\alst{w}egas ęndi \alst{w}aldas hwílon. \hld\ Þat wári u̇s allaro \alst{w}illjono mêsta, &
þat wí ina \alst{s}elvon gi·\alst{s}ehan móstin, \hld\ wissin, hwár wí ina \alst{s}ókjan skoldin, &
þana \alst{k}uning an þesumu \alst{k}êsur-dóma. \hld\ Saga u̇s, undar hwi-likumu hé sí þesaro \alst{k}unnjo a·fódit.“ &
Þó warð \alst{E}rodesa \hld\ \alst{i}nnan briostun &
\alst{h}arm wið \alst{h}erta, \hld\ bi·gan im is \alst{h}ugi wallan, &
\alst{s}evo mid \alst{s}orgun: \hld\ gi·hôrde \alst{s}ęggjan þó, &
þat hé þár \alst{o}var-hôvdon \hld\ \alst{ê}gan skoldi, &
\alst{k}raftagoron \alst{k}uning \hld\ \alst{k}unnjes gódes, &
\alst{s}áligoron undar þem gi·\alst{s}ïðja. \hld\ Þó hé \alst{s}amnon hét, &
só hwat só an \alst{J}erusalem \hld\ \alst{g}ódaro manno &
allaro \alst{sp}áhoston \hld\ \alst{sp}rákono wárun &
ęndi an iro \alst{b}rioston \hld\ \alst{b}ók-kraftes mêst &
\alst{w}issun te \alst{w}árun, \hld\ ęndi hé sie mid \alst{w}ordun fragn, &
swíðo \alst{n}iud-líko \hld\ \alst{n}íð-hugdig man, &
\alst{k}uning þero liudjo, \hld\ hwár \alst{K}rist gi·boran &
an \alst{w}er-old-ríkja \hld\ \alst{w}erðan skoldi, &
\alst{f}riðu-gumono bętst. \hld\ Þó sprak im eft þat \alst{f}olk an·gęgin, &
þat \alst{w}erod \alst{w}ár-líko, \hld\ kwáðun þat sie \alst{w}issin garo, &
þat hé skoldi an \alst{B}ethleem gi·\alst{b}oran werðan: \hld\ „só is an u̇sun \alst{b}ókun gi·skrivan, &
\alst{w}ís-líko gi·\alst{w}ritan, \hld\ só it \alst{w}ár-sagon, &
swíðo \alst{g}lawa \alst{g}umon \hld\ bi \alst{g}odes krafta &
\alst{f}il-wíse man \hld\ \alst{f}urn gi·sprákun, &
þat skoldi fon \alst{B}ethleem \hld\ \alst{b}urgo hirdi, &
\alst{l}iof \alst{l}andes ward \hld\ an þit \alst{l}ioht kuman, &
\alst{r}íki \alst{r}ád-gevo, \hld\ þe \alst{r}ihtjen skal &
\alst{J}udeono \alst{g}um-skępi \hld\ ęndi is \alst{g}eva wesan &
\alst{m}ildi ovar \alst{m}iddil-gard \hld\ \alst{m}anagun þiodun.“\eva

\bvb TODO.\evb\evg

\bvg\bva[8][630]%
Þó gi·fragn ik þat \alst{s}án aftar þiu \hld\ \alst{s}líð-mód kuning &
þero \alst{w}ár-sagono \alst{w}ord \hld\ þem \alst{w}rękkjun sagda, &
þea þár an \alst{ę}li-lęndi \hld\ \alst{e}rlos wárun &
\alst{f}erran gi·\alst{f}arana, \hld\ ęndi hé \alst{f}rágoda aftar þiu, &
hwan sie an \alst{ô}star-wegun \hld\ \alst{ê}rist gi·sáhin &
þana \alst{k}uning-sterron \alst{k}uman, \hld\ \alst{k}umbạl liuhtjen &
\alst{h}êdro fon \alst{h}imile. \hld\ Sie ni weldun is im þó \alst{h}elen eo·wiht, &
ak \alst{s}agdun it im \alst{s}ȯð-líko. \hld\ Þó hét hé sie an þana \alst{s}ïð faran, &
hét þat sie ira \alst{â}rundi \alst{a}l \hld\ \alst{u}ndar fundin &
umbi þes \alst{k}indes \alst{k}umi, \hld\ ęndi þe \alst{k}uning selvo gi·bôd &
swíðo \alst{h}ard-liko, \hld\ \alst{h}êrro Judeono, &
þem \alst{w}ísun mannun, \hld\ êr þan sie fórin \alst{w}estan forð, &
þat sie im eft gi·\alst{k}u̇ðdin, \hld\ hwár hé þana \alst{k}uning skoldi &
\alst{s}ókjan at is \alst{s}ęlðon; \hld\ kwað þat hé þár weldi mid is gi·\alst{s}ïðun tó, &
\alst{b}edan te þem \alst{b}arne. \hld\ Þan hogda hé im te \alst{b}anon werðan &
\alst{w}ápnes ęggjun. \hld\ Þan eft \alst{w}aldand god &
\alst{þ}ȧhte wið þem \alst{þ}inga: \hld\ hé mahta a·\alst{þ}ęngjan mêr, &
gi·\alst{l}êstjan an þesum \alst{l}iohte: \hld\ þat is noh \alst{l}ango skín, &
gi·\alst{k}u̇ðid \alst{k}raft godes. \hld\ Þó géngun eft þiu \alst{k}umbl forð &
\alst{w}ánum undar \alst{w}olknun. \hld\ Þó wárun þea \alst{w}íson man &
\alst{f}u̇sa te \alst{f}aranne: \hld\ gi·witun im \alst{f}orð þanan &
\alst{b}alda an \alst{b}od-skępi: \hld\ weldun þat \alst{b}arn godes &
\alst{s}elvon \alst{s}ókjan. \hld\ Sie ni habdun þanan gi·\alst{s}ïðjas mêr, &
b·útan þat sie \alst{þ}ríe wárun: \hld\ wissun im \alst{þ}ingo gi·skêð, &
wárun im \alst{g}lawe \alst{g}umon, \hld\ þe þea \alst{g}eva lêddun. &
Þan sáhun sie só \alst{w}ís-líko \hld\ undar þana \alst{w}olknes skion, &
up te þem \alst{h}ôhon \alst{h}imile, \hld\ hwó fórun þea \alst{h}wíton sterron &
—ant·\alst{k}ęndun sie þat \alst{k}umbạl godes—, \hld\ þiu wárun þurh \alst{K}rista herod &
gi·\alst{w}arht te þesero \alst{w}er-oldi. \hld\ Þea \alst{w}eros aftar géngun, &
\alst{f}olgodun \alst{f}erạht-líko \hld\ —sie \alst{f}rumide þe mahte— &
ant-þat sie gi·\alst{s}áhun, \hld\ \alst{s}ïð-wórige man, &
\alst{b}erht \alst{b}ôkạn godes, \hld\ \alst{b}lêk an himile &
\alst{st}illo gi·\alst{st}anden. \hld\ Þe \alst{st}erro liohto skên &
\alst{h}wít ovar þem \alst{h}úse, \hld\ þár þat \alst{h}êlage barn &
\alst{w}onode an \alst{w}illjon \hld\ ęndi ina þat \alst{w}íf bi·held, &
þiu \alst{þ}iorne gi·\alst{þ}iudo. \hld\ Þó warð þero \alst{þ}egno hugi &
\alst{b}líði an iro \alst{b}riostun: \hld\ bi þem \alst{b}ôkna for·stódun, &
þat sie þat \alst{f}riðu-barn godes \hld\ \alst{f}unden habdun, &
\alst{h}êlagna \alst{h}evan-kuning. \hld\ Þó sie an þat \alst{h}ús innan &
mid iro \alst{g}evun \alst{g}éngun, \hld\ \alst{g}umon ôstr-onja, &
\alst{s}ïð-wórige man: \hld\ \alst{s}án ant·kęndun &
þea \alst{w}eros \alst{w}aldand Krist. \hld\ Þea \alst{w}rękkjon fellun &
te þem \alst{k}inde an \alst{k}neo-beda \hld\ ęndi ina an \alst{k}uning-wísa &
\alst{g}ódan \alst{g}róttun \hld\ ęndi im þea \alst{g}eva drógun, &
\alst{g}old ęndi wíh-rôk \hld\ bi \alst{g}odes têknun &
*ęndi \alst{m}yrra þár \alst{m}id. \hld\ Þea \alst{m}an stódun garowa, &
\alst{h}olde for iro \alst{h}êrron, \hld\ þea it mid iro \alst{h}andun sán &
\alst{f}agạro ant·\alst{f}éngun. \hld\ Þó gi·witun im þea \alst{f}erạhton man, &
\alst{s}ęggi te \alst{s}ęlðon \hld\ \alst{s}ïð-wórige, &
\alst{g}umon an \alst{g}ast-sęli. \hld\ Þár im \alst{g}odes ęngil &
\alst{s}lápandjun an naht \hld\ \alst{s}wevan gi·tôgde, &
gi·\alst{d}rog im an \alst{d}rôme, \hld\ al so it \alst{d}rohtin self, &%TODO: Form of vowel in gi·drog unclear. Looks like a verb.
\alst{w}aldand \alst{w}elde, \hld\ þat im þu̇hte þat man im mid \alst{w}ordun gi·budi, &
þat sie im* þanan \alst{ȯ}ðran weg, \hld\ \alst{e}rlos fórin, &
\alst{l}iðodin sie te \alst{l}ande \hld\ ęndi þana \alst{l}êðan man, &
\alst{E}rodesan \hld\ \alst{e}ft ni sóhtin, &
\alst{m}ódagna kuning. \hld\ Þó warð \alst{m}organ kuman &
\alst{w}ánum te þesero \alst{w}er-oldi. \hld\ Þó bi·gunnun þea \alst{w}íson man &
\alst{s}ęggjan iro \alst{s}wevanos; \hld\ \alst{s}elvon ant·kęndun &
\alst{w}aldandes \alst{w}ord, \hld\ hwand sie gi·\alst{w}it mikil &
\alst{b}árun an iro \alst{b}riostun: \hld\ \alst{b}ádun alo-waldon, &
\alst{h}êron \alst{h}evan-kuning, \hld\ þat sie móstin is \alst{h}uldi forð, &
gi·\alst{w}irkjan is \alst{w}illjon, \hld\ kwáðun þat sea ti im habdin gi·\alst{w}ęndit hugi, &
*iro \alst{m}ód \alst{m}organ gi·hwem. \hld\ Þó fórun eft þie \alst{m}an þanan, &
\alst{e}rlos \alst{ô}str-onje, \hld\ al só im þe \alst{ę}ngil godes &
\alst{w}ordun gi·\alst{w}ísde: \hld\ námun im \alst{w}eg ȯðran, &
ful-\alst{g}éngun \alst{g}odes lêrun: \hld\ ni weldun þemu \alst{J}udeo kuninge &
umbi þes \alst{b}arnes gi·\alst{b}urd \hld\ \alst{b}odon ôstr-onje, &
\alst{s}ïð-wórige man \hld\ \alst{s}ęggjan gio·wiht, &
ak \alst{w}endun im eft an iro \alst{w}illjon.\eva

\bvb TODO.\evb\evg

\bvg\bva[9][699]%
\hspace*{100pt} Þó warð sán aftar þiu \alst{w}aldandes, &%NOTE: In cæsura
\alst{g}odes ęngil kumen \hld\ \alst{J}osepe te sprákun, &
\alst{s}agde im an \alst{s}wefne \hld\ \alst{s}lápandjum an naht, &
\alst{b}odo drohtines, \hld\ þat þat \alst{b}arn godes &
\alst{s}líð-mód kuning \hld\ \alst{s}ókjan welda, &
\alst{á}htjan is \alst{a}ldres; \hld\ „nú skaltu ine an \alst{A}egypteo &
\alst{l}and ant·\alst{l}êdjan \hld\ ęndi undar þem \alst{l}iudjun wesan &
mid þiu \alst{g}odes barnu \hld\ ęndi mid þeru \alst{g}ódan þior*nan, &
\alst{w}unon undar þemu \alst{w}erode, \hld\ unt-þat þi \alst{w}ord kume &
\alst{h}êrron þínes, \hld\ þat þú þat \alst{h}êlage barn &
eft te þesum \alst{l}and-skępi \hld\ \alst{l}êdjan mótis, &
\alst{d}rohtin þínen.“ \hld\ Þó fon þem \alst{d}rôma an·sprang &
\alst{J}oseph an is \alst{g}ęst-sęli, \hld\ ęndi þat \alst{g}odes gi·bod &
\alst{s}án ant·kęnda: \hld\ gi·wêt im an þana \alst{s}ïð þanen &
þe \alst{þ}egạn mid þeru \alst{þ}iornon, \hld\ sóhta im \alst{þ}iod ȯðra &
ovar \alst{b}rêdan \alst{b}erg: \hld\ welda þat \alst{b}arn godes &
\alst{f}íundun ant·\alst{f}órjan. \hld\ *Þó gi·\alst{f}rang aftar þiu &%NOTE: gi·frang [sic]
E\alst{r}ódes þe kuning, \hld\ þár hé an is \alst{r}íkja sat, &
þat \alst{w}árun þea \alst{w}íson man \hld\ \alst{w}estan gi·hworvan &
\alst{ô}star an iro \alst{ó}ðil \hld\ ęndi fórun im \alst{ȯ}ðran weg: &
wisse þat sie im þat \alst{â}rundi \hld\ \alst{e}ft ni weldun &
\alst{s}ęggjan an is \alst{s}ęlðon. \hld\ Þó warð im þes an \alst{s}orgun hugi, &
\alst{m}ód \alst{m}ornondi, \hld\ kwað þat it im þie \alst{m}an dedin, &
\alst{h}ęliðos* te \alst{h}ônðun. \hld\ Þó hé só \alst{h}riuwig sat, &
\alst{b}alg ina an is \alst{b}riostun, \hld\ kwað þat hé is mahti \alst{b}ętaron rád, &
\alst{ȯ}ðran gi·þęnkjen: \hld\ „nú ik is \alst{a}ldạr kan, &
\alst{w}êt is \alst{w}intẹr-gi·talu: \hld\ nú ik gi·\alst{w}innan mag, &
þat hé io ovar þesaro \alst{e}rðu \hld\ \alst{a}ld ni wirðit, &
\alst{h}êr undar þesum \alst{h}ęri-skępi.“ \hld\ Þó hé só \alst{h}ardo gi·bôd, &
E\alst{r}ódes ovar is \alst{r}íki, \hld\ hét þó is \alst{r}inkos faran &
\alst{k}uning þero liudjo, \hld\ hét þat sie \alst{k}inda só filo &
þurh iro \alst{h}and-magen \hld\ \alst{h}ôvdu bi·námin, &
só manag \alst{b}arn umbi \alst{B}ethleem, \hld\ só filo só þár gi·\alst{b}oran wurði, &
an \alst{t}wêm gę́run a·\alst{t}ogan. \hld\ \alst{T}ionon frumidon &
þes \alst{k}uninges gi·sïðos. \hld\ Þó skolda þár só manag \alst{k}indisk man &
\alst{s}weltan \alst{s}undjono lôs. \hld\ Ni warð \alst{s}íð noh êr &
\alst{j}ámar-líkara for·\alst{g}ang \hld\ \alst{j}ungaro manno, &
\alst{a}rm-líkara dôð. \hld\ \alst{I}disi wiopun, &
\alst{m}ódar \alst{m}anaga, \hld\ gi·sáhun iro \alst{m}ęgi spildjan: &
ni mahte siu im nio gi·\alst{f}ormon, \hld\ þoh siu mid iro \alst{f}aðmon twêm &
iro \alst{ê}gan barn \hld\ \alst{a}rmun bi·féngi, &
\alst{l}iof ęndi \alst{l}uttil, \hld\ þoh skolda is simbla þat \alst{l}íf gevan, &
þe \alst{m}agu for þeru \alst{m}ódar. \hld\ \alst{M}ênes ni sáhun, &
\alst{w}ítjes þie \alst{w}am-skaðon: \hld\ \alst{w}ápnes ęggjun &
\alst{f}ręmidun \alst{f}irin-werk mikil. \hld\ \alst{F}ellun managa &
\alst{m}agu-junge \alst{m}an. \hld\ Þia \alst{m}ódar wiopun &
\alst{k}ind-jungaro \alst{k}walm; \hld\ \alst{k}ara was an Bethleem, &
\alst{h}ofno \alst{h}lúdost: \hld\ þoh man im iro \alst{h}erton an twê &
\alst{s}niði mid \alst{s}werdu, \hld\ þoh ni mohta im gio \alst{s}êrara dád &
\alst{w}erðan an þesaro \alst{w}er-oldi, \hld\ \alst{w}ívun managun, &
\alst{b}rúdjun an \alst{B}ethleem: \hld\ gi·sáhun iro \alst{b}arn bi·foran, &
\alst{k}ind-junge man, \hld\ \alst{k}walmu sweltan &
\alst{b}lódag an iro \alst{b}armun. \hld\ Þie \alst{b}anon wítnodun &
un·\alst{sk}uldige \alst{sk}ole: \hld\ ni bi·\alst{sk}rivun gio·wiht &
þea \alst{m}an umbi \alst{m}ên-werk: \hld\ weldun \alst{m}ahtigna, &
\alst{K}rist selvon a·\alst{k}węlljan. \hld\ Þan habde ina \alst{k}raftag god &
gi·\alst{n}ęridan wið iro \alst{n}íðe, \hld\ þat inan \alst{n}ahtes þanan &
an \alst{A}egypteo land \hld\ \alst{e}rlos ant·lêddun, &
\alst{g}umon mid \alst{J}osepe \hld\ an þana \alst{g}rónjon wang, &
an \alst{e}rðono bętstun, \hld\ þár ên \alst{a}ha fliutid, &
\alst{N}íl-strôm mikil \hld\ \alst{n}orð te sêwa, &
\alst{f}lódo \alst{f}agọrosta. \hld\ Þár þat \alst{f}riðu-barn godes &
\alst{w}onoda an \alst{w}illjon, \hld\ ant-þat \alst{w}urd for·nam &
\alst{E}rodes þana kuning, \hld\ þat hé for·lét \alst{ę}ldjo barn, &
\alst{m}ódag \alst{m}anno drôm. \hld\ Þó skolda þero \alst{m}arka gi·wald &
\alst{ê}gan is \alst{ę}rvi-ward: \hld\ þe was \alst{A}rkheláus &
\alst{h}êtan, \alst{h}ęri-togo \hld\ \alst{h}elm-berandero: &
þe skolda umbi \alst{J}erusalem \hld\ \alst{J}udeono folkes, &
\alst{w}erodes gi·\alst{w}aldan. \hld\ Þó warð \alst{w}ord kuman &
þár an \alst{E}gypti \hld\ \alst{ę}ðiljun manne, &
þat hé þár te \alst{J}osepe, \hld\ \alst{g}odes ęngil sprak, &
\alst{b}odo drohtines, \hld\ hét ina eft þat \alst{b}arn þanan &
\alst{l}êdjen te \alst{l}ande. \hld\ „nú havað þit \alst{l}ioht af·geven“, kwað hé, &
„\alst{E}rodes þe kuning; \hld\ hé welde is \alst{á}htjen giu, &
\alst{f}rêson is \alst{f}erạhas. \hld\ Nú maht þú an \alst{f}riðu lêdjen &
þat \alst{k}ind undar euwa \alst{k}unni, \hld\ nú þe \alst{k}uning ni livod, &
\alst{e}rl \alst{o}var-módig.“ \hld\ \alst{A}l ant·kęnde &
\alst{J}osep \alst{g}odes têkạn: \hld\ \alst{g}ęrịwide ina sniumo &
þe \alst{þ}egạn mit þera \alst{þ}iornun, \hld\ þó sie \alst{þ}anan weldun &
\alst{b}êðju mid þiu \alst{b}arnu: \hld\ lêstun þiu \alst{b}erhton gi·skapu, &
\alst{w}aldandes \alst{w}illjon, \hld\ al só hé im êr mid is \alst{w}ordun gi·bôd.\eva

\bvb TODO.\evb\evg

\bvg\bva[10][780]%
Gi·witun im þó eft an \alst{G}alilea-land \hld\ \alst{J}oseph ęndi Maria, &
\alst{h}êlag \alst{h}íwiski \hld\ \alst{h}evan-kuninges, &
wárun im an \alst{N}azareth-burg. \hld\ Þár þe \alst{n}ęrjondjo Krist &
\alst{w}óhs undar þem \alst{w}erode, \hld\ warð gi·\alst{w}ittjes ful, &
\alst{a}n was imu \alst{a}nst godes, \hld\ hé was \alst{a}llun liof &
\alst{m}ódar-\alst{m}águn: \hld\ hé ni was ȯðrun \alst{m}annun gi·lík, &
þe \alst{g}umo an sínera \alst{g}ódi. \hld\ Þó hé \alst{g}ę́r-talo &
\alst{t}we-livi habde, \hld\ þó warð þiu \alst{t}íd kuman, &
þat sie þár te \alst{J}erusalem, \hld\ \alst{J}uðeo liudi &
iro \alst{þ}iod-gode \hld\ \alst{þ}ionon skoldun, &
\alst{w}irkjan is \alst{w}illjon. \hld\ Þó warð þár an þana \alst{w}íh innan &
þár te \alst{J}erusalem \hld\ \alst{J}udeono gi·samnod &
\alst{m}an-kraft \alst{m}ikil. \hld\ Þár \alst{M}aria was &
\alst{s}elf an gi·\alst{s}ïðja \hld\ ęndi iru \alst{s}unu habda, &
\alst{g}odes êgan barn. \hld\ Þó sie þat \alst{g}eld habdun, &
\alst{e}rlos an þem \alst{a}lạha, \hld\ só it an iro \alst{ê}wa gi·bôd, &
gi·\alst{l}êstid te iro \alst{l}and-wísun, \hld\ þó fórun im eft þie \alst{l}iudi þanan, &
\alst{w}eros an iro \alst{w}illjon \hld\ ęndi þár an þem \alst{w}íha af·stód &
\alst{m}ahtig barn godes, \hld\ só ina þiu \alst{m}ódar þár &
ni \alst{w}issa te \alst{w}áron; \hld\ ak siu wánda þat hé mid þem \alst{w}eroda forð, &
\alst{f}óri mit iro \alst{f}riundun. \hld\ Gi·\alst{f}rang aftar þiu &
eft an \alst{ȯ}ðrun daga \hld\ \alst{a}ðal-kunnjes wíf, &
\alst{s}álig þiorna, \hld\ þat hé undar þem gi·\alst{s}ïðja ni was. &
warð \alst{M}ariun þó \hld\ \alst{m}ód an sorgun, &
\alst{h}riuwig umbi iro \alst{h}erta, \hld\ þó siu þat \alst{h}êlaga barn &
ni \alst{f}and undar þem \alst{f}olka: \hld\ \alst{f}ilu gornoda &
þiu \alst{g}odes þiorna. \hld\ Gi·witun im þó eft te \alst{J}erusalem &
iro \alst{s}unu \alst{s}ókjan, \hld\ fundun ina \alst{s}ittjan þár &
an þem \alst{w}íha innan, \hld\ þár þe \alst{w}ísa man, &
swíðo \alst{g}lauwa \alst{g}umon \hld\ an \alst{g}odes êwa &
\alst{l}ásun ęnde \alst{l}ínodun, \hld\ hwó sie \alst{l}of skoldin &%NOTE: the ms spelling <ende> for <endi> only occurs here and 38 lines below
\alst{w}irkjan mid iro \alst{w}ordun þem, \hld\ þe þesa \alst{w}er-old gi·skóp. &
Þár sat undar \alst{m}iddjun \hld\ \alst{m}ahtig barn godes, &
\alst{K}rist alo-waldo, \hld\ só is þea ni mahtun ant·\alst{k}ęnnjan wiht, &
þe þes \alst{w}íhes þár \hld\ \alst{w}ardon skoldun, &
ęndi \alst{f}rágoda sie \hld\ \alst{f}iri-wit-líko &
\alst{w}ísera \alst{w}ordo. \hld\ Sie \alst{w}undradun alle, &
\edtext{bi·hwí}{\Afootnote{\emph{hwó} \textbf{C}}} gio só \alst{k}indisk man \hld\ su·lika \alst{k}widi mahti &
\edtext{mid is \alst{m}u̇ðu gi·\alst{m}ênjan}{\Afootnote{\emph{gi·mahljan mid is mu̇ðu} \textbf{C}}}. \hld\ Þár ina þiu \alst{m}ódar fand &
\alst{s}ittjan under þem gi·\alst{s}ïðja \hld\ ęndi iro \alst{s}unu grótta, &
\alst{w}ísan undar þem \alst{w}eroda, \hld\ sprak im mid ira \alst{w}ordun tó: &
„hwí weldes þú þínera \alst{m}ódar, \hld\ \alst{m}anno liovosto, &
gi·\alst{s}idon su·lika \alst{s}orga, \hld\ þat ik þí só \alst{s}êrag-mód, &%NOTE:Checked "gi·sidon"
\alst{i}dis \alst{a}rm-hugdig \hld\ \alst{ê}skon skolda &
undar þesun \alst{b}urg-liudjun?“ \hld\ Þó sprak iru eft þat \alst{b}arn an·gęgin &
\alst{w}ísun \alst{w}ordun: \hld\ „Hwat þú \alst{w}êst garo“, kwað hé, &
„þat ik þár gi·\alst{r}ísu, \hld\ þár ik bi \alst{r}ehton skal &
\alst{w}onon an \alst{w}illjon, \hld\ þár gi·\alst{w}ald havad &%TODO: Check "wonon"
\alst{m}ín \alst{m}ahtig fader.“ \hld\ Þie \alst{m}an ni for·stódun, &
þie \alst{w}eros an þem \alst{w}íha, \hld\ bi·hwí hé só þat \alst{w}ord gi·sprak, &
gi·\alst{m}ênda mid is \alst{m}u̇ðu: \hld\ \alst{M}aria al bi·held, &
gi·\alst{b}arg an ira \alst{b}reostun, \hld\ só hwat só siu gi·hôrda ira \alst{b}arn sprekan &
\alst{w}ísaro \alst{w}ordo. \hld\ Gi·\alst{w}itun im þó eft þanan &
fon \alst{J}erusalem \hld\ \alst{J}oseph ęndi Maria, &
habdun im te gi·\alst{s}ïðja \hld\ \alst{s}unu drohtines, &
allaro \alst{b}arno \alst{b}ętsta, \hld\ þero þe io gi·\alst{b}oran wurði &
\alst{m}agu fon \alst{m}ódar: \hld\ habdun im þár \alst{m}innja tó &
þurh \alst{h}luttran \alst{h}ugi, \hld\ ęndi hé só gi·\alst{h}ôrig was, &
\alst{g}odes êgan barn \hld\ \alst{g}aduling-mágun &
þurh is \alst{ô}d-módi, \hld\ \alst{a}ldron sínun: &
ni welda an is \alst{k}indiski þó noh \hld\ is \alst{k}raft mikil &
\alst{m}annun \alst{m}árjan, \hld\ þat hé su·lik \alst{m}ęgin êhta, &
gi·\alst{w}ald an þesaro \alst{w}er-oldi, \hld\ ak hé im an is \alst{w}illjon bêd &
gi·\alst{þ}iudo undar þero \alst{þ}iodu \hld\ \alst{þ}rí-tig gę́ro, &
êr þan hé þár \alst{t}êkạn ênig \hld\ \alst{t}ôgjan weldi, &
\alst{s}ęggjan þem gi·\alst{s}ïðja, \hld\ þat hé \alst{s}elvo was &
an þesaro \alst{m}iddil-gard \hld\ \alst{m}anno drohtin. &
\alst{H}abda im só bi·\alst{h}alden \hld\ \alst{h}êlag barn godes &
\alst{w}ord ęndi \alst{w}ís-dóm \hld\ ęnde allaro gi·\alst{w}ittjo mêst, &
tulgo \alst{sp}áhan hugi: \hld\ ni mahta man is an is \alst{sp}rákun werðan, &
an is \alst{w}ordun gi·\alst{w}ar, \hld\ þat hé su·lik gi·\alst{w}it êhta, &
\alst{þ}egạn su·lika gi·\alst{þ}ȧhti, \hld\ ak hé im só gi·\alst{þ}iudo bêd &
\alst{t}orhtaro \alst{t}êkno. \hld\ Ni was noh þan þiu \alst{t}íd kuman, &
þat hé ina ovar þesan \alst{m}iddil-gard \hld\ \alst{m}árjan skolda, &
\alst{l}êrjan þie \alst{l}iudi, \hld\ hwó sie skoldin iro gi·\alst{l}ôvon haldan, &
\alst{w}irkjan \alst{w}illjon godes; \hld\ \alst{w}issun þat þoh managa &
\alst{l}iudi aftar þem \alst{l}anda, \hld\ þat hé was an þit \alst{l}ioht kuman, &
þoh sie ina \alst{k}u̇ð-líko \hld\ an·\alst{k}ęnnjan ni mahtin, &
êr þan hé ina \alst{s}elvo \hld\ \alst{s}ęggjan welda.\eva

\bvb TODO.\evb\evg

\bvg\bva[11][859]%
Þan was im \alst{J}ohannes \hld\ fon is \alst{j}uguð-hêdi &
a·\alst{w}ahsan an ênero \alst{w}óstunni; \hld\ þár ni was \alst{w}erodes þan mêr, &
b·útan þat hé þár \alst{ê}n-kora \hld\ \alst{a}lo-waldon gode, &
\alst{þ}egạn \alst{þ}ionoda: \hld\ for·lét \alst{þ}ioda gi·mang, &
\alst{m}anno gi·\alst{m}ênðon. \hld\ Þár warð im \alst{m}ahtig kuman &
an þero \alst{w}óstunni \hld\ \alst{w}ord fon himila, &
\alst{g}ód-lík stemna \alst{g}odes, \hld\ ęndi \alst{J}ohanne gi·bod, &
þat hé \alst{K}ristes \alst{k}umi \hld\ ęndi is \alst{k}raft mikil &
ovar þesan \alst{m}iddil-gard \hld\ \alst{m}árjan skoldi; &
hét ina \alst{w}ár-líko \hld\ \alst{w}ordun sęggjan, &
þat wári \alst{h}evan-ríki \hld\ \alst{h}ęliðo barnun &
an þem \alst{l}and-skępi, \hld\ \alst{l}iudjun gi·náhid, &
\alst{w}elono \alst{w}un-samost. \hld\ Im was þó \alst{w}illjo mikil, &
þat hé fon \alst{s}u·likun \alst{s}áldun \hld\ \alst{s}ęggjan mósti. &
Gi·wêt im þó \alst{g}angan, \hld\ al só \alst{J}ordan flót, &
\alst{w}atar an \alst{w}illjon, \hld\ ęndi þem \alst{w}eroda allan dag, &
aftar þem \alst{l}and-skępi \hld\ þem \alst{l}iudjun ku̇ðda, &
þat sie mid \alst{f}astunnju \hld\ \alst{f}irin-werk manag, &
iro \alst{s}elvoro \hld\ \alst{s}undja bóttin, &
„þat gí werðan \alst{h}rênja“, \hld[kwað hé;] „\alst{h}evan-ríki is &
gi·náhid \alst{m}anno barnun. \hld\ Nú látad eu an euwan \alst{m}ód-sevon &
euwar \alst{s}elvoro \hld\ \alst{s}undja hreuwan, &
\alst{l}êdas þat gí an þesun \alst{l}iohta fręmidun, \hld\ ęndi mínun \alst{l}êrun hôrjad, &
\alst{w}ęndjat aftar mínun \alst{w}ordun. \hld\ Ik eu an \alst{w}atara skal &
gi·\alst{d}ôpjan \alst{d}iur-líko, \hld\ þoh ik euwa \alst{d}ádi ne mugi, &
euwar \alst{s}elvaro \hld\ \alst{s}undja a·látan, &
þat gí þurh mín \alst{h}and-gi·werk \hld\ \alst{h}luttra werðan &
\alst{l}êðaro gi·\alst{l}êsto: \hld\ ak þe is an þit \alst{l}ioht kuman, &
\alst{m}ahtig te \alst{m}annun \hld\ ęndi undar eu \alst{m}iddjun stéd, &
—þoh gí ina \alst{s}elvun \hld\ gi·\alst{s}ehan ni willjan—, &
þe eu gi·\alst{d}ôpjan skal \hld\ an euwes \alst{d}rohtines namon &
an þana \alst{h}âlagon gêst. \hld\ Þat is \alst{h}êrro ovar al: &
hé mag allaro \alst{m}anno gi·hwena \hld\ \alst{m}ên-gi·þȧhtjo, &
\alst{s}undjono \alst{s}ikoron, \hld\ só hwene só só \alst{s}álig mót &
\alst{w}erðen an þesaro \alst{w}er-oldi, \hld\ þat þes \alst{w}illjon havad, &
þat hé só gi·\alst{l}êstja, \hld\ só hé þesun \alst{l}iudjun wili, &
gi·\alst{b}ioden \alst{b}arn godes. \hld\ Ik bium an is \alst{b}od-skępi herod &
an þesa \alst{w}er-old kumen \hld\ ęndi skal im þana \alst{w}eg rúmjen, &
\alst{l}êrjan þesa \alst{l}iudi, \hld\ hwó sea skulin iro gi·\alst{l}ôvon haldan &
þurh \alst{h}luttran \alst{h}ugi, \hld\ ęndi þat sie an \alst{h}ęllja ni þurvin, &
\alst{f}aran an \alst{f}ern þat hêta. \hld\ Þes wirðid só \alst{f}agan an is móde &
man te só \alst{m}anagaro stundu, \hld\ só hwe só þat \alst{m}ên for·látid, &
\alst{g}erno þes \alst{g}ramon an-busni, \hld\ —só mag im þes \alst{g}ódon gi·wirkjan, &
\alst{h}uldi \alst{h}evan-kuninges,— \hld\ só hwe só havad \alst{h}luttra treuwa &
up te þem \alst{a}lo-mahtigon gode.“ \hld\ \alst{E}rlos managa &
bi þem \alst{l}êrun þó, \hld\ \alst{l}iudi wándun, &
\alst{w}eros \alst{w}ár-líko, \hld\ þat þat \alst{w}aldand Krist &
\alst{s}elbo wári, \hld\ hwanda hé só filu \alst{s}ȯðes gi·sprak, &
\alst{w}ároro \alst{w}ordo. \hld\ Þó warð þat só \alst{w}ído ku̇ð &
ovar þat for·\alst{g}evana land \hld\ \alst{g}umono gi·hwi-likum, &
\alst{s}ęggjun at iro \alst{s}ęlðun: \hld\ þó kwámun ina \alst{s}ókjan þarod &
fon \alst{J}erusalem \hld\ \alst{J}udeo liudjo &
\alst{b}odon fon þeru \alst{b}urgi \hld\ ęndi frágodun, ef hé wári þat \alst{b}arn godes, &
„þat hér \alst{l}ango giu“, \hld[kwáðun sie,] „\alst{l}iudi sagdun, &
\alst{w}eros \alst{w}ár-líko, \hld\ þat hé skoldi an þesa \alst{w}er-old kuman“. &
\alst{J}ohannes þó gi·mahạlde \hld\ ęndi te·\alst{g}ęgnes sprak &
þem \alst{b}odun \alst{b}ald-líko: \hld\ „ni bium ik“, kwað hé, „þat \alst{b}arn godes, &
\alst{w}ár \alst{w}aldand Krist, \hld\ ak ik skal im þana \alst{w}eg rúmjen, &
\alst{h}êrron mínumu.“ \hld\ Þea \alst{h}ęliðos frugnun, &
þea þár an þem \alst{â}rundje \hld\ \alst{e}rlos wárun, &
\alst{b}odon fon þero \alst{b}urgi: \hld\ „ef þú nú ni bist þat \alst{b}arn godes, &
bist þú þan þoh \alst{E}lias, \hld\ þe hér an \alst{ê}r-dagun &
\alst{w}as undar þesumu \alst{w}erode? \hld\ hé is \alst{w}is-kumo &
eft an þesan \alst{m}iddil-gard. \hld\ Saga u̇s, hwat þú \alst{m}anno sís! &
Bist þú \alst{ê}nig þero, \hld\ þe hér \alst{ê}r wári &
\alst{w}ísaro \alst{w}ár-saguno? \hld\ Hwat skulun wí þem \alst{w}erode fon þí &
\edtext{\alst{s}ęggjan te \alst{s}ȯðon}{\Bfootnote{Formulaic, also found in \Heliand\ 2077a, 4018a, 4988a, along with \Beowulf\ 51a: \emph{sęcgan tó sȯðe}.}}? \hld\ Neo hér êr \alst{s}u·lik ni warð &
an þesun \alst{m}iddil-gard \hld\ \alst{m}an ȯðar kuman &
\alst{d}ádjun só mári. \hld\ Bi·hwí þú hér \alst{d}ôpisli &
\alst{f}ręmis undar þesumu \alst{f}olke, \hld\ ef þú þaro \alst{f}ora·sagono &
\alst{ê}n-hwi-lik ni bist?“ \hld\ Þó habde \alst{e}ft garo &
\alst{J}ohannes þe \alst{g}ódo \hld\ \alst{g}lau and-wordi: &
„Ik bium \alst{f}ora-bodo \hld\ \alst{f}râon mínes, &
\alst{l}ioves hêrron; \hld\ ik skal þit \alst{l}and rekon, &
þit \alst{w}erod aftar is \alst{w}illjon. \hld\ Ik hębbju fon is \alst{w}orde mid mí &
\alst{st}ranga \alst{st}emna, \hld\ þoh sie hér ni willje for·\alst{st}andan filo &
\alst{w}erodes an þesaro \alst{w}óstunni. \hld\ Ni bium ik mid \alst{w}ihti gi·lík &
\alst{d}rohtine mínumu: \hld\ hé is mid is \alst{d}ádjun só strang, &
só \alst{m}ári ęndi só \alst{m}ahtig \hld\ —þat wirðid \alst{m}anagun ku̇ð, &
\alst{w}erun aftar þesaro \alst{w}er-oldi— \hld\ þat ik þes \alst{w}irðig ni bium, &
þat ik móti an is gi·\alst{sk}uoha, \hld\ þoh ik sí is \alst{sk}alk êgan, &
an só \alst{r}íkjumu drohtine, \hld\ þea \alst{r}eomon ant·bindan: &
só mikilu is hé \alst{b}ętara þan ik. \hld\ Nis þes \alst{b}odon gi·mako &
\alst{ê}nig ovar \alst{e}rðu, \hld\ ne nú \alst{a}ftar ni skal &
\alst{w}erðan an þesaro \alst{w}er-oldi. \hld\ Hębbjad euwan \alst{w}illjon þarod, &
\alst{l}iudi euwan gi·\alst{l}ôvon: \hld\ þan eu \alst{l}ango skal &
wesan euwa \alst{h}ugi \alst{h}rómag; \hld\ þan gí \alst{h}ęlli-gi·þwing, &
for·\alst{l}átad \alst{l}êðaro drôm \hld\ ęndi sókjad eu \alst{l}ioht godes, &
\alst{u}p-\alst{ô}des hêm, \hld\ \alst{ê}wig ríki, &
\alst{h}ôhan \alst{h}evan-wang. \hld\ Ne látad euwan \alst{h}ugi twífljen!“\eva

\bvb TODO.\evb\evg

\bvg\bva[12][949]%
Só sprak þó \alst{j}ung \alst{g}umo \hld\ bi \alst{g}odes lêrun &
\alst{m}annun te \alst{m}árðu. \hld\ \alst{M}anag samnoda &
þár te \alst{B}ethania \hld\ \alst{b}arn Israheles; &
\alst{k}wámun þár te Johannese \hld\ \alst{k}uningo gi·sïðos, &
\alst{l}iudi te \alst{l}êrun \hld\ ęndi iro gi·\alst{l}ôvon ant·féngun. &
Hé \alst{d}ôpte sie \alst{d}ago gi·hwi-likes \hld\ ęndi im iro \alst{d}ádi lóg, &%TODO: check lóg
\alst{w}rêðaro \alst{w}illjon, \hld\ ęndi lovode im \alst{w}ord godes, &
\alst{h}êrron sínes: \hld\ „\alst{h}evan-ríki wirðid“, kwað hé, &
„\alst{g}aru \alst{g}umono só hwem, \hld\ só ti \alst{g}ode þęnkid &
ęndi an þana \alst{h}êljand *wili \hld\ \alst{h}luttro gi·lôvjan, &%NOTE ms. -- wili] P 1r.
\alst{l}êstjan is \alst{l}êra“. \hld\ Þó ni was \alst{l}ang te þiu, &
þat im fon \alst{G}alilea gi·wêt \hld\ \alst{g}odes êgan barn, &
*\alst{d}iur-lík \alst{d}rohtines sunu, \hld\ \alst{d}ôpi suokjan. &
\alst{w}as im þuo an is \alst{w}astme \hld\ \alst{w}aldandes barn*, &
al só hé mid þero \alst{þ}iodu \hld\ \alst{þ}rí-tig habdi &
\alst{w}intro an is \alst{w}er-oldi. \hld\ Þó hé an is \alst{w}illjon kwam, &
þár \alst{J}ohannes \hld\ an \alst{J}ordana strôme &
allan \alst{l}angan dag \hld\ \alst{l}iudi manage &
\alst{d}ôpte \alst{d}iur-líko. \hld\ Reht só hé þó is \alst{d}rohtin gi·sah, &
\alst{h}oldan \alst{h}êrron, \hld\ só warð im is \alst{h}ugi blíði, &
þes im þe \alst{w}illjo gi·stód, \hld\ ęndi sprak im þó mid is \alst{w}ordun tó, &
swíðo \alst{g}ód \alst{g}umo, \hld\ \alst{J}ohannes te Kriste: &
„nú kumis þú te mínero \alst{d}ôpi, \hld\ \alst{d}rohtin frô mín, &
\alst{þ}iod-gumono bętsto: \hld\ só skolde ik te \alst{þ}ínero duan, &
hwand þú bist allaro \alst{k}uningo \alst{k}raftigost.“ \hld\ \alst{K}rist selvo gi·bôd, &
\alst{w}aldand \alst{w}ár-líko, \hld\ þat hé ni spráki þero \alst{w}ordo þan mêr: &
„wêst þú, þat u̇s só gi·\alst{r}ísid“, \hld[kwað hé,] „allaro \alst{r}ehto gi·hwi-lik &
te gi·\alst{f}ulljanne \hld\ \alst{f}orð-wardes nú &
an \alst{g}odes willjon“. \hld\ \alst{J}ohannes stód, &
\alst{d}ôpte allan \alst{d}ag \hld\ \alst{d}ruht-folk mikil, &
\alst{w}erod an \alst{w}atere \hld\ ęndi ôk \alst{w}aldand Krist, &
\alst{h}êran \alst{h}evan-kuning \hld\ \alst{h}andun sínun &
an allaro \alst{b}aðo þem \alst{b}ętston \hld\ ęndi im þár te \alst{b}edu gi·hnêg &
an \alst{k}neo \alst{k}raftag. \hld\ \alst{K}rist up gi·wêt &
\alst{f}agạr fon þem \alst{f}lóde, \hld\ \alst{f}riðu-barn godes, &
\alst{l}iof \alst{l}iudjo ward. \hld\ Só hé þó þat \alst{l}and af·stóp, &%TODO: check af·stóp
só ant·\alst{h}lidun þó \alst{h}imiles doru, \hld\ ęndi kwam þe \alst{h}êlago gêst &
fon þem \alst{a}lo-waldon \hld\ \alst{o}vane te Kriste: &
—was im an gi·\alst{l}ík-nissje \hld\ \alst{l}ungras fugles, &
\alst{d}iur-líkara \alst{d}úvun— \hld\ ęndi sat im uppan u̇ses \alst{d}rohtines ahslu, &
\alst{w}onoda im ovar þem \alst{w}aldandes barne. \hld\ Aftar kwam þár \alst{w}ord fon himile, &
\alst{h}lúd fon þem \alst{h}ôhon radura \hld\ ęndi grótta þane \alst{h}êljand selvon, &
\alst{K}rista, allaro \alst{k}uningo bętston, \hld\ kwað þat hé ina gi·\alst{k}orana habdi &
\alst{s}elvo fon \alst{s}ínun ríkja, \hld\ kwað þat im þe \alst{s}unu líkodi &
\alst{b}ętst allaro gi·\alst{b}oranaro manno, \hld\ kwað þat hé im wári allaro \alst{b}arno liovost. &
Þat móste \alst{J}ohannes þó, \hld\ al só it \alst{g}od welde, &
gi·\alst{s}ehan ęndi gi·hôrjan. \hld\ Hé gi·deda it \alst{s}án aftar þiu &
\alst{m}annun \alst{m}ári, \hld\ þat sie þár \alst{m}ahtigna &
\alst{h}êrron \alst{h}abdun: \hld\ „Þit is“, kwað hé, „\alst{h}evan-kuninges sunu, &
\alst{ê}n \alst{a}lo-waldand: \hld\ þesas willjo ik \alst{u}r-kundjo &
\alst{w}esan an þesaro \alst{w}er-oldi, \hld\ hwand it sagda mí \alst{w}ord godes, &
\alst{d}rohtines stemne, \hld\ þó hé mí \alst{d}ôpjan hét &
\alst{w}eros an \alst{w}atare, \hld\ só hwár só ik gi·sáwi \alst{w}ár-líko &
þana \alst{h}êlagon gêst \hld\ *fan \alst{h}evan-wange &
an þesan \alst{m}iddil-gard \hld\ ênigan \alst{m}an waron, &
\alst{k}uman mid \alst{k}raftu; \hld\ þat kwað, þat skoldi \alst{K}rist wesan, &
\alst{d}iur-lík \alst{d}rohtines suno. \hld\ Hie \alst{d}ôpjan skal &
an þana \alst{h}êlagan gêst \hld\ ęndi \alst{h}êljan managa &%NOTE ms. -- þana] P end.
\alst{m}anno \alst{m}ên-dádi. \hld\ Hé havad \alst{m}aht fon gode, &
þat hé a·\alst{l}átan mag \hld\ \alst{l}iudjo gi·hwi-likun &
\alst{s}aka ęndi \alst{s}undja. \hld\ Þit is \alst{s}elvo Krist, &
\alst{g}odes êgan barn, \hld\ \alst{g}umono bętsto, &
\alst{f}riðu wið \alst{f}íundun. \hld\ Wala þat eu þes mag \alst{f}râh-mód hugi &
\alst{w}esan an þesaro \alst{w}er-oldi, \hld\ þes eu þe \alst{w}illjo gi·stód, &
þat gí só \alst{l}ibbjanda \hld\ þana \alst{l}andes ward &
\alst{s}elvon gi·\alst{s}áhun. \hld\ Ní mót sliumo \alst{s}undjono lôs &
manag \alst{g}êst faran \hld\ an \alst{g}odes willjon &
\alst{t}ionon a·\alst{t}ómid, \hld\ þe mid \alst{t}reuwon wili &%TODO: check a·tómid
wið is \alst{w}ini \alst{w}irkjan \hld\ ęndi an \alst{w}aldand Krist &
\alst{f}asto gi·lôvjan. \hld\ Þat skal te \alst{f}rumun werðen &
\alst{g}umono só hwi-likun, \hld\ só þat \alst{g}erno dót“.\eva

\bvb TODO.\evb\evg

\bvg\bva[13][1020]%
Só ge·fragn ik þat \alst{J}ohannes þó \hld\ \alst{g}umono gi·hwi-likun, &
\alst{l}ovoda þem \alst{l}iudjun \hld\ \alst{l}êra Kristes, &
\alst{h}êrron sínes, \hld\ ęndi \alst{h}evan-ríki &
te gi·\alst{w}innanne, \hld\ \alst{w}elono þane mêston, &
\alst{s}álig \alst{s}in-líf. \hld\ Þó hé im \alst{s}elvo gi·wêt &
aftar þem \alst{d}ôpislja, \hld\ \alst{d}rohtin þe gódo, &
an êna \alst{w}óstunnja, \hld\ \alst{w}aldandes sunu; &
was im þár an þero \alst{ê}n-\alst{ô}di \hld\ \alst{e}rlo drohtin &
\alst{l}ange hwíla; \hld\ ne habda \alst{l}iudjo þan mêr, &
\alst{s}ęggjo te gi·\alst{s}ïðun, \hld\ al só hé im \alst{s}elvo gi·kôs: &
welda is þár látan \alst{k}oston \hld\ \alst{k}raftiga wihti, &
\alst{s}elvon \alst{S}atanasan, \hld\ þe gio an \alst{s}undja spęnit, &
\alst{m}an an \alst{m}ên-werk: \hld\ hé konsta is \alst{m}ód-sevon, &
\alst{w}rêðan \alst{w}illjon, \hld\ hwó hé þesa \alst{w}er-old êrist, &
an þem \alst{a}n-ginnja \hld\ \alst{i}rmin-þioda &
bi·\alst{s}wêk mit \alst{s}undjun, \hld\ þó hé þiu \alst{s}in-híun twê, &
\alst{Á}daman ęndi \alst{É}wan, \hld\ þurh \alst{u}n-treuwa &
for·\alst{l}êdda mid \alst{l}uginun, \hld\ þat \alst{l}iudo barn &
aftar iro \alst{h}in-fęrdi \hld\ \alst{h}ęllja sóhtun, &
\alst{g}umono \alst{g}êstos. \hld\ Þó welda þat \alst{g}od mahtig, &
\alst{w}aldand \alst{w}ęndjan \hld\ ęndi welda þesum \alst{w}erode for·geven &
\alst{h}ôh \alst{h}imil-ríki: \hld\ be·þiu hé herod \alst{h}êlagna bodon, &
is \alst{s}unu \alst{s}ęnda. \hld\ Þat was \alst{S}atanase &
tulgo \alst{h}arm an is \alst{h}ugi: \hld\ afonsta \alst{h}evan-ríkjes &
\alst{m}anno kunnje: \hld\ welda þó \alst{m}ahtigna &
mid þem \alst{s}elvon \alst{s}akun \hld\ \alst{s}unu drohtines, &
þem hé \alst{Á}daman \hld\ an \alst{ê}r-dagun &
\alst{d}arnungo bi·\alst{d}róg, \hld\ þat hé warð is \alst{d}rohtine lêð, &
bi·\alst{s}wêk ina mid \alst{s}undjun \hld\ —só welda hé þó \alst{s}elvan dón &
\alst{h}êlandjan Krist. \hld\ Þan habda hé is \alst{h}ugi fasto &
wið þana \alst{w}am-skaðon, \hld\ \alst{w}aldandes barn, &
\alst{h}erte só gi·\alst{h}ęrdid: \hld\ welda \alst{h}evan-ríki &
\alst{l}iudjun gi·\alst{l}êstjan. \hld\ Was im þes \alst{l}andes ward &
an \alst{f}astunnja \hld\ \alst{f}ior-tig nahto, &
\alst{m}anno drohtin, \hld\ só hé þár \alst{m}ates ni ant·bêt; &
þan langa ni gi·\alst{d}orstun \hld\ im \alst{d}ęrnja wihti, &
\alst{n}íð-hugdig fíund, \hld\ \alst{n}áhor gangan, &
\alst{g}rótjan ina \alst{g}ęgin-warðan: \hld\ wánde þat hé \alst{g}od ên-fald, &
for·útar \alst{m}an-kunnjes wiht \hld\ \alst{m}ahtig wári, &
\alst{h}êleg \alst{h}imiles ward. \hld\ Só hé ina þó ge·\alst{h}ungrjan lét, &
þat ina bi·gan bi þero \alst{m}ęnnisko \hld\ \alst{m}óses lustjan &
aftar þem \alst{f}iuwar-tig dagun, \hld\ þe \alst{f}íund náhor géng, &
\alst{m}irki \alst{m}ên-skaðo: \hld\ wánda þat hé \alst{m}an ên-fald &
\alst{w}ári \alst{w}issungo, \hld\ sprak im þó mid is \alst{w}ordun tó, &
\alst{g}rótta ina þe \alst{g}êr-fíund: \hld\ „ef þú sís \alst{g}odes sunu“, kwað hé, &
„be·hwí ni hêtis þú þan \alst{w}erðan, \hld\ ef þú gi·\alst{w}ald haves, &
allaro \alst{b}arno \alst{b}ętst, \hld\ \alst{b}rôd af þesun stênun? &
Ge·\alst{h}êli þínna \alst{h}ungạr!“ \hld\ Þó sprak eft þe \alst{h}êlago Krist: &
„ni mugun \alst{ę}ldi-barn“, \hld[kwað hé,] „\alst{ê}n-faldes brôdes, &
\alst{l}iudi \alst{l}ibbjen, \hld\ ak sie skulun þurh \alst{l}êra godes &
\alst{w}esan an þesero \alst{w}er-oldi \hld\ ęndi skulun þiu \alst{w}erk frummjen, &
þea þár werðad a·\alst{h}lúdid \hld\ fon þero \alst{h}êlogun tungun, &
fon þem \alst{g}alme \alst{g}odes: \hld\ þat is \alst{g}umono líf &
\alst{l}iudjo só hwi-likon, \hld\ só þat \alst{l}êstjan wili, &
þat fon \alst{w}aldandes \hld\ \alst{w}orde ge·biudid.“ &
Þó bi·gan eft \alst{n}iuson \hld\ ęndi \alst{n}áhor géng &
\alst{u}n-hiuri fíund \hld\ \alst{ȯ}ðru sïðu, &
\alst{f}andoda is \alst{f}rôhan. \hld\ Þat \alst{f}riðu-barn þolode &
\alst{w}rêðes \alst{w}illjon \hld\ ęndi im gi·\alst{w}ald for·gaf, &
þat hé umbi is \alst{k}raft mikil \hld\ \alst{k}oston mósti, &
\alst{l}ét ina þó \alst{l}êdjan \hld\ þana \alst{l}iud-skaðon, &
þat hé ina an \alst{J}erusalem \hld\ te þem \alst{g}odes wíha, &
\alst{a}lles \alst{o}van-wardan, \hld\ \alst{u}p gi·sętta &
an allaro \alst{h}úso \alst{h}ôhost, \hld\ ęndi \alst{h}osk-wordun sprak, &
þe \alst{g}ramo þurh \alst{g}elp mikil: \hld\ „ef þú sís \alst{g}odes sunu“, kwað hé, &
„\alst{sk}ríd þi te erðu hinan. \hld\ Ge·\alst{sk}rivan was it giu lango, &
an \alst{b}ókun ge·writen, \hld\ hwó gi·\alst{b}oden havad &
is \alst{ę}ngilun \hld\ \alst{a}lo-mahtig fader, &
þat sie þi at \alst{w}ege ge·hwem \hld\ \alst{w}ardos sinðun, &
\alst{h}aldad þi undar iro \alst{h}andun. \hld\ Hwat þú \alst{h}wargin ni þarft &
mid þínun \alst{f}ótun \hld\ an \alst{f}elis be·spurnan, &
an \alst{h}ardan stên.“ \hld\ Þó sprak eft þe \alst{h}êlago Krist, &
allaro \alst{b}arno \alst{b}ętst: \hld\ „só is ôk an \alst{b}ókun ge·skrivan“, kwað hé, &
„þat þú te \alst{h}ardo ni skalt \hld\ \alst{h}êrran þínes, &
\alst{f}andon þínes \alst{f}rôhan: \hld\ þat nis þí allaro \alst{f}rumono neg·ên.“ &
Lét ina þó an þana \alst{þ}riddjan sïð \hld\ þana \alst{þ}iod-skaðon &
gi·\alst{b}rengen uppan ênan \alst{b}erg þen hôhon: \hld\ þár ina þe \alst{b}alo-wíso &
lét \alst{a}l \alst{o}var-sehan \hld\ \alst{i}rmin-þiode, &
\alst{w}onod-saman \alst{w}elon \hld\ ęndi \alst{w}er-old-ríki &
ęndi all su·lik \alst{ô}des, \hld\ só þius \alst{e}rða bi·havad &
\alst{f}agọroro \alst{f}rumono, \hld\ ęndi sprak im þó þe \alst{f}íund an·gęgin, &
kwað þat hé im þat al só \alst{g}ód-lík \hld\ for·\alst{g}even weldi, &
\alst{h}ôha \alst{h}ęri-dómos, \hld\ „ef þú wilt \alst{h}nígan te mí, &
\alst{f}allan te mínun \alst{f}ótun \hld\ ęndi mí for \alst{f}rôhan havas, &
\alst{b}edos te mínun \alst{b}arma. \hld\ Þan látu ik þí \alst{b}rúkan wel &
\alst{a}lles þes \alst{ô}d-welon, \hld\ þes ik þí hębbju gi·\alst{ô}git hír.“ &
Þó ni welda þes \alst{l}êðan word \hld\ \alst{l}ęngeron hwíle &
\alst{h}ôrjan þe \alst{h}êlago Krist, \hld\ ak hé ina fon is \alst{h}uldi for·drêf, &
\alst{S}atanasan for·\alst{s}wêp, \hld\ ęndi \alst{s}án aftar sprak &
allaro \alst{b}arno \alst{b}ętst, \hld\ kwað þat man \alst{b}edon skoldi &
\alst{u}p te þem \alst{a}lo-mahtigon gode \hld\ ęndi im \alst{ê}num þionon &
swíðo \alst{þ}io-liko \hld\ \alst{þ}egnos managa, &
\alst{h}ęliðos aftar is \alst{h}uldi: \hld\ „þár ist þiu \alst{h}elpa ge·lang &
\alst{m}anno ge·hwi-likun.“ \hld\ Þó gi·wêt im þe \alst{m}ên-skaðo, &
\alst{s}wíðo \alst{s}êrag-mód \hld\ \alst{S}atanas þanan, &
\alst{f}íund undar \alst{f}ern-dalu. \hld\ Warð þár \alst{f}olk mikil &
fon þem \alst{a}lo-waldan \hld\ \alst{o}vana te Kriste &
\alst{g}odes ęngilo kumen, \hld\ þie im sïðor \alst{j}ungar-dóm, &
skoldun \alst{a}mbaht-skępi \hld\ \alst{a}ftar lêstjen, &
\alst{þ}ionon \alst{þ}io-líko: \hld\ só skal man \alst{þ}iod-gode, &
\alst{h}êrron aftar \alst{h}uldi, \hld\ \alst{h}evan-kuninge.\eva

\bvb TODO.\evb\evg

\bvg\bva[14][1121]%
Was im an þem \alst{s}in-węldi \hld\ \alst{s}álig barn godes &
\alst{l}ange hwíle, \hld\ unt-þat im þó \alst{l}iovora warð, &
þat hé is \alst{k}raft mikil \hld\ \alst{k}u̇ðjen wolda &
\alst{w}eroda te \alst{w}illjon. \hld\ Þó for·lét hé \alst{w}aldes hleo, &%NOTE: Not hlêo.
\alst{ê}n-ôdjes \alst{a}rd \hld\ ęndi sóhte im eft \alst{e}rlo ge·mang, &
\alst{m}ári \alst{m}ęgin-þiode \hld\ ęndi \alst{m}anno drôm, &
géng im þó bi \alst{J}ordanes staðe: \hld\ þár ina \alst{J}ohannes ant·fand, &
þat \alst{f}riðu-barn godes, \hld\ \alst{f}rôhan sínan, &
\alst{h}êlagana \alst{h}evan-kuning, \hld\ ęndi þem \alst{h}ęliðun sagda, &
\alst{J}ohannes is \alst{j}ungurun, \hld\ þó hé ina \alst{g}angan ge·sah: &
„þit is þat \alst{l}amb godes, \hld\ þat þár \alst{l}ôsjan skal &
af þesaro \alst{w}ídon \alst{w}er-old \hld\ \alst{w}rêða sundja, &
\alst{m}an-kunnjas \alst{m}ên, \hld\ \alst{m}ári drohtin, &
\alst{k}uningo \alst{k}raftigost.“ \hld\ \alst{K}rist im forð gi·wêt &
an \alst{G}alileo land, \hld\ \alst{g}odes êgan barn, &
\alst{f}ór im te þem \alst{f}riundun, \hld\ þár hé a·\alst{f}ódit was, &
\alst{t}ír-líko a·\alst{t}ogan, \hld\ ęndi \alst{t}alda mid wordun &
\alst{K}rist undar is \alst{k}unnje, \hld\ \alst{k}uningo ríkjost, &
hwó sie \alst{s}koldin iro \alst{s}elvoro \hld\ \alst{s}undja bótjan, &
hét þat sie im iro \alst{h}arm-werk manag \hld\ \alst{h}reuwan létin, &
\alst{f}eldin iro \alst{f}irin-dádi: \hld\ „nú is it all ge·\alst{f}ullot só, &
só hír \alst{a}lde man \hld\ \alst{ê}r hwanna sprákun, &
ge·\alst{h}étun eu te \alst{h}elpu \hld\ \alst{h}evan-ríki: &
nú is it giu gi·\alst{n}áhid þurh þes \alst{n}ęrjandan kraft: \hld\ þes mótun gí \alst{n}eotan forð, &
só hwe só \alst{g}erno wili \hld\ \alst{g}ode þeonogjan, &
\alst{w}irkjan aftar is \alst{w}illjon.“ \hld\ Þó warð þes \alst{w}erodes filu, &
þero \alst{l}iudjo an \alst{l}ustun: \hld\ wurðun im þea \alst{l}êra Kristes, &
só \alst{s}wótja þem gi·\alst{s}ïðja. \hld\ Hé bi·gan im \alst{s}amnon þó &
\alst{g}umono te \alst{j}ungoron, \hld\ \alst{g}ódoro manno, &
\alst{w}ord-spáha \alst{w}eros. \hld\ Géng im þó bi ênes \alst{w}atares staðe, &
þat þár habda \alst{J}ordan \hld\ a·nevan \alst{G}alileo land &
ênna \alst{s}ê ge·warhtan. \hld\ Þár hé \alst{s}ittjan fand &
\alst{A}ndreas ęndi Petrus \hld\ bi þem \alst{a}ha-strôme, &
\alst{b}êðja þea ge·\alst{b}róðar, \hld\ þár sie an \alst{b}rêd watar &
swíðo \alst{n}iud-líko \hld\ \alst{n}ętti þenidun, &
\alst{f}iskodun im an þem \alst{f}lóde. \hld\ Þár sie þat \alst{f}riðu-barn godes &
bi þes \alst{s}êes staðe \hld\ \alst{s}elvo grótta, &
hét þat sie im \alst{f}olgodin, \hld\ kwað þat hé im só \alst{f}ilu woldi &
\alst{g}odes ríkjas for·\alst{g}even; \hld\ „al só git hír an \alst{J}ordanes strôme &
\alst{f}iskos \alst{f}ȧhat, \hld\ só skulun git noh \alst{f}iriho barn &
\alst{h}alon te inkun \alst{h}andun, \hld\ þat sie an \alst{h}evan-ríki &
þurh inka \alst{l}êra \hld\ \alst{l}íðan mótin, &
\alst{f}aran \alst{f}olk manag.“ \hld\ Þó warð \alst{f}rô-mód hugi &
\alst{b}êðjun þem gi·\alst{b}róðrun: \hld\ ant·kęndun þat \alst{b}arn godes, &
\alst{l}iovan hêrron: \hld\ for·\alst{l}étun al saman &
\alst{A}ndreas ęndi Petrus, \hld\ só hwat só sie bi þeru \alst{a}hu habdun, &
ge·\alst{w}unstes bi þem \alst{w}atare: \hld\ was im \alst{w}illjo mikil, &
þat sie mid þem \alst{g}odes barne \hld\ \alst{g}angan móstin, &
\alst{s}amad an is gi·\alst{s}ïðja, \hld\ skoldun \alst{s}álig-líko &
\alst{l}ôn ant·fȧhan: \hld\ só dót \alst{l}iudjo so hwi-lik, &
só þes \alst{h}êrran wili \hld\ \alst{h}uldi gi·þionon, &
ge·\alst{w}irkjan is \alst{w}illjon. \hld\ Þó sie bi þes \alst{w}atares staðe &
\alst{f}urðor kwámun, \hld\ þó fundun sie þár ênna \alst{f}ródan man &
\alst{s}ittjan bi þem \alst{s}êwa \hld\ ęndi is \alst{s}uni twêne, &
\alst{J}akobus ęndi \alst{J}ohannes: \hld\ wárun im \alst{j}unga man. &
\alst{S}átun im þá ge·\alst{s}un-fader \hld\ an ênumu \alst{s}ande uppen, &
\alst{b}rugdun ęndi \alst{b}óttun \hld\ \alst{b}êðjum handun &
þiu \alst{n}ętti \alst{n}iud-líko, \hld\ þea sie habdun \alst{n}ahtes êr &
for·\alst{s}liten an þem \alst{s}êwa. \hld\ Þár sprak im \alst{s}elvo tó &
\alst{s}álig barn godes, \hld\ hét þat sie an þana \alst{s}ïð mid im, &
\alst{J}akobus ęndi \alst{J}ohannes, \hld\ \alst{g}éngin bêðje, &
\alst{k}ind-junge man. \hld\ Þó wárun im \alst{K}ristes word &
só \alst{w}irðig an þesaro \alst{w}er-oldi, \hld\ þat sie bi þes \alst{w}atares staðe &
iro \alst{a}ldan fader \hld\ \alst{ê}nna for·létun, &
\alst{f}ródan bi þem \alst{f}lóde, \hld\ ęndi al þat sie þár \alst{f}ehas êhtun, &
\alst{n}ęttju ęndi \alst{n}ęglit-skipu, \hld\ ge·kurun im þana \alst{n}ęrjandan Krist, &
\alst{h}êlagna te \alst{h}êrron, \hld\ was im is \alst{h}elpono þarf &
te gi·\alst{þ}iononne: \hld\ só is allaro \alst{þ}egno ge·hwem, &
\alst{w}ero an þesero \alst{w}er-oldi. \hld\ Þó gi·wêt im þe \alst{w}aldandes sunu &
mid þem \alst{f}iuwarjun \alst{f}orð, \hld\ ęndi im þó þana \alst{f}ïfton gi·kôs &
\alst{K}rist an ênero \alst{k}ôp-stędi, \hld\ \alst{k}uninges jungoron, &
\alst{m}ód-spáhana man: \hld\ \alst{M}attheus was hé hêtan, &
was im \alst{a}mbahtjo \hld\ \alst{ę}ðilero manno, &
skolda þár te is \alst{h}êrron \hld\ \alst{h}andun ant·fȧhan &
\alst{t}ins ęndi \alst{t}olna; \hld\ \alst{t}reuwa habda hé góda, &
\alst{a}ðal-and·bári: \hld\ for·lét \alst{a}l saman &
\alst{g}old ęndi silụvar \hld\ ęndi \alst{g}eva managa, &
\alst{d}iurje mêðmos, \hld\ ęndi warð im u̇ses \alst{d}rohtines man; &
\alst{k}ôs im þe \alst{k}uninges þegn \hld\ \alst{K}rist te hêrran, &
\alst{m}ilderan \alst{m}êðọm-gevon, \hld\ þan êr is \alst{m}an-drohtin &
\alst{w}ári an þesero \alst{w}er-oldi: \hld\ féng im \alst{w}óðera þing, &
\alst{l}ang-samoron rád. \hld\ Þó warð it allun þem \alst{l}iudjun ku̇ð, &
fon allaro \alst{b}urgo gi·hwem, \hld\ hwó þat \alst{b}arn godes &
\alst{s}amnode ge·\alst{s}ïðos \hld\ ęndi \alst{s}elvo ge·sprak &
só manag \alst{w}ís-lík \alst{w}ord \hld\ ęndi \alst{w}áres só filu, &
\alst{t}orhtes gi·\alst{t}ôgde \hld\ ęndi \alst{t}êkạn manag &
ge·\alst{w}arhte an þesero \alst{w}er-oldi. \hld\ Was þat an is \alst{w}ordun skín &
iak an is \alst{d}ádjun só same, \hld\ þat hé \alst{d}rohtin was, &
\alst{h}imilisk \alst{h}êrro \hld\ ęndi te \alst{h}elpu kwam &
an þesan \alst{m}iddil-gard \hld\ \alst{m}anno barnun, &
\alst{l}iudjun te þesun \alst{l}iohta. \hld\ Oft ge·deda hé þat an þem \alst{l}ande skín, &
þan hé þár \alst{t}orht-líko \hld\ só manag \alst{t}êkạn gi·warhte, &
þár hé \alst{h}êlde mid is \alst{h}andun \hld\ \alst{h}alte ęndi blinde, &
\alst{l}ôsde af þeru \alst{l}éf-hêdi \hld\ \alst{l}iudi manage, &
af \alst{s}u·likun \alst{s}uhtjun, \hld\ só þan allaro \alst{s}wároston &
an \alst{f}iriho barn \hld\ \alst{f}íund bi·wurpun, &
tulgo \alst{l}ang-sam \alst{l}egar.\eva

\bvb TODO.\evb\evg

\bvg\bva[15][1217]%
\hspace*{100pt} Þó fórun þár þie \alst{l}iudi tó &%NOTE: In cæsura.
allaro \alst{d}ago ge·hwi-likes, \hld\ þár u̇sa \alst{d}rohtin was &
\alst{s}elvo undar þem gi·\alst{s}ïðje, \hld\ unt-þat þár ge·\alst{s}amnod warð &
\alst{m}ęgin-folk \alst{m}ikil \hld\ \alst{m}anagero þiodo, &
þoh sie þár alle be ge·\alst{l}íkumu \hld\ ge·\alst{l}ôvon ni kwámin. &
\alst{w}eros þurh ênan \alst{w}illjon: \hld\ sume sóhtun sie þat \alst{w}aldandes barn, &
\alst{a}rmoro manno filu \hld\ —was im \alst{á}tes þarf—, &
þat sie im þár at þeru \alst{m}ęnigi \hld\ \alst{m}ates ęndi drankes, &
\alst{þ}igidin at þeru \alst{þ}iodu; \hld\ hwand þár was manag \alst{þ}egạn só gód, &
þie ira \alst{a}lamosnje \hld\ \alst{a}rmun mannun &
\alst{g}erno \alst{g}ávun. \hld\ Sume wárun sie im eft \alst{J}udeono kunnjes, &
\alst{f}êgni \alst{f}olk-skępi: \hld\ wárun þár ge·\alst{f}arana te þiu, &
þat sie u̇ses \alst{d}rohtines \hld\ \alst{d}ádjo ęndi wordo &
\alst{f}áron woldun, \hld\ habdun im \alst{f}êgnjen hugi, &
\alst{w}rêðen \alst{w}illjon: \hld\ woldun \alst{w}aldand Krist &
a·\alst{l}êdjen þem \alst{l}iudjun, \hld\ þat sie is \alst{l}êron ni hôrdin, &
ne \alst{w}ęndin aftar is \alst{w}illjon. \hld\ Suma wárun sie im eft só \alst{w}íse man, &
wárun im \alst{g}lawe \alst{g}umon \hld\ ęndi \alst{g}ode werðe, &
a·\alst{l}esane undar þem \alst{l}iudjun, \hld\ kwámun im þarod be þem \alst{l}êron Kristes, &
þat sie is \alst{h}êlag word \hld\ \alst{h}ôrjen móstin, &
\alst{l}ínon ęndi \alst{l}êstjen: \hld\ habdun mid iro ge·\alst{l}ôvon te im &
\alst{f}asto ge·\alst{f}angen, \hld\ habdun im \alst{f}erhten hugi, &
wurðun is \alst{þ}egnos te þiu, \hld\ þat hé sie an \alst{þ}iod-welon &
\alst{a}ftar iro \alst{ê}n-dagon \hld\ \alst{u}p ge·brȧhti, &
an \alst{g}odes ríki. \hld\ Hé só \alst{g}erno ant·féng &
\alst{m}an-kunnjes \alst{m}anag \hld\ ęndi \alst{m}und-burd gi·hét &
te \alst{l}angaru hwílu, \hld\ ęndi mahta só gi·\alst{l}êstjen wel. &
Þó warð þár \alst{m}ęgin só \alst{m}ikil \hld\ umbi þana \alst{m}árjon Krist, &
\alst{l}iudjo ge·samnod: \hld\ þó gi·sah hé fon allun \alst{l}andun kuman, &
fon allun \alst{w}ídun \alst{w}egun \hld\ \alst{w}erod te·samne &
\alst{l}ungro \alst{l}iudjo: \hld\ is \alst{l}of was só wído &
\alst{m}anagun ge·\alst{m}árid. \hld\ Þó gi·wêt im \alst{m}ahtig self &
an ênna \alst{b}erg uppan, \hld\ \alst{b}arno ríkjost, &
\alst{s}undạr ge·\alst{s}ittjen, \hld\ ęndi im \alst{s}elvo ge·kôs &
\alst{t}we-livi ge·\alst{t}alda, \hld\ \alst{t}reu-hafta man, &
\alst{g}ódoro \alst{g}umono, \hld\ þea hé im te \alst{j}ungoron forð &
allaro \alst{d}ago ge·hwi-likes, \hld\ \alst{d}rohtin welda &
an is ge·\alst{s}ïð-skępja \hld\ \alst{s}imblon hębbjan. &
\alst{N}ęmnida sie þó bi \alst{n}aman \hld\ ęndi hét sie im þó \alst{n}áhor gangan, &
\alst{A}ndreas ęndi Petrus \hld\ \alst{ê}rist sána, &
ge·\alst{b}róðar twêne, \hld\ ęndi \alst{b}êðje mid im, &
\alst{J}akobus ęndi \alst{J}ohannes: \hld\ sie wárun \alst{g}ode werðe; &
\alst{m}ildi was hé im an is \alst{m}óde; \hld\ sie wárun ênes \alst{m}annes suni &
\alst{b}êðje bi ge·\alst{b}urdjun; \hld\ sie kôs þat \alst{b}arn godes &
\alst{g}óde te \alst{j}ungoron \hld\ ęndi \alst{g}umono filu, &
\alst{m}árjero \alst{m}anno: \hld\ \alst{M}attheus ęndi Þomas, &
\alst{J}udasas twêna \hld\ ęndi \alst{J}akob ȯðran, &
is \alst{s}elves \alst{s}wiri: \hld\ sie wárun fon gi·\alst{s}ustruonjon twêm &
\alst{k}nósles \alst{k}umana, \hld\ \alst{K}rist ęndi Jakob, &
\alst{g}óde \alst{g}adulingos. \hld\ Þó habda þero \alst{g}umono þár &
þe \alst{n}ęrjendo Krist \hld\ \alst{n}iguni ge·talde, &%TODO: check niguni
\alst{t}reu-hafte man: \hld\ þó hét hé ôk þana \alst{t}e·handon gangan &
\alst{s}elvo mid þem gi·\alst{s}ïðun: \hld\ \alst{S}ímon was hé hêtan; &
hét ôk \alst{B}artholomeus \hld\ an þana \alst{b}erg uppan &
\alst{f}aran fan þem \alst{f}olke áðrum \hld\ ęndi \alst{Ph}ilippus mid im, &
\alst{t}reu-hafte man. \hld\ Þó géngun sie \alst{t}we-livi samad, &
\alst{r}inkos te þeru \alst{r}únu, \hld\ þár þe \alst{r}ádand sat, &
\alst{m}anagoro \alst{m}und-boro, \hld\ þe allumu \alst{m}an-kunnje &
wið \alst{h}ęllje ge·þwing \hld\ \alst{h}elpan welde, &
\alst{f}ormon wið þem \alst{f}erne, \hld\ só hwem só \alst{f}rummjen wili &
só \alst{l}iov-líka \alst{l}êra, \hld\ só hé þem \alst{l}iudjun þár &
þurh is gi·\alst{w}it mikil \hld\ \alst{w}ísjan hogda.\eva

\bvb TODO.\evb\evg

\bvg\bva[16][1279]%
Þó umbi þana \alst{n}ęrjandon Krist \hld\ \alst{n}áhor géngun &%NOTE ms. -- Þó] V 1 (27r)
\alst{s}u-lika ge·\alst{s}ïðos, \hld\ só hé im \alst{s}elvo ge·kôs, &
\alst{w}aldand undar þem \alst{w}erode. \hld\ Stódun \alst{w}ísa man, &
\alst{g}umon umbi þana \alst{g}odes sunu \hld\ \alst{g}erno swíðo, &
\alst{w}eros an \alst{w}illjon: \hld\ was im þero \alst{w}ordo niud, &
\alst{þ}ȧhtun ęndi \alst{þ}agodun, \hld\ hwat im þero \alst{þ}iodo drohtin, &
\alst{w}eldi \alst{w}aldand self \hld\ \alst{w}ordun ku̇ðjan &
þesum \alst{l}iudjun te \alst{l}iove. \hld\ Þan sat im þe \alst{l}andes hirdi &
\alst{g}ęgin-ward for þem \alst{g}umun, \hld\ \alst{g}odes êgan barn: &
welda mid is \alst{sp}rákun \hld\ \alst{sp}áh-word manag &
\alst{l}êrjan þea \alst{l}iudi, \hld\ hwó sie \alst{l}of gode &
an þesum \alst{w}er-old-ríkja \hld\ \alst{w}irkjan skoldin. &
\alst{S}at im þó ęndi \alst{s}wígoda \hld\ ęndi \alst{s}ah sie an lango, &
was im \alst{h}old an is \alst{h}ugi \hld\ \alst{h}êlag drohtin, &
\alst{m}ildi an is \alst{m}óde, \hld\ ęndi þó is \alst{m}und ant·lôk, &
\alst{w}ísde mid \alst{w}ordun \hld\ \alst{w}aldandes sunu &
\alst{m}anag \alst{m}ár-lík þing \hld\ ęndi þem \alst{m}annum sagde &
\alst{sp}áhun wordun, \hld\ þem þe hé te þeru \alst{sp}ráku þarod, &
\alst{K}rist alo-waldo, \hld\ ge·\alst{k}oran habda, &
hwi-like wárin \alst{a}llaro \hld\ \alst{i}rmin-manno &
\alst{g}ode werðoston \hld\ \alst{g}umono kunnjes; &
\alst{s}agde im þó te \alst{s}ȯðan, \hld\ kwað þat þie \alst{s}áliga wárin, &
\alst{m}an an þesoro \alst{m}iddil-gardun, \hld\ þie hér an iro \alst{m}óde wárin &
\alst{a}rme þurh \alst{ô}d-módi: \hld\ „þem is þat \alst{ê}wana ríki, &
swíðo \alst{h}êlag-lík \hld\ an \alst{h}evan-wange &
\alst{s}in-líf far·geven.“ \hld\ Kwað þat ôk \alst{s}álige wárin &
\alst{m}áð-mundje \alst{m}an: \hld\ „þie mótun þie \alst{m}árjon erðe, &
of·\alst{s}ittjen þat \alst{s}elve ríki.“ \hld\ Kwað þat ôk \alst{s}álige wárin, &
þie hír \alst{w}iopin iro \alst{w}ammun dádi; \hld\ „þie mótun eft \alst{w}illjon ge·bídan, &
\alst{f}rófre an iro \alst{f}râhon ríkja. \hld\ Sálige sind ôk, þe sie hír \alst{f}rumono gi·lustid, &
\alst{r}inkos, þat sie \alst{r}ehto a·dómjen. \hld\ Þes mótun sie werðan an þem \alst{r}íkja drohtines &
gi·\alst{f}ullit þurh iro \alst{f}erhton dádi: \hld\ su-líkoro mótun sie \alst{f}rumono bi·knégan &
þie \alst{r}inkos, þie hír \alst{r}ehto a·dómjad, \hld\ ne willjad an \alst{r}únun be·swíkan &
\alst{m}an, þár sie at \alst{m}ahle sittjad. \hld\ Sálige sind ôk þem hír \alst{m}ildi wirðit &
\alst{h}ugi an \alst{h}ęliðo briostun: \hld\ þem wirðit þe \alst{h}êlego drohtin, &
\alst{m}ildi \alst{m}ahtig selvo. \hld\ Sálige sind ôk undar þesaro \alst{m}anagon þiodu, &
þie hębbjad iro \alst{h}erta gi·\alst{h}rênod: \hld\ þie mótun þane \alst{h}evanes waldand &
\alst{s}ehan an \alst{s}ínum ríkja.“ \hld\ Kwað þat ôk \alst{s}álige wárin, &
„þie þe \alst{f}riðu-samo undar þesumu \alst{f}olke libbjod \hld\ ęndi ni willjad êniga \alst{f}ehta ge·wirken, &
\alst{s}aka mid iro \alst{s}elvoro dádjun: \hld\ þie mótun wesan \alst{s}uni drohtines ge·nęmnide, &
hwande hé im wil ge·\alst{n}ádig werðen; \hld\ þes mótun sie \alst{n}iotan lango &
\alst{s}elvon þes \alst{s}ínes ríkjes.“ \hld\ Kwað þat ôk \alst{s}álige wárin &
þie \alst{r}inkos, þe \alst{r}ehto weldin, \hld\ „ęndi þurh þat þolod \alst{r}íkjoro manno &
\alst{h}ęti ęndi \alst{h}arm-kwidi: \hld\ þem is ôk an \alst{h}imile eft &
\alst{g}odes wang for·\alst{g}even \hld\ ęndi \alst{g}êst-lík \edtext{líf}{\Afootnote{end \textbf{V}/27r; text continues on 32v.}} &
\alst{a}ftar te \alst{ê}wan-dage, \hld\ só is io \alst{ę}ndi ni kumit, &
\alst{w}elan \alst{w}un-sames.“ \hld\ Só habde þó \alst{w}aldand Krist &
for þem \alst{e}rlom þár \hld\ \alst{a}hto ge·talda &
\alst{s}álda ge·\alst{s}agda; \hld\ mid þem skal \alst{s}imbla gi·hwe &
\alst{h}imil-ríki ge·\alst{h}alon, \hld\ ef hé it \alst{h}ębbjan wili, &
eþþo hé skal te \alst{ê}wan-daga \hld\ \alst{a}ftar þarvon &
\alst{w}elon ęndi \alst{w}illjon, \hld\ sïðor hé þese \alst{w}er-old a·givid, &
\alst{e}rð-lívi-gi·skapu, \hld\ ęndi sókit im \alst{ȯ}ðar lioht &
só \alst{l}iof só \alst{l}êð, \hld\ só hé mid þesun \alst{l}iudjun hér &
gi·\alst{w}erkod an þesoro \alst{w}er-oldi, \hld\ al só it þár þó mid is \alst{w}ordun sagde &
\alst{K}rist alo-waldo, \hld\ \alst{k}uningo ríkjost &
\alst{g}odes êgan barn \hld\ \alst{j}ungorun sínun: &
„Ge werðat ôk só \alst{s}álige“, \hld[kwað hé,] „þes iu \alst{s}aka biodat &
\alst{l}iudi aftar þeson \alst{l}ande \hld\ ęndi \alst{l}êð sprekat, &
\alst{h}ębbjad iu te \alst{h}oska \hld\ ęndi \alst{h}armes filu &
ge·\alst{w}irkjad an þesoro \alst{w}er-oldi \hld\ ęndi \alst{w}íti ge·frummjad, &
\alst{f}ęlgjad iu \alst{f}irin-spráka \hld\ ęndi \alst{f}íund-skępi, &
\alst{l}âgnjad iuwa \alst{l}êra, \hld\ dót iu \alst{l}êðes filu, &
\alst{h}armes þurh iuwan \alst{h}êrron. \hld\ Þes látad gí iuwan \alst{h}ugi simbla, &
\alst{l}íf an \alst{l}ustun, \hld\ hwand iu þat \alst{l}ôn stęndit &
an \alst{g}odes ríkja \alst{g}aru, \hld\ \alst{g}ódo ge·hwi-likes, &
\alst{m}ikil ęndi \alst{m}anag-fald: \hld\ þat is iu te \alst{m}édu far·gevan, &
hwand gí hér \alst{ê}r bi·foran \hld\ \alst{a}rvid þolodun, &
\alst{w}íti an þesoro \alst{w}er-oldi. \hld\ \alst{W}irs is þem ȯðrum, &
\alst{g}iviðig \alst{g}rimmora þing, \hld\ þem þe hér \alst{g}ód êgun, &
\alst{w}ídan \alst{w}orold-\alst{w}elon: \hld\ þie for·slítat iro \alst{w}unnja hér; &
ge·\alst{n}iudot sie ge·\alst{n}óges, \hld\ skulun eft \alst{n}arowaro þing &
aftar iro \alst{h}in-fęrdi \hld\ \alst{h}ęliðos þolojan. &
Þan \alst{w}ópjan þár \alst{w}an-skęfti, \hld\ þie hér êr an \alst{w}unnjon sín, &
\alst{l}ibbjad an allon \alst{l}ustun, \hld\ ne willjad þes far·\alst{l}átan wiht, &
\alst{m}êni-gi·þȧhtjo, \hld\ þes sie an iro \alst{m}ód spęnit, &
\alst{l}êðoro gi·\alst{l}êstjo. \hld\ Þan im þat \alst{l}ôn kumid, &
\alst{u}vil \alst{a}rvêd-sam, \hld\ þan sie is þane \alst{ę}ndi skulun &
\alst{s}orgondi ge·\alst{s}ehan. \hld\ Þan wirðid im \alst{s}êr hugi, &
þes \edtext{sie}{\Afootnote{cuts off \textbf{V}}} þesero \alst{w}er-oldes só filu \hld\ \alst{w}illjan ful-géngun, &
\alst{m}an an iro \alst{m}ód-sevon. \hld\ Nú skulun gí im þat \alst{m}ên lahan, &
\alst{w}ęrjan mid \alst{w}ordun, \hld\ al só ik giu nú ge·\alst{w}ísjan mag, &
\alst{s}ęggjan \alst{s}ȯð-líko, \hld\ ge·\alst{s}ïðos míne, &
\alst{w}árun \alst{w}ordun, \hld\ þat gí þesoro \alst{w}er-oldes nú forð &
skulun \alst{s}alt wesan, \hld\ \alst{s}undigero manno, &
\alst{b}ótjan iro \alst{b}alu-dádi, \hld\ þat sie an \alst{b}ętara þing, &
\alst{f}olk far·\alst{f}ȧhan \hld\ ęndi for·látan \alst{f}íundes gi·werk, &
\alst{d}iuvales ge·\alst{d}ádi, \hld\ ęndi sókjan iro \alst{d}rohtines ríki. &
Só skulun gí mid iuwon \alst{l}êrun \hld\ \alst{l}iud-folk manag &
\alst{w}ęndjan aftar mínon \alst{w}illjon. \hld\ Ef iuwar þan a·\alst{w}irðid hwi-lik, &
far·\alst{l}átid þea \alst{l}êra, \hld\ þea hé \alst{l}êstjan skal, &
þan is im só þem \alst{s}alte, \hld\ þe man bi \alst{s}êes staðe &
\alst{w}ído te·\alst{w}irpit: \hld\ þan it te \alst{w}ihti ni dôg, &
ak it \alst{f}iriho barn \hld\ \alst{f}ótun spurnat, &
\alst{g}umon an \alst{g}reote. \hld\ Só wirðid þem, þe þat \alst{g}odes word skal &
\alst{m}annum \alst{m}árjan: \hld\ ef hé im þan látid is \alst{m}ód twehon, &
þat hí ne willja mid \alst{h}luttro \alst{h}ugi \hld\ te \alst{h}evan-ríkja &
\alst{sp}anen mid is \alst{sp}ráku \hld\ ęndi sęggjan \alst{sp}el godes, &
ak \alst{w}ęnkid þero \alst{w}ordo, \hld\ þan wirðid im \alst{w}aldand gram, &
\alst{m}ahtig \alst{m}ódag, \hld\ ęndi só samo \alst{m}anno barn; &
wirðid \alst{a}llun þan \hld\ \alst{i}rmin-þiodun, &
\alst{l}iudjun a·\alst{l}êðid, \hld\ ef is \alst{l}êra ni dugun.“\eva

\bvb TODO.\evb\evg

\bvg\bva[17][1381]%
Só \alst{sp}rak hé þó \alst{sp}áh-líko \hld\ ęndi sagda \alst{sp}el godes, &
\alst{l}êrde þe \alst{l}andes ward \hld\ \alst{l}iudi síne &
mid \alst{h}luttru \alst{h}ugju. \hld\ \alst{H}ęliðos stódun, &
\alst{g}umon umbi þana \alst{g}odes sunu \hld\ \alst{g}erno swíðo, &
\alst{w}eros an \alst{w}illjon: \hld\ was im þero \alst{w}ordo niud, &
\alst{þ}ȧhtun ęndi \alst{þ}agodun, \hld\ gi·hôrdun þero \alst{þ}iodo drohtin &
sęggjan \alst{ê}w godes \hld\ \alst{ę}ldi-barnun; &
gi·\alst{h}ét im \alst{h}evan-ríki \hld\ ęndi te þem \alst{h}ęliðun sprak: &
„Ôk mag ik iu \alst{s}ęggjan, \hld\ ge·\alst{s}ïðos mína, &
\alst{w}árun \alst{w}ordun, \hld\ þat gí þesoro \alst{w}er-oldes nú forð &
skulun \alst{l}ioht wesan \hld\ \alst{l}iudjo barnun, &
\alst{f}agạr mid \alst{f}irihun \hld\ ovar \alst{f}olk manag, &
\alst{w}litig ęndi \alst{w}un-sam: \hld\ ni mugun iuwa \alst{w}erk mikil &
bi·\alst{h}olan werðan, \hld\ mid hwi-liko gí sea \alst{h}ugi ku̇ðjat: &
þan mêr þe þiu \alst{b}urg ni mag, \hld\ þiu an \alst{b}erge stáð, &
\alst{h}ôh \alst{h}olm-klivu, \hld\ bi·\alst{h}olen werðen, &
\alst{w}risi-lík gi·\alst{w}erk, \hld\ ni mugun iuwa \alst{w}ord þan mêr &
an þesoro \alst{m}iddil-gard \hld\ \alst{m}annum werðen, &
iuwa \alst{d}ádi bi·\alst{d}ęrnit. \hld\ \alst{D}ót, só ik iu lêrju: &
\alst{l}átad iuwa \alst{l}ioht mikil \hld\ \alst{l}iudjun skínan, &
\alst{m}anno barnun, \hld\ þat sie far·standan iuwan \alst{m}ód-sevon, &
iuwa \alst{w}erk ęndi iuwan \alst{w}illjon, \hld\ ęndi þes \alst{w}aldand god &
mid \alst{h}luttro \alst{h}ugju, \hld\ \alst{h}imiliskan fader, &
\alst{l}ovon an þesumu \alst{l}iohte, \hld\ þes hé iu su·lika \alst{l}êra far·gaf. &
Ni skal neoman \alst{l}ioht, þe it havad, \hld\ \alst{l}iudjun dęrnjan, &
te \alst{h}ardo be·\alst{h}węlvjan, \hld\ ak hé it \alst{h}ôho skal &
an \alst{s}ęli \alst{s}ęttjan, \hld\ þat þea ge·\alst{s}ehan mugin &
\alst{a}lla ge·líko, \hld\ þea þár \alst{i}nna sind, &
\alst{h}ęliðos an \alst{h}allu. \hld\ Þan hald ni skulun gí iuwa \alst{h}êlag word &
an þesumu \alst{l}and-skępa \hld\ \alst{l}iudjun dęrnjen, &
\alst{h}ęlið-kunnje far·\alst{h}elan, \hld\ ak ge it \alst{h}ôho skulun &
\alst{b}rêdjan, þat gi·\alst{b}od godes, \hld\ þat it allaro \alst{b}arno ge·hwi-lik, &
ovar al þit \alst{l}and-skępi \hld\ \alst{l}iudi far·standan &
ęndi só ge·\alst{f}rummjen, \hld\ só it an \alst{f}orn-dagun &
tulgo \alst{w}íse man \hld\ \alst{w}ordun ge·sprákun, &
þan sie þana \alst{a}ldan \alst{ê}w \hld\ \alst{e}rlos heldun, &
ęndi ôk \alst{s}u·liku \alst{s}wíðor, \hld\ só ik iu nú \alst{s}ęggjan mag, &
alloro \alst{g}umono ge·hwi-lik \hld\ \alst{g}ode þionojan, &
þan it þár an þem \alst{a}ldom \hld\ \alst{ê}wa ge·beode. &
Ni \alst{w}ánjat gí þes mit \alst{w}ihtju, \hld\ þat ik bi þiu an þesa \alst{w}er-old kwámi, &
þat ik þana \alst{a}ldan \alst{ê}w \hld\ \alst{i}rrjen willje, &
\alst{f}ęlljan undar þesumu \alst{f}olke \hld\ efþo þero \alst{f}ora-sagono &
\alst{w}ord \alst{w}iðar-\alst{w}erpen, \hld\ þea hér só gi·\alst{w}árja man &
\alst{b}ar-líko ge·\alst{b}udun. \hld\ Êr skal \alst{b}êðju te·faran, &
\alst{h}imil ęndi erðe, \hld\ þiu nú bi·\alst{h}lidan standat, &
êr þan þero \alst{w}ordo \hld\ \alst{w}iht bi·líva &
un·\alst{l}êstid an þesumu \alst{l}iohte, \hld\ þea sie þesum \alst{l}iudjun hér &
\alst{w}ár-líko ge·budun. \hld\ Ni kwam ik an þesa \alst{w}er-old te þiu, &
þat ik \alst{f}eldi þero \alst{f}ora-sagono word, \hld\ ak ik siu \alst{f}ulljen skal, &
\alst{ô}kjon ęndi nígjan \hld\ \alst{ę}ldi-barnum, &
þesumu \alst{f}olke te \alst{f}rumu. \hld\ Þat was \alst{f}orn ge·skrivan &
an þem \alst{a}ldon \alst{ê}o \hld\ —ge hôrdun it \alst{o}ft sprekan &
\alst{w}ord-\alst{w}íse man—: \hld\ só hwe só þat an þesoro \alst{w}er-oldi gi·dót, &
þat hé \alst{ȧ}ðrana \hld\ \alst{a}ldru bi·neote, &
\alst{l}ívu bi·\alst{l}ôsje, \hld\ þem skulun \alst{l}iudjo barn &
\alst{d}ôd a·\alst{d}êljan. \hld\ Þan willjo ik it iu \alst{d}iopor nú, &
\alst{f}urður bi·\alst{f}ȧhan: \hld\ só hwe só ina þurh \alst{f}íund-skępi, &
\alst{m}an wiðar ȯðrana \hld\ an is \alst{m}ód-sevon &
\alst{b}ilgit an is \alst{b}reostun \hld\ —hwand sie alle ge·\alst{b}róðar sint, &
\alst{s}álig folk godes, \hld\ \alst{s}ibbjon bi·tengja, &%TODO: Check etymology of bi·tengja.
\alst{m}an mid \alst{m}ág-skępi—, \hld\ þan wirðit þoh hwe ȯðrumu an is \alst{m}óde só gram, &
\alst{l}íbes weldi ina bi·\alst{l}ôsjen, \hld\ of hé mahti gi·\alst{l}êstjen só: &
þan is hé sán a·\alst{f}éhit \hld\ ęndi is þes \alst{f}erạhas skolo, &
\alst{a}l su·likes \alst{u}r-dêljes \hld\ só þe \alst{ȯ}ðar was, &
þe þurh is \alst{h}and-męgin \hld\ \alst{h}ôvdo bi·lôsde &
\alst{e}rl \alst{ȯ}ðarna. \hld\ Ôk is an þem \alst{ê}o ge·skrivan &
\alst{w}árun \alst{w}ordun, \hld\ só gí \alst{w}iton alle, &
þan man is \alst{n}áhiston \hld\ \alst{n}iud-líko skal &
\alst{m}innjan an is \alst{m}óde, \hld\ wesen is \alst{m}águn hold, &
\alst{g}adulingun \alst{g}ód, \hld\ wesen is \alst{g}eva mildi, &
\alst{f}râhon is \alst{f}riunda ge·hwane, \hld\ ęndi skal is \alst{f}íund hatan, &
wiðẹr·\alst{st}anden þem mid \alst{st}rídu \hld\ ęndi mid \alst{st}arku hugi, &
\alst{w}ęrjan wiðar \alst{w}rêðun. \hld\ Þan sęggjo ik iu te \alst{w}áron nú, &
\alst{f}ul-líkur for þesumu \alst{f}olke, \hld\ þat gí iuwa \alst{f}íund skulun &
\alst{m}innjon an iuwomu \alst{m}óde, \hld\ só samo só gí iuwa \alst{m}ágos dót, &
an \alst{g}odes namon. \hld\ Dót im \alst{g}ódes filu, &
tôgjat im \alst{h}luttran \alst{h}ugi, \hld\ \alst{h}olda treuwa, &
\alst{l}iof wiðar ira \alst{l}êðe. \hld\ Þat is \alst{l}ang-sam rád &
\alst{m}anno só hwi-likumu, \hld\ só is \alst{m}ód te þiu &
ge·\alst{f}líhit wiðar is \alst{f}íunde. \hld\ Þan mótun gí þea \alst{f}ruma êgan, &
þat gí mótun \alst{h}êten \hld\ \alst{h}evan-kuninges suni, &
is \alst{b}líði \alst{b}arn. \hld\ Ne mugun gí iu \alst{b}ętaran rád &
ge·\alst{w}innan an þesoro \alst{w}er-oldi. \hld\ Þan sęggjo ik iu te \alst{w}áron ôk, &
\alst{b}arno ge·hwi-likum, \hld\ þat gí ne mugun mid gi·\alst{b}olgono hugi &
iuwas \alst{g}ódes wiht \hld\ te \alst{g}odes húsun &
\alst{w}aldande far·gevan, \hld\ þat it imu \alst{w}irðig sí &
te ant·\alst{f}ȧhanne, \hld\ só lango só þú \alst{f}íund-skępjes wiht, &
wiðẹr \alst{ȯ}ðran man \hld\ \alst{i}n-wid hugis. &
Êr skalt þú þi \alst{s}imbla ge·\alst{s}ónjen \hld\ wið þana \alst{s}ak-waldand, &
ge·\alst{m}ódi gi·\alst{m}ahljan: \hld\ sïðor maht þú \alst{m}êðmos þína &
te þem \alst{g}odes altere a·\alst{g}evan: \hld\ þan sind sie þemu \alst{g}ódan werðe, &
\alst{h}evan-kuninge. \hld\ Mêr skulun gí aftar is \alst{h}uldi þionon, &
\alst{g}odes willjon ful·\alst{g}án, \hld\ þan ȯðra \alst{J}udeon duon, &
ef gí willjat \alst{ê}gan \hld\ \alst{ê}wan ríki, &
\alst{s}in-líf \alst{s}ehan. \hld\ Ôk skal ik iu \alst{s}ęggjan noh, &
hwó it þár an þem \alst{a}ldon \hld\ \alst{ê}o ge·biudid, &
þat \alst{ê}nig \alst{e}rl \alst{ȯ}ðres \hld\ \alst{i}dis ni bi·swíka, &
\alst{w}íf mid \alst{w}ammu. \hld\ Þan sęggjo ik iu te \alst{w}áron ôk, &
þat þár man is \alst{s}iuni mugun \hld\ \alst{s}wíðo far·lêdjan &
an \alst{m}irki \alst{m}ên, \hld\ ef hí ina látid is \alst{m}ód spanen, &
þat hé be·\alst{g}inna þero \alst{g}irnjan, \hld\ þiu imu ge·\alst{g}angan ni skal. &
Þan haved hé an imu \alst{s}elvon \alst{s}án \hld\ \alst{s}undja ge·warhta, &
ge·\alst{h}ęftid an is \alst{h}ertan \hld\ \alst{h}ęlli-wíti. &
Ef þan þana man is \alst{s}iun wili \hld\ eþþa is \alst{s}wíðare hand &
far·\alst{l}êdjen is \alst{l}iðo hwi-lik \hld\ an \alst{l}êðan weg, &
þan is \alst{e}rlo ge·hwem \hld\ \alst{ȯ}ðar bętara, &
\alst{f}iriho barno, \hld\ þat hé ina \alst{f}ram werpa &
ęndi þana \alst{l}ið \alst{l}ôsje \hld\ af is \alst{l}ík-hamon &
ęndi ina \alst{á}no kuma \hld\ \alst{u}p te himile, &
þan hé só mid \alst{a}llun \hld\ te þem \alst{I}nferne, &
\alst{h}werve mid só \alst{h}êlun \hld\ an \alst{h}ęlli-grund. &
Þan mênid þiu \alst{l}éf-hêd, \hld\ þat ênig \alst{l}iudjo ni skal &
far·\alst{f}olgan is \alst{f}riunde, \hld\ ef hé ina an \alst{f}irina spanit, &
\alst{s}wás man an \alst{s}aka: \hld\ þan ne sí hé imu eo só swíðo an \alst{s}ibbjun bi·lang, &
ne iro \alst{m}ág-skępi só \alst{m}ikil, \hld\ ef hé ina an \alst{m}orð spęnit, &
\alst{b}édid \alst{b}alu-werko; \hld\ \alst{b}ętera is imu þan ȯðar, &
þat hé þana \alst{f}riund fan imu \hld\ \alst{f}er far·werpa, &
\alst{m}íðe þes \alst{m}áges \hld\ ęndi ni hębbja þár êniga \alst{m}innja tó, &
þat hé móti \alst{ê}no \hld\ \alst{u}p ge·stígan &
\edtext{\alst{h}ôh}{\Afootnote{TODO: Critical note (ms. apparently has hô)}} \alst{h}imil-ríki, \hld\ þan sie \alst{h}ęlli-ge·þwing, &
\alst{b}rêd \alst{b}alu-wíti \hld\ \alst{b}êðja gi·sókjan, &
\alst{u}vil \alst{a}rvidi.\eva

\bvb TODO.\evb\evg

\bvg\bva[18][1502]%
\hspace*{100pt} Ôk is an þem \alst{ê}o ge·skrivan &%NOTE: In cæsura.
\alst{w}árun \alst{w}ordun, \hld\ só gí \alst{w}itun alle, &
þat \alst{m}íðe \alst{m}ên-êðos \hld\ \alst{m}an-kunnjes ge·hwi-lik, &
ni for·\alst{s}węrje ina \alst{s}elvon, \hld\ hwand þat is \alst{s}undje te mikil, &
far·\alst{l}êdid \alst{l}iudi \hld\ an \alst{l}êðan weg. &
Þan willjo ik iu eft \alst{s}ęggjan, \hld\ þan sán ni \alst{s}węrja neo-man &
\alst{ê}nigan \alst{ê}ð-staf \hld\ \alst{ę}ldi-barno, &
ne bi \alst{h}imile þemu \alst{h}ôhon, \hld\ hwand þat is þes \alst{h}êrron stól, &
ne bi \alst{e}rðu þár \alst{u}ndar, \hld\ hwand þat is þes \alst{a}lo-waldon &
\alst{f}agạr \alst{f}ót-skamel, \hld\ nek ênig \alst{f}iriho barno &
ne \alst{s}węrja bi is \alst{s}elves hôvde, \hld\ hwand hé ni mag þár ne \alst{s}wart ne hwít &
ênig \alst{h}ár ge·wirkjan, \hld\ b·útan só it þe \alst{h}êlago god, &
ge·\alst{m}arkode \alst{m}ahtig; \hld\ be·þiu skulun \alst{m}íðan filu &
\alst{e}rlos \alst{ê}ð-wordo. \hld\ Só hwe só it \alst{o}fto dót, &
só \alst{w}irðid is simbla \alst{w}irsa, \hld\ hwand hé imu gi·\alst{w}ardon ni mag. &
Bi·þiu skal ik iu nú te \alst{w}árun \hld\ \alst{w}ordun gi·beodan, &
þat gí neo ne \alst{s}węrjen \hld\ \alst{s}wíðoron êðos, &
\alst{m}éron \alst{m}et \alst{m}annun, \hld\ b·útan só ik iu mid \alst{m}ínun hér &
swíðo \alst{w}ár-liko \hld\ \alst{w}ordun ge·biudu: &
ef man hwemu \alst{s}aka \alst{s}ókja, \hld\ bi·\alst{s}ęggja þat wáre, &
kweðe \alst{j}á, gef it sí, \hld\ \alst{g}eha þes þár wár is, &
kweðe \alst{n}ên, af it \alst{n}is, \hld\ láta im ge·\alst{n}óg an þiu; &
só hwat só is \alst{m}êr ovar þat \hld\ \alst{m}an ge·frummjad, &
só kumid it \alst{a}l fan \alst{u}vile \hld\ \alst{ę}ldi-barnun, &
þat \alst{e}rl þurh \alst{u}n-treuwa \hld\ \alst{ȯ}ðres ni wili &
\alst{w}ordo ge·lôvjan. \hld\ Þan sęggjo ik iu te \alst{w}áron ôk, &
hwó it þár an þem \alst{a}ldon \hld\ \alst{ê}o ge·biudit: &
só hwe só \alst{ô}gon ge·nimid \hld\ \alst{ȯ}ðres mannes, &
\alst{l}ôsid af is \alst{l}ík-haman, \hld\ eþþa is \alst{l}iðo hwi-likan, &
þat hé it eft mid is \alst{s}elves skal \hld\ \alst{s}án ant·gelden &
mid ge·\alst{l}íkun \alst{l}iðjon. \hld\ Þan willjo ik iu \alst{l}êrjan nú, &
þat gí só ni \alst{w}rekan \hld\ \alst{w}rêða dádi, &
ak þat gí þurh \alst{ô}d-módi \hld\ \alst{a}l ge·þologjan &
\alst{w}ítjes ęndi \alst{w}ammes, \hld\ só hwat só man iu an þesoro \alst{w}er-oldi ge·dóe. &
Dóe \alst{a}lloro \alst{e}rlo ge·hwi-lik \hld\ \alst{ȯ}ðrom manne &
\alst{f}rume ęndi ge·\alst{f}óri, \hld\ só hé willje, þat im \alst{f}iriho barn &
\alst{g}ódes an·\alst{g}ęgin dóen. \hld\ Þan wirðit im \alst{g}od mildi, &
\alst{l}iudjo só hwi-likum, \hld\ só þat \alst{l}êstjen wili. &
\alst{Ê}rod gí \alst{a}rme man, \hld\ dêljad iuwan \alst{ô}d-welon &
undar þero \alst{þ}urftigon \alst{þ}iodu; \hld\ ne rókjad, hweðar gí is ênigan \alst{þ}ank ant·fȧhan &
efþo lôn an þesoro \alst{l}êhnjon wer-oldi, \hld\ ak huggjat te iuwomu \alst{l}eovon hêrran &
þero \alst{g}evono te \alst{g}elde, \hld\ þat sie iu \alst{g}od lôno, &
\alst{m}ahtig \alst{m}und-boro, \hld\ só hwat só gí is þurh is \alst{m}innes gi·dót. &
Ef þú þan \alst{g}evogjan wili \hld\ \alst{g}ódun mannun &
\alst{f}agạre \alst{f}eho-skattos, \hld\ þár þú eft \alst{f}rumono hugis &
\alst{m}êr ant·fȧhan, \hld\ te hwí havas þú þes êniga \alst{m}éda fon gode &
eþþa \alst{l}ôn an þemu is \alst{l}iohte? \hld\ hwand þat is \alst{l}êhni feho. &
Só is þes \alst{a}lles ge·hwat, \hld\ þe þú \alst{ȯ}ðrun ge·duos &
\alst{l}iudjon te \alst{l}eove, \hld\ þár þú hugis eft ge·\alst{l}ík neman &
þero \alst{w}ordo ęndi þero \alst{w}erko: \hld\ te hwí wêt þi þes u̇sa \alst{w}aldand þank, &
þes þú þín só bi·\alst{f}ilhis \hld\ ęndi ant·\alst{f}áhis eft þan þú wili? &
\alst{i}uwan \alst{ô}ð-welon \hld\ gevan gí þem \alst{a}rmun mannun, &
þe ina iu an þesoro \alst{w}er-oldi ne lônon \hld\ ęndi rómot te iuwes \alst{w}aldandes ríkja. &
Te \alst{h}lúd ni dó þú it, \hld\ þan þú mid þínun \alst{h}andun bi·felhas &
þína \alst{a}lamosna þemu \alst{a}rmon manne, \hld\ ak dó im þurh \alst{ô}d-módjen &
\alst{g}erno þurh \alst{g}odes þank: \hld\ þan móst þú eft \alst{g}eld niman, &
swíðo \alst{l}iof-lík \alst{l}ôn, \hld\ þár þú is \alst{l}ango bi·þarft, &
\alst{f}agạroro \alst{f}rumono. \hld\ Só hwat só þú is só þurh \alst{f}erhtan hugi &
\alst{d}arno ge·\alst{d}êljas, \hld\ —so is u̇sumu \alst{d}rohtine werð— &
ne \alst{g}alpo þú far þínun \alst{g}evun te swíðo, \hld\ noh ênig \alst{g}umono ne skal, &
þat siu im þurh \alst{í}dale hróm \hld\ \alst{e}ft ni werðe &
\alst{l}êð-líko far·\alst{l}oren. \hld\ Þanna þú skalt \alst{l}ôn nemen &
fora \alst{g}odes ôgun \hld\ \alst{g}ódero werko. &
Ôk skal ik iu ge·\alst{b}eodan, \hld\ þan gí willjad te \alst{b}edu hnígan &
ęndi willjad te iuwomu \alst{h}êrron \hld\ \alst{h}elpono biddjan, &
þat hé iu a·\alst{l}áte \hld\ \alst{l}êðes þinges, &
þero \alst{s}akono ęndi þero \alst{s}undjono, \hld\ þea gí iu \alst{s}elvon hír &
\alst{w}rêða ge·\alst{w}irkjad, \hld\ þat gí it þan for ȯðrumu \alst{w}erode ni duad: &
ni \alst{m}árjad it far \alst{m}ęnigi, \hld\ þat iu þes \alst{m}an ni lovon, &
ni \alst{d}iurjan þero \alst{d}ádjo, \hld\ þat gí iuwes \alst{d}rohtines gi·bed &
þurh þat \alst{í}dala hróm \hld\ \alst{a}l ne far·leosan. &
Ak þan gí willjan te iuwomo \alst{h}êrron \hld\ \alst{h}elpono biddjan, &
\alst{þ}iggjan \alst{þ}eo-líko, \hld\ —þes iu is \alst{þ}arf mikil— &
þat iu \alst{s}igi-drohtin \hld\ \alst{s}undjono tómja, &
þan \alst{d}ót gí þat só \alst{d}arno: \hld\ þoh wêt it iuwe \alst{d}rohtin self &
\alst{h}êlag an \alst{h}imile, \hld\ hwand imu nis bi·\alst{h}olan n·eo·wiht &
ne \alst{w}ordo ne \alst{w}erko. \hld\ Hé látid it þan al ge·\alst{w}erðan só, &
só gí ina þan \alst{b}iddjad, \hld\ þan gí te þero \alst{b}edo hnígad &
mid \alst{h}luttru \alst{h}ugi.“ \hld\ \alst{H}ęliðos stódun, &
\alst{g}umon umbi þana \alst{g}odes sunu \hld\ \alst{g}erno swíðo, &
\alst{w}eros an \alst{w}illjon: \hld\ was im þero \alst{w}ordo niud, &
\alst{þ}ȧhtun ęndi \alst{þ}agodun, \hld\ was im \alst{þ}arf mikil, &
þat sie þat eft ge·\alst{h}ogdin, \hld\ þat im þat \alst{h}êlaga barn &
an þana \alst{f}orman sïð \hld\ \alst{f}ilu mid wordun &
\alst{t}orhtes ge·\alst{t}alde. \hld\ Þó sprak im eft ên þero \alst{t}we-livjo an·gęgin, &
\alst{g}lauworo \alst{g}umono, \hld\ te þem \alst{g}odes barne:\eva

\bvb TODO.\evb\evg

\bvg\bva[19][1588]%
„\alst{H}êrro þe gódo“, \hld[kwað hé,] „u̇s is þínoro \alst{h}uldi þarf, &
te gi·\alst{w}irkenne þínna \alst{w}illjon, \hld\ ęndi ôk þínoro \alst{w}ordo só self, &
allaro \alst{b}arno \alst{b}ętst, \hld\ þat þú u̇s \alst{b}edon lêres, &
\alst{j}ungoron þíne, \hld\ só \alst{J}ohannes duot, &
\alst{d}iur-lík \alst{d}ôperi, \hld\ \alst{d}ago ge·hwi-likas &
is \alst{w}erod mid \alst{w}ordun, \hld\ hwí sie \alst{w}aldand skulun, &
\alst{g}ódan \alst{g}rótjan. \hld\ Dó þína \alst{j}ungorun só self: &
ge·\alst{r}ihti u̇s þat ge·\alst{r}úni.“ \hld\ Þó habda eft þe \alst{r}íkjo garu &
\alst{s}án aftar þiu, \hld\ \alst{s}unu drohtines, &
\alst{g}ód word an·\alst{g}ęgin: \hld\ „Þan gí \alst{g}od willjan“, kwað hé, &
„\alst{w}eros mid iuwon \alst{w}ordun \hld\ \alst{w}aldand grótjan, &
allaro \alst{k}uningo \alst{k}raftigostan, \hld\ þan \alst{k}weðad gi, só ik iu lêrju: &
‚\alst{F}adar u̇sa \hld\ \alst{f}iriho barno, &
þú bist an þem \alst{h}ôhon \hld\ \alst{h}imila ríkja, &
ge·\alst{w}íhid sí þín namo \hld\ \alst{w}ordo ge·hwi-liko. &
\alst{K}uma þín \hld\ \alst{k}raftag ríki. &
\alst{W}erða þín \alst{w}illjo \hld\ ovar þesa \alst{w}er-old alla, &
só sama an erðo, \hld\ só þár \alst{u}ppa ist &
an þem \alst{h}ôhon \hld\ \alst{h}imilo ríkja. &
Gef u̇s \alst{d}ago ge·hwi-likes rád, \hld\ \alst{d}rohtin þe gódo, &
þína \alst{h}êlaga \alst{h}elpa, \hld\ ęndi a·lát u̇s, \alst{h}evanes ward, &
\alst{m}anagoro \alst{m}ên-skuldjo, \hld\ al só we ȯðrum \alst{m}annum dóan. &
Ne lát u̇s far·\alst{l}êdjan \hld\ \alst{l}êða wihti &
só forð an iro \alst{w}illjon, \hld\ só wí \alst{w}irðige sind, &
ak help u̇s wiðar \alst{a}llun \hld\ \alst{u}vilon dádjun.‘ &
Só skulun gí \alst{b}iddjan, \hld\ þan gí te \alst{b}ede hnígad &
\alst{w}eros mid iuwom \alst{w}ordun, \hld\ þat iu \alst{w}aldand god &
\alst{l}êðes a·\alst{l}áte \hld\ an \alst{l}eut-kunnja. &
Ef gí þan willjad a·\alst{l}átan \hld\ \alst{l}iudjo ge·hwi-likun &
þero \alst{s}akono ęndi þero \alst{s}undjono, \hld\ þe sie wið iu \alst{s}elvon hír &
\alst{w}rêða ge·\alst{w}irkjat, \hld\ þan a·látid iu \alst{w}aldand god, &
\alst{f}adar ala-mahtig \hld\ \alst{f}irin-werk mikil, &
\alst{m}anagoro \alst{m}ên-skuldjo. \hld\ Ef iu þan wirðid iuwa \alst{m}ód te stark, &
þat gí ne wiljat \alst{ȯ}ðrun \hld\ \alst{e}rlun a·látan, &
\alst{w}eron \alst{w}am-dádi, \hld\ þan ne wil iu ôk \alst{w}aldand god &
\alst{g}rim-werk far·\alst{g}evan, \hld\ ak gí skulun is \alst{g}eld niman, &
swíðo \alst{l}êð-lik \alst{l}ôn \hld\ te \alst{l}anguru hwílu, &
\alst{a}lles þes \alst{u}n-rehtes, \hld\ þes gí \alst{ȯ}ðrum hír &
gi·\alst{l}êstjad an þesumu \alst{l}iohte \hld\ ęndi þan wið \alst{l}iudjo barn &
þea \alst{s}aka ni gi·\alst{s}ónjad, \hld\ êr gí an þana \alst{s}ïð faran, &
\alst{w}eros fon þesoro \alst{w}er-oldi. \hld\ Ok skal ik iu te \alst{w}árun sęggjan, &
hwó gí \alst{l}êstjan skulun \hld\ \alst{l}êra mína: &
þan gí iuwa \alst{f}astonnja \hld\ \alst{f}rummjan willjan, &
\alst{m}inson iuwa \alst{m}ên-dádi, \hld\ þan ni duad gí þat te \alst{m}anagom ku̇ð, &
ak \alst{m}íðad is far ȯðrum \alst{m}annun: \hld\ þoh wêt \alst{m}ahtig god, &
\alst{w}aldand iuwan \alst{w}illjan, \hld\ þoh iu \alst{w}erod ȯðar, &
\alst{l}iudjo barn ne \alst{l}ovon. \hld\ Hé gildid is iu \alst{l}ôn aftar þiu, &
iuwa \alst{h}êlag fadar \hld\ an \alst{h}imil-ríkja, &
þes ge im mid su·likum \alst{ô}d-módja, \hld\ \alst{e}rlos þeonod, &
só \alst{f}erht-líko undar þesumu \alst{f}olke. \hld\ Ne willjat \alst{f}eho winnan &
\alst{e}rlos an \alst{u}n-reht, \hld\ ak wirkjad \alst{u}p te gode &
\alst{m}an aftar \alst{m}édu: \hld\ þat is \alst{m}êra þing, &
þan man hír an \alst{e}rðu \hld\ \alst{ô}dag libbja, &
\alst{w}er-old-skattes ge·\alst{w}ono. \hld\ Ef gí willjad mínun \alst{w}ordun hôrjan, &
þan ne \alst{s}amnod gí hír \alst{s}ink mikil \hld\ \alst{s}ilọvres ne goldes &
an þesoro \alst{m}iddil-gard, \hld\ \alst{m}êðọm-hordes, &
hwand it \alst{r}otat hír an \alst{r}oste, \hld\ ęndi \alst{r}ęgin-þeovos far·stelad, &
\alst{w}urmi a·\alst{w}ardjad, \hld\ wirðid þat gi·\alst{w}ádi far·slitan, &
ti·\alst{g}angid þe \alst{g}old-welo. \hld\ Lêstjad iuwa \alst{g}ódon werk, &
samnod iu an \alst{h}imile \hld\ \alst{h}ord þat méra, &
\alst{f}agạra \alst{f}eho-skattos: \hld\ þat ni mag iu ênig \alst{f}íund be·niman, &
ne-\alst{w}iht an·\alst{w}ęndjan, \hld\ hwand þe \alst{w}elo standid &
\alst{g}aru iu te·\alst{g}ęgnes, \hld\ só hwat só gí \alst{g}ódes þarod, &
an þat \alst{h}imil-ríki \hld\ \alst{h}ordes ge·samnod, &
\alst{h}ęliðos þurh iuwa \alst{h}and-geva, \hld\ ęndi hębbjad þarod iuwan \alst{h}ugi fasto; &
hwand þár ist alloro \alst{m}anno gi·hwes \hld\ \alst{m}ód-ge·þȧhti, &
\alst{h}ugi ęndi \alst{h}erta, \hld\ þár is \alst{h}ord ligid, &
\alst{s}ink ge·\alst{s}amnod. \hld\ Nis eo só \alst{s}álig man, &
þat mugi an þesoro \alst{b}rêdon wer-old \hld\ \alst{b}êðju ant·hengjan, &
ge þat hí an þesoro \alst{e}rðo \hld\ \alst{ô}dag libbja, &
an allun \alst{w}er-old-lustun \alst{w}esa, \hld\ ge þoh \alst{w}aldand gode &
te \alst{þ}anke ge·\alst{þ}eono: \hld\ ak hé skal alloro \alst{þ}ingo gi·hwes &
simbla \alst{ȯ}ðar-hweðar \hld\ \alst{ê}n far·látan &
eþþo \alst{l}usta þes \alst{l}ík-hamon \hld\ eþþo \alst{l}íf êwig. &
Be·þiu ni \alst{g}ornot gí umbi iuwa ge·\alst{g}aruwi, \hld\ ak huggjad te \alst{g}ode fasto, &
ne \alst{m}ornont an iuwomu \alst{m}óde, \hld\ hwat gí eft an \alst{m}organ skulin &
\alst{e}tan efþo drinkan \hld\ eþþo \alst{a}n hębbjan &
\alst{w}eros te ge·\alst{w}ę́dja: \hld\ it wêt al \alst{w}aldand god, &
hwes þea bi·\alst{þ}urvun, \hld\ þea im hír \alst{þ}ionod wel, &
\alst{f}olgod iro \alst{f}rôhan willjon. \hld\ Hwat gí þat bi þesun \alst{f}uglun mugun &
\alst{w}ár-líko undar·\alst{w}itan, \hld\ þea hír an þesoro \alst{w}er-oldi sint, &
\alst{f}arad an \alst{f}eðar-hamun: \hld\ sie ni kunnun ênig \alst{f}eho winnan, &
þoh givid im \alst{d}rohtin god \hld\ \alst{d}ago ge·hwi-likes &
\alst{h}elpa wiðar \alst{h}ungre. \hld\ Ôk mugun gí an iuwom \alst{h}ugi markon, &
\alst{w}eros umbi iuwa ge·\alst{w}ádi, \hld\ hwó þie \alst{w}urti sint &
\alst{f}agọro ge·\alst{f}ratohot, \hld\ þea hír an \alst{f}elde stád, &
\alst{b}erht-líko ge·\alst{b}lóid: \hld\ ne mahta þe \alst{b}urges ward, &
\alst{S}alomon þe \alst{s}uning, \hld\ þe habda \alst{s}ink mikil, &
\alst{m}êðọm-hordas \alst{m}êst, \hld\ þero þe ênig \alst{m}an êhti, &
\alst{w}elono ge·\alst{w}unnan \hld\ ęndi allaro ge·\alst{w}ádjo kust,— &
þoh ni mohte hé an is \alst{l}íve, \hld\ þoh hé habdi alles þeses \alst{l}andes ge·wald, &
a·\alst{w}innan su·lik ge·\alst{w}ádi, \hld\ só þiu \alst{w}urt havad, &
þiu hír an \alst{f}elde stád \hld\ \alst{f}agọro ge·gariwit, &
\alst{l}illi mid só \alst{l}iof-líku blómon: \hld\ ina wádit þe \alst{l}andes waldand &
hér fan \alst{h}evanes wange. \hld\ Mér is im þoh umbi þit \alst{h}ęliðo kunni, &
\alst{l}iudi sint im \alst{l}iovoron mikilu, \hld\ þea hé im an þesumu \alst{l}ande ge·warhte, &
\alst{w}aldand an \alst{w}illjon sínan. \hld\ Be·þiu ne þurvon gí umbi iuwa ge·\alst{w}ádi sorgon, &
ne \alst{g}ornot gí umbi iuwa ge·\alst{g}ariwi te swíðo: \hld\ \alst{g}od wili is alles rádan, &
\alst{h}elpan fan \alst{h}evanes wange, \hld\ ef gí willjad aftar is \alst{h}uldi þeonon. &
\alst{G}erot gí simbla êrist þes \alst{g}odes ríkjas, \hld\ ęndi þan duat aftar þem is \alst{g}ódun werkun, &
\alst{r}ómod gí \alst{r}ehtoro þingo: \hld\ þan wili iu þe \alst{r}íkjo drohtin &
\alst{g}evon mid alloro \alst{g}ódu ge·hwi-liku, \hld\ ef gí im þus ful·\alst{g}angan willjad, &
só ik iu te \alst{w}árun hír \hld\ \alst{w}ordun sęggjo.\eva

\bvb TODO.\evb\evg

\bvg\bva[20][1691]%
Ne skulun gí \alst{ê}nigumu manne \hld\ \alst{u}n-rehtes wiht, &
\alst{d}ęrvjes a·\alst{d}êljan, \hld\ hwand þe \alst{d}óm eft kumid &
ovar þana \alst{s}elvon man, \hld\ þár it im te \alst{s}orgon skal, &
\alst{w}erðan þem te \alst{w}ítja, \hld\ þe hír mid is \alst{w}ordun ge·sprikid &
\alst{u}n-reht \alst{ȯ}ðrum. \hld\ Neo þat iuwar \alst{ê}nig ne dua &
\alst{g}umono an þesom \alst{g}ardon \hld\ \alst{g}eldes eþþo kôpes, &
þat hí \alst{u}n-reht gi·met \hld\ \alst{ȯ}ðrumu manne &
\alst{m}ên-ful \alst{m}ako, \hld\ hwand it simbla \alst{m}ótjan skal &
\alst{e}rlo ge·hwi-likomu, \hld\ su·lik só hé it \alst{ȯ}ðrumu ge·dód, &
só kumid it im eft te·\alst{g}ęgnes, \hld\ þár hé \alst{g}erno ne wili &
ge·\alst{s}ehan is \alst{s}undjon. \hld\ Ôk skal ik iu \alst{s}ęggjan noh, &
hwár gí iu \alst{w}ardon skulun \hld\ \alst{w}ítjo mêsta, &
\alst{m}ên-werk \alst{m}anag: \hld\ te hwí skalt þú ênigan \alst{m}an be·sprekan, &
\alst{b}róðar þínan, \hld\ þat þú undar is \alst{b}ráhon ge·sehas &
\alst{h}alm an is ôgon, \hld\ ęndi ge·\alst{h}uggjan ni wili &
þana \alst{s}wáran balkon, \hld\ þe þú an þínoro \alst{s}iuni havas, &
\alst{h}ard trio ęndi \alst{h}ęvig. \hld\ Lát þi þat an þínan \alst{h}ugi fallan, &
hwó þú þana êrist a·\alst{l}ôsjas: \hld\ þan skínid þí \alst{l}ioht be·foran, &
\alst{ô}gun werðad þí ge·\alst{o}ponot; \hld\ þan maht þú \alst{a}ftar þiu &
\alst{s}wáses mannes ge·\alst{s}iun \hld\ \alst{s}ïðor ge·bótjan, &
ge·\alst{h}êljan an is \alst{h}ôvde. \hld\ Só mag þat an is \alst{h}ugi méra &
an þesoro \alst{m}iddil-gard \hld\ \alst{m}anno ge·hwi-likumu, &
\alst{w}esan an þesoro \alst{w}er-oldi, \hld\ þat hí hír \alst{w}ammas ge·duot, &
þan hí \alst{a}htogja \hld\ \alst{ȯ}ðres mannes &
\alst{s}aka ęndi \alst{s}undja, \hld\ ęndi havad im \alst{s}elvo mêr &
\alst{f}irin-werko ge·\alst{f}rumid. \hld\ Ef hé wili is \alst{f}ruma lêstjan, &
þan skal hí ina \alst{s}elvon êr \hld\ \alst{s}undjono a·tómjan, &
\alst{l}êð-werko \alst{l}ôson: \hld\ sïðor mag hí mid is \alst{l}êrun werðan &
\alst{h}ęliðun te \alst{h}elpu, \hld\ sïðor hí ina \alst{h}luttran wêt, &
\alst{s}undjono \alst{s}ikoran. \hld\ Ne skulun gí \alst{s}wínum te·foran &
iuwa \alst{m}ęre-gríton makon \hld\ eþþo \alst{m}êðmo ge·striuni, &
\alst{h}êlag \alst{h}als-męni, \hld\ hwand siu it an \alst{h}oru spurnat, &
\alst{s}ulwjad an \alst{s}ande: \hld\ ne witun \alst{s}úvrjas ge·skêð, &
\alst{f}agạroro \alst{f}ratoho. \hld\ Su-lik sint hír \alst{f}olk manag, &
þe iuwa \alst{h}êlag word \hld\ \alst{h}ôrjan ne willjad, &
ful-gangan \alst{g}odes lêrun: \hld\ ne witun \alst{g}ódes ge·skêð, &
ak sind im \alst{l}ári word \hld\ \alst{l}eovoron mikilu, &
umbi·\alst{þ}arvi \alst{þ}ing, \hld\ þanna \alst{þ}eot-godes &
\alst{w}erk ęndi \alst{w}illjo. \hld\ Ne sind sie \alst{w}irðige þan, &
þat sie ge·\alst{h}ôrjan iuwa \alst{h}êlag word, \hld\ ef sie is ne willjad an iro \alst{h}ugi þęnkjan, &
ne \alst{l}ínon ne \alst{l}êstjan. \hld\ Þem ni sęggjan gí iuworo \alst{l}êron wiht, &
þat gí þea \alst{sp}ráka godes \hld\ ęndi \alst{sp}el managu &
ne far·\alst{l}eosan an þem \alst{l}iudjun, \hld\ þea þár ne willjan gi·\alst{l}ôvjan tó, &
\alst{w}ároro \alst{w}ordo. \hld\ Ôk skulun gí iu \alst{w}ardon filu &
\alst{l}istjun undar þesun \alst{l}iudjun, \hld\ þár gí aftar þesumu \alst{l}ande farad, &
þat iu þea \alst{l}uggjon ne mugin \hld\ \alst{l}êron be·swíkan &
ni mid \alst{w}ordun ni mid \alst{w}erkun. \hld\ Sie kumad an su·likom ge·\alst{w}ádjon te iu, &
\alst{f}agọron \alst{f}ratohon: \hld\ þoh hębbjad sie \alst{f}êknan hugi: &
þea mugun gí sán ant·\alst{k}ęnnjan, \hld\ só gí sie \alst{k}uman ge·sehad: &
sie sprekad \alst{w}ís-lík \alst{w}ord, \hld\ þoh iro \alst{w}erk ne dugin, &
þero \alst{þ}egno ge·\alst{þ}ȧhti. \hld\ Hwand gí witun, þat eo an \alst{þ}ornjun ne skulun &
\alst{w}ín-beri \alst{w}esan \hld\ efþa \alst{w}elon eo·wiht, &
\alst{f}agọroro \alst{f}ruhtjo, \hld\ nek ôk \alst{f}ígun ne lesad &
\alst{h}ęliðos an \alst{h}iopon. \hld\ Þat mugun gí undar·\alst{h}uggjan wel, &
þat \alst{e}o þe \alst{u}vilo bôm, \hld\ þár hé an \alst{e}rðu stád, &
\alst{g}óden wastum ne \alst{g}ivid, \hld\ nek it ôk \alst{g}od ni ge·skóp, &
þat þe \alst{g}ódo bôm \hld\ \alst{g}umono barnun &
\alst{b}ári \alst{b}ittres wiht, \hld\ ak kumid fan alloro \alst{b}âmo ge·hwi-likumu &
su·lik \alst{w}astom te þesero \alst{w}er-oldi, \hld\ só im fan is \alst{w}urtjon ge·dregid, &
eþþa \alst{b}erht eþþa \alst{b}ittar. \hld\ Þat mênid þoh \alst{b}reost-hugi, &
\alst{m}anagoro \alst{m}ód-sevon \hld\ \alst{m}anno kunnjes, &
hwó \alst{a}lloro \alst{e}rlo ge·hwi-lik \hld\ \alst{ô}git selvo, &
\alst{m}eldod mid is \alst{m}u̇ðu, \hld\ hwi-likan hé \alst{m}ód havad, &
\alst{h}ugi umbi is \alst{h}erte: \hld\ þes ni mag hé far·\alst{h}elan eo·wiht, &
ak kumad fan þem \alst{u}vilan man \hld\ \alst{i}n-wid-rádos, &
\alst{b}ittara \alst{b}alu-spráka, \hld\ su·lik só hí an is \alst{b}reostun havad &
ge·\alst{h}ęftid umbi is \alst{h}erte: \hld\ simbla is \alst{h}ugi ku̇ðid, &
is \alst{w}illjon mid is \alst{w}ordun, \hld\ ęndi farad is \alst{w}erk aftar þiu. &
Só kumad fan þemu \alst{g}ódan manne \hld\ \alst{g}lau and-wordi, &
\alst{w}ís-lík fan is ge·\alst{w}ittja, \hld\ þat hí simbla mid is \alst{w}ordu ge·sprikid, &
\alst{m}an mid is \alst{m}íðu su·lik, \hld\ só hé an is \alst{m}óde havad &
\alst{h}ord umbi is \alst{h}erte. \hld\ Þanan kumad þea \alst{h}êlagan lêra, &
swíðo \alst{w}un-sam \alst{w}ord, \hld\ ęndi skulun is \alst{w}erk aftar þiu &
\alst{þ}eodu ge·\alst{þ}íhan, \hld\ \alst{þ}egnun managun &
\alst{w}erðan te \alst{w}illjon, \hld\ al só it \alst{w}aldand self &
\alst{g}ódun mannun far·\alst{g}ivid, \hld\ \alst{g}od alo-mahtig, &
\alst{h}imilisk \alst{h}êrro, \hld\ hwand sie áno is \alst{h}elpa ni mugun &
ne mid \alst{w}ordun ne mid \alst{w}erkun \hld\ \alst{w}iht a·þęngjan &
\alst{g}ódes an þesun \alst{g}ardun. \hld\ Be·þiu skulun \alst{g}umono barn &
an is \alst{ê}nes kraft \hld\ \alst{a}lle gi·lôvjan.\eva

\bvb TODO.\evb\evg

\bvg\bva[21][1771]%
Ôk skal ik iu \alst{w}ísjan, \hld\ hwó hír \alst{w}egos twêna &
\alst{l}iggjad an þesumu \alst{l}iohte, \hld\ þea farad \alst{l}iudjo barn, &
\alst{a}l \alst{i}rmin-þiod. \hld\ Þero is \alst{ȯ}ðar sán &
\alst{w}íd stráta ęndi brêd, \hld\ —farid sie \alst{w}erodes filu, &
\alst{m}an-kunnjes \alst{m}anag, \hld\ hwand sie þarod iro \alst{m}ód spęnit, &
\alst{w}er-old-lusta \alst{w}eros— \hld\ þiu an þea \alst{w}irson hand &
\alst{l}iudi \alst{l}êdid, \hld\ þár sie te far·\alst{l}ora werðad, &
\alst{h}ęliðos an \alst{h}ęllju, \hld\ þár is \alst{h}êt ęndi swart, &
\alst{ę}gis-lík an \alst{i}nnan: \hld\ \alst{ó}ði ist þarod te faranne &
\alst{ę}ldi-barnun, \hld\ þoh it im at þemu \alst{ę}ndje ni dugi. &
Þan ligid \alst{e}ft \alst{ȯ}ðar \hld\ \alst{ę}ngira mikilu &
\alst{w}eg an þesoro \alst{w}er-oldi, \hld\ fęrid ina \alst{w}erodes lút, &
\alst{f}áho \alst{f}olk-skępi: \hld\ ni willjad ina \alst{f}iriho barn &
\alst{g}erno \alst{g}angan, \hld\ þoh hé te \alst{g}odes ríkja, &
an þat \alst{ê}wiga líf, \hld\ \alst{e}rlos lêdja. &
Þan nimad gí iu þana \alst{ę}ngjan: \hld\ þoh hé só \alst{ó}ði ne sí &
\alst{f}irihon te \alst{f}aranne, \hld\ þoh skal hí te \alst{f}rumu werðan &
só hwemu só ina þurh·\alst{g}ęngid, \hld\ só skal is \alst{g}eld niman, &
swíðo \alst{l}ang-sam \alst{l}ôn \hld\ ęndi \alst{l}íf êwig, &
\alst{d}iur-líkan \alst{d}rôm. \hld\ Eo gí þes \alst{d}rohtin skulun, &
\alst{w}aldand biddjen, \hld\ þat gí þana \alst{w}eg mótin &
\alst{f}an foran ant·\alst{f}ȧhan \hld\ ęndi \alst{f}orð þurh gi·gangan &
an þat \alst{g}odes ríki. \hld\ Hé ist \alst{g}aru simbla &
wiðar þiu te \alst{g}evanne, \hld\ þe man ina \alst{g}erno bidid, &
\alst{f}ergot \alst{f}iriho barn. \hld\ Sókjad \alst{f}adar iuwan &
\alst{u}p te þemu \alst{ê}winom ríkja: \hld\ þan mótun gí ina \alst{a}ftar þiu &
te iuworu \alst{f}rumu \alst{f}ïðan. \hld\ Ku̇ðjad iuwa \alst{f}ard þarod &
at iuwas \alst{d}rohtines \alst{d}urun: \hld\ þan werðad iu an·\alst{d}ón aftar þiu, &
\alst{h}imil-portun ant·\alst{h}lidan, \hld\ þat gí an þat \alst{h}êlage lioht, &
an þat \alst{g}odes ríki \hld\ \alst{g}angan mótun, &
\alst{s}in-líf \alst{s}ehan. \hld\ Ôk skal ik iu \alst{s}ęggjan noh &
far þesumu \alst{w}erode allun \hld\ \alst{w}ár-lík biliði, &
þat alloro \alst{l}iudjo só hwi-lik, \hld\ só þesa mína \alst{l}êra wili &
ge·\alst{h}aldan an is \alst{h}erton \hld\ ęndi wil iro an is \alst{h}ugi a·þęnkjan, &
\alst{l}êstjan sea an þesumu \alst{l}ande, \hld\ þe gi·\alst{l}íko duot &
\alst{w}ísumu manne, \hld\ þe gi·\alst{w}it havad, &
\alst{h}orska \alst{h}ugi-skęfti, \hld\ ęndi \alst{h}ús-stędi kiusid &
an \alst{f}astoro \alst{f}oldun \hld\ ęndi an \alst{f}elisa uppan &
\alst{w}égos \alst{w}irkid, \hld\ þár im \alst{w}ind ni mag, &
ne wág ne \alst{w}atares strôm \hld\ \alst{w}ihtju ge·tiunjan, &
ak mag im þár wið \alst{u}n-gi·widerjon \hld\ \alst{a}llun standan &
an þemu \alst{f}elise uppan, \hld\ hwand it só \alst{f}asto warð &
gi·\alst{st}ellit an þemu \alst{st}êne: \hld\ ant·havad it þiu \alst{st}ędi niðana, &
\alst{w}ręðid wiðar \alst{w}inde, \hld\ þat it \alst{w}íkan ni mag. &
Só duot eft \alst{m}anno só hwi-lik, \hld\ só þesun \alst{m}ínun ni wili &
\alst{l}êrun hôrjen \hld\ ne þero \alst{l}êstjen wiht; &
só duot þe \alst{u}n-wíson \hld\ \alst{e}rla ge·líko, &
un-ge·\alst{w}ittigon \alst{w}ere, \hld\ þe im be \alst{w}atares staðe &
an \alst{s}ande wili \hld\ \alst{s}ęli-hús wirkjan, &
þár it \alst{w}estrani \alst{w}ind \hld\ ęndi \alst{w}ágo strôm, &
\alst{s}êes u̇ðjon te·\alst{s}láad; \hld\ ne mag im \alst{s}and ęndi greot &
ge·\alst{w}ręðjan wið þemu \alst{w}inde, \hld\ ak wirðid te·\alst{w}orpan þan, &
te·\alst{f}allen an þemu \alst{f}lóde, \hld\ hwand it an \alst{f}astoro nis &
\alst{e}rðu ge·timbrod. \hld\ Só skal allaro \alst{e}rlo ge·hwes &
\alst{w}erk ge·þïhan \alst{w}iðar þiu, \hld\ þe hí þius mín \alst{w}ord frumid, &
\alst{h}aldid \alst{h}êlag ge·bod.“ \hld\ Þó bi·gunnun an iro \alst{h}ugi wundron &
\alst{m}ęgin-folk \alst{m}ikil: \hld\ ge·hôrdun \alst{m}ahtiges godes &
\alst{l}iof-líka \alst{l}êra; \hld\ ne wárun an þemu \alst{l}ande ge·wuno, &
þat sie eo fan \alst{s}u·likun êr \hld\ \alst{s}ęggjan ge·hôrdin &
\alst{w}ordun eþþo \alst{w}erkun. \hld\ Far·stódun \alst{w}íse man, &
þat hé só \alst{l}êrde, \hld\ \alst{l}iudjo drohtin, &
\alst{w}árun \alst{w}ordun, \hld\ só hé ge·\alst{w}ald habde, &
\alst{a}llun þem \alst{u}n-ge·líko, \hld\ þe þár an \alst{ê}r-dagun &
undar þem \alst{l}iud-skępja \hld\ \alst{l}êrjon wárun &
a·\alst{k}oran undar þemu \alst{k}unnje: \hld\ ne habdun þiu \alst{K}ristes word &
ge·\alst{m}akon mid \alst{m}annun, \hld\ þe hé far þero \alst{m}ęnigi sprak, &
ge·\alst{b}ôd uppan þemu \alst{b}erge.\eva

\bvb TODO.\evb\evg

\bvg\bva[22][1837]%
\hspace*{100pt} Hé im þó \alst{b}êðju be·falh &
te ge·\alst{s}ęggennja \hld\ \alst{s}ínom wordun, &
hwó man \alst{h}imil-ríki \hld\ ge·\alst{h}alon skoldi, &
\alst{w}íd-brêdan \alst{w}elan, \hld\ gia hé im ge·\alst{w}ald far·gaf, &
þat sie móstin \alst{h}êljan \hld\ \alst{h}alte ęndi blinde, &
\alst{l}iudjo \alst{l}éf-hêdi, \hld\ \alst{l}egar-będ manag, &
\alst{s}wára \alst{s}uhti, \hld\ giak hé im \alst{s}elvo ge·bôd, &
þat sie at ênigumu \alst{m}anne \hld\ \alst{m}éde ne námin, &
\alst{d}iurje mêðmos: \hld\ „ge·huggjad gi“, kwað hé, —„hwand iu is þiu \alst{d}ád kuman, &
þat ge·\alst{w}it ęndi þe \alst{w}ís-dóm, \hld\ ęndi iu þea ge·\alst{w}ald far·givid &
alloro \alst{f}iriho \alst{f}adar, \hld\ só gí sie ni þurvun mid ênigo \alst{f}eho kôpon, &
\alst{m}édjan mid ênigun \alst{m}êðmun,— \hld\ só wesat gí iro \alst{m}annun forð &
an iuwon \alst{h}ugi-skęftjun \hld\ \alst{h}elpono mildja, &
\alst{l}êrjad gí \alst{l}iudjo barn \hld\ \alst{l}ang-samna rád, &
\alst{f}ruma \alst{f}orð-wardes; \hld\ \alst{f}irin-werk lahad, &
\alst{s}wára \alst{s}undjon. \hld\ Ne látad iu \alst{s}ilọvar nek gold &
\alst{w}ihti þes \alst{w}irðig, \hld\ þat it eo an iuwa ge·\alst{w}ald kuma, &
\alst{f}agạra \alst{f}eho-skattos: \hld\ it ni mag iu te ênigoro \alst{f}rumu hwęrgin, &
\alst{w}erðan te ênigumu \alst{w}illjon. \hld\ Ne skulun gí ge·\alst{w}ádjas þan mêr &
\alst{e}rlos \alst{ê}gan, \hld\ b·útan só gí þan \alst{a}n hębbjan, &
\alst{g}umon te \alst{g}arẹwja, \hld\ þan gí \alst{g}angan skulun &
an þat gi·\alst{m}ang innan. \hld\ Neo gí umbi iuwan \alst{m}ęti ni sorgot, &
\alst{l}ęng umbi iuwa \alst{l}íf-nare, \hld\ hwand þene \alst{l}êrjand skulun &
\alst{f}ódjan þat \alst{f}olk-skępi: \hld\ þes sint þea \alst{f}ruma werða, &
\alst{l}eov-líkes \alst{l}ônes, \hld\ þe hí þem \alst{l}iudjun sagad. &
\alst{w}irðig is þe \alst{w}urhtjo, \hld\ þat man ina \alst{w}el fódja, &
þana \alst{m}an mid \alst{m}ósu, \hld\ þe só \alst{m}anagoro skal &
\alst{s}eola bi·\alst{s}organ \hld\ ęndi an þana \alst{s}ïð spanen, &
\alst{g}êstos an \alst{g}odes wang. \hld\ Þat is \alst{g}rôtara þing, &
þat man bi·\alst{s}orgon skal \hld\ \alst{s}eolun managa, &
hwó man þea ge·\alst{h}alde \hld\ te \alst{h}evan-ríkja, &
þan man þene \alst{l}ík-hamon \hld\ \alst{l}iudi-barno &
\alst{m}ósu bi·\alst{m}orna. \hld\ Be·þiu \alst{m}an skulun &
\alst{h}aldan þene \alst{h}old-líko, \hld\ þe im te \alst{h}evan-ríkja &
þene \alst{w}eg wísit \hld\ ęndi sie \alst{w}am-skaðun, &
\alst{f}eondun wit·\alst{f}áhit \hld\ ęndi \alst{f}irin-werk lahid, &
\alst{s}wára \alst{s}undjon. \hld\ Nú ik iu \alst{s}ęndjan skal &
aftar þesumu \alst{l}and-skępje \hld\ só \alst{l}amb undar wulvos: &
só skulun gí undar iuwa \alst{f}íund \alst{f}aren, \hld\ undar \alst{f}ilu þeodo, &
undar \alst{m}is-líke \alst{m}an. \hld\ Hębbjad iuwan \alst{m}ód wiðar þem &
só \alst{g}lawan te·\alst{g}ęgnes, \hld\ só samo só þe \alst{g}elwo wurm, &
\alst{n}ádra þiu fêha, \hld\ þár siu iro \alst{n}íð-skępjes, &
\alst{w}itodes \alst{w}ánit, \hld\ þat man iu undar þemu \alst{w}erode ne mugi &
be·\alst{s}wíkan an þemu \alst{s}ïðe. \hld\ Far þiu gí \alst{s}orgon skulun, &
þat iu þea \alst{m}an ni \alst{m}ugin \hld\ \alst{m}ód-ge·þȧhti, &
\alst{w}illjan a·\alst{w}ardjen. \hld\ Wesat iu so \alst{w}ara wiðar þiu, &
wið iro \alst{f}êknjon dádjun, \hld\ só man wiðar \alst{f}íundun skal. &
Þan wesat gí eft an iuwon \alst{d}ádjun \hld\ \alst{d}úvon ge·líka, &
hębbjad wið \alst{e}rlo ge·hwene \hld\ \alst{ê}n-faldan hugi, &
\alst{m}ildjan \alst{m}ód-sevon, \hld\ þat þár \alst{m}an neg·ên &
þurh iuwa \alst{d}ádi \hld\ be·\alst{d}rogan ne werðe, &
be·\alst{s}wikan þurh iuwa \alst{s}undja. \hld\ Nú skulun gí an þana \alst{s}ïð faran, &
an þat \alst{â}rundi: \hld\ þár skulun gí \alst{a}rvidjes só filu &
ge·\alst{þ}olon undar þeru \alst{þ}iod \hld\ ęndi ge·\alst{þ}wing só samo &
\alst{m}anag ęndi \alst{m}is-lík, \hld\ hwand gí an \alst{m}ínumu namon &
þea \alst{l}iudi \alst{l}êrjat. \hld\ Be·þiu skulun gí þár \alst{l}êðes filu &
fora \alst{w}er-old-kuningun, \hld\ \alst{w}ítjas ant·fȧhan. &
Oft skulun gí þár for \alst{r}íkja \hld\ þurh þius mín \alst{r}ehtun word &
ge·\alst{b}undane standen \hld\ ęndi \alst{b}êðju ge·þologjan, &
ge \alst{h}osk ge \alst{h}arm-kwidi: \hld\ umbi þat ne látad gí iuwan \alst{h}ugi twíflon, &
\alst{s}evon \alst{s}wíkandjan: \hld\ gí ni þurvun an ênigun \alst{s}orgun wesan &
an iuwomu \alst{h}ugi \alst{h}węrgin, \hld\ þan man iu for þea \alst{h}êri forð &
an þene \alst{g}ast-sęli \hld\ \alst{g}angan hêtid, &
hwat gí im þan te·\alst{g}ęgnes skulin \hld\ \alst{g}ódoro wordo, &
\alst{sp}áh-líkoro ge·\alst{sp}rekan, \hld\ hwand iu þiu \alst{sp}ód kumid, &
\alst{h}elpe fon \alst{h}imile, \hld\ ęndi sprikid þe \alst{h}êlogo gêst, &
\alst{m}ahtig fon iuwomu \alst{m}unde. \hld\ Be·þiu ne an·drádad gí iu þero \alst{m}anno níð &
ne \alst{f}orhtjat iro \alst{f}íund-skępi: \hld\ þoh sie hębbjan iuwas \alst{f}erạhes ge·wald, &
þat sie mugin þene \alst{l}ík-hamon \hld\ \alst{l}ívu be·neotan, &
a·\alst{s}lahan mid \alst{s}werde, \hld\ þoh sie þeru \alst{s}eolun ne mugun &
\alst{w}iht a·\alst{w}ardjan. \hld\ Ant·drádad iu \alst{w}aldand god, &
\alst{f}orhtjad \alst{f}ader iuwan, \hld\ \alst{f}rummjad gerno &
is ge·\alst{b}od-skępi, \hld\ hwand hí havad \alst{b}êðjes gi·wald, &
\alst{l}iudjo \alst{l}íves \hld\ ęndi ôk iro \alst{l}ík-hamon &
gek þero \alst{s}eolon só \alst{s}elf: \hld\ ef gí iuwa an þem \alst{s}ïðe þarod &
far·\alst{l}iosat þurh þesa \alst{l}êra, \hld\ þan mótun gí sie eft an þemu \alst{l}iohte godes &
be·\alst{f}oran \alst{f}ïðan, \hld\ hwand sie \alst{f}ader iuwa, &
\alst{h}aldid \alst{h}êlag god \hld\ an \alst{h}imil-ríkja.\eva

\bvb TODO.\evb\evg

\bvg\bva[23][1915]%
Ne kumat þea alle te \alst{h}imile, \hld\ þea þe hér \alst{h}rópat te mí &
\alst{m}anno te \alst{m}und-burd. \hld\ \alst{M}anaga sind þero, &
þea willjad alloro \alst{d}ago ge·hwi-likes \hld\ te \alst{d}rohtine hnígan, &
\alst{h}rópad þár te \alst{h}elpu \hld\ ęndi \alst{h}uggjad an ȯðar, &
\alst{w}irkjad \alst{w}am-dádi: \hld\ ne sind im þan þiu \alst{w}ord fruma, &
ak þea mótun \alst{h}wervan \hld\ an þat \alst{h}imiles lioht, &
\alst{g}angan an þat \alst{g}odes ríki, \hld\ þea þes \alst{g}erne sint, &
þat sie hír ge·\alst{f}rummjen \hld\ \alst{f}ader ala-waldan &
\alst{w}erk ęndi \alst{w}illjon. \hld\ Þea ni þurvun mid \alst{w}ordun só fílu &
\alst{h}rópan te \alst{h}elpu, \hld\ hwanda þe \alst{h}êlogo god &
wêt alloro \alst{m}anno ge·hwes \hld\ \alst{m}ód-ge·þȧhti, &
\alst{w}ord ęndi \alst{w}illjon, \hld\ ęndi gildid im is \alst{w}erko lôn. &
Be·þiu skulun gí \alst{s}orgon, \hld\ þan gí an þene \alst{s}ïð farad, &
hwó gí þat \alst{â}rundi \hld\ ti \alst{ę}ndja be·brengen. &
Þan gí \alst{l}íðan skulun \hld\ aftar þesumu \alst{l}and-skępja, &
\alst{w}ído aftar þesoro \alst{w}er-oldi, \hld\ al só iu \alst{w}egos lêdjad, &
\alst{b}rêd stráta te \alst{b}urg, \hld\ simbla sókjad gí iu þene \alst{b}ętston sán &
\alst{m}an undar þeru \alst{m}ęnegi \hld\ ęndi ku̇ðjad imu iuwan \alst{m}óð-sevon &
\alst{w}árun \alst{w}ordun. \hld\ Ef sie þan þes \alst{w}irðige sint, &
þat sie iuwa \alst{g}ódun werk \hld\ \alst{g}erno ge·lêstjen &
mid \alst{h}luttru \alst{h}ugi, \hld\ þan gí an þemu \alst{h}úse mid im &
\alst{w}onod an \alst{w}illjon \hld\ ęndi im \alst{w}el lônod, &
\alst{g}eldad im mid \alst{g}ódu \hld\ ęndi sie te \alst{g}ode selvon &
\alst{w}ordun ge·\alst{w}íhad \hld\ ęndi sęggjad im \alst{w}issan friðu, &
\alst{h}êlaga \alst{h}elpa \hld\ \alst{h}evan-kuninges. &
Ef sie þan só \alst{s}áliga \hld\ þurh iro \alst{s}elvoro dád &
\alst{w}erðan ni mótun, \hld\ þat sie iuwa \alst{w}erk frummjen, &
\alst{l}êstjen iuwa \alst{l}êra, \hld\ þan gí fan þem \alst{l}iudjun sán, &
\alst{f}arad fan þemu \alst{f}olke, \hld\ —þe iuwa \alst{f}riðu hwirvid &
eft an iuworo \alst{s}elvoro \alst{s}ïð,— \hld\ ęndi látad sie mid \alst{s}undjun forð, &
mid \alst{b}alu-werkun \alst{b}úan \hld\ ęndi sókjad iu \alst{b}urg ȯðra, &
\alst{m}ikil \alst{m}an-werod, \hld\ ęndi ne látad þes \alst{m}elmes wiht &
\alst{f}olgan an iuwom \alst{f}ótun, \hld\ þanan þe man iu ant·\alst{f}ȧhan ne wili, &
ak \alst{sk}uddjat it fan iuwon \alst{sk}óhun, \hld\ þat it im eft te \alst{sk}amu werðe, &
þemu \alst{w}erode te ge·\alst{w}it-skępje, \hld\ þat iro \alst{w}illjo ne dôg. &
Þan sęggjo ik iu te \alst{w}árun, \hld\ só hwan só þius \alst{w}er-old ęndjad &
ęndi þe \alst{m}árjo dag \hld\ ovar \alst{m}an farid, &
þat þan \alst{S}odomo-burg, \hld\ þiu hír þurh \alst{s}undjon warð &
an \alst{a}f-grundi \hld\ \alst{ê}ldes kraftu, &
\alst{f}iuru bi·\alst{f}allen, \hld\ þat þiu þan havad \alst{f}riðu méran, &
\alst{m}ildiran \alst{m}und-burd, \hld\ þan þea \alst{m}an êgin, &
þe iu hír \alst{w}iðar-\alst{w}erpat \hld\ ęndi ne willjad iuwa \alst{w}ord frummjen. &
Só hwe só iu þan ant·\alst{f}áhit \hld\ þurh \alst{f}erhtan hugi, &
þurh \alst{m}ildjan \alst{m}ód, \hld\ só havad \alst{m}ínan forð &
\alst{w}illjon ge·\alst{w}arhten \hld\ ęndi ôk \alst{w}aldand god, &
ant·\alst{f}angan \alst{f}ader iuwan, \hld\ \alst{f}iriho drohtin, &
\alst{r}íkjan \alst{r}ád-gevon, \hld\ þene þe al \alst{r}eht bi·kan. &
\alst{w}êt \alst{w}aldand self, \hld\ ęndi \alst{w}illjan lônot &
\alst{g}umono ge·hwi-likumu, \hld\ só hwat só hí hír \alst{g}ódes ge·duot, &
þoh hí þurh \alst{m}innja godes \hld\ \alst{m}anno hwi-likumu &
\alst{w}illjandi far·geve \hld\ \alst{w}atares drinkan, &
þat hí \alst{þ}urftigumu manne \hld\ \alst{þ}urst ge·hêlje, &
\alst{k}aldes brunnan. \hld\ Þesa \alst{k}widi werðad wára, &
þat eo ne bi·\alst{l}ívid, \hld\ ne hí þes \alst{l}ôn skuli, &
fora \alst{g}odes ôgun \hld\ \alst{g}eld ant·fȧhan, &
\alst{m}éda \alst{m}anag-falde, \hld\ só hwat só hí is þurh mína \alst{m}innja ge·duot. &
Só hwe só mín þan far·\alst{l}ôgnid \hld\ \alst{l}iudi-barno, &
\alst{h}ęliðo for þesoro \alst{h}ęrju, \hld\ só dóm ik is an \alst{h}imile só self &
þár \alst{u}ppe far þem \alst{a}lo-waldan fader \hld\ ęndi for allumu is \alst{ę}ngilo krafte, &
far þeru \alst{m}ikilon \alst{m}ęnigi. \hld\ Só hwi-lik só þan eft \alst{m}anno barno &
an þesoro \alst{w}er-oldi ne wili \hld\ \alst{w}ordun míðan, &
ak \alst{g}ihit far \alst{g}um-skępi, \hld\ þat hé mín \alst{j}ungoro sí, &
þene willju ek \alst{e}ft \alst{ó}gjan \hld\ far \alst{ô}gun godes, &
fora alloro \alst{f}iriho \alst{f}ader, \hld\ þár \alst{f}olk manag &
for þene \alst{a}lo-waldon \hld\ \alst{a}lla gangad &
\alst{r}eðinon wið þene \alst{r}íkjon. \hld\ Þár willju ik imu an \alst{r}eht wesan &
\alst{m}ildi \alst{m}und-boro, \hld\ só hwemu só \alst{m}ínun hír &
\alst{w}ordun hôrid \hld\ ęndi þiu \alst{w}erk frumid, &
þea ik hír an þesumu \alst{b}erge uppan \hld\ ge·\alst{b}oden hębbju.“ &
Habda þó te \alst{w}árun \hld\ \alst{w}aldandes sunu &
ge·\alst{l}êrid þea \alst{l}iudi, \hld\ hwó sie \alst{l}of gode &
\alst{w}irkjan skoldin. \hld\ Þó lét hí þat \alst{w}erod þanan &
an alloro \alst{h}alva ge·hwi-lika, \hld\ \alst{h}ęri-skępi manno &
\alst{s}ïðon te \alst{s}ęlðon. \hld\ Habdun \alst{s}elves word, &
ge·\alst{h}ôrid \alst{h}evan-kuninges \hld\ \alst{h}êlaga lêra, &
só eo te \alst{w}er-oldi sint \hld\ \alst{w}ordo ęndi dádjo, &
\alst{m}an-kunnjes \alst{m}anag \hld\ ovar þesan \alst{m}iddil-gard &
\alst{sp}rákono þiu \alst{sp}áhiron, \hld\ só hwe só þiu \alst{sp}el ge·frang, &
þea þár an þemu \alst{b}erge ge·sprak \hld\ \alst{b}arno ríkjast.\eva

\bvb TODO.\evb\evg

\bvg\bva[24][1994]%
Ge·wêt imu þó umbi \alst{þ}rea naht aftar þiu \hld\ þesoro \alst{þ}iodo drohtin &
an \alst{G}alileo land, \hld\ þár hé te ênum \alst{g}ômum warð, &
ge·\alst{b}edan þat \alst{b}arn godes: \hld\ þár skolda man êna \alst{b}rúd gevan, &
\alst{m}una-líka \alst{m}agað. \hld\ Þár \alst{M}aria was, &
mid iro \alst{s}uni \alst{s}elvo, \hld\ \alst{s}álig þiorna, &
\alst{m}ahtiges \alst{m}óder. \hld\ \alst{M}anagoro drohtin &
\alst{g}éng imu þó mid is \alst{j}ungoron, \hld\ \alst{g}odes êgan barn, &
an þat \alst{h}ôha \alst{h}ús, \hld\ þár þe \alst{h}ęri drank, &
þea \alst{J}udeon an þemu \alst{g}ast-sęli: \hld\ hé im ôk at þem \alst{g}ômun was, &
giak hí þár ge·\alst{k}u̇ðde, \hld\ þat hí habda \alst{k}raft godes, &
\alst{h}elpa fan \alst{h}imil-fader, \hld\ \alst{h}êlagna gêst, &
\alst{w}aldandes \alst{w}ís-dóm. \hld\ \alst{W}erod blíðode, &
wárun þár an \alst{l}uston \hld\ \alst{l}iudi at·samne, &
\alst{g}umon \alst{g}lad-módje. \hld\ \alst{G}éngun ambaht-man, &
\alst{sk}ęnkjon mid \alst{sk}álun, \hld\ drógun \alst{sk}írjane wín &
mid \alst{o}rkun ęndi mid \alst{a}lo-fatun; \hld\ was þár \alst{e}rlo drôm &
\alst{f}agạr an \alst{f}lęttja, \hld\ þó þár \alst{f}olk undar im &
an þem \alst{b}ęnkjon só \alst{b}ętst \hld\ \alst{b}líðsja af·hóvun, &
\alst{w}árun þár an \alst{w}unnjun. \hld\ Þó im þes \alst{w}ínes brast, &
þem \alst{l}iudjun þes \alst{l}íðes: \hld\ is ni was far·\alst{l}êvid wiht &
\alst{h}węrgin an þemu \alst{h}úse, \hld\ þat for þene \alst{h}ęri forð &
\alst{sk}ęnkjon drógin, \hld\ ak þiu \alst{sk}apu wárun &
\alst{l}íðes a·\alst{l}árid. \hld\ Þó ni was \alst{l}ang te þiu, &
þat it sán ant·\alst{f}unda \hld\ \alst{f}río skônjosta, &
\alst{K}ristes móder: \hld\ géng wið iro \alst{k}ind sprekan, &
wið iro \alst{s}unu \alst{s}elvon, \hld\ \alst{s}agda im mid wordun, &
þat þea \alst{w}erdos þó mêr \hld\ \alst{w}ínes ne habdun &
þem \alst{g}ęstjun te \alst{g}ômun. \hld\ Siu þó \alst{g}erno bad, &
þat is þe \alst{h}êlogo Krist \hld\ \alst{h}elpa ge·riedi &
þemu \alst{w}erode te \alst{w}illjon. \hld\ Þó habda eft is \alst{w}ord garu &
\alst{m}ahtig barn godes \hld\ ęndi wið is \alst{m}óder sprak: &
„Hwat ist \alst{m}í ęndi þí“, \hld[kwað hé,] „umbi þesoro \alst{m}anno lið, &
umbi þeses \alst{w}erodes \alst{w}ín? \hld\ Te hwí sprikis þú þes, \alst{w}íf, só filu, &
\alst{m}anos mí far þesoro \alst{m}ęnigi? \hld\ Ne sint \alst{m}ína noh &
\alst{t}ídi kumana.“ \hld\ Þan þoh gi·\alst{t}rúoda siu wel &
an iro \alst{h}ugi-skęftjun, \hld\ \alst{h}êlag þiorne, &
þat is aftar þem \alst{w}ordun \hld\ \alst{w}aldandes barn, &
\alst{h}êljandoro bętst \hld\ \alst{h}elpan weldi. &
Hét þó þea \alst{a}mbaht-man \hld\ \alst{i}diso skônjost, &
\alst{sk}ęnkjon ęndi \alst{sk}ap-wardos, \hld\ þea þár skoldun þero \alst{sk}olu þionon, &
þat sie þes ne \alst{w}ord ne \alst{w}erk \hld\ \alst{w}iht ne far·létin, &
þes sie þe \alst{h}êlogo Krist \hld\ \alst{h}êtan weldi &
\alst{l}êstjan far þem \alst{l}iudjun. \hld\ \alst{L}árja stódun þár &
\alst{st}ên-fatu sehsi. \hld\ Þó só \alst{st}illo ge·bôd &
\alst{m}ahtig barn godes, \hld\ só it þár \alst{m}anno filu &
ne \alst{w}issa te \alst{w}árun, \hld\ hwó hé it mid is \alst{w}ordu ge·sprak; &
hé hét þea \alst{sk}ęnkjon \hld\ þó \alst{sk}írjas watares &
þiu \alst{f}atu \alst{f}ulljen, \hld\ ęndi hí þár mid is \alst{f}ingrun þó, &
\alst{s}egnade \alst{s}elvo \hld\ \alst{s}ínun handun, &
\alst{w}arhte it te \alst{w}íne \hld\ ęndi hét is an ên \alst{w}êgi hlaðen, &
\alst{sk}ęppjen mid ênoro \alst{sk}álon, \hld\ ęndi þó te þem \alst{sk}ęnkjon sprak, &
hét is þero \alst{g}ęstjo, \hld\ þe at þem \alst{g}ômun was &
þemu \alst{h}êroston \hld\ an \alst{h}and gevan, &
\alst{f}ul mid \alst{f}olmun, \hld\ þemu þe þes \alst{f}olkes þár &
ge·\alst{w}eld aftar þemu \alst{w}erde. \hld\ Reht só hí þes \alst{w}ínes ge·drank, &
só ni \alst{m}ahte hé be·\alst{m}íðan, \hld\ ne hí far þeru \alst{m}ęnigi sprak &
te þemu \alst{b}rúdi-gumon, \hld\ kwað þat simbla þat \alst{b}ętste líð &
\alst{a}lloro \alst{e}rlo ge·hwi-lik \hld\ \alst{ê}rist skoldi &
\alst{g}evan at is \alst{g}ômun: \hld\ „undar þiu wirðid þero \alst{g}umono hugi &
a·\alst{w}ękid mid \alst{w}ínu, \hld\ þat sie \alst{w}el blíðod, &
\alst{d}runkan \alst{d}rômjad. \hld\ Þan mag man þár \alst{d}ragan aftar þiu &
\alst{l}íht-\alst{l}íkora \alst{l}íð: \hld\ só ist þesoro \alst{l}iudjo þau. &
Þan havas þú nú \alst{w}undẹr-líko \hld\ \alst{w}erd-skępi þínan &
ge·\alst{m}arkod far þesoro \alst{m}ęnigi: \hld\ hétis far þit \alst{m}anno folk &
alles þínes \alst{w}ínes \hld\ þat \alst{w}irsiste &
þíne \alst{a}mbaht-man \hld\ \alst{ê}rist brengjan, &
\alst{g}evan at þínun \alst{g}ômun. \hld\ Nú sint þína \alst{g}ęsti sade, &
sint þíne \alst{d}ruhtingos \hld\ \alst{d}runkane swíðo, &
is þit \alst{f}olk \alst{f}rô-mód: \hld\ nú hétis þú hír \alst{f}orð dragan &
alloro \alst{l}íðo \alst{l}of-samost, \hld\ þero þe ik eo an þesumu \alst{l}iohte ge·sah &
\alst{h}węrgin \alst{h}ębbjan. \hld\ Mid þius skoldis þú u̇s \alst{h}in-dag êr &
\alst{g}evon ęndi \alst{g}ômjan: \hld\ þan it alloro \alst{g}umono ge·hwi-lik &
ge·\alst{þ}igedi te \alst{þ}anke.“ \hld\ Þó warð þár \alst{þ}egạn manag &
ge·\alst{w}ar aftar þem \alst{w}ordun, \hld\ sïðor sie þes \alst{w}ínes ge·drunkun, &
þat þár þe \alst{h}êlogo Krist \hld\ an þemu \alst{h}úse innan &
\alst{t}êkạn warhte: \hld\ \alst{t}rúodun sie sïðor &
þiu \alst{m}êr an is \alst{m}und-burd, \hld\ þat hí habdi \alst{m}aht godes, &
ge·\alst{w}ald an þesoro \alst{w}er-oldi. \hld\ Þó warð þat só \alst{w}ído ku̇ð &
ovar \alst{G}alileo land \hld\ \alst{J}udeo liudjun, &
hwó þár \alst{s}elvo ge·deda \hld\ \alst{s}unu drohtines &
\alst{w}ater te \alst{w}íne: \hld\ þat warð þár \alst{w}undro êrist, &
þero þe hí þár an \alst{G}alilea \hld\ \alst{J}udeo liudjon, &
\alst{t}êkno ge·\alst{t}ôgdi. \hld\ Ne mag þat ge·\alst{t}ęlljan man, &
ge·\alst{s}ęggjan te \alst{s}ȯðan, \hld\ hwat þár \alst{s}ïðor warð &
\alst{w}undres undar þemu \alst{w}erode, \hld\ þár \alst{w}aldand Krist &
an \alst{g}odes namon \hld\ \alst{J}udeo liudjon &
allan \alst{l}angan dag \hld\ \alst{l}êra sagde, &
gi·\alst{h}ét im \alst{h}evan-ríki \hld\ ęndi \alst{h}ęlljo ge·þwing &
\alst{w}ęride mid \alst{w}ordun, \hld\ hét sie \alst{w}ara godes, &
\alst{s}in-líf \alst{s}ókjan: \hld\ þár is \alst{s}eolono lioht, &
\alst{d}rôm \alst{d}rohtines \hld\ ęndi \alst{d}ag-skímon, &
\alst{g}ód-lík-nissja \alst{g}odes; \hld\ þár \alst{g}êst manag &
\alst{w}unod an \alst{w}illjan, \hld\ þe hír \alst{w}el þęnkid, &
þat hé \alst{h}ír bi·\alst{h}alde \hld\ \alst{h}evan-kuninges ge·bod.\eva

\bvb TODO.\evb\evg

\bvg\bva[25][2088]%
Ge·wêt imu þó mid is \alst{j}ungoron \hld\ fan þem \alst{g}ômun forð &%TODO: check gômun.
\alst{K}ristus te \alst{K}apharnaum, \hld\ \alst{k}uningo ríkjost, &
te þeru \alst{m}árjon burg. \hld\ \alst{M}ęgin samnode, &
\alst{g}umon imu te·\alst{g}ęgnes, \hld\ \alst{g}ódoro manno &
\alst{s}álig ge·\alst{s}ïði: \hld\ weldun þiu is \alst{s}wótjan word &
\alst{h}êlag \alst{h}ôrjen. \hld\ Þár im ên \alst{h}unno kwam, &
ên \alst{g}ód man an·\alst{g}ęgin \hld\ ęndi ina \alst{g}erno bad &
\alst{h}elpan \alst{h}êlagne, \hld\ kwað þat hí undar is \alst{h}íwiskja &
ênna \alst{l}efna \alst{l}amon \hld\ \alst{l}ango habdi, &
\alst{s}eokan an is \alst{s}ęlðon: \hld\ „só ina ênig \alst{s}ęggjo ne mag &
\alst{h}andun ge·\alst{h}êljen. \hld\ Nú is im þínoro \alst{h}elpono þarf, &
\alst{f}rô mín þe gódo.“ \hld\ Þó sprak im eft þat \alst{f}riðu-barn godes &
\alst{s}án aftar þiu \hld\ \alst{s}elvo te·gęgnes, &
kwað þat hé þár \alst{k}wámi \hld\ ęndi þat \alst{k}ind weldi &
\alst{n}ęrjan af þeru \alst{n}ôdi. \hld\ Þó im \alst{n}áhor géng &
þe man far þeru \alst{m}ęnigi \hld\ wið só \alst{m}ahtigna &
\alst{w}ordun \alst{w}ehslan: \hld\ „ik þes \alst{w}irðig ne bium,“ kwað hé, &
„\alst{h}êrro þe gódo, \hld\ þat þú an mín \alst{h}ús kumes, &
\alst{s}ókjas mína \alst{s}ęliða, \hld\ hwand ik bium só \alst{s}undig man &
mid \alst{w}ordun ęndi mid \alst{w}erkun. \hld\ Ik ge·lôvju þat þú ge·\alst{w}ald havas, &
þat þú ina \alst{h}inana maht \hld\ \alst{h}êlan ge·wirkjan, &
\alst{w}aldand frô mín: \hld\ ef þú it mid þínun \alst{w}ordun ge·sprikis, &
þan is sán þiu \alst{l}éf-hêd \alst{l}ôsot \hld\ ęndi wirðid is \alst{l}ík-hamo &
\alst{h}êl ęndi \alst{h}rêni, \hld\ ef þú im þína \alst{h}elpa far·givis. &
Ik bium mí \alst{a}mbaht-man, \hld\ hębbju mí \alst{ô}des ge·nóg, &
\alst{w}elono ge·\alst{w}unnen: \hld\ þoh ik undar ge·\alst{w}ęldi sí &
\alst{a}ðal-kuninges, \hld\ þoh hębbju ik \alst{e}rlo ge·trôst, &
\alst{h}olde \alst{h}ęri-rinkos, \hld\ þea mí só ge·\alst{h}ôriga sint, &
þat sie þes ne \alst{w}ord ne \alst{w}erk \hld\ \alst{w}iht ne far·látad, &
þes ik sie an þesumu \alst{l}and-skępje \hld\ \alst{l}êstjan héte, &
ak sie \alst{f}arad ęndi \alst{f}rummjad \hld\ ęndi eft te iro \alst{f}rôhan kumad, &
\alst{h}olde te iro \alst{h}êrron. \hld\ Þoh ik at mínumu \alst{h}ús êgi &
\alst{w}íd-brêdene \alst{w}elon \hld\ ęndi \alst{w}erodes ge·nóg, &
\alst{h}ęliðos \alst{h}ugi-dęrvje, \hld\ þoh ni gi·dar ik þi só \alst{h}êlagna &
\alst{b}iddjen, \alst{b}arn godes, \hld\ þat þú an mín \alst{b}ú gangas, &
\alst{s}ókjas mína \alst{s}ęliða, \hld\ hwand ik só \alst{s}undig bium, &
\alst{w}êt mína far·\alst{w}urhti.“ \hld\ Þó sprak eft \alst{w}aldand Krist, &
þe \alst{g}umo wið is \alst{j}ungoron, \hld\ kwað þat hí an \alst{J}udeon hwęrgin &
\alst{u}ndar \alst{I}sraheles \hld\ \alst{a}voron ne fundi &
ge·\alst{m}akon þes \alst{m}annes, \hld\ þe io \alst{m}êr te gode &
an þemu \alst{l}and-skępi \hld\ ge·\alst{l}ôvon habdi, &
þan \alst{h}luttron te \alst{h}imile: \hld\ „nú látu ik iu þár \alst{h}ôrjen tó, &
þár ik it iu te \alst{w}árun hír \hld\ \alst{w}ordun sęggjo, &
þat noh skulun \alst{ę}li-þeoda \hld\ \alst{ô}stane ęndi westane, &
\alst{m}an-kunnjes kuman \hld\ \alst{m}anag te·samne, &
\alst{h}êlag folk godes \hld\ an \alst{h}evan-ríki: &
þea motun þár an \alst{A}brahames \hld\ ęndi an \alst{I}saakes só self &
ęndi ôk an \alst{J}akobes, \hld\ \alst{g}ódoro manno, &
\alst{b}armun restjen \hld\ ęndi \alst{b}êðju ge·þologjan, &
\alst{w}elon ęndi \alst{w}illjon \hld\ ęndi \alst{w}onod-sam líf, &
\alst{g}ód lioht mid \alst{g}ode. \hld\ Þan skal \alst{J}udeono filu, &
þeses \alst{r}íkjas suni \hld\ be·\alst{r}ôvode werðen, &
be·\alst{d}êlide su·likoro \alst{d}iurðo, \hld\ ęndi skulun an \alst{d}alun þiustron &
an þemu alloro \alst{f}erristan \hld\ \alst{f}erne liggen. &
Þár mag man ge·\alst{h}ôrjen \hld\ \alst{h}ęliðos kwíðjan, &
þár sie iro \alst{t}orn manag \hld\ \alst{t}andon bítad; &
þár ist \alst{g}rist-grimmo \hld\ ęndi \alst{g}rádag fiur, &
\alst{h}ard \alst{h}ęlljo ge·þwing, \hld\ \alst{h}êt ęndi þiustri, &
\alst{s}wart \alst{s}in-nahti \hld\ \alst{s}undja te lône, &
\alst{w}rêðoro ge·\alst{w}urhtjo, \hld\ só hwemu só þes \alst{w}illjon ne havad, &
þat hé ina a·\alst{l}ôsje, \hld\ êr hí þit \alst{l}ioht a·geve, &
\alst{w}ęndje fan þesoro \alst{w}er-oldi. \hld\ Nú maht þú þi an þínan \alst{w}illjon forð &
\alst{s}ïðon te \alst{s}ęlðun; \hld\ þan findis þú ge·\alst{s}undan at hús &
\alst{m}ago-jungan \alst{m}an: \hld\ \alst{m}ód is imu an luston, &
þat \alst{b}arn is ge·hêlid, \hld\ só þú \alst{b}édi te mí: &
it wirðid al só ge·\alst{l}êstid, \hld\ só þú ge·\alst{l}ôvon havas &
an þínumu \alst{h}ugi \alst{h}ardo.“ \hld\ Þó sagde \alst{h}evan-kuninge, &
þe \alst{a}mbaht-man \hld\ \alst{a}lo-waldon gode &
\alst{þ}ank for þero \alst{þ}iodo, \hld\ þes hé imu at su·likun \alst{þ}arvun halp. &
Habda þo gi·\alst{â}rundid, \hld\ \alst{a}l só hé welde, &
\alst{s}álig-líko: \hld\ gi·wêt imu an þana \alst{s}ïð þanan, &
\alst{w}ende an is \alst{w}illjan, \hld\ þár hé \alst{w}elon êhte, &
\alst{b}ú ęndi \alst{b}odlos: \hld\ fand þat \alst{b}arn ge·sund, &
\alst{k}ind-jungan man. \hld\ \alst{K}ristes wárun þó &
\alst{w}ord ge·fullot: \hld\ hí ge·\alst{w}ald habda &
te \alst{t}ôgjanna \alst{t}êkạn, \hld\ só þat ni mag gi·\alst{t}ęlljen man, &
ge·\alst{a}hton ovar þesoro \alst{e}rðu, \hld\ hwat hé þurh is \alst{ê}nes kraft &
an þesaro \alst{m}iddil-gard \hld\ \alst{m}áriða ge·frumide, &
\alst{w}undres ge·\alst{w}arhte, \hld\ hwand al an is ge·\alst{w}ęldi stád, &
\alst{h}imil ęndi erðe.\eva

\bvb TODO.\evb\evg

\bvg\bva[26][2167]%
\hspace*{100pt} Þó ge·wêt imu þe \alst{h}êlogo Krist &%NOTE: In cæsura.
\alst{f}orð-wardes \alst{f}aren, \hld\ \alst{f}ręmide alo-mahtig &
alloro \alst{d}ago ge·hwi-likes, \hld\ \alst{d}rohtin þe gódo, &
\alst{l}iudjo barnum \alst{l}eof, \hld\ \alst{l}êrde mid wordun &
\alst{g}odes willjon \alst{g}umun, \hld\ habda imu \alst{j}ungorono filu &
\alst{s}imbla te gi·\alst{s}ïðun, \hld\ \alst{s}álig folk godes, &
\alst{m}anno \alst{m}ęgin-kraft, \hld\ \alst{m}anagoro þeodo, &
\alst{h}êlag \alst{h}ęri-skępi, \hld\ was is \alst{h}elpono gód, &
\alst{m}annun \alst{m}ildi. \hld\ Þó hí mid þeru \alst{m}ęnigi kwam, &
mid þiu \alst{b}rahtmu þat \alst{b}arn godes \hld\ te \alst{b}urg þeru hôhon, &
þe \alst{n}ęrjendo te \alst{N}aim: \hld\ þár skolde is \alst{n}amo werðen &
\alst{m}annun ge·\alst{m}árid. \hld\ Þó géng \alst{m}ahtig tó &
\alst{n}ęrjendo Krist, \hld\ ant-tat hé gi·\alst{n}áhid was, &
\alst{h}êljandero bętst: \hld\ þó sáhun sie þár ên \alst{h}rêo dragan, &
ênan \alst{l}íf-lôsan \alst{l}ík-hamon \hld\ þea \alst{l}iudi fórjen, &
\alst{b}eran an ênaru \alst{b}áru \hld\ út at þera \alst{b}urges dore, &
\alst{m}agu-jungan \alst{m}an. \hld\ Þiu \alst{m}óder aftar géng &
an iro \alst{h}ugi \alst{h}riuwig \hld\ ęndi \alst{h}andun slóg, &
\alst{k}arode ęndi \alst{k}úmde \hld\ iro \alst{k}indes dôð, &
\alst{i}dis \alst{a}rm-skapan; \hld\ it was ira \alst{ê}nag barn: &
siu was iru \alst{w}idowa, \hld\ ne habda \alst{w}unnja þan mêr, &
bi·úten te þemu \alst{ê}nagun sunje \hld\ \alst{a}l ge·láten &
\alst{w}unnja ęndi \alst{w}illjan, \hld\ ant-tat ina iru \alst{w}urd be·nam, &
\alst{m}ári \alst{m}etodo-ge·skapu. \hld\ \alst{M}ęgin folgode, &
\alst{b}urg-liudjo ge·\alst{b}rak, \hld\ þár man ina an \alst{b}áru dróg, &
\alst{j}ungan man te \alst{g}rave. \hld\ Þár warð imu þe \alst{g}odes sunu, &
\alst{m}ahtig \alst{m}ildi \hld\ ęndi te þeru \alst{m}óder sprak, &
hét þat þiu \alst{w}idowa \hld\ \alst{w}óp far·léti, &
\alst{k}ara aftar þemu \alst{k}inde: \hld\ „þú skalt hír \alst{k}raft sehan, &
\alst{w}aldandes gi·\alst{w}erk: \hld\ þi skal hír \alst{w}illjo ge·standen, &
\alst{f}rófra far þesumu \alst{f}olke: \hld\ ne þarft þú \alst{f}erạh karon &
\alst{b}arnes þínes.“ \hld\ *Þuo hie ti þero \alst{b}áron géng &
iak hie ina \alst{s}elvo ant·hrên, \hld\ \alst{s}uno drohtines, &
\alst{h}êlagon \alst{h}andon, \hld\ ęndi ti þem \alst{h}ęliðe sprak, &
hiet ina só \alst{a}la-jungan \hld\ \alst{u}p a·standan, &
a·\alst{r}ísan fan þeru \alst{r}estun. \hld\ Þie \alst{r}ink up a·sat, &
þat \alst{b}arn an þero \alst{b}árun: \hld\ warð im eft an is \alst{b}riost kuman &
þie \alst{g}êst þuru \alst{g}odes kraft, \hld\ ęndi hie te·\alst{g}ęgnes sprak, &
þe \alst{m}an wið is \alst{m}ágos. \hld\ Þuo ina eft þero \alst{m}uoder bi·falạh &
\alst{h}êlandi Krist an \alst{h}and: \hld\ \alst{h}ugi warð iro te frovra, &
þes \alst{w}íves an \alst{w}unnjon, \hld\ hwand iro þár su·lik \alst{w}illjo gi·stuod. &
\alst{F}éll siu þó te \alst{f}uotun Kristes \hld\ ęndi þena \alst{f}olko drohtin &
\alst{l}ovoda for þero \alst{l}iudjo męnigi, \hld\ hwand hie iro at só \alst{l}iobes ferạhe &
\alst{m}undoda wiðẹr \alst{m}etodi-gi·skęftje: \hld\ far·stuod siu þat hie was þie \alst{m}ahtigo drohtin, &
þie \alst{h}êlago, þie \alst{h}imiles gi·waldid, \hld\ ęndi þat hie mahti gi·\alst{h}elpan managon, &
\alst{a}llon \alst{i}rmin-þiedon. \hld\ Þuo bi·gunnun þat \alst{a}hton managa, &
þat \alst{w}undẹr, þat under þem \alst{w}eroda gi·burida, \hld\ kwáðun þat \alst{w}aldand selvo, &
\alst{m}ahtig kwámi þarod is \alst{m}ęnigi wíson, \hld\ ęndi þat hie im só \alst{m}árjan sandi &
\alst{w}ár-sagon an þero \alst{w}er-oldes ríki, \hld\ þie im þár su·likan \alst{w}illjon frumidi. &
warð þár þuo \alst{e}rl manag \hld\ \alst{ę}gison bi·fangan, &
þat \alst{f}olk warð an \alst{f}orọhton: \hld\ gi·sáhun þena is \alst{f}erạh êgan, &
\alst{d}ages lioht sehan, \hld\ þena þe êr \alst{d}ôð for·nam, &
an \alst{s}uht-będdjon \alst{s}walt: \hld\ þuo was im eft gi·\alst{s}und after þiu, &
\alst{k}ind-jung a·\alst{k}wikot. \hld\ Þuo warð þat \alst{k}u̇ð obar all &
\alst{a}varon \alst{I}sraheles. \hld\ Reht só þuo \alst{á}vand kwam, &
\alst{s}ó warð þár all gi·\alst{s}amnod \hld\ \alst{s}eokora manno, &
\alst{h}altaro ęndi \alst{h}ávaro, \hld\ só hwat só þár \alst{h}węrgin was, &
þia \alst{l}évun under þem \alst{l}iudjon, \hld\ ęndi wurðun þár gi·\alst{l}êdit tuo, &
\alst{k}umana te \alst{K}riste, \hld\ þár hie im þuru is \alst{k}raft mikil &
\alst{h}alp ęndi sie \alst{h}êlda, \hld\ ęndi liet sia eft gi·\alst{h}aldana þanan &
\alst{w}endan an iro \alst{w}illjon. \hld\ Be·þiu skal man is \alst{w}erk lovon, &
\alst{d}iuran is \alst{d}ádi, \hld\ hwand hie is \alst{d}rohtin self, &
\alst{m}ahtig \alst{m}und-boro \hld\ \alst{m}anno kunnje, &
\alst{l}iudjo só hwi-likon, \hld\ só þár gi·\alst{l}ôbit tuo &
an is \alst{w}ord ęndi an is \alst{w}erk.\eva

\bvb TODO.\evb\evg

\bvg\bva[27][2231]%
\hspace*{100pt} Þuo was þár \alst{w}erodes só filo &%NOTE: In cæsura.
\alst{a}llaro \alst{ę}li-þiodo kuman \hld\ te þem \alst{ê}ron Kristes, &
te só \alst{m}ahtiges \alst{m}und-burd. \hld\ Þuo welda hie þár êna \alst{m}ęri líðan, &
þie \alst{g}odes suno mid is \alst{j}ungron \hld\ a·nevan \alst{G}alilea-land, &
\alst{w}aldand ênna \alst{w}ágo strôm. \hld\ Þuo hiet hie þat \alst{w}erod ȯðar &
\alst{f}orð-werdes \alst{f}aran, \hld\ ęndi hie gi·wêt im \alst{f}ahora sum &
an ênna \alst{n}akon innan, \hld\ \alst{n}ęrjendi Krist, &
\alst{s}lápan \alst{s}ïð-wórig. \hld\ \alst{S}egel up dádun &
\alst{w}edẹr-wísa \alst{w}eros, \hld\ lietun \alst{w}ind after &
\alst{m}anon ovar þena \alst{m}ęri-strôm, \hld\ unþat hie te \alst{m}iddjan kwam, &
\alst{w}aldand mid is \alst{w}erodu. \hld\ Þuo bi·gan þes \alst{w}edares kraft, &
\alst{u̇}st up stígan, \hld\ \alst{u̇}ðjun wahsan; &
\alst{s}wang gi·\alst{s}werk an gi·mang: \hld\ þie \alst{s}êw warð an hruoru, &
wan \alst{w}ind ęndi \alst{w}ater; \hld\ \alst{w}eros sorọgodun, &
þiu \alst{m}ęri warð só \alst{m}uodag, \hld\ ni wánda þero \alst{m}anno nig·ên &
\alst{l}ęngron \alst{l}íves. \hld\ Þuo sia \alst{l}andes ward &
\alst{w}ękidun mid iro \alst{w}ordon \hld\ ęndi sagdun im þes \alst{w}edares kraft, &
bádun þat im gi·\alst{n}áðig \hld\ \alst{n}ęrjendi Krist &
\alst{w}urði wið þem \alst{w}atare: \hld\ „efþa wí skulun hier te \alst{w}undẹr-kwálu &
\alst{s}weltan an þeson \alst{s}êwe.“ \hld\ \alst{S}elf up a·rês &
þie \alst{g}uodo \alst{g}odes suno \hld\ ęndi te is \alst{j}ungron sprak, &
hiet þat sia im \alst{w}edares gi·\alst{w}in \hld\ \alst{w}iht ni an·drédin: &
„te hwí sind gí só \alst{f}orhta?“ \hld[kwaþ-hie.] „Nis iu noh \alst{f}ast hugi, &
gi·\alst{l}ôvo is iu te \alst{l}uttil. \hld\ Nis nú \alst{l}ang te þiu, &
þat þia \alst{st}rômos skulun \hld\ \alst{st}ilrun werðan &
gi þit *\alst{w}edar \alst{w}un-sam.“ \hld\ Þo hí te þem \alst{w}inde sprak &
ge te þemu \alst{s}êwa só \alst{s}elf \hld\ ęndi sie \alst{s}multro hét &
\alst{b}êðja ge·\alst{b}árjan. \hld\ Sie gi·\alst{b}od lêstun, &
\alst{w}aldandes \alst{w}ord: \hld\ \alst{w}edẹr stillodun, &
\alst{f}agạr warð an \alst{f}lóde. \hld\ Þó bi·gan þat \alst{f}olk undar im, &
\alst{w}erod \alst{w}undrajan, \hld\ ęndi suma mid iro \alst{w}ordun sprákun, &
hwi-lik þat só \alst{m}ahtigoro \hld\ \alst{m}anno wári, &
þat imu só þe \alst{w}ind ęndi þe \alst{w}ág \hld\ \alst{w}ordu hôrdin, &
\alst{b}êðja is gi·\alst{b}od-skępjes. \hld\ Þó habda sie þat \alst{b}arn godes &
gi·\alst{n}ęrid fan þeru \alst{n}ôdi: \hld\ þe \alst{n}ako furðor \edtext{skręid}{\Afootnote{A rare occurrence of the original diphthong; see note to line 359 above.}}, &
\edtrans{\alst{h}ôh \alst{h}urnid-skip}{high horned ship}{\Bfootnote{A formulaic expression for a high-prowed longship, apparently an inheritance from earlier pagan Saxon poetry.  The line reoccurs below at 2907a.}}; \hld\ \alst{h}ęliðos kwámun, &
\alst{l}iudi te \alst{l}ande, \hld\ sagdun \alst{l}of gode, &
\alst{m}áridun is \alst{m}ęgin-kraft. \hld\ Kwam þár \alst{m}anno filu &
an·\alst{g}ęgin þemu \alst{g}odes sunje; \hld\ hé sie \alst{g}erno ant·féng, &
só hwene só þár mid \alst{h}luttru \alst{h}ugi \hld\ \alst{h}elpa sóhte; &
\alst{l}êrde sie iro gi·\alst{l}ôvon \hld\ ęndi iro \alst{l}ík-hamon &
\alst{h}andun \alst{h}êlde: \hld\ nio þe man só \alst{h}ardo ni was &
gi·\alst{s}êrit mid \alst{s}uhtjun: \hld\ þoh ina \alst{S}atanases &
\alst{f}êknja jungoron \hld\ \alst{f}íundes kraftu &
\alst{h}abdin undar \alst{h}andun \hld\ ęndi is \alst{h}ugi-skęfti, &
gi·\alst{w}it a·\alst{w}ardid, \hld\ þat hé \alst{w}ódjendi &
\alst{f}óri undar þemu \alst{f}olke, \hld\ þoh im simbla \alst{f}erh far·gaf &
\alst{h}êlandjo Krist, \hld\ ef hé te is \alst{h}andun kwam, &
\alst{d}rêf þea \alst{d}iuvlas þanan \hld\ \alst{d}rohtines kraftu, &
\alst{w}árun \alst{w}ordun, \hld\ ęndi im is ge·\alst{w}it far·gaf, &
lét ina þan \alst{h}êlan \hld\ wiðẹr \alst{h}ęttjandun, &
gaf im wið þie \alst{f}íund \alst{f}riðu, \hld\ ęndi im \alst{f}orð gi·wêt &
an só hwi-lik þero \alst{l}ando, \hld\ só im þan \alst{l}eovost was.\eva

\bvb TODO.\evb\evg

\bvg\bva[28][2284]%
Só deda þe \alst{d}rohtines sunu \hld\ \alst{d}ago ge·hwi-likes &
\alst{g}ód werk mid is \alst{j}ungeron, \hld\ só neo \alst{J}udeon umbi þat &
an þea is \alst{m}ikilun kraft \hld\ þiu \alst{m}êr ne ge·lôvdun, &
þat hé \alst{a}lo-waldo \hld\ \alst{a}lles wári, &
\alst{l}andes ęndi \alst{l}iudjo: \hld\ þes sie noh \alst{l}ôn nimat, &
\alst{w}ídana \alst{w}rak-sïð, \hld\ þes sie þár þat ge·\alst{w}in drivun &
wið \alst{s}elvan þene \alst{s}unu drohtines. \hld\ Þó hé im mid is ge·\alst{s}ïðon gi·wêt &
eft an \alst{G}alilaeo land, \hld\ \alst{g}odes êgan barn, &
\alst{f}ór im te þem \alst{f}riundun, \hld\ þár hé a·\alst{f}ódid was &
ęndi al undar is \alst{k}unnje \hld\ \alst{k}ind-jung a·wóhs, &
þe \alst{h}êlago \alst{h}êljand. \hld\ Umbi ina \alst{h}ęri-skępi, &
\alst{þ}eoda \alst{þ}rungun; \hld\ þár was \alst{þ}egạn manag &
só \alst{s}álig undar þem ge·\alst{s}ïðe. \hld\ Þár drógun ênna \alst{s}eokan man &
\alst{e}rlos an iro \alst{a}rmun: \hld\ weldun ina for \alst{ô}gun Kristes, &
\alst{b}rengjan for þat \alst{b}arn godes \hld\ —was im \alst{b}ótono þarf, &
þat ina ge·\alst{h}êldi \hld\ \alst{h}evanes waldand, &
\alst{m}anno \alst{m}und-boro—, \hld\ þe was êr só \alst{m}anagan dag &
\alst{l}iðu-wastmon bi·\alst{l}amod, \hld\ ni mahte is \alst{l}ík-hamon &
\alst{w}iht ge·\alst{w}aldan. \hld\ Þan was þár \alst{w}erodes só filu, &
þat sie ina fora þat \alst{b}arn godes \hld\ \alst{b}rengjan ni mahtun, &
ge·\alst{þ}ringan þurh þea \alst{þ}ioda, \hld\ þat sie só \alst{þ}urftiges &
\alst{s}unnja ge·\alst{s}agdin. \hld\ Þó gi·wêt imu an ênna \alst{s}ęli innan &
\alst{h}êljando Krist; \hld\ \alst{h}warf warð þár umbi, &
\alst{m}ęgin-þeodo ge·\alst{m}ang. \hld\ Þó bi·gunnun þea \alst{m}an spreken, &
þe þene \alst{l}éfna \alst{l}amon \hld\ \alst{l}ango fórdun, &
\alst{b}árun mid is \alst{b}ęddju, \hld\ hwó sie ina ge·drógin fora þat \alst{b}arn godes, &
an þat \alst{w}erod innan, \hld\ þár ina \alst{w}aldand Krist &
\alst{s}elvo gi·\alst{s}áwi. \hld\ Þó géngun þea ge·\alst{s}ïðos tó, &
\alst{h}óvun ina mid iro \alst{h}andun \hld\ ęndi uppan þat \alst{h}ús stigun, &
\alst{s}litun þene \alst{s}ęli ovana \hld\ ęndi ina mid \alst{s}élun létun &
an þene \alst{r}akud innan, \hld\ þár þe \alst{r}íkjo was, &
\alst{k}uningo \alst{k}raftigost. \hld\ Reht só hé ina þó \alst{k}uman gi·sah &
þurh þes \alst{h}úses \alst{h}róst, \hld\ só hé þó an iro \alst{h}ugi far·stód, &
an þero \alst{m}anno \alst{m}ód-sevon, \hld\ þat sie \alst{m}ikilana te imu &
ge·\alst{l}ôvon habdun, \hld\ þó hé for þen \alst{l}iudjun sprak, &
kwað þat hé þene \alst{s}iakon man \hld\ \alst{s}undjono tómjan &
\alst{l}átan weldi. \hld\ Þó sprákun im eft þea \alst{l}iudi an·gęgin, &
\alst{g}ram-harde \alst{J}udeon, \hld\ þea þes \alst{g}odes barnes &
\alst{w}ord aftar \alst{w}arodun, \hld\ kwáðun þat þat ni mahti gi·\alst{w}erðen só, &
\alst{g}rim-werk far·\alst{g}even, \hld\ bi·útan \alst{g}od êno, &
\alst{w}aldand þesaro \alst{w}er-oldes. \hld\ Þó habda eft is \alst{w}ord garu &
\alst{m}ahtig barn godes: \hld\ „ik gi·dón þat“, kwað hé, „an þesumu \alst{m}anne skín, &
þe hír só \alst{s}iak ligid \hld\ an þesumu \alst{s}ęli innan, &
te \alst{w}undron gi·\alst{w}êgid, \hld\ þat ik ge·\alst{w}ald hębbju &
\alst{s}undja te far·gevanne \hld\ ęndi ôk \alst{s}eokan man &
te ge·\alst{h}êljanne, \hld\ só ik ina \alst{h}rínan ni þarf.“ &
\alst{M}anoda ina þó \hld\ þe \alst{m}árjo drohtin, &
\alst{l}iggjandjan \alst{l}amon, \hld\ hét ina far þem \alst{l}iudjun a·standan &
up \alst{a}lo-hêlan \hld\ ęndi hét ina an is \alst{a}hslun niman, &
is \alst{b}ęd-gi·wádi te \alst{b}aka; \hld\ hé þat gi·\alst{b}od lêste &
\alst{s}niumo for þemu gi·\alst{s}ïðja \hld\ ęndi géng imu eft ge·\alst{s}und þanan, &
\alst{h}êl fan þemu \alst{h}úse. \hld\ Þó þes só manag \alst{h}êðin man, &
\alst{w}eros \alst{w}undradun, \hld\ kwáðun þat imu \alst{w}aldand self, &
\alst{g}od alo-mahtig \hld\ far·\alst{g}evan habdi &
\alst{m}éron \alst{m}ahti \hld\ þan elkor ênigumu \alst{m}annes sunje, &
\alst{k}raft ęndi \alst{k}usti; \hld\ sie ni weldun ant·\alst{k}ęnnjan þoh, &
\alst{J}udeo liudi, \hld\ þat hé \alst{g}od wári, &
ne ge·\alst{l}ôvdun is \alst{l}êran, \hld\ ak habdun im \alst{l}êðan stríd, &
\alst{w}unnun wiðar is \alst{w}ordun: \hld\ þes sie \alst{w}erk hlutun, &
\alst{l}êð-lík \alst{l}ôn-geld, \hld\ ęndi só noh \alst{l}ango skulun, &
þes sie ni weldun \alst{h}ôrjen \hld\ \alst{h}evan-kuninges, &
\alst{K}ristes lêrun, \hld\ þea hé \alst{k}u̇ðde ovar al, &
\alst{w}ído aftar þesaro \alst{w}er-oldi, \hld\ ęndi lét sie is \alst{w}erk sehan &
allaro \alst{d}ago ge·hwi-likes, \hld\ is \alst{d}ádi skawon, &
\alst{h}ôrjen is \alst{h}êlag word, \hld\ þe hé te \alst{h}elpu ge·sprak &
\alst{m}anno barnun, \hld\ ęndi só manag \alst{m}ahtig-lík &
\alst{t}êkạn ge·\alst{t}ôgda, \hld\ þat sie gi·\alst{t}rúodin þiu bet, &
gi·\alst{l}ôvdin an is \alst{l}êra. \hld\ Hé só managan \alst{l}ík-hamon &
\alst{b}alu-suhtjo ant·\alst{b}and \hld\ ęndi \alst{b}óta ge·skęride, &
far·gaf \alst{f}êgjun \alst{f}erạh, \hld\ þem þe \alst{f}u̇sid was &
\alst{h}ęlið an \alst{h}ęl-sïð: \hld\ þan gi·deda ina þe \alst{h}êland self, &
\alst{K}rist þurh is \alst{k}raft mikil \hld\ \alst{k}wikan aftar dôða, &
lét ina an þesaro \alst{w}er-oldi forð \hld\ \alst{w}unnjono neotan.\eva

\bvb TODO.\evb\evg

\bvg\bva[29][2357]%
Só \alst{h}êlde hé þea \alst{h}altun man \hld\ ęndi þea \alst{h}ávon só self, &
\alst{b}ótta þem þár \alst{b}linde wárun, \hld\ lét sie þat \alst{b}erhte lioht, &
\alst{s}in-skôni \alst{s}ehan, \hld\ \alst{s}undja lôsda, &
\alst{g}umono \alst{g}rim-werk. \hld\ Ni was gio \alst{J}udeono be·þiu, &
\alst{l}êðes \alst{l}iud-skępjes \hld\ gi·\alst{l}ôvo þiu bętara &
an þene \alst{h}êlagon Krist, \hld\ ak habdun im \alst{h}ardene mód, &
swíðo \alst{st}arkan \alst{st}ríd, \hld\ far·\alst{st}andan ni weldun, &
þat sie habdun for·\alst{f}angan \hld\ \alst{f}íundun an willjan, &
\alst{l}iudi mid iro ge·\alst{l}ôvun. \hld\ Ni was gio þiu \alst{l}atoro be·þiu &
\alst{s}unu drohtines, \hld\ ak hé \alst{s}agde mid wordun, &
hwó sie skoldin ge·\alst{h}alon \hld\ \alst{h}imiles ríki, &
\alst{l}êrde aftar þemu \alst{l}ande, \hld\ habde imu þero \alst{l}iudjo só filu &
gi·\alst{w}enid mid is \alst{w}ordun, \hld\ þat im \alst{w}erod mikil, &
\alst{f}olk \alst{f}olgoda, \hld\ ęndi hé im \alst{f}ilu sagda, &
be \alst{b}iliðjun þat \alst{b}arn godes, \hld\ þes sie ni mahtun an iro \alst{b}reostun far·standan, &
undar·\alst{h}uggjan an iro \alst{h}erton, \hld\ êr it im þe \alst{h}êlago Krist &
ovar þat \alst{e}rlo folk \hld\ \alst{o}ponun wordun &
þurh is \alst{s}elves kraft \hld\ \alst{s}ęggjan welda, &
\alst{m}árjan hwat hé \alst{m}ênde. \hld\ Þár ina \alst{m}ęgin umbi, &
\alst{þ}ioda \alst{þ}rungun: \hld\ was im \alst{þ}arf mikil &
te gi·\alst{h}ôrjenne \hld\ \alst{h}evan-kuninges &
\alst{w}ár-fastun \alst{w}ord. \hld\ Hé stód imu þó bi ênes \alst{w}atares staðe, &
ni welde þó bi þemu ge·\alst{þ}ringe \hld\ ovar þat \alst{þ}egno folk &
an þemu \alst{l}ande uppan \hld\ þea \alst{l}êra ku̇ðjan, &
ak \alst{g}éng imu þó þe \alst{g}ódo \hld\ ęndi is \alst{j}ungaron mid imu, &
\alst{f}riðu-barn godes, \hld\ þemu \alst{f}lóde náhor &
an ên \alst{sk}ip innan, \hld\ ęndi it \alst{sk}alden hét &
\alst{l}ande rúmur, \hld\ þat ina þea \alst{l}iudi só filu, &
\alst{þ}ioda ni \alst{þ}rungi. \hld\ Stód \alst{þ}egạn manag, &
\alst{w}erod bi þemu \alst{w}atare, \hld\ þár \alst{w}aldand Krist &
ovar þat \alst{l}iudjo folk \hld\ \alst{l}êra sagde: &
„Hwat ik iu \alst{s}ęggjan mag“, \hld[kwað hé,] „ge·\alst{s}ïðos míne, &
hwó imu \alst{ê}n \alst{e}rl bi·gan \hld\ an \alst{e}rðu sájan &
\alst{h}rên-korni mid is \alst{h}andun. \hld\ Sum it an \alst{h}ardan stên &
\alst{o}van-wardan fel, \hld\ \alst{e}rðon ni habda, &
þat it þár mahti \alst{w}ahsan \hld\ efþa \alst{w}urtjo gi·fȧhan, &
\alst{k}ínan efþa bi·\alst{k}líven, \hld\ ak warð þat \alst{k}orn far·loren, &
þat þár an þeru \alst{l}éian gi·\alst{l}ag. \hld\ Sum it eft an \alst{l}and bi·fel, &
an \alst{e}rðun \alst{a}ðal-kunnjes: \hld\ bi·gan imu \alst{a}ftar þiu &
\alst{w}ahsen \alst{w}án-líko \hld\ ęndi \alst{w}urtjo fȧhan, &
\alst{l}ód an \alst{l}ustun: \hld\ was þat \alst{l}and só gód, &
\alst{f}ránisko gi·\alst{f}ehod. \hld\ Sum it eft bi·\alst{f}allen warð &
an êna \alst{st}arka \alst{st}rátun, \hld\ þár \alst{st}ópon géngun, &
\alst{h}rosso \alst{h}óf-slaga \hld\ ęndi \alst{h}ęliðo tráda; &
warð imu þár an \alst{e}rðu \hld\ ęndi eft \alst{u}p gi·géng, &
bi·gan imu an þemu \alst{w}ege \alst{w}ahsen; \hld\ þó it eft þes \alst{w}erodes far·nam, &
þes \alst{f}olkes \alst{f}ard mikil \hld\ ęndi \alst{f}uglos a·lásun, &
þat is þemu \alst{é}ksan wiht \hld\ \alst{a}ftar ni móste &%TODO: check éksan
\alst{w}erðan te \alst{w}illjan, \hld\ þes þár an þene \alst{w}eg bi·fel. &
Sum warð it þan bi·\alst{f}allen, \hld\ þár só \alst{f}ilu stódun &
\alst{þ}ikkero \alst{þ}orno \hld\ an \alst{þ}emu dage; &
warð imu þár an \alst{e}rðu \hld\ ęndi eft \alst{u}p gi·géng, &%TODO: is the repeated line due to an error?
\alst{k}én imu þár ęndi \alst{k}livode. \hld\ Þó slógun þár eft \alst{k}rúd an gi·mang, &
\alst{w}ęridun imu þene \alst{w}astom: \hld\ habda it þes \alst{w}aldes hlea &
\alst{f}orana ovar-\alst{f}angan, \hld\ þat it ni mahte te ênigaro \alst{f}rumu werðen, &
ef it þea \alst{þ}ornos \hld\ só \alst{þ}ringan móstun.“ &
Þó \alst{s}átun ęndi \alst{s}wígodun \hld\ ge·\alst{s}ïðos Kristes, &
\alst{w}ord-spáha \alst{w}eros: \hld\ was im \alst{w}undạr mikil, &
be hwi-likun \alst{b}iliðjun \hld\ þat \alst{b}arn godes &
su·lik \alst{s}ȯð-lík spel \hld\ \alst{s}ęggjan bi·gunni. &
Þó bi·gan is þero \alst{e}rlo \hld\ \alst{ê}n frágojan &
\alst{h}oldan \alst{h}êrron, \hld\ \alst{h}nêg imu te·gęgnes &
tulgo \alst{w}erð-liko: \hld\ „Hwat þú ge·\alst{w}ald havas“, kwað hé, &
„ia an \alst{h}imile ia an erðu, \hld\ \alst{h}êlag drohtin, &
\alst{u}ppa ęndi niðara, \hld\ bist þú \alst{a}lo-waldo &
\alst{g}umono \alst{g}êsto, \hld\ ęndi wí þíne \alst{j}ungaron sind, &
an u̇sumu \alst{h}ugi \alst{h}olde. \hld\ \alst{H}êrro þe gódo, &
ef it þín \alst{w}illjo sí, \hld\ lát u̇s þínaro \alst{w}ordo þár &
\alst{ę}ndi gi·hôrjen, \hld\ þat wí it \alst{a}ftar þi &
ovar al \alst{K}ristin-folk \hld\ \alst{k}u̇ðjan mótin. &
wí \alst{w}itun þat þínun \alst{w}ordun \hld\ \alst{w}ár-lík biliði &
\alst{f}orð \alst{f}olgojad, \hld\ ęndi u̇s is \alst{f}irinun þarf, &
þat wí þín \alst{w}ord ęndi þín \alst{w}erk, \hld\ —hwand it fan su·likumu ge·\alst{w}ittja kumid— &
þat wí it an þesumu \alst{l}ande \hld\ at þi \alst{l}ínon mótin.“\eva

\bvb TODO.\evb\evg

\bvg\bva[30][2431]%
Þó im eft te·\alst{g}ęgnes \hld\ \alst{g}umono bętsta &
\alst{a}nd-wordi ge·sprak: \hld\ „ni mênde ik \alst{e}lkor wiht“, kwað hé, &
„te bi·\alst{d}ęrnjenne \hld\ \alst{d}ádjo mínaro, &
\alst{w}ordo efþa \alst{w}erko; \hld\ þit skulun gí \alst{w}itan alle, &
\alst{j}ungaron míne, \hld\ hwand iu far·\alst{g}even havad &
\alst{w}aldand þesaro \alst{w}er-oldes, \hld\ þat gí \alst{w}itan mótun &
an iuwom \alst{h}ugi-skęftjun \hld\ \alst{h}imilisk ge·rúni; &
þem ȯðrun skal man be \alst{b}iliðjun \hld\ þat gi·\alst{b}od godes &
\alst{w}ordun \alst{w}ísjen. \hld\ Nú willju ik iu te \alst{w}árun hier &
\alst{m}árjen, hwat ik \alst{m}ênde, \hld\ þat gí \alst{m}ína þiu bet &
ovar al þit \alst{l}and-skępi \hld\ \alst{l}êra far·standan. &
Þat \alst{s}ád, þat ik iu \alst{s}agda, \hld\ þat is \alst{s}elves word, &
þiu \alst{h}êlaga lêra \hld\ \alst{h}evan-kuninges, &
hwó \alst{m}an þea \alst{m}árjen skal \hld\ ovar þene \alst{m}iddil-gard, &
\alst{w}ído aftar þesaro \alst{w}er-oldi. \hld\ \alst{W}eros sind im gi·hugide, &
\alst{m}an \alst{m}is-líko: \hld\ sum su·likan \alst{m}ód dręgid, &
\alst{h}arda \alst{h}ugi-skęfti \hld\ ęndi \alst{h}rêan sevon, &
þat ina ni ge·\alst{w}erðod, \hld\ þat hé it be iuwon \alst{w}ordun due, &
þat hé þesa mína \alst{l}êra forð \hld\ \alst{l}êstjen willje, &
ak werðad þár só far·\alst{l}orana \hld\ \alst{l}êra mína, &
\alst{g}odes ambusni \hld\ ęndi iuwaro \alst{g}umono word &
an þemu \alst{u}vilon manne, \hld\ só ik iu \alst{ê}r sagda, &
þat þat \alst{k}orn far·warð, \hld\ þat þár mid \alst{k}íðun ni mahte &
an þemu \alst{st}êne uppan \hld\ \alst{st}ędi-haft werðan. &
Só wirðid \alst{a}l far·loran \hld\ \alst{ę}ðilero spráka, &
\alst{â}rundi godes, \hld\ só hwat só man þemu \alst{u}vilon manne &
\alst{w}ordun ge·\alst{w}ísid, \hld\ ęndi hé an þea \alst{w}irson hand, &
undar \alst{f}íundo \alst{f}olk \hld\ \alst{f}ard ge·kiusid, &
an \alst{g}odes un-wiljan \hld\ ęndi an \alst{g}ramono hróm &
ęndi an \alst{f}iures \alst{f}arm. \hld\ \alst{F}orð skal hé hêtjan &
mid is \alst{b}reost-hugi \hld\ \alst{b}rêda logna. &
Nio gí an þesumu \alst{l}ande þiu \alst{l}és \hld\ \alst{l}êra mína &
\alst{w}ordun ni \alst{w}ísjad: \hld\ is þeses \alst{w}erodes só filu, &
\alst{e}rlo aftar þesaro \alst{e}rðun: \hld\ bi·stéd þár \alst{ȯ}ðar man, &
þe is imu \alst{j}ung ęndi \alst{g}lau, \hld\ —ęndi havad imu \alst{g}ódan mód—, &
\alst{sp}rákono \alst{sp}áhi \hld\ ęndi wêt iuwaro \alst{sp}ello gi·skêð, &
\alst{h}ugid is þan an is \alst{h}erton \hld\ ęndi \alst{h}ôrid þár mid is ôrun tó &
swíðo \alst{n}iud-líko \hld\ ęndi \alst{n}áhor stéd, &
an is \alst{b}reost hlędid \hld\ þat gi·\alst{b}od godes, &
\alst{l}ínod ęndi \alst{l}êstid: \hld\ is is gi·\alst{l}ôvo só gód, &
talod imu, hwó hé \alst{ȯ}ðrana \hld\ \alst{ę}ft gi·hwęrvje &
\alst{m}ên-dádigan \alst{m}an, \hld\ þat is \alst{m}ód draga &
\alst{h}luttra treuwa \hld\ te \alst{h}evan-kuninge. &
Þan \alst{b}rêdid an þes \alst{b}reostun \hld\ þat gi·\alst{b}od godes, &
þie \alst{l}uvigo gi·\alst{l}ôbo, \hld\ só an þemu \alst{l}ande duod &
þat \alst{k}orn mid \alst{k}íðun, \hld\ þár it gi·\alst{k}und havad &
ęndi imu þiu \alst{w}urð bi·hagod \hld\ ęndi \alst{w}edẹres gang, &
\alst{r}ęgin ęndi sunne, \hld\ þat it is \alst{r}eht havad. &
Só duod þiu \alst{g}odes lêra \hld\ an þemu \alst{g}ódun manne &
\alst{d}ages ęndi nahtes, \hld\ ęndi gangid imu \alst{d}iuval fer, &
\alst{w}rêða \alst{w}ihti \hld\ ęndi þe \alst{w}ard godes &
\alst{n}áhor mikilu \hld\ \alst{n}ahtes ęndi dages, &
ant-tat sie ina \alst{b}rengjad, \hld\ þat þár \alst{b}êðju wirðid &
ia þiu \alst{l}êra te frumu \hld\ \alst{l}iudjo barnun, &
þe fan is \alst{m}u̇ðe kumid, \hld\ iak wirðid þe \alst{m}an gode; &
havad só gi·\alst{w}ehslod \hld\ te þesaro \alst{w}er-old-stundu &
mid is \alst{h}ugi-skęftjun \hld\ \alst{h}imil-ríkjas gi·dêl, &
\alst{w}elono þene mêstan: \hld\ farid imu an gi·\alst{w}ald godes, &
\alst{t}ionuno \alst{t}ómig. \hld\ \alst{T}reuwa sind só góda &
\alst{g}umono ge·hwi-likumu, \hld\ só nis \alst{g}oldes hord &
ge·\alst{l}ík su·likumu gi·\alst{l}ôvon. \hld\ Wesad iuwaro \alst{l}êrono forð &
\alst{m}an-kunnje \alst{m}ildje; \hld\ sie sind só \alst{m}is-líka, &
\alst{h}ęliðos ge·\alst{h}ugda: \hld\ sum havad iro \alst{h}ardan stríd, &
\alst{w}rêðan \alst{w}illjan, \hld\ \alst{w}ankolna hugi, &
is imu \alst{f}êknes \alst{f}ul \hld\ ęndi \alst{f}irin-werko. &
\alst{Þ}an bi·ginnid imu \alst{þ}unkjan, \hld\ þan hé undar þeru \alst{þ}iodu stád &
ęndi þár gi·\alst{h}ôrid \hld\ ovar \alst{h}lust mikil &
þea \alst{g}odes lêra, \hld\ þan þunkid imu, þat hé sie \alst{g}erno forð &
\alst{l}êstjen willje; \hld\ þan bi·ginnid imu þiu \alst{l}êra godes &
an is \alst{h}ugi \alst{h}afton, \hld\ ant-tat imu þan eft an \alst{h}and kumid &
\alst{f}eho te gi·\alst{f}órja \hld\ ęndi \alst{f}ręmiði skat. &
Þan far·\alst{l}êdjad ina \hld\ \alst{l}êða wihti, &
þan hé imu far·\alst{f}áhid \hld\ an \alst{f}eho-giri, &
a·\alst{l}ęskid þene gi·\alst{l}ôbon: \hld\ þan was imu þat \alst{l}uttil fruma, &
þat hé it gio an is \alst{h}ertan ge·\alst{h}ugda, \hld\ ef hé it \alst{h}alden ne wili. &
Þat is só þe \alst{w}astom, \hld\ þe an þemu \alst{w}ege be·gan, &
\alst{l}iodan an þemu \alst{l}ande: \hld\ þó far·nam ina eft þero \alst{l}iudjo fard. &
Só duot þea \alst{m}ęgin-sundjon \hld\ an þes \alst{m}annes hugi &
þea \alst{g}odes lêra, \hld\ ef hé is ni \alst{g}ômid wel; &
elkor bi·\alst{f}ęlljad sia ina \hld\ \alst{f}erne te boðme, &
an þene \alst{h}êtan \alst{h}ęl, \hld\ þár hé \alst{h}evan-kuninge &
ni wirðid \alst{f}urður te \alst{f}rumu, \hld\ ak ina \alst{f}íund skulun &
\alst{w}ítju gi·\alst{w}aragjan. \hld\ Simla gí mid \alst{w}ordun forð &
\alst{l}êrjad an þesumu \alst{l}ande: \hld\ *ik kan þesaro \alst{l}iudjo hugi, &
só \alst{m}is-líkan \alst{m}uod-sevon \hld\ \alst{m}anno kunnjes, &
só \alst{w}anda \alst{w}ísa \hld\ {[...]} &
Sum havit all te þiu is \alst{m}uod gi·látan \hld\ ęndi \alst{m}êr sorọgot, &
hwó hie þat \alst{h}ord bi·\alst{h}alde, \hld\ þan hwó hie \alst{h}evan-kuninges &
\alst{w}illjon gi·\alst{w}irkje. \hld\ Be·þiu þár \alst{w}ahsan ni mag &
þat \alst{h}êlaga gi·bod godes, \hld\ þoh it þár a·\alst{h}afton mugi, &
\alst{w}urtjon bi·\alst{w}erpan, \hld\ hwand it þie \alst{w}elo þringit. &
Só samo só þat \alst{k}rúd ęndi þie þorn \hld\ þat \alst{k}orn ant·fȧhat, &
\alst{w}ęrjat im þena \alst{w}astom, \hld\ só duot þie \alst{w}elo manne: &
gi·\alst{h}ęftid is \alst{h}erta, \hld\ þat hie it gi·\alst{h}uggjan ni muot, &
þie \alst{m}an an is \alst{m}uode, \hld\ þes hie \alst{m}êst bi·þarf, &
hwó hie þat gi·\alst{w}irkje, \hld\ þan lang þie hie an þesaro \alst{w}er-oldi sí, &
þat hie ti \alst{ê}won-dage \hld\ \alst{a}fter muoti &
\alst{h}ębbjan þuru is \alst{h}êrren þank \hld\ \alst{h}imiles ríki, &
só \alst{ę}ndi-lôsan welon, \hld\ só þat ni mag \alst{ê}nig man &
\alst{w}itan an þesaro \alst{w}er-oldi. \hld\ Nio hie só \alst{w}ído ni kan &
te gi·\alst{þ}ęnkjanne, \hld\ \alst{þ}egạn an is muode, &
þat it bi·\alst{h}aldan mugi \hld\ \alst{h}erta þes mannes, &
þat hie þat ti \alst{w}áron witi, \hld\ hwat \alst{w}aldand god havit &
\alst{g}uodes gi·\alst{g}ęrẹwid, \hld\ þat all \alst{g}ęgin-werd stéð &
\alst{m}anno só hwi-likon, \hld\ só ina hier \alst{m}innjot wel &
ęndi \alst{s}elvo te þiu \hld\ is \alst{s}eola gi·haldit, &
þat hie an \alst{l}ioht godes \hld\ \alst{l}íðan muoti.“\eva

\bvb TODO.\evb\evg

\bvg\bva[31][2538]%
Só \alst{w}ísda hie þuo mid \alst{w}ordon, \hld\ stuod \alst{w}erod mikil &
umbi þat \alst{b}arn godes, \hld\ ge·hôrdun ina bi \alst{b}iliðon filo &
umbi þesaro \alst{w}er-oldes gi·\alst{w}and \hld\ \alst{w}ordon tęlljan; &
kwað þat im ôk \alst{ê}n \alst{a}ðales man \hld\ an is \alst{a}kker sáidi &
\alst{h}luttạr \alst{h}rên-korni \hld\ \alst{h}andon sínon: &
\alst{w}olda im þár só \alst{w}un-sames \hld\ \alst{w}astmes tiljan, &
\alst{f}agạres \alst{f}ruhtes. \hld\ Þuo géng þár is \alst{f}íond aftar &
þuru \alst{d}ęrnjan hugi, \hld\ ęndi it all mid \alst{d}urðu ovar-séu, &%TODO: séu unclear.
mid \alst{w}eodo \alst{w}irsiston. \hld\ Þuo \alst{w}óhsun sia bêðju, &
ge þat \alst{k}orn ge þat \alst{k}rúd. \hld\ Só \alst{k}wámun gangan &
is \alst{h}aga-stoldos te \alst{h}ús, \hld\ iro \alst{h}êrren sagdun, &
\alst{þ}egnos iro \alst{þ}iodne \hld\ \alst{þ}rístjon wordon: &
„Hwat þú sáidos \alst{h}luttạr korn, \hld\ \alst{h}êrro þie guodo, &
\alst{ê}n-fald an þínon \alst{a}kkar: \hld\ nú ni gi·sihit ênig \alst{e}rlo þan mêr &
\alst{w}eodes \alst{w}ahsan. \hld\ Hwí mohta þat gi·\alst{w}erðan só?“ &
Þuo sprak eft þie \alst{a}ðales man \hld\ þem \alst{e}rlon te·gęgnes, &
\alst{þ}iodan wið is \alst{þ}egnos, \hld\ kwað þat hie it mahti undar·\alst{þ}ęnkjan wel, &
þat im þár \alst{u}n-hold man \hld\ \alst{a}ftar sáida, &
\alst{f}íond \alst{f}êkni krúd: \hld\ „ne gionsta mí þero \alst{f}ruhtjo wel, &
a·\alst{w}erda mí þena \alst{w}astom.“ \hld\ Þuo þár eft \alst{w}ini sprákun, &
is \alst{j}ungron te·\alst{g}ęgnes, \hld\ kwáðun þat sia þár weldin \alst{g}angan tuo, &
\alst{k}uman mid \alst{k}raftu \hld\ ęndi lôsjan þat \alst{k}rúd þanan, &
\alst{h}alon it mid iro \alst{h}andon. \hld\ Þuo sprak im eft iro \alst{h}êrro an·gęgin: &
„ne \alst{w}ęlljo ik, þat gí it \alst{w}iodon“, \hld[kwaþ-hie,] „hwand gí bi·\alst{w}ardon ni mugun, &
gi·\alst{g}ômjan an iuwon \alst{g}ange, \hld\ þoh gí it \alst{g}erno ni duan, &
ni gí þes \alst{k}ornes te filo, \hld\ \alst{k}íðo a·węrdjat, &
\alst{f}ęlljat under iuwa \alst{f}uoti. \hld\ Láte man sia \alst{f}orð hinan &
\alst{b}êðju wahsan, \hld\ und êr \alst{b}ewod kume &
ęndi an þem \alst{f}elde sind \hld\ \alst{f}ruhti rípja, &
\alst{a}roa an þem \alst{a}kkare: \hld\ þan faran wí þár \alst{a}lla tuo, &
\alst{h}alon it mid u̇ssan \alst{h}andon \hld\ ęndi þat \alst{h}rên-kurni lesan &
\alst{s}úvro te·\alst{s}amne \hld\ ęndi it an mínon \alst{s}ęli duojan, &
\alst{h}ębbjan it þár gi·\alst{h}aldan, \hld\ þat it \alst{h}węrgin ni mugi &
\alst{w}iht a·\alst{w}ęrdjan, \hld\ ęndi þat \alst{w}iod niman, &
\alst{b}indan it te \alst{b}urðinnjon \hld\ ęndi werpan it an \alst{b}ittạr fiur, &
láton it þár \alst{h}alojan \hld\ \alst{h}êta logna, &
\alst{ạ}ld \alst{u}n-fuodi.“ \hld\ Þuo stuod \alst{e}rl manag, &
\alst{þ}egnos \alst{þ}agjandi, \hld\ hwat \alst{þ}iod-gomo, &
*\alst{m}ári \alst{m}ahtig Krist \hld\ \alst{m}ênjan weldi, &
\alst{b}ôknjen mid þiu \alst{b}iliðju \hld\ \alst{b}arno ríkjost. &
Bádun þó só \alst{g}erno \hld\ \alst{g}ódan drohtin &
ant·\alst{l}úkan þea \alst{l}êra, \hld\ þat sia móstin þea \alst{l}iudi forð, &
\alst{h}êlaga \alst{h}ôrjan. \hld\ Þó sprak im eft iro \alst{h}êrro an·gęgin, &
\alst{m}ári \alst{m}ahtig Krist: \hld\ „þat is“, kwað hé, „\alst{m}annes sunu: &
ik \alst{s}elvo bium, þat þár \alst{s}áiu, \hld\ ęndi sind þesa \alst{s}áliga man &
þat \alst{h}luttra \alst{h}rên-korni, \hld\ þea mí hér \alst{h}ôrjad wel, &
\alst{w}irkjad mínan \alst{w}illjan; \hld\ þius \alst{w}er-old is þe akkar, &
þit \alst{b}rêda \alst{b}ú-land \hld\ \alst{b}arno man-kunnjes; &
\alst{S}atanas \alst{s}elvo is, \hld\ þat þár \alst{s}áid aftar &
só \alst{l}êð-líka \alst{l}êra: \hld\ havad þesaro \alst{l}iudjo só filu, &
\alst{w}erodes a·\alst{w}ardid, \hld\ þat sie \alst{w}am frummjad, &
\alst{w}irkjad aftar is \alst{w}illjon; \hld\ þoh skulun sie hér \alst{w}ahsen forð, &
þea for·\alst{g}riponon \alst{g}umon, \hld\ só samo só þea \alst{g}ódun man, &
ant-tat \alst{M}úd-spelles mę9gin \hld\ ovar \alst{m}an fęrid, &
\alst{ę}ndi þesaro wer-oldes. \hld\ Þan is allaro \alst{a}kkaro ge·hwi-lik &
ge·\alst{r}ípod an þesumu \alst{r}íkja: \hld\ skulun iro \alst{r}egan-gi·skapu &
\alst{f}rummjen \alst{f}iriho barn. \hld\ Þan te·\alst{f}arid erða: &
þat is allaro \alst{b}ewo \alst{b}rêdost; \hld\ þan kumid þe \alst{b}erhto drohtin &
\alst{o}vana mid is \alst{ę}ngilo kraftu, \hld\ ęndi kumad \alst{a}lle te·samne &
\alst{l}iudi, þe io þit \alst{l}ioht gi·sáun, \hld\ ęndi skulun þan \alst{l}ôn ant·fȧhan &
\alst{u}viles ęndi gódes. \hld\ Þan gangad \alst{ę}ngilos godes, &
\alst{h}êlage \alst{h}evan-wardos, \hld\ ęndi lesat þea \alst{h}luttron man &
\alst{s}undọr te·\alst{s}amne, \hld\ ęndi duat sie an \alst{s}in-skôni, &
\alst{h}ôh \alst{h}imiles lioht, \hld\ ęndi þea ȯðra an \alst{h}ęllja grund, &
\alst{w}erpad þea far·\alst{w}arhton \hld\ an \alst{w}allandi fiur; &
þár skulun sie gi·\alst{b}undene \hld\ \alst{b}ittra logna, &
\alst{þ}rá-werk \alst{þ}olon, \hld\ ęndi þea ȯðra \alst{þ}iod-welon &
an \alst{h}evan-ríkja, \hld\ \alst{h}wítaro sunnon &
\alst{l}iohtjan ge·\alst{l}íko. \hld\ Su-lik \alst{l}ôn nimad &
\alst{w}eros \alst{w}al-dádjo. \hld\ Só hwe só gi·\alst{w}it êgi, &
ge·\alst{h}ugdi an is \alst{h}ertan, \hld\ eþþa gi·\alst{h}ôrjen mugi, &
\alst{e}rl mid is \alst{ô}run, \hld\ só láta imu þit an \alst{i}nnan sorga, &
an is \alst{m}ód-sevon, \hld\ hwó hé skal an þemu \alst{m}árjon dage &
wið þene \alst{r}íkjon god \hld\ an \alst{r}ęðju standen &%TODO: check reðju
\alst{w}ordo ęndi \alst{w}erko allaro, \hld\ þe hé an þesaro \alst{w}er-oldi gi·duod. &
Þat is \alst{ę}gis-líkost \hld\ \alst{a}llaro þingo, &
\alst{f}orht-líkost \alst{f}iriho barnun, \hld\ þat sie skulun wið iro \alst{f}râhon mahljen, &
\alst{g}umon wið þene \alst{g}ódan drohtin: \hld\ þan weldi \alst{g}erno ge·hwe wesan, &
allaro \alst{m}anno ge·hwi-lik \hld\ \alst{m}ênes tómig, &
\alst{s}líðero \alst{s}akono. \hld\ Aftar þiu skal \alst{s}orgon êr &
allaro \alst{l}iudjo ge·hwi-lik, \hld\ êr hé þit \alst{l}ioht af·geve, &
þe þan \alst{ê}gan wili \hld\ \alst{a}lungan tír, &
\alst{h}ôh \alst{h}evan-ríki \hld\ ęndi \alst{h}uldi godes.“\eva

\bvb TODO.\evb\evg

\bvg\bva[32][2621]%
Só gi·fragn ik þat þó \alst{s}elvo \hld\ \alst{s}unu drohtines, &
allaro \alst{b}arno \alst{b}ętst \hld\ \alst{b}iliðjo sagda, &
hwi-lik þero \alst{w}ári \hld\ an \alst{w}er-old-ríkja &
undar \alst{h}ęlið-kunnje \hld\ \alst{h}imil-ríkje ge·lík; &
kwað þat oft \alst{l}uttiles hwat \hld\ \alst{l}iohtora wurði, &
só \alst{h}ôho af·\alst{h}uovi, \hld\ „so duot \alst{h}imil-ríki: &
þat is simla \alst{m}êra, \hld\ þan is \alst{m}an ênig &
\alst{w}ánje an þesaro \alst{w}er-oldi. \hld\ Ôk is imu þat \alst{w}erk ge·lík, &
þat man an \alst{s}êo innan \hld\ \alst{s}ęgina wirpit, &
\alst{f}isk-nęt an \alst{f}lód \hld\ ęndi \alst{f}áhit bêðju, &
\alst{u}vile ęndi góde, \hld\ tiuhid \alst{u}p te staðe, &
\alst{l}iðod sie te \alst{l}ande, \hld\ \alst{l}isit aftar þiu &
þea \alst{g}ódun an \alst{g}reote \hld\ ęndi látid þea ȯðra eft an \alst{g}rund faran, &
an \alst{w}ídan \alst{w}ág. \hld\ Só duod \alst{w}aldand god &
an þemu \alst{m}árjon dage \hld\ \alst{m}ęnniskono barn: &
brengid \alst{i}rmin-þiod, \hld\ \alst{a}lle te·samne, &
lisit imu þan þea \alst{h}luttron \hld\ an \alst{h}evan-ríki, &
látid þea far·\alst{g}riponon \hld\ an \alst{g}rund faren &
\alst{h}ęllje fiures. \hld\ Ni wêt \alst{h}ęliðo man &
þes \alst{w}ítjes \alst{w}iðar-lága, \hld\ þes þár \alst{w}eros þiggjat, &
an þemu \alst{I}nferne \hld\ \alst{i}rmin-þioda. &
Þan hald ni mag þera \alst{m}édan \alst{m}an \hld\ gi·\alst{m}akon fïðen, &
ni þes \alst{w}elon ni þes \alst{w}illjon, \hld\ þes þár \alst{w}aldand skerid, &%NOTE: skerid uncertain; skeran or skęrjan?
\alst{g}ildid \alst{g}od selvo \hld\ \alst{g}umono só hwi-likumu, &
só ina \alst{h}ér gi·\alst{h}aldid, \hld\ þat hé an \alst{h}evan-ríki, &
an þat \alst{l}ang-same \alst{l}ioht \hld\ \alst{l}íðan móti.“ &
Só \alst{l}êrda hé þó mid \alst{l}istjun. \hld\ Þan fórun þár þea \alst{l}iudi tó &
ovar al \alst{G}alilaeo land \hld\ þat \alst{g}odes barn sehan: &
dádun it bi þemu \alst{w}undre, \hld\ hwanen imu mahti su·lik \alst{w}ord kumen, &
só \alst{sp}áh-líko gi·\alst{sp}rokan, \hld\ þat hé \alst{sp}el godes &
gio só \alst{s}ȯð-líko \hld\ \alst{s}ęggjan konsti, &
só \alst{k}raftig-líko gi·\alst{k}weðen: \hld\ „Hé is þeses \alst{k}unnjes hinen“, kwáðun sie, &
„þe man þurh \alst{m}ág-skępi: \hld\ hér is is \alst{m}óder mid u̇s, &
\alst{w}íf undar þesumu \alst{w}erode. \hld\ Hwat wí þe hér \alst{w}itun alle, &
só \alst{k}u̇ð is u̇s is \alst{k}uni-burd \hld\ ęndi is \alst{k}nósles ge·hwat; &
a·\alst{w}óhs al undar þesumu \alst{w}erode: \hld\ hwanen skoldi imu su·lik ge·\alst{w}it kuman, &
\alst{m}éron \alst{m}ahti, \hld\ þan hér ȯðra \alst{m}an êgin?“ &
Só far·\alst{m}unste ina þat \alst{m}anno folk \hld\ ęndi sprákun im gi·\alst{m}êd-lik word, &
far·\alst{h}ogdun ina só \alst{h}êlagna, \hld\ \alst{h}ôrjen ni weldun &
is gi·\alst{b}od-skępjes. \hld\ Ni hé þár ôk \alst{b}iliðjo filu &
þurh iro \alst{u}n-gi·lôvon \hld\ \alst{ó}gjan ni welde, &
\alst{t}orhtero \alst{t}êkno, \hld\ hwand hé wisse iro \alst{t}wífljan hugi, &
iro \alst{w}rêðan \alst{w}illjan, \hld\ þat ni wárun \alst{w}eros ȯðra &
só \alst{g}rimme under \alst{J}udeon, \hld\ só wárun umbi \alst{G}alilaeo land, &
só \alst{h}ardo ge·\alst{h}ugide: \hld\ só þár was þe \alst{h}êlago Krist, &
gi·\alst{b}oren þat \alst{b}arn godes, \hld\ si ni weldun is gi·\alst{b}od-skępi þoh &
ant·\alst{f}ȧhan \alst{f}erht-líko, \hld\ ak bi·gan þat \alst{f}olk undar im, &
\alst{r}inkos \alst{r}ádan, \hld\ hwó sie þene \alst{r}íkjon Krist &
\alst{w}êgdin te \alst{w}undron. \hld\ Hétun þó iro \alst{w}erod kumen, &
ge·\alst{s}ïði te·\alst{s}amne: \hld\ \alst{s}undja weldun &
an þene \alst{g}odes sunu \hld\ \alst{g}erno gi·tęlljen &
\alst{w}rêðes \alst{w}illjon; \hld\ ni was im is \alst{w}ordo niud, &
\alst{sp}áharo \alst{sp}ello, \hld\ ak sie bi·gunnun \alst{sp}rekan undar im, &
hwó sie ina só \alst{k}raftagne \hld\ fan ênumu \alst{k}live wurpin, &
ovar ênna \alst{b}erges wal: \hld\ weldun þat \alst{b}arn godes &
\alst{l}ivu bi·\alst{l}ôsjen. \hld\ Þó hé imu mid þem \alst{l}iudjun samad &
\alst{f}rô-líko \alst{f}ór: \hld\ ni was imu \alst{f}orạht hugi, &
—wisse þat imu ni \alst{m}ahtun \hld\ \alst{m}ęnniskono barn, &
bi þeru \alst{g}od-kundi \hld\ \alst{J}udeo liudi &
êr is \alst{t}ídjun wiht \hld\ \alst{t}eonon gi·frummjen, &
\alst{l}êðaro gi·\alst{l}êsto—, \hld\ ak hé imu mid þem \alst{l}iudjun samad &
\alst{st}êg uppen þene \alst{st}ên-holm, \hld\ ant-þat sie te þeru \alst{st}ędi kwámun, &
þár sie ine fan þemu \alst{w}alle niðer \hld\ \alst{w}erpen hugdun, &
\alst{f}ęlljen te \alst{f}oldu, \hld\ þat hé wurði is \alst{f}erhes lôs, &
is \alst{a}ldres at \alst{ę}ndje. \hld\ Þó warð þero \alst{e}rlo hugi, &
an þemu \alst{b}erge uppen \hld\ \alst{b}ittra gi·þȧhti &
\alst{J}uðeono te·\alst{g}angen, \hld\ þat iro ênig ni habde só \alst{g}rimmon sevon &
ni só \alst{w}rêðen \alst{w}illjon, \hld\ þat sie mahtin þene \alst{w}aldandes sunu, &
\alst{K}rist ant·\alst{k}ęnnjen; \hld\ hé ni was iro \alst{k}u̇ð ênigumu, &
þat sie ina þó undar·\alst{w}issin. \hld\ Só mahte hé undar ira \alst{w}erode standen &
ęndi an iro gi·\alst{m}ange \hld\ \alst{m}iddjumu gangen, &
\alst{f}aren undar iro \alst{f}olke. \hld\ Hé dede imu þene \alst{f}riðu selvo, &
\alst{m}und-burd wið þeru \alst{m}ęnegi \hld\ ęndi gi·wêt imu þurh \alst{m}iddi þanan &
þes \alst{f}íundo \alst{f}olkes, \hld\ \alst{f}ór imu þó, þár hé welde, &
an êne \alst{w}óstunnje \hld\ \alst{w}aldandes sunu, &
\alst{k}uningo \alst{k}raftigost: \hld\ habde þero \alst{k}ustes gi·wald, &
hwár imu an þemu \alst{l}ande \hld\ \alst{l}eovost wári &
te \alst{w}esanne an þesaru \alst{w}er-oldi.\eva

\bvb TODO.\evb\evg

\bvg\bva[33][2698]%
\hspace*{100pt} Þan fór imu an \alst{w}eg ȯðran &%NOTE: In cæsura.
\alst{J}ohannes mid is \alst{j}ungarun, \hld\ \alst{g}odes ambaht-man, &
\alst{l}êrde þea \alst{l}iudi \hld\ \alst{l}ang-samane rád, &
hét þat sie \alst{f}rume fręmidin, \hld\ \alst{f}irina far·létin, &
\alst{m}ên ęndi \alst{m}orð-werk. \hld\ Hé was þár \alst{m}anagumu liof &
\alst{g}ódaro \alst{g}umono. \hld\ Hé sóhte imu þó þene \alst{J}udeono kuning, &
þene \alst{h}ęri-togon at \alst{h}ús, \hld\ þe \alst{h}êten was &
\alst{E}rodes aftar is \alst{ę}ldiron, \hld\ \alst{o}var-módig man: &
\alst{b}úide imu be þeru \alst{b}rúdi, \hld\ þiu êr sínes \alst{b}róðer was, &
\alst{i}dis an \alst{ê}hti, \hld\ ant-tat hé \alst{ę}lljor skók, &
\alst{w}er-old \alst{w}eslode. \hld\ Þó imu þat \alst{w}íf gi·nam &
þe \alst{k}uning te \alst{k}wenun; \hld\ êr wárun iro \alst{k}ind ôdan, &
\alst{b}arn be is \alst{b}róðer. \hld\ Þó bi·gan imu þea \alst{b}rúd lahan &
\alst{J}ohannes þe \alst{g}ódo, \hld\ kwað þat it \alst{g}ode wári, &
\alst{w}aldande \alst{w}iðẹr-mód, \hld\ þat it ênig \alst{w}ero frumidi, &
þat \alst{b}róðer \alst{b}rúd \hld\ an is \alst{b}ęd námi, &
\alst{h}ębbje sie imu te \alst{h}íwun. \hld\ „Ef þú mí \alst{h}ôrjen wili, &
gi·\alst{l}ôvjen mínun \alst{l}êrun, \hld\ ni skalt þú sie \alst{l}ęng êgan, &
ak míð ire an þínumu \alst{m}óde: \hld\ ni hava þár su·lika \alst{m}innja tó, &
ni \alst{s}undjo þi te \alst{s}wíðo.“ \hld\ Þó warð an \alst{s}orgun hugi &
þes \alst{w}íves aftar þem \alst{w}ordun; \hld\ an·dréd þat hé þene \alst{w}er-old-kuning &
\alst{sp}rákono ge·\alst{sp}óni \hld\ ęndi \alst{sp}áhun wordun, &
þat hé sie far·\alst{l}éti. \hld\ Be·gan siu imu þó \alst{l}êðes filu &
\alst{r}áden an \alst{r}únon, \hld\ ęndi ine \alst{r}inkos hét, &
\alst{u}n-sundigane \hld\ \alst{e}rlos fȧhan &
ęndi ine an ênumu \alst{k}arkerja \hld\ \alst{k}lústar-bęndjun, &
\alst{l}iðo-kospun bi·\alst{l}úkan: \hld\ be þem \alst{l}iudjun ne gi·dorstun &
ine \alst{f}erạhu bi·lôsjen, \hld\ hwand sie wárun imu \alst{f}riund alle, &
wissun ine só \alst{g}óden \hld\ ęndi \alst{g}ode werðen, &
habdun ina for \alst{w}ár-sagon, \hld\ só sia \alst{w}ela mahtun. &
Þó wurðun an þemu \alst{g}ę́r-tale \hld\ \alst{J}udeo kuninges &
\alst{t}ídi kumana, \hld\ só þár gi·\alst{t}ald habdun &
\alst{f}róde \alst{f}olk-weros, \hld\ þó hé gi·\alst{f}ódid was, &
an \alst{l}ioht kuman. \hld\ Só was þero \alst{l}iudjo þau, &
þat þat \alst{e}rlo ge·hwi-lik \hld\ \alst{ó}vjan skolde, &
\alst{J}udeono mid \alst{g}ômun. \hld\ Þó warð þár an þene \alst{g}ast-sęli &
\alst{m}ęgin-kraft \alst{m}ikil \hld\ \alst{m}anno ge·samnod, &
\alst{h}ęri-togono an þat \alst{h}ús, \hld\ þár iro \alst{h}êrro was &
an is \alst{k}uning-stóle. \hld\ \alst{K}wámun managa &
\alst{J}udeon an þene \alst{g}ast-sęli; \hld\ warð im þár \alst{g}lad-mód hugi, &
\alst{b}líði an iro \alst{b}reostun: \hld\ gi·sáhun iro \alst{b}âg-gevon &
\alst{w}esen an \alst{w}unnjon. \hld\ Dróg man \alst{w}ín an flęt &
\alst{sk}íri mid \alst{sk}álun, \hld\ \alst{sk}ęnkjon hwurvun, &
\alst{g}éngun mid \alst{g}old-fatun: \hld\ \alst{g}aman was þár inne &
\alst{h}lúd an þero \alst{h}allu, \hld\ \alst{h}ęliðos drunkun. &
Was þes an \alst{l}ustun \hld\ \alst{l}andes hirdi, &
hwat hé þemu \alst{w}erode mêst \hld\ te \alst{w}unnjun gi·fręmidi. &
Hét hé þó \alst{g}angen forð \hld\ \alst{g}êla þiornun, &
is \alst{b}róder \alst{b}arn, \hld\ þár hé an is \alst{b}ęnki sat &
\alst{w}ínu gi·\alst{w}lęnkid, \hld\ ęndi þó te þemu \alst{w}íve sprak; &
\alst{g}rótte sie fora þemu \alst{g}um-skępje \hld\ ęndi \alst{g}erno bad, &
þat siu þár fora þem \alst{g}astjun \hld\ \alst{g}aman af·hóvi &
\alst{f}agạr an \alst{f}lęttje: \hld\ „lát þit \alst{f}olk sehan, &
hwó þú ge·\alst{l}ínod havas \hld\ \alst{l}iudjo męnegi &
te \alst{b}líðsjanne an \alst{b}ęnkjun; \hld\ ef þú mí þera \alst{b}ede tugiðos, &
mín \alst{w}ord for þesumu \alst{w}erode, \hld\ þan willju ik it hér te \alst{w}árun ge·kweðen, &
\alst{l}iahto fora þesun \alst{l}iudjun \hld\ ęndi ôk gi·\alst{l}êstjen só, &
þat ik þí þan \alst{a}ftar þiu \hld\ \alst{ê}ron willju, &
só hwes só þú mí \alst{b}idis \hld\ for þesun mínun \alst{b}âg-winjun: &
þoh þú mí þesaro \alst{h}ęri-dómo \hld\ \alst{h}alvaro fergos, &
\alst{r}íkjas mínes, \hld\ þoh gi·dón ik, þat it ênig \alst{r}inko ni mag &
\alst{w}ordun gi·\alst{w}ęndjen, \hld\ ęndi it skal gi·\alst{w}erðen só.“ &
Þó warð þera \alst{m}agað aftar þiu \hld\ \alst{m}ód gi·hworven, &
\alst{h}ugi aftar iro \alst{h}êrron, \hld\ þat siu an þemu \alst{h}úse innen, &
an þemu \alst{g}ast-sęli \hld\ \alst{g}amen up a·huof, &
al só þero \alst{l}iudjo \hld\ \alst{l}and-wíse gi·dróg, &
þero \alst{þ}iodo \alst{þ}au. \hld\ Þiu \alst{þ}iorne spilode &
\alst{h}rór aftar þemu \alst{h}úse: \hld\ \alst{h}ugi was an lustun, &%NOTE: hrór checked.
\alst{m}anagaro \alst{m}ód-sevo. \hld\ Þó þiu \alst{m}agað habda &
gi·\alst{þ}ionod te \alst{þ}anke \hld\ \alst{þ}iod-kuninge &
ęndi \alst{a}llumu þemu \alst{e}rl-skępje, \hld\ þe þár \alst{i}nne was &
\alst{g}ódaro \alst{g}umono, \hld\ siu welde þó ira \alst{g}eva êgan, &
þiu \alst{m}agað for þeru \alst{m}ęnegi: \hld\ géng þó wið iro \alst{m}ódar sprekan &
ęndi \alst{f}rágode sie \hld\ \alst{f}iri-wit-líko, &
hwes siu þene \alst{b}urges ward \hld\ \alst{b}iddjen skoldi. &
Þó \alst{w}ísde siu aftar iro \alst{w}illjon, \hld\ hét þat siu \alst{w}ihtes þan êr &
ni \alst{g}ęrodi for þemu \alst{g}um-skępje, \hld\ bi·útan þat man iru \alst{J}ohannes &
an þeru \alst{h}allu innan \hld\ \alst{h}ôvid gávi &
a·\alst{l}ôsid af is \alst{l}ík-hamon. \hld\ Þat was allun þem \alst{l}iudjun harm, &
þem \alst{m}annun an iro \alst{m}óde, \hld\ þó sie þat gi·hôrdun þea \alst{m}agað sprekan; &
só was it ôk þemu \alst{k}uninge: \hld\ hé ni mahte is \alst{k}widi liagan, &
is \alst{w}ord \alst{w}ęndjen: \hld\ hét þó is \alst{w}ę́pạn-berand &
\alst{g}angen fan þemu \alst{g}ast-sęli \hld\ ęndi hét þene \alst{g}odes man &
\alst{l}ívu bi·\alst{l}ôsjen. \hld\ Þó ni was \alst{l}ang te þiu, &
þat man an þea \alst{h}alla \hld\ \alst{h}ôvid brȧhte &
\alst{þ}es \alst{þ}iod-gumon, \hld\ ęndi it þár þeru \alst{þ}iornun far·gaf, &
\alst{m}agað for þeru \alst{m}ęnegi: \hld\ siu dróg it þeru \alst{m}óder forð. &
Þó was \alst{ê}n-dago \hld\ \alst{a}llaro manno &
þes \alst{w}ísoston, \hld\ þero þe gio an þesa \alst{w}er-old kwámi, &
þero þe \alst{k}wene ênig \hld\ \alst{k}ind gi·bári, &
\alst{i}dis fan \alst{e}rle, \hld\ lét man simla þen \alst{ê}non bi·foran, &
þe þiu \alst{þ}iorne gi·dróg, \hld\ þe gio \alst{þ}egnes ni warð &
\alst{w}ís an iro \alst{w}er-oldi, \hld\ bi·útan só ine \alst{w}aldand god &
fan \alst{h}evan-wange \hld\ \alst{h}êlages gêstes &
gi·\alst{m}arkode \alst{m}ahtig: \hld\ þe ni habde ênigan gi·\alst{m}akon hwęrgin &
\alst{ê}r nek \alst{a}ftar. \hld\ \alst{E}rlos hwurvun, &%NOTE: nek checked.
\alst{g}umon umbi \alst{J}ohannen, \hld\ is \alst{j}ungaron managa, &
\alst{s}álig ge·\alst{s}ïði, \hld\ ęndi ine an \alst{s}ande bi·gróvun, &
\alst{l}eoves \alst{l}ík-hamon: \hld\ wissun þat hé \alst{l}ioht godes, &
\alst{d}iur-líkan \alst{d}rôm \hld\ mid is \alst{d}rohtine samad, &
\alst{u}p-\alst{ô}das hêm \hld\ \alst{ê}gan móste, &
\alst{s}álig \alst{s}ókjan.\eva

\bvb TODO.\evb\evg

\bvg\bva[34][2799]%
\hspace*{60pt} Þó ge·witun im þea ge·\alst{s}ïðos þanen, &%NOTE: In cæsura.
\alst{J}ohannes \alst{j}ungaron \hld\ \alst{j}ámer-móde, &
\alst{h}êlag-ferạha: \hld\ was im iro \alst{h}êrron dôð &
\alst{s}wíðo an \alst{s}orgun. \hld\ Ge·witun im \alst{s}ókjan þó &
an þeru \alst{w}óstunni \hld\ \alst{w}aldandes sunu, &
\alst{k}raftigana \alst{K}rist \hld\ ęndi imu \alst{k}u̇ð gi·dedun &
\alst{g}ódes mannes for·\alst{g}ang, \hld\ hwó habde þe \alst{J}udeono kuning &
\alst{m}anno þene \alst{m}árjostan \hld\ \alst{m}ákjas ęggjun &
\alst{h}ôvdu bi·\alst{h}auwan: \hld\ hé ni welde is ênigen \alst{h}arm spreken, &
\alst{s}unu drohtines; \hld\ hé wisse þat þiu \alst{s}eole was &
\alst{h}êlag gi·\alst{h}alden \hld\ wiðẹr \alst{h}ęttjandjon, &
an \alst{f}riðe wiðẹr \alst{f}íundun. \hld\ Þó só gi·\alst{f}rági warð &
aftar þem \alst{l}and-skępjun \hld\ \alst{l}êrjandero bętst &
an þeru \alst{w}óstunni: \hld\ \alst{w}erod samnode, &
\alst{f}ór \alst{f}olkun tó: \hld\ was im \alst{f}iri-wit mikil &
\alst{w}ísaro \alst{w}ordo; \hld\ imu was ôk \alst{w}illjo só samo, &
\alst{s}unje drohtines, \hld\ þat hé su·lik ge·\alst{s}ïðo folk &
an þat \alst{l}ioht godes \hld\ \alst{l}aðojan mósti, &
\alst{w}ęnnjen mid \alst{w}illjon. \hld\ \alst{W}aldand lêrde &
allan \alst{l}angan dag \hld\ \alst{l}iudi managa, &
\alst{ę}li-þeodige man, \hld\ ant-tat an \alst{á}vand sêg &
\alst{s}unne te \alst{s}edle. \hld\ Þó géngun is ge·\alst{s}ïðos twe-livi, &
\alst{g}umon te þemu \alst{g}odes barne \hld\ ęndi sagdun iro \alst{g}ódumu hêrron, &
mid hwi-liku \alst{a}rvêdju þár þea \alst{e}rlos livdin, \hld\ kwáðun þat sie is \alst{ê}ra bi·þorftin, &
\alst{w}eros an þemu \alst{w}óstjon lande: \hld\ „sie ni mugun sie hér mid \alst{w}ihti ant·hębbjen, &
\alst{h}ęliðos bi \alst{h}ungres ge·þwinge. \hld\ Nú lát þú sie, \alst{h}êrro þe gódo, &
\alst{s}ïðon, þár sie \alst{s}ęliða fïðen. \hld\ Náh sind hér ge·\alst{s}etana burgi &
\alst{m}anaga mid \alst{m}ęgin-þiodun: \hld\ þár fïðad sie \alst{m}ęti te kôpe, &
\alst{w}eros aftar þem \alst{w}íkjon.“ \hld\ Þó sprak eft \alst{w}aldand Krist, &
\alst{þ}ioda drohtin, \hld\ kwað þat þes êniga \alst{þ}urụfti ni wárin, &
„þat sie þurh \alst{m}ęti-lôsi \hld\ \alst{m}ína far·látan &
\alst{l}eov-líka \alst{l}êra. \hld\ Gevad gí þesun \alst{l}iudjun gi·nóg, &
\alst{w}ęnnjad sie hér mid \alst{w}illjon.“ \hld\ Þó habde eft is \alst{w}ord garu &
\alst{Ph}ilippus \alst{f}ród gumo, \hld\ kwað þat þár só \alst{f}ilu wári &
\alst{m}anno \alst{m}ęnigi: \hld\ „þoh wí hér te \alst{m}ęti habdin &
\alst{g}aru im te \alst{g}evanne, \hld\ só wí mahtin far·\alst{g}elden mêst, &
ef wí hér gi·\alst{s}aldin \hld\ \alst{s}ilụver-skatto &
\alst{t}wê hund samad, \hld\ \alst{t}weho wári is noh þan, &
þat iro \alst{ê}nig þár \hld\ \alst{ê}nes gi·námi: &
só \alst{l}uttik wári þat þesun \alst{l}iudjun.“ \hld\ Þó sprak eft þe \alst{l}andes ward &%NOTE: luttik checked.
ęndi \alst{f}rágode sie \hld\ \alst{f}iri-wit-líko, &
\alst{m}anno drohtin, \hld\ hwat sie þár te \alst{m}ęti habdin &
\alst{w}istes ge·\alst{w}unnin. \hld\ Þó sprak imu eft mid is \alst{w}ordun an·gęgin &
\alst{A}ndreas fora þem \alst{e}rlun \hld\ ęndi þemu \alst{a}lo-waldon &
\alst{s}elvumu \alst{s}agde, \hld\ þat sie an iro gi·\alst{s}ïðje þan mêr &
\alst{g}arowes ni habdin, \hld\ „bi·útan \alst{g}irstin brôd &
\alst{f}ïvi an u̇saru \alst{f}ęrdi \hld\ ęndi \alst{f}iskos twêne. &
Hwat mag þat þoh þesaru \alst{m}ęnigi?“ \hld\ Þó sprak imu eft \alst{m}ahtig Krist, &
þe \alst{g}ódo \alst{g}odes sunu, \hld\ ęndi hét þat \alst{g}umono folk &
\alst{sk}ęrjen ęndi \alst{sk}êðen \hld\ ęndi hét þea \alst{sk}ola sęttjen, &
\alst{e}rlos aftar þeru \alst{e}rðu, \hld\ \alst{i}rmin-þioda &
an \alst{g}rase \alst{g}ruonimu, \hld\ ęndi þó te is \alst{j}ungarun sprak, &
allaro \alst{b}arno \alst{b}ętst, \hld\ hét imu þiu \alst{b}rôd halon &
ęndi þea \alst{f}iskos \alst{f}orð. \hld\ Þat \alst{f}olk stillo bêd, &
\alst{s}at ge·\alst{s}ïði mikil; \hld\ undar þiu hé þurh is \alst{s}elves kraft, &
\alst{m}anno drohtin, \hld\ þene \alst{m}ęti wíhide, &
\alst{h}êlag \alst{h}evan-kuning, \hld\ ęndi mid is \alst{h}andun brak, &
\alst{g}af it is \alst{j}ungarun forð, \hld\ ęndi it sie undar þemu \alst{g}um-skępje hét &
\alst{d}ragan ęndi \alst{d}êljen. \hld\ Sie lêstun iro \alst{d}rohtines word, &
is \alst{g}eva \alst{g}erno drógun \hld\ \alst{g}umono gi·hwemu, &
\alst{h}êlaga \alst{h}elpa. \hld\ It undar iro \alst{h}andun wóhs, &
\alst{m}ęti \alst{m}anno gi·hwemu: \hld\ þeru \alst{m}ęgin-þiodu warð &
\alst{l}íf an \alst{l}ustun, \hld\ þea \alst{l}iudi wurðun alle, &
\alst{s}ade \alst{s}álig folk, \hld\ só hwat só þár gi·\alst{s}amnod was &
fan allun \alst{w}ídun \alst{w}egun. \hld\ Þó hét \alst{w}aldand Krist &
\alst{g}angen is \alst{j}ungaron \hld\ ęndi hét sie \alst{g}ômjen wel, &
þat þiu \alst{l}éva þár \hld\ far·\alst{l}oren ni wurði; &
hét \alst{s}ie þó \alst{s}amnon, \hld\ þó þár \alst{s}ade wárun &
\alst{m}an-kunnjes \alst{m}anag. \hld\ Þár \alst{m}óses warð, &
\alst{b}rôdes te lévu, \hld\ þat man \alst{b}irilos gi·las &
\alst{t}we-livi fulle: \hld\ þat was \alst{t}êkạn mikil, &
\alst{g}rôt kraft \alst{g}odes, \hld\ hwand þár was \alst{g}umono gi·tald &
áno \alst{w}íf ęndi kind, \hld\ \alst{w}erodes at·samme &
\alst{f}ïf þúsundig. \hld\ Þat \alst{f}olk al far·stód, &
þea \alst{m}an an iro \alst{m}óde, \hld\ þat sie þár \alst{m}ahtigna &
\alst{h}êrron \alst{h}abdun. \hld\ Þó sie \alst{h}evan-kuning, &
þea \alst{l}iudi \alst{l}ovodun, \hld\ kwáðun þat gio ni wurði an þit \alst{l}ioht kuman &
\alst{w}ísaro \alst{w}ár-sago, \hld\ efþa þat hé gi·\alst{w}ald mid gode &
an þesaru \alst{m}iddil-gard \hld\ \alst{m}éron habdi, &
\alst{ê}n-faldaran hugi. \hld\ \alst{A}lle gi·sprákun, &
þat hé \alst{w}ári \alst{w}irðig \hld\ \alst{w}elono ge·hwi-likes, &
þat hé \alst{e}rð-ríki \hld\ \alst{ê}gan mósti, &
\alst{w}ídene \alst{w}er-old-stól, \hld\ „nú hé su·lik ge·\alst{w}it havad, &
só \alst{g}rôte kraft mid \alst{g}ode.“ \hld\ Þea \alst{g}umon alle gi·warð, &
þat sie ine gi·\alst{h}óvin \hld\ te \alst{h}êrosten, &
gi·\alst{k}urin ine te \alst{k}uninge: \hld\ þat \alst{K}riste ni was &
\alst{w}ihtes \alst{w}irðig, \hld\ hwand hé þit \alst{w}er-old-ríki, &
\alst{e}rðe ęndi \alst{u}p-himil \hld\ þurh is \alst{ê}nes kraft &
\alst{s}elvo gi·warhte \hld\ ęndi \alst{s}ïðor gi·held, &
\alst{l}and ęndi \alst{l}iud-skępi, \hld\ —þoh þes ênigan gi·\alst{l}ôvon ni dedin &
\alst{w}rêðe \alst{w}iðẹr-sakon— \hld\ þat al an is gi·\alst{w}alde stád, &
\alst{k}uning-ríkjo \alst{k}raft \hld\ ęndi \alst{k}êsur-dómes, &
\alst{m}ęgin-þiodo \alst{m}ahal. \hld\ Be·þiu ni welde hé þurh þero \alst{m}anno spráka &
\alst{h}ębbjan ênigan \alst{h}êr-dóm, \hld\ \alst{h}êlag drohtin, &
\alst{w}er-old-kuninges namon; \hld\ ni hé þó mid \alst{w}ordun stríd &
ni af·hóf wið þat \alst{f}olk \alst{f}urður, \hld\ ak \alst{f}ór imu þó, þár hé welde, &
an ên ge·\alst{b}irgi uppan: \hld\ flóh þat \alst{b}arn godes &
\alst{g}êlaro \alst{g}elp-kwidi \hld\ ęndi is \alst{j}ungaron hét &
ovar ênne \alst{s}êo \alst{s}ïðon \hld\ ęndi im \alst{s}elvo gi·bôd, &
hwár sie im eft te·\alst{g}ęgnes \hld\ \alst{g}angen skoldin.\eva

\bvb TODO.\evb\evg

\bvg\bva[35][2899]%
Þó te·\alst{l}ét þat \alst{l}iud-werod \hld\ aftar þemu \alst{l}ande allumu, &
te·\alst{f}ór \alst{f}olk mikil, \hld\ sïðor iro \alst{f}râho gi·wêt &
an þat ge·\alst{b}irgi uppan, \hld\ \alst{b}arno ríkjost, &
\alst{w}aldand an is \alst{w}illjon. \hld\ Þó te þes \alst{w}atares staðe &
\alst{s}amnodun þea ge·\alst{s}ïðos Kristes, \hld\ þe hé imu habde \alst{s}elvo gi·korane, &
sie \alst{t}welivi þurh iro \alst{t}reuwa góda: \hld\ ni was im \alst{t}weho nigijan, &
nevu sie an þat \alst{g}odes þionost \hld\ \alst{g}erno weldin &
ovar þene \alst{s}êo \alst{s}ïðon. \hld\ Þó létun sie \alst{s}wíðjan strôm, &
\alst{h}ôh \alst{h}urnid-skip \hld\ \alst{h}luttron u̇ðjon, &
\alst{sk}êðan \alst{sk}ír water. \hld\ \alst{Sk}rêd lioht dages, &
\alst{s}unne warð an \alst{s}edle; \hld\ þe \alst{s}êo-líðandjan &
\alst{n}aht \alst{n}evulo bi·warp; \hld\ \alst{n}áðidun erlos &
\alst{f}orð-wardes an \alst{f}lód; \hld\ warð þiu \alst{f}iorðe tíd &
þera \alst{n}ahtes kuman \hld\ —\alst{n}ęrjendo Krist &
\alst{w}arode þea \alst{w}ág-líðand—: \hld\ þó warð \alst{w}ind mikil, &
\alst{h}ôh wedẹr af·\alst{h}aven: \hld\ \alst{h}lamodun u̇ðjon, &
\alst{st}rôm an \alst{st}amne; \hld\ \alst{st}rídjun fęridun &
þea \alst{w}eros wiðẹr \alst{w}inde, \hld\ was im \alst{w}rêð hugi, &
\alst{s}evo \alst{s}orgono ful: \hld\ \alst{s}elvon ni wándun &
\alst{l}agu-\alst{l}íðandja \hld\ an \alst{l}and kumen &
þurh þes \alst{w}ederes ge·\alst{w}in. \hld\ Þó gi·sáhun sie \alst{w}aldand Krist &
an þemu \alst{s}êe uppan \hld\ \alst{s}elvun gangan, &
\alst{f}aran an \alst{f}áðjon: \hld\ ni mahte an þene \alst{f}lód innan, &
an þene \alst{s}êo \alst{s}inkan, \hld\ hwand ine is \alst{s}elves kraft &
\alst{h}êlag ant·\alst{h}abde. \hld\ \alst{H}ugi warð an forhtun, &
þero \alst{m}anno \alst{m}ód-sevo: \hld\ an·drédun þat it im \alst{m}ahtig fíund &
te gi·\alst{d}roge \alst{d}ádi. \hld\ Þó sprak im iro \alst{d}rohtin tó, &
\alst{h}êlag \alst{h}evan-kuning, \hld\ ęndi sagde im þat hé iro \alst{h}êrro was &
\alst{m}ári ęndi \alst{m}ahtig: \hld\ „nú gí \alst{m}ódes skulun &
\alst{f}astes \alst{f}áhen; \hld\ ne sí iu \alst{f}orht hugi, &
gi·\alst{b}árjad gí \alst{b}ald-líko: \hld\ ik \alst{b}ium þat barn godes, &
is \alst{s}elves \alst{s}unu, \hld\ þe iu wið þesumu \alst{s}êe skal, &
\alst{m}undon wið þesan \alst{m}ęri-strôm.“ \hld\ Þó sprak imu ên þero \alst{m}anno an·gęgin &
ovar \alst{b}ord skipes, \hld\ \alst{b}ar-wirðig gumo, &
\alst{P}etrus þe gódo \hld\ —ni welde \alst{p}íne þolon, &
\alst{w}atares \alst{w}íti—: \hld\ „ef þú it \alst{w}aldand sís“, kwað hé, &
„\alst{h}êrro þe gódo, \hld\ só mí an mínumu \alst{h}ugi þunkit, &
hêt mí þan þarod \alst{g}angan te þí \hld\ ovar þesen \alst{g}evenes strôm, &
\alst{d}rokno ovar \alst{d}iap water, \hld\ ef þú mín \alst{d}rohtin sís, &
\alst{m}anagoro \alst{m}und-boro.“ \hld\ Þó hét ine \alst{m}ahtig Krist &
\alst{g}angan imu te·\alst{g}ęgnes. \hld\ Hé warð \alst{g}aru sáno, &
\alst{st}ôp af þemu \alst{st}amne \hld\ ęndi \alst{st}rídjun géng &
\alst{f}orð te is \alst{f}rôjan. \hld\ Þiu \alst{f}lód ant·habde &
þene \alst{m}an þurh \alst{m}aht godes, \hld\ ant-tat hé imu an is \alst{m}óde bi·gan &
an·\alst{d}ráden \alst{d}iap water, \hld\ þó hé \alst{d}ríven gi·sah &
þene \alst{w}ég mid \alst{w}indu: \hld\ \alst{w}undun ina u̇ðjon, &
\alst{h}ôh strôm umbi·\alst{h}ring. \hld\ Reht só hé þó an is \alst{h}ugi twehode, &
só \alst{w}êk imu þat \alst{w}ater under, \hld\ ęndi hé an þene \alst{w}ág innan, &
\alst{s}ank an þene \alst{s}êo-strôm, \hld\ ęndi hé hriop \alst{s}án aftar þiu &
\alst{g}áhon te þemu \alst{g}odes sunje \hld\ ęndi \alst{g}erno bad, &
þat hé ine þó ge·\alst{n}ęridi, \hld\ þó hé an \alst{n}ôdjun was, &
\alst{þ}egạn an ge·\alst{þ}winge. \hld\ \alst{Þ}iodo drohtin &
ant·\alst{f}éng ine mid is \alst{f}aðmun \hld\ ęndi \alst{f}rágode sána, &
te hwí hé þó ge·\alst{t}wehodi: \hld\ „Hwat þú mahtes ge·\alst{t}rúojan wel, &
\alst{w}iten þat te \alst{w}árun, \hld\ þat þi \alst{w}atares kraft &
an þemu \alst{s}êe innen \hld\ þínes \alst{s}ïðes ni mahte, &
\alst{l}agu-strôm gi·\alst{l}ęttjen, \hld\ só lango só þú habdes ge·\alst{l}ôvon te mí &
an þínumu \alst{h}ugi \alst{h}ardo. \hld\ Nú willju ik þi an \alst{h}elpun wesen, &
\alst{n}ęrjen þi an þesaru \alst{n}ôdi“. \hld\ Þó \alst{n}am ine alo-mahtig, &
\alst{h}êlag bi \alst{h}andun: \hld\ þó warð imu eft \alst{h}lutter water &
\alst{f}ast under \alst{f}ótun, \hld\ ęndi sie an \alst{f}áði samad &
\alst{b}êðja géngun, \hld\ ant-tat sie ovar \alst{b}ord skipes &
\alst{st}ópun fan þemu \alst{st}rôme, \hld\ ęndi an þemu \alst{st}amne ge·sat &
allaro \alst{b}arno \alst{b}ętst. \hld\ Þó warð \alst{b}rêd water, &
\alst{st}rômos ge·\alst{st}illid, \hld\ ęndi sie te \alst{st}aðe kwámun, &
\alst{l}agu-\alst{l}íðandja \hld\ an \alst{l}and samen &
þurh þes \alst{w}ateres ge·\alst{w}in, \hld\ sagdun þo \alst{w}aldande þank, &
\alst{d}iurden iro \alst{d}rohtin \hld\ \alst{d}ádjun ęndi wordun, &
\alst{f}ellun imu te \alst{f}ótun \hld\ ęndi \alst{f}ilu sprákun &
\alst{w}ísaro \alst{w}ordo, \hld\ kwáðun þat sie \alst{w}issin garo, &
þat hé wári \alst{s}elvo \hld\ \alst{s}unu drohtines &
\alst{w}ár an þesaru \alst{w}er-oldi \hld\ ęndi ge·\alst{w}ald habdi &
ovar \alst{m}iddil-gard, \hld\ ęndi þat hé mahti allaro \alst{m}anno gi·hwes &
\alst{f}erạhe gi·\alst{f}ormon, \hld\ al só hé im an þemu \alst{f}lóde dede &
wið þes \alst{w}atares ge·\alst{w}in.\eva

\bvb TODO.\evb\evg

\bvg\bva[36][2973]%
\hspace*{100pt} Þó gi·wêt imu \alst{w}aldand Krist &%NOTE: in cæsura
\alst{s}ïðon fan þemu \alst{s}êe, \hld\ \alst{s}unu drohtines, &
\alst{ê}nag barn godes. \hld\ \alst{Ę}li-þioda kwam imu, &
\alst{g}umon te·\alst{g}ęgnes: \hld\ wárun is \alst{g}ódun werk &
\alst{f}erran ge·\alst{f}rági, \hld\ þat hé só \alst{f}ilu sagde &
\alst{w}ároro \alst{w}ordo: \hld\ imu was \alst{w}illjo mikil, &
þat hé su·lik \alst{f}olk-skępi \hld\ \alst{f}rummjen mósti, &
þat sie simla \alst{g}erno \hld\ \alst{g}ode þionodin, &
wárin ge·\alst{h}ôrige \hld\ \alst{h}evan-kuninge &
\alst{m}an-kunnjes \alst{m}anag. \hld\ Þó gi·wêt hé imu over þea \alst{m}arka Judeono, &
\alst{s}óhte imu \alst{S}idono burg, \hld\ habde ge·\alst{s}ïðos mid imu, &
\alst{g}óde \alst{j}ungaron. \hld\ Þár imu te·\alst{g}ęgnes kwam &
ên \alst{i}dis fan \alst{ȧ}ðrom þiodun; \hld\ siu was iru \alst{a}ðali-ge·burdjo, &
\alst{k}unnjes fan \alst{K}ananeo lande; \hld\ siu bad þene \alst{k}raftagan drohtin, &
\alst{h}êlagna, þat hé iru \alst{h}elpe ge·rédi, \hld\ kwað þat iru wári \alst{h}arm gi·standen, &
\alst{s}orọga at iru \alst{s}elvaru dohter, \hld\ kwað þat siu wári mid \alst{s}uhtjun bi·fangen: &
„be·\alst{d}rogan habbjad sie \alst{d}ęrnja wihti. \hld\ Nú is iro \alst{d}ôd at hęndi, &
þea \alst{w}rêðon habbjad sie ge·\alst{w}ittju be·numane. \hld\ Nú biddju ik þi, \alst{w}aldand frô min, &
\alst{s}elvo \alst{s}unu Dawides, \hld\ þat sie af su·likum \alst{s}uhtjun a·tómjes, &
þat þú sie só \alst{a}rma \hld\ \alst{ê}-gróht-fullo &
\alst{w}am-skaðon bi·\alst{w}eri.“ \hld\ Ni gaf iru þó noh \alst{w}aldand Krist &
\alst{ê}nig \alst{a}nd-wordi; \hld\ siu imu \alst{a}ftar géng, &
\alst{f}olgode \alst{f}ruokno, \hld\ ant-tat siu te is \alst{f}ótun kwam, &
\alst{g}rótte ina \alst{g}reatandi. \hld\ \alst{J}ungaron Kristes &
bádun iro \alst{h}êrron, \hld\ þat hé an is \alst{h}ugja mildi &
\alst{w}urði þemu \alst{w}íve. \hld\ Þó habde eft is \alst{w}ord garu &
\alst{s}unu drohtines \hld\ ęndi te is ge·\alst{s}ïðun sprak: &
„\alst{ê}rist skal ik \alst{I}sraheles \hld\ \alst{a}voron werðen, &
\alst{f}olk-skępi te \alst{f}rumu, \hld\ þat sie \alst{f}erhtan hugi &
\alst{h}ębbjan te iro \alst{h}êrron: \hld\ im is \alst{h}elpono þarf, &
þea \alst{l}iudi sind far·\alst{l}orane, \hld\ far·\alst{l}áten habbjad &
\alst{w}aldandes \alst{w}ord, \hld\ þat \alst{w}erod is ge·twíflid, &
\alst{d}rívad im \alst{d}ęrnjan hugi, \hld\ ne willjad iro \alst{d}rohtine hôrjen &
\alst{I}srahelo \alst{e}rl-skępi, \hld\ \alst{u}n-gi·lôviga sind &
\alst{h}ęliðos iro \alst{h}êrron: \hld\ þoh skal þanen \alst{h}elpe kumen &
\alst{a}llun \alst{ę}li-þiodun.“ \hld\ \alst{A}galêto bad &
þat \alst{w}íf mid iro \alst{w}ordun, \hld\ þat iru \alst{w}aldand Krist &
an is \alst{m}ód-sevon \hld\ \alst{m}ildi wurði, &
þat siu iro \alst{b}arnes forð \hld\ \alst{b}rúkan mósti, &
\alst{h}ębbjan sie \alst{h}êle. \hld\ Þó sprak iru \alst{h}êrro an·gęgin, &
\alst{m}ári ęndi \alst{m}ahtig: \hld\ „nis þat“, kwað hé, „\alst{m}annes reht, &
\alst{g}umono nig·ênum \hld\ \alst{g}ód te gi·frummjenne &
þat hé is \alst{b}arnun \hld\ \alst{b}rôdes af·tíhe, &
\alst{w}ęrnje im ovar \alst{w}illjon, \hld\ láte sie \alst{w}íti þoljan, &
\alst{h}ungạr \alst{h}ęti-grimmen, \hld\ ęndi fódje is \alst{h}undos mid þiu.“ &
„\alst{W}ár is þat, \alst{w}aldand“, \hld[kwað siu,] „þat þú mid þínun \alst{w}ordun sprikis, &
\alst{s}ȯð-líko \alst{s}agis: \hld\ Hwat þoh oft an \alst{s}ęli innen &
undar iro \alst{h}êrron diske \hld\ \alst{h}welpos \alst{h}wervad &%NOTE: unusual alliteration
\alst{b}rosmono fulle \hld\ þero fan þemu \alst{b}iode niðer &
ant·\alst{f}allat iro \alst{f}rôjan.“ \hld\ Þó gi·hôrde þat \alst{f}riðu-barn godes &
\alst{w}illjan þes \alst{w}íves \hld\ ęndi sprak iru mid is \alst{w}ordun tó: &
„\alst{w}ela þat þú \alst{w}íf haves \hld\ \alst{w}illjan góden! &
\alst{M}ikil is þín gi·lôvo \hld\ an þea \alst{m}aht godes, &
an þene \alst{l}iudjo drohtin. \hld\ Al wirðid gi·\alst{l}êstid só &
umbi þínes \alst{b}arnes líf, \hld\ só þú \alst{b}ádi te mí.“ &
Þó warð siu sán gi·\alst{h}êlid, \hld\ só it þe \alst{h}êlago ge·sprak &
\alst{w}ordun \alst{w}ár-fastun: \hld\ þat \alst{w}íf fagonode, &
þes siu iro \alst{b}arnes forð \hld\ \alst{b}rúkan móste; &
\alst{h}abde iru gi·\alst{h}olpen \hld\ \alst{h}êljando Krist, &
habde sie far·\alst{f}angane \hld\ \alst{f}íundo kraftu, &
\alst{w}am-skaðun bi·\alst{w}ęrid. \hld\ Þó gi·wêt imu \alst{w}aldand forð, &
\alst{b}arno þat \alst{b}ętste, \hld\ sóhte imu \alst{b}urg ȯðre, &
þiu só \alst{þ}ikko was \hld\ mid þeru \alst{þ}iodu Judeono, &
mid \alst{s}u̇ðar-liudjun gi·\alst{s}eten. \hld\ Þár gi·fragn ik þat hé is ge·\alst{s}ïðos grótte, &
þe \alst{j}ungaron þe hé imu habde be is \alst{g}óde gi·korane, \hld\ þat sie mid imu \alst{g}erno ge·wunodun, &
\alst{w}eros þurh is \alst{w}íson spráka: \hld\ „alle skal ik iu“, kwað hé, „mid \alst{w}ordun frágon, &
\alst{j}ungaron míne: \hld\ hwat kweðat þese \alst{J}udeo liudi, &
\alst{m}ári \alst{m}ęgin-þioda, \hld\ hwat ik \alst{m}anno sí?“ &
Imu and-wordidun \alst{f}rô-líko \hld\ is \alst{f}riund an·gęgin, &
\alst{j}ungaron síne: \hld\ „nis þit \alst{J}udeono folk, &
\alst{e}rlos \alst{ê}n-wordje: \hld\ sum sagad þat þú \alst{E}lias sís, &
\alst{w}ís \alst{w}ár-sago, \hld\ þe hér giu \alst{w}as lango, &
\alst{g}ód undar þesumu \alst{g}um-skępje, \hld\ sum sagad þat þú \alst{J}ohannes sís, &
\alst{d}iur-lík \alst{d}rohtines bodo, \hld\ þe hér \alst{d}ôpte iu &
\alst{w}erod an \alst{w}atere; \hld\ alle sie mid \alst{w}ordun sprekad, &
þat þú \alst{ê}n-hwi-lik sís \hld\ \alst{ę}ðilero manno, &
þero \alst{w}ár-sagono, \hld\ þe hér mid \alst{w}ordun giu &
\alst{l}êrdun þese \alst{l}iudi, \hld\ ęndi þat þú sís eft an þit \alst{l}ioht kumen &
te \alst{w}ísjanne þesumu \alst{w}erode.“ \hld\ Þó sprak eft \alst{w}aldand Krist: &
„Hwe kweðad \alst{g}í, þat ik sí“, \hld[kwað hé,] „\alst{j}ungaron míne, &
\alst{l}iovon \alst{l}iud-weros?“ \hld\ Þó te \alst{l}at ni warð &
\alst{S}ímon Petrus: \hld\ sprak \alst{s}án an·gęgin &
\alst{ê}no for im \alst{a}llun \hld\ —habde imu \alst{ę}lljen gód, &
\alst{þ}rístja gi·\alst{þ}ȧhti, \hld\ was is \alst{þ}eodone hold—:\eva

\bvb TODO.\evb\evg

\bvg\bva[37][3057]%
„Þú bist þe \alst{w}áro \hld\ \alst{w}aldandes sunu, &
\alst{l}ibbjendes godes, \hld\ þe þit \alst{l}ioht gi·skóp, &
\alst{K}rist \alst{k}uning êwig: \hld\ só willjad wí \alst{k}weðen alle, &
\alst{j}ungaron þíne, \hld\ þat þú sís \alst{g}od selvo, &
\alst{h}êljandero bętst.“ \hld\ Þó sprak imu eft is \alst{h}êrro an·gęgin: &
„\alst{s}álig bist þú \alst{S}ímon“, kwað hé, „\alst{s}unu Jonases; \hld\ ni mahtes þú þat \alst{s}elvo ge·huggjan, &
gi·\alst{m}arkon an þínun \alst{m}ód-gi·þȧhtjun, \hld\ ne it ni mahte þi \alst{m}annes tunge &
\alst{w}ordun ge·\alst{w}ísjen, \hld\ ak dede it þi \alst{w}aldand selvo, &
\alst{f}ader allaro \alst{f}iriho barno, \hld\ þat þú só \alst{f}orð gi·spráki, &
só \alst{d}iapo bi \alst{d}rohtin þínen. \hld\ \alst{D}iur-líko skalt þú þes lôn ant·fáhen, &
\alst{h}luttro havas þú an þínan \alst{h}êrron gi·lôvon, \hld\ \alst{h}ugi-skęfti sind þíne stêne ge·líka, &
só \alst{f}ast bist þú só \alst{f}elis þe hardo; \hld\ hêten skulun þi \alst{f}iriho barn &
\alst{s}ankte Péter: \hld\ ovar þemu stêne skal man mínen \alst{s}ęli wirkjan, &
\alst{h}êlag \alst{h}ús godes; \hld\ þár skal is \alst{h}íwiski tó &
\alst{s}álig \alst{s}amnon: \hld\ ni mugun wið þem þínun \alst{s}wíðjun krafte &
an·þebbjen \alst{h}ęllje portun. \hld\ Ik far·givu þi \alst{h}imil-ríkjas slutilas, &%TODO: Etymology an·þebbjen
þat þú móst \alst{a}ftar mí \hld\ \alst{a}llun gi·waldan &
\alst{k}ristinum folke; \hld\ \alst{k}umad alle te þi &
\alst{g}umono \alst{g}êstos; \hld\ þú have \alst{g}rôte gi·wald, &
hwene þú hér an \alst{e}rðu \hld\ \alst{ę}ldi-barno &
ge·\alst{b}inden willjes: \hld\ þemu is \alst{b}êðju gi·duan, &
\alst{h}imil-ríki bi·loken, \hld\ ęndi \alst{h}ęllje sind imu opana, &
\alst{b}rinnandi fiur; \hld\ só hwene só þú eft ant·\alst{b}inden wili, &
an-þeftjen is \alst{h}ęndi, \hld\ þemu is \alst{h}imil-ríki, &
ant·\alst{l}oken \alst{l}iohto mêst \hld\ ęndi \alst{l}íf êwig, &
\alst{g}róni \alst{g}odes wang. \hld\ Mid su·likaru ik þi \alst{g}evu willju &
\alst{l}ônon þínen gi·\alst{l}ôvon. \hld\ Ni willju ik, þat gí þesun \alst{l}iudjun noh, &
\alst{m}árjen þesaru \alst{m}ęnigi, \hld\ þat ik bium \alst{m}ahtig Krist, &
\alst{g}odes êgan barn. \hld\ Mí skulun \alst{J}udeon noh, &
\alst{u}n-skuldigna \hld\ \alst{e}rlos binden, &
\alst{w}êgjan mí te \alst{w}undrun \hld\ —dót mí \alst{w}ítjes filo— &
innan \alst{J}erusalem \hld\ \alst{g}êres ordun, &
\alst{á}htjen mínes \alst{a}ldres \hld\ \alst{ę}ggjun skarpun, &
bi·\alst{l}ôsjen mí \alst{l}ívu. \hld\ Ik an þesumu \alst{l}iohte skal &
þurh u̇ses \alst{d}rohtines kraft \hld\ fan \alst{d}ôde a·standen &
an \alst{þ}riddjumu dage“. \hld\ Þó warð \alst{þ}egno bętst &
\alst{s}wíðo an \alst{s}orgun, \hld\ \alst{S}ímon Petrus, &
warð imu \alst{h}ugi \alst{h}riuwig, \hld\ ęndi te is \alst{h}êrron sprak &
\alst{r}ink an \alst{r}únun: \hld\ „ni skal þat \alst{r}íki god“, kwað hé, &
„\alst{w}aldand \alst{w}illjen, \hld\ þat þú eo su·lik \alst{w}íti mikil &
gi·\alst{þ}olos undar þesaru \alst{þ}iod: \hld\ nis þes \alst{þ}arf nigijan, &%TODO: Check nigijan
\alst{h}êlag drohtin.“ \hld\ Þó sprak imu eft is \alst{h}êrro an·gęgin, &
\alst{m}ári \alst{m}ahtig Krist \hld\ —was imu an is \alst{m}óde hold—: &
„Hwat þú nú \alst{w}iðẹr-\alst{w}ard bist“, \hld[kwað hé,] „\alst{w}illjon mínes, &
\alst{þ}egno bętsto! \hld\ Hwat þú þesaro \alst{þ}iodo kanst &
\alst{m}ęnniskan sidu: \hld\ þú ni wêst þe \alst{m}aht godes, &
þe ik gi·\alst{f}rummjen skal. \hld\ Ik mag þi \alst{f}ilu sęggjan &
\alst{w}árun \alst{w}ordun, \hld\ þár hér undar þesumu \alst{w}erode standad &
ge·\alst{s}ïðos míne, \hld\ þea ni mótun \alst{s}welten êr, &
\alst{h}werven an \alst{h}inen-fard \hld\ êr sie \alst{h}imiles lioht, &
\alst{g}odes ríki sehat.“ \hld\ Kôs imu \alst{j}ungarono þó &
\alst{s}án aftar þiu \hld\ \alst{S}ímon Petrus, &
\alst{J}akob ęndi \alst{J}ohannes, \hld\ ea \alst{g}umon twêne, &
\alst{b}êðja þea gi·\alst{b}róðer, \hld\ ęndi imu þó uppen þene \alst{b}erg gi·wêt &
\alst{s}under mid þem ge·\alst{s}ïðun, \hld\ \alst{s}álig barn godes, &
mid þem \alst{þ}egnun \alst{þ}rim, \hld\ \alst{þ}iodo drohtin, &
\alst{w}aldand þesaro \alst{w}er-oldes: \hld\ welde im þár \alst{w}undres filu, &
\alst{t}êkno \alst{t}ôgjan, \hld\ þat sie gi·\alst{t}rúodin þiu bet, &
þat hé \alst{s}elvo was \hld\ \alst{s}unu drohtines, &
\alst{h}êlag \alst{h}evan-kuning. \hld\ Þó sie an \alst{h}ôhan wall &
\alst{st}igun \alst{st}ên ęndi berg, \hld\ ant-tat sie te þeru \alst{st}ędi kwámun, &
\alst{w}eros wiðẹr \alst{w}olkạn, \hld\ þár \alst{w}aldand Krist, &
\alst{k}uningo \alst{k}raftigost \hld\ gi·\alst{k}oren habde, &
þat hé is \alst{g}od-kundi \hld\ \alst{j}ungarun sínun &
þurh is \alst{ê}nes kraft \hld\ \alst{ó}gjan welde, &
\alst{b}erht-lík \alst{b}iliði.\eva

\bvb TODO.\evb\evg

\bvg\bva[38][3122]%
\hspace*{100pt}Þó imu þár te \alst{b}edu gi·hnêg, &
þó warð imu þár \alst{u}ppe \hld\ \alst{ȯ}ðar-líkora &
\alst{w}liti ęndi gi·\alst{w}ádi: \hld\ wurðun imu is \alst{w}angun liohte, &
\alst{b}líkandi só þiu \alst{b}erhte sunne: \hld\ só skên þat \alst{b}arn godes, &
\alst{l}iuhte is \alst{l}ík-hamo: \hld\ \alst{l}iomon stódun &
\alst{w}ánamo fan þemu \alst{w}aldandes barne; \hld\ warð is ge·\alst{w}ádi só hwít &
só \alst{s}nêw te \alst{s}ehanne. \hld\ Þó warð þár \alst{s}eld-lík þing &
gi·\alst{ô}gid aftar þiu: \hld\ \alst{E}lias ęndi Moyses &
\alst{k}wámun þár te \alst{K}riste \hld\ wið só \alst{k}raftagne &
\alst{w}ordun \alst{w}ehsljan. \hld\ Þár warð só \alst{w}un-sam spráka, &
só \alst{g}ód word undar \alst{g}umun, \hld\ þár þe \alst{g}odes sunu &
wið þea \alst{m}árjan \alst{m}an \hld\ \alst{m}ahljen welde, &
só \alst{b}líði warð uppan þemu \alst{b}erge: \hld\ skên þat \alst{b}erhte lioht, &
was þár \alst{g}ard \alst{g}ód-lík \hld\ ęndi \alst{g}róni wang, &
\alst{P}aradíse ge·lík. \hld\ \alst{P}etrus þó gi·mahạlde, &
\alst{h}ęlið \alst{h}ard-módig \hld\ ęndi te is \alst{h}êrron sprak, &
\alst{g}rótte þene \alst{g}odes sunu: \hld\ „\alst{g}ód is it hér te wesanne, &
ef þú it gi·\alst{k}iosan wili, \hld\ \alst{K}rist alo-waldo, &
þat man þí \alst{h}ér an þesaru \alst{h}ôhe \hld\ ên \alst{h}ús ge·wirkja, &
\alst{m}ár-líko ge·\alst{m}ako \hld\ ęndi \alst{M}oysese ȯðer &
ęndi \alst{E}liase þriddja: \hld\ þit is \alst{ô}das hêm, &
\alst{w}elono \alst{w}un-samost.“ \hld\ Reht só hé þó þat \alst{w}ord ge·sprak, &
só ti·\alst{l}ét þiu \alst{l}uft an twê: \hld\ \alst{l}ioht wolkạn skên, &
\alst{g}lítandi \alst{g}límo, \hld\ ęndi þea \alst{g}ódun man &
\alst{w}liti-skôni be·\alst{w}arp. \hld\ Þó fan þemu \alst{w}olkne kwam &
\alst{h}êlag stemne godes, \hld\ ęndi þem \alst{h}ęliðun þár &
\alst{s}elvo \alst{s}agde, \hld\ þat þat is \alst{s}unu wári, &
\alst{l}ibbjendero \alst{l}iovost: \hld\ „an þemu mí \alst{l}íkod wel &
an mínun \alst{h}ugi-skęftjun. \hld\ Þemu gí \alst{h}ôrjen skulun, &
ful·\alst{g}angad imu \alst{g}erno.“ \hld\ Þó ni mahtun þea \alst{j}ungaron Kristes &
þes \alst{w}olknes \alst{w}liti \hld\ ęndi \alst{w}ord godes, &
þea is \alst{m}ikilon \alst{m}aht \hld\ þea \alst{m}an ant·standen, &
ak sie bi·\alst{f}ellun þó \alst{f}orð-wardes: \hld\ \alst{f}erhes ni wándun, &
\alst{l}ęngiron \alst{l}íves. \hld\ Þó géng im tó þe \alst{l}andes ward, &
be·\alst{h}rên sie mid is \alst{h}andun \hld\ \alst{h}êljandero bętst, &
hét þat sie im ni an·\alst{d}rédin: \hld\ „ni skal iu hér \alst{d}erjen eo·wiht, &
þes gí hér \alst{s}eld-líkes \hld\ gi·\alst{s}ehen habbjad, &
\alst{m}érjaro þingo.“ \hld\ Þó eft þem \alst{m}annun warð &
\alst{h}ugi at iro \alst{h}erton \hld\ ęndi gi·\alst{h}êlid mód, &
gi·\alst{b}ade an iro \alst{b}reostun: \hld\ gi·sáhun þat \alst{b}arn godes &
\alst{ê}nna standen, \hld\ was þat \alst{ȯ}ðer þó, &
be·\alst{h}liden \alst{h}imiles lioht. \hld\ Þó gi·wêt imu þe \alst{h}êlago Krist &
fan þemu \alst{b}erge niðer; \hld\ gi·\alst{b}ôd aftar þiu &
\alst{j}ungarun sínun, \hld\ þat sie ovar \alst{J}udeono folk &
ni \alst{s}agdin þea gi·\alst{s}ioni: \hld\ „er þan ik \alst{s}elvo hér &
swíðo \alst{d}iur-líko \hld\ fan \alst{d}ôðe a·stande, &
a·\alst{r}íse fan þeru \alst{r}estu: \hld\ sïðor mugun gí it \alst{r}ękkjen forð, &
\alst{m}árjen ovar \alst{m}iddil-gard \hld\ \alst{m}anagun þiodun &
\alst{w}ído aftar þesaru \alst{w}er-oldi.“\eva

\bvb TODO.\evb\evg

\bvg\bva[39][3170]%
\hspace*{100pt}Þó gi·wêt imu \alst{w}aldand Krist &
eft an \alst{G}alileo land, \hld\ sóhte is \alst{g}adulingos, &
\alst{m}ahtig is \alst{m}ágo hêm, \hld\ sagde þár \alst{m}anages hwat &
\alst{b}erhtero \alst{b}iliðjo, \hld\ ęndi þat \alst{b}arn godes &
þem is \alst{s}áligun ge·\alst{s}ïðun \hld\ \alst{s}org-spell ni for·hal, &
ak hé im \alst{o}pen-líko \hld\ \alst{a}llun sagde, &
þem is \alst{g}ódun \alst{j}ungarun, \hld\ hwó ine skolde þat \alst{J}udeono folk &
\alst{w}êgjan te \alst{w}undrun. \hld\ Þes wurðun þár \alst{w}íse man &
\alst{s}wíðo an \alst{s}orgun, \hld\ warð im \alst{s}êr hugi, &
\alst{h}riuwig umbi iro \alst{h}erte: \hld\ gi·hôrdun iro \alst{h}êrron þó, &
\alst{w}aldandes sunu \hld\ \alst{w}ordun tęlljen, &
hwat hé undar þeru \alst{þ}iodu \hld\ \alst{þ}olojan skolde, &
\alst{w}illjendi undar þemu \alst{w}erode. \hld\ Þó gi·wêt imu \alst{w}aldand Krist, &
\alst{g}umo fan \alst{G}alilea, \hld\ sóhte imu \alst{J}udeono burg, &
\alst{k}wámun im te \alst{K}afarnaum. \hld\ Þár fundun sie ênan \alst{k}uninges þegạn &
\alst{w}lankan undar þemu \alst{w}erode: \hld\ kwað þat hé wári gi·\alst{w}ęldig bodo &
\alst{a}ðal-kêsures; \hld\ hé grótte \alst{a}ftar þiu &
\alst{S}ímon Petrusen, \hld\ kwað þat hé wári gi·\alst{s}ęndid þarod, &
þat hé þár gi·\alst{m}anodi \hld\ \alst{m}anno ge·hwi-liken &
þero \alst{h}ôvid-skatto, \hld\ þe sie te þemu \alst{h}ove skoldin &
\alst{t}insi gelden: \hld\ „nis þes \alst{t}weho ênig &
\alst{g}umono ni-gj·ênumu, \hld\ ne sie ina far·\alst{g}elden sán &
\alst{m}êðmo kustjon, \hld\ bi·úten iuwe \alst{m}êster êno &
havad it far·\alst{l}áten. \hld\ Ni skal þat \alst{l}íkon wel &
mínumu \alst{h}êrron, \hld\ só man it imu at is \alst{h}ove ku̇ðid, &
\alst{a}ðal-kêsure.“ \hld\ Þó géng \alst{a}ftar þiu &
\alst{S}ímon Petrus, \hld\ welde it \alst{s}ęggjan þó &
\alst{h}êrron sínumu: \hld\ hé was is an is \alst{h}ugi iu þan, &%TODO: Check sínumu.
gi·\alst{w}aro \alst{w}aldand Krist: \hld\ —imu ni mahte \alst{w}ord ênig &
bi·\alst{h}olen werðen, \hld\ hé wisse \alst{h}ugi-skęfti &
\alst{m}anno ge·hwi-likes—: \hld\ hét þó þene is \alst{m}árjan þegạn, &
\alst{S}ímon Petrus \hld\ an þene \alst{s}êo innen &
\alst{a}ngul werpen: \hld\ „su·liken só þú þár \alst{ê}rist mugis &
\alst{f}isk gi·\alst{f}áhen“, \hld[kwað hé,] „só teoh þú þene fan þemu \alst{f}lóde te þi, &
ant·\alst{k}lęmmi imu þea \alst{k}inni: \hld\ þár maht þú undar þem \alst{k}aflon nimen &
\alst{g}uldine skattos, \hld\ þat þú far·\alst{g}elden maht &
þemu \alst{m}anne te gi·\alst{m}ódja \hld\ \alst{m}ínen ęndi þínen &
\alst{t}insjo só hwi-likan, \hld\ só hé u̇s \alst{t}ó sókid.“ &
Hé ni þorfte imu þó \alst{a}ftar þiu \hld\ \alst{ȯ}ðaru wordu &
\alst{f}urður gi·bioden: \hld\ géng \alst{f}iskari gód, &
\alst{S}ímon Petrus, \hld\ warp an þene \alst{s}êo innen &
\alst{a}ngul an \alst{u̇}ðjon \hld\ ęndi \alst{u}p gi·tôh &
\alst{f}isk an \alst{f}lóde \hld\ mid is \alst{f}olmun twêm, &
te·\alst{k}lóf imu þea \alst{k}inni \hld\ ęndi undar þem \alst{k}aflun nam &
\alst{g}uldine skattos: \hld\ dede al, só imu þe \alst{g}odes sunu &
\alst{w}ordun ge·\alst{w}ísde. \hld\ Þár was þó \alst{w}aldandes &
\alst{m}ęgin-kraft gi·\alst{m}árid, \hld\ hwó skal allaro \alst{m}anno ge·hwi-lik &
swíðo \alst{w}illjendi \hld\ is \alst{w}er-old-hêrron &
\alst{sk}uldi ęndi \alst{sk}attos, \hld\ þea imu gi·\alst{sk}ęride sind, &
\alst{g}erno \alst{g}elden: \hld\ ni skal ine far·\alst{g}úmon eo·wiht, &
ni far·\alst{m}uni ine an is \alst{m}óde, \hld\ ak wese imu \alst{m}ildi an is hugi, &
\alst{þ}iono imu \alst{þ}io-líko: \hld\ an þiu mag hé \alst{þ}iod-godes &
\alst{w}illjan ge·\alst{w}irkjan \hld\ ęndi ôk is \alst{w}er-old-hêrron &
\alst{h}uldi \alst{h}abbjen.\eva

\bvb TODO.\evb\evg

\bvg\bva[40][3223]%
Só lêrde þe \alst{h}êlago Krist &
þea is \alst{g}ódon \alst{j}ungaron: \hld\ „ef ênig \alst{g}umono wið iu“, kwað hé, &
„\alst{s}undja ge·wirkja, \hld\ þan nim þú ina \alst{s}undạr te þi, &
þene \alst{r}ink an \alst{r}úna \hld\ ęndi imu is \alst{r}ád saga, &
\alst{w}ísi imu mid \alst{w}ordun. \hld\ Ef imu þan þes \alst{w}erð ne sí, &
þat hé þí gi·\alst{h}ôrje, \hld\ \alst{h}ala þí þár ȯðara tó &
\alst{g}ódaro \alst{g}umono, \hld\ ęndi lah imu is \alst{g}rimmun werk, &
\alst{s}ak ina \alst{s}ȯð-wordun. \hld\ Ef imu þan is \alst{s}undja aftar þiu, &
\alst{l}ôs-werk ni \alst{l}êðon, \hld\ gi·duo it ȯðrun \alst{l}iudjun ku̇ð, &
\alst{m}ári it þan for \alst{m}ęnegi \hld\ ęndi lát \alst{m}anno filu &
\alst{w}iten is far·\alst{w}urhti: \hld\ ôðo be·ginnad imu þan is \alst{w}erk tregan, &
an is \alst{h}ugi \alst{h}reuwen, \hld\ þan hé it gi·hôrid \alst{h}ęliðo filu, &
\alst{a}hton \alst{ę}ldi-barn \hld\ ęndi imu is \alst{u}vilon dád &
\alst{w}ęrjad mid \alst{w}ordun. \hld\ Ef hé þan ôk \alst{w}ęndjen ne wili, &
ak far·\alst{m}ódat su·lika \alst{m}ęnegi, \hld\ þan lát þú þene \alst{m}an faren, &
\alst{h}ava ina þan far \alst{h}êðinen \hld\ ęndi lát ina þi an þínumu \alst{h}ugi lêðen, &
\alst{m}íð is an þínumu \alst{m}óde, \hld\ ne sí þat imu eft \alst{m}ildi god, &
\alst{h}êr \alst{h}evan-kuning \hld\ \alst{h}elpe far·líhe, &
\alst{f}ader allaro \alst{f}iriho barno.“ \hld\ Þó \alst{f}rágode Petrus, &
allaro \alst{þ}egno bętst \hld\ \alst{þ}eodan sínan: &
„hwó oft skal ik þem \alst{m}annun, \hld\ þe wið \alst{m}í habbjad &
\alst{l}êð-werk gi·duan, \hld\ \alst{l}eovo drohtin, &
skal ik im \alst{s}ivun \alst{s}ïðun \hld\ iro \alst{s}undja a·láten, &
\alst{w}rêðaro \alst{w}erko, \hld\ êr þan ik is êniga \alst{w}réka frummje, &%TODO: Check wréka.
\alst{l}êðes te \alst{l}ône?“ \hld\ Þó sprak eft þe \alst{l}andes ward, &
an·gęgin þe \alst{g}odes sunu \hld\ \alst{g}ódumu þegne: &
„ni \alst{s}ęggju ik þi fan \alst{s}ivunjun, \hld\ só þú \alst{s}elvo sprikis, &
\alst{m}ahlis mid þínu \alst{m}u̇ðu, \hld\ ik duom þi \alst{m}êra þár tó: &
\alst{s}ivun \alst{s}ïðun \alst{s}ivun-tig \hld\ só skalt þú \alst{s}undja ge·hwemu, &
\alst{l}êðes a·\alst{l}áten: \hld\ só willju ik þi te \alst{l}êrun geven &
\alst{w}ordun \alst{w}ár-fastun. \hld\ Nú ik þí su·lika gi·\alst{w}ald far·gaf, &
þat þú mínes \alst{h}íwiskes \hld\ \alst{h}êrost wáris, &
\alst{m}anages \alst{m}ann-kunnjes, \hld\ nú skalt þú im \alst{m}ildi wesen, &
\alst{l}iudjun \alst{l}íði.“ \hld\ Þó þár te þemu \alst{l}êrjande kwam &
ên \alst{j}ung man an·\alst{g}ęgin \hld\ ęndi frágode \alst{J}esu Krist: &
„\alst{m}êster þe gódo“, \hld[kwað hé,] „hwat skal ik \alst{m}anages duan, &
an þiu þe ik \alst{h}evan-ríki \hld\ ge·\alst{h}alan móti?“ &
Habde imu \alst{ô}d-welon \hld\ \alst{a}llen ge·wunnen, &
\alst{m}êðọm-hord \alst{m}anag, \hld\ þoh hé \alst{m}ildjan hugi &
\alst{b}ári an is \alst{b}reostun. \hld\ Þó sprak imu þat \alst{b}arn godes: &
„hwat kwiðis þú umbi \alst{g}ódon? \hld\ nis þat \alst{g}umono ênig &
bi·útan þe \alst{ê}no, \hld\ þe þár \alst{a}l ge·skóp, &
\alst{w}er-old ęndi \alst{w}unnja. \hld\ Ef þú is \alst{w}illjan havas, &
þat þú an \alst{l}ioht godes \hld\ \alst{l}íðan mótis, &
þan skalt þú bi·\alst{h}alden \hld\ þea \alst{h}êlagon lêra, &
þe þár an þemu \alst{a}ldon \hld\ \alst{ê}wa ge·biudid, &
þat þú \alst{m}an ni slah, \hld\ ni þú \alst{m}ênes ni sweri, &
far·\alst{l}egar-nessi far·\alst{l}át \hld\ ęndi \alst{l}uggi ge·wit-skępi, &
\alst{st}ríd ęndi \alst{st}ulina; \hld\ ne wis þú te \alst{st}ark an hugi, &
ne \alst{n}íðin ne hatul, \hld\ ni \alst{n}ôd-róf ni fręmi; &
\alst{a}v-unst \alst{a}lla far·lát; \hld\ wis þínun \alst{ę}ldirun gód, &
\alst{f}ader ęndi móder, \hld\ ęndi þínun \alst{f}riundun hold, &
þem \alst{n}áhistun gi·\alst{n}áðig. \hld\ Þan þú þi gi·\alst{n}iodon móst &
\alst{h}imilo ríkjas, \hld\ ef þú it bi·\alst{h}alden wili, &
ful-\alst{g}angan \alst{g}odes lêrun.“ \hld\ Þó sprak eft þe \alst{j}ungo man &
„al hębbju ik só gi·\alst{l}êstid“, \hld[kwað hé,] „só þú mí \alst{l}êris nú, &
\alst{w}ordun \alst{w}ísis, \hld\ só ik is eo \alst{w}iht ni far·lét &
fan mínero \alst{k}indiski.“ \hld\ Þó bi·gan ina \alst{K}rist sehan &
\alst{a}n mid is \alst{ô}gun: \hld\ „\alst{ê}n is þár noh nú“, kwað hé, &
„\alst{w}an þero \alst{w}erko: \hld\ ef þú is \alst{w}illjon havas, &
þat þú \alst{þ}urh-fręmid \hld\ \alst{þ}ionon mótis &
\alst{h}êrron þínumu, \hld\ þan skalt þú þat þín \alst{h}ord nimen, &
skalt þínan \alst{ô}d-welon \hld\ \alst{a}llan far·kôpjen, &
\alst{d}iurje mêðmos, \hld\ ęndi \alst{d}êljen hét &
\alst{a}rmun mannun: \hld\ þan havas þú \alst{a}ftar þiu &
\alst{h}ord an \alst{h}imile; \hld\ kum þi þan gi·\alst{h}alden te mí, &
\alst{f}olgo þi mínaro \alst{f}ęrdi: \hld\ þan havas þú \alst{f}riðu sïður.“ &
Þó wurðun \alst{K}ristes word \hld\ \alst{k}ind-jungumu manne &
\alst{s}wíðo an \alst{s}orgun, \hld\ was imu \alst{s}êr hugi, &
\alst{m}ód umbi herte: \hld\ habde \alst{m}êðmo filu, &
\alst{w}elono ge·\alst{w}unnen; \hld\ \alst{w}ęnde imu eft þanen, &
was imu \alst{u}n-\alst{ô}ðo \hld\ \alst{i}nnan breostun, &
an is \alst{s}evon \alst{s}wáro. \hld\ \alst{S}ah imu aftar þó &
\alst{K}rist alo-waldo, \hld\ \alst{k}wað it þó, þár hé welde, &
te þem is \alst{j}ungarun \alst{g}ęgin-wardun, \hld\ þat wári an \alst{g}odes ríki &
\alst{u}n-óði \alst{ô}dagumu manne \hld\ \alst{u}p te kumanne: &
„\alst{ô}ður mag man \alst{o}lvundjon, \hld\ þoh hé sí \alst{u}n-met grôt, &%TODO: check ódur
þurh \alst{n}áðlan gat, \hld\ þoh it sí \alst{n}aru swíðo, &
\alst{s}áftur þurh·\alst{s}lópjen, \hld\ þan mugi kuman þiu \alst{s}iole te himile &
þes \alst{ô}dagan mannes, \hld\ þe hér \alst{a}l havad &
gi·\alst{w}ęndid an þene \alst{w}er-old-skat \hld\ \alst{w}illjon sínen, &
\alst{m}ód-gi·þȧhti, \hld\ ęndi ni hugid umbi þie \alst{m}aht godes.“\eva

\bvb TODO.\evb\evg

\bvg\bva[41][3305]%
Imu \alst{a}nd-wordjade \hld\ \alst{ê}r-þungan gumo, &
\alst{S}ímon Petrus, \hld\ ęndi \alst{s}ęggjan bad &
\alst{l}eovan hêrron: \hld\ „Hwat skulun wí þes te \alst{l}ône nimen“, kwað hé, &
„\alst{g}ódes te \alst{g}elde, \hld\ þes wí þurh þín \alst{j}ungar-dóm &
\alst{ê}gan ęndi \alst{ę}rvi \hld\ \alst{a}l far·létun &
\alst{h}ovos ęndi \alst{h}íwiski \hld\ ęndi þi te \alst{h}êrron gi·kurun, &
\alst{f}olgodun þínaru \alst{f}ęrdi: \hld\ hwat skal u̇s þes te \alst{f}rumu werðen, &
\alst{l}anges te \alst{l}ône?“ \hld\ \alst{L}iudjo drohtin &
\alst{s}agde im þó \alst{s}elvo: \hld\ „Þan ik \alst{s}ittjen kumu“, kwað hé, &
„an þie \alst{m}ikilan \alst{m}aht \hld\ an þemu \alst{m}árjan dage, &
þár ik \alst{a}llun skal \hld\ \alst{i}rmin-þiodun &
\alst{d}ómos a·\alst{d}êljen, \hld\ þan mótun gí mid iuwomu \alst{d}rohtine þár &
\alst{s}elvon \alst{s}ittjen \hld\ ęndi mótun þera \alst{s}aka waldan: &
mótun gí \alst{I}srahelo \hld\ \alst{ę}ðili-folkun &
a·\alst{d}êljen aftar iro \alst{d}ádjun: \hld\ só mótun gí þár gi·\alst{d}iuride wesen. &
Þan sęggju ik iu te \alst{w}áran: \hld\ só hwe só þat an þesaru \alst{w}er-oldi gi·duot, &
þat hé þurh \alst{m}ína \alst{m}innja \hld\ \alst{m}ágo ge·sidli &
\alst{l}iof far·\alst{l}étid, \hld\ þes skal hí hér \alst{l}ôn niman &
\alst{t}ehan sïðun \alst{t}ehin-fald, \hld\ ef hé it mid \alst{t}reuwon duot, &
mid \alst{h}luttru \alst{h}ugi. \hld\ Ovar þat havad hé ôk \alst{h}imiles lioht, &
\alst{o}pen \alst{ê}wig líf.“ \hld\ Bi·gan imu þó \alst{a}ftar þiu &
allaro \alst{b}arno \alst{b}ętst \hld\ ên \alst{b}iliði sęggjan, &
kwað þat þár \alst{ê}n \alst{ô}dag man \hld\ an \alst{ê}r-dagun &
\alst{w}ári undar þemu \alst{w}erode: \hld\ þe habde \alst{w}elono ge·nóg, &
\alst{s}inkas gi·\alst{s}amnod \hld\ ęndi imu \alst{s}imlun was &
\alst{g}aru mid \alst{g}oldu \hld\ ęndi mid \alst{g}odo-wębbju, &
\alst{f}agạrun \alst{f}ratahun \hld\ ęndi imu so \alst{f}ilu habde &
\alst{g}ódes an is \alst{g}ardun \hld\ ęndi imu at \alst{g}ômun sat &
allaro \alst{d}ago ge·hwi-likes: \hld\ habde imu \alst{d}iur-lík líf, &
\alst{b}líðsja an is \alst{b}ęnkjun. \hld\ Þan was þár eft ên \alst{b}iddjendi man, &
gi·\alst{l}évod an is \alst{l}ík-hamon, \hld\ \alst{L}azarus was hé hêten, &
lag imu \alst{d}ago ge·hwi-likes \hld\ at þem \alst{d}urun foren, &
þár hé þene \alst{ô}dagan man \hld\ \alst{i}nne wisse &
an is \alst{g}ęst-sęli \hld\ \alst{g}ôme þiggjan, &
\alst{s}ittjen at \alst{s}umble, \hld\ ęndi hé \alst{s}imlun bêd &
gi·\alst{a}rmod þár \alst{ú}te: \hld\ ni móste þár \alst{i}n kuman, &
ne hé ni mahte ge·\alst{b}iddjen, \hld\ þat man imu þes \alst{b}rôdes þarod &
gi·\alst{d}ragan weldi, \hld\ þes þár fan þemu \alst{d}iske niðer &
ant·\alst{f}el undar iro \alst{f}óti: \hld\ ni mahte imu þár ênig \alst{f}ruma werðen &
fan þemu \alst{h}êroston, þe þes \alst{h}úses gi·weld, \hld\ bi·útan þat þár géngun is \alst{h}undos tó, &
\alst{l}ikkodun is \alst{l}ík-wundon, \hld\ þár hé \alst{l}iggjandi &
\alst{h}ungạr þolode; \hld\ ni kwam imu þár te \alst{h}elpu wiht &
fan þemu \alst{r}íkjon manne. \hld\ Þó gi·fragn ik þat ina is \alst{r}egano gi·skapu, &
þene \alst{a}rmon man \hld\ is \alst{ê}n-dago &
gi·\alst{m}anoda \alst{m}ahtjun swíð, \hld\ þat hé \alst{m}anno drôm &
a·\alst{g}even skolde. \hld\ \alst{G}odes ęngilos &
ant·\alst{f}éngun is \alst{f}erh \hld\ ęndi lêddun ine \alst{f}orð þanen, &
þat sie an \alst{A}brahames barm \hld\ þes \alst{a}rmon mannes &
\alst{s}iole gi·\alst{s}ęttun: \hld\ þár móste hé \alst{s}imlun forð &
\alst{w}esen an \alst{w}unnjun. \hld\ Þó kwámun ôk \alst{w}urde-gi·skapu, &
þemu \alst{ô}dagan man \hld\ \alst{o}r-lag-hwíle, &
þat hé þit \alst{l}ioht far·\alst{l}ét: \hld\ \alst{l}êða wihti &
be·\alst{s}inkodun is \alst{s}iole \hld\ an þene \alst{s}warton hęl, &
an þat \alst{f}ern innen \hld\ \alst{f}íundun te willjan, &
be·\alst{g}róvun ine an \alst{g}ramono hêm. \hld\ Þanen mahte hé þene \alst{g}ódan skawon, &
\alst{A}braham ge·sehen, \hld\ þár hé \alst{u}ppe was &
\alst{l}íves an \alst{l}ustun, \hld\ ęndi \alst{L}azarus sat &
\alst{b}líði an is \alst{b}arme, \hld\ \alst{b}erht lôn ant·féng &
\alst{a}llaro is \alst{a}rm-ódjo, \hld\ ęndi lag þe \alst{ô}dago man &
\alst{h}êto an þeru \alst{h}ęllju, \hld\ \alst{h}riop up þanen: &
„\alst{f}ader Abraham“, \hld[kwað hé,] „mí is \alst{f}irinun þarf, &
þat þú \alst{m}í an þínumu \alst{m}ód-sevon \hld\ \alst{m}ildi werðes, &
\alst{l}íði an þesaru \alst{l}ognu: \hld\ sęndi mí \alst{L}azarus herod, &
þat hé mí ge·\alst{f}órja \hld\ an þit \alst{f}ern innan &
\alst{k}aldes wateres. \hld\ Ik hér \alst{k}wik brinnu &
\alst{h}êto an þesaru \alst{h}ęllju: \hld\ nú is mí þínaro \alst{h}elpono þarf, &
þat hé mí a·\alst{l}ęskje \hld\ mid is \alst{l}uttikon fingru &
\alst{t}ungon míne, \hld\ nú siu \alst{t}êkạn havad, &
\alst{u}vil \alst{a}rvêdi. \hld\ \alst{I}nwid-rádo, &
\alst{l}êðaro spráka, \hld\ alles is mí nú þes \alst{l}ôn kumen.“ &
Imu \alst{a}nd-wordjade þó \alst{A}braham \hld\ —þat was \alst{a}ld-fader—: &
„ge·\alst{h}ugi þú an þínumu \alst{h}erton“, \hld[kwað hé,] „hwat þú \alst{h}abdes iu &
\alst{w}elono an \alst{w}er-oldi. \hld\ Hwat þú þár alle þíne \alst{w}unnja far·sliti, &
\alst{g}ódes an \alst{g}ardun, \hld\ só hwat só þi \alst{g}iviðig forð &
\alst{w}erðen skolde. \hld\ \alst{W}íti þolode &
\alst{L}azarus an þemu \alst{l}iohte, \hld\ habde þár \alst{l}êðes filu, &
\alst{w}ítjas an \alst{w}er-oldi. \hld\ Be·þiu skal hé nú \alst{w}elon êgan, &
\alst{l}ibbjen an \alst{l}ustun: \hld\ þú skalt þea \alst{l}ogna þolan, &
\alst{b}rinnendi fiur: \hld\ ni mag is þi ênig \alst{b}óte kumen &
\alst{h}inana te \alst{h}ęllju: \hld\ it havad þe \alst{h}êlago god &
só gi·\alst{f}astnod mid is \alst{f}aðmun: \hld\ ni mag þár \alst{f}aren ênig &
\alst{þ}egno þurh þat \alst{þ}iustri: \hld\ it is hér só \alst{þ}ikki undar u̇s.“ &
Þó sprak eft \alst{A}brahame \hld\ þe \alst{e}rl te·gęgnes &
fan þeru \alst{h}êtan \alst{h}ęll \hld\ ęndi \alst{h}elpono bad, &
þat hé \alst{L}azarus \hld\ an \alst{l}iudjo drôm &
\alst{s}elvon \alst{s}andi: \hld\ „þat hé ge·\alst{s}ęggja þár &
\alst{b}róðarun mínun, \hld\ hwó ik hér \alst{b}rinnendi &
\alst{þ}rá-werk \alst{þ}olon; \hld\ si þár undar þeru \alst{þ}iodu sind, &
si \alst{f}ïvi undar þemu \alst{f}olke: \hld\ ik an \alst{f}orhtun bium, &
þat sie im þár far·\alst{w}irkjen, \hld\ þat sie skulin ôk an þit \alst{w}íti te mí, &
an só \alst{g}rádag fiur.“ \hld\ Þó imu eft te·\alst{g}ęgnes sprak &
\alst{A}braham \alst{a}ld-fader, \hld\ kwað þat sie þár \alst{ê}o godes &
an þemu \alst{l}and-skępi, \hld\ \alst{l}iudi habdin, &
\alst{M}oyseses gi·bôd \hld\ ęndi þár \alst{m}anagaro tó &
\alst{w}ár-saguno \alst{w}ord: \hld\ „ef sie is \alst{w}illige sind, &
þat sie þat bi·\alst{h}alden, \hld\ þan ni þurvun sie an þea \alst{h}ęll innen, &
an þat \alst{f}ern \alst{f}aren, \hld\ ef sie ge·\alst{f}rummjad só, &
só þea ge·\alst{b}iodad, \hld\ þe þea \alst{b}ók lesat &
þem \alst{l}iudjun te \alst{l}êrun. \hld\ Ef sie þes þan ni willjad \alst{l}êstjen wiht, &
þanne ni \alst{h}ôrjad sie ôk \hld\ þemu þe \alst{h}inan a·stád, &
\alst{m}an fan dôðe. \hld\ Láte man sie an iro \alst{m}ód-sevon &
\alst{s}elvon keosen, \hld\ hweðer im \alst{s}wótjera þunkje &
te gi·\alst{w}innanne, \hld\ só lango só sie an þesaru \alst{w}er-oldi sind, &
þat sie \alst{e}ft \alst{u}vil eþþa gód \hld\ \alst{a}ftar habbjen.“\eva

\bvb TODO.\evb\evg

\bvg\bva[42][3409]%
Só \alst{l}êrde hé þó þea \alst{l}iudi \hld\ \alst{l}iohton wordon, &
allaro \alst{b}arno \alst{b}ętst, \hld\ ęndi \alst{b}iliði sagde &
\alst{m}anag \alst{m}an-kunnje \hld\ \alst{m}ahtig drohtin, &
kwað þat imu ên \alst{s}álig gumo \hld\ \alst{s}amnon bi·gunni &
\alst{m}an an \alst{m}orgen, \hld\ „ęndi im \alst{m}éda gi·hét, &
þe \alst{h}êrosto þes \alst{h}íwiskjas, \hld\ swíðo *\alst{h}old-lík lôn“, &
kwað þat hie iro \alst{a}llaro gi·hwem \hld\ \alst{ê}nna gávi &
\alst{s}ilọvrinna skat. \hld\ „Þuo \alst{s}amnodun managa &
\alst{w}eros an is \alst{w}ín-gardon, \hld\ —ęndi hie im \alst{w}erk bi·falạh— &
\alst{á}dro an \alst{ú}htan. \hld\ Sum kwam þár ôk an \alst{u}ndorn tuo, &
sum kwam þár an \alst{m}iddjan dag, \hld\ \alst{m}an te þem werke, &
sum kwam þár te \alst{n}ónu, \hld\ þuo was þiu \alst{n}iguða tíd &
\alst{s}umar-langes dages; \hld\ sum þár ôk \alst{s}ïðor kwam &
an þia \alst{ę}lliftun tíd. \hld\ Þuo géng þár \alst{á}vand tuo, &
\alst{s}unna ti \alst{s}edle. \hld\ Þuo hie \alst{s}elvo gi·bôd &
\alst{i}s \alst{a}mbahtjon, \hld\ \alst{e}rlo drohtin, &
þat \alst{m}an þero \alst{m}anno gi·hwem \hld\ is \alst{m}eoda for·guldi, &
þem \alst{e}rlon \alst{a}rvid-lôn; \hld\ hiet þiem at \alst{ê}rist gevan. &
þia þár at \alst{l}ętst wárun, \hld\ \alst{l}iudi kumana, &
\alst{w}eros te þem \alst{w}erke, \hld\ ęndi mid is \alst{w}ordon gi·bôd, &
þat man þem \alst{m}annon iro \hld\ \alst{m}ieda for·guldi &
\alst{a}lles at \alst{a}ftan, \hld\ þem þár kwámun at \alst{ê}rist tuo &
\alst{w}illendi te þem \alst{w}erke. \hld\ \alst{W}ándun sia swíðo, &
þat man im \alst{m}êra lôn \hld\ gi·\alst{m}akod habdi &
wið iro \alst{a}rạvedje: \hld\ þan man im \alst{a}llon gaf, &
þem \alst{l}iudjon gi·\alst{l}íko. \hld\ \alst{L}êð was þat swíðo, &
\alst{a}llon þem \alst{a}ndo, \hld\ þem þár kwámun at \alst{ê}rist tuo: &
„wí kwámun hier an \alst{m}orạgan“, \hld[kwáðun sia,] „ęndi þolodun hier \alst{m}anag te dage &
\alst{a}rạvid-werko, \hld\ hwílon \alst{u}n-met hét, &
\alst{sk}ínandja sunna: \hld\ nú ni givis þú u̇s \alst{sk}attes þan mêr, &
þie þú þem \alst{ȯ}ðron duos, \hld\ þia hier \alst{ê}na hwíla &
\alst{w}áron an þínon \alst{w}erke.“ \hld\ Þuo habda eft is \alst{w}ord garo &
þie \alst{h}êrosto þes \alst{h}íwiskes, \hld\ kwað þat hie im ni habdi gi·\alst{h}êtan þan mêr &
\alst{w}erðes wið iro \alst{w}erke: \hld\ „Hwat ik gi·\alst{w}ald hębbju“, kwaþ-hie, &
„þat ik iu allon gi·\alst{l}íko \hld\ muot \alst{l}ôn for·geldan, &
iuwes \alst{w}erkes \alst{w}erð.“ \hld\ Þan \alst{w}aldandi Krist &
\alst{m}ênda im þoh \alst{m}éra þing, \hld\ þoh hie ovar þat \alst{m}anno folk &
fan þem \alst{w}ín-gardon só \hld\ \alst{w}ordon spráki, &
hwó þár \alst{u}n-\alst{e}fno \hld\ \alst{e}rlos kwámun, &
\alst{w}eros te þem \alst{w}erke. \hld\ Só skulun fan þero \alst{w}er-oldi duon &
\alst{m}ann-kunnjes barn \hld\ an þat \alst{m}árjo lioht, &
\alst{g}umon an \alst{g}odes wang: \hld\ sum bi·ginnit ina \alst{g}iriwan sán &
an is \alst{k}indiski, \hld\ havit im gi·\alst{k}oranan muod, &
\alst{w}illjon guodan, \hld\ \alst{w}er-old-saka míðit, &
far·\alst{l}átit is \alst{l}usta; \hld\ ni mag ina is \alst{l}ík-hamo &
an un·\alst{sp}uod for·\alst{sp}anan: \hld\ \alst{sp}áhiða línot, &
\alst{g}odes êw, \hld\ \alst{g}ramono for·látit, &
\alst{w}rêðaro \alst{w}illjon, \hld\ duot im só te is \alst{w}er-oldi forð, &
\alst{l}êstit só an þeson \alst{l}iohte, \hld\ ant-þat im is \alst{l}íves kumit, &
\alst{a}ldres \alst{á}vand; \hld\ gi·wítit im þan \alst{u}p-wegos: &
þár wirðit im is \alst{a}rạvedi \hld\ \alst{a}ll gi·lônot, &
far·\alst{g}oldan mid \alst{g}uodu \hld\ an \alst{g}odes ríkje. &
Þat mêndun þia \alst{w}urụhtjon, \hld\ þia an þem \alst{w}ín-gardon &
\alst{á}dro an \alst{ú}hta \hld\ \alst{a}rvid-líko &
\alst{w}erk bi·gunnun \hld\ ęndi þuru·\alst{w}onodun forð, &
\alst{e}rlos unt \alst{á}vand. \hld\ Sum þár ôk an \alst{u}ndern kwam, &
habda þuo far·\alst{m}ęrrid, \hld\ þia \alst{m}orạgan-stunda &
þes \alst{d}ag-werkes for·\alst{d}uolon; \hld\ só duot \alst{d}oloro filo, &
gi·\alst{m}êdaro \alst{m}anno: \hld\ drívit im \alst{m}is-lík þing &
\alst{g}erno an is \alst{j}uguði, \hld\ —havit im \alst{g}elp-kwidi &
\alst{l}êða gi·\alst{l}ínot \hld\ ęndi \alst{l}ôs-word manag—, &
ant-þat is \alst{k}indiski \hld\ far·\alst{k}uman wirðit, &
þat ina after is \alst{j}uguði \hld\ \alst{g}odes anst manot &
\alst{b}líði an is \alst{b}rioston; \hld\ fáhit im te \alst{b}ęteron þan &
\alst{w}ordon ęndi \alst{w}erkon, \hld\ lêdit im is \alst{w}er-old mid þiu, &
is \alst{a}ldạr ant þena \alst{ę}ndi: \hld\ kumit im \alst{a}lles lôn &
an \alst{g}odes ríkje, \hld\ \alst{g}ódaro werko. &
Sum \alst{m}ann þan \alst{m}id-firi \hld\ \alst{m}ên far·látid, &
\alst{s}wára \alst{s}undjun, \hld\ fáhit im an \alst{s}álig þing, &
bi·\alst{g}innit im þuru \alst{g}odes kraft \hld\ \alst{g}uodaro werko, &
\alst{b}uotit \alst{b}alo-spráka, \hld\ látit im is \alst{b}ittrun dád &
an is \alst{h}ugje \alst{h}reuwan; \hld\ kumit im þiu \alst{h}elpa fon gode, &
þat im gi·\alst{l}êstid þie gi·\alst{l}ôvo, \hld\ só lango só im is \alst{l}íf warod; &
\alst{f}arit im \alst{f}orð mid þiu, \hld\ ant·\alst{f}áhit is mieda, &
\alst{g}uod lôn at \alst{g}ode; \hld\ ni sindun êniga \alst{g}eva bęteran. &
Sum bi·ginnit þan ôk \alst{f}urðor, \hld\ þan hie ist \alst{f}ruodot mêr, &
is \alst{a}ldạres af·hęldit, \hld\ —þan bi·ginnat im is \alst{u}vilon werk &
\alst{l}êðon an þeson \alst{l}iohte, \hld\ þan ina \alst{l}êra godes &
gi·\alst{m}anod an is \alst{m}uode: \hld\ wirðit im \alst{m}ildera hugi, &
þuru·\alst{g}ęngit im mid \alst{g}uodu \hld\ ęndi \alst{g}eld nimit, &
\alst{h}ôh \alst{h}imil-ríki, \hld\ þan hie \alst{h}inan węndit, &
wirðit im is \alst{m}ieda só sama, \hld\ só þem \alst{m}an *nun warð, &
þea þár te \alst{n}ónu dages, \hld\ an þea \alst{n}igunda tíd, &
an þene \alst{w}ín-gardon \hld\ \alst{w}irkjan kwámun. &
Sum wirðid þan só \alst{s}wíðo ge·fródot, \hld\ só hé ni wili is \alst{s}undja bótjen, &
ak hé \alst{ô}kid sie mid \alst{u}vilu ge·hwi-liku, \hld\ ant-tat imu is \alst{á}vand náhid, &
is \alst{w}er-old ęndi is \alst{w}unnja far·slítid; \hld\ þan be·ginnid hé imu \alst{w}íti an·dréden, &
is \alst{s}undjon werðad imu \alst{s}orga an móde: \hld\ ge·hugid hwat hé \alst{s}elvo ge·frumide &
\alst{g}rimmes þan lango, þe hé móste is \alst{j}uguðjo neoten; \hld\ ni mag þan mid ȯðru \alst{g}ódu gi·bótjen &
þea \alst{d}ádi, þea hé só \alst{d}ęrvja ge·frumide, \hld\ ak hé slęhit allaro \alst{d}ago ge·hwi-likes &
an is \alst{b}reost mid \alst{b}êðjun handun \hld\ ęndi wópit sie mid \alst{b}ittrun trahnun, &
\alst{h}lúdo hé sie mid \alst{h}ofnu kúmid, \hld\ bidid þene \alst{h}êlagon drohtin &
\alst{m}ahtigne, þat hé imu \alst{m}ildi werðe: \hld\ ni látid imu sïðor is \alst{m}ód gi·twífljen; &
só \alst{ê}-gróht-ful is, þe þár \alst{a}lles ge·węldid: \hld\ hé ni wili ênigumu \alst{i}rmin-manne &
far·\alst{w}ęrnjen \alst{w}illjan sínes; \hld\ far·givid imu \alst{w}aldand selvo &
\alst{h}êlag \alst{h}imil-ríki: \hld\ þan is imu gi·\alst{h}olpen sïður. &
\alst{A}lle skulun sie þár \alst{ê}ra ant·fáhen, \hld\ þoh sie þarod te \alst{ê}naru tídi &
ni \alst{k}umen, þat \alst{k}unni manno, \hld\ þoh wili imu þe \alst{k}raftigo drohtin, &
gi·\alst{l}ônon allaro \alst{l}iudjo só hwi-likumu, \hld\ só hér is gi·\alst{l}ôvon ant·fáhit: &
\alst{ê}n himil-ríki \hld\ givid hé \alst{a}llun þeodun, &
\alst{m}annun te \alst{m}édu. \hld\ Þat mênde \alst{m}ahtig Krist, &
\alst{b}arno þat \alst{b}ętste, \hld\ þó hé þat \alst{b}iliði sprak, &
hwó þár te þem \alst{w}ín-gardun \hld\ \alst{w}urhtjon kwámin, &
\alst{m}an \alst{m}is-líko: \hld\ þoh nam is \alst{m}éde ge·hwe &
\alst{f}ulle te is \alst{f}rôjan. \hld\ Só skulun \alst{f}iriho barn &
at \alst{g}ode selvumu \hld\ \alst{g}eld ant·fáhen, &
swíðo \alst{l}eov-lík \alst{l}ôn, \hld\ þoh sie sume só \alst{l}ate werðan.\eva

\bvb TODO.\evb\evg

\bvg\bva[43][3516]%
Hét imu þó þea is \alst{g}ódan \hld\ \alst{j}ungaron náhor &
\alst{t}we-livi gangan \hld\ —þea wárun imu \alst{t}riuwiston &
\alst{m}an ovar erðu—, \hld\ sagde im \alst{m}ahtig selvo &
\alst{ȯ}ðer-sïðu, \hld\ hwi-lik imu þár \alst{a}rvêdi &
\alst{t}ó-ward wárun: \hld\ „þes ni mag ênig \alst{t}weho werðen“, kwað hé; &
kwað þat sie þó te \alst{J}erusalem \hld\ an þat \alst{J}udeono folk &
\alst{l}íðan skoldin: \hld\ „þár wirðid all gi·\alst{l}êstid só, &
ge·\alst{f}rumid undar þemu \alst{f}olke, \hld\ só it an \alst{f}urn-dagun &
\alst{w}íse man be mí \hld\ \alst{w}ordun ge·sprákun. &
Þár skulun mí far·\alst{k}ôpon \hld\ undar þea \alst{k}raftigon þiod, &
\alst{h}ęliðos te þeru \alst{h}êri; \hld\ þár werðat mína \alst{h}ęndi ge·bundana, &
\alst{f}aðmos werðad mí þár ge·\alst{f}astnod; \hld\ \alst{f}ilu skal ik þár gi·þolojan, &
\alst{h}oskes gi·\alst{h}ôrjen \hld\ ęndi \alst{h}arm-kwidi, &
\alst{b}ismer-spráka \hld\ ęndi \alst{b}i-hêt-word manag; &
sie \alst{w}êgjat mí te wundron \hld\ \alst{w}ápnes ęggjun, &
bi·\alst{l}ôsjad mí \alst{l}ívu: \hld\ ik te þesumu \alst{l}iohte skal &
þurh \alst{d}rohtines kraft \hld\ fan \alst{d}ôðe a·standen &
an \alst{þ}riddjon dage. \hld\ Ni kwam ik undar þesa \alst{þ}eoda herod &
te þiu, þat mín \alst{ę}ldi-barn \hld\ \alst{a}rvêd habdin, &
þat mí \alst{þ}ionodi þius \alst{þ}iod: \hld\ ni willju ik is sie \alst{þ}iggjen nú, &
\alst{f}ergon þit \alst{f}olk-skępi, \hld\ ak ik skal imu te \alst{f}rumu werðen, &
\alst{þ}eonon imu \alst{þ}eo-líko \hld\ ęndi for alla þesa \alst{þ}eoda geven &
\alst{s}eole míne. \hld\ Ik willju sie \alst{s}elvo nú &
\alst{l}ôsjen mid mínu \alst{l}ívu, \hld\ þea hér \alst{l}ango bidun, &
\alst{m}an-kunnjes \alst{m}anag, \hld\ \alst{m}ínara helpa.“ &
\alst{F}ór imu þó \alst{f}orð-wardes \hld\ —habde imu \alst{f}asten hugi, &
\alst{b}líðjan an is \alst{b}reostun \hld\ \alst{b}arn drohtines— &
welda im te \alst{J}erusalem \hld\ \alst{J}udeo folkes &
\alst{w}illjon \alst{w}ísan: \hld\ hé konste þes \alst{w}erodes só garo &
\alst{h}ęti-grimmen \alst{h}ugi \hld\ ęndi \alst{h}ardan stríd, &
\alst{w}rêðan \alst{w}illjon. \hld\ \alst{W}erod sïðode &
furi \alst{J}erikho-burg; \hld\ was þe \alst{g}odes sunu, &
\alst{m}ahtig undar þero \alst{m}ęnigi. \hld\ Þár sátun twênje \alst{m}an bi wege, &
\alst{b}linde wárun sie \alst{b}êðje: \hld\ was im \alst{b}ótono þarf, &
þat sie ge·\alst{h}êldi \hld\ \alst{h}evanes waldand, &
hwand sie só \alst{l}ango \hld\ \alst{l}iohtes þolodun, &
\alst{m}anaga hwíla. \hld\ Sie gi·hôrdun þó þat \alst{m}ęgin faren &
ęndi \alst{f}rágodun sán \hld\ \alst{f}iri-wit-líko &
\alst{r}ęgini-blindun, \hld\ hwi-lik þár \alst{r}íki man &
undar þemu \alst{f}olk-skępi \hld\ \alst{f}urista wári, &
\alst{h}êrost an \alst{h}ôvid. \hld\ Þó sprak im ên \alst{h}ęlið an·gęgin, &
kwað þat þár \alst{J}esu Krist \hld\ fan \alst{G}alilea-lande, &
\alst{h}êljandero bętst \hld\ \alst{h}êrost wári, &
\alst{f}óri mid is \alst{f}olku. \hld\ Þó warð \alst{f}ráh-mód hugi &
\alst{b}êðjun þem \alst{b}lindun mannun, \hld\ þó sie þat \alst{b}arn godes &
\alst{w}issun under þemu \alst{w}erode: \hld\ hreopun im þó mid iro \alst{w}ordun tó, &
\alst{h}lúdo te þemu \alst{h}êlagon Kriste, \hld\ bádun þat hé im \alst{h}elpe ge·rédi: &
„\alst{d}rohtin \alst{D}awides sunu: \hld\ wis u̇s mid þínun \alst{d}ádjun mildi, &
\alst{n}ęri u̇s af þesaru \alst{n}ôdi, \hld\ só þú gi·\alst{n}óge dós &
\alst{m}anno kunnjes: \hld\ þú bist \alst{m}anagun gód, &
\alst{h}ilpis ęndi \alst{h}êlis.“ \hld\ Þo bi·gan im þat \alst{h}ęliðo folk &
\alst{w}ęrjen mid \alst{w}ordun, \hld\ þat sie an \alst{w}aldand Krist &
só \alst{h}lúdo ni \alst{h}riopin. \hld\ Si ni weldun im \alst{h}ôrjen te þiu, &
ak sie simla \alst{m}êr ęndi \alst{m}êr \hld\ ovar þat \alst{m}anno folk &
\alst{h}lúdo \alst{h}reopun. \hld\ \alst{H}éljand ge·stód, &
allaro \alst{b}arno \alst{b}ętst, \hld\ hét sie þó \alst{b}rengjen te imu, &
\alst{l}êdjen þurh þea \alst{l}iudi, \hld\ sprak im \alst{l}istjun tó &
\alst{m}ild-líko for þeru \alst{m}ęnegi: \hld\ „hwat willjad git \alst{m}ínaro hér“, kwað hé, &
„\alst{h}elpono \alst{h}abbjen?“ \hld\ Sie bádun ina \alst{h}êlagna, &
þat hé im ira \alst{ô}gon \hld\ \alst{o}pana gi·dádi, &
far·\alst{l}iwi þeses \alst{l}iohtes, \hld\ þat sie \alst{l}iudjo drôm, &
\alst{s}wigle \alst{s}unnun skín \hld\ gi·\alst{s}ehen móstin, &
\alst{w}liti-skônje \alst{w}er-old. \hld\ \alst{W}aldand frumide, &
\alst{h}rên sie þó mid is \alst{h}andun, \hld\ dede is \alst{h}elpe þár tó, &
þat þem \alst{b}lindun þó \hld\ \alst{b}êðjum wurðun &
\alst{ô}gon gi·\alst{o}ponod, \hld\ þat sie \alst{e}rðe ęndi himil &
þurh \alst{k}raft godes \hld\ ant·\alst{k}iennjen mahtun, &
\alst{l}ioht ęndi \alst{l}iudi. \hld\ Þó sagdun sie \alst{l}of gode, &
\alst{d}iurdun u̇san \alst{d}rohtin, \hld\ þes sie \alst{d}ages liohtes &
\alst{b}rúkan móstun: \hld\ ge·witun im \alst{b}êðje mid imu, &
\alst{f}olgodun is \alst{f}ęrdi: \hld\ was im þiu \alst{f}ruma giviðig, &
ęndi ôk \alst{w}aldandes \alst{w}erk \hld\ \alst{w}ído ge·ku̇ðid, &
\alst{m}anagun gi·\alst{m}árid.\eva

\bvb TODO.\evb\evg

\bvg\bva[44][3588]%
\hspace*{100pt}Þár was só \alst{m}ahtig-lík &
\alst{b}iliði gi·\alst{b}ôknid, \hld\ þár þe \alst{b}lindon man &
bi þemu \alst{w}ege sátun, \hld\ \alst{w}íti þolodun, &
\alst{l}iohtes \alst{l}ôse: \hld\ þat mênid þoh \alst{l}iudjo barn, &
al \alst{m}an-kunni, \hld\ hwó sie \alst{m}ahtig god &
an þemu \alst{a}na·ginne \hld\ þurh is \alst{ê}nes kraft &
\alst{s}in-híun twê \hld\ \alst{s}elvo gi·warhte, &
\alst{Á}dam ęndi \alst{É}wan: \hld\ far·gaf im \alst{u}p-wegos, &
\alst{h}imilo ríki; \hld\ ak þó warð im þe \alst{h}atola te náh, &
\alst{f}íund mid \alst{f}êknu \hld\ ęndi mid \alst{f}irin-werkun, &
bi·\alst{s}wêk sie mid \alst{s}undjun, \hld\ þat sie \alst{s}in-skôni, &
\alst{l}ioht far·\alst{l}étun: \hld\ wurðun an \alst{l}êðaron stędi, &
an þesen \alst{m}iddil-gard \hld\ \alst{m}an far·worpen, &
\alst{þ}olodun hér an \alst{þ}iustrju \hld\ \alst{þ}iod-arvêdi, &
\alst{w}unnun \alst{w}rak-sïðos, \hld\ \alst{w}elon þarvodun: &
far·\alst{g}átun \alst{g}odes ríkjes, \hld\ \alst{g}ramon þeonodun, &
\alst{f}íundo barnun; \hld\ sie guldun is im mid \alst{f}iuru lôn &
an þeru \alst{h}êton \alst{h}ęllju. \hld\ Be·þiu wárun siu an iro \alst{h}ugi blinda &
an þesaru \alst{m}iddil-gard, \hld\ \alst{m}ęnniskono barn, &
hwand siu ine ni ant·\alst{k}iendun, \hld\ \alst{k}raftagne god, &
\alst{h}imilisken \alst{h}êrron, \hld\ þene þe sie mid is \alst{h}andun gi·skóp, &
gi·\alst{w}arhte an is \alst{w}illjon. \hld\ Þius \alst{w}er-old was þó só far·hwęrvid, &
bi·\alst{þ}wungen an \alst{þ}iustrje, \hld\ an \alst{þ}iod-arvidi, &
an \alst{d}ôðes \alst{d}alu: \hld\ sátun im þó bi þeru \alst{d}rohtines strátun &
\alst{j}ámar-móde, \hld\ \alst{g}odes helpe bidun: &
siu ni mahte im þó êr \alst{w}erðen, \hld\ êr þan \alst{w}aldand god &
an þesan \alst{m}iddil-gard, \hld\ \alst{m}ahtig drohtin, &
is \alst{s}elves \alst{s}unu \hld\ \alst{s}ęndjen weldi &
þat hé \alst{l}ioht ant·\alst{l}uki \hld\ \alst{l}iudjo barnun, &
\alst{o}ponodi im \alst{ê}wig líf, \hld\ þat sie þene \alst{a}lo-waldon &
mahtin ant·\alst{k}ęnnjen wel, \hld\ \alst{k}raftagna god. &
Ôk mag ik giu gi·\alst{t}ęlljen, \hld\ of gí þár \alst{t}ó willjad &
\alst{h}uggjen ęndi \alst{h}ôrjen, \hld\ þat gí þes \alst{h}êljandes mugun &
\alst{k}raft ant·\alst{k}ęnnjen, \hld\ hwó is \alst{k}umi wurðun &
an þesaru \alst{m}iddil-gard \hld\ \alst{m}anagun te helpu, &
ia hwat hé mid þem \alst{d}ádjun \hld\ \alst{d}rohtin selvo &
\alst{m}anages \alst{m}ênde, \hld\ ia be·hwiu þiu \alst{m}árje burg &
\alst{J}erikho hêtid, \hld\ þiu þár an \alst{J}udeon stád &
gi·\alst{m}akod mid \alst{m}úrun: \hld\ þiu is aftar þemu \alst{m}ánen gi·nęmnid, &
aftar þemu \alst{t}orhten \alst{t}ungle: \hld\ hé ni mag is \alst{t}ídi be·míðen, &
ak hé \alst{d}ago ge·hwi-likes \hld\ \alst{d}uod ȯðer-hweðer, &
\alst{w}anod ohþo \alst{w}ahsid. \hld\ Só dód an þesaro \alst{w}er-oldi hér, &%TODO: check ohþo
an þesaru \alst{m}iddil-gard \hld\ \alst{m}ęnniskono barn: &
\alst{f}arad ęndi \alst{f}olgod, \hld\ \alst{f}róde stervad, &
werðad \alst{e}ft junga \hld\ \alst{a}ftar kumane, &
\alst{w}eros a·\alst{w}ahsane, \hld\ unt-tat sie eft \alst{w}urd far·nimid. &
Þat mênde þat \alst{b}arn godes, \hld\ þó hé fon þeru \alst{b}urgi fór, &
þe \alst{g}ódo fan \alst{J}erikho, \hld\ þat ni mahte êr werðen \alst{g}umono barnun &
þiu \alst{b}lindja gi·\alst{b}ótid, \hld\ þat sie þat \alst{b}erhte lioht, &
gi·\alst{s}áhin \alst{s}in-skôni, \hld\ êr þan hé \alst{s}elvo hér &
an þesaru \alst{m}iddil-gard \hld\ \alst{m}ęnniski ant·féng, &
\alst{f}lêsk ęndi lík-hamon. \hld\ Þó wurðun þes \alst{f}iriho barn &
gi·\alst{w}ar an þesaru \alst{w}er-oldi, \hld\ þe hér an \alst{w}ítje êr, &
\alst{s}átun an \alst{s}undjun \hld\ gi·\alst{s}iunjes lôse, &
\alst{þ}olodun an \alst{þ}iustrje, \hld\ —sie af·sóvun þat was þesaru \alst{þ}iod kuman &
\alst{h}êljand te \alst{h}elpu \hld\ fan \alst{h}evan-ríkje, &
\alst{K}rist allaro \alst{k}uningo bęst; \hld\ sie mahtun is ant·\alst{k}ęnnjen sán, &
gi·\alst{f}óljen is \alst{f}ardjo. \hld\ Þó sie só \alst{f}ilu hriopun, &
þe \alst{m}an te þemu \alst{m}ahtigon gode, \hld\ þat im \alst{m}ildi aftar þiu &
\alst{w}aldand \alst{w}urði. \hld\ Þan \alst{w}ęridun im swíðo &
þia \alst{s}wárun \alst{s}undjon, \hld\ þe sie im êr \alst{s}elvon gi·dádun, &
\alst{l}ettun sie þes gi·\alst{l}ôbon. \hld\ Sie ni mahtun þem \alst{l}iudjun þoh &%TODO: check "lettun"
bi·\alst{w}ęrjen iro \alst{w}illjon, \hld\ ak sie an \alst{w}aldand god &
\alst{h}lúdo \alst{h}riopun, \hld\ ant-tat hé im iro \alst{h}êli far·gaf, &
þat sie \alst{s}in-líf \hld\ gi·\alst{s}ehen móstin, &
\alst{o}pen \alst{ê}wig lioht \hld\ ęndi \alst{a}n faren &
an þiu \alst{b}erhtun \alst{b}ú. \hld\ Þat mêndun þea \alst{b}lindun man, &
þe þár bi \alst{J}erikho-burg \hld\ te þemu \alst{g}odes barne &
\alst{h}lúdo \alst{h}riopun, \hld\ þat hé im iro \alst{h}êli far·lihi, &
\alst{l}iohtes an þesumu \alst{l}íve: \hld\ þan im þea \alst{l}iudi só filu &
\alst{w}ęridun mid \alst{w}ordun, \hld\ þea þár an þemu \alst{w}ege fórun &
bi·\alst{f}oren ęndi bi·hinden: \hld\ só dót þea \alst{f}irin-sundjon &
an þesaru \alst{m}iddil-gard \hld\ \alst{m}an-kunnje. &
hôrjad nú hwó þie \alst{b}lindun, \hld\ sïður im gi·\alst{b}ótid warð, &
þat sie \alst{s}unnun lioht \hld\ ge·\alst{s}ehen móstun, &
hwó si þó \alst{d}ádun: \hld\ ge·witun im mid iro \alst{d}rohtine samad, &
\alst{f}olgodun is \alst{f}ęrdi, \hld\ sprákun \alst{f}ilu wordo &
þemu \alst{l}andes hirdje te \alst{l}ove: \hld\ só dód im noh \alst{l}iudjo barn &
\alst{w}ído aftar þesaru \alst{w}er-oldi, \hld\ sïður im \alst{w}aldand Krist &
ge·\alst{l}iuhte mid is \alst{l}êrun \hld\ ęndi im \alst{l}íf êwig, &
\alst{g}odes ríki far·\alst{g}af \hld\ \alst{g}ódun mannun, &
\alst{h}ôh \alst{h}imiles lioht \hld\ ęndi is \alst{h}elpe þár tó, &
só hwemu só þat gi·\alst{w}erkod, \hld\ þat hé móti þemu is \alst{w}ege folgon.\eva

\bvb TODO.\evb\evg

\bvg\bva[45][3671]%
Þó \alst{n}áhide \hld\ \alst{n}ęrjendo Krist, &
þe \alst{g}ódo te \alst{J}erusalem. \hld\ Kwam imu þár te·\alst{g}ęgnes filu &
\alst{w}erodes an \alst{w}illjon \hld\ \alst{w}el huggendjes, &
ant·\alst{f}éngun ina \alst{f}agạro \hld\ ęndi imu bi·\alst{f}oren stręidun &%NOTE: ęi is original.
þene \alst{w}eg mid iro gi·\alst{w}ádjun \hld\ ęndi mid \alst{w}urtjun só same, &
mid \alst{b}erhtun \alst{b}lómun \hld\ ęndi mid \alst{b}ômo tógun, &
þat \alst{f}eld mid \alst{f}agạron palmun, \hld\ al só is \alst{f}ard ge·buride, &
þat þe \alst{g}odes sunu \hld\ \alst{g}angan welde &
te þeru \alst{m}árjan burg. \hld\ Hwarf ina \alst{m}ęgin umbi &
\alst{l}iudjo an \alst{l}ustun, \hld\ ęndi \alst{l}of-sang a·hóf &
þat \alst{w}erod an \alst{w}illjon: \hld\ sagdun \alst{w}aldande þank, &
þes þár \alst{s}elvo kwam \hld\ \alst{s}unu Dawides &
\alst{w}íson þes \alst{w}erodes. \hld\ Þó ge·sah \alst{w}aldand Krist &
þe \alst{g}ódo te \alst{J}erusalem, \hld\ \alst{g}umono bętsta, &
\alst{b}líkan þene \alst{b}urges wal \hld\ ęndi \alst{b}ú Judeono, &
\alst{h}ôha \alst{h}orn-sęli \hld\ ęndi ôk þat \alst{h}ús godes, &
allaro \alst{w}ího \alst{w}un-samost. \hld\ Þó \alst{w}el imu an innen &
\alst{h}ugi wið is \alst{h}erte: \hld\ þó ni mahte þat \alst{h}êlage barn &
\alst{w}ópu a·\alst{w}ísjen, \hld\ sprak þó \alst{w}ordo filu &
\alst{h}riuwig-líko \hld\ —was imu is \alst{h}ugi sêreg—: &
„\alst{w}ê warð þí, Jerusalem“, \hld[kwað hé,] „þes þú te \alst{w}árun ni wêst &
þea \alst{w}urde-gi·skęfti, \hld\ þe þí noh gi·\alst{w}erðen skulun, &
\alst{h}wó þú noh wirðis be·\alst{h}abd \hld\ \alst{h}ęrjes kraftu &
ęndi þí bi·\alst{s}ittjad \hld\ \alst{s}líð-móde man, &
\alst{f}íund mid \alst{f}olkun. \hld\ Þan ni havas þú \alst{f}riðu hwęrgin, &
\alst{m}und-burd mid \alst{m}annun: \hld\ lêdjad þi hér \alst{m}anage tó &
\alst{o}rdos ęndi \alst{ę}ggja, \hld\ \alst{o}r-legas word, &
far·\alst{f}ioþ þín \alst{f}olk-skępi \hld\ \alst{f}iures liomon, &
þese \alst{w}íki a·\alst{w}óstjad, \hld\ \alst{w}allos hôha &
\alst{f}ęlljad te \alst{f}oldun: \hld\ ni af·stád is \alst{f}elis nígijan, &
\alst{st}ên ovar ȯðrumu, \hld\ ak werðad þesa \alst{st}ędi wóstja &
umbi \alst{J}erusalem \hld\ \alst{J}udeo liudjo, &
hwand sie ni ant·\alst{k}ęnnjad, \hld\ þat im \alst{k}umana sind &
iro \alst{t}ídi \alst{t}ó-wardes, \hld\ ak sie habbjad im \alst{t}wífljen hugi, &
ni \alst{w}itun þat iro \alst{w}ísad \hld\ \alst{w}aldandes kraft.“ &
Gi·wêt imu þó mid þeru \alst{m}ęnegi \hld\ \alst{m}anno drohtin &
an þea \alst{b}erhton \alst{b}urg. \hld\ Só þó þat \alst{b}arn godes &
innan \alst{J}erusalem \hld\ mid þiu \alst{g}umono folku, &
\alst{s}êg mid þiu ge·\alst{s}ïðu, \hld\ þó warð þár allaro \alst{s}ango mêst, &
\alst{h}lúd stemnje af·\alst{h}aven \hld\ \alst{h}êlagun wordun, &
\alst{l}ovodun þene \alst{l}andes ward \hld\ \alst{l}iudjo męnegi, &
\alst{b}arno þat \alst{b}ętste; \hld\ þiu \alst{b}urg warð an hróru, &
þat \alst{f}olk warð an \alst{f}orhtun \hld\ ęndi \alst{f}rágodun sán, &
hwe þat \alst{w}ári, \hld\ þat þár mid þiu \alst{w}erodu kwam, &
mid þeru \alst{m}ikilon \alst{m}ęnegi. \hld\ Þó sprak im ên \alst{m}an an·gęgin, &
kwað þat þár \alst{J}esu Krist \hld\ fan \alst{G}alileo lande, &
fan \alst{N}azareth-burg \hld\ \alst{n}ęrjand kwámi, &
\alst{w}itig \alst{w}ár-sago \hld\ þemu \alst{w}erode te helpu. &
Þó was þem \alst{J}udiun, \hld\ þe imu êr \alst{g}rame wárun, &
un·\alst{h}olde an \alst{h}ugi, \hld\ \alst{h}arm an móde, &
þat imu þea \alst{l}iudi só filu \hld\ \alst{l}of-sang warhtun, &
\alst{d}iurdun iro drohtin. \hld\ Þó géngun \alst{d}ol-móde, &
þat sie wið \alst{w}aldand Krist \hld\ \alst{w}ordun sprákun, &
bádun þat hé þat ge·\alst{s}ïði \hld\ \alst{s}wígon héti, &
\alst{l}etti þea \alst{l}iudi, \hld\ þat sie imu \alst{l}of só filu &
\alst{w}ordun ni \alst{w}arhtin: \hld\ „it is þesumu \alst{w}erode lêð“, kwáðun sie, &
„þesun \alst{b}urg-liudjun.“ \hld\ Þó sprak eft þat \alst{b}arn godes: &
„ef gí sie a·\alst{m}ęrrjad“, \hld[kwað hé,] „þat hér ni mótin \alst{m}anno barn &
\alst{w}aldandes kraft \hld\ \alst{w}ordun diurjen, &
þan skulun it \alst{h}rópen þoh \hld\ \alst{h}arde stênos &
for þesumu \alst{f}olk-skępi, \hld\ \alst{f}elisos starka, &
êr þan it eo be·\alst{l}íve, \hld\ nevo man is \alst{l}of spreke &
\alst{w}ído aftar þesaru \alst{w}er-oldi.“ \hld\ Þó hé an þene \alst{w}íh innen, &
\alst{g}éng an þat \alst{g}odes hús: \hld\ fand þár \alst{J}udeono filu, &
\alst{m}is-líke \alst{m}an, \hld\ \alst{m}anage at·samne, &
þea im þár \alst{k}ôp-stędi \hld\ gi·\alst{k}oran habdun, &
\alst{m}angodun im þár mid \alst{m}anages hwí: \hld\ \alst{m}unitęrjas sátun &
an þemu \alst{w}íhe innan, \hld\ habdun iro \alst{w}esl gi·dago &
\alst{g}aru te \alst{g}evanne. \hld\ Þat was þemu \alst{g}odes barne &
\alst{a}l an \alst{a}ndun: \hld\ drêf sie \alst{ú}t þanen &
\alst{r}úmo fan þemu \alst{r}akude, \hld\ kwað þat wári \alst{r}ehtara dád, &
þat þár te \alst{b}edu fórin \hld\ \alst{b}arn Israheles &%TODO: check bedu
„ęndi an þesumu mínumu \alst{h}úse \hld\ \alst{h}elpono biddjan, &
þat sia \alst{s}igi-drohtin \hld\ \alst{s}undjono tuomje, &
þan hér \alst{þ}eovas \hld\ an \alst{þ}ing-stędi halden, &
þea far·\alst{w}arhton \alst{w}eros \hld\ \alst{w}ehsal drívan, &
\alst{u}n-reht \alst{ê}n-fald. \hld\ Ne gí êniga \alst{ê}ra ni witun &
þeses \alst{g}odes húses, \hld\ \alst{J}udeo liudi.“ &
Só \alst{r}úmde hé þó ęndi \alst{r}ekode, \hld\ \alst{r}íki drohtin, &
þat \alst{h}êlaga \alst{h}ús \hld\ ęndi an \alst{h}elpun was &
\alst{m}anagumu \alst{m}an-kunnje, \hld\ þem þe is \alst{m}ikilon kraft &
\alst{f}errene ge·\alst{f}rugnun \hld\ ęndi þár gi·\alst{f}aran kwámun &
ovar \alst{l}angan weg. \hld\ Warð þár \alst{l}éf so manag, &
\alst{h}alt gi·\alst{h}êlid \hld\ ęndi \alst{h}áf só same, &
\alst{b}lindun gi·\alst{b}ótid. \hld\ Só dede þat \alst{b}arn godes &
\alst{w}illjendi þemu \alst{w}erode, \hld\ hwand al an is gi·\alst{w}ęldi stéd &
umbi þesaro \alst{l}iudjo \alst{l}íf \hld\ ęndi ôk umbi þit \alst{l}and só same.\eva

\bvb TODO.\evb\evg

\bvg\bva[46][3758]%
Stód imu þó fora þemu \alst{w}íhe \hld\ \alst{w}aldandjo Krist, &
\alst{l}iof \alst{l}andes ward, \hld\ ęndi imu þero \alst{l}iudjo hugi, &
iro \alst{w}illjon aftar·\alst{w}arode: \hld\ gi·sah \alst{w}erod mikil &
an þat \alst{m}árje hús \hld\ \alst{m}êðmos fórjen, &
\alst{g}evon mid \alst{g}oldu \hld\ ęndi mid \alst{g}odu-wębbju, &
\alst{d}iurjun fratahun. \hld\ Þat al \alst{d}rohtin Krist &
\alst{w}arode \alst{w}ís-líko. \hld\ Þó kwam þár ôk ên \alst{w}idowa tó, &
\alst{i}dis \alst{a}rm-skapen, \hld\ ęndi te þemu \alst{a}lạha géng &
ęndi siu an þat \alst{t}resur-hús \hld\ \alst{t}wêne lęgde &
\alst{ê}ríne skattos: \hld\ was iru \alst{ê}n-fald hugi, &
\alst{w}illjan gódes. \hld\ Þó sprak \alst{w}aldand Krist, &
þe \alst{g}umo wið is \alst{j}ungaron, \hld\ kwað þat siu þár \alst{g}eva brȧhti &
\alst{m}êron \alst{m}ikilu þan ęlkor \hld\ ênig \alst{m}annes sunu: &
„ef hér \alst{ô}daga man“, \hld[kwað hé,] „\alst{ê}ra brȧhtun, &
\alst{m}êðọm-hord \alst{m}anag, \hld\ sie létun im \alst{m}êr at hús &
\alst{w}elona ge·\alst{w}unnen. \hld\ Ni dede þius \alst{w}idowa só, &
ak siu te þesumu \alst{a}lạhe gaf \hld\ \alst{a}l þat siu habde &
\alst{w}elono ge·\alst{w}unnen, \hld\ só siu iru \alst{w}iht ni far·lét &
\alst{g}ódes an iro \alst{g}ardun. \hld\ Be·þiu sind ira \alst{g}eva mêron, &
\alst{w}aldande \alst{w}erða, \hld\ hwand siu it mid su·likumu \alst{w}illjon dede &
te þesumu \alst{g}odes húse. \hld\ Þes skal siu \alst{g}eld niman, &
swíðo \alst{l}ang-sam \alst{l}ôn, \hld\ þes siu su·likan gi·\alst{l}ôvon havad.“ &
Só gi·fragn ik þat þár an þemu \alst{w}íhe \hld\ \alst{w}aldandjo Krist &
allaro \alst{d}ago ge·hwi-likes, \hld\ \alst{d}rohtin manno, &
\alst{w}ísde mid \alst{w}ordun. \hld\ Stód ine \alst{w}erod umbi, &
\alst{g}rôt folk \alst{J}udeono, \hld\ gi·hôrdun is \alst{g}ódan word, &
\alst{s}wótja \alst{s}ęggjan. \hld\ Sum só \alst{s}álig warð &
\alst{m}anno undar þeru \alst{m}ęnegi, \hld\ þat it bi·gan an is \alst{m}ód hladen; &
\alst{l}ínodun im þea \alst{l}êra, \hld\ þe þe \alst{l}andes ward &
al be \alst{b}iliðjun sprak, \hld\ \alst{b}arn drohtines. &
Sumun wárun eft so \alst{l}êða \hld\ \alst{l}êra Kristes, &
\alst{w}aldandes \alst{w}ord: \hld\ was im \alst{w}iðẹr-mód hugi &
allun þem, þe an þemu \alst{h}ęri-skępi \hld\ \alst{h}êrost wárun, &
\alst{f}uriston an þemu \alst{f}olke: \hld\ \alst{f}áres hugdun &
\alst{w}rêða mid iro \alst{w}ordun \hld\ —habdun im \alst{w}iðẹr-sakon &
gi·\alst{h}aloden te \alst{h}elpu, \hld\ þes \alst{h}êroston man, &
\alst{E}rodeses þegạn, \hld\ þe þár \alst{a}nd-ward stód &
\alst{w}rêðes \alst{w}illjan, \hld\ þat hé iro \alst{w}ord ovar-hôrdi— &
ef sie ina for·\alst{f}éngin, \hld\ þat sie ina þan \alst{f}eteros an, &
þea \alst{l}iudi \alst{l}iðo-bęndi \hld\ \alst{l}ęggjen móstin, &
\alst{s}undja lôsan. \hld\ Þó géngun im þea ge·\alst{s}ïðos tó &
\alst{b}ittra gi·hugde, \hld\ þat sie wið þat \alst{b}arn godes, &
\alst{w}rêða \alst{w}iðẹr-sakon \hld\ \alst{w}ordun sprákun: &
„Hwat þú bist \alst{ê}o-sago“, \hld[kwáðun sie,] „\alst{a}llun þiodun, &
\alst{w}ísis \alst{w}áres só filu: \hld\ nis þi \alst{w}erð eo·wiht &
te bi·\alst{m}íðanne \hld\ \alst{m}anno ni-ênumu &
umbi is \alst{r}íki-dóm, \hld\ nevo þú simlun þat \alst{r}eht sprikis &
ęndi an þene \alst{g}odes weg \hld\ \alst{g}umono ge·sïði &
\alst{l}êdis mid þinun \alst{l}êrun: \hld\ ni mag þi \alst{l}aster man &
\alst{f}ïðan undar þesumu \alst{f}olke. \hld\ Nú wí þi \alst{f}rágon skulun. &
\alst{r}íki þiodan, \hld\ hwi-lik \alst{r}eht havad &
þe \alst{k}êsur fan Rúmu, \hld\ þe imu te þesumu \alst{k}unnje herod &
\alst{t}insi sókid \hld\ ęndi gi·\alst{t}ald havad, &
hwat wí imu \alst{g}elden skulin \hld\ \alst{g}ę́ro ge·hwi-likes &
\alst{h}ôvid-skatto. \hld\ Saga hwat þí þes an þínumu \alst{h}ugi þunkja: &
is it \alst{r}eht þe nis? \hld\ \alst{R}ád for þínun &
\alst{l}and-mę́gun wel: \hld\ u̇s is þínaro \alst{l}êrono þarf.“ &
Sie weldun þat hé it ant·\alst{k}wáði: \hld\ þan mahte hé þoh ant·\alst{k}ęnnjen wel &
iro \alst{w}rêðon \alst{w}illjon: \hld\ „te hwí gí \alst{w}ár-logon“, kwað hé, &
„\alst{f}andot mín só \alst{f}rókno? \hld\ Ni skal iu þat te \alst{f}rumu werðen, &
þat gí \alst{d}reogerjas \hld\ \alst{d}arnungo nú &
willjad mí far·\alst{f}áhen.“ \hld\ Hét hé þó \alst{f}orð dragan &
te \alst{sk}awonne þe \alst{sk}attos, \hld\ „þe gí \alst{sk}uldige sind &
an þat \alst{g}eld \alst{g}even.“ \hld\ \alst{J}udeon drógun &
ênna \alst{s}ilụvrinna forð: \hld\ \alst{s}áhun manage tó, &
hwó hé was ge·\alst{m}unitod: \hld\ was an \alst{m}iddjen skín &
þes \alst{k}êsures biliði \hld\ —þat mahtun sie ant·\alst{k}ęnnjen wel—, &
iro \alst{h}êrron \alst{h}ôvid-mál. \hld\ Þó frágode sie þe \alst{h}êlago Krist, &
aftar hwemu þiu ge·\alst{l}ík-nessi \hld\ gi·\alst{l}egid wári. &
Sie kwáðun þat it \alst{w}ári \hld\ \alst{w}er-old-kêsures &
fan \alst{R}úmu-burg, \hld\ „þes þe alles þeses \alst{r}íkes havad &
ge·\alst{w}ald an þesaru \alst{w}er-oldi.“ \hld\ „Þan willju ik iu te \alst{w}árun hér“, kwað hé, &
„\alst{s}elvo \alst{s}ęggjan, \hld\ þat gí imu \alst{s}ín gevad, &
\alst{w}er-old-hêrron is ge·\alst{w}unst, \hld\ ęndi \alst{w}aldand gode &
\alst{s}ęlljad, þat þár \alst{s}ín ist: \hld\ þat skulun iuwa \alst{s}eolon wesen, &
\alst{g}umono \alst{g}êstos.“ \hld\ Þó warð þero \alst{J}udeono hugi &
ge·\alst{m}insod an þemu \alst{m}ahle: \hld\ ni mahtun þe \alst{m}ên-skaðon &
\alst{w}ordun ge·\alst{w}innen, \hld\ só iro \alst{w}illjo géng, &
þat sie ina far·\alst{f}éngin, \hld\ hwand imu þat \alst{f}riðu-barn godes &
\alst{w}ardode wið þe \alst{w}rêðon \hld\ ęndi im \alst{w}ár an·gęgin, &
\alst{s}ȯð-spel \alst{s}agde, \hld\ þoh sie ni wárin só \alst{s}álige te þiu, &
þat sie it só far·\alst{f}éngin, \hld\ só it iro \alst{f}ruma wári.\eva

\bvb TODO.\evb\evg

\bvg\bva[47][3840]%
Sie ni weldun it þoh far·\alst{l}áten, \hld\ ak hétun þár \alst{l}êdjen forð &
ên \alst{w}íf for þemu \alst{w}erode, \hld\ þiu habde \alst{w}am ge·frumid, &
\alst{u}n-reht \alst{ê}n-fald: \hld\ þiu \alst{i}dis was bi·fangen &
an far·\alst{l}egar-nessi, \hld\ was iro \alst{l}íves skolo, &
þat sie \alst{f}iriho barn \hld\ \alst{f}erạhu bi·námin, &
\alst{ê}htin iro \alst{a}ldres: \hld\ só was an iro \alst{ê}w ge·skriven. &
Sie bi·gunnun ina þó \alst{f}rágon, \hld\ \alst{f}ruokne liudi, &
\alst{w}rêða mid iro \alst{w}ordun, \hld\ hwat sie skoldin þemu \alst{w}íve duan, &
hweðer sie sie \alst{k}węlidin, \hld\ þe sie sie \alst{k}wika létin, &
þe hwat hé umbi su·lika \alst{d}ádi \hld\ a·\alst{d}êljen weldi: &
„þú wêst, hwó þesaru \alst{m}ęnegi“, \hld[kwáðun sie,] „\alst{M}oyses gi·bôd &
\alst{w}árun \alst{w}ordun, \hld\ þat allaro \alst{w}ívo ge·hwi-lik &
an far·\alst{l}egar-nessi \hld\ \alst{l}íves far·warhti &
ęndi þat sie þan a·\alst{w}urpin \hld\ \alst{w}eros mid handun, &
\alst{st}arkun \alst{st}ênun: \hld\ nú maht þú sie sehan \alst{st}anden hér &
an \alst{s}undjun bi·fangan: \hld\ \alst{s}aga hwat þú is willjes.“ &
\alst{w}eldun ine þea \alst{w}iðẹr-sakon \hld\ \alst{w}ordun far·fáhen, &
ef hé þat gi·\alst{k}wáði, \hld\ þat sie sie \alst{k}wika létin, &
\alst{f}riðodi ira \alst{f}erạhe, \hld\ þan weldi þat \alst{f}olk Judeono &
kweðen, þat hé iro \alst{a}ldiron \hld\ \alst{ê}o wiðẹr-sagdi, &
þero \alst{l}iudjo \alst{l}and-reht; \hld\ ef hé sie þan héti \alst{l}ívu bi·nimen, &
þea \alst{m}agað fur þeru \alst{m}ęnegi, \hld\ þan weldin sie kweðen, þat hé só \alst{m}ildjene hugi &
ni \alst{b}ári an is \alst{b}reostun, \hld\ só skoldi habbjen \alst{b}arn godes: &
weldun sie só \alst{h}weðeres \hld\ \alst{h}êlagne Krist &
þero \alst{w}ordo ge·\alst{w}ítnon, \hld\ só hé þár for þemu \alst{w}erode ge·spráki, &
a·\alst{d}êldi te \alst{d}óme. \hld\ Þan wisse \alst{d}rohtin Krist &
þero \alst{m}anno só garo \hld\ \alst{m}ód-gi·þȧhti, &
iro \alst{w}rêðon \alst{w}illjon; \hld\ þó hé te þemu \alst{w}erode sprak, &
te \alst{a}llun þem \alst{e}rlun: \hld\ „só hwi-lik só iuwar \alst{á}no sí“, kwað hé, &
„\alst{s}líðja \alst{s}undjon, \hld\ só ganga iru \alst{s}elvo tó &
ęndi sie at \alst{ê}rist \hld\ \alst{e}rl mid is handun &
\alst{st}ên ana werpe.“ \hld\ Só \alst{st}ódun Judeon, &
\alst{þ}ȧhtun ęndi \alst{þ}agodun: \hld\ ni mahte \alst{þ}egạn nigijan &
wið þem \alst{w}ord-kwidi \hld\ \alst{w}iðẹr-saka finden: &
ge·hugde \alst{m}anno ge·hwi-lik \hld\ \alst{m}ên-gi·þȧhti, &
is \alst{s}elves \alst{s}undja: \hld\ ni was iro só \alst{s}ikur ênig, &
þat hé bi þemu \alst{w}orde \hld\ þemu \alst{w}íve ge·dorsti &
\alst{st}ên an werpen, \hld\ ak létun sie \alst{st}anden þár &
\alst{ê}nan þár inne \hld\ ęndi im \alst{ú}t þanen &
\alst{g}éngun \alst{g}ram-harde \hld\ \alst{J}udeo liudi, &
\alst{ê}n aftar \alst{ȯ}ðrumu, \hld\ ant-tat iro þár \alst{ê}nig ni was &
þes \alst{f}íundo \alst{f}olkes, \hld\ þe iro \alst{f}erhes þó, &
þeru \alst{i}dis \alst{a}ldạr-lago \hld\ \alst{á}htjen weldi. &
Þó gi·\alst{f}ragn ik þat sie \alst{f}rágode \hld\ \alst{f}riðu-barn godes, &
allaro \alst{g}umono bętst: \hld\ „Hwár kwámun þit \alst{J}udeono folk“, kwað hé, &
„þíne \alst{w}iðẹr-sakon, \hld\ þea þi hér \alst{w}rógdun te mí? &
Ne sie þí \alst{h}iudu wiht \hld\ \alst{h}armes ne gi·dádun, &
þea \alst{l}iudi \alst{l}êðes, \hld\ þe þí weldun \alst{l}ívu be·niman, &
\alst{w}êgjan te \alst{w}undrun?“ \hld\ Þó sprak imu eft þat \alst{w}íf an·gęgin, &
kwað þat iru þár \alst{n}io·man \hld\ þurh þes \alst{n}ęrjandan &
\alst{h}êlaga \alst{h}elpa \hld\ \alst{h}arm ne gi·frumidi &
\alst{w}ammes te lône. \hld\ Þó sprak eft \alst{w}aldand Krist, &
\alst{d}rohtin manno: \hld\ „ne ik þi geþ ni \alst{d}ęrju n·eo·wiht“, kwað hé, &
„ak gang þí \alst{h}êl \alst{h}inen, \hld\ lát þi an þínumu \alst{h}ugi sorga, &
þat þú nio \alst{s}ïð aftar þius \hld\ \alst{s}undig ni werðes.“ &
\alst{H}abde iru þó gi·\alst{h}olpen \hld\ \alst{h}êlag barn godes, &
ge·\alst{f}riðot iro \alst{f}erạhe. \hld\ Þan stód þat \alst{f}olk Judeono &
\alst{u}viles \alst{a}n-mód \hld\ só fan \alst{ê}ristan, &
\alst{w}rêðes \alst{w}illjan, \hld\ hwó sie \alst{w}ord-hęti &
wið þat \alst{f}riðu-barn godes \hld\ \alst{f}rummjen móstin. &
Habdun þea \alst{l}iudi an twê \hld\ mid iro gi·\alst{l}ôvon gi·fangan: &
was þiu \alst{s}male þioda \hld\ \alst{s}ínes willjan &
\alst{g}ernora mikilu, \hld\ þes \alst{g}odes barnes word &
te ge·\alst{f}rummjenne, \hld\ só im iro \alst{f}râho gi·bôd: &
\alst{r}ómodun te \alst{r}ehta \hld\ bet þan þie \alst{r}íkjon man, &
\alst{h}abdun ina far iro \alst{h}êrron \hld\ ia far \alst{h}evan-kuning, &
ful·\alst{g}éngun imu \alst{g}erno. \hld\ Þó gi·wêt imu þe \alst{g}odes sunu &
an þene \alst{w}íh innan: \hld\ hwarf ina \alst{w}erod umbi, &
\alst{m}ęgin-þiodo gi·\alst{m}ang. \hld\ Hé an \alst{m}iddjen stód, &
\alst{l}êrde þea \alst{l}iudi \hld\ \alst{l}iohtun wordun, &
\alst{h}lúdero stemnun: \hld\ was \alst{h}lust mikil, &
\alst{þ}agode \alst{þ}egạn manag, \hld\ ęndi hé þeru \alst{þ}iod gi·bôd, &
só hwe só þár mid \alst{þ}urstu \hld\ bi·\alst{þ}wungan wári, &
„só ganga imu herod \alst{d}rinkan te mí“, \hld[kwað hé,] „\alst{d}ago ge·hwi-likes &
\alst{s}wótjes brunnan. \hld\ Ik mag \alst{s}ęggjan iu, &
só hwe só hér gi·\alst{l}ôvid te mí \hld\ \alst{l}iudjo barno &
\alst{f}asto undar þesumu \alst{f}olke, \hld\ þat imu þan \alst{f}lioten skulun &
fan is \alst{l}ík-hamon \hld\ \alst{l}ibbjendi flód, &
\alst{i}rnandi water, \hld\ \alst{a}ho-spring mikil, &
\alst{k}umad þanen \alst{k}wika brunnon. \hld\ Þesa \alst{k}widi werðad wára, &
\alst{l}iudjun gi·\alst{l}êstid, \hld\ só hwemu só hér gi·\alst{l}ôvid te mí.“ &
Þan mênde mid þiu \alst{w}ataru \hld\ \alst{w}aldandjo Krist, &
\alst{h}êr \alst{h}evan-kuning \hld\ \alst{h}êlagna gêst, &
hwó þene \alst{f}iriho barn \hld\ ant·\alst{f}áhen skoldin, &
\alst{l}ioht ęndi \alst{l}isti \hld\ ęndi \alst{l}íf êwig, &
\alst{h}ôh \alst{h}evan-ríki \hld\ ęndi \alst{h}uldi godes.\eva

\bvb TODO.\evb\evg

\bvg\bva[48][3926]%
Wurðun þó þea \alst{l}iudi \hld\ umbi þea \alst{l}êra Kristes, &
umbi þiu \alst{w}ord an ge·\alst{w}inne: \hld\ stódun \alst{w}lanka man, &
\alst{g}êl-móde Judeon, \hld\ sprákun \alst{g}elp mikil, &
\alst{h}abdun it im te \alst{h}oska, \hld\ kwáðun þat sie mahtin gi·\alst{h}ôrjen wel, &
þat imu \alst{m}ahlidin fram \hld\ \alst{m}ódaga wihti, &
\alst{u}n-holde \alst{ú}t: \hld\ „nú hé an \alst{a}vu lêrid“, kwáðun sie, &
„\alst{w}ordu ge·hwi-liku.“ \hld\ Þó sprak eft þat \alst{w}erod ȯðar: &
„ni þurvun gí þene \alst{l}êrjand \alst{l}ahan“, \hld[kwáðun sie:] „kumad \alst{l}íves word &
\alst{m}ahtig fan is \alst{m}u̇de; \hld\ hé wirkid \alst{m}anages hwat, &
\alst{w}undres an þesaru \alst{w}er-oldi: \hld\ nis þat \alst{w}rêðaro dád, &
\alst{f}íundo kraftes: \hld\ nio it þan te su·likaru \alst{f}rumu ni wurði, &
ak it \alst{g}egnungo \hld\ fan \alst{g}ode alo-waldon, &
\alst{k}umid fan is \alst{k}rafte. \hld\ Þat mugun gí ant·\alst{k}ęnnjen wel &
an þem is \alst{w}árun \alst{w}ordun, \hld\ þat hé gi·\alst{w}ald havad &
\alst{a}lles ovar \alst{e}rðu.“ \hld\ Þó weldun ina þe \alst{a}nd-sakon þár &
an \alst{st}ędi fáhen \hld\ efþa \alst{st}ên ana werpen, &
ef sie im þero \alst{m}anno \hld\ \alst{m}ęnigi ni an·drédin, &
ni \alst{f}orhtodin þat \alst{f}olk-skępi. \hld\ Þó sprak þat \alst{f}riðu-barn godes: &
„ik tôgju iu \alst{g}ódes só filu“, \hld[kwað hé,] „fan \alst{g}ode selvumu, &
\alst{w}ordo ęndi \alst{w}erko: \hld\ nú willjad gí mí \alst{w}ítnon hér &
þurh iuwan \alst{st}arkan hugi, \hld\ \alst{st}ên ana werpen, &
bi·\alst{l}ôsjen mí \alst{l}ívu.“ \hld\ Þó sprákun imu eft þea \alst{l}iudi an·gęgin, &
\alst{w}rêða \alst{w}iðẹr-sakon: \hld\ „ne wí it be þínun \alst{w}erkun ni duat“, kwáðun sia, &
„þat wí þí \alst{a}ldres \hld\ tó \alst{á}htjen willjad, &
ak wí duat it be þínun \alst{w}ordun, \hld\ hwand þú su·lik \alst{w}áh sprikis, &
*hwand þú þik só \alst{m}áris \hld\ ęndi su·lik \alst{m}ên sagis, &
\alst{g}ihis for þeson \alst{J}udeon, \hld\ þat þú sís \alst{g}od selvo, &
\alst{m}ahtig drohtin, \hld\ ęndi bist þi þoh \alst{m}an só wi, &
\alst{k}uman fan þeson \alst{k}unnje.“ \hld\ \alst{K}rist alo-waldo &
ne wolda þero \alst{J}udeono þuo lęng \hld\ \alst{g}elpes hôrjan, &
\alst{w}rêðaro \alst{w}illjon, \hld\ ak hie im af þem \alst{w}íhe fuor &
ovar \alst{J}ordanes strôm; \hld\ habda \alst{j}ungron mid im, &
þia is \alst{s}áligun gi·\alst{s}ïðos, \hld\ þia im \alst{s}imlon mid im &
\alst{w}illjon \alst{w}onodun: \hld\ suohta \alst{w}erod ȯðer, &
\alst{d}eda þár só hie gi·wonoda, \hld\ \alst{d}rohtin selvo, &
\alst{l}êrda þia \alst{l}iudi: \hld\ gi·\alst{l}ôvda þie wolda &
an is hêlagun word. \hld\ Þat skolda sinnon wel &%NOTE: alliteration is indeed missing.
\alst{m}anno só hwi-likon, \hld\ só þat an is \alst{m}uod gi·nam. &
Þuo gi·frang ik þat þár te \alst{K}riste \hld\ \alst{k}umana wurðun &%NOTE: gi·frang] Checked according to C.
\alst{b}odon fan \alst{B}ethaniu \hld\ ęndi sagdun þem \alst{b}arne godes, &
þat sia an þat \alst{â}rundi þarod \hld\ \alst{i}disi sęndin, &
\alst{M}aria ęndi \alst{M}artha, \hld\ \alst{m}agað frí-líka, &
swíðo \alst{w}un-sama \alst{w}íf; \hld\ þia \alst{w}issa hie bêðja, &
wárun im gi·\alst{s}wester twá, \hld\ þia hie \alst{s}elvo êr &
\alst{m}innjoda an is \alst{m}uode \hld\ þuru iro \alst{m}ildjan hugi, &
þiu \alst{w}íf þuru iro \alst{w}illjon guodan. \hld\ Sia im te \alst{w}áron þuo &
an·\alst{b}udun fon \alst{B}ethaniu, \hld\ þat iro \alst{b}ruoðer was &
\alst{L}azarus \alst{l}egar-fast \hld\ ęndi þat sia is \alst{l}íves ni wándun; &
bádun þat þarod \alst{k}wámi \hld\ \alst{K}rist alo-waldo &
\alst{h}êlag te \alst{h}elpu. \hld\ Reht só hie sia gi·\alst{h}ôrda þuo &
\alst{s}ęggjan fan só \alst{s}iekon, \hld\ só sprak hie \alst{s}án an·gęgin, &
kwað þat \alst{L}azaruses \hld\ \alst{l}egar ni wári &
gi·\alst{d}uan im te \alst{d}ôðe, \hld\ „ak þár skal \alst{d}rohtines lof“, kwaþ-hie, &
„gi·\alst{f}rumid werðan: \hld\ nis it im te ȯðron \alst{f}rêson gi·duan.“ &
was im þár þuo \alst{s}elvo \hld\ \alst{s}uno drohtines &
\alst{t}wá naht ęndi dagas. \hld\ Þiu \alst{t}íd was þuo ge·náhit, &
þat hie eft te \alst{J}erusalem \hld\ \alst{J}udeo liudjo &
\alst{w}íson \alst{w}elda, \hld\ só hie gi·\alst{w}ald habda. &
\alst{S}agda þuo is gi·\alst{s}ïðon \hld\ \alst{s}uno drohtines, &
þat hie eft ovar \alst{J}ordan \hld\ \alst{J}udeo liudi &
\alst{s}uokjan welda. \hld\ Þuo sprákun im \alst{s}án an·gęgin &
\alst{j}ungron sína: \hld\ „te hwí bist þú só \alst{g}ern þarod“, kwaðun sia, &
„\alst{f}rô mín, te \alst{f}aranne? \hld\ Ni þat nú \alst{f}urn ni was, &
þat sia þik þínero \alst{w}ordo \hld\ \alst{w}ítnon hogdun, &
weldun þi mid \alst{st}ênon \alst{st}arkan a·werpan? \hld\ nú þú eft undar þia \alst{st}rídigun þioda &
\alst{f}undos te \alst{f}aranne, \hld\ þár ist \alst{f}íondo gi·nuog, &
\alst{e}rlos \alst{o}var-muoda?“ \hld\ Þuo \alst{ê}n þero twe-livjo, &
\alst{Þ}uomas gi·málda \hld\ —was im gi·\alst{þ}ungan mann, &
\alst{d}iur-lík \alst{d}rohtines þegạn—: \hld\ „ne skulun wí im þia \alst{d}ád lahan“, kwaþ-hie, &
„ni \alst{w}ęrnjan wí im þes \alst{w}illjen, \hld\ ak wita im \alst{w}onjan mid, &%TODO: check "wonjan"
\alst{þ}uolojan mid u̇sson \alst{þ}iodne: \hld\ þat ist \alst{þ}egnes kust, &
þat hie mid is \alst{f}râhon samad \hld\ \alst{f}asto gi·stande, &
\alst{d}ôje mid im þár an \alst{d}uome. \hld\ \alst{D}uan u̇s alla só, &
\alst{f}olgon im te þero \alst{f}ęrdi: \hld\ ni látan u̇se \alst{f}erạh wið þiu &
\alst{w}ihtes \alst{w}irðig, \hld\ neva wí an þem \alst{w}erode mid im, &
\alst{d}ôjan mid u̇son \alst{d}rohtine. \hld\ Þan lêvot u̇s þoh \alst{d}uom after, &
\alst{g}uod word for \alst{g}umon.“ \hld\ Só wurðun þuo \alst{j}ungron Kristes, &
\alst{e}rlos \alst{a}ðal-borana \hld\ an \alst{ê}n-falden hugje, &
\alst{h}êrren te willjen. \hld\ Þuo sagda \alst{h}êlag Krist &
\alst{s}elvo is gi·\alst{s}ïðon \hld\ þat a·\alst{s}lápan was &
\alst{L}azarus fan þem \alst{l}egare, \hld\ „havit þit \alst{l}ioht a·gevan, &
an·\alst{s}wevit ist an \alst{s}elmon. \hld\ Nú wí an þena \alst{s}ïð faran &
ęndi ina a·\alst{w}ękkjan, \hld\ þat hie muoti eft þesa \alst{w}er-old sehan, &
\alst{l}ibbjandi \alst{l}ioht: \hld\ þan wirðit iuwa gi·\alst{l}ôvo after þiu &
\alst{f}orð-werd gi·\alst{f}ęstid.“ \hld\ Þuo gi·wêt hie im ovar þia \alst{f}luod þanan, &
þie \alst{g}uodo \alst{g}odes suno, \hld\ an-þat hie mid is \alst{j}ungron kwam &
þár te \alst{B}ithaniu, \hld\ \alst{b}arn drohtines &
\alst{s}elvo mid is gi·\alst{s}ïðon, \hld\ þár þia gi·\alst{s}wester twá, &
\alst{M}aria ęndi \alst{M}artha \hld\ an \alst{m}uod-karon &
\alst{s}êraga \alst{s}átun. \hld\ Was þár gi·\alst{s}amnot filo &
fan \alst{J}erusalem \hld\ \alst{J}udeo liudo, &
þia þiu *\alst{w}íf \alst{w}eldun \hld\ \alst{w}ordun fruovrjan, &
þat sie só ni \alst{k}arodin \hld\ \alst{k}ind-jungas dôð, &
\alst{L}azaruses far·\alst{l}ust. \hld\ Só þó þe \alst{l}andes ward &
\alst{g}éng an þiu \alst{g}ardos, \hld\ só wurðun þes \alst{g}odes barnes &
\alst{k}umi þár gi·\alst{k}u̇ðid, \hld\ þat hé só \alst{k}raftig was &
bi þeru \alst{b}urg úten. \hld\ Þó im \alst{b}êðjun was, &
þem \alst{w}ívun su·lik \alst{w}illjo, \hld\ þat sie im \alst{w}aldand tó, &
þat \alst{f}riðu-barn godes, \hld\ \alst{f}arandjen wissun.\eva

\bvb TODO.\evb\evg

\bvg\bva[49][4025]%
Þó þem \alst{w}ívun was \hld\ \alst{w}illjono mêsta &
\alst{k}umi drohtines \hld\ ęndi \alst{K}ristes word &
te gi·\alst{h}ôrjenne. \hld\ \alst{H}eovandi géng &
\alst{M}artha \alst{m}ód-karag \hld\ wið só \alst{m}ahtigne &
\alst{w}ordun \alst{w}ehslan \hld\ ęndi wið \alst{w}aldand sprak &
an iro \alst{h}ugi \alst{h}riuwig: \hld\ „Þár þú mí, \alst{h}êrro mín“, kwað siu, &
„\alst{n}ęrjendero bętst, \hld\ \alst{n}áhor wáris, &
\alst{h}êljand þe gódo, \hld\ þan ni þorfti ik nú su·lik \alst{h}arm þolon, &
\alst{b}ittra \alst{b}reost-kara, \hld\ þan ni wári nú mín \alst{b}róðer dôd, &
\alst{L}azarus fan þesumu \alst{l}iohte, \hld\ ak hé imu mahti \alst{l}ibbjen forð &
\alst{f}erạhes ge·\alst{f}ullid. \hld\ Ik þoh, \alst{f}rô mín, te þí &
\alst{l}iohto gi·\alst{l}ôvju, \hld\ \alst{l}êrjandero bętst, &
só hwes só þú \alst{b}iddjen wili \hld\ \alst{b}erhton drohtin, &
þat hé it þi sán far·\alst{g}ivid, \hld\ \alst{g}od alo-mahtig, &
gi·\alst{w}erðot þínan \alst{w}illjan.“ \hld\ Þó sprak eft \alst{w}aldand Krist &
þeru \alst{i}dis \alst{a}nd-wordi: \hld\ „Ni lát þú þí an \alst{i}nnan þes“, kwað hé, &
„þínan \alst{s}evon \alst{s}werkan: \hld\ ik þí \alst{s}ęggjan mag &
\alst{w}árun \alst{w}ordun, \hld\ þat þes nis gi·\alst{w}and ênig, &
nevu þín \alst{b}róðer skal \hld\ þurh gi·\alst{b}od godes, &
þurh \alst{d}rohtines kraft \hld\ fan \alst{d}ôðe a·standen &
an is \alst{l}ík-hamon.“ \hld\ „All hębbju ik gi·\alst{l}ôvon só“, kwað siu, &
„þat it só gi·\alst{w}erðen skal, \hld\ só hwan só þius \alst{w}er-old ęndjod &
ęndi þe \alst{m}árjo dag \hld\ ovar \alst{m}an fęrid, &
þat hé þan fan \alst{e}rðu skal \hld\ \alst{u}p a·standen &
an þemu \alst{d}ómes \alst{d}aga, \hld\ þan werðad fan \alst{d}ôðe kwika &
þurh \alst{m}aht godes \hld\ \alst{m}an-kunnjes ge·hwi-lik, &
a·\alst{r}ísad fan \alst{r}estu.“ \hld\ Þó sagde \alst{r}íkjo Krist &
þeru \alst{i}dis \alst{a}lo-mahtig \hld\ \alst{o}ponun wordun, &
þat hé \alst{s}elvo was \hld\ \alst{s}unu drohtines, &
bêðju ia \alst{l}íf ia \alst{l}ioht \hld\ \alst{l}iudjo barnon &
te a·\alst{st}andanne: \hld\ „nio þe \alst{st}erven ni skal, &
\alst{l}íf far·\alst{l}iosen, \hld\ þe hér gi·\alst{l}ôvid te mí: &
þoh ina \alst{ę}ldi-barn \hld\ \alst{e}rðu bi·þękkjen, &
\alst{d}iapo bi·\alst{d}elven, \hld\ nis hé \alst{d}ôd þiu mêr: &
þat \alst{f}lêsk is bi·\alst{f}olhen, \hld\ þat \alst{f}erạh is gi·halden, &
is þiu \alst{s}iola gi·\alst{s}und.“ \hld\ Þó sprak imu eft \alst{s}án an·gęgin &
þat \alst{w}íf mid iro \alst{w}ordun: \hld\ „ik gi·lôvju þat þú þe \alst{w}áro bist“, kwað siu, &
„\alst{K}rist godes sunu: \hld\ þat mag man ant·\alst{k}ęnnjen wel, &
\alst{w}iten an þínun \alst{w}ordun, \hld\ þat þú gi·\alst{w}ald haves &
þurh þiu \alst{h}êlagon gi·skapu \hld\ \alst{h}imiles ęndi erðun.“ &
Þó ge·fragn ik þat þár þero \alst{i}disjo kwam \hld\ \alst{ȯ}ðar gangan &
\alst{M}aria \alst{m}ód-karag: \hld\ géngun iro \alst{m}anaga aftar &
\alst{J}udeo liudi. \hld\ Þó siu þemu \alst{g}odes barne &
\alst{s}agde \alst{s}êrag-mód, \hld\ hwat iru te \alst{s}orgun gi·stód &
an iro \alst{h}ugi \alst{h}armes: \hld\ \alst{h}ofnu kúmde &
\alst{L}azaruses far·\alst{l}ust, \hld\ \alst{l}iaves mannes, &
\alst{g}riat \alst{g}ornundi, \hld\ ant-tat þemu \alst{g}odes barne &
\alst{h}ugi warð gi·\alst{h}rórid: \hld\ \alst{h}ête trahni &
\alst{w}ópu a·\alst{w}ellun, \hld\ ęndi þó te þem \alst{w}ívun sprak, &
hét ina þó \alst{l}êdjen, \hld\ þár \alst{L}azarus was &
\alst{f}oldu bi·\alst{f}olhen. \hld\ Lag þár ên \alst{f}elis bi·ovan, &
\alst{h}ard stên be·\alst{h}liden. \hld\ Þó hét þe \alst{h}êlago Krist &
ant·\alst{l}úkan þea \alst{l}éia, \hld\ þat hé mósti þat \alst{l}ík sehan, &
\alst{h}rêo skawojen. \hld\ Þó ni mahte an iro \alst{h}ugi míðan &
\alst{M}arþa for þeru \alst{m}ęnegi, \hld\ wið \alst{m}ahtigne sprak: &
„\alst{f}rô mín þe gódo“, \hld[kwað siu,] „ef man þene \alst{f}elis nimid, &
þene \alst{st}ên ant·lúkid, \hld\ þan wániu ik þat þanen \alst{st}ank kume, &
un·\alst{s}wóti \alst{s}wek, \hld\ hwand ik þi \alst{s}ęggjan mag &
\alst{w}árun \alst{w}ordun, \hld\ þat þes nis gi·\alst{w}and ênig, &
þat hé þár nú bi·\alst{f}olhen was \hld\ \alst{f}iuwar naht ęndi dagos &
an þemu \alst{e}rð-grave.“ \hld\ \alst{A}nd-wordi gaf &
\alst{w}aldand þemu \alst{w}íve: \hld\ „Hwat ni sagde ik þí te \alst{w}árun êr“, kwað hé, &
„ef þú gi·\alst{l}ôvjen wili, \hld\ þan nis nú \alst{l}ang te þiu, &
þat þú hér ant·\alst{k}ęnnjen skalt \hld\ \alst{k}raft drohtines, &
þe \alst{m}ikilon \alst{m}aht godes?“ \hld\ Þó géngun \alst{m}anage tó, &
af·\alst{h}óvun \alst{h}arden stên. \hld\ Þó sah þe \alst{h}êlago Krist &
\alst{u}p mid is \alst{ô}gun, \hld\ \alst{á}-lát sagde &
þemu þe þese \alst{w}er-old gi·skóp, \hld\ „þes þú mín \alst{w}ord gi·hôris“, kwað hé, &
„\alst{s}igi-drohtin \alst{s}elvo; \hld\ ik wêt þat þú só \alst{s}imlun duos, &
ak ik duom it be þesumu \alst{g}rôton \hld\ \alst{J}udeono folke, &
þat sie þat te \alst{w}árun \alst{w}itin, \hld\ þat þú mí an þese \alst{w}er-old sęndes &
þesun \alst{l}iudjun te \alst{l}êrun.“ \hld\ Þó hé te \alst{L}azaruse hriop &
\alst{st}arkaru \alst{st}emnju \hld\ ęndi hét ina \alst{st}anden up &
ia fan þemu \alst{g}rave \alst{g}angan. \hld\ Þó warð þe \alst{g}êst kumen &
an þene \alst{l}ík-hamon: \hld\ hé bi·gan is \alst{l}iði hrórjen, &
ant·\alst{w}arp undar þemu gi·\alst{w}ę́dje: \hld\ was imo só be·\alst{w}unden þó noh, &
an \alst{h}rêo-będdjon bi·\alst{h}elid. \hld\ Hét imu \alst{h}elpen þó &
\alst{w}aldandjo Krist. \hld\ \alst{W}eros géngun tó, &
ant·\alst{w}undun þat ge·\alst{w}ádi. \hld\ \alst{W}ánum up a·rês &
\alst{L}azarus te þesumu \alst{l}iohte: \hld\ was imu is \alst{l}íf far·geven, &
þat hé is \alst{a}ldạr-lagu \hld\ \alst{ê}gan mósti, &
\alst{f}riðu \alst{f}orð-wardes. \hld\ Þó \alst{f}agonadun bêðja, &
\alst{M}aria ęndi \alst{M}artha: \hld\ ni mag þat \alst{m}an ȯðrumu &
gi·\alst{s}ęggjan te \alst{s}ȯðe, \hld\ hwó þea ge·\alst{s}wester twó &
\alst{m}ęndjodun an iro \alst{m}óde. \hld\ \alst{M}aneg wundrode &
\alst{J}udeo liudjo, \hld\ þó sie ina fan þemu \alst{g}rave sáhun &
\alst{s}ïðon ge·\alst{s}unden, \hld\ þene þe êr \alst{s}uht far·nam &
ęndi sie bi·\alst{d}ulvun \hld\ \alst{d}iapo undar erðu &
\alst{l}íves \alst{l}ôsen: \hld\ þó móste imu \alst{l}ibbjen forð &
\alst{h}êl an \alst{h}êmun. \hld\ Só mag \alst{h}evan-kuninges, &
þiu \alst{m}ikile \alst{m}aht godes \hld\ \alst{m}anno ge·hwi-likes &
\alst{f}erạhe gi·\alst{f}ormon \hld\ ęndi wið \alst{f}íundo níð &
\alst{h}êlag \alst{h}elpen, \hld\ só hwemu só hé is \alst{h}uldi far·givid.\eva

\bvb TODO.\evb\evg

\bvg\bva[50][4118]%
Þó warð þár só \alst{m}anagumu \alst{m}anne \hld\ \alst{m}ód aftar Kriste, &
gi·\alst{h}worven \alst{h}ugi-skęfti, \hld\ sïðor sie is \alst{h}êlagon werk &
\alst{s}elvon gi·\alst{s}áhun, \hld\ hwand eo êr \alst{s}u·lik ni warð &
\alst{w}undẹr an \alst{w}er-oldi. \hld\ Þan was eft þes \alst{w}erodes só filu, &
só \alst{m}ód-starke \alst{m}an: \hld\ ni weldon þe \alst{m}aht godes &
ant·\alst{k}ęnnjen \alst{k}u̇ð-líko, \hld\ ak sie wið is \alst{k}raft mikil &
\alst{w}unnun mid iro \alst{w}ordun: \hld\ \alst{w}árun im waldandes &
\alst{l}êra so \alst{l}êða: \hld\ sóhtun im \alst{l}iudi ȯðra &
an \alst{J}erusalem, \hld\ þár \alst{J}udeono was &
\alst{h}êri \alst{h}and-mahạl \hld\ ęndi \alst{h}ôvid-stędi, &
\alst{g}rôt \alst{g}um-skępi \hld\ \alst{g}rimmaro þioda. &
Sie \alst{k}u̇ðdun im þó \alst{K}ristes werk, \hld\ kwáðun þat sie \alst{k}wikan sáhin &
þene \alst{e}rl mid iro \alst{ô}gun, \hld\ þe an \alst{e}rðu was, &
\alst{f}oldu bi·\alst{f}olhen \hld\ \alst{f}iuwar naht ęndi dagos, &
\alst{d}ôd bi·\alst{d}olven, \hld\ ant-tat hé ina mid is \alst{d}ádjun selvo, &
mid is wordun a·\alst{w}ękide, \hld\ þat hé mósti þese \alst{w}er-old sehan. &
Þó was þat só \alst{w}iðẹr-\alst{w}ard \hld\ \alst{w}lankun mannun, &
\alst{J}udeo liudjun: \hld\ hétun iro \alst{g}um-skępi þó, &
\alst{w}erod samnojan \hld\ ęndi \alst{w}arvos fáhen, &
\alst{m}ęgin-þioda gi·\alst{m}ang, \hld\ an \alst{m}ahtigna Krist &
\alst{r}iedun an \alst{r}únun: \hld\ „nis þat \alst{r}ád ênig“, kwáðun sie, &
„þat wí þat gi·\alst{þ}olojan: \hld\ wili þesaro \alst{þ}ioda te filu &
gi·\alst{l}ôvjen aftar is \alst{l}êrun. \hld\ Þan u̇s \alst{l}iudi farad, &
an \alst{e}o-rid-folk, \hld\ werðat u̇sa \alst{o}var-hôvdun &
\alst{r}inkos fan \alst{R}úmu. \hld\ Þan wí þeses \alst{r}íkjes skulun &
\alst{l}ôse \alst{l}ibbjen \hld\ efþa wí skulun u̇ses \alst{l}íves þolon, &
\alst{h}ęliðos u̇saro \alst{h}ôvdo.“ \hld\ Þó sprak þár ên gi·\alst{h}êrod man &
ovar \alst{w}arf \alst{w}ero, \hld\ þe was þes \alst{w}erodes þó &
an þeru \alst{b}urg innan \hld\ \alst{b}iskop þero liudjo &
—\alst{K}aiphas was hé hêten; \hld\ habdun ina gi·\alst{k}oranen te þiu &
an þeru \alst{g}ę́r-talu \hld\ \alst{J}udeo liudi, &
þat hé þes \alst{g}odes húses \hld\ \alst{g}ômjen skoldi, &
\alst{w}ardon þes \alst{w}íhes—: \hld\ „Mí þunkid \alst{w}undẹr mikil“, kwað hé, &
„\alst{m}ári þioda, \hld\ —gí kunnun \alst{m}anages gi·skêð— &
hwí gí þat te \alst{w}árun ni \alst{w}itin, \hld\ \alst{w}erod Judeono, &
þat hér is \alst{b}ętera rád \hld\ \alst{b}arno ge·hwi-likumu, &
þat man hér \alst{ê}nne man \hld\ \alst{a}ldru bi·lôsje &
ęndi þat hé þurh iuwa \alst{d}ádi \hld\ \alst{d}rôreg sterve, &
for þesumu \alst{f}olk-skępi \hld\ \alst{f}erạh far·láte, &
þan al þit \alst{l}iud-werod \hld\ far·\alst{l}oren werðe.“ &
Ni was it þoh is \alst{w}illjan, \hld\ þat hé só \alst{w}ár ge·sprak, &
só \alst{f}orð for þemu \alst{f}olke, \hld\ \alst{f}rume man-kunnjes &
gi·\alst{m}ênde for þeru \alst{m}ęnegi, \hld\ ak it kwam imu fan þeru \alst{m}aht godes &
þurh is \alst{h}êlagan \alst{h}êd, \hld\ hwand hé þat \alst{h}ús godes &
þár an \alst{J}erusalem \hld\ bi·\alst{g}angan skolde, &
\alst{w}ardon þes \alst{w}íhes: \hld\ be·þiu hé só \alst{w}ár gi·sprak, &
\alst{b}iskop þero liudjo, \hld\ hwó skoldi þat \alst{b}arn godes &
\alst{a}lla \alst{i}rmin-þiod \hld\ mid is \alst{ê}nes ferhe, &
mid is \alst{l}ívu a·\alst{l}ôsjen: \hld\ þat was allaro þesaro \alst{l}iudjo rád, &
\alst{h}wand hé gi·\alst{h}alode \hld\ mid þiu \alst{h}êðina liudi, &
\alst{w}eros an is \alst{w}illjon \hld\ \alst{w}aldandjo Krist. &
Þó wurðun \alst{ê}n-wordje \hld\ \alst{o}var-módje man, &
\alst{w}erod Judeono, \hld\ ęndi an iro \alst{w}arve gi·sprákun, &
\alst{m}ári þioda, \hld\ þat sie im ni létin iro \alst{m}ód twehon: &
só hwe só ina undar þemu \alst{f}olke \hld\ \alst{f}inden mahti, &
þat ina sán gi·\alst{f}éngi \hld\ ęndi \alst{f}orð brȧhti &
an þero \alst{þ}iodo \alst{þ}ing; \hld\ kwáðun þat sie ni mahtin gi·\alst{þ}olojan lęng, &
þat sie þe \alst{ê}no man \hld\ só \alst{a}lla weldi, &
\alst{w}erod far·winnen. \hld\ Þan wisse \alst{w}aldand Krist &
þero \alst{m}anno só garo \hld\ \alst{m}ód-gi·þȧhti, &
\alst{h}ęti-grimmon \alst{h}ugi, \hld\ hwand imu ni was bi·\alst{h}olen eo·wiht &
an þesaru \alst{m}iddil-gard: \hld\ hé ni welde þó an þie \alst{m}ęnigi innen &
sïður \alst{o}pen-líko, \hld\ under þat \alst{e}rlo folk, &
\alst{g}angan under þea \alst{J}udeon: \hld\ bêd þe \alst{g}odes sunu &
þero \alst{t}orọhtjon \alst{t}íd, \hld\ þe imu \alst{t}ó-ward was, &
þat hé far \alst{þ}esa \alst{þ}ioda \hld\ \alst{þ}olojan welde, &
far þit \alst{w}erod \alst{w}íti: \hld\ \alst{w}isse imu selvo &
þat \alst{d}ag-þingi garo. \hld\ Þó gi·wêt imu u̇se \alst{d}rohtin forð &
ęndi imu þó an \alst{E}ffrem \hld\ \alst{a}lo-waldo Krist &
an þeru \alst{h}ôhon burg \hld\ \alst{h}êlag drohtin &
\alst{w}unode mid is \alst{w}erodu, \hld\ ant-tat hé an is \alst{w}illjan hwarf &
eft te \alst{B}ethania \hld\ \alst{b}rahtmu þiu mikilun, &
mid þiu is \alst{g}ódum \alst{g}um-skępi. \hld\ \alst{J}udeon bi·sprákun þat &
\alst{w}ordu ge·hwi-liku, \hld\ þó sie imu su·lik \alst{w}erod mikil &
\alst{f}olgon gi·sáhun: \hld\ „nis \alst{f}rume ênig“, kwáðun sie, &
„u̇ses \alst{r}íkjes gi·\alst{r}ádi, \hld\ þoh wí \alst{r}eht sprekan, &
ni \alst{þ}íhit u̇ses \alst{þ}inges wiht: \hld\ þius \alst{þ}iod wili &
\alst{w}ęndjen after is \alst{w}illjan; \hld\ imu all þius \alst{w}er-old folgot, &
\alst{l}iudi bi þem is \alst{l}êrun, \hld\ þat wí imu \alst{l}êðes wiht &
for þesumu \alst{f}olk-skępi \hld\ gi·\alst{f}rummjen ni mótun.“\eva

\bvb TODO.\evb\evg

\bvg\bva[51][4198]%
Gi·wêt imu þó þat \alst{b}arn godes \hld\ innan \alst{B}ethania &
\alst{s}ehs nahtun êr, \hld\ þan þiu \alst{s}amnunga &
þár an \alst{J}erusalem \hld\ \alst{J}udeo liudjo &
an þem \alst{w}íh-dagun \hld\ \alst{w}erðen skolde, &
þat sie skoldun \alst{h}aldan \hld\ þea \alst{h}êlagon tídi, &
\alst{J}udeono paskha. \hld\ Béd þe \alst{g}odes sunu, &
\alst{m}ahtig under þeru \alst{m}ęnegi: \hld\ was þár \alst{m}anno kraft, &
\alst{w}erodes bi þem is \alst{w}ordun. \hld\ Þár géngun ina twê \alst{w}íf umbi, &
\alst{M}aria ęndi \alst{M}artha, \hld\ mid \alst{m}ildju hugi, &
\alst{þ}ionodun imu \alst{þ}eo-líko. \hld\ \alst{Þ}iodo drohtin &
gaf im \alst{l}ang-sam \alst{l}ôn: \hld\ lét sea \alst{l}êðes gi·hwes, &
\alst{s}undjono \alst{s}ikora, \hld\ ęndi \alst{s}elvo gi·bôd, &
þat sea an \alst{f}riðe \alst{f}órin \hld\ wiðẹr \alst{f}íundo níð, &
þea \alst{i}disa mid is \alst{o}rlovu gódu: \hld\ habdun iro \alst{a}mbaht-skępi &
bi·\alst{w}ęndid an is \alst{w}illjon. \hld\ Þó gi·wêt imu \alst{w}aldand Krist &
\alst{f}orð mid þiu \alst{f}olku, \hld\ \alst{f}iriho drohtin, &
innan \alst{J}erusalem, \hld\ þár \alst{J}udeono was &
\alst{h}ęte-lík \alst{h}ard-buri, \hld\ þár sie þea \alst{h}êlagon tíd &
\alst{w}arodun at þemu \alst{w}íhe; \hld\ was þár \alst{w}erodes só filu, &
\alst{k}raftigaro \alst{k}unnjo, \hld\ þie ni weldun \alst{K}ristes word &
\alst{g}erno hôrjen \hld\ ni te þemu \alst{g}odes barne &
an iro \alst{m}ód-sevon \hld\ \alst{m}innje ni habdun, &
ak \alst{w}árun im só \alst{w}rêða \hld\ \alst{w}lanka þioda, &
\alst{m}ódeg \alst{m}an-kunni, \hld\ habdun im \alst{m}orð-hugi, &
\alst{i}n-wid an \alst{i}nnan: \hld\ an \alst{a}vuh far·féngun &
\alst{K}ristes lêre, \hld\ weldun ina \alst{k}raftigna &
\alst{w}ítnon þero \alst{w}ordo; \hld\ ak was þár \alst{w}erodes só filu, &
umbi \alst{e}rl-skępi \hld\ \alst{a}nt-langana dag, &
habde ine þiu \alst{s}male þiod \hld\ þurh is \alst{s}wótjun word &
\alst{w}erodu bi·\alst{w}orpen, \hld\ þat ine þie \alst{w}iðẹr-sakon &
under þemu \alst{f}olk-skępi \hld\ \alst{f}áhen ne gi·dorstun, &
ak \alst{m}iðun is bi þeru \alst{m}ęnegi. \hld\ Þan stód \alst{m}ahtig Krist &
an þemu \alst{w}íhe innan, \hld\ sagde \alst{w}ord manag &
\alst{f}iriho barnun te \alst{f}rumu. \hld\ Was þár \alst{f}olk umbi &
allan \alst{l}angan dag, \hld\ ant-tat þiu \alst{l}iohte gi·wêt &
\alst{s}unne te \alst{s}edle. \hld\ Þó te \alst{s}ęliðun fór &
\alst{m}an-kunnjes \alst{m}anag. \hld\ Þan was þár ên \alst{m}ári berg &
bi þeru \alst{b}urg úten, \hld\ þe was \alst{b}rêd ęndi hôh, &
\alst{g}róni ęndi skôni: \hld\ hétun ina \alst{J}udeo liudi &
\alst{O}liueti bi namon. \hld\ Þár imu \alst{u}p gi·wêt &
\alst{n}ęrjendjo Krist, \hld\ só ina þiu \alst{n}aht bi·féng, &
was imu þár mid is \alst{j}ungarun, \hld\ só ine þár \alst{J}udeono ênig &
ni \alst{w}isse ti \alst{w}árun, \hld\ hwand hé an þemu \alst{w}íhe stód, &
\alst{l}iudjo drohtin, \hld\ só \alst{l}ioht ôstene kwam, &
ant·\alst{f}éng þat \alst{f}olk-skępi \hld\ ęndi im \alst{f}ilu sagde &
\alst{w}ároro \alst{w}ordo, \hld\ só nis an þesaru \alst{w}er-oldi ênig, &
an þesaru \alst{m}iddil-gard \hld\ \alst{m}anno só spáhi, &
\alst{l}iudjo barno nig·ên, \hld\ þat þero \alst{l}êrono mugi &
\alst{ę}ndi gi·tęlljen, \hld\ þe hé þár an þemu \alst{a}lạhe gi·sprak, &
\alst{w}aldand an þemu \alst{w}íhe, \hld\ ęndi simlun mid is \alst{w}ordun gi·bôd, &
þat sie sie \alst{g}ęrẹwidin \hld\ te \alst{g}odes ríkje, &
allaro \alst{m}anno ge·hwi-lik, \hld\ þat sie móstin an þemu \alst{m}árjon daga &
iro \alst{d}rohtines \hld\ \alst{d}iuriða ant·fáhen. &
Sagde im hwat sie it \alst{s}undjun frumidun \hld\ ęndi \alst{s}imlun gi·bôd, &
þat sie þea a·\alst{l}ęskidin; \hld\ hét sie \alst{l}ioht godes &
\alst{m}innjon an iro \alst{m}óde, \hld\ \alst{m}ên far·láten, &
\alst{a}voha \alst{o}var-hugdi, \hld\ \alst{ô}d-módi niman, &
\alst{h}laðen þat an iro \alst{h}ertan; \hld\ kwað þat im þan wári \alst{h}evan-ríki, &
\alst{g}aru \alst{g}ódo mêst. \hld\ Þó warð þár \alst{g}umono só filu &
gi·\alst{w}ęndid aftar is \alst{w}illjon, \hld\ sïður sie þat \alst{w}ord godes &
\alst{h}êlag gi·\alst{h}ôrdun, \hld\ \alst{h}evan-kuninges, &
ant·\alst{k}ęndun \alst{k}raft mikil, \hld\ \alst{k}umi drohtines, &
\alst{h}êrron \alst{h}elpe, \hld\ ia þat \alst{h}evan-ríki was, &
\alst{n}ęrjendi gi·\alst{n}áhid \hld\ ęndi \alst{n}áða godes &
\alst{m}anno barnun. \hld\ Sum só \alst{m}ódeg was &
\alst{J}udeo folkes, \hld\ habdun \alst{g}rimman hugi, &
\alst{s}líð-móden \alst{s}evon \hld\ {[...]}, &
ni weldun is \alst{w}orde gi·lôvjen, \hld\ ak habdun im ge·\alst{w}in mikil &
wið þea \alst{K}ristes kraft: \hld\ \alst{k}umen ni móstun &
þea \alst{l}iudi þurh \alst{l}êðen stríd, \hld\ þat sie gi·\alst{l}ôvon te imu &
\alst{f}asto gi·\alst{f}éngin; \hld\ ni was im þiu \alst{f}rume giviðig, &
þat sie \alst{h}evan-ríki \hld\ \alst{h}abbjen móstin. &
\alst{G}éng imu þó þe \alst{g}odes sunu \hld\ ęndi is \alst{j}ungaron mid imu, &
\alst{w}aldand fan þemu \alst{w}íhe, \hld\ all só is \alst{w}illjo géng, &
iak imu uppen þene \alst{b}erg gi·stêg \hld\ \alst{b}arn drohtines: &
\alst{s}at imu þár mid is ge·\alst{s}ïðun \hld\ ęndi im \alst{s}agde filu &
\alst{w}ároro \alst{w}ordo. \hld\ Sí bi·gunnun im þó umbi þene \alst{w}íh sprekan, &
þie \alst{g}umon umbi þat \alst{g}odes hús, \hld\ kwáðun þat ni wári \alst{g}ód-líkora &
\alst{a}lạh ovar \alst{e}rðu \hld\ þurh \alst{e}rlo hand, &
þurh \alst{m}annes gi·werk \hld\ mid \alst{m}ęgin-kraftu &
\alst{r}akud a·\alst{r}ihtid. \hld\ Þó þe \alst{r}íkjo sprak, &
\alst{h}êr \alst{h}evan-kuning \hld\ —\alst{h}ôrdun þe ȯðra—: &
„ik mag iu gi·\alst{t}ęlljen“, \hld[kwað hé,] „þat noh wirðid þiu \alst{t}íd kumen, &
þat is af·\alst{st}anden ni skal \hld\ \alst{st}ên ovar ȯðrumu, &
ak it \alst{f}allid ti \alst{f}oldu \hld\ ęndi \alst{f}iur nimid, &
\alst{g}rádag logna, \hld\ þoh it nú só \alst{g}ód-lík sí, &
só \alst{w}ís-líko gi·\alst{w}arht, \hld\ ęndi só dód all þesaro \alst{w}er-oldes gi·skapu, &
te·\alst{g}lídid \alst{g}róni wang.“ \hld\ Þó géngun imu is \alst{j}ungaron tó, &
frágodun ina só \alst{st}illo: \hld\ „hwó lango skal \alst{st}anden noh“, kwáðun sie, &
„þius \alst{w}er-old an \alst{w}unnjun, \hld\ êr þan þat gi·\alst{w}and kume, &
þat þe \alst{l}asto dag \hld\ \alst{l}iohtes skíne &
þurh \alst{w}olkạn-skion, \hld\ efþo hwan is þín eft \alst{w}án kumen &
an þene \alst{m}iddil-gard, \hld\ \alst{m}anno kunnje &
te a·\alst{d}êljenne, \hld\ \alst{d}ôdun ęndi kwikun? &
\alst{f}rô mín þe gódo, \hld\ u̇s is þes \alst{f}iri-wit mikil, &
\alst{w}aldandjo Krist, \hld\ hwan þat gi·\alst{w}erðen skuli.“\eva

\bvb TODO.\evb\evg

\bvg\bva[52][4294]%
Þó im \alst{a}nd-wordi \hld\ \alst{a}lo-waldo Krist &
\alst{g}ód-lík far·\alst{g}af \hld\ þem \alst{g}umun selvo: &
„þat havad só bi·\alst{d}ęrnid“, \hld[kwað hé,] „\alst{d}rohtin þe gódo, &
iak só \alst{h}ardo far·\alst{h}olen \hld\ \alst{h}imil-ríkjes fader, &
\alst{w}aldand þesaro \alst{w}er-oldes, \hld\ só þat \alst{w}iten ni mag &
ênig \alst{m}annisk barn, \hld\ hwan þiu \alst{m}árje tíd &
gi·\alst{w}irðid an þesaru \alst{w}er-oldi, \hld\ ne it ôk te \alst{w}áran ni kunnun &
\alst{g}odes ęngilos, \hld\ þie for imu \alst{g}ęgin-warde &
\alst{s}imlun \alst{s}indun: \hld\ sie it ôk gi·\alst{s}ęggjan ni mugun &
te \alst{w}áran mid iro \alst{w}ordun, \hld\ hwan þat gi·\alst{w}erðen skuli, &
þat hé willje an þesan \alst{m}iddil-gard, \hld\ \alst{m}ahtig drohtin, &
\alst{f}iriho \alst{f}andon. \hld\ \alst{F}ader wêt it êno &
\alst{h}êlag fan \alst{h}imile: \hld\ elkur is it bi·\alst{h}olen allun, &
\alst{k}wikun ęndi dôdun, \hld\ hwan is \alst{k}umi werðad. &
Ik mag iu þoh gi·\alst{t}ęlljen, \hld\ hwi-lik hér \alst{t}êkạn bi·foran &
gi·\alst{w}erðad \alst{w}undẹr-lík, \hld\ êr þan hé an þese \alst{w}er-old kume &
an þemu \alst{m}árjon daga: \hld\ þat wirðid hér êr an þemu \alst{m}ánon skín &
iak an þeru \alst{s}unnon só \alst{s}ame; \hld\ gi·\alst{s}werkad siu bêðju, &
mid \alst{f}inistre werðad bi·\alst{f}angan; \hld\ \alst{f}allad sterron, &
\alst{h}wít \alst{h}evan-tungạl, \hld\ ęndi \alst{h}risid erðe, &
\alst{b}ivod þius \alst{b}rêde wer-old \hld\ —wirðid su·likaro \alst{b}ôkno filu—: &
\alst{g}rimmid þe \alst{g}rôto sêo, \hld\ wirkid þie \alst{g}evenes strôm &
\alst{ę}gison mid is \alst{u̇}ðjun \hld\ \alst{e}rð-búandjun. &
Þan \alst{þ}orrot þiu \alst{þ}iod \hld\ þurh þat ge·\alst{þ}wing mikil, &
\alst{f}olk þurh þea \alst{f}orhta: \hld\ þan nis \alst{f}riðu hwęrgin, &
ak \alst{w}irðid \alst{w}íg só maneg \hld\ ovar þese \alst{w}er-old alla &
\alst{h}ęte-lík af·\alst{h}aben, \hld\ ęndi \alst{h}ęri lêdid &
\alst{k}unni ovar ȯðar: \hld\ wirðid \alst{k}uningo gi·win, &
\alst{m}ęgin-fard \alst{m}ikil: \hld\ wirðid \alst{m}anagoro kwalm, &
\alst{o}pen \alst{u}r-lagi \hld\ —þat is \alst{ę}gis-lík þing, &
þat io su·lik \alst{m}orð \hld\ skulun \alst{m}an af·hębbjen—, &
\alst{w}irðid \alst{w}ól só mikil \hld\ ovar þese \alst{w}er-old alle, &
\alst{m}an-stervono \alst{m}êst, \hld\ þero þe gio an þesaru \alst{m}iddil-gard &
\alst{s}wulti þurh \alst{s}uhti: \hld\ liggjad \alst{s}eoka man, &
\alst{d}riosat ęndi \alst{d}ôjat \hld\ ęndi iro \alst{d}ag ęndjad, &
\alst{f}ulljad mid iro \alst{f}erạhu; \hld\ \alst{f}ęrid un·met grôt &
\alst{h}ungạr \alst{h}ęti-grim \hld\ ovar \alst{h}ęliðo barn, &
\alst{m}ęti-gêdjono \alst{m}êst: \hld\ nis þat \alst{m}inniste &
þero \alst{w}ítjo an þesaru \alst{w}er-oldi, \hld\ þe hér gi·\alst{w}erðen skulun &
êr \alst{d}ómes \alst{d}age. \hld\ Só hwan só gí þea \alst{d}ádi gi·sehan &
gi·\alst{w}erðen an þesaru \alst{w}er-oldi, \hld\ só mugun gí þan te \alst{w}áran far·standen, &
þat þan þe \alst{l}atsto dag \hld\ \alst{l}iudjun náhid &
\alst{m}ári te \alst{m}annun \hld\ ęndi \alst{m}aht godes, &
\alst{h}imil-kraftes \alst{h}róri \hld\ ęndi þes \alst{h}êlagon kumi, &
\alst{d}rohtines mid is \alst{d}iuriðun. \hld\ Hwat gí þesaro \alst{d}ádjo mugun &
bi þesun \alst{b}ômun \hld\ \alst{b}iliði ant·kęnnjen: &
þan sie \alst{b}rustjad ęndi \alst{b}lójat \hld\ ęndi \alst{b}ladu tôgjat, &
\alst{l}ôf ant·\alst{l}úkad, \hld\ þan witun \alst{l}iudjo barn, &
þat þan is \alst{s}án after þiu \hld\ \alst{s}umer gi·náhid &
\alst{w}arm ęndi \alst{w}un-sam \hld\ ęndi \alst{w}edẹr skôni. &
Só witin gí ôk bi þesun \alst{t}êknun, \hld\ þe ik iu \alst{t}alde hér, &
hwan þe \alst{l}atsto dag \hld\ \alst{l}iudjun náhid. &
Þan sęggjo ik iu te \alst{w}áran, \hld\ þat êr þit \alst{w}erod ni mót, &
te·\alst{f}aran þit \alst{f}olk-skępi, \hld\ êr þan werðe ge·\alst{f}ullid só, &
mínu \alst{w}ord gi·\alst{w}árod. \hld\ Noh gi·\alst{w}and kumid &
\alst{h}imiles ęndi erðun, \hld\ ęndi stéid mín \alst{h}êlag word &
\alst{f}ast \alst{f}orð-wardes \hld\ ęndi wirðid al ge·\alst{f}ullod só, &
gi·\alst{l}êstid an þesumu \alst{l}iohte, \hld\ só ik for þesun \alst{l}iudjun ge·spriku. &
\alst{w}akot gí \alst{w}ar-líko: \hld\ iu is \alst{w}is-kumo &
\alst{d}uom-\alst{d}ag þe márjo \hld\ ęndi iuwes \alst{d}rohtines kraft, &
þiu \alst{m}ikilo \alst{m}ęgin-strengi \hld\ ęndi þiu \alst{m}árje tíd, &
gi·\alst{w}and þesaro \alst{w}er-oldes. \hld\ Fora þiu gí \alst{w}ardon skulun, &
þat hé iu \alst{s}lápandje \hld\ an \alst{s}wef-restu &
\alst{f}árungo ni bi·\alst{f}ȧhe \hld\ an \alst{f}irin-werkun, &
\alst{m}ênes fulle. \hld\ \alst{M}út-spelli kumit &
an \alst{þ}iustrja naht, \hld\ al só \alst{þ}iof fęrid &
\alst{d}arno mid is \alst{d}ádjun, \hld\ só kumid þe \alst{d}ag mannun, &
þe \alst{l}atsto þeses \alst{l}iohtes, \hld\ só it êr þese \alst{l}iudi ni witun, &
só samo só þiu \alst{f}lód deda \hld\ an \alst{f}urn-dagun, &
þe þár mid \alst{l}agu-strômun \hld\ \alst{l}iudi far·tęride &
bi \alst{N}óeas tídjun, \hld\ bi·útan þat ina \alst{n}ęride god &
mid is \alst{h}íwiskja, \hld\ \alst{h}êlag drohtin, &
wið þes \alst{f}lódes \alst{f}arm: \hld\ só warð ôk þat \alst{f}iur kuman &
\alst{h}êt fan \alst{h}imile, \hld\ þat þea \alst{h}ôhon burgi &
umbi \alst{S}odomo land \hld\ \alst{s}wart logna bi·féng &
\alst{g}rim ęndi \alst{g}rádag, \hld\ þat þár n·ênig \alst{g}umono ni gi·nas &
bi·útan \alst{L}oth êno: \hld\ ina ant·\alst{l}êddun þanen &
\alst{d}rohtines ęngilos \hld\ ęndi is \alst{d}ohter twá &
an ênan \alst{b}erg uppen: \hld\ þat ȯðar al \alst{b}rinnandi fiur, &
ia \alst{l}and ia \alst{l}iudi \hld\ \alst{l}ogna far·tęride: &
só \alst{f}árungo warð þat \alst{f}iur kumen, \hld\ só warð êr þe \alst{f}lód só samo: &
só wirðid þe \alst{l}atsto dag. \hld\ For þiu skal allaro \alst{l}iudjo ge·hwi-lik &
\alst{þ}ęnkjan fora þemu \alst{þ}inge; \hld\ þes is \alst{þ}arf mikil &
\alst{m}anno ge·hwi-likumu: \hld\ be·þiu látad iu an iuwan \alst{m}ód sorga.\eva

\bvb TODO.\evb\evg

\bvg\bva[53][4378]%
Hwand só hwan só þat ge·\alst{w}irðid, \hld\ þat \alst{w}aldand Krist, &
\alst{m}ári \alst{m}annes sunu \hld\ mid þeru \alst{m}aht godes, &
\alst{k}umit mid þiu \alst{k}raftu \hld\ \alst{k}uningo ríkjost &
\alst{s}ittjan an is \alst{s}elves maht \hld\ ęndi \alst{s}amod mid imu &
\alst{a}lle þea \alst{ę}ngilos, \hld\ þe þár \alst{u}ppa sind &
\alst{h}êlaga an \alst{h}imile, \hld\ þan skulun þarod \alst{h}ęliðo barn, &
\alst{ę}li-þeoda kuman \hld\ \alst{a}lla te·samne &
\alst{l}ibbjandero \alst{l}iudjo, \hld\ só hwat só io an þesumu \alst{l}iohte warð &
\alst{f}iriho a·\alst{f}ódid. \hld\ Þár hé þemu \alst{f}olke skal, &
allumu \alst{m}an-kunnje \hld\ \alst{m}ári drohtin &
a·\alst{d}êljen aftar iro \alst{d}ádjun. \hld\ Þan skêðid hé þea far·\alst{d}uanan man, &
þea far·\alst{w}arhton \alst{w}eros \hld\ an þea \alst{w}inistron hand: &
só duot hé ôk þea \alst{s}áligon \hld\ an þea \alst{s}wíðeron half; &
\alst{g}rótid hé þan þea \alst{g}ódun \hld\ ęndi im te·\alst{g}ęgnes sprikid: &
„\alst{K}umad gí“, kwiðid hé, „þea þár gi·\alst{k}orene sindun, \hld\ ęndi ant·fȧhad þit \alst{k}raftiga ríki, &
þat \alst{g}óde, þat þár gi·\alst{g}ęrẹwid stęndid, \hld\ þat þár warð \alst{g}umono barnun &
gi·\alst{w}arht fan þesaro \alst{w}er-oldes ęndje: \hld\ iu havad ge·\alst{w}íhid selvo &
\alst{f}ader allaro \alst{f}iriho barno: \hld\ gí mótun þesaro \alst{f}rumono neotan, &
ge·\alst{w}aldon þeses \alst{w}ídon ríkjas, \hld\ hwand gí oft mínan \alst{w}illjon frumidun, &
ful·\alst{g}éngun mí \alst{g}erno \hld\ ęndi wárun mí iuwaro \alst{g}evo mildje, &
\alst{þ}an ik bi·\alst{þ}wungan was \hld\ \alst{þ}urstu ęndi hungru, &
\alst{f}rostu bi·\alst{f}angan \hld\ efþo an \alst{f}eteron lag, &
bi·\alst{k}lęmmid an \alst{k}arkare: \hld\ oft wurðun mí \alst{k}umana þarod &
\alst{h}elpa fan iuwun \alst{h}andun: \hld\ gí wárun mí an iuwomu \alst{h}ugi mildje, &
\alst{w}ísodun mín \alst{w}erð-liko.“ \hld\ Þan sprikid imu eft þat \alst{w}erod an·gęgin: &
„\alst{F}rô mín þe gódo“, \hld[kweðat sie,] „hwan wári þú bi·\alst{f}angan só, &
be·\alst{þ}wungan an su·likun \alst{þ}arạvun, \hld\ só þú fora þesaru \alst{þ}iod tęlis, &
\alst{m}ahtig \alst{m}ênis? \hld\ Hwan gi·sah þí \alst{m}an ênig &
be·\alst{þ}wungen an su·likun \alst{þ}arạvun? \hld\ Hwat þú haves allaro \alst{þ}iodo gi·wald &
iak só samo þero \alst{m}êðmo, \hld\ þero þe io \alst{m}anno barn &
ge·\alst{w}unnun an þesaro \alst{w}er-oldi.“ \hld\ Þan sprikid im eft \alst{w}aldand god: &
„só hwat só gí \alst{d}ádun“, \hld[kwiðit hé,] „an iuwes \alst{d}rohtines namon, &
\alst{g}ódes far·\alst{g}ávun \hld\ an \alst{g}odes êra &
þem \alst{m}annun, þe hér \alst{m}inniston sindun, \hld\ þero nú undar þesaru \alst{m}ęnegi standad &
ęndi þurh \alst{ô}d-módi \hld\ \alst{a}rme wárun &
\alst{w}eros, hwand sie mínan \alst{w}illjon fręmidun \hld\ —só hwat só gí im iuwaro \alst{w}elono far·gávun, &
gi·\alst{d}ádun þurh \alst{d}iuriða, \hld\ þat ant·féng iuwa \alst{d}rohtin selvo, &
þiu \alst{h}elpe kwam te \alst{h}evan-kuninge. \hld\ Be·þiu wili iu þe \alst{h}êlago drohtin &
\alst{l}ônon iuwan gi·\alst{l}ôvon: \hld\ givid iu \alst{l}íf êwig.“ &
\alst{W}ęndid ina þan \alst{w}aldand \hld\ an þea \alst{w}inistron hand, &
\alst{d}rohtin te þem far·\alst{d}uanun mannun, \hld\ sagad im þat sie skulin þea \alst{d}ád ant·gelden, &
þea \alst{m}an iro \alst{m}ên-gi·werk: \hld\ „nú gí fan \alst{m}í skulun“, kwiðit hé, &
„\alst{f}aran só for·\alst{f}lókane \hld\ an þat \alst{f}iur êwig, &
þat þár gi·\alst{g}arẹwid warð \hld\ \alst{g}odes and-sakun, &
\alst{f}íundo \alst{f}olke \hld\ be \alst{f}irin-werkun, &
\alst{h}wand gí mí ni \alst{h}ulpun, \hld\ þan mí \alst{h}unger ęndi þurst &
\alst{w}êgde te \alst{w}undrun \hld\ efþa ik ge·\alst{w}ádjes lôs &
\alst{g}éng \alst{j}ámer-mód, \hld\ was mí \alst{g}rôtun þarf, &
þan ni habde ik þár ênige \alst{h}elpe, \hld\ þan ik ge·\alst{h}ęftid was, &
an \alst{l}iðo-kospun bi·\alst{l}okan, \hld\ efþa mí \alst{l}egar bi·féng, &
\alst{s}wára \alst{s}uhti: \hld\ þan ni weldun gí mín \alst{s}iokes þár &
\alst{w}íson mid \alst{w}ihti: \hld\ ni was iu \alst{w}erð eo·wiht, &
þat gí mín ge·\alst{h}ugdin. \hld\ Be·þiu gí an \alst{h}ęllje skulun &
\alst{þ}olon an \alst{þ}iustre.“ \hld\ Þan sprikid imu eft þiu \alst{þ}iod an·gęgin: &
„\alst{W}ola \alst{w}aldand god“, \hld[kweðad sie,] „hwí wilt þú só wið þit \alst{w}erod sprekan, &
\alst{m}ahljen wið þese \alst{m}ęnegi? \hld\ Hwan was þí io \alst{m}anno þarf, &
\alst{g}umono \alst{g}ódes? \hld\ Hwat sie it al be þínun \alst{g}evun êgun, &
\alst{w}elon an þesaro \alst{w}er-oldi“. \hld\ Þan sprikid eft \alst{w}aldand god: &
„þan gí þea \alst{a}rmostun“, \hld[kwiðid hé,] „\alst{ę}ldi-barno, &
\alst{m}anno þea \alst{m}inniston \hld\ an iuwomu \alst{m}ód-sevon &
\alst{h}ęliðos far·\alst{h}ugdun, \hld\ létun sea iu an iuwomu \alst{h}ugi lêðe, &
be·\alst{d}êldun sie iuwaro \alst{d}iurða, \hld\ þan dádun gí iuwana \alst{d}rohtin só sama, &
gi·\alst{w}ęrnidun imu iuwaro \alst{w}elono: \hld\ be·þiu ni wili iu \alst{w}aldand god, &
ant·\alst{f}áhen \alst{f}ader iuwa, \hld\ ak gí an þat \alst{f}iur skulun, &
an þene \alst{d}iopun \alst{d}ôð, \hld\ \alst{d}iuvlun þionon, &
\alst{w}rêðun \alst{w}iðẹr-sakun, \hld\ hwand gí só \alst{w}arhtun bi·foran.“ &
Þan aftar þem \alst{w}ordun skêðit \hld\ þat \alst{w}erod an twê, &
þea \alst{g}ódun ęndi þea uvilon: \hld\ farad þea far·\alst{g}riponon man &
an þea \alst{h}êtan \alst{h}ęl \hld\ \alst{h}riuwig-móde, &
þea far·\alst{w}arhton \alst{w}eros, \hld\ \alst{w}íti ant·fȧhat, &
\alst{u}vil \alst{ę}ndi-lôs. \hld\ Lêdid \alst{u}p þanen &
\alst{h}êr \alst{h}evan-kuning \hld\ þea \alst{h}luttạron þeoda &
an þat \alst{l}ang-same \alst{l}ioht: \hld\ þár is \alst{l}íf êwig, &
gi·\alst{g}arẹwid \alst{g}odes ríki \hld\ \alst{g}ódaro þiado.“\eva

\bvb TODO.\evb\evg

\subsection{Passion.}

\bvg\bva[54][4452]%
Só ge·fragn ik þat þem \alst{r}inkun þó \hld\ \alst{r}íki drohtin &
umbi þesaro \alst{w}er-oldes gi·\alst{w}and \hld\ \alst{w}ordun talde, &
hwó þiu \alst{f}orð \alst{f}ęrid, \hld\ þan lango þe sie \alst{f}iriho barn &
\alst{a}rdon mótun, \hld\ ia hwó siu an þemu \alst{ę}ndje skal &
te·\alst{g}líden ęndi te·\alst{g}angen. \hld\ Hé sagde ôk is \alst{j}ungarun þár &
\alst{w}árun \alst{w}ordun: \hld\ „Hwat gí \alst{w}itun alle“, kwað hé, &
„þat nú ovar \alst{t}wá naht \hld\ sind \alst{t}ídi kumana, &
\alst{J}udeono paskha, \hld\ þat sie skulun iro \alst{g}ode þionon, &
\alst{w}eros an þemu \alst{w}íhe. \hld\ Þes nis ge·\alst{w}and ênig, &
þat þár wirðid \alst{m}annes sunu \hld\ te þeru \alst{m}ęgin-þiodu &
\alst{k}raftag far·\alst{k}ôpot \hld\ ęndi an \alst{k}rúke a·slagan, &
\alst{þ}olod \alst{þ}iad-kwála.“ \hld\ Þó warð þár \alst{þ}egạn manag &
\alst{s}líð-mód gi·\alst{s}amnod, \hld\ \alst{s}u̇ðar-liudjo, &
\alst{J}udeono \alst{g}um-skępi, \hld\ þár sie skoldun iro \alst{g}ode þionon. &
wurðun \alst{ê}o-sagon \hld\ \alst{a}lle kumane, &
an \alst{w}arf \alst{w}eros, \hld\ þe sie þó \alst{w}ísostun &
undar þeru \alst{m}ęnegi \hld\ \alst{m}anno taldun, &
\alst{k}raftag \alst{k}uni-burd. \hld\ Þár \alst{K}aiphas was, &
\alst{b}iskop þero liudjo. \hld\ Sie rédun þó an þat \alst{b}arn godes, &
hwó sie ina a·\alst{s}luogin \hld\ \alst{s}undja lôsan, &
kwáðun þat sie ina an þemu \alst{h}êlagon daga \hld\ \alst{h}rínen ni skoldin &
undar þero \alst{m}anno \alst{m}ęnegi, \hld\ „þat ni werðe þius \alst{m}ęgin-þioda, &
\alst{h}ęliðos an \alst{h}róru, \hld\ hwand ina þit \alst{h}ęri-skępi wili &
far·\alst{st}anden mid \alst{st}rídu. \hld\ Wí só \alst{st}illo skulun &
\alst{f}rêson is \alst{f}erạhes, \hld\ þat þit \alst{f}olk Judeono &
an þesun \alst{w}íh-dagun \hld\ \alst{w}róht ni af·hębbjen.“ &
Þó géng imu þár \alst{J}údas forð, \hld\ \alst{j}ungaro Kristes, &
\alst{ê}n þero twe-livjo, \hld\ þár þat \alst{a}ðali sat, &
\alst{J}udeono \alst{g}um-skępi; \hld\ kwað þat hé is im \alst{g}ódan rád &
\alst{s}ęggjan mahti: \hld\ „hwat willjad gí mí \alst{s}ęlljen hér“, kwað hé, &
„\alst{m}êðmo te \alst{m}édu, \hld\ ef ik iu þene \alst{m}an givu &
áno \alst{w}íg ęndi áno \alst{w}róht?“ \hld\ Þó warð þes \alst{w}erodes hugi, &
þero \alst{l}iudjo an \alst{l}ustun: \hld\ „ef þú wili gi·\alst{l}êstjen só“, kwáðun sie, &
„þín \alst{w}ord gi·\alst{w}áron, \hld\ þan þú gi·\alst{w}ald haves, &
hwat þú at \alst{þ}esaru \alst{þ}iodu \hld\ \alst{þ}iggjan willjes &
\alst{g}ódaro mêðmo.“ \hld\ Þó gi·hét imu þat \alst{g}um-skępi þár &
an is \alst{s}elves dóm \hld\ \alst{s}ilụvar-skatto &
\alst{þ}rí-tig at·samne, \hld\ ęndi hé te þeru \alst{þ}iodu gi·sprak &
\alst{d}ęrẹvjun wordun, \hld\ þat hé gávi is \alst{d}rohtin wið þiu. &
\alst{w}ende ina þó fan þemu \alst{w}erode: \hld\ was im \alst{w}rêð hugi, &
\alst{t}alode im só \alst{t}reu-lôs, \hld\ hwan êr wurði imu þiu \alst{t}íd kuman, &
þat hé ina mahti far·\alst{w}ísjen \hld\ \alst{w}rêðaro þiodo, &
\alst{f}íundo \alst{f}olke. \hld\ Þan wisse þat \alst{f}riðu-barn godes, &
\alst{w}ár \alst{w}aldand Krist, \hld\ þat hé þese \alst{w}er-old skolde, &
a·\alst{g}even þese \alst{g}ardos \hld\ ęndi sókjen imu \alst{g}odes ríki, &
gi·\alst{f}aren is \alst{f}ader-óðil. \hld\ Þó ni gi·sah ênig \alst{f}iriho barno &
\alst{m}êron \alst{m}innje, \hld\ þan hé þó te þem \alst{m}annun gi·nam, &
te þem is \alst{g}ódun \alst{j}ungaron: \hld\ \alst{g}ôme warhte, &
\alst{s}ętte sie \alst{s}wás-líko \hld\ ęndi im \alst{s}agde filu &
\alst{w}ároro \alst{w}ordo. \hld\ Skrêd \alst{w}estẹr dag, &
\alst{s}unne te \alst{s}edle. \hld\ Þó hé \alst{s}elvo gi·bôd, &
\alst{w}aldand mid is \alst{w}ordun, \hld\ hét im \alst{w}ater dragan &
\alst{h}luttạr te \alst{h}andun, \hld\ ęndi rês þó þe \alst{h}êlago Krist, &
þe \alst{g}ódo at þem \alst{g}ômun \hld\ ęndi þár is \alst{j}ungarono þwóg &
\alst{f}óti mid is \alst{f}olmun \hld\ ęndi swarf sie mid is \alst{f}anon aftar, &
\alst{d}ruknide sie \alst{d}iur-líka. \hld\ Þó wið is \alst{d}rohtin sprak &
\alst{S}ímon Petrus: \hld\ „Ni þunkid mí þit \alst{s}ómi þing“, kwað hé, &
„\alst{f}rô mín þe gódo, \hld\ þat þú míne \alst{f}óti þwahes &
mid þem þínun \alst{h}êlagun \alst{h}andun.“ \hld\ Þó sprak imu eft is \alst{h}êrro an·gęgin, &
\alst{w}aldand mid is \alst{w}ordun: \hld\ „Ef þú is \alst{w}illjan ni haves“, kwað hé, &
„te ant·\alst{f}ȧhanne, \hld\ þat ik þíne \alst{f}óti þwahe &
þurh su·lika \alst{m}innja, \hld\ só ik þesun ȯðrun \alst{m}annun hér &
\alst{d}óm þurh \alst{d}iurða, \hld\ þan ni haves þú ênigan \alst{d}êl mid mí &
an \alst{h}evan-ríkja.“ \hld\ \alst{H}ugi warð þó gi·węndid &
\alst{S}ímon Petruse: \hld\ „Þú hava þí \alst{s}elvo gi·wald“, kwað hé, &
„\alst{f}rô mín þe gódo, \hld\ \alst{f}óto ęndi hando &
ęndi mínes \alst{h}ôvdes só sama, \hld\ \alst{h}andun þínun, &
\alst{þ}iadan, te \alst{þ}wahanne, \hld\ te þiu þak ik móti \alst{þ}ína forð &
\alst{h}uldi \alst{h}ębbjan \hld\ ęndi \alst{h}evan-ríkjes &
su·lik gi·\alst{d}êli, \hld\ só þú mí, \alst{d}rohtin, wili &
far·\alst{g}even þurh þína \alst{g}ódi.“ \hld\ \alst{J}ungaron Kristes, &
þene \alst{a}mbaht-skępi \hld\ \alst{e}rlos þolodun, &
\alst{þ}egnos mid gi·\alst{þ}uldjon, \hld\ só hwat só im iro \alst{þ}iodan dede, &
\alst{m}ahtig þurh þea \alst{m}innja, \hld\ ęndi mênde imu al \alst{m}éra þing &
\alst{f}irihon te gi·\alst{f}rummjenne.\eva

\bvb TODO.\evb\evg

\bvg\bva[55][4526]%
\hspace*{100pt}\alst{F}riðu-barn godes &
géng imu þó eft gi·\alst{s}ittjen \hld\ under þat ge·\alst{s}ïðo folk &
ęndi im sagda filu \alst{l}ang-samna rád. \hld\ Warð eft \alst{l}ioht kuman, &
\alst{m}orgen te \alst{m}annun. \hld\ \alst{M}ahtigne Krist &
\alst{g}róttun is \alst{j}ungaron ęndi frágodun, \hld\ hwár sie is \alst{g}ôma þó &
an þemu \alst{w}íh-dage \hld\ \alst{w}irkjen skoldin, &
\alst{h}war hé weldi \alst{h}alden \hld\ þea \alst{h}êlagon tídi &
\alst{s}elvo mid is ge·\alst{s}ïðun. \hld\ Þó hé sie \alst{s}ókjen hét, &
þea \alst{g}umon \alst{J}erusalem: \hld\ „só gí þan \alst{g}angan kumad“, kwað hé, &
„an þea \alst{b}urg innan \hld\ —þár is \alst{b}raht mikil, &
\alst{m}ęgin-þiodo gi·\alst{m}ang—, \hld\ þár mugun gí ênan \alst{m}an sehan &
an is \alst{h}andun dragen \hld\ \alst{h}luttres watares &
\alst{f}ul mid \alst{f}olmun. \hld\ Þemu gí \alst{f}olgon skulun &
an só hwi-like \alst{g}ardos, \hld\ só gí ina \alst{g}angan gi·sehat, &
ia gí þan þemu \alst{h}êrron, \hld\ þe þie \alst{h}ovos êgi, &
\alst{s}elvon \alst{s}ęggjad, \hld\ þat ik iu \alst{s}ęnde þarod &
te gi·\alst{g}aruwenne mína \alst{g}ôma. \hld\ Þan tôgid hé iu ên \alst{g}ód-lík hús, &
\alst{h}ôhan sóleri, \hld\ þe is bi·\alst{h}angen al &
\alst{f}agạrun \alst{f}ratahun. \hld\ Þár gí \alst{f}rummjen skulun &
\alst{w}erd-skępi mínan. \hld\ Þár bium ik \alst{w}is-kumo &
\alst{s}elvo mid mínun ge·\alst{s}ïðun.“ \hld\ Þó wurðun \alst{s}án aftar þiu &
þár te \alst{J}erusalem \hld\ \alst{j}ungaron Kristes &
\alst{f}orð-ward an \alst{f}ęrdi, \hld\ \alst{f}undun all só hé sprak &
\alst{w}ord-têkạn \alst{w}ár: \hld\ ni was þes gi·\alst{w}and ênig. &
Þár \alst{g}ęrẹwidun sie þea \alst{g}ôma. \hld\ Warð þe \alst{g}odes sunu, &
\alst{h}êlag drohtin \hld\ an þat \alst{h}ús kuman, &
þár sie þe \alst{l}and-wíse \hld\ \alst{l}êstjen skoldun, &
ful·\alst{g}angan \alst{g}odes gi·bode, \hld\ al só \alst{J}udeono was &
\alst{ê}o ęndi \alst{a}ld-sidu \hld\ an \alst{ê}r-dagun. &
Gi·wêt imu þó an þemu \alst{á}vande \hld\ \alst{a}lo-waldand Krist &
an þene \alst{s}ęli \alst{s}ittjen; \hld\ hét þár is ge·\alst{s}ïðos te imu &
\alst{t}we-livi gangan, \hld\ þea im gi·\alst{t}riuwiston &
an iro \alst{m}ód-sevon \hld\ \alst{m}anno wárun &
bi \alst{w}ordun ęndi bi \alst{w}ísun: \hld\ \alst{w}isse imu selvo &
iro \alst{h}ugi-skęfti \hld\ \alst{h}êlag drohtin. &
\alst{G}rótte sie þó ovar þem \alst{g}ômun: \hld\ „\alst{G}ern bium ik swíðo“, kwað hé, &
„þat ik \alst{s}amad mid iu \hld\ \alst{s}ittjen móti, &
\alst{g}ômono neoten, \hld\ \alst{J}udeono paskha &
\alst{d}êljen mid iu só \alst{d}iurjun. \hld\ Nú ik iu iuwes \alst{d}rohtines skal &
\alst{w}illjon sęggjan, \hld\ þat ik an þesaro \alst{w}er-oldi ni mót &
mid \alst{m}annun \alst{m}êr \hld\ \alst{m}óses an·bíten &
\alst{f}urður mid \alst{f}irihun, \hld\ êr þan gi·\alst{f}ullod wirðid &
\alst{h}imilo ríki. \hld\ Mí is an \alst{h}andun nú &
\alst{w}íti ęndi \alst{w}undẹr-kwále, \hld\ þea ik for þesumu \alst{w}erode skal, &
\alst{þ}olon for þesaru \alst{þ}iodu.“ \hld\ Só hé þó só te þem \alst{þ}egnun sprak, &
\alst{h}êlag drohtin, \hld\ só warð imu is \alst{h}ugi dróvi, &
warð imu gi·\alst{s}worken \alst{s}evo, \hld\ ęndi eft te þem ge·\alst{s}ïðun sprak, &
þe \alst{g}ódo te þem is \alst{j}ungarun: \hld\ „Hwat ik iu \alst{g}odes ríki“, kwað hé, &
„gi·\alst{h}ét \alst{h}imiles lioht, \hld\ ęndi gí mí \alst{h}old-líko &
iuwan \alst{þ}egạn-skępi. \hld\ Nú ni willjat gí a·\alst{þ}ęngjan só, &
ak \alst{w}ęnkjat þero \alst{w}ordo. \hld\ Nú sęggju ik iu te \alst{w}áran hér, &
þat wili iuwar \alst{t}we-livjo ên \hld\ \alst{t}reuwana swíkan, &
wili mí far·\alst{k}ôpon \hld\ undar þit \alst{k}unni Judeono, &
gi·\alst{s}ęlljen wiðẹr \alst{s}ilụvre, \hld\ ęndi wili imu þár \alst{s}ink niman, &
\alst{d}iurje mêðmos, \hld\ ęndi geven is \alst{d}rohtin wið þiu, &
\alst{h}oldan \alst{h}êrran. \hld\ Þat imu þoh te \alst{h}arme skal, &
\alst{w}erðan te \alst{w}ítje; \hld\ be þat hé þea \alst{w}urdi far·sihit &
ęndi hé þes \alst{a}rvêdjes \hld\ \alst{ę}ndi skawot, &
þan \alst{w}êt hé þat te \alst{w}áran, \hld\ þat imu wári \alst{w}óðjera þing, &
\alst{b}ętera mikilu, \hld\ þat hé gio gi·\alst{b}oran ni wurði &
\alst{l}ibbjendi te þesumu \alst{l}iohte, \hld\ þan hé þat \alst{l}ôn nimid, &
\alst{u}vil \alst{a}rvêdi \hld\ \alst{i}n-wid-rádo.“ &
Þó bi·gan þero \alst{e}rlo ge·hwi-lik \hld\ te \alst{ȯ}ðrumu skawon, &
\alst{s}orgondi \alst{s}ehan; \hld\ was im \alst{s}êr hugi, &
\alst{h}riuwig umbi iro \alst{h}erta: \hld\ gi·hôrdun iro \alst{h}êrron þó &
\alst{g}orn-word sprekan. \hld\ Þea \alst{g}umon sorgodun, &
hwi-likan hé þero \alst{t}we-livjo \hld\ te þiu \alst{t}ęlljen weldi, &
\alst{sk}uldigna \alst{sk}aðon, \hld\ þat hé habdi þea \alst{sk}attos þár &
ge·\alst{þ}ingod at þeru \alst{þ}iod. \hld\ Ni was þero \alst{þ}egno ênigumu &
su·likes \alst{i}n-widdjes \hld\ \alst{ó}ði te gehanne, &
\alst{m}ên-gi·þȧhtjo \hld\ —ant·suok þero \alst{m}anno ge·hwi-lik—, &
wurðun alle an \alst{f}orhtun, \hld\ \alst{f}rágon ne gi·dorstun, &
êr þan þó ge·\alst{b}ôknide \hld\ \alst{b}ar-wirðig gumo, &
\alst{S}ímon Petrus \hld\ —ne gi·dorste it \alst{s}elvo sprekan— &
te \alst{J}ohanne þemu \alst{g}ódon: \hld\ hé was þemu \alst{g}odes barne &
an \alst{þ}em dagun \hld\ \alst{þ}egno liovost, &
\alst{m}êst an \alst{m}innjun \hld\ ęndi móste þár þó an þes \alst{m}ahtiges Kristes &
\alst{b}arme restjen \hld\ ęndi an is \alst{b}reostun lag, &
\alst{h}linode mid is \alst{h}ôvdu: \hld\ þár nam hé só manag \alst{h}êlag ge·rúni, &
\alst{d}iapa gi·þȧhti, \hld\ ęndi þó te is \alst{d}rohtine sprak, &
be·gan ina þó \alst{f}rágon: \hld\ „hwe skal þat, \alst{f}rô mín, wesen“, kwað hé, &
„þat þi far·\alst{k}ôpon wili, \hld\ \alst{k}uningo ríkjost, &
undar þínaro \alst{f}íundo \alst{f}olk? \hld\ U̇s wári þes \alst{f}iri-wit mikil, &
\alst{w}aldand, te \alst{w}itanne.“ \hld\ Þó habde eft is \alst{w}ord garu &
\alst{h}êljando Krist: \hld\ „seh þi, hwemu ik hér an \alst{h}and geve &
mínes \alst{m}óses for þesun \alst{m}annun: \hld\ þe haved \alst{m}ên-gi·þȧht, &
\alst{b}irid \alst{b}ittran hugi; \hld\ þe skal mí an \alst{b}anono ge·wald, &
\alst{f}íundun bi·\alst{f}elhen, \hld\ þár man mínes \alst{f}erhes skal, &
\alst{a}ldres \alst{á}htjen.“ \hld\ Nam hé þó \alst{a}ftar þiu &
þes \alst{m}óses for þem \alst{m}annun \hld\ ęndi gaf is þemu \alst{m}ên-skaðen, &
\alst{J}udase an hand \hld\ ęndi imu te·\alst{g}ęgnes sprak &
\alst{s}elvo for þem is ge·\alst{s}ïðun \hld\ ęndi ina \alst{s}niumo hét &
\alst{f}aran fan þemu is \alst{f}olke: \hld\ „\alst{f}rumi só þú þęnkis“, kwað hé, &
„\alst{d}ó þat þú \alst{d}uan skalt: \hld\ þú ni maht bi·\alst{d}ęrnjen lęng &
\alst{w}illjon þínan. \hld\ Þiu \alst{w}urd is at handun, &
þea \alst{t}ídi sind nú gi·náhid.“ \hld\ Só þó þe \alst{t}reu-logo &
þat \alst{m}ós ant·féng \hld\ ęndi mid is \alst{m}u̇ðu an·bêt, &
só af·\alst{g}af ina þó þiu \alst{g}odes kraft, \hld\ \alst{g}ramon in ge·witun &
an þene \alst{l}ík-hamon, \hld\ \alst{l}êða wihti, &
warð imu \alst{S}atanas \hld\ \alst{s}êro bi·tęngi, &
\alst{h}ardo umbi is \alst{h}erte, \hld\ sïður ine þiu \alst{h}elpe godes &
far·\alst{l}ét an þesumu \alst{l}iohte. \hld\ Só is þena \alst{l}iudjo wê, &
þe só undar þesumu \alst{h}imile skal \hld\ \alst{h}êrron wehslon.\eva

\bvb TODO.\evb\evg

\bvg\bva[56][4629]%
Gi·wêt imu þó \alst{ú}t þanen \hld\ \alst{i}n-widjas gern &
\alst{J}udas \alst{g}angan: \hld\ habde imu \alst{g}rimmen hugi &
\alst{þ}egạn wið is \alst{þ}iodan. \hld\ Was þó iu \alst{þ}iustri naht, &
\alst{s}wíðo gi·\alst{s}worken. \hld\ \alst{S}unu drohtines &
was ima at þem \alst{g}ômun forð \hld\ ęndi is \alst{j}ungarun þár &
\alst{w}aldand \alst{w}ín ęndi brôd \hld\ \alst{w}íhide bêðju, &
\alst{h}êlagode \alst{h}evan-kuning, \hld\ mid is \alst{h}andun brak, &
\alst{g}af it undar þem is \alst{j}ungarun \hld\ ęndi \alst{g}ode þankode, &
sagde þem \alst{á}-lát, \hld\ þe þár \alst{a}l gi·skóp, &
\alst{w}er-old ęndi \alst{w}unnja, \hld\ ęndi sprak \alst{w}ord manag: &
„gi·\alst{l}ôvjot gí þes \alst{l}iohto“, \hld[kwað hé,] „þat þit is mín \alst{l}ík-hamo &
ęndi mín \alst{b}lód só same: \hld\ givu ik iu hér \alst{b}êðju samad &
\alst{e}tan ęndi drinkan. \hld\ Þit ik an \alst{e}rðu skal &
\alst{g}evan ęndi \alst{g}eotan \hld\ ęndi iu te \alst{g}odes ríkje &
\alst{l}ôsjen mid mínu \alst{l}ík-hamen \hld\ an \alst{l}íf êwig, &
an þat \alst{h}imiles lioht. \hld\ Gi·\alst{h}uggjat gí simlun, &
þat \alst{g}í þiu ful·\alst{g}angan, \hld\ þiu ik an þesun \alst{g}ômun dón; &
\alst{m}árjad þit for \alst{m}ęnegi: \hld\ þit is \alst{m}ahtig þing, &
mid þius skulun gí iuwomu \alst{d}rohtine \hld\ \alst{d}iuriða frummjen, &
\alst{h}abbjad þit mín te gi·\alst{h}ugdjun, \hld\ \alst{h}êlag biliði, &
þat it \alst{ę}ldi-barn \hld\ \alst{a}ftar lêstjen, &
\alst{w}aron an þesaru \alst{w}er-oldi, \hld\ þat þat \alst{w}itin alle, &
\alst{m}an ovar þesan \alst{m}iddil-gard, \hld\ þat it is þurh mína \alst{m}innja gi·duan &
\alst{h}êrron te \alst{h}uldi. \hld\ Ge·\alst{h}uggjad gí simlun, &
hweo ik iu hér ge·\alst{b}iudu, \hld\ þat gí iuwan \alst{b}róðer-skępi &
\alst{f}asto \alst{f}rummjad: \hld\ habbjad \alst{f}erhtan hugi, &
\alst{m}innjod iu an iuwomu \alst{m}óde, \hld\ þat þat \alst{m}anno barn &
\alst{o}var \alst{i}rmin-þiod \hld\ \alst{a}lle far·standen, &
þat \alst{g}í sind \alst{g}egnungo \hld\ \alst{j}ungaron míne. &
Ôk skal ik iu \alst{k}u̇ðjen, \hld\ hwó hér wili \alst{k}raftag fíund, &
\alst{h}ęttjand \alst{h}eru-grim, \hld\ umbi iuwan \alst{h}ugi niusjen, &
\alst{S}atanas \alst{s}elvo: \hld\ hé kumid iuwaro \alst{s}eolono herod &
\alst{f}rókno \alst{f}rêson. \hld\ Simlun gí \alst{f}asto te gode &
\alst{b}erad iuwa \alst{b}reost-gi·þȧht: \hld\ ik skal an iuwaru \alst{b}edu standen, &
þat iu ni \alst{m}ugi þe \alst{m}ên-skaðo \hld\ \alst{m}ód ge·twífljan; &
ik \alst{f}ul-lêstju iu wiðẹr þemu \alst{f}íunde. \hld\ Ôk kwam hé herod giu \alst{f}rêson mín, &
þoh imu is \alst{w}illjon hér \hld\ \alst{w}iht ne gi·stódi, &
\alst{l}ioves an þemu mínumu \alst{l}ík-hamon. \hld\ Nú ni willju ik iu \alst{l}ęng helen, &
hwat iu hér nú \alst{s}niumo skal \hld\ te \alst{s}orgu gi·standen: &
gí skulun mí ge·\alst{s}wíkan, \hld\ ge·\alst{s}ïðos míne, &
iuwes \alst{þ}egạn-skępjes, \hld\ êr þan þius \alst{þ}iustrje naht &
\alst{l}iudi far·\alst{l}íða \hld\ ęndi eft \alst{l}ioht kume, &
\alst{m}organ te \alst{m}annun.“ \hld\ Þó warð \alst{m}ód gumon &
\alst{s}wíðo gi·\alst{s}worken \hld\ ęndi \alst{s}êr hugi, &
\alst{h}riuwig umbi iro \alst{h}erte \hld\ ęndi iro \alst{h}êrron word &
\alst{s}wíðo an \alst{s}orgun. \hld\ \alst{S}ímon Petrus þó, &
\alst{þ}egạn wið is \alst{þ}iodan \hld\ \alst{þ}ríst-wordun sprak &
bí \alst{h}uldi *wið is \alst{h}êrron: \hld\ „þoh þí all þit \alst{h}ęliðo folk“, kwaþ-hie, &
„gi·\alst{s}wíkan þína gi·\alst{s}ïðos, \hld\ þoh ik \alst{s}innon mid þí &
at allon \alst{þ}arạvon \hld\ \alst{þ}olojan willju. &
Ik biun \alst{g}aro sinnon, \hld\ ef mí \alst{g}od látið, &
þat ik an þínon \alst{f}ul-lêstje \hld\ \alst{f}asto gi·stande; &
þoh sia þi an \alst{k}arkarjes \hld\ \alst{k}lústron hardo, &
þesa \alst{l}iudi bi·\alst{l}úkan, \hld\ þoh ist mí \alst{l}uttil tweho, &
ne ik an þem \alst{b}ęndjon mid þi \hld\ \alst{b}ídan willje, &
\alst{l}iggjan mid þi só \alst{l}ieven; \hld\ ef sia þínes \alst{l}íves þan &
þuru \alst{ę}ggja níð \hld\ \alst{á}htjan willjad, &
\alst{f}rô mín þie guodo, \hld\ ik givu mín \alst{f}erạh furi þik &
an \alst{w}ápno spil: \hld\ nis mí \alst{w}erð iowiht &
te bi·\alst{m}íðanne, \hld\ só lango só mí \alst{m}ín warod &
\alst{h}ugi ęndi \alst{h}and-kraft.“ \hld\ Þuo sprak im eft is \alst{h}êrro an·gęgin: &
„Hwat þú þik bi·\alst{w}ánis“, \hld[kwaþ-hie,] „\alst{w}issaro treuwono, &
\alst{þ}rístero \alst{þ}ingo: \hld\ þú havis \alst{þ}egnes hugi, &
\alst{w}illjon guodan. \hld\ Ik mag þi sęggjan, hwó it þoh gi·\alst{w}erðan skal, &
þat þú \alst{w}irðis só \alst{w}êk-muod, \hld\ þoh þú nú ni \alst{w}ánjes só, &
þat þú þínes \alst{þ}iadnes te naht \hld\ \alst{þ}ríwo far·lôgnis &
êr \alst{h}ano-krádi ęndi kwiðis, \hld\ þak ik þín \alst{h}êrro ni sí, &
ak þú far·\alst{m}anst mína \alst{m}und-burd.“ \hld\ Þuo sprak eft þie \alst{m}an an·gęgin: &
„ef it gio an \alst{w}er-oldi“, \hld[kwaþ-hie,] „gi·\alst{w}erðan muosti, &
þat ik \alst{s}amad midi þí \hld\ \alst{s}weltan muosti, &
\alst{d}ôjan \alst{d}iur-líko, \hld\ þan ne wurði gio þie \alst{d}ag kuman, &
þat ik þín far·\alst{l}ôgnidi, \hld\ \alst{l}ievo drohtin, &
\alst{g}erno for þeson \alst{J}uðeon.“ \hld\ Þuo kwáðun alla þia \alst{j}ungron só, &
þat sia þár an þem \alst{þ}ingon mid im \hld\ \alst{þ}oljan weldin.\eva

\bvb TODO.\evb\evg

\bvg\bva[57][4703]%
Þuo im eft mid is \alst{w}ordon gi·bôd \hld\ \alst{w}aldand selvo, &
\alst{h}êr \alst{h}evan-kuning, \hld\ þat sia im ni lietin iro \alst{h}ugi twífljan, &
hiet þat sia ni weldin {[...]} \hld\ \alst{d}iopa gi·þȧhti: &
„Ne \alst{d}ruovje iuwa herta \hld\ þuru iuwes \alst{d}rohtines word, &
ne \alst{f}orọhtjat te filo: \hld\ ik skal \alst{f}ader u̇san &
\alst{s}elvan \alst{s}uokjan \hld\ ęndi iu \alst{s}ęndjan skal &
fan \alst{h}evan-ríkje \hld\ \alst{h}êlagna gêst: &
þie skal iu eft gi·\alst{f}ruofrjan \hld\ ęndi te \alst{f}rumu werðan, &
\alst{m}anon iu þero \alst{m}ahlo, \hld\ þie ik iu \alst{m}anag hębbju &
\alst{w}ordon gi·\alst{w}ísid. \hld\ Hie givit iu gi·\alst{w}it an briost, &
\alst{l}ust-sama \alst{l}êra, \hld\ þat gí \alst{l}êstjan forð &
þiu \alst{w}ord ęndi þiu \alst{w}erk, \hld\ þia ik iu an þesaro \alst{w}er-oldi gi·bôd.“ &
A·\alst{r}ês im þuo þe \alst{r}íkjo \hld\ an þemo \alst{r}akode innan, &
\alst{n}ęrjendo Krist \hld\ ęndi gi·wêt im \alst{n}ahtes þanan &
\alst{s}elvo mid is gi·\alst{s}ïðon: \hld\ \alst{s}êrago géngun &
swíðo \alst{g}ornondja \hld\ \alst{j}ungron Kristes, &
\alst{h}riuwig-muoda. \hld\ Þuo hie im an þena \alst{h}ôhan gi·wêt &
\alst{O}liueti-berg: \hld\ þár was hie \alst{u}p gi·wuno &
\alst{g}angan mid is \alst{j}ungron. \hld\ Þat wissa \alst{J}udas wel, &
\alst{b}alo-hugdig man, \hld\ hwand hie was oft an þem \alst{b}erẹge mid im. &
Þár \alst{g}ruotta þie \alst{g}odes suno \hld\ \alst{j}u̇gron sína: &
„Gí sind nú só \alst{d}ruovja“, \hld[kwaþ-hie,] „nú gí mínan \alst{d}ôð witun; &
nú \alst{g}ornonð gí ęndi \alst{g}riotand, \hld\ ęndi þesa \alst{J}uðeon sind an luston, &
\alst{m}ęndit þius \alst{m}ęnigi, \hld\ sindun an iro \alst{m}uode fráha, &
þius \alst{w}er-old ist an \alst{w}unnjon. \hld\ Þes wirðit þoh gi·\alst{w}and kuman &
\alst{s}niumo tulgo: \hld\ þan wirðit im \alst{s}êr hugi, &
þan \alst{m}ornjat sia an iro \alst{m}óde, \hld\ ęndi gí \alst{m}ęndjan skulun &
\alst{a}fter te \alst{ê}won-dage, \hld\ hwand gio \alst{ę}ndi ni kumið, &
iuwes \alst{w}el-líves gi·\alst{w}and: \hld\ be·þiu ne þurvun iu þius \alst{w}erk tregan, &
\alst{h}reuwan mín \alst{h}in-fard, \hld\ hwand þanan skal þiu \alst{h}elpa kuman &
\alst{g}umono barnon.“ \hld\ Þuo hiet hie is \alst{j}ungron þár &
\alst{b}ídan uppan þemo \alst{b}erge, \hld\ kwað þat hie ti \alst{b}edu weldi &
an þiu \alst{h}olm-klivu \hld\ \alst{h}ôhor stígan; &
hiet þuo \alst{þ}ria mid im \hld\ \alst{þ}egnos gangan, &
\alst{J}akobe ęndi \alst{J}ohannese \hld\ ęndi þena \alst{g}uodan Petruse, &
\alst{þ}ríst-muodjan \alst{þ}egạn. \hld\ Þuo sia mid iro \alst{þ}iedne samad &
\alst{g}erno \alst{g}éngun. \hld\ Þuo hiet sia þie \alst{g}odes suno &
an \alst{b}erge uppan \hld\ te \alst{b}edu hnígan, &
hiet sia \alst{g}od \alst{g}ruotjan, \hld\ *\alst{g}erno biddjan, &
þat hé im þero \alst{k}ostondero \hld\ \alst{k}raft far·stódi, &
\alst{w}rêðaro \alst{w}illjon, \hld\ þat im þe \alst{w}iðẹr-sako, &
ni \alst{m}ahti þe \alst{m}ên-skaðo \hld\ \alst{m}ód gi·twífljan, &
iak imu þó \alst{s}elvo gi·hnêg \hld\ \alst{s}unu drohtines &
\alst{k}raftag an \alst{k}nio-beda, \hld\ \alst{k}uningo ríkjost, &
\alst{f}orð-ward te \alst{f}oldu: \hld\ \alst{f}ader alo-þiado &
\alst{g}ódan \alst{g}rótte, \hld\ \alst{g}orn-wordun sprak &
\alst{h}riuwig-líko: \hld\ was imu is \alst{h}ugi dróvi, &
bi þeru \alst{m}ęnniski \hld\ \alst{m}ód gi·hrórid, &
is \alst{f}lêsk was an \alst{f}orhtun: \hld\ \alst{f}ellun imo trahni, &
\alst{d}rôp is \alst{d}iur-lík swêt, \hld\ al só \alst{d}rôr kumid &
\alst{w}allan fan \alst{w}undun. \hld\ Was an ge·\alst{w}inne þó &
an þemu \alst{g}odes barne \hld\ þe \alst{g}êst ęndi þe lík-hamo: &
ȯðar was \alst{f}u̇sid \hld\ an \alst{f}orð-wegos, &
þe \alst{g}êst an \alst{g}odes ríki, \hld\ ȯðar \alst{j}ámar stód, &
\alst{l}ík-hamo Kristes: \hld\ ni welde þit \alst{l}ioht a·geven, &
ak \alst{d}róvde for þemu \alst{d}ôðe. \hld\ Simla hé hreop te \alst{d}rohtine forð &
þiu \alst{m}êr aftar þiu \hld\ \alst{m}ahtigna grótte, &
\alst{h}ôhan \alst{h}imil-fader, \hld\ \alst{h}êlagna god, &
\alst{w}aldand mid is \alst{w}ordun: \hld\ „ef nú \alst{w}erðen ni mag“, kwað hé, &
„\alst{m}an-kunni ge·nęrid, \hld\ ne sí þat ik \alst{m}ínan geve &
\alst{l}iovan \alst{l}ík-hamon \hld\ for \alst{l}iudjo barn &
te \alst{w}êgjanne te \alst{w}undrun, \hld\ it sí þan þín \alst{w}illjo só, &
ik willju is þan gi·\alst{k}oston: \hld\ ik nimu þene \alst{k}ęlik an hand, &
\alst{d}rinku ina þi te \alst{d}iurðu, \hld\ \alst{d}rohtin frô mín, &
\alst{m}ahtig \alst{m}und-boro. \hld\ Ni seh þú \alst{m}ínes hér &
\alst{f}lêskes gi·\alst{f}órjes. \hld\ Ik \alst{f}ullon skal &
\alst{w}illjon þínen: \hld\ þú haves ge·\alst{w}ald ovar al.“ &
Gi·wêt imu þó \alst{g}angen, \hld\ þár hé êr is \alst{j}ungaron lét &
\alst{b}ídan uppan þemu \alst{b}erge; \hld\ fand sie þat \alst{b}arn godes &
\alst{s}lápen \alst{s}organdje: \hld\ was im \alst{s}êr hugi, &
þes sie fan iro \alst{d}rohtine \hld\ \alst{d}êljen skoldun. &
Só sind þat \alst{m}ód-þraka \hld\ \alst{m}anno ge·hwi-likumu, &
þat hé far·\alst{l}áten skal \hld\ \alst{l}iavane hêrron, &
af·\alst{g}even þene só \alst{g}ódene. \hld\ Þó hé te is \alst{j}ungarun sprak, &
\alst{w}ahte sie \alst{w}aldand \hld\ ęndi \alst{w}ordun grótte: &
„Hwí willjad gí \alst{s}ó \alst{s}lápen?“ \hld[kwað hé;] „ni mugun \alst{s}amad mid mí &
\alst{w}akon êne tíd? \hld\ Þiu \alst{w}urd is at handun, &
þat it só gi·\alst{g}angen skal, \hld\ só it \alst{g}od fader &
gi·\alst{m}arkode \alst{m}ahtig. \hld\ Mí nis an mínumu \alst{m}óde tweho: &
mín \alst{g}êst is \alst{g}aru \hld\ an \alst{g}odes willjan, &
\alst{f}u̇s te \alst{f}aranne: \hld\ mín \alst{f}lêsk is an sorgun, &
\alst{l}ętid mik mín \alst{l}ík-hamo: \hld\ \alst{l}êð is imu swíðo &
\alst{w}íti te þolonne. \hld\ Ik þoh \alst{w}illjan skal &
mínes \alst{f}ader ge·\alst{f}rummjen; \hld\ hębbjad gí \alst{f}asten hugi.“ &
Gi·wêt imu þó \alst{e}ft þanan \hld\ \alst{ȯ}ðer-sïðu &
an þene \alst{b}erg uppen \hld\ te \alst{b}edu gangan, &
\alst{m}ári drohtin, \hld\ ęndi þár só \alst{m}anag gi·sprak &
\alst{g}ódoro wordo. \hld\ \alst{G}odes ęngil kwam &
\alst{h}êlag fan \alst{h}imile, \hld\ is \alst{h}ugi fastnode, &
\alst{b}ęldide te þem \alst{b}ęndjun. \hld\ Hé was an þeru \alst{b}edu simla &
\alst{f}orð an \alst{f}líte \hld\ ęndi is \alst{f}ader grótte, &
\alst{w}aldand mid is \alst{w}ordun: \hld\ „ef it nú \alst{w}esen ni mag“, kwað hé, &
„\alst{m}ári drohtin, \hld\ nevu ik for þit \alst{m}anno folk &
\alst{þ}iod-kwále \alst{þ}oloje, \hld\ ik an \alst{þ}ínan skal &
\alst{w}illjan \alst{w}onjan.“ \hld\ Gi·\alst{w}êt imu þó eft þanen &
\alst{s}ókjan is ge·\alst{s}ïðos: \hld\ fand sie \alst{s}lápandje, &
\alst{g}rótte sie \alst{g}áhun. \hld\ \alst{G}éng imu eft þanen &
\alst{þ}riddjon sïðu te bedu \hld\ ęndi sprak \alst{þ}iod-kuning &
al þiu \alst{s}elvon word, \hld\ \alst{s}unu drohtines, &
te þemu \alst{a}lo-waldon fader, \hld\ só hé \alst{ê}r dede, &
\alst{m}anode \alst{m}ahtigna \hld\ \alst{m}anno frumana &
swíðo \alst{n}iud-líko \hld\ \alst{n}ęrjando Krist, &
\alst{g}éng imu þó eft te þem is \alst{j}ungarun, \hld\ \alst{g}rótte sie sáno: &
„\alst{s}lápad gí ęndi ręstjad“, \hld[kwað hé,] „nú wirðid \alst{s}niumo herod &
\alst{k}uman mid \alst{k}raftu, \hld\ þe mí far·\alst{k}ôpot havad, &
\alst{s}undja lôsan gi·\alst{s}ald.“ \hld\ Ge·\alst{s}ïðos Kristes &
\alst{w}akodun þó aftar þem \alst{w}ordun \hld\ ęndi gi·sáhun þó þat \alst{w}erod kuman &
an þene \alst{b}erg uppen \hld\ \alst{b}rahtmu þiu mikilon, &
\alst{w}rêða \alst{w}ápạn-berand.\eva

\bvb TODO.\evb\evg

\bvg\bva[58][4811]%
\hspace*{100pt} \alst{W}ísde im Judas, &
\alst{g}ram-hugdig man; \hld\ \alst{J}udeon aftar sigun, &
\alst{f}íundo \alst{f}olk-skępi; \hld\ dróg man \alst{f}iur an gi·mang, &
\alst{l}ogna an \alst{l}ioht-fatun, \hld\ \alst{l}êdde man faklon &
\alst{b}rinnandja fan \alst{b}urg, \hld\ þár sie an þene \alst{b}erg uppan &
\alst{st}igun mid \alst{st}rídu. \hld\ Þea \alst{st}ędi wisse Judas wel, &
hwár hé þea \alst{l}iudi \hld\ tó \alst{l}êdjan skolde. &
Sagde imu þó te \alst{t}êkne, \hld\ þó sie þár \alst{t}ó fórun &
þemu \alst{f}olke bi·\alst{f}oran, \hld\ te þiu þat sie ni far·\alst{f}éngin þár, &
\alst{e}rlos \alst{ȯ}ðren man: \hld\ „ik gangu imu at \alst{ê}rist tó“, kwað hé, &
„\alst{k}ussju ine ęndi \alst{k}waddju: \hld\ þat is \alst{K}rist selvo. &
Þene gí \alst{f}áhen skulun \hld\ \alst{f}olko kraftu, &
\alst{b}inden ina uppan þemu \alst{b}erge \hld\ ęndi ina te \alst{b}urg hinan &
\alst{l}êdjen undar þea \alst{l}iudi: \hld\ hé is \alst{l}íves havad &
mid is \alst{w}ordun far·\alst{w}erkod.“ \hld\ \alst{W}erod sïðode þó, &
ant-tat sie te \alst{K}riste \hld\ \alst{k}umane wurðun, &
\alst{g}rim folk \alst{J}udeono, \hld\ þár hé mid is \alst{j}ungarun stód, &
\alst{m}ári drohtin: \hld\ bêd \alst{m}etodo-gi·skapu, &
\alst{t}orhtero \alst{t}ídjo. \hld\ Þó géng imu \alst{t}reu-lôs man, &
\alst{J}udas te·\alst{g}ęgnes \hld\ ęndi te þemu \alst{g}odes barne &
\alst{h}nêg mid is \alst{h}ôvdu \hld\ ęndi is \alst{h}êrron kwędde, &
\alst{k}uste ina \alst{k}raftagne \hld\ ęndi is \alst{k}widi lêste, &
\alst{w}ísde ina þemu \alst{w}erode, \hld\ al só hé êr mid \alst{w}ordun ge·hét. &
Þat \alst{þ}olode al mid gi·\alst{þ}uldjun \hld\ \alst{þ}iodo drohtin, &
\alst{w}aldand þesara \alst{w}er-oldes \hld\ ęndi sprak imu mid is \alst{w}ordun tó, &
\alst{f}rágode ine \alst{f}rókno: \hld\ „be·hwí kumis þú só mid þius \alst{f}olku te mí, &
be·hwí \alst{l}êdis þú mí só þese \alst{l}iudi tó \hld\ ęndi mí te þesare \alst{l}êðan þiode sprekan, &
far·\alst{k}ôpos mid þínu \alst{k}ussu \hld\ under þit \alst{k}unni Judeono, &
\alst{m}eldos mí te þesaru \alst{m}ęnegi?“ \hld\ Géng imu þó wið þea \alst{m}an &
wið þat \alst{w}erod ȯðar \hld\ ęndi sie mid is \alst{w}ordun fragn, &
hwene sie mid þiu ge·\alst{s}ïðju \hld\ \alst{s}ókjan kwámin &
só \alst{n}iud-liko an \alst{n}aht, \hld\ „so gí willjan \alst{n}ôd frummjen &
\alst{m}anno hwi-likumu.“ \hld\ Þó sprak imu eft þiu \alst{m}ęnegi an·gęgin, &
kwáðun þat im \alst{h}êljand \hld\ þár an þemu \alst{h}olme uppan &
ge·\alst{w}ísid \alst{w}ári, \hld\ „þe þit gi·\alst{w}er frumid &
\alst{J}udeo liudjun \hld\ ęndi ina \alst{g}odes sunu &
\alst{s}elvon hêtid. \hld\ Ina kwámun wí \alst{s}ókjan herod, &
weldin ina \alst{g}erno bi·\alst{g}eten: \hld\ hé is fan \alst{G}alileo lande, &
fan \alst{N}azareth-burg.“ \hld\ Só im þó þe \alst{n}ęrjendjo Krist &
\alst{s}agde te \alst{s}ȯðan, \hld\ þat hé it \alst{s}elvo was, &
só wurðun þó an \alst{f}orhtun \hld\ \alst{f}olk Judeono, &
wurðun under·\alst{b}adode, \hld\ þat sie under \alst{b}ak fellun &
\alst{a}lle \alst{e}fno sán, \hld\ \alst{e}rðe gi·sóhtun, &
wiðẹr·\alst{w}ardes þat \alst{w}erod: \hld\ ni mahte þat \alst{w}ord godes, &
þie \alst{st}emnje ant·\alst{st}andan: \hld\ wárun þoh só \alst{st}rídige man, &
a·\alst{h}liopun eft up an þemu \alst{h}olme, \hld\ \alst{h}ugi fastnodun, &
\alst{b}undun \alst{b}riost-gi·þȧht, \hld\ gi·\alst{b}olgane géngun &
\alst{n}áhor mid \alst{n}íðu, \hld\ ant-tat sie þene \alst{n}ęrjendjon Krist &
\alst{w}erodo bi·\alst{w}urpun. \hld\ Stódun \alst{w}íse man, &
swíðo \alst{g}ornundje \hld\ \alst{j}ungaron Kristes &
bi·foran þeru \alst{d}ęrẹvjon \alst{d}ádi \hld\ ęndi te iro \alst{d}rohtine sprákun: &
„\alst{w}ári it nú þín \alst{w}illjo“, \hld[kwáðun sie,] „\alst{w}aldand frô mín, &
þat sie u̇s hér an \alst{sp}eres ordun \hld\ \alst{sp}ildjen móstin &
\alst{w}ápnun \alst{w}unde, \hld\ þan ni wári u̇s \alst{w}iht só gód, &
só þat wí hér for u̇sumu \alst{d}rohtine \hld\ \alst{d}óan móstin &
\alst{b}ęniðjun \alst{b}lêka“. \hld\ Þó gi·\alst{b}olgan warð &
\alst{s}nel \alst{s}werd-þegạn, \hld\ \alst{S}ímon Petrus, &
\alst{w}ell imu innan hugi, \hld\ þat hé ni mahte ênig \alst{w}ord sprekan: &
só \alst{h}arm warð imu an is \alst{h}ertan, \hld\ þat man is \alst{h}êrron þár &
\alst{b}inden welde. \hld\ Þó hé gi·\alst{b}olgan géng, &
swíðo \alst{þ}ríst-mód \alst{þ}egạn \hld\ for is \alst{þ}iodan standen, &
\alst{h}ard for is \alst{h}êrron: \hld\ ni was imu is \alst{h}ugi twífli, &
\alst{b}lóð an is \alst{b}reostun, \hld\ ak hé is \alst{b}il a·tôh, &
\alst{s}werd bi \alst{s}ídu, \hld\ \alst{s}lóg imu te·gęgnes &
an þene \alst{f}uriston \alst{f}íund \hld\ \alst{f}olmo krafto, &
þat þó \alst{M}alkhus warð \hld\ \alst{m}ákjas ęggjun, &
an þea \alst{s}wíðaron half \hld\ \alst{s}werdu gi·málod: &
þiu \alst{h}lust warð imu far·\alst{h}awan, \hld\ hé warð an þat \alst{h}ôvid wund, &
þat imu \alst{h}eru-drôrag \hld\ \alst{h}lear ęndi ôre &
\alst{b}ęni-wundun \alst{b}rast: \hld\ \alst{b}lód aftar sprang, &
\alst{w}ell fan \alst{w}undun. \hld\ Þó was an is \alst{w}angun skard &
þe \alst{f}uristo þero \alst{f}íundo. \hld\ Þó stód þat \alst{f}olk an rúm: &
an·drédun im þes \alst{b}illes \alst{b}iti. \hld\ Þó sprak þat \alst{b}arn godes &
\alst{s}elvo te \alst{S}ímon Petruse, \hld\ hét þat hé is \alst{s}werd dedi &
\alst{sk}arp an \alst{sk}êðja: \hld\ „ef ik wið þesa \alst{sk}ola weldi“, kwað hé, &
„wið þeses \alst{w}erodes ge·\alst{w}in \hld\ \alst{w}íg-saka frummjen, &
þan \alst{m}anodi ik þene \alst{m}árjon \hld\ \alst{m}ahtigne god, &
\alst{h}êlagne fader \hld\ an \alst{h}imil-ríkja, &
þat hé mí só managan \alst{ę}ngil herod \hld\ \alst{o}vana sandi &
\alst{w}íges só \alst{w}ísen, \hld\ só ni mahtin iro \alst{w}ápạn-þręki &
\alst{m}an a·dôgjan: \hld\ iro ni stódi gio su·lik \alst{m}ęgin samad, &
\alst{f}olkes gi·\alst{f}astnod, \hld\ þat im iro \alst{f}erh aftar þiu &
\alst{w}erðen mahti. \hld\ Ak it havad \alst{w}aldand god, &
\alst{a}lo-mahtig fader \hld\ an \alst{ȯ}ðar gi·markot, &
þat wí gi·\alst{þ}olojan skulun, \hld\ só hwat só u̇s þius \alst{þ}ioda tó &
\alst{b}ittres \alst{b}rengit: \hld\ ni skulun u̇s \alst{b}elgan wiht, &
\alst{w}rêðjan wið iro ge·\alst{w}inne; \hld\ hwand só hwe só \alst{w}ápno níð, &
\alst{g}rimman \alst{g}êr-hęti wili \hld\ \alst{g}erno frummjen, &
hé \alst{s}wiltit imu \hld\ eft \alst{s}werdes ęggjun, &
\alst{d}óit im bi·\alst{d}rôregan: \hld\ wí mid u̇sun \alst{d}ádjun ni skulun &
\alst{w}iht a·\alst{w}ęrdjan.“ \hld\ Géng hé þó te þemu \alst{w}undon manne, &
\alst{l}ęgde mid \alst{l}istjun \hld\ \alst{l}ík te·samne, &
\alst{h}ôvid-wundon, \hld\ þat siu sán gi·\alst{h}êlid warð, &
þes \alst{b}illes \alst{b}iti, \hld\ ęndi sprak þat \alst{b}arn godes &
wið þat \alst{w}rêðe \alst{w}erod: \hld\ „mí þunkid \alst{w}undẹr mikil“, kwað hé, &
„ef gí mí \alst{l}êðes wiht \hld\ \alst{l}êstjen weldun, &
hwí gí mí þó ni \alst{f}éngun, \hld\ þan ik undar iuwomu \alst{f}olke stód, &
an þemu \alst{w}íhe innan \hld\ ęndi þár \alst{w}ord manag &
\alst{s}ȯð-lík \alst{s}agde. \hld\ Þan was \alst{s}unnon skín, &
\alst{d}iur-lik \alst{d}ages lioht, \hld\ þan ni weldun gí mí \alst{d}óan eo·wiht &
\alst{l}êðes an þesumu \alst{l}iohte, \hld\ ęndi nú lêdjad mí iuwa \alst{l}iudi tó &
an \alst{þ}iustrje naht, \hld\ al só man \alst{þ}iove dót, &
þan man þene \alst{f}ȧhan wili \hld\ ęndi hé is \alst{f}erhes havad &
far·\alst{w}erkot, \alst{w}am-skaðo.“ \hld\ \alst{w}erod Judeono &
\alst{g}ripun þó an þene \alst{g}odes sunu, \hld\ \alst{g}rimma þioda, &
\alst{h}atandjero \alst{h}óp, \hld\ \alst{h}wurvun ina umbi &
\alst{m}ódag \alst{m}anno folk \hld\ —\alst{m}ênes ni sáhun—, &
\alst{h}ęftun \alst{h}eru-bęndjun \hld\ \alst{h}andi te·samne, &
\alst{f}aðmos mid \alst{f}iterjun. \hld\ Im ni was su·likaro \alst{f}irin-kwála &
\alst{þ}arf te gi·\alst{þ}olonne, \hld\ \alst{þ}iod-arvêdjes, &
te \alst{w}innanne su·lik \alst{w}íti, \hld\ ak hé it þurh þit \alst{w}erod deda, &
hwand hé \alst{l}iudjo barn \hld\ \alst{l}ôsjen welda, &
\alst{h}alon fan \alst{h}ęllju \hld\ an \alst{h}imil-ríki, &
an þene \alst{w}ídon \alst{w}elon: \hld\ be·þiu hé þes \alst{w}iht ne bi·sprak, &
þes sie imu þurh \alst{i}n-wid-níð \hld\ \alst{ó}gjan weldun.\eva%TODO: ógjan or ôgjan?

\bvb TODO.\evb\evg

\bvg\bva[59][4926]%
Þó wurðun þes só \alst{m}alske \hld\ \alst{m}ódag folk Judeono, &
þiu \alst{h}êri warð þes só \alst{h}rómeg, \hld\ þes sie þena \alst{h}êlagon Krist &
an \alst{l}iðo-bęndjon \hld\ \alst{l}êdjan muostun, &
\alst{f}órjan an \alst{f}iterjun. \hld\ Þie \alst{f}íund eft ge·witun &
fan þemu \alst{b}erge te \alst{b}urg. \hld\ Géng þat \alst{b}arn godes &
undar þemu \alst{h}ęri-skępi \hld\ \alst{h}andun ge·bunden, &
\alst{d}rúvondi te \alst{d}ale. \hld\ Wárun imu þea is \alst{d}iurjon þó &
ge·\alst{s}ïðos ge·\alst{s}wikane, \hld\ al só hé im êr \alst{s}elvo gi·sprak: &
ni was it þoh be ênigaru \alst{b}lóði, \hld\ þat sie þat \alst{b}arn godes, &
\alst{l}ioven far·\alst{l}étun, \hld\ ak it was só \alst{l}ango bi·foren &
\alst{w}ár-sagono \alst{w}ord, \hld\ þat it skoldi gi·\alst{w}erðen só: &
be·þiu ni \alst{m}ahtun sie is be·\alst{m}íðan. \hld\ Þan aftar þeru \alst{m}ęnegi géngun &
\alst{J}ohannes ęndi Petrus, \hld\ þie \alst{g}umon twêne, &
\alst{f}olgodun \alst{f}errane: \hld\ was im \alst{f}iri-wit mikil, &
hwat þea \alst{g}rimmon \alst{J}udeon \hld\ þemu \alst{g}odes barne, &
weldin iro \alst{d}rohtine \alst{d}óen. \hld\ Þó sie te \alst{d}ale kwámun &
fan þemu \alst{b}erge te \alst{b}urg, \hld\ þár iro \alst{b}iskop was, &
iro \alst{w}íhes \alst{w}ard, \hld\ þár lêddun ina \alst{w}lanke man, &
\alst{e}rlos undar \alst{e}deros. \hld\ Þár was \alst{ê}ld mikil, &
\alst{f}iur an \alst{f}ríd-hove \hld\ þemu \alst{f}olke te·gęgnes, &
ge·\alst{w}arht for þemu \alst{w}erode: \hld\ þár géngun sie im \alst{w}ęrmjen tó, &
\alst{J}udeo liudi, \hld\ létun þene \alst{g}odes sunu &
\alst{b}ídon an \alst{b}ęndjun. \hld\ Was þár \alst{b}raht mikil, &
\alst{g}êl-módigaro \alst{g}alm. \hld\ \alst{J}ohannes was êr &
þemu \alst{h}êroston ku̇ð: \hld\ be·þiu móste hé an þene \alst{h}of innan &
\alst{þ}ringan mid þeru \alst{þ}ioda. \hld\ Stód allaro \alst{þ}egno bętsto, &
\alst{P}etrus þár úte: \hld\ ni lét ina þe \alst{p}ortun ward &
\alst{f}olgon is \alst{f}rôen, \hld\ êr it at is \alst{f}riunde a·bad, &
\alst{J}ohannes at ênumu \alst{J}udeon, \hld\ þat man ina \alst{g}angan lét &
\alst{f}orð an þene \alst{f}ríd-hof. \hld\ Þár kwam im ên \alst{f}êkni wíf &
\alst{g}angan te·\alst{g}ęgnes, \hld\ þiu ênas \alst{J}udeon was, &
iro \alst{þ}eodanes \alst{þ}iw, \hld\ ęndi þó te þemu \alst{þ}egne sprak &
\alst{m}agað un·wán-lík: \hld\ „Hwat þú mahtis \alst{m}an wesan“, kwað siu, &
„\alst{j}ungaro fan \alst{G}alilea, \hld\ þes þe þár \alst{g}enower stéd &
\alst{f}aðmun gi·\alst{f}astnod.“ \hld\ Þó an \alst{f}orhtun warð &
\alst{S}ímon Petrus \alst{s}án, \hld\ \alst{s}lak an is móde, &
kwað þat hé þes \alst{w}íves \hld\ \alst{w}ord ni bi·konsti &
ni þes \alst{þ}eodanes \hld\ \alst{þ}egạn ni wári: &
\alst{m}êð is þó for þeru \alst{m}ęnegi, \hld\ kwað þat hé þena \alst{m}an ni ant·kęndi: &
„ni sind mí þíne \alst{k}widi \alst{k}u̇ðe“, \hld[kwað hé;] was imu þiu \alst{k}raft godes, &
þe \alst{h}ęrdislo fan þemu \alst{h}ertan. \hld\ \alst{H}warạvondi géng &
\alst{f}orð undar þemu \alst{f}olke, \hld\ ant-tat hé te þemu \alst{f}iure kwam; &
gi·\alst{w}êt ina þó \alst{w}armjen. \hld\ Þár im ôk ên \alst{w}íf bi·gan &
\alst{f}ęlgjan \alst{f}irin-spráka: \hld\ „hér mugun gí“, kwað siu, „an iuwan \alst{f}íund sehan: &
þit is \alst{g}egnungo \hld\ \alst{j}ungaro Kristes, &
is \alst{s}elves ge·\alst{s}ïð.“ \hld\ Þó géngun imu \alst{s}án aftar þiu &
\alst{n}áhor \alst{n}íð-hwata \hld\ ęndi ina \alst{n}iud-líko &
\alst{f}rágodun \alst{f}íundo barn, \hld\ hwi-likes hé \alst{f}olkes wári: &
“ni bist þú þesoro \alst{b}urg-liudjo“, \hld[kwáðun sie;] „þat mugun wí an þínumu gi·\alst{b}árje gi·sehan, &
an þínun \alst{w}ordun ęndi an þínaru \alst{w}íson, \hld\ þat þú þeses \alst{w}erodes ni bist, &
ak þú bist \alst{g}aliléisk man.“ \hld\ hé ni welda þes þó \alst{g}ehan eo·wiht, &
ak \alst{st}ód þó ęndi \alst{st}rídda \hld\ ęndi \alst{st}arkan êð &
\alst{s}wíð-líko ge·\alst{s}wór, \hld\ þat hé þes ge·\alst{s}ïðes ni wári. &
Ni habda is \alst{w}ordo ge·\alst{w}ald: \hld\ it skolde gi·\alst{w}erðen só, &
só it þe ge·\alst{m}arkode, \hld\ þe \alst{m}an-kunnjes &
far·\alst{w}ardot an þesaru \alst{w}er-oldi. \hld\ Þó kwam imu ôk an þemu \alst{w}arve tó &
þes \alst{m}annes \alst{m}ág-wini, \hld\ þe hé êr mid is \alst{m}ákjo gi·héw, &
\alst{s}werdu þiu skarpon, \hld\ kwað þat hé ina \alst{s}áhi þár &
an þemu \alst{b}erge uppan, \hld\ „þár wí an þemu \alst{b}ôm-gardon &
\alst{h}êrron þínumu \hld\ \alst{h}ęndi bundun, &
\alst{f}astnodun is \alst{f}olmos.“ \hld\ Hé þó þurh \alst{f}orhtan hugi &
for·\alst{l}ôgnide þes is \alst{l}ioves hêrron, \hld\ kwað þat hé weldi wesan þes \alst{l}íves skolo, &
ef it mahti \alst{ê}nig þár \hld\ \alst{i}rmin-manno &
gi·\alst{s}ęggjan te \alst{s}ȯðan, \hld\ þat hé þes ge·\alst{s}ïðes wári, &
\alst{f}olgodi þeru \alst{f}ęrdi. \hld\ Þó warð an þena \alst{f}ormon sïð &
\alst{h}ano-krád af·\alst{h}aven. \hld\ Þó sah þe \alst{h}êlago Krist, &
\alst{b}arno þat \alst{b}ętste, \hld\ þár hé ge·\alst{b}unden stóð, &
\alst{s}elvo te \alst{S}ímon Petruse, \hld\ \alst{s}unu drohtines &
te þemu \alst{e}rle ovar is \alst{a}hsla. \hld\ Þó warð imu an \alst{i}nnan sán, &
\alst{S}ímon Petruse \hld\ \alst{s}êr an is móde, &
\alst{h}arm an is \alst{h}ertan \hld\ ęndi is \alst{h}ugi dróvi, &
\alst{s}wíðo warð imu an \alst{s}orgun, \hld\ þat hé êr \alst{s}elvo ge·sprak: &
gi·hugde þero \alst{w}ordo þó, \hld\ þe imu êr \alst{w}aldand Krist &
\alst{s}elvo \alst{s}agda, \hld\ þat hé an þeru \alst{s}wartan naht &
êr \alst{h}ano-krádi \hld\ is \alst{h}êrron skoldi &
\alst{þ}ríwo far·lôgnjen. \hld\ Þes \alst{þ}ram imu an innan mód &
\alst{b}ittro an is \alst{b}reostun, \hld\ ęndi géng imu þó gi·\alst{b}olgan þanen &
þe \alst{m}an fan þeru \alst{m}ęnigi \hld\ an \alst{m}ód-karu, &
\alst{s}wíðo an \alst{s}orgun, \hld\ ęndi is \alst{s}elves word, &
\alst{w}am-skęfti \alst{w}eop, \hld\ ant-tat imu \alst{w}allan kwámun &
þurh þea \alst{h}ert-kara \hld\ \alst{h}ête trahni, &
\alst{b}lódage fan is \alst{b}reostun. \hld\ Hé ni wánde þat hé is mahti gi·\alst{b}ótjen wiht, &
\alst{f}irin-werko \alst{f}urður \hld\ efþa te is \alst{f}râhon kuman, &
\alst{h}êrron \alst{h}uldi: \hld\ nis ênig \alst{h}ęliðo só ald, &
þat io \alst{m}annes sunu \hld\ \alst{m}êr gi·sáhi &
is \alst{s}elves word \hld\ \alst{s}êrur hreuwan, &
\alst{k}aron efþa \alst{k}úmjen: \hld\ „Wola \alst{k}rafteg god“, kwað hé, &
þat ik hębbju mí só for·\alst{w}erkot, \hld\ só ik mínaro \alst{w}er-oldes ni þarf &
\alst{ó}-lát sęggjan. \hld\ Ef ik nú te \alst{a}ldre skal &
\alst{h}uldjo þínaro \hld\ ęndi \alst{h}evan-ríkjas, &
\alst{þ}eoden, \alst{þ}olojan, \hld\ þan ni þarf mí þes ênig \alst{þ}ank wesan, &
\alst{l}iovo drohtin, \hld\ þat ik io te þesumu \alst{l}iohte kwam. &
Ni bium ik nú þes \alst{w}irðig, \hld\ \alst{w}aldand frô mín, &
þat ik under þíne \alst{j}ungaron \hld\ \alst{g}angan móti, &
þus \alst{s}undig under þíne ge·\alst{s}ïðos: \hld\ ik iro \alst{s}elvo skal &
\alst{m}íðan an mínumu \alst{m}óde, \hld\ nú ik mí su·lik \alst{m}ên ge·sprak.“ &
Só \alst{g}ornode \hld\ \alst{g}umono bętsta, &
\alst{h}rau im só \alst{h}ardo, \hld\ þat hé habde is \alst{h}êrren þó &
\alst{l}eoves far·\alst{l}ôgnid. \hld\ Þan ni þurvun þes \alst{l}iudjo barn, &
\alst{w}eros \alst{w}undrojan, \hld\ be·hwí it \alst{w}eldi god, &
þat só \alst{l}ioven man \hld\ \alst{l}êð gi·stódi, &
þat hé só \alst{h}ôn-líko \hld\ \alst{h}êrron sínes &
þurh þera \alst{þ}iwun word, \hld\ \alst{þ}egno snellost, &
far·\alst{l}ôgnide só \alst{l}ioves: \hld\ it was al bi þesun \alst{l}iudjun gi·duan, &
\alst{f}iriho barnun te \alst{f}rumu. \hld\ Hé welde ina te \alst{f}uriston dóan, &
\alst{h}êrost ovar is \alst{h}íwiski, \hld\ \alst{h}êlag drohtin: &
lét ina ge·\alst{k}unnon, \hld\ hwi-like \alst{k}raft havet &
þe \alst{m}ęnniska \alst{m}ód \hld\ áno þe \alst{m}aht godes; &
lét ina ge·\alst{s}undjon, \hld\ þat hé \alst{s}ïðor þiu bet &
\alst{l}iudjun gi·\alst{l}ôvdi, \hld\ hwó \alst{l}iof is þár &
\alst{m}anno gi·hwi-likumu, \hld\ þan hé \alst{m}ên ge·frumit, &
þat man ina a·\alst{l}áte \hld\ \alst{l}êðes þinges, &
\alst{s}akono ęndi \alst{s}undjono, \hld\ só im þó \alst{s}elvo dede &
\alst{h}evan-ríki god \hld\ \alst{h}arm-ge·wurhti.\eva

\bvb TODO.\evb\evg

\bvg\bva[60][5040]%
Be þiu nis \alst{m}annes bág \hld\ \alst{m}ikilun bi·þęrvi, &
\alst{h}agu-staldes \alst{h}róm: \hld\ ef imu þiu \alst{h}elpe godes &
ge·\alst{s}wíkid þurh is \alst{s}undjon, \hld\ þan is imu \alst{s}án aftar þiu &
\alst{b}reost-hugi \alst{b}lóðora, \hld\ þoh hé êr \alst{b}i-hêt spreka, &
\alst{h}rómje fan is \alst{h}ildi \hld\ ęndi fan is \alst{h}and-krafti, &
þe \alst{m}an fan is \alst{m}ęgine. \hld\ Þat warð þár an þemu \alst{m}árjon skín, &
\alst{þ}egno bętston, \hld\ þó imu is \alst{þ}iodanes gi·swêk &
\alst{h}êlag \alst{h}elpe. \hld\ Be·þiu ni skoldi \alst{h}rómjen man &
te \alst{s}wíðo fan imu \alst{s}elvon, \hld\ hwand imu þár \alst{s}wíkid oft &
\alst{w}án ęndi \alst{w}illjo, \hld\ ef imu \alst{w}aldand god, &
\alst{h}êr \alst{h}evan-kuning \hld\ \alst{h}erte ni stęrkit. &
Þan bêd allaro \alst{b}arno \alst{b}ętst, \hld\ \alst{b}ęndi þolode &
þurh \alst{m}an-kunni. \hld\ Hwurvun ina \alst{m}anaga umbi &
\alst{J}udeono liudi, \hld\ sprákun \alst{g}elp mikil, &
\alst{h}abdun ina te \alst{h}oska, \hld\ þár hé gi·\alst{h}ęftid stód, &
\alst{þ}olode mid ge·\alst{þ}uldjun, \hld\ só hwat só imu þiu \alst{þ}iod deda, &
\alst{l}iudi \alst{l}êðes. \hld\ Þó warð eft \alst{l}ioht kuman, &
\alst{m}organ te \alst{m}annun. \hld\ \alst{M}anag samnoda &
\alst{h}ęri Judeono: \hld\ habdun im \alst{h}ugi wulvo, &
\alst{i}n-wid an \alst{i}nnan. \hld\ Warð þár \alst{ê}o-sago &
an \alst{m}organ-tíd \hld\ \alst{m}anag gi·samnod &
\alst{i}rri ęndi \alst{ê}n-hard, \hld\ \alst{i}n-widjas gern, &
\alst{w}rêðes \alst{w}illjan. \hld\ Géngun im an \alst{w}arf samad &
\alst{r}inkos an \alst{r}úna, \hld\ bi·gunnun im \alst{r}ádan þó, &
hwó sie ge·\alst{w}ísadin \hld\ mid \alst{w}ár-lôsun, &
\alst{m}annun \alst{m}ên-ge·witun \hld\ an \alst{m}ahtigna Krist &
te gi·\alst{s}ęggjanne \alst{s}undja \hld\ þurh is \alst{s}elves word, &
þat sie ina þan te \alst{w}undẹr-kwálu \hld\ \alst{w}êgjan móstin, &
a·\alst{d}êljen te \alst{d}ôðe. \hld\ Sie ni mahtun an þemu \alst{d}age finden &
só \alst{w}rêð ge·\alst{w}it-skępi, \hld\ þat sie imu \alst{w}íti be·þiu &
a·\alst{d}êljen gi·\alst{d}orstin \hld\ efþa \alst{d}ôð frummjen, &
\alst{l}ívu bi·\alst{l}ôsjen. \hld\ Þó kwámun þár at \alst{l}atstan forð &
an þena \alst{w}arf \alst{w}ero \hld\ \alst{w}ár-lôse man &
\alst{t}wêne gangan \hld\ ęndi bi·gunnun im \alst{t}ęlljen an, &
kwáðun þat sie ina \alst{s}elvon \hld\ \alst{s}ęggjan gi·hôrdin, &
þat hé mahti te·\alst{w}erpen \hld\ þena \alst{w}íh godes, &
allaro \alst{h}úso \alst{h}ôhost \hld\ ęndi þurh is \alst{h}and-męgin, &
þurh is \alst{ê}nes kraft \hld\ \alst{u}p a·rihtjen &
an \alst{þ}riddjon daga, \hld\ só is elkor ni þorfti be·\alst{þ}íhan man. &
Hé \alst{þ}agoda ęndi \alst{þ}oloda: \hld\ ni sprak imu io þiu \alst{þ}iod só filu, &
þea \alst{l}iudi mid \alst{l}uginun, \hld\ þat hé it mid \alst{l}êðun an·gęgin &
\alst{w}ordun \alst{w}ráki. \hld\ Þó þár undar þemu \alst{w}erode a·rês &
\alst{b}alu-hugdig man, \hld\ \alst{b}iskop þero liudjo, &
þe \alst{f}uristo þes \alst{f}olkes \hld\ ęndi \alst{f}rágode Krist &
iak ina be imu \alst{s}elvon bi·\alst{s}wór \hld\ \alst{s}wíðon êðun, &
\alst{g}rótte ina an \alst{g}odes namon \hld\ ęndi \alst{g}erno bad, &
þat hé im þat gi·\alst{s}agdi, \hld\ ef hé \alst{s}unu wári &
þes \alst{l}ibbjendjes godes: \hld\ „þes þit \alst{l}ioht ge·skóp, &
\alst{K}rist \alst{k}uning êwig. \hld\ Wí ni mugun is ant·\alst{k}ięnnjen wiht &
ne an þínun \alst{w}ordun ni an þínun \alst{w}erkun.“ \hld\ Þó sprak imu eft þe \alst{w}áro an·gęgin, &
þe \alst{g}ódo \alst{g}odes sunu: \hld\ „þú kwiðis it for þesun \alst{J}udeon nú, &
\alst{s}ȯð-líko \alst{s}ęgis, \hld\ þat ik it \alst{s}elvo bium. &
Þes ni gi·\alst{l}ôvjad mí þese \alst{l}iudi: \hld\ ni willjad mí for·\alst{l}átan be·þiu; &
ni sind im mín \alst{w}ord \alst{w}irðig. \hld\ Nú sęggju ik iu te \alst{w}árun þoh, &
þat gí noh skulun \alst{s}ittjen gi·\alst{s}ehan \hld\ an þe \alst{s}wíðaron half godes &
\alst{m}árjan \alst{m}annes sunu, \hld\ an \alst{m}ęgin-krafte &
þes \alst{a}lo-walden fader, \hld\ ęndi þanan \alst{e}ft kuman &
an \alst{h}imil-wolknun \alst{h}erod \hld\ ęndi allumu \alst{h}ęliðo kunnje &
mid is \alst{w}ordun a·dêljen, \hld\ al só iro ge·\alst{w}urhti sind.“ &
Þo \alst{b}alg ina þe \alst{b}iskop, \hld\ habde \alst{b}ittren hugi, &
\alst{w}rêðida wið þemu \alst{w}orde \hld\ ęndi is gi·\alst{w}ádi slêt, &
\alst{b}rak for is \alst{b}reostun: \hld\ „Nú ni þurvun gí \alst{b}ídan lęng“, kwað hé, &
„þit \alst{w}erod ge·\alst{w}it-skępjes, \hld\ nú im su·lik \alst{w}ord farad, &
\alst{m}ên-spráka fan is \alst{m}u̇ðe. \hld\ Þat gi·hôrid hér nú \alst{m}anno filu, &
\alst{r}inko an þesumu \alst{r}akude, \hld\ þat hé ina só \alst{r}íkjan telit, &
\alst{g}ihid þat hé \alst{g}od sí. \hld\ Hwat willjad gí \alst{J}udeon þes &
a·\alst{d}êljen te \alst{d}óme? \hld\ Is hé \alst{d}ôðes nú &
\alst{w}irðig be su·likun \alst{w}ordun?“\eva

\bvb TODO.\evb\evg

\bvg\bva[61][5107]%
\hspace*{100pt} Þat \alst{w}erod al ge·sprak, &
\alst{f}olk Judeono, \hld\ þat hé wári þes \alst{f}erhes skolo, &
\alst{w}ítjes só \alst{w}irðig. \hld\ Ni was it þoh be is ge·\alst{w}urhtjun gi·dóen, &
þat ine þár an \alst{J}erusalem \hld\ \alst{J}udeo liudi, &
\alst{s}unu drohtines \hld\ \alst{s}undja lôsen &
a·\alst{d}êldun te \alst{d}ôðe. \hld\ Þó was þero \alst{d}ádjo hróm &
\alst{J}udeo liudjun, \hld\ hwat sie þemu \alst{g}odes barne mahtin &
só \alst{h}aftemu mêst, \hld\ \alst{h}armes ge·frummjen. &
Be·\alst{w}urpun ina þó mid \alst{w}erodu \hld\ ęndi ina an is \alst{w}angon slógun, &
an is \alst{h}leor mid iro \alst{h}andun \hld\ —al was imu þat te \alst{h}oske gi·dóen—, &
\alst{f}ęlgidun imu \alst{f}irin-word \hld\ \alst{f}íundo męnegi, &
\alst{b}ismer-spráka. \hld\ Stód þat \alst{b}arn godes &
\alst{f}ast under \alst{f}íundun: \hld\ wárun imu is \alst{f}aðmos ge·bundene, &
\alst{þ}olode mid gi·\alst{þ}uldjun, \hld\ só hwat só imu þiu \alst{þ}ioda tó &
\alst{b}ittres \alst{b}ráhte: \hld\ ni \alst{b}alg ina n·eo·wiht &
wið þes \alst{w}erodes ge·\alst{w}in. \hld\ Þó námon ina \alst{w}rêðe man &
só gi·\alst{b}undanan, \hld\ þat \alst{b}arn godes, &
ęndi ina þó \alst{l}êddun, \hld\ þár þero \alst{l}iudjo was, &
þere \alst{þ}iade \alst{þ}ing-hús. \hld\ Þár \alst{þ}egạn manag &
\alst{h}wurvun umbi iro \alst{h}ęri-togon. \hld\ Þár was iro \alst{h}êrron bodo &
fan \alst{R}úmu-burg, \hld\ þes þe þó þes \alst{r}íkjas gi·weld: &
\alst{k}umen was hé fan þemu \alst{k}êsure, \hld\ gi·sęndid was hé undar þat \alst{k}unni Judeono &
te \alst{r}ihtjenne þat \alst{r}íki, \hld\ was þár \alst{r}ád-gevo: &
\alst{P}ilatus was hé hêten; \hld\ hé was fan \alst{P}onteo lande &
\alst{k}nósles \alst{k}ęnnit. \hld\ Habde imu \alst{k}raft mikil, &
an þemu \alst{þ}ing-húse \hld\ \alst{þ}iod gi·samnod, &
an \alst{w}arf \alst{w}eros; \hld\ \alst{w}ár-lôse man &
a·\alst{g}ávun þó þena \alst{g}odes sunu, \hld\ \alst{J}udeo liudi, &
under \alst{f}íundo \alst{f}olk, \hld\ kwáðun þat hé wári þes \alst{f}erhes skolo, &
þat man ina \alst{w}ítnodi \hld\ \alst{w}ápnes ęggjun, &
\alst{sk}arpun \alst{sk}úrun. \hld\ Ni welde þiu \alst{sk}ole Judeono &
\alst{þ}ringan an þat \alst{þ}ing-hús, \hld\ ak þiu \alst{þ}iod úte stód, &
\alst{m}ahlidun þanen wið þea \alst{m}ęnegi: \hld\ ni weldun an þat gi·\alst{m}ang faren, &
an \alst{ę}li-landige man, \hld\ þat sie þár \alst{u}n-reht word, &
an þemu \alst{d}age \alst{d}ęrvjes wiht \hld\ a·\alst{d}êljan ne gi·hôrdin, &
ak kwáðun þat sie im só \alst{h}luttro \hld\ \alst{h}êlaga tídi, &
weldin iro \alst{p}askha halden. \hld\ \alst{P}ilatus ant·féng &
at þem \alst{w}am-skaðun \hld\ \alst{w}aldandes barn, &
\alst{s}undja lôsen. \hld\ Þó an \alst{s}orgun warð &
\alst{J}udases hugi, \hld\ þó hé a·\alst{g}evan gi·sah &
is \alst{d}rohtin te \alst{d}ôðe, \hld\ þó bi·gan imu þiu \alst{d}ád aftar þiu &
an is \alst{h}ugja \alst{h}reuwan, \hld\ þat hé habde is \alst{h}êrron êr &
\alst{s}undja lôsen gi·\alst{s}ald. \hld\ Nam imu þó þat \alst{s}ilụvar an hand, &
\alst{þ}rí-tig skatto, \hld\ þat man imu êr wið is \alst{þ}iodane gaf, &
\alst{g}éng imu þó te þem \alst{J}udiun \hld\ ęndi im is \alst{g}rimmon dád, &
\alst{s}undjon \alst{s}agde, \hld\ ęndi im þat \alst{s}ilụvar bôd &
\alst{g}erno te a·\alst{g}evanne: \hld\ „ik hębbju it só \alst{g}rio-líko“, kwað hé, &
„mínes \alst{d}rohtines \hld\ \alst{d}rôru gi·kôpot, &
só ik wêt þat it mí ni \alst{þ}íhit.“ \hld\ \alst{Þ}iod Judeono &
ni weldun it þó ant·\alst{f}ȧhan, \hld\ ak hétun ina \alst{f}orð aftar þiu &
umbi \alst{s}u·lika \alst{s}undja \hld\ \alst{s}elvon ahton, &
hwat hé wið is \alst{f}râhon \hld\ ge·\alst{f}rumid habdi: &
„Þú \alst{s}áhi þi \alst{s}elvo þes“, \hld[kwáðun sie;] „hwat wili þú þes nú \alst{s}óken te u̇s? &
Ne \alst{w}ít þú þat þesumu \alst{w}erode!“ \hld\ Þó gi·\alst{w}êt imu eft þanan &
\alst{J}udas \alst{g}angan \hld\ te þemu \alst{g}odes wíhe &
\alst{s}wíðo an \alst{s}orgun \hld\ ęndi þat \alst{s}ilụvar warp &
an þena \alst{a}lạh innan, \hld\ ne gi·dorste it \alst{ê}gan lęng; &
\alst{f}ór imu þó só an \alst{f}orhtun, \hld\ só ina \alst{f}íundo barn &
\alst{m}ódage \alst{m}anodun: \hld\ habdun þes \alst{m}annes hugi &
\alst{g}ramon under·\alst{g}ripanen, \hld\ was imu \alst{g}od a·bolgan, &
þat hé imu \alst{s}elvon þó \hld\ \alst{s}ímon warhte, &
\alst{h}nêg þó an \alst{h}eru-sêl \hld\ an \alst{h}inginna, &
\alst{w}arạg an \alst{w}urgil \hld\ ęndi \alst{w}íti ge·kôs, &
\alst{h}ard \alst{h}ęllje ge·þwing, \hld\ \alst{h}êt ęndi þiustri, &
\alst{d}iap \alst{d}ôðes \alst{d}alu, \hld\ hwand hé êr umbi is \alst{d}rohtin swêk.\eva

\bvb TODO.\evb\evg

\bvg\bva[62][5172]%
Þan \alst{b}êd þat \alst{b}arn godes \hld\ —\alst{b}ęndi þolode &
an \alst{þ}emu \alst{þ}ing-húse—, \hld\ hwan êr þiu \alst{þ}iod under im, &
\alst{e}rlos \alst{ê}n-wordje \hld\ \alst{a}lle wurðin, &
hwat sie imu þan te \alst{f}erạh-kwálu \hld\ \alst{f}rummjan weldin. &
Þó þár an þem \alst{b}ęnkjun a·rês \hld\ \alst{b}odo kêsures &
fan \alst{R}úmu-burg \hld\ ęndi géng imu wið þat \alst{r}íki Judeono &
\alst{m}ódag \alst{m}ahljen, \hld\ þár þiu \alst{m}ęnigi stód &
aftar þemu \alst{h}ove \alst{h}warvon: \hld\ ni weldun an þat \alst{h}ús kuman &
an þemu \alst{p}askha-dage. \hld\ \alst{P}ilatus bi·gan &
\alst{f}rókno \alst{f}rágon \hld\ ovar þat \alst{f}olk Judeono, &
mid hwiu þe \alst{m}an habdi \hld\ \alst{m}orðes gi·skuldit, &
\alst{w}ítjes gi·\alst{w}erkot: \hld\ „be hwí gí imu só \alst{w}rêðe sind, &
an iuwomu \alst{h}ugja \alst{h}ótje?“ \hld\ Sie kwáðun þat hé im habdi \alst{h}armes só filu, &
\alst{l}êðes gi·\alst{l}êstid: \hld\ „ni gávin ina þesa \alst{l}iudi þi, &
þár sie ina \alst{ê}r bi·foran \hld\ \alst{u}vilan ni wissin, &
\alst{w}ordun far·\alst{w}arhten. \hld\ Hé havat þeses \alst{w}erodes só filu &
far·\alst{l}êdid mid is \alst{l}êrun \hld\ —ęndi þesa \alst{l}iudi męrrid, &
dóit im iro \alst{h}ugi twífljen—, \hld\ þat wí ni mótun te þemu \alst{h}ove kêsures &
\alst{t}insi gelden; \hld\ þat mugun wí ina gi·\alst{t}ęlljen an &
mid \alst{w}áru ge·\alst{w}it-skępi. \hld\ Hé sprikid ôk \alst{w}ord mikil, &
\alst{k}wiðit þat hé \alst{K}rist sí, \hld\ \alst{k}uning ovar þit ríki, &
be·\alst{g}ihit ina só \alst{g}rôtes.“ \hld\ Þó im eft te·\alst{g}ęgnes sprak &
\alst{b}odo kêsures: \hld\ „ef hé só \alst{b}ar-líko“, kwað hé, &
„under þesaru \alst{m}ęnigi \hld\ \alst{m}ên-werk frumid, &
ant·\alst{f}ȧhad ina þan eft under iuwe \alst{f}olk-skępi, \hld\ ef hé sí is \alst{f}erhes skolo, &
ęndi imu só a·\alst{d}êljad, \hld\ ef hé sí \alst{d}ôðes werð, &
só it an \alst{i}uwaro \alst{a}ldrono \hld\ \alst{ê}o ge·biode.“ &
Sie kwáðun þó, þat sie ni \alst{m}óstin \hld\ \alst{m}anno nig·ênumu &
an þea \alst{h}êlagon tíd \hld\ te \alst{h}and-banon, &
\alst{w}erðen mid \alst{w}ápnun \hld\ an \alst{þ}emu wíh-dage. &
Þó \alst{w}ęnde ina fan þemu \alst{w}erode \hld\ \alst{w}rêð-hugdig man, &
\alst{þ}egạn kêsures, \hld\ þe ovar þea \alst{þ}ioda was &
\alst{b}odo fan Rúmu-burg—: \hld\ hét imu þó þat \alst{b}arn godes &
\alst{n}áhor gangan \hld\ ęndi ina \alst{n}iud-líko, &
\alst{f}rágoda \alst{f}rókno, \hld\ ef hé ovar þat \alst{f}olk kuning &
þes \alst{w}erodes \alst{w}ári. \hld\ Þó habde eft is \alst{w}ord garu &
\alst{s}unu drohtines: \hld\ „hweðer þú þat fan þi \alst{s}elvumu sprikis“, kwað hé, &
„þe it þi \alst{ȯ}ðre hér \hld\ \alst{e}rlos sagdun, &
\alst{k}wáðun umbi mínan \alst{k}uning-duom?“ \hld\ Þó sprak eft þe \alst{k}êsures bodo &
\alst{w}lank ęndi \alst{w}rêð-mód, \hld\ þár hé wið \alst{w}aldand Krist &
\alst{r}eðjode an þem \alst{r}akude: \hld\ „ni bium ik þeses \alst{r}íkjes hinan“, kwað hé, &
„\alst{J}udeo liudjo, \hld\ ni \alst{g}adoling þín, &
þesaro \alst{m}anno \alst{m}ág-wini, \hld\ ak mí þí þius \alst{m}ęnigi bi·falạh, &
a·\alst{g}ávun þí þína \alst{g}adulingos mí, \hld\ \alst{J}udeo liudi, &
\alst{h}aftan te \alst{h}andun. \hld\ Hwat havas þú \alst{h}armes gi·duan, &
þat þú só \alst{b}ittro skalt \hld\ \alst{b}ęndi þolojan, &
\alst{k}walm undar þínumu \alst{k}unnje?“ \hld\ Þó sprak imu eft \alst{K}rist an·gęgin, &
\alst{h}êlendero bętst, \hld\ þár hé gi·\alst{h}ęftid stód &
an þemu \alst{r}akude innan: \hld\ „nis mín \alst{r}íki hinan“, kwað hé, &
„fan þesaru \alst{w}er-old-stundu. \hld\ Ef it þoh \alst{w}ári só, &
þan wárin só \alst{st}ark-móde \hld\ wiðẹr \alst{st}ríd-hugi, &
wiðẹr \alst{g}rama þioda \hld\ \alst{j}ungaron míne, &
só man mí ni \alst{g}ávi \hld\ \alst{J}udeo liudjun, &
\alst{h}ęttendjun an \alst{h}and \hld\ an \alst{h}eru-bęndjun &
te \alst{w}êgjanne te \alst{w}undrun. \hld\ Te þiu warð ik an þesaru \alst{w}er-oldi gi·boran, &
þat ik ge·\alst{w}it-skępi giu \hld\ \alst{w}áres þinges &
mid mínun \alst{k}umjun \alst{k}u̇ðdi. \hld\ Þat mugun ant·\alst{k}ęnnjen wel &
þe \alst{w}eros, þe sind fan \alst{w}áre kumane: \hld\ þe mugun mín \alst{w}ord far·standen, &
gi·\alst{l}ôvjen mínun \alst{l}êrun.“ \hld\ Þó ni mahte \alst{l}asteres wiht &
an þem \alst{b}arne godes \hld\ \alst{b}odo kêsures, &
\alst{f}indan \alst{f}êknja word, \hld\ þat hé is \alst{f}erhes be·þiu &
\alst{sk}uldig wári. \hld\ Þó géng hé im eft wið þea \alst{sk}ola Judeono &
\alst{m}ódag \alst{m}ahljen \hld\ ęndi þeru \alst{m}ęnigi sagde &
ovar \alst{h}lust mikil, \hld\ þat hé an þemu \alst{h}afton manne &
su·lika \alst{f}irin-spráka \hld\ \alst{f}inden ni mahti &
for þem \alst{f}olk-skipje, \hld\ só hé wári is \alst{f}erhes skolo, &
\alst{d}ôðes wirðig. \hld\ Þan stódun \alst{d}ol-móde &
\alst{J}udeo liudi \hld\ ęndi þane \alst{g}odes sunu &
\alst{w}ordun \alst{w}rógdun: \hld\ kwáðun þat hé gi·\alst{w}er êrist &
be·\alst{g}unni an \alst{G}alileo lande, \hld\ „ęndi ovar \alst{J}udeon fór &
\alst{h}erod-wardes þanan, \hld\ \alst{h}ugi twíflode, &
\alst{m}anno \alst{m}ód-sevon, \hld\ só hé is \alst{m}orðes werð, &
þat man ina \alst{w}ítnoje \hld\ \alst{w}ápnes ęggjun, &
ef eo man mid su·likun \alst{d}ádjun mag \hld\ \alst{d}ôðes ge·skuldjen.“\eva

\bvb TODO.\evb\evg

\bvg\bva[63][5246]%
Só \alst{w}rógdun ina mid \alst{w}ordun \hld\ \alst{w}erod Judeono &
þurh \alst{h}ótjan \alst{h}ugi. \hld\ Þó þe \alst{h}ęri-togo, &
\alst{s}líð-módig man \hld\ \alst{s}ęggjan gi·hôrde, &
fan hwi-likumu \alst{k}unnje was \hld\ \alst{K}rist a·fódid, &
\alst{m}anno þe bętsto: \hld\ hé was fan þeru \alst{m}árjan þiadu, &
þe \alst{g}ódo fan \alst{G}alilea-lande; \hld\ þár was \alst{g}um-skępi &
\alst{ę}ðiljero manno; \hld\ \alst{E}rodes bi·held þár &
\alst{k}raftagne \alst{k}uning-dóm, \hld\ só ina imu þe \alst{k}êsur far·gaf, &
þe \alst{r}íkjo fan \alst{R}úmu, \hld\ þat hé þár \alst{r}ehto ge·hwi-lik &
ge·\alst{f}rumidi undar þemu \alst{f}olke \hld\ ęndi \alst{f}riðu lêsti, &
\alst{d}ómos a·\alst{d}êldi. \hld\ Hé was ôk an þemu \alst{d}age selvo &
an \alst{J}erusalem \hld\ mid is \alst{g}um-skępi, &
mid is \alst{w}erode at þemu \alst{w}íhe: \hld\ só was iro \alst{w}íse þan, &
þat sie þár þia \alst{h}êlagun tíd \hld\ \alst{h}aldan skoldun, &
\alst{p}askha Judeono. \hld\ \alst{P}ilatus gi·bôd þó, &
þat þena \alst{h}afton man \hld\ \alst{h}ęliðos námin &
só gi·\alst{b}undanan, \hld\ þat \alst{b}arn godes, &
hét þat sie ina \alst{E}rodese, \hld\ \alst{e}rlos brȧhtin &
\alst{h}aften te \alst{h}andun, \hld\ hwand hé fan is \alst{h}ęri-skępi was, &
fan is \alst{w}erodes ge·\alst{w}ald. \hld\ \alst{W}ígand frumidun &
iro \alst{h}êrron word: \hld\ \alst{h}êlagne Krist &
\alst{f}órdun an \alst{f}iterjun \hld\ for þena \alst{f}olk-togun, &
allaro \alst{b}arno \alst{b}ętst, \hld\ þero þe io gi·\alst{b}oren wurði &
an \alst{l}iudjo \alst{l}ioht; \hld\ an \alst{l}iðu-bęndjun géng, &
ant-tat sie ina \alst{b}ráhtun, \hld\ þár hé an is \alst{b}ęnkja sat, &
\alst{k}uning Erodes: \hld\ umbi·hwarf ina \alst{k}raft wero, &
\alst{w}lanke \alst{w}ígandos: \hld\ was im \alst{w}illjo mikil, &
þat sie þár \alst{s}elvon Krist \hld\ gi·\alst{s}ehan móstin: &
wándun þat hé im sum \alst{t}êkạn \hld\ þár \alst{t}ôgjan skoldi, &
\alst{m}ári ęndi \alst{m}ahtig, \hld\ só hé \alst{m}anagun dede &
þurh is \alst{g}od-kundi \hld\ \alst{J}udeo *liudjon. &
\alst{F}rágoda ina þuo þie \alst{f}olk-kuning \hld\ \alst{f}iri-wit-líko &
\alst{m}anagon wordon, \hld\ wolda is \alst{m}uod-sevon &
\alst{f}orð undar·\alst{f}indan, \hld\ hwat hie te \alst{f}rumu mohti &
\alst{m}annon gi·\alst{m}arkon. \hld\ Þan stuod \alst{m}ahtig Krist, &
\alst{þ}agoda ęndi \alst{þ}oloda: \hld\ ne wolda þem \alst{þ}ied-kuninge, &
\alst{E}rodese ne is \alst{e}rlon \hld\ \alst{a}nt-swór gevan &
\alst{w}ordo nig·ênon. \hld\ Þan stuod þiu \alst{w}rêða þiod, &
\alst{J}udeo liudi \hld\ ęndi þena \alst{g}odes suno &
\alst{w}urrun ęndi \alst{w}ruogdun, \hld\ anþat im warð þie \alst{w}er-old-kuning &
an is \alst{h}uge \alst{h}uoti \hld\ ęndi all is \alst{h}ęri-skipi, &
far·\alst{m}uonstun ina an iro \alst{m}uode: \hld\ ne ant·kęndun \alst{m}aht godes, &
\alst{h}imiliskan \alst{h}êrron, \hld\ ak was im iro \alst{h}ugi þiustri, &
\alst{b}aluwes gi·\alst{b}landan. \hld\ \alst{B}arn drohtines &
iro \alst{w}rêðun \alst{w}erk, \hld\ \alst{w}ord ęndi dádi &
þuru \alst{ô}d-muodi \hld\ \alst{a}ll gi·þoloda, &
só hwat só sia im \alst{t}ionono þuo \hld\ \alst{t}uogjan woldun. &
Sia \alst{h}ietun im þuo te \alst{h}oske \hld\ \alst{h}wít gi·wádi &
umbi is \alst{l}iði \alst{l}ęggjan, \hld\ þiu mêr hie wurði þem \alst{l}iudjon þár, &
\alst{j}ungron te \alst{g}amne. \hld\ \alst{J}udeon faganodun, &
þuo sia ina te \alst{h}oske \hld\ \alst{h}ębbjan gi·sáhun, &
\alst{e}rlos \alst{o}var-muoda. \hld\ Þuo sęnda ina \alst{e}ft þanan &
\alst{E}rodes se kuning \hld\ an þat \alst{ȯ}ðer folk; &
a·\alst{l}êdjan hiet ina \alst{l}ungra mann, \hld\ ęndi \alst{l}astar sprákun, &
\alst{f}elgidun im \alst{f}irin-word, \hld\ þár hie an \alst{f}eteron géng &
bi·\alst{h}lagan mid \alst{h}osku: \hld\ ni was im \alst{h}ugi twífli, &
neva hie it þuru \alst{ô}d-muodi \hld\ \alst{a}ll gi·þoloda; &
ne welda iro \alst{u}vilun word \hld\ \alst{i}dug-lônon, &
\alst{h}osk ęndi \alst{h}arm-kwidi. \hld\ Þuo brȧhtun sia ina eft an þat \alst{h}ús innan, &
an þia \alst{p}alenkja uppan, \hld\ þár \alst{P}ilatus was &
an þero \alst{þ}ing-stędi. \hld\ \alst{Þ}egnos a·gávun &
\alst{b}arno þat \alst{b}ęsta \hld\ \alst{b}anon te handon &
\alst{s}undi-lôsjan, \hld\ só hie \alst{s}elvo gi·kôs: &
welda \alst{m}anno barn \hld\ \alst{m}orðes a·tuomjan, &
\alst{n}ęrjan af \alst{n}ôdi. \hld\ Stuodun \alst{n}íð-hwata, &
\alst{J}udeon far þem \alst{g}ast-sęlje: \hld\ habdun sia \alst{g}ramono barn, &
þia \alst{sk}ola far·\alst{sk}undid, \hld\ þat sia ne be·\alst{sk}rivun iowiht &
\alst{g}rimmera dádjo. \hld\ Þuo gi·wêt im \alst{g}angan þarod &
\alst{þ}egạn kêsures \hld\ wið þia \alst{þ}iod sprekan, &
\alst{h}ard \alst{h}ęri-togo: \hld\ „Hwat gí mí þesan \alst{h}aftan mann“, kwaþ-hie, &
„an þesan \alst{s}ęli \alst{s}ęndun \hld\ ęndi \alst{s}elvon an·budun, &
þat hie iuwes \alst{w}erodes só filo \hld\ a·\alst{w}erdit habdi, &
far·\alst{l}êdid mid is \alst{l}êron. \hld\ Nú ik mid þeson \alst{l}iudon ni mag, &
\alst{f}indan mid þius \alst{f}olku, \hld\ þat hie is \alst{f}erạhes sí &
furi þesaro \alst{sk}olu \alst{sk}uldig. \hld\ \alst{Sk}ín was þat hiudu: &
\alst{E}rodes mohta, \hld\ þie iuwan \alst{ê}o bi·kan, &
iuwaro \alst{l}iudo \alst{l}and-reht, \hld\ hie ni mahta is \alst{l}íves gi·frêson, &
þat hie hier þuru êniga \alst{s}undja te dage \hld\ \alst{s}weltan skoldi, &
\alst{l}íf far·\alst{l}átan. \hld\ Nú willju ik ina for þeson \alst{l}iudjon hier &
gi·\alst{þ}róon mid \alst{þ}ingon, \hld\ \alst{þ}rístjon wordun, &
\alst{b}uotjan im is \alst{b}riost-hugi, \hld\ látan ina \alst{b}rúkan forð &
\alst{f}erạhes mid \alst{f}irjon.“ \hld\ \alst{F}olk Judeono &
\alst{h}reopun þuo alla samad \hld\ \alst{h}lúdero stemnu, &
hietun \alst{f}lít-líko \hld\ \alst{f}erạhes áhtjan &
\alst{K}rist mid \alst{k}walmu \hld\ ęndi an \alst{k}rúki slahan, &
\alst{w}êgjan te \alst{w}undron: \hld\ „hie mid is \alst{w}ordon havit &
\alst{d}ôðes gi·skuldid: \hld\ sagit þat hie \alst{d}rohtin sí, &
\alst{g}egnungo \alst{g}odes suno. \hld\ Þat hie a·\alst{g}eldan skal, &
\alst{i}n-wid-spráka, \hld\ só is an u̇son \alst{ê}we gi·skrivan, &
þat man su·lika \alst{f}irin-kwidi \hld\ \alst{f}erạhu kôpo.“\eva

\bvb TODO.\evb\evg

\bvg\bva[64][5336]%
Þuo warð þie an \alst{f}orạhton, \hld\ þie þes \alst{f}olkes gi·weld, &
\alst{m}ikilon an is \alst{m}uode, \hld\ þuo hie gi·hôrda þia \alst{m}an sprekan, &
þat sia ina \alst{s}elvon \hld\ \alst{s}ęggjan gi·hôrdin, &
\alst{g}ehan fur þem \alst{g}um-skipe, \hld\ þat hie wári \alst{g}odes suno. &
Þuo hwarf im eft þie \alst{h}ęri-togo \hld\ an þat \alst{h}ús innan &
te þero \alst{þ}ing-stędi, \hld\ \alst{þ}rístjon wordon &
\alst{g}ruotta þena \alst{g}odes suno \hld\ ęndi frágoda, hwat hie \alst{g}umono wári: &
„hwat bist þú \alst{m}anno?“ \hld[kwaþ-hie.] „Te hwí þú mí só þínan \alst{m}uod hilis, &
\alst{d}ęrnis \alst{d}iop-gi·þȧht? \hld\ Wêst þú þat it all an mínon \alst{d}uome stéd &%TODO: Check stéd.
umbi þínes \alst{l}íves gi·\alst{l}agu? \hld\ Mí þi hębbjat þesa \alst{l}iudi far·gevan, &
\alst{w}erod Judeono, \hld\ þat ik gi·\alst{w}aldan muot &
só þik te \alst{sp}ildjanne \hld\ an \alst{sp}eres orde, &
só ti \alst{k}węlljanne an \alst{k}rúkjum, \hld\ só \alst{k}wikan látan, &
só hweðer sí mí \alst{s}elvon \hld\ \alst{s}uotera þunkit &
te gi·\alst{f}rummjanne mid mínu \alst{f}olku.“ \hld\ Þuo sprak eft þat \alst{f}riðu-barn godes: &
„\alst{W}êst þú þat te \alst{w}áron“, \hld[kwaþ-hie,] „þat þú gi·\alst{w}ald ovar mik &
\alst{h}ębbjan ni mohtis, \hld\ ne wári þat it þí \alst{h}êlag god &
\alst{s}elvo far·gávi? \hld\ Ôk hębbjat þia \alst{s}undjono mêr, &
þia mik þi bi·\alst{f}ulhun \hld\ þuru \alst{f}íond-skipi, &
gi·\alst{s}aldun an \alst{s}ímon haftan.“ \hld\ Þuo welda ina \alst{s}ïð after þiu &
\alst{g}ram-hugdig man \hld\ \alst{g}erno far·látan, &
\alst{þ}egạn kêsures, \hld\ þár hie is havdi for þero \alst{þ}ioda gi·wald; &
ak sia \alst{w}ęridun im þena \alst{w}illjon \hld\ \alst{w}ordu gi·hwi-liku, &
\alst{k}unni Judeono: \hld\ „ne bist þú“, kwáðun sia, „þes \alst{k}êsures friund, &
þínon \alst{h}êrren \alst{h}old, \hld\ ef þú ina \alst{h}inan látis &
\alst{s}ïðon gi·\alst{s}undon: \hld\ þat þi noh te \alst{s}orạgan mag, &
\alst{w}erðan te \alst{w}íte, \hld\ hwand só hwe só su·lik \alst{w}ord sprikit, &
a·\alst{h}avið ina só \alst{h}ôho, \hld\ kwiðit þat hie \alst{h}ębbjan mugi &
\alst{k}uning-duomes namon, \hld\ ne sí þat ina im þie \alst{k}êsur geve, &
hie \alst{w}irrid im is \alst{w}er-uld-ríki \hld\ ęndi is \alst{w}ord far·hugid, &
far·\alst{m}an ina an is \alst{m}uode. \hld\ Be·þiu skalt þú su·lik \alst{m}ên wrekan, &
\alst{h}osk-word manag, \hld\ ef þú umbi þínes \alst{h}êrren ruokis, &
umbi þínes \alst{f}rôhon \alst{f}riund-skipi, \hld\ þan skalt þú ina þiu \alst{f}erhu be·niman.“ &
Þuo gi·\alst{h}ôrda þie \alst{h}ęri-togo \hld\ þia \alst{h}êri Juðeono &
\alst{þ}rêgjan fan is \alst{þ}iodne; \hld\ þuo hie far þero \alst{þ}ing-stędi géng &
\alst{s}elvo gi·\alst{s}ittjan, \hld\ þár gi·\alst{s}amnod was &
só mikil \alst{w}arf \alst{w}erodes, \hld\ hiet \alst{w}aldand Krist &
\alst{l}êdjan for þia \alst{l}iudi. \hld\ \alst{L}angoda Judeon, &
hwan êr sia þat \alst{h}êlaga barn \hld\ \alst{h}angon gi·sáwin, &
\alst{k}węlan an \alst{k}rúkje; \hld\ sia kwáðun þat sia \alst{k}uning ȯðran &
ne \alst{h}avdin undar iro \alst{h}ęri-skipje, \hld\ nevan þena \alst{h}êran kêsar &
fan \alst{R}úmu-burg: \hld\ „þie havit hier \alst{r}íki over u̇s. &
Be·þiu ni skalt þú þesan far·\alst{l}átan; \hld\ hie havit u̇s só filo \alst{l}êðes gi·sprokan, &
far·\alst{d}uan havit hie im mid is \alst{d}ádjon. \hld\ Hie skal \alst{d}ôð þolon, &
\alst{w}íti ęndi \alst{w}undạr-kwála.“ \hld\ \alst{W}erod Judeono &
só \alst{m}anag \alst{m}is-lík þing \hld\ an \alst{m}ahtigna Krist &
\alst{s}agdun te \alst{s}undjun. \hld\ Hie \alst{s}wígondi stuod &
þuru \alst{ô}ð-muodi, \hld\ ne \alst{a}nt-wordida n·io·wiht &
wið iro \alst{w}rêðun \alst{w}ord: \hld\ wolda þesa \alst{w}er-old alla &
\alst{l}ôsjan mid is \alst{l}ívu: \hld\ bi·þiu liet hie ina þia \alst{l}êðun þiod &
\alst{w}êgjan te \alst{w}undron, \hld\ all só iro \alst{w}illjo géng: &
ni wolda im \alst{o}pan-líko \hld\ \alst{a}llon ku̇ðjan &
\alst{J}udeo liudjon, \hld\ þat hie was \alst{g}od selvo; &
hwand \alst{w}issin sia þat te \alst{w}áron, \hld\ þat hie su·lika gi·\alst{w}ald havdi &
ovar þeson \alst{m}iddil-gard, \hld\ þan wurði im iro \alst{m}uod-sevo &
gi·\alst{b}lôðit an iro \alst{b}rioston: \hld\ þan ne gi·dorstin sia þat \alst{b}arn godes &
\alst{h}andon ant·\alst{h}rínan: \hld\ þan ni wurði \alst{h}evan-ríki, &
ant·\alst{l}okan \alst{l}iohto mêst \hld\ \alst{l}iudjo barnon. &
Be·þiu \alst{m}êð hie is só an is \alst{m}uode, \hld\ ne lét þat \alst{m}anno folk &
\alst{w}itan, hwat sia \alst{w}arạhtun. \hld\ Þiu \alst{w}urd náhida þuo, &
\alst{m}ári \alst{m}aht godes \hld\ ęndi \alst{m}iddi dag, &
þat sia þia \alst{f}erạh-kwála \hld\ \alst{f}rummjan skoldun. &
Þan lag þár ôk an \alst{b}ęndjon \hld\ an þero \alst{b}urg innan &
ên \alst{r}uof \alst{r}ęgin-skaðo, \hld\ þie habda under þem \alst{r}íke só filo &
\alst{m}orðes gi·rádan \hld\ ęndi \alst{m}an-slahta gi·frumid, &
was \alst{m}ári \alst{m}ęgin-þiof: \hld\ ni was þár is gi·\alst{m}ako hwęrgin; &
was þár ôk bi \alst{s}ínon \hld\ \alst{s}undjon gi·hęftid, &
\alst{B}arrabas was hie hêtan; \hld\ hie after þem \alst{b}urgjon was &
þuru is \alst{m}ên-dádi \hld\ \alst{m}anogon gi·ku̇ðid. &
Þan was \alst{l}and-wísa \hld\ \alst{l}iudjo Judeono, &
þat sia \alst{j}áro gi·hwen \hld\ an \alst{g}odes minnja &
an þem \alst{h}êlagon dage \hld\ ênna \alst{h}aftan mann &
a·\alst{b}iddjan skoldun, \hld\ þat im iro \alst{b}urges ward, &
iro \alst{f}olk-togo \hld\ \alst{f}erạh far·gávi. &
Þuo bi·gan þie \alst{h}ęri-togo \hld\ þia \alst{h}êri Judeono, &
þat \alst{f}olk \alst{f}rágojan, \hld\ þár sia im \alst{f}ora stuodun, &
hweðeron sia þero \alst{t}wejo \hld\ \alst{t}uomjan weldin, &
\alst{f}erạhes biddjan: \hld\ „þia hier an \alst{f}eteron sind &
\alst{h}aft undar þeson \alst{h}ęri-skipje?“ \hld\ Þiu \alst{h}êri Judeono &
habdun þuo þia \alst{a}rạmun man \hld\ \alst{a}lla gi·spanana, &
þat sia þemo \alst{l}and-skaðen \hld\ \alst{l}íf a·bádin, &
gi·\alst{þ}ingodin þem \alst{þ}iove, \hld\ þie oft an \alst{þ}iustrja naht &
\alst{w}am gi·\alst{w}arạhta, \hld\ ęndi \alst{w}aldand Krist &
\alst{k}węlidin an \alst{k}rúkje. \hld\ Þuo warð þat \alst{k}u̇ð ovar all, &
hwó þiu þiod havda \alst{d}uomos a·\alst{d}êlid. \hld\ Þuo skoldun sia þia \alst{d}ád frummjan, &
\alst{h}ȧhan þat \alst{h}êlaga barn. \hld\ Þat warð þem \alst{h}ęri-togen &
\alst{s}ïðor te \alst{s}orgon, \hld\ þat hie þia \alst{s}aka wissa, &
þat sia þuru \alst{n}íð-skipi \hld\ \alst{n}ęrjendon Krist, &
\alst{h}atoda þiu \alst{h}êri, \hld\ ęndi hie im \alst{h}ôrda te þiu, &
\alst{w}arạhta iro \alst{w}illjon: \hld\ þes hie \alst{w}íti ant·féng, &
\alst{l}ôn an þeson \alst{l}iohte \hld\ ęndi \alst{l}ang after, &
\alst{w}ói sïðor \alst{w}ann, \hld\ sïðor hie þesa \alst{w}er-old a·gaf.\eva%NOTE: wói checked.

\bvb TODO.\evb\evg

\bvg\bva[65][5428]%
Þuo warð þas þie \alst{w}rêðo gi·\alst{w}aro, \hld\ \alst{w}am-skaðono mêst, &
\alst{S}atanas \alst{s}elvo, \hld\ þuo þiu \alst{s}eola kwam &
\alst{J}udases an \alst{g}rund \hld\ \alst{g}rimmaro hęlljun— &
þuo \alst{w}issa hie te \alst{w}áren, \hld\ þat þat was \alst{w}aldand Krist, &
\alst{b}arn drohtines, \hld\ þat þár gi·\alst{b}undan stuod; &
\alst{w}issa þuo te \alst{w}áron, \hld\ þat hie welda þesa \alst{w}er-old alla &
mid is \alst{h}ęnginnja \hld\ \alst{h}ęllja gi·þwinges, &
\alst{l}iudi a·\alst{l}ôsjan \hld\ an \alst{l}ioht godes. &
Þat was \alst{S}atanase \hld\ \alst{s}êr an muode, &
tulgo \alst{h}arm an is \alst{h}ugje: \hld\ welda is \alst{h}elpan þuo, &
þat im \alst{l}iudjo barn \hld\ \alst{l}íf ne bi·námin, &
ne \alst{k}węlidin an \alst{k}rúkje, \hld\ ak hie welda, þat hie \alst{k}wik livdi, &
te þiu þat \alst{f}iriho barn \hld\ \alst{f}ernes ne wurðin, &
\alst{s}undjono \alst{s}ikura. \hld\ \alst{S}atanas gi·wêt im þuo, &
þár þes \alst{h}ęri-togen \hld\ \alst{h}íwiski was &
an þero \alst{b}urg innan. \hld\ Hie þero is \alst{b}rúdi bi·gann, &
þera idis \alst{o}pan-líko \hld\ \alst{u}n-hiuri fíond &
\alst{w}undẹr tôgjan, \hld\ þat sia an \alst{w}ord-helpon &
\alst{K}riste wári, \hld\ þat hie muosti \alst{k}wik libbjan, &
\alst{d}rohtin manno \hld\ —hie was iu þan te \alst{d}ôðe gi·skęrid— &
\alst{w}issa þat te \alst{w}áron, \hld\ þat hie im skoldi þia gi·\alst{w}ald bi·niman, &
þat hie sia ovar þesan \alst{m}iddil-gard \hld\ só \alst{m}ikila ni havdi, &
ovar \alst{w}ída \alst{w}er-old. \hld\ Þat \alst{w}íf warð þuo an forạhton, &
\alst{s}wíðo an \alst{s}orọgon, \hld\ þuo iru þiu gi·\alst{s}iuni kwámun &
þuru þes \alst{d}ęrnjen \alst{d}ád \hld\ an \alst{d}ages liohte, &
an \alst{h}ęlið-helme bi·\alst{h}elid. \hld\ Þuo siu te iru \alst{h}êrren an·bôd, &
þat \alst{w}íf mid iro \alst{w}ordon \hld\ ęndi im te \alst{w}áren hiet &
\alst{s}elvon \alst{s}ęggjan, \hld\ hwat iro þár te gi·\alst{s}iunjon kwam &
þuru þena \alst{h}êlagan mann, \hld\ ęndi im \alst{h}elpan bad, &
\alst{f}ormon is \alst{f}erhe: \hld\ „ik hębbju hier só \alst{f}ilo þuru ina &
\alst{s}eld-líkes gi·\alst{s}ewan, \hld\ só ik wêt, þat þia \alst{s}undjun skulun &
\alst{a}llaro \alst{e}rlo gi·hwem \hld\ \alst{u}vilo gi·þíhan, &
só im \alst{f}ruokno tuo \hld\ \alst{f}erạhes áhtið.“ &
Þie \alst{s}ęgg warð þuo an \alst{s}ïðe, \hld\ ant-tat hie \alst{s}ittjan fand &
þena \alst{h}ęri-togon \hld\ an \alst{h}warạve innan &
an þem \alst{st}ên-wege, \hld\ þár þiu \alst{st}ráta was &
\alst{f}elison gi·\alst{f}uogid. \hld\ Þár hie te is \alst{f}rôhon géng, &
sagda im þes \alst{w}íves \alst{w}ord. \hld\ Þuo warð im \alst{w}rêð hugi, &
þem \alst{h}ęri-togen, \hld\ —\alst{h}warạvoda an innan—, &
gi·\alst{b}lôðit \alst{b}riost-gi·þȧht: \hld\ was im \alst{b}êðjes wê, &
gie þat sea ina \alst{s}luogin \hld\ \alst{s}undja lôsan, &
gie it bi þem \alst{l}iudjon þuo \hld\ for·\alst{l}átan ne gi·dorsta &
þuru þes \alst{w}erodes word. \hld\ Warð im gi·\alst{w}ęndid þuo &
\alst{h}ugi an \alst{h}erten \hld\ after þero \alst{h}êri Judeono, &
te \alst{w}erkjanne iro \alst{w}illjon: \hld\ ne \alst{w}ardoda im nie-wiht &
þia \alst{s}wárun \alst{s}undjun, \hld\ þia hie im þár þuo \alst{s}elvo gi·deda. &
Hiet im þuo te is \alst{h}andon dragan \hld\ \alst{h}luttran brunnjon, &
\alst{w}atar an \alst{w}égje, \hld\ þár hie furi þem \alst{w}erode sat, &
\alst{þ}wóg ina þár for þero \alst{þ}ioda \hld\ \alst{þ}egạn kêsures, &
\alst{h}ard \alst{h}ęri-togo \hld\ ęndi þuo fur þero \alst{h}êri sprak, &
kwað þat hie ina þero \alst{s}undjono þár \hld\ \alst{s}ikoran dádi, &
\alst{w}rêðero \alst{w}erko: \hld\ „ne willju ik þes \alst{w}ihtes plegan“, kwaþ-hie, &
„umbi þesan \alst{h}êlagan mann, \hld\ ak \alst{h}leotad gí þes alles, &
gie \alst{w}ordo gie \alst{w}erko, \hld\ þes gí im hér te \alst{w}ítje gi·duan.“ &
Þuo \alst{h}reop all saman \hld\ \alst{h}ęri-skipi Judeono, &
þiu \alst{m}ikila \alst{m}ęnigi, \hld\ kwáðun þat sia weldin umbi þena \alst{m}an plegan &
\alst{d}ęrạvoro \alst{d}ádjo: \hld\ „fare is \alst{d}rôr ovar u̇s, &
is \alst{b}luod ęndi is \alst{b}aneði \hld\ ęndi ovar u̇sa \alst{b}arn só samo, &
ovar \alst{u̇}sa \alst{a}varon þár after \hld\ —wí willjat is \alst{a}lles plegan“, kwaðun sia, &
„umbi þena \alst{s}lęgi \alst{s}elvon,— \hld\ ef wí þár êniga \alst{s}undja gi·duan!“ &
A·\alst{g}evan warð þár þuo furi þem \alst{J}udeon \hld\ allaro \alst{g}umono bęsta &
\alst{h}ęttendjon an \alst{h}and, \hld\ an \alst{h}eru-bęndjon &
\alst{n}arạwo gi·\alst{n}ôdid, \hld\ þár ina \alst{n}íð-hwata, &
\alst{f}íond ant·\alst{f}éngun: \hld\ \alst{f}olk ina umbi·hwarf, &
\alst{m}ên-skaðono \alst{m}ęgin. \hld\ \alst{M}ahtig drohtin &
\alst{þ}oloda gi·\alst{þ}uldjon, \hld\ só hwat só im þiu \alst{þ}ioda deda. &
Sia hietun ina þuo \alst{f}illjan, \hld\ êr þan sia im \alst{f}erạhes tuo, &
\alst{a}ldres \alst{á}htin, \hld\ ęndi im undar is \alst{ô}gun spiwun, &
dedun im þat te \alst{h}oske, \hld\ þat sia mid iro \alst{h}andon slógun, &
\alst{w}eros an is \alst{w}angun \hld\ ęndi im is gi·\alst{w}ádi bi·námun, &
\alst{r}ôvodun ina þia \alst{r}ęgin-skaðon, \hld\ \alst{r}ôdes lakanes &
dedun im eft \alst{ȯ}ðer \alst{a}n \hld\ þuru \alst{u}n-huldi; &
\alst{h}ietun þuo \alst{h}ôvid-band \hld\ \alst{h}ardaro þorno &
\alst{w}undron \alst{w}indan \hld\ ęndi an \alst{w}aldand Krist &
\alst{s}elvon \alst{s}ęttjan, \hld\ ęndi géngun im þia gi·\alst{s}ïðos tuo, &
\alst{k}węddun ina an \alst{k}uning-wísu \hld\ ęndi þár an \alst{k}nio fellun, &
\alst{h}nigun im mid iro \alst{h}ôvdu: \hld\ all was im þat te \alst{h}oske gi·duan, &
þoh hie it all gi·\alst{þ}olodi, \hld\ \alst{þ}iodo drohtin, &
\alst{m}ahtig þuru þia \alst{m}innja \hld\ \alst{m}anno kunnjes. &
Hietun sia þuo \alst{w}irkjan \hld\ \alst{w}ápnes ęggjon &
\alst{h}ęliðos mid iro \alst{h}andon \hld\ \alst{h}ardes bômes &
\alst{k}raftiga \alst{k}rúki \hld\ ęndi hietun sia \alst{K}ristan þuo, &
\alst{s}álig barn godes \hld\ \alst{s}elvon fuorjan, &
\alst{d}ragan hietun sia u̇san \alst{d}rohtin, \hld\ þár hie be·\alst{d}rôragad skolda &
\alst{s}weltan \alst{s}undjono lôs. \hld\ \alst{S}ïðodun Judeon, &
\alst{w}eros an \alst{w}illon, \hld\ lêddun \alst{w}aldand Krist, &
\alst{d}rohtin te \alst{d}ôðe. \hld\ Þár mohta man þuo \alst{d}erẹvi þing &
\alst{h}arm-lík gi·\alst{h}ôrjan: \hld\ \alst{h}iovandi þár after &
géngun \alst{w}íf mid \alst{w}ópu, \hld\ \alst{w}eros gnornodun, &
þia fan \alst{G}alilea mid im \hld\ \alst{g}angan kwámun, &
\alst{f}olgodun ovar \alst{f}err-wegos: \hld\ was im iro \alst{f}rôhon dôð &
\alst{s}wíðo an \alst{s}orạgan. \hld\ Þuo hie \alst{s}elvo sprak, &
\alst{b}arno þat \alst{b}ęsta \hld\ ęndi under \alst{b}ak be·sah, &
hiet þat sia ni \alst{w}épin: \hld\ „ni þarf iu \alst{w}iht tregan“, kwaþ-hie, &
„mínero \alst{h}in-fęrdjo, \hld\ ak gí mid \alst{h}ofnu mugun &
iuwa \alst{w}rêðan \alst{w}erk \hld\ \alst{w}ópu kúmjan, &
\alst{t}ornon \alst{t}rahnon. \hld\ Noh wirðið þiu \alst{t}íd kuman, &
þat þia \alst{m}uoder þes \hld\ \alst{m}ęndendja sind, &%TODO: check męndendja
\alst{b}rúdi Judeono, \hld\ þem gio \alst{b}arn ni warð &
\alst{ô}dan an \alst{a}ldre. \hld\ Þan gí iuwa \alst{i}n-wid skulun &
\alst{g}rimmo an·\alst{g}eldan; \hld\ þan gí só \alst{g}erna sind, &
þat iu \alst{h}ier bi·\alst{h}lídan \hld\ \alst{h}ôha bergos, &
\alst{d}iopo be·\alst{d}elvan; \hld\ \alst{d}ôð wári iu þan allon &
\alst{l}iovera an þeson \alst{l}ande \hld\ þan su·lik \alst{l}iudjo kwalm &
te gi·\alst{þ}oljanne, \hld\ só hier þan þesaro \alst{þ}ioda kumid.“\eva

\bvb TODO.\evb\evg

\bvg\bva[66][5533]%
Þuo sia þár an \alst{g}riete \hld\ \alst{g}algon rihtun, &
an þem \alst{f}elde uppan \hld\ \alst{f}olk Judeono, &
\alst{b}ôm an \alst{b}erẹge, \hld\ ęndi þár an þat \alst{b}arn godes &
\alst{k}węlidun an \alst{k}rúkje: \hld\ slógun \alst{k}ald ísarn, &
\alst{n}iwa \alst{n}aglos \hld\ \alst{n}íðon skarpa &
\alst{h}ardo mid \alst{h}amuron \hld\ þuru is \alst{h}ęndi ęndi þuru is fuoti, &
\alst{b}ittra \alst{b}ęndi: \hld\ is \alst{b}lód ran an erða, &
\alst{d}rôr fan u̇son \alst{d}rohtine. \hld\ Hie ni welda þoh þia \alst{d}ád wrekan &
\alst{g}rimma an þem \alst{J}udeon, \hld\ ak hie þes \alst{g}od fader &
\alst{m}ahtigna bad, \hld\ þat hie ni wári þem \alst{m}anno folke, &
þem \alst{w}erode þiu \alst{w}rêðra: \hld\ „hwand sia ni \alst{w}itun, hwat sia duot“, kwaþ-hie. &
Þuo þia \alst{w}ígandos \hld\ gi·\alst{w}ádi Kristes, &
\alst{d}rohtines \alst{d}êldun, \hld\ \alst{d}ęrẹvja mann, &
þes \alst{r}íken gi·\alst{r}ôbi. \hld\ Þia \alst{r}inkos ni mahtun &
umbi þena \alst{s}elvon {[...]} \hld\ \alst{s}am-wurdi gi·sprekan, &
êr sia an iro \alst{h}warạve \hld\ \alst{h}lôtos wurpun, &
\alst{h}wi-lik iro skoldi \alst{h}ębbjan \hld\ þia \alst{h}êlagun pêda, &
allaro gi·\alst{w}ádjo \alst{w}un-samost. \hld\ Þes \alst{w}erodes hirdi &
\alst{h}iet þuo, þe \alst{h}ęri-togo, \hld\ ovar þem \alst{h}ôvde selves &
\alst{K}ristes an \alst{k}rúke skrívan, \hld\ þat þat wári \alst{k}uning Judeono, &
Jesus fan \alst{N}azareth-burh, \hld\ þie þár \alst{n}ęglid stuod &
an \alst{n}iwon galgon \hld\ þuru \alst{n}íð-skipi, &
an \alst{b}ômin treo. \hld\ Þuo \alst{b}ádun þia liudi &
þat \alst{w}ord \alst{w}ęndjan, \hld\ kwáðun þat hie im só an is \alst{w}illjon spráki, &
\alst{s}elvo \alst{s}agdi, \hld\ þat hie habdi þes gi·\alst{s}ïðes gi·wald, &
\alst{k}uning wári ovar Judeon. \hld\ Þuo sprak eft þie \alst{k}êsures bodo, &
\alst{h}ard \alst{h}ęri-togo: \hld\ „it ist iu só ovar is \alst{h}ôvde gi·skrivan, &
\alst{w}ís-líko gi·\alst{w}ritan, \hld\ só ik it nú \alst{w}ęndjan ni mag.“ &
Dádun þuo þár te \alst{w}ítje \hld\ \alst{w}erod Judeono &
\alst{t}wêna far·\alst{t}alda man \hld\ an \alst{t}wá halva &
\alst{K}ristes an \alst{k}rúki: \hld\ lietun sia \alst{k}walm þolon &
an þem \alst{w}arạg-trewe \hld\ \alst{w}erko te lône, &
\alst{l}êðaro dádjo. \hld\ Þia \alst{l}iudi sprákun &
\alst{h}osk-word manag \hld\ \alst{h}êlagon Kriste, &
\alst{g}rottun ina mid \alst{g}elpu: \hld\ sáwun allaro \alst{g}umono þen bęston &
\alst{k}węlan an þemo \alst{k}rúkje: \hld\ „ef þú sís \alst{k}uning ovar all“, kwáðun sia, &
„\alst{s}uno drohtines, \hld\ só þú havis \alst{s}elvo gi·sprokan, &
\alst{n}ęri þik fan þero \alst{n}ôdi \hld\ ęndi \alst{n}íðes a·tuomi, &
gang þi \alst{h}êl \alst{h}erod; \hld\ þan węlljat an þik \alst{h}ęliðo barn, &
þesa \alst{l}iudi gi·\alst{l}ôvjan.“ \hld\ Sum imo ôk \alst{l}astar sprak &
swíðo \alst{g}êl-hert \alst{J}udeo, \hld\ þár hie fur þem \alst{g}algon stuod: &
„\alst{W}ah warð þesaro \alst{w}er-oldi“, \hld[kwaþ-hie,] „ef þú iro skoldis gi·\alst{w}ald êgan. &
Þú sagdas þat þú mahtis an \alst{ê}non dage \hld\ \alst{a}ll te·werpan &
þat \alst{h}ôha \alst{h}ús \hld\ \alst{h}evan-kuninges, &
\alst{st}ên-werko mêst \hld\ ęndi eft \alst{st}andan gi·duon &
an \alst{þ}riddjon dage, \hld\ só is elkor ni þorfti bi·\alst{þ}íhan mann &
þeses \alst{f}olkes \alst{f}urðor. \hld\ Sínu hwó þú nú gi·\alst{f}astnod stés, &
\alst{s}wíðo gi·\alst{s}êrid: \hld\ ni maht þi \alst{s}elvon wiht &
\alst{b}alowes gi·\alst{b}uotjan.“ \hld\ Þuo þár ôk an þem \alst{b}ęndjon sprak &
þero \alst{þ}eovo ȯðer, \hld\ all só hie þia \alst{þ}ioda gi·hôrda, &
\alst{w}rêðon \alst{w}ordon \hld\ —ne was is \alst{w}illjo guod, &
þes \alst{þ}egnes gi·\alst{þ}ȧht—: \hld\ „ef þú sís \alst{þ}iod-kuning“, kwaþ-hie, &
„\alst{K}rist, godes suno, \hld\ gang þi þan fan þem \alst{k}rúke niðer, &
\alst{s}lópi þi fan þem \alst{s}ímon \hld\ ęndi u̇s \alst{s}amad allon &
\alst{h}ilp ęndi \alst{h}êli. \hld\ Ef þú sís \alst{h}evan-kuning, &
\alst{w}aldand þesaro \alst{w}er-oldes, \hld\ gi·duo it þan an þínon \alst{w}erkon skín, &
\alst{m}ári þik fur þesaro \alst{m}ęnigi.“ \hld\ Þuo sprak þero \alst{m}anno ȯðer &
an þero \alst{h}ęnginna, \hld\ þár hie gi·\alst{h}ęftid stuod, &
\alst{w}an \alst{w}undẹr-kwála: \hld\ „Be·hwí wilt þú su·lik \alst{w}ord sprekan, &
\alst{g}ruotis ina mid \alst{g}elpu? \hld\ Stés þí hier an \alst{g}algen haft, &
gi·\alst{b}rokan an \alst{b}ôme. \hld\ Wit hier \alst{b}êðja þolod &
\alst{s}êr þuru unka \alst{s}undjun: \hld\ is unk unkero \alst{s}elvero dád &
\alst{w}orðan te \alst{w}ítje. \hld\ Hie stéd hier \alst{w}ammes lôs, &
allaro \alst{s}undjono \alst{s}ikur, \hld\ só hie \alst{s}elvo gio &
\alst{f}irina ni gi·\alst{f}rumida, \hld\ botan þat hie þuru þeses \alst{f}olkes nið &
\alst{w}illendi an þesaro \alst{w}er-uldi \hld\ \alst{w}íti ant·fáhid. &
Ik willju þár gi·\alst{l}ôvjan tuo“, \hld[kwaþ-hie,] „ęndi willju þena \alst{l}andes ward, &
þena \alst{g}odes suno \hld\ \alst{g}erno biddjan, &
þat þú mín gi·\alst{h}uggjes \hld\ ęndi an \alst{h}elpun sís, &
\alst{r}ádendero bęst, \hld\ þan þú an þín \alst{r}íki kumis: &
wes mí þan gi·\alst{n}áðig.“ \hld\ Þuo sprak im eft \alst{n}ęrjendo Krist &
\alst{w}ordon te·gęgnes: \hld\ „Ik sęggju þí te \alst{w}áron hier“, kwaþ-hie, &
„þat þú noh \alst{h}iu-du móst \hld\ an \alst{h}imil-ríke &
mid mí \alst{s}amad \hld\ \alst{s}ehan lioht godes, &
an þemo \alst{P}aradýse, \hld\ þoh þú nú an su·likoro \alst{p}ínu sís.“ &
Þan stuod þár ôk \alst{M}aria, \hld\ \alst{m}uoder Kristes, &
\alst{b}lêk under þem \alst{b}ôme, \hld\ gi·sah iro \alst{b}arn þolon, &
\alst{w}innan \alst{w}undẹr-kwála. \hld\ Ôk wárun þár \alst{w}íf mid iro &
an só \alst{m}ahtiges \hld\ \alst{m}innja kumana— &
þan stuod þár ôk \alst{J}ohannes, \hld\ \alst{j}ungro Kristes, &
\alst{h}riuwi undar is \alst{h}êrren, \hld\ was im is \alst{h}ugi sêrag— &
\alst{d}rúvodun fur þem \alst{d}ôðe. \hld\ Þár sprak \alst{d}rohtin Krist &
\alst{m}ahtig te þero \alst{m}uoder: \hld\ „nú ik þí hier \alst{m}ínemo skal &
\alst{j}ungron be·felhan, \hld\ þem þí hier \alst{g}ęgin-ward stéd: &
wis þí an is gi·\alst{s}ïðje \alst{s}amad: \hld\ þú skalt ina furi \alst{s}uno hębbjan.“ &
\alst{G}rótta hie þuo \alst{J}ohannes, \hld\ hiet þat hie iru ful-\alst{g}éngi wel, &
\alst{m}innjodi sia só \alst{m}ildo, \hld\ só man is \alst{m}uoder skal, &
\alst{i}dis \alst{u}n-wamma. \hld\ Þuo hie sia an is \alst{ê}ra ant·féng &
þuru \alst{h}luttran \alst{h}ugi, \hld\ só im is \alst{h}êrro gi·bôd.\eva

\bvb TODO.\evb\evg

\bvg\bva[67][5622]%
Þuo warð þár an \alst{m}iddjan dag \hld\ \alst{m}ahtig têkạn, &
\alst{w}undạr-lík gi·\alst{w}arạht \hld\ ovar þesan \alst{w}er-old allan, &
þuo man þena \alst{g}odes suno \hld\ an þena \alst{g}algon huof, &
\alst{K}rist an þat \alst{k}rúki: \hld\ þuo warð it \alst{k}u̇ð ovar all, &
hwó þiu \alst{s}unna warð gi·\alst{s}workan: \hld\ ni mahta \alst{s}wigli lioht &
\alst{sk}ôni gi·\alst{sk}ínan, \hld\ ak sia \alst{sk}ado far·féng, &
\alst{þ}imm ęndi \alst{þ}iustri \hld\ ęndi só gi·\alst{þ}rusmod neval. &
Warð allaro \alst{d}ago \alst{d}ruovost, \hld\ \alst{d}unkar swíðo &
ovar þesan \alst{w}ídun \alst{w}er-uld, \hld\ só lango só \alst{w}aldand Krist &
\alst{k}wal an þemo \alst{k}rúkje, \hld\ \alst{k}uningo ríkost, &
ant \alst{n}uon dages. \hld\ Þuo þie \alst{n}eval ti·skrêd, &
þat gi·\alst{s}werk warð þuo te·\alst{s}wungan, \hld\ bi·gan \alst{s}unnun lioht &
\alst{h}êdron an \alst{h}imile. \hld\ Þuo \alst{h}reop up te gode &
allaro \alst{k}uningo \alst{k}raftigost, \hld\ þuo hie an þemo \alst{k}rúkje stuod &
\alst{f}aðmon gi·\alst{f}astnot: \hld\ „\alst{f}ader alo-mahtig“, kwaþ-hie, &
„te hwí þú mik só far·\alst{l}ieti, \hld\ \alst{l}ievo drohtin, &
\alst{h}êlag \alst{h}evan-kuning, \hld\ ęndi þína \alst{h}elpa dedos, &
\alst{f}ullisti só \alst{f}err? \hld\ Ik standu under þeson \alst{f}íondon hier &
\alst{w}undron gi·\alst{w}êgid.“ \hld\ \alst{W}erod Judeono &
\alst{h}lógun is im þuo te \alst{h}oske: \hld\ gi·\alst{h}ôrdun þena hêlagun Krist, &
\alst{d}rohtin furi þem \alst{d}ôðe \hld\ \alst{d}rinkan biddjan, &
kwað þat ina \alst{þ}urstidi. \hld\ Þiu \alst{þ}ioda ne latta, &
\alst{w}rêða \alst{w}iðar-sakon: \hld\ was im \alst{w}illjo mikil, &
hwat sia im \alst{b}ittres tuo \hld\ \alst{b}ringan mahtin. &
Habdun im \alst{u}n-swóti \hld\ \alst{ę}kid ęndi galla &
gi·\alst{m}ęngid þia \alst{m}ên-hwaton; \hld\ stuod ên \alst{m}ann garo, &
swíðo \alst{sk}uldig \alst{sk}aðo, \hld\ þena habdun sia gi·\alst{sk}ęrid te þiu, &
far·\alst{sp}anan mid \alst{sp}rákon, \hld\ þat hie sia en êna \alst{sp}unsja nam, &
\alst{l}íðo þes \alst{l}êðosten, \hld\ druog it an ênon \alst{l}angan skafte, &
gi·\alst{b}undan an ênon \alst{b}ôme \hld\ ęndi deda it þem \alst{b}arne godes, &
\alst{m}ahtigon te \alst{m}u̇ðe. \hld\ Hie an·kęnda iro \alst{m}irkjun dádi, &
gi·\alst{f}uolda iro \alst{f}égnes: \hld\ \alst{f}urðor ni welda &%TODO: check fégnes
is só \alst{b}ittres an·\alst{b}ítan, \hld\ ak hreop þat \alst{b}arn godes &
\alst{h}lúdo te þem \alst{h}imiliskon fader: \hld\ „ik an þina \alst{h}ęndi be·filhu“, kwaþ-hie, &
„mínon \alst{g}êst an \alst{g}odes willjon; \hld\ hie ist nú \alst{g}aro te þiu, &
\alst{f}u̇s te \alst{f}aranne.“ \hld\ \alst{F}iriho drohtin &
gi·\alst{h}nêgida þuo is \alst{h}ôvid, \hld\ \alst{h}êlagon áðom &
\alst{l}iet fan þemo \alst{l}ík-hamen. \hld\ Só þuo þie \alst{l}andes ward &
\alst{s}walt an þem \alst{s}ímon, \hld\ só warð \alst{s}án after þiu &
\alst{w}undạr-têkạn gi·\alst{w}arạht, \hld\ þat þár \alst{w}aldandes dôð &
un·\alst{k}weðandes só filo \hld\ ant·\alst{k}ęnnjan skolda, &
þiadnes \alst{ê}n-dagon: \hld\ \alst{e}rða bivoda, &
\alst{h}risidun þia \alst{h}ôhun bergos, \hld\ \alst{h}arda stênos kluvun, &
\alst{f}elisos after þem \alst{f}elde, \hld\ ęndi þat \alst{f}êha lakan te·brast &
an \alst{m}iddjon an twê, \hld\ þat êr \alst{m}anagan dag &
an þemo \alst{w}íhe innan \hld\ \alst{w}undron gi·striunid &
\alst{h}êl \alst{h}angoda \hld\ —ni muostun \alst{h}ęliðo barn, &
þia \alst{l}iudi skawon, \hld\ hwat under þemo \alst{l}akane was &
\alst{h}êlages be·\alst{h}angan: \hld\ þuo mohtun an þat \alst{h}orð sehan &
\alst{J}udeo liudi— \hld\ \alst{g}ravu wurðun gi·opanod &
\alst{d}ôdero manno, \hld\ ęndi sia þuru \alst{d}rohtines kraft &
an iro \alst{l}ík-hamon \hld\ \alst{l}ibbjandi a·stuodun &
\alst{u}p fan \alst{e}rðu \hld\ ęndi wurðun gi·\alst{ô}gida þár &
\alst{m}annon te \alst{m}árðu. \hld\ Þat was só \alst{m}ahtig þing, &
þat þár \alst{K}ristes dôð \hld\ ant·\alst{k}ęnnjan skoldun, &
só \alst{f}ilo þes gi·\alst{f}uoljan, \hld\ þie gio mid \alst{f}irihon ne sprak &
\alst{w}ord an þesaro \alst{w}er-oldi. \hld\ \alst{W}erod Judeono &
\alst{s}áwun \alst{s}eld-lík þing, \hld\ ak was im iro \alst{s}líði hugi &
só far·\alst{h}ardod an iro \alst{h}erten, \hld\ þat þár io só \alst{h}êlag ni warð &
\alst{t}êkạn gi·\alst{t}ôgid, \hld\ þat sia \alst{t}rúodin þiu bat &
an þia \alst{K}ristes \alst{k}raft, \hld\ þat hie \alst{k}uning ovar all, &
þes \alst{w}erodes \alst{w}ári. \hld\ Suma sia þár mid iro \alst{w}ordon gi·sprákun, &
þia þes \alst{h}rêwes þár \hld\ \alst{h}uodjan skoldun, &
þat þat \alst{w}ári te \alst{w}áren \hld\ \alst{w}aldandes suno, &
\alst{g}odes \alst{g}egnungo, \hld\ þat þár an þem \alst{g}algon swalt, &
\alst{b}arno þat \alst{b}ęsta. \hld\ Slógun an iro \alst{b}riost filo &
\alst{w}ópjandero \alst{w}ívo: \hld\ was im þiu \alst{w}undẹr-kwála &
\alst{h}arm an iro \alst{h}erten \hld\ ęndi iro \alst{h}êrren dôð &
\alst{s}wíðo an \alst{s}orọgon. \hld\ Þan was \alst{s}ido Judeono, &
þat sia þia \alst{h}aftun þuru þena \alst{h}êlagon dag \hld\ \alst{h}angon ni lietin &
\alst{l}ęngerun hwíla, \hld\ þan im þat \alst{l}íf skriði, &
þiu \alst{s}eola be·\alst{s}unki: \hld\ \alst{s}líð-muoda mann &
géngun im mid \alst{n}íð-skipju \alst{n}áhor, \hld\ þár só be·\alst{n}ęglida stuodun &
\alst{þ}eovos twêna, \hld\ \alst{þ}olodun bêðja &
\alst{k}wála bi \alst{K}riste: \hld\ wárun im \alst{k}wika noh þan, &
unt-þat sia þia \alst{g}rimmun \hld\ \alst{J}udeo liudi &
\alst{b}ênon be·\alst{b}rákon, \hld\ þat sia \alst{b}êðja samad &
\alst{l}íf far·\alst{l}ietun, \hld\ suohtun im \alst{l}ioht ȯðer. &
Sia ni þorftun \alst{d}rohtin Krist \hld\ \alst{d}ôðes bêdjan &
\alst{f}urðor mid ênigon \alst{f}irinon: \hld\ fundun ina gi·\alst{f}aranan þuo iu: &
is \alst{s}eola was gi·\alst{s}ęndid \hld\ an \alst{s}uȯðan weg, &
an \alst{l}ang-sam \alst{l}ioht, \hld\ is \alst{l}iði kuolodun; &
þat \alst{f}erạh was af þem \alst{f}lêske. \hld\ Þuo géng im ên þero \alst{f}íondo tuo &
an \alst{n}íð-hugi, \hld\ druog \alst{n}ęgilid sper &
\alst{h}ard an is \alst{h}andon, \hld\ mid \alst{h}eru-þrummjon stak, &
liet \alst{w}ápnes ord \hld\ \alst{w}undum sníðan, &
þat an \alst{s}elves warð \hld\ \alst{s}ídu Kristes &
ant·\alst{l}okan is \alst{l}ík-hamo. \hld\ Þia \alst{l}iudi gi·sáwun, &
þat þanan \alst{b}luod ęndi water \hld\ \alst{b}êðju sprungun, &
\alst{w}ellun fan þero \alst{w}undun, \hld\ all só is \alst{w}illjo géng &
ęndi hie habda gi·\alst{m}arkod êr \hld\ \alst{m}anno kunnje, &
\alst{f}iriho barnon te \alst{f}rumu: \hld\ þuo was it all gi·\alst{f}ullid só.\eva

\bvb TODO.\evb\evg

\bvg\bva[68][5714]%
Só þuo gi·\alst{s}êgid warð \hld\ \alst{s}edle náhor &
\alst{h}êdra sunna \hld\ mid \alst{h}evan-tunglon &
an þem \alst{d}ruoven \alst{d}age, \hld\ þuo géng im u̇ses \alst{d}rohtines þegạn &
—was im \alst{g}lau \alst{g}umo, \hld\ \alst{j}ungro Kristes &
\alst{m}anaga hwíla, \hld\ só it þár \alst{m}anno filo &
ne \alst{w}issa te \alst{w}áron, \hld\ hwand hie it mid is \alst{w}ordon hal &
\alst{J}uðeono \alst{g}um-skipje: \hld\ \alst{J}oseph was hie hêtan, &
\alst{d}arnungo was hie u̇ses \alst{d}rohtines jungro: \hld\ hie ni welda þero far·\alst{d}uanun þiod &
\alst{f}olgon te ênigon \alst{f}irin-werkon, \hld\ ak hie bêd im under þem \alst{f}olke Judeono, &
\alst{h}êlag \alst{h}imilo ríkjes— \hld\ hie géng im þuo wið þena \alst{h}ęri-togon mahljan, &
\alst{þ}ingon wið þena \alst{þ}egạn kêsures, \hld\ \alst{þ}igida ina gerno, &
þat hie muosti a·\alst{l}ôsjan \hld\ þena \alst{l}ík-hamon &
\alst{K}ristes fan þemo \alst{k}rúkje, \hld\ þie þár gi·\alst{k}węlmid stuod, &
þes \alst{g}uoden fan þem \alst{g}algen \hld\ ęndi an \alst{g}raf lęggjan, &
\alst{f}oldu bi·\alst{f}elạhan. \hld\ Im ni welda þie \alst{f}olk-togo þuo &
\alst{w}ęrnjan þes \alst{w}illjen, \hld\ ak im gi·\alst{w}ald far·gaf, &
þat hie só muosti gi·\alst{f}rummjan. \hld\ Hie gi·wêt im þuo \alst{f}orð þanan &
\alst{g}angan te þem \alst{g}algon, \hld\ þár hie wissa þat \alst{g}odes barn, &
\alst{h}rêo \alst{h}angondi \hld\ \alst{h}êrren sínes, &
nam ina þuo an þero \alst{n}iwun ruodun \hld\ ęndi ina fan \alst{n}aglon a·tuomda, &
ant·\alst{f}éng ina mid is \alst{f}aðmon, \hld\ só man is \alst{f}rôhon skal, &
\alst{l}ioves \alst{l}ík-hamon, \hld\ ęndi ina an \alst{l}íne bi·wand, &
\alst{d}ruog ina \alst{d}iur-líko \hld\ —só was þie \alst{d}rohtin werð—, &
þár sia þia \alst{st}ędi havdun \hld\ an ênon \alst{st}êne innan &
\alst{h}andon gi·\alst{h}auwan, \hld\ þár gio \alst{h}ęliðo barn &
\alst{g}umon ne bi·\alst{g}ruovon. \hld\ Þár sia þat \alst{g}odes barn &
te iro \alst{l}and-wísu, \hld\ \alst{l}íko hêlgost &
\alst{f}oldu bi·\alst{f}ulhun \hld\ ęndi mid ênu \alst{f}elisu be·lukun &
allaro \alst{g}ravo \alst{g}uod-líkost. \hld\ \alst{G}riotandi sátun &
\alst{i}disi \alst{a}rm-skapana, \hld\ þia þat \alst{a}ll for·sáwun, &
þes \alst{g}umen \alst{g}rimman dôð. \hld\ Gi·witun im þuo \alst{g}angan þanan &
\alst{w}ópjandi \alst{w}íf \hld\ ęndi \alst{w}ara námun, &
hwó sia eft te þem \alst{g}rave \hld\ \alst{g}angan mahtin: &
havdun im far·\alst{s}ewana \hld\ \alst{s}orọga gi·nuogja, &
\alst{m}ikila \alst{m}uod-kara: \hld\ \alst{M}aria wárun sia hêtana, &
\alst{i}disi \alst{a}rm-skapana. \hld\ Þuo warð \alst{á}vand kuman, &
\alst{n}aht mid \alst{n}eflu. \hld\ \alst{N}íð-folk Judeono &
warð an \alst{m}orạgan eft, \hld\ \alst{m}ęnigi gi·samnod, &
\alst{r}ękidun an \alst{r}únon: \hld\ „Hwat þú wêst, hwó þit \alst{r}íki was &
þuru þesan \alst{ê}nan man \hld\ \alst{a}ll gi·twíflid, &
\alst{w}erod gi·\alst{w}orran: \hld\ nú ligid hie \alst{w}undon siok, &
\alst{d}iopa bi·\alst{d}olvan. \hld\ Hie sagda simnen, þat hie skoldi fan \alst{d}ôðe a·standan &
an \alst{þ}riddjan dage. \hld\ Þius \alst{þ}iod gi·lôvit te filo, &
þit \alst{w}erod after is \alst{w}ordon. \hld\ Nú þú hier \alst{w}ardon hét, &
ovar þem \alst{g}rave \alst{g}ômjan, \hld\ þat ina is \alst{j}ungron þár &
ne far·\alst{st}elan an þemo \alst{st}êne \hld\ ęndi sęggjan þan, þat hie a·\alst{st}andan sí, &
\alst{r}íki fan \alst{r}aston: \hld\ þan wirðit þit \alst{r}inko folk &
\alst{m}êr gi·\alst{m}ęrrid, \hld\ ef sia it bi·ginnat \alst{m}árjan hier.“ &
Þuo wurðun þár gi·\alst{sk}ęrida \hld\ fan þero \alst{sk}olu Judeono &
\alst{w}eros te þero \alst{w}ahtu: \hld\ gi·witun im mid iro gi·\alst{w}ápnjon þarod &
te þem \alst{g}rave \alst{g}angan, \hld\ þár sia skoldun þes \alst{g}odes barnes &
\alst{h}rêwes \alst{h}uodjan. \hld\ Warð þie \alst{h}êlago dag &
\alst{J}udeono far·\alst{g}angan. \hld\ Sia ovar þemo \alst{g}rave sátun, &
\alst{w}eros an þero \alst{w}ahtun \hld\ \alst{w}annom nahton, &
\alst{b}idun undar iro \alst{b}ordon, \hld\ hwan êr þie \alst{b}erẹhto dag &
ovar \alst{m}iddil-gard \hld\ \alst{m}annon kwámi, &
\alst{l}iudon te \alst{l}iohte. \hld\ Þuo ni was \alst{l}ang te þiu, &
þat þár warð þie \alst{g}êst kuman \hld\ be \alst{g}odes krafte, &
\alst{h}âlag áðom \hld\ undar þena \alst{h}ardon stên &
an þena \alst{l}ík-hamon. \hld\ \alst{L}ioht was þuo gi·opanod &
\alst{f}iriho barnon te \alst{f}rumu: \hld\ was \alst{f}erkal manag &
ant·\alst{h}ęftid fan \alst{h}ęll-doron \hld\ ęndi te \alst{h}imile weg &
gi·\alst{w}arạht fan þesaro \alst{w}er-oldi. \hld\ \alst{W}ánom up a·stuod &
\alst{f}riðu-barn godes, \hld\ \alst{f}uor im þuo þár hie welda, &
só þia \alst{w}ardos þes \hld\ \alst{w}iht ni af·swovun, &
\alst{d}ęrvja liudi, \hld\ hwan hie fan þem \alst{d}ôðe a·stuod, &
a·\alst{r}ês fan þero \alst{r}astun. \hld\ \alst{R}inkos sátun &
umbi þat \alst{g}raf útan, \hld\ \alst{J}udeo liudi, &
\alst{sk}ola mid iro \alst{sk}ildjon. \hld\ \alst{Sk}rêd forð-wardes &
\alst{s}wigli \alst{s}unnun lioht. \hld\ \alst{S}ïðodun idisi &
te þem \alst{g}rave \alst{g}angan, \hld\ \alst{g}um-kunnjes wíf, &
\alst{M}ariun \alst{m}uni-líka: \hld\ habdun \alst{m}êðmo filo &
gi·\alst{s}ald wiðẹr \alst{s}alvum, \hld\ \alst{s}ilụvres ęndi goldes, &
\alst{w}erðes wiðẹr \alst{w}urtjon, \hld\ só sia mahtun a·\alst{w}innan mêst, &
þat sia þena \alst{l}ík-hamon \hld\ \alst{l}ioves hêrren, &
\alst{s}uno drohtines, \hld\ \alst{s}alvon muostin, &
\alst{w}undun \alst{w}ritanan. \hld\ Þiu \alst{w}íf sorạgodun &
an iro \alst{s}evon \alst{s}wíðo, \hld\ ęndi \alst{s}uma sprákun, &
hwie im þena \alst{g}rôtan stên \hld\ fan þemo \alst{g}rave skoldi &
gi·\alst{h}węrẹvjan an \alst{h}alva, \hld\ þe sia ovar þat \alst{h}rêo sáwun &
þia \alst{l}iudi \alst{l}ęggjan, \hld\ þuo sia þena \alst{l}ík-hamon þár &
be·\alst{f}ulhun an þemo \alst{f}elise. \hld\ Só þiu \alst{f}rí havdun &
ge·\alst{g}angan te þem \alst{g}ardon, \hld\ þat sia te þem \alst{g}rave mahtun &
gi·\alst{s}ehan \alst{s}elvon, \hld\ þuo þár \alst{s}wógan kwam &
\alst{ę}ngil þes \alst{a}lo-waldon \hld\ \alst{o}vana fan radure, &
\alst{f}aran an \alst{f}eðer-hamon, \hld\ þat all þiu \alst{f}olda an skian, &
þiu \alst{e}rða dunida \hld\ ęndi þia \alst{e}rlos wurðun &
an \alst{w}êkan hugje, \hld\ \alst{w}ardos Juðeono, &
bi·\alst{f}ellun bi þem \alst{f}orạhton: \hld\ ne wándun ira \alst{f}erạh êgan, &
\alst{l}íf \alst{l}angerun hwíl.\eva

\bvb TODO.\evb\evg

\bvg\bva[69][5803]%
\hspace*{100pt} \alst{L}águn þá wardos, &
þia gi·\alst{s}ïðos \alst{s}ám-kwika: \hld\ \alst{s}án up a·hlâd &
þie \alst{g}rôto stên fan þem \alst{g}rave, \hld\ só ina þie \alst{g}odes ęngil &
gi·\alst{h}węrịvida an \alst{h}alva, \hld\ ęndi im uppan þem \alst{h}lêwe gi·sat &
\alst{d}iur-lík \alst{d}rohtines bodo. \hld\ Hie was an is \alst{d}ádjon ge·lík, &
an is \alst{a}n-siunjon, \hld\ só hwem só ina muosta undar is \alst{ô}gon skawon, &
só \alst{b}erẹht ęndi só \alst{b}líði \hld\ all só \alst{b}liksmun lioht; &
was im is gi·\alst{w}ádi \hld\ \alst{w}intạr-kaldon &
\alst{s}nêwe gi·líkost. \hld\ Þuo sáwun sia ina \alst{s}ittjan þár, &
þiu \alst{w}íf uppan þem gi·\alst{w}ęndidan stêne, \hld\ ęndi im fan þem \alst{w}litje kwámun, &
þem \alst{i}dison su·lika \alst{ę}gison te·gęgnes: \hld\ \alst{a}ll wurðun fan þem grurje &
þiu \alst{f}rí an \alst{f}orạhton mikilon, \hld\ \alst{f}urðor ne gi·dorstun &
te þemo \alst{g}rave \alst{g}angan, \hld\ êr sia þie \alst{g}odes ęngil, &
\alst{w}aldandes bodo \hld\ \alst{w}ordon gruotta, &
kwað þat hie iro \alst{â}rundi \hld\ \alst{a}ll bi·kunsti, &
\alst{w}erk ęndi \alst{w}illjon \hld\ ęndi þero \alst{w}ívo hugi, &
hiet þat sia im ne an·\alst{d}rédin: \hld\ „ik wêt þat gí iuwan \alst{d}rohtin suokat, &
\alst{n}ęrjendon Krist \hld\ fan \alst{N}azareth-burg, &
þena þi hier \alst{k}węlidun \hld\ ęndi an \alst{k}rúki slógun &
\alst{J}udeo liudi \hld\ ęndi an \alst{g}raf lagdun &
\alst{s}undi-lôsjan. \hld\ Nú nist hie \alst{s}elvo hier, &
ak hie ist a·\alst{st}andan iu, \hld\ ęndi sind þesa \alst{st}ędi lárja, &%NOTE ms. -- a·standan] L 1r.
þit \alst{g}raf an þeson \alst{g}riote. \hld\ Nú mugun gí \alst{g}angan herod &
\alst{n}áhor mikilu \hld\ —ik wêt þat is iu ist \alst{n}iud sehan &
an þeson \alst{st}êne innan—: \hld\ hier sind noh þia \alst{st}ędi skína, &
þár is \alst{l}ík-hamo \alst{l}ag.“ \hld\ \alst{L}ungra féngun &
gi·\alst{b}ada an iro \alst{b}rioston \hld\ \alst{b}lêka idisi, &
\alst{w}liti-skôni \alst{w}íf: \hld\ was im \alst{w}il-spell mikil &
te gi·\alst{h}ôrjanne, \hld\ þat im fan iro \alst{h}êrren sagda &
\alst{ę}ngil þes \alst{a}lo-walden. \hld\ Hiet sia \alst{e}ft þanan &
fan þem \alst{g}rave \alst{g}angan ęndi faran \hld\ te þem \alst{j}ungron Kristes, &
\alst{s}ęggjan þem is gi·\alst{s}ïðon \hld\ \alst{s}uoðon wordon, &
þat iro \alst{d}rohtin was \hld\ fan \alst{d}ôðe a·standan. &
Hiet ôk an \alst{s}undron \hld\ \alst{S}ímon Petruse &
\alst{w}ill-spell mikil \hld\ \alst{w}ordon ku̇ðjan, &
\alst{k}umi drohtines, \hld\ gie þat \alst{K}rist selvo &
was an \alst{G}alileo land, \hld\ „þár ina eft is \alst{j}ungron skulun, &
gi·\alst{s}ehan is gi·\alst{s}ïðos, \hld\ só hie im êr \alst{s}elvo gi·sprak &
\alst{w}árom \alst{w}ordon.“ \hld\ Reht só þuo þiu \alst{w}íf þanan &
\alst{g}angan weldun, \hld\ só stuodun im te·\alst{g}ęgnes þár &
\alst{ę}ngilos twêna \hld\ an \alst{a}la-hwíton &
\alst{w}ánamon gi·\alst{w}ádjom \hld\ ęndi sprákun im mid iro \alst{w}ordon tuo &
\alst{h}êlag-líko: \hld\ \alst{h}ugi warð gi·blôðid &
þen \alst{i}dison an \alst{ę}gison: \hld\ ne mahtun an þia \alst{ę}ngilos godes &
bi þemo \alst{w}lite skawon: \hld\ was im þiu \alst{w}ánami te strang, &%NOTE ms. -- strang] L 1v.
te \alst{s}wíði te \alst{s}ehanne. \hld\ Þuo sprákun \edtext{im \alst{s}án}{\Afootnote{so C; om. L}} an·gęgin &
\alst{w}aldandes bodun \hld\ ęndi þiu \alst{w}íf frágodun, &
te hwí sia \alst{K}ristan þarod \hld\ \alst{k}wikan mid dôdon, &
\alst{s}uno drohtines \hld\ \alst{s}uokjan kwámin &
\alst{f}erạhes \alst{f}ullan; \hld\ „nú gí ina ni \alst{f}indat hier &
an þeson \alst{st}ên-grave, \hld\ ak hie ist a·\alst{st}andan nú &
an is \alst{l}ík-hamon: \hld\ þes gí gi·\alst{l}ôvjan skulun &
ęndi gi·huggjan þero \alst{w}ordo, \hld\ þe hie iu te \alst{w}áron oft &
\alst{s}elvo \alst{s}agda, \hld\ þan hie an iuwon ge·\alst{s}ïðja was &
an \alst{G}alilea-lande, \hld\ hwó hie skoldi gi·\alst{g}evan werðan, &
gi·\alst{s}ald \alst{s}elvo \hld\ an \alst{s}undigaro manno, &
\alst{h}ęttjandero hand, \hld\ \alst{h}êlag drohtin, &
þat sea ina \alst{k}węlidin \hld\ ęndi an \alst{k}rúki slógin, &
\alst{d}ôdan gi·\alst{d}ádin \hld\ ęndi þat hie skoldi þuruh \alst{d}rohtines kraft &
an \alst{þ}riddjon dage \hld\ \alst{þ}ioda te willjan &
\alst{l}ibbjandi a·standan. \hld\ Nú havat hie all gi·\alst{l}êstid só, &
ge·\alst{f}rumid mid \alst{f}irihon: \hld\ íljat gí nú \alst{f}orð hinan, &
\alst{g}angat \alst{g}áh-líko \hld\ ęndi duot it þem is \alst{j}ungron ku̇ð.\eva

\bvb TODO.\evb\evg

\bvg\bva[70][5866]%
Hie havat sia iu fur·\alst{f}arana \hld\ ęndi ist im \alst{f}orð hinan &
an \alst{G}alileo land, \hld\ þár ina eft is \alst{j}ungron skulun, &
gi·\alst{s}ehan is ge·\alst{s}ïðos.“ \hld\ Þuo warð \edtext{\alst{s}án}{\Afootnote{so L; om. C}} after þiu &
þem \alst{w}ívon an \alst{w}illjon, \hld\ þat sia gi·hôrdun su·lik \alst{w}ord sprekan, &
\alst{k}u̇ðjan þia \alst{k}raft godes \hld\ —wárun im só a·\alst{k}umana þuo noh &
gie só \alst{f}orạhta ge·\alst{f}rumida—: \hld\ gi·witun im \alst{f}orð þanan &%NOTE ms. -- forahta] L end.
fan þem \alst{g}rave \alst{g}angan \hld\ ęndi sagdun þem \alst{j}ungron Kristes &
\alst{s}eld-lík gi·\alst{s}iuni, \hld\ þár sia \alst{s}orọgondi &
\alst{b}idun su·likero \alst{b}uota. \hld\ Þuo wurðun ôk an þia \alst{b}urg kumana &
\alst{J}udeono wardos, \hld\ þia ovar þemo \alst{g}rave sátun &
alla \alst{l}anga naht \hld\ ęndi þes \alst{l}ík-hamen þár, &
\alst{h}uodun þes \alst{h}rêwes. \hld\ Sia sagdun þero \alst{h}êri Judeono, &
hwi-lika im þár \alst{a}nd-warda \hld\ \alst{ę}gison kwámun, &
\alst{s}eld-lík gi·\alst{s}iuni, \hld\ \alst{s}agdun mid wordon, &
al só it gi·\alst{d}uan was \hld\ an þero \alst{d}rohtines kraft, &
ni \alst{m}iðun an iro \alst{m}uode. \hld\ Þuo budun im \alst{m}êðmo filo &
\alst{J}udeo liudi, \hld\ \alst{g}old ęndi silụvar, &
\alst{s}aldun im \alst{s}ink manag, \hld\ te þiu þat sia it ni \alst{s}agdin forð, &
ne \alst{m}áridin þero \alst{m}ęnigi: \hld\ „ak kweðat þat iu \alst{m}óði hugi &
an·\alst{s}wevidi mid \alst{s}lápu \hld\ ęndi þat þár kwámin is gi·\alst{s}ïðos tuo, &
far·\alst{st}álin ina an þem \alst{st}êne. \hld\ Simnen wesat gí an \alst{st}ríde mid þiu, &
\alst{f}orð an \alst{f}líte: \hld\ ef it wirðit þem \alst{f}olk-togen ku̇ð, &
wí gi·\alst{h}elpat iu wið þena \alst{h}êrosten, \hld\ þat hie iu \alst{h}armes wiht, &
\alst{l}êðes ni gi·\alst{l}êstid.“ \hld\ Þuo námun sia an þem \alst{l}iudon filo &
\alst{d}iurero mêðmo, \hld\ \alst{d}ádun all só sia bi·gunnun &
—ne gi·\alst{w}eldun iro \alst{w}illjon— \hld\ dádun só \alst{w}ído ku̇ð &
þem \alst{l}iudon after þem \alst{l}ande, \hld\ þat sia su·lika \alst{l}ugina woldun &
a·\alst{h}ębbjan be þan \alst{h}êlagan drohtin. \hld\ Þan was eft gi·\alst{h}êlid hugi &
\alst{j}ungron Kristes, \hld\ þuo sia gi·hôrdun þiu \alst{g}uodun wíf &
\alst{m}árjan þia \alst{m}aht godes; \hld\ þuo wárun sia an iro \alst{m}uode fráha, &
gie im te þem \alst{g}rave bêðja, \hld\ \alst{J}ohannes ęndi Petrus &
runnun \alst{o}vast-líko: \hld\ warð \alst{ê}r kuman &
\alst{J}ohannes þie \alst{g}uodo, \hld\ ęndi im ovar þem \alst{g}rave gi·stuod, &
ant-at þár \alst{s}án after kwam \hld\ \alst{S}ímon Petrus, &
\alst{e}rl \alst{ę}llan-ruof \hld\ ęndi im þár \alst{i}n gi·wêt &
an þat \alst{g}raf \alst{g}angan: \hld\ gi·sah þár þes \alst{g}odes barnes, &
\alst{h}rêo-gi·wádi \hld\ \alst{h}êrren sínes &
\alst{l}ínin \alst{l}iggjan, \hld\ mid þiu was êr þie \alst{l}ík-hamo &
\alst{f}agạro bi·\alst{f}angan; \hld\ lag þie \alst{f}ano sundạr, &
mit þem was þat \alst{h}ôvid bi·\alst{h}elid \hld\ \alst{h}êlages Kristes, &
\alst{r}íkjes drohtines, \hld\ þan hie an þesaro \alst{r}astu was. &
Þuo \alst{g}éng im ôk \alst{J}ohannes \hld\ an þat \alst{g}raf innan &
\alst{s}ehan \alst{s}eld-lík þing; \hld\ warð im \alst{s}án after þiu &
ant·\alst{l}okan is gi·\alst{l}ôvo, \hld\ þat hie wissa, þat skolda eft an þit \alst{l}ioht kuman &
is \alst{d}rohtin diur-líko, \hld\ fan \alst{d}ôðe a·standan &
\alst{u}p fan \alst{e}rðu. \hld\ Þuo gi·witun im \alst{e}ft þanan &
\alst{J}ohannes ęndi Petrus, \hld\ ęndi kwámun þia \alst{j}ungron Kristes, &
þia gi·\alst{s}ïðos te·\alst{s}amne. \hld\ Þan stuod \alst{s}êrag-muod &
\alst{ê}n þera \alst{i}diso \hld\ \alst{ȯ}ðer-sïðu &
\alst{g}riotandi ovar þem \alst{g}rave, \hld\ was iro \alst{j}ámar muod— &
\alst{M}aria was þat \alst{M}agdalena—, \hld\ was iro \alst{m}uod-gi·þȧht, &
\alst{s}evo mit \alst{s}orọgon gi·blandan, \hld\ ne wissa hwárod siu \alst{s}ókjan skolda &
þena \alst{h}êrron, þár iro wárun at þia \alst{h}elpa gi·langa. \hld\ Siu ni mohta þuo \alst{h}ofnu a·wísan, &
þat \alst{w}íf ni mahta \alst{w}óp for·látan: \hld\ ne wissa hwárod siu sia \alst{w}ęndjan skolda; &
gi·\alst{m}ęrrid wárun iro þes \alst{m}uod-gi·þȧhti. \hld\ Þuo gi·sah siu þena \alst{m}ahtigan þár &
\alst{K}riste standan, \hld\ þuoh siu ina \alst{k}u̇ð-líko &
ant·\alst{k}ęnnjan ni mohti, \hld\ êr þan hie ina \alst{k}u̇ðjan welda, &
\alst{s}ęggjan þat hie it \alst{s}elvo wári. \hld\ Hie frágoda hwat siu só \alst{s}êro bi·wiepi, &
só \alst{h}armo mid \alst{h}êton trahnin. \hld\ Siu kwað, þat siu umbi iro \alst{h}êrron ni wissi &
te \alst{w}áren, hwárod hie \alst{w}erðan skoldi: \hld\ „ef þú ina mí gi·\alst{w}ísan mohtis, &
\alst{f}rô mín, ef ik þik \alst{f}rágon gi·dorsti, \hld\ ef þú ina hier an þeson \alst{f}elise gi·námis, &
\alst{w}ísi ina mí mid \alst{w}ordon þínon: \hld\ þan wári mí allaro \alst{w}illjono mêsta, &
þat ik ina \alst{s}elvo gi·\alst{s}áhi.“ \hld\ Sia ni wissa, þat sia þie \alst{s}uno drohtines &
\alst{g}ruotta mid \alst{g}ódaro sprákun: \hld\ siu wánda þat it þie \alst{g}ardari wári, &
\alst{h}of-ward \alst{h}êrren sínes. \hld\ Þuo gruotta sia þie \alst{h}êlago drohtin, &
bi \alst{n}amen \alst{n}ęrjendero bęst: \hld\ siu géng im þuo \alst{n}áhor sniumo, &
þat \alst{w}íf mid \alst{w}illjon guodan, \hld\ ant·kęnda iro \alst{w}aldand selvan, &
\alst{m}íðan siu is þuru þia \alst{m}innja ni wissa: \hld\ welda ina mid iro \alst{m}undon grípan, &
þiu \alst{f}êhmja an þena \alst{f}olko drohtin, \hld\ novan þat iro \alst{f}riðu-barn godes &
\alst{w}ęrida mid \alst{w}ordon sínon, \hld\ kwað þat siu ina mid \alst{w}ihti ni mósti &
\alst{h}andon ant·\alst{h}rínan: \hld\ „ik ni stêg noh“, kwaþ-hie, „te þem \alst{h}imiliskon fader; &
ak \alst{í}li þú nú \alst{o}fst-líko \hld\ ęndi þem \alst{e}rlon ku̇ði, &
\alst{b}ruoðron mínon, \hld\ þat ik u̇ser \alst{b}êðero fader &
\alst{a}la-waldan, \hld\ \alst{i}uwan ęndi mínan &
\alst{s}uȯð-fastan god \hld\ \alst{s}uokjan willju.“\eva

\bvb TODO.\evb\evg

\bvg\bva[71][5941]%
Þat \alst{w}íf warð þuo an \alst{w}unnon, \hld\ þat siu muosta su·likan \alst{w}illjon ku̇ðjan, &
\alst{s}ęggjan fan im gi·\alst{s}undon: \hld\ warð \alst{s}án garo &
þiu \alst{i}dis an þat \alst{â}rundi \hld\ ęndi þem \alst{e}rlon brȧhta, &
\alst{w}ill-spel \alst{w}eron, \hld\ þat siu \alst{w}aldand Krist &
gi·\alst{s}undan gi·\alst{s}áwi, \hld\ ęndi sagda hwó hé iru \alst{s}elvo gi·bôd &
\alst{t}orọhtero \alst{t}êkno. \hld\ Sia ni weldun gi·\alst{t}rúojan þuo noh &
þes \alst{w}íves \alst{w}ordon, \hld\ þat siu su·lik \alst{w}ill-spel brȧhte &
\alst{g}egnungo fan þemo \alst{g}odes suno, \hld\ ak sia sátun im \alst{j}ámor-muoda, &
\alst{h}ęliðos \alst{h}riuwonda. \hld\ Þuo warð þie \alst{h}êlago Krist &
eft \alst{o}pan-líko \hld\ \alst{ȯ}ðer-sïðu, &
\alst{d}rohtin gi·tôgid, \hld\ sïðor hie fan \alst{d}ôðe a·stuod, &
þan \alst{w}ívon an \alst{w}illjon, \hld\ þat hie im þár an \alst{w}ege muotta. &
\alst{k}wędda sia \alst{k}u̇ð-líko, \hld\ ęndi sia te is \alst{k}neohon hnigun, &
\alst{f}ellun im tó \alst{f}uoton. \hld\ Hie hét þat sia \alst{f}orạhtan hugi &
ne \alst{b}árin an iro \alst{b}rioston: \hld\ „ak gí mínon \alst{b}ruoðron skulun &
þesa \alst{k}widi \alst{k}u̇ðjan, \hld\ þat sia \alst{k}uman after mí &
an \alst{G}alileo land; \hld\ þár ik im eft te·\alst{g}ęgnes biun.“ &
Þan fuorun im ôk fan \alst{J}erusalem \hld\ þero \alst{j}ungrono twêna &
an þem \alst{s}elvon daga \hld\ \alst{s}án an morgan, &
\alst{e}rlos an iro \alst{â}rundi: \hld\ weldun im te \alst{E}maus &
þat \alst{k}astel suokan. \hld\ Þuo bi·gunnun im \alst{k}widi managa &
under þem \alst{w}eron \alst{w}ahsan, \hld\ þár sia after þem \alst{w}ege fuorun, &
þem \alst{h}ęliðon umbi iro \alst{h}êrron. \hld\ Þuo kwam im þár þie \alst{h}êlago tuo &
\alst{g}angandi \alst{g}odes suno. \hld\ Sia ni mahtun ina \alst{g}aro-líko &
ant·\alst{k}ęnnan \alst{k}raftigna: \hld\ hie ni welda ina þuo noh \alst{k}u̇ðjan te im; &
was im þoh an iro gi·\alst{s}ïðje \alst{s}amad \hld\ ęndi frágoda, umbi hwi-lika sia \alst{s}aka sprákin: &
„hwí \alst{g}angat gí só \alst{g}ornondja?“ \hld[kwaþ-hie;] „Ist ink \alst{j}ámer hugi, &
\alst{s}evo \alst{s}orạgono full.“ \hld\ Sia sprákun im \alst{s}án an·gęgin, &
þia \alst{e}rlos \alst{a}nd·wurdi: \hld\ „te hwí þú þes \alst{ê}skos só“, kwáðun sia; &
„bist þí fan \alst{J}erusalem \hld\ \alst{J}udeono folkas &
\skipnumbering{[...]}“\eva

\bvb TODO.\evb\evg

\bvg\bva[][5971]%
\skipnumbering„{[...]} &
\edtext{\edtext{\alst{h}êlagumu gêste \hld\ fan \alst{h}evan-wange}{\Afootnote{Partly scraped off, but still just about readable in \textbf{M}.}}, &
mid þem \alst{g}rôtun \alst{g}odes kraft.“ \hld\ Nam is \alst{j}ungaron þó, &
\alst{e}rlos góde, \hld\ lêdda sie \alst{ú}t þanan, &
ant-tat hé sie \alst{b}rȧhte \hld\ an \alst{B}ethanía; &
þár \alst{h}óf hé is \alst{h}ęndi up \hld\ ęndi \alst{h}êlegoda sie alle, &
\alst{w}íhida sie mid is \alst{w}ordun. \hld\ Gi·\alst{w}êt imo up þanan, &
sóhta imo þat \alst{h}ôha \alst{h}imilo ríki \hld\ ęndi þena is \alst{h}êlagon stól: &
\alst{s}itit imo þár \hld\ an þea \alst{s}wíðron half godes, &
\alst{a}lo-mahtiges fader \hld\ ęndi þanan \alst{a}ll ge·sihit &
\alst{w}aldandjo Krist, \hld\ só hwat só þius \alst{w}er-old be·havet. &
Þó an þeru \alst{s}elvon stędi \hld\ ge·\alst{s}ïðos góde &
te \alst{b}edu fellun \hld\ ęndi im eft te \alst{b}urg þanan &
þár te \alst{J}erusalem \hld\ \alst{j}ungaron Kristes &
\alst{f}órun \alst{f}aganondi: \hld\ was im \alst{f}ráh-mód hugi, &
\alst{w}árun im þár at þemu \alst{w}íhe. \hld\ \alst{W}aldandes kraft}{\lemma{hêlagumu \dots\ kraft}\Afootnote{Only in \textbf{M}.}} &
\edtext{[...]}{\Afootnote{Four lines are scraped off and entirely illegible in \textbf{M}.}}\eva

\bvb TODO.\evb\evg

\sectionline
