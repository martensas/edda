 ǫ́ðarb·útanlòvo ant· te· un· bi·hòrj\bookStart{Heliand}

Manuscripts in chronological order:
  Source: https://escholarship.org/content/qt19k1z5h8/qt19k1z5h8_noSplash_88c7cf5d8c1e26a95c76028adf012df9.pdf?t=mtfq0h p. 11.
  See p. 66 for parallel texts with C, M, P and L.


  L 840–850 (Thomas 4073 (Ms))
  P 840–850 (R 56/2537 (PA))
  V 800–850 (Palatini Latini 1447)
  S 850 (cgm. 8840)
  M 850–875 (cgm. 25)
  C 950–1000 (Cotton Caligula A. VII sign. 3-11)

Fragments L and P appear to originally belong to the same codex?


Notes on the normalization:
  \begin{itemize}
    \item Long vowels are marked by the acute rather than by the circumflex. This is both faithful to the original manuscripts and concordant with my practice in normalising other Germanic languages.
    \item Long vowels \emph{ò} and \emph{è} resulting from monophthongisation of \emph{au} and \emph{ai} are, however, written with the grave.
    \item When attested in all mss., epenthetic (svarabhakti) vowels are written with a dot beneath them. Otherwise they are deleted.
    \item ms. \emph{e} and \emph{i}, when occuring, between vowels are written as \emph{j}.
    \item ms. \emph{e} as resulting from \emph{i}-mutation is written as \emph{ę}.
    \item ms. \emph{b} or \emph{ƀ}, when representing the voiced bilabial fricative, is written as \emph{v}.
    \item ms. \emph{th} is written as \emph{þ}.
    \item ms. \emph{uu} is written as \emph{w}.
  \end{itemize}

\sectionline

Manega wáron, \hld\ þe sia iro mód ge·spón,
þat sia bi—gunnun word godes,
rękkjan þat gi—rúni, \hld\ þat þie ríkjo Krist
undar man-kunnja \hld\ máriða gi·frumida
mid wordun endi mid werkun. \hld\ Þat wolda þó wísara filo
liudo barno lovon, \hld\ lèra Kristes,
hèlag word godas, \hld\ endi mid iro handon skrívan
berẹht-líko an buok, \hld\ hwó sia is gi·bod-skip skoldin
frummjan, firiho barn. \hld\ Þan wárun þoh sia fiori te þiu
under þera menigo, \hld\ þia habdon maht godes,
helpa fan himila, \hld\ hèlagna gèst,
kraft fan Kriste; \hld\ sia wurðun gi·korana te þio,
þat sie þan Éwangelium \hld\ ènan skoldun
an buok skrívan \hld\ endo só manag gi·bod godes,
hèlag himilisk word: \hld\ sia ne muosta hęliðo þan mér,
firiho barno frummjan, \hld\ neuan þat sia fiori te þio
þuru kraft godas \hld\ ge·korana wurðun,
Matheus endi Markus, \hld\ —só wárun þia man hètana—
Lukas endi Johannes; \hld\ sia wárun gode lieva,
wirðiga ti þem gi·wirkje. \hld\ Habda im waldand god,
þem hęliðon an iro hertan \hld\ hèlagna gèst
fasto bi·folhan \hld\ endi ferahtan hugi,
só manag wís-lík word \hld\ endi gi·wit mikil,
þat sea skoldin a·hębbjan \hld\ hèlagaro stemnun
god-spell þat guoda, \hld\ þat ni havit ènigan gi·gadon hwergin,
þiu word an þesaro wer-oldi, \hld\ þat io waldand mér,
drohtin diurje \hld\ efþo dervi þing,
firin-werk fellje \hld\ efþo fíundo níð,
stríd wiðer-stande—, \hld\ hwand hie habda starkan hugi,
mildjan endi guodan, \hld\ þie þe mèster was,
aðal-ord-frumo \hld\ alo-mahtig.
Þat skoldun sea fiori \hld\ þuo fingron skrívan,
settjan endi singan \hld\ endi sęggjan forð,
þat sea fan Kristes \hld\ krafte þem mikilon
gi·sáhun endi gi·hòrdun, \hld\ þes hie selvo gi·sprak,
gi·wísda endi gi·warahta, \hld\ wundạr-líkas filo,
só manag mid mannon \hld\ mahtig drohtin,
all so hie it fan þem an·ginne \hld\ þuru is ènes kraht,
waldand gi·sprak, \hld\ þuo hie èrist þesa wer-old gi·skuop
endi þuo all bi·fieng \hld\ mid ènu wordo,
himil endi erða \hld\ endi al þat sea bi·hlidan égun
gi·warahtes endi gi·wahsanes: \hld\ þat warð þuo all mid wordon godas
fasto bi·fangan, \hld\ endi gi·frumid after þiu,
hwi-lik þan liud-skępi \hld\ landes skoldi
wídost gi·waldan, \hld\ efþo hwar þiu wer-old-aldar
endon skoldin. \hld\ Én was iro þuo noh þan
firiho barnun bi·foran, \hld\ endi þiu fívi wárun a·gangan:
skolda þuo þat sehsta \hld\ sálig-líko
kuman þuru kraft godes \hld\  endi Kristas gi·burd,
hélandero bestan, \hld\ hèlagas gèstes,
an þesan middil-gard \hld\ managon te helpun,
firjo barnon ti frumon \hld\ wið fíundo níð,
wið dernero dwalm. \hld\ Þan habda þuo drohtin god
Rómano-liudjon far·liwan \hld\ ríkjo mèsta,
habda þem heri-skipje \hld\ herta gi·sterkid,
þat sia habdon bi·þwungana \hld\ þiedo gi·hwi-lika,
habdun fan Rúmu-burg \hld\ ríki gi·wunnan
helm-gi·tróstjon, \hld\ sáton iro hęri-togon
an lando gi·hwem, \hld\ habdun liudjo gi·wald,
allon ęli-þeodon. \hld\ Erodes was
an Hjerusalem \hld\ over þat Judeono folk
gi·koran te kuninge, \hld\ só ina þie kèser þarod,
fon Rúmu-burg \hld\ ríki þiodan
satta undar þat gi·síði. \hld\ Hie ni was þoh mid sibbjon bi·lang
avaron Israheles, \hld\ eðili-gi·burdi,
kuman fon iro knuosle, \hld\ neuan þat hie þuru þes kèsures þank
fan Rúmu-burg \hld\ ríki habda,
þat im wárun só gi·hòriga \hld\ hildi-skalkos,
avaron Israheles \hld\ ęlljan-ruova:
swíðo un·wanda wini, \hld\ þan lang hie gi·wald éhta,
Erodes þes ríkjas \hld\ endi rád-burdjon held
Judeo liudi. \hld\ Þan was þar èn gi·gamalod mann,
þat was fruod gomo, \hld\ habda ferehtan hugi,
was fan þem liudjon \hld\ Lewias kunnes,
Jakobas sunjas, \hld\ guodero þiedo:
Zakharias was hie hètan. \hld\ Þat was só sálig man,
hwand hie simblon gerno \hld\ gode þeonoda,
warahta after is willjon; \hld\ deda is wíf só self
—was iru gi·aldrod idis: \hld\ ni muosta im ęrvi-ward
an iro juguð-hédi \hld\ giviðig werðan—
libdun im far·úter laster, \hld\ waruhtun lof goda,
wárun só gi·hòriga \hld\ hevan-kuninge,
diuridon úsan drohtin: \hld\ ni weldun dervjas wiht
under man-kunnje, \hld\ mènes gi·frummjan,
ne *saka ne sundja; \hld\ was im þoh an sorgun hugi,
þat sie ęrvi-ward \hld\ ègan ni móstun,
ak wárun im barno-lòs. \hld\ Þan skolda he gi·bod godes
þar an Hjerusalem, \hld\ só oft só is gi·gengi gi·stód,
þat ina torht-líko \hld\ tídi gi·manodun,
só skolda he at þem wíha \hld\ waldandes geld
hèlag bi·hwervan, \hld\ hevan-kuninges,
godes jungar-skępi: \hld\ gern was he swíðo,
þat he it þurh ferhtan hugi \hld\ frummjan mósti.
Þó warð þiu tíd kuman, \hld\ —þat þar gi·tald habdun %NOTE: Fitt 2.
wísa man mid wordun,— \hld\ þat skolda þana wíh godes
Zakharias bi·sehan. \hld\ Þó warð þar gi·samnod filu
þar te Hjerusalem \hld\ Judeo liudi,
werodes te þem wíha, \hld\ þar sie waldand god
swíðo þeo-líko \hld\ þiggjan skoldun,
hérron is huldi, \hld\ þat sie hevan-kuning
lèðes a·léti. \hld\ Þea liudi stódun
umbi þat hèlaga hús, \hld\ endi geng im þe gi·hérodo man
an þana wíh innan. \hld\ Þat werod ǫ́ðar béd
umbi þana alah útan, \hld\ Ebreo liudi,
hwan ér þe fródo man \hld\ gi·frumid habdi
waldandes willjon. \hld\ Só he þó þana wí-ròk dróg,
ald aftar þem alaha, \hld\ endi umbi þana altari geng
mid is ròk-fatun \hld\ ríkjun þionon,
—fremida ferht-líko \hld\ fráon sínes,
godes jungar-skępi \hld\ gerno swíðo
mid hluttru hugi, \hld\ *só man hérren skal
gerno ful-gangan—, \hld\ grurios kwámun im,
egison an þem alahe: \hld\ hie gi·sah þar aftar þiu ènna ęngil godes
an þem wíhe innan, \hld\ hie sprak im mid is wordun tuo,
hiet þat fruod gumo \hld\ foroht ni wári,
hiet þat hie im ni and-riede: \hld\ þína dádi sind“, kwat-hie*,
„waldanda werðe \hld\ endi þín word só self,
þín þionost is im an þanke, \hld\ þat þu su·lika gi·þáht haves
an is ènes kraft. \hld\ Ik is ęngil bium,
Gabriel bium ik hètan, \hld\ þe gio for goda standu,
and-ward for þem alo-waldon, \hld\ ne sí þat he me an is árundi hwarod
sęndjan willja. \hld\ Nu hiet he me an þesan síð faran,
hiet þat ik þi þoh gi·kúðdi, \hld\ þat þi kind gi·boran,
fon þínera alderu idis \hld\ ódan skoldi
werðan an þesero wer-oldi, \hld\ wordun spáhi.
Þat ni skal an is liva gio \hld\ líðes an·bítan,
wínes an is wer-oldi: \hld\ só haved im wurd-gi·skapu,
metod gi·markod \hld\ endi maht godes.
Hét þat ik þi þoh sagdi, \hld\ þat it skoldi gi·síð wesan
hevan-kuninges, \hld\ hét þat git it heldin wel,
tuhin þurh trewa, \hld\ kwað þat he im tíras só filu
an godes ríkja \hld\ for·gevan weldi.
He kwað þat þe gódo gumo \hld\ Johannes te namon
hębbjan skoldi, \hld\ gi·bòd þat git it hétin só,
þat kind, þan it kwámi, \hld\ kwað þat it Kristes gi·síð
an þesaro wídun wer-old \hld\ werðan skoldi,
is selves sunjes, \hld\ endi kwað þat sie sliumo herod
an is bod-skępi \hld\ béðe kwámin.“
Zakharias þó gi·mahalda \hld\ endi wið selvan sprak
drohtines ęngil, \hld\ endi im þero dádjo bi·gan,
wundron þero wordo: \hld\ „hwó mag þat gi·werðan só“, kwað he,
„aftar an aldre? \hld\ it is unk al te lat
só te gi·winnanne, \hld\ só þu mid þínun wordun gi·sprikis.
Hwanda wit habdun aldres \hld\ ér efno twèn-tig
wintro an unkro wer-oldi, \hld\ ér þan kwámi þit wíf te mi;
þan wárun wit nu at-samna \hld\ ant·sivunta wintro
gi·bęnkjon endi gi·będdjon, \hld\ siðor ik sie mi te brúdi ge·kòs.
Só wit þes an unkro juguði \hld\ gi·girnan ni mohtun,
þat wit ęrvi-ward \hld\ ègan móstin,
fódjan an unkun flęttja, \hld\ nu wit sus gi·fródod sint
—havad unk eldi bi·noman \hld\ ęlljan-dádi,
þat wit sint an unkro siuni gi·slekit \hld\ endi an unkun sídun lat;
flésk is unk ant·fallan, \hld\ fel un·skóni,
is unka lud gi·liðen, \hld\ lík gi·drusnod,
sind unka and-bári \hld\ ǫ́ðar-líkaron,
mód endi męgin-kraft—, \hld\ só wit giu só managan dag
wárun an þesero wer-oldi, \hld\ só mi þes wundạr þunkit,
hwó it só gi·werðan mugi, \hld\ só þu mid þínun wordun gi·sprikis.
Þó warð þat heven-kuninges bodon \hld\ harm an is móde,%NOTE: Fitt 3.
þat he is gi·werkes \hld\ só wundron skolda
endi þat ni welda gi·huggjan, \hld\ þat ina mahta hèlag god
só ala-jungan, \hld\ só he fon èrist was,
selvo gi·wirkjan, \hld\ of he só weldi.
Skerida im þó te wítja, \hld\ þat he ni mahte ènig word sprekan,
gi·mahljen mid is múðu, \hld\ „ér þan þi magu wirðid,
fon þínero aldero idis \hld\ erl a·fódit,
kind-jung gi·boran \hld\ kunnjes gódes,
wánum te þesero wer-oldi. \hld\ Þan skalt þu eft word sprekan,
hębbjan þínaro stemna gi·wald; \hld\ ni þarft þu stum wesan
lengron hwíla.“ \hld\ Þó warð it sán gi·léstid só,
gi·worðan te wáron, \hld\ só þar an þem wíha gi·sprak
ęngil þes alo-waldon: \hld\ warð ald gumo
spráka bi·lòsit, \hld\ þoh he spáhan hugi
bári an is breostun. \hld\ Bidun allan dag
þat werod for þem wíha \hld\ endi wundrodun alla,
bi·hwí he þar só lango, \hld\ lof-sálig man,
swíðo fród gumo \hld\ fráon sínun
þionon þorfti, \hld\ só þar ér ènig þegno ni deda,
þan sie þar at þem wíha \hld\ waldandes geld
folmon frumidun. \hld\ Þó kwam fród gumo
út fon þem alaha. \hld\ Erlos þrungun
náhor mikilu: \hld\ was im niud mikil,
hwat he im sǫ́ð-líkes \hld\ sęggjan weldi,
wísjan te wáron. \hld\ He ni mohta þó ènig word sprekan,
gi·sęggjan þem gi·siðea, \hld\ b·útan þat he mid is swíðron hand
wísda þem weroda, \hld\ þat sie úses waldandes
lèra léstin. \hld\ Þea liudi for·stódun,
þat he þar habda gegnungo \hld\ god-kundes hwat
for·sehen selvo, \hld\ þoh he is ni mahti gi·sęggjan wiht,
gi·wísjan te wáron. \hld\ Þó habda he úses waldandes
geld gi·léstid, \hld\ al só is gi·gengi was
gi·markod mid mannun. \hld\ Þó warð sán aftar þiu maht godes,
gi·kúðid is kraft mikil: \hld\ warð þiu kwán ókan,
idis an ira eldju: \hld\ skolda im ęrvi-ward,
swíðo god-kund gumo \hld\ giviðig werðan,
barn an burgun. \hld\ Béd aftar þiu
þat wíf wurdi-gi·skapu. \hld\ Skréd þe wintạr forð,
geng þes géres gi·tal. \hld\ Johannes kwam
an liudjo lioht: \hld\ lík was im skóni,
was im fel fagar, \hld\ fahs endi naglos,
wangun wárun im wlitige. \hld\ Þó fórun þar wíse man,
snelle te·samne, \hld\ þea swásostun mèst,
wundrodun þes werkes, \hld\ bi·hwí it gio mahti gi·werðan só,
þat undar só aldun twèm \hld\ ódan wurði
barn an gi·burdjon, \hld\ ni wári þat it gi·bod godes
selves wári: \hld\ af·suovun sie garo,
þat it elkor só wán-lík \hld\ werðan ni mahti.
Þó sprak þar èn gi·fródot man, \hld\ þe só filo konsta
wísaro wordo, \hld\ habde gi·wit mikil,
frágode niud-líko, \hld\ hwat is namo skoldi
wesan an þesaro wer-oldi: \hld\ „mi þunkid an is wísu gi·lík
iak an is gi·bárja, \hld\ þat he sí bętara þan wi,
só ik wániu, \hld\ þat ina ús gegnungo god fon himila
selvo sendi“. \hld\ Þó sprak sán aftar
þiu módar þes kindes, \hld\ þiu þana magu habda,
þat barn an ire barme: \hld\ „hér kwam gi·bod godes“, kwað siu,
„fernun gére, \hld\ furmon wordu
gi·bòd, þat he Johannes \hld\ bi godes lèrun
hètan skoldi. \hld\ Þat ik an mínumu hugi ni gi·dar
węndjan mid wihti, \hld\ of ik is gi·waldan mót“.
Þó sprak èn gèl-hert man, \hld\ þe ira gaduling was:
„ne hét ér gio·wiht só“, \hld\ kwað he, „aðal-boranes
úses kunnjes efþo knósles; \hld\ wita kiasan im óðrana
niud-samna namon: \hld\ he niate of he móti“.
Þó sprak eft þe fródo man, \hld\ þe þar konsta filo mahljan:
„ni givu ik þat te ráde“, \hld\ kwað he, „rinko neg·ènun,
þat he word godes \hld\ węndjan biginna;
ak wita is þana fader frágon, \hld\ þe þar só gi·fródod sitit,
wís an is wín-sęli: \hld\ þoh he ni mugi ènig word sprekan,
þoh mag he bi bók-stavon \hld\ bréf ge·wirkjan,
namon gi·skrívan“. \hld\ Þó he náhor geng,
legda im èna bók an barm \hld\ endi bad gerno
wrítan wís-líko \hld\ word-gi·merkjun,
hwat sie þat hèlaga barn \hld\ hètan skoldin.
Þó nam he þia bók an hand \hld\ endi an is hugi þáhte
swíðo gerno te gode: \hld\ Johannes namon
wís-líko gi·wrét \hld\ endi ók aftar mid is wordu gi·sprak
swíðo spáh-líko: \hld\ habda im eft is spráka gi·wald,
gi·wittjas endi wísun. \hld\ Þat wíti was þó a·gangan,
hard harm-skare, \hld\ þe im hèlag god
mahtig makode, \hld\ þat he an is mód-sevon
godes ni for·gáti, \hld\ þan he im eft sendi is jungron tó.
Þó ni was lang aftar þiu, \hld\ ne it al só gi·léstid warð,%NOTE: Fitt 4.
só he man-kunnja \hld\ managa hwíla,
god alo-mahtig \hld\ for·geven habda,
þat he is himilisk barn \hld\ herod te wer-oldi,
si selves sunu \hld\ sęndjan weldi,
te þiu þat he hér a·lòsdi \hld\ al liud-stamna,
werod fon wítja. \hld\ Þó warð is wisbodo
an Galilea-land, \hld\ Gabriel kuman,
ęngil þes alo-waldon, \hld\ þar he ène idis wisse,
muni-líka magað: \hld\ Maria was siu hèten,
was iru þiorna gi·þigan. \hld\ Sea èn þegạn habda,
Joseph gi·mahlit, \hld\ gódes kunnjes man,
þea Dawides dohter: \hld\ þat was só diur-lík wíf,
idis ant·héti. \hld\ Þar sie þe ęngil godes
an Nazareth-burg \hld\ bi namon selvo
grótte gegin-warde \hld\ endi sie fon gode kwędda:
„Hél wis þu, Maria“, \hld\ kwað he, „þu bist þínun hérron liof,
waldande wirðig, \hld\ hwand þu gi·wit haves,
idis enstio fol. \hld\ Þu skalt for allun wesan
wívun gi·wíhit. \hld\ Ne have þu wèkan hugi,
ne forhti þu þínun ferhe: \hld\ ne kwam ik þi te ènigun fréson herod,
ne dragu ik ènig drugi·þing. \hld\ Þu skalt úses drohtines wesan
módar mid mannun \hld\ endi skalt þana magu fódjan,
þes hòhon hevan-kuninges suno. \hld\ Þe skal hèljand te namon
ègan mid eldjun. \hld\ Neo endi ni kumid,
þes wídon ríkjas gi·wand, \hld\ þe he gi·waldan skal,
mári þeodan.“ \hld\ Þó sprak im eft þiu magað an·gegin,
wið þana ęngil godes \hld\ idiso skónjost,
allaro wívo wlitigost: \hld\ „hwó mag þat gi·werðen só“, kwað siu,
„þat ik magu fódje? \hld\ Ne ik gio mannes ni warð
wís an mínera wer-oldi.“ \hld\ Þó habde eft is word garu
ęngil þes alo-waldon \hld\ þero idisiu te·gegnes:
„an þi skal hèlag gèst \hld\ fon hevan-wange
kuman þurh kraft godes. \hld\ Þanan skal þi kind ódan
werðan an þesaro wer-oldi; \hld\ waldandes kraft
skal þi fon þem hòhoston \hld\ hevan-kuninge
skadowan mid skimon. \hld\ Ni warð skónjera gi·burd,
ne só mári mid mannun, \hld\ hwand siu kumid þurh maht godes
an þese wídon wer-old.“ \hld\ Þó warð eft þes wíbes hugi
aftar þem árundje \hld\ al gi·hworven
an godes willjon. \hld\ „Þan ik hér garu standu“, kwað siu,
„te su·likun ambaht-skępi, \hld\ só he mi ègan wili.
Þiu bium ik þeot-godes. \hld\ Nu ik þeses þinges gi·trúon;
werðe mi aftar þínun wordun, \hld\ al só is willjo sí,
hérron mínes; \hld\ nis mi hugi twífli,
ne word ne wísa.“ \hld\ Só gi·fragn ik, þat þat wíf ant·feng
þat godes árundi \hld\ gerno swíðo
mid leohtu hugi \hld\ endi mid gi·lòvon gódun
endi mid hluttrun trewun; \hld\ warð þe hèlago gèst,
þat barn an ira bósma; \hld\ endi siu ira breostun for·stód
iak an ire sevon selvo, \hld\ sagda þem siu welda,
þat sie habde giókana \hld\ þes alo-waldon kraft
hèlag fon himile. \hld\ Þó warð hugi Josepes,
is mód gi·worrid, \hld\ þe im ér þea magað habda,
þea idis ant·héttja, \hld\ aðal-knósles wíf
gi·boht im te brúdju. \hld\ He afsóf þat siu habda barn undar iru:
ni wánda þes mid wihti, \hld\ þat iru þat wíf habdi
gi·wardod só waro-líko: \hld\ ni wisse waldandes þó noh
blíði gi·bod-skępi. \hld\ Ni welda sia imo te brúdi þó,
halon imo te híwon, \hld\ ak bi·gan im þó an hugi þęnkjan,
hwó he sie só for·léti, \hld\ só iru þar nu wurði lèdes wiht,
ódan arvides. \hld\ Ni welda sie aftar þiu
meldon for męnigi: \hld\ antd-réd þat sie manno barn
lívu binámin. \hld\ Só was þan þero liudjo þau
þurh þen aldon éu, \hld\ Ebreo folkes,
só hwi-lik só þar an un·reht \hld\ idis gi·híwida,
þat siu simbla þana bed-skępi \hld\ buggjan skolda,
frí mid ira ferhu: \hld\ ni was gio þiu fémea só gód,
þat siu mid þem liudun leng \hld\ libbjen mósti,
wesan undar þem weroda. \hld\ Bi·gan im þe wíso mann,
swíðo gód gumo, \hld\ Joseph an is móda
þęnkjan þero þingo, \hld\ hwó he þea þiornun þó
listjun for·léti. \hld\ Þó ni was lang te þiu,
þat im þar an dròma \hld\ kwam drohtines ęngil,
hevan-kuninges bodo, \hld\ endi hét sie ina haldan wel,
minnjon sie an is móde: \hld\ „Ni wis þu“, kwað he, „Mariun wrèð,
þiornun þínaro; \hld\ siu is gi·þungan wíf;
ne for·hugi þu sie te hardo; \hld\ þu skalt sie haldan wel,
wardon ira an þesaro wer-oldi. \hld\ Lésti þu inka wini-trewa
forð só þu dádi, \hld\ endi hald inkan friund-skępi wel!
Ne lát þu sie þi þiu lèðaron, \hld\ þoh siu undar ira liðon égi,
barn an ira bósma. \hld\ It kumid þurh gi·bod godes,
hèlages gèstes \hld\ fon hevan-wanga:
þat is Jésu Krist, \hld\ godes ègan barn,
waldandes sunu. \hld\ Þu skalt sie wel haldan,
hèlag-líko. \hld\ Ne lát þu þi þínan hugi twífljen,
merrjan þína mód-gi·þáht.“ \hld\ Þó warð eft þes mannes hugi
gi·wendid aftar þem wordun, \hld\ þat he im te þem wíva genam,
te þera magað minnja: \hld\ ant·kenda maht godes,
waldandes gi·bod; \hld\ was im willjo mikil,
þat he sia só hèlag-líko \hld\ haldan mósti:
bi·sorgoda sie an is gi·síðea, \hld\ endi siu só súvro dróg
al te huldi godes \hld\ hèlagna gèst,
gód-líkan gumon, \hld\ antþat sie godes gi·skapu
mahtig gi·manodun, \hld\ þat siu ina an manno lioht,
allaro barno bętst, \hld\ brengjan skolda.
Þó warð fon Rúmu-burg \hld\ ríkes mannes%NOTE: Fitt 5.
ovar alla þesa irmin-þiod \hld\ Oktawiánas
ban endi bod-skępi \hld\ ovar þea is brèdon gi·wald
kuman fon þem kèsure \hld\ kuningo gi·hwi-likun,
hèm-sittjandjun, \hld\ só wído só is hęri-togon
ovar al þat land-skępi \hld\ liudjo gi·weldun.
Hiet man þat alla þea ęli-lęndjun man \hld\ iro óðil sóhtin,
hęliðos iro hand-mahal \hld\ an·gegen iro hérron bodon,
kwámi te þem knósla gi·hwe, \hld\ þanan he kunnjas was,
gi·boran fon þem burgjun. \hld\ Þat gi·bod warð gi·léstid
ovar þesa wídon wer-old; \hld\ werod samnoda
te allaro burgeo gi·hwem. \hld\ Fórun þea bodon ovar all,
þea fon þem kèsura \hld\ kumana wá*run,
bók-spáha weros, \hld\ endi an bréf skrivun
swíðo niud-líko \hld\ namono gi·hwi-likan,
ia land ia liudi, \hld\ þat im ni mahti a·lettjan mann
gumono su·lika gambra, \hld\ só im skolda geldan gi·hwe
hęliðo fon is hòvda. \hld\ Þó gi·wèt im ók mid is híwiska
Joseph þe gódo, \hld\ só it god mahtig,
waldand welda: \hld\ sóhta im þiu wánamon hèm,
þea burg an Bethleem, \hld\ þar iro beiðero was,
þes hęliðes hand-mahal* \hld\ endi ók þera hèlagun þiornun,
Mariun þera gódun. \hld\ Þar was þes márjon stól
an ér-dagun, \hld\ aðalkuninges,
Dawides þes gódon, \hld\ þan langa þe he þana druht-skępi þar,
erl undar Ebreon \hld\ ègan mósta,
haldan hòh-gi·setu. \hld\ Sie wárun is híwiskas,
kuman fon is knósla, \hld\ kunnjas gódes,
bèðju bi gi·burdjun. \hld\ Þar gi·fragn ik, þat sie þiu berhtun gi·skapu,
Mariun gi·manodun \hld\ *endi maht godes,
þat iru an þem síða \hld\ sunu ódan warð,
gi·boran an Bethleem \hld\ barno strangost,
allaro kuningo kraftigost: \hld\ kuman warð þe márjo,
mahtig an manno lioht, \hld\ só is ér managan dag
biliði wárun \hld\ endi bókno filu
gi·worðen an þesero wer-oldi. \hld\ Þó was it all gi·wárod só,
só it ér spáha man \hld\ gi·sprokan habdun,
þurh hwi-lik òd-módi \hld\ he þit erð-ríki herod
þurh is selves kraft \hld\ sókjan welda,
managaro mund-boro. \hld\ Þó ina þiu módar nam,
bi·wand ina mid wádju \hld\ wívo skónjost,
fagaron fratahun, \hld\ endi ina mid iro folmon twèm
legda liov-líko \hld\ luttilna man,
þat kind an èna kribbjun, \hld\ þoh he habdi kraft godes,
manno drohtin. \hld\ Þar sat þiu módar bi·foran,
wíf wakogjandi, \hld\ war*doda selvo,
held þat hèlaga barn: \hld\ ni was ira hugi twífli,
þera magað ira mód-sevo. \hld\ Þó warð þat managun kúð
ovar þesa wídon wer-old, \hld\ wardos ant·fundun,
þea þar ehu-skalkos \hld\ úta wárun,
weros an wahtu, \hld\ wiggjo gòmjan,
fehas aftar fel*da: \hld\ gi·sáhun finistri an twè
te·látan an lufte, \hld\ endi kwam lioht godes
wánum þurh þiu wolkan \hld\ endi þea wardos þar
bi·feng an þem felda. \hld\ Sie wurðun an forhtun þó,
þea man an ira móda: \hld\ gi·sáhun þar mahtigna
godes ęngil kuman, \hld\ þe im te·gegnes sprak,
hét þat im þea wardos \hld\ wiht ne antd-rédin
lèðes fon þem liohta: \hld\ „ik skal eu“, kwað he, „liovara þing,
swíðo wár-líko \hld\ willjon sęggjan,
kúðjan kraft mikil: \hld\ nu is Krist ge·boran
an þeser*o selvun naht, \hld\ sálig barn godes,
an þera Dawides burg, \hld\ drohtin þe gódo.
Þat is mendislo \hld\ manno kunnjas,
allaro firiho fruma. \hld\ Þar gi ina fíðan mugun,
an Bethlema-burg \hld\ barno ríkjost:
hębbjad þat te tèkna, \hld\ þat ik eu gi·tęlljan mag
wárun wordun, \hld\ þat he þar bi·wundan ligid,
þat kind an ènera kribbjun, \hld\ þoh he sí kuning ovar al
erðun endi himiles \hld\ endi ovar eldeo barn,
wer-oldes waldand“. \hld\ Reht só he þó þat word gi·sprak,
só warð þar ęngilo te þem ènun \hld\ un·rím kuman,
hèlag hęri-skępi \hld\ fon hevan-wanga,
fagar folk godes, \hld\ endi filu sprákun,
lof-word manag \hld\ liudjo hérron.
Af·hóvun þó hèlagna sang, \hld\ þó sie eft te hevan-wanga
wundun þurh þiu wolkan. \hld\ Þea wardos hòrdun,
hwó þiu ęngilo kraft \hld\ alo-mahtigna god
swíðo werð-líko \hld\ wordun lovodun:
„diuriða sí nu“, \hld\ kwáðun sie, „drohtine selvun
an þem hòhoston \hld\ himilo ríkja
ęndi friðu an erðu \hld\ firiho barnun,
gód-willigun gumun, \hld\ þem þe god ant·kęnnjad
þurh hluttran hugi.“ \hld\ Þea hirdjo for·stódun,
þat sie mahtig þing \hld\ gi·manod habda,
blíð-lík bod-skępi: \hld\ gi·witun im te Bethleem þanan
nahtes síðon; \hld\ was im niud mikil,
þat sie selvon Krist \hld\ gi·sehan móstin.
Habda im þe ęngil godes \hld\ al gi·wísid%NOTE: Fitt 6.
torhtun tèknun, \hld\ þat sie im tó selvun,
te þem godes barne \hld\ gangan mahtun,
endi fundun sán \hld\ folko drohtin,
liudjo hérron. \hld\ Sagdun þó lof goda,
waldande mid iro wordun \hld\ endi wído kúðdun
ovar þea berhtun burg, \hld\ hwi-lik im þar biliði warð
fon hevan-wanga \hld\ hèlag gi·tògit,
fagar an felde. \hld\ Þat frí al bi·held
an ira hugi-skęftjun, \hld\ hèlag þiorna,
þiu magað an ira móde, \hld\ só hwat só siu gi·hòrda þea mann sprekan.
Fódda ina þó fagaro \hld\ frího skánjosta,
þiu módar þurh minnja \hld\ managaro drohtin,
hèlag himilisk barn. \hld\ hęliðos gi·sprákun
an þem ahtodon daga \hld\ erlos managa,
swíðo glawa gumon \hld\ mid þera godes þiornun,
þat he hèljand te namon \hld\ hębbjan skoldi,
só it þe godes ęngil \hld\ Gabriel gi·sprak
wáron wordun \hld\ endi þem wíve gi·bòd,
bodo drohtines, \hld\ þó siu èrist þat barn ant·feng
wánum te þesero wer-oldi; \hld\ was iru willjo mikil,
þat siu ina só hèlag-líko \hld\ haldan mósti,
ful-geng im þó só gerno. \hld\ Þat gér furðor skrèd
untþat þat friðu-barn godes \hld\ fiar-tig habda
dago endi nahto. \hld\ Þó skoldun sie þar èna dád frummjan,
þat sie ina te Hjerusalem \hld\ for·gevan skoldun
waldanda te þem wíha. \hld\ Só was iro wísa þan,
þero liudjo land-sidu, \hld\ þat þat ni mósta for·látan ne-gèn
idis undar Ebreon, \hld\ ef iru at èrist warð
sunu a·fódit, \hld\ ne siu ina simbla þarod
te þem godes wíha \hld\ for·gevan skolda.
Gi·witun im þó þiu gódun twè, \hld\ Joseph endi Maria
bèðju fon Bethleem: \hld\ habdun þat barn mid im,
hèlagna Krist, \hld\ sóhtun im hús godes
an Hjerusalem; \hld\ þar skoldun sie is geld frummjan
waldanda at þem wíha \hld\ wísa léstjan
Judeo folkes. \hld\ Þar fundun sea ènna gódan man
aldan at þem alaha, \hld\ aðal-boranan,
þe habda at þem wíha só filu \hld\ wintro endi sumaro
gi·libd an þem liohta: \hld\ oft warhta he þar lof goda
mid hluttru hugi; \hld\ habda im hèlagna gèst,
sálig-líkan sevon; \hld\ Simeon was he hètan.
Im habda gi·wísid \hld\ waldandas kraft
langa hwíla, \hld\ þat he ni mósta ér þit lioht a·gevan,
węndjan af þesero wer-oldi, \hld\ ér þan im þe willjo gi·stódi,
þat he selvan Krist \hld\ gi·sehan mósti,
hèlagna hevan-kuning. \hld\ Þó warð im is hugi swíðo
blíði an is briostun, \hld\ þó he gi·sah þat barn kuman
an þena wíh innan. \hld\ Þuo sagda hie waldande þank,
al-mahtigon gode, \hld\ þes he ina mid is ògun gi·sah.
Geng im þó te·gegnes \hld\ endi ina gerno ant·feng
ald mid is armun: \hld\ al ant·kende
bókan endi biliði \hld\ endi ók þat barn godes,
hèlagna hevan-kuning. \hld\ „Nu ik þi, hérro, skal“, kwað he,
„gerno biddjan, \hld\ nu ik sus gigamalod bium,
þat þu þínan holdan skalk \hld\ nu hinan hwervan látas,
an þína friðu-wára faran, \hld\ þar ér mína forðrun dedun,
weros fon þesero wer-oldi, \hld\ nu mi þe willjo gi·stód,
dago liovosto, \hld\ þat ik mínan drohtin gi·sah,
holdan hérron, \hld\ só mi gi·hètan was
langa hwíla. \hld\ Þu bist lioht mikil
allun ęli-þiodun, \hld\ þea ér þes alo-waldon
kraft ne ant·kendun. \hld\ Þína kumi sindun
te dóma endi te diurðon, \hld\ drohtin fró mín,
avarun Israhelas, \hld\ èganumu folke,
þínun liovun *liudjun.“ \hld\ Listiun talde þó
þe aldo man an þem alaha \hld\ idis þero gódun,
sagda sǫ́ð-líko, \hld\ hwó iro sunu skolda
ovar þesan middil-gard \hld\ managun werðan
sumun te falle, sumun te fróvru \hld\ firiho barnun,
þem liudjun te leova, \hld\ þe is lèrun gi·hòrdin,
endi þem te harma, \hld\ þe hòrjen ni weldin
Kristas lèron. \hld\ „Þu skalt noh“, kwað he, „kara þiggjan,
harm an þínumu herton, \hld\ þan ina hęliðo barn
wápnun wítnod. \hld\ Þat wirðid þi werk mikil,
þrim te gi·þolonna.“ \hld\ Þiu þiorna al for·stód
wísas mannas word. \hld\ Þó kwam þar ók èn wíf gangan
ald innan þem alaha: \hld\ Anna was siu hètan,
dohtar Fanueles; \hld\ siu habde ira drohtine wel
gi·þionod te þanka, \hld\ was iru gi·þungan wíf.
Siu mósta aftar ira magað-hédi, \hld\ síðor siu mannes warð,
erles an éhti \hld\ eðili þiorne,
só mósta siu mid ira brúdi-gumon \hld\ bódlo gi·waldan
sivun wintạr saman. \hld\ Þó gi·fragn ik þat iru þar sorga gi·stód
þat sie þiu mikila maht \hld\ metodes te·dèlda,
wrèð wurdi-gi·skapu. \hld\ Þó was siu widowa aftar þiu
at þem friðu-wíha \hld\ fior endi ant·ahtoda
wintro an iro wer-oldi, \hld\ só siu nia þana wíh ni for·lét,
ak siu þar ira drohtine wel \hld\ dages endi nahtes,
gode þionode. \hld\ Siu kwam þar ók gangan tó
an þea selvun tíd: \hld\ sán ant·kende
þat hèlage barn godes \hld\ endi þem hęliðon kúðde,
þem weroda aftar þem wíha \hld\ wil-spel mikil,
kwað þat im nęrjandas ginist \hld\ gi·náhid wári,
helpa heven-kuninges: \hld\ „nu is þe hèlago Krist,
waldand selvo \hld\ an þesan wíh kuman
te a·lòsjenne þea liudi, \hld\ þe hér nu lango bidun
an þesara middil-gard, \hld\ managa hwíla,
þurftig þioda, \hld\ só nu þes þinges mugun
mendjan man-kunni.“ \hld\ Manag fagonoda
werod aftar þem wíha: \hld\ gi·hòrdun wil-spel mikil
fon gode sęggjan. \hld\ Þat geld habde þó gi·léstid
þiu idis an þem alaha, \hld\ al só it im an ira éwa gi·bòd
endi an þera berhtun burg \hld\ bók gi·wísdun,
hèlagaro hand-gi·werk. \hld\ Gi·witun im þó te hús þanan
fon Hjerusalem \hld\ Joseph endi Maria,
hèlag híwiski: \hld\ habdun im heven-kuning
simbla te gi·síða, \hld\ sunu drohtines,
managaro mund-boron, \hld\ só it gio mári ni warð
þan wídor an þesaro wer-oldi, \hld\ b·útan só is willjo geng,
heven-kuninges hugi. \hld\ Þoh þar þan gi·hwi-lik hèlag man
Krist ant·kendi, \hld\ þoh ni warð it gio te þes kuninges hove
þem mannun gi·márid, \hld\ þea im an iro mód-sevon
holde ni wárun, \hld\ ak was im só bi·halden forð
mid wordun endi mid werkun, \hld\ antþat þar weros óstan,
swíðo glawa gumon \hld\ gangan kwámun
þrea te þero þiodu, \hld\ þegnos snelle,
an langan weg \hld\ ovar þat land þarod:
folgodun ènun berhtun bókne \hld\ endi sóhtun þat barn godes
mid hluttru hugi: \hld\ weldun im hnígan tó,
gehan im te jungrun: \hld\ drivun im godes gi·skapu.
Þó sie Erodesan þar \hld\ ríkjan fundun
an is sęli sittjen, \hld\ slíð-wurdjan kuning,
módagna mid is mannun: \hld\ —simbla was he morðes gern—
þó kwaddun sie ina kúsko \hld\ an kuning-wísun,
fagaro an is flęttje, \hld\ endi he frágoda sán,
hwi-lik sie árundi \hld\ úta gi·bráhti,
weros an þana wrak-síð: \hld\ „hweðer lèdjad gi wundan gold
te gevu hwi-likun gumuno? \hld\ te hwí gi þus an ganga kumad,
gi·faran an fóðju? \hld\ Hwat, gi néþwanan ferran sind
erlos fon óðrun þiodun. \hld\ Ik gi·sihu þat gi sind eðili-gi·burdjun
kunnjes fon knósle gódun: \hld\ nio hér ér su·lika kumana ni wurðun
éri fon óðrun þiodun, \hld\ síðor ik mósta þesas erlo folkes,
gi·waldan þesas wídon ríkjas. \hld\ Gi skulun mi te wárun sęggjan
for þesun liudjo folke, \hld\ bi·hwí gi sín te þesun lande kumana“.
Þó sprákun im eft te·gegnes \hld\ gumon òstr-onja,
word-spáhe weros: \hld\ „wi þi te wárun mugun“, kwáðun sie,
„úse árundi \hld\ óðo gi·tęlljen,
gi·sęggjan sǫ́ð-líko, \hld\ bi·hwí wi kwámun an þesan sið herod
fon óstan te þesaro erðu. \hld\ Giu wárun þar aðaljes man,
gód-sprákja gumon, \hld\ þea ús gódes só filu,
helpa gi·hétun \hld\ fon heven-kuninge
wárum wordun. \hld\ Þan was þar èn gi·wittig man,
fród endi fil-wís \hld\ —forn was þat giu—,
úse aldiro óstar hinan, \hld\ —þar ni warð síðor ènig man
sprákono só spáhi—; \hld\ he mahte rekkjen spel godes,
hwand im habde for·liwan \hld\ liudjo hérro,
þat he mahte fon erðu \hld\ up gi·hòrjan
waldandes word: \hld\ bi·þiu was is gi·wit mikil,
þes þegnes gi·þáhti. \hld\ Þó he þanan skolda,
a-geven gardos, \hld\ gadulingo gi·mang,
for·láten liudjo dròm, \hld\ sókjen lioht ǫ́ðar,
þó he is jungron \hld\ hét gangan náhor,
ęrvi-wardos, \hld\ endi is erlun þó
sagde sǫ́ð-líko: \hld\ —þat al síðor kwam,
gi·warð* an þesaro wer-oldi—: \hld\ þó sagda he þat hér skoldi kuman èn wís-kuning
mári endi mahtig \hld\ an þesan middil-gard
þes bętston gi·burdjes; \hld\ kwað þat it skoldi wesan barn godes,
kwað þat he þesero wer-oldes \hld\ waldan skoldi
gio te éwan-daga, \hld\ erðun endi himiles.
He kwað þat an þem selvon daga, \hld\ þe ina sáligna
an þesan middil-gard \hld\ módar gi·drógi,
só kwað he þat óstana \hld\ èn skoldi skínan
himil-tungal hwít, \hld\ su·lik só wi hér ne habdin ér
undar-twisk erða endi himil \hld\ ǫ́ðar hwerigin,
ne su·lik barn ne su·lik bókan. \hld\ Hét þat þar te bedu fórin
þrea man fon þero þiodu, \hld\ hét sie þęnkjan wel,
hwan ér sie gi·sáwin óstana \hld\ up síðogjan,
þat godes bókan gangan, \hld\ hét sie garwjan sán,
hét þat wi im folgodin, \hld\ só it furi wurði,
westar ovar þesa wer-oldi. \hld\ Nu is it al gi·wárod só,
kuman þurh kraft godes: \hld\ þe kuning is gi·fódit,
gi·boran bald endi strang: \hld\ wi gi·sáhun is bókan skínan
hédro fon himiles tunglun, \hld\ só ik wèt, þat it hèlag drohtin,
markoda mahtig selvo; \hld\ wi gi·sáhun morgno gi·hwi-likes
blíkan þana berhton sterron, \hld\ endi wi gengun aftar þem bókna herod
wegas endi waldas hwílon. \hld\ Þat wári ús allaro willjono mèsta,
þat wi ina selvon gi·sehan móstin, \hld\ wissin, hwar wi ina sókjan skoldin,
þana kuning an þesumu kèsur-dóma. \hld\ Saga ús, undar hwi-likumu he sí þesaro kunnjo a·fódit.“
Þó warð Erodesa \hld\ innan briostun
harm wið herta, \hld\ bi·gan im is hugi wallan,
sevo mid sorgun: \hld\ gi·hòrde sęggjan þó,
þat he þar ovar-hòvdon \hld\ ègan skoldi,
kraftagoron kuning \hld\ kunnjes gódes,
sáligoron undar þem gi·síðja. \hld\ Þó he samnon hét,
só hwat só an Hjerusalem \hld\ gódaro manno
allaro spáhoston \hld\ sprákono wárun
endi an iro brioston \hld\ bók-kraftes mèst
wissun te wárun, \hld\ endi he sie mid wordun fragn,
swíðo niud-líko \hld\ níð-hugdig man,
kuning þero liudjo, \hld\ hwar Krist gi·boran
an wer-old-ríkja \hld\ werðan skoldi,
friðu-gumono bętst. \hld\ Þó sprak im eft þat folk an·gegin,
þat werod wár-líko, \hld\ kwáðun þat sie wissin garo,
þat he skoldi an Bethleem gi·boran werðan: \hld\ „só is an úsun bókun gi·skrivan,
wís-líko gi·writan, \hld\ só it wár-sagon,
swíðo glawa gumon \hld\ bi godes krafta
fil-wíse man \hld\ furn gi·sprákun,
þat skoldi fon Bethleem \hld\ burgo hirdi,
liof landes ward \hld\ an þit lioht kuman,
ríki rád-gevo, \hld\ þe rihtjen skal
Judeono gum-skępi \hld\ endi is geva wesan
mildi ovar middil-gard \hld\ managun þiodun.“
Þó gi·fragn ik þat sán aftar þiu \hld\ slíð-mód kuning
þero wár-sagono word \hld\ þem wrękkjun sagda,
þea þar an ęli-lęndi \hld\ erlos wárun
ferran gi·farana, \hld\ endi he frágoda aftar þiu,
hwan sie an óstar-wegun \hld\ èrist gi·sáhin
þana kuning-sterron kuman, \hld\ kumbal liuhtjen
hédro fon himile. \hld\ Sie ni weldun is im þó helen eo·wiht,
ak sagdun it im sǫ́ð-líko. \hld\ Þó hét he sie an þana síð faran,
hét þat sie ira árundi al \hld\ undar-fundin
umbi þes kindes kumi, \hld\ endi þe kuning selvo gi·bòd
swíðo hard-liko, \hld\ hérro Judeono,
þem wísun mannun, \hld\ ér þan sie fórin westan forð,
þat sie im eft gi·kúðdin, \hld\ hwar he þana kuning skoldi
sókjan at is selðon; \hld\ kwað þat he þar weldi mid is gi·síðun tó,
bedan te þem barne. \hld\ Þan hogda he im te banon werðan
wápnes ęggjun. \hld\ Þan eft waldand god
þáhte wið þem þinga: \hld\ he mahta a·þengjan mér,
gi·léstjan an þesum liohte: \hld\ þat is noh lango skín,
gi·kúðid kraft godes. \hld\ Þó gengun eft þiu kumbl forð
wánum undar wolknun. \hld\ Þó wárun þea wíson man
fúsa te faranne: \hld\ gi·witun im forð þanan
balda an bod-skępi: \hld\ weldun þat barn godes
selvon sókjan. \hld\ Sie ni habdun þanan gi·síðjas mér,
b·útan þat sie þrie wárun: \hld\ wissun im þingo gi·skéð,
wárun im glawe gumon, \hld\ þe þea geva lèddun.
Þan sáhun sie só wís-líko \hld\ undar þana wolknes skion,
up te þem hòhon himile, \hld\ hwó fórun þea hwíton sterron
—ant·kendun sie þat kumbal godes—, \hld\ þiu wárun þurh Krista herod
gi·warht te þesero wer-oldi. \hld\ Þea weros aftar gengun,
folgodun feraht-líko \hld\ —sie frumide þe mahte—
antþat sie gi·sáhun, \hld\ síð-wórige man,
berht bókan godes, \hld\ blèk an himile
stillo gi·standen. \hld\ Þe sterro liohto skèn
hwít ovar þem húse, \hld\ þar þat hèlage barn
wonode an willjon \hld\ endi ina þat wíf bi·held,
þiu þiorne gi·þiudo. \hld\ Þó warð þero þegno hugi
blíði an iro briostun: \hld\ bi þem bókna for·stódun,
þat sie þat friðu-barn godes \hld\ funden habdun,
hèlagna heven-kuning. \hld\ Þó sie an þat hús innan
mid iro gevun gengun, \hld\ gumon òstr-onja,
síð-wórige man: \hld\ sán ant·kendun
þea weros waldand Krist. \hld\ Þea wrekkion fellun
te þem kinde an kneobeda \hld\ endi ina an kuning-wísa
gódan gróttun \hld\ endi im þea geva drógun,
gold endi wíh-ròk \hld\ bi godes tèknun
*endi myrra þar mid. \hld\ Þea man stódun garowa,
holde for iro hérron, \hld\ þea it mid iro handun sán
fagaro ant·fengun. \hld\ Þó gi·witun im þea ferahton man,
seggi te selðon \hld\ síð-wórige,
gumon an gast-sęli. \hld\ Þar im godes ęngil
slápandjun an naht \hld\ swevan gi·tògde,
gi·drog im an dròme, \hld\ al so it drohtin self,
waldand welde, \hld\ þat im þúhte þat man im mid wordun gi·budi,
þat sie im* þanan óðran weg, \hld\ erlos fórin,
liðodin sie te lande \hld\ endi þana lèðan man,
Erodesan \hld\ eft ni sóhtin,
módagna kuning. \hld\ Þó warð morgan kuman
wánum te þesero wer-oldi. \hld\ Þó bi·gunnun þea wíson man
sęggjan iro swevanos; \hld\ selvon ant·kendun
waldandes word, \hld\ hwand sie gi·wit mikil
bárun an iro briostun: \hld\ bádun alo-waldon,
héron heven-kuning, \hld\ þat sie móstin is huldi forð,
gi·wirkjan is willjon, \hld\ kwáðun þat sea ti im habdin gi·wendit hugi,
*iro mód morgan gi·hwem. \hld\ Þó fórun eft þie man þanan,
erlos òstr-onje, \hld\ al só im þe ęngil godes
wordun gi·wísde: \hld\ námun im weg óðran,
ful-gengun godes lèrun: \hld\ ni weldun þemu Judeo kuninge
umbi þes barnes gi·burd \hld\ bodon òstr-onje,
síð-wórige man \hld\ sęggjan gio·wiht,
ak wendun im eft an iro willjon. \hld\ Þó warð sán aftar þiu waldandes,
godes ęngil kumen \hld\ Josepe te sprákun,
sagde im an swefne \hld\ slápandjum an naht,
bodo drohtines, \hld\ þat þat barn godes
slíð-mód kuning \hld\ sókjan welda,
áhtjan is aldres; \hld\ „nu skaltu ine an Aegypteo
land ant·lèdjan \hld\ endi undar þem liudjun wesan
mid þiu godes barnu \hld\ endi mid þeru gódan þior*nan,
wunon undar þemu werode, \hld\ untþat þi word kume
hérron þínes, \hld\ þat þu þat hèlage barn
eft te þesum land-skępi \hld\ lèdjan mótis,
drohtin þínen.“ \hld\ Þó fon þem dròma an·sprang
Joseph an is gęst-sęli, \hld\ endi þat godes gi·bod
sán ant·kenda: \hld\ gi·wèt im an þana síð þanen
þe þegạn mid þeru þiornon, \hld\ sóhta im þiod óðra
ovar brèdan berg: \hld\ welda þat barn godes
fíundun ant·fórjan. \hld\ *Þó gi·frang aftar þiu % NOTE: gi·frang [sic]
Erodes þe kuning, \hld\ þar he an is ríkja sat,
þat wárun þea wíson man \hld\ westan gi·hworvan
óstar an iro óðil \hld\ endi fórun im óðran weg:
wisse þat sie im þat árundi \hld\ eft ni weldun
sęggjan an is selðon. \hld\ Þó warð im þes an sorgun hugi,
mód mornondi, \hld\ kwað þat it im þie man dedin,
hęliðos* te hónðun. \hld\ Þó he só hriwig sat,
balg ina an is briostun, \hld\ kwað þat he is mahti bętaron rád,
óðran gi·þęnkjen: \hld\ „nu ik is aldar kan,
wèt is winter-gi·talu: \hld\ nu ik gi·winnan mag,
þat he io ovar þesaro erðu \hld\ ald ni wirðit,
hér undar þesum hęri-skępi.“ \hld\ Þó he só hardo gi·bòd,
Erodes ovar is riki, \hld\ hét þó is rinkos faran
kuning þero liudjo, \hld\ hét þat sie kinda só filo
þurh iro hand-magen \hld\ hòvdu bi·námin,
só manag barn umbi Bethleem, \hld\ só filo só þar gi·boran wurði,
an twèm gèrun a·togan. \hld\ Tionon frumidon
þes kuninges gi·síðos. \hld\ Þó skolda þar só manag kindisk man
sweltan sundjono lòs. \hld\ Ni warð sið noh ér
giámar-líkara for·gang \hld\ jungaro manno,
arm-líkara dòð. \hld\ Idisi wiopun,
módar managa, \hld\ gi·sáhun iro megi spildjan:
ni mahte siu im nio gi·formon, \hld\ þoh siu mid iro faðmon twèm
iro ègan barn \hld\ armun bi·fengi,
liof endi luttil, \hld\ þoh skolda is simbla þat líf gevan,
þe magu for þeru módar. \hld\ Mènes ni sáhun,
wítjes þie wam-skaðon: \hld\ wápnes ęggjun
fremidun firin-werk mikil. \hld\ Fellun managa
magu-junge man. \hld\ Þia módar wiopun
kind-jungaro kwalm; \hld\ kara was an Bethleem,
hofno hlúdost: \hld\ þoh man im iro herton an twè
sniði mid swerdu, \hld\ þoh ni mohta im gio sèrara dád
werðan an þesaro wer-oldi, \hld\ wívun managun,
brúdjun an Bethleem: \hld\ gi·sáhun iro barn bi·foran,
kind-junge man, \hld\ kwalmu sweltan
blódag an iro barmun. \hld\ Þie banon wítnodun
un·skuldige skole: \hld\ ni bi·skrivun gio·wiht
þea man umbi mèn-werk: \hld\ weldun mahtigna,
Krist selvon a·kwęlljan. \hld\ Þan habde ina kraftag god
gineridan wið iro níðe, \hld\ þat inan nahtes þanan
an Aegypteo land \hld\ erlos ant·lèddun,
gumon mid Josepe \hld\ an þana grónjon wang,
an erðono bętstun, \hld\ þar èn aha fliutid,
Níl-stròm mikil \hld\ norð te sèwa,
flódo fagorosta. \hld\ Þar þat friðu-barn godes
wonoda an willjon, \hld\ antþat wurd for·nam
Erodes þana kuning, \hld\ þat he for·lét eldeo barn,
módag manno dròm. \hld\ Þó skolda þero marka gi·wald
ègan is ęrvi-ward: \hld\ þe was Arkheláus
hètan, hęri-togo \hld\ helm-berandero:
þe skolda umbi Hjerusalem \hld\ Judeono folkes,
werodes gi·waldan. \hld\ Þó warð word kuman
þar an Egypti \hld\ eðiliun manne,
þat he þar te Josepe, \hld\ godes ęngil sprak,
bodo drohtines, \hld\ hét ina eft þat barn þanan
lèdjen te lande. \hld\ „nu havað þit lioht afgeven“, kwað he,
„Erodes þe kuning; \hld\ he welde is áhtjen giu,
fréson is ferahas. \hld\ Nu maht þu an friðu lèdjen
þat kind undar ewa kunni, \hld\ nu þe kuning ni livod,
erl ovar-módig.“ \hld\ Al ant·kende
Josep godes tèkạn: \hld\ geriwide ina sniumo
þe þegạn mit þera þiornun, \hld\ þó sie þanan weldun
bèðju mid þiu barnu: \hld\ léstun þiu berhton gi·skapu,
waldandes willjon, \hld\ al só he im ér mid is wordun gi·bòd.
Gi·witun im þó eft an Galilea-land \hld\ Joseph endi Maria,
hèlag híwiski \hld\ heven-kuninges,
wárun im an Nazareth-burg. \hld\ Þar þe nęrjondio Krist
wóhs undar þem werode, \hld\ warð gi·wittjes ful,
an was imu anst godes, \hld\ he was allun liof
módar-mágun: \hld\ he ni was óðrun mannun gi·lík,
þe gumo an sínera gódi. \hld\ Þó he gér-talo
twelivi habde, \hld\ þó warð þiu tíd kuman,
þat sie þar te Hjerusalem, \hld\ Juðeo liudi
iro þiod-gode \hld\ þionon skoldun,
wirkjan is willjon. \hld\ Þó warð þar an þana wíh innan
þar te Hjerusalem \hld\ Judeono gi·samnod
man-kraft mikil. \hld\ Þar Maria was
self an gi·síðja \hld\ endi iru sunu habda,
godes ègan barn. \hld\ Þó sie þat geld habdun,
erlos an þem alaha, \hld\ só it an iro éwa gi·bòd,
gi·léstid te iro land-wísun, \hld\ þó fórun im eft þie liudi þanan,
weros an iro willjon \hld\ endi þar an þem wíha afstód
mahtig barn godes, \hld\ só ina þiu módar þar
ni wissa te wáron; \hld\ ak siu wánda þat he mid þem weroda forð,
fóri mit iro friundun. \hld\ Gifrang aftar þiu
eft an óðrun daga \hld\ aðal-kunnjes wíf,
sálig þiorna, \hld\ þat he undar þem gi·síðia ni was.
warð Mariun þó \hld\ mód an sorgun,
hriwig umbi iro herta, \hld\ þó siu þat hèlaga barn
ni fand undar þem folka: \hld\ filu gornoda
þiu godes þiorna. \hld\ Gi·witun im þó eft te Hjerusalem
iro sunu sókjan, \hld\ fundun ina sittjan þar
an þem wíha innan, \hld\ þar þe wísa man,
swíðo glauwa gumon \hld\ an godes éwa
lásun ende línodun, \hld\ hwó sie lof skoldin
wirkjan mid iro wordun þem, \hld\ þe þesa wer-old gi·skóp.
Þar sat undar middjun \hld\ mahtig barn godes,
Krist alo-waldo, \hld\ só is þea ni mahtun ant·kęnnjan wiht,
þe þes wíhes þar \hld\ wardon skoldun,
endi frágoda sie \hld\ firi-wit-líko
wísera wordo. \hld\ Sie wundradun alle,
bu-hwí gio só kindisk man \hld\ su·lika kwidi mahti
mid is múðu gi·mènjan. \hld\ Þar ina þiu módar fand
sittjan under þem gi·síðja \hld\ endi iro sunu grótta,
wísan undar þem weroda, \hld\ sprak im mid ira wordun tó:
„hwí weldes þu þínera módar, \hld\ manno liovosto,
gi·sidon su·lika sorga, \hld\ þat ik þi só sèrag-mód,
idis arm-hugdig \hld\ éskon skolda
undar þesun burg-liudjun?“ \hld\ Þó sprak iru eft þat barn an·gegin
wísun wordun: \hld\ „hwat, þu wèst garo“, kwað he,
„þat ik þar gi·rísu, \hld\ þar ik bi rehton skal
wonon an willjon, \hld\ þar gi·wald havad
mín mahtig fader.“ \hld\ Þie man ni for·stódun,
þie weros an þem wíha, \hld\ bi·hwí he só þat word gi·sprak,
gi·mènda mid is múðu: \hld\ Maria al bi·held,
gi·barg an ira breostun, \hld\ só hwat só siu gi·hòrda ira barn sprekan
wisaro wordo. \hld\ Gi·witun im þó eft þanan
fon Hjerusalem \hld\ Joseph endi Maria,
habdun im te gi·síðja \hld\ sunu drohtines,
allaro barno bętsta, \hld\ þero þe io gi·boran wurði
magu fon módar: \hld\ habdun im þar minnja tó
þurh hluttran hugi, \hld\ endi he só gi·hòrig was,
godes ègan barn \hld\ gaduling-mágun
þurh is òd-módi, \hld\ aldron sínun:
ni welda an is kindiski þó noh \hld\ is kraft mikil
mannun márjan, \hld\ þat he su·lik męgin éhta,
gi·wald an þesaro wer-oldi, \hld\ ak he im an is willjon béd
gi·þiudo undar þero þiodu \hld\ þrí-tig géro,
ér þan he þar tèkạn ènig \hld\ tògjan weldi,
sęggjan þem gi·síðja, \hld\ þat he selvo was
an þesaro middil-gard \hld\ manno drohtin.
Habda im só bi·halden \hld\ hèlag barn godes
word endi wís-dóm \hld\ ende allaro gi·wittjo mèst,
tulgo spáhan hugi: \hld\ ni mahta man is an is sprákun werðan,
an is wordun gi·war, \hld\ þat he su·lik gi·wit éhta,
þegạn su·lika gi·þáhti, \hld\ ak he im só gi·þiudo béd
torhtaro tèkno. \hld\ Ni was noh þan þiu tíd kuman,
þat he ina ovar þesan \hld\ middil-gard márjan skolda,
lèrjan þie liudi, \hld\ hwó sie skoldin iro gi·lòvon haldan,
wirkjan willjon godes; \hld\ wissun þat þoh managa
liudi aftar þem landa, \hld\ þat he was an þit lioht kuman,
þoh sie ina kúð-líko \hld\ an·kennjan ni mahtin,
ér þan he ina selvo \hld\ sęggjan welda.
Þan was im Johannes \hld\ fon is juguð-hédi
awahsan an ènero wóstunni; \hld\ þar ni was werodes þan mér,
b·útan þat he þar èn-kora \hld\ alo-waldon gode,
þegạn þionoda: \hld\ for·lét þioda gi·mang,
manno gi·mènðon. \hld\ Þar warð im mahtig kuman
an þero wóstunni \hld\ word fon himila,
gód-lík stemna godes, \hld\ endi Johanne gi·bod,
þat he Kristes kumi \hld\ endi is kraft mikil
ovar þesan middil-gard \hld\ márjan skoldi;
hét ina wár-líko \hld\ wordun sęggjan,
þat wári hevan-riki \hld\ hęliðo barnun
an þem land-skępi, \hld\ liudjun gi·náhid,
welono wun-samost. \hld\ Im was þó willjo mikil,
þat he fon su·likun sáldun \hld\ sęggjan mósti.
Gi·wèt im þó gangan, \hld\ al só Jordan flót,
watar an willjon, \hld\ endi þem weroda allan dag,
aftar þem land-skępi \hld\ þem liudjun kúðda,
þat sie mid fastunnju \hld\ firin-werk manag,
iro selvoro \hld\ sundja bóttin,
„þat gi werðan hrènja“, \hld\ kwað he. „Hevan-riki is
gi·náhid manno barnun. \hld\ Nu látad eu an ewan mód-sevon
ewar selvoro \hld\ sundja hrewan,
lèdas þat gi an þesun liohta fremidun, \hld\ endi mínun lèrun hòrjad,
węndjat aftar mínun wordun. \hld\ Ik eu an watara skal
gi·dòpjan diur-líko, \hld\ þoh ik ewa dádi ne mugi,
ewar selvaro \hld\ sundja a·látan,
þat gi þurh mín hand-gi·werk \hld\ hluttra werðan
lèðaro gi·lésto: \hld\ ak þe is an þit lioht kuman,
mahtig te mannun \hld\ endi undar eu middjun stéd,
—þoh gi ina selvun \hld\ gi·sehan ni willjan—,
þe eu gi·dòpjan skal \hld\ an ewes drohtines namon
an þana hálagon gèst. \hld\ Þat is hérro ovar al:
he mag allaro manno gi·hwena \hld\ mèn-gi·þáhtjo,
sundjono sikoron, \hld\ só hwene só só sálig mót
werðen an þesaro wer-oldi, \hld\ þat þes willjon havad,
þat he só gi·léstja, \hld\ só he þesun liudjun wili,
gi·bioden barn godes. \hld\ Ik bium an is bod-skępi herod
an þesa wer-old kumen \hld\ endi skal im þana weg rúmien,
lèrjan þesa liudi, \hld\ hwó sea skulin iro gi·lòvon haldan
þurh hluttran hugi, \hld\ endi þat sie an hęllja ni þurvin,
faran an fern þat hèta. \hld\ Þes wirðid só fagan an is móde
man te só managaro stundu, \hld\ só hwe só þat mèn for·látid,
gerno þes gramon anbusni, \hld\ —só mag im þes gódon gi·wirkjan,
huldi heven-kuninges,— \hld\ só hwe só havad hluttra trewa
up te þem alo-mahtigon gode.“ \hld\ Erlos managa
bi þem lèrun þó, \hld\ liudi wándun,
weros wár-líko, \hld\ þat þat waldand Krist
selbo wári, \hld\ hwanda he só filu sǫ́ðes gi·sprak,
wároro wordo. \hld\ Þó warð þat só wído kúð
ovar þat for·gevana land \hld\ gumono gi·hwi-likum,
sęggjun at iro selðun: \hld\ þó kwámun ina sókjan þarod
fon Hjerusalem \hld\ Judeo liudjo
bodon fon þeru burgi \hld\ endi frágodun, ef he wári þat barn godes,
„þat hér lango giu“, \hld\ kwaðun sie, „liudi sagdun,
weros wár-líko, \hld\ þat he skoldi an þesa wer-old kuman“.
Johannes þó gi·mahalde \hld\ endi te·gegnes sprak
þem bodun bald-líko: \hld\ „ni bium ik“, kwað he, „þat barn godes,
wár waldand Krist, \hld\ ak ik skal im þana weg rúmien,
hérron mínumu.“ \hld\ Þea hęliðos frugnun,
þea þar an þem árundje \hld\ erlos wárun,
bodon fon þero burgi: \hld\ „ef þu nu ni bist þat barn godes,
bist þu þan þoh Elias, \hld\ þe hér an ér-dagun
was undar þesumu werode? \hld\ He is wiskumo
eft an þesan middil-gard. \hld\ Saga ús hwat þu manno sís!
Bist þu ènig þero, \hld\ þe hér ér wári
wísaro wár-saguno? \hld\ Hwat skulun wi þem werode fon þi
sęggjan te sǫ́ðon? \hld\ Neo hér ér su·lik ni warð
an þesun middil-gard \hld\ man ǫ́ðar kuman
dádjun só mári. \hld\ Bi·hwí þu hér dòpisli
fremis undar þesumu folke, \hld\ ef þu þaro fora-sagono
èn-hwi-lik ni bist?“ \hld\ Þó habde eft garo
Johannes þe gódo \hld\ glau and-wordi:
„Ik bium fora-bodo \hld\ fráon mínes,
lioves hérron; \hld\ ik skal þit land rekon,
þit werod aftar is willjon. \hld\ Ik hębbju fon is worde mid mi
stranga stemna, \hld\ þoh sie hér ni willje for·standan filo
werodes an þesaro wóstunni. \hld\ Ni bium ik mid wihti gi·lík
drohtine mínumu: \hld\ he is mid is dádjun só strang,
só mári endi só mahtig \hld\ —þat wirðid managun kúð,
werun aftar þesaro wer-oldi— \hld\ þat ik þes wirðig ni bium,
þat ik móti an is gi·skuoha, \hld\ þoh ik sí is skalk ègan,
an só ríkjumu drohtine, \hld\ þea reomon ant·bindan:
só mikilu is he bętara þan ik. \hld\ Nis þes bodon gi·mako
ènig ovar erðu, \hld\ ne nu aftar ni skal
werðan an þesaro wer-oldi. \hld\ Hębbjad ewan willjon þarod,
liudi ewan gi·lòvon: \hld\ þan eu lango skal
wesan ewa hugi hrómag; \hld\ þan gi hęlli-gi·þwing,
for·látad lèðaro dròm \hld\ endi sókjad eu lioht godes,
up-òdes hèm, \hld\ èwig ríki,
hòhan heven-wang. \hld\ Ne látad ewan hugi twífljen!“
Só sprak þó jung gumo \hld\ bi godes lèrun
mannun te márðu. \hld\ Manag samnoda
þar te Bethania \hld\ barn Israheles;
kwámun þar te Johannese \hld\ kuningo gi·síðos,
liudi te lèrun \hld\ endi iro gi·lòvon ant·fengun.
He dòpte sie dago gi·hwi-likes \hld\ endi im iro dádi lóg,
wrèðaro willjon, \hld\ endi lovode im word godes,
hérron sínes: \hld\ „heven-ríki wirðid“, kwað he,
„garu gumono só hwem, \hld\ só ti gode þęnkid
endi an þana hèljand *wili \hld\ hluttro gi·lòvjan, %NOTE: wili] P 1r.
léstjan is lèra“. \hld\ Þó ni was lang te þiu,
þat im fon Galilea gi·wèt \hld\ godes ègan barn,
*diur-lík drohtines sunu, \hld\ dòpi suokjan.
was im þuo an is wastme \hld\ waldandes barn*,
al só he mid þero þiodu \hld\ þrí-tig habdi
wintro an is wer-oldi. \hld\ Þó he an is willjon kwam,
þar Johannes \hld\ an Jordana stròme
allan langan dag \hld\ liudi manage
dòpte diur-líko. \hld\ Reht só he þó is drohtin gi·sah,
holdan hérron, \hld\ só warð im is hugi blíði,
þes im þe willjo gi·stód, \hld\ endi sprak im þó mid is wordun tó,
swíðo gód gumo, \hld\ Johannes te Kriste:
„nu kumis þu te mínero dòpi, \hld\ drohtin fró mín,
þiod-gumono bętsto: \hld\ só skolde ik te þínero duan,
hwand þu bist allaro kuningo kraftigost.“ \hld\ Krist selvo gi·bòd,
waldand wár-líko, \hld\ þat he ni spráki þero wordo þan mér:
„wèst þu, þat ús só gi·rísid“, \hld\ kwað he, „allaro rehto gi·hwi-lik
te gi·fulljanne \hld\ forð-wardes nu
an godes willjon“. \hld\ Johannes stód,
dòpte allan dag \hld\ druht-folk mikil,
werod an watere \hld\ endi ók waldand Krist,
héran heven-kuning \hld\ handun sínun
an allaro baðo þem bętston \hld\ endi im þar te bedu gi·hnèg
an kneo kraftag. \hld\ Krist up gi·wèt
fagar fon þem flóde, \hld\ friðu-barn godes,
liof liudjo ward. \hld\ Só he þó þat land af·stóp,
só ant·hlidun þó himiles doru, \hld\ endi kwam þe hèlago gèst
fon þem alo-waldon \hld\ ovane te Kriste:
—was im an gi·lík-nissje \hld\ lungras fugles,
diur-líkara dúvun— \hld\ endi sat im uppan úses drohtines ahslu,
wonoda im ovar þem waldandes barne. \hld\ Aftar kwam þar word fon himile,
hlúd fon þem hòhon radura \hld\ endi grótta þane hèljand selvon,
Krista, allaro kuningo bętston, \hld\ kwað þat he ina gi·korana habdi
selvo fon sínun ríkja, \hld\ kwað þat im þe sunu líkodi
bętst allaro gi·boranaro manno, \hld\ kwað þat he im wári allaro barno liovost.
Þat móste Johannes þó, \hld\ al só it god welde,
gi·sehan endi gi·hòrjan. \hld\ He gi·deda it sán aftar þiu
mannun mári, \hld\ þat sie þar mahtigna
hérron habdun: \hld\ „Þit is“, kwað he, „heven-kuninges sunu,
èn alo-waldand: \hld\ þesas willjo ik ur-kundjo
wesan an þesaro wer-oldi, \hld\ hwand it sagda mi word godes,
drohtines stemne, \hld\ þó he mi dòpjan hét
weros an watare, \hld\ só hwar só ik gi·sáwi wár-líko
þana hèlagon gèst \hld\ *fan hevan-wange
an þesan middil-gard \hld\ ènigan man waron,
kuman mid kraftu; \hld\ þat kwað, þat skoldi Krist wesan,
diur-lík drohtines suno. \hld\ Hie dòpjan skal
an þana hèlagan gèst \hld\ endi hèljan managa %NOTE: þana] P end.
manno mèn-dádi. \hld\ He havad maht fon gode,
þat he a·látan mag \hld\ liudjo gi·hwi-likun
saka endi sundja. \hld\ Þit is selvo Krist,
godes ègan barn, \hld\ gumono bętsto,
friðu wið fíundun. \hld\ Wala þat eu þes mag fráh-mód hugi
wesan an þesaro wer-oldi, \hld\ þes eu þe willjo gi·stód,
þat gi só libbjanda \hld\ þana landes ward
selvon gi·sáhun. \hld\ Nu mót sliumo sundjono lòs
manag gèst faran \hld\ an godes willjon
tionon a·tómid, \hld\ þe mid trewon wili
wið is wini wirkjan \hld\ endi an waldand Krist
fasto gi·lòvjan. \hld\ Þat skal te frumun werðen
gumono só hwi-likun, \hld\ só þat gerno dót“.
Só ge·fragn ik þat Johannes \hld\ þó gumono gi·hwi-likun,
lovoda þem liudjun \hld\ lèra Kristes,
hérron sínes, \hld\ endi heven-ríki
te gi·winnanne, \hld\ welono þane mèston,
sálig sin-líf. \hld\ Þó he im selvo gi·wèt
aftar þem dòpislja, \hld\ drohtin þe gódo,
an èna wóstunnja, \hld\ waldandes sunu;
was im þar an þero èn-ódi \hld\ erlo drohtin
lange hwíla; \hld\ ne habda liudjo þan mér,
seggjo te gi·síðun, \hld\ al só he im selvo gi·kòs:
welda is þar látan koston \hld\ kraftiga wihti,
selvon Satanasan, \hld\ þe gio an sundja spenit,
man an mèn-werk: \hld\ he konsta is mód-sevon,
wrèðan willjon, \hld\ hwó he þesa wer-old èrist,
an þem an·ginnja \hld\ irmin-þioda
bi·swèk mit sundjun, \hld\ þó he þiu sinhíun twè,
Ádaman endi Éuan, \hld\ þurh un·trewa
for·lèdda mid luginun, \hld\ þat liudo barn
aftar iro hin-ferdi \hld\ hęllja sóhtun,
gumono gèstos. \hld\ Þó welda þat god mahtig,
waldand węndjan \hld\ endi welda þesum werode for·geven
hòh himil-ríki: \hld\ be·þiu he herod hèlagna bodon,
is sunu senda. \hld\ Þat was Satanase
tulgo harm an is hugi: \hld\ afonsta hevan-ríkjes
manno kunnje: \hld\ welda þó mahtigna
mid þem selvon sakun \hld\ sunu drohtines,
þem he Ádaman \hld\ an ér-dagun
darnungo bi-dróg, \hld\ þat he warð is drohtine lèð,
bi·swèk ina mid sundjun \hld\ —só welda he þó selvan dón
hélandjan Krist. \hld\ Þan habda he is hugi fasto
wið þana wam-skaðon, \hld\ waldandes barn,
herte só gi·herdid: \hld\ welda heven-ríki
liudjun gi·léstjan. \hld\ Was im þes landes ward
an fastunnea \hld\ fior-tig nahto,
manno drohtin, \hld\ só he þar mates ni antbét;
þan langa ni gi·dorstun \hld\ im dernja wihti,
níð-hugdig fíund, \hld\ náhor gangan,
grótjan ina gegin-warðan: \hld\ wánde þat he god èn-fald,
for·útar man-kunnjes wiht \hld\ mahtig wári,
hèleg himiles ward. \hld\ Só he ina þó ge·hungrjan lét,
þat ina bi·gan bi þero męnnisko \hld\ móses lustjan
aftar þem fiwar-tig dagun, \hld\ þe fíund náhor geng,
mirki mèn-skaðo: \hld\ wánda þat he man èn-fald
wári wissungo, \hld\ sprak im þó mid is wordun tó,
grótta ina þe gèr-fíund: \hld\ „ef þu sís godes sunu“, kwað he,
„be·hwí ni hétis þu þan werðan, \hld\ ef þu gi·wald haves,
allaro barno bętst, \hld\ bròd af þesun stènun?
Ge·héli þínna hungar.“ \hld\ Þó sprak eft þe hèlago Krist:
„ni mugun eldi-barn“, \hld\ kwað he, „èn-faldes bròdes,
liudi libbjen, \hld\ ak sie skulun þurh lèra godes
wesan an þesero wer-oldi \hld\ endi skulun þiu werk frummjen,
þea þar werðad a·hlúdid \hld\ fon þero hélogun tungun,
fon þem galme godes: \hld\ þat is gumono líf
liudjo só hwi-likon, \hld\ só þat léstjan wili,
þat fon waldandes \hld\ worde ge·biudid.“
Þó bi·gan eft niuson \hld\ endi náhor geng
un·hiuri fíund \hld\ óðru síðu,
fandoda is fròhan. \hld\ Þat friðu-barn þolode
wrèðes willjon \hld\ endi im gi·wald for·gaf,
þat he umbi is kraft mikil \hld\ koston mósti,
lét ina þó lèdjan \hld\ þana liud-skaðon,
þat he ina an Hjerusalem \hld\ te þem godes wíha,
alles ovan-wardan, \hld\ up gi·setta
an allaro húso hòhost, \hld\ endi hosk-wordun sprak,
þe gramo þurh gelp mikil: \hld\ „ef þu sís godes sunu“, kwað he,
„skríd þi te erðu hinan. \hld\ Ge·skrivan was it giu lango,
an bókun ge·writen, \hld\ hwó gi·boden havad
is ęngilun \hld\ alo-mahtig fader,
þat sie þi at wege ge·hwem \hld\ wardos sinðun,
haldad þi undar iro handun. \hld\ Hwat, þu hwargin ni þarft
mid þínun fótun \hld\ an felis be·spurnan,
an hardan stèn.“ \hld\ Þó sprak eft þe hèlago Krist,
allaro barno bętst: \hld\ „só is ók an bókun ge·skrivan“, kwað he,
„þat þu te hardo ni skalt \hld\ hérran þínes,
fandon þínes fròhan: \hld\ þat nis þi allaro frumono negèn.“
Lét ina þó an þana þriddjan síð \hld\ þana þiod-skaðon
gi·brengen uppan ènan berg þen hòhon: \hld\ þar ina þe balo-wíso
lét al ovar-sehan \hld\ irmin-þiode,
wonod-saman welon \hld\ endi wer-old-ríki
endi all su·lik ódes, \hld\ só þius erða bi·havad
fagororo frumono, \hld\ endi sprak im þó þe fíund an·gegin,
kwað þat he im þat al só gód-lík \hld\ for·geven weldi,
hòha heri-dómos, \hld\ „ef þu wilt hnígan te mi,
fallan te mínun fótun \hld\ endi mi for fròhan havas,
bedos te mínun barma. \hld\ Þan látu ik þi brúkan wel
alles þes òd-welon, \hld\ þes ik þi hębbju gi·ógit hír.“
Þó ni welda þes lèðan word \hld\ lengeron hwíle
hòrjan þe hèlago Krist, \hld\ ak he ina fon is huldi for·dréf,
Satanasan for·swép, \hld\ endi sán aftar sprak
allaro barno bętst, \hld\ kwað þat man bedon skoldi
up te þem alo-mahtigon gode \hld\ endi im ènum þionon
swíðo þio-liko \hld\ þegnos managa,
hęliðos aftar is huldi: \hld\ „þar ist þiu helpa gelang
manno ge·hwi-likun.“ \hld\ Þó gi·wèt im þe mèn-skaðo,
swíðo sèrag-mód \hld\ Satanas þanan,
fíund undar fern-dalu. \hld\ Warð þar folk mikil
fon þem alo-waldan \hld\ ovana te Kriste
godes ęngilo kumen, \hld\ þie im síðor jungar-dóm,
skoldun ambaht-skępi \hld\ aftar léstjen,
þionon þiolíko: \hld\ só skal man þiod-gode,
hérron aftar huldi, \hld\ hevan-kuninge.
was im an þem sin-weldi \hld\ sálig barn godes
lange hwíle, \hld\ untþat im þó liovora warð,
þat he is kraft mikil \hld\ kúðjen wolda
weroda te willjon. \hld\ Þó for·lét he waldes hléo,
èn-ódjes ard \hld\ endi sóhte im eft erlo ge·mang,
mári męgin-þiode \hld\ endi manno dròm,
geng im þó bi Jordanes staðe: \hld\ þar ina Johannes ant·fand,
þat friðu-barn godes, \hld\ fròhan sínan,
hèlagana heven-kuning, \hld\ endi þem hęliðun sagda,
Johannes is jungurun, \hld\ þó he ina gangan ge·sah:
„þit is þat lamb godes, \hld\ þat þar lòsjan skal
af þesaro wídon wer-old \hld\ wrèða sundja,
man-kunnjas mèn, \hld\ mári drohtin,
kuningo kraftigost.“ \hld\ Krist im forð gi·wèt
an Galileo land, \hld\ godes ègan barn,
fór im te þem friundun, \hld\ þar he a·fódit was,
tír-líko atogan, \hld\ endi talda mid wordun
Krist undar is kunnje, \hld\ kuningo ríkjost,
hwó sie skoldin iro selvoro \hld\ sundja bótjan,
hét þat sie im iro harm-werk manag \hld\ hrewan létin,
feldin iro firin-dádi: \hld\ „nu is it all ge·fullot só,
só hír alde man \hld\ ér hwanna sprákun,
ge·hétun eu te helpu \hld\ heven-ríki:
nu is it giu gi·náhid þurh þes nęrjandan kraft: \hld\ þes mótun gi neotan forð,
só hwe só gerno wili \hld\ gode þeonogjan,
wirkjan aftar is willjon.“ \hld\ Þó warð þes werodes filu,
þero liudjo an lustun: \hld\ wurðun im þea lèra Kristes,
só swótja þem gi·síðja. \hld\ He bi·gan im samnon þó
gumono te jungoron, \hld\ gódoro manno,
word-spáha weros. \hld\ Geng im þó bi ènes watares staðe,
þat þar habda Jordan \hld\ anevan Galileo land
ènna sè ge·warhtan. \hld\ Þar he sittjan fand
Andreas endi Petrus \hld\ bi þem aha-stròme,
bèðja þea ge·bróðar, \hld\ þar sie an brèd watar
swíðo niud-líko \hld\ nętti þenidun,
fiskodun im an þem flóde. \hld\ Þar sie þat friðu-barn godes
bi þes sèes staðe \hld\ selvo grótta,
hét þat sie im folgodin, \hld\ kwað þat he im só filu woldi
godes ríkjas for·geven; \hld\ „al só git hír an Jordanes stròme
fiskos fáhat, \hld\ só skulun git noh firiho barn
halon te inkun handun, \hld\ þat sie an heven-ríki
þurh inka lèra \hld\ líðan mótin,
faran folk manag.“ \hld\ Þó warð fró-mód hugi
bèðjun þem gi·bróðrun: \hld\ ant·kendun þat barn godes,
liovan hérron: \hld\ for·létun al saman
Andreas endi Petrus, \hld\ só hwat só sie bi þeru ahu habdun,
ge·wunstes bi þem watare: \hld\ was im willjo mikil,
þat sie mid þem godes barne \hld\ gangan móstin,
samad an is gi·síðja, \hld\ skoldun sálig-líko
lòn ant·fáhan: \hld\ só dót liudjo so hwi-lik,
só þes hérran wili \hld\ huldi gi·þionon,
ge·wirkjan is willjon. \hld\ Þó sie bi þes watares staðe
furðor kwámun, \hld\ þó fundun sie þar ènna fródan man
sittjan bi þem sèwa \hld\ endi is suni twène,
Jakobus endi Johannes: \hld\ wárun im junga man.
Sátun im þá ge·sun-fader \hld\ an ènumu sande uppen,
brugdun endi bóttun \hld\ bèðjum handun
þiu nętti niud-líko, \hld\ þea sie habdun nahtes ér
for·sliten an þem sèwa. \hld\ Þar sprak im selvo tó
sálig barn godes, \hld\ hét þat sie an þana síð mid im,
Jakobus endi Johannes, \hld\ gengin bèðje,
kind-junge man. \hld\ Þó wárun im Kristes word
só wirðig an þesaro wer-oldi, \hld\ þat sie bi þes watares staðe
iro aldan fader \hld\ ènna for·létun,
fródan bi þem flóde, \hld\ endi al þat sie þar fehas éhtun,
nęttju endi nęglit-skipu, \hld\ ge·kurun im þana nęrjandan Krist,
hèlagna te hérron, \hld\ was im is helpono þarf
te gi·þiononne: \hld\ só is allaro þegno ge·hwem,
wero an þesero wer-oldi. \hld\ Þó gi·wèt im þe waldandes sunu
mid þem fiwarjun forð, \hld\ endi im þó þana fífton gi·kòs
Krist an ènero kòp-stędi, \hld\ kuninges jungoron,
mód-spáhana man: \hld\ Mattheus was he hètan,
was im ambahtjo \hld\ eðilero manno,
skolda þar te is hérron \hld\ handun ant·fáhan
tins endi tolna; \hld\ trewa habda he góda,
aðal-and-bári: \hld\ for·lét al saman
gold endi siluvar \hld\ endi geva managa,
diurje mèðmos, \hld\ endi warð im úses drohtines man;
kòs im þe kuninges þegn \hld\ Krist te hérran,
milderan mèðom-gevon, \hld\ þan ér is man-drohtin
wári an þesero wer-oldi: \hld\ feng im wóðera þing,
lang-samoron rád. \hld\ Þó warð it allun þem liudjun kúð,
fon allaro burgo gi·hwem, \hld\ hwó þat barn godes
samnode ge·síðos \hld\ endi selvo ge·sprak
só manag wís-lík word \hld\ endi wáres só filu,
torhtes gi·tògde \hld\ endi tèkạn manag
ge·warhte an þesero wer-oldi. \hld\ Was þat an is wordun skín
iak an is dádjun só same, \hld\ þat he drohtin was,
himilisk hérro \hld\ endi te helpu kwam
an þesan middil-gard \hld\ manno barnun,
liudjun te þesun liohta. \hld\ Oft ge·deda he þat an þem lande skín,
þan he þar torht-líko \hld\ só manag tèkạn gi·warhte,
þar he hélde mid is handun \hld\ halte endi blinde,
lòsde af þeru léf-hédi \hld\ liudi manage,
af su·likun suhtjun, \hld\ só þan allaro swároston
an firiho barn \hld\ fíund bi·wurpun,
tulgo lang-sam legar. \hld\ Þó fórun þar þie liudi tó
allaro dago ge·hwi-likes, \hld\ þar úsa drohtin was
selvo undar þem gi·síðje, \hld\ untþat þar ge·samnod warð
męgin-folk mikil \hld\ managero þiodo,
þoh sie þar alle be ge·líkumu \hld\ ge·lòvon ni kwámin.
weros þurh ènan willjon: \hld\ sume sóhtun sie þat waldandes barn,
armoro manno filu \hld\ —was im átes þarf—,
þat sie im þar at þeru męnigi \hld\ mates endi drankes,
þigidin at þeru þiodu; \hld\ hwand þar was manag þegạn só gód,
þie ira alamosnje \hld\ armun mannun
gerno gávun. \hld\ Sume wárun sie im eft Judeono kunnjes,
fégni folk-skępi: \hld\ wárun þar ge·farana te þiu,
þat sie úses drohtines \hld\ dádjo endi wordo
fáron woldun, \hld\ habdun im fégnjen hugi,
wrèðen willjon: \hld\ woldun waldand Krist
a-lèdjen þem liudjun, \hld\ þat sie is lèron ni hòrdin,
ne wendin aftar is willjon. \hld\ Suma wárun sie im eft só wíse man,
wárun im glawe gumon \hld\ endi gode werðe,
alesane undar þem liudjun, \hld\ kwámun im þarod be þem lèron Kristes,
þat sie is hèlag word \hld\ hòrjen móstin,
línon endi léstjen: \hld\ habdun mid iro ge·lòvon te im
fasto gefangen, \hld\ habdun im ferhten hugi,
wurðun is þegnos te þiu, \hld\ þat he sie an þiod-welon
aftar iro èn-dagon \hld\ up ge·bráhti,
an godes ríki. \hld\ He só gerno ant·feng
man-kunnjes manag \hld\ endi mund-burd gi·hét
te langaru hwílu, \hld\ endi mahta só gi·léstjen wel.
Þó warð þar męgin só mikil \hld\ umbi þana márjon Krist,
liudjo ge·samnod: \hld\ þó gi·sah he fon allun landun kuman,
fon allun wídun wegun \hld\ werod te·samne
lungro liudjo: \hld\ is lof was só wído
managun ge·márid. \hld\ Þó gi·wèt im mahtig self
an ènna berg uppan, \hld\ barno ríkjost,
sundar ge·sittjen, \hld\ endi im selvo ge·kòs
twelivi ge·talda, \hld\ trew-hafta man,
gódoro gumono, \hld\ þea he im te jungoron forð
allaro dago ge·hwi-likes, \hld\ drohtin welda
an is ge·síð-skępja \hld\ simblon hębbjan.
Nemnida sie þó bi naman \hld\ endi hét sie im þó náhor gangan,
Andreas endi Petrus \hld\ èrist sána,
ge·bróðar twène, \hld\ endi bèðje mid im,
Jakobus endi Johannes: \hld\ sie wárun gode werðe;
mildi was he im an is móde; \hld\ sie wárun ènes mannes suni
bèðje bi ge·burdjun; \hld\ sie kòs þat barn godes
góde te jungoron \hld\ endi gumono filu,
márjero manno: \hld\ Mattheus endi Þomas,
Judasas twèna \hld\ endi Jakob óðran,
is selves swiri: \hld\ sie wárun fon gi·sustruonjon twèm
knósles kumana, \hld\ Krist endi Jakob,
góde gadulingos. \hld\ Þó habda þero gumono þar
þe nęrjendo Krist \hld\ niguni ge·talde, %TODO: check niguni
trew-hafte man: \hld\ þó hét he ók þana te·handon gangan
selvo mid þem gi·síðun: \hld\ Símon was he hètan;
hét ók Bartholomeus \hld\ an þana berg uppan
faran fan þem folke áðrum \hld\ endi Philippus mid im,
trew-hafte man. \hld\ Þó gengun sie twelivi samad,
rinkos te þeru rúnu, \hld\ þar þe rádand sat,
managoro mund-boro, \hld\ þe allumu man-kunnje
wið hęllje ge·þwing \hld\ helpan welde,
formon wið þem ferne, \hld\ só hwem só frummjen wili
só liov-líka lèra, \hld\ só he þem liudjun þar
þurh is gi·wit mikil \hld\ wísjan hogda.
*Þó umbi þana nęrjendon Krist \hld\ náhor gengun
su·like ge·síðos, \hld\ só he im selvo ge·kòs,
waldand undar þem werode. \hld\ Stódun wísa man,
gumon umbi þana godes sunu \hld\ gerno swíðo,
weros an willjon: \hld\ was im þero wordo niud,
þáhtun endi þagodun, \hld\ hwat im þero þiodo drohtin,
weldi waldand self \hld\ wordun kúðjen
þesum liudjun te liove. \hld\ Þan sat im þe landes hirdi
gegin-ward for þem gumun, \hld\ godes ègan barn:
welda mid is sprákun \hld\ spáh-word manag
lèrjan þea liudi, \hld\ hwó sie lof gode
an þesum wer-old-ríkja \hld\ wirkjan skoldin.
Sat im þó endi swígoda \hld\ endi sah sie an lango,
was im hold an is hugi \hld\ hèlag drohtin,
mildi an is móde, \hld\ endi þó is mund ant·lók,
wísde mid wordun \hld\ waldandes sunu
manag már-lík þing \hld\ endi þem mannum sagde
spáhun wordun, \hld\ þem þe he te þeru spráku þarod,
Krist alo-waldo, \hld\ ge·koran habda,
hwi-like wárin allaro \hld\ irmin-manno
gode werðoston \hld\ gumono kunnjes;
sagde im þó te sǫ́ðan, \hld\ kwað þat þie sáliga wárin,
man an þesoro middil-gardun, \hld\ þie hér an iro móde wárin
arme þurh òd-módi: \hld\ „þem is þat éwana ríki,
swíðo hèlag-lík \hld\ an hevan-wange
sin-líf far·geven.“ \hld\ Kwað þat ók sálige wárin
máð-mundje man: \hld\ „þie mótun þie márjon erðe,
of-sittjen þat selve ríki.“ \hld\ Kwað þat ók sálige wárin,
þie hír wiopin iro wammun dádi; \hld\ „þie mótun eft willjon ge·bídan,
frófre an iro fráhon ríkja. \hld\ Sálige sind ók, þe sie hír frumono gi·lustid,
rinkos, þat sie rehto a·dómien. \hld\ Þes mótun sie werðan an þem ríkja drohtines
gi·fullit þurh iro ferhton dádi: \hld\ su-líkoro mótun sie frumono bi·knégan
þie rinkos, þie hír rehto a·dómjad, \hld\ ne willjad an rúnun be·swíkan
man, þar sie at mahle sittjad. \hld\ Sálige sind ók þem hír mildi wirðit
hugi an hęliðo briostun: \hld\ þem wirðit þe hèlego drohtin,
mildi mahtig selvo. \hld\ Sálige sind ók undar þesaro managon þiodu,
þie hębbjad iro herta gi·hrènod: \hld\ þie mótun þane hevenes waldand
sehan an sínum ríkja.“ \hld\ Kwað þat ók sálige wárin,
„þie þe friðu-samo undar þesumu folke libbiod \hld\ endi ni willjad èniga fehta ge·wirken,
saka mid iro selvoro dádjun: \hld\ þie mótun wesan suni drohtines ge·nemnide,
hwande he im wil ge·nádig werðen; \hld\ þes mótun sie niotan lango
selvon þes sínes ríkjes.“ \hld\ Kwað þat ók sálige wárin
þie rinkos, þe rehto weldin, \hld\ „endi þurh þat þolod ríkjoro manno
hęti endi harm-kwidi: \hld\ þem is ók an himile eft
godes wang for·geven \hld\ endi gèst-lík líf
aftar te éwan-dage, \hld\ só is io endi ni kumit,
welan wun-sames.“ \hld\ Só habde þó waldand Krist
for þem erlon þar \hld\ ahto ge·talda
sálda ge·sagda; \hld\ mid þem skal simbla gi·hwe
himil-rík ge·halon, \hld\ ef he it hębbjen wili,
etþo he skal te éwan-daga \hld\ aftar þarvon
welon endi willjon, \hld\ síðor he þese wer-old agivid,
erð-lívi-gi·skapu, \hld\ endi sókit im ǫ́ðar lioht
só liof só lèð, \hld\ só he mid þesun liudjun hér
gi·werkod an þesoro wer-oldi, \hld\ al só it þar þó mid is wordun sagde
Krist alo-waldo, \hld\ kuningo ríkjost
godes égen barn \hld\ jungorun sínun:
„Ge werðat ók só sálige“, \hld\ kwað he, „þes iu saka biodat
liudi aftar þeson lande \hld\ endi lèð sprekat,
hębbjad iu te hoska \hld\ endi harmes filu
ge·wirkjad an þesoro wer-oldi \hld\ endi wíti ge·frummjad,
felgjad iu firin-spráka \hld\ endi fíund-skępi,
lágnjad iuwa lèra, \hld\ dót iu lèðes filu,
harmes þurh iuwen hérron. \hld\ Þes látad gi ewan hugi simbla,
líf an lustun, \hld\ hwand iu þat lòn stendit
an godes ríkja garu, \hld\ gódo ge·hwi-likes,
mikil endi manag-fald: \hld\ þat is iu te médu far·geven,
hwand gi hér ér bi·foran \hld\ arvid þolodun,
wíti an þesoro wer-oldi. \hld\ Wirs is þem óðrun,
giviðig grimmora þing, \hld\ þem þe hér gód égun,
wídan worold-welon: \hld\ þie for·slítat iro wunnja hér;
ge·niudot sie ge·nóges, \hld\ skulun eft narowaro þing
aftar iro hin-ferdi \hld\ hęliðos þolojan.
Þan wópjan þar wan-skęfti, \hld\ þie hér ér an wunnjon sín,
libbjad an allon lustun, \hld\ ne willjad þes far·látan wiht,
mèni-gi·þáhtio, \hld\ þes sie an iro mód spenit,
lèðoro gi·léstio. \hld\ Þan im þat lòn kumid,
uvil arvet-sam, \hld\ þan sie is þane endi skulun
sorgondi ge·sehan. \hld\ Þan wirðid im sèr hugi,
þes sie* þesero wer-oldes só filu \hld\ willjan ful-gengun,
man an iro mód-sevon. \hld\ Nu skulun gi im þat mèn lahan,
węrjan mid wordun, \hld\ al só ik giu nu ge·wísjan mag,
sęggjan sǫ́ð-líko, \hld\ ge·síðos míne,
wárun wordun, \hld\ þat gi þesoro wer-oldes nu forð
skulun salt wesan, \hld\ sundigero manno,
bótjan iro balu-dádi, \hld\ þat sie an bętara þing,
folk far·fáhan endi for·látan \hld\ fíundes gi·werk,
diuvales ge·dádi, \hld\ endi sókjan iro drohtines ríki.
Só skulun gi mid iuwon lèrun \hld\ liud-folk manag
węndjan aftar mínon willjon. \hld\ Ef iuwar þan a·wirðid hwi-lik,
far·látid þea lèra, \hld\ þea he léstjan skal,
þan is im só þem salte, \hld\ þe man bi sèes staðe
wído te·wirpit: \hld\ þan it te wihti ni dóg,
ak it firiho barn \hld\ fótun spurnat,
gumon an greote. \hld\ Só wirðid þem, þe þat godes word skal
mannum márjan: \hld\ ef he im þan látid is mód twehon,
þat hi ne willja mid hluttro hugi \hld\ te heven-ríkja
spanen mid is spráku \hld\ endi sęggjan spel godes,
ak wenkid þero wordo, \hld\ þan wirðid im waldand gram,
mahtig módag, \hld\ endi só samo manno barn;
wirðid allun þan \hld\ irmin-þiodun,
liudjun a·lèðid, \hld\ ef is lèra ni dugun.“
So sprak he þó spáh-líko \hld\ endi sagda spel godes,
lèrde þe landes ward \hld\ liudi síne
mid hluttru hugi. \hld\ Hęliðos stódun,
gumon umbi þana godes sunu \hld\ gerno swíðo,
weros an willjon: \hld\ was im þero wordo niud,
þáhtun endi þagodun, \hld\ gi·hòrdun þero þiodo drohtin
sęggjan éu godes \hld\ eldi-barnun;
gi·hét im heven-ríki \hld\ endi te þem hęliðun sprak:
„ók mag ik iu sęggjan, \hld\ ge·síðos mína,
wárun wordun, \hld\ þat gi þesoro wer-oldes nu forð
skulun lioht wesan \hld\ liudjo barnun,
fagar mid firihun \hld\ ovar folk manag,
wlitig endi wun-sam: \hld\ ni mugun iuwa werk mikil
bi·holan werðan, \hld\ mid hwi-liko gi sea hugi kúðjat:
þan mér þe þiu burg ni mag, \hld\ þiu an berge stáð,
hòh holm-klivu, \hld\ bi·holen werðen,
wrisi-lík gi·werk, \hld\ ni mugun iuwa word þan mér
an þesoro middil-gard \hld\ mannum werðen,
iuwa dádi bi·dernit. \hld\ Dót, só ik iu lèrju:
látad iuwa lioht mikil \hld\ liudjun skínan,
manno barnun, \hld\ þat sie far·standan iuwan mód-sevon,
iuwa werk endi iuwan willjon, \hld\ endi þes waldand god
mit hluttro hugi, \hld\ himiliskan fader,
lovon an þesumu liohte, \hld\ þes he iu su·lika lèra far·gaf.
Ni skal neoman lioht, þe it havad, \hld\ liudjun dernjan,
te hardo be·hwelvean, \hld\ ak he it hòho skal
an sęli sęttjan, \hld\ þat þea ge·sehan mugin
alla ge·liko, \hld\ þea þar inna sind,
hęliðos an hallu. \hld\ Þan hald ni skulun gi iuwa hèlag word
an þesumu land-skępa \hld\ liudjun dernjen,
hęlið-kunnje far·helan, \hld\ ak ge it hòho skulun
brèdjan, þat gi·bod godes, \hld\ þat it allaro barno ge·hwi-lik,
ovar al þit land-skępi \hld\ liudi far·standan
endi só ge·frummjen, \hld\ só it an forn-dagun
tulgo wíse man \hld\ wordun ge·sprákun,
þan sie þana aldan éw \hld\ erlos heldun,
endi ók su·liku swíðor, \hld\ só ik iu nu sęggjan mag,
alloro gumono ge·hwi-lik \hld\ gode þionojan,
þan it þar an þem aldom \hld\ éwa ge·beode.
Ni wánjat gi þes mit wihtju, \hld\ þat ik bi þiu an þesa wer-old kwámi,
þat ik þana aldan éu \hld\ irrjen willje,
fellean undar þesumu folke \hld\ efþo þero fora-sagono
word wiðar-werpen, \hld\ þea hér só gi·wárja man
bar-líko ge·budun. \hld\ Ér skal bèðju te·faran,
himil endi erðe, \hld\ þiu nu bi·hlidan standat,
ér þan þero wordo \hld\ wiht bi·líva
un·léstid an þesumu liohte, \hld\ þea sie þesum liudjun hér
wár-líko ge·budun. \hld\ Ni kwam ik an þesa wer-old te þiu,
þat ik feldi þero fora-sagono word, \hld\ ak ik siu fulljen skal,
ókjon endi nígjan \hld\ eldi-barnum,
þesumu folke te frumu. \hld\ Þat was forn ge·skrivan
an þem aldon éo \hld\ —ge hòrdun it oft sprekan
word-wíse man—: \hld\ só hwe só þat an þesoro wer-oldi gi·dót,
þat he áðrana \hld\ aldru bi·neote,
lívu bi·lòsje, \hld\ þem skulun liudjo barn
dòd a·dèljan. \hld\ Þan willjo ik it iu diopor nu,
furður bi·fáhan: \hld\ só hwe só ina þurh fíund-skępi,
man wiðar óðrana \hld\ an is mód-sevon
bilgit an is breostun \hld\ —hwand sie alle ge·bróðar sint,
sálig folk godes, \hld\ sibbjon bi·tengja,
man mid mág-skępi—, \hld\ þan wirðit þoh hwe óðrumu an is móde só gram,
líbes weldi ina bi·lòsjen, \hld\ of he mahti gi·léstjen só:
þan is he sán a·féhit \hld\ endi is þes ferahas skolo,
al su·likes ur-dèljes \hld\ só þe ǫ́ðar was,
þe þurh is hand-męgin \hld\ hòvdo bi·lòsde
erl ǫ́ðarna. \hld\ Ók is an þem éo ge·skrivan
wárun wordun, \hld\ só gi witon alle,
þan man is náhiston \hld\ niud-líko skal
minnjan an is móde, \hld\ wesen is mágun hold,
gadulingun gód, \hld\ wesen is geva mildi,
fráhon is friunda ge·hwane, \hld\ endi skal is fíund hatan,
wiðer-standen þem mid strídu \hld\ endi mid starku hugi,
węrjan wiðar wrèðun. \hld\ Þan sęggjo ik iu te wáron nu,
fullíkur for þesumu folke, \hld\ þat gi iuwa fíund skulun
minnjon an iuwomu móde, \hld\ só samo só gi iuwa mágos dót,
an godes namon. \hld\ Dót im gódes filu,
tògjat im hluttran hugi, \hld\ holda trewa,
liof wiðar ira lèðe. \hld\ Þat is lang-sam rád
manno só hwi-likumu, \hld\ só is mód te þiu
ge·flíhit wiðar is fíunde. \hld\ Þan mótun gi þea fruma ègan,
þat gi mótun hèten \hld\ heven-kuninges suni,
is blíði barn. \hld\ Ne mugun gi iu bętaran rád
ge·winnan an þesoro wer-oldi. \hld\ Þan sęggjo ik iu te wáron ók,
barno ge·hwi-likum, \hld\ þat gi ne mugun mid gi·bolgono hugi
iuwas gódes wiht \hld\ te godes húsun
waldande far·gevan, \hld\ þat it imu wirðig sí
te ant·fáhanne, \hld\ só lango só þu fíund-skępjes wiht,
wiðer óðran man \hld\ in·wid hugis.
Ér skalt þu þi simbla ge·sónjen \hld\ wið þana sak-waldand,
ge·módi gi·mahlean: \hld\ síðor maht þu mèðmos þína
te þem godes altere a·gevan: \hld\ þan sind sie þemu gódan werðe,
heven-kuninge. \hld\ Mér skulun gi aftar is huldi þionon,
godes willjon ful-gán, \hld\ þan óðra Judeon duon,
ef gi willjat ègan \hld\ éwan ríki,
sin-líf sehan. \hld\ Ók skal ik iu sęggjan noh,
hwó it þar an þem aldon \hld\ éo ge·biudid,
þat ènig erl óðres \hld\ idis ni bi·swíka,
wíf mid wammu. \hld\ Þan sęggjo ik iu te wáron ók,
þat þar man is siuni mugun \hld\ swíðo far·lèdjan
an mirki mèn, \hld\ ef hi ina látid is mód spanen,
þat he be·ginna þero girnean, \hld\ þiu imu ge·gangan ni skal.
Þan haved he an imu selvon sán \hld\ sundja ge·warhta,
ge·hęftid an is hertan \hld\ hęlli-wíti.
Ef þan þana man is siun wili \hld\ etþa is swíðare hand
far·lèdjen is liðo hwi-lik \hld\ an lèðan weg,
þan is erlo ge·hwem \hld\ ǫ́ðar bętara,
firiho barno, \hld\ þat he ina fram werpa
endi þana lið lòsje \hld\ af is lík-hamon
endi ina áno kuma \hld\ up te himile,
þan he só mid allun \hld\ te þem Inferne,
hwerve mid só hélun \hld\ an hęlli-grund.
Þan mènid þiu léf-héd, \hld\ þat ènig liudjo ni skal
far·folgan is friunde, \hld\ ef he ina an firina spanit,
swás man an saka: \hld\ þan ne sí he imu eo só swíðo an sibbjun bi·lang,
ne iro mág-skępi só mikil, \hld\ ef he ina an morð spenit,
bédid balu-werko; \hld\ bętera is imu þan ǫ́ðar,
þat he þana friund fan imu \hld\ fer far·werpa,
míðe þes máges \hld\ endi ni hębbja þar èniga minnja tó,
þat he móti èno \hld\ up ge·stígan
hó himil-ríki, \hld\ þan sie hęlli-ge·þwing,
brèd bal-wíti \hld\ bèðja gi·sókjan,
uvil arvidi. \hld\ Ók is an þem éo ge·skrivan
wárun wordun, \hld\ só gi witun alle,
þat míðe mèn-éðos \hld\ man-kunnjes ge·hwi-lik,
ni for·swęrje ina selvon, \hld\ hwand þat is sundje te mikil,
far·lèdid liudi \hld\ an lèðan weg.
Þan willjo ik iu eft sęggjan, \hld\ þan sán ni swęrja neoman
ènigan éð-staf \hld\ eldi-barno,
ne bi himile þemu hòhon, \hld\ hwand þat is þes hérron stól,
ne bi erðu þar undar, \hld\ hwand þat is þes alo-waldon
fagar fót-skamel, \hld\ nek ènig firiho barno
ne swęrja bi is selves hòvde, \hld\ hwand he ni mag þar ne swart ne hwít
ènig hár ge·wirkjan, \hld\ b·útan só it þe hèlago god,
ge·markode mahtig; \hld\ be·þiu skulun míðan filu
erlos éð-wordo. \hld\ Só hwe só it ofto dót,
só wirðid is simbla wirsa, \hld\ hwand he imu gi·wardon ni mag.
Bi·þiu skal ik iu nu te wárun \hld\ wordun gi·beodan,
þat gi neo ne swęrjen \hld\ swíðoron éðos,
méron met mannun, \hld\ b·útan só ik iu mid mínun hér
swíðo wár-liko \hld\ wordun ge·biudu:
ef man hwemu saka sókja, \hld\ bi·sęggja þat wáre,
kweðe iá, gef it sí, \hld\ geha þes þar wár is,
kweðe nèn, af it nis, \hld\ láta im ge·nóg an þiu;
só hwat só is mér ovar þat \hld\ man ge·frummjad,
só kumid it al fan uvile \hld\ eldi-barnun,
þat erl þurh un·trewa \hld\ óðres ni wili
wordo ge·lòvjan. \hld\ Þan sęggjo ik iu te wáron ók,
hwó it þar an þem aldon \hld\ éo ge·biudit:
só hwe só ògon ge·nimid \hld\ óðres mannes,
lòsid af is lík-haman, \hld\ etþa is liðo hwi-likan,
þat he it eft mid is selves skal \hld\ sán ant·gelden
mid ge·líkun liðjon. \hld\ Þan willjo ik iu lèrjan nu,
þat gi só ni wrekan \hld\ wrèða dádi,
ak þat gi þurh òd-módi \hld\ al ge·þologjan
wítjes endi wammes, \hld\ só hwat só man iu an þesoro wer-oldi gedóe.
Dóe alloro erlo ge·hwi-lik \hld\ óðrom manne
frume endi ge·fóri, \hld\ só he willje, þat im firiho barn
gódes an·gegin dóen. \hld\ Þan wirðit im god mildi,
liudjo só hwi-likum, \hld\ só þat léstjen wili.
Érod gi arme man, \hld\ dèljad iwan òd-welon
undar þero þurftigon þiodu; \hld\ ne rókjad, hweðar gi is ènigan þank ant·fáhan
efþo lòn an þesoro léhneon wer-oldi, \hld\ ak huggjat te iuwomu leovon hérran
þero gevono te gelde, \hld\ þat sie iu god lòno,
mahtig mund-boro, \hld\ só hwat só gi is þurh is minnes gi·dót.
Ef þu þan gevogjan wili \hld\ gódun mannun
fagare feho-skattos, \hld\ þar þu eft frumono hugis
mér ant·fáhan, \hld\ te hwí havas þu þes èniga méda fon gode
etþa lòn an þemu is liohte? \hld\ hwand þat is léhni feho.
Só is þes alles ge·hwat, \hld\ þe þu óðrun ge·duos
liudjon te leove, \hld\ þar þu hugis eft ge·lík neman
þero wordo endi þero werko: \hld\ te hwí wèt þi þes úsa waldand þank,
þes þu þín só bi·filhis \hld\ endi ant·fáhis eft þan þu wili?
iuwan òð-welon \hld\ gevan gi þem armun mannun,
þe ina iu an þesoro wer-oldi ne lònon \hld\ endi rómot te iuwes waldandes ríkja.
Te hlúd ni dó þu it, \hld\ þan þu mid þínun handun bi·felhas
þína alamosna þemu armon manne, \hld\ ak dó im þurh òd-módjen
gerno þurh godes þank: \hld\ þan móst þu eft geld niman,
swíðo liof-lík lòn, \hld\ þar þu is lango bi·þarft,
fagaroro frumono. \hld\ Só hwat só þu is só þurh ferhtan hugi
darno ge·dèljas, \hld\ —so is úsumu drohtine werð—
ne galpo þu far þínun gevun te swíðo, \hld\ noh ènig gumono ne skal,
þat siu im þurh ídale hróm \hld\ eft ni werðe
lèð-líko far·loren. \hld\ Þanna þu skalt lòn nemen
fora godes ògun \hld\ gódero werko.
Ók skal ik iu ge·beodan, \hld\ þan gi willjad te bedu hnígan
endi willjad te iuwomu hérron \hld\ helpono biddjan,
þat he iu a·láte \hld\ lèðes þinges,
þero sakono endi þero sundjono, \hld\ þea gi iu selvon hír
wrèða ge·wirkjad, \hld\ þat gi it þan for óðrumu werode ni duad:
ni márjad it far męnigi, \hld\ þat iu þes man ni lovon,
ni diurjan þero dádjo, \hld\ þat gi iuwes drohtines gi·bed
þurh þat ídala hróm \hld\ al ne far·leosan.
Ak þan gi willjan te iuwomo hérron \hld\ helpono biddjan,
þiggjan þeo-líko, \hld\ —þes iu is þarf mikil—
þat iu sigi-drohtin \hld\ sundjono tómja,
þan dót gi þat só darno: \hld\ þoh wèt it iuwe drohtin self
hèlag an himile, \hld\ hwand imu nis bi·holan n·eo·wiht
ne wordo ne werko. \hld\ He látid it þan al ge·werðan só,
só gi ina þan biddjad, \hld\ þan gi te þero bedo hnígad
mid hluttru hugi.“ \hld\ Hęliðos stódun,
gumon umbi þana godes sunu \hld\ gerno swíðo,
weros an willjon: \hld\ was im þero wordo niud,
þáhtun endi þagodun, \hld\ was im þarf mikil,
þat sie þat eft ge·hogdin, \hld\ þat im þat hèlaga barn
an þana forman sið \hld\ filu mid wordun
torhtes ge·talde. \hld\ Þó sprak im eft èn þero twelivjo an·gegin,
glauworo gumono, \hld\ te þem godes barne:
„Hérro þe gódo“, \hld\ kwað he, „ús is þínoro huldi þarf,
te gi·wirkenne þínna willjon, \hld\ endi ók þínoro wordo só self,
allaro barno bętst, \hld\ þat þu ús bedon lères,
jungoron þíne, \hld\ só Johannes duot,
diur-lík dòperi, \hld\ dago ge·hwi-likas
is werod mid wordun, \hld\ hwí sie waldand skulun,
gódan grótjan. \hld\ Dó þína jungorun só self:
ge·rihti ús þat ge·rúni.“ \hld\ Þó habda eft þe ríkjo garu
sán aftar þiu, \hld\ sunu drohtines,
gód word an·gegin: \hld\ „Þan gi god willjan“, kwað he,
„weros mid iuwon wordun \hld\ waldand grótjan,
allaro kuningo kraftigostan, \hld\ þan kweðad gi, só ik iu lèrju:
Fadar úsa \hld\ firiho barno,
þu bist an þem hòhon \hld\ himila ríkja,
ge·wíhid sí þín namo \hld\ wordo ge·hwi-liko.
Kuma þín \hld\ kraftag ríki.
Werða þín willjo \hld\ ovar þesa wer-old alla,
só sama an erðo, \hld\ só þar uppa ist
an þem hòhon \hld\ himilo ríkja.
Gef ús dago ge·hwi-likes rád, \hld\ drohtin þe gódo,
þína hèlaga helpa, \hld\ endi a·lát ús, hevenes ward,
managoro mèn-skuldjo, \hld\ al só we óðrum mannum dóan.
Ne lát ús far·lèdjan \hld\ lèða wihti
só forð an iro willjon, \hld\ só wi wirðige sind,
ak help ús wiðar allun \hld\ uvilon dádjun.
Só skulun gi biddjan, \hld\ þan gi te bede hnígad
weros mid iuwom wordun, \hld\ þat iu waldand god
lèðes a·láte \hld\ an leut-kunnja.
Ef gi þan willjad a·látan \hld\ liudjo ge·hwi-likun
þero sakono endi þero sundjono, \hld\ þe sie wið iu selvon hír
wrèða ge·wirkjat, \hld\ þan a·látid iu waldand god,
fadar ala-mahtig \hld\ firin-werk mikil,
managoro mèn-skuldjo. \hld\ Ef iu þan wirðid iuwa mód te stark,
þat gi ne wileat óðrun \hld\ erlun a·látan,
weron wam-dádi, \hld\ þan ne wil iu ók waldand god
grim-werk far·gevan, \hld\ ak gi skulun is geld niman,
swíðo lèð-lik lòn \hld\ te languru hwílu,
alles þes un·rehtes, \hld\ þes gi óðrum hír
gi·léstjad an þesumu liohte \hld\ endi þan wið liudjo barn
þea saka ni gi·sónjad, \hld\ ér gi an þana síð faran,
weros fon þesoro wer-oldi. \hld\ Ok skal ik iu te wárun sęggjan,
hwó gi léstjan skulun \hld\ lèra mína:
þan gi iuwa fastonnja \hld\ frummjan willjan,
minson iuwa mèn-dádi, \hld\ þan ni duad gi þat te managom kúð,
ak míðad is far óðrum mannun: \hld\ þoh wèt mahtig god,
waldand iuwan willjan, \hld\ þoh iu werod ǫ́ðar,
liudjo barn ne lovon. \hld\ He gildid is iu lòn aftar þiu,
iuwa hèlag fadar \hld\ an himil-ríkja,
þes ge im mid su·likum òd-módja, \hld\ erlos þeonod,
só ferhtlíko undar þesumu folke. \hld\ Ne willjat feho winnan
erlos an un·reht, \hld\ ak wirkjad up te gode
man aftar médu: \hld\ þat is méra þing,
þan man hír an erðu \hld\ òdag libbja,
wer-old-skattes ge·wono. \hld\ Ef gi willjad mínun wordun hòrjan,
þan ne samnod gi hír sink mikil \hld\ silovres ne goldes
an þesoro middil-gard, \hld\ mèðom-hordes,
hwand it rotat hír an roste, \hld\ endi ręgin-þeovos far·stelad,
wurmi a·wardjad, \hld\ wirðid þat gi·wádi far·slitan,
ti-gangid þe gold-welo. \hld\ Léstjad iuwa gódon werk,
samnod iu an himile \hld\ hord þat méra,
fagara feho-skattos: \hld\ þat ni mag iu ènig fíund beniman,
ne-wiht an·węndjan, \hld\ hwand þe welo standid
garu iu te·gegnes, \hld\ só hwat só gi gódes þarod,
an þat himil-ríki \hld\ hordes ge·samnod,
hęliðos þurh iuwa hand-geva, \hld\ endi hębbjad þarod iuwan hugi fasto;
hwand þar ist alloro manno gi·hwes \hld\ mód-ge·þáhti,
hugi endi herta, \hld\ þar is hord ligid,
sink ge·samnod. \hld\ Nis eo só sálig man,
þat mugi an þesoro brèdon wer-old \hld\ bèðju ant·hengjan,
ge þat hi an þesoro erðo \hld\ òdag libbja,
an allun wer-old-lustun wesa, \hld\ ge þoh waldand gode
te þanke ge·þeono: \hld\ ak he skal alloro þingo gi·hwes
simbla ǫ́ðar-hweðar \hld\ èn far·látan
etþo lusta þes lík-hamon \hld\ etþo líf èwig.
Be·þiu ni gornot gi umbi iuwa ge·garuwi, \hld\ ak huggjad te gode fasto,
ne mornont an iuwomu móde, \hld\ hwat gi eft an morgan skulin
etan efþo drinkan \hld\ etþo an hębbjan
weros te ge·wédja: \hld\ it wèt al waldand god,
hwes þea bi·þurvun, \hld\ þea im hír þionod wel,
folgod iro fròhan willjon. \hld\ Hwat, gi þat bi þesun fuglun mugun
wár-líko undar-witan, \hld\ þea hír an þesoro wer-oldi sint,
farad an feðar-hamun: \hld\ sie ni kunnun ènig feho winnan,
þoh givid im drohtin god \hld\ dago ge·hwi-likes
helpa wiðar hungre. \hld\ Ók mugun gi an iuwom hugi markon,
weros umbi iuwa ge·wádi, \hld\ hwó þie wurti sint
fagoro ge·fratohot, \hld\ þea hír an felde stád,
berht-líko ge·blóid: \hld\ ne mahta þe burges ward,
Salomon þe suning, \hld\ þe habda sink mikil,
mèðom-hordas mèst, \hld\ þero þe ènig man éhti,
welono ge·wunnan \hld\ endi allaro ge·wádjo kust, —
þoh ni mohte he an is líve, \hld\ þoh he habdi alles þeses landes ge·wald,
a-winnan su·lik ge·wádi, \hld\ só þiu wurt havad,
þiu hír an felde stád \hld\ fagoro ge·gariwit,
lilli mid só liof-líku blómon: \hld\ ina wádit þe landes waldand
hér fan hevenes wange. \hld\ Mér is im þoh umbi þit hęliðo kunni,
liudi sint im liovoron mikilu, \hld\ þea he im an þesumu lande ge·warhte,
waldand an willjon sínan. \hld\ Be·þiu ne þurvon gi umbi iuwa ge·wádi sorgon,
ne gornot gi umbi iuwa ge·gariwi te swíðo: \hld\ god wili is alles rádan,
helpan fan hevenes wange, \hld\ ef gi willjad aftar is huldi þeonon.
Gerot gi simbla èrist þes godes ríkjas, \hld\ endi þan duat aftar þem is gódun werkun,
rómod gi rehtoro þingo: \hld\ þan wili iu þe ríkjo drohtin
gevon mid alloro gódu ge·hwi-liku, \hld\ ef gi im þus ful-gangan willjad,
só ik iu te wárun hír \hld\ wordun seggjo.
Ne skulun gi ènigumu manne \hld\ un·rehtes wiht,
dervjes a·dèljan, \hld\ hwand þe dóm eft kumid
ovar þana selvon man, \hld\ þar it im te sorgon skal,
werðan þem te wítja, \hld\ þe hír mid is wordun ge·sprikid
un·reht óðrum. \hld\ Neo þat iuwar ènig ne dua
gumono an þesom gardon \hld\ geldes etþo kòpes,
þat hi un·reht gi·met \hld\ óðrumu manne
mèn-ful mako, \hld\ hwand it simbla mótjan skal
erlo ge·hwi-likomu, \hld\ su·lik só he it óðrumu ge·dód,
só kumid it im eft te·gegnes, \hld\ þar he gerno ne wili
ge·sehan is sundjon. \hld\ Ók skal ik iu sęggjan noh,
hwar gi iu wardon skulun \hld\ wítjo mèsta,
mèn-werk manag: \hld\ te hwí skalt þu ènigan man besprekan,
bróðar þínan, \hld\ þat þu undar is bráhon ge·sehas
halm an is ògon, \hld\ endi ge·huggjan ni wili
þana swáran balkon, \hld\ þe þu an þínoro siuni havas,
hard trio endi hevig. \hld\ Lát þi þat an þínan hugi fallan,
hwó þu þana èrist a·lòsjas: \hld\ þan skínid þi lioht be·foran,
ògun werðad þi ge·oponot; \hld\ þan maht þu aftar þiu
swáses mannes gesiun \hld\ síðor ge·bótjan,
ge·hèljan an is hòvde. \hld\ Só mag þat an is hugi méra
an þesoro middil-gard \hld\ manno ge·hwi-likumu,
wesan an þesoro wer-oldi, \hld\ þat hi hír wammas ge·duot,
þan hi ahtogja \hld\ óðres mannes
saka endi sundja, \hld\ endi havad im selvo mér
firin-werko ge·frumid. \hld\ Ef he wili is fruma léstjan,
þan skal hi ina selvon ér \hld\ sundjono a·tómjan,
lèð-werko lòson: \hld\ síðor mag hi mid is lèrun werðan
hęliðun te helpu, \hld\ síðor hi ina hluttran wèt,
sundjono sikoran. \hld\ Ne skulun gi swínum te·foran
iuwa mere-gríton makon \hld\ etþo mèðmo ge·striuni,
hèlag hals-męni, \hld\ hwand siu it an horu spurnat,
sulwjad an sande: \hld\ ne witun súvrjas ge·skéð,
fagaroro fratoho. \hld\ Su-lik sint hír folk manag,
þe iuwa hèlag word \hld\ hòrjan ne willjad,
ful-gangan godes lèrun: \hld\ ne witun gódes ge·skéð,
ak sind im lári word \hld\ leovoron mikilu,
umbi·þarvi þing, \hld\ þanna þeot-godes
werk endi willjo. \hld\ Ne sind sie wirðige þan,
þat sie ge·hòrjan iuwa hèlag word, \hld\ ef sie is ne willjad an iro hugi þęnkjan,
ne línon ne léstjan. \hld\ Þem ni sęggjan gi iuworo lèron wiht,
þat gi þea spráka godes \hld\ endi spel managu
ne far·leosan an þem liudjun, \hld\ þea þar ne willjan gi·lòvjan tó,
wároro wordo. \hld\ Ók skulun gi iu wardon filu
listjun undar þesun liudjun, \hld\ þar gi aftar þesumu lande farad,
þat iu þea luggjon ne mugin \hld\ lèron be·swíkan
ni mid wordun ni mid werkun. \hld\ Sie kumad an su·likom ge·wádjon te iu,
fagoron fratohon: \hld\ þoh hębbjad sie féknan hugi:
þea mugun gi sán ant·kęnnjan, \hld\ só gi sie kuman ge·sehad:
sie sprekad wís-lík word, \hld\ þoh iro werk ne dugin,
þero þegno ge·þáhti. \hld\ Hwand gi witun, þat eo an þorniun ne skulun
wín-beri wesan \hld\ efþa welon eo·wiht,
fagororo fruhtjo, \hld\ nek ók fígun ne lesad
hęliðos an hiopon. \hld\ Þat mugun gi undar-huggjan wel,
þat eo þe uvilo bóm, \hld\ þar he an erðu stád,
góden wastum ne givid, \hld\ nek it ók god ni ge·skóp,
þat þe gódo bóm \hld\ gumono barnun
bári bittres wiht, \hld\ ak kumid fan alloro bámo ge·hwi-likumu
su·lik wastom te þesero wer-oldi, \hld\ só im fan is wurtjon ge·dregid,
etþa berht etþa bittar. \hld\ Þat mènid þoh breost-hugi,
managoro mód-sevon \hld\ manno kunnjes,
hwó alloro erlo ge·hwi-lik \hld\ ógit selvo,
meldod mid is múðu, \hld\ hwi-likan he mód havad,
hugi umbi is herte: \hld\ þes ni mag he far·helan eo·wiht,
ak kumad fan þem uvilan man \hld\ in·wid-rádos,
bittara balu-spráka, \hld\ su·lik só hi an is breostun havad
ge·hęftid umbi is herte: \hld\ simbla is hugi kúðid,
is willjon mid is wordun, \hld\ endi farad is werk aftar þiu.
Só kumad fan þemu gódan manne \hld\ glau and-wordi,
wís-lík fan is ge·wittja, \hld\ þat hi simbla mid is wordu ge·sprikid,
man mid is míðu su·lik, \hld\ só he an is móde havad
hord umbi is herte. \hld\ Þanan kumad þea hèlagan lèra,
swíðo wun-sam word, \hld\ endi skulun is werk aftar þiu
þeodu ge·þíhan, \hld\ þegnun managun
werðan te willjon, \hld\ al só it waldand self
gódun mannun far·givid, \hld\ god alo-mahtig,
himilisk hérro, \hld\ hwand sie áno is helpa ni mugun
ne mid wordun ne mid werkun \hld\ wiht a·þengjan
gódes an þesun gardun. \hld\ Be·þiu skulun gumono barn
an is ènes kraft \hld\ alle gi·lòvjan.
Ók skal ik iu wísjan, \hld\ hwó hír wegos twèna
liggjad an þesumu liohte, \hld\ þea farad liudjo barn,
al irmin-þiod. \hld\ Þero is ǫ́ðar sán
wíd stráta endi brèd, \hld\ —farid sie werodes filu,
man-kunnjes manag, \hld\ hwand sie þarod iro mód spenit,
wer-old-lusta weros— \hld\ þiu an þea wirson hand
liudi lèdid, \hld\ þar sie te far·lora werðad,
hęliðos an hęllju, \hld\ þar is hèt endi swart,
egis-lík an innan: \hld\ óði ist þarod te faranne
eldi-barnun, \hld\ þoh it im at þemu endje ni dugi.
Þan ligid eft ǫ́ðar \hld\ engira mikilu
weg an þesoro wer-oldi, \hld\ fęrid ina werodes lút,
fáho folk-skępi: \hld\ ni willjad ina firiho barn
gerno gangan, \hld\ þoh he te godes ríkja,
an þat èwiga líf, \hld\ erlos lèdja.
Þan nimad gi iu þana engjan: \hld\ þoh he só óði ne sí
firihon te faranne, \hld\ þoh skal hi te frumu werðan
só hwemu só ina þurh-gengid, \hld\ só skal is geld niman,
swíðo lang-sam lòn \hld\ endi líf èwig,
diur-líkan dròm. \hld\ Eo gi þes drohtin skulun,
waldand biddjen, \hld\ þat gi þana weg mótin
fan foran ant·fáhan \hld\ endi forð þurh gi·gangan
an þat godes ríki. \hld\ He ist garu simbla
wiðar þiu te gevanne, \hld\ þe man ina gerno bidid,
fergot firiho barn. \hld\ Sókjad fadar iuwan
up te þemu éwinom ríkja: \hld\ þan mótun gi ina aftar þiu
te iuworu frumu fíðan. \hld\ Kúðead iuwa fard þarod
at iuwas drohtines durun: \hld\ þan werðad iu andón aftar þiu,
himil-portun ant·hlidan, \hld\ þat gi an þat hèlage lioht,
an þat godes ríki \hld\ gangan mótun,
sin-líf sehan. \hld\ Ók skal ik iu sęggjan noh
far þesumu werode allun \hld\ wár-lík biliði,
þat alloro liudjo só hwi-lik, \hld\ só þesa mína lèra wili
ge·haldan an is herton \hld\ endi wil iro an is hugi a·þęnkjan,
léstjan sea an þesumu lande, \hld\ þe gi·líko duot
wísumu manne, \hld\ þe gi·wit havad,
horska hugi-skęfti, \hld\ endi hús-stędi kiusid
an fastoro foldun \hld\ endi an felisa uppan
wégos wirkid, \hld\ þar im wind ni mag,
ne wág ne watares stròm \hld\ wihtiu ge·tiunjan,
ak mag im þar wið un·gi·widereon \hld\ allun standan
an þemu felise uppan, \hld\ hwand it só fasto warð
gi·stellit an þemu stène: \hld\ anthavad it þiu stędi niðana,
wreðid wiðar winde, \hld\ þat it wíkan ni mag.
Só duot eft manno só hwi-lik, \hld\ só þesun mínun ni wili
lèrun hòrjen ne þero \hld\ léstjen wiht,
só duot þe un·wíson \hld\ erla ge·líko,
un·ge·wittigon were, \hld\ þe im be watares staðe
an sande wili \hld\ sęli-hús wirkjan,
þar it westrani wind \hld\ endi wágo stròm,
sèes úðjon te·sláad; \hld\ ne mag im sand endi greot
ge·wreðjen wið þemu winde, \hld\ ak wirðid te·worpan þan,
te·fallen an þemu flóde, \hld\ hwand it an fastoro nis
erðu ge·timbrod. \hld\ Só skal allaro erlo ge·hwes
werk ge·þíhan wiðar þiu, \hld\ þe hi þius mín word frumid,
haldid hèlag ge·bod.“ \hld\ Þó bi·gunnun an iro hugi wundron
męgin-folk mikil: \hld\ ge·hòrdun mahtiges godes
liof-líka lèra; \hld\ ne wárun an þemu lande ge·wuno,
þat sie eo fan su·likun ér \hld\ sęggjan ge·hòrdin
wordun etþo werkun. \hld\ Far·stódun wíse man,
þat he só lèrde, \hld\ liudjo drohtin,
wárun wordun, \hld\ só he ge·wald habde,
allun þem un·ge·líko, \hld\ þe þar an ér-dagun
undar þem liud-skępja \hld\ lèrjon wárun
a-koran undar þemu kunnje: \hld\ ne habdun þiu Kristes word
ge·makon mid mannun, \hld\ þe he far þero męnigi sprak,
ge·bód uppan þemu berge. \hld\ He im þó bèðju be·falh
ge te seggennja \hld\ sínom wordun,
hwó man himil-ríki \hld\ ge·halon skoldi,
wíd-brèdan welan, \hld\ gia he im ge·wald far·gaf,
þat sie móstin hèljan \hld\ halte endi blinde,
liudjo léf-hédi, \hld\ legar-będ manag,
swára suhti, \hld\ giak he im selvo ge·bód,
þat sie at ènigumu manne \hld\ méde ne námin,
diurje mèðmos: \hld\ „ge·huggjad gi“, kwað he, —„hwand iu is þiu dád kuman,
þat ge·wit endi þe wís-dóm, \hld\ endi iu þea ge·wald far·givid
alloro firiho fadar, \hld\ só gi sie ni þurvun mid ènigo feho kòpon,
médjan mid ènigun mèðmun,— \hld\ só wesat gi iro mannun forð
an iuwon hugi-skęftjun \hld\ helpono mildja,
lèrjad gi liudjo barn \hld\ lang-samna rád,
fruma forð-wardes; \hld\ firin-werk lahad,
swára sundjon. \hld\ Ne látad iu silovar nek gold
wihti þes wirðig, \hld\ þat it eo an iuwa ge·wald kuma,
fagara feho-skattos: \hld\ it ni mag iu te ènigoro frumu hwergin,
werðan te ènigumu willjon. \hld\ Ne skulun gi ge·wádjas þan mér
erlos ègan, \hld\ b·útan só gi þan an hębbjan,
gumon te garewea, \hld\ þan gi gangan skulun
an þat gi·mang innan. \hld\ Neo gi umbi iuwan męti ni sorgot,
leng umbi iuwa líf-nare, \hld\ hwand þene lèrjand skulun
fódjan þat folk-skępi: \hld\ þes sint þea fruma werða,
leov-líkes lònes, \hld\ þe hi þem liudjun sagad.
wirðig is þe wurhtjo, \hld\ þat man ina wel fódja,
þana man mid mósu, \hld\ þe só managoro skal
seola bi·sorgan \hld\ endi an þana síð spanen,
gèstos an godes wang. \hld\ Þat is gròtara þing,
þat man bi·sorgon skal \hld\ seolun managa,
hwó man þea ge·halde \hld\ te heven-ríkja,
þan man þene lík-hamon \hld\ liudi-barno
mósu bi·morna. \hld\ Be·þiu man skulun
haldan þene hold-líko, \hld\ þe im te heven-ríkja
þene weg wísit \hld\ endi sie wam-skaðun,
feondun wit-fáhit \hld\ endi firin-werk lahid,
swára sundjon. \hld\ Nu ik iu sęndjan skal
aftar þesumu land-skępje \hld\ só lamb undar wulvos:
só skulun gi undar iuwa fíund faren, \hld\ undar filu þeodo,
undar mis-líke man. \hld\ Hębbjad iuwan mód wiðar þem
só glawan te·gegnes, \hld\ só samo só þe gelwo wurm,
nádra þiu féha, \hld\ þar siu iro níð-skępjes,
witodes wánit, \hld\ þat man iu undar þemu werode ne mugi
be·swíkan an þemu síðe. \hld\ Far þiu gi sorgon skulun,
þat iu þea man ni mugin \hld\ mód-ge·þáhti,
willjan a·wardjen. \hld\ Wesat iu so wara wiðar þiu,
wið iro fékneon dádjun, \hld\ só man wiðar fíundun skal.
Þan wesat gi eft an iuwon dádjun \hld\ dúvon ge·líka,
hębbjad wið erlo ge·hwene \hld\ èn-faldan hugi,
mildjan mód-sevon, \hld\ þat þar man negèn
þurh iuwa dádi \hld\ be·drogan ne werðe,
be·swikan þurh iuwa sundja. \hld\ Nu skulun gi an þana síð faran,
an þat árundi: \hld\ þar skulun gi arvidjes só filu
ge·þolon undar þeru þiod \hld\ endi ge·þwing só samo
manag endi mis-lík, \hld\ hwand gi an mínumu namon
þea liudi lèrjat. \hld\ Be·þiu skulun gi þar lèðes filu
fora wer-old-kuningun, \hld\ wítjas ant·fáhan.
Oft skulun gi þar for ríkja \hld\ þurh þius mín rehtun word
ge·bundane standen \hld\ endi bèðju ge·þologjan,
ge hosk ge harm-kwidi: \hld\ umbi þat ne látad gi iuwan hugi twíflon,
sevon swíkandjan: \hld\ gi ni þurvun an ènigun sorgun wesan
an iuwomu hugi hwergin, \hld\ þan man iu for þea héri forð
an þene gast-sęli \hld\ gangan hétid,
hwat gi im þan te·gegnes skulin \hld\ gódoro wordo,
spáh-líkoro ge·sprekan, \hld\ hwand iu þiu spód kumid,
helpe fon himile, \hld\ endi sprikid þe hélogo gèst,
mahtig fon iuwomu munde. \hld\ Be·þiu ne and-rádad gi iu þero manno níð
ne forhtjat iro fíund-skępi: \hld\ þoh sie hębbjan iuwas ferahes ge·wald,
þat sie mugin þene lík-hamon \hld\ lívu beneotan,
a-slahan mid swerde, \hld\ þoh sie þeru seolun ne mugun
wiht a·wardjan. \hld\ Antd-rádad iu waldand god,
forhtjad fader iuwan, \hld\ frummjad gerno
is ge·bod-skępi, \hld\ hwand hi havad bèðjes gi·wald,
liudjo líves \hld\ endi ók iro lík-hamon
gek þero seolon só self: \hld\ ef gi iuwa an þem síðe þarod
far·liosat þurh þesa lèra, \hld\ þan mótun gi sie eft an þemu liohte godes
be·foran fíðan, \hld\ hwand sie fader iuwa,
haldid hèlag god \hld\ an himil-ríkja.
Ne kumat þea alle te himile, \hld\ þea þe hír hrópat te mi
manno te mund-burd. \hld\ Managa sind þero,
þea willjad alloro dago ge·hwi-likes \hld\ te drohtine hnígan,
hrópad þar te helpu \hld\ endi huggjad an ǫ́ðar,
wirkjad wam-dádi: \hld\ ne sind im þan þiu word fruma,
ak þea mótun hwervan \hld\ an þat himiles lioht,
gangan an þat godes ríki, \hld\ þea þes gerne sint,
þat sie hír ge·frummjen \hld\ fader ala-waldan
werk endi willjon. \hld\ Þea ni þurvun mid wordun só fílu
hrópan te helpu, \hld\ hwanda þe hélogo god
wèt alloro manno ge·hwes \hld\ mód-ge·þáhti,
word endi willjon, \hld\ endi gildid im is werko lòn.
Be·þiu skulun gi sorgon, \hld\ þan gi an þene síð farad,
hwó gi þat árundi \hld\ ti ęndja be·brengen.
Þan gi líðan skulun \hld\ aftar þesumu land-skępja,
wído aftar þesoro wer-oldi, \hld\ al só iu wegos lèdjad,
brèd stráta te burg, \hld\ simbla sókjad gi iu þene bętston sán
man undar þeru męnegi \hld\ endi kúðjad imu iuwan móð-sevon
wárun wordun. \hld\ Ef sie þan þes wirðige sint,
þat sie iuwa gódun werk \hld\ gerno ge·léstjen
mid hluttru hugi, \hld\ þan gi an þemu húse mid im
wonod an willjon \hld\ endi im wel lònod,
geldad im mid gódu \hld\ endi sie te gode selvon
wordun ge·wíhad \hld\ endi sęggjad im wissan friðu,
hèlaga helpa \hld\ heven-kuninges.
Ef sie þan só sáliga \hld\ þurh iro selvoro dád
werðan ni mótun, \hld\ þat sie iuwa werk frummjen,
léstjen iuwa lèra, \hld\ þan gi fan þem liudjun sán,
farad fan þemu folke, \hld\ —þe iuwa friðu hwirvid
eft an iuworo selvoro síð,— \hld\ endi látad sie mid sundjun forð,
mid balu-werkun búan \hld\ endi sókjad iu burg óðra,
mikil man-werod, \hld\ endi ne látad þes melmes wiht
folgan an iuwom fótun, \hld\ þanan þe man iu ant·fáhan ne wili,
ak skuddjat it fan iuwon skóhun, \hld\ þat it im eft te skamu werðe,
þemu werode te ge·wit-skępje, \hld\ þat iro willjo ne dóg.
Þan sęggjo ik iu te wárun, \hld\ só hwan só þius wer-old endjad
endi þe márjo dag \hld\ ovar man farid,
þat þan Sodomo-burg, \hld\ þiu hír þurh sundjon warð
an af·grundi \hld\ éldes kraftu,
fiuru bi·fallen, \hld\ þat þiu þan havad friðu méran,
mildiran mund-burd, \hld\ þan þea man égin,
þe iu hír wiðar-werpat \hld\ endi ne willjad iuwa word frummjen.
Só hwe só iu þan ant·fáhit \hld\ þurh ferhtan hugi,
þurh mildjan mód, \hld\ só havad mínan forð
willjon ge·warhten \hld\ endi ók waldand god,
ant·fangan fader iuwan, \hld\ firiho drohtin,
ríkjan rád-gevon, \hld\ þene þe al reht bikan.
wèt waldand self, \hld\ endi willjan lònot
gumono ge·hwi-likumu, \hld\ só hwat só hi hír gódes geduot,
þoh hi þurh minnja godes \hld\ manno hwi-likumu
willjandi far·geve \hld\ watares drinkan,
þat hi þurftigumu manne \hld\ þurst ge·hèlje,
kaldes brunnan. \hld\ Þesa kwidi werðad wára,
þat eo ne bi·lívid, \hld\ ne hi þes lòn skuli,
fora godes ògun \hld\ geld ant·fáhan,
méda manag-falde, \hld\ só hwat só hi is þurh mína minnja geduot.
Só hwe só mín þan far·lógnid \hld\ liudi-barno,
hęliðo for þesoro hęrju, \hld\ só dóm ik is an himile só self
þar uppe far þem alo-waldan fader \hld\ endi for allumu is ęngilo krafte,
far þeru mikilon męnigi. \hld\ Só hwi-lik só þan eft manno barno
an þesoro wer-oldi ne wili \hld\ wordun míðan,
ak gihit far gum-skępi, \hld\ þat he mín jungoro sí,
þene willju ek eft ógjan \hld\ far ògun godes,
fora alloro firiho fader, \hld\ þar folk manag
for þene alo-waldon \hld\ alla gangad
reðinon wið þene ríkjon. \hld\ Þar willju ik imu an reht wesan
mildi mund-boro, \hld\ só hwemu só mínun hír
wordun hòrid \hld\ endi þiu werk frumid,
þea ik hír an þesumu berge uppan \hld\ ge·boden hębbju.“
Habda þó te wárun \hld\ waldandes sunu
ge·lèrid þea liudi, \hld\ hwó sie lof gode
wirkjan skoldin. \hld\ Þó lét hi þat werod þanan
an alloro halva ge·hwi-lika, \hld\ hęri-skępi manno
síðon te selðon. \hld\ Habdun selves word,
ge·hòrid heven-kuninges \hld\ hèlaga lèra,
só eo te wer-oldi sint \hld\ wordo endi dádjo,
man-kunnjes manag \hld\ ovar þesan middil-gard
sprákono þiu spáhiron, \hld\ só hwe só þiu spel ge·frang,
þea þar an þemu berge ge·sprak \hld\ barno ríkjast.
Ge·wèt imu þó umbi þrea naht aftar þiu \hld\ þesoro þiodo drohtin
an Galileo land, \hld\ þar he te ènum gòmum warð,
ge·bedan þat barn godes: \hld\ þar skolda man èna brúd gevan,
muna-líka magað. \hld\ Þar Maria was,
mid iro suni selvo, \hld\ sálig þiorna,
mahtiges móder. \hld\ Managoro drohtin
geng imu þó mid is jungoron, \hld\ godes ègan barn,
an þat hòha hús, \hld\ þar þe hęri drank,
þea Judeon an þemu gast-sęli: \hld\ he im ók at þem gòmun was,
giak hi þar ge·kúðde, \hld\ þat hi habda kraft godes,
helpa fan himil-fader, \hld\ hèlagna gèst,
waldandes wís-dóm. \hld\ Werod blíðode,
wárun þar an luston \hld\ liudi at-samne,
gumon glad-módje. \hld\ Gengun ambaht-man,
skęnkjon mid skálun, \hld\ drógun skírjane wín
mid orkun endi mid alo-fatun; \hld\ was þar erlo dròm
fagar an flęttja, \hld\ þó þar folk undar im
an þem bęnkjon só bętst \hld\ blíðsea af·hóvun,
wárun þar an wunnjun. \hld\ Þó im þes wínes brast,
þem liudjun þes líðes: \hld\ is ni was far·lévid wiht
hwergin an þemu húse, \hld\ þat for þene hęri forð
skęnkjon drógin, \hld\ ak þiu skapu wárun
líðes a·lárid. \hld\ Þó ni was lang te þiu,
þat it sán ant·funda \hld\ frío skónjosta,
Kristes móder: \hld\ geng wið iro kind sprekan,
wið iro sunu selvon, \hld\ sagda im mid wordun,
þat þea werdos þó mér \hld\ wínes ne habdun
þem gęstjun te gòmun. \hld\ Siu þó gerno bad,
þat is þe hélogo Krist \hld\ helpa ge·riedi
þemu werode te willjon. \hld\ Þó habda eft is word garu
mahtig barn godes \hld\ endi wið is móder sprak:
„hwat ist mi endi þi“, \hld\ kwað he, „umbi þesoro manno lið,
umbi þeses werodes wín? \hld\ Te hwí sprikis þu þes, wíf, só filu,
manos mi far þesoro męnigi? \hld\ Ne sint mína noh
tídi kumana.“ \hld\ Þan þoh gi·trúoda siu wel
an iro hugi-skęftjun, \hld\ hèlag þiorne,
þat is aftar þem wordun \hld\ waldandes barn,
hèljandoro bętst \hld\ helpan weldi.
Hét þó þea ambaht-man \hld\ idiso skónjost,
skęnkjon endi skap-wardos, \hld\ þea þar skoldun þero skolu þionon,
þat sie þes ne word ne werk \hld\ wiht ne far·létin,
þes sie þe hélogo Krist \hld\ hètan weldi
léstjan far þem liudjun. \hld\ Lárja stódun þar
stèn-fatu sehsi. \hld\ Þó só stillo ge·bód
mahtig barn godes, \hld\ só it þar manno filu
ne wissa te wárun, \hld\ hwó he it mid is wordu ge·sprak;
he hét þea skęnkjon \hld\ þó skírjas watares
þiu fatu fulljen, \hld\ endi hi þar mid is fingrun þó,
segnade selvo \hld\ sínun handun,
warhte it te wíne \hld\ endi hét is an èn wégi hlaðen,
skęppjen mid ènoro skálon, \hld\ endi þó te þem skęnkjon sprak,
hét is þero gęstjo, \hld\ þe at þem gòmun was
þemu héroston \hld\ an hand gevan,
ful mid folmun, \hld\ þemu þe þes folkes þar
ge·weld aftar þemu werde. \hld\ Reht só hi þes wínes ge·drank,
só ni mahte he be·míðan, \hld\ ne hi far þeru męnigi sprak
te þemu brúdi-gumon, \hld\ kwað þat simbla þat bętste líð
alloro erlo ge·hwi-lik \hld\ èrist skoldi
gevan at is gòmun: \hld\ „undar þiu wirðid þero gumono hugi
a-wękid mid wínu, \hld\ þat sie wel blíðod,
drunkan dròmjad. \hld\ Þan mag man þar dragan aftar þiu
líht-líkora líð: \hld\ só ist þesoro liudjo þau.
Þan havas þu nu wunder-líko \hld\ werd-skępi þínan
ge·markod far þesoro męnigi: \hld\ hétis far þit manno folk
alles þínes wínes \hld\ þat wirsiste
þíne ambaht-man \hld\ èrist brengjan,
gevan at þínun gòmun. \hld\ Nu sint þína gęsti sade,
sint þíne druhtingos \hld\ drunkane swíðo,
is þit folk fró-mód: \hld\ nu hétis þu hír forð dragan
alloro líðo lof-samost, \hld\ þero þe ik eo an þesumu liohte gesah
hwergin hębbjan. \hld\ Mid þius skoldis þu ús hindag ér
gevon endi gòmjan: \hld\ þan it alloro gumono ge·hwi-lik
ge·þigedi te þanke.“ \hld\ Þó warð þar þegạn manag
ge·war aftar þem wordun, \hld\ síðor sie þes wínes ge·drunkun,
þat þar þe hélogo Krist \hld\ an þemu húse innan
tèkạn warhte: \hld\ trúodun sie síðor
þiu mér an is mund-burd, \hld\ þat hi habdi maht godes,
ge·wald an þesoro wer-oldi. \hld\ Þó warð þat só wído kúð
ovar Galileo land \hld\ Judeo liudjun,
hwó þar selvo ge·deda \hld\ sunu drohtines
water te wíne: \hld\ þat warð þar wundro èrist,
þero þe hi þar an Galilea \hld\ Judeo liudjon,
tèkno ge·tògdi. \hld\ Ne mag þat ge·tęlljan man,
ge·sęggjan te sǫ́ðan, \hld\ hwat þar síðor warð
wundres undar þemu werode, \hld\ þar waldand Krist
an godes namon \hld\ Judeo liudjon
allan langan dag \hld\ lèra sagde,
gi·hét im heven-ríki \hld\ endi hęlljo ge·þwing
weride mid wordun, \hld\ hét sie wara godes,
sin-líf sókjan: \hld\ þar is seolono lioht,
dròm drohtines \hld\ endi dag-skímon,
gód-lík-nissja godes; \hld\ þar gèst manag
wunod an willjan, \hld\ þe hír wel þęnkid,
þat he hír bi·halde \hld\ heven-kuninges ge·bod.
Ge·wèt imu þó mid is jungoron \hld\ fan þem gòmun forð
Kristus te Kapharnaum, \hld\ kuningo ríkjost,
te þeru márjon burg. \hld\ Megin samnode,
gumon imu te·gegnes, \hld\ gódoro manno
sálig ge·síði: \hld\ weldun þiu is swótjan word
hèlag hòrjen. \hld\ Þar im èn hunno kwam,
èn gód man an·gegin \hld\ endi ina gerno bad
helpan hèlagne, \hld\ kwað þat hi undar is híwiskja
ènna lefna lamon \hld\ lango habdi,
seokan an is selðon: \hld\ „só ina ènig seggjo ne mag
handun ge·héljen. \hld\ Nu is im þínoro helpono þarf,
fró mín þe gódo.“ \hld\ Þó sprak im eft þat friðu-barn godes
sán aftar þiu \hld\ selvo te·gegnes,
kwað þat he þar kwámi \hld\ endi þat kind weldi
nęrjan af þeru nòdi. \hld\ Þó im náhor geng
þe man far þeru męnigi \hld\ wið só mahtigna
wordun wehslan: \hld\ „ik þes wirðig ne bium,“ kwað he,
„hérro þe gódo, \hld\ þat þu an mín hús kumes,
sókjas mína seliða, \hld\ hwand ik bium só sundig man
mid wordun endi mid werkun. \hld\ Ik ge·lòvju þat þu ge·wald havas,
þat þu ina hinana maht \hld\ hélan ge·wirkjan,
waldand fró mín: \hld\ ef þu it mid þínun wordun ge·sprikis,
þan is sán þiu léf-héd lòsot \hld\ endi wirðid is lík-hamo
hél endi hrèni, \hld\ ef þu im þína helpa far·givis.
Ik bium mi ambaht-man, \hld\ hębbju mi ódes ge·nóg,
welono ge·wunnen: \hld\ þoh ik undar ge·weldi sí
aðal-kuninges, \hld\ þoh hębbju ik erlo ge·tróst,
holde hęri-rinkos, \hld\ þea mi só ge·hòriga sint,
þat sie þes ne word ne werk \hld\ wiht ne far·látad,
þes ik sie an þesumu land-skępje \hld\ léstjan héte,
ak sie farad endi frummjad \hld\ endi eft te iro fròhan kumad,
holde te iro hérron. \hld\ Þoh ik at mínumu hús égi
wíd-brèdene welon \hld\ endi werodes ge·nóg,
hęliðos hugi-dervje, \hld\ þoh ni gi·dar ik þi só hèlagna
biddjen, barn godes, \hld\ þat þu an mín bú gangas,
sókjas mína seliða, \hld\ hwand ik só sundig bium,
wèt mína far·wurhti.“ \hld\ Þó sprak eft waldand Krist,
þe gumo wið is jungoron, \hld\ kwað þat hi an Judeon hwergin
undar Israheles \hld\ avoron ne fundi
ge·makon þes mannes, \hld\ þe io mér te gode
an þemu land-skępi \hld\ ge·lòvon habdi,
þan hluttron te himile: \hld\ „nu látu ik iu þar hòrjen tó,
þar ik it iu te wárun hír \hld\ wordun seggjo,
þat noh skulun ęli-þeoda \hld\ óstane endi uestane,
man-kunnjes kuman \hld\ manag te·samne,
hèlag folk godes \hld\ an heven-ríki:
þea motun þar an Abrahames \hld\ endi an Isaakes só self
endi ók an Jakobes, \hld\ gódoro manno,
barmun restjen \hld\ endi bèðju ge·þologjan,
welon endi willjon \hld\ endi wonod-sam líf,
gód lioht mid gode. \hld\ Þan skal Judeono filu,
þeses ríkjas suni \hld\ beróvode werðen,
be·dèlide su·likoro diurðo, \hld\ endi skulun an dalun þiustron
an þemu alloro ferristan \hld\ ferne liggen.
Þar mag man ge·hòrjen \hld\ hęliðos kwíðjan,
þar sie iro torn manag \hld\ tandon bítad;
þar ist grist-grimmo \hld\ endi grádag fiur,
hard hęlljo ge·þwing, \hld\ hèt endi þiustri,
swart sin-nahti \hld\ sundja te lòne,
wrèðoro ge·wurhtjo, \hld\ só hwemu só þes willjon ne havad,
þat he ina a·lòsje, \hld\ ér hi þit lioht a·geve,
węndje fan þesoro wer-oldi. \hld\ Nu maht þu þi an þínan willjon forð
síðon te selðun; \hld\ þan findis þu ge·sundan at hús
mago-jungan man: \hld\ mód is imu an luston,
þat barn is ge·hélid, \hld\ só þu bédi te mi:
it wirðid al só ge·léstid, \hld\ só þu ge·lòvon havas
an þínumu hugi hardo.“ \hld\ Þó sagde heven-kuninge,
þe ambaht-man \hld\ alo-waldon gode
þank for þero þiodo, \hld\ þes he imu at su·likun þarvun halp.
Habda þo gi·árundid, \hld\ al só he welde,
sálig-líko: \hld\ gi·wèt imu an þana síð þanan,
wende an is willjan, \hld\ þar he welon éhte,
bú endi bód-los: \hld\ fand þat barn ge·sund,
kind-jungan man. \hld\ Kristes wárun þó
word ge·fullot: \hld\ hi ge·wald habda
te tògjanna tèkạn, \hld\ só þat ni mag gi·tęlljen man,
ge·ahton ovar þesoro erðu, \hld\ hwat he þurh is ènes kraft
an þesaro middil-gard \hld\ máriða ge·frumide,
wundres ge·warhte, \hld\ hwand al an is ge·weldi stád,
himil endi erðe. \hld\ Þó ge·wèt imu þe hélogo Krist
forð-wardes faren, \hld\ fremide alo-mahtig
alloro dago ge·hwi-likes, \hld\ drohtin þe gódo,
liudjo barnum leof, \hld\ lèrde mid wordun
godes willjon gumun, \hld\ habda imu jungorono filu
simbla te gi·síðun, \hld\ sálig folk godes,
manno męgin-kraft, \hld\ managoro þeodo,
hèlag hęri-skępi, \hld\ was is helpono gód,
mannun mildi. \hld\ Þó hi mid þeru męnigi kwam,
mid þiu brahtmu þat barn godes \hld\ te burg þeru hòhon,
þe nęrjendo te Naim: \hld\ þar skolde is namo werðen
mannun ge·márid. \hld\ Þó geng mahtig tó
nęrjendo Krist, \hld\ antat he gi·náhid was,
hèljandero bętst: \hld\ þó sáhun sie þar èn hréo dragan,
ènan líf-lòsan lík-hamon \hld\ þea liudi fórjen,
beran an ènaru báru \hld\ út at þera burges dore,
magu-jungan man. \hld\ Þiu móder aftar geng
an iro hugi hriwig \hld\ endi handun slóg,
karode endi kúmde \hld\ iro kindes dòð,
idis arm-skapan; \hld\ it was ira ènag barn:
siu was iru widowa, \hld\ ne habda wunnja þan mér,
bi·úten te þemu ènagun \hld\ sunje al geláten
wunnja endi willjan, \hld\ anttat ina iru wurd benam,
mári metodo-ge·skapu. \hld\ Megin folgode,
burg-liudjo ge·brak, \hld\ þar man ina an báru dróg,
jungan man te grave. \hld\ Þar warð imu þe godes sunu,
mahtig mildi \hld\ endi te þeru móder sprak,
hét þat þiu widowa \hld\ wóp far·léti,
kara aftar þemu kinde: \hld\ „þu skalt hír kraft sehan,
waldandes gi·werk: \hld\ þi skal hír willjo ge·standen,
frófra far þesumu folke: \hld\ ne þarft þu ferah karon
barnes þínes.“ \hld\ *Þuo hie ti þero báron geng
iak hie ina selvo ant·hrèn, \hld\ suno drohtines,
hèlagon handon, \hld\ endi ti þem hęliðe sprak,
hiet ina só ala-jungan \hld\ up a·standan,
a-rísan fan þeru restun. \hld\ Þie rink up asat,
þat barn an þero bárun: \hld\ warð im eft an is briost kuman
þie gèst þuru godes kraft, \hld\ endi hie te·gegnes sprak,
þe man wið is mágos. \hld\ Þuo ina eft þero muoder bi·falah
hélandi Krist an hand: \hld\ hugi warð iro te frovra,
þes wíves an wunnjon, \hld\ hwand iro þar su·lik willjo gi·stuod.
Fell siu þó te fuotun Kristes \hld\ endi þena folko drohtin
lovoda for þero liudjo męnigi, \hld\ hwand hie iro at só liobes ferahe
mundoda wiðer metodi-gi·skęftje: \hld\ far·stuod siu þat hie was þie mahtigo drohtin,
þie hèlago, þie himiles gi·waldid, \hld\ endi þat hie mahti gi·helpan managon,
allon irmin-þiedon. \hld\ Þuo bi·gunnun þat ahton managa,
þat wunder, þat under þem weroda gi·burida, \hld\ kwáðun þat waldand selvo,
mahtig kwámi þarod is męnigi wíson, \hld\ endi þat hie im só márjan sandi
wár-sagon an þero wer-oldes ríki, \hld\ þie im þar su·likan willjon frumidi.
warð þar þuo erl manag \hld\ egison bi·fangan,
þat folk warð an forohton: \hld\ gi·sáhun þena is ferah ègan,
dages lioht sehan, \hld\ þena þe ér dòð for·nam,
an suht-będdjon swalt: \hld\ þuo was im eft gi·sund after þiu,
kind-jung a·kwikot. \hld\ Þuo warð þat kúð obar all
avaron Israheles. \hld\ Reht só þuo ávand kwam,
só warð þar all gi·samnod \hld\ seokora manno,
haltaro endi hávaro, \hld\ só hwat só þar hwergin was,
þia lévun under þem liudjon, \hld\ endi wurðun þar gi·lèdit tuo,
kumana te Kriste, \hld\ þar hie im þuru is kraft mikil
halp endi sie hélda, \hld\ endi liet sia eft gi·haldana þanan
wendan an iro willjon. \hld\ Be·þiu skal man is werk lovon,
diuran is dádi, \hld\ hwand hie is drohtin self,
mahtig mund-boro \hld\ manno kunnje,
liudjo só hwi-likon, \hld\ só þar gi·lóbit tuo
an is word endi an is werk. \hld\ Þuo was þar werodes só filo
allaro ęli-þiodo \hld\ kuman te þem éron Kristes,
te só mahtiges mund-burd. \hld\ Þuo welda hie þar èna męri líðan,
þie godes suno mid is jungron \hld\ anevan Galilea-land,
waldand ènna wágo stròm. \hld\ Þuo hiet hie þat werod ǫ́ðar
forð-werdes faran, \hld\ endi hie gi·wèt im fahora sum
an ènna nakon innan, \hld\ nęrjendi Krist,
slápan síð-wórig. \hld\ Segel up dádun
weder-wísa weros, \hld\ lietun wind after
manon ovar þena męri-stròm, \hld\ unþat hie te middjan kwam,
waldand mid is werodu. \hld\ Þuo bi·gan þes wedares kraft,
úst up stígan, \hld\ úðjun wahsan;
swang gi·swerk an gi·mang: \hld\ þie sèu warð an hruoru,
wan wind endi water; \hld\ weros sorogodun,
þiu męri warð só muodag, \hld\ ni wánda þero manno nig·èn
lęngron líves. \hld\ Þuo sia landes ward
wękidun mid iro wordon \hld\ endi sagdun im þes wedares kraft,
bádun þat im gi·náðig \hld\ nęrjendi Krist
wurði wið þem watare: \hld\ „efþa wi skulun hier te wunder-kwálu
sweltan an þeson sèwe.“ \hld\ Self up a·rés
þie guodo godes suno \hld\ endi te is jungron sprak,
hiet þat sia im wedares gi·win \hld\ wiht ni and-rédin:
„te hwí sind gi só forhta?“ \hld\ kwat-hie. „Nis iu noh fast hugi,
gi·lòvo is iu te luttil. \hld\ Nis nu lang te þiu,
þat þia stròmos skulun \hld\ stilrun werðan
gi þit *wedar wun-sam.“ \hld\ Þo hi te þem winde sprak
ge te þemu sèwa só self \hld\ endi sie smultro hét
bèðja ge·bárjan. \hld\ Sie gi·bod léstun,
waldandes word: \hld\ weder stillodun,
fagar warð an flóde. \hld\ Þó bi·gan þat folk undar im,
werod wundrajan, \hld\ endi suma mid iro wordun sprákun,
hwi-lik þat só mahtigoro \hld\ manno wári,
þat imu só þe wind endi þe wág \hld\ wordu hòrdin,
bèðja is gi·bod-skępjes. \hld\ Þó habda sie þat barn godes
ginerid fan þeru nòdi: \hld\ þe nako furðor skreid,
hòh hurnid-skip; \hld\ hęliðos kwámun,
liudi te lande, \hld\ sagdun lof gode,
máridun is męgin-kraft. \hld\ Kwam þar manno filu
an·gegin þemu godes sunje; \hld\ he sie gerno ant·feng,
só hwene só þar mid hluttru hugi \hld\ helpa sóhte;
lèrde sie iro gi·lòvon \hld\ endi iro lík-hamon
handun hélde: \hld\ nio þe man só hardo ni was
gi·sèrit mid suhtjun: \hld\ þoh ina Satanases
féknea jungoron \hld\ fíundes kraftu
habdin undar handun \hld\ endi is hugi-skęfti,
gi·wit a·wardid, \hld\ þat he wódjendi
fóri undar þemu folke, \hld\ þoh im simbla ferh far·gaf
hélandjo Krist, \hld\ ef he te is handun kwam,
dréf þea diuvlas þanan \hld\ drohtines kraftu,
wárun wordun, \hld\ endi im is ge·wit far·gaf,
lét ina þan hélan \hld\ wiðer hettjandun,
gaf im wið þie fíund friðu, \hld\ endi im forð gi·wèt
an só hwi-lik þero lando, \hld\ só im þan leovost was.
Só deda þe drohtines sunu \hld\ dago ge·hwi-likes
gód werk mid is jungeron, \hld\ só neo Judeon umbi þat
an þea is mikilun kraft \hld\ þiu mér ne ge·lòvdun,
þat he alo-waldo \hld\ alles wári,
landes endi liudjo: \hld\ þes sie noh lòn nimat,
wídana wrak-síð, \hld\ þes sie þar þat ge·win drivun
wið selvan þene sunu drohtines. \hld\ Þó he im mid is ge·síðon gi·wèt
eft an Galilaeo land, \hld\ godes ègan barn,
fór im te þem friundun, \hld\ þar he a·fódid was
endi al undar is kunnje \hld\ kind-jung a·wóhs,
þe hèlago hèljand. \hld\ Umbi ina hęri-skępi,
þeoda þrungun; \hld\ þar was þegạn manag
só sálig undar þem ge·síðe. \hld\ Þar drógun ènna seokan man
erlos an iro armun: \hld\ weldun ina for ògun Kristes,
brengjan for þat barn godes \hld\ —was im bótono þarf,
þat ina ge·héldi \hld\ hevenes waldand,
manno mund-boro—, \hld\ þe was ér só managan dag
liðu-wastmon bi·lamod, \hld\ ni mahte is lík-hamon
wiht ge·waldan. \hld\ Þan was þar werodes só filu,
þat sie ina fora þat barn godes \hld\ brengjan ni mahtun,
ge·þringan þurh þea þioda, \hld\ þat sie só þurftiges
sunnea ge·sagdin. \hld\ Þó gi·wèt imu an ènna sęli innan
hèljando Krist; \hld\ hwarf warð þar umbi,
męgin-þeodo ge·mang. \hld\ Þó bi·gunnun þea man spreken,
þe þene léfna lamon \hld\ lango fórdun,
bárun mid is będdju, \hld\ hwó sie ina ge·drógin fora þat barn godes,
an þat werod innan, \hld\ þar ina waldand Krist
selvo gi·sáwi. \hld\ Þó gengun þea ge·síðos tó,
hóvun ina mid iro handun \hld\ endi uppan þat hús stigun,
slitun þene sęli ovana \hld\ endi ina mid sélun létun
an þene rakud innan, \hld\ þar þe ríkjo was,
kuningo kraftigost. \hld\ Reht só he ina þó kuman gi·sah
þurh þes húses hróst, \hld\ só he þó an iro hugi far·stód,
an þero manno mód-sevon, \hld\ þat sie mikilana te imu
ge·lòvon habdun, \hld\ þó he for þen liudjun sprak,
kwað þat he þene siakon man \hld\ sundjono tómjan
látan weldi. \hld\ Þó sprákun im eft þea liudi an·gegin,
gram-harde Judeon, \hld\ þea þes godes barnes
word aftar-warodun, \hld\ kwáðun þat þat ni mahti gi·werðen só,
grim-werk far·geven, \hld\ bi·útan god èno,
waldand þesaro wer-oldes. \hld\ Þó habda eft is word garu
mahtig barn godes: \hld\ „ik gi·dón þat“, kwað he, „an þesumu manne skín,
þe hír só siak ligid \hld\ an þesumu sęli innan,
te wundron gi·wégid, \hld\ þat ik ge·wald hębbju
sundja te far·gevanne \hld\ endi ók seokan man
te ge·hèljanne, \hld\ só ik ina hrínan ni þarf.“
Manoda ina þó \hld\ þe márjo drohtin,
liggjandjan lamon, \hld\ hét ina far þem liudjun a·standan
up alo-hélan \hld\ endi hét ina an is ahslun niman,
is bed-gi·wádi te baka; \hld\ he þat gi·bod léste
sniumo for þemu gi·síðja \hld\ endi geng imu eft ge·sund þanan,
hél fan þemu húse. \hld\ Þó þes só manag héðin man,
weros wundradun, \hld\ kwáðun þat imu waldand self,
god alo-mahtig \hld\ far·gevan habdi
méron mahti \hld\ þan elkor ènigumu mannes sunje,
kraft endi kústi; \hld\ sie ni weldun ant·kęnnjan þoh,
Judeo liudi, \hld\ þat he god wári,
ne ge·lòvdun is lèran, \hld\ ak habdun im lèðan stríd,
wunnun wiðar is wordun: \hld\ þes sie werk hlutun,
lèð-lík lòn-geld, \hld\ endi só noh lango skulun,
þes sie ni weldun hòrjen \hld\ heven-kuninges,
Kristes lèrun, \hld\ þea he kúðde ovar al,
wído aftar þesaro wer-oldi, \hld\ endi lét sie is werk sehan
allaro dago ge·hwi-likes, \hld\ is dádi skawon,
hòrjen is hèlag word, \hld\ þe he te helpu ge·sprak
manno barnun, \hld\ endi só manag mahtig-lík
tèkạn ge·tògda, \hld\ þat sie gi·trúodin þiu bet,
gi·lòvdin an is lèra. \hld\ He só managan lík-hamon
balu-suhtjo ant·band \hld\ endi bóta ge·skeride,
far·gaf fégjun ferah, \hld\ þem þe fúsid was
hęlið an hęl-síð: \hld\ þan gi·deda ina þe héland self,
Krist þurh is kraft mikil \hld\ kwikan aftar dòða,
lét ina an þesaro wer-oldi forð \hld\ wunnjono neotan.
Só hélde he þea haltun man \hld\ endi þea hávon só self,
bótta, þem þar blinde wárun, \hld\ lét sie þat berhte lioht,
sin-skóni sehan, \hld\ sundja lòsda,
gumono grim-werk. \hld\ Ni was gio Judeono be·þiu,
lèðes liud-skępjes \hld\ gi·lòvo þiu bętara
an þene hèlagon Krist, \hld\ ak habdun im hardene mód,
swíðo starkan stríd, \hld\ far·standan ni weldun,
þat sie habdun for·fangan \hld\ fíundun an willjan,
liudi mid iro ge·lòvun. \hld\ Ni was gio þiu latoro be·þiu
sunu drohtines, \hld\ ak he sagde mid wordun,
hwó sie skoldin ge·halon \hld\ himiles ríki,
lèrde aftar þemu lande, \hld\ habde imu þero liudjo só filu
gi·wenid mid is wordun, \hld\ þat im werod mikil,
folk folgoda, \hld\ endi he im filu sagda,
be biliðjun þat barn godes, \hld\ þes sie ni mahtun an iro breostun far·standan,
undar-huggjan an iro herton, \hld\ ér it im þe hèlago Krist
ovar þat erlo folk \hld\ oponun wordun
þurh is selves kraft \hld\ sęggjan welda,
márjan hwat he mènde. \hld\ Þar ina męgin umbi,
þioda þrungun: \hld\ was im þarf mikil
te gi·hòrjenne \hld\ heven-kuninges
wár-fastun word. \hld\ He stód imu þó bi ènes watares staðe,
ni welde þó bi þemu ge·þringe \hld\ ovar þat þegno folk
an þemu lande uppan \hld\ þea lèra kúðjan,
ak geng imu þó þe gódo \hld\ endi is jungaron mid imu,
friðu-barn godes, \hld\ þemu flóde náhor
an èn skip innan, \hld\ endi it skalden hét
lande rúmur, \hld\ þat ina þea liudi só filu,
þioda ni þrungi. \hld\ Stód þegạn manag,
werod bi þemu watare, \hld\ þar waldand Krist
ovar þat liudjo folk \hld\ lèra sagde:
„hwat, ik iu sęggjan mag“, \hld\ kwað he, „ge·síðos míne,
hwó imu èn erl bi·gan \hld\ an erðu sájan
hrèn-korni mid is handun. \hld\ Sum it an hardan stèn
ovan-wardan fel, \hld\ erðon ni habda,
þat it þar mahti wahsan \hld\ efþa wurtjo gi·fáhan,
kínan efþa bi·klíven, \hld\ ak warð þat korn far·loren,
þat þar an þeru léian gi·lag. \hld\ Sum it eft an land bi·fel,
an erðun aðal-kunnjes: \hld\ bi·gan imu aftar þiu
wahsen wán-líko \hld\ endi wurtjo fáhan,
lód an lustun: \hld\ was þat land só gód,
fránisko gi·fehod. \hld\ Sum it eft bi·fallen warð
an èna starka strátun, \hld\ þar stópon gengun,
hrosso hóf-slaga \hld\ endi hęliðo tráda;
warð imu þar an erðu \hld\ endi eft up gi·geng,
bi·gan imu an þemu wege wahsen; \hld\ þó it eft þes werodes far·nam,
þes folkes fard mikil \hld\ endi fuglos a·lásun,
þat is þemu éksan wiht \hld\ aftar ni móste
werðan te willjan, \hld\ þes þar an þene weg bi·fel.
Sum warð it þan bi·fallen, \hld\ þar só filu stódun
þikkero þorno \hld\ an þemu dage;
warð imu þar an erðu \hld\ endi eft up gi·geng,
kén imu þar endi klivode. \hld\ Þó slógun þar eft krúd an gi·mang,
weridun imu þene wastom: \hld\ habda it þes waldes hlea
forana ovar-fangan, \hld\ þat it ni mahte te ènigaro frumu werðen,
ef it þea þornos \hld\ só þringan móstun.“
Þó sátun endi swígodun \hld\ ge·síðos Kristes,
word-spáha weros: \hld\ was im wundạr mikil,
be hwi-likun biliðjun \hld\ þat barn godes
su·lik sǫ́ð-lík spel \hld\ sęggjan bi·gunni.
Þó bi·gan is þero erlo \hld\ èn frágojan
holdan hérron, \hld\ hnèg imu te·gegnes
tulgo werð-liko: \hld\ „hwat, þu ge·wald havas“, kwað he,
„ia an himile ia an erðu, \hld\ hèlag drohtin,
uppa endi niðara, \hld\ bist þu alo-waldo
gumono gèsto, \hld\ endi wi þíne jungaron sind,
an úsumu hugi holde. \hld\ Hérro þe gódo,
ef it þín willjo sí, \hld\ lát ús þínaro wordo þar
endi gi·hòrjen, \hld\ þat wi it aftar þi
ovar al Kristin-folk \hld\ kúðjan mótin.
wi witun þat þínun wordun \hld\ wár-lík biliði
forð folgojad, \hld\ endi ús is firinun þarf,
þat wi þín word endi þín werk, \hld\ —hwand it fan su·likumu ge·wittja kumid—
þat wi it an þesumu lande \hld\ at þi línon mótin.“
Þó im eft te·gegnes \hld\ gumono bętsta
and-wordi ge·sprak: \hld\ „ni mènde ik elkor wiht“, kwað he,
„te bi·dernjenne \hld\ dádjo mínaro,
wordo efþa werko; \hld\ þit skulun gi witan alle,
jungaron míne, \hld\ hwand iu far·geven havad
waldand þesaro wer-oldes, \hld\ þat gi witan mótun
an iuwom hugi-skęftjun \hld\ himilisk ge·rúni;
þem óðrun skal man be biliðjun \hld\ þat gi·bod godes
wordun wísjen. \hld\ Nu willju ik iu te wárun hier
márjen, hwat ik mènde, \hld\ þat gi mína þiu bet
ovar al þit land-skępi \hld\ lèra far·standan.
Þat sád, þat ik iu sagda, \hld\ þat is selves word,
þiu hèlaga lèra \hld\ heven-kuninges,
hwó man þea márjen skal \hld\ ovar þene middil-gard,
wído aftar þesaro wer-oldi. \hld\ Weros sind im gi·hugide,
man mis-líko: \hld\ sum su·likan mód dregid,
harda hugi-skęfti \hld\ endi hréan sevon,
þat ina ni ge·werðod, \hld\ þat he it be iuwon wordun due,
þat he þesa mína lèra forð \hld\ léstjen willje,
ak werðad þar só far·lorana \hld\ lèra mína,
godes ambusni \hld\ endi iuwaro gumono word
an þemu uvilon manne, \hld\ só ik iu ér sagda,
þat þat korn far·warð, \hld\ þat þar mid kíðun ni mahte
an þemu stène uppan \hld\ stędi-haft werðan.
Só wirðid al far·loran \hld\ eðilero spráka,
árundi godes, \hld\ só hwat só man þemu uvilon manne
wordun ge·wísid, \hld\ endi he an þea wirson hand,
undar fíundo folk \hld\ fard ge·kiusid,
an godes un·wiljan \hld\ endi an gramono hróm
endi an fiures farm. \hld\ Forð skal he hétjan
mid is breost-hugi \hld\ brèda logna.
Nio gi an þesumu lande þiu lés \hld\ lèra mína
wordun ni wísjad: \hld\ is þeses werodes só filu,
erlo aftar þesaro erðun: \hld\ bi·stéd þar ǫ́ðar man,
þe is imu jung endi glau, \hld\ —endi havad imu gódan mód—,
sprákono spáhi \hld\ endi wèt iuwaro spello gi·skéð,
hugid is þan an is herton \hld\ endi hòrid þar mid is órun tó
swíðo niud-líko \hld\ endi náhor stéd,
an is breost hledid \hld\ þat gi·bod godes,
línod endi léstid: \hld\ is is gi·lòvo só gód,
talod imu, \hld\ hwó he óðrana eft gi·hwervje
mèn-dádigan man, \hld\ þat is mód draga
hluttra trewa \hld\ te heven-kuninge.
Þan brèdid an þes breostun \hld\ þat gi·bod godes,
þie luvigo gi·lóbo, \hld\ só an þemu lande duod
þat korn mid kíðun, \hld\ þar it gi·kund havad
endi imu þiu wurð bi·hagod \hld\ endi wederes gang,
ręgin endi sunne, \hld\ þat it is reht havad.
Só duod þiu godes lèra \hld\ an þemu gódun manne
dages endi nahtes, \hld\ endi gangid imu diuval fer,
wrèða wihti \hld\ endi þe ward godes
náhor mikilu \hld\ nahtes endi dages,
anttat sie ina brengjad, \hld\ þat þar bèðju wirðid
ia þiu lèra te frumu \hld\ liudjo barnun,
þe fan is múðe kumid, \hld\ iak wirðid þe man gode;
havad só gi·wehslod \hld\ te þesaro wer-old-stundu
mid is hugi-skęftjun \hld\ himil-ríkjas gi·dèl,
welono þene mèstan: \hld\ farid imu an gi·wald godes,
tionuno tómig. \hld\ Trewa sind só góda
gumono ge·hwi-likumu, \hld\ só nis goldes hord
ge·lík su·likumu gi·lòvon. \hld\ Wesad iuwaro lèrono forð
man-kunnje mildje; \hld\ sie sind só mis-líka,
hęliðos ge·hugda: \hld\ sum havad iro hardan stríd,
wrèðan willjan, \hld\ wankolna hugi,
is imu féknes ful \hld\ endi firin-werko.
Þan bi·ginnid imu þunkjan, \hld\ þan he undar þeru þiodu stád
endi þar gi·hòrid \hld\ ovar hlust mikil
þea godes lèra, \hld\ þan þunkid imu, þat he sie gerno forð
léstjen willje; \hld\ þan bi·ginnid imu þiu lèra godes
an is hugi hafton, \hld\ anttat imu þan eft an hand kumid
feho te gi·fórja \hld\ endi fremiði skat.
Þan far·lèdjad ina \hld\ lèða wihti,
þan he imu far·fáhid \hld\ an feho-giri,
a-leskid þene gi·lóbon: \hld\ þan was imu þat luttil fruma,
þat he it gio an is hertan ge·hugda, \hld\ ef he it halden ne wili.
Þat is só þe wastom, \hld\ þe an þemu wege began,
liodan an þemu lande: \hld\ þó far·nam ina eft þero liudjo fard.
Só duot þea męgin-sundjon \hld\ an þes mannes hugi
þea godes lèra, \hld\ ef he is ni gòmid wel;
elkor bi·fęlljad sia ina \hld\ ferne te boðme,
an þene hètan hęl, \hld\ þar he heven-kuninge
ni wirðid furður te frumu, \hld\ ak ina fíund skulun
wítiu gi·waragjan. \hld\ Simla gi mid wordun forð
lèrjad an þesumu lande: \hld\ *ik kan þesaro liudjo hugi,
só mis-líkan muod-sevon \hld\ manno kunnjes,
só wanda wísa \hld\ [...]
Sum havit all te þiu is muod gi·látan \hld\ endi mér sorogot,
hwó hie þat hord bi·halde, \hld\ þan hwó hie hevan-kuninges
willjon gi·wirkje. \hld\ Be·þiu þar wahsan ni mag
þat hèlaga gi·bod godes, \hld\ þoh it þar a·hafton mugi,
wurtjon bi·werpan, \hld\ hwand it þie welo þringit.
Só samo só þat krúd endi þie þorn \hld\ þat korn ant·fáhat,
werjat im þena wastom, \hld\ só duot þie welo manne:
gi·heftid is herta, \hld\ þat hie it gi·huggjan ni muot,
þie man an is muode, \hld\ þes hie mèst bi·þarf,
hwó hie þat gi·wirkje, \hld\ þan lang þie hie an þesaro wer-oldi sí,
þat hie ti éwon-dage \hld\ after muoti
hębbjan þuru is hérren þank \hld\ himiles ríki,
só ęndi-lòsan welon, \hld\ só þat ni mag ènig man
witan an þesaro wer-oldi. \hld\ Nio hie só wído ni kan
te gi·þęnkjanne, \hld\ þegạn an is muode,
þat it bi·haldan mugi \hld\ herta þes mannes,
þat hie þat ti wáron witi, \hld\ hwat waldand god havit
guodes gi·gerewid, \hld\ þat all gegin-werd stéð
manno só hwi-likon, \hld\ só ina hier minnjot wel
endi selvo te þiu \hld\ is seola gi·haldit,
þat hie an lioht godes \hld\ líðan muoti.“
Só wísda hie þuo mid wordon, \hld\ stuod werod mikil
umbi þat barn godes, \hld\ ge·hòrdun ina bi biliðon filo
umbi þesaro wer-oldes gi·wand \hld\ wordon tęlljan;
kwat þat im ók èn aðales man \hld\ an is akker sáidi
hluttar hrèn-korni \hld\ handon sínon:
wolda im þar só wun-sames \hld\ wastmes tiljan,
fagares fruhtes. \hld\ Þuo geng þar is fíond aftar
þuru dernjan hugi, \hld\ endi it all mid durðu ovar-séu,
mid weodo wirsiston. \hld\ Þuo wóhsun sia bèðju,
ge þat korn ge þat krúd. \hld\ Só kwámun gangan
is haga-stoldos te hús, \hld\ iro hérren sagdun,
þegnos iro þiodne \hld\ þrístion wordon:
„hwat, þu sáidos hluttar korn, \hld\ hérro þie guodo,
èn-fald an þínon akkar: \hld\ nu ni gi·sihit ènig erlo þan mér
weodes wahsan. \hld\ Hwí mohta þat gi·werðan só?“
Þuo sprak eft þie aðales man \hld\ þem erlon te·gegnes,
þiodan wið is þegnos, \hld\ kwat þat hie it mahti undar-þęnkjan wel,
þat im þar un·hold man \hld\ aftar sáida,
fíond fékni krúd: \hld\ „ne gionsta mi þero fruhtio wel,
a-werda mi þena wastom.“ \hld\ Þuo þar eft wini sprákun,
is jungron te·gegnes, \hld\ kwáðun þat sia þar weldin gangan tuo,
kuman mid kraftu \hld\ endi lòsjan þat krúd þanan,
halon it mid iro handon. \hld\ Þuo sprak im eft iro hérro an·gegin:
„ne welleo ik, þat gi it wiodon“, \hld\ kwat-hie, „hwand gi bi·wardon ni mugun,
gi·gòmjan an iuwon gange, \hld\ þoh gi it gerno ni duan,
ni gi þes kornes te filo, \hld\ kíðo a·werdjat,
fęlljat under iuwa fuoti. \hld\ Láte man sia forð hinan
bèðju wahsan, \hld\ und ér bewod kume
endi an þem felde sind \hld\ fruhti rípja,
aroa an þem akkare: \hld\ þan faran wi þar alla tuo,
halon it mid ússan handon \hld\ endi þat hrèn-kurni lesan
súvro te·samne \hld\ endi it an mínon sęli duojan,
hębbjan it þar gi·haldan, \hld\ þat it hwergin ni mugi
wiht a·werdjan, \hld\ endi þat wiod niman,
bindan it te burðinnjon \hld\ endi werpan it an bittar fiur,
láton it þar halojan \hld\ hèta lógna,
éld un·fuodi.“ \hld\ Þuo stuod erl manag,
þegnos þagjandi, \hld\ hwat þiod-gomo,
*mári mahtig Krist \hld\ mènjan weldi,
bóknjen mid þiu biliðju \hld\ barno ríkjost.
Bádun þó só gerno \hld\ gódan drohtin
ant·lúkan þea lèra, \hld\ þat sia móstin þea liudi forð,
hèlaga hòrjan. \hld\ Þó sprak im eft iro hérro an·gegin,
mári mahtig Krist: \hld\ „þat is“, kwað he, „mannes sunu:
ik selvo bium, þat þar sáiu, \hld\ endi sind þesa sáliga man
þat hluttra hrèn-korni, \hld\ þea mi hér hòrjad wel,
wirkjad mínan willjan; \hld\ þius wer-old is þe akkar,
þit brèda bú-land \hld\ barno man-kunnjes;
Satanas selvo is, \hld\ þat þar sáid aftar
só lèð-líka lèra: \hld\ havad þesaro liudjo só filu,
werodes a·wardid, \hld\ þat sie wam frummjad,
wirkjad aftar is willjon; \hld\ þoh skulun sie hér wahsen forð,
þea for·griponon gumon, \hld\ só samo só þea gódun man,
anttat Múdspelles męgin \hld\ ovar man fęrid,
endi þesaro wer-oldes. \hld\ Þan is allaro akkaro ge·hwi-lik
ge·rípod an þesumu ríkja: \hld\ skulun iro regan-gi·skapu
frummjen firiho barn. \hld\ Þan te·farid erða:
þat is allaro bewo brèdost; \hld\ þan kumid þe berhto drohtin
ovana mid is ęngilo kraftu, \hld\ endi kumad alle te·samne
liudi, þe io þit lioht gi·sáun, \hld\ endi skulun þan lòn ant·fáhan
uviles endi gódes. \hld\ Þan gangad ęngilos godes,
hèlage heven-wardos, \hld\ endi lesat þea hluttron man
sundor te·samne, \hld\ endi duat sie an sin-skóni,
hòh himiles lioht, \hld\ endi þea óðra an hęllja grund,
werpad þea far·warhton \hld\ an wallandi fiur;
þar skulun sie gi·bundene \hld\ bittra logna,
þrá-werk þolon, \hld\ endi þea óðra þiod-welon
an heven-ríkja, \hld\ hwítaro sunnon
liohtjan ge·líko. \hld\ Su-lik lòn nimad
weros wal-dádjo. \hld\ Só hwe só gi·wit égi,
ge·hugdi an is hertan, \hld\ etþa gi·hòrjen mugi,
erl mid is órun, \hld\ só láta imu þit an innan sorga,
an is mód-sevon, \hld\ hwó he skal an þemu márjon dage
wið þene ríkjon god \hld\ an reðju standen
wordo endi werko allaro, \hld\ þe he an þesaro wer-oldi gi·duod.
Þat is egis-líkost \hld\ allaro þingo,
forht-líkost firiho barnun, \hld\ þat sie skulun wið iro fráhon mahljen,
gumon wið þene gódan drohtin: \hld\ þan weldi gerno ge·hwe wesan,
allaro manno ge·hwi-lik \hld\ mènes tómig,
slíðero sakono. \hld\ Aftar þiu skal sorgon ér
allaro liudjo ge·hwi-lik, \hld\ ér he þit lioht afgeve,
þe þan ègan wili \hld\ alungan tír,
hòh heven-ríki \hld\ endi huldi godes.“
Só gi·fragn ik þat þó selvo \hld\ sunu drohtines,
allaro barno bętst \hld\ biliðjo sagda,
hwi-lik þero wári \hld\ an wer-old-ríkja
undar hęlið-kunnje \hld\ himil-ríkje ge·lík;
kwað þat oft luttiles hwat \hld\ liohtora wurði,
só hòho af·huovi, \hld\ „so duot himil-ríki:
þat is simla méra, \hld\ þan is man ènig
wánje an þesaro wer-oldi. \hld\ Ók is imu þat werk ge·lík,
þat man an sèo innan \hld\ segina wirpit,
fisk-nęt an flód \hld\ endi fáhit bèðju,
uvile endi góde, \hld\ tiuhid up te staðe,
liðod sie te lande, \hld\ lisit aftar þiu
þea gódun an greote \hld\ endi látid þea óðra eft an grund faran,
an wídan wág. \hld\ Só duod waldand god
an þemu márjon dage \hld\ męnniskono barn:
brengid irmin-þiod, \hld\ alle te·samne,
lisit imu þan þea hluttron \hld\ an heven-ríki,
látid þea far·griponon \hld\ an grund faren
hęllje fiures. \hld\ Ni wèt hęliðo man
þes wítjes wiðar-lága, \hld\ þes þar weros þiggjat,
an þemu Inferne \hld\ irmin-þioda.
Þan hald ni mag þera médan man \hld\ gi·makon fíðen,
ni þes welon ni þes willjon, \hld\ þes þar waldand skerid,
gildid god selvo \hld\ gumono só hwi-likumu,
só ina hér gi·haldid, \hld\ þat he an heven-ríki,
an þat lang-same lioht \hld\ líðan móti.“
Só lèrda he þó mid listjun. \hld\ Þan fórun þar þea liudi tó
ovar al Galilaeo land \hld\ þat godes barn sehan:
dádun it bi þemu wundre, \hld\ hwanen imu mahti su·lik word kumen,
só spáh-líko gi·sprokan, \hld\ þat he spel godes
gio só sǫ́ð-líko \hld\ sęggjan konsti,
só kraftig-líko gi·kweðen: \hld\ „he is þeses kunnjes hinen“, kwáðun sie,
„þe man þurh mág-skępi: \hld\ hér is is móder mid ús,
wíf undar þesumu werode. \hld\ Hwat, wi þe hér witun alle,
só kúð is ús is kuni-burd \hld\ endi is knósles ge·hwat;
a-wóhs al undar þesumu werode: \hld\ hwanen skoldi imu su·lik ge·wit kuman,
méron mahti, \hld\ þan hér óðra man égin?“
Só far·munste ina þat manno folk \hld\ endi sprákun im gi·méd-lik word,
far·hogdun ina só hèlagna, \hld\ hòrjen ni weldun
is gi·bod-skępjes. \hld\ Ni he þar ók biliðjo filu
þurh iro un·gi·lòvon \hld\ ógjan ni welde,
torhtero tèkno, \hld\ hwand he wisse iro twíflean hugi,
iro wrèðan willjan, \hld\ þat ni wárun weros óðra
só grimme under Judeon, \hld\ só wárun umbi Galilaeo land,
só hardo ge·hugide: \hld\ só þar was þe hèlago Krist,
gi·boren þat barn godes, \hld\ si ni weldun is gi·bod-skępi þoh
ant·fáhan ferht-líko, \hld\ ak bi·gan þat folk undar im,
rinkos rádan, \hld\ hwó sie þene ríkjon Krist
wégdin te wundron. \hld\ Hétun þó iro werod kumen,
ge·síði te·samne: \hld\ sundja weldun
an þene godes sunu \hld\ gerno gi·tęlljen
wrèðes willjon; \hld\ ni was im is wordo niud,
spáharo spello, \hld\ ak sie bi·gunnun sprekan undar im,
hwó sie ina só kraftagne \hld\ fan ènumu klive wurpin,
ovar ènna berges wal: \hld\ weldun þat barn godes
livu bi·lòsjen. \hld\ Þó he imu mid þem liudjun samad
fró-líko fór: \hld\ ni was imu foraht hugi,
—wisse þat imu ni mahtun \hld\ męnniskono barn,
bi þeru god-kundi \hld\ Judeo liudi
ér is tídjun wiht \hld\ teonon gi·frummjen,
lèðaro gi·lésto—, \hld\ ak he imu mid þem liudjun samad
stèg uppen þene stèn-holm, \hld\ antþat sie te þeru stędi kwámun,
þar sie ine fan þemu walle niðer \hld\ werpen hugdun,
felljen te foldu, \hld\ þat he wurði is ferhes lòs,
is aldres at endje. \hld\ Þó warð þero erlo hugi,
an þemu berge uppen \hld\ bittra gi·þáhti
Juðeono te·gangen, \hld\ þat iro ènig ni habde só grimmon sevon
ni só wrèðen willjon, \hld\ þat sie mahtin þene waldandes sunu,
Krist ant·kęnnjen; \hld\ he ni was iro kúð ènigumu,
þat sie ina þó undar-wissin. \hld\ Só mahte he undar ira werode standen
endi an iro gi·mange \hld\ middjumu gangen,
faren undar iro folke. \hld\ He dede imu þene friðu selvo,
mund-burd wið þeru męnegi \hld\ endi gi·wèt imu þurh middi þanan
þes fíundo folkes, \hld\ fór imu þó, þar he welde,
an ène wóstunnje \hld\ waldandes sunu,
kuningo kraftigost: \hld\ habde þero kustes gi·wald,
hwar imu an þemu lande \hld\ leovost wári
te wesanne an þesaru wer-oldi. \hld\ Þan fór imu an weg óðran
Johannes mid is jungarun, \hld\ godes ambaht-man,
lèrde þea liudi \hld\ lang-samane rád,
hét þat sie frume fremidin, \hld\ firina far·létin,
mèn endi morð-werk. \hld\ He was þar managumu liof
gódaro gumono. \hld\ He sóhte imu þó þene Judeono kuning,
þene hęri-togon at hús, \hld\ þe hèten was
Erodes aftar is eldiron, \hld\ ovar-módig man:
búide imu be þeru brúdi, \hld\ þiu ér sínes bróðer was,
idis an éhti, \hld\ anttat he elljor skók,
wer-old weslode. \hld\ Þó imu þat wíf gi·nam
þe kuning te kwenun; \hld\ ér wárun iro kind ódan,
barn be is bróðer. \hld\ Þó bi·gan imu þea brúd lahan
Johannes þe gódo, \hld\ kwað þat it gode wári,
waldande wiðer-mód, \hld\ þat it ènig wero frumidi,
þat bróðer brúd \hld\ an is bed námi,
hębbje sie imu te híwun. \hld\ „Ef þu mi hòrjen wili,
gi·lòvjen mínun lèrun, \hld\ ni skalt þu sie leng ègan,
ak míð ire an þínumu móde: \hld\ ni hava þar su·lika minnja tó,
ni sundjo þi te swíðo.“ \hld\ Þó warð an sorgun hugi
þes wíves aftar þem wordun; \hld\ and-réd þat he þene wer-old-kuning
sprákono ge·spóni \hld\ endi spáhun wordun,
þat he sie far·léti. \hld\ Be·gan siu imu þó lèðes filu
ráden an rúnon, \hld\ endi ine rinkos hét,
un·sundigane \hld\ erlos fáhan
endi ine an ènumu karkerea \hld\ klústar-bęndjun,
liðo-kospun bi·lúkan: \hld\ be þem liudjun ne gi·dorstun
ine ferahu bi·lòsjen, \hld\ hwand sie wárun imu friund alle,
wissun ine só góden \hld\ endi gode werðen,
habdun ina for wár-sagon, \hld\ só sia wela mahtun.
Þó wurðun an þemu gér-tale \hld\ Judeo kuninges
tídi kumana, \hld\ só þar gi·tald habdun
fróde folk-weros, \hld\ þó he gi·fódid was,
an lioht kuman. \hld\ Só was þero liudjo þau,
þat þat erlo ge·hwi-lik \hld\ óvean skolde,
Judeono mid gòmun. \hld\ Þó warð þar an þene gast-sęli
męgin-kraft mikil \hld\ manno ge·samnod,
hęri-togono an þat hús, \hld\ þar iro hérro was
an is kuning-stóle. \hld\ Kwámun managa
Judeon an þene gast-sęli; \hld\ warð im þar glad-mód hugi,
blíði an iro breostun: \hld\ gi·sáhun iro bág-gevon
wesen an wunnjon. \hld\ Dróg man wín an flęt
skíri mid skálun, \hld\ skęnkjon hwurvun,
gengun mid gold-fatun: \hld\ gaman was þar inne
hlúd an þero hallu, \hld\ hęliðos drunkun.
was þes an lustun \hld\ landes hirdi,
hwat he þemu werode mèst \hld\ te wunnjun gi·fręmidi.
Hét he þó gangen forð \hld\ géla þiornun,
is bróder barn, \hld\ þar he an is bęnki sat
wínu gi·wlenkid, \hld\ endi þó te þemu wíbe sprak;
grótte sie fora þemu gum-skępje \hld\ endi gerno bad,
þat siu þar fora þem gastjun \hld\ gaman af·hóvi
fagar an flęttje: \hld\ „lát þit folk sehan,
hwó þu gelínod havas \hld\ liudjo męnegi
te blíðseanne an bęnkjun; \hld\ ef þu mi þera bede tugiðos,
mín word for þesumu werode, \hld\ þan willju ik it hér te wárun ge·kweðen,
liahto fora þesun liudjun \hld\ endi ók gi·léstjen só,
þat ik þi þan aftar þiu \hld\ éron willju,
só hwes só þu mi bidis \hld\ for þesun mínun bág-winjun:
þoh þu mi þesaro hęri-dómo \hld\ halvaro fergos,
ríkjas mínes, \hld\ þoh gi·dón ik, þat it ènig rinko ni mag
wordun gi·węndjen, \hld\ endi it skal gi·werðen só.“
Þó warð þera magað aftar þiu \hld\ mód gi·hworven,
hugi aftar iro hérron, \hld\ þat siu an þemu húse innen,
an þemu gast-sęli \hld\ gamen up a·huof,
al só þero liudjo \hld\ land-wíse gi·dróg,
þero þiodo þau. \hld\ Þiu þiorne spilode
hrór aftar þemu húse: \hld\ hugi was an lustun,
managaro mód-sevo. \hld\ Þó þiu magað habda
gi·þionod te þanke \hld\ þiod-kuninge
endi allumu þemu erl-skępje, \hld\ þe þar inne was
gódaro gumono, \hld\ siu welde þó ira geva ègan,
þiu magað for þeru męnegi: \hld\ geng þó wið iro módar sprekan
endi frágode sie \hld\ firi-wit-líko,
hwes siu þene burges ward \hld\ biddjen skoldi.
Þó wísde siu aftar iro willjon, \hld\ hét þat siu wihtes þan ér
ni gerodi for þemu gum-skępje, \hld\ bi·útan þat man iru Johannes
an þeru hallu innan \hld\ hòvid gávi
a-lòsid af is lík-hamon. \hld\ Þat was allun þem liudjun harm,
þem mannun an iro móde, \hld\ þó sie þat gi·hòrdun þea magað sprekan;
só was it ók þemu kuninge: \hld\ he ni mahte is kwidi liagan,
is word węndjen: \hld\ hét þó is wépan-berand
gangen fan þemu gast-sęli \hld\ endi hét þene godes man
lívu bi·lòsjen. \hld\ Þó ni was lang te þiu,
þat man an þea halla \hld\ hòvid bráhte
þes þiod-gumon, \hld\ endi it þar þeru þiornun far·gaf,
magað for þeru męnegi: \hld\ siu dróg it þeru móder forð.
Þó was èn-dago \hld\ allaro manno
þes wísoston, \hld\ þero þe gio an þesa wer-old kwámi,
þero þe kwene ènig \hld\ kind gi·bári,
idis fan erle, \hld\ lét man simla þen ènon bi·foran,
þe þiu þiorne gi·dróg, \hld\ þe gio þegnes ni warð
wís an iro wer-oldi, \hld\ bi·útan só ine waldand god
fan heven-wange \hld\ hèlages gèstes
gi·markode mahtig: \hld\ þe ni habde ènigan gi·makon hwergin
ér nek aftar. \hld\ Erlos hwurvun,
gumon umbi Johannen, \hld\ is jungaron managa,
sálig ge·síði, \hld\ endi ine an sande bi·gróvun,
leoves lík-hamon: \hld\ wissun þat he lioht godes,
diur-líkan dròm \hld\ mid is drohtine samad,
up-òdas hèm \hld\ ègan móste,
sálig sókjan. \hld\ Þó ge·witun im þea ge·síðos þanen,
Johannes gjungaron \hld\ giámer-móde,
hèlag-feraha: \hld\ was im iro hérron dòð
swíðo an sorgun. \hld\ Ge·witun im sókjan þó
an þeru wóstunni \hld\ waldandes sunu,
kraftigana Krist \hld\ endi imu kúð gi·dedun
gódes mannes for·gang, \hld\ hwó habde þe Judeono kuning
manno þene márjostan \hld\ mákjas ęggjun
hòvdu bi·hauwan: \hld\ he ni welde is ènigen harm spreken,
sunu drohtines; \hld\ he wisse þat þiu seole was
hèlag gi·halden \hld\ wiðer hettjandeon,
an friðe wiðer fíundun. \hld\ Þó só gi·frági warð
aftar þem land-skępjun \hld\ lèrjandero bętst
an þeru wóstunni: \hld\ werod samnode,
fór folkun tó: \hld\ was im firi-wit mikil
wísaro wordo; \hld\ imu was ók willjo só samo,
sunje drohtines, \hld\ þat he su·lik ge·síðo folk
an þat lioht godes \hld\ laðojan mósti,
wennjen mid willjon. \hld\ Waldand lèrde
allan langan dag \hld\ liudi managa,
ęli-þeodige man, \hld\ anttat an ávand ség
sunne te sedle. \hld\ Þó gengun is ge·síðos twelivi,
gumon te þemu godes barne \hld\ endi sagdun iro gódumu hérron,
mid hwi-liku arvediu þar þea erlos livdin, \hld\ kwáðun þat sie is éra bi·þorftin,
weros an þemu wóstjon lande: \hld\ „sie ni mugun sie hér mid wihti ant·hębbjen,
hęliðos bi hungres ge·þwinge. \hld\ Nu lát þu sie, hérro þe gódo,
síðon, þar sie sęliða fíðen. \hld\ Náh sind hér ge·setana burgi
managa mid męgin-þiodun: \hld\ þar fíðad sie męti te kòpe,
weros aftar þem wíkjon.“ \hld\ Þó sprak eft waldand Krist,
þioda drohtin, \hld\ kwað þat þes èniga þurufti ni wárin,
„þat sie þurh męti-lòsi \hld\ mína far·látan
leov-líka lèra. \hld\ Gevad gi þesun liudjun gi·nóg,
wennjad sie hér mid willjon.“ \hld\ Þó habde eft is word garu
Philippus fród gumo, \hld\ kwað þat þar só filu wári
manno męnigi: \hld\ „þoh wi hér te męti habdin
garu im te gevanne, \hld\ só wi mahtin far·gelden mèst,
ef wi hér gi·saldin \hld\ siluver-skatto
twè hund samad, \hld\ tweho wári is noh þan,
þat iro ènig þar \hld\ ènes gi·námi:
só luttik wári þat þesun liudjun.“ \hld\ Þó sprak eft þe landes ward
endi frágode sie \hld\ firi-wit-líko,
manno drohtin, \hld\ hwat sie þar te męti habdin
wistes ge·wunnin. \hld\ Þó sprak imu eft mid is wordun an·gegin
Andreas fora þem erlun \hld\ endi þemu alo-waldon
selvumu sagde, \hld\ þat sie an iro gi·síðje þan mér
garowes ni habdin, \hld\ „bi·útan girstin bròd
fívi an úsaru ferdi \hld\ endi fiskos twène.
Hwat mag þat þoh þesaru męnigi?“ \hld\ Þó sprak imu eft mahtig Krist,
þe gódo godes sunu, \hld\ endi hét þat gumono folk
skerjen endi skéðen \hld\ endi hét þea skola settjen,
erlos aftar þeru erðu, \hld\ irmin-þioda
an grase gruonimu, \hld\ endi þó te is jungarun sprak,
allaro barno bętst, \hld\ hét imu þiu bròd halon
endi þea fiskos forð. \hld\ Þat folk stillo béd,
sat ge·síði mikil; \hld\ undar þiu he þurh is selves kraft,
manno drohtin, \hld\ þene męti wíhide,
hèlag heven-kuning, \hld\ endi mid is handun brak,
gaf it is jungarun forð, \hld\ endi it sie undar þemu gum-skępje hét
dragan endi dèljen. \hld\ Sie léstun iro drohtines word,
is geva gerno drógun \hld\ gumono gi·hwemu,
hèlaga helpa. \hld\ It undar iro handun wóhs,
męti manno gi·hwemu: \hld\ þeru męgin-þiodu warð
líf an lustun, \hld\ þea liudi wurðun alle,
sade sálig folk, \hld\ só hwat só þar gi·samnod was
fan allun wídun wegun. \hld\ Þó hét waldand Krist
gangen is jungaron \hld\ endi hét sie gòmjen wel,
þat þiu léva þar \hld\ far·loren ni wurði;
hét sie þó samnon, \hld\ þó þar sade wárun
man-kunnjes manag. \hld\ Þar móses warð,
bròdes te lévu, \hld\ þat man birilos gi·las
twelivi fulle: \hld\ þat was tèkạn mikil,
gròt kraft godes, \hld\ hwand þar was gumono gi·tald
áno wíf endi kind, \hld\ werodes at-samme
fíf þúsundig. \hld\ Þat folk al far·stód,
þea man an iro móde, \hld\ þat sie þar mahtigna
hérron habdun. \hld\ Þó sie heven-kuning,
þea liudi lovodun, \hld\ kwáðun þat gio ni wurði an þit lioht kuman
wísaro wár-sago, \hld\ efþa þat he gi·wald mid gode
an þesaru middil-gard \hld\ méron habdi,
èn-faldaran hugi. \hld\ Alle gi·sprákun,
þat he wári wirðig \hld\ welono ge·hwi-likes,
þat he erð-ríki \hld\ ègan mósti,
wídene wer-old-stól, \hld\ „nu he su·lik ge·wit havad,
só gròte kraft mid gode.“ \hld\ Þea gumon alle gi·warð,
þat sie ine gi·hóvin \hld\ te hérosten,
gi·kurin ine te kuninge: \hld\ þat Kriste ni was
wihtes wirðig, \hld\ hwand he þit wer-old-ríki,
erðe endi up-himil \hld\ þurh is ènes kraft
selvo gi·warhte \hld\ endi síðor gi·held,
land endi liud-skępi, \hld\ —þoh þes ènigan gi·lòvon ni dedin
wrèðe wiðer-sakon— \hld\ þat al an is gi·walde stád,
kuning-ríkjo kraft \hld\ endi kèsur-dómes,
męgin-þiodo mahal. \hld\ Be·þiu ni welde he þurh þero manno spráka
hębbjan ènigan hér-dóm, \hld\ hèlag drohtin,
wer-old-kuninges namon; \hld\ ni he þó mid wordun stríd
ni af·hóf wið þat folk furður, \hld\ ak fór imu þó, þar he welde,
an èn ge·birgi uppan: \hld\ flóh þat barn godes
gélaro gelp-kwidi \hld\ endi is jungaron hét
ovar ènne sèo síðon \hld\ endi im selvo gi·bòd,
hwar sie im eft te·gegnes \hld\ gangen skoldin.
Þó te·lét þat liud-werod \hld\ aftar þemu lande allumu,
te·fór folk mikil, \hld\ síðor iro fráho gi·wèt
an þat ge·birgi uppan, \hld\ barno ríkjost,
waldand an is willjon. \hld\ Þó te þes watares staðe
samnodun þea ge·síðos Kristes, \hld\ þe he imu habde selvo gi·korane,
sie twelivi þurh iro trewa góda: \hld\ ni was im tweho nigijan,
nevu sie an þat godes þionost \hld\ gerno weldin
ovar þene sèo síðon. \hld\ Þó létun sie swíðean stròm,
hòh hurnid-skip \hld\ hluttron úðjon,
skéðan skír water. \hld\ Skréd lioht dages,
sunne warð an sedle; \hld\ þe sèo-líðandjan
naht nevulo bi·warp; \hld\ náðidun erlos
forð-wardes an flód; \hld\ warð þiu fiorðe tid
þera nahtes kuman \hld\ —nęrjendo Krist
warode þea wág-líðand—: \hld\ þó warð wind mikil,
hòh weder af·haven: \hld\ hlamodun úðjon,
stròm an stamne; \hld\ strídjun fęridun
þea weros wiðer winde, \hld\ was im wrèð hugi,
sevo sorgono ful: \hld\ selvon ni wándun
lagu-líðandja \hld\ an land kumen
þurh þes wederes ge·win. \hld\ Þó gi·sáhun sie waldand Krist
an þemu sèe uppan \hld\ selvun gangan,
faran an fáðjon: \hld\ ni mahte an þene flód innan,
an þene sèo sinkan, \hld\ hwand ine is selves kraft
hèlag ant·habde. \hld\ Hugi warð an forhtun,
þero manno mód-sevo: \hld\ and-rédun þat it im mahtig fíund
te gi·droge dádi. \hld\ Þó sprak im iro drohtin tó,
hèlag heven-kuning, \hld\ endi sagde im þat he iro hérro was
mári endi mahtig: \hld\ „nu gi módes skulun
fastes fáhen; \hld\ ne sí iu forht hugi,
gi·bárjad gi bald-líko: \hld\ ik bium þat barn godes,
is selves sunu, \hld\ þe iu wið þesumu sèe skal,
mundon wið þesan męri-stròm.“ \hld\ Þó sprak imu èn þero manno an·gegin
ovar bord skipes, \hld\ bar-wirðig gumo,
Petrus þe gódo \hld\ —ni welde píne þolon,
watares wíti—: \hld\ „ef þu it waldand sís“, kwað he,
„hérro þe gódo, \hld\ só mi an mínumu hugi þunkit,
hét mi þan þarod gangan te þi \hld\ ovar þesen gevenes stròm,
drokno ovar diap water, \hld\ ef þu mín drohtin sís,
managoro mund-boro.“ \hld\ Þó hét ine mahtig Krist
gangan imu te·gegnes. \hld\ He warð garu sáno,
stóp af þemu stamne \hld\ endi strídjun geng
forð te is frójan. \hld\ Þiu flód ant·habde
þene man þurh maht godes, \hld\ antat he imu an is móde bi·gan
and-ráden diap water, \hld\ þó he dríven gi·sah
þene wég mid windu: \hld\ wundun ina úðjon,
hòh stròm umbi·hring. \hld\ Reht só he þó an is hugi twehode,
só wèk imu þat water under, \hld\ endi he an þene wág innan,
sank an þene sèo-stròm, \hld\ endi he hriop sán aftar þiu
gáhon te þemu godes sunje \hld\ endi gerno bad,
þat he ine þó ge·neridi, \hld\ þó he an nòdjun was,
þegạn an ge·þwinge. \hld\ Þiodo drohtin
ant·feng ine mid is faðmun \hld\ endi frágode sána,
te hwí he þó ge·twehodi: \hld\ „hwat, þu mahtes ge·trúojan wel,
witen þat te wárun, \hld\ þat þi watares kraft
an þemu sèe innen \hld\ þínes síðes ni mahte,
lagu-stròm gi·lęttjen, \hld\ só lango só þu habdes ge·lòvon te mi
an þínumu hugi hardo. \hld\ Nu willju ik þi an helpun wesen,
nęrjen þi an þesaru nòdi“. \hld\ Þó nam ine alo-mahtig,
hèlag bi handun: \hld\ þó warð imu eft hlutter water
fast under fótun, \hld\ endi sie an fáði samad
bèðja gengun, \hld\ antat sie ovar bord skipes
stópun fan þemu stròme, \hld\ endi an þemu stamne gesat
allaro barno bętst. \hld\ Þó warð brèd water,
stròmos ge·stillid, \hld\ endi sie te staðe kwámun,
lagu-líðandja \hld\ an land samen
þurh þes wateres ge·win, \hld\ sagdun þo waldande þank,
diurden iro drohtin \hld\ dádjun endi wordun,
fellun imu te fótun \hld\ endi filu sprákun
wísaro wordo, \hld\ kwáðun þat sie wissin garo,
þat he wári selvo \hld\ sunu drohtines
wár an þesaru wer-oldi \hld\ endi ge·wald habdi
ovar middil-gard, \hld\ endi þat he mahti allaro manno gi·hwes
ferahe gi·formon, \hld\ al só he im an þemu flóde dede
wið þes watares ge·win. \hld\ Þó gi·wèt imu waldand Krist
síðon fan þemu sèe, \hld\ sunu drohtines,
ènag barn godes. \hld\ Eli-þioda kwam imu,
gumon te·gegnes: \hld\ wárun is gódun werk
ferran ge·frági, \hld\ þat he só filu sagde
wároro wordo: \hld\ imu was willjo mikil,
þat he su·lik folk-skępi \hld\ frummjen mósti,
þat sie simla gerno \hld\ gode þionodin,
wárin ge·hòrige \hld\ heven-kuninge
man-kunnjes manag. \hld\ Þó gi·wèt he imu over þea marka Judeono,
sóhte imu Sidono burg, \hld\ habde ge·síðos mid imu,
góde jungaron. \hld\ Þar imu te·gegnes kwam
èn idis fan áðrom þiodun; \hld\ siu was iru aðali-ge·burdjo,
kunnjes fan Kananeo lande; \hld\ siu bad þene kraftagan drohtin,
hèlagna, þat he iru helpe ge·rédi, \hld\ kwað þat iru wári harm gi·standen,
soroga at iru selvaru dohter, \hld\ kwað þat siu wári mid suhtjun bi·fangen:
„be·drogan habbjad sie dernja wihti. \hld\ Nu is iro dòd at hendi,
þea wrèðon habbjad sie ge·wittju be·numane. \hld\ Nu biddju ik þi, waldand fró min,
selvo sunu Dawides, \hld\ þat sie af su·likum suhtjun a·tómjes,
þat þu sie só arma \hld\ égroht-fullo
wam-skaðon bi·weri.“ \hld\ Ni gaf iru þó noh waldand Krist
ènig and-wordi; \hld\ siu imu aftar geng,
folgode fruokno, \hld\ antat siu te is fótun kwam,
grótte ina greatandi. \hld\ Gjungaron Kristes
bádun iro hérron, \hld\ þat he an is hugja mildi
wurði þemu wíve. \hld\ Þó habde eft is word garu
sunu drohtines \hld\ endi te is ge·síðun sprak:
„èrist skal ik Israheles \hld\ avoron werðen,
folk-skępi te frumu, \hld\ þat sie ferhtan hugi
hębbjan te iro hérron: \hld\ im is helpono þarf,
þea liudi sind far·lorane, \hld\ far·láten habbjad
waldandes word, \hld\ þat werod is ge·twíflid,
drívad im dernjan hugi, \hld\ ne willjad iro drohtine hòrjen
Israhelo erl-skępi, \hld\ un·gi·lòviga sind
hęliðos iro hérron: \hld\ þoh skal þanen helpe kumen
allun ęli-þiodun.“ \hld\ Agaléto bad
þat wíf mid iro wordun, \hld\ þat iru waldand Krist
an is mód-sevon \hld\ mildi wurði,
þat siu iro barnes forð \hld\ brúkan mósti,
hębbjan sie héle. \hld\ Þó sprak iru hérro an·gegin,
mári endi mahtig: \hld\ „nis þat“, kwað he, „mannes reht,
gumono nig·ènum \hld\ gód te gi·frummjenne
þat he is barnun \hld\ bròdes af·tíhe,
węrnje im ovar willjon, \hld\ láte sie wíti þolean,
hungar hęti-grimmen, \hld\ endi fódje is hundos mid þiu.“
„wár is þat, waldand“, \hld\ kwað siu, „þat þu mid þínun wordun sprikis,
sǫ́ð-líko sagis: \hld\ hwat, þoh oft an sęli innen
undar iro hérron diske \hld\ hwelpos hwervad
brosmono fulle \hld\ þero fan þemu biode niðer
ant·fallat iro frójan.“ \hld\ Þó gi·hòrde þat friðu-barn godes
willjan þes wíves \hld\ endi sprak iru mid is wordun tó:
„wela þat þu wíf haves \hld\ willjan góden!
Mikil is þín gi·lòvo \hld\ an þea maht godes,
an þene liudjo drohtin. \hld\ Al wirðid gi·léstid só
umbi þínes barnes líf, \hld\ só þu bádi te mi.“
Þó warð siu sán gi·hélid, \hld\ só it þe hèlago ge·sprak
wordun wár-fastun: \hld\ þat wíf fagonode,
þes siu iro barnes forð \hld\ brúkan móste;
habde iru gi·holpen \hld\ hèljando Krist,
habde sie far·fangane \hld\ fíundo kraftu,
wam-skaðun bi·werid. \hld\ Þó gi·wèt imu waldand forð,
barno þat bętste, \hld\ sóhte imu burg óðre,
þiu só þikko was \hld\ mid þeru þiodu Judeono,
mid súðar-liudjun gi·seten. \hld\ Þar gi·fragn ik þat he is ge·síðos grótte,
þe jungaron þe he imu habde be is góde gi·korane, \hld\ þat sie mid imu gerno ge·wunodun,
weros þurh is wíson spráka: \hld\ „alle skal ik iu“, kwað he, „mid wordun frágon,
jungaron míne: \hld\ hwat kweðat þese Judeo liudi,
mári męgin-þioda, \hld\ hwat ik manno sí?“
Imu and-wordidun fró-líko \hld\ is friund an·gegin,
jungaron síne: \hld\ „nis þit Judeono folk,
erlos èn-wordje: \hld\ sum sagad þat þu Elias sís,
wís wár-sago, \hld\ þe hér giu was lango,
gód undar þesumu gum-skępje, \hld\ sum sagad þat þu Johannes sís,
diur-lík drohtines bodo, \hld\ þe hér dòpte iu
werod an watere; \hld\ alle sie mid wordun sprekad,
þat þu èn-hwi-lik sís \hld\ eðilero manno,
þero wár-sagono, \hld\ þe hér mid wordun giu
lèrdun þese liudi, \hld\ endi þat þu sís eft an þit lioht kumen
te wísjanne þesumu werode.“ \hld\ Þó sprak eft waldand Krist:
„hwe kweðad gi, þat ik sí“, \hld\ kwað he, „jungaron míne,
liovon liud-weros?“ \hld\ Þó te lat ni warð
Símon Petrus: \hld\ sprak sán an·gegin
èno for im allun \hld\ —habde imu ęlljen gód,
þrístja gi·þáhti, \hld\ was is þeodone hold—:
„þu bist þe wáro \hld\ waldandes sunu,
libbjendes godes, \hld\ þe þit lioht gi·skóp,
Krist kuning èwig: \hld\ só willjad wi kweðen alle,
jungaron þíne, \hld\ þat þu sís god selvo,
hèljandero bętst.“ \hld\ Þó sprak imu eft is hérro an·gegin:
„sálig bist þu Símon“, kwað he, „sunu Jonases; \hld\ ni mahtes þu þat selvo ge·huggjan,
gi·markon an þínun mód-gi·þáhtiun, \hld\ ne it ni mahte þi mannes tunge
wordun ge·wísjen, \hld\ ak dede it þi waldand selvo,
fader allaro firiho barno, \hld\ þat þu só forð gi·spráki,
só diapo bi drohtin þínen. \hld\ Diur-líko skalt þu þes lòn ant·fáhen,
hluttro havas þu an þínan hérron gi·lòvon, \hld\ hugi-skęfti sind þíne stène ge·líka,
só fast bist þu só felis þe hardo; \hld\ hèten skulun þi firiho barn
sankte Péter: \hld\ ovar þemu stène skal man mínen sęli wirkjan,
hèlag hús godes; \hld\ þar skal is híwiski tó
sálig samnon: \hld\ ni mugun wið þem þínun swíðjun krafte
an-þebbjen hęllje portun. \hld\ Ik far·givu þi himil-ríkjas slutilas,
þat þu móst aftar mi \hld\ allun gi·waldan
kristinum folke; \hld\ kumad alle te þi
gumono gèstos; \hld\ þu have gròte gi·wald,
hwene þu hér an erðu \hld\ eldi-barno
ge·binden willjes: \hld\ þemu is bèðju gi·duan,
himil-ríki bi·loken, \hld\ endi hęllje sind imu opana,
brinnandi fiur; \hld\ só hwene só þu eft ant·binden wili,
an-þeftjen is hendi, \hld\ þemu is himil-ríki,
ant·loken liohto mèst \hld\ endi líf èwig,
gróni godes wang. \hld\ Mid su·likaru ik þi gevu willju
lònon þínen gi·lòvon. \hld\ Ni willju ik, þat gi þesun liudjun noh,
márjen þesaru męnigi, \hld\ þat ik bium mahtig Krist,
godes ègan barn. \hld\ Mi skulun Judeon noh,
un·skuldigna \hld\ erlos binden,
wégjan mi te wundrun \hld\ —dót mi wítjes filo—
innan Hjerusalem \hld\ gères ordun,
áhtjen mínes aldres \hld\ ęggjun skarpun,
bi·lòsjen mi lívu. \hld\ Ik an þesumu liohte skal
þurh úses drohtines kraft \hld\ fan dòde a·standen
an þriddjumu dage“. \hld\ Þó warð þegno bętst
swíðo an sorgun, \hld\ Símon Petrus,
warð imu hugi hriwig, \hld\ endi te is hérron sprak
rink an rúnun: \hld\ „ni skal þat ríki god“, kwað he,
„waldand willjen, \hld\ þat þu eo su·lik wíti mikil
gi·þolos undar þesaru þiod: \hld\ nis þes þarf nigijan,
hèlag drohtin.“ \hld\ Þó sprak imu eft is hérro an·gegin,
mári mahtig Krist \hld\ —was imu an is móde hold—:
„hwat, þu nu wiðer-ward bist“, \hld\ kwað he, „willjon mínes,
þegno bętsto! \hld\ Hwat, þu þesaro þiodo kanst
męnniskan sidu: \hld\ þu ni wèst þe maht godes,
þe ik gi·frummjen skal. \hld\ Ik mag þi filu sęggjan
wárun wordun, \hld\ þar hér undar þesumu werode standad
ge·síðos míne, \hld\ þea ni mótun swelten ér,
hwerven an hinen-fard \hld\ ér sie himiles lioht,
godes ríki sehat.“ \hld\ Kòs imu jungarono þó
sán aftar þiu \hld\ Símon Petrus,
Jakob endi Johannes, \hld\ ea gumon twène,
bèðja þea gi·bróðer, \hld\ endi imu þó uppen þene berg gi·wèt
sunder mid þem ge·síðun, \hld\ sálig barn godes,
mid þem þegnun þrim, \hld\ þiodo drohtin,
waldand þesaro wer-oldes: \hld\ welde im þar wundres filu,
tèkno tògjan, \hld\ þat sie gi·trúodin þiu bet,
þat he selvo was \hld\ sunu drohtines,
hèlag heven-kuning. \hld\ Þó sie an hòhan wall
stigun stèn endi berg, \hld\ antat sie te þeru stędi kwámun,
weros wiðer wolkan, \hld\ þar waldand Krist,
kuningo kraftigost \hld\ gi·koren habde,
þat he is god-kundi \hld\ jungarun sínun
þurh is ènes kraft \hld\ ógjan welde,
berht-lík biliði. \hld\ Þó imu þar te bedu gi·hnèg,
þó warð imu þar uppe \hld\ ǫ́ðar-líkora
wliti endi gi·wádi: \hld\ wurðun imu is wangun liohte,
blíkandi só þiu berhte sunne: \hld\ só skèn þat barn godes,
liuhte is lík-hamo: \hld\ liomon stódun
wánamo fan þemu waldandes barne; \hld\ warð is ge·wádi só hwít
só snéu te sehanne. \hld\ Þó warð þar seld-lík þing
gi·ógid aftar þiu: \hld\ Elias endi Moyses
kwámun þar te Kriste \hld\ wið só kraftagne
wordun wehsljan. \hld\ Þar warð só wun-sam spráka,
só gód word undar gumun, \hld\ þar þe godes sunu
wið þea márjan man \hld\ mahljen welde,
só blíði warð uppan þemu berge: \hld\ skèn þat berhte lioht,
was þar gard gód-lík \hld\ endi gróni wang,
Paradise ge·lík. \hld\ Petrus þó gi·mahalde,
hęlið hard-módig \hld\ endi te is hérron sprak,
grótte þene godes sunu: \hld\ „gód is it hér te wesanne,
ef þu it gi·kiosan wili, \hld\ Krist alo-waldo,
þat man þi hér an þesaru hòhe \hld\ èn hús ge·wirkja,
már-líko ge·mako \hld\ endi Moysese óðer
endi Eliase þriddja: \hld\ þit is ódas hèm,
welono wun-samost.“ \hld\ Reht só he þó þat word ge·sprak,
só ti-lét þiu luft an twè: \hld\ lioht wolkan skèn,
glítandi glímo, \hld\ endi þea gódun man
wliti-skóni be·warp. \hld\ Þó fan þemu wolkne kwam
hèlag stemne godes, \hld\ endi þem hęliðun þar
selvo sagde, \hld\ þat þat is sunu wári,
libbjendero liovost: \hld\ „an þemu mi líkod wel
an mínun hugi-skęftjun. \hld\ Þemu gi hòrjen skulun,
ful-gangad imu gerno.“ \hld\ Þó ni mahtun þea jungaron Kristes
þes wolknes wliti \hld\ endi word godes,
þea is mikilon maht \hld\ þea man ant·standen,
ak sie bi·fellun þó forð-wardes: \hld\ ferhes ni wándun,
lengiron líves. \hld\ Þó geng im tó þe landes ward,
be·hrèn sie mid is handun \hld\ hèljandero bętst,
hét þat sie im ni and-rédin: \hld\ „ni skal iu hér derjen eo·wiht,
þes gi hér seld-líkes \hld\ gi·sehen habbjad,
mérjaro þingo.“ \hld\ Þó eft þem mannun warð
hugi at iro herton \hld\ endi gi·hélid mód,
gi·bade an iro breostun: \hld\ gi·sáhun þat barn godes
ènna standen, \hld\ was þat oðer þó,
be·hliden himiles lioht. \hld\ Þó gi·wèt imu þe hèlago Krist
fan þemu berge niðer; \hld\ gi·bòd aftar þiu
jungarun sínun, \hld\ þat sie ovar Judeono folk
ni sagdin þea gi·sioni: \hld\ „er þan ik selvo hér
swíðo diur-líko \hld\ fan dòðe a·stande,
a-ríse fan þeru restu: \hld\ síðor mugun gi it rekkjen forð,
márjen ovar middil-gard \hld\ managun þiodun
wído aftar þesaru wer-oldi.“ \hld\ Þó gi·wèt imu waldand Krist
eft an Galileo land, \hld\ sóhte is gadulingos,
mahtig is mágo hèm, \hld\ sagde þar manages hwat
berhtero biliðjo, \hld\ endi þat barn godes
þem is sáligun ge·síðun \hld\ sorg-spell ni for·hal,
ak he im open-líko \hld\ allun sagde,
þem is gódun jungarun, \hld\ hwó ine skolde þat Judeono folk
wégjan te wundrun. \hld\ Þes wurðun þar wíse man
swíðo an sorgun, \hld\ warð im sèr hugi,
hriwig umbi iro herte: \hld\ gi·hòrdun iro hérron þó,
waldandes sunu \hld\ wordun tęlljen,
hwat he undar þeru þiodu \hld\ þolojan skolde,
willjendi undar þemu werode. \hld\ Þó gi·wèt imu waldand Krist,
gumo fan Galilea, \hld\ sóhte imu Judeono burg,
kwámun im te Kafarnaum. \hld\ Þar fundun sie ènan kuninges þegạn
wlankan undar þemu werode: \hld\ kwað þat he wári gi·weldig bodo
aðal-kèsures; \hld\ he grótte aftar þiu
Símon Petrusen, \hld\ kwað þat he wári gi·sęndid þarod,
þat he þar gi·manodi \hld\ manno ge·hwi-liken
þero hòvid-skatto, \hld\ þe sie te þemu hove skoldin
tinsi gelden: \hld\ „nis þes tweho ènig
gumono ni-giènumu, \hld\ ne sie ina far·gelden sán
mèðmo kustjon, \hld\ bi·úten iuwe mèster èno
havad it far·láten. \hld\ Ni skal þat líkon wel
mínumu hérron, \hld\ só man it imu at is hove kúðid,
aðal-kèsure.“ \hld\ Þó geng aftar þiu
Símon Petrus, \hld\ welde it sęggjan þó
hérron sínumu: \hld\ he was is an is hugi iu þan, %TODO: Check sínumu.
gi·waro waldand Krist: \hld\ —imu ni mahte word ènig
bi·holen werðen, \hld\ he wisse hugi-skęfti
manno ge·hwi-likes—: \hld\ hét þó þene is márjan þegạn,
Símon Petrus \hld\ an þene sèo innen
angul werpen: \hld\ „su·liken só þu þar èrist mugis
fisk gi·fáhen“, \hld\ kwað he, „só teoh þu þene fan þemu flóde te þi,
ant·klemmi imu þea kinni: \hld\ þar maht þu undar þem kaflon nimen
guldine skattos, \hld\ þat þu far·gelden maht
þemu manne te gi·módja \hld\ mínen endi þínen
tinseo só hwi-likan, \hld\ só he ús tó sókid.“
He ni þorfte imu þó aftar þiu \hld\ ǫ́ðaru wordu
furður gi·bioden: \hld\ geng fiskari gód,
Símon Petrus, \hld\ warp an þene sèo innen
angul an úðjon \hld\ endi up gi·tóh
fisk an flóde \hld\ mid is folmun twèm,
te·klóf imu þea kinni \hld\ endi undar þem kaflun nam
guldine skattos: \hld\ dede al, só imu þe godes sunu
wordun ge·wísde. \hld\ Þar was þó waldandes
męgin-kraft gi·márid, \hld\ hwó skal allaro manno ge·hwi-lik
swíðo willjendi \hld\ is wer-old-hérron
skuldi endi skattos, \hld\ þea imu gi·skeride sind,
gerno gelden: \hld\ ni skal ine far·gúmon eo·wiht,
ni far·muni ine an is móde, \hld\ ak wese imu mildi an is hugi,
þiono imu þio-líko: \hld\ an þiu mag he þiodgodes
willjan ge·wirkjan \hld\ endi ók is wer-old-hérron
huldi habbjen. \hld\ Só lèrde þe hèlago Krist
þea is gódon jungaron: \hld\ „ef ènig gumono wið iu“, kwað he,
„sundja ge·wirkja, \hld\ þan nim þu ina sundar te þi,
þene rink an rúna \hld\ endi imu is rád saga,
wísi imu mid wordun. \hld\ Ef imu þan þes werð ne sí,
þat he þi gi·hòrje, \hld\ hala þi þar ǫ́ðara tó
gódaro gumono, \hld\ endi lah imu is grimmun werk,
sak ina sǫ́ð-wordun. \hld\ Ef imu þan is sundja aftar þiu,
lòs-werk ni lèðon, \hld\ gi·duo it óðrun liudjun kúð,
mári it þan for męnegi \hld\ endi lát manno filu
witen is far·wurhti: \hld\ óðo be·ginnad imu þan is werk tregan,
an is hugi hrewen, \hld\ þan he it gi·hòrid hęliðo filu,
ahton eldi-barn \hld\ endi imu is uvilon dád
węrjad mid wordun. \hld\ Ef he þan ók węndjen ne wili,
ak far·módat su·lika męnegi, \hld\ þan lát þu þene man faren,
hava ina þan far héðinen \hld\ endi lát ina þi an þínumu hugi lèðen,
míð is an þínumu móde, \hld\ ne sí þat imu eft mildi god,
hér heven-kuning \hld\ helpe far·líhe,
fader allaro firiho barno.“ \hld\ Þó frágode Petrus,
allaro þegno bętst \hld\ þeodan sínan:
„hwó oft skal ik þem mannun, \hld\ þe wið mi habbjad
lèð-werk gi·duan, \hld\ leovo drohtin,
skal ik im sivun síðun \hld\ iro sundja a·láten,
wrèðaro werko, \hld\ ér þan ik is èniga wréka frummje,
lèðes te lòne?“ \hld\ Þó sprak eft þe landes ward,
an·gegin þe godes sunu \hld\ gódumu þegne:
„ni sęggju ik þi fan sivunjun, \hld\ só þu selvo sprikis,
mahlis mid þínu múðu, \hld\ ik duom þi méra þar tó:
sivun síðun sivuntig \hld\ só skalt þu sundja ge·hwemu,
lèðes a·láten: \hld\ só willju ik þi te lèrun geven
wordun wár-fastun. \hld\ Nu ik þi su·lika gi·wald far·gaf,
þat þu mínes híwiskes \hld\ hérost wáris,
manages mann-kunnjes, \hld\ nu skalt þu im mildi wesen,
liudjun líði.“ \hld\ Þó þar te þemu lèrjande kwam
èn jung man an·gegin \hld\ endi frágode Jesu Krist:
„mèster þe gódo“, \hld\ kwað he, „hwat skal ik manages duan,
an þiu þe ik heven-ríki \hld\ ge·halan móti?“
Habde imu òd-welon \hld\ allen ge·wunnen,
mèðom-hord manag, \hld\ þoh he mildjan hugi
bári an is breostun. \hld\ Þó sprak imu þat barn godes:
„hwat kwiðis þu umbi gódon? \hld\ nis þat gumono ènig
bi·útan þe èno, \hld\ þe þar al ge·skóp,
wer-old endi wunnja. \hld\ Ef þu is willjan havas,
þat þu an lioht godes \hld\ líðan mótis,
þan skalt þu bi·halden \hld\ þea hèlagon lèra,
þe þar an þemu aldon \hld\ éwa ge·biudid,
þat þu man ni slah, \hld\ ni þu mènes ni sweri,
far·legar-nessi far·lát \hld\ endi luggi ge·wit-skępi,
stríd endi stulina; \hld\ ne wis þu te stark an hugi,
ne níðin ne hatul, \hld\ ni nòd-róf ni fremi;
avunst alla far·lát; \hld\ wis þínun eldirun gód,
fader endi móder, \hld\ endi þínun friundun hold,
þem náhistun gi·náðig. \hld\ Þan þu þi gi·niodon móst
himilo ríkjas, \hld\ ef þu it bi·halden wili,
ful-gangan godes lèrun.“ \hld\ Þó sprak eft þe jungo man
„al hębbju ik só gi·léstid“, \hld\ kwað he, „só þu mi lèris nu,
wordun wísis, \hld\ só ik is eo wiht ni far·lét
fan mínero kindiski.“ \hld\ Þó bi·gan ina Krist sehan
an mid is ògun: \hld\ „èn is þar noh nu“, kwað he,
„wan þero werko: \hld\ ef þu is willjon havas,
þat þu þurh-fremid \hld\ þionon mótis
hérron þínumu, \hld\ þan skalt þu þat þín hord nimen,
skalt þínan òd-welon \hld\ allan far·kòpjen,
diurje mèðmos, \hld\ endi dèljen hét
armun mannun: \hld\ þan havas þu aftar þiu
hord an himile; \hld\ kum þi þan gi·halden te mi,
folgo þi mínaro ferdi: \hld\ þan havas þu friðu síður.“
Þó wurðun Kristes word \hld\ kind-jungumu manne
swíðo an sorgun, \hld\ was imu sèr hugi,
mód umbi herte: \hld\ habde mèðmo filu,
welono ge·wunnen; \hld\ wende imu eft þanen,
was imu unóðo \hld\ innan breostun,
an is sevon swáro. \hld\ Sah imu aftar þó
Krist alo-waldo, \hld\ kwað it þó, þar he welde,
te þem is jungarun gegin-wardun, \hld\ þat wári an godes ríki
un·óði òdagumu manne \hld\ up te kumanne:
„óður mag man olvundjon, \hld\ þoh he sí un·met gròt,
þurh náðlan gat, \hld\ þoh it sí naru swíðo,
sáftur þurh-slópien, \hld\ þan mugi kuman þiu siole te himile
þes òdagan mannes, \hld\ þe hér al havad
gi·wendid an þene wer-old-skat \hld\ willjon sínen,
mód-gi·þáhti, \hld\ endi ni hugid umbi þie maht godes.“
Imu and-wordjade \hld\ ér-þungan gumo,
Símon Petrus, \hld\ endi sęggjan bad
leovan hérron: \hld\ „hwat skulun wi þes te lòne nimen“, kwað he,
„gódes te gelde, \hld\ þes wi þurh þín jungar-dóm
ègan endi ęrvi \hld\ al far·létun
hovos endi híwiski \hld\ endi þi te hérron gi·kurun,
folgodun þínaru ferdi: \hld\ hwat skal ús þes te frumu werðen,
langes te lòne?“ \hld\ liudjo drohtin
sagde im þó selvo: \hld\ „þan ik sittjen kumu“, kwað he,
„an þie mikilan maht \hld\ an þemu márjan dage,
þar ik allun skal \hld\ irmin-þiodun
dómos a·dèljen, \hld\ þan mótun gi mid iuwomu drohtine þar
selvon sittjen \hld\ endi mótun þera saka waldan:
mótun gi Israhelo \hld\ eðili-folkun
a-dèljen aftar iro dádjun: \hld\ só mótun gi þar gi·diuride wesen.
Þan sęggju ik iu te wáran: \hld\ só hwe só þat an þesaru wer-oldi gi·duot,
þat he þurh mína minnja \hld\ mágo ge·sidli
liof far·létid, \hld\ þes skal hi hér lòn niman
tehan síðun tehin-fald, \hld\ ef he it mid trewon duot,
mid hluttru hugi. \hld\ Ovar þat havad he ók himiles lioht,
open èwig líf.“ \hld\ Bi·gan imu þó aftar þiu
allaro barno bętst \hld\ èn biliði sęggjan,
kwað þat þar èn òdag man \hld\ an ér-dagun
wári undar þemu werode: \hld\ „þe habde welono ge·nóg,
sinkas gi·samnod \hld\ endi imu simlun was
garu mid goldu \hld\ endi mid godo-wębbju,
fagarun fratahun \hld\ endi imu so filu habde
gódes an is gardun \hld\ endi imu at gòmun sat
allaro dago ge·hwi-likes: \hld\ habde imu diur-lík líf,
blíðsea an is bęnkjun. \hld\ Þan was þar eft èn biddjendi man,
gi·lévod an is lík-hamon, \hld\ Lazarus was he hèten,
lag imu dago ge·hwi-likes \hld\ at þem durun foren,
þar he þene òdagan man \hld\ inne wisse
an is gęst-sęli \hld\ gòme þiggjan,
sittjen at sumble, \hld\ endi he simlun béd
gi·armod þar úte: \hld\ ni móste þar in kuman,
ne he ni mahte ge·biddjen, \hld\ þat man imu þes bròdes þarod
gi·dragan weldi, \hld\ þes þar fan þemu diske niðer
ant·fel undar iro fóti: \hld\ ni mahte imu þar ènig fruma werðen
fan þemu héroston, þe þes húses gi·weld, \hld\ bi·útan þat þar gengun is hundos tó,
likkodun is lík-wundon, \hld\ þar he liggjandi
hungar þolode; \hld\ ni kwam imu þar te helpu wiht
fan þemu ríkjon manne. \hld\ Þó gi·fragn ik þat ina is regano-gi·skapu,
þene armon man \hld\ is èn-dago
gi·manoda mahtiun swíð, \hld\ þat he manno dròm
a-geven skolde. \hld\ Godes ęngilos
ant·fengun is ferh \hld\ endi lèddun ine forð þanen,
þat sie an Abrahames barm \hld\ þes armon mannes
siole gi·settun: \hld\ þar móste he simlun forð
wesen an wunnjun. \hld\ Þó kwámun ók wurde-gi·skapu,
þemu òdagan man \hld\ or-lag-hwíle,
þat he þit lioht far·lét: \hld\ lèða wihti
be·sinkodun is siole \hld\ an þene swarton hęl,
an þat fern innen \hld\ fíundun te willjan,
be·gróvun ine an gramono hèm. \hld\ Þanen mahte he þene gódan skawon,
Abraham ge·sehen, \hld\ þar he uppe was
líves an lustun, \hld\ endi Lazarus sat
blíði an is barme, \hld\ berht lòn ant·feng
allaro is arm-ódjo, \hld\ endi lag þe òdago man
hèto an þeru hęllju, \hld\ hriop up þanen:
„fader Abraham“, \hld\ kwað he, „mi is firinun þarf,
þat þu mi an þínumu mód-sevon \hld\ mildi werðes,
líði an þesaru lognu: \hld\ sendi mi Lazarus herod,
þat he mí ge·fórja \hld\ an þit fern innan
kaldes wateres. \hld\ Ik hér kwik brinnu
hèto an þesaru hęllju: \hld\ nu is mi þínaro helpono þarf,
þat he mi a·leskje \hld\ mid is luttikon fingru
tungon míne, \hld\ nu siu tèkạn havad,
uvil arvedi. \hld\ Inwid-rádo,
lèðaro spráka, \hld\ alles is mi nu þes lòn kumen.“
Imu and-wordjade þó Abraham \hld\ —þat was ald-fader—:
„ge·hugi þu an þínumu herton“, \hld\ kwað he, „hwat þu habdes iu
welono an wer-oldi. \hld\ Hwat, þu þar alle þíne wunnja far·sliti,
gódes an gardun, \hld\ só hwat só þi giviðig forð
werðen skolde. \hld\ Wíti þolode
Lazarus an þemu liohte, \hld\ habde þar lèðes filu,
wítjas an wer-oldi. \hld\ Be·þiu skal he nu welon ègan,
libbjen an lustun: \hld\ þu skalt þea logna þolan,
brinnendi fiur: \hld\ ni mag is þi ènig bóte kumen
hinana te hęllju: \hld\ it havad þe hèlago god
só gi·fastnod mid is faðmun: \hld\ ni mag þar faren ènig
þegno þurh þat þiustri: \hld\ it is hér só þikki undar ús.“
Þó sprak eft Abrahame \hld\ þe erl te·gegnes
fan þeru hètan hęll \hld\ endi helpono bad,
þat he Lazarus \hld\ an liudjo dròm
selvon sandi: \hld\ „þat he ge·sęggja þar
bróðarun mínun, \hld\ hwó ik hér brinnendi
þrá-werk þolon; \hld\ si þar undar þeru þiodu sind,
si fívi undar þemu folke: \hld\ ik an forhtun bium,
þat sie im þar far·wirkjen, \hld\ þat sie skulin ók an þit wíti te mi,
an só grádag fiur.“ \hld\ Þó imu eft te·gegnes sprak
Abraham ald-fader, \hld\ kwað þat sie þar éo godes
an þemu land-skępi, \hld\ liudi habdin,
Moyseses gi·bòd \hld\ endi þar managaro tó
wár-saguno word: \hld\ „ef sie is willige sind,
þat sie þat bi·halden, \hld\ þan ni þurvun sie an þea hęll innen,
an þat fern faren, \hld\ ef sie ge·frummjad só,
só þea ge·biodad, \hld\ þe þea bók lesat
þem liudjun te lèrun. \hld\ Ef sie þes þan ni willjad léstjen wiht,
þanne ni hòrjad sie ók \hld\ þemu þe hinan a·stád,
man fan dòðe. \hld\ Láte man sie an iro mód-sevon
selvon keosen, \hld\ hweðer im swótjera þunkje
te gi·winnanne, \hld\ só lango só sie an þesaru wer-oldi sind,
þat sie eft uvil etþa gód \hld\ aftar habbjen.“
Só lèrde he þó þea liudi \hld\ liohton wordon,
allaro barno bętst, \hld\ endi biliði sagde
manag man-kunnje \hld\ mahtig drohtin,
kwað þat imu èn sálig gumo \hld\ samnon bi·gunni
man an morgen, \hld\ „endi im méda gi·hét,
þe hérosto þes híwiskjas, \hld\ swíðo *hold-lík lòn“,
kwat þat hie iro allaro gi·hwem \hld\ ènna gávi
silovrinna skat. \hld\ „Þuo samnodun managa
weros an is wín-gardon, \hld\ —endi hie im werk bi·falah—
ádro an úhtan. \hld\ Sum kwam þar ók an undorn tuo,
sum kwam þar an middjan dag, \hld\ man te þem werke,
sum kwam þar te nónu, \hld\ þuo was þiu niguða tíd
sumar-langes dages; \hld\ sum þar ók síðor kwam
an þia elliftun tíd. \hld\ Þuo geng þar ávand tuo,
sunna ti sedle. \hld\ Þuo hie selvo gi·bòd
is ambahtjon, \hld\ erlo drohtin,
þat man þero manno gi·hwem \hld\ is meoda for·guldi,
þem erlon arvid-lòn; \hld\ hiet þiem at èrist gevan.
þia þar at letst wárun, \hld\ liudi kumana,
weros te þem werke, \hld\ endi mid is wordon gi·bòd,
þat man þem mannon iro \hld\ mieda for·guldi
alles at aftan, \hld\ þem þar kwámun at èrist tuo
willendi te þem werke. \hld\ Wándun sia swíðo,
þat man im méra lòn \hld\ gi·makod habdi
wið iro aravedje: \hld\ þan man im allon gaf,
þem liudjon gi·líko. \hld\ Léð was þat swíðo,
allon þem ando, \hld\ þem þar kwámun at èrist tuo:
„wi kwámun hier an moragan“, \hld\ kwáðun sia, „endi þolodun hier manag te dage
aravid-werko, \hld\ hwílon un·met hét,
skínandja sunna: \hld\ nu ni givis þu ús skattes þan mér,
þie þu þem óðron duos, \hld\ þia hier èna hwíla
wáron an þínon werke.“ \hld\ Þuo habda eft is word garo
þie hérosto þes híwiskes, \hld\ kwat þat hie im ni habdi gi·hètan þan mér
werðes wið iro werke: \hld\ „hwat, ik gi·wald hębbju“, kwat-hie,
„þat ik iu allon gi·líko \hld\ muot lòn for·geldan,
iwes werkes werð.“ \hld\ Þan waldandi Krist
mènda im þoh méra þing, \hld\ þoh hie ovar þat manno folk
fan þem wín-gardon só \hld\ wordon spráki,
hwó þar un·efno \hld\ erlos kwámun,
weros te þem werke. \hld\ Só skulun fan þero wer-oldi duon
mann-kunnjes barn \hld\ an þat márjo lioht,
gumon an godes wang: \hld\ sum bi·ginnit ina giriwan sán
an is kindiski, \hld\ havit im gi·koranan muod,
willjon guodan, \hld\ wer-old-saka míðit,
far·látit is lusta; \hld\ ni mag ina is lík-hamo
an un·spuod for·spanan: \hld\ spáhiða línot,
godes éu, \hld\ gramono for·látit,
wrèðaro willjon, \hld\ duot im só te is wer-oldi forð,
léstit só an þeson liohte, \hld\ antþat im is líves kumit,
aldres ávand; \hld\ gi·wítit im þan up-wegos:
þar wirðit im is aravedi \hld\ all gi·lònot,
far·goldan mid guodu \hld\ an godes ríkje.
Þat mèndun þia wuruhtjon, \hld\ þia an þem wín-gardon
ádro an úhta \hld\ arvid-líko
werk bi·gunnun \hld\ endi þuru-wonodun forð,
erlos unt ávand. \hld\ Sum þar ók an undern kwam,
habda þuo far·merrid, \hld\ þia moragan-stunda
þes dag-werkes for·duolon; \hld\ só duot doloro filo,
gi·médaro manno: \hld\ drívit im mis-lík þing
gerno an is juguði, \hld\ —havit im gelp-kwidi
lèða gi·línot \hld\ endi lòs-word manag—,
antþat is kindiski \hld\ far·kuman wirðit,
þat ina after is juguði \hld\ godes anst manot
blíði an is brioston; \hld\ fáhit im te bęteron þan
wordon endi werkon, \hld\ lèdit im is wer-old mid þiu,
is aldar ant þena endi: \hld\ kumit im alles lòn
an godes ríkje, \hld\ gódaro werko.
Sum mann þan mid-firi \hld\ mèn far·látid,
swára sundjun, \hld\ fáhit im an sálig þing,
biginnit im þuru godes kraft \hld\ guodaro werko,
buotit balo-spráka, \hld\ látit im is bittrun dád
an is hugje hrewan; \hld\ kumit im þiu helpa fon gode,
þat im gi·léstid þie gi·lòvo, \hld\ só lango só im is líf warod;
farit im forð mid þiu, \hld\ ant·fáhit is mieda,
guod lòn at gode; \hld\ ni sindun èniga geva bęteran.
Sum biginnit þan ók furðor, \hld\ þan hie ist fruodot mér,
is aldares af·heldit, \hld\ —þan bi·ginnat im is uvilon werk
lèðon an þeson liohte, \hld\ þan ina lèra godes
gi·manod an is muode: \hld\ wirðit im mildera hugi,
þuru-gengit im mid guodu \hld\ endi geld nimit,
hòh himil-ríki, \hld\ þan hie hinan wendit,
wirðit im is mieda só sama, \hld\ só þem man *nun warð,
þea þar te nónu dages, \hld\ an þea nigunda tíd,
an þene wín-gardon \hld\ wirkjan kwámun.
Sum wirðid þan só swíðo ge·fródot, \hld\ só he ni wili is sundja bótjen,
ak he ókid sie mid uvilu ge·hwi-liku, \hld\ antat imu is ávand náhid,
is wer-old endi is wunnja far·slítid; \hld\ þan be·ginnid he imu wíti and-réden,
is sundjon werðad imu sorga an móde: \hld\ ge·hugid hwat he selvo ge·frumide
grimmes þan lango, þe he móste is juguðjo neoten; \hld\ ni mag þan mid óðru gódu gi·bótjen
þea dádi, þea he só dervja ge·frumide, \hld\ ak he slehit allaro dago ge·hwi-likes
an is breost mid bèðjun handun \hld\ endi wópit sie mid bittrun trahnun,
hlúdo he sie mid hofnu kúmid, \hld\ bidid þene hèlagon drohtin
mahtigne, þat he imu mildi werðe: \hld\ ni látid imu síðor is mód gi·twífljen;
só é-groht-ful is, þe þar alles ge·weldid: \hld\ he ni wili ènigumu irmin-manne
far·węrnjen willjan sínes; \hld\ far·givid imu waldand selvo
hèlag himil-ríki: \hld\ þan is imu gi·holpen síður.
Alle skulun sie þar éra ant·fáhen, \hld\ þoh sie þarod te ènaru tídi
ni kumen, þat kunni manno, \hld\ þoh wili imu þe kraftigo drohtin,
gi·lònon allaro liudjo só hwi-likumu, \hld\ só hér is gi·lòvon ant·fáhit:
èn himil-ríki \hld\ givid he allun þeodun,
mannun te médu. \hld\ Þat mènde mahtig Krist,
barno þat bętste, \hld\ þó he þat biliði sprak,
hwó þar te þem wín-gardun \hld\ wurhtjon kwámin,
man mis-líko: \hld\ þoh nam is méde ge·hwe
fulle te is frójan. \hld\ Só skulun firiho barn
at gode selvumu \hld\ geld ant·fáhen,
swíðo leov-lík lòn, \hld\ þoh sie sume só late werðan.
Hét imu þó þea is gódan \hld\ jungaron náhor
twelivi gangan \hld\ —þea wárun imu triuwiston
man ovar erðu—, \hld\ sagde im mahtig selvo
óðer-síðu, \hld\ hwi-lik imu þar arvedi
tóward wárun: \hld\ „þes ni mag ènig tweho werðen“, kwað he;
kwað þat sie þó te Hjerusalem \hld\ an þat Judeono folk
líðan skoldin: \hld\ „þar wirðid all gi·léstid só,
ge·frumid undar þemu folke, \hld\ só it an furn-dagun
wíse man be mi \hld\ wordun ge·sprákun.
Þar skulun mi far·kòpon \hld\ undar þea kraftigon þiod,
hęliðos te þeru héri; \hld\ þar werðat mína hendi ge·bundana,
faðmos werðad mi þar gefastnod; \hld\ filu skal ik þar gi·þolojan,
hoskes gi·hòrjen \hld\ endi harm-kwidi,
bismerspráka \hld\ endi bi·hét-word manag;
sie wégjat mi te wundron \hld\ wápnes ęggjun,
bi·lòsjad mi lívu: \hld\ ik te þesumu liohte skal
þurh drohtines kraft \hld\ fan dòðe a·standen
an þriddjon dage. \hld\ Ni kwam ik undar þesa þeoda herod
te þiu, þat mín eldi-barn \hld\ arved habdin,
þat mi þionodi þius þiod: \hld\ ni willju ik is sie þiggjen nu,
fergon þit folk-skępi, \hld\ ak ik skal imu te frumu werðen,
þeonon imu þeo-líko \hld\ endi for alla þesa þeoda geven
seole míne. \hld\ Ik willju sie selvo nu
lòsjen mid mínu lívu, \hld\ þea hér lango bidun,
man-kunnjes manag, \hld\ mínara helpa.“
Fór imu þó forð-wardes \hld\ —habde imu fasten hugi,
blíðean an is breostun \hld\ barn drohtines—
welda im te Hjerusalem \hld\ Judeo folkes
willjon wísan: \hld\ he konste þes werodes só garo
hęti-grimmen hugi \hld\ endi hardan stríd,
wrèðan willjon. \hld\ Werod síðode
furi Hjerikho-burg; \hld\ was þe godes sunu,
mahtig undar þero męnigi. \hld\ Þar sátun twènje man bi wege,
blinde wárun sie bèðje: \hld\ was im bótono þarf,
þat sie ge·héldi \hld\ hevenes waldand,
hwand sie só lango \hld\ liohtes þolodun,
managa hwíla. \hld\ Sie gi·hòrdun þó þat męgin faren
endi frágodun sán \hld\ firi-wit-líko
ręgini-blindun, \hld\ hwi-lik þar ríki man
undar þemu folk-skępi \hld\ furista wári,
hérost an hòvid. \hld\ Þó sprak im èn hęlið an·gegin,
kwað þat þar Hjesu Krist \hld\ fan Galilea-lande,
hèljandero bętst \hld\ hérost wári,
fóri mid is folku. \hld\ Þó warð fráh-mód hugi
bèðjun þem blindun mannun, \hld\ þó sie þat barn godes
wissun under þemu werode: \hld\ hreopun im þó mid iro wordun tó,
hlúdo te þemu hèlagon Kriste, \hld\ bádun þat he im helpe gerédi:
„drohtin Dawides sunu: \hld\ wis ús mid þínun dádjun mildi,
neri ús af þesaru nòdi, \hld\ só þu gi·nóge dós
manno kunnjes: \hld\ þu bist managun gód,
hilpis endi hélis.“ \hld\ Þo bi·gan im þat hęliðo folk
werjen mid wordun, \hld\ þat sie an waldand Krist
só hlúdo ni hriopin. \hld\ Si ni weldun im hòrjen te þiu,
ak sie simla mér endi mér \hld\ ovar þat manno folk
hlúdo hreopun. \hld\ Héljand ge·stód,
allaro barno bętst, \hld\ hét sie þó brengjen te imu,
lèdjen þurh þea liudi, \hld\ sprak im listjun tó
mild-líko for þeru męnegi: \hld\ „hwat willjad git mínaro hér“, kwað he,
„helpono habbjen?“ \hld\ Sie bádun ina hèlagna,
þat he im ira ògon \hld\ opana gi·dádi,
far·liwi þeses liohtes, \hld\ þat sie liudjo dròm,
swigle sunnun skín \hld\ gi·sehen móstin,
wliti-skónje wer-old. \hld\ Waldand frumide,
hrèn sie þó mid is handun, \hld\ dede is helpe þar tó,
þat þem blindun þó \hld\ bèðjum wurðun
ògon gi·oponod, \hld\ þat sie erðe endi himil
þurh kraft godes \hld\ ant·kiennjen mahtun,
lioht endi liudi. \hld\ Þó sagdun sie lof gode,
diurdun úsan drohtin, \hld\ þes sie dages liohtes
brúkan móstun: \hld\ ge·witun im bèðje mid imu,
folgodun is ferdi: \hld\ was im þiu fruma giviðig,
endi ók waldandes werk \hld\ wído ge·kúðid,
managun gi·márid. \hld\ Þar was só mahtiglík
biliði gi·bóknid, \hld\ þar þe blindon man
bi þemu wege sátun, \hld\ wíti þolodun,
liohtes lòse: \hld\ þat mènid þoh liudjo barn,
al man-kunni, \hld\ hwó sie mahtig god
an þemu ana-ginne \hld\ þurh is ènes kraft
sin-híun twè \hld\ selvo gi·warhte,
Ádam endi Éwan: \hld\ far·gaf im up-wegos,
himilo ríki; \hld\ ak þó warð im þe hatola te náh,
fíund mid féknu \hld\ endi mid firin-werkun,
bi·swèk sie mid sundjun, \hld\ þat sie sin-skóni,
lioht far·létun: \hld\ wurðun an lèðaron stędi,
an þesen middil-gard \hld\ man far·worpen,
þolodun hér an þiustrju \hld\ þiod-arvedi,
wunnun wrak-síðos, \hld\ welon þarvodun:
far·gátun godes ríkjes, \hld\ gramon þeonodun,
fíundo barnun; \hld\ sie guldun is im mid fiuru lòn
an þeru hèton hęllju. \hld\ Be·þiu wárun siu an iro hugi blinda
an þesaru middil-gard, \hld\ męnniskono barn,
hwand siu ine ni ant·kiendun, \hld\ kraftagne god,
himilisken hérron, \hld\ þene þe sie mid is handun gi·skóp,
gi·warhte an is willjon. \hld\ Þius wer-old was þó só far·hwervid,
bi·þwungen an þiustrje, \hld\ an þiod-arvidi,
an dòðes dalu: \hld\ sátun im þó bi þeru drohtines strátun
iámar-móde, \hld\ godes helpe bidun:
siu ni mahte im þó ér werðen, \hld\ ér þan waldand god
an þesan middil-gard, \hld\ mahtig drohtin,
is selves sunu \hld\ sęndjen weldi
þat he lioht ant·luki \hld\ liudjo barnun,
oponodi im èwig líf, \hld\ þat sie þene alo-waldon
mahtin ant·kęnnjen wel, \hld\ kraftagna god.
Ók mag ik giu gi·tęlljen, \hld\ of gi þar tó willjad
huggjen endi hòrjen, \hld\ þat gi þes hèljandes mugun
kraft ant·kęnnjen, \hld\ hwó is kumi wurðun
an þesaru middil-gard \hld\ managun te helpu,
ia hwat he mid þem dádjun \hld\ drohtin selvo
manages mènde, \hld\ ia be·hwiu þiu márje burg
Hjerikho hétid, \hld\ þiu þar an Judeon stád
gi·makod mid múrun: \hld\ þiu is aftar þemu mánen gi·nemnid,
aftar þemu torhten tungle: \hld\ he ni mag is tídi be·míðen,
ak he dago ge·hwi-likes \hld\ duod óðer-hweðer,
wanod ohþo wahsid. \hld\ Só dód an þesaro wer-oldi hér,
an þesaru middil-gard \hld\ męnniskono barn:
farad endi folgod, \hld\ fróde stervad,
werðad eft junga \hld\ aftar kumane,
weros a·wahsane, \hld\ unttat sie eft wurd far·nimid.
Þat mènde þat barn godes, \hld\ þó he fon þeru burgi fór,
þe gódo fan Hjerikho, \hld\ þat ni mahte ér werðen gumono barnun
þiu blindja gi·bótid, \hld\ þat sie þat berhte lioht,
gi·sáhin sin-skóni, \hld\ ér þan he selvo hér
an þesaru middil-gard \hld\ męnniski ant·feng,
flésk endi lík-hamon. \hld\ Þó wurðun þes firiho barn
gi·war an þesaru wer-oldi, \hld\ þe hér an wítje ér,
sátun an sundjun \hld\ gi·siunjes lòse,
þolodun an þiustrje, \hld\ —sie af·sóvun þat was þesaru þiod kuman
hèljand te helpu \hld\ fan heven-ríkje,
Krist allaro kuningo best; \hld\ sie mahtun is ant·kęnnjen sán,
gi·fóljen is fardio. \hld\ Þó sie só filu hriopun, %TODO: Check fardio
þe man te þemu mahtigon gode, \hld\ þat im mildi aftar þiu
waldand wurði. \hld\ Þan weridun im swíðo
þia swárun sundjon, \hld\ þe sie im ér selvon gi·dádun,
lettun sie þes gi·lóbon. \hld\ Sie ni mahtun þem liudjun þoh
bi·werjen iro willjon, \hld\ ak sie an waldand god
hlúdo hriopun, \hld\ antat he im iro héli far·gaf,
þat sie sin-líf \hld\ gi·sehen móstin,
open èwig lioht \hld\ endi an faren
an þiu berhtun bú. \hld\ Þat mèndun þea blindun man,
þe þar bi Hjerikho-burg \hld\ te þemu godes barne
hlúdo hriopun, \hld\ þat he im iro héli far·lihi,
liohtes an þesumu líve: \hld\ þan im þea liudi só filu
weridun mid wordun, \hld\ þea þar an þemu wege fórun
bi·foren endi bi·hinden: \hld\ só dót þea firin-sundjon
an þesaru middil-gard \hld\ man-kunnje.
hòrjad nu hwó þie blindun, \hld\ síður im gi·bótid warð,
þat sie sunnun lioht \hld\ ge·sehen móstun,
hwó si þó dádun: \hld\ ge·witun im mid iro drohtine samad,
folgodun is ferdi, \hld\ sprákun filu wordo
þemu landes hirdje te love: \hld\ só dód im noh liudjo barn
wído aftar þesaru wer-oldi, \hld\ síður im waldand Krist
ge·liuhte mid is lèrun \hld\ endi im líf èwig,
godes ríki far·gaf \hld\ gódun mannun,
hòh himiles lioht \hld\ endi is helpe þar tó,
só hwemu só þat gi·werkod, \hld\ þat he móti þemu is wege folgon.
Þó náhide \hld\ nęrjendo Krist,
þe gódo te Hjerusalem. \hld\ Kwam imu þar te·gegnes filu
werodes an willjon \hld\ wel huggendjes,
ant·fengun ina fagaro \hld\ endi imu bi·foren streidun
þene weg mid iro gi·wádjun \hld\ endi mid wurtjun só same,
mid berhtun blómun \hld\ endi mid bómo tógun,
þat feld mid fagaron palmun, \hld\ al só is fard ge·buride,
þat þe godes sunu \hld\ gangan welde
te þeru márjan burg. \hld\ Hwarf ina męgin umbi
liudjo an lustun, \hld\ endi lof-sang a·hóf
þat werod an willjon: \hld\ sagdun waldande þank,
þes þar selvo kwam \hld\ sunu Dawides
wíson þes werodes. \hld\ Þó gesah waldand Krist
þe gódo te Hjerusalem, \hld\ gumono bętsta,
blíkan þene burges wal \hld\ endi bú Judeono,
hòha horn-sęli \hld\ endi ók þat hús godes,
allaro wího wun-samost. \hld\ Þó wel imu an innen
hugi wið is herte: \hld\ þó ni mahte þat hèlage barn
wópu a·wísjen, \hld\ sprak þó wordo filu
hriwig-líko \hld\ —was imu is hugi sèreg—:
„wé warð þi, Hjerusalem“, \hld\ kwað he, „þes þu te wárun ni wèst
þea wurde-gi·skęfti, \hld\ þe þi noh gi·werðen skulun,
hwó þu noh wirðis be·habd \hld\ hęrjes kraftu
endi þi bi·sittjad \hld\ slíð-móde man,
fíund mid folkun. \hld\ Þan ni havas þu friðu hwergin,
mund-burd mid mannun: \hld\ lèdjad þi hér manage tó
ordos endi ęggja, \hld\ or-legas word,
far·fioþ þín folk-skępi \hld\ fiures liomon,
þese wíki a·wóstjad, \hld\ wallos hòha
fęlljad te foldun: \hld\ ni afstád is felis nigijan,
stèn ovar óðrumu, \hld\ ak werðad þesa stędi wóstja
umbi Hjerusalem \hld\ Judeo liudjo,
hwand sie ni ant·kęnnjad, \hld\ þat im kumana sind
iro tídi tó-wardes, \hld\ ak sie habbjad im twífljen hugi,
ni witun þat iro wísad \hld\ waldandes kraft.“
Gi·wèt imu þó mid þeru męnegi \hld\ manno drohtin
an þea berhton burg. \hld\ Só þó þat barn godes
innan Hjerusalem \hld\ mid þiu gumono folku,
ség mid þiu ge·síðu, \hld\ þó warð þar allaro sango mèst,
hlúd stemnje af·haven \hld\ hèlagun wordun,
lovodun þene landes ward \hld\ liudjo męnegi,
barno þat bętste; \hld\ þiu burg warð an hróru,
þat folk warð an forhtun \hld\ endi frágodun sán,
hwe þat wári, \hld\ þat þar mid þiu werodu kwam,
mid þeru mikilon męnegi. \hld\ Þó sprak im èn man an·gegin,
kwað þat þar Hjesu Krist \hld\ fan Galileo lande,
fan Nazareth-burg \hld\ nęrjand kwámi,
witig wár-sago \hld\ þemu werode te helpu.
Þó was þem Judiun, \hld\ þe imu ér grame wárun,
un·holde an hugi, \hld\ harm an móde,
þat imu þea liudi só filu \hld\ lof-sang warhtun,
diurdun iro drohtin. \hld\ Þó gengun dolmóde,
þat sie wið waldand Krist \hld\ wordun sprákun,
bádun þat he þat ge·síði \hld\ swígon héti,
letti þea liudi, \hld\ þat sie imu lof só filu
wordun ni warhtin: \hld\ „it is þesumu werode lèð“, kwáðun sie,
„þesun burg-liudjun.“ \hld\ Þó sprak eft þat barn godes:
„ef gi sie a·merrjad“, \hld\ kwað he, „þat hér ni mótin manno barn
waldandes kraft \hld\ wordun diurjen,
þan skulun it hrópen þoh \hld\ harde stènos
for þesumu folk-skępi, \hld\ felisos starka,
ér þan it eo be·líve, \hld\ nevo man is lof spreke
wído aftar þesaru wer-oldi.“ \hld\ Þó he an þene wíh innen,
geng an þat godes hús: \hld\ fand þar Judeono filu,
mis-líke man, \hld\ manage at-samne,
þea im þar kòp-stędi \hld\ gi·koran habdun,
mangodun im þar mid manages hwí: \hld\ muniterjas sátun
an þemu wíhe innan, \hld\ habdun iro wesl gi·dago
garu te gevanne. \hld\ Þat was þemu godes barne
al an andun: \hld\ dréf sie ut þanen
rúmo fan þemu rakude, \hld\ kwað þat wári rehtara dád,
þat þar te bedu fórin \hld\ barn Israheles
„endi an þesumu mínumu húse \hld\ helpono biddjan,
þat sia sigi-drohtin \hld\ sundjono tuomie,
þan hér þeovas \hld\ an þing-stędi halden,
þea far·warhton weros \hld\ wehsal drívan,
un·reht èn-fald. \hld\ Ne gi èniga éra ni witun
þeses godes húses, \hld\ Judeo liudi.“
Só rúmde he þó endi rekode, \hld\ ríki drohtin,
þat hèlaga hús \hld\ endi an helpun was
managumu man-kunnje, \hld\ þem þe is mikilon kraft
ferrene ge·frugnun \hld\ endi þar gi·faran kwámun
ovar langan weg. \hld\ Warð þar léf so manag,
halt gi·hélid \hld\ endi háf só same,
blindun gi·bótid. \hld\ Só dede þat barn godes
willjendi þemu werode, \hld\ hwand al an is gi·weldi stéd
umbi þesaro liudjo líf \hld\ endi ók umbi þit land só same.
Stód imu þó fora þemu wíhe \hld\ waldandeo Krist,
liof landes ward, \hld\ endi imu þero liudjo hugi,
iro willjon aftar-warode: \hld\ gi·sah werod mikil
an þat márje hús \hld\ mèðmos fórjen,
gevon mid goldu \hld\ endi mid godu-wębbju,
diurjun fratahun. \hld\ Þat al drohtin Krist
warode wís-líko. \hld\ Þó kwam þar ók èn widowa tó,
idis arm-skapen, \hld\ endi te þemu alaha geng
endi siu an þat tresur-hús \hld\ twène legde
éríne skattos: \hld\ was iru èn-fald hugi,
willjan gódes. \hld\ Þó sprak waldand Krist,
þe gumo wið is gjungaron, \hld\ kwað þat siu þar geva bráhti
méron mikilu þan elkor \hld\ ènig mannes sunu:
„ef hér òdaga man“, \hld\ kwað he, „éra bráhtun,
mèðom-hord manag, \hld\ sie létun im mér at hús
welona ge·wunnen. \hld\ Ni dede þius widowa só,
ak siu te þesumu alahe gaf \hld\ al þat siu habde
welono ge·wunnen, \hld\ só siu iru wiht ni far·lét
gódes an iro gardun. \hld\ Be·þiu sind ira geva méron,
waldande werða, \hld\ hwand siu it mid su·likumu willjon dede
te þesumu godes húse. \hld\ Þes skal siu geld niman,
swíðo lang-sam lòn, \hld\ þes siu su·likan gi·lòvon havad.“
Só gi·fragn ik þat þar an þemu wíhe \hld\ waldandeo Krist
allaro dago ge·hwi-likes, \hld\ drohtin manno,
wísde mid wordun. \hld\ Stód ine werod umbi,
gròt folk Judeono, \hld\ gi·hòrdun is gódan word,
swótja sęggjan. \hld\ Sum só sálig warð
manno undar þeru męnegi, \hld\ þat it bi·gan an is mód hladen;
línodun im þea lèra, \hld\ þe þe landes ward
al be biliðjun sprak, \hld\ barn drohtines.
Sumun wárun eft so lèða \hld\ lèra Kristes,
waldandes word: \hld\ was im wiðer-mód hugi
allun þem, þe an þemu hęri-skępi \hld\ hérost wárun,
furiston an þemu folke: \hld\ fáres hugdun
wrèða mid iro wordun \hld\ —habdun im wiðer-sakon
gi·haloden te helpu, \hld\ þes héroston man,
Erodeses þegạn, \hld\ þe þar and-ward stód
wrèðes willjan, \hld\ þat he iro word ovar-hòrdi—
ef sie ina for·fengin, \hld\ þat sie ina þan feteros an,
þea liudi liðo-bendi \hld\ lęggjen móstin,
sundja lòsan. \hld\ Þó gengun im þea ge·síðos tó
bittra gi·hugde, \hld\ þat sie wið þat barn godes,
wrèða wiðer-sakon \hld\ wordun sprákun:
„hwat, þu bist èo-sago“, \hld\ kwáðun sie, „allun þiodun,
wísis wáres só filu: \hld\ nis þi werð eo·wiht
te bi·míðanne \hld\ manno ni-ènumu
umbi is ríki-dóm, \hld\ nevo þu simlun þat reht sprikis
endi an þene godes weg \hld\ gumono ge·síði
lèdis mid þinun lèrun: \hld\ ni mag þi laster man
fíðan undar þesumu folke. \hld\ Nu wi þi frágon skulun.
ríki þiodan, \hld\ hwi-lik reht havad
þe kèsur fan Rúmu, \hld\ þe imu te þesumu kunnje herod
tinsi sókid \hld\ endi gi·tald havad,
hwat wi imu gelden skulin \hld\ géro ge·hwi-likes
hòvid-skatto. \hld\ Saga hwat þi þes an þínumu hugi þunkja:
is it reht þe nis? \hld\ Rád for þínun
land-mégun wel: \hld\ ús is þínaro lèrono þarf.“
Sie weldun þat he it ant·kwáði: \hld\ þan mahte he þoh ant·kęnnjen wel
iro wrèðon willjon: \hld\ „te hwí gi wár-logon“, kwað he,
„fandot mín só frókno? \hld\ Ni skal iu þat te frumu werðen,
þat gi dreogerjas \hld\ darnungo nu
willjad mi far·fáhen.“ \hld\ Hét he þó forð dragan
te skawonne þe skattos, \hld\ „þe gi skuldige sind
an þat geld geven.“ \hld\ Judeon drógun
ènna siluvrinna forð: \hld\ sáhun manage tó,
hwó he was ge·munitod: \hld\ was an middjen skín
þes kèsures biliði \hld\ —þat mahtun sie ant·kęnnjen wel—,
iro hérron hòvid-mál. \hld\ Þó frágode sie þe hèlago Krist,
aftar hwemu þiu ge·lík-nessi \hld\ gi·legid wári.
Sie kwáðun þat it wári \hld\ wer-old-kèsures
fan Rúmu-burg, \hld\ „þes þe alles þeses ríkes havad
ge·wald an þesaru wer-oldi.“ \hld\ „Þan willju ik iu te wárun hér“, kwað he,
„selvo sęggjan, \hld\ þat gi imu sín gevad,
wer-old-hérron is ge·wunst, \hld\ endi waldand gode
sęlljad, þat þar sín ist: \hld\ þat skulun iuwa seolon wesen,
gumono gèstos.“ \hld\ Þó warð þero Judeono hugi
ge·minsod an þemu mahle: \hld\ ni mahtun þe mèn-skaðon
wordun ge·winnen, \hld\ só iro willjo geng,
þat sie ina far·fengin, \hld\ hwand imu þat friðu-barn godes
wardode wið þe wrèðon \hld\ endi im wár an·gegin,
sǫ́ð-spel sagde, \hld\ þoh sie ni wárin só sálige te þiu,
þat sie it só far·fengin, \hld\ só it iro fruma wári.
Sie ni weldun it þoh far·láten, \hld\ ak hétun þar lèdjen forð
èn wíf for þemu werode, \hld\ þiu habde wam ge·frumid,
un·reht èn-fald: \hld\ þiu idis was bi·fangen
an far·legar-nessi, \hld\ was iro líves skolo,
þat sie firiho barn \hld\ ferahu bi·námin,
éhtin iro aldres: \hld\ só was an iro éu ge·skriven.
Sie bi·gunnun ina þó frágon, \hld\ fruokne liudi,
wrèða mid iro wordun, \hld\ hwat sie skoldin þemu wíve duan,
hweðer sie sie kwęlidin, \hld\ þe sie sie kwika létin,
þe hwat he umbi su·lika dádi \hld\ a·dèljen weldi:
„þu wèst, hwó þesaru męnegi“, \hld\ kwáðun sie, „Moyses gi·bòd
wárun wordun, \hld\ þat allaro wívo ge·hwi-lik
an far·legar-nessi \hld\ líves far·warhti
endi þat sie þan a·wurpin \hld\ weros mid handun,
starkun stènun: \hld\ nu maht þu sie sehan standen hér
an sundjun bi·fangan: \hld\ saga hwat þu is willjes.“
weldun ine þea wiðer-sakon \hld\ wordun far·fáhen,
ef he þat gi·kwáði, \hld\ þat sie sie kwika létin,
friðodi ira ferahe, \hld\ þan weldi þat folk Judeono
kweðen, þat he iro aldiron \hld\ éo wiðer-sagdi,
þero liudjo land-reht; \hld\ ef he sie þan héti lívu bi·nimen,
þea magað fur þeru męnegi, \hld\ þan weldin sie kweðen, þat he só mildjene hugi
ni bári an is breostun, \hld\ só skoldi habbjen barn godes:
weldun sie só hweðeres \hld\ hèlagne Krist
þero wordo ge·wítnon, \hld\ só he þar for þemu werode ge·spráki,
a-dèldi te dóme. \hld\ Þan wisse drohtin Krist
þero manno só garo \hld\ mód-gi·þáhti,
iro wrèðon willjon; \hld\ þó he te þemu werode sprak,
te allun þem erlun: \hld\ „só hwi-lik só iuwar áno sí“, kwað he,
„slíðja sundjon, \hld\ só ganga iru selvo tó
endi sie at èrist \hld\ erl mid is handun
stèn ana werpe.“ \hld\ Só stódun Judeon,
þáhtun endi þagodun: \hld\ ni mahte þegạn nigijan
wið þem word-kwidi \hld\ wiðer-saka finden:
ge·hugde manno ge·hwi-lik \hld\ mèn-gi·þáhti,
is selves sundja: \hld\ ni was iro só sikur ènig,
þat he bi þemu worde \hld\ þemu wíve ge·dorsti
stèn an werpen, \hld\ ak létun sie standen þar
ènan þar inne \hld\ endi im út þanen
gengun gram-harde \hld\ Judeo liudi,
èn aftar óðrumu, \hld\ antat iro þar ènig ni was
þes fíundo folkes, \hld\ þe iro ferhes þó,
þeru idis aldar-lago \hld\ áhtjen weldi.
Þó gi·fragn ik þat sie frágode \hld\ friðu-barn godes,
allaro gumono bętst: \hld\ „hwar kwámun þit Judeono folk“, kwað he,
„þine wiðer-sakon, \hld\ þea þi hér wrógdun te mi?
Ne sie þi hiudu wiht \hld\ harmes ne gi·dádun,
þea liudi lèðes, \hld\ þe þi weldun lívu beniman,
wégjan te wundrun?“ \hld\ Þó sprak imu eft þat wíf an·gegin,
kwað þat iru þar nio·man \hld\ þurh þes nęrjandan
hèlaga helpa \hld\ harm ne gi·frumidi
wammes te lòne. \hld\ Þó sprak eft waldand Krist,
drohtin manno: \hld\ „ne ik þi geþ ni derju n·eo·wiht“, kwað he,
„ak gang þi hél hinen, \hld\ lát þi an þínumu hugi sorga,
þat þu nio síð aftar þius \hld\ sundig ni werðes.“
Habde iru þó gi·holpen \hld\ hèlag barn godes,
ge·friðot iro ferahe. \hld\ Þan stód þat folk Judeono
uviles an·mód \hld\ só fan èristan,
wrèðes willjan, \hld\ hwó sie word-hęti
wið þat friðu-barn godes \hld\ frummjen móstin.
Habdun þea liudi an twè \hld\ mid iro gi·lòvon gi·fangan:
was þiu smale þioda \hld\ sínes willjan
gernora mikilu, \hld\ þes godes barnes word
te ge·frummjenne, \hld\ só im iro fráho gi·bòd:
rómodun te rehta \hld\ bet þan þie ríkjon man,
habdun ina far iro hérron \hld\ ia far heven-kuning,
ful-gengun imu gerno. \hld\ Þó gi·wèt imu þe godes sunu
an þene wíh innan: \hld\ hwarf ina werod umbi,
męgin-þiodo gi·mang. \hld\ He an middjen stód,
lèrde þea liudi \hld\ liohtun wordun,
hlúdero stemnun: \hld\ was hlust mikil,
þagode þegạn manag, \hld\ endi he þeru þiod gi·bòd,
só hwe só þar mid þurstu \hld\ bi·þwungan wári,
„só ganga imu herod drinkan te mi“, \hld\ kwað he, „dago ge·hwi-likes
swótjes brunnan. \hld\ Ik mag sęggjan iu,
só hwe só hér gi·lòvid te mi \hld\ liudjo barno
fasto undar þesumu folke, \hld\ þat imu þan flioten skulun
fan is lík-hamon \hld\ libbjendi flód,
irnandi water, \hld\ aho-spring mikil,
kumad þanen kwika brunnon. \hld\ Þesa kwidi werðad wára,
liudjun gi·léstid, \hld\ só hwemu só hér gi·lòvid te mi.“
Þan mènde mid þiu wataru \hld\ waldandeo Krist,
hér heven-kuning \hld\ hèlagna gèst,
hwó þene firiho barn \hld\ ant·fáhen skoldin,
lioht endi listi \hld\ endi líf èwig,
hòh heven-ríki \hld\ endi huldi godes.
wurðun þó þea liudi \hld\ umbi þea lèra Kristes,
umbi þiu word an ge·winne: \hld\ stódun wlanka man,
gél-móde Judeon, \hld\ sprákun gelp mikil,
habdun it im te hoska, \hld\ kwaðun þat sie mahtin gi·hòrjen wel,
þat imu mahlidin fram \hld\ módaga wihti,
un·holde út: \hld\ „nu he an avu lèrid“, kwáðun sie,
„wordu ge·hwi-liku.“ \hld\ Þó sprak eft þat werod ǫ́ðar:
„ni þurvun gi þene lèrjand lahan“, \hld\ kwáðun sie: „kumad líves word
mahtig fan is múde; \hld\ he wirkid manages hwat,
wundres an þesaru wer-oldi: \hld\ nis þat wrèðaro dád,
fíundo kraftes: \hld\ nio it þan te su·likaru frumu ni wurði,
ak it gegnungo \hld\ fan gode alo-waldon,
kumid fan is krafte. \hld\ Þat mugun gi ant·kęnnjen wel
an þem is wárun wordun, \hld\ þat he gi·wald havad
alles ovar erðu.“ \hld\ Þó weldun ina þe andsakon þar
an stędi fáhen \hld\ efþa stèn ana werpen,
ef sie im þero manno \hld\ męnigi ni and-rédin,
ni forhtodin þat folk-skępi. \hld\ Þó sprak þat friðu-barn godes:
„ik tògju iu gódes só filu“, \hld\ kwað he, „fan gode selvumu,
wordo endi werko: \hld\ nu willjad gi mi wítnon hér
þurh iuwan starkan hugi, \hld\ stèn ana werpen,
bi·lòsjen mi lívu.“ \hld\ Þó sprákun imu eft þea liudi an·gegin,
wrèða wiðer-sakon: \hld\ „ne wi it be þínun werkun ni duat“, kwáðun sia,
„þat wi þi aldres \hld\ tó áhtjen willjad,
ak wi duat it be þínun wordun, \hld\ hwand þu su·lik wáh sprikis,
*hwand þu þik só máris \hld\ endi su·lik mèn sagis,
gihis for þeson Judeon, \hld\ þat þu sís god selvo,
mahtig drohtin, \hld\ endi bist þi þoh man só wi,
kuman fan þeson kunnje.“ \hld\ Krist alo-waldo
ne wolda þero Judeono þuo leng \hld\ gelpes hòrjan,
wrèðaro willjon, \hld\ ak hie im af þem wíhe fuor
ovar Jordanes stròm; \hld\ habda jungron mid im,
þia is sáligun gi·síðos, \hld\ þia im simlon mid im
willjon wonodun: \hld\ suohta werod óðer,
deda þar só hie gi·wonoda, \hld\ drohtin selvo,
lèrda þia liudi: \hld\ gi·lòvda þie wolda
an is hèlagun word. \hld\ Þat skolda sinnon wel
manno só hwi-likon, \hld\ só þat an is muod gi·nam.
Þuo gi·frang ik þat þar te Kriste \hld\ kumana wurðun %NOTE: gi·frang] Checked according to C.
bodon fan Bethaniu \hld\ endi sagdun þem barne godes,
þat sia an þat árundi þarod \hld\ idisi sendin,
Maria endi Marþa, \hld\ magað frí-líka,
swíðo wun-sama wíf; \hld\ þia wissa hie bèðja,
wárun im gi·swester twá, \hld\ þia hie selvo ér
minnjoda an is muode \hld\ þuru iro mildjan hugi,
þiu wíf þuru iro willjon guodan. \hld\ Sia im te wáron þuo
an-budun fon Bethaniu, \hld\ þat iro bruoðer was
Lazarus legar-fast \hld\ endi þat sia is líves ni wándun;
bádun þat þarod kwámi \hld\ Krist alo-waldo
hèlag te helpu. \hld\ Reht só hie sia gi·hòrda þuo
sęggjan fan só siekon, \hld\ só sprak hie sán an·gegin,
kwað þat Lazaruses \hld\ legar ni wári
gi·duan im te dòðe, \hld\ „ak þar skal drohtines lof“, kwat-hie,
„gi·frumid werðan: \hld\ nis it im te óðron fréson gi·duan.“
was im þar þuo selvo \hld\ suno drohtines
twá naht endi dagas. \hld\ Þiu tíd was þuo ge·náhit,
þat hie eft te Hjerusalem \hld\ Judeo liudjo
wíson welda, \hld\ só hie gi·wald habda.
Sagda þuo is gi·síðon \hld\ suno drohtines,
þat hie eft ovar Jordan \hld\ Judeo liudi
suokjan welda. \hld\ Þuo sprákun im sán an·gegin
jungron sína: \hld\ „te hwí bist þu só gern þarod“, kwaðun sia,
„frò mín, te faranne? \hld\ Ni þat nu furn ni was,
þat sia þik þínero wordo \hld\ wítnon hogdun,
weldun þi mid stènon starkan a·werpan? \hld\ nu þu eft undar þia strídigun þioda
fundos te faranne, \hld\ þar ist fíondo ginuog,
erlos ovar-muoda?“ \hld\ Þuo èn þero twelivjo,
Þuomas gi·málda \hld\ —was im gi·þungan mann,
diur-lík drohtines þegạn—: \hld\ „ne skulun wi im þia dád lahan“, kwat-hie,
„ni węrnjan wi im þes willjen, \hld\ ak wita im wonjan mid,
þuolojan mid ússon þiodne: \hld\ þat ist þegnes kust,
þat hie mid is fráhon samad \hld\ fasto gi·stande,
dòje mid im þar an duome. \hld\ Duan ús alla só,
folgon im te þero ferdi: \hld\ ni látan úse ferah wið þiu
wihtes wirðig, \hld\ neva wi an þem werode mid im,
dòjan mid úson drohtine. \hld\ Þan lévot ús þoh duom after,
guod word for gumon.“ \hld\ Só wurðun þuo jungron Kristes,
erlos aðal-borana \hld\ an èn-falden hugje,
hérren te willjen. \hld\ Þuo sagda hèlag Krist
selvo is gi·síðon \hld\ þat a·slápan was
Lazarus fan þem legare, \hld\ „havit þit lioht a·gevan,
an-swevit ist an selmon. \hld\ Nu wi an þena síð faran
endi ina a·wękkjan, \hld\ þat hie muoti eft þesa wer-old sehan,
libbjandi lioht: \hld\ þan wirðit iuwa gi·lòvo after þiu
forð-werd gi·fęstid.“ \hld\ Þuo gi·wèt hie im ovar þia fluod þanan,
þie guodo godes suno, \hld\ anþat hie mid is jungron kwam
þar te Bithaniu, \hld\ barn drohtines
selvo mid is gi·síðon, \hld\ þar þia gi·swester twá,
Maria endi Marþa \hld\ an muod-karon
sèraga sátun. \hld\ Was þar gi·samnot filo
fan Hjerusalem \hld\ Judeo liudo,
þia þiu *wíf weldun \hld\ wordun fruovrjan,
þat sie só ni karodin \hld\ kind-jungas dòð,
Lazaruses far·lust. \hld\ Só þó þe landes ward
geng an þiu gardos, \hld\ só wurðun þes godes barnes
kumi þar gi·kúðid, \hld\ þat he só kraftig was
bi þeru burg úten. \hld\ Þó im bèðjun was,
þem wívun su·lik willjo, \hld\ þat sie im waldand tó,
þat friðu-barn godes, \hld\ farandjen wissun.
Þó þem wíbun was \hld\ willjono mèsta
kumi drohtines \hld\ endi Kristes word
te gi·hòrjenne. \hld\ Heovandi geng
Martha mód-karag \hld\ wið só mahtigne
wordun wehslan \hld\ endi wið waldand sprak
an iro hugi hriwig: \hld\ „þar þu mi, hérro mín“, kwað siu,
„nęrjendero bętst, \hld\ náhor wáris,
hèljand þe gódo, \hld\ þan ni þorfti ik nu su·lik harm þolon,
bittra breost-kara, \hld\ þan ni wári nu mín bróðer dòd,
Lazarus fan þesumu liohte, \hld\ ak he imu mahti libbjen forð
ferahes ge·fullid. \hld\ Ik þoh, fró mín, te þi
liohto gi·lòvju, \hld\ lèrjandero bętst,
só hwes só þu biddjen wili \hld\ berhton drohtin,
þat he it þi sán far·givid, \hld\ god alo-mahtig,
gi·werðot þínan willjan.“ \hld\ Þó sprak eft waldand Krist
þeru idis and-wordi: \hld\ „ni lát þu þi an innan þes“, kwað he,
„þínan sevon swerkan: \hld\ ik þi sęggjan mag
wárun wordun, \hld\ þat þes nis gi·wand ènig,
nevu þín bróðer skal \hld\ þurh gi·bod godes,
þurh drohtines kraft \hld\ fan dòðe a·standen
an is lík-hamon.“ \hld\ „All hębbju ik gi·lòvon só“, kwað siu,
„þat it só gi·werðen skal, \hld\ só hwan só þius wer-old endjod
endi þe márjo dag \hld\ ovar man fęrid,
þat he þan fan erðu skal \hld\ up a·standen
an þemu dómes daga, \hld\ þan werðad fan dòðe kwika
þurh maht godes \hld\ man-kunnjes ge·hwi-lik,
a-rísad fan restu.“ \hld\ Þó sagde ríkjo Krist
þeru idis alo-mahtig \hld\ oponun wordun,
þat he selvo was \hld\ sunu drohtines,
bèðju ia líf ia lioht \hld\ liudjo barnon
te a·standanne: \hld\ „nio þe sterven ni skal,
líf far·liosen, \hld\ þe hér gi·lòvid te mi:
þoh ina eldi-barn \hld\ erðu bi·þękkjen,
diapo bi·delven, \hld\ nis he dòd þiu mér:
þat flésk is bi·folhen, \hld\ þat ferah is gi·halden,
is þiu siola gi·sund.“ \hld\ Þó sprak imu eft sán an·gegin
þat wíf mid iro wordun: \hld\ „ik gi·lòvju þat þu þe wáro bist“, kwað siu,
„Krist godes sunu: \hld\ þat mag man ant·kęnnjen wel,
witen an þínun wordun, \hld\ þat þu gi·wald haves
þurh þiu hèlagon gi·skapu \hld\ himiles endi erðun.“
Þó ge·fragn ik þat þar þero idisio kwam \hld\ ǫ́ðar gangan
Maria mód-karag: \hld\ gengun iro managa aftar
Judeo liudi. \hld\ Þó siu þemu godes barne
sagde sèrag-mód, \hld\ hwat iru te sorgun gi·stód
an iro hugi harmes: \hld\ hofnu kúmde
Lazaruses far·lust, \hld\ liaves mannes,
griat gornundi, \hld\ antat þemu godes barne
hugi warð gi·hrórid: \hld\ hète trahni
wópu a·wellun, \hld\ endi þó te þem wívun sprak,
hét ina þó lèdjen, \hld\ þar Lazarus was
foldu bi·folhen. \hld\ Lag þar èn felis bi·ovan,
hard stèn be·hliden. \hld\ Þó hét þe hèlago Krist
ant·lúkan þea léia, \hld\ þat he mósti þat lík sehan,
hréo skawojen. \hld\ Þó ni mahte an iro hugi míðan
Marþa for þeru męnegi, \hld\ wið mahtigne sprak:
„fró mín þe gódo“, \hld\ kwað siu, „ef man þene felis nimid,
þene stèn ant·lúkid, \hld\ þan wániu ik þat þanen stank kume,
un·swóti swek, \hld\ hwand ik þi sęggjan mag
wárun wordun, \hld\ þat þes nis gi·wand ènig,
þat he þar nu bi·folhen was \hld\ fiuwar naht endi dagos
an þemu erð-grave.“ \hld\ And-wordi gaf
waldand þemu wíbe: \hld\ „hwat, ni sagde ik þi te wárun ér“, kwað he,
„ef þu gi·lòvjen wili, \hld\ þan nis nu lang te þiu,
þat þu hér ant·kęnnjen skalt \hld\ kraft drohtines,
þe mikilon maht godes?“ \hld\ Þó gengun manage tó,
af·hóvun harden stèn. \hld\ Þó sah þe hèlago Krist
up mid is ògun, \hld\ ǫ́·lát sagde %NOTE: ǫ́·lát = álát
þemu þe þese wer-old gi·skóp, \hld\ „þes þu mín word gi·hòris“, kwað he,
„sigi-drohtin selvo; \hld\ ik wèt þat þu só simlun duos,
ak ik duom it be þesumu gròton \hld\ Judeono folke,
þat sie þat te wárun witin, \hld\ þat þu mi an þese wer-old sendes
þesun liudjun te lèrun.“ \hld\ Þó he te Lazaruse hriop
starkaru stemnju \hld\ endi hét ina standen up
ia fan þemu grave gangan. \hld\ Þó warð þe gèst kumen
an þene lík-hamon: \hld\ he bi·gan is liði hrórjen,
ant·warp undar þemu gi·wédje: \hld\ was imo só be·wunden þó noh,
an hréo-będdjon bi·helid. \hld\ Hét imu helpen þó
waldandeo Krist. \hld\ Weros gengun tó,
ant·wundun þat ge·wádi. \hld\ Wánum up a·rés
Lazarus te þesumu liohte: \hld\ was imu is líf far·geven,
þat he is aldar-lagu \hld\ ègan mósti,
friðu forð-wardes. \hld\ Þó fagonadun bèðja,
Maria endi Marþa: \hld\ ni mag þat man óðrumu
gi·sęggjan te sǫ́ðe, \hld\ hwó þea gesuester twó
mendjodun an iro móde. \hld\ Maneg wundrode
Judeo liudjo, \hld\ þó sie ina fan þemu grave sáhun
síðon ge·sunden, \hld\ þene þe ér suht far·nam
endi sie bi·dulvun \hld\ diapo undar erðu
líves lòsen: \hld\ þó móste imu libbjen forð
hél an hèmun. \hld\ Só mag heven-kuninges,
þiu mikile maht godes \hld\ manno ge·hwi-likes
ferahe gi·formon \hld\ endi wið fíundo níð
hèlag helpen, \hld\ só hwemu só he is huldi far·givid
Þó warð þar só managumu manne \hld\ mód aftar Kriste,
gi·hworven hugi-skęfti, \hld\ síðor sie is hèlagon werk
selvon gi·sáhun, \hld\ hwand eo ér su·lik ni warð
wunder an wer-oldi. \hld\ Þan was eft þes werodes só filu,
só mód-starke man: \hld\ ni weldon þe maht godes
ant·kęnnjen kúð-líko, \hld\ ak sie wið is kraft mikil
wunnun mid iro wordun: \hld\ wárun im waldandes
lèra so lèða: \hld\ sóhtun im liudi óðra
an Hjerusalem, \hld\ þar Judeono was
héri hand-mahal \hld\ endi hòvid-stędi,
gròt gum-skępi \hld\ grimmaro þioda.
Sie kúðdun im þó Kristes werk, \hld\ kwáðun þat sie kwikan sáhin
þene erl mid iro ògun, \hld\ þe an erðu was,
foldu bi·folhen \hld\ fiuwar naht endi dagos,
dòd bi·dolven, \hld\ antat he ina mid is dádjun selvo,
mid is wordun a·wekide, \hld\ þat he mósti þese wer-old sehan.
Þó was þat só wiðer-ward \hld\ wlankun mannun,
Judeo liudjun: \hld\ hétun iro gum-skępi þó,
werod samnojan \hld\ endi warvos fáhen,
męgin-þioda gi·mang, \hld\ an mahtigna Krist
riedun an runun: \hld\ „nis þat rád ènig“, kwáðun sie,
„þat wi þat gi·þolojan: \hld\ wili þesaro þioda te filu
gi·lòvjen aftar is lèrun. \hld\ Þan ús liudi farad,
an eo-rid-folk, \hld\ werðat úsa ovar-hòvdun
rinkos fan Rúmu. \hld\ Þan wi þeses ríkjes skulun
lòse libbjen \hld\ efþa wi skulun úses líves þolon,
hęliðos úsaro hòvdo.“ \hld\ Þó sprak þar èn gi·hérod man
ovar warf wero, \hld\ þe was þes werodes þó
an þeru burg innan \hld\ biskop þero liudjo
—Kaiphas was he hèten; \hld\ habdun ina gi·koranen te þiu
an þeru gér-talu \hld\ Judeo liudi,
þat he þes godes húses \hld\ gòmjen skoldi,
wardon þes wíhes—: \hld\ „mi þunkid wunder mikil“, kwað he,
„mári þioda, \hld\ —gi kunnun manages gi·skéð—
hwí gi þat te wárun ni witin, \hld\ werod Judeono,
þat hér is bętera rád \hld\ barno ge·hwi-likumu,
þat man hér ènne man \hld\ aldru bi·lòsje
endi þat he þurh iuwa dádi \hld\ dròreg sterve,
for þesumu folk-skępi \hld\ ferah far·láte,
þan al þit liud-werod \hld\ far·loren werðe.“
Ni was it þoh is willjan, \hld\ þat he só wár ge·sprak,
só forð for þemu folke, \hld\ frume man-kunnjes
gi·mènde for þeru męnegi, \hld\ ak it kwam imu fan þeru maht godes
þurh is hèlagan héd, \hld\ hwand he þat hús godes
þar an Hjerusalem \hld\ bi·gangan skolde,
wardon þes wíhes: \hld\ be·þiu he só wár gi·sprak,
biskop þero liudjo, \hld\ hwó skoldi þat barn godes
alla irmin-þiod \hld\ mid is ènes ferhe,
mid is lívu a·lòsjen: \hld\ þat was allaro þesaro liudjo rád,
hwand he gi·halode \hld\ mid þiu héðina liudi,
weros an is willjon \hld\ waldandio Krist.
Þó wurðun èn-wordje \hld\ ovar-módje man,
werod Judeono, \hld\ endi an iro warve gi·sprákun,
mári þioda, \hld\ þat sie im ni létin iro mód twehon:
só hwe só ina undar þemu folke \hld\ finden mahti,
þat ina sán gi·fengi \hld\ endi forð bráhti
an þero þiodo þing; \hld\ kwáðun þat sie ni mahtin gi·þolojan leng,
þat sie þe èno man \hld\ só alla weldi,
werod far·winnen. \hld\ Þan wisse waldand Krist
þero manno só garo \hld\ mód-gi·þáhti,
hęti-grimmon hugi, \hld\ hwand imu ni was bi·holen eo·wiht
an þesaru middil-gard: \hld\ he ni welde þó an þie męnigi innen
síður open-líko, \hld\ under þat erlo folk,
gangan under þea Judeon: \hld\ béd þe godes sunu
þero torohtjon tíd, \hld\ þe imu tóward was,
þat he far þesa þioda \hld\ þolojan welde,
far þit werod wíti: \hld\ wisse imu selvo
þat dag-þingi garo. \hld\ Þó gi·wèt imu úse drohtin forð
endi imu þó an Effrem \hld\ alo-waldo Krist
an þeru hòhon burg \hld\ hèlag drohtin
wunode mid is werodu, \hld\ antat he an is willjan hwarf
eft te Bethania \hld\ brahtmu þiu mikilun,
mid þiu is gódum gum-skępi. \hld\ Judeon bisprákun þat
wordu ge·hwi-liku, \hld\ þó sie imu su·lik werod mikil
folgon gi·sáhun: \hld\ „nis frume ènig“, kwáðun sie,
„úses ríkjes gi·rádi, \hld\ þoh wi reht sprekan,
ni þíhit úses þinges wiht: \hld\ þius þiod wili
węndjen after is willjan; \hld\ imu all þius wer-old folgot,
liudi bi þem is lèrun, \hld\ þat wi imu lèðes wiht
for þesumu folk-skępi \hld\ gi·frummjen ni mótun.“
Gi·wèt imu þó þat barn godes \hld\ innan Bethania
sehs nahtun ér, \hld\ þan þiu samnunga
þar an Hjerusalem \hld\ Judeo liudjo
an þem wíh-dagun \hld\ werðen skolde,
þat sie skoldun haldan \hld\ þea hèlagon tídi,
Judeono paskha. \hld\ Béd þe godes sunu,
mahtig under þeru męnegi: \hld\ was þar manno kraft,
werodes bi þem is wordun. \hld\ Þar gengun ina twè wíf umbi,
Maria endi Marþa, \hld\ mid mildju hugi,
þionodun imu þeo-líko. \hld\ Þiodo drohtin
gaf im lang-sam lòn: \hld\ lét sea lèðes gi·hwes,
sundjono sikora, \hld\ endi selvo gi·bòd,
þat sea an friðe fórin \hld\ wiðer fíundo níð,
þea idisa mid is orlovu gódu: \hld\ habdun iro ambaht-skępi
bi·wendid an is willjon. \hld\ Þó gi·wèt imu waldand Krist
forð mid þiu folku, \hld\ firiho drohtin,
innan Hjerusalem, \hld\ þar Judeono was
hete-lík hard-buri, \hld\ þar sie þea hèlagon tíd
warodun at þemu wíhe; \hld\ was þar werodes só filu,
kraftigaro kunnjo, \hld\ þie ni weldun Kristes word
gerno hòrjen \hld\ ni te þemu godes barne
an iro mód-sevon \hld\ minnje ni habdun,
ak wárun im só wrèða \hld\ wlanka þioda,
módeg man-kunni, \hld\ habdun im morð-hugi,
in·wid an innan: \hld\ an avuh far·fengun
Kristes lère, \hld\ weldun ina kraftigna
wítnon þero wordo; \hld\ ak was þar werodes só filu,
umbi erl-skępi \hld\ ant·langana dag,
habde ine þiu smale þiod \hld\ þurh is swótiun word
werodu bi·worpen, \hld\ þat ine þie wiðer-sakon
under þemu folk-skępi \hld\ fáhen ne gi·dorstun,
ak miðun is bi þeru męnegi. \hld\ Þan stód mahtig Krist
an þemu wíhe innan, \hld\ sagde word manag
firiho barnun te frumu. \hld\ Was þar folk umbi
allan langan dag, \hld\ antat þiu liohte gi·wèt
sunne te sedle. \hld\ Þó te seliðun fór
man-kunnjes manag. \hld\ Þan was þar èn mári berg
bi þeru burg úten, \hld\ þe was brèd endi hòh,
gróni endi skóni: \hld\ hétun ina Judeo liudi
Oliueti bi namon. \hld\ Þar imu up gi·wèt
nęrjendjo Krist, \hld\ só ina þiu naht bi·feng,
was imu þar mid is jungarun, \hld\ só ine þar Judeono ènig
ni wisse ti wárun, \hld\ hwand he an þemu wíhe stód,
liudjo drohtin, \hld\ só lioht óstene kwam,
ant·feng þat folk-skępi \hld\ endi im filu sagde
wároro wordo, \hld\ só nis an þesaru wer-oldi ènig,
an þesaru middil-gard \hld\ manno só spáhi,
liudjo barno nig·èn, \hld\ þat þero lèrono mugi
endi gi·tęlljen, \hld\ þe he þar an þemu alahe gi·sprak,
waldand an þemu wíhe, \hld\ endi simlun mid is wordun gi·bòd,
þat sie sie gerewidin \hld\ te godes ríkje,
allaro manno ge·hwi-lik, \hld\ þat sie móstin an þemu márjon daga
iro drohtines \hld\ diuriða ant·fáhen.
Sagde im hwat sie it sundjun frumidun \hld\ endi simlun gi·bòd,
þat sie þea a·leskidin; \hld\ hét sie lioht godes
minnjon an iro móde, \hld\ mèn far·láten,
avoha ovar-hugdi, \hld\ òd-módi niman,
hlaðen þat an iro hertan; \hld\ kwað þat im þan wári heven-ríki,
garu gódo mèst. \hld\ Þó warð þar gumono só filu
gi·wendid aftar is willjon, \hld\ síður sie þat word godes
hèlag gi·hòrdun, \hld\ heven-kuninges,
ant·kendun kraft mikil, \hld\ kumi drohtines,
hérron helpe, \hld\ ia þat heven-ríki was,
nęrjendi gi·náhid \hld\ endi náða godes
manno barnun. \hld\ Sum só módeg was
Judeo folkes, \hld\ habdun grimman hugi,
slíð-móden sevon \hld\ [...],
ni weldun is worde gi·lòvjen, \hld\ ak habdun im ge·win mikil
wið þea Kristes kraft: \hld\ kumen ni móstun
þea liudi þurh lèðen stríd, \hld\ þat sie gi·lòvon te imu
fasto gi·fengin; \hld\ ni was im þiu frume giviðig,
þat sie heven-ríki \hld\ habbjen móstin.
Geng imu þó þe godes sunu \hld\ endi is jungaron mid imu,
waldand fan þemu wíhe, \hld\ all só is willjo geng,
iak imu uppen þene berg gi·stèg \hld\ barn drohtines:
sat imu þar mid is ge·síðun \hld\ endi im sagde filu
wároro wordo. \hld\ Sí bi·gunnun im þó umbi þene wíh sprekan,
þie gumon umbi þat godes hús, \hld\ kwáðun þat ni wári gód-líkora
alah ovar erðu \hld\ þurh erlo hand,
þurh mannes gi·werk \hld\ mid męgin-kraftu
rakud a·rihtid. \hld\ Þó þe ríkjo sprak,
hér heven-kuning \hld\ —hòrdun þe óðra—:
„ik mag iu gi·tęlljen“, \hld\ kwað he, „þat noh wirðid þiu tíd kumen,
þat is af·standen ni skal \hld\ stèn ovar óðrumu,
ak it fallid ti foldu \hld\ endi fiur nimid,
grádag logna, \hld\ þoh it nu só gód-lík sí,
só wís-líko gi·warht, \hld\ endi só dód all þesaro wer-oldes gi·skapu,
te·glídid gróni wang.“ \hld\ Þó gengun imu is jungaron tó,
frágodun ina só stillo: \hld\ „hwó lango skal standen noh“, kwáðun sie,
„þius wer-old an wunnjun, \hld\ ér þan þat gi·wand kume,
þat þe lasto dag \hld\ liohtes skíne
þurh wolkan-skion, \hld\ efþo hwan is þín eft wán kumen
an þene middil-gard, \hld\ manno kunnje
te a·dèljenne, \hld\ dòdun endi kwikun?
fró mín þe gódo, \hld\ ús is þes firi-wit mikil,
waldandeo Krist, \hld\ hwan þat gi·werðen skuli.“
Þó im and-wordi \hld\ alo-waldo Krist
gód-lík far·gaf \hld\ þem gumun selvo:
„þat havad só bi·dernid“, \hld\ kwað he, „drohtin þe gódo,
iak só hardo far·holen \hld\ himil-ríkjes fader,
waldand þesaro wer-oldes, \hld\ só þat witen ni mag
ènig mannisk barn, \hld\ hwan þiu márje tíd
gi·wirðid an þesaru wer-oldi, \hld\ ne it ók te wáran ni kunnun
godes ęngilos, \hld\ þie for imu gegin-warde
simlun sindun: \hld\ sie it ók gi·sęggjan ni mugun
te wáran mid iro wordun, \hld\ hwan þat gi·werðen skuli,
þat he willje an þesan middil-gard, \hld\ mahtig drohtin,
firiho fandon. \hld\ Fader wèt it èno
hèlag fan himile: \hld\ elkur is it bi·holen allun,
kwikun endi dòdun, \hld\ hwan is kumi werðad,
Ik mag iu þoh gi·tęlljen, \hld\ hwi-lik hér tèkạn bi·foran
gi·werðad wunder-lík, \hld\ ér þan he an þese wer-old kume
an þemu márjon daga: \hld\ þat wirðid hér ér an þemu mánon skín
iak an þeru sunnon só same; \hld\ gi·swerkad siu bèðju,
mid finistre werðad bi·fangan; \hld\ fallad sterron,
hwít heven-tungal, \hld\ endi hrisid erðe,
bivod þius brède wer-old \hld\ —wirðid su·likaro bókno filu—:
grimmid þe gròto sèo, \hld\ wirkid þie gevenes stròm
egison mid is úðjun \hld\ erð-búandjun.
Þan þorrot þiu þiod \hld\ þurh þat ge·þwing mikil,
folk þurh þea forhta: \hld\ þan nis friðu hwergin,
ak wirðid wíg só maneg \hld\ ovar þese wer-old alla
hete-lík af·haben, \hld\ endi hęri lèdid
kunni ovar ǫ́ðar: \hld\ wirðid kuningo gi·win,
męgin-fard mikil: \hld\ wirðid managoro kwalm,
open ur-lagi \hld\ —þat is egis-lík þing,
þat io su·lik morð \hld\ skulun man af·hębbjen—,
wirðid wól só mikil \hld\ ovar þese wer-old alle,
man-stervono mèst, \hld\ þero þe gio an þesaru middil-gard
swulti þurh suhti: \hld\ liggjad seoka man,
driosat endi dòjat \hld\ endi iro dag ęndjad,
fulljad mid iro ferahu; \hld\ fęrid un·met gròt
hungar hęti-grim \hld\ ovar hęliðo barn,
męti-gédjono mèst: \hld\ nis þat minniste
þero wítjo an þesaru wer-oldi, \hld\ þe hér gi·werðen skulun
ér dómes dage. \hld\ Só hwan só gi þea dádi gi·sehan
gi·werðen an þesaru wer-oldi, \hld\ só mugun gi þan te wáran far·standen,
þat þan þe latsto dag \hld\ liudjun náhid
mári te mannun \hld\ endi maht godes,
himil-kraftes hróri \hld\ endi þes hèlagon kumi,
drohtines mid is diuriðun. \hld\ Hwat, gi þesaro dádjo mugun
bi þesun bómun \hld\ biliði ant·kęnnjen:
þan sie brustjad endi blójat \hld\ endi bladu tògjat,
lóf ant·lúkad, \hld\ þan witun liudjo barn,
þat þan is sán after þiu \hld\ sumer gi·náhid
warm endi wun-sam \hld\ endi weder skóni.
Só witin gi ók bi þesun tèknun, \hld\ þe ik iu talde hér,
hwan þe latsto dag \hld\ liudjun náhid.
Þan sęggjo ik iu te wáran, \hld\ þat ér þit werod ni mót,
te·faran þit folk-skępi, \hld\ ér þan werðe ge·fullid só,
mínu word gi·wárod. \hld\ Noh gi·wand kumid
himiles endi erðun, \hld\ endi steid mín hèlag word
fast forð-wardes \hld\ endi wirðid al ge·fullod só,
gi·léstid an þesumu liohte, \hld\ só ik for þesun liudjun ge·spriku.
wakot gi war-líko: \hld\ iu is wis-kumo
duom-dag þe márjo \hld\ endi iuwes drohtines kraft,
þiu mikilo męgin-strengi \hld\ endi þiu márje tíd,
gi·wand þesaro wer-oldes. \hld\ Fora þiu gi wardon skulun,
þat he iu slápandje \hld\ an swef-restu
fárungo ni bi·fáhe \hld\ an firin-werkun,
mènes fulle. \hld\ Mútspelli kumit
an þiustrja naht, \hld\ al só þiof fęrid
darno mid is dádjun, \hld\ só kumid þe dag mannun,
þe latsto þeses liohtes, \hld\ só it ér þese liudi ni witun,
só samo só þiu flód deda \hld\ an furn-dagun,
þe þar mid lagu-stròmun \hld\ liudi far·teride
bi Nóeas tídjun, \hld\ bi·útan þat ina neride god
mid is híwiskja, \hld\ hèlag drohtin,
wið þes flódes farm: \hld\ só warð ók þat fiur kuman
hèt fan himile, \hld\ þat þea hòhon burgi
umbi Sodomo land \hld\ swart logna bi·feng
grim endi grádag, \hld\ þat þar n·ènig gumono ni gi-nas
bi·útan Loth èno: \hld\ ina ant·lèddun þanen
drohtines ęngilos \hld\ endi is dohter twá
an ènan berg uppen: \hld\ þat ǫ́ðar al brinnandi fiur,
ia land ia liudi \hld\ logna far·teride:
só fárungo warð þat fiur kumen, \hld\ só warð ér þe flód só samo:
só wirðid þe latsto dag. \hld\ For þiu skal allaro liudjo ge·hwi-lik
þęnkjan fora þemu þinge; \hld\ þes is þarf mikil
manno ge·hwi-likumu: \hld\ be·þiu látad iu an iuwan mód sorga.
Hwand só hwan só þat ge·wirðid, \hld\ þat waldand Krist,
mári mannes sunu \hld\ mid þeru maht godes,
kumit mid þiu kraftu \hld\ kuningo ríkjost
sittjan an is selves maht \hld\ endi samod mid imu
alle þea ęngilos, \hld\ þe þar uppa sind
hèlaga an himile, \hld\ þan skulun þarod hęliðo barn,
ęli-þeoda kuman \hld\ alla te·samne
libbjandero liudjo, \hld\ só hwat só io an þesumu liohte warð
firiho a·fódid. \hld\ Þar he þemu folke skal,
allumu man-kunnje \hld\ mári drohtin
a-dèljen aftar iro dádjun. \hld\ Þan skéðid he þea far·duanan man,
þea far·warhton weros \hld\ an þea winistron hand:
só duot he ók þea sáligon \hld\ an þea swíðeron half;
grótid he þan þea gódun \hld\ endi im te·gegnes sprikid:
„kumad gi“, kwiðid he, „þea þar gi·korene sindun, \hld\ endi ant·fáhad þit kraftiga ríki,
þat góde, þat þar gi·gerewid stendid, \hld\ þat þar warð gumono barnun
gi·warht fan þesaro wer-oldes endje: \hld\ iu havad ge·wíhid selvo
fader allaro firiho barno: \hld\ gi mótun þesaro frumono neotan,
ge·waldon þeses wídon ríkjas, \hld\ hwand gi oft mínan willjon frumidun,
ful-gengun mi gerno \hld\ endi wárun mi iuwaro gevo mildje,
þan ik bi·þwungan was \hld\ þurstu endi hungru,
frostu bi·fangan \hld\ efþo an feteron lag,
bi·klemmid an karkare: \hld\ oft wurðun mi kumana þarod
helpa fan iuwn handun: \hld\ gi wárun mi an iuwomu hugi mildje,
wísodun mín werðliko.“ \hld\ Þan sprikid imu eft þat werod an·gegin:
„fró mín þe gódo“, \hld\ kweðat sie, „hwan wári þu bi·fangan só,
be·þwungan an su·likun þarạvun, \hld\ só þu fora þesaru þiod telis,
mahtig mènis? \hld\ Hwan gi·sah þi man ènig
be·þwungen an su·likun þarạvun? \hld\ Hwat, þu haves allaro þiodo gi·wald
iak só samo þero mèðmo, \hld\ þero þe io manno barn
ge·wunnun an þesaro wer-oldi.“ \hld\ Þan sprikid im eft waldand god:
„só hwat só gi dádun“, \hld\ kwiðit he, „an iuwes drohtines namon,
gódes far·gávun \hld\ an godes éra
þem mannun, þe hér minniston sindun, \hld\ þero nu undar þesaru męnegi standad
endi þurh òd-módi \hld\ arme wárun
weros, hwand sie mínan willjon fremidun \hld\ —só hwat só gi im iuwaro welono far·gávun,
gi·dádun þurh diuriða, \hld\ þat ant·feng iuwa drohtin selvo,
þiu helpe kwam te heven-kuninge. \hld\ Be·þiu wili iu þe hèlago drohtin
lònon iuwan gi·lòvon: \hld\ givid iu líf èwig.“
wendid ina þan waldand \hld\ an þea winistron hand,
drohtin te þem far·duanun mannun, \hld\ sagad im þat sie skulin þea dád ant·gelden,
þea man iro mèn-gi·werk: \hld\ „nu gi fan mi skulun“, kwiðit he.
„faran só for·flókane \hld\ an þat fiur èwig,
þat þar gi·garewid warð \hld\ godes and-sakun,
fíundo folke \hld\ be firin-werkun,
hwand gi mi ni hulpun, \hld\ þan mi hunger endi þurst
wégde te wundrun \hld\ efþa ik ge·wádjes lòs
geng jámer-mód, \hld\ was mi gròtun þarf,
þan ni habde ik þar ènige helpe, \hld\ þan ik ge·hęftid was,
an liðokospun bi·lokan, \hld\ efþa mi legar bi·feng,
swára suhti: \hld\ þan ni weldun gi mín siokes þar
wíson mid wihti: \hld\ ni was iu werð eo·wiht,
þat gi mín ge·hugdin. \hld\ Be·þiu gi an hęllje skulun
þolon an þiustre.“ \hld\ Þan sprikid imu eft þiu þiod an·gegin:
„wola waldand god“, \hld\ kweðad sie, „hwí wilt þu só wið þit werod sprekan,
mahljen wið þese męnegi? \hld\ Hwan was þi io manno þarf,
gumono gódes? \hld\ Hwat, sie it al be þínun gevun égun,
welon an þesaro wer-oldi“. \hld\ Þan sprikid eft waldand god:
„þan gi þea armostun“, \hld\ kwiðid he, „eldi-barno,
manno þea minniston \hld\ an iuwomu mód-sevon
hęliðos far·hugdun, \hld\ létun sea iu an iuwomu hugi lèðe,
be·dèldun sie iuwaro diurða, \hld\ þan dádun gi iuwana drohtin só sama,
gi·węrnidun imu iuwaro welono: \hld\ be·þiu ni wili iu waldand god,
ant·fáhen fader iuwa, \hld\ ak gi an þat fiur skulun,
an þene diopun dòð, \hld\ diuvlun þionon,
wrèðun wiðer-sakun, \hld\ hwand gi só warhtun bi·foran.“
Þan aftar þem wordun skéðit \hld\ þat werod an twè,
þea gódun endi þea uvilon: \hld\ farad þea far·griponon man
an þea hètan hęl \hld\ hriwig-móde,
þea far·warhton weros, \hld\ wíti ant·fáhat,
uvil ęndi-lòs. \hld\ Lédid up þanen
hér heven-kuning \hld\ þea hluttaron þeoda
an þat lang-same lioht: \hld\ þar is líf èwig,
gi·garewid godes ríki \hld\ gódaro þiado.“
Só ge·fragn ik þat þem rinkun þo \hld\ ríki drohtin
umbi þesaro wer-oldes gi·wand \hld\ wordun talde,
hwó þiu forð fęrid, \hld\ þan lango þe sie firiho barn
ardon mótun, \hld\ ia hwó siu an þemu endje skal
te·glíden endi te·gangen. \hld\ He sagde ók is jungarun þar
wárun wordun: \hld\ „hwat, gi witun alle“, kwað he,
„þat nu ovar twá naht \hld\ sind tídi kumana,
Gjudeono paskha, \hld\ þat sie skulun iro gode þionon,
weros an þemu wíhe. \hld\ Þes nis ge·wand ènig,
þat þar wirðid mannes sunu \hld\ te þeru męgin-þiodu
kraftag far·kòpot \hld\ endi an krúke a·slagan,
þolod þiad-kwála.“ \hld\ Þó warð þar þegạn manag
slíð-mód gi·samnod, \hld\ súðar-liudjo,
Judeono gum-skępi, \hld\ þar sie skoldun iro gode þionon.
wurðun éo-sagon \hld\ alle kumane,
an warf weros, \hld\ þe sie þó wísostun
undar þeru męnegi \hld\ manno taldun,
kraftag kuni-burd. \hld\ Þar Kaiphas was,
biskop þero liudjo. \hld\ Sie rédun þó an þat barn godes,
hwó sie ina a·sluogin \hld\ sundja lòsan,
kwáðun þat sie ina an þemu hèlagon daga \hld\ hrínen ni skoldin
undar þero manno męnegi, \hld\ „þat ni werðe þius męgin-þioda,
hęliðos an hróru, \hld\ hwand ina þit hęri-skępi wili
far·standen mid strídu. \hld\ Wi só stillo skulun
fréson is ferahes, \hld\ þat þit folk Judeono
an þesun wíh-dagun \hld\ wróht ni af·hębbjen.“
Þó geng imu þar Júdas forð, \hld\ jungaro Kristes,
èn þero twelivjo, \hld\ þar þat aðali sat,
Judeono gum-skępi; \hld\ kwað þat he is im gódan rád
sęggjan mahti: \hld\ „hwat willjad gi mi sęlljen hér“, kwað he,
„mèðmo te médu, \hld\ ef ik iu þene man givu
áno wíg endi áno wróht?“ \hld\ Þó warð þes werodes hugi,
þero liudjo an lustun: \hld\ „ef þu wili gi·léstjen só“, kwáðun sie,
„þín word gi·wáron, \hld\ þan þu gi·wald haves,
hwat þu at þesaru þiodu \hld\ þiggjan willjes
gódaro mèðmo.“ \hld\ Þó gi·hét imu þat gum-skępi þar
an is selves dóm \hld\ siluvar-skatto
þrí-tig at-samne, \hld\ endi he te þeru þiodu gi·sprak
dereveun wordun, \hld\ þat he gávi is drohtin wið þiu.
wende ina þó fan þemu werode: \hld\ was im wrèð hugi,
talode im só treu-lòs, \hld\ hwan ér wurði imu þiu tíd kuman,
þat he ina mahti far·wísjen \hld\ wrèðaro þiodo,
fíundo folke. \hld\ Þan wisse þat friðu-barn godes,
wár waldand Krist, \hld\ þat he þese wer-old skolde,
a-geven þese gardos \hld\ endi sókjen imu godes ríki,
gi·faren is fader-oðil. \hld\ Þó ni gi·sah ènig firiho barno
méron minnje, \hld\ þan he þó te þem mannun gi·nam,
te þem is gódun jungaron: \hld\ gòme warhte,
sette sie swás-líko \hld\ endi im sagde filu
wároro wordo. \hld\ Skréd wester dag,
sunne te sedle. \hld\ Þó he selvo gi·bòd,
waldand mid is wordun, \hld\ hét im water dragan
hluttar te handun, \hld\ endi rés þó þe hèlago Krist,
þe gódo at þem gòmun \hld\ endi þar is jungarono þwóg
fóti mid is folmun \hld\ endi suarf sie mid is fanon aftar,
druknide sie diur-líka. \hld\ Þó wið is drohtin sprak
Símon Petrus: \hld\ „ni þunkid mi þit sómi þing“, kwað he,
„fró mín þe gódo, \hld\ þat þu míne fóti þwahes
mid þem þínun hèlagun handun.“ \hld\ Þó sprak imu eft is hérro an·gegin,
waldand mid is wordun: \hld\ „ef þu is willjan ni haves“, kwað he,
„te ant·fáhanne, \hld\ þat ik þíne fóti þwahe
þurh su·lika minnja, \hld\ só ik þesun óðrun mannun hér
dóm þurh diurða, \hld\ þan ni haves þu ènigan dèl mid mi
an heven-ríkja.“ \hld\ Hugi warð þó gi·wendid
Símon Petruse: \hld\ „þu hava þi selvo gi·wald“, kwað he,
„frò mín þe gódo, \hld\ fóto endi hando
endi mínes hòvdes só sama, \hld\ handun þínun,
þiadan, te þwahanne, \hld\ te þiu þak ik móti þína forð
huldi hębbjan \hld\ endi heven-ríkjes
su·lik gi·dèli, \hld\ só þu mi, drohtin, wili
far·geven þurh þína gódi.“ \hld\ Jungaron Kristes,
þene ambaht-skępi \hld\ erlos þolodun,
þegnos mid gi·þuldeon, \hld\ só hwat só im iro þiodan dede,
mahtig þurh þea minnja, \hld\ endi mènde imu al méra þing
firihon te gi·frummjenne. \hld\ friðu-barn godes
geng imu þó eft gi·sittjen \hld\ under þat ge·síðo folk
endi im sagda filu lang-samna rád. \hld\ Warð eft lioht kuman,
morgen te mannun. \hld\ Mahtigne Krist
gróttun is jungaron endi frágodun, \hld\ hwar sie is gòma þó
an þemu wíh-dage \hld\ wirkjen skoldin,
hwar he weldi halden \hld\ þea hèlagon tídi
selvo mid is ge·síðun. \hld\ Þó he sie sókjen hét,
þea gumon Hjerusalem: \hld\ „só gi þan gangan kumad“, kwað he,
„an þea burg innan \hld\ —þar is braht mikil,
męgin-þiodo gi·mang—, \hld\ þar mugun gi ènan man sehan
an is handun dragen \hld\ hluttres watares
ful mid folmun. \hld\ Þemu gi folgon skulun
an só hwi-like gardos, \hld\ só gi ina gangan gi·sehat,
ia gi þan þemu hérron, \hld\ þe þie hovos égi,
selvon sęggjad, \hld\ þat ik iu sende þarod
te gi·garuwenne mína gòma. \hld\ Þan tògid he iu èn gód-lík hús,
hòhan soleri, \hld\ þe is bi·hangen al
fagarun fratahun. \hld\ Þar gi frummjen skulun
werd-skępi mínan. \hld\ Þar bium ik wiskumo
selvo mid mínun ge·síðun.“ \hld\ Þó wurðun sán aftar þiu
þar te Hjerusalem \hld\ jungaron Kristes
forð-ward an ferdi, \hld\ fundun all só he sprak
word-tèkạn wár: \hld\ ni was þes gi·wand ènig.
Þar gerewidun sie þea gòma. \hld\ Warð þe godes sunu,
hèlag drohtin \hld\ an þat hús kuman,
þar sie þe land-wíse \hld\ léstjen skoldun,
ful-gangan godes gi·bode, \hld\ al só Judeono was
éo endi ald-sidu \hld\ an ér-dagun.
Gi·wèt imu þó an þemu ávande \hld\ alo-waldand Krist
an þene sęli sittjen; \hld\ hét þar is ge·síðos te imu
twelivi gangan, \hld\ þea im gi·triwiston
an iro mód-sevon \hld\ manno wárun
bi wordun endi bi wísun: \hld\ wisse imu selvo
iro hugi-skęfti \hld\ hèlag drohtin.
Grótte sie þó ovar þem gòmun: \hld\ „gern bium ik swíðo“, kwað he,
„þat ik samad mid iu \hld\ sittjen móti,
gòmono neoten, \hld\ Judeono paskha
dèljen mid iu só diurjun. \hld\ Nu ik iu iuwes drohtines skal
willjon sęggjan, \hld\ þat ik an þesaro wer-oldi ni mót
mid mannun mér \hld\ móses an·bíten
furður mid firihun, \hld\ ér þan gi·fullod wirðid
himilo ríki. \hld\ Mi is an handun nu
wíti endi wunder-kwále, \hld\ þea ik for þesumu werode skal,
þolon for þesaru þiodu.“ \hld\ Só he þó só te þem þegnun sprak,
hèlag drohtin, \hld\ só warð imu is hugi dróvi,
warð imu gi·sworken sevo, \hld\ endi eft te þem ge·síðun sprak,
þe gódo te þem is jungarun: \hld\ „hwat, ik iu godes ríki“, kwað he,
„gi·hét himiles lioht, \hld\ endi gi mi hold-líko
iuwan þegạn-skępi. \hld\ Nu ni willjat gi a·þengjan só,
ak wenkjat þero wordo. \hld\ Nu sęggju ik iu te wáran hér,
þat wili iuwar twelivjo èn \hld\ trewana swíkan,
wili mi far·kòpon \hld\ undar þit kunni Judeono,
gi·sęlljen wiðer siluvre, \hld\ endi wili imu þar sink niman,
diurje mèðmos, \hld\ endi geven is drohtin wið þiu,
holdan hérran. \hld\ Þat imu þoh te harme skal,
werðan te wítje; \hld\ be þat he þea wurdi far·sihit
endi he þes arvedjes \hld\ endi skawot,
þan wèt he þat te wáran, \hld\ þat imu wári wóðjera þing,
bętera mikilu, \hld\ þat he gio gi·boran ni wurði
libbjendi te þesumu liohte, \hld\ þan he þat lòn nimid,
uvil arvedi \hld\ in·wid-rádo.“
Þó bi·gan þero erlo ge·hwi-lik \hld\ te óðrumu skawon,
sorgondi sehan; \hld\ was im sèr hugi,
hriwig umbi iro herta: \hld\ gi·hòrdun iro hérron þó
gorn-word sprekan. \hld\ Þea gumon sorgodun,
hwi-likan he þero twelivjo \hld\ te þiu tęlljen weldi,
skuldigna skaðon, \hld\ þat he habdi þea skattos þar
ge·þingod at þeru þiod. \hld\ Ni was þero þegno ènigumu
su·likes in·widdjes \hld\ óði te gehanne,
mèn-gi·þáhtio \hld\ —ant·suok þero manno ge·hwi-lik—,
wurðun alle an forhtun, \hld\ frágon ne gi·dorstun,
ér þan þó ge·bóknide \hld\ bar-wirðig gumo,
Símon Petrus \hld\ —ne gi·dorste it selvo sprekan—
te Johanne þemu gódon: \hld\ he was þemu godes barne
an þem dagun \hld\ þegno liovost,
mèst an minnjun \hld\ endi móste þar þó an þes mahtiges Kristes
barme restjen \hld\ endi an is breostun lag,
hlinode mid is hòvdu: \hld\ þar nam he só manag hèlag ge·rúni,
diapa gi·þáhti, \hld\ endi þó te is drohtine sprak,
be·gan ina þó frágon: \hld\ „hwe skal þat, fró mín, wesen“, kwað he,
„þat þi far·kòpon wili, \hld\ kuningo ríkjost,
undar þínaro fíundo folk? \hld\ Ús wári þes firi-wit mikil,
waldand, te witanne.“ \hld\ Þó habde eft is word garu
hèljando Krist: \hld\ „seh þi, hwemu ik hér an hand geve
mínes móses for þesun mannun: \hld\ þe haved mèn-gi·þáht,
birid bittran hugi; \hld\ þe skal mi an banono ge·wald,
fíundun bi·felhen, \hld\ þar man mínes ferhes skal,
aldres áhtjen.“ \hld\ Nam he þó aftar þiu
þes móses for þem mannun \hld\ endi gaf is þemu mèn-skaðen,
Judase an hand \hld\ endi imu te·gegnes sprak
selvo for þem is ge·síðun \hld\ endi ina sniumo hét
faran fan þemu is folke: \hld\ „frumi só þu þęnkis“, kwað he,
„dó þat þu duan skalt: \hld\ þu ni maht bi·dernjen leng
willjon þínan. \hld\ Þiu wurd is at handun,
þea tídi sind nu gi·náhid.“ \hld\ Só þó þe treu-logo
þat mós ant·feng \hld\ endi mid is múðu anbét,
só afgaf ina þó þiu godes kraft, \hld\ gramon in ge·witun
an þene lík-hamon, \hld\ lèða wihti,
warð imu Satanas \hld\ sèro bi·tengi,
hardo umbi is herte, \hld\ síður ine þiu helpe godes
far·lét an þesumu liohte. \hld\ Só is þena liudjo wé,
þe só undar þesumu himile skal \hld\ hérron wehslon.
Gi·wèt imu þó út þanen \hld\ in·widjas gern
Judas gangan: \hld\ habde imu grimmen hugi
þegạn wið is þiodan. \hld\ Was þó iu þiustri naht,
swíðo gi·sworken. \hld\ Sunu drohtines
was ima at þem gòmun forð \hld\ endi is jungarun þar
waldand wín endi bròd \hld\ wíhide bèðju,
hèlagode heven-kuning, \hld\ mid is handun brak,
gaf it undar þem is jungarun \hld\ endi gode þankode,
sagde þem ǫ́·lát, \hld\ þe þar al gi·skóp,
wer-old endi wunnja, \hld\ endi sprak word manag:
„gi·lòvjot gi þes liohto“, \hld\ kwað he, „þat þit is mín lík-hamo
endi mín blód só same: \hld\ givu ik iu hér bèðju samad
etan endi drinkan. \hld\ Þit ik an erðu skal
gevan endi geotan \hld\ endi iu te godes ríkje
lòsjen mid mínu lík-hamen \hld\ an líf èwig,
an þat himiles lioht. \hld\ Gi·huggjat gi simlun,
þat gi þiu ful-gangan, \hld\ þiu ik an þesun gòmun dón;
márjad þit for męnegi: \hld\ þit is mahtig þing,
mid þius skulun gi iuwomu drohtine \hld\ diuriða frummjen,
habbjad þit mín te gi·hugdjun, \hld\ hèlag biliði,
þat it eldi-barn \hld\ aftar léstjen,
waron an þesaru wer-oldi, \hld\ þat þat witin alle,
man ovar þesan middil-gard, \hld\ þat it is þurh mína minnja gi·duan
hérron te huldi. \hld\ Ge·huggjad gi simlun,
hweo ik iu hér ge·biudu, \hld\ þat gi iuwan bróðer-skępi
fasto frummjad: \hld\ habbjad ferhtan hugi,
minnjod iu an iuwomu móde, \hld\ þat þat manno barn
ovar irmin-þiod \hld\ alle far·standen,
þat gi sind gegnungo \hld\ jungaron míne.
Ók skal ik iu kúðjen, \hld\ hwó hér wili kraftag fíund,
hettjand heru-grim, \hld\ umbi iuwan hugi niusjen,
Satanas selvo: \hld\ he kumid iuwaro seolono herod
frókno fréson. \hld\ Simlun gi fasto te gode
berad iuwa breost-gi·þáht: \hld\ ik skal an iuwaru bedu standen,
þat iu ni mugi þe mèn-skaðo \hld\ mód ge·twíflean;
ik fulléstiu iu wiðer þemu fíunde. \hld\ Ók kwam he herod giu fréson mín,
þoh imu is willjon hér \hld\ wiht ne gi·stódi,
lioves an þemu mínumu lík-hamon. \hld\ Nu ni willju ik iu leng helen,
hwat iu hér nu sniumo skal \hld\ te sorgu gi·standen:
gi skulun mi ge·swíkan, \hld\ ge·síðos míne,
iuwes þegạn-skępjes, \hld\ ér þan þius þiustrje naht
liudi far·líða \hld\ endi eft lioht kume,
morgan te mannun.“ \hld\ Þó warð mód gumon
swíðo gi·sworken \hld\ endi sèr hugi,
hriwig umbi iro herte \hld\ endi iro hérron word
swíðo an sorgun. \hld\ Símon Petrus þó,
þegạn wið is þiodan \hld\ þríst-wordun sprak
bi huldi *wið is hérron: \hld\ „þoh þi all þit hęliðo folk“, kwat-hie,
„gi·swíkan þína gi·síðos, \hld\ þoh ik sinnon mid þi
at allon þarạvon \hld\ þolojan willju.
Ik biun garo sinnon, \hld\ ef mi god látið,
þat ik an þínon ful-léstje \hld\ fasto gi·stande;
þoh sia þi an karkarjes \hld\ klústron hardo,
þesa liudi bi·lúkan, \hld\ þoh ist mi luttil tweho,
ne ik an þem bęndjon mid þi \hld\ bídan willje,
liggjan mid þi só lieven; \hld\ ef sia þínes líves þan
þuru ęggja níð \hld\ áhtjan willjad,
fró mín þie guodo, \hld\ ik givu mín ferah furi þik
an wápno spil: \hld\ nis mi werð iowiht
te bi·míðanne, \hld\ só lango só mi mín warod
hugi endi hand-kraft.“ \hld\ Þuo sprak im eft is hérro an·gegin:
„hwat, þu þik bi·wánis“, \hld\ kwat-hie, „wissaro trewono,
þrístero þingo: \hld\ þu havis þegnes hugi,
willjon guodan. \hld\ Ik mag þi sęggjan, hwó it þoh gi·werðan skal,
þat þu wirðis só wèk-muod, \hld\ þoh þu nu ni wánjes só,
þat þu þínes þiadnes te naht \hld\ þríwo far·lógnis
ér hano-krádi endi kwiðis, \hld\ þak ik þín hérro ni sí,
ak þu far·manst mína mund-burd.“ \hld\ Þuo sprak eft þie man an·gegin:
„ef it gio an wer-oldi“, \hld\ kwat-hie, „gi·werðan muosti,
þat ik samad midi þi \hld\ sweltan muosti,
dòjan diur-líko, \hld\ þan ne wurði gio þie dag kuman,
þat ik þín far·lógnidi, \hld\ lievo drohtin,
gerno for þeson Juðeon.“ \hld\ Þuo kwáðun alla þia jungron só,
þat sia þar an þem þingon mid im \hld\ þoljan weldin
Þuo im eft mid is wordon gi·bòd \hld\ waldand selvo,
hér hevan-kuning, \hld\ þat sia im ni lietin iro hugi twífljan,
hiet þat sia ni weldin[...] \hld\ diopa gi·þáhti:
„ne druovie iuwa herta \hld\ þuru iuwes drohtines word,
ne forohtjat te filo: \hld\ ik skal fader úsan
selvan suokjan \hld\ endi iu sęndjan skal
fan hevan-ríkje \hld\ hèlagna gèst:
þie skal iu eft gi·fruofrjan \hld\ endi te frumu werðan,
manon iu þero mahlo, \hld\ þie ik iu manag hębbju
wordon gi·wísid. \hld\ Hie givit iu gi·wit an briost,
lust-sama lèra, \hld\ þat gi léstjan forð
þiu word endi þiu werk, \hld\ þia ik iu an þesaro wer-oldi gi·bòd.“
A-rés im þuo þe ríkjo \hld\ an þemo rakode innan,
nęrjendo Krist \hld\ endi gi·wèt im nahtes þanan
selvo mid is gi·síðon: \hld\ sèrago gengun
swíðo gornondja \hld\ jungron Kristes,
hriwig-muoda. \hld\ Þuo hie im an þena hòhan gi·wèt
Oliueti-berg: \hld\ þar was hie up gi·wuno
gangan mid is jungron. \hld\ Þat wissa Judas wel,
balo-húgdig man, \hld\ hwand hie was oft an þem berege mid im.
Þar gruotta þie godes suno \hld\ iúgron sína:%TODO: Check iúgron.
„gi sind nu só druovja“, \hld\ kwat-hie, „nu gi mínan dòð witun;
nu gornonð gi endi griotand, \hld\ endi þesa Juðeon sind an luston,
mendit þius męnigi, \hld\ sindun an iro muode fráha,
þius wer-old ist an wunnjon. \hld\ Þes wirðit þoh gi·wand kuman
sniumo tulgo: \hld\ þan wirðit im sèr hugi,
þan mornjat sia an iro móde, \hld\ endi gi mendjan skulun
after te éwon-dage, \hld\ hwand gio endi ni kumið,
iuwes wellíves gi·wand: \hld\ be·þiu ne þurvun iu þius werk tregan,
hrewan mín hin-fard, \hld\ hwand þanan skal þiu helpa kuman
gumono barnon.“ \hld\ Þuo hiet hie is jungron þar
bídan uppan þemo berge, \hld\ kwað þat hie ti bedu weldi
an þiu holm-klivu \hld\ hòhor stígan;
hiet þuo þria mid im \hld\ þegnos gangan,
Jakobe endi Johannese \hld\ endi þena guodan Petruse,
þríst-muodjan þegạn. \hld\ Þuo sia mid iro þiedne samad
gerno gengun. \hld\ Þuo hiet sia þie godes suno
an berge uppan \hld\ te bedu hnígan,
hiet sia god gruotjan, \hld\ *gerno biddjan,
þat he im þero kostondero \hld\ kraft far·stódi,
wrèðaro willjon, \hld\ þat im þe wiðer-sako,
ni mahti þe mèn-skaðo \hld\ mód gi·twíflean,
iak imu þó selvo gi·hnèg \hld\ sunu drohtines
kraftag an knio-beda, \hld\ kuningo ríkjost,
forð-ward te foldu: \hld\ fader alo-þiado
gódan grótte, \hld\ gorn-wordun sprak
hriwig-líko: \hld\ was imu is hugi dróvi,
bi þeru męnniski \hld\ mód gi·hrórid,
is flésk was an forhtun: \hld\ fellun imo trahni,
dróp is diur-lík swèt, \hld\ al só dròr kumid
wallan fan wundun. \hld\ Was an ge·winne þó
an þemu godes barne \hld\ þe gèst endi þe lík-hamo:
ǫ́ðar was fúsid \hld\ an forð-wegos,
þe gèst an godes ríki, \hld\ ǫ́ðar giámar stód,
lík-hamo Kristes: \hld\ ni welde þit lioht a·geven,
ak drovde for þemu dòðe. \hld\ Simla he hreop te drohtine forð
þiu mér aftar þiu \hld\ mahtigna grótte,
hòhan himil-fader, \hld\ hèlagna god,
waldand mid is wordun: \hld\ „ef nu werðen ni mag“, kwað he,
„man-kunni ge·nerid, \hld\ ne sí þat ik mínan geve
liovan lík-hamon \hld\ for liudjo barn
te wégjanne te wundrun, \hld\ it sí þan þín willjo só,
ik willju is þan gi·koston: \hld\ ik nimu þene kelik an hand,
drinku ina þi te diurðu, \hld\ drohtin fró mín,
mahtig mund-boro. \hld\ Ni seh þu mínes hér
fléskes gi·fórjes. \hld\ Ik fullon skal
willjon þínen: \hld\ þu haves ge·wald ovar al.“
Gi·wèt imu þó gangen, \hld\ þar he ér is jungaron lét
bídan uppan þemu berge; \hld\ fand sie þat barn godes
slápen sorgandje: \hld\ was im sèr hugi,
þes sie fan iro drohtine \hld\ dèljen skoldun.
Só sind þat mód-þraka \hld\ manno ge·hwi-likumu,
þat he far·láten skal \hld\ liavane hérron,
af·geven þene só gódene. \hld\ Þó he te is jungarun sprak,
wahte sie waldand \hld\ endi wordun grótte:
„hwí willjad gi só slápen?“ \hld\ kwað he; „ni mugun samad mid mi
wakon ène tíd? \hld\ Þiu wurd is at handun,
þat it só gi·gangen skal, \hld\ só it god fader
gi·markode mahtig. \hld\ Mi nis an mínumu móde tweho:
mín gèst is garu \hld\ an godes willjan,
fús te faranne: \hld\ mín flésk is an sorgun,
letid mik mín lík-hamo: \hld\ lèð is imu swíðo
wíti te þolonne. \hld\ Ik þoh willjan skal
mínes fader ge·frummjen; \hld\ hębbjad gi fasten hugi.“
Gi·wèt imu þó eft þanan \hld\ óðer-síðu
an þene berg uppen \hld\ te bedu gangan,
mári drohtin, \hld\ endi þar só manag gi·sprak
gódoro wordo. \hld\ Godes ęngil kwam
hèlag fan himile, \hld\ is hugi fastnode,
beldide te þem bęndjun. \hld\ He was an þeru bedu simla
forð an flíte \hld\ endi is fader grótte,
waldand mid is wordun: \hld\ „ef it nu wesen ni mag“, kwað he,
„mári drohtin, \hld\ nevu ik for þit manno folk
þiod-kwále þoloie, \hld\ ik an þínan skal
willjan wonjan.“ \hld\ Gi·wèt imu þó eft þanen
sókjan is ge·síðos: \hld\ fand sie slápandje,
grótte sie gáhun. \hld\ Geng imu eft þanen
þriddjon síðu te bedu \hld\ endi sprak þiod-kuning
al þiu selvon word, \hld\ sunu drohtines,
te þemu alo-waldon fader, \hld\ só he ér dede,
manode mahtigna \hld\ manno frumana
swíðo niud-líko \hld\ nęrjando Krist,
geng imu þó eft te þem is jungarun, \hld\ grótte sie sáno:
„slápad gi endi restjad“, \hld\ kwað he. „Nu wirðid sniumo herod
kuman mid kraftu, \hld\ þe mi far·kòpot havad,
sundja lòsan gi·sald.“ \hld\ Ge·síðos Kristes
wakodun þó aftar þem wordun \hld\ endi gi·sáhun þó þat werod kuman
an þene berg uppen \hld\ brahtmu þiu mikilon,
wrèða wápan-berand. \hld\ Wísde im Judas,
gram-hugdig man; \hld\ Judeon aftar sigun,
fíundo folk-skępi; \hld\ dróg man fiur an gi·mang,
logna an lioht-fatun, \hld\ lèdde man faklon
brinnandja fan burg, \hld\ þar sie an þene berg uppan
stigun mid strídu. \hld\ Þea stędi wisse Judas wel,
hwar he þea liudi \hld\ tó lèdjan skolde.
Sagde imu þó te tèkne, \hld\ þó sie þar tó fórun
þemu folke bi·foran, \hld\ te þiu þat sie ni far·fengin þar,
erlos óðren man: \hld\ „ik gangu imu at èrist tó“, kwað he,
„kussju ine endi kwaddju: \hld\ þat is Krist selvo.
Þene gi fáhen skulun \hld\ folko kraftu,
binden ina uppan þemu berge \hld\ endi ina te burg hinan
lèdjen undar þea liudi: \hld\ he is líves havad
mid is wordun far·werkod.“ \hld\ Werod síðode þó,
antat sie te Kriste \hld\ kumane wurðun,
grim folk Judeono, \hld\ þar he mid is jungarun stód,
mári drohtin: \hld\ béd metodo-gi·skapu,
torhtero tídeo. \hld\ Þó geng imu treu-lòs man,
Judas te·gegnes \hld\ endi te þemu godes barne
hnèg mid is hòvdu \hld\ endi is hérron kwędde,
kuste ina kraftagne \hld\ endi is kwidi léste,
wísde ina þemu werode, \hld\ al só he ér mid wordun ge·hét.
Þat þolode al mid gi·þuldjun \hld\ þiodo drohtin,
waldand þesara wer-oldes \hld\ endi sprak imu mid is wordun tó,
frágode ine frókno: \hld\ „be·hwí kumis þu só mid þius folku te mi,
be·hwí lèdis þu mi só þese liudi tó \hld\ endi mi te þesare lèðan þiode sprekan,
far·kòpos mid þínu kussu \hld\ under þit kunni Judeono,
meldos mi te þesaru męnegi?“ \hld\ Geng imu þó wið þea man
wið þat werod ǫ́ðar \hld\ endi sie mid is wordun fragn,
hwene sie mid þiu ge·síðju \hld\ sókjan kwámin
só niud-liko an naht, \hld\ „so gi willjan nòd frummjen
manno hwi-likumu.“ \hld\ Þó sprak imu eft þiu męnegi an·gegin,
kwáðun þat im hèljand \hld\ þar an þemu holme uppan
ge·wísid wári, \hld\ „þe þit gi·wer frumid
Judeo liudjun \hld\ endi ina godes sunu
selvon hétid. \hld\ Ina kwámun wi sókjan herod,
weldin ina gerno bi·geten: \hld\ he is fan Galileo lande,
fan Nazareth-burg.“ \hld\ Só im þó þe nęrjendjo Krist
sagde te sǫ́ðan, \hld\ þat he it selvo was,
só wurðun þó an forhtun \hld\ folk Judeono,
wurðun under-badode, \hld\ þat sie under bak fellun
alle efno sán, \hld\ erðe gi·sóhtun,
wiðer-wardes þat werod: \hld\ ni mahte þat word godes,
þie stemnje ant·standan: \hld\ wárun þoh só strídige man,
a-hliopun eft up an þemu holme, \hld\ hugi fastnodun,
bundun briost-gi·þáht, \hld\ gi·bolgane gengun
náhor mid níðu, \hld\ anttat sie þene nęrjendjon Krist
werodo bi·wurpun. \hld\ Stódun wíse man,
swíðo gornundje \hld\ gjungaron Kristes
bi·foran þeru dereveon dádi \hld\ endi te iro drohtine sprákun:
„wári it nu þín willjo“, \hld\ kwáðun sie, „waldand fró mín,
þat sie ús hér an speres ordun \hld\ spildjen móstin
wápnun wunde, \hld\ þan ni wári ús wiht só gód,
só þat wi hér for úsumu drohtine \hld\ dóan móstin
beniðjun blèka“. \hld\ Þó gi·bolgan warð
snel swerd-þegạn, \hld\ Símon Petrus,
well imu innan hugi, \hld\ þat he ni mahte ènig word sprekan:
só harm warð imu an is hertan, \hld\ þat man is hérron þar
binden welde. \hld\ Þó he gi·bolgan geng,
swíðo þríst-mód þegạn \hld\ for is þiodan standen,
hard for is hérron: \hld\ ni was imu is hugi twífli,
blóð an is breostun, \hld\ ak he is bil a·tóh,
swerd bi sídu, \hld\ slóg imu te·gegnes
an þene furiston fíund \hld\ folmo krafto,
þat þó Malkhus warð \hld\ mákjas ęggjun,
an þea swíðaron half \hld\ swerdu gi·málod:
þiu hlust warð imu far·hawan, \hld\ he warð an þat hòvid wund,
þat imu heru-dròrag \hld\ hlear endi óre
bęni-wundun brast: \hld\ blód aftar sprang,
well fan wundun. \hld\ Þó was an is wangun skard
þe furisto þero fíundo. \hld\ Þó stód þat folk an rúm:
and-rédun im þes billes biti. \hld\ Þó sprak þat barn godes
selvo te Símon Petruse, \hld\ hét þat he is swerd dedi
skarp an skéðia: \hld\ „ef ik wið þesa skola weldi“, kwað he,
„wið þeses werodes ge·win \hld\ wíg-saka frummjen,
þan manodi ik þene márjon \hld\ mahtigne god,
hèlagne fader \hld\ an himil-ríkja,
þat he mi só managan ęngil herod \hld\ ovana sandi
wíges só wísen, \hld\ só ni mahtin iro wápan-þręki
man a·dógen: \hld\ iro ni stódi gio su·lik męgin samad,
folkes gi·fastnod, \hld\ þat im iro ferh aftar þiu
werðen mahti. \hld\ Ak it havad waldand god,
alo-mahtig fader \hld\ an ǫ́ðar gi·markot,
þat wi gi·þolojan skulun, \hld\ só hwat só ús þius þioda tó
bittres brengit: \hld\ ni skulun ús belgan wiht,
wrèðean wið iro ge·winne; \hld\ hwand só hwe só wápno níð,
grimman gèr-hęti wili \hld\ gerno frummjen,
he swiltit imu \hld\ eft swerdes ęggjun,
dóit im bi·dròregan: \hld\ wi mid úsun dádjun ni skulun
wiht a·werdjan.“ \hld\ Geng he þó te þemu wundon manne,
legde mid listjun \hld\ lík te·samne,
hòvid-wundon, \hld\ þat siu sán gi·hélid warð,
þes billes biti, \hld\ endi sprak þat barn godes
wið þat wrèðe werod: \hld\ „mi þunkid wunder mikil“, kwað he,
„ef gi mi lèðes wiht \hld\ léstjen weldun,
hwí gi mi þó ni fengun, \hld\ þan ik undar iuwomu folke stód,
an þemu wíhe innan \hld\ endi þar word manag
sǫ́ð-lík sagde. \hld\ Þan was sunnon skín,
diur-lik dages lioht, \hld\ þan ni weldun gi mi dóan eo·wiht
lèðes an þesumu liohte, \hld\ endi nu lèdjad mi iuwa liudi tó
an þiustrje naht, \hld\ al só man þiove dót,
þan man þene fáhan wili \hld\ endi he is ferhes havad
far·werkot, wam-skaðo.“ \hld\ werod Judeono
gripun þó an þene godes sunu, \hld\ grimma þioda,
hatandjero hóp, \hld\ hwurvun ina umbi
módag manno folk \hld\ —mènes ni sáhun—,
heftun heru-bęndjun \hld\ handi te·samne,
faðmos mid fitereun. \hld\ Im ni was su·likaro firin-kwála
þarf te gi·þolonne, \hld\ þiod-arvedjes,
te winnanne su·lik wíti, \hld\ ak he it þurh þit werod deda,
hwand he liudjo barn \hld\ lòsjen welda,
halon fan hęllju \hld\ an himil-ríki,
an þene wídon welon: \hld\ be·þiu he þes wiht ne bisprak,
þes sie imu þurh in·wid-níð \hld\ ógjan weldun.
Þó wurðun þes só malske \hld\ módag folk Judeono,
þiu héri warð þes só hrómeg, \hld\ þes sie þena hèlagon Krist
an liðo-bęndjon \hld\ lèdjan muostun,
fórjan an fiterjun. \hld\ Þie fíund eft ge·witun
fan þemu berge te burg. \hld\ Geng þat barn godes
undar þemu hęri-skępi \hld\ handun ge·bunden,
drúvondi te dale. \hld\ Wárun imu þea is diurion þó
ge·síðos ge·swikane, \hld\ al só he im ér selvo gi·sprak:
ni was it þoh be ènigaru blóði, \hld\ þat sie þat barn godes,
lioven far·létun, \hld\ ak it was só lango bi·foren
wár-sagono word, \hld\ þat it skoldi gi·werðen só:
be·þiu ni mahtun sie is be·míðan. \hld\ Þan aftar þeru męnegi gengun
Johannes endi Petrus, \hld\ þie gumon twène,
folgodun ferrane: \hld\ was im firi-wit mikil,
hwat þea grimmon Judeon \hld\ þemu godes barne,
weldin iro drohtine dóen. \hld\ Þó sie te dale kwámun
fan þemu berge te burg, \hld\ þar iro biskop was,
iro wíhes ward, \hld\ þar lèddun ina wlanke man,
erlos undar ederos. \hld\ Þar was éld mikil,
fiur an fríd-hove \hld\ þemu folke te·gegnes,
ge·warht for þemu werode: \hld\ þar gengun sie im wermien tó,
Judeo liudi, \hld\ létun þene godes sunu
bídon an bęndjun. \hld\ Was þar braht mikil,
gél-módigaro galm. \hld\ Johannes was ér
þemu héroston kúð: \hld\ be·þiu móste he an þene hof innan
þringan mid þeru þioda. \hld\ Stód allaro þegno bętsto,
Petrus þar úte: \hld\ ni lét ina þe portun ward
folgon is fróen, \hld\ ér it at is friunde a·bad,
Johannes at ènumu Judeon, \hld\ þat man ina gangan lét
forð an þene fríd-hof. \hld\ Þar kwam im èn fékni wíf
gangan te·gegnes, \hld\ þiu ènas Judeon was,
iro þeodanes þiw, \hld\ endi þó te þemu þegne sprak
magað un·wán-lík: \hld\ „hwat, þu mahtis man wesan“, kwað siu,
„gjungaro fan Galilea, \hld\ þes þe þar genower stéd
faðmun gi·fastnod.“ \hld\ Þó an forhtun warð
Símon Petrus sán, \hld\ slak an is móde,
kwað þat he þes wíves \hld\ word ni bi·konsti
ni þes þeodanes \hld\ þegạn ni wári:
méð is þó for þeru męnegi, \hld\ kwað þat he þena man ni ant·kendi:
„ni sind mi þíne kwidi kúðe“, \hld\ kwað he; was imu þiu kraft godes,
þe herdislo fan þemu hertan. \hld\ Hwarạvondi geng
forð undar þemu folke, \hld\ antat he te þemu fiure kwam;
gi·wèt ina þó warmien. \hld\ Þar im ók èn wíf bi·gan
felgjan firin-spráka: \hld\ „hér mugun gi“, kwað siu, „an iuwan fíund sehan:
þit is gegnungo \hld\ gjungaro Kristes,
is selves ge·síð.“ \hld\ Þó gengun imu sán aftar þiu
náhor níð-hwata \hld\ endi ina niud-líko
frágodun fíundo barn, \hld\ hwi-likes he folkes wári:
“ni bist þu þesoro burg-liudjo“, \hld\ kwáðun sie; „þat mugun wi an þínumu gi·bárje gi·sehan,
an þínun wordun endi an þínaru wíson, \hld\ þat þu þeses werodes ni bist,
ak þu bist galiléisk man.“ \hld\ He ni welda þes þó gehan eo·wiht,
ak stód þó endi strídda \hld\ endi starkan éð
swíð-líko ge·swór, \hld\ þat he þes ge·síðes ni wári.
Ni habda is wordo ge·wald: \hld\ it skolde gi·werðen só,
só it þe ge·markode, \hld\ þe man-kunnjes
far·wardot an þesaru wer-oldi. \hld\ Þó kwam imu ók an þemu warve tó
þes mannes mág-wini, \hld\ þe he ér mid is mákjo giheu, %NOTE: giheu checked.
swerdu þiu skarpon, \hld\ kwað þat he ina sáhi þar
an þemu berge uppan, \hld\ „þar wi an þemu bóm-gardon
hérron þínumu \hld\ hendi bundun,
fastnodun is folmos.“ \hld\ He þó þurh forhtan hugi
for·lógnide þes is lioves hérron, \hld\ kwað þat he weldi wesan þes líves skolo,
ef it mahti ènig þar \hld\ irmin-manno
gi·sęggjan te sǫ́ðan, \hld\ þat he þes ge·síðes wári,
folgodi þeru ferdi. \hld\ Þó warð an þena formon síð
hano-krád af·haven. \hld\ Þó sah þe hèlago Krist,
barno þat bętste, \hld\ þar he ge·bunden stóð,
selvo te Símon Petruse, \hld\ sunu drohtines
te þemu erle ovar is ahsla. \hld\ Þó warð imu an innan sán,
Símon Petruse \hld\ sèr an is móde,
harm an is hertan \hld\ endi is hugi dróvi,
swíðo warð imu an sorgun, \hld\ þat he ér selvo ge·sprak:
gi·hugde þero wordo þó, \hld\ þe imu ér waldand Krist
selvo sagda, \hld\ þat he an þeru swartan naht
ér hano-krádi \hld\ is hérron skoldi
þríwo far·lógnjen. \hld\ Þes þram imu an innan mód
bittro an is breostun, \hld\ endi geng imu þó gi·bolgan þanen
þe man fan þeru męnigi \hld\ an mód-karu,
swíðo an sorgun, \hld\ endi is selves word,
wam-skęfti weop, \hld\ antat imu wallan kwámun
þurh þea hert-kara \hld\ hète trahni,
blódage fan is breostun. \hld\ He ni wánde þat he is mahti gi·bótjen wiht,
firin-werko furður \hld\ efþa te is fráhon kuman,
hérron huldi: \hld\ nis ènig hęliðo só ald,
þat io mannes sunu \hld\ mér gi·sáhi
is selves word \hld\ sèrur hrewan, %NOTE: sèrur hrewan checked.
karon efþa kúmien: \hld\ „wola krafteg god“, kwað he,
þat ik hębbju mi só for·werkot, \hld\ só ik mínaro wer-oldes ni þarf
ǫ́·lát sęggjan. \hld\ Ef ik nu te aldre skal
huldjo þínaro \hld\ endi heven-ríkjas,
þeoden, þolojan, \hld\ þan ni þarf mi þes ènig þank wesan,
liovo drohtin, \hld\ þat ik io te þesumu liohte kwam.
Ni bium ik nu þes wirðig, \hld\ waldand fró mín,
þat ik under þíne jungaron \hld\ gangan móti,
þus sundig under þíne ge·síðos: \hld\ ik iro selvo skal
míðan an mínumu móde, \hld\ nu ik mi su·lik mèn ge·sprak.“
Só gornode \hld\ gumono bętsta,
hrau im só hardo, \hld\ þat he habde is hérren þó
leoves far·lógnid. \hld\ Þan ni þurvun þes liudjo barn,
weros wundrojan, \hld\ be·hwí it weldi god,
þat só lioven man \hld\ lèð gi·stódi,
þat he só hón-líko \hld\ hérron sínes
þurh þera þiwun word, \hld\ þegno snellost,
far·lógnide só lioves: \hld\ it was al bi þesun liudjun gi·duan,
firiho barnun te frumu. \hld\ He welde ina te furiston dóan,
hérost ovar is híwiski, \hld\ hèlag drohtin:
lét ina ge·kunnon, \hld\ hwi-like kraft havet
þe męnniska mód \hld\ áno þe maht godes;
lét ina ge·sundjon, \hld\ þat he síðor þiu bet
liudjun gi·lòvdi, \hld\ hwó liof is þar
manno gi·hwi-likumu, \hld\ þan he mèn ge·frumit,
þat man ina a·láte \hld\ lèðes þinges,
sakono endi sundjono, \hld\ só im þó selvo dede
heven-ríki god \hld\ harm-ge·wurhti.
Be þiu nis mannes bág \hld\ mikilun bi·þervi,
hagu-staldes hróm: \hld\ ef imu þiu helpe godes
ge·swíkid þurh is sundjon, \hld\ þan is imu sán aftar þiu
breost-hugi blóðora, \hld\ þoh he ér bi·hét spreka,
hrómje fan is hildi \hld\ endi fan is hand-krafti,
þe man fan is męgine. \hld\ Þat warð þar an þemu márjon skín,
þegno bętston, \hld\ þó imu is þiodanes gi·swèk
hèlag helpe. \hld\ Be·þiu ni skoldi hrómjen man
te swíðo fan imu selvon, \hld\ hwand imu þar swíkid oft
wán endi willjo, \hld\ ef imu waldand god,
hér heven-kuning \hld\ herte ni sterkit.
Þan béd allaro barno bętst, \hld\ bendi þolode
þurh man-kunni. \hld\ Hwurvun ina managa umbi
Judeono liudi, \hld\ sprákun gelp mikil,
habdun ina te hoska, \hld\ þar he gi·heftid stód,
þolode mid ge·þuldjun, \hld\ só hwat só imu þiu þiod deda,
liudi lèðes. \hld\ Þó warð eft lioht kuman,
morgan te mannun. \hld\ Manag samnoda
heri Judeono: \hld\ habdun im hugi wulvo,
in·wid an innan. \hld\ Warð þar éo-sago
an morgan-tíd \hld\ manag gi·samnod
irri endi èn-hard, \hld\ in·widjas gern,
wrèðes willjan. \hld\ Gengun im an warf samad
rinkos an rúna, \hld\ bi·gunnun im rádan þó,
hwó sie ge·wísadin \hld\ mid wár-lòsun,
mannun mèn-ge·witun \hld\ an mahtigna Krist
te gi·sęggjanne sundja \hld\ þurh is selves word,
þat sie ina þan te wunder-kwálu \hld\ wégjan móstin,
a-dèljen te dòðe. \hld\ Sie ni mahtun an þemu dage finden
só wrèð ge·wit-skępi, \hld\ þat sie imu wíti be·þiu
a-dèljen gi·dorstin \hld\ efþa dòð frummjen,
lívu bi·lòsjen. \hld\ Þó kwámun þar at latstan forð
an þena warf wero \hld\ wár-lòse man
twène gangan \hld\ endi bi·gunnun im tęlljen an,
kwáðun þat sie ina selvon \hld\ sęggjan gi·hòrdin,
þat he mahti te·werpen \hld\ þena wíh godes,
allaro húso hòhost \hld\ endi þurh is hand-męgin,
þurh is ènes kraft \hld\ up a·rihtjen
an þriddjon daga, \hld\ só is elkor ni þorfti be·þíhan man.
He þagoda endi þoloda: \hld\ ni sprak imu io þiu þiod só filu,
þea liudi mid luginun, \hld\ þat he it mid lèðun an·gegin
wordun wráki. \hld\ Þó þar undar þemu werode a·rés
balu-hugdig man, \hld\ biskop þero liudjo,
þe furisto þes folkes \hld\ endi frágode Krist
iak ina be imu selvon bi·swór \hld\ swíðon éðun,
grótte ina an godes namon \hld\ endi gerno bad,
þat he im þat gi·sagdi, \hld\ ef he sunu wári
þes libbjendjes godes: \hld\ „þes þit lioht ge·skóp,
Krist kuning èwig. \hld\ Wi ni mugun is ant·kiennjen wiht
ne an þínun wordun ni an þínun werkun.“ \hld\ Þó sprak imu eft þe wáro an·gegin,
þe gódo godes sunu: \hld\ „þu kwiðis it for þesun Judeon nu,
sǫ́ð-líko segis, \hld\ þat ik it selvo bium.
Þes ni gi·lòvjad mi þese liudi: \hld\ ni willjad mi for·látan be·þiu;
ni sind im mín word wirðig. \hld\ Nu sęggju ik iu te wárun þoh,
þat gi noh skulun sittjen gi·sehan \hld\ an þe swíðaron half godes
márjan mannes sunu, \hld\ an męgin-krafte
þes alo-walden fader, \hld\ endi þanan eft kuman
an himil-wolknun herod \hld\ endi allumu hęliðo kunnje
mid is wordun a·dèljen, \hld\ al só iro ge·wurhti sind.“
Þo balg ina þe biskop, \hld\ habde bittren hugi,
wrèðida wið þemu worde \hld\ endi is gi·wádi slét,
brak for is breostun: \hld\ „nu ni þurvun gi bídan leng“, kwað he,
„þit werod ge·wit-skępjes, \hld\ nu im su·lik word farad,
mèn-spráka fan is múðe. \hld\ Þat gi·hòrid hér nu manno filu,
rinko an þesumu rakude, \hld\ þat he ina só ríkjan telit,
gihid þat he god sí. \hld\ Hwat willjad gi Judeon þes
a-dèljen te dóme? \hld\ Is he dòðes nu
wirðig be su·likun wordun?“ \hld\ Þat werod al ge·sprak,
folk Judeono, \hld\ þat he wári þes ferhes skolo,
wítjes só wirðig. \hld\ Ni was it þoh be is ge·wurhtiun gi·dóen,
þat ine þar an Hjerusalem \hld\ Judeo liudi,
sunu drohtines \hld\ sundja lòsen
a-dèldun te dòðe. \hld\ Þó was þero dádjo hróm
Judeo liudjun, \hld\ hwat sie þemu godes barne mahtin
só haftemu mèst, \hld\ harmes ge·frummjen.
Be·wurpun ina þó mid werodu \hld\ endi ina an is wangon slógun,
an is hleor mid iro handun \hld\ —al was imu þat te hoske gi·dóen—,
felgidun imu firin-word \hld\ fíundo męnegi,
bismer-spráka. \hld\ Stód þat barn godes
fast under fíundun: \hld\ wárun imu is faðmos ge·bundene,
þolode mid gi·þuldjun, \hld\ só hwat só imu þiu þioda tó
bittres bráhte: \hld\ ni balg ina n·eo·wiht
wið þes werodes ge·win. \hld\ Þó námon ina wrèðe man
só gi·bundanan, \hld\ þat barn godes,
endi ina þó lèddun, \hld\ þar þero liudjo was,
þere þiade þing-hús. \hld\ Þar þegạn manag
hwurvun umbi iro hęri-togon. \hld\ Þar was iro hérron bodo
fan Rúmu-burg, \hld\ þes þe þó þes ríkjas gi·weld:
kumen was he fan þemu kèsure, \hld\ gi·sendid was he undar þat kunni Judeono
te rihtjenne þat ríki, \hld\ was þar rád-gevo:
Pilatus was he hèten; \hld\ he was fan Ponteo lande
knósles kennit. \hld\ Habde imu kraft mikil,
an þemu þing-húse \hld\ þiod gi·samnod,
an warf weros; \hld\ wár-lòse man
a-gávun þó þena godes sunu, \hld\ Judeo liudi,
under fíundo folk, \hld\ kwáðun þat he wári þes ferhes skolo,
þat man ina wítnodi \hld\ wápnes ęggjun,
skarpun skúrun. \hld\ Ni welde þiu skole Judeono
þringan an þat þing-hús, \hld\ ak þiu þiod úte stód,
mahlidun þanen wið þea męnegi: \hld\ ni weldun an þat gi·mang faren,
an ęli-landige man, \hld\ þat sie þar un·reht word,
an þemu dage dervjes wiht \hld\ a·dèljan ne gi·hòrdin,
ak kwáðun þat sie im só hluttro \hld\ hèlaga tídi,
weldin iro paskha halden. \hld\ Pilatus ant·feng
at þem wam-skaðun \hld\ waldandes barn,
sundja lòsen. \hld\ Þó an sorgun warð
Judases hugi, \hld\ þó he a·gevan gi·sah
is drohtin te dòðe, \hld\ þó bi·gan imu þiu dád aftar þiu
an is hugja hrewan, \hld\ þat he habde is hérron ér
sundja lòsen gi·sald. \hld\ Nam imu þó þat siluvar an hand,
þrí-tig skatto, \hld\ þat man imu ér wið is þiodane gaf,
geng imu þó te þem Judiun \hld\ endi im is grimmon dád,
sundjon sagde, \hld\ endi im þat siluvar bód
gerno te a·gevanne: \hld\ „ik hębbju it só grio-líko“, kwað he,
„mínes drohtines \hld\ dròru gi·kòpot,
só ik wèt þat it mi ni þíhit.“ \hld\ Þiod Judeono
ni weldun it þó ant·fáhan, \hld\ ak hétun ina forð aftar þiu
umbi su·lika sundja \hld\ selvon ahton,
hwat he wið is fráhon \hld\ ge·frumid habdi:
„þu sáhi þi selvo þes“, \hld\ kwaðun sie; „hwat wili þu þes nu sóken te ús?
Ne wít þu þat þesumu werode!“ \hld\ Þó gi·wèt imu eft þanan
Judas gangan \hld\ te þemu godes wíhe
swíðo an sorgun \hld\ endi þat siluvar warp
an þena alah innan, \hld\ ne gi·dorste it ègan leng;
fór imu þó só an forhtun, \hld\ só ina fíundo barn
módage manodun: \hld\ habdun þes mannes hugi
gramon under-gripanen, \hld\ was imu god a·bolgan,
þat he imu selvon þó \hld\ símon warhte,
hnèg þó an heru-sél \hld\ an hinginna,
warag an wurgil \hld\ endi wíti ge·kòs,
hard hęllje ge·þwing, \hld\ hèt endi þiustri,
diap dòðes dalu, \hld\ hwand he ér umbi is drohtin swèk.
Þan béd þat barn godes \hld\ —bendi þolode
an þemu þing-húse—, \hld\ hwan ér þiu þiod under im,
erlos èn-wordje \hld\ alle wurðin,
hwat sie imu þan te ferah-kwálu \hld\ frummjan weldin.
Þó þar an þem bęnkjun a·rés \hld\ bodo kèsures
fan Rúmu-burg \hld\ endi geng imu wið þat ríki Judeono
módag mahljen, \hld\ þar þiu męnigi stód
aftar þemu hove hwarvon: \hld\ ni weldun an þat hús kuman
an þemu paskha-dage. \hld\ Pilatus bi·gan
frókno frágon \hld\ ovar þat folk Judeono,
mid hwiu þe man habdi \hld\ morðes gi·skuldit,
wítjes gi·werkot: \hld\ „be hwí gi imu só wrèðe sind,
an iuwomu hugja hótje?“ \hld\ Sie kwáðun þat he im habdi harmes só filu,
lèðes gi·léstid: \hld\ „ni gávin ina þesa liudi þi,
þar sie ina ér bi·foran \hld\ uvilan ni wissin,
wordun far·warhten. \hld\ He havat þeses werodes só filu
far·lèdid mid is lèrun \hld\ —endi þesa liudi merrid,
dóit im iro hugi twífljen—, \hld\ þat wi ni mótun te þemu hove kèsures
tinsi gelden; \hld\ þat mugun wi ina gi·tęlljen an
mid wáru ge·wit-skępi. \hld\ He sprikid ók word mikil,
kwiðit þat he Krist sí, \hld\ kuning ovar þit ríki,
be·gihit ina só gròtes.“ \hld\ Þó im eft te·gegnes sprak
bodo kèsures: \hld\ „ef he só bar-líko“, kwað he,
„under þesaru męnigi \hld\ mèn-werk frumid,
ant·fáhad ina þan eft under iuwe folk-skępi, \hld\ ef he sí is ferhes skolo,
endi imu só a·dèljad, \hld\ ef he sí dòðes werð,
só it an iuwaro aldrono \hld\ éo ge·biode.“
Sie kwáðun þó, þat sie ni móstin \hld\ manno nig·ènumu
an þea hèlagon tíd \hld\ te hand-banon,
werðen mid wápnun \hld\ an þemu wíh-dage.
Þó wende ina fan þemu werode \hld\ wrèð-hugdig man,
þegạn kèsures, \hld\ þe ovar þea þioda was
bodo fan Rúmu-burg—: \hld\ hét imu þó þat barn godes
náhor gangan \hld\ endi ina niud-líko,
frágoda frókno, \hld\ ef he ovar þat folk kuning
þes werodes wári. \hld\ Þó habde eft is word garu
sunu drohtines: \hld\ „hweðer þu þat fan þi selvumu sprikis“, kwað he,
„þe it þi óðre hér \hld\ erlos sagdun,
kwáðun umbi mínan kuning-duom?“ \hld\ Þó sprak eft þe kèsures bodo
wlank endi wrèð-mód, \hld\ þar he wið waldand Krist
reðjode an þem rakude: \hld\ „ni bium ik þeses ríkjes hinan“, kwað he,
„Gjudeo liudjo, \hld\ ni gadoling þín,
þesaro manno mág-wini, \hld\ ak mi þi þius męnigi bi·falah,
a-gávun þi þína gadulingos mi, \hld\ Judeo liudi,
haftan te handun. \hld\ Hwat havas þu harmes gi·duan,
þat þu só bittro skalt \hld\ bendi þolojan,
kwalm undar þínumu kunnje?“ \hld\ Þó sprak imu eft Krist an·gegin,
hélendero bętst, \hld\ þar he gi·heftid stód
an þemu rakude innan: \hld\ „nis mín ríki hinan“, kwað he,
„fan þesaru wer-old-stundu. \hld\ Ef it þoh wári só,
þan wárin só stark-móde \hld\ wiðer stríd-hugi,
wiðer grama þioda \hld\ jungaron míne,
só man mi ni gávi \hld\ Judeo liudjun,
hettendjun an hand \hld\ an heru-bęndjun
te wégjanne te wundrun. \hld\ Te þiu warð ik an þesaru wer-oldi gi·boran,
þat ik ge·wit-skępi giu \hld\ wáres þinges
mid mínun kumiun kúðdi. \hld\ Þat mugun ant·kęnnjen wel
þe weros, þe sind fan wáre kumane: \hld\ þe mugun mín word far·standen,
gi·lòvjen mínun lèrun.“ \hld\ Þó ni mahte lasteres wiht
an þem barne godes \hld\ bodo kèsures,
findan féknea word, \hld\ þat he is ferhes be·þiu
skuldig wári. \hld\ Þó geng he im eft wið þea skola Judeono
módag mahljen \hld\ endi þeru męnigi sagde
ovar hlust mikil, \hld\ þat he an þemu hafton manne
su·lika firin-spráka \hld\ finden ni mahti
for þem folk-skipje, \hld\ só he wári is ferhes skolo,
dòðes wirðig. \hld\ Þan stódun dol-móde
Judeo liudi \hld\ endi þane godes sunu
wordun wrógdun: \hld\ kwáðun þat he gi·wer èrist
be·gunni an Galileo lande, \hld\ „endi ovar Judeon fór
herod-wardes þanan, \hld\ hugi twíflode,
manno mód-sevon, \hld\ só he is morðes werð,
þat man ina wítnoje \hld\ wápnes ęggjun,
ef eo man mid su·likun dádjun mag \hld\ dòðes ge·skuldjen.“
Só wrógdun ina mid wordun \hld\ werod Judeono
þurh hótjan hugi. \hld\ Þó þe hęri-togo,
slíð-módig man \hld\ sęggjan gi·hòrde,
fan hwi-likumu kunnje was \hld\ Krist a·fódid,
manno þe bętsto: \hld\ he was fan þeru márjan þiadu,
þe gódo fan Galilea-lande; \hld\ þar was gum-skępi
eðiljero manno; \hld\ Erodes bi·held þar
kraftagne kuning-dóm, \hld\ só ina imu þe kèsur far·gaf,
þe ríkjo fan Rúmu, \hld\ þat he þar rehto ge·hwi-lik
ge·frumidi undar þemu folke \hld\ endi friðu lésti,
dómos a·dèldi. \hld\ He was ók an þemu dage selvo
an Hjerusalem \hld\ mid is gum-skępi,
mid is werode at þemu wíhe: \hld\ só was iro wíse þan,
þat sie þar þia hèlagun tíd \hld\ haldan skoldun,
paskha Judeono. \hld\ Pilatus gi·bòd þó,
þat þena hafton man \hld\ hęliðos námin
só gi·bundanan, \hld\ þat barn godes,
hét þat sie ina Erodese, \hld\ erlos bráhtin
haften te handun, \hld\ hwand he fan is hęri-skępi was,
fan is werodes ge·wald. \hld\ Wígand frumidun
iro hérron word: \hld\ hèlagne Krist
fórdun an fiterjun \hld\ for þena folk-togun,
allaro barno bętst, \hld\ þero þe io gi·boren wurði
an liudjo lioht; \hld\ an liðu-bęndjun geng,
antat sie ina bráhtun, \hld\ þar he an is bęnkja sat,
kuning Erodes: \hld\ umbi·hwarf ina kraft wero,
wlanke wígandos: \hld\ was im willjo mikil,
þat sie þar selvon Krist \hld\ gi·sehan móstin:
wándun þat he im sum tèkạn \hld\ þar tògjan skoldi,
mári endi mahtig, \hld\ só he managun dede
þurh is god-kundi \hld\ Judeo *liudjon.
Frágoda ina þuo þie folk-kuning \hld\ firi-wit-líko
managon wordon, \hld\ wolda is muod-sevon
forð undar-findan, \hld\ hwat hie te frumu mohti
mannon gi·markon. \hld\ Þan stuod mahtig Krist,
þagoda endi þoloda: \hld\ ne wolda þem þied-kuninge,
Erodese ne is erlon \hld\ ant·swór gevan
wordo nig·ènon. \hld\ Þan stuod þiu wrèða þiod,
Judeo liudi \hld\ endi þena godes suno
wurrun endi wruogdun, \hld\ anþat im warð þie wer-old-kuning
an is huge huoti \hld\ endi all is hęri-skipi,
far·muonstun ina an iro muode: \hld\ ne ant·kendun maht godes,
himiliskan hérron, \hld\ ak was im iro hugi þiustri,
baluwes gi·blandan. \hld\ Barn drohtines
iro wrèðun werk, \hld\ word endi dádi
þuru òd-muodi \hld\ all gi·þoloda,
só hwat só sia im tionono þuo \hld\ tuogjan woldun.
Sia hietun im þuo te hoske \hld\ hwít gi·wádi
umbi is liði lęggjan, \hld\ þiu mér hie wurði þem liudjon þar,
jungron te gamne. \hld\ Judeon faganodun,
þuo sia ina te hoske \hld\ hębbjan gi·sáhun,
erlos ovar-muoda. \hld\ Þuo senda ina eft þanan
Erodes se kuning \hld\ an þat óðer folk;
a-lèdjan hiet ina lungra mann, \hld\ endi lastar sprákun,
felgidun im firin-word, \hld\ þar hie an feteron geng
bi·hlagan mid hosku: \hld\ ni was im hugi twífli,
neva hie it þuru òd-muodi \hld\ all gi·þoloda;
ne welda iro uvilun word \hld\ idug-lònon,
hosk endi harmkwidi. \hld\ Þuo bráhtun sia ina eft an þat hús innan,
an þia palenkja uppan, \hld\ þar Pilatus was
an þero þing-stędi. \hld\ Þegnos a·gávun
barno þat besta \hld\ banon te handon
sundi-lòsjan, \hld\ só hie selvo gi·kòs:
welda manno barn \hld\ morðes a·tuomjan,
nęrjan af nòdi. \hld\ Stuodun níð-hwata,
Judeon far þem gast-sęlje: \hld\ habdun sia gramono barn,
þia skola far·skundid, \hld\ þat sia ne be·skrivun iowiht
grimmera dádjo. \hld\ Þuo gi·wèt im gangan þarod
þegạn kèsures \hld\ wið þia þiod sprekan,
hard hęri-togo: \hld\ „hwat, gi mi þesan haftan mann“, kwat-hie,
„an þesan sęli sendun \hld\ endi selvon an·budun,
þat hie iuwes werodes só filo \hld\ a·werdit habdi,
far·lèdid mid is lèron. \hld\ Nu ik mid þeson liudon ni mag,
findan mid þius folku, \hld\ þat hie is ferahes sí
furi þesaro skolu skuldig. \hld\ Skín was þat hiudu:
Erodes mohta, \hld\ þie iuwan éo bikan,
iuwaro liudo land-reht, \hld\ hie ni mahta is líves gi·fréson,
þat hie hier þuru èniga sundja te dage \hld\ sweltan skoldi,
líf far·látan. \hld\ Nu willju ik ina for þeson liudjon hier
gi·þróon mid þingon, \hld\ þrístion wordun,
buotjan im is briost-hugi, \hld\ látan ina brúkan forð
ferahes mid firjon.“ \hld\ Folk Judeono
hreopun þuo alla samad \hld\ hlúdero stemnu,
hietun flít-líko \hld\ ferahes áhtjan
Krist mid kwalmu \hld\ endi an krúki slahan,
wégjan te wundron: \hld\ „hie mid is wordon havit
dòðes gi·skuldid: \hld\ sagit þat hie drohtin sí,
gegnungo godes suno. \hld\ Þat hie a·geldan skal,
in·wid-spráka, \hld\ só is an úson éwe gi·skrivan,
þat man su·lika firin-kwidi \hld\ ferahu kòpo.“
Þuo warð þie an forahton, \hld\ þie þes folkes gi·weld,
mikilon an is muode, \hld\ þuo hie gi·hòrda þia man sprekan,
þat sia ina selvon \hld\ sęggjan gi·hòrdin,
gehan fur þem gum-skipe, \hld\ þat hie wári godes suno.
Þuo hwarf im eft þie hęri-togo \hld\ an þat hús innan
te þero þing-stędi, \hld\ þrístion wordon
gruotta þena godes suno \hld\ endi frágoda, hwat hie gumono wári:
„hwat bist þu manno?“ kwat-hie. \hld\ „Te hwí þu mi só þínan muod hilis,
dernis diop-gi·þáht? \hld\ Wèst þu þat it all an mínon duome stéd %TODO: Check stéd.
umbi þínes líves gi·lagu? \hld\ Mi þi hębbjat þesa liudi far·gevan,
werod Judeono, \hld\ þat ik gi·waldan muot
só þik te spildjanne \hld\ an speres orde,
só ti kwęlljanne an krúkjum, \hld\ só kwikan látan,
só hweðer só mi selvon \hld\ suotera þunkit
te gi·frummjanne mid mínu folku.“ \hld\ Þuo sprak eft þat friðu-barn godes:
„wèst þu þat te wáron“, \hld\ kwat-hie, „þat þu gi·wald ovar mik
hębbjan ni mohtis, \hld\ ne wári þat it þi hèlag god
selvo far·gávi? \hld\ Ók hębbjat þia sundjono mér,
þia mik þi bi·fulhun \hld\ þuru fíond-skipi,
gi·saldun an símon haftan.“ \hld\ Þuo welda ina síð after þiu
gram-hugdig man \hld\ gerno far·látan,
þegạn kèsures, \hld\ þar hie is havdi for þero þioda gi·wald;
ak sia weridun im þena willjon \hld\ wordu gi·hwi-liku,
kunni Judeono: \hld\ „ne bist þu“, kwáðun sia, „þes kèsures friund,
þínon hérren hold, \hld\ ef þu ina hinan látis
síðon gi·sundon: \hld\ þat þi noh te soragan mag,
werðan te wíte, \hld\ hwand só hwe só su·lik word sprikit,
a-havið ina só hòho, \hld\ kwiðit þat hie hębbjan mugi
kuning-duomes namon, \hld\ ne sí þat ina im þie kèsur geve,
hie wirrid im is weruld-ríki \hld\ endi is word far·hugid,
far·man ina an is muode. \hld\ Be·þiu skalt þu su·lik mèn wrekan,
hosk-word manag, \hld\ ef þu umbi þínes hérren ruokis,
umbi þínes fròhon friund-skipi, \hld\ þan skalt þu ina þiu ferhu beniman.“
Þuo gi·hòrda þie hęri-togo \hld\ þia héri Juðeono
þrégjan fan is þiodne; \hld\ þuo hie far þero þing-stędi geng
selvo gi·sittjan, \hld\ þar gi·samnod was
só mikil warf werodes, \hld\ hiet waldand Krist
lèdjan for þia liudi. \hld\ Langoda Judeon,
hwan ér sia þat hèlaga barn \hld\ hangon gi·sáwin,
kwęlan an krúkje; \hld\ sia kwáðun þat sia kuning óðran
ne havdin undar iro hęri-skipje, \hld\ nevan þena héran késar
fan Rúmu-burg: \hld\ „þie havit hier ríki over ús.
Be·þiu ni skalt þu þesan far·látan; \hld\ hie havit ús só filo lèðes gi·sprokan,
far·duan havit hie im mid is dádjon. \hld\ Hie skal dòð þolon,
wíti endi wundạr-kwála.“ \hld\ Werod Judeono
só manag mis-lík þing \hld\ an mahtigna Krist
sagdun te sundjun. \hld\ Hie swígondi stuod
þuru óð-muodi, \hld\ ne ant·wordida n·io·wiht
wið iro wrèðun word: \hld\ wolda þesa wer-old alla
lòsjan mid is lívu: \hld\ bi·þiu liet hie ina þia lèðun þiod
wégjan te wundron, \hld\ all só iro willjo geng:
ni wolda im opan-líko \hld\ allon kúðjan
Judeo liudjon, \hld\ þat hie was god selvo;
hwand wissin sia þat te wáron, \hld\ þat hie su·lika gi·wald havdi
ovar þeson middil-gard, \hld\ þan wurði im iro muod-sevo
gi·blóðit an iro brioston: \hld\ þan ne gi·dorstin sia þat barn godes
handon ant·hrínan: \hld\ þan ni wurði hevan-ríki,
ant·lokan liohto mèst \hld\ liudjo barnon.
Be·þiu méð hie is só an is muode, \hld\ ne lét þat manno folk
witan, hwat sia warahtun. \hld\ Þiu wurd náhida þuo,
mári maht godes \hld\ endi middi dag,
þat sia þia ferah-kwála \hld\ frummjan skoldun.
Þan lag þar ók an bęndjon \hld\ an þero burg innan
èn ruof ręgin-skaðo, \hld\ þie habda under þem ríke só filo
morðes gi·rádan \hld\ endi man-slahta gi·frumid,
was mári męgin-þiof: \hld\ ni was þar is gi·mako hwergin;
was þar ók bi sínon \hld\ sundjon gi·hęftid,
Barrabas was hie hètan; \hld\ hie after þem burgjon was
þuru is mèn-dádi \hld\ manogon gi·kúðid.
Þan was land-wísa \hld\ liudjo Judeono,
þat sia iáro gi·hwen \hld\ an godes minnja
an þem hèlagon dage \hld\ ènna haftan mann
a-biddjan skoldun, \hld\ þat im iro burges ward,
iro folk-togo \hld\ ferah far·gávi.
Þuo bi·gan þie hęri-togo \hld\ þia héri Judeono,
þat folk frágojan, \hld\ þar sia im fora stuodun,
hweðeron sia þero twejo \hld\ tuomjan weldin,
ferahes biddjan: \hld\ „þia hier an feteron sind
haft undar þeson hęri-skipje?“ \hld\ Þiu héri Judeono
habdun þuo þia aramun man \hld\ alla gi·spanana,
þat sia þemo land-skaðen \hld\ líf a·bádin,
gi·þingodin þem þiove, \hld\ þie oft an þiustrja naht
wam gi·warahta, \hld\ endi waldand Krist
kwęlidin an krúkje. \hld\ Þuo warð þat kúð ovar all,
hwó þiu þiod havda duomos a·dèlid. \hld\ Þuo skoldun sia þia dád frummjan,
háhan þat hèlaga barn. \hld\ Þat warð þem hęri-togen
síðor te sorgon, \hld\ þat hie þia saka wissa,
þat sia þuru níð-skipi \hld\ nęrjendon Krist,
hatoda þiu héri, \hld\ endi hie im hòrda te þiu,
warahta iro willjon: \hld\ þes hie wíti ant·feng,
lòn an þeson liohte \hld\ endi lang after,
wói síðor wann, \hld\ síðor hie þesa wer-old a·gaf. %NOTE: wói [sic.]
Þuo warð þas þie wrèðo gi·waro, \hld\ wam-skaðono mèst,
Satanas selvo, \hld\ þuo þiu seola kwam
Judases an grund \hld\ grimmaro hęlljun—
þuo wissa hie te wáren, \hld\ þat þat was waldand Krist,
barn drohtines, \hld\ þat þar gi·bundan stuod;
wissa þuo te wáron, \hld\ þat hie welda þesa wer-old alla
mid is henginnja \hld\ hęllja gi·þwinges,
liudi a·lòsjan \hld\ an lioht godes.
Þat was Satanase \hld\ sèr an muode,
tulgo harm an is hugje: \hld\ welda is helpan þuo,
þat im liudjo barn \hld\ líf ne binámin,
ne kwęlidin an krúkje, \hld\ ak hie welda, þat hie kwik livdi,
te þiu þat firiho barn \hld\ fernes ne wurðin,
sundjono sikura. \hld\ Satanas gi·wèt im þuo,
þar þes hęri-togen \hld\ híwiski was
an þero burg innan. \hld\ Hie þero is brúdi bi·gann,
þera idis opan-líko \hld\ un·hiuri fíond
wunder tògjan, \hld\ þat sia an word-helpon
Kriste wári, \hld\ þat hie muosti kwik libbjan,
drohtin manno \hld\ —hie was iu þan te dòðe gi·skerid—
wissa þat te wáron, \hld\ þat hie im skoldi þia gi·wald biniman,
þat hie sia ovar þesan middil-gard \hld\ só mikila ni havdi,
ovar wída wer-old. \hld\ Þat wíf warð þuo an forahton,
swíðo an sorogon, \hld\ þuo iru þiu gi·siuni kwámun
þuru þes dernjen dád \hld\ an dages liohte,
an hęlið-helme bi·helid. \hld\ Þuo siu te iru hérren anbód,
þat wíf mid iro wordon \hld\ endi im te wáren hiet
selvon sęggjan, \hld\ hwat iro þar te gi·siunjon kwam
þuru þena hèlagan mann, \hld\ endi im helpan bad,
formon is ferhe: \hld\ „ik hębbju hier só filo þuru ina
seld-líkes gi·sewan, \hld\ só ik wèt, þat þia sundjun skulun
allaro erlo gi·hwem \hld\ uvilo gi·þíhan,
só im fruokno tuo \hld\ ferahes áhtið.“
Þie segg warð þuo an síðe, \hld\ antat hie sittjan fand
þena hęri-togon \hld\ an hwarạve innan
an þem stèn-wege, \hld\ þar þiu stráta was
felison gi·fuogid. \hld\ Þar hie te is fròhon geng,
sagda im þes wíves word. \hld\ Þuo warð im wrèð hugi,
þem hęri-togen, \hld\ —hwarạvoda an innan—,
gi·blóðit briost-gi·þáht: \hld\ was im bèðjes wé,
gie þat sea ina sluogin \hld\ sundja lòsan,
gie it bi þem liudjon þuo \hld\ for·látan ne gi·dorsta
þuru þes werodes word. \hld\ Warð im gi·wendid þuo
hugi an herten \hld\ after þero héri Judeono,
te werkjanne iro willjon: \hld\ ne wardoda im nie-wiht
þia swárun sundjun, \hld\ þia hie im þar þuo selvo gi·deda.
Hiet im þuo te is handon dragan \hld\ hluttran brunnjon,
watar an wégje, \hld\ þar hie furi þem werode sat,
þwóg ina þar for þero þioda \hld\ þegạn kèsures,
hard hęri-togo \hld\ endi þuo fur þero héri sprak,
kwað þat hie ina þero sundjono þar \hld\ sikoran dádi,
wrèðero werko: \hld\ „ne willju ik þes wihtes plegan“, kwat-hie,
„umbi þesan hèlagan mann, \hld\ ak hleotad gi þes alles,
gie wordo gie werko, \hld\ þes gi im hér te wítje gi·duan.“
Þuo hreop all saman \hld\ hęri-skipi Judeono,
þiu mikila męnigi, \hld\ kwáðun þat sia weldin umbi þena man plegan
deravoro dádjo: \hld\ „fare is dròr ovar ús,
is bluod endi is baneði \hld\ endi ovar úsa barn só samo,
ovar úsa avaron þar after \hld\ —wi willjat is alles plegan“, kwaðun sia,
„umbi þena slegi selvon,— \hld\ ef wi þar èniga sundja gi·duan!“
A-gevan warð þar þuo furi þem Judeon \hld\ allaro gumono besta
hettendjon an hand, \hld\ an heru-bęndjon
narawo gi·nòdid, \hld\ þar ina níð-hwata,
fíond ant·fengun: \hld\ folk ina umbi·hwarf,
mèn-skaðono męgin. \hld\ Mahtig drohtin
þoloda gi·þuldion, \hld\ só hwat só im þiu þioda deda.
Sia hietun ina þuo filljan, \hld\ ér þan sia im ferahes tuo,
aldres áhtin, \hld\ endi im undar is ògun spiwun,
dedun im þat te hoske, \hld\ þat sia mid iro handon slógun,
weros an is wangun \hld\ endi im is gi·wádi bi·námun,
róvodun ina þia ręgin-skaðon, \hld\ ródes lakanes
dedun im eft óðer an \hld\ þuru un·huldi;
hietun þuo hòvid-band \hld\ hardaro þorno
wundron windan \hld\ endi an waldand Krist
selvon sęttjan, \hld\ endi gengun im þia gi·síðos tuo,
kwęddun ina an kuning-wísu \hld\ endi þar an knio fellun,
hnigun im mid iro hòvdu: \hld\ all was im þat te hoske gi·duan,
þoh hie it all gi·þolodi, \hld\ þiodo drohtin,
mahtig þuru þia minnja \hld\ manno kunnjes.
Hietun sia þuo wirkjan \hld\ wápnes ęggjon
hęliðos mid iro handon \hld\ hardes bómes
kraftiga krúki \hld\ endi hietun sia Kristan þuo,
sálig barn godes \hld\ selvon fuorjan,
dragan hietun sia úsan drohtin, \hld\ þar hie be·dròragad skolda
sweltan sundjono lòs. \hld\ Síðodun Judeon,
weros an willon, \hld\ lèddun waldand Krist,
drohtin te dòðe. \hld\ Þar mohta man þuo derevi þing
harmlík gi·hòrjan: \hld\ hiovandi þar after
gengun wíf mid wópu, \hld\ weros gnornodun,
þia fan Galilea mid im \hld\ gangan kwámun,
folgodun ovar ferr-wegos: \hld\ was im iro fròhon dòð
swíðo an soragan. \hld\ Þuo hie selvo sprak,
barno þat besta \hld\ endi under bak besah,
hiet þat sia ni wépin: \hld\ „ni þarf iu wiht tregan“, kwat-hie,
„mínero hin-ferdjo, \hld\ ak gi mid hofnu mugun
iuwa wrèðan werk \hld\ wópu kúmian,
tornon trahnon. \hld\ Noh wirðið þiu tíd kuman,
þat þia muoder þes \hld\ mendendja sind,
brúdi Judeono, \hld\ þem gio barn ni warð
ódan an aldre. \hld\ Þan gi iuwa in·wid skulun
grimmo an·geldan; \hld\ þan gi só gerna sind,
þat iu hier bi·hlídan \hld\ hòha bergos,
diopo be·delvan; \hld\ dòð wári iu þan allon
liovera an þeson lande \hld\ þan su·lik liudjo kwalm
te gi·þoljanne, \hld\ só hier þan þesaro þioda kumid.“
Þuo sia þar an griete \hld\ galgon rihtun,
an þem felde uppan \hld\ folk Judeono,
bóm an berege, \hld\ endi þar an þat barn godes
kwęlidun an krúkje: \hld\ slógun kald ísarn,
niwa naglos \hld\ níðon skarpa
hardo mid hamuron \hld\ þuru is hendi endi þuru is fuoti,
bittra bendi: \hld\ is blód ran an erða,
dròr fan úson drohtine. \hld\ Hie ni welda þoh þia dád wrekan
grimma an þem Judeon, \hld\ ak hie þes god fader
mahtigna bad, \hld\ þat hie ni wári þem manno folke,
þem werode þiu wrèðra: \hld\ „hwand sia ni witun, hwat sia duot“, kwat-hie.
Þuo þia wígandos \hld\ gi·wádi Kristes,
drohtines dèldun, \hld\ derevia mann,
þes ríken gi·róbi. \hld\ Þia rinkos ni mahtun
umbi þena selvon {[...]} \hld\ sam-wurdi gi·sprekan,
ér sia an iro hwarạve \hld\ hlótos wurpun,
hwi-lik iro skoldi hębbjan \hld\ þia hèlagun péda,
allaro gi·wádjo wun-samost. \hld\ Þes werodes hirdi
hiet þuo, þe hęri-togo, \hld\ ovar þem hòvde selves
Kristes an krúke skrívan, \hld\ þat þat wári kuning Judeono,
Jesus fan Nazareth-burh, \hld\ þie þar neglid stuod
an niwon galgon þuru \hld\ níð-skipi,
an bómin treo. \hld\ Þuo bádun þia liudi
þat word węndjan, \hld\ kwáðun þat hie im só an is willjon spráki,
selvo sagdi, \hld\ þat hie habdi þes gi·síðes gi·wald,
kuning wári ovar Judeon. \hld\ Þuo sprak eft þie kèsures bodo,
hard hęri-togo: \hld\ „it ist iu só ovar is hòvde gi·skrivan,
wís-líko gi·writan, \hld\ só ik it nu węndjan ni mag.“
Dádun þuo þar te wítje \hld\ werod Judeono
twèna far·talda man \hld\ an twá halva
Kristes an krúki: \hld\ lietun sia kwalm þolon
an þem warạg-trewe \hld\ werko te lòne,
lèðaro dádjo. \hld\ Þia liudi sprákun
hosk-word manag \hld\ hèlagon Kriste,
grottun ina mid gelpu: \hld\ sáwun allaro gumono þen beston
kwęlan an þemo krúkje: \hld\ „ef þu sís kuning ovar all“, kwáðun sia,
„suno drohtines, \hld\ só þu havis selvo gi·sprokan,
neri þik fan þero nòdi \hld\ endi níðes a·tuomi,
gang þi hél herod; \hld\ þan welljat an þik hęliðo barn,
þesa liudi gi·lòvjan.“ \hld\ Sum imo ók lastar sprak
swíðo gél-hert Judeo, \hld\ þar hie fur þem galgon stuod:
„wah warð þesaro wer-oldi“, \hld\ kwat-hie, „ef þu iro skoldis gi·wald ègan.
Þu sagdas þat þu mahtis an ènon dage \hld\ all te·werpan
þat hòha hús \hld\ hevan-kuninges,
stèn-werko mèst \hld\ endi eft standan gi·duon
an þriddjon dage, \hld\ só is elkor ni þorfti bi·þíhan mann
þeses folkes furðor. \hld\ Sínu hwó þu nu gi·fastnod stés,
swíðo gi·sèrid: \hld\ ni maht þi selvon wiht
balowes gi·buotjan.“ \hld\ Þuo þar ók an þem bęndjon sprak
þero þeovo óðer, \hld\ all só hie þia þioda gi·hòrda,
wrèðon wordon \hld\ —ne was is willjo guod,
þes þegnes gi·þáht—: \hld\ „ef þu sís þiod-kuning“, kwat-hie,
„Krist, godes suno, \hld\ gang þi þan fan þem krúke niðer,
slópi þi fan þem símon \hld\ endi ús samad allon
hilp endi héli. \hld\ Ef þu sís hevan-kuning,
waldand þesaro wer-oldes, \hld\ gi·duo it þan an þínon werkon skín,
mári þik fur þesaro męnigi.“ \hld\ Þuo sprak þero manno óðer
an þero hęnginna, \hld\ þar hie gi·hęftid stuod,
wan wunder-kwála: \hld\ „be·hwí wilt þu su·lik word sprekan,
gruotis ina mid gelpu? \hld\ stés þi hier an galgen haft,
gi·brókan an bóme. \hld\ Wit hier bèðja þolod
sèr þuru unka sundjun: \hld\ is unk unkero selvero dád
worðan te wítje. \hld\ Hie stéd hier wammes lòs,
allaro sundjono sikur, \hld\ só hie selvo gio
firina ni gi·frumida, \hld\ botan þat hie þuru þeses folkes nið
willendi an þesaro weruldi \hld\ wíti ant·fáhid.
Ik willju þar gi·lòvjan tuo“, \hld\ kwat-hie, „endi willju þena landes ward,
þena godes suno \hld\ gerno biddjan,
þat þu mín gi·huggjes \hld\ endi an helpun sís,
rádendero best, \hld\ þan þu an þín ríki kumis:
wes mi þan gi·náðig.“ \hld\ Þuo sprak im eft nęrjendo Krist
wordon te·gegnes: \hld\ „ik sęggju þi te wáron hier“, kwat-hie,
„þat þu noh hiudu móst \hld\ an himil-ríke
mid mi samad \hld\ sehan lioht godes,
an þemo Paradyse, \hld\ þoh þu nu an su·likoro pínu sís.“
Þan stuod þar ók Maria, \hld\ muoder Kristes,
blèk under þem bóme, \hld\ gi·sah iro barn þolon,
winnan wunder-kwála. \hld\ Ók wárun þar wíf mid iro
an só mahtiges \hld\ minnja kumana—
þan stuod þar ók Johannes, \hld\ jungro Kristes,
hriwi undar is hérren, \hld\ was im is hugi sèrag—
drúvodun fur þem dòðe. \hld\ Þar sprak drohtin Krist
mahtig te þero muoder: \hld\ „nu ik þi hier mínemo skal
jungron be·felhan, \hld\ þem þi hier gegin-ward stéd:
wis þi an is gi·síðje samad: \hld\ þu skalt ina furi suno hębbjan.“
Grótta hie þuo Johannes, \hld\ hiet þat hie iru ful-gengi wel,
minnjodi sia só mildo, \hld\ só man is muoder skal,
idis un·wamma. \hld\ Þuo hie sia an is éra ant·feng
þuru hluttran hugi, \hld\ só im is hérro gi·bòd.
Þuo warð þar an middjan dag \hld\ mahtig tèkạn,
wundạr-lík gi·waraht \hld\ ovar þesan wer-old allan,
þuo man þena godes suno \hld\ an þena galgon huof,
Krist an þat krúki: \hld\ þuo warð it kúð ovar all,
hwó þiu sunna warð gi·sworkan: \hld\ ni mahta swigli lioht
skóni gi·skínan, \hld\ ak sia skado far·feng,
þimm endi þiustri \hld\ endi só gi·þrusmod neval.
Warð allaro dago druovost, \hld\ dunkar swíðo
ovar þesan wídun weruld, \hld\ só lango só waldand Krist
kwal an þemo krúkje, \hld\ kuningo ríkost,
ant nuon dages. \hld\ Þuo þie neval ti-skrèd,
þat gi·swerk warð þuo te·swungan, \hld\ bi·gan sunnun lioht
hédron an himile. \hld\ Þuo hreop up te gode
allaro kuningo kraftigost, \hld\ þuo hie an þemo krúkje stuod
faðmon gi·fastnot: \hld\ „fader alo-mahtig“, kwat-hie,
„te hwí þu mik só far·lieti, \hld\ lievo drohtin,
hèlag hevan-kuning, \hld\ endi þína helpa dedos,
fullisti só ferr? \hld\ Ik standu under þeson fíondon hier
wundron gi·wégid.“ \hld\ Werod Judeono
hlógun is im þuo te hoske: \hld\ gi·hòrdun þena hèlagun Krist,
drohtin furi þem dòðe \hld\ drinkan biddjan,
kwað þat ina þurstidi. \hld\ Þiu þioda ne latta,
wrèða wiðar-sakon: \hld\ was im willjo mikil,
hwat sia im bittres tuo \hld\ bringan mahtin.
Habdun im un·swóti \hld\ ekid endi galla
gi·mengid þia mèn-hwaton; \hld\ stuod èn mann garo,
swíðo skuldig skaðo, \hld\ þena habdun sia gi·skerid te þiu,
far·spanan mid sprákon, \hld\ þat hie sia en èna spunsia nam,
líðo þes lèðosten, \hld\ druog it an ènon langan skafte,
gi·bundan an ènon bóme \hld\ endi deda it þem barne godes,
mahtigon te múðe. \hld\ Hie an·kenda iro mirkjun dádi,
gi·fuolda iro fégnes: \hld\ furðor ni welda
is só bittres an·bítan, \hld\ ak hreop þat barn godes
hlúdo te þem himiliskon fader: \hld\ „ik an þina hendi be·filhu“, kwat-hie,
„mínon gèst an godes willjon; \hld\ hie ist nu garo te þiu,
fús te faranne.“ \hld\ Firiho drohtin
gi·hnègida þuo is hòvid, \hld\ hèlagon áðom
liet fan þemo lík-hamen. \hld\ Só þuo þie landes ward
swalt an þem símon, \hld\ só warð sán after þiu
wundạr-tèkạn gi·waraht, \hld\ þat þar waldandes dòð
un·kweðandes só filo \hld\ ant·kęnnjan skolda,
þiadnes èn-dagon: \hld\ erða bivoda,
hrisidun þia hòhun bergos, \hld\ harda stènos kluvun,
felisos after þem felde, \hld\ endi þat féha lakan tebrast
an middjon an twè, \hld\ þat ér managan dag
an þemo wíhe innan \hld\ wundron gi·striunid
hél hangoda \hld\ —ni muostun hęliðo barn,
þia liudi skawon, \hld\ hwat under þemo lakane was
hèlages be·hangan: \hld\ þuo mohtun an þat horð sehan
Judeo liudi— \hld\ gravu wurðun gi·opanod
dòdero manno, \hld\ endi sia þuru drohtines kraft
an iro lík-hamon \hld\ libbjandi a·stuodun
up fan erðu \hld\ endi wurðun gi·ógida þar
mannon te márðu. \hld\ Þat was só mahtig þing,
þat þar Kristes dòð \hld\ ant·kęnnjan skoldun,
só filo þes gi·fuoljan, \hld\ þie gio mid firihon ne sprak
word an þesaro wer-oldi. \hld\ Werod Judeono
sáwun seld-lík þing, \hld\ ak was im iro slíði hugi
só far·hardod an iro herten, \hld\ þat þar io só hèlag ni warð
tèkạn gi·tògid, \hld\ þat sia trúodin þiu bat
an þia Kristes kraft, \hld\ þat hie kuning ovar all,
þes werodes wári. \hld\ Suma sia þar mid iro wordon gi·sprákun,
þia þes hréwes þar \hld\ huodjan skoldun,
þat þat wári te wáren \hld\ waldandes suno,
godes gegnungo, \hld\ þat þar an þem galgon swalt,
barno þat besta. \hld\ Slógun an iro briost filo
wópjandero wívo: \hld\ was im þiu wunder-kwála
harm an iro herten \hld\ endi iro hérren dòð
swíðo an sorogon. \hld\ Þan was sido Judeono,
þat sia þia haftun þuru þena hèlagon dag \hld\ hangon ni lietin
lengerun hwíla, \hld\ þan im þat líf skriði,
þiu seola besunki: \hld\ slíð-muoda mann
gengun im mid níð-skipiu náhor, \hld\ þar só be·nęglida stuodun
þeovos twèna, \hld\ þolodun bèðja
kwála bi Kriste: \hld\ wárun im kwika noh þan,
untþat sia þia grimmun \hld\ Judeo liudi
bènon be·brákon, \hld\ þat sia bèðja samad
líf far·lietun, \hld\ suohtun im lioht óðer.
Sia ni þorftun drohtin Krist \hld\ dòðes bédjan
furðor mid ènigon firinon: \hld\ fundun ina gi·faranan þuo iu:
is seola was gi·sendid \hld\ an suoðan weg,
an lang-sam lioht, \hld\ is liði kuolodun,
þat ferah was af þem fléske. \hld\ Þuo geng im èn þero fíondo tuo
an níð-hugi, \hld\ druog negilid sper
hard an is handon, \hld\ mid heru-þrummjon stak,
liet wápnes ord \hld\ wundum sníðan,
þat an selves warð \hld\ sídu Kristes
ant·lokan is lík-hamo. \hld\ Þia liudi gi·sáwun,
þat þanan bluod endi water \hld\ bèðju sprungun,
wellun fan þero wundun, \hld\ all só is willjo geng
endi hie habda gi·markod ér \hld\ manno kunnje,
firiho barnon te frumu: \hld\ þuo was it all gi·fullid só.
Só þuo gi·ségid warð \hld\ seðle náhor
hédra sunna \hld\ mid hevan-tunglon
an þem druoven dage, \hld\ þuo geng im úses drohtines þegạn
—was im glau gumo, \hld\ jungro Kristes
managa hwíla, \hld\ só it þar manno filo
ne wissa te wáron, \hld\ hwand hie it mid is wordon hal
Juðeono gum-skipje: \hld\ Joseph was hie hètan,
darnungo was hie úses drohtines jungro: \hld\ hie ni welda þero far·duanun þiod
folgon te ènigon firin-werkon, \hld\ ak hie béd im under þem folke Judeono,
hèlag himilo ríkjes— \hld\ hie geng im þuo wið þena hęri-togon mahljan,
þingon wið þena þegạn kèsures, \hld\ þigida ina gerno,
þat hie muosti a·lòsjan \hld\ þena lík-hamon
Kristes fan þemo krúkje, \hld\ þie þar gi·kwelmid stuod,
þes guoden fan þem galgen \hld\ endi an graf lęggjan,
foldu bi·felahan. \hld\ Im ni welda þie folk-togo þuo
węrnjan þes willjen, \hld\ ak im gi·wald far·gaf,
þat hie só muosti gi·frummjan. \hld\ Hie gi·wèt im þuo forð þanan
gangan te þem galgon, \hld\ þar hie wissa þat godes barn,
hréo hangondi \hld\ hérren sínes,
nam ina þuo an þero niwun ruodun \hld\ endi ina fan naglon a·tuomda,
ant·feng ina mid is faðmon, \hld\ só man is fròhon skal,
lioves lík-hamon, \hld\ endi ina an líne bi·wand,
druog ina diur-líko \hld\ —só was þie drohtin werð—,
þar sia þia stędi havdun \hld\ an ènon stène innan
handon gi·hauwan, \hld\ þar gio hęliðo barn
gumon ne bi·gruovon. \hld\ Þar sia þat godes barn
te iro land-wísu, \hld\ líko hèlgost
foldu bi·fulhun \hld\ endi mid ènu felisu be·lukun
allaro gravo guod-líkost. \hld\ Griotandi sátun
idisi arm-skapana, \hld\ þia þat all for·sáwun,
þes gumen grimman dòð. \hld\ Gi·witun im þuo gangan þanan
wópjandi wíf \hld\ endi wara námun,
hwó sia eft te þem grave \hld\ gangan mahtin:
havdun im far·sewana \hld\ soroga gi·nuogja,
mikila muod-kara: \hld\ Maria wárun sia hètana,
idisi arm-skapana. \hld\ Þuo warð ávand kuman,
naht mid neflu. \hld\ Niðfolk Judeono
warð an moragan eft, \hld\ męnigi gi·samnod,
rekidun an rúnon: \hld\ „hwat, þu wèst, hwó þit ríki was
þuru þesan ènan man \hld\ all gi·twíflid,
werod gi·worran: \hld\ nu ligid hie wundon siok,
diopa bi·dolvan. \hld\ Hie sagda simnen, þat hie skoldi fan dòðe a·standan
an þriddjan dage. \hld\ Þius þiod gi·lòvit te filo,
þit werod after is wordon. \hld\ Nu þu hier wardon hét,
ovar þem grave gòmjan, \hld\ þat ina is jungron þar
ne far·stelan an þemo stène \hld\ endi sęggjan þan, þat hie a·standan sí,
ríki fan raston: \hld\ þan wirðit þit rinko folk
mér gi·merrid, \hld\ ef sia it biginnat márjan hier.“
Þuo wurðun þar gi·skerida \hld\ fan þero skolu Judeono
weros te þero wahtu: \hld\ gi·witun im mid iro gi·wápnjon þarod
te þem grave gangan, \hld\ þar sia skoldun þes godes barnes
hréwes huodjan. \hld\ Warð þie hèlago dag
Judeono far·gangan. \hld\ Sia ovar þemo grave sátun,
weros an þero wahtun \hld\ wannom nahton,
bidun undar iro bordon, \hld\ hwan ér þie berẹhto dag
ovar middil-gard \hld\ mannon kwámi,
liudon te liohte. \hld\ Þuo ni was lang te þiu,
þat þar warð þie gèst kuman \hld\ be godes krafte,
hálag áðom \hld\ undar þena hardon stèn
an þena lík-hamon. \hld\ Lioht was þuo gi·opanod
firiho barnon te frumu: \hld\ was ferkal manag
ant·hęftid fan hęll-doron \hld\ endi te himile weg
gi·waraht fan þesaro wer-oldi. \hld\ Wánom up astuod
friðu-barn godes, \hld\ fuor im þuo þar hie welda,
só þia wardos þes \hld\ wiht ni af·swovun,
dervja liudi, \hld\ hwan hie fan þem dòðe astuod,
a-rés fan þero rastun. \hld\ Rinkos sátun
umbi þat graf útan, \hld\ Judeo liudi,
skola mid iro skildion. \hld\ Skréd forð-wardes
swigli sunnun lioht. \hld\ Síðodun idisi
te þem grave gangan, \hld\ gum-kunnjes wíf,
Mariun muni-líka: \hld\ habdun mèðmo filo
gi·sald wiðer salvum, \hld\ siluvres endi goldes,
werðes wiðer wurtjon, \hld\ só sia mahtun a·winnan mèst,
þat sia þena lík-hamon \hld\ lioves hérren,
suno drohtines, \hld\ salvon muostin,
wundun writanan. \hld\ Þiu wíf soragodun
an iro sevon swíðo, \hld\ endi suma sprákun,
hwie im þena gròtan stèn \hld\ fan þemo grave skoldi
gi·hwerevian an halva, \hld\ þe sia ovar þat hréo sáwun
þia liudi lęggjan, \hld\ þuo sia þena lík-hamon þar
be·fulhun an þemo felise. \hld\ Só þiu frí havdun
ge·gangan te þem gardon, \hld\ þat sia te þem grave mahtun
gi·sehan selvon, \hld\ þuo þar swógan kwam
ęngil þes alo-waldon \hld\ ovana fan radure,
faran an feðer-hamon, \hld\ þat all þiu folda an skian,
þiu erða dunida \hld\ endi þia erlos wurðun
an wèkan hugje, \hld\ wardos Juðeono,
bi·fellun bi þem forahton: \hld\ ne wándun ira ferah ègan,
líf langerun hwíl. \hld\ Lágun þa wardos,
þia gi·síðos sám-kwika: \hld\ sán up a·hléd
þie gròto stèn fan þem grave, \hld\ só ina þie godes ęngil
gi·hwerivida an halva, \hld\ endi im uppan þem hlèwe gi·sat
diur-lík drohtines bodo. \hld\ Hie was an is dádjon ge·lík,
an is an·siunjon, \hld\ só hwem só ina muosta undar is ògon skawon,
só berẹht endi só blíði \hld\ all só bliksmun lioht;
was im is gi·wádi \hld\ wintạr-kaldon
snéwe gi·líkost. \hld\ Þuo sáwun sia ina sittjan þar,
þiu wíf uppan þem gi·wendidan stène, \hld\ endi im fan þem wlitje kwámun,
þem idison su·lika egison te·gegnes: \hld\ all wurðun fan þem grurje
þiu frí an forahton mikilon, \hld\ furðor ne gi·dorstun
te þemo grave gangan, \hld\ ér sia þie godes ęngil,
waldandes bodo \hld\ wordon gruotta,
kwað þat hie iro árundi \hld\ all bi·kunsti,
werk endi willjon \hld\ endi þero wívo hugi,
hiet þat sia im ne and-rédin: \hld\ „ik wèt þat gi iuwan drohtin suokat,
nęrjendon Krist \hld\ fan Nazareth-burg,
þena þi hier kwęlidun \hld\ endi an krúki slógun
Judeo liudi \hld\ endi an graf lagdun
sundi-lòsjan. \hld\ Nu nist hie selvo hier,
ak hie ist a·standan iu, \hld\ endi sind þesa stędi lárja, %NOTE: a·standan] L 1r.
þit graf an þeson griote. \hld\ Nu mugun gi gangan herod
náhor mikilu \hld\ —ik wèt þat is iu ist niud sehan
an þeson stène innan—: \hld\ hier sind noh þia stędi skína,
þar is lík-hamo lag.“ \hld\ Lungra fengun
gi·bada an iro brioston \hld\ blèka idisi,
wliti-skóni wíf: \hld\ was im wil-spell mikil
te gi·hòrjanne, \hld\ þat im fan iro hérren sagda
ęngil þes alo-walden. \hld\ Hiet sia eft þanan
fan þem grave gangan endi faran \hld\ te þem jungron Kristes,
sęggjan þem is gi·síðon \hld\ suoðon wordon,
þat iro drohtin was \hld\ fan dòðe a·standan.
Hiet ók an sundron \hld\ Símon Petruse
will-spell mikil \hld\ wordon kúðjan,
kumi drohtines, \hld\ gie þat Krist selvo
was an Galileo land, \hld\ „þar ina eft is jungron skulun,
gi·sehan is gi·síðos, \hld\ só hie im ér selvo gi·sprak
wárom wordon.“ \hld\ Reht só þuo þiu wíf þanan
gangan weldun, \hld\ só stuodun im te·gegnes þar
ęngilos twèna \hld\ an ala-hwíton
wánamon gi·wádjom \hld\ endi sprákun im mid iro wordon tuo
hèlag-líko: \hld\ hugi warð gi·blóðid
þen idison an egison: \hld\ ne mahtun an þia ęngilos godes
bi þemo wlite skawon: \hld\ was im þiu wánami te strang, %NOTE: strang] L 1v.
te swíði te sehanne. \hld\ Þuo sprákun \edtext{im sán}{\Afootnote{so C; om. L}} an·gegin
waldandes bodun \hld\ endi þiu wíf frágodun,
te hwí sia Kristan þarod \hld\ kwikan mid dòdon,
suno drohtines \hld\ suokjan kwámin
ferahes fullan; \hld\ „nu gi ina ni findat hier
an þeson stèn-grave, \hld\ ak hie ist a·standan nu
an is lík-hamon: \hld\ þes gi gi·lòvjan skulun
endi gi·huggjan þero wordo, \hld\ þe hie iu te wáron oft
selvo sagda, \hld\ þan hie an iuwon ge·síðja was
an Galilea-lande, \hld\ hwó hie skoldi gi·gevan werðan,
gi·sald selvo \hld\ an sundigaro manno,
hettjandero hand, \hld\ hèlag drohtin,
þat sea ina kwęlidin \hld\ endi an krúki slógin,
dòdan gi·dádin \hld\ endi þat hie skoldi þuruh drohtines kraft
an þriddjon dage \hld\ þioda te willjan
libbjandi a·standan. \hld\ Nu havat hie all gi·léstid só,
ge·frumid mid firihon: \hld\ íljat gi nu forð hinan,
gangat gáh-líko \hld\ endi duot it þem is jungron kúð.
Hie havat sia iu fur·farana \hld\ endi ist im forð hinan
an Galileo land, \hld\ þar ina eft is jungron skulun,
gi·sehan is ge·síðos.“ \hld\ Þuo warð \edtext{sán}{\Afootnote{so L; om. C}} after þiu
þem wívon an willjon, \hld\ þat sia gi·hòrdun su·lik word sprekan,
kúðjan þia kraft godes \hld\ —wárun im só a·kumana þuo noh
gie só forahta ge·frumida—: \hld\ gi·witun im forð þanan %NOTE: forahta] L end.
fan þem grave gangan \hld\ endi sagdun þem jungron Kristes
seld-lík gi·siuni, \hld\ þar sia sorogondi
bidun su·likero buota. \hld\ Þuo wurðun ók an þia burg kumana
Judeono wardos, \hld\ þia ovar þemo grave sátun
alla langa naht \hld\ endi þes lík-hamen þar,
huodun þes hréwes. \hld\ Sia sagdun þero héri Judeono,
hwi-lika im þar and-warda \hld\ egison kwámun,
seld-lík gi·siuni, \hld\ sagdun mid wordon,
al só it gi·duan was \hld\ an þero drohtines kraft,
ni miðun an iro muode. \hld\ Þuo budun im mèðmo filo
Judeo liudi, \hld\ gold endi siluvar,
saldun im sink manag, \hld\ te þiu þat sia it ni sagdin forð,
ne máridin þero męnigi: \hld\ „ak kweðat þat iu móði hugi
an·swevidi mid slápu \hld\ endi þat þar kwámin is gi·síðos tuo,
far·stálin ina an þem stène. \hld\ Simnen wesat gi an stríde mid þiu,
forð an flíte: \hld\ ef it wirðit þem folk-togen kúð,
wi gi·helpat iu wið þena hérosten, \hld\ þat hie iu harmes wiht,
lèðes ni gi·léstid.“ \hld\ Þuo námun sia an þem liudon filo
diurero mèðmo, \hld\ dádun all só sia bi·gunnun
—ne gi·weldun iro willjon— \hld\ dádun só wído kúð
þem liudon after þem lande, \hld\ þat sia su·lika lugina woldun
a-hębbjan be þan hèlagan drohtin. \hld\ Þan was eft gi·hélid hugi
jungron Kristes, \hld\ þuo sia gi·hòrdun þiu guodun wíf
márjan þia maht godes; \hld\ þuo wárun sia an iro muode fráha,
gie im te þem grave bèðja, \hld\ Johannes endi Petrus
runnun ovast-líko: \hld\ warð ér kuman
Johannes þie guodo, \hld\ endi im ovar þem grave gi·stuod,
antat þar sán after kwam \hld\ Símon Petrus,
erl ęllan-ruof \hld\ endi im þar in gi·wèt
an þat graf gangan: \hld\ gi·sah þar þes godes barnes,
hréo-gi·wádi \hld\ hérren sínes
línin liggjan, \hld\ mid þiu was ér þie lík-hamo
fagaro bi·fangan; \hld\ lag þie fano sundar,
mit þem was þat hòvid bi·helid \hld\ hèlages Kristes,
ríkjes drohtines, \hld\ þan hie an þesaro rastu was.
Þuo geng im ók Johannes \hld\ an þat graf innan
sehan seld-lík þing; \hld\ warð im sán after þiu
ant·lokan is gi·lòvo, \hld\ þat hie wissa, þat skolda eft an þit lioht kuman
is drohtin diur-líko, \hld\ fan dòðe a·standan
up fan erðu. \hld\ Þuo gi·witun im eft þanan
Johannes endi Petrus, \hld\ endi kwámun þia jungron Kristes,
þia gi·síðos te·samne. \hld\ Þan stuod sèrag-muod
èn þera idiso \hld\ óðer-síðu
griotandi ovar þem grave, \hld\ was iro iámar muod—
Maria was þat Magdalena—, \hld\ was iro muod-gi·þáht,
sevo mit sorogon gi·blandan, \hld\ ne wissa hwarod siu sókjan skolda
þena hérron, þar iro wárun at þia helpa gi·langa. \hld\ Siu ni mohta þuo hofnu a·wísan,
þat wíf ni mahta wóp for·látan: \hld\ ne wissa hwarod siu sia węndjan skolda;
gi·merrid wárun iro þes muod-gi·þáhti. \hld\ Þuo gi·sah siu þena mahtigan þar
Kriste standan, \hld\ þuoh siu ina kúð-líko
ant·kęnnjan ni mohti, \hld\ ér þan hie ina kúðjan welda,
sęggjan þat hie it selvo wári. \hld\ Hie frágoda hwat siu só sèro bi·wiepi,
só harmo mid hèton trahnin. \hld\ Siu kwað, þat siu umbi iro hérron ni wissi
te wáren, hwarod hie werðan skoldi: \hld\ „ef þu ina mi gi·wísan mohtis,
fró mín, ef ik þik frágon gi·dorsti, \hld\ ef þu ina hier an þeson felise gi·námis,
wísi ina mi mid wordon þínon: \hld\ þan wári mi allaro willjono mèsta,
þat ik ina selvo gi·sáhi.“ \hld\ Sia ni wissa, þat sia þie suno drohtines
gruotta mid gódaro sprákun: \hld\ siu wánda þat it þie gardari wári,
hof-ward hérren sínes. \hld\ Þuo gruotta sia þie hèlago drohtin,
bi namen nęrjendero best: \hld\ siu geng im þuo náhor sniumo,
þat wíf mid willjon guodan, \hld\ ant·kenda iro waldand selvan,
míðan siu is þuru þia minnja ni wissa: \hld\ welda ina mid iro mundon grípan,
þiu féhmia an þena folko drohtin, \hld\ novan þat iro friðu-barn godes
werida mid wordon sínon, \hld\ kwað þat siu ina mid wihti ni mósti
handon ant·hrínan: \hld\ „ik ni stèg noh“, kwat-hie, „te þem himiliskon fader;
ak íli þu nu ofst-líko \hld\ endi þem erlon kúði,
bruoðron mínon, \hld\ þat ik úser béðero fader
ala-waldan, \hld\ iuwan endi mínan
suoð-fastan god \hld\ suokjan willju.“
Þat wíf warð þuo an wunnon, \hld\ þat siu muosta su·likan willjon kúðjan,
sęggjan fan im gi·sundon: \hld\ warð sán garo
þiu idis an þat árundi \hld\ endi þem erlon bráhta,
will-spel weron, \hld\ þat siu waldand Krist
gi·sundan gi·sáwi, \hld\ endi sagda hwó he iru selvo gi·bòd
torohtero tèkno. \hld\ Sia ni weldun gi·trúojan þuo noh
þes wíbes wordon, \hld\ þat siu su·lik will-spel bráhte
gegnungo fan þemo godes suno, \hld\ ak sia sátun im iámor-muoda,
hęliðos hriwonda. \hld\ Þuo warð þie hèlago Krist
eft opan-líko \hld\ óðer-síðu,
drohtin gi·tògid, \hld\ síðor hie fan dòðe a·stuod,
þan wívon an willjon, \hld\ þat hie im þar an wege muotta.
kwędda sia kúð-líko, \hld\ endi sia te is kneohon hnigun,
fellun im tó fuoton. \hld\ Hie hét þat sia forahtan hugi
ne bárin an iro brioston: \hld\ „ak gi mínon bruoðron skulun
þesa kwidi kúðjan, \hld\ þat sia kuman after mi
an Galileo land; \hld\ þar ik im eft te·gegnes biun.“
Þan fuorun im ók fan Hjerusalem \hld\ þero jungrono twèna
an þem selvon daga \hld\ sán an morgan,
erlos an iro árundi: \hld\ weldun im te Emaus
þat kastel suokan. \hld\ Þuo bi·gunnun im kwidi managa
under þem weron wahsan, \hld\ þar sia after þem wege fuorun,
þem hęliðon umbi iro hérron. \hld\ Þuo kwam im þar þie hèlago tuo
gangandi godes suno. \hld\ Sia ni mahtun ina garo-líko
ant·kęnnan kraftigna: \hld\ hie ni welda ina þuo noh kúðjan te im;
was im þoh an iro gi·síðje samad endi frágoda, \hld\ umbi hwi-lika sia saka sprákin:
„hwí gangat gi só gornondja?“ \hld\ kwat-hie; „Ist ink jámer hugi,
sevo soragono full.“ \hld\ Sia sprákun im sán an·gegin,
þia erlos and-wurdi: \hld\ „te hwí þu þes éskos só“, kwáðun sia;
„bist þi fan Hjerusalem \hld\ Judeono folkas
hèlagumu gèste \hld\ fan heven-wange,
mid þem gròtun godes kraft.“ \hld\ Nam is jungaron þó,
erlos góde, \hld\ lèdda sie út þanan,
antat he sie bráhte \hld\ an Bethania;
þar hóf he is hendi up \hld\ endi hèlegoda sie alle,
wíhida sie mid is wordun. \hld\ Gi·wèt imo up þanan,
sóhta imo þat hòha himilo ríki \hld\ endi þena is hèlagon stól:
sitit imo þar \hld\ an þea swíðron half godes,
alo-mahtiges fader \hld\ endi þanan all ge·sihit
waldandeo Krist, \hld\ só hwat só þius wer-old be·havet.
Þó an þeru selvon stędi \hld\ ge·síðos góde
te bedu fellun \hld\ endi im eft te burg þanan
þar te Hjerusalem \hld\ jungaron Kristes
fórun faganondi: \hld\ was im fráh-mod hugi,
wárun im þar at þemu wíhe. \hld\ Waldandes kraft
[...]
