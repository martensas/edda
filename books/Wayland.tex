\bookStart{Lay of Wayland}[Vǫlundarkviða]
\def\thisBookCode{Volundarkvida}

\begin{flushright}%
\textbf{Dating} \parencite{Sapp2022}: C10th (0.428)–early C11th (0.475)

\textbf{Meter:} \Fornyrdislag%
\end{flushright}%

\section{Introduction}

The \textbf{Lay of Wayland} (\Volundarkvida) is a psychologically complex, well wrought poem.  The verses themselves are preserved only in \Regius, but the beginning of the foreword is found on the very last page of \AM.

\Volundarkvida\ is a narrative poem telling the story of Wayland the Smith.  Wayland was one of the most famous figures in Germanic legend, and independent versions of his tale are found in Germany, England, and Iceland.  In his most archetypal form, Wayland (ON \emph{Vǫlundr}, OE \emph{Wéland} or \emph{Wélund}, MHG \emph{*Welent}) is an uncannily talented smith who is taken captive and hamstrung by the greedy tyrant Nithad (ON \emph{Níðuðr}, OE \emph{Níþhad}, MHG \emph{*Nídung}), who forces him to make jewels for him and his family.  Wayland plans a cruel revenge against the king: he murders his two sons and rapes his daughter, Beadhild (ON \emph{Bǫðvildr}, OE \emph{Beaduhild}, MHG \emph{*Botil}), making her pregnant.  At last, he escapes in a self-made flight suit, having regaining his mobility.

Wayland gets his revenge on the whole royal household. He murders Nithad’s two young sons (affectionately, his “bear-cubs”) and thus ends his male lineage. Likewise he defangs Nithad’s “cunning wife” (she is never called anything else) by reducing her once powerful counsels to cold words; and finally he rapes Beadhild, depriving her of her maidenhood and value in marriage. They are thus reduced to the same state of complete powerlessness as he himself experienced, something clearly seen in the repetition of the adjective \emph{viljalauss} ‘powerless’; in st. 12 it describes Wayland after he wakes in shackles, but in st. 31 Nithad uses it to refer to his own mental state after the deaths of his sons. This sense of hopelessness concludes the poem in Beadhild’s haunting words: “I nowise knew withstand him; I nowise could withstand him.”

From the other versions of the story it is known that Beadhild gave birth to a son, Woody (OE \emph{Wudga}, \ThidreksSaga\ \emph{Viðga}, in Danish ballads \emph{Vidrik Verlandsøn}). He went on to become a great hero, and in the later heroic ballads by far eclipses his father. His birth seems heavily foreshadowed by Wayland forcing Nithad to swear an oath in st. 33, but he is nowhere directly mentioned in the poem, probably for artistic reasons.

Apart from this lay there is one other telling of the full story, namely the Strand of Wayland the Smith in \ThidreksSaga. While written in Old Norse, it is clear from the proper names and content that it is based on German sources (probably heroic ballads). Thus the native form \emph{Vǫlundr} is replaced with the Low German \emph{Velent} [\emph{sic}], \emph{Níðuðr} with \emph{Niðungr}. Interestingly there is a note within it showing that the native form was still known, namely about “Velent, the excellent smith, whom Warrings (\emph{væringjar}) call Wayland (\emph{Vǫlundr})”. Apparently Wayland was so famous that “all men seem to praise his workmanship so, that the maker of any smith’s work which is made better than other works, is called a Wayland (\emph{Vǫlundr}) with regards to workmanship.”

Far more stark than minor differences of language is that of tone. The psychological complexity and tension of the older redaction is almost entirely gone: Wayland is no longer a mysterious wild man, but a chivalrous knight who can escape from any peril through his ingenuity and craftmanship. He is not kidnapped out of Nithad’s greed, nor hamstrung out of the suspicion of his cruel wife, but rather a loyal servant of Nithad’s, banished from the kingdom after defending himself against the king’s corrupt steward, and hamstrung after being caught attempting to poison the king’s food in revenge.

Most frustratingly the personality of Beadhild is entirely expulged. She is the anonymous “king’s daughter”, an unnamed maiden (\emph{jungfrú}, a borrowing from Low German) who is peacefully seduced by Wayland and quickly falls in love with him. Likewise the person of Nithad’s cunning wife is completely gone, and the murder of his sons no longer ends his lineage, since he has another, older son who survives him and takes over the kingdom. Wayland still flies away laughing after telling Nithad what he has done, but only four years (his son with Beadhild is three years old) later reconciliates with Nithad’s son, retrieves Beadhild and their son and lives a long life as a famous craftsman.

Thus, by the time of the \ThidreksSaga\ the old story of Wayland had been heavily distorted, a tragic victim of chivalric sensibilities.  This younger version does not have any high literary value, but is of course still of interest since it shows the wide reception and variation of the narrative.

Finally there are also traces of the story in the Anglo-Saxon tradition, where it is alluded to in both \Waldere\ and \Deor, the latter of which particularly emphasising the powerlessness felt by Wayland and Beadhild (thus being much closer in spirit to the present poem than to \ThidreksSaga). Parts of the narrative are depicted on the early C8th Frank’s casket, where it is as prominent as the depiction of the Adoration of the Magi—a true testament to the weight with which it was regarded within that culture.

\section{From Wayland (\emph{Frá Vǫlundi})}

\bpg\bpa\mssnote{\Regius~18r/4, \AM~6v/26}Níðuðr hét konungr í Svíþjóð.
Hann átti tvá sonu ok eina dóttur; \edtrans{hon hét}{she was called}{\Afootnote{so \Regius; ok hét hon ‘and she was called’ \AM}} Bǫðvildr.
Brǿðr \edtrans{vǫ́ru}{were}{\Afootnote{so \AM; om. \Regius}} þrír, synir Finna konungs. Hét einn Slagfiðr, annarr Egill, þriði Vǫlundr.
Þeir skriðu ok veiddu dýr. Þeir kvǫ́mu í Úlfdali ok gerðu \edtext{sér þar hús.
Þar er vatn, er heitir Úlfsjár.
Snemma of morgin fundu þeir á vatsstrǫndu konur þrjár, ok spunnu lín. Þar váru hjá þeim álftarhamir þeira; þat váru valkyrjur.
Þar váru tvę́r dǿtr Hlǫðvés konungs: Hlaðguðr svanhvít ok Hervǫr alvitr. In þriðja var Ǫlrún \edtext{Kjárs dóttir af Vallandi}{\lemma{Kjárs [\dots] af Vallandi ‘Choser of Walland’}\Bfootnote{I.e. “Cæsar of Rome”; a legendary form of the Roman emperor. See Index.}}.
Þeir hǫfðu þę́r heim til skála með sér. Fekk Egill Ǫlrúnar, en Slagfiðr Svanhvítrar, en Vǫlundr Alvitrar.
Þau bjuggu sjau vetr. Þá flugu þę́r at vitja víga ok kvǫ́mu eigi aptr.
Þá skreið Egill at leita Ǫlrúnar, en Slagfiðr leitaði Svanhvítrar, en Vǫlundr sat í Úlfdǫlum.
Hann var hagastr maðr, svá at menn viti í fornum sǫgum.
Níðuðr konungr lét hann hǫndum taka, svá sem hér er um kveðit:}{\lemma{sér þar hús \dots\ um kveðit ‘for themselves houses \dots\ sung of’}\Afootnote{so \Regius; om. (due to loss of the following foll. in the ms.) \AM}}\epa

\bpb Nithad was a king called in Sweden.
He had two sons and one daughter; she was called Beadhild.
Three brothers were there; the sons of a king of the Finns. One was called Slayfinn, the other Eyel, the third Wayland.
They fared on skis and hunted wild beasts. They came into the Wolfdales and made for themselves houses there.
There is a lake there which is called the Wolfsea.
Early in the morning they found on the lake-shore three women, and they span linen. There were by them their swan-\inx[C]{hame}[hames]; those were Walkirries.
There were two daughters of king Ladwigh: Ladguth Swanwhite and Harware Elwight. The third was Alerune, daughter of \inx[P]{Choser} of \inx[G]{Walland}.
The men took the women to their halls with them.  Eyel got Alerune, and Slayfinn Swanwhite, and Wayland the Elwight.
The couples lived there for seven winters; then the women left to attend battles, and did not come back.
Then Eyel fared on skis to search for Alerune, but Slayfinn searched for Swanwhite—but Wayland stayed in the Wolfdales.
He was the most skilled craftsman whom men know of in the ancient saws. King Nithad had him taken, as it is here sung of:\epb\epg

\sectionline

\section{The Lay of Wayland}

\bvg\bva\mssnote{\Regius~18r/19}\alst{M}ęyjar flugu sunnan \hld\ \edtrans{\alst{M}yrk-við}{Mirkwood}{\Bfootnote{A great border forest, surely referenced for its association with the war-ravaged lands of the Gots and Huns; a natural environment for \inx[G]{Walkirries}.}} í gǫgnum &
\edtrans{\alst{a}l-vitr}{elwights}{\Bfootnote{“Strange beings, foreign wights”, reflecting a hypothetical \emph{*alja-wihtiz}.}} \alst{u}ngar, \hld\ \edtrans{\alst{ø}r-lǫg drýgja;}{fulfill orlay}{\Bfootnote{That is, to fulfill their preordained destinies, and act according to their innate nature as described in P1 and st. 3.  \textcite[103]{MCR2005} and some other editors see these words as a sign of English influence and translate \emph{drýgja ør-lǫg} as “engage in war”, considering \emph{ør-lǫg} a semantic borrowing from the OE \emph{or-leg} which is taken to mean the same as Dutch \emph{oorlog} ‘war’.  This is unnecessary; ON \emph{ør-lǫg} otherwise means ‘fate, destiny’, and so may its OE cognate as seen by the equivalent phrase found in l. 29 of a poem on the Christian Doomsday (TODO?), where a man going to Hell for his sins \emph{þǫnne â tó ealdre \hld\ or·leg dreógeð} ‘then for ever and ever [he] suffers his orlay’.}} &
þę́r á \alst{s}ę́var-strǫnd \hld\ \alst{s}ęttusk at hvílask, &
\alst{d}rósir suð-rǿnar \hld\ \alst{d}ýrt lín spunnu.\eva

\bvb Maidens flew from the south through \inx[L]{Mirkwood} \\
—young elwights—to fulfill \inx[C]{orlay}. \\
They on the lake-shore set down to rest; \\
the southern ladies span costly linen.\evb\evg


\bvg\bva\mssnote{\Regius~18r/21}\alst{Ęi}n nam þęira \hld\ \alst{Ę}gil at vęrja &
\alst{f}ǫgr mę́r \alst{f}ira \hld\ \alst{f}aðmi ljósum; &
ǫnnur vas \alst{S}vanhvít, \hld\ \alst{s}van-fjaðrar dró, &
\edtext{[...]}{\Bfootnote{A line mentioning Slayfinn has probably been lost here.}} &
en hin \alst{þ}riðja \hld\ \alst{þ}ęira systir &
varði \edtrans{\alst{h}vítan}{white}{\Bfootnote{Pale skin being a sign of noble ancestry; cf. 17/3.}} \hld\ \alst{h}als Vǫlundar.\eva

\bvb One of them took to embrace Eyel \\
—the fair maiden among men—in her pale bosom. \\
Second was Swanwhite; her swan-feathers she rustled, \\
{[...]} \\
And the third sister among them \\
embraced the white throat of Wayland.\evb\evg


\bvg\bva\mssnote{\Regius~18r/24}\alst{S}ǫ́tu \alst{s}íðan \hld\ \alst{s}jau vetr at þat, &
en hinn \alst{á}tta \hld\ \alst{a}llan þrǫ́ðu, &
en hinn \alst{n}íunda \hld\ \alst{n}auðr of skilði, &
\alst{m}ęyjar fýstusk \hld\ á \alst{m}yrkvan við, &
\alst{a}l-vitr \alst{u}ngar \hld\ \alst{ø}r-lǫg drýgja.\eva

\bvb They stayed then seven winters after that, \\
and all the eighth they yearned, \\
and the ninth did need divorce them. \\
The maidens longed for the Mirky Wood: \\
the young elwights, to fulfill orlay.\evb\evg


\bvg\bva\mssnote{\Regius~18r/26}Kom þar af \alst{v}ęiði \hld\ \alst{v}eðr-ęygr skyti &
\edtext{Vǫlundr \alst{l}íðandi \hld\ of \alst{l}angan veg,}{\lemma{Vǫlundr \dots\ veg ‘Wayland \dots\ way’}\Afootnote{emend. based on st. 9/3–4; om. \Regius}} &
\alst{S}lagfiðr ok Ęgill, \hld\ \alst{s}ali fundu auða, &
gingu \alst{ú}t ok \alst{i}nn \hld\ ok \alst{u}mb sǫ́usk.\eva

\bvb Came there from the hunt the stormy-eyed shooter: \\
Wayland passing over a long way. \\
Slayfinn and Eyel found the halls deserted; \\
they walked out and in, and looked about.\evb\evg


\bvg\bva\mssnote{\Regius~18r/27}\alst{Au}str skręið \alst{Ę}gill \hld\ at \alst{Ǫ}lrúnu, &
en \alst{s}uðr \alst{S}lagfiðr \hld\ at \alst{S}vanhvítu, &
en \alst{ęi}nn Vǫlundr \hld\ sat í \alst{U}lf-dǫlum.\eva

\bvb East skied Eyel after Alerune, \\
and south Slayfinn after Swanwhite, \\
and alone Wayland stayed in the Wolfdales.\evb\evg


\bvg\bva\mssnote{\Regius~18r/29}Hann sló \alst{g}ull rautt \hld\ við \alst{g}im fastan, &
\alst{l}ukði alla \hld\ \edtrans{\alst{l}inn-baugum}{serpent-bighs}{\Bfootnote{It is unclear whether this word refers to rings actually shaped like snakes or is merely a poetic description of twisted rings.  Archeological examples of the former include the so-called “snake-head rings” (German \emph{Schlangenkopfringe}, Swedish \emph{ormhuvudringar}) from the Migration Period, and the snake- or dragon-shaped armlet from the Wiking Age found in a hoard in Undrom, Ångermanland, northern Sweden (108822 HST). https://samlingar.shm.se/object/5C5658C4-0813-4DFF-947F-E5E4C4BAB965.}} vęl; &
\alst{s}vá bęið hann \hld\ \alst{s}innar ljóssar &
\alst{k}vánar, ef hǫ́num \hld\ \alst{k}oma gęrði.\eva

\bvb He struck red gold by fastened gem; \\
he enclosed all the serpent-\inx[C]{bigh}[bighs] well; \\
so he awaited his own bright wife, \\
if to him she might come.\evb\evg


\bvg\bva\mssnote{\Regius~18r/31}Þat spyrr \alst{N}íðuðr, \hld\ \edtrans{\alst{N}íara}{the Nears}{\Bfootnote{An obscure tribe, perhaps the residents of \emph{Närke}, an ancient province of Sweden. See Index.}} dróttinn, &
at \alst{ęi}nn Vǫlundr \hld\ sat í \alst{U}lf-dǫlum; &
\alst{n}ǫ́ttum fóru sęggir, \hld\ \edtrans{\alst{n}ęglðar vǫ́ru brynjur}{nailed were their byrnies}{\Bfootnote{The “byrnies” here are apparently some kind of costly plate armour.}}, &
\alst{sk}ildir bliku þęira \hld\ við hinn \alst{sk}arða mána.\eva

\bvb This learns Nithad, lord of the \inx[G]{Nears}, \\
that alone Wayland stayed in the Wolfdales. \\
Nightily journeyed warriors—nailed were their byrnies— \\
their shields gleamed by the sickle moon.\evb\evg


\bvg\bva\mssnote{\Regius~18r/33}Stigu ór \alst{s}ǫðlum \hld\ at \alst{s}alar gafli, &
\edtext{gingu \alst{i}nn þaðan \hld\ \alst{ę}nd-langan sal}{\lemma{gingu \dots\ sal ‘went \dots\ hall’}\Bfootnote{Formulaic. The fixed variant line \emph{hón/hann inn of gekk \hld\ ęnd-langan sal} ‘he/she inside did go the endlong hall’ (i.e. ‘through the entire length of the hall’, cf. English “livelong”) occurs in three other places: sts. 16 and 30 of the present poem, and st. 3 of \Oddrunargratr. \emph{ęnd-langr salr} ‘endlong hall’ occurs in two additional places: st. 27 of \Thrymskvida\ and st. 3 of \Skirnismal.}}, &
sǫ́u á \alst{b}ast \hld\ \alst{b}auga dręgna, &
\alst{s}jau hundruð allra, \hld\ es sá \alst{s}ęggr átti.\eva

\bvb They stepped off their saddles by the hall’s gables; \\
went thence inside the endlong hall; \\
saw they on a bast-rope bighs drawn up, \\
seven hundred in all, which that man owned.\evb\evg


\bvg\bva\mssnote{\Regius~18v/2}Ok þęir \alst{a}f tóku \hld\ ok þęir \alst{á} létu &
\edtrans{fyr \alst{ęi}nn \alst{ú}tan, \hld\ es \alst{a}f létu}{save for one, which off they slid}{\Bfootnote{This bigh is probably the one mentioned in sts. 17 and 26, since Beadhild has it already when Wayland is brought back after being captured. It may have been kept for its particular beauty. \textcite{FinnurEdda}\ writes (\emph{my translation from the Danish}): “The ring which Nithad kept must have had special properties, and distinguished itself before others.  There is no doubt that the ring is a flight ring; whether this was clear to the poet is however questionable.  This much is certain, that Wayland seems to be able to fly away only after he has got back the ring; that is, the one which Beadhild brings him.”  This is by no means certain.  Wayland was a craftsman of legendary skill and could certainly have built wings for himself without a magical flight-ring.  That is what he does in the Low German version; it is also what happens in the related Daidalos myth.  For both of these see the introduction to the present poem.}}. &
Kom þar af \alst{v}ęiði \hld\ \alst{v}eðr-ęygr skyti &
Vǫlundr \alst{l}íðandi \hld\ of \alst{l}angan veg.\eva

\bvb And they took off and they slid on, \\
save for one which they slid off.— \\
Came there from the hunt the stormy-eyed shooter: \\
Wayland passing over a long way.\evb\evg


\bvg\bva\mssnote{\Regius~18v/4}Gekk hann \alst{b}rúnni \hld\ \alst{b}eru hold stęikja; &
\edtext{\alst{á}r}{\Afootnote{metr. and sens. emend.; \emph{hár} \Regius}} brann hrísi \hld\ \alst{a}ll-þurr fura, &
\alst{v}iðr hinn \alst{v}ind-þurri, \hld\ fyr \alst{V}ǫlundi.\eva

\bvb Went he the brown she-bear’s flesh to roast; \\
in early morning burned the twigs of all-dry pine— \\
the wood wind-dry—before Wayland.\evb\evg


\bvg\bva\mssnote{\Regius~18v/5}Sat á \alst{b}er-fjalli, \hld\ \edtrans{\alst{b}auga talði}{bighs he counted}{\Bfootnote{Wayland’s grief and loneliness are skilfully illustrated by his counting all seven hundred rings, something which had apparently become a habit for him.}}, &
\edtrans{\alst{a}lfa ljóði}{prince of elves}{\Bfootnote{Probably referring to Wayland’s nature as a Wild Man, something also seen by his hunting of bears, skiing, and fierce gaze, all associated with his Finnish or Saami ancestry.  Cf. 14/2b and 32/1b, where Nithad calls him \emph{vísi alfa} ‘chief of elves’.}} \hld\ \alst{ęi}ns saknaði; &
\alst{h}ugði at \alst{h}ęfði \hld\ \alst{H}lǫðvés dóttir, &
\alst{a}l-vitr \alst{u}nga \hld\ vę́ri \alst{a}ptr komin.\eva

\bvb Sat he on the bear-pelt, bighs he counted— \\
the prince of elves was missing one! \\
Thought he that Ladwigh’s daughter \ken*{= Harware} might have it, \\
that the young elwight might be come back.\evb\evg


\bvg\bva\mssnote{\Regius~18v/7}\alst{S}at \alst{s}vá lęngi, \hld\ at \alst{s}ofnaði, &
ok \alst{v}aknaði \hld\ \alst{v}ilja-lauss; &
vissi sér á \alst{h}ǫndum \hld\ \alst{h}ǫfgar nauðir, &
en á \alst{f}ótum \hld\ \alst{f}jǫtur of spęnntan.\eva

\bvb Sat he so long that asleep he fell, \\
and he awoke, powerless. \\
He knew on his hands heavy restraints, \\
and on his feet a fetter tight.\evb\evg


\bvg\bva\mssnote{\Regius~18v/9}\speakernote{[Vǫlundr kvað:]}%
„Hvęrir ’ru \alst{jǫ}frar \hld\ þęir’s \alst{á} lǫgðu &
\alst{b}ęsti-síma \hld\ ok \alst{b}undu mik?“\eva

\bvb\speakernoteb{[Wayland quoth:]}%
“Which are the princes that laid on \\
the bast-cordage, and bound me?”\evb\evg


\bvg\bva\mssnote{\Regius~18v/10}Kallaði \alst{n}ú \alst{N}íðuðr, \hld\ \alst{N}íara dróttinn: &
„Hvar gatst, \alst{V}ǫlundr, \hld\ \alst{v}ísi alfa, &
\alst{ó}ra \alst{au}ra, \hld\ í \alst{U}lf-dǫlum? &
\alst{G}ull vas þar ęigi \hld\ á \alst{G}rana lęiðu, &
\alst{f}jarri hugða’k várt land \hld\ \alst{f}jǫllum Rínar.“\eva

\bvb Now called Nithad, lord of the Nears: \\
“Where didst thou, Wayland, chief of elves, \\
get \emph{our} ounces in the Wolfdales? \\
Gold was there not on \inx[P]{Grane}’s path; \\
far I thought our land from the fells of the Rhine.\footnoteB{Grane was the horse of the legendary hero \inx[P]{Siward}, who slew the dragon \inx[P]{Fathomer} and took his gold.  Nithad’s speech is sarcastic: “Is there a dragon’s hoard in the Wolfdales?”}”\evb\evg


\bvg\bva\mssnote{\Regius~18v/13}\speakernote{[Vǫlundr kvað:]}%
„\alst{M}an’k at \alst{m}ęiri \hld\ \alst{m}ę́ti ǫ́ttum, &
es vér \alst{h}ęil \alst{h}jú \hld\ \alst{h}ęima vǫ́rum: &
\alst{H}laðguðr ok \alst{H}ęrvǫr \hld\ borin vas \alst{H}lǫðvé, &
\alst{k}unn vas Ǫlrún \hld\ \alst{K}íars dóttir.“\eva

\bvb\speakernoteb{[Wayland quoth:]}%
“I recall that we owned greater wealth \\
when we a whole household were at home. \\
Ladguth and Harware were born to Ladwigh; \\
known was Alerune, Choser’s daughter.”\footnoteB{Wayland responds rather cryptically and almost seems to be speaking to himself.  By asserting the noble lineages of the three swan-wives he gives a legitimate origin for his wealth, but he is aware that Nithad neither believes him nor cares.}\evb\evg

\sectionline

\bvg\bva\mssnote{\Regius~18v/15}%
\edtext{Úti stóð \alst{k}unnig \hld\ \alst{k}vǫ́n Níðaðar,}{\lemma{Úti \dots\ Níðaðar ‘Outside \dots\ of Nithad’}\Afootnote{emend. based on st. 30/1–2; om. \Regius}} &
\edtext{hón \alst{i}nn of gekk \hld\ \alst{ę}nd-langan sal}{\lemma{hón \dots\ sal ‘she went \dots\ hall’}\Bfootnote{Formulaic, also occuring in st. 30 of the present poem and in \Oddrunargratr\ 3.}}, &
\alst{st}óð á golfi, \hld\ \alst{st}ilti rǫddu: &
„es-a sá nú \alst{h}ýrr, \hld\ es ór \alst{h}olti fęrr.“\eva

\bvb Outside stood the cunning wife of Nithad; \\
she went inside the endlong hall, \\
stood on the floor, steered her voice: \\
“He is not mild now, who comes out of the wood.”\evb\evg


\bpg\bpa\mssnote{\Regius~18v/16}%
Níðuðr konungr gaf dóttur sinni Bǫðvildi gull-hring þann er hann tók af bastinu at Vǫlundar, en hann sjalfr bar sverðit er Vǫlundr átti. En dróttning kvað:\epa

\bpb King Nithad gave his daughter Beadhild the golden ring which he took from the bast rope in Wayland’s hall, but he himself carried the sword which Wayland had owned. And the queen quoth:\epb\epg


\bvg\bva\mssnote{\Regius~18v/19}\alst{T}ęnn hǫ́num \alst{t}ęygjask \hld\ es hǫ́num ’s \alst{t}ét sverð, &
ok hann \alst{B}ǫðvildar \hld\ \alst{b}aug of þękkir, &
\alst{ǫ́}mun eru \alst{au}gu \hld\ \alst{o}rmi hinum frána; &
\alst{s}níðið ér hann \hld\ \alst{s}ina magni, &
ok \alst{s}ętið hann \alst{s}íðan \hld\ í \alst{S}ę́varstǫð.“\eva

\bvb His teeth are bared when he is shown the sword, \\
and Beadhild’s bigh he recognizes; \\
reminiscent are his eyes to the gleaming serpent’s. \\
Snithe ye from him the might of his sinews, \\
and set him thereafter on Seastead!”\evb\evg


\bpg\bpa\mssnote{\Regius~18v/21}%
Svá var gǫrt, at skornar váru sinar í knés-fótum ok settr í holm einn, er þar var fyrir landi, er hét Sę́varstaðr. Þar smíðaði hann konungi alls-kyns gǫr-simar; engi maðr þorði at fara til hans, nema konungr einn. Vǫlundr kvað:\epa

\bpb So it was done that the sinews in his houghs were cut, and he was placed on the lonely islet which there lay before the land, which was called Seastead. There he forged for the king every kind of jewelry.  No man dared go to him save the king alone.  Wayland quoth:\epb\epg


\bvg\bva\mssnote{\Regius~18v/24}%
„\edtrans{Skínn}{shines}{\Bfootnote{Metrically deficient, since \emph{sk-} and \emph{s-} cannot alliterate.  A possible emendation is \emph{se’k} ‘I see’.}} Níðaði \hld\ sverð á linda, &
þat’s ek \alst{h}vęsta \hld\ sęm \alst{h}agast kunna’k &
ok ek \alst{h}ęrða’k \hld\ sęm \alst{h}ǿgst þótti; &
sá ’s mér \alst{f}ránn mę́kir \hld\ ę́ \alst{f}jarri borinn; &
\alst{s}é’k-a þann Vǫlundi \hld\ til \alst{s}miðju borinn.\eva

\bvb “The sword shines on Nithad’s belt, \\
which I sharpened as most handily I could, \\
and I hardened as most pleasingly seemed. \\
That gleaming blade is ever further from me carried; \\
I see it not for Wayland to the smithy carried!\evb\evg


\bvg\bva\mssnote{\Regius~18v/27}%
Nú \alst{b}err \alst{B}ǫðvildr \hld\ \alst{b}rúðar minnar &
—\alst{b}íð’k-a þęss \alst{b}ót— \hld\ \alst{b}auga rauða.“\eva

\bvb Now does Beadhild bear my bride’s \\
—I await no recompense for that—red bighs.”\evb\evg


\bvg\bva\mssnote{\Regius~18v/28}%
\edtrans{\alst{S}at—né \alst{s}vaf á-valt—}{He sat—never slept—}{\Bfootnote{Compare \Gudrunarhvot\ TODO: \emph{hófu mik—né drękkðu—} ‘they lifted me—they drowned [me] not—’.}} \hld\ ok \alst{s}ló hamri; &
vél gęrði \alst{h}ęldr \hld\ \alst{h}vatt Níðaði; &
\alst{d}rifu ungir tvęir \hld\ á \alst{d}ýr séa &
\alst{s}ynir Níðaðar \hld\ í \alst{S}ę́varstǫð.\eva

\bvb He sat—never slept—and struck the hammer; \\
wiles he most boldly planned for Nithad. \\
Two young ones were drifting to see costly things: \\
Nithad’s sons, to Seastead.\evb\evg


\bvg\bva\mssnote{\Regius~18v/30}%
\alst{K}vǫ́mu til \alst{k}istu, \hld\ \alst{k}rǫfðu lukla, &
\alst{o}pin vas \alst{i}ll-úð, \hld\ es þęir \alst{í} sǫ́u, &
fjǫlð vas þar \alst{m}ęina, \hld\ es \alst{m}ǫgum sýndisk &
at vę́ri \alst{g}ull rautt \hld\ ok \alst{g}ǫr-simar.\eva

\bvb Came they to the chest, demanded the keys; \\
open was the evil when inside they saw. \\
A host was there of harms, which to the lads seemed \\
like were they red gold and jewelry.\evb\evg


\bvg\bva\mssnote{\Regius~18v/33}\speakernote{[Vǫlundr kvað:]}%
„Komið \alst{ęi}nir tvęir, \hld\ komið \alst{a}nnars dags; &
ykkr lę́t’k þat \alst{g}ull \hld\ of \alst{g}efit verða; &
\alst{s}ęgið-a męyjum \hld\ né \alst{s}al-þjóðum, &
\alst{m}anni øngum, \hld\ at \alst{m}ik fyndið.“\eva

\bvb\speakernoteb{[Wayland quoth:]}%
“Come alone ye two, come another day; \\
to you, I say, this gold will be given. \\
Tell no maidens nor hall-folk \\
—not a man!—that \emph{me} ye met.”\evb\evg


\bvg\bva\mssnote{\Regius~19r/1}\alst{S}nimma kallaði \hld\ \alst{s}ęggr á annan, &
\alst{b}róðir á \alst{b}róður: \hld\ „gǫngum \alst{b}aug séa!“ &
\alst{K}vǫ́mu til \alst{k}istu, \hld\ \alst{k}rǫfðu lukla, &
\alst{o}pin vas \alst{i}ll-úð \hld\ es þęir \alst{í} litu.\eva

\bvb Early called one youth to another, \\
brother to brother: “Let us go see the bighs!” \\
Came they to the chest, demanded the keys; \\
open was the evil when inside they looked.\evb\evg


\bvg\bva\mssnote{\Regius~19r/3}Snęið af \alst{h}ǫfuð \hld\ \edtrans{\alst{h}úna}{bear-cubs}{\Bfootnote{An affectionate term for young boys, perhaps relating to warrior-initiations done in bear-skins.  This word is repeated by Nithad in st. 32 and mirrored by Wayland in st. 34.}} þęira &
ok und \edtrans{\alst{f}ęn \alst{f}jǫturs}{the fetter’s fen}{\Bfootnote{Unclear.  The smithy or islet may be Wayland’s “fetter”, in which case he buried them in a fen on the island.}} \hld\ \alst{f}ǿtr of lagði, &
ęn \edtrans{þę́r \alst{sk}álar, \hld\ es und \alst{sk}ǫrum vǫ́ru}{those bowls which were under their curls}{\Bfootnote{i.e. their skulls.}}, &
\alst{s}vęip útan \alst{s}ilfri, \hld\ \alst{s}ęldi Níðaði.\eva

\bvb He sliced off the heads of those bear-cubs, \\
and under the fetter’s fen their feet he laid. \\
And the bowls which were under their curls \\
he coated with silver, gave to Nithad.\evb\evg


\bvg\bva\mssnote{\Regius~19r/5}\alst{E}n ór \alst{au}gum \hld\ \edtrans{\alst{ja}rkna-stęina}{arkenstones}{\Bfootnote{Probably round crystals.}} &
sęndi \alst{k}unnigri \hld\ \alst{k}vǫ́n Níðaðar; &
en ór \alst{t}ǫnnum \hld\ \alst{t}vęggja þęira &
\alst{s}ló brjóst-kringlur, \hld\ \alst{s}ęndi Bǫðvildi.\eva

\bvb And from the eyes arkenstones \\
he sent to the cunning wife of Nithad. \\
And from the teeth of the two \\
he struck breast-brooches, sent to Beadhild.\evb\evg

\sectionline

{\small Something appears to be missing here, but the narrative can be gleaned.  Beadhild breaks the bigh given to her by Nithad (mentioned above in sts. 10—see note there—and 17), and fears her father’s anger.  She goes to Wayland in secret and asks him to mend it.  The sight of this ring reminds Wayland of his wife, and he decides to rape Beadhild.}

\sectionline

\bvg\bva\mssnote{\Regius~19r/7}%
Þá nam \alst{B}ǫðvildr \hld\ \alst{b}augi at hrósa &
\edtext{[...]}{\Bfootnote{The meter requires a half-line here, perhaps containing a repetition of 1a: \emph{baugi at hrósa} ‘the bigh to praise’.}}\ \hld\ es brotit hafði, &
„\alst{þ}ori’g-a’k sęgja, \hld\ nema \alst{þ}ér ęinum.“\eva

\bvb Then Beadhild began the bigh to praise, \\
{[...]} which she had broken, \\
“I dare not tell, save to thee alone.”\evb\evg


\bvg\bva\mssnote{\Regius~19r/8}\speakernote{Vǫlundr kvað:}%
„Ek \alst{b}ǿti svá \hld\ \alst{b}rest á gulli, &
at \alst{f}ęðr þínum \hld\ \alst{f}ęgri þykkir, &
ok \alst{m}ǿðr þinni \hld\ \alst{m}iklu bętri, &
ok \alst{s}jalfri þér \hld\ at \alst{s}ama hófi.“\eva

\bvb\speakernoteb{Wayland quoth:}
“I will so mend the crack on the gold, \\
that to thy father it fairer seems, \\
and to thy mother even better, \\
and to thyself of the same rank.”\evb\evg


\bvg\bva\mssnote{\Regius~19r/10}\alst{B}ar hána \alst{b}jóri, \hld\ \edtrans{því’t \alst{b}ętr kunni}{for he knew better}{\Bfootnote{i.e. he was more cunning than her.}}, &
\alst{s}vá’t hǫ́n í \alst{s}essi \hld\ of \alst{s}ofnaði. &
„Nú \alst{h}ęfi’k \alst{h}ęfnt \hld\ \alst{h}arma minna &
\alst{a}llra \edtrans{nema \alst{ę}inna}{save one}{\Bfootnote{Presumably the deprivation of his mobility due to the hamstringing, which he resolves by crafting his flight suit.}} \hld\ \edtrans{\alst{í}við-gjarna}{insidious ones}{\Bfootnote{King Nithad and his house.}}.“\eva

\bvb He overcame her with beer—for he knew better— \\
so that she in the seat did fall asleep. \\
“Now have I avenged my harms, \\
all, save one, on the insidious ones.”\evb\evg

\sectionline

\bvg\bva\mssnote{\Regius~19r/12}„\alst{V}ęl ek,“ kvað \alst{V}ǫlundr, \hld\ „\alst{v}erða’k á \edtrans{fitjum}{paddles}{\Bfootnote{\CV: \emph{fit} ‘the webbed foot of water-birds’, here a reference to the flight-suit which allows Wayland to regain his freedom.}}, &
þęim’s mik \alst{N}íðaðar \hld\ \alst{n}ǫ́mu rekkar.“ &
\alst{H}lę́jandi Vǫlundr \hld\ \alst{h}ófsk at lopti, &
\alst{g}rátandi Bǫðvildr \hld\ \alst{g}ekk ór ęyju. &
tregði \alst{f}ǫr \alst{f}riðils \hld\ ok \alst{f}ǫður ręiði.\eva

\bvb “Well I”, quoth Wayland, “fall on my paddles; \\
those of which Nithad’s men bereaved me!” \\
Laughing, Wayland threw himself in the air; \\
weeping, Beadhild went from the island, \\
grieved the lover’s flight and the father’s wrath.\evb\evg

\sectionline

\bvg\bva\mssnote{\Regius~19r/14}%
Úti stęndr \alst{k}unnig \hld\ \alst{k}vǫ́n Níðaðar, &
ok hón \alst{i}nn of gekk \hld\ \alst{ę}nd-langan sal, &
en hann á \alst{s}al-garð \hld\ \alst{s}ęttisk at hvílask, &
„Vakir þú \alst{N}íðuðr, \hld\ \alst{N}íara dróttinn?“\eva

\bvb Outside stands the cunning wife of Nithad, \\
and she inside did go the endlong hall. \\
But he on the courtyard set down to rest. \\
“Art thou awake, O Nithad, lord of the Nears?”\evb\evg


\bvg\bva\mssnote{\Regius~19r/17}\speakernote{[Níðuðr kvað:]}%
„\edtrans{\alst{V}aki’k á-valt \hld\ \alst{v}ilja-lauss}{I am always awake, powerless}{\Bfootnote{This line references sts. 12 and 20, but there Wayland was the powerless man who never slept.  By his revenge the suffering has been transferred onto Nithad.}}, &
\alst{s}ofna’k minst, \hld\ síðst \alst{s}onu dauða, &
\alst{k}ęll mik í hǫfuð, \hld\ \edtrans{\alst{k}ǫld erumk rǫ́ð þín}{cold seem thy counsels}{\Bfootnote{A severe insult to a woman of power, for such counsels to her husband was how she would influence worldly affairs.  In this way Wayland’s revenge reaches also Nithad’s wife.}}, &
\alst{v}ilnumk þęss nú, \hld\ at við \alst{V}ǫlund dǿma’k.“\eva

\bvb\speakernoteb{[Nithad quoth:]}%
“I am always awake, powerless; \\
I sleep the least since my sons died. \\
My head turns cold; cold seem thy counsels— \\
I would now but that I with Wayland may speak.”\evb\evg

\sectionline

\bvg\bva\mssnote{\Regius~19r/19}\speakernote{[Níðuðr kvað:]}%
„Sęg mér þat \alst{V}ǫlundr, \hld\ \alst{v}ísi alfa, &
af \alst{h}ęilum \alst{h}vat varð \hld\ \alst{h}únum mínum?“\eva

\bvb\speakernoteb{[Nithad quoth:]}%
“Tell me this, O Wayland, chief of elves: \\
what became of my healthy bear-cubs?”\evb\evg


\bvg\bva\mssnote{\Regius~19r/20}\speakernote{[Vǫlundr kvað:]}%
„\alst{Ęi}ða skalt mér \alst{á}ðr \hld\ \alst{a}lla vinna, &
\edtext{at \alst{sk}ips borði \hld\ ok at \alst{sk}jaldar rǫnd, &
at \alst{m}ars bǿgi \hld\ ok at \alst{m}ę́kis ęgg}{\lemma{at skips \dots\ ęgg ‘by deck \dots\ of sword’}\Bfootnote{Nithad must swear the oaths by his tools of trade as a warrior; by extension on his martial honour.  Cf. \HelgakvidaTwo, where broken oaths are to come back “biting” the oath-breaker by cursing his ship, horse, and sword, in that order.}} &
at þú \edtrans{\alst{k}vęlj-at}{shalt not torment}{\Bfootnote{A negative imperative.  The normal 2nd. sg. imper. of \emph{kvęlja} is \emph{kvęl}, but the negative clitic \emph{-at} causes the \emph{-j-} to reappear in a rare \emph{liaison} effect.  See Rosenberg (2024): “A Norse sandhi?” (TODO: add to bibliography).}} \hld\ \edtext{\alst{k}vǫ́n Vǫlundar, &
né \alst{b}rúði minni}{\lemma{kvǫ́n Vǫlundar ‘wife of Wayland’, brúði minni ‘my bride’}\Bfootnote{Beadhild, who is now pregnant.}} \hld\ at \alst{b}ana verðir, &
þótt kvǫ́n \alst{ęi}gim, \hld\ þá’s \alst{é}r kunnið, &
eða \alst{jó}ð \alst{ęi}gim \hld\ \alst{i}nnan hallar.\eva

\bvb\speakernoteb{[Wayland quoth:]}%
“Oaths shalt thou first all swear to me— \\
by the ship’s wall and the shield’s rim, \\
by the steed’s bough and the sword’s edge— \\
that thou shalt not torment the wife of Wayland, \\
nor of my bride become the bane, \\
though a wife we might own whom ye might know; \\
or a babe might own within the hall.\evb\evg


\bvg\bva\mssnote{\Regius~19r/24}%
\alst{G}akk til smiðju, \hld\ þęirar’s \alst{g}ørðir, &
þar fiðr \alst{b}ęlgi \hld\ \alst{b}lóði stokna, &
snęið’k af \alst{h}ǫfuð \hld\ \alst{h}úna þinna &
ok und \alst{f}ęn \alst{f}jǫturs \hld\ \alst{f}ǿtr of lagða’k.\eva

\bvb Go to the smithy which thou madest; \\
there wilt thou find bellows blood-besprinkled. \\
I sliced off the heads of thy bear-cubs, \\
and under the fetter’s fen their feet I laid.\evb\evg


\bvg\bva\mssnote{\Regius~19r/26}%
En þę́r \alst{sk}álar, \hld\ es und \alst{sk}ǫrum vǫ́ru, &
\alst{s}vęip’k útan \alst{s}ilfri, \hld\ \alst{s}ęlda’k Níðaði, &
\alst{e}n ór \alst{au}gum \hld\ \alst{ja}rkna-stęina, &
sęnda’k \alst{k}unnigri \hld\ \alst{k}vǫ́n Níðaðar.\eva

\bvb And the bowls which were under their curls, \\
I coated with silver, gave to Nithad. \\
And from the eyes arkenstones \\
I sent to the cunning wife of Nithad.\evb\evg


\bvg\bva\mssnote{\Regius~19r/28}%
En ór \alst{t}ǫnnum \hld\ \alst{t}vęggja þęira &
\alst{s}ló’k brjóst-kringlur, \hld\ \alst{s}ęnda’k Bǫðvildi; &
nú gęngr \alst{B}ǫðvildr \hld\ \alst{b}arni aukin, &
\edtrans{\alst{ęi}nga dóttir \hld\ \alst{y}kkur bęggja.}{the only daughter of you both}{\Bfootnote{Formulaic, near-identical to \HervararSaga\ st. 25/1–2: (\emph{Vaki, Angantýr, \hld\ vękr þik Hęrvǫr, // ęinga dóttir \hld\ ykkur Svǫ́fu.} ‘Wake, Ongentew: Harware awakes thee, the only daughter of thee and Sweve.’ Cf. also \Beowulf\ 375a, 2997b: \emph{ángan dohtor} ‘only daughter (accusative)’.)}}“\eva

\bvb And from the teeth of the two \\
I struck breast-brooches, sent to Beadhild. \\
Now goes Beadhild swollen with child; \\
the only daughter of you both.”\evb\evg


\bvg\bva\mssnote{\Regius~19r/30}\speakernote{[Níðuðr kvað:]}%
„\alst{M}ę́ltir-a þat \alst{m}ál, \hld\ es mik \alst{m}ęirr tregi, &
né þik \alst{v}ilja’k \alst{V}ǫlundr \hld\ \alst{v}err of níta; &
es-at svá maðr \alst{h}ǫ́r, \hld\ at þik af \alst{h}ęsti taki, &
\alst{n}é svá ǫflugr, \hld\ at þik \alst{n}eðan skjóti, &
þar’s þú \alst{sk}ollir \hld\ við \alst{sk}ý uppi.“\eva

\bvb\speakernoteb{[Nithad quoth:]}%
“Thou couldst not have spoken a speech which would grieve me more; \\
nor could I worse wish, Wayland, to deny thee. \\
There is no man so high that he might take thee from a horse, \\
nor so strong that he might shoot thee from below, \\
where thou dost jeer by the clouds above!”\evb\evg


\bvg\bva\mssnote{\Regius~19v/1}%
\alst{H}lę́jandi Vǫlundr \hld\ \alst{h}ófsk at lopti, &
en \alst{ó}-kátr Níðuðr \hld\ sat þá \alst{ę}ptir.\eva

\bvb Laughing, Wayland threw himself in the air; \\
but, gloomy, Nithad stayed behind.\evb\evg

\sectionline

\bvg\bva\mssnote{\Regius~19v/2}\speakernote{[Níðuðr kvað:]}%
„Upp rís \edtrans{\alst{Þ}akkráðr}{Thankred}{\Bfootnote{A German name never found elsewhere in ON, but equivalent to MHG \emph{Dancrát}.}}, \hld\ \alst{þ}rę́ll minn batsti, &
\alst{b}ið \alst{B}ǫðvildi, \hld\ \edtext{męy hina \alst{b}rá-hvítu, &
gangi \alst{f}agr-varið}{\lemma{męy hina brá-hvítu \dots\ fagr-varið ‘the brow-white maiden \dots\ fair-clothed’}\Bfootnote{Nithad still has some doubt in his heart and by these words tries to convince himself of the innocence of his daughter (\emph{mę́r} ‘maiden, virgin’).}} \hld\ við \alst{f}ǫður rǿða.“\eva

\bvb\speakernoteb{[Nithad quoth:]}%
“Rise up, Thankred, my best thrall; \\
bid Beadhild, the brow-white maiden, \\
to go, fair-clothed, with her father to counsel.”\evb\evg

\sectionline

\bvg\bva\mssnote{\Regius~19v/3}\speakernote{[Níðuðr kvað:]}%
„Es þat \alst{s}att Bǫðvildr, \hld\ es \alst{s}ǫgðu mér, &
\alst{s}ǫ́tuð it Vǫlundr \hld\ \alst{s}aman í holmi?“\eva

\bvb\speakernoteb{[Nithad quoth:]}%
“Is it true, Beadhild, as they told me— \\
stayed thou and Wayland together on the islet?”\evb\evg


\bvg\bva\mssnote{\Regius~19v/4}\speakernote{[Bǫðvildr kvað:]}%
„\alst{S}att ’s þat Níðuðr \hld\ es \edtrans{\alst{s}agði}{\emph{he} told}{\Bfootnote{Beadhild knows that Wayland is the only one aware of the rape and thus deduces that \emph{he} told her father.  She makes a subtle change in the conjugation from her father’s general third person plural (“what they told”), to the specific singular form (“what \emph{he} told”).}} þér: &
\alst{s}ǫ́tum vit Vǫlundr \hld\ \alst{s}aman í holmi &
\alst{ęi}na \alst{ǫ}gur-stund, \hld\ \alst{ę́}va skyldi; &
ek \alst{v}ę́tr hǫ́num \hld\ \edtext{\alst{v}inna}{\Afootnote{metr. and sens. emend.; om. \Regius}} \edtext{kunna’k, &
ek \alst{v}ę́tr hǫ́num \hld\ \alst{v}inna mátta’k}{\lemma{kunna’k ‘knew’, mátta’k ‘could’}\Bfootnote{Beadhild could defend herself neither mentally (\emph{kunna} ‘to know, understand’) nor physically (\emph{mega} ‘to have strength to do, avail’).  A powerful final stanza.}}.“\eva

\bvb\speakernoteb{[Beadhild quoth:]}%
“True it is, Nithad, as \emph{he} told thee— \\
I and Wayland stayed together on the islet \\
for one heavy hour—it should never have been. \\
I nowise knew withstand him; \\
I nowise could withstand him.”\evb\evg

\sectionline
