\bookStart{The Lay of Wayland}[Vǫlundarkviða]

\begin{flushright}%
Dating \parencite{Sapp2022}: C10th (0.428)–early C11th (0.475)

Meter: \Fornyrdislag%
\end{flushright}%

% Introduction

The \textbf{Lay of Wayland} (\Volundarkvida) is a story of immense psychological complexity, one of the masterpieces of Norse poetry.

The poem begins with a prose introduction, which survives in both \Regius\ and \AM.

Wayland gets his revenge on the whole royal household. He murders Nithad’s two young sons (affectionately, his “bear-cubs”) and thus ends his male lineage. Likewise he defangs Nithad’s “cunning wife” (she is never called anything else) by reducing her once powerful counsels to cold words; and finally he rapes Beadhild, depriving her of her maidenhood and value in marriage. They are thus reduced to the same state of complete powerlessness as he himself experienced, something clearly seen in the repetition of the adjective \emph{viljalauss} ‘powerless’; in v. 12 it describes Wayland after he wakes in shackles, but in v. 31 Nithad uses it to refer to his own mental state after the deaths of his sons. This sense of hopelessness is also seen in Beadhild’s haunting concluding speech. “I knew by naught struggle against him; I could by naught struggle against him.”

From the other versions of the story it is known that Beadhild gave birth to a son, Woody (OE \emph{Wudga}, \ThidreksSaga\ \emph{Viðga}, in Danish ballads \emph{Vidrik Verlandsøn}). He went on to become a great hero, and in the later heroic ballads by far eclipses his father. His birth seems heavily foreshadowed by Wayland forcing Nithad to swear an oath in v. 33, but he is nowhere directly mentioned in the poem, probably for artistic reasons.

Apart from this lay there is one other telling of the full story, namely the Strand of Wayland the Smith in \ThidreksSaga. While written in Old Norse, it is clear from the proper names and content that it is based on German sources (probably heroic ballads). Thus the native form \emph{Vǫlundr} is replaced with \emph{Velent} [\emph{sic}], \emph{Níðuðr} with \emph{Niðungr}. Interestingly there is a note within it showing that the native form was still known, namely about “Velent, the excellent smith, whom Warrings (\emph{Væringjar}) call Wayland (\emph{Vǫlundr})”. Apparently Wayland was so famous that “all men seem to praise his workmanship so, that the maker of any smith’s work which is made better than other works, is called a Wayland (\emph{Vǫlundr}) with regards to workmanship.”

Far more stark than minor differences of language is that of tone. The psychological complexity and tension of the older redaction is almost entirely gone: Wayland is no longer a mysterious wild man, but a chivalrous knight who can escape from any peril through his ingenuity and craftmanship. He is not kidnapped out of Nithad’s greed, nor hamstrung out of the suspicion of his cruel wife, but rather a loyal servant of Nithad’s, banished from the kingdom after defending himself against the king’s corrupt steward, and hamstrung after being caught attempting to poison the king’s food in revenge.

Most frustratingly the personality of Beadhild is entirely expulged. She is the anonymous “king’s daughter”, an unnamed maiden (\emph{jungfrú}, a borrowing from Low German) who is peacefully seduced by Wayland and quickly falls in love with him. Likewise the person of Nithad’s cunning wife is completely gone, and the murder of his sons no longer ends his lineage, since he has another, older son who survives him and takes over the kingdom. Wayland still flies away laughing after telling Nithad what he has done, but only four years (his son with Beadhild is three years old) later reconciliates with Nithad’s son, retrieves Beadhild and their son and lives a long life as a famous craftsman.

With this it is clearly seen that the story by the time of the \ThidreksSaga\ had been heavily distorted, a tragic victim of medieval romantic sensibilities. It does not have any high literary value, but is of interest since it shows the wide reception and variation of the narrative.

Finally there are also traces of the story in the Anglo-Saxon tradition, where it is alluded to in both \Waldere\ and \Deor, the latter of which particularly emphasising the powerlessness felt by Wayland and Beadhild (thus being much closer in spirit to the present poem than to \ThidreksSaga). Parts of the narrative are depicted on the early C8th Frank’s casket, where it is as prominent as the depiction of the Adoration of the Magi—a true testament to the weight with which it was regarded within that culture.

To illustrate the narrative correspondences and differences of the various redactions, I present the following table:
\begin{longtabu} to \textwidth {|c c c c c c|}
	\hline
	Person & \Volundarkvida & \ThidreksSaga & \Deor & \Waldere & Frank’s casket \\ [0.5ex]
	\hline\hline
  Wayland & \emph{Vǫlundr} & \emph{Velent} & \emph{Welund} & \emph{Weland} & + \\
  Wayland’s brothers & Agle and Slayfinn & Agle & − & − & Agle? \\
  Father of the brothers & A Finnish king & The riser Wade (\emph{Vaði}) of Zealand & − & − & − \\
  Nithad & \emph{Níðuðr}, lord of the Nears in Sweden & \emph{Niðungr}, king of \inx[G]{Thede} in Jutland & \emph{Niðhad} & \emph{Niðhad} & − \\
  Nithad’s daughter (Beadhild) & \emph{Bǫðvildr} & Unnamed & \emph{Beadohild} & − & + \\
  Nithad’s sons & Two & Three & More than one & − & At least one \\
  Wayland and Beadhild’s son (Woody) & − & Viðga & − & Widia & − \\
  Wives of the brothers & Wayland and Allwit, Agle and Alerune, Slayfinn and Swanwhite & Agle and Alerune & − & − & − \\ [0.5ex]
  \hline
  — & Wayland and his brothers ski and hunt animals. They settle in Wolfdales, and one day find their wives. When they suddenly leave, Slayfinn and Agle go out to search for them, while Wayland remains alone, smithing rings and longing for his wife. & Wade sends Wayland away to learn smithing from dwarfs, and he becomes exceptionally skilled. He enters into the service of Nithad and becomes his trusted friend. One day Nithad asks Wayland to retrieve his victory-stone (the owner of which will always have victory in battle), in exchange for which he will be given half of his kingdom and his daughter in marriage. After retrieving it, Wayland is ambushed by Nithad’s steward, who asks him for the stone. When he refuses, the steward attacks him, but Wayland easily slays the steward, whose men flee. Wayland is banished for this, but returns and attempts to poison the king’s food. & − & − & − \\
  — & Nithad learns that Wayland is alone, and arrives with a large number of warriors. He abducts him, and on the counsel of his wife has him hamstrung. & After the trick is discovered, Wayland is caught and hamstrung by the king. & − & − & − \\
  — & Wayland is placed on the island Seastead, and forced to make jewellery for him. & Wayland pretends to reconcile with the king, acknowledging his error, and saying that he will never flee, even if he is able to. In return for this he is given a smithy, and as much gold and silver as he asks for. & − & − & − \\ [1ex]
	\hline
\end{longtabu}

\sectionline

\section{Regarding Wayland (\emph{Frá Vǫlundi})}

\bpg\bpa Níðuðr hét konungr í Svíþjóð.
Hann átti tvá sonu ok eina dóttur. Hon hét Bǫðvildr.
Brę́ðr váru þrír, synir Finnakonungs.
Hét einn Slagfiðr, annarr Egill, þriði Vǫlundr.
Þeir skriðu ok veiddu dýr. Þeir kómu í Úlfdali ok gerðu sér þar hús.
Þar er vatn, er heitir Úlfsjár.
Snemma of morgin fundu þeir á vatnsstrǫndu konur þrjár, ok spunnu lín.
Þar váru hjá þeim álftarhamir þeira. Þat váru valkyrjur.
Þar váru tvę́r dę́tr Hlǫðvés konungs, Hlaðguðr svanhvít ok Hervǫr alvitr, in þriðja var Ǫlrún Kjársdóttir af Vallandi.
Þeir hǫfðu þę́r heim til skála með sér. Fekk Egill Ǫlrúnar, en Slagfiðr Svanhvítrar, en Vǫlundr Alvitrar.
Þau bjuggu sjau vetr. Þá flugu þę́r at vitja víga ok kómu eigi aftr.
Þá skreið Egill at leita Ǫlrúnar, en Slagfiðr leitaði Svanhvítrar, en Vǫlundr sat í Úlfdǫlum.
Hann var hagastr maðr, svá’t menn viti í fornum sǫgum.
Níðuðr konungr lét hann hǫndum taka, svá sem hér er um kveðit:\epa

\bpb Nithad was named a king in Sweden.
He owned two sons and one daughter; she was called Beadhild.
There were three brothers, the sons of a king of the Finns.
One was called Slayfinn, another Agle, the third Wayland.
They travelled on skis and hunted wild animals. They came into the Wolfdales and made for themselves houses there.
There is a water there, called Wolfsea.
Early in the morning they found on the lake-shore three women, and they were spinning linen.
By them were their swan-\inx[C]{hame}[hames]; they were Walkirries.
Two of them were the daughters of king Ladwigh: Ladguth Swanwhite and Harware Allwit, the third was Alerune, daughter of \inx[P]{Kear} of \inx[G]{Walland}\footnoteB{The Roman emperor; see Encyclopedia.}.
The brothers brought the maidens with them to their halls. Agle got Alerune, but Slayfinn Swanwhite, but Wayland Allwit.
They lived there for seven winters, then they left to attend battles, and did not return.
Then Agle left on skis to look for Alerune, but Slayfinn sought out Swanwhite; but Wayland stayed in the Wolfdales.
He was the most skilled craftsman, as men know, in the ancient saws.
King Nithad had him captured, about which this has been sung:\epb\epg

\sectionline

\bvg
\bva Męyjar flugu sunnan \hld\ Myrkvið í gǫgnum &
alvitr ungar, \hld\ ørlǫg drýgja; &
þę́r á sę́varstrǫnd \hld\ sęttusk at hvílask &
drósir suðrǿnar, \hld\ dýrt lín spunnu.\eva

\bvb Maidens flew from the south through Mirkwood\footnoteB{Mirkwood is surely referenced for its association with the war-ravaged lands of the Gots and Huns; a natural environment for Walkirries.}—young allwits\footnoteB{Maybe look at what this means. TODO.}—to fulfill \inx[C]{orlay}. They on the lake-shore set down to rest; the southern ladies span expensive linen.\evb
\evg


\bvg
\bva Ęin nam þęira \hld\ Ęgil at vęrja &
fǫgr mę́r fira \hld\ faðmi ljósum; &
ǫnnur vas Svanhvít, \hld\ svanfjaðrar dró, &
\edtext{[...]}{\Bfootnote{A line mentioning the name of Slayfinn has certainly gone missing here.}} &
en hin þriðja \hld\ þęira systir &
varði hvítan \hld\ hals Vǫlundar.\eva

\bvb One of them began—the fair maiden of men—to embrace Agle in her light bosom. Another was Swanwhite—her swan-feathers she pulled; but the third sister warded the white throat of Wayland.\evb
\evg


\bvg
\bva Sǫ́tu síðan \hld\ sjau vetr at þat, &
en hinn átta \hld\ allan þrǫ́ðu, &
en hinn níunda \hld\ nauðr of skilði, &
męyjar fýstusk \hld\ á myrkvan við, &
alvitr ungar \hld\ ørlǫg drýgja.\eva

\bvb Then they stayed for seven winters at that, but all the eighth they yearned, but the ninth did need divorce them: the maidens longed for the mirky wood: the young allwits, to fulfill orlay.\footnoteB{As Walkirries the \emph{orlay} of the sisters is to preside over battles for Weden. Remembering this duty they become increasingly anxious, until they one day decide to finally leave, as seen from the next verse without telling their husbands. For the significance of Mirkwood, see note to v. 1.}\evb
\evg


\bvg
\bva Kom þar af vęiði \hld\ veðręygr skyti &
Vǫlundr líðandi \hld\ of langan veg, &
Slagfiðr ok Ęgill, \hld\ sali fundu auða, &
gingu út ok inn \hld\ ok umb sǫ́usk.\eva

\bvb Came there from the hunt the weather-eyed shooter: Wayland passing over a long way. Slayfinn and Agle found the halls deserted; they walked out and in, and looked about.\evb
\evg


\bvg
\bva Austr skręið Ęgill \hld\ at Ǫlrúnu, &
en suðr Slagfiðr \hld\ at Svanhvítu, &
en ęinn Vǫlundr \hld\ sat í Ulfdǫlum.\eva

\bvb East skied Agle for Alerune, but south Slayfinn for Swanwhite; but alone Wayland stayed in the Wolfdales.\evb
\evg


\bvg
\bva Hann sló goll rautt \hld\ við gim fastan, &
lukði hann alla \hld\ linnbaugum vęl; &
svá bęið hann \hld\ sinnar ljóssar &
kvánar, ef hǫ́num \hld\ of koma gęrði.\eva

\bvb He struck the red gold by fastened gemstone, enclosed he all the serpent-\inx[C]{bigh}[bighs]\footnoteB{Armlets, torcs resembling serpents, perhaps even literally shaped like them; cf. the Viking age armlet found in a hoard in Undrom, Ångermanland, northern Sweden. Museum ID 108822 HST. TODO: Maybe include photo?} well; thus awaited he his bright wife, if to him she might come.\evb
\evg


\bvg
\bva Þat spyrr Níðuðr, \hld\ Níara dróttinn, &
at ęinn Vǫlundr \hld\ sat í Ulfdǫlum; &
nǫ́ttum fóru sęggir, \hld\ nęglðar vǫ́ru brynjur, &
skildir bliku þęira \hld\ við hinn skarða mána.\eva

\bvb It learns Nithad, lord of the \inx[G]{Nears}, that alone Wayland stayed in the Wolfdales. By night travelled warriors—nailed were their byrnies;\footnoteB{The soldiers had plated armour.} their shields gleamed by the waning moon.\evb
\evg


\bvg
\bva Stigu ór sǫðlum \hld\ at salar gafli, &
gingu inn þaðan \hld\ ęndlangan sal, &
sǫ́u þęir á bast \hld\ bauga dręgna, &
sjau hundruð allra, \hld\ es sá sęggr átti.\eva

\bvb They stepped out of the saddles, towards the hall’s gables; went inside thence, through the endlong hall. Saw they on a bast-rope bighs drawn up: seven hundred in all, which that man owned.\evb
\evg


\bvg
\bva Ok þęir af tóku \hld\ ok þęir á létu &
fyr ęinn útan, \hld\ es af létu; &
kom þar af vęiði \hld\ veðręygr skyti &
Vǫlundr líðandi \hld\ of langan veg.\eva

\bvb And they took off and they put back on; but for one, which away they put.\footnoteB{That this is the bigh mentioned by itself in vv. 17 and 26 seems likely. \textcite{FinnurEdda}\ writes: “The ring which Nithad kept must have had special properties, and distinguished itself before others. There is no doubt that the ring is a flight ring; whether this was clear to the poet is however questionable. This much is certain, that Wayland seems to be able to fly away only after he has got back the ring; that is, the one which Beadhild brings him.” (\emph{My translation from the Danish.})—The reader may for himself judge the plausibility of this, but it seems that Wayland, being an exceptionally handy craftsman, may just as well have crafted wings for himself without need for magical rings. This agrees with the Low German verison and the Daedalus myth, for both of which see the introduction to the poem.}—Came there from the hunt the weather-eyed shooter: Wayland passing over a long way.\evb
\evg


\bvg
\bva Gekk brúnni \hld\ beru hold stęikja, &
ár brann hrísi \hld\ allþurru fura, &
viðr hinn vindþurri, \hld\ fyr Vǫlundi.\eva

\bvb Went he the brown she-bear’s hull to roast; early burned the twigs of all-dry pine—the wind-dry wood—before Wayland.\evb
\evg


\bvg
\bva Sat á berfjalli, \hld\ bauga talði, &
alfa ljóði \hld\ ęins saknaði. &
hugði at hęfði \hld\ Hlǫðvés dóttir, &
Alvitr unga, \hld\ vę́ri aptr komin.\eva

\bvb Sat he on the bear-skin, bighs he counted—the prince of elves was missing one! Thought he that Ladwigh’s daughter might have it; that the young Allwit might be come back.\evb
\evg


\bvg
\bva Sat hann svá lęngi, \hld\ at hann sofnaði, &
ok hann vaknaði \hld\ viljalauss; &
vissi sér á hǫndum \hld\ hǫfgar nauðir, &
en á fótum \hld\ fjǫtur of spęntan.\eva

\bvb Sat he so long that asleep he fell, and he awoke, powerless. He knew on his hands tortuous restraints, and on his feet were fetters tightened.\evb
\evg


\bvg {\small [Wayland quoth:]}
\bva „Hvęrir ’ru jǫfrar \hld\ þęir’s á lǫgðu &
bęstisíma \hld\ ok bundu mik?“\eva

\bvb “Which are the princes, those that laid on thick bast-ropes, and bound me?”\evb
\evg


\bvg
\bva Kallaði nú Níðuðr, \hld\ Níara dróttinn: &
„Hvar gazt Vǫlundr, \hld\ vísi alfa, &
óra aura, \hld\ í Ulfdǫlum? &
Goll vas þar ęigi \hld\ á Grana lęiðu, &
fjarri hugða’k várt land \hld\ fjǫllum Rínar.“\eva

\bvb Out called Nithad, lord of the Nears: “Where gottest thou, Wayland, leader of elves, \emph{our} ounces in the Wolfdales? Gold was there not on \inx[P]{Grane}’s path; far I thought our land from the fells of the Rhine.\footnoteB{Grane was the horse of the legendary hero \inx[P]{Siward}, slayer of the dragon \inx[P]{Fathomer}. These events were thought to have taken place in Germany. The sense of the is thus sarcastic: “Where did you get that gold? A dragon’s hoard?”.}”\evb
\evg


\bvg {\small [Wayland quoth:]}
\bva „Man’k at męiri \hld\ mę́ti ǫ́ttum, &
es vér hęil hjú \hld\ hęima vǫ́rum. &
Hlaðguðr ok Hervǫr \hld\ borin vas Hlǫðvé, &
kunn vas Ǫlrún \hld\ Kíars dóttir.“\eva

\bvb “I remember that we owned greater wealth, when we a whole household were at home: Ladguth, and Harware was born to Ladwigh; known was Alerune, Kear’s daughter.”\footnoteB{Wayland responds rather cryptically. It seems that by asserting the noble lineage of the three swan-wives he gives a legitimate reason for his wealth, although he seems to be aware, judging by the tone, that the greedy Nithad neither cares nor believes him.}\evb
\evg


\bvg
\bva Úti stóð kunnig \hld\ kvǫ́n Níðaðar, &
hón inn of gekk \hld\ ęndlangan sal, &
stóð á golfi, \hld\ stilti rǫddu: &
„es-a sá nú hýrr, \hld\ es ór holti fęrr.\eva

\bvb Outside stood the cunning wife of Nithad; she inside did walk across the length of the hall; stood she on the floor, steered her voice: “That one\footnoteB{The abducted Wayland.} is not mild now, who comes out of the wood.\evb
\evg


\bvg
\bva Tęnn hǫ́num tęygjask \hld\ es hǫ́num’s tét sverð &
ok hann Bǫðvildar \hld\ baug of þękkir. &
Ǫ́mun eru augu \hld\ ormi hinum frána, &
sníðið ér hann \hld\ sina magni, &
ok sętið hann síðan \hld\ í Sę́varstǫð.“\eva

\bvb His teeth are bared when he is shown the sword, and he recognizes Beadhild’s bigh. Reminiscent are the eyes to the gleaming snake’s. Cut ye from him the might of his sinews, and set him thereafter on Seastead!”\evb
\evg


\bvg
\bva[P] Svá var gǫrt, at skornar váru sinar í knésfótum ok settr í holm einn, er þar var fyrir landi, er hét Sę́varstaðr. Þar smíðaði hann konungi allskyns gǫrsimar; engi maðr þorði at fara til hans, nema konungr einn. Vǫlundr kvað:\eva

\bvb[P] Thus was done, that the sinews in his houghs were cut, and he was placed on a lonely islet which there lay before the land, which was called Seastead. There he smithed for the king all manner of jewels. No man dared journey to him, save for the king alone. Wayland quoth:\evb
\evg


\bvg
\bva „Sé’k Níðaði \hld\ sverð á linda, &
þat’s ek hvęsta \hld\ sęm hagast kunna’k &
ok ek hęrða’k \hld\ sęm hǿgst þótti; &
sá ’s mér fránn mę́kir \hld\ ę́ fjarri borinn. &
sé’kk-a þann Vǫlundi \hld\ til smiðju borinn.\eva

\bvb “I see a sword on Nithad’s belt, that one I sharpened as most handily I knew, and hardened as most pleasingly seemed. Now that gleaming blade is ever far from me carried; I see it not for Wayland to the smithy carried.\evb
\evg


\bvg
\bva Nú berr Bǫðvildr \hld\ brúðar minnar, &
bíð’k-a þess bót, \hld\ bauga rauða.“\eva

\bvb Now Beadhild bears my bride’s—I get no bettering for that—red bighs.”
\evg


\bvg
\bva Sat né svaf ávalt \hld\ ok sló hamri; &
vél gęrði hęldr \hld\ hvatt Níðaðí; &
drifu ungir tvęir \hld\ á dýr séa &
synir Níðaðar \hld\ í Sę́varstǫð.\eva

\bvb He sat—never slept—and struck the hammer; he very boldly planned wiles for Nithad.—Two young ones hurried to look at precious things: Nithad’s sons, to Seastead.\evb
\evg


\bvg
\bva Kvǫ́mu til kistu, \hld\ krǫfðu lukla, &
opin vas illúð, \hld\ es í sǫ́u, &
fjǫlð vas þar męina, \hld\ es mǫgum sýndisk &
at vę́ri goll rautt \hld\ ok gǫrsimar.\eva

\bvb Came they to the chest, demanded the keys; open was the evil when inside they looked. A great deal was there of harms, which to the lads seemed like were it red gold and jewels.\evb
\evg


\bvg {\small [Wayland quoth:]}
\bva „Komið ęinir tvęir, \hld\ komið annars dags; &
ykkr lę́t’k þat goll \hld\ of gefit verða; &
sęgið-a męyjum \hld\ né salþjóðum, &
manni ęngum, \hld\ at mik fyndið.“\eva

\bvb “Come alone ye two, come another day; to you I will let that gold be given. Say not to maidens, nor to the people of the hall; to no man, that ye met me.”\evb
\evg


\bvg
\bva Snimma kallaði \hld\ sęggr á annan, &
bróðir á bróður: \hld\ „gǫngum baug séa!“ &
Kómu til kistu, \hld\ krǫfðu lukla, &
opin vas illúð \hld\ es í litu.\eva

\bvb Early called one youth to another, brother to brother: “Let us go see the bighs!”. Came they to the chest, demanded the keys; open was the evil when inside they looked.\evb
\evg


\bvg
\bva Snęið af hǫfuð \hld\ húna þęira &
ok und fęn fjǫturs \hld\ fǿtr of lagði, &
ęn þę́r skálar, \hld\ es und skǫrum vǫ́ru, &
svęip útan silfri, \hld\ sęldi Níðaði.\eva

\bvb He sliced off the heads of those bear-cubs\footnoteB{An affectionate term for the young boys. TODO: Relate to Bearserks?} \ken{boys}, and under the fetter’s fen\footnoteB{Very unclear. TODO.} their feet did lay; but the bowls\footnoteB{Their skulls.}, which were under their curls, he coated with silver and gave to Nithad.\evb
\evg


\bvg
\bva En ór augum \hld\ jarknastęina &
sęndi kunnigri \hld\ kvǫ́n Níðaðar; &
en ór tǫnnum \hld\ tvęggja þęira &
sló brjóstkringlur, \hld\ sęndi Bǫðvildi.\eva

\bvb But out of the eyes, earkenstones he sent to the cunning wife of Nithad; but out of the teeth of the two, he struck breast-brooches, sent to Beadhild.\evb
\evg


\bvg
\bva Þá nam Bǫðvildr \hld\ baugi at hrósa &
\edtext{[...]}{\Bfootnote{The meter requires a half-line here, likely containing a more specific description of the bigh.}}\ \hld\ es brotit hafði, &
„þori’k-a’k sęgja, \hld\ nema þér ęinum.“\eva

\bvb Then Beadhild began to praise the ring,\footnoteB{The verse is without doubt incomplete, but the story can be gleaned: Beadhild breaks the bigh she has been given by her parents (previously mentioned in vv. 10 (see note there) and 17), and is afraid that her parents may become upset. She thus goes to Wayland in secret, asking him to repair it.} [...] which she had broken, “I dare not tell it, save to thee alone.”\evb
\evg


\bvg {\small [Wayland quoth:]}
\bva „Ek bǿti svá \hld\ brest á golli, &
at fęðr þínum \hld\ fęgri þykkir, &
ok mǿðr þinni \hld\ miklu bętri, &
ok sjalfri þér \hld\ at sama hófi.“\eva

\bvb “I mend such the crack on the gold, that to thy father it fairer seems, and to thy mother far better, and to thyself of the same rank.”\evb
\evg


\bvg
\bva Bar hann hána bjóri, \hld\ þvíat hann bętr kunni, &
svát hón í sessi \hld\ of sofnaði. &
„Nú hęfk hęfnt \hld\ harma minna &
allra nema ęinna \hld\ íviðgjǫrnum.“\eva

\bvb He overcame her with beer—for he was more cunning—so that she in the seat asleep did fall. “Now have I avenged my harms—all but one\footnoteB{Presumably the deprivation of his mobility due to the hamstringing, which he resolves in the following stanza.}—on the insidious ones.\footnoteB{King Nithad and his family.}”\evb
\evg


\bvg
\bva „Vęl ek,“ kvað Vǫlundr, \hld\ „verða’k á fitjum, &
þęim’s mik Níðaðar \hld\ nǫ́mu rekkar.“ &
Hlę́jandi Vǫlundr \hld\ hófsk at lopti, &
grátandi Bǫðvildr \hld\ gekk ór ęyju. &
tregði fǫr friðils \hld\ ok fǫður vręiði.\eva

\bvb “Well I”, quoth Wayland, “fall on my paddles; those which Nithad’s men bereaved me of!\footnoteB{\emph{C-V}: \emph{fit} ‘the webbed foot of water-birds’, the reader may picture for himself. Wayland has crafted a mechanism to take flight, regaining his mobility which he lost when he was hamstrung.}” Laughing Wayland threw himself in the air; weeping Beadhild went from the island: she grieved the lover’s flight, and the father’s fury.\evb
\evg


\bvg
\bva Úti stóð kunnig \hld\ kvǫ́n Níðaðar, &
ok hón inn of gekk \hld\ ęndlangan sal, &
en hann á salgarð \hld\ sęttisk at hvílask, &
„Vakir þú Níðuðr, \hld\ Níara dróttinn?“\eva

\bvb Outside stood the cunning wife of Nithad; she walked inside across the length of the hall—but he, on the courtyard, set down to rest. “Art thou awake, Nithad, lord of the Nears?”\evb
\evg


\bvg {\small [Nithad quoth:]}
\bva „Vaki’k ávalt \hld\ viljalauss, &
sofna’k minst, \hld\ síz sonu dauða, &
kęll mik í hǫfuð, \hld\ kǫld erumk rǫ́ð þín, &
vilnumk þess nú, \hld\ at við Vǫlund dǿma’k.“\eva

\bvb “I am always awake, powerless; I fall asleep the least, since the death of my sons. My head freezes; cold are thy counsels—I wish now but that: to speak with Wayland.”\evb
\evg


\bvg {\small [Nithad quoth:]}
\bva „Sęg mér þat Vǫlundr, \hld\ vísi alfa, &
af hęilum hvat varð \hld\ húnum mínum?“\eva

\bvb “Say it to me, Wayland, leader of elves: what became of my healthy bear-cubs \ken{boys}?”\evb
\evg


\bvg {\small [Wayland quoth:]}
\bva „Ęiða skalt mér áðr \hld\ alla vinna, &
at skips borði \hld\ ok at skjaldar rǫnd, &
at mars bǿgi \hld\ ok at mę́kis ęgg &
at þú kvęlj-at \hld\ kvǫ́n Vǫlundar, &
né brúði minni \hld\ at bana verðir, &
þótt kvǫ́n ęigim, \hld\ þá’s ér kunnið, &
eða jóð ęigim \hld\ innan hallar.\eva

\bvb “Before that shalt thou swear to me all oaths:—by the deck of the ship and the rim of the shield, by the bough of the steed and the edge of the sword—that thou wilt not torment the wife of Wayland, nor of my bride become the bane, though a wife we might own, which ye know; or a babe might own, inside of the hall.\footnoteB{Wayland has Nithad swear an oath that he will not harm Beadhild, nor their (yet unborn) child.}\evb
\evg


\bvg
\bva Gakk til smiðju, \hld\ es gęrðir þú, &
þar fiðr þú bęlgi \hld\ blóði stokna, &
snęið’k af hǫfuð \hld\ húna þinna &
ok und fęn fjǫturs \hld\ fǿtr of lagða’k.\eva

\bvb Go to the smithy, which thou madest; there wilt thou find bellows, sprinkled with blood. I sliced off the heads of thy bear-cubs \ken{boys}, and under the fetter’s fen their feet did I lay.\evb
\evg


\bvg
\bva En þę́r skálar, \hld\ es und skǫrum vǫ́ru, &
svęip’k útan silfri, \hld\ sęlda’k Níðaði, &
en ór augum \hld\ jarknastęina, &
sęnda'k kunnigri \hld\ kvǫ́n Níðaðar.\eva

\bvb But the bowls, which were under their curls, I coated with silver and gave to Nithad. But out of the eyes, earkenstones I sent to the cunning wife of Nithad.\evb
\evg


\bvg
\bva En ór tǫnnum \hld\ tvęggja þęira &
sló’k brjóstkringlur, \hld\ sęnda’k Bǫðvildi; &
nú gęngr Bǫðvildr \hld\ barni aukin, &
ęingadóttir \hld\ ykkur bęggja.“\eva

\bvb But out of the teeth of the two, I struck breast-brooches, sent to Beadhild. Now walks Beadhild, swollen with child; the only daughter of you both.”\evb
\evg


\bvg {\small [Nithad quoth:]}
\bva „Mę́ltir-a þú þat mál, \hld\ es mik męir tregi, &
né þik vilja’k Vǫlundr \hld\ verr of níta; &
es-at svá maðr hǫ́r, \hld\ at þik af hęsti taki, &
né svá ǫflugr, \hld\ at þik neðan skjóti. &
þar’s þú skollir \hld\ við ský uppi.“\eva

\bvb “Thou spokest not that speech which might grieve me more; nor could I worse wish, Wayland, to deny thee. There is no man so high that he from horse might take thee, nor so mighty that he might shoot thee down, there where thou jeerest against the cloud-cover above!”\evb
\evg


\bvg
\bva Hlę́jandi Vǫlundr \hld\ hófsk at lopti, &
en ókátr Níðuðr \hld\ þá ęptir sat.\eva

\bvb Laughing Wayland threw himself in the air, but gloomy Nithad thereafter stayed.\evb
\evg


\bvg {\small [Nithad quoth:]}
\bva „Upp rís Þakkráðr, \hld\ þrę́ll minn bazti, &
bið Bǫðvildi, \hld\ męy hina bráhvítu, &
gangi fagrvarið \hld\ við fǫður rǿða.“\eva

\bvb “Rise up Thankred, my best thrall; ask Beadhild—the brow-white maiden—to go fair-clothed, with her father to counsel.”\evb
\evg


\bvg {\small [Nithad quoth:]}
\bva „Es þat satt Bǫðvildr, \hld\ es sǫgðu mér, &
sǫ́tuð it Vǫlundr \hld\ saman í holmi?“\eva

\bvb “Is it true, Beadhild, as they said to me: stayed thou and Wayland together on the island?”\evb
\evg


\bvg {\small [Beadhild quoth:]}
\bva „Satt ’s þat Níðuðr \hld\ es sagði þér: &
sǫ́tum vit Vǫlundr \hld\ saman í holmi &
ęina ǫgurstund, \hld\ ę́va skyldi; &
ek vę́tr hǫ́num \hld\ vinna kunna’k, &
ek vę́tr hǫ́num \hld\ vinna mátta’k.“\eva

\bvb “’Tis true, Nithad, as \emph{he} said\footnoteB{Beadhild, knowing that the only one who is aware of what happened is Wayland, makes the subtle change in the conjugation, from her father’s general plural (“what \emph{they} said”), to the specific singular (“what \emph{he} said”).} to thee: I and Wayland stayed together on the island, for one burdensome hour—it should never [have been]! I knew by naught struggle against him; I could by naught struggle against him.\footnoteB{She was both mentally (\emph{kunna} ‘to know, understand’) and physically (\emph{mega} ‘to have strength to do, avail’) incapable of struggling against him. — As \textcite{FinnurEdda} comments, an unsurpassed final verse.}”\evb
\evg
