\bookStart{The Lay of Wayland}[Vǫlundarkviða]

\begin{flushright}%
\textbf{Dating} \parencite{Sapp2022}: C10th (0.428)–early C11th (0.475)

\textbf{Meter:} \Fornyrdislag%
\end{flushright}%

% Introduction

The \textbf{Lay of Wayland} (\Volundarkvida) is a story of immense psychological complexity, one of the masterpieces of Norse narrative poetry.

The poem begins with a prose introduction, which survives in both \Regius\ and \AM.

Wayland gets his revenge on the whole royal household. He murders Nithad’s two young sons (affectionately, his “bear-cubs”) and thus ends his male lineage. Likewise he defangs Nithad’s “cunning wife” (she is never called anything else) by reducing her once powerful counsels to cold words; and finally he rapes Beadhild, depriving her of her maidenhood and value in marriage. They are thus reduced to the same state of complete powerlessness as he himself experienced, something clearly seen in the repetition of the adjective \emph{viljalauss} ‘powerless’; in v. 12 it describes Wayland after he wakes in shackles, but in v. 31 Nithad uses it to refer to his own mental state after the deaths of his sons. This sense of hopelessness is also seen in Beadhild’s haunting concluding speech. “I knew by naught struggle against him; I could by naught struggle against him.”

From the other versions of the story it is known that Beadhild gave birth to a son, Woody (OE \emph{Wudga}, \ThidreksSaga\ \emph{Viðga}, in Danish ballads \emph{Vidrik Verlandsøn}). He went on to become a great hero, and in the later heroic ballads by far eclipses his father. His birth seems heavily foreshadowed by Wayland forcing Nithad to swear an oath in v. 33, but he is nowhere directly mentioned in the poem, probably for artistic reasons.

Apart from this lay there is one other telling of the full story, namely the Strand of Wayland the Smith in \ThidreksSaga. While written in Old Norse, it is clear from the proper names and content that it is based on German sources (probably heroic ballads). Thus the native form \emph{Vǫlundr} is replaced with \emph{Velent} [\emph{sic}], \emph{Níðuðr} with \emph{Niðungr}. Interestingly there is a note within it showing that the native form was still known, namely about “Velent, the excellent smith, whom Warrings (\emph{væringjar}) call Wayland (\emph{Vǫlundr})”. Apparently Wayland was so famous that “all men seem to praise his workmanship so, that the maker of any smith’s work which is made better than other works, is called a Wayland (\emph{Vǫlundr}) with regards to workmanship.”

Far more stark than minor differences of language is that of tone. The psychological complexity and tension of the older redaction is almost entirely gone: Wayland is no longer a mysterious wild man, but a chivalrous knight who can escape from any peril through his ingenuity and craftmanship. He is not kidnapped out of Nithad’s greed, nor hamstrung out of the suspicion of his cruel wife, but rather a loyal servant of Nithad’s, banished from the kingdom after defending himself against the king’s corrupt steward, and hamstrung after being caught attempting to poison the king’s food in revenge.

Most frustratingly the personality of Beadhild is entirely expulged. She is the anonymous “king’s daughter”, an unnamed maiden (\emph{jungfrú}, a borrowing from Low German) who is peacefully seduced by Wayland and quickly falls in love with him. Likewise the person of Nithad’s cunning wife is completely gone, and the murder of his sons no longer ends his lineage, since he has another, older son who survives him and takes over the kingdom. Wayland still flies away laughing after telling Nithad what he has done, but only four years (his son with Beadhild is three years old) later reconciliates with Nithad’s son, retrieves Beadhild and their son and lives a long life as a famous craftsman.

With this it is clearly seen that the story by the time of the \ThidreksSaga\ had been heavily distorted, a tragic victim of medieval romantic sensibilities. It does not have any high literary value, but is of interest since it shows the wide reception and variation of the narrative.

Finally there are also traces of the story in the Anglo-Saxon tradition, where it is alluded to in both \Waldere\ and \Deor, the latter of which particularly emphasising the powerlessness felt by Wayland and Beadhild (thus being much closer in spirit to the present poem than to \ThidreksSaga). Parts of the narrative are depicted on the early C8th Frank’s casket, where it is as prominent as the depiction of the Adoration of the Magi—a true testament to the weight with which it was regarded within that culture.

To illustrate the narrative correspondences and differences of the various redactions, I present the following table:
\begin{longtabu} to \textwidth {|c c c c c c|}
	\hline
	Person & \Volundarkvida & \ThidreksSaga & \Deor & \Waldere & Frank’s casket \\ [0.5ex]
	\hline\hline
  Wayland & \emph{Vǫlundr} & \emph{Velent} & \emph{Welund} & \emph{Weland} & + \\
  Wayland’s brothers & Eyel and Slayfinn & Eyel & − & − & Eyel? \\
  Father of the brothers & A Finnish king & The riser Wade (\emph{Vaði}) of Zealand & − & − & − \\
  Nithad & \emph{Níðuðr}, lord of the Nears in Sweden & \emph{Niðungr}, king of \inx[G]{Thede} in Jutland & \emph{Niðhad} & \emph{Niðhad} & − \\
  Nithad’s daughter (Beadhild) & \emph{Bǫðvildr} & Unnamed & \emph{Beadohild} & − & + \\
  Nithad’s sons & Two & Three & More than one & − & At least one \\
  Wayland and Beadhild’s son (Woody) & − & Viðga & − & Widia & − \\
  Wives of the brothers & Wayland and Elwight, Eyel and Alerune, Slayfinn and Swanwhite & Eyel and Alerune & − & − & − \\ [0.5ex]
  \hline
  — & Wayland and his brothers ski and hunt animals. They settle in the Wolfdales, and one day find their wives. When they suddenly leave, Slayfinn and Eyel go out to search for them, while Wayland remains alone, smithing rings and longing for his wife. & Wade sends Wayland away to learn smithing from dwarfs, and he becomes exceptionally skilled. He enters into the service of Nithad and becomes his trusted friend. One day Nithad asks Wayland to retrieve his victory-stone (the owner of which will always have victory in battle), in exchange for which he will be given half of his kingdom and his daughter in marriage. After retrieving it, Wayland is ambushed by Nithad’s steward, who asks him for the stone. When he refuses, the steward attacks him, but Wayland easily slays the steward, whose men flee. Wayland is banished for this, but returns and attempts to poison the king’s food. & − & − & − \\
  — & Nithad learns that Wayland is alone, and arrives with a large number of warriors. He abducts him, and on the counsel of his wife has him hamstrung. & After the trick is discovered, Wayland is caught and hamstrung by the king. & − & − & − \\
  — & Wayland is placed on the island Seastead, and forced to make jewellery for him. & Wayland pretends to reconcile with the king, acknowledging his error, and saying that he will never flee, even if he is able to. In return for this he is given a smithy, and as much gold and silver as he asks for. & − & − & − \\ [1ex]
	\hline
\end{longtabu}

\sectionline

\section{Regarding Wayland (\emph{Frá Vǫlundi})}

\bpg\bpa\mssnote{\Regius~18r/4, \AM~6v/26}Níðuðr hét konungr í Svíþjóð.
Hann átti tvá sonu ok eina dóttur; \edtrans{hon hét}{she was called}{\Afootnote{so \Regius; ok hét hon ‘and she was called’ \AM}} Bǫðvildr.
Brǿðr \edtrans{vǫ́ru}{were}{\Afootnote{so \AM; om. \Regius}} þrír, synir Finna konungs. Hét einn Slagfiðr, annarr Egill, þriði Vǫlundr.
Þeir skriðu ok veiddu dýr. Þeir kvǫ́mu í Úlfdali ok gerðu \edtext{sér þar hús.
Þar er vatn, er heitir Úlfsjár.
Snemma of morgin fundu þeir á vatsstrǫndu konur þrjár, ok spunnu lín. Þar váru hjá þeim álftarhamir þeira; þat váru valkyrjur.
Þar váru tvę́r dǿtr Hlǫðvés konungs: Hlaðguðr svanhvít ok Hervǫr alvitr. In þriðja var Ǫlrún \edtext{Kjárs dóttir af Vallandi}{\lemma{Kjárs [\dots] af Vallandi ‘Choser of Walland’}\Bfootnote{i.e. ‘Cæsar of Rome’; a legendary form of the Roman emperor. See Encyclopedia.}}.
Þeir hǫfðu þę́r heim til skála með sér. Fekk Egill Ǫlrúnar, en Slagfiðr Svanhvítrar, en Vǫlundr Alvitrar.
Þau bjuggu sjau vetr. Þá flugu þę́r at vitja víga ok kvǫ́mu eigi aptr.
Þá skreið Egill at leita Ǫlrúnar, en Slagfiðr leitaði Svanhvítrar, en Vǫlundr sat í Úlfdǫlum.
Hann var hagastr maðr, svá at menn viti í fornum sǫgum.
Níðuðr konungr lét hann hǫndum taka, svá sem hér er um kveðit:}{\lemma{sér þar hús \dots\ um kveðit ‘for themselves houses \dots\ sung of’}\Afootnote{so \Regius; om. (due to loss of the following foll. in the ms.) \AM}}\epa

\bpb Nithad was a king called in Sweden.
He had two sons and one daughter; she was called Beadhild.
Three brothers were there; the sons of a king of the Finns. One was called Slayfinn, the other Eyel, the third Wayland.
They fared on skis and hunted wild beasts. They came into the Wolfdales and made for themselves houses there.
There is a lake there which is called the Wolfsea.
Early in the morning they found on the lake-shore three women, and they span linen. There were by them their swan-\inx[C]{hame}[hames]; those were Walkirries.
There were two daughters of king Ladwigh: Ladguth Swanwhite and Harware Elwight. The third was Alerune, daughter of \inx[P]{Choser} of \inx[G]{Walland}.
The men took the women to their halls with them.  Eyel got Alerune, and Slayfinn Swanwhite, and Wayland the Elwight.
The couples lived there for seven winters; then the women left to attend battles, and did not come back.
Then Eyel fared on skis to search for Alerune, but Slayfinn searched for Swanwhite—but Wayland stayed in the Wolfdales.
He was the most skilled craftsman whom men know of in the ancient saws. King Nithad had him taken, as it is here sung of:\epb\epg

\sectionline

\bvg\bva\mssnote{\Regius~18r/19}\alst{M}ęyjar flugu sunnan \hld\ \edtrans{\alst{M}yrk-við}{Mirkwood}{\Bfootnote{Mirkwood is surely referenced for its association with the war-ravaged lands of the Gots and Huns; a natural environment for Walkirries.}} í gǫgnum &
\edtrans{\alst{a}l-vitr}{elwights}{\Bfootnote{i.e. “strange beings, foreign wights”, continuing a hypothetical \emph{*alja-wihtiz}.}} \alst{u}ngar, \hld\ \edtrans{\alst{ø}r-lǫg drýgja;}{fulfill orlay}{\Bfootnote{That is, to fulfill their preordained destinies, and act according to their innate nature, as described in P1 and st. 3.  \textcite{MCR2005}[103] and some other editors see a sign of English influence in these words; they translate \emph{drýgja ør-lǫg} as “engage in war”, considering \emph{ør-lǫg} a semantic borrowing from the OE cognate of Dutch \emph{oorlog} ‘war’.  This is unneccessary; ON \emph{ør-lǫg} otherwise means ‘fate, destiny’, and so may its OE cognate, as seen by the equivalent phrase found in l. 29 of a poem on the Christian Doomsday (TODO?), where a man going to Hell for his sins \emph{ǫnd þǫnne â tó ealdre \hld\ or·leg dreógeð} ‘and then for ever and ever [he] suffers his orlay’.}} &
þę́r á \alst{s}ę́var-strǫnd \hld\ \alst{s}ęttusk at hvílask &
\alst{d}rósir suð-rǿnar, \hld\ \alst{d}ýrt lín spunnu.\eva

\bvb Maidens flew from the south through Mirkwood \\
—young elwights— to fulfill \inx[C]{orlay}. \\
They on the lake-shore set down to rest, \\
southern ladies, they span costly linen.\evb\evg


\bvg\bva\mssnote{\Regius~18r/21}\alst{Ęi}n nam þęira \hld\ \alst{Ę}gil at vęrja &
\edtrans{\alst{f}ǫgr mę́r \alst{f}ira}{fair maiden of men}{\Bfootnote{i.e. “fair maiden in human shape”.}} \hld\ \alst{f}aðmi ljósum; &
ǫnnur vas \alst{S}vanhvít, \hld\ \alst{s}van-fjaðrar dró, &
\edtext{[...]}{\Bfootnote{A line mentioning Slayfinn has probably been lost here.}} &
en hin \alst{þ}riðja \hld\ \alst{þ}ęira systir &
varði \alst{h}vítan \hld\ \alst{h}als Vǫlundar.\eva

\bvb One of them began—the fair maiden of men— \\
to embrace Eyel in her bosom bright. \\
Second was Swanwhite—her swan-feathers she rustled. \\
{[...]} \\
But the third of those sisters \\
embraced the white throat of Wayland.\evb\evg


\bvg\bva\mssnote{\Regius~18r/24}\alst{S}ǫ́tu \alst{s}íðan \hld\ \alst{s}jau vetr at þat, &
en hinn \alst{á}tta \hld\ \alst{a}llan þrǫ́ðu, &
en hinn \alst{n}íunda \hld\ \alst{n}auðr of skilði, &
\alst{m}ęyjar fýstusk \hld\ á \alst{m}yrkvan við, &
\alst{a}l-vitr \alst{u}ngar \hld\ \alst{ø}r-lǫg drýgja.\eva

\bvb They stayed then for seven winters after that, \\
but all the eighth they yearned, \\
and the ninth did need divorce them.— \\
The maidens longed for the Mirky wood: \\
the young elwights, to fulfill orlay.\evb\evg


\bvg\bva\mssnote{\Regius~18r/26}Kom þar af \alst{v}ęiði \hld\ \alst{v}eðr-ęygr skyti &
\edtext{Vǫlundr \alst{l}íðandi \hld\ of \alst{l}angan veg,}{\lemma{Vǫlundr \dots\ veg ‘Wayland \dots\ way’}\Afootnote{emend. based on st. 9/3–4 below; om. \Regius}} &
\alst{S}lagfiðr ok Ęgill, \hld\ \alst{s}ali fundu auða, &
gingu \alst{ú}t ok \alst{i}nn \hld\ ok \alst{u}mb sǫ́usk.\eva

\bvb Came there from the hunt the stormy-eyed shooter: \\
Wayland passing over a long way. \\
Slayfinn and Eyel found the halls deserted; \\
they walked out and in, and looked around.\evb\evg


\bvg\bva\mssnote{\Regius~18r/27}\alst{Au}str skręið \alst{Ę}gill \hld\ at \alst{Ǫ}lrúnu, &
en \alst{s}uðr \alst{S}lagfiðr \hld\ at \alst{S}vanhvítu, &
en \alst{ęi}nn Vǫlundr \hld\ sat í \alst{U}lf-dǫlum.\eva

\bvb East skied Eyel after Alerune, \\
but south Slayfinn after Swanwhite— \\
but alone Wayland stayed in the Wolfdales.\evb\evg


\bvg\bva\mssnote{\Regius~18r/29}Hann sló \alst{g}ull rautt \hld\ við \alst{g}im fastan, &
\alst{l}ukði alla \hld\ \edtrans{\alst{l}inn-baugum}{serpent-bighs}{\Bfootnote{Armlets, torcs resembling or shaped like serpents.  Cf. the snake- or dragon-shaped Wiking age armlet 108822 HST found in a hoard in Undrom, Ångermanland, northern Sweden. https://samlingar.shm.se/object/5C5658C4-0813-4DFF-947F-E5E4C4BAB965.}} vęl; &
\alst{s}vá bęið hann \hld\ \alst{s}innar ljóssar &
\alst{k}vánar, ef hǫ́num \hld\ \alst{k}oma gęrði.\eva

\bvb He struck red gold by fastened gem; \\
he enclosed all the serpent-\inx[C]{bigh}[bighs] well; \\
thus he awaited his own bright wife, \\
if to him she might come.\evb\evg


\bvg\bva\mssnote{\Regius~18r/31}Þat spyrr \alst{N}íðuðr, \hld\ \edtrans{\alst{N}íara}{the Nears}{\Bfootnote{An obscure tribe, perhaps the residents of \emph{Närke}, an ancient province of Sweden. See Encyclopedia.}} dróttinn, &
at \alst{ęi}nn Vǫlundr \hld\ sat í \alst{U}lf-dǫlum; &
\alst{n}ǫ́ttum fóru sęggir, \hld\ \edtrans{\alst{n}ęglðar vǫ́ru brynjur}{nailed were their byrnies}{\Bfootnote{The “byrnies” here are apparently some kind of costly plate armour.}}, &
\alst{sk}ildir bliku þęira \hld\ við hinn \alst{sk}arða mána.\eva

\bvb This learns Nithad, lord of the \inx[G]{Nears}, \\
that alone Wayland stayed in the Wolfdales. \\
Nightily journeyed warriors—nailed were their byrnies— \\
their shields gleamed by the waning moon.\evb\evg


\bvg\bva\mssnote{\Regius~18r/33}Stigu ór \alst{s}ǫðlum \hld\ at \alst{s}alar gafli, &
\edtext{gingu \alst{i}nn þaðan \hld\ \alst{ę}nd-langan sal}{\lemma{gingu \dots\ sal ‘went \dots\ hall’}\Bfootnote{Formulaic. The fixed variant line \emph{hón/hann inn of gekk \hld\ ęnd-langan sal} ‘he/she inside did go the endlong hall’ (i.e. ‘through the entire length of the hall’, cf. English “livelong”) occurs in three other places: sts. 16 and 30 of the present poem, and st. 3 of \Oddrunargratr. \emph{ęnd-langr salr} ‘endlong hall’ occurs in two additional places: st. 27 of \Thrymskvida\ and st. 3 of \Skirnismal.}}, &
sǫ́u á \alst{b}ast \hld\ \alst{b}auga dręgna, &
\alst{s}jau hundruð allra, \hld\ es sá \alst{s}ęggr átti.\eva

\bvb They stepped off their saddles by the hall’s gables; \\
went thence inside the endlong hall; \\
saw they on a bast-rope bighs drawn up: \\
seven hundred in all, which that man owned.\evb\evg


\bvg\bva\mssnote{\Regius~18v/2}Ok þęir \alst{a}f tóku \hld\ ok þęir \alst{á} létu &
\edtrans{fyr \alst{ęi}nn \alst{ú}tan, \hld\ es \alst{a}f létu}{save for one, which off they slid}{\Bfootnote{This bigh is probably the one mentioned in sts. 17 and 26, since Beadhild has it already when Wayland is brought back after being captured. It may have been kept for its particular beauty. \textcite{FinnurEdda}\ writes (\emph{my translation from the Danish}): “The ring which Nithad kept must have had special properties, and distinguished itself before others.  There is no doubt that the ring is a flight ring; whether this was clear to the poet is however questionable.  This much is certain, that Wayland seems to be able to fly away only after he has got back the ring; that is, the one which Beadhild brings him.”  This is by no means certain.  Wayland was a craftsman of legendary skill and could certainly have built wings for himself without a magical flight-ring.  That is what he does in the Low German version; it is also what happens in the related Daidalos myth.  For both of these see the introduction to the present poem.}}. &
Kom þar af \alst{v}ęiði \hld\ \alst{v}eðr-ęygr skyti &
Vǫlundr \alst{l}íðandi \hld\ of \alst{l}angan veg.\eva

\bvb And they took off, and they slid on; \\
save for one, which off they slid.— \\
Came there from the hunt the stormy-eyed shooter: \\
Wayland passing over a long way.\evb\evg


\bvg\bva\mssnote{\Regius~18v/4}Gekk hann \alst{b}rúnni \hld\ \alst{b}eru hold stęikja; &
\edtext{\alst{á}r}{\Afootnote{metr. and sens. emend.; \emph{hár} \Regius}} brann hrísi \hld\ \alst{a}ll-þurr fura, &
\alst{v}iðr hinn \alst{v}ind-þurri, \hld\ fyr \alst{V}ǫlundi.\eva

\bvb Went he the brown she-bear’s flesh to roast; \\
in early morning burned the twigs of all-dry pine— \\
the wood wind-dry—before Wayland.\evb\evg


\bvg\bva\mssnote{\Regius~18v/5}Sat á \alst{b}er-fjalli, \hld\ \edtrans{\alst{b}auga talði}{bighs he counted}{\Bfootnote{Wayland’s grief and loneliness are skilfully illustrated by his counting all seven hundred rings, something which had apparently become a habit for him.}}, &
\edtrans{\alst{a}lfa ljóði}{prince of elves}{\Bfootnote{Probably referring to Wayland’s nature as a half-dæmonic Wild Man, something also seen by his hunting of bears, skiing, and fierce gaze.  Cf. 14/2b and 32/1b, where Nithad calls him \emph{vísi alfa} ‘overseer of elves’.}} \hld\ \alst{ęi}ns saknaði; &
\alst{h}ugði at \alst{h}ęfði \hld\ \alst{H}lǫðvés dóttir, &
\alst{a}l-vitr \alst{u}nga \hld\ vę́ri \alst{a}ptr komin.\eva

\bvb Sat he on the bear-pelt, bighs he counted— \\
the prince of elves was missing one! \\
Thought he that Ladwigh’s daughter \ken*{= Harware} might have it, \\
that the young elwight might be come back.\evb\evg


\bvg\bva\mssnote{\Regius~18v/7}\alst{S}at \alst{s}vá lęngi, \hld\ at \alst{s}ofnaði, &
ok \alst{v}aknaði \hld\ \alst{v}ilja-lauss; &
vissi sér á \alst{h}ǫndum \hld\ \alst{h}ǫfgar nauðir, &
en á \alst{f}ótum \hld\ \alst{f}jǫtur of spęnntan.\eva

\bvb Sat he so long that asleep he fell, \\
and he awoke, powerless. \\
He knew on his hands tortuous restraints, \\
and on his feet were fetters tightened.\evb\evg


\bvg\bva\mssnote{\Regius~18v/9}\speakernote{[Vǫlundr kvað:]}%
„Hvęrir ’ru \alst{jǫ}frar \hld\ þęir’s \alst{á} lǫgðu &
\alst{b}ęsti-síma \hld\ ok \alst{b}undu mik?“\eva

\bvb\speakernoteb{[Wayland quoth:]}%
“Which are the princes that laid on \\
the bast-cordage, and bound me?”\evb\evg


\bvg\bva\mssnote{\Regius~18v/10}Kallaði \alst{n}ú \alst{N}íðuðr, \hld\ \alst{N}íara dróttinn: &
„Hvar gatst, \alst{V}ǫlundr, \hld\ \alst{v}ísi alfa, &
\alst{ó}ra \alst{au}ra, \hld\ í \alst{U}lf-dǫlum? &
\alst{G}ull vas þar ęigi \hld\ á \alst{G}rana lęiðu, &
\alst{f}jarri hugða’k várt land \hld\ \alst{f}jǫllum Rínar.“\eva

\bvb Now called Nithad, lord of the Nears: \\
“Where gottest thou, Wayland, overseer of elves, \\
\emph{our} ounces, in the Wolfdales? \\
Gold was there not on \inx[P]{Grane}’s path; \\
far I’ve thought our land from the fells of the Rhine.\footnoteB{Grane was the horse of the legendary hero \inx[P]{Siward}, slayer of the dragon \inx[P]{Fathomer}. These events were thought to have taken place in Germany. Nithad’s speech is thus sarcastic: “Where did you get that gold? I have never heard of a dragon’s hoard in the Wolfdales!”, the implication being that Wayland has stolen the gold (from king Nithad).}”\evb\evg


\bvg\bva\mssnote{\Regius~18v/13}\speakernote{[Vǫlundr kvað:]}%
„\alst{M}an’k at \alst{m}ęiri \hld\ \alst{m}ę́ti ǫ́ttum, &
es vér \alst{h}ęil \alst{h}jú \hld\ \alst{h}ęima vǫ́rum: &
\alst{H}laðguðr ok \alst{H}ęrvǫr \hld\ borin vas \alst{H}lǫðvé, &
\alst{k}unn vas Ǫlrún \hld\ \alst{K}íars dóttir.“\eva

\bvb\speakernoteb{[Wayland quoth:]}%
“I recall that we owned greater wealth, \\
when we a whole household were at home: \\
Ladguth and Harware were born to Ladwigh; \\
known was Alerune, Choser’s daughter.”\footnoteB{Wayland responds rather cryptically and almost seems to be speaking to himself.  It seems that by asserting the noble lineages of the three swan-wives he gives a legitimate reason for his wealth, but, judging by the tone, he is aware that Nithad neither believes him nor cares.}\evb\evg

\sectionline

\bvg\bva\mssnote{\Regius~18v/15}\edtext{Úti stóð \alst{k}unnig \hld\ \alst{k}vǫ́n Níðaðar,}{\lemma{Úti \dots\ Níðaðar ‘Outside \dots\ of Nithad’}\Afootnote{emend. based on st. 30/1–2; om. \Regius}} &
\edtext{hón \alst{i}nn of gekk \hld\ \alst{ę}nd-langan sal}{\lemma{hón \dots\ sal ‘she went \dots\ hall’}\Bfootnote{Formulaic, also occuring in st. 30 of the present poem and in \Oddrunargratr\ 3.}}, &
\alst{st}óð á golfi, \hld\ \alst{st}ilti rǫddu: &
„es-a sá nú \alst{h}ýrr, \hld\ es ór \alst{h}olti fęrr.“\eva

\bvb Outside stood the cunning wife of Nithad, \\
she went inside the endlong hall, \\
stood on the floor, steered her voice: \\
“He is not mild now, who comes out of the wood.”\evb\evg


\bpg\bpa\mssnote{\Regius~18v/16}Níðuðr konungr gaf dóttur sinni Bǫðvildi gull-hring þann er hann tók af bastinu at Vǫlundar, en hann sjalfr bar sverðit er Vǫlundr átti. En dróttning kvað:\epa

\bpb King Nithad gave his daughter Beadhild the golden ring which he took from the bast rope in Wayland’s hall, but he himself carried the sword which Wayland had owned. But the queen quoth:\epb\epg


\bvg\bva\mssnote{\Regius~18v/19}\alst{T}ęnn hǫ́num \alst{t}ęygjask \hld\ es hǫ́num’s \alst{t}ét sverð, &
ok hann \alst{B}ǫðvildar \hld\ \alst{b}aug of þękkir, &
\alst{ǫ́}mun eru \alst{au}gu \hld\ \alst{o}rmi hinum frána; &
\alst{s}níðið ér hann \hld\ \alst{s}ina magni, &
ok \alst{s}ętið hann \alst{s}íðan \hld\ í \alst{S}ę́varstǫð.“\eva

\bvb His teeth are bared when he is shown the sword, \\
and Beadhild’s bigh he recognizes; \\
reminiscent are his eyes to the gleaming serpent’s.— \\
Snithe ye from him the might of his sinews, \\
and set him thereafter on Seastead!”\evb\evg


\bpg
\bpa\mssnote{\Regius~18v/21}Svá var gǫrt, at skornar váru sinar í knés-fótum ok settr í holm einn, er þar var fyrir landi, er hét Sę́varstaðr. Þar smíðaði hann konungi alls-kyns gǫr-simar; engi maðr þorði at fara til hans, nema konungr einn. Vǫlundr kvað:\epa

\bpb So it was done that the sinews in his houghs were cut, and he was placed on a lonely islet lying there before the land, which was called Seastead. There he smithed for the king every kind of jewelry. No man dared go to him save the king alone. Wayland quoth:\epb
\epg


\bvg\bva\mssnote{\Regius~18v/24}„\edtrans{\alst{S}é’k}{I see}{\Afootnote{metr. emend.; \emph{skínn} ‘shines’ \Regius}} Níðaði \hld\ \alst{s}verð á linda, &
þat’s ek \alst{h}vęsta \hld\ sęm \alst{h}agast kunna’k &
ok ek \alst{h}ęrða’k \hld\ sęm \alst{h}ǿgst þótti; &
sá ’s mér \alst{f}ránn mę́kir \hld\ ę́ \alst{f}jarri borinn; &
\alst{s}é’k-a þann Vǫlundi \hld\ til \alst{s}miðju borinn.\eva

\bvb “I see the sword on Nithad’s belt, \\
which I sharpened as most handily I could, \\
and I hardened as most pleasingly seemed.— \\
That gleaming blade is ever further from me carried; \\
I see it not for Wayland to the smithy carried!\evb\evg


\bvg\bva\mssnote{\Regius~18v/27}Nú \alst{b}err \alst{B}ǫðvildr \hld\ \alst{b}rúðar minnar &
—\alst{b}íð’k-a þęss \alst{b}ót— \hld\ \alst{b}auga rauða.“\eva

\bvb Now does Beadhild bear my bride’s \\
—I await no recompense for that—red bighs.”\evb\evg


\bvg\bva\mssnote{\Regius~18v/28}\edtrans{\alst{S}at—né \alst{s}vaf á-valt—}{He sat—he slept never—}{\Bfootnote{Compare \Gudrunarhvot\ TODO: \emph{hófu mik—né drękkðu—} ‘they lifted me—they drowned [me] not—’.}} \hld\ ok \alst{s}ló hamri; &
vél gęrði \alst{h}ęldr \hld\ \alst{h}vatt Níðaði; &
\alst{d}rifu ungir tvęir \hld\ á \alst{d}ýr séa &
\alst{s}ynir Níðaðar \hld\ í \alst{S}ę́varstǫð.\eva

\bvb He sat—he slept never—and struck the hammer; \\
he very boldly planned wiles for Nithad.— \\
Two young ones were drifting to see costly things: \\
Nithad’s sons, to Seastead.\evb\evg


\bvg\bva\mssnote{\Regius~18v/30}\alst{K}vǫ́mu til \alst{k}istu, \hld\ \alst{k}rǫfðu lukla, &
\alst{o}pin vas \alst{i}llúð, \hld\ es \alst{í} sǫ́u, &
fjǫlð vas þar \alst{m}ęina, \hld\ es \alst{m}ǫgum sýndisk &
at vę́ri \alst{g}ull rautt \hld\ ok \alst{g}ǫr-simar.\eva

\bvb Came they to the chest, demanded the keys; \\
open was the evil when inside they looked. \\
A great deal was there of harms, which to the lads seemed \\
like were it red gold and jewelry.\evb\evg


\bvg\bva\mssnote{\Regius~18v/33}\speakernote{[Vǫlundr kvað:]}%
„Komið \alst{ęi}nir tvęir, \hld\ komið \alst{a}nnars dags; &
ykkr lę́t’k þat \alst{g}ull \hld\ of \alst{g}efit verða; &
\alst{s}ęgið-a męyjum \hld\ né \alst{s}al-þjóðum, &
\alst{m}anni øngum, \hld\ at \alst{m}ik fyndið.“\eva

\bvb\speakernoteb{[Wayland quoth:]}%
“Come alone ye two; come another day! \\
To you, I declare, this gold will be given. \\
Tell not maidens nor the folk of the hall \\
—no man!—that \emph{me} ye met.”\evb\evg


\bvg\bva\mssnote{\Regius~19r/1}\alst{S}nimma kallaði \hld\ \alst{s}ęggr á annan, &
\alst{b}róðir á \alst{b}róður: \hld\ „gǫngum \alst{b}aug séa!“ &
\alst{K}vǫ́mu til \alst{k}istu, \hld\ \alst{k}rǫfðu lukla, &
\alst{o}pin vas \alst{i}llúð \hld\ es \alst{í} litu.\eva

\bvb Early called one youth to another, \\
brother to brother: “Let us go see the bighs!” \\
Came they to the chest, demanded the keys; \\
open was the evil when inside they looked.\evb\evg


\bvg\bva\mssnote{\Regius~19r/3}Snęið af \alst{h}ǫfuð \hld\ \edtrans{\alst{h}úna}{bear-cubs}{\Bfootnote{An affectionate term for the young boys, perhaps relating to warrior-initiations done in bear-skins.}} þęira &
ok und \edtrans{\alst{f}ęn \alst{f}jǫturs}{the fetter’s fen}{\Bfootnote{Unclear.  The smithy or islet may be Wayland’s “fetter”, in which case he buried them in a bog close-by.}} \hld\ \alst{f}ǿtr of lagði, &
ęn \edtrans{þę́r \alst{sk}álar, \hld\ es und \alst{sk}ǫrum vǫ́ru}{those bowls which were under their curls}{\Bfootnote{i.e. their skulls.}}, &
\alst{s}vęip útan \alst{s}ilfri, \hld\ \alst{s}ęldi Níðaði.\eva

\bvb He sliced off the heads of those bear-cubs, \\
and under the fetter’s fen their feet he laid; \\
but those bowls which were under their curls \\
he coated with silver and gave to Nithad.\evb\evg


\bvg\bva\mssnote{\Regius~19r/5}\alst{E}n ór \alst{au}gum \hld\ \edtrans{\alst{ja}rkna-stęina}{arkenstones}{\Bfootnote{Probably round crystals.}} &
sęndi \alst{k}unnigri \hld\ \alst{k}vǫ́n Níðaðar; &
en ór \alst{t}ǫnnum \hld\ \alst{t}vęggja þęira &
\alst{s}ló brjóst-kringlur, \hld\ \alst{s}ęndi Bǫðvildi.\eva

\bvb But out of the eyes arkenstones \\
he sent to the cunning wife of Nithad; \\
but out of the teeth of the two lads \\
he struck breast-brooches; sent [them] to Beadhild.\evb\evg

\sectionline

Something appears to be missing here, but the narrative can be gleaned.  Beadhild breaks the bigh stolen by Nithad (mentioned above in sts. 10 (see note there) and 17), and is afraid that her parents will be angry about it.  She thus goes to Wayland in secret and asks him to mend it.  The sight of this ring may be what angers Wayland, and makes him take it out on Beadhild.

\sectionline

\bvg\bva\mssnote{\Regius~19r/7}Þá nam \alst{B}ǫðvildr \hld\ \alst{b}augi at hrósa &
\edtext{[...]}{\Bfootnote{The meter requires a half-line here, likely containing a more specific description of the bigh.}}\ \hld\ es brotit hafði, &
„\alst{þ}ori’g-a’k sęgja, \hld\ nema \alst{þ}ér ęinum.“\eva

\bvb Then Beadhild began to praise the ring, \\
{[...]} which she had broken, \\
“I dare not tell save to thee alone.”\evb\evg


\bvg\bva\mssnote{\Regius~19r/8}\speakernote{Vǫlundr kvað:}%
„Ek \alst{b}ǿti svá \hld\ \alst{b}rest á gulli, &
at \alst{f}ęðr þínum \hld\ \alst{f}ęgri þykkir, &
ok \alst{m}ǿðr þinni \hld\ \alst{m}iklu bętri, &
ok \alst{s}jalfri þér \hld\ at \alst{s}ama hófi.“\eva

\bvb\speakernoteb{Wayland quoth:}
“I [will] so mend the crack on the gold, \\
that to thy father it fairer seems, \\
and to thy mother much better, \\
and to thyself of the same rank.”\evb\evg


\bvg\bva\mssnote{\Regius~19r/10}\alst{B}ar hána \alst{b}jóri, \hld\ \edtrans{því-at \alst{b}ętr kunni}{for he knew better}{\Bfootnote{i.e. he was more cunning than her.}}, &
\alst{s}vá’t hǫ́n í \alst{s}essi \hld\ of \alst{s}ofnaði. &
„Nú \alst{h}ęfi’k \alst{h}ęfnt \hld\ \alst{h}arma minna &
\alst{a}llra \edtrans{nema \alst{ę}inna}{save one}{\Bfootnote{Presumably the deprivation of his mobility due to the hamstringing, which he resolves by crafting his flight suit.}} \hld\ \edtrans{\alst{í}-við-gjarna}{insidious ones}{\Bfootnote{King Nithad and his house.}}.“\eva

\bvb He overcame her with beer—for he knew better— \\
so that she in the seat asleep did fall. \\
“Now have I avenged my harms, \\
all, save one, on the insidious ones.”\evb\evg

\sectionline

\bvg\bva\mssnote{\Regius~19r/12}„\alst{V}ęl ek,“ kvað \alst{V}ǫlundr, \hld\ „\alst{v}erða’k á \edtrans{fitjum}{paddles}{\Bfootnote{\CV: \emph{fit} ‘the webbed foot of water-birds’, here a reference to the flight-suit which allows Wayland to regain his freedom.}}, &
þęim’s mik \alst{N}íðaðar \hld\ \alst{n}ǫ́mu rekkar.“ &
\alst{H}lę́jandi Vǫlundr \hld\ \alst{h}ófsk at lopti, &
\alst{g}rátandi Bǫðvildr \hld\ \alst{g}ekk ór ęyju. &
tregði \alst{f}ǫr \alst{f}riðils \hld\ ok \alst{f}ǫður ręiði.\eva

\bvb “Well I”, quoth Wayland, “fall on my paddles; \\
those which Nithad’s men bereaved me of!” \\
Laughing, Wayland threw himself in the air; \\
weeping, Beadhild went from the island; \\
grieved the lover’s flight, and the father’s fury.\evb\evg

\sectionline

\bvg\bva\mssnote{\Regius~19r/14}Úti stęndr \alst{k}unnig \hld\ \alst{k}vǫ́n Níðaðar, &
ok hón \alst{i}nn of gekk \hld\ \alst{ę}nd-langan sal, &
en hann á \alst{s}al-garð \hld\ \alst{s}ęttisk at hvílask, &
„Vakir þú \alst{N}íðuðr, \hld\ \alst{N}íara dróttinn?“\eva

\bvb Outside stands the cunning wife of Nithad, \\
and she inside did go the endlong hall— \\
but he, on the courtyard, set down to rest. \\
“Art thou awake, O Nithad, lord of the Nears?”\evb\evg


\bvg\bva\mssnote{\Regius~19r/17}\speakernote{[Níðuðr kvað:]}%
„\alst{V}aki’k á-\alst{v}alt \hld\ \edtrans{\alst{v}ilja-lauss}{powerless}{\Bfootnote{Used earlier of Wayland in st. 12, immediately after his binding.}}, &
\alst{s}ofna’k minst, \hld\ síðst \alst{s}onu dauða, &
\alst{k}ęll mik í hǫfuð, \hld\ \edtrans{\alst{k}ǫld erumk rǫ́ð þín}{cold seem thy counsels}{\Bfootnote{A severe insult to a woman, evenmoreso to a queen, for such counsels to their husbands were how they could influence worldly affairs.}}, &
\alst{v}ilnumk þęss nú, \hld\ at við \alst{V}ǫlund dǿma’k.“\eva

\bvb\speakernoteb{[Nithad quoth:]}%
“I am always awake, powerless; \\
I fall asleep the least since my sons have died. \\
My head turns cold; cold seem thy counsels— \\
I wish now but this: to speak with Wayland.”\evb\evg

\sectionline

\bvg\bva\mssnote{\Regius~19r/19}\speakernote{[Níðuðr kvað:]}%
„Sęg mér þat \alst{V}ǫlundr, \hld\ \alst{v}ísi alfa, &
af \alst{h}ęilum \alst{h}vat varð \hld\ \alst{h}únum mínum?“\eva

\bvb\speakernoteb{[Nithad quoth:]}%
“Tell me this, O Wayland, overseer of elves: \\
what became of my healthy bear-cubs?”\evb\evg


\bvg\bva\mssnote{\Regius~19r/20}\speakernote{[Vǫlundr kvað:]}%
„\alst{Ęi}ða skalt mér \alst{á}ðr \hld\ \alst{a}lla vinna, &
\edtext{at \alst{sk}ips borði \hld\ ok at \alst{sk}jaldar rǫnd, &
at \alst{m}ars bǿgi \hld\ ok at \alst{m}ę́kis ęgg}{\lemma{at skips \dots\ ęgg ‘by deck \dots\ of sword’}\Bfootnote{Which are all tools of war; in this way Wayland asks Nithad to swear on his honour as a warrior.  A familiar oath-formula; TODO.}} &
at þú \alst{k}vęlj-at \hld\ \edtext{\alst{k}vǫ́n Vǫlundar, &
né \alst{b}rúði minni}{\lemma{kvǫ́n Vǫlundar ‘wife of Wayland’, brúði minni ‘my bride’}\Bfootnote{i.e. Beadhild, who is now pregnant.}} \hld\ at \alst{b}ana verðir, &
þótt kvǫ́n \alst{ęi}gim, \hld\ þá’s \alst{é}r kunnið, &
eða \alst{jó}ð \alst{ęi}gim \hld\ \alst{i}nnan hallar.\eva

\bvb\speakernoteb{[Wayland quoth:]}%
“All oaths shalt thou first swear to me, \\
by deck of ship and rim of shield, \\
by bough of steed and edge of sword— \\
that thou wilt not torment the wife of Wayland, \\
nor of my bride become the bane, \\
though a wife we might own whom ye might know; \\
or a babe might own within the hall.\evb\evg


\bvg\bva\mssnote{\Regius~19r/24}\alst{G}akk til smiðju, \hld\ þęirar’s \alst{g}ørðir, &
þar fiðr \alst{b}ęlgi \hld\ \alst{b}lóði stokna, &
snęið’k af \alst{h}ǫfuð \hld\ \alst{h}úna þinna &
ok und \alst{f}ęn \alst{f}jǫturs \hld\ \alst{f}ǿtr of lagða’k.\eva

\bvb Go to the smithy, which \emph{thou} didst make; \\
there wilt thou find bellows sprinkled with blood. \\
I sliced off the heads of thy bear-cubs, \\
and under the fetter’s fen their feet I laid.\evb\evg


\bvg\bva\mssnote{\Regius~19r/26}En þę́r \alst{sk}álar, \hld\ es und \alst{sk}ǫrum vǫ́ru, &
\alst{s}vęip’k útan \alst{s}ilfri, \hld\ \alst{s}ęlda’k Níðaði, &
\alst{e}n ór \alst{au}gum \hld\ \alst{ja}rkna-stęina, &
sęnda’k \alst{k}unnigri \hld\ \alst{k}vǫ́n Níðaðar.\eva

\bvb But the bowls which were under their curls, \\
I coated with silver and gave to Nithad. \\
But out of the eyes arkenstones \\
I sent to the cunning wife of Nithad.\evb\evg


\bvg\bva\mssnote{\Regius~19r/28}En ór \alst{t}ǫnnum \hld\ \alst{t}vęggja þęira &
\alst{s}ló’k brjóst-kringlur, \hld\ \alst{s}ęnda’k Bǫðvildi; &
nú gęngr \alst{B}ǫðvildr \hld\ \alst{b}arni aukin, &
\edtrans{\alst{ęi}nga dóttir \hld\ \alst{y}kkur bęggja.}{the only daughter of you both}{\Bfootnote{Formulaic, near-identical to \HervararSaga\ st. 25/1–2: (\emph{Vaki, Angantýr, \hld\ vękr þik Hęrvǫr, // ęinga dóttir \hld\ ykkr Svǫ́fu.} ‘Wake, Ongentew: Harware awakes thee, the only daughter of thee and Sweve.’ Cf. also \Beowulf\ 375a, 2997b: \emph{ángan dohtor} ‘only daughter (accusative)’.)}}“\eva

\bvb But out of the teeth of the two, \\
I struck breast-brooches; sent [them] to Beadhild. \\
Now goes Beadhild swollen with child; \\
the only daughter of you both.”\evb\evg


\bvg\bva\mssnote{\Regius~19r/30}\speakernote{[Níðuðr kvað:]}%
„\alst{M}ę́ltir-a þat \alst{m}ál, \hld\ es mik \alst{m}ęirr tregi, &
né þik \alst{v}ilja’k \alst{V}ǫlundr \hld\ \alst{v}err of níta; &
es-at svá maðr \alst{h}ǫ́r, \hld\ at þik af \alst{h}ęsti taki, &
\alst{n}é svá ǫflugr, \hld\ at þik \alst{n}eðan skjóti, &
þar’s þú \alst{sk}ollir \hld\ við \alst{sk}ý uppi.“\eva

\bvb\speakernoteb{[Nithad quoth:]}%
“Thou mightst not have spoken a speech which might grieve me more; \\
nor could I worse wish, O Wayland, to deny thee.— \\
No man is so high that he from horse might take thee, \\
nor so mighty that he might shoot thee from below, \\
there as thou jeerest against the clouds above!”\evb\evg


\bvg\bva\mssnote{\Regius~19v/1}\alst{H}lę́jandi Vǫlundr \hld\ \alst{h}ófsk at lopti, &
en \alst{ó}-kátr Níðuðr \hld\ sat þá \alst{ę}ptir.\eva

\bvb Laughing, Wayland threw himself in the air; \\
but, gloomy, Nithad stayed behind.\evb\evg

\sectionline

\bvg\bva\mssnote{\Regius~19v/2}\speakernote{[Níðuðr kvað:]}%
„Upp rís \alst{Þ}akkráðr, \hld\ \alst{þ}rę́ll minn batsti, &
\alst{b}ið \alst{B}ǫðvildi, \hld\ \edtext{męy hina \alst{b}rá-hvítu, &
gangi \alst{f}agr-varið}{\lemma{męy hina brá-hvítu \dots\ fagr-varið ‘the brow-white maiden \dots\ fair-clothed’}\Bfootnote{With these expressions Nithad strongly stresses the purity of his daughter (\emph{mę́r} ‘maiden’ here simply meaning ‘virgin’). Perhaps he thinks that her innocence can be restored if she dresses in fair clothes, but it will not be so.}} \hld\ við \alst{f}ǫður rǿða.“\eva

\bvb\speakernoteb{[Nithad quoth:]}%
“Rise up, O Thankred, my best thrall; \\
bid Beadhild, the brow-white maiden, \\
to go, fair-clothed, with her father to counsel.”\evb\evg

\sectionline

\bvg\bva\mssnote{\Regius~19v/3}\speakernote{[Níðuðr kvað:]}%
„Es þat \alst{s}att Bǫðvildr, \hld\ es \alst{s}ǫgðu mér, &
\alst{s}ǫ́tuð it Vǫlundr \hld\ \alst{s}aman í holmi?“\eva

\bvb\speakernoteb{[Nithad quoth:]}%
“Is it true, Beadhild, as they told me: \\
stayed thou and Wayland together on the islet?”\evb\evg


\bvg\bva\mssnote{\Regius~19v/4}\speakernote{[Bǫðvildr kvað:]}%
„\alst{S}att ’s þat Níðuðr \hld\ es \edtrans{\alst{s}agði}{\emph{he} told}{\Bfootnote{Beadhild knows that Wayland is the only one aware of the rape and thus deduces that \emph{he} told her father.  She makes a subtle change in the conjugation from her father’s general third person plural (“what they told”), to the specific singular form (“what \emph{he} told”).}} þér: &
\alst{s}ǫ́tum vit Vǫlundr \hld\ \alst{s}aman í holmi &
\alst{ęi}na \alst{ǫ}gur-stund, \hld\ \alst{ę́}va skyldi; &
ek \alst{v}ę́tr hǫ́num \hld\ \edtext{\alst{v}inna}{\Afootnote{metr. and sens. emend.; om. \Regius}} \edtext{kunna’k, &
ek \alst{v}ę́tr hǫ́num \hld\ \alst{v}inna mátta’k}{\lemma{kunna’k ‘knew’, mátta’k ‘could’}\Bfootnote{Beadhild was totally incapable of defending her honour, both mentally (\emph{kunna} ‘to know, understand’) and physically (\emph{mega} ‘to have strength to do, avail’. — As \textcite{FinnurEdda} comments, an excellent final stanza.}}.“\eva

\bvb\speakernoteb{[Beadhild quoth:]}%
“’Tis true, Nithad, as \emph{he} told thee: \\
I and Wayland stayed together on the islet, \\
for one heavy hour—it should never [have been]! \\
I by naught against him \emph{knew} struggle; \\
I by naught against him \emph{could} struggle.”\evb\evg

\sectionline
