\bookStart{The Lay of Hymer}[Hymiskviða]

% Introduction.
Attested in two manuscripts, \Regius\ and \AM. The two are surprisingly consistent.

Þórr dró Miðgarðsorm. % TODO: Format as header.

Thunder pulled up the Middenyardsworm.\footnotetext{This is the only title the poem has in \Regius. \AM\ has the proper title \emph{Hymiskviða} instead.}


\bvg
\bva Ár valtívar \hld\ vęiðar nǫ́mu &
ok sumblsamir \hld\ áðr saðir yrði, &
hristu tęina \hld\ ok á hlaut sǫ́u, &
fundu þęir at Ę́gis \hld\ ørkost hvera.\eva

\bvb Of yore the slaughter-Tues had caught game\footnoteB{Lit. ‘took game’}, and banqueting before they might eat\footnoteB{Lit. ‘might become sated’}, they shook the twigs and looked at the \inx[C]{leat}; they found at Eagre’s a great choice of cauldrons.\footnoteB{The gods sprinkled the leat (sacrificial blood) of the beasts and interpreted the pattern; they found it most auspicious to feast at Eagre’s.}\evb
\evg


\bvg
\bva Sat bergbúi \hld\ barntęitr fyrir, &
mjǫk glíkr męgi \hld\ Miskorblinda, &
lęit í augu \hld\ Yggs barn í þrá: &
„þú skalt ǫ́sum \hld\ opt sumbl \edtext{gęra}{\lemma{gęra “host”}\Afootnote{gefa “give” \AM}}!“\eva

\bvb — Sat the mountain-dweller \ken*{= Eagre} there, joyous like a child, much like the lad of Misherblind\footnoteB{A reference to a lost myth? Unless Misherblind is an alternative name for Firneet, Eagre’s father.}; into his eyes looked the child of Ug \name{= Weden} \ken*{= Thunder} in defiance: “Thou shalt for the Ease oft’ host banquets!”\footnoteB{Having seen that Eagre has a great store of cauldrons, Thunder orders him to host future banquets for the Ease.}\evb
\evg


\bvg
\bva Ǫnn fekk jǫtni \hld\ orðbę́ginn halr, &
hugði at hefndum \hld\ hann nę́st við goð, &
bað hann Sifjar ver \hld\ sér fǿra hver, &
„þann’s ek ǫllum ǫl \hld\ yðr of hęita.“\eva

\bvb Great toil for the ettin the word-peevish man \ken*{= Thunder} caused; thought he \ken*{= Eagre} of revenge, soon, against the god: asked he Sib’s husband \ken*{= Thunder} to bring him a cauldron, “that one with which I for you all ale might brew.\footnoteB{Eagre asks Thunder to find a single cauldron which can hold enough ale to supply all the Ease.}”\evb
\evg


\bvg
\bva Né þat mǫ́ttu \hld\ mę́rir tívar &
ok ginnręgin \hld\ of geta hvęrgi, &
unz af tryggðum \hld\ Týr Hlórriða &
ástráð mikit \hld\ ęinum sagði:\eva

\bvb But that might the renowned \inx[G]{Tues} and the \inx[G]{gin-Reins} nowhere get ahold of; until out of loyalty, a great word of loving advice Tue to Loride \name{= Thunder} alone did say:\evb
\evg


\bvg
\bva „Býr fyr austan \hld\ Élivága &
hundvíss Hymir \hld\ at himins ęnda, &
á minn faðir \hld\ móðugr kętil, &
\edtext{rúmbrugðinn}{\Afootnote{‘rumbrygðan’ \AM}} hver \hld\ rastar djúpan.“\eva

\bvb “Lives to the east of the \inx[L]{Ilewaves} the hound-wise Hymer, at the end of heaven. Owns my father\footnoteB{Hymer being Tue’s father.}, fierce, a kettle; a size-renowned cauldron a \inx[C]{rest} deep.”\evb
\evg


\bvg
\bva „Veiztu, ef þiggjum \hld\ þann lǫgvelli?“ &
„Ef, vinr, vélar \hld\ vit gørvum til!“\eva

\bvb “Knowest thou if we will receive that liquid-boiler \ken{cauldron}?” — “If, friend, we two make use of wiles!”\footnoteB{The speakers are not indicated, but it is most sensible that Thunder asks and Tue answers.}\evb
\evg

\bvg
\bva Fóru drjúgum \hld\ \edtext{dag þann framan}{\lemma{dag þann framan “from the beginning of the day”}\Afootnote{\emph{emend. according to} \textcite{FinnurEdda}; dag þann fram “on that day forth” \Regius; dag fráliga “swiftly at day” \AM}} &
Ásgarði frá \hld\ unz til \edtext{Ęgils}{\lemma{Ęgils “Agle’s”}\Afootnote{\emph{thus} \Regius; Ę́gis “Eagre’s” \AM; — \AM\ \emph{reading possibly from confusion with Eagre described earlier in the poem, but or the shepherd did share his name.}}} kvǫ́mu. &
Hirði hann hafra \hld\ horngǫfgasta; &
hurfu at hǫllu \hld\ es Hymir átti.\eva

\bvb — Journeyed they with great strides from the beginning of the day, from Osyard, until to Agle’s they came—he herded the horn-noblest he-goats—they turned to the hall which Hymer owned.\evb
\evg


\bvg
\bva Mǫgr fann ǫmmu, \hld\ mjǫk lęiða sér, &
hafði hǫfða \hld\ hundruð níu. &
ęn ǫnnur gekk \hld\ algollin framm &
brúnhvít bera \hld\ bjórvęig syni.\eva

\bvb The lad \ken*{= Tue} found his grandmother greatly loathsome; heads she had, nine hundred.—But another woman, all-golden, stepped forth: white-browed, she carried a beer-draught for her son \ken*{= Tue}:\evb
\evg


\bvg
\bva „Áttniðr jǫtna \hld\ ek vilja’k ykr &
hugfulla tvá \hld\ und hvera sętja; &
es mínn \edtext{fríi}{\lemma{fríi “lover”}\Afootnote{\emph{thus} \Regius; faðir “father” \AM}} \hld\ mǫrgu sinni &
gløggr við gęsti \hld\ gǫrr ills hugar.“\eva

\bvb “Kinsman of ettins \ken*{= Tue}! I would wish to set you high-mettled two under the cauldrons; my lover \ken*{= Hymer} has many a time been stingy against guests, quick to ill temper.”\footnoteB{Tue’s mother hides him and Thunder, lest Hymer find them.}\evb
\evg


\bvg
\bva Ęn váskapaðr \hld\ varð \edtext{síðbúinn}{\Afootnote{\emph{om.} \AM}}, &
harðráðr Hymir, \hld\ hęim af vęiðum; &
gekk inn í sal, \hld\ glumðu jǫklar, &
vas karls, es kom, \hld\ kinnskógr frørinn.\eva

\bvb But the misshapen one was come late—the hard-minded Hymer—home from the hunt. He entered the hall—icicles clattered—on the churl who came \ken*{= Hymer} was the cheek-shaw \ken{beard} frozen.\evb
\evg


\bvg {\small [Tue’s mother quoth:]}
\bva „Ves þú hęill, Hymir, \hld\ í hugum góðum! &
Nú ’s sonr kominn \hld\ til sala þinna, &
sá’s vit vę́ttum \hld\ af vęgi lǫngum; &
fylgir hǫ́num \hld\ Hróðrs andskoti, &
vinr verliða; \hld\ Véurr hęitir sá.\eva

\bvb “Be thou hale, Hymer, in good spirits!\footnoteB{Formula also seeen in runic inscription N B380: \emph{Heill sé þú · ok í hugum góðum. \\ Þórr þik þiggi, \\ Óðinn þik eigi.} \\ “May thou be hale, and in good spirits! May Thunder receive thee, may Weden own thee.” \\ Cf. also \Beowulf\ l. 407: \emph{Wæs þú Hróðgár hál!} “Be thou, Rothgar, hale!”} Now the son \ken*{= Tue} is come to thy halls, the one whom we two have been expecting, from a long way off. Follows him the opponent of Rooder <ettin> \ken*{= Thunder}, the friend of manly retinues \ken*{= Thunder}; Wighward \name{= Thunder} is that one called.\evb
\evg


\bvg
\bva Sé þú hvar sitja \hld\ und salar gafli, &
svá \edtext{forða sér}{\Afootnote{forðask \AM}}, \hld\ stęndr \edtext{súl}{\Afootnote{‘sol’ \AM}} fyrir.“ &
Sundr stǫkk súla \hld\ fyr sjón jǫtuns, &
ęn \edtext{allr}{\Afootnote{áðr \Regius\AM TODO: elaborate, mention Finnur}} í tvau \hld\ áss brotnaði.\eva

\bvb See where they sit, ’neath the hall’s gable: thus they hide themselves—a pillar stands before them!\footnoteB{Tue’s mother reveals the hiding place of the gods.}” The pillars sprang asunder before the sight of the ettin, but all in two the beam was broken.\evb
\evg


\bvg
\bva Stukku átta, \hld\ ęn ęinn af þęim &
hverr harðslęginn \hld\ hęill af þolli; &
framm gingu þęir, \hld\ ęn forn jǫtunn &
sjónum lęiddi \hld\ sinn andskota.\eva

\bvb Eight\footnoteB{Eight cauldrons.} sprung apart, but one of them, a hard-forged cauldron, [came] whole off its peg\footnoteB{Presumably the one in which Tue and Thunder were hiding.}. Forth went they, but the ancient ettin with his sight beheld\footnoteB{Literally “led with his sight”.} his opponent \ken*{= Thunder}.\evb
\evg


\bvg
\bva Sagðit hǫ́num \hld\ hugr vęl þá’s sá &
gýgjar \edtext{grǿti}{\lemma{grǿti “distresser”}\Afootnote{gę́ti “keeper, warder” \AM}} \hld\ á golf kominn, &
þar vǫ́ru þjórar \hld\ þrír of tęknir, &
bað \edtext{sęnn}{\Afootnote{‘sun’ \AM}} jǫtunn \hld\ sjóða ganga.\eva

\bvb His heart was not pleased then, when he saw the distresser of troll-women \ken*{= Thunder} come on the floor. There were three bulls taken: bade the ettin at once them be cooked.\evb
\evg


\bvg
\bva Hvęrn létu þęir \hld\ hǫfði skęmra &
ok á sęyði \hld\ síðan bǫ́ru, &
át Sifjar verr \hld\ áðr sofa gingi, &
ęinn með ǫllu \hld\ øxn tvá Hymis.\eva

\bvb Each [bull] they let shorten by a head, and onto the fire-pit then carried: ate the husband of Sib \ken*{= Thunder}—before he might go to sleep—alone all together, two of Hymer’s oxen.\evb
\evg


\bvg
\bva Þótti hǫ́rum \hld\ Hrungnis spjalla &
verðr Hlórriða \hld\ vęl fullmikill, &
„munum at aptni \hld\ ǫðrum verða &
við vęiðimat \hld\ vér þrír lifa.“\eva

\bvb To the hoary friend of Rungner \name{ettin} \ken*{= Hymer} seemed Loride’s meal far too great; “next evening will we three by game-meat have to live.\footnoteB{Hymer’s stinginess—he refuses to share more of his own food, forcing his guests to go hunt—goes against all Indo-European rules of hospitality and illustrates the otherness of the Ettins. See introduction to the poem.}”\evb
\evg


\bvg
\bva Véurr kvaðzk vilja \hld\ á vág róa, &
ef ballr jǫtunn \hld\ bęitur gę́fi. &
„Hverf þú til \edtext{hjarðar}{\Afootnote{hallar \emph{(corr.)} \AM}}, \hld\ ef hug trúir, &
brjótr berg-Dana, \hld\ bęitur sǿkja.\eva

\bvb Wighward \name{= Thunder} called himself willing to row on the wave, if the baleful ettin might give pieces of bait. “Turn to the herd, if thou trust in thy heart—breaker of boulder-Danes \ken*{\textsc{ettins} > = Thunder}!—to seek pieces of bait.\evb
\evg


\bvg
\bva Þess \edtext{vę́ntir mik}{\Afootnote{vę́nti ek \Regius}}, \hld\ at þér \edtext{mynit}{\lemma{mynit “will not”}\Afootnote{myni ”will” \Regius}} &
ǫgn at oxa \hld\ auðfeng vesa.“ &
Svęinn sýsliga \hld\ svęif til skógar, &
þar’s oxi stóð \hld\ alsvartr fyrir.\eva

\bvb I expect that the oxen for bait will not be easily caught by thee.”—The swain \name{= Thunder} sharply turned to the woods, there where an ox stood, all-black, before [him].\evb
\evg


\bvg
\bva Braut af þjóri \hld\ þurs ráðbani &
hǫ́tún ofan \hld\ horna tveggja. &
„Verk þikkja þín \hld\ verri myklu &
kjóla valdi \hld\ an kyrr sitir.“\eva

\bvb From the bull broke the treacherous slayer of the thurse \ken*{= Thunder} off the high meadow of the two horns \ken{head} from above.—“Thy works seem far worse to the wielder of keels \ken*{= Hymer = me}, than if thou didst sit calm.\footnoteB{Hymer snidely belittles Thunder’s feat of pulling off the head of the ox (presumably by the horns).}”\evb
\evg

\bvg
\bva Bað hlunngota \hld\ hafra dróttinn &
\edtext{áttrunn}{\Afootnote{‘atrænn’ \AM}} apa \hld\ útar fǿra, &
ęn sá jǫtunn \hld\ sína \edtext{talði}{\Afootnote{‘milldi’ \emph{(corr.)} \AM}}, &
lítla fýsi \hld\ lęngra at róa.\eva

\bvb The lord of he-goats \ken*{= Thunder} bade the kinsman of the \inx[C]{ape}\footnoteB{The specific sense of \emph{api} is uncertain. It seems to generally refer to a fool, but see Index.}\ \ken*{\textsc{ettin} = Hymer} to push the launching-steed \ken{boat} further out; but that ettin told of his scarce wish to row longer.\footnoteB{The parallelism is notable, as Hymer, who just mocked Thunder, is now forced to do his willing by rowing.}\evb
\evg


\bvg
\bva Dró \edtext{mę́rr}{\Afootnote{\emph{thus} \Regius; ‘mæirr’ \AM}} Hymir \hld\ móðugr hvala &
ęinn á ǫngli \hld\ upp sęnn tváa, &
ęn aptr í skut \hld\ Óðni sifjaðr &
Véurr við vélar \hld\ vað gęrði sér.\eva

\bvb Pulled the renowned Hymer, fierce, up whales: one on the hook, soon up two; but back in the stern the Weden-related Wighward \name{= Thunder}, cleverly\footnoteB{lit. ‘by wiles’.} made himself a fishing-line.\evb
\evg


\bvg
\bva Ęgnði á ǫngul \hld\ sá’s ǫldum bergr, &
orms ęinbani \hld\ oxa hǫfði; &
gęin við \edtext{agni}{\lemma{agni “bait”}\Bfootnote{\emph{thus} \AM; ǫngli ‘hook’ \Regius}}, \hld\ sú’s goð fía, &
umbgjǫrð neðan \hld\ allra landa.\eva

\bvb On the hook fastened he who saves men \ken*{= Thunder}—the lone slayer of the Worm \ken*{= Thunder}—the head of the ox. At the bait snapped the one whom the gods hate \ken*{= Middenyardsworm}; the encircler of all lands\footnoteB{This kenning occurs identically in a fragment by 9th century scold Alewigh Snub (Ǫlv \emph{Þórr}, edited by Margaret Clunies Ross in \emph{SkP} III).} \ken*{= Middenyardsworm} from below.\evb
\evg


\bvg
\bva Dró djarfliga \hld\ dáðrakkr Þórr &
orm ęitrfáan \hld\ upp at borði; &
hamri kníði \hld\ hǫ́fjall skarar &
ofljótt ofan \hld\ ulfs hnitbróður.\eva

\bvb Daringly pulled deed-bold Thunder the venom-glistening Worm up on the gunwale; with the hammer he struck the high mountain of hair\footnoteB{A rather unfitting kenning, since serpents do not have hair.} \ken{head}—greatly hideous, from above—of the clash-brother of the Wolf \ken*{= Middenyardsworm}.\evb
\evg


\bvg
\bva \edtext{Hraungǫlkn}{\lemma{hraungǫlkn}\Afootnote{\emph{emend.}; hręingǫlkn \Regius\AM}} \edtext{hrutu}{\Afootnote{\emph{thus} \AM; hlumðu \Regius}}, \hld\ ęn hǫlkn þutu, &
fór hin forna \hld\ fold ǫll saman; &
søkkðisk síðan \hld\ sá fiskr í mar.\eva

\bvb The rock-monsters \ken{ettins} bounded,\footnoteB{\emph{hraun-gǫlkn} “rock-monsters”. Both mss. have \emph{hręin-}, which if retained the meaningless and unparalleled “reindeer-monsters”. On the other hand \emph{hraun} \ONP: ‘stone/barren area, wasteland; lava-field’ is well attested in Scoldish kennings for ettins. The precise meaning of \emph{galkn} ‘monster’ (plural \emph{gǫlkn}) is unclear; apart from this, it is attested in three Scoldish verses, always in kennings of the type “troll-woman of the shield \ken{axe}”. While the mss. ‘\emph{galkn}’ (norm. \emph{gálkn}) could be both singular and plural, the form of the verb precludes the former. This means that the word cannot be referring to the Middenyardsworm, refuting the renderings of Crawford (“the monster howled”) and Larrington (“the sea-wolf shrieked”).} but the bedrock resounded; moved the ancient earth all at once; sank thereafter that fish \ken*{= Middenyardsworm} into the sea.\evb
\evg


\bvg
\bva Ótęitr jǫtunn, \hld\ es aptr røru, &
\edtext{[...]}{\Bfootnote{There is without doubt a line missing here, the grammar and sense require it.}} &
svá’t \edtext{ár}{\lemma{ár “in the early morning”}\Afootnote{\textcite{FinnurEdda}\ \emph{suggests} svá’t at ǫ́r “so that by the oar”}} Hymir \hld\ ękki mę́lti, &
vęifði rǿði \hld\ veðrs annars til.\eva

\bvb The not joyous ettin, as they rowed back, [...], so that in the early morning\footnoteB{Assuming this is the correct reading, it would seem like the group has spent the whole night at sea, with Hymer being the only one rowing.} Hymer spoke nothing; he pulled the oar around, against the storm:\evb
\evg


\bvg {\small [Hymer quoth:]}
\bva „Mundu of vinna \hld\ verk halft við mik, &
at hęim hvala \hld\ haf til bǿjar &
eða flotbrúsa \hld\ fęstir okkarn.“\eva

\bvb “Thou wilt win half the work by me,\footnoteB{Hymer offers Thunder, who now has nothing to show for the trip, that he can share with him half the glory of pulling up the whales if he does what he asks.} if thou carry the whales home to the farm, or our float-jar \ken{boat} do fasten.”\evb
\evg


\bvg
\bva Gekk Hlórriði \hld\ gręip \edtext{á}{\Afootnote{til á \Regius}} stafni &
vatt með austri \hld\ upp lǫgfáki; &
ęinn með ǫ́rum \hld\ ok með austskotu &
bar hann til bǿjar \hld\ brimsvín jǫtuns &
ok \edtext{holtriða}{\Afootnote{holtriba \Regius}} \hld\ hver í gegnum. \eva

\bvb Went Loride \name{= Thunder}; grasped the stern; hurled with the bilge-water the lake-nag \ken{boat} up. Alone with the oars and the bilge-bucket, he bore to the farm the brim-swines \ken{whales} of the ettin, even through the cauldron of woodland ridges\footnoteB{TODO. What do other editors and translators say?} \ken{valley?}.\evb
\evg


\bvg
\bva Ok ęnn jǫtunn \hld\ of afręndi, &
þrágirni vanr, \hld\ við Þór sęnti, &
kvað-at mann ramman, \hld\ þótt róa kynni, &
krǫpturligan, \hld\ nema kalk bryti.\eva

\bvb And still the ettin, used to stubbornness, about [his] strength of hand jibed at Thunder;\footnoteB{i.e. Hymer accused him of weak physical strength.} he called no man strong, although he could row, mightily, unless he broke the chalice.\evb
\evg


\bvg
\bva Ęn Hlórriði, \hld\ es at hǫndum kom, &
brátt lét bresta \hld\ brattstęin glęri, &
sló sitjandi \hld\ súlur í gǫgnum; &
bǫ́ru þó hęilan \hld\ fyr Hymi síðan.\eva

\bvb But Loride \name{= Thunder}, when [it] came in his hands, impatiently crashed sharp stone\footnoteB{Stone pillars.} with the glass;\footnoteB{The chalice seems to have been glazed.} he struck, sitting, right through the pillars; yet they\footnoteB{Presumably Hymer’s servants.} carried it whole before Hymer afterwards.\evb
\evg


\bvg
\bva Unz þat hin fríða \hld\ friðla kęndi &
ástráð mikit, \hld\ ęitt es vissi, &
„drep við haus Hymis, \hld\ hann ’s harðari, &
kostmóðs jǫtuns, \hld\ kalki hvęrjum.“\eva

\bvb Until the handsome mistress gave a great word of loving advice, the one she knew: “Strike against Hymer’s skull; it is harder—on the choice-weary\footnoteB{A reference to the gods having eaten up his best food.} ettin—than every chalice.”\evb
\evg


\bvg
\bva Harðr \edtext{ręis}{\Afootnote{\emph{om.} \AM}} á kné \hld\ hafra dróttinn, &
fǿrðisk allra \hld\ í ásmęgin; &
hęill vas karli \hld\ hjalmstofn ofan, &
ęn vínfęrill \hld\ valr rifnaði.\eva

\bvb Hard rose on the knees the lord of he-goats \ken*{= Thunder}; he summoned his highest os-might.\footnoteB{Compare \Gylfaginning\ in its description of Thunder attempting to pull up the Worm: \emph{Þá varð Þórr reiðr ok fę́rðist í ásmegin} “Then Thunder became wroth, and summoned his os-might.”} Whole was on the churl \ken*{= Hymer} the helmet-stump \ken{head} above, but the round wine-track \ken{chalice} rent apart.\evb
\evg


\bvg
\bva „Mǫrg vęitk mę́ti \hld\ mér gingin frá, &
\edtext{es}{\Afootnote{\emph{om.} \Regius}} kalki sé’k \hld\ \edtext{fyr}{\Afootnote{‘yr’ \Regius}} knéum hrundit,“ &
karl orð of kvað: \hld\ „kná’k-at sęgja &
aptr ę́vagi: \hld\ þú est ǫlðr of hęitt.\eva

\bvb “I know many good things have gone from me, when I see the chalice thrown before [his] knees;”—the churl \ken*{= Hymer} then words did speak: “I cannot say it, ever again: ‘Thou art, ale, [well] brewed!\footnoteB{Hymer laments that since his finest vessel is now broken, he will never again be able to enjoy strong drink.}’.\evb
\evg


\bvg
\bva Þat ’s til kostar \hld\ ef koma mę́ttið &
út ór óru \hld\ ǫlkjól hofi.“ &
Týr lęitaði \hld\ tysvar hrǿra; &
stóð at hvǫ́ru \hld\ hverr kyrr fyrir.\eva

\bvb It would be well done, if ye might make the ale-keel\footnoteB{\emph{ǫlkjól} is the accusative form, but in this sense (\CV: \emph{koma}, B) we would expect the dative \emph{ǫlkjóli}, something that the meter does not allow for.} \ken{cauldron} to come out of our hall.\footnoteB{\emph{hof} ‘hall’ usually means ‘hove; temple’.}” Tue attempted, twice, to move it; stood nevertheless the cauldron still before [him].\evb
\evg


\bvg
\bva Faðir Móða \hld\ fekk á þręmi &
ok í gǫgnum sté \hld\ golf niðr í sal; &
hóf sér á hǫfuð upp \hld\ hver Sifjar verr, &
ęn á hę́lum \hld\ hringar skullu.\eva

\bvb The father of Moody \ken*{= Thunder} grasped the brim, and stepped down through the floor in the hall;\footnoteB{In the account of \Gylfaginning\ Thunder is said to have stepped through the boat when trying to pull up the Middenyardsworm. This detail is also seen on the carving of the Altuna stone from Uppland, Sweden; it may have been transposed to this place in the narrative.} heaved the husband of Sib \ken*{= Thunder} up onto his head the cauldron, but on his heels rings clattered.\footnoteB{The rings from the cauldron-chain; this detail is mentioned in an example sentence contrasting long and short phonemes in the \FGT: \emph{heyrði til hǫddu, þá er Þórr bar hverinn} “one heard the pot-links when Thunder bore the kettle”. According to \textcite{FinnurEdda}\ this chain reached from one end of the kettle to another, in which case this would be an oblique reference to the cauldron’s size, its diameter being the same as Thunder’s height.}\evb
\evg


\bvg
\bva Fórut lęngi, \hld\ áðr líta nam &
aptr Óðins sonr \hld\ ęinu sinni; &
sá hann ór hręysum \hld\ með Hymi austan &
folkdrótt fara \hld\ fjǫlhǫfðaða.\eva

\bvb They journeyed for long, before the son of Weden \ken*{= Thunder} took to look back, a single time;—saw he out of stone-heaps, with Hymer from the east, a many-headed folk-troop faring.\evb
\evg


\bvg
\bva Hóf sér af hęrðum \hld\ hver standandi, &
vęifði Mjǫlni \hld\ morðgjǫrnum framm, &
ok hraunhvala \hld\ hann alla drap.\eva

\bvb Heaved he off from his shoulders the cauldron, [while] standing; he swung the murder-eager Millner forth, and the rock-whales \ken{= ettins} he slew all.\evb
\evg


\bvg
\bva Fórut lęngi, \hld\ áðr liggja nam &
hafr Hlórriða \hld\ halfdauðr fyrir, &
vas \edtext{skę́r}{\Afootnote{\emph{emend. from meaningless} ‘skirr’ \Regius\AM}} skǫkuls \hld\ skakkr á bęini, &
ęn því hinn lę́vísi \hld\ Loki of olli.\eva

\bvb They journeyed not for long, before the he-goat of Loride \name{= Thunder} took to lie half-dead before [them]; the steed of the cart-pole \ken{goat} was halt in the leg, but that the deceitful Lock did wield.\footnoteB{Apparently Lock (who has not been mentioned previously in the poem) was placing curses on the returning party.}\evb
\evg


\bvg
\bva Ęn ér hęyrt hafið, \hld\ hvęrr kann of þat &
goðmǫ́lugra \hld\ gørr at skilja, &
hvęr af hraunbúa \hld\ hann laun of fekk, &
es bę́ði galt \hld\ bǫrn sín fyrir.\eva

\bvb But ye have heard; each god-knowledgeable\footnoteB{\emph{goð-mǫ́lugr} ‘able to speak about the god-lore; versed in the mythology’ is a \emph{hapax}.} man knows about this more clearly discern: which rewards he \ken*{= Lock} from the rock-dweller got, as he yielded up both his own children for it.\footnoteB{As pointed out in \textcite{FinnurEdda}\, a verse containing such an address to the audience is otherwise unheard of. — What myth is being referred to is unclear. TODO: What do other authors write}\evb
\evg


\bvg
\bva Þróttǫflugr kom \hld\ á þing goða
ok hafði hver, \hld\ þann’s Hymir átti;
ęn Véar hvęrjan \hld\ vęl skulu drekka
ǫlðr at Ę́gis \hld\ \edtext{ęitt hǫrmęitið}{\lemma{ęitt hǫrmęitið “one ... flax-cutting”}\Bfootnote{A very obscure kenning. \LaFarge\ give several interpretations, viz. \emph{ęitr-hǫr-męitir} ‘poison-rope-cutter \ken{snake > winter}’, \emph{ęitr-orm-męiðir} ‘poison-worm-injurer’ \ken{winter}. The solution with the minimal amount of emendation is to read \emph{ęitt} ‘one’ as modifying \emph{ǫlðr} ‘ale-feast’, and \emph{hvęrjan} ‘every’ as modifying \emph{hǫr-męitiðr} ‘flax-cutting’, a compound made up of \emph{hǫrr} ‘flax, cord’ and \emph{męita} ‘to cut’ and referring to an obscure harvest festival. The interpretation is by no means certain.}}.\eva

\bvb The valour-mighty one \ken*{= Thunder} came onto the \inx[C]{Thing} of the gods, and had that cauldron which Hymer owned; but the \inx[G]{Wighers} \name{= Gods} shall well drink an ale-feast at Eagre’s every flax-cutting \ken{fall?}.\evb
\evg
