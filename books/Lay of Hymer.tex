\bookStart{The Lay of Hymer}[Hymiskviða]

% Introduction.
Attested in two manuscripts, \Regius\ and \AM. The two are surprisingly consistent; all verses are shared, and come in the same order. The title \emph{Hymiskvida} ‘the Lay of Hymer’ comes from \AM. \Regius\ instead has in the usual red ink the header \emph{Þórr dró Miðgarðsorm} ‘Thunder pulled the Middenyardsworm’.

\sectionline

\bvg
\bva\mssnote{\Regius\ 13v/26, \AM\ 5v/25}Ár \alst{v}altívar \hld\ \alst{v}ęiðar nǫ́mu &
ok \alst{s}umblsamir \hld\ áðr \alst{s}aðir yrði, &
\alst{h}ristu tęina \hld\ ok á \alst{h}laut sǫ́u, &
fundu at \alst{Ę́}gis \hld\ \alst{ø}rkost hvera.\eva

\bvb Of yore the slain-Tues \ken{gods} had caught game\footnoteB{Lit. ‘took game’}, and banqueting before they might eat\footnoteB{Lit. ‘might become sated’}, they shook the twigs and looked at the \inx[C]{leat}; they found at Eagre’s a great choice of cauldrons.\footnoteB{The gods sprinkled the leat (\emph{hlaut} ‘sacrificial blood’) of the beasts and interpreted the pattern; they found it most auspicious to feast at Eagre’s. TODO: reference to leat-twigs.}\evb
\evg


\bvg
\bva\mssnote{\Regius\ 13v/28, \AM\ 5v/27}Sat \alst{b}erg\alst{b}úi \hld\ \alst{b}arntęitr fyrir, &
\alst{m}jǫk glíkr \alst{m}ęgi \hld\ \alst{M}iskorblinda, &
lęit í \alst{au}gu \hld\ \alst{Y}ggs barn í þrá: &
„þú skalt \alst{ǫ́}sum \hld\ \alst{o}pt sumbl \edtext{gęra}{\lemma{gęra ‘host’}\Afootnote{\emph{gefa} ‘give’ \AM}}!“\eva

\bvb — Sat the mountain-dweller \ken*{\textsc{ettin} = Eagre} there, merry like a child, much alike to the lad of Misherblind;\footnoteB{A reference to a lost myth? Unless Misherblind is an alternative name for Firneet, Eagre’s father.} into his eyes looked the child of Ug \name{= Weden} \ken*{= Thunder} in stubbornness: “Thou shalt for the Ease oft host banquets!”\footnoteB{Having seen that Eagre has a great store of cauldrons, Thunder orders him to host future banquets for the Ease.}\evb
\evg


\bvg
\bva\mssnote{\Regius\ 13v/31, \AM\ 5v/29}\alst{Ǫ}nn fekk \alst{jǫ}tni \hld\ \alst{o}rðbę́ginn halr, &
\alst{h}ugði at \alst{h}efndum \hld\ \alst{h}ann nę́st við goð, &
bað hann \alst{S}ifjar ver \hld\ \alst{s}ér fǿra hver, &
„þann’s ek \alst{ǫ}llum \alst{ǫ}l \hld\ \alst{y}ðr of hęita.“\eva

\bvb Great toil for the ettin the word-peevish man \ken*{= Thunder} caused; he \ken*{= Eagre} thought of revenge, soon, against the god; he bade Sib’s husband \ken*{= Thunder} bring him a cauldron, “that one with which I for you all ale might heat.\footnoteB{Eagre gets back at Thunder by telling him that he needs a single cauldron which can hold enough ale to supply all the Ease.}”\evb
\evg


\bvg
\bva\mssnote{\Regius\ 14r/1, \AM\ 5v/30}Né þat \alst{m}ǫ́ttu \hld\ \alst{m}ę́rir tívar &
ok \alst{g}innręgin \hld\ of \alst{g}eta hvęrgi, &
unz af \alst{t}ryggðum \hld\ \alst{T}ýr Hlórriða &
\alst{á}stráð mikit \hld\ \alst{ęi}num sagði:\eva

\bvb But that one might the renowned \inx[G]{Tues} and the \inx[G]{gin-Reins} nowhere get ahold of—until, out of loyalty, a great loving counsel Tue to Loride \name{= Thunder} alone did say:\evb
\evg


\bvg
\bva\mssnote{\Regius\ 14r/3, \AM\ 6r/2}„Býr fyr \alst{au}stan \hld\ \alst{É}livága &
\alst{h}undvíss \alst{H}ymir \hld\ at \alst{h}imins ęnda, &
á \alst{m}inn faðir \hld\ \alst{m}óðugr kętil, &
\edtext{\alst{r}úmbrugðinn}{\Afootnote{\emph{†rumbrygðan†} \AM}} hver \hld\ \alst{r}astar djúpan.“\eva

\bvb “Dwells to the east of the \inx[L]{Ilewaves} the hound-wise Hymer, at heaven’s end.\footnoteB{According to \Vafthrudnismal\ 31 the Ilewaves were the poisonous wild rushes out of which the ettins emerged, and so it only makes sense that they would be found in the east, where the ettins dwell. Hymer’s dwelling even further east than them illustrates his fierceness.} Owns my father \ken*{= Hymer}, fierce, a kettle; a size-renowned cauldron, a \inx[C]{rest} deep.”\evb
\evg


\bvg {\small [Thunder quoth:]}
\bva\mssnote{\Regius\ 14r/4, \AM\ 6r/4}„Vęizt, ef \alst{þ}iggjum \hld\ \alst{þ}ann lǫgvelli?“ &
{\small [Tue quoth:]} „Ef, \alst{v}inr, \alst{v}élar \hld\ \alst{v}it gørvum til!“\eva

\bvb “Knowest thou if we will receive that liquid-boiler \ken{cauldron}?” — “If, friend, we two make use of wiles!”\footnoteB{Like elsewhere in this poem the speakers are not indicated, but it is most sensible that Thunder asks and Tue answers.}\evb
\evg

\bvg
\bva\mssnote{\Regius\ 14r/5, \AM\ 6r/4}Fóru drjúgum \hld\ \edtext{dag þann framan}{\lemma{dag þann framan ‘from the beginning of the day’}\Afootnote{emend. following \textcite{FinnurEdda}; \emph{dag þann fram} ‘on that day forth’ \Regius; \emph{dag fráliga} ‘swiftly at day’ \AM}} &
Ásgarði frá \hld\ unz til \edtext{Ęgils}{\lemma{Ęgils ‘Agle’s [home]’}\Afootnote{thus \Regius; \emph{Ę́gis} ‘Eagre’s [home]’ \AM is probably from confusion with Eagre (the ettin) described earlier in the poem, alternatively the shepherd shared his name.}} kvǫ́mu; &
hirði hafra \hld\ horngǫfgasta; &
hurfu at hǫllu \hld\ es Hymir átti.\eva

\bvb — Journeyed they with great strides from the beginning of the day, from Osyard, until to Agle’s [home] they came—he herded the horn-noblest he-goats\footnoteB{Thunder left his goats in the care of Agle, whose identity is unclear, but is also mentionde in Snorri TODO.}—they turned to the hall which Hymer owned.\evb
\evg


\bvg
\bva\mssnote{\Regius\ 14r/7, \AM\ 6r/6}Mǫgr fann ǫmmu, \hld\ mjǫk lęiða sér, &
hafði hǫfða \hld\ hundruð níu. &
ęn ǫnnur gekk \hld\ algollin framm &
brúnhvít bera \hld\ bjórvęig syni.\eva

\bvb The lad \ken*{= Tue} found his grandmother very loathsome; heads she had, nine hundred.—But another woman, all-golden, stepped forth: white-browed, she carried a beer-draught for her son \ken*{= Tue}:\evb
\evg


\bvg
\bva\mssnote{\Regius\ 14r/9, \AM\ 6r/8}„Áttniðr jǫtna \hld\ ek vilja’k ykr &
hugfulla tvá \hld\ und hvera sętja; &
es mínn \edtext{fríi}{\lemma{fríi ‘lover’}\Afootnote{thus \Regius; \emph{faðir} ‘father’ \AM}} \hld\ mǫrgu sinni &
gløggr við gęsti \hld\ gǫrr ills hugar.“\eva

\bvb “Descendant of ettins \ken*{= Tue}! I would wish to set you high-mettled two under the cauldrons; my lover \ken*{= Hymer} has many a time been stingy against guests, quick to ill temper.”\footnoteB{Tue’s mother hides him and Thunder, lest Hymer find them.}\evb
\evg


\bvg
\bva\mssnote{\Regius\ 14r/11, \AM\ 6r/9}Ęn váskapaðr \hld\ varð \edtext{síðbúinn}{\lemma{síðbúinn ‘come late’}\Afootnote{om. \AM}}, &
harðráðr Hymir, \hld\ hęim af vęiðum; &
gekk inn í sal, \hld\ glumðu jǫklar, &
vas karls, es kom, \hld\ kinnskógr frørinn.\eva

\bvb But the misshapen one was come late—the hard-minded Hymer—home from the hunt. He entered the hall—icicles clattered\footnoteB{In Icelandic the word \emph{jǫkull} comes to specifically mean ‘glacier’, but this development is peculiar and its base meaning is ‘icicle’, a word with which it is also cognate. The icicles are certainly those in Hymer’s beard.}—on the churl who came \ken*{= Hymer} was the cheek-shaw \ken{beard} frozen.\evb
\evg


\bvg {\small [Tue’s mother quoth:]}
\bva\mssnote{\Regius\ 14r/13, \AM\ 6r/11}„Ves þú hęill, Hymir, \hld\ í hugum góðum! &
Nú ’s sonr kominn \hld\ til sala þinna, &
sá’s vit vę́ttum \hld\ af vęgi lǫngum; &
fylgir hǫ́num \hld\ Hróðrs andskoti, &
vinr verliða; \hld\ Véurr hęitir sá.\eva

\bvb “Be thou hale, Hymer, in good spirits!\footnoteB{This formula is very closely paralleled in runic inscription N B380 (edited under Charms and Spells). Cf. also \Beowulf\ 407a: \emph{Wæs þú Hróðgár hál} ‘Be thou, Rothgar, hale!’} Now the son \ken*{= Tue} is come to thy halls, the one whom we two have been awaiting from a long way off. Follows him the opponent of Rooder \name{ettin} \ken*{= Thunder}, the friend of manly retinues \ken*{= Thunder}; \inx[P]{Wighward} \name{= Thunder} is that one called.\evb
\evg


\bvg
\bva\mssnote{\Regius\ 14r/15, \AM\ 6r/13}Sé þú hvar sitja \hld\ und salar gafli, &
svá \edtext{forða sér}{\Afootnote{forðask \AM}}, \hld\ stęndr \edtext{súl}{\Afootnote{\emph{†sol†} \AM}} fyrir.“ &
Sundr stǫkk súla \hld\ fyr sjón jǫtuns, &
ęn \edtext{allr}{\Afootnote{\emph{áðr} ‘earlier, before that’ \Regius\AM TODO: elaborate, mention Finnur}} í tvau \hld\ áss brotnaði.\eva

\bvb See where they sit ’neath the hall’s gable: thus they protect themselves—a pillar stands before them!\footnoteB{Tue’s mother reveals the hiding place of the gods.}” The pillars sprang asunder before the sight of the ettin, but all in two the roof-beam was broken.\evb
\evg


\bvg
\bva\mssnote{\Regius\ 14r/17, \AM\ 6r/15}Stukku átta, \hld\ ęn ęinn af þęim &
hverr harðslęginn \hld\ hęill af þolli; &
framm gingu þęir, \hld\ ęn forn jǫtunn &
sjónum lęiddi \hld\ sinn andskota.\eva

\bvb Eight [cauldrons] sprung apart, but one of them—a hard-forged cauldron—[came] whole off its peg.\footnoteB{The cauldrons were presumably hanging on the roof-beam. Eight of them broke, but a single one remained whole.} Forth went they, but the ancient ettin with his sight closely followed his opponent \ken*{= Thunder}.\evb
\evg


\bvg
\bva\mssnote{\Regius\ 14r/19, \AM\ 6r/16}Sagði-t hǫ́num \hld\ hugr vęl þá’s sá &
gýgjar \edtext{grǿti}{\lemma{grǿti ‘distresser’}\Afootnote{\emph{gę́ti} ‘keeper, warder’ \AM}} \hld\ á golf kominn, &
þar vǫ́ru þjórar \hld\ þrír of tęknir, &
bað \edtext{sęnn}{\lemma{sęnn ‘at once’}\Afootnote{\emph{sun} ‘[his] son \ken*{= Tue}?’ \AM}} jǫtunn \hld\ sjóða ganga.\eva

\bvb His \ken*{Hymer’s} heart was not pleased then, when he saw the distresser of troll-women \ken*{= Thunder} come on the floor. There were three bulls taken: bade the ettin at once [his servants] to go roast [them].\evb
\evg


\bvg
\bva\mssnote{\Regius\ 14r/21, \AM\ 6r/18}Hvęrn létu þęir \hld\ hǫfði skęmra &
ok á sęyði \hld\ síðan bǫ́ru, &
át Sifjar verr \hld\ áðr sofa gingi, &
ęinn með ǫllu \hld\ øxn tvá Hymis.\eva

\bvb Each [bull] they let shorten by a head, and onto the fire-pit then carried: ate the husband of Sib \ken*{= Thunder}—before he might go to sleep—alone by himself two of Hymer’s oxen.\footnoteB{Cf. \Thrymskvida\ 24.}\evb
\evg


\bvg
\bva\mssnote{\Regius\ 14r/23, \AM\ 6r/19}Þótti hǫ́rum \hld\ Hrungnis spjalla &
verðr Hlórriða \hld\ vęl fullmikill, &
„munum at aptni \hld\ ǫðrum verða &
við vęiðimat \hld\ vér þrír lifa.“\eva

\bvb To the hoary friend of Rungner \name{ettin} \ken*{= Hymer} seemed Loride’s meal well full-great; “next evening will we three by game-meat have to live.\footnoteB{The construction is difficult, but should probably be read in prose word order as \emph{vér þrír munum at ǫðrum aptni verða lifa við vęiðimat}, where \emph{verða} has a similar use as its modern German cognate \emph{werden}. Hymer’s stinginess—he refuses to share more of his own food, forcing his guests to go hunt—breaks all Indo-European rules of hospitality and illustrates the otherness of the Ettins. See Introduction to the poem.}”\evb
\evg


\bvg
\bva\mssnote{\Regius\ 14r/24, \AM\ 6r/21}Véurr kvaðsk vilja \hld\ á vág róa, &
ef ballr jǫtunn \hld\ bęitur gę́fi. &
„Hverf þú til \edtext{hjarðar}{\Afootnote{\emph{hallar} corr. \AM}}, \hld\ ef hug trúir, &
brjótr berg-Dana, \hld\ bęitur sǿkja.\eva

\bvb Wighward \name{= Thunder} called himself willing to row on the wave, if the baleful ettin might give pieces of bait. “Turn to the herd if thou trust in thy heart—breaker of boulder-Danes \ken*{\textsc{ettins} > = Thunder}!—to seek pieces of bait.\evb
\evg


\bvg
\bva\mssnote{\Regius\ 14r/26, \AM\ 6r/23}Þess \edtext{vę́ntir mik}{\Afootnote{\emph{vę́nti ek} \Regius}}, \hld\ at þér \edtext{mynit}{\lemma{mynit ‘will not’}\Afootnote{thus \AM; \emph{myni} ‘will’ \Regius. I prefer the \AM\ reading since it makes this the first of Hymer’s several challenges to Thunder, ones which the god easily accomplishes.}} &
ǫgn at oxa \hld\ auðfeng vesa.“ &
Svęinn sýsliga \hld\ svęif til skógar, &
þar’s oxi stóð \hld\ alsvartr fyrir.\eva

\bvb I expect that the oxen for bait will not be an easy catch for thee.”—The swain \name{= Thunder} sharply turned to the woods, there where an ox stood, all-black, before [him].\evb
\evg


\bvg
\bva\mssnote{\Regius\ 14r/28, \AM\ 6r/24}Braut af þjóri \hld\ þurs ráðbani &
hǫ́tún ofan \hld\ horna tveggja. &
„Verk þikkja þín \hld\ verri myklu &
kjóla valdi \hld\ an kyrr sitir.“\eva

\bvb Off from the bull broke the counsel-slayer of the thurse \ken*{= Thunder} the high meadow of the two horns \ken{head} from above.—“Thy works seem far worse to the wielder of keels \ken*{= Hymer = me}, than if thou calm did sit.\footnoteB{Hymer snidely belittles Thunder’s feat of pulling off the head of the ox (presumably by the horns).}”\evb
\evg


\bvg
\bva\mssnote{\Regius\ 14r/30, \AM\ 6r/26}Bað hlunngota \hld\ hafra dróttinn &
\edtext{áttrunn}{\Afootnote{\emph{†atrænn†} \AM}} apa \hld\ útar fǿra, &
ęn sá jǫtunn \hld\ sína \edtext{talði}{\Afootnote{\emph{milldi} (corr.) \AM}}, &
lítla fýsi \hld\ \edtext{lęngra at róa}{\Afootnote{metr. emend.; \emph{at róa lęngra} \Regius\AM}}.\eva

\bvb The lord of he-goats \ken*{= Thunder} bade the kinsman of the \inx[C]{ape}\footnoteB{The specific sense of \emph{api} is uncertain. It seems to generally refer to a fool, but see Encyclopedia.}\ \ken*{\textsc{ettin} = Hymer} to push the launching-steed \ken{boat} further out; but that ettin told of his scarce wish to row longer.\footnoteB{There is some humour in the situation as Hymer, who just mocked Thunder, is now forced to do his willing by rowing.}\evb
\evg


\bvg
\bva\mssnote{\Regius\ 14r/31, \AM\ 6r/27}Dró \edtext{mę́rr}{\lemma{mę́rr ‘renowned’}\Afootnote{thus \Regius; \emph{męirr} ‘more, further’ \AM}} Hymir \hld\ móðugr hvala &
ęinn á ǫngli \hld\ upp sęnn tváa, &
ęn aptr í skut \hld\ Óðni sifjaðr &
Véurr við vélar \hld\ vað gęrði sér.\eva

\bvb Pulled renowned Hymer—fierce—whales: one on the hook, soon up two; but back in the stern the Weden-related Wighward \name{= Thunder} wilily\footnoteB{Probably in the sense that he made the fishing line behind Hymer’s back when he was distracted pulling up the whales.} made himself a fishing-line.\evb
\evg


\bvg
\bva\mssnote{\Regius\ 14v/1, \AM\ 6r/29}Ęgnði á ǫngul \hld\ sá’s ǫldum bergr, &
orms ęinbani \hld\ oxa hǫfði; &
gęin við \edtext{agni}{\lemma{agni ‘bait’}\Afootnote{thus \AM; \emph{ǫngli} ‘hook’ \Regius}}, \hld\ sú’s goð fía, &
umbgjǫrð neðan \hld\ allra landa.\eva

\bvb On the hook fastened he who saves men \ken*{= Thunder}—the lone slayer of the Worm \ken*{= Thunder}—the head of the ox. At the bait snapped the one whom the gods hate \ken*{= Middenyardsworm}—the encircler of all lands\footnoteB{This kenning occurs identically in a fragment by 9th century scold Alewigh Snub (Ǫlv \emph{Þórr}, edited by Margaret Clunies Ross in \emph{SkP} III).} \ken*{= Middenyardsworm}—from below.\evb
\evg


\bvg
\bva\mssnote{\Regius\ 14v/3, \AM\ 6v/1}Dró djarfliga \hld\ dáðrakkr Þórr &
orm ęitrfáan \hld\ upp at borði; &
hamri kníði \hld\ hǫ́fjall skarar &
ofljótt ofan \hld\ ulfs hnitbróður.\eva

\bvb Pulled boldly deed-bold Thunder the venom-glistening Worm up on the gunwale; with the hammer he struck the high mountain of hair\footnoteB{A rather unfitting kenning, since serpents do not have hair.} \ken{head}—very hideous, from above—on the clash-brother of the Wolf \ken*{= Middenyardsworm}.\evb
\evg


\bvg
\bva\mssnote{\Regius\ 14v/5, \AM\ 6v/2}\edtext{Hraungǫlkn}{\lemma{hraungǫlkn ‘lavafield-monsters’}\Bfootnote{Both mss. have \emph{hręin-}, which may mean either ‘clean’ or ‘reindeer’, neither of which fit. On the other hand \emph{hraun} \ONP: ‘stone/barren area, wasteland; lava-field’ is well attested in Scoldish kennings for ettins. The precise meaning of \emph{galkn} ‘monster’ (plural \emph{gǫlkn}) is unclear; but it is attested in three Scoldish verses, always in kennings of the type “troll-woman of the shield \ken{axe}”. While the mss. ‘\emph{galkn}’ (norm. \emph{gálkn}) could be both singular and plural, the form of the verb precludes the former. This means that the word cannot be referring to the Middenyardsworm, refuting the interpretation of \textcite{LarringtonEdda}: “the sea-wolf shrieked”.}} \edtext{hrutu}{\Afootnote{thus \AM; \emph{hlumðu} ‘dashed’ \Regius}}, \hld\ ęn hǫlkn þutu, &
fór hin forna \hld\ fold ǫll saman; &
søkkðisk síðan \hld\ sá fiskr í mar.\eva

\bvb The lavafield-monsters \ken{ettins} bounded, but the bedrock resounded; moved the ancient earth all at once; sank thereafter that fish \ken*{= Middenyardsworm} into the sea.\evb
\evg


\bvg
\bva\mssnote{\Regius\ 14v/6, \AM\ 6v/3}Ótęitr jǫtunn, \hld\ es aptr røru, &
\skipnumbering\edtext{[...]}{\Bfootnote{There is without doubt a line missing here; the meter usually requires four lines, and the first half of the sentence is incomplete without a verb (unless one understands an implied “was”, so that the verse would begin “Unmerry was the ettin”).}} &
svá’t \edtext{ár}{\lemma{\emph{ár} ‘in the early morning’}\Bfootnote{\textcite{FinnurEdda}\ suggests \emph{svá’t at ǫ́r} ‘so that by the oar’. Assuming my interpretation is correct, the three would have been fishing}} Hymir \hld\ ękki mę́lti, &
vęifði rǿði \hld\ veðrs annars til.\eva

\bvb The unmerry ettin \ken*{= Hymer}, as they rowed back, [...], so that in the early morning Hymer spoke nothing; he pulled the oar around, against the storm:\evb
\evg


\bvg {\small [Hymer quoth:]}
\bva\mssnote{\Regius\ 14v/8, \AM\ 6v/4}„Munt of vinna \hld\ verk halft við mik, &
at hęim hvala \hld\ haf til bǿjar &
eða flotbrúsa \hld\ fęstir okkarn.“\eva

\bvb “Thou wilt win a half work by me if thou carry the whales home to the farm, or our float-jar \ken{boat} do fasten.\footnoteB{Hymer tells Thunder, who since he did not actually pull up the Worm now has nothing to show for the trip, that he can accomplish something half as good as the pulling of the whales if he carries them home, or if he fastens the boat (by the shore).}”\evb
\evg


\bvg
\bva\mssnote{\Regius\ 14v/9, \AM\ 6v/6}Gekk Hlórriði \hld\ gręip \edtext{á}{\Afootnote{\emph{til á} \Regius}} stafni &
vatt með austri \hld\ upp lǫgfáki; &
ęinn með ǫ́rum \hld\ ok með austskotu &
bar til bǿjar \hld\ brimsvín jǫtuns &
ok \edtext{holtriða}{\Afootnote{\emph{†holtriba†} \Regius}} \hld\ hver í gegnum. \eva

\bvb Went Loride \name{= Thunder}, grasped the stern; hurled with the bilge-water the lake-nag \ken{boat} up.\footnoteB{Thunder did not pour the bilge-water, something that makes its weight considerably heavier, out of the boat. This was a great work of strength.} Alone with the oars and the bilge-bucket he bore to the farm the brim-swines \ken{whales} of the ettin, even through the cauldron of woodland ridges\footnoteB{TODO. What do other editors and translators say?} \ken{valley?}.\evb
\evg


\bvg
\bva\mssnote{\Regius\ 14v/12, \AM\ 6v/7}\edtext{Ok}{\Afootnote{\emph{enn} \AM}} ęnn jǫtunn \hld\ umb afręndi, &
þrágirni vanr, \hld\ við Þór sęnti, &
kvað-at mann ramman, \hld\ þótt róa kynni, &
krǫpturligan, \hld\ nema kalk bryti.\eva

\bvb And yet the ettin, used to stubbornness, regarding strength of hand flyted with Thunder; he called not the man strong—although he could row, mightily—unless he broke the chalice.\footnoteB{Hymer accuses Thunder of weakness, refusing to call him strong unless he breaks a certain chalice.}\evb
\evg


\bvg
\bva\mssnote{\Regius\ 14v/14, \AM\ 6v/9}Ęn Hlórriði, \hld\ es at hǫndum kom, &
brátt lét bresta \hld\ brattstęin glęri, &
sló sitjandi \hld\ súlur í gǫgnum; &
bǫ́ru þó hęilan \hld\ fyr Hymi síðan.\eva

\bvb But Loride \name{= Thunder}, when [it] came in his hands, impatiently crashed steep stone\footnoteB{\textcite{FinnurEdda} interprets the word as referring to stone pillars.} with the glass \ken*{= chalice}; he struck right through the fastened\footnoteB{\emph{sitjandi} ‘sitting’ is ambiguous and can modify either Thunder or the (roof-bearing) pillars. I think it is more likely to modify the pillars, signifying their stability.} pillars; yet they \ken*{= Hymer’s servants?} bore it whole before Hymer afterwards.\evb
\evg


\bvg
\bva\mssnote{\Regius\ 14v/16, \AM\ 6v/10}Unz þat hin fríða \hld\ friðla kęndi &
ástráð mikit, \hld\ ęitt es vissi, &
„drep við haus Hymis, \hld\ hann ’s harðari, &
kostmóðs jǫtuns, \hld\ kalki hvęrjum.“\eva

\bvb Until the handsome mistress \ken*{= Tue’s mother} gave a great loving counsel, the one she knew: “Strike against Hymer’s skull; it is harder—on the choice-weary\footnoteB{A reference to the gods having eaten up his choicest food.} ettin—than every chalice.”\evb
\evg


\bvg
\bva\mssnote{\Regius\ 14v/18, \AM\ 6v/12}Harðr \edtext{ręis}{\Afootnote{om. \AM}} á kné \hld\ hafra dróttinn, &
fǿrðisk allra \hld\ í ásmęgin; &
hęill vas karli \hld\ hjalmstofn ofan, &
ęn vínfęrill \hld\ valr rifnaði.\eva

\bvb Hard rose on the knees the lord of he-goats \ken*{= Thunder}; he summoned his highest os-might.\footnoteB{Compare \Gylfaginning\ in its description of Thunder attempting to pull up the Worm: \emph{Þá varð Þórr reiðr ok fę́rðist í ásmegin} “Then Thunder became wroth, and summoned his os-might.”} Whole was on the churl \ken*{= Hymer} the helmet-stump \ken{head} above, but the round wine-track \ken{chalice} rent apart.\evb
\evg


\bvg {\small [Hymer quoth:]}
\bva\mssnote{\Regius\ 14v/20, \AM\ 6v/13}„Mǫrg vęit’k mę́ti \hld\ mér gingin frá, &
\edtext{es}{\Afootnote{om. \Regius}} kalki sé’k \hld\ \edtext{fyr}{\Afootnote{\emph{†yr†} \Regius}} knéum hrundit,“ &
karl orð of kvað: \hld\ „kná’k-at sęgja &
aptr ę́vagi: \hld\ þú est ǫlðr of hęitt.\eva

\bvb “I know many good things to be gone from me when I see the chalice thrown before [his] knees;”—the churl \ken*{= Hymer} then words did speak: “I cannot say it, ever again: ‘Thou art, ale, [well] heated!\footnoteB{Hymer laments that since his finest vessel is now broken, he will never again be able to enjoy strong drink.}’\evb
\evg


\bvg
\bva\mssnote{\Regius\ 14v/22, \AM\ 6v/15}Þat ’s til kostar \hld\ ef koma mę́ttið &
út ór óru \hld\ ǫlkjól hofi.“ &
Týr lęitaði \hld\ tysvar hrǿra; &
stóð at hvǫ́ru \hld\ hverr kyrr fyrir.\eva

\bvb It would be well done, if ye might make the ale-keel\footnoteB{\emph{ǫlkjól} is the accusative form, but in this sense (\CV: \emph{koma}, B) we would expect the dative \emph{ǫlkjóli}, something that the meter does not allow for.} \ken{cauldron} to come out of our hall.\footnoteB{\emph{hof} ‘hall’ usually means ‘hove; temple’.}” Tue attempted, twice, to move it; stood nevertheless the cauldron still before [him].\evb
\evg


\bvg
\bva\mssnote{\Regius\ 14v/24, \AM\ 6v/16}Faðir Móða \hld\ fekk á þręmi &
ok í gǫgnum sté \hld\ golf niðr í sal; &
hóf sér á hǫfuð upp \hld\ hver Sifjar verr, &
ęn á hę́lum \hld\ hringar skullu.\eva

\bvb The father of Moody \ken*{= Thunder} grasped the brim, and stepped down through the floor in the hall;\footnoteB{In the account of \Gylfaginning\ Thunder is said to have stepped through the boat when trying to pull up the Middenyardsworm. This detail is also seen on the carving of the Altuna stone from Uppland, Sweden; it may have been transposed to this place in the narrative.} heaved the husband of Sib \ken*{= Thunder} up onto his head the cauldron, but on his heels rings clattered.\footnoteB{The rings from the cauldron-chain; this detail is mentioned in an example sentence contrasting long and short phonemes in \FGT: \emph{heyrði til hǫddu, þá er Þórr bar hverinn} “one heard the pot-links when Thunder bore the kettle”. According to \textcite{FinnurEdda}\ this chain reached from one end of the kettle to another, in which case this would be an oblique reference to the cauldron’s size, its diameter being the same as Thunder’s height.}\evb
\evg


\bvg
\bva\mssnote{\Regius\ 14v/26, \AM\ 6v/18}Fóru-t lęngi, \hld\ áðr líta nam &
aptr Óðins sonr \hld\ ęinu sinni; &
sá hann ór hręysum \hld\ með Hymi austan &
folkdrótt fara \hld\ fjǫlhǫfðaða.\eva

\bvb They journeyed not for long before the son of Weden \ken*{= Thunder} took to look back, a single time;—saw he out of stone-heaps, with Hymer from the east, a many-headed folk-troop \ken{= ettins} journeying.\evb
\evg


\bvg
\bva\mssnote{\Regius\ 14v/28, \AM\ 6v/19}Hóf sér af hęrðum \hld\ hver standandi, &
vęifði Mjǫlni \hld\ morðgjǫrnum framm, &
ok hraunhvala \hld\ hann alla drap.\eva

\bvb Heaved he off from his shoulders the cauldron, [while] standing; he swung the murder-eager Millner forth, and the rock-whales \ken{= ettins} all he slew.\evb
\evg


\bvg
\bva\mssnote{\Regius\ 14v/30, \AM\ 6v/21}Fóru-t lęngi, \hld\ áðr liggja nam &
hafr Hlórriða \hld\ halfdauðr fyrir, &
vas \edtext{skę́r}{\Afootnote{emend. from meaningless \emph{†skirr†} \Regius\AM}} skǫkuls \hld\ skakkr á bęini, &
ęn því hinn lę́vísi \hld\ Loki of olli.\eva

\bvb They journeyed not for long before the he-goat of Loride \name{= Thunder} took to lie half-dead before [them]; the steed of the cart-pole \ken{goat} was halt in the leg, but that the deceitful Lock did cause.\footnoteB{Apparently Lock (who is not mentioned earlier in the poem) was placing curses on the returning party. Snorre mentions this, TODO.}\evb
\evg


\bvg
\bva\mssnote{\Regius\ 14v/32, \AM\ 6v/22}Ęn ér hęyrt hafið, \hld\ hvęrr kann of þat &
goðmǫ́lugra \hld\ gørr at skilja, &
hvęr af hraunbúa \hld\ hann laun of fekk, &
es bę́ði galt \hld\ bǫrn sín fyrir.\eva

\bvb But ye have heard—each god-knowledgeable\footnoteB{\emph{goð-mǫ́lugr} ‘able to speak about the god-lore; versed in the mythology’ is a \emph{hapax}.} man knows about this more clearly discern—which rewards he \ken*{= Lock} from the rock-dweller \ken{ettin} got, as he yielded up both his own children for it.\footnoteB{As pointed out in \textcite{FinnurEdda}\, a verse containing such an address to the audience is otherwise unheard of. — What myth is being referred to is unclear. TODO: What do other authors write}\evb
\evg


\bvg
\bva\mssnote{\Regius\ 15r/1, \AM\ 6v/24}Þróttǫflugr kom \hld\ á þing goða &
ok hafði hver, \hld\ þann’s Hymir átti; &
ęn véar hvęrjan \hld\ vęl skulu drekka &
ǫlðr at Ę́gis \hld\ \edtext{ęitt hǫrmęitið}{\lemma{ęitt hǫrmęitið “one \dots\ flax-cutting”}\Bfootnote{A very obscure kenning. \textcite{LaFargeGlossary} give several interpretations, viz. \emph{ęitr-hǫr-męitir} ‘poison-rope-cutter \ken{snake > winter}’, \emph{ęitr-orm-męiðir} ‘poison-worm-injurer’ \ken{winter}. The solution with the minimal amount of emendation is to read \emph{ęitt} ‘one’ as modifying \emph{ǫlðr} ‘ale-feast’, and \emph{hvęrjan} ‘every’ as modifying \emph{hǫr-męitiðr} ‘flax-cutting’, a compound made up of \emph{hǫrr} ‘flax, cord’ and \emph{męita} ‘to cut’ and referring to an obscure harvest festival. The interpretation is by no means certain.}}.\eva

\bvb The valour-mighty one \ken*{= Thunder} came onto the \inx[C]{Thing} of the gods, and had that cauldron which Hymer owned; but the \inx[G]{Wigh-beings} \name{= gods} shall well drink an ale-feast at Eagre’s, every flax-cutting \ken{fall?}.\evb
\evg
