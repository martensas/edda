Þórr dró Miðgarðsorm. TO-DO: Format as header.

Thunder pulled the Middenyardsworm.\footnotemark[1]
\footnotetext[1]{Clearly not part of the original poem, this is the only title (written with red ink) the poem has in \emph{R}. \emph{748} has the proper title \emph{Hymiskviða} instead.}

\bvg
\bva Ár valtívar \hld vęiðar nǫ́mu &
ok sumblsamir \hld áðr saðir yrði, &
hristu tęina \hld ok á hlaut sǫ́u, &
fundu þęir at Ægis \hld ørkost hvera.\eva

\bvb Of yore the slaughter-Tues had hunted game\footnoteB{Lit. ‘took game’}, and banqueting before they might eat\footnoteB{Lit. ‘before they might become sated’}, they shook the twigs and looked at the \inx{leat}; they found at Eyer’s a great choice of cauldrons.\footnoteB{The gods sprinkled the leat (sacrificial blood) from the caught animals and interpreted the pattern; they found it most auspicious to feast at Eyer’s.}


\bvg
\bva Sat bergbúi \hld barntęitr fyr, &
mjǫk glíkr męgi \hld Miskorblinda, &
lęit í augu \hld Yggs barn í þrá: &
“þú skalt ǫ́sum \hld opt sumbl gęra!”\eva

\bvb — Sat the mountain-dweller \ken{Eyer}[1] there, joyous like a child, much like the lad of Misherblind\footnoteB{A reference to a lost myth? Unless Misherblind is an alternative name for Firneet, Eyer’s father.}; into his eyes looked the child of Ug <= Weden> \ken{Thunder}[1] in defiance: “Thou shalt for the Ease oft’ host banquets!”\footnoteB{Having seen that Eyer has a great store of cauldrons, Thunder orders him to host future banquets for the Ease.}\evb
\evg


\bvg
\bva Ǫnn fekk jǫtni \hld orðbæginn halr, &
hugði at hefndum \hld hann næst við goð, &
bað hann Sifjar ver \hld sér fǿra hver, &
“þann’s ek ǫllum ǫl \hld yðr of hęita”.\eva

\bvb Great toil for the ettin the word-peevish man \ken{Thunder}[1] caused; thought he of revenge, soon, against the god: asked he Sib’s husband to bring him a cauldron, “that one with which I for you all ale might brew.”\footnoteB{Eyer asks Thunder to find a single cauldron which can hold enough ale to supply all the Ease.}
\evg


\bvg
\bva Né þat mǫ́ttu \hld mærir tívar &
ok ginnręgin \hld of geta hvęrgi, &
unz af tryggðum \hld Týr Hlórriða &
ástráð mikit \hld ęinum sagði:\eva

\bvb But that might the renowned Tues and the \inx{Gin-Reins} nowhere get ahold of, until out of loyalty, a great word of loving advice Tue to Loride <= Thunder> alone did say:\evb
\evg


\bvg
\bva “Býr fyr austan \hld Élivága &
hundvíss Hymir \hld at himins ęnda, &
á minn faðir \hld móðugr kętil, &
rúmbrugðinn hver \hld rastar djúpan.”\eva

\bvb “Lives to the east of the Ilewaves the houndwise Hymer, at the end of heaven. Owns my father\footnoteB{Hymer being Tue’s father.}, fierce, a kettle; a size-renowned cauldron one \inx{rest} deep.\evb
\evg


\bvg
\bva “Veiztu, ef þiggjum \hld þann lǫgvelli?” &
“Ef, vinr, vélar \hld vit gørvum til!”\eva

\bvb “Knowest thou if we will receive that ale-boiler?” — “If, friend, we two make use of wiles!” \evb
\evg

\bvg
\bva Fóru drjúgum \hld dag þann fram &
Ásgarði frá \hld unz til Ęgils kvǫ́mu. &
Hirði hann hafra \hld horngǫfgasta; &
hurfu at hǫllu \hld es Hymir átti.\eva

\bvb — They travelled quickly throughout the day, from Osyard, until to Agle’s they came; he was herding bucks with the noblest of horns; they turned to the hall which Hymer owned.\evb
\evg


\bvg
\bva Mǫgr fann ǫmmu, \hld mjǫk lęiða sér, &
hafði hǫfða \hld hundruð níu. &
ęn ǫnnur gekk \hld algollin framm &
brúnhvít bera \hld bjórvęig syni.\eva

\bvb The lad found his grand-mother greatly loathsome; she had of heads nine hundred. But another woman, all-golden, stepped forth: white-browed, she carried a beer-draught for her son.\evb
\evg

