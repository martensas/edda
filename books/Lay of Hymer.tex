\bookStart{\emph{Hymiskviða} — The Lay of Hymer.}

% Introduction.
Attested in two manuscripts, \Regius\ and \AM. The two are surprisingly consistent.

Þórr dró Miðgarðsorm. % TO-DO: Format as header.

Thunder pulled the Middenyardsworm.\footnotetext{This is the only title the poem has in \Regius. \AM\ has the proper title \emph{Hymiskviða} instead.}


\bvg
\bva Ár valtívar \hld vęiðar nǫ́mu &
ok sumblsamir \hld áðr saðir yrði, &
hristu tęina \hld ok á hlaut sǫ́u, &
fundu þęir at Ægis \hld ørkost hvera.\eva

\bvb Of yore the slaughter-Tues had caught game\footnoteB{Lit. ‘took game’}, and banqueting before they might eat\footnoteB{Lit. ‘might become sated’}, they shook the twigs and looked at the \inx{leat}; they found at Eyer’s a great choice of cauldrons.\footnoteB{The gods sprinkled the leat (sacrificial blood) of the beasts and interpreted the pattern; they found it most auspicious to feast at Eyer’s.}\evb
\evg


\bvg
\bva Sat bergbúi \hld barntęitr fyr, &
mjǫk glíkr męgi \hld Miskorblinda, &
lęit í augu \hld Yggs barn í þrá: &
“þú skalt ǫ́sum \hld opt sumbl \edtext{gęra}{\lemma{gęra “host”}\Afootnote{gefa “give” \AM}}!”\eva

\bvb — Sat the mountain-dweller \ken{Eyer}[1] there, joyous like a child, much like the lad of Misherblind\footnoteB{A reference to a lost myth? Unless Misherblind is an alternative name for Firneet, Eyer’s father.}; into his eyes looked the child of Ug <= Weden> \ken{Thunder}[1] in defiance: “Thou shalt for the Ease oft’ host banquets!”\footnoteB{Having seen that Eyer has a great store of cauldrons, Thunder orders him to host future banquets for the Ease.}\evb
\evg


\bvg
\bva Ǫnn fekk jǫtni \hld orðbæginn halr, &
hugði at hefndum \hld hann næst við goð, &
bað hann Sifjar ver \hld sér fǿra hver, &
“þann’s ek ǫllum ǫl \hld yðr of hęita”.\eva

\bvb Great toil for the ettin the word-peevish man \ken{Thunder}[1] caused; thought he of revenge, soon, against the god: asked he Sib’s husband to bring him a cauldron, “that one with which I for you all ale might brew.”\footnoteB{Eyer asks Thunder to find a single cauldron which can hold enough ale to supply all the Ease.}
\evg


\bvg
\bva Né þat mǫ́ttu \hld mærir tívar &
ok ginnręgin \hld of geta hvęrgi, &
unz af tryggðum \hld Týr Hlórriða &
ástráð mikit \hld ęinum sagði:\eva

\bvb But that might the renowned Tues and the \inx{Gin-Reins} nowhere get ahold of, until out of loyalty, a great word of loving advice Tue to Loride <= Thunder> alone did say:\evb
\evg


\bvg
\bva “Býr fyr austan \hld Élivága &
hundvíss Hymir \hld at himins ęnda, &
á minn faðir \hld móðugr kętil, &
\edtext{rúmbrugðinn}{\Afootnote{‘rumbrygðan’ \AM}} hver \hld rastar djúpan.”\eva

\bvb “Lives to the east of the Ilewaves the houndwise Hymer, at the end of heaven. Owns my father\footnoteB{Hymer being Tue’s father.}, fierce, a kettle; a size-renowned cauldron one \inx{rest} deep.”\evb
\evg


\bvg
\bva “Veiztu, ef þiggjum \hld þann lǫgvelli?” &
“Ef, vinr, vélar \hld vit gørvum til!”\eva

\bvb “Knowest thou if we will receive that ale-boiler?” — “If, friend, we two make use of wiles!”\footnoteB{The speakers are not indicated, but it is most sensible that Thunder asks and Tue answers.}\evb
\evg

\bvg
\bva Fóru drjúgum \hld \edtext{dag þann framan}{\lemma{dag þann framan “from the beginning of the day”}\Afootnote{\emph{Emendation from Finnur 1932}; dag þann fram “on that day forth” \Regius; dag fráliga “swiftly at day” \AM}} &
Ásgarði frá \hld unz til \edtext{Ęgils}{\lemma{Ęgils “Agle’s”}\Afootnote{\emph{thus} \Regius; Ægis “Eyer’s” \AM; — \AM\ \emph{reading possibly from confusion with Eyer described earlier in the poem, but or the shepherd did share his name.}}} kvǫ́mu. &
Hirði hann hafra \hld horngǫfgasta; &
hurfu at hǫllu \hld es Hymir átti.\eva

\bvb — They travelled with great strides from the beginning of the day, from Osyard, until to Agle’s they came—he herded bucks with the noblest of horns—they turned to the hall which Hymer owned.\evb
\evg


\bvg
\bva Mǫgr fann ǫmmu, \hld mjǫk lęiða sér, &
hafði hǫfða \hld hundruð níu. &
ęn ǫnnur gekk \hld algollin framm &
brúnhvít bera \hld bjórvęig syni.\eva

\bvb The lad found his grandmother greatly loathsome; she had of heads nine hundred. But another woman, all-golden, stepped forth: white-browed, she carried a beer-draught for the son \ken{Tue}[1].\evb
\evg


\bvg
\bva Áttniðr jǫtna \hld ek vilja’k ykr &
hugfulla tvá \hld und hvera sętja; &
es mínn \edtext{fríi}{\lemma{fríi “lover”}\Afootnote{\emph{thus} \Regius; faðir “father” \AM}} \hld mǫrgu sinni &
gløggr við gęsti \hld gǫrr ills hugar.\eva

\bvb “Kinsman of ettins! I would wish to set you high-mettled two under the cauldrons; my lover has many a time been stingy against guests, quick to ill temper.”\footnoteB{Tue’s mother (the all-golden woman in previous v.) wishes to hide him and Thunder, lest her husband (Hymer) find them.}\evb
\evg


\bvg
\bva Ęn váskapaðr \hld varð \edtext{síðbúinn}{\Afootnote{\emph{om.} \AM}}, &
harðráðr Hymir, \hld hęim af vęiðum; &
gekk inn í sal, \hld glumðu jǫklar, &
vas karls, es kom, \hld kinnskógr frørinn.\eva

\bvb But the misshapen one was come late—the hard-minded Hymer—home from the hunt. He entered the hall—icicles clattered—frozen was the cheek-forest \ken{beard} of the churl who came.\evb
\evg


\bvg
\bva Ves þú hęill, Hymir, \hld í hugum góðum! &
Nú ’s sonr kominn \hld til sala þinna, &
sá’s vit vættum \hld af vęgi lǫngum; &
fylgir hǫ́num \hld Hróðrs andskoti, &
vinr verliða; \hld Véurr hęitir sá.\eva

\bvb “Be thou hale, Hymer, in good spirits!\footnoteB{Formula identically mirrored in runic inscription N B380: \emph{Heill sé þú / ok í hugum góðum. / Þórr þik þiggi, / Óðinn þik eigi.} “May thou be hale, and in good spirits! May Thunder receive thee, may Weden own thee.” Cf. also \Beowulf\ l. 407: \emph{Wæs þú Hróðgár hál!} “Be thou, Rothgar, hale!”} Now the son is come to thy halls, the one whom we two have been expecting, from a long way off. Follows him the opponent of Rooder <ettin> \ken{Thunder}[1], the friend of manly retinues \ken{Thunder}[1]; Wighward he is called.\evb
\evg


\bvg
\bva Sé þú hvar sitja \hld und salar gafli, &
svá \edtext{forða sér}{\Afootnote{forðask \AM}}, \hld stęndr \edtext{súl}{\Afootnote{‘sol’ \AM}} fyrir. &
Sundr stǫkk súla \hld fyr sjón jǫtuns, &
ęn \edtext{allr}{\Afootnote{áðr \Regius\AM TODO: elaborate, mention Finnur}} í tvau \hld áss brotnaði.\eva

\bvb See where they sit, ’neath the hall’s gable: thus they hide themselves—a pillar stands before them!” The pillars sprang asunder before the sight of the ettin, but all in two the beam was broken.\evb
\evg


\bvg
\bva Stukku átta, \hld ęn ęinn af þęim &
hverr harðslęginn \hld hęill af þolli; &
framm gingu þęir, \hld ęn forn jǫtunn &
sjónum lęiddi \hld sinn andskota.\eva

\bvb Eight\footnoteB{Eight kettles.} sprung apart, but one of them, a hard-forged kettle, [came] whole off its peg\footnoteB{Presumably the one in which Tue and Thunder were hiding.}. Forth went they, but the ancient ettin with his sight beheld\footnoteB{Literally “led with his sight”.} his opponent.\evb
\evg


\bvg
\bva Sagðit hǫ́num \hld hugr vęl þá’s sá &
gýgjar \edtext{grǿti}{\lemma{grǿti “distresser”}\Afootnote{gæti “keeper, warder” \AM}} \hld á golf kominn, &
þar vǫ́ru þjórar \hld þrír of tęknir, &
bað \edtext{sęnn}{\Afootnote{‘sun’ \AM}} jǫtunn \hld sjóða ganga.\eva

\bvb His heart was not pleased then, when he saw the distresser of Gows <ettin-women> \ken{Thunder}[1] come on the floor. There were three bulls taken: the ettin at once bade them be cooked.\evb
\evg


\bvg
\bva Hvęrn létu þęir \hld hǫfði skęmra &
ok á sęyði \hld síðan bǫ́ru, &
át Sifjar verr \hld áðr sofa gingi, &
ęinn með ǫllu \hld øxn tvá Hymis.\eva

\bvb Each one they let shorten by a head, and then bore onto the fire-pit; ate the husband of Sib \ken{Thunder}[1], before he might go to sleep, all together alone, two of Hymer’s oxen.\evb
\evg
