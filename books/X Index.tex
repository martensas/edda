Poems (by CR order)

\emph{Wal} — Spae of the Wallow (\emph{Vǫluspǫ́})
\emph{High} — Speeches of the High One (\emph{Hávamǫ́l})
\emph{Web} — Speeches of Webthrithen (\emph{Vafþrúðnismǫ́l})
\emph{Reed} — From the Sons of king Reeding (\emph{Frá sonum Hrauðungs konungs})
\emph{Grim} — Speeches of Grimen (\emph{Grímnismǫ́l})
\emph{Shr} — Speeches of Shearen (\emph{Skírnismǫ́l})
\emph{Hoar} — Leed of Hoarbeard (\emph{Hárbarðsljóð})
\emph{Hyme} — Lay of Hyme (\emph{Hymiskviða})
(\emph{Frá Ægi ok goðum})
(\emph{Lokasęnna})
(\emph{Frá Loka})
\emph{Thr} — Lay of Thrim (\emph{Þrymskviða})
(\emph{Frá Vǫlundi})
\emph{Way} — Lay of Wayland (\emph{Vǫlundarkviða})
\emph{Alw} — Speeches of Allwise (\emph{Alvíssmǫ́l})
\emph{I HHb} — First Lay of Hallow Hundingsbane (\emph{Helgakviða Hundingsbana I})
\emph{HHw} — Lay of Hallow Herwardsson (\emph{Helgakviða Hjǫrvarðssonar})
\emph{II HHb} — Second Lay of Hallow Hundingsbane (\emph{Helgakviða Hundingsbana II})
(\emph{Grípisspǫ́})
(\emph{Reginsmǫ́l})
(\emph{Fáfnismǫ́l})
(\emph{Sigrdrífumǫ́l})
\emph{SiL} — Fragment of the Lay of Siward (\emph{Brot af Sigurðarkviða})
\emph{SDe} — From the Death of Siward. (\emph{Frá dauða Sigurðar})
\emph{I Gr} — First Lay of Guthrun (\emph{Guðrúnarkviða I})
\emph{ShS} — Short Lay of Siward (\emph{Sigurðarkviða in skamma})
\emph{BHel} — Hellride of Byrnhild (\emph{Helreið Brynhildar})
\emph{Nifl} — The Killing of the Niflings (\emph{Dráp Niflunga})
\emph{II Gr} — Second Lay of Guthrun (\emph{Guðrúnarkviða II})
\emph{III Gr} — Third Lay of Guthrun (\emph{Guðrúnarkviða III})
\emph{BnOr} — From Burgny and Ordrun (\emph{Frá Borgnýju ok Oddrúnu})
\emph{OdW} — Weeping of Ordrun (\emph{Oddrúnargrátr})
\emph{AtD} — Death of Attle (\emph{Dauði Atla})
\emph{AtL} — Lay of Attle (\emph{Atlakviða})
\emph{AtS} — Speeches of Attle (\emph{Atlamǫ́l})
\emph{FGr} — From Guthrun (\emph{Frá Guðrúnu})
\emph{GrB} — Boldness of Guthrun (\emph{Guðrúnarhvǫt})
\emph{Ham} — Speeches of Hamthew (\emph{Hamðismǫ́l})
\emph{Hind} — The Leed of Hindle (\emph{Hyndluljóð})



Cultural and religious terms

† \textbf{aught} (ON. \emph{ætt}, OE. \emph{ǣht})
 The Nordic (paternal) clan or family line.

† \textbf{fimble-} (ON. \emph{fimbul-})
 The ultimate, final, greatest. See \textbf{Fimble-thyle}, \textbf{Fimble-winter}.

† \textbf{hame} (ON. \emph{hamr})
 A skin, shape. Individuals can through magic “shift hames” (ON. \emph{skipta hǫmum}), and leave their human \emph{hames} behind, instead entering into the shapes of wolves, bears, birds. During this process the original hame would be sleeping in a vulnerable state, as described in the Saw of the Walsings, chap. TO-DO: . See also \textbf{feather-hame}, \textbf{town-riders}, \textbf{evening-riders}.

† \textbf{harrow} (ON. \emph{hǫrgr}, OE. \emph{hearg}, NWGmc. \emph{*harugaʀ})
 A cairn constructed for ritual purposes. \emph{Hind} 10 describes one: “A \textbf{harrow} he made for me, loaded with stones; now that stone-pile is become into glass. He reddened [it] in fresh blood of oxen; Oughthere ever trusted on the osennies.” See also \textbf{wigh}.

† \textbf{leed} (ON. \emph{ljóð}, OE. \emph{lēod})
 A song or chant with magical qualities.

† \textbf{thyle} (ON. \emph{þulr}, OE. \emph{þyle}, NWGmc. \emph{*þuliʀ})
 A sage who through rote learning has acquired a large amount of mythological lore (cf. \emph{þula} 'a list in poetic form; a meaningless poem' and \emph{þylja} 'to recite, to chant'). Thus \textbf{Weden} is the \textbf{Fimble-thyle}, being the unbeaten master of lore, as can be seen in his wisdom contests (see \emph{Alw}, \emph{Web}). Runic inscription DR 248 (Snoldelev) suggests the thyle was somehow bound to a specific place, and in Beowulf it seems to have been a court position, with \textbf{Unferth} being described as the "thyle of Rothgore".

† \textbf{wigh} (ON. \emph{vé}, OE. \emph{wēoh}, \emph{wīh}, NWGmc. \emph{*wīhą})
 A holy shrine or sanctuary. It seems that where the \textbf{harrow} was a pile of stones or cairn used for carrying out rituals, the \emph{wigh} was an enclosed space. The earliest Norse attestation is the runic inscription Ög N288 (Oklunda), which reads: “Guthhere <= Gunnarr> painted these runes, and he fled, guilty. Sought this wigh, and he fled into this clearing. And he bound. [...]” The implication seems to be that the wigh was considered so sacred that Guthhere could not be apprehended or punished for his crime while in it. — In Old English the word means ‘pagan idol’. It is not immediately clear which meaning is the original one, but in this edition the Norse sense has been adopted, since the Anglo-Saxon sources are all of a Christian nature. The \emph{Beewolf} name \emph{Wighstone} (\emph{Wīh-} or \emph{Wēohstān}) in any case suggests it is the Norse meaning, since ‘idol-stone’ makes little sense.

† \textbf{wode} (ON. \emph{óðr}, OE. \emph{wōd}, NWGmc. \emph{*wōþuʀ})
 \textbf{Hean}'s gift to men, though the name would suggest it be from \textbf{Weden}. The word has several related meanings: ‘poetic inspiration’, ‘madness’, ‘rage’.



Individuals

† \textbf{Earp and Oatle} (ON. \emph{Erpr ok Eitill}})
 The sons of \textbf{Attle} and \textbf{Guthrun}.

† \textbf{Attle} (\emph{Attila}, ON. \emph{Atli}, OE. \emph{Ætla}, MHG. \emph{Etzel}, NWGmc. \emph{*Attilō})
 The ruler of the \textbf{Huns} (historically from 434–453). Husband of \textbf{Guthrun}, and with her father of \textbf{Earp and Oatle}. and murderer of
 I HHb 54, SiL 11, I Gr 23, ShS 28, 29, 33, 37, 54, 56, 57, II Gr 26, 38, 45, III Gr 1, 9, BnOr 0, OdW A, 2, 22, 23, 25, 26, 30, 31, AtD 0, AtL 1, 3, 15, 17, 18, 27, 31, 32, 34, 36, 37, 38, 41, 43, B, AtS 2, 4, 21, 22, 44, 52, 60, 64, 71, 73, 77, 80, 86, 87, 97, 98, 108, 113, 117, FGr 0, GrB 12, Ham 6.

† \textbf{Guthrun} (ON. \emph{Guðrún})
 Daughter of king \textbf{Yivick}, sister of \textbf{Guthhere} and \textbf{Hain}. The wife of \textbf{Attle}.

† \textbf{Hain} 1 (ON. \emph{Hǫgni}, OE. \emph{Haguna}, \emph{Hagena}, OHG. \emph{Hagano}, Ger. \emph{Hagen}, NWGmc. \emph{*Hagunō})
 A \textbf{Nifling} and \textbf{Yifking}, son of king \textbf{Yivick}, brother of \textbf{Guthhere} and \textbf{Guthrun}. In \emph{AtL} he defeats seven warriors before being captured by \textbf{Attle}, who has his heart cut out at the request of Guthhere.

† \textbf{Hain} 2 A petty king of \textbf{East Geatland}, contemporary with \textbf{Granmer}, the king of \textbf{Southmanland} and Ingeld Illrede, the \textbf{Ingling} king of \textbf{Upland}.

† \textbf{Hindle} A witch awoken by Frow in \emph{Hind}.

† \textbf{Oughthere} (ON. \emph{Óttarr}, OE. \emph{Ōhthere}, NWGmc. \emph{*Ōhtaharjaʀ})
 TO-DO

† \textbf{Rotholf} (ON. \emph{Hrólfr kraki}, OE. \emph{Hrōþulf}, NWGmc. \emph{*Hrōþiwulfaʀ})
 A king of the \textbf{Shieldings}, son of \textbf{Hallow} and \textbf{Irse}, and nephew of \textbf{Rothgore}. As foreshadowed in \emph{Beewolf} (1017–9, 1180–90), he betrays the sons of \textbf{Rothgore}, his cousins \textbf{Rethrich and Rothmund}, in order to take the throne for himself. In the later Icelandic tradition this has been forgotten, and he is consistently portrayed as a heroic king.

† \textbf{Rothgore} (ON. \emph{Hróarr}, OE. \emph{Hrōþgār}, NWGmc. \emph{*Hrōþigaiʀaʀ})
 A king of the \textbf{Shieldings}, one of the main characters in \emph{Beewolf}. Son of \textbf{Halfdane}, husband of \textbf{Walthew}, by whom father of \textbf{Rethrich and Rothmund}. Uncle of \textbf{Rotholf}, who after his death betrays and slays his sons.

† \textbf{Weden} (rhymes with \emph{leaden}; ON. \emph{Óðinn}, OE. \emph{Wōden}, \emph{Wēden}, OHG. \emph{Wuotan}, NWGmc. \emph{*Wōdanaʀ})
 Chief of the \textbf{Ease}, his name is clearly related to \textbf{wode}, referring to his role as the patron of \textbf{scolds} and \textbf{bearserks}. For the meaning of his other names see \textbf{Fimblethyle}, \textbf{Harn} TO-DO. Husband of \textbf{Frie}, and by her father of \textbf{Bolder}. Also father of \textbf{Thunder} by \textbf{Earth}. Brother of \textbf{Hean} and \textbf{Lother}.

† \textbf{Yivick} (ON. \emph{Gjúki}, OE. \emph{Gifica}, OHG. \emph{Gibicho}, MHG. \emph{Gibeche})
 King of the \textbf{Burgends} (historically from late 300s–407) of the Nifling dynasty, founder of the \textbf{Yifking} aught†. Father of \textbf{Guthrun}, \textbf{Guthhere} and \textbf{Hain}.



Nations, tribes and clans (of divine and human beings)

TO-DO: Map of rough tribal areas..

† \textbf{Danes} (ON. \emph{danir}, OE. \emph{Dene})
 A tribe in eastern modern-day Denmark and southern Sweden. They probably originated in Scania in southern Sweden, before moving westwards into the Danish isles and eventually Jutland, driving out the \textbf{Earls} and \textbf{Jutes}.
 Noted members: TO-DO
 Attestations: TO-DO

† \textbf{Ease} (rhymes with \emph{these}; ON. \emph{æsir}, OE. \emph{ēse}, NWGmc. \emph{*ansiwiʀ})
 A group of Gods, though the word can also refer to all the Gods. Singular \textbf{os}. See \textbf{Gods}, \textbf{Tues}, \textbf{Wanes}, \textbf{Powers}.
 Noted members: \textbf{Weden}, \textbf{Thunder}, \textbf{Frie}, \textbf{Hath} and \textbf{Bolder}
 Attestations: TO-DO

† \textbf{Ettins} (ON. \emph{jǫtnar}, OE. \emph{eotenas}, NWGmc. \emph{*etunōʀ})
 The fundamental enemies of the Gods, the agents of chaos and disorder. See \textbf{Rises}, \textbf{Thurses}.
 Noted members: \textbf{Thrym}
 Attestations: TO-DO

† \textbf{Geats} (ON. \emph{gautar}, OE. \emph{Gēatas}, NWGmc. \emph{*gautōʀ})
 A tribe in what is today southern-central Sweden. See also \textbf{Geatland}.
 Noted members: TO-DO
 Attestations: TO-DO

† \textbf{Gods} (ON. \emph{goð}, OE. \emph{godu}, OHG. \emph{gota})
 TO-DO.
 Noted members: TO-DO
 Attestations: TO-DO

† \textbf{Huns} (ON. \emph{húnir}, OE. \emph{Hūne}, OHG. \emph{Hūni}, \emph{Hunni}, NWGmc. \emph{*hūnīʀ})
 TO-DO.
 Noted members: TO-DO
 Attestations: TO-DO

† \textbf{os} (ON. \emph{áss}, OE. \emph{ōs}, NWGmc. \emph{*ansuʀ})
 A member of the Ease, or a god in general. See \textbf{Ease}, \textbf{Gods}.
 Noted members: TO-DO
 Attestations: TO-DO

† \textbf{Saxes} (ON. \emph{saxar}, OE. \emph{Seaxan}, \emph{Seaxe})
 TO-DO.
 Noted members: TO-DO
 Attestations: TO-DO

† \textbf{Shieldings} (ON. \emph{Skjǫldungar}, OE. \emph{Scyldingas})
 The descendants of \textbf{Shield}, the legendary ruling dynasty of the \textbf{Danes}. With \textbf{Harward}'s death after his slaying of \textbf{Rotholf} their rule ended. TO-DO
 Noted members: TO-DO
 Attestations: TO-DO

† \textbf{Shilvings} (ON. \emph{Skilfingar}, OE. \emph{Scilfingas})
 The exact difference between Shilvings and \textbf{Inglings} is unclear. According to the Saw of Geatrich
 Noted members: TO-DO
 Attestations: \emph{Hind} 15, 20

† \textbf{Swedes} (ON. \emph{svíar}, OE. \emph{Swēon})
 TO-DO.
 Noted members: TO-DO
 Attestations: TO-DO

† \textbf{Thurses} (ON. \emph{þursar}, OE. \emph{þyrs}, OS. \emph{thuris}, OHG. \emph{duris}, NWGmc. \emph{*þurisaʀ})
 Possibly a poetic synonym for \textbf{Ettins}. See also \textbf{Rime-Thurse}
 Noted members: TO-DO
 Attestations: Wal 8, Shr 31, 35, 36, Hyme 17, Thr 5, 10, 21, 24, 29, 30, Alw 2, I HHb 40, HHw 27.
 
 † \textbf{Tues} (ON. \emph{tívar})
 A poetic synonym for \textbf{Gods}.
 Noted members: —
 Attestations: TO-DO

† \textbf{Yifkings} (ON. \emph{Gjúkungar})
 The descendants of \textbf{Yivick}, including \textbf{Guthhere}, \textbf{Guthrun} and \textbf{Hain}.
 Noted members: TO-DO
 Attestations: TO-DO



Other proper nouns (objects and places)

† \textbf{Feather-hame} (ON. \emph{fjaðrhamr})
 A \emph{hame} owned by the Ease that lets the wearer fly like a bird, more specifically a falcon.

† \textbf{Mielden} (ON. \emph{Mjǫllnir}, OE. \emph{*Meolden}, NWGmc. \emph{*Meldunjaʀ})
 A powerful hammer owned by Thunder, it serves as the Ease’s most powerful weapon against the ettins (cf. Thr. 17: “Shortly the ettins will Osyard inhabit, unless thou thy hammer for thyself dost fetch!”).



Bibliography

Streitberg, Wilhelm. 1910. \emph{Die gotische Bibel. Zweiter Teil: Gotisch-griechisch-deutsches Wörterbuch.} Heidelberg: Winter.

SkP = Skaldic Poetry of the Scandinavian Middle Ages. Turnhout: Brepols.

SkP III = \emph{Poetry from Treatises on Poetics}. Ed. Kari Ellen Gade in collaboration with Edith Marold. 2017.
