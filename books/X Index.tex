Poems

WSp. — Spae of the Wale (\emph{Vǫluspǫ́})
High. — Speeches of the High One (\emph{Hávamǫ́l})
Web. — Speeches of Webthrithen (\emph{Vafþrúðnismǫ́l})
Reed. — From the Sons of king Reeding (\emph{Frá sonum Hrauðungs konungs})
Grim. — Speeches of Grimen (\emph{Grímnismǫ́l})
Shr. — Speeches of Shearen (\emph{Skírnismǫ́l})
Hoar. — Leed of Hoarbeard (\emph{Hárbarðsljóð})
Hyme. — Lay of Hyme (\emph{Hymiskviða})

Thr. — Lay of Thrime (\emph{Þrymskviða})

Way. — Lay of Wayland
Alw. — Speeches of Allwise (\emph{Alvíssmǫ́l})
I HHb. — First Lay of Hallow Hundingsbane
HHw. — Lay of Hallow Herwardsson
II HHb. — Second Lay of Hallow Hundingsbane

SiL. — Fragment of the Lay of Siward (\emph{Brot af Sigurðarkviða})
SDe. — From the Death of Siward. (\emph{Frá dauða Sigurðar})
I Gr. — First Lay of Guthrun (\emph{Guðrúnarkviða I})
ShS. — Short Lay of Siward (\emph{Sigurðarkviða in skamma})
BHel. — Hellride of Byrnhild (\emph{Helreið Brynhildar})
Nifl. — The Killing of the Niflings (\emph{Dráp Niflunga})
II Gr. — Second Lay of Guthrun (\emph{Guðrúnarkviða II})
III Gr. — Third Lay of Guthrun (\emph{Guðrúnarkviða III})
BnOr. — From Burgny and Ordrun (\emph{Frá Borgnýju ok Oddrúnu})
OdW. — Weeping of Ordrun (\emph{Oddrúnargrátr})
EtD. — Death of Ettle (\emph{Dauði Atla})
EtL. — Lay of Ettle (\emph{Atlakviða})
EtS. — Speeches of Ettle (\emph{Atlamǫ́l})
FGr. — From Guthrun (\emph{Frá Guðrúnu})
GrB. — Boldness of Guthrun (\emph{Guðrúnarhvǫt})
Ham. — Speeches of Hamthew (\emph{Hamðismǫ́l})
Hind. — The Leed of Hindle (\emph{Hyndluljóð})



Cultural and religious terms

† \textbf{aught} (ON. \emph{ætt})
 The Nordic (paternal) clan or family line.
 
† \textbf{harrow} (ON. \emph{hǫrgr}, OE. \emph{hearg})
 A cairn constructed for ritual purposes. \emph{Hind} 10 describes one: “A \textbf{harrow} he made for me, loaded with stones; now that stone-pile is become into glass. He reddened [it] in fresh blood of oxen; Oughthere ever trusted on the osennies.”
 
† \textbf{leed} (ON. \emph{ljóð}, OE. \emph{lēod})
 A song or chant with magical qualities.
 
† \textbf{wode} (ON. \emph{óðr}, OE. \emph{wōd})
 \textbf{Hean}'s gift to men, although the name would suggest it be from \textbf{Weden}. This word has several related meanings; ‘poetic inspiration’, ‘madness’, ‘rage’, although it in the later Skaldic corpus simply serves as a synonym for ‘poetry’.



Individuals

† \textbf{Earp and Oatle} (ON. \emph{Erpr ok Eitill}})
 The sons of \textbf{Ettle} and \textbf{Guthrun}.

† \textbf{Ettle} (\emph{Attila}, ON. \emph{Atli}, OE. \emph{Ætla}, MHG. \emph{Etzel})
 The ruler of the \textbf{Huns} (historically from 434—453). Husband of \textbf{Guthrun}, and with her father of \textbf{Earp and Oatle}. and murderer of
 I HHb 54, SiL 11, I Gr 23, ShS 28, 29, 33, 37, 54, 56, 57, II Gr 26, 38, 45, III Gr 1, 9, BnOr 0, OdW A, 2, 22, 23, 25, 26, 30, 31, EtD 0, EtL 1, 3, 15, 17, 18, 27, 31, 32, 34, 36, 37, 38, 41, 43, B, EtS 2, 4, 21, 22, 44, 52, 60, 64, 71, 73, 77, 80, 86, 87, 97, 98, 108, 113, 117, FGr 0, GrB 12, Ham 6.

† \textbf{Hain} (ON. \emph{Hǫgni}, OE. \emph{Haguna}, \emph{Hagena}, OHG. \emph{Hagano}, Ger. \emph{Hagen})
 A \textbf{Nifling} and \textbf{Yifking}, son of \textbf{Yivick}, brother of \textbf{Guthhere} and \textbf{Guthrun}. In \emph{EtL} he defeats seven warriors before being captured by \textbf{Ettle}, who has his heart cut out at the request of Guthhere.
 
 A petty king of \textbf{East Geatland}, contemporary with \textbf{Granmer}, the king of \textbf{Southmanland} and Ingeld Illrede, the \textbf{Ingling} king of \textbf{Upland}.
 
† \textbf{Weden} (rhymes with \emph{leaden}; ON. \emph{Óðinn}, OE. \emph{Wōden}, \emph{Wēden}, OHG. \emph{Wuotan})
 Chief of the \textbf{Ese}, his name is clearly related to \textbf{wode}, referring to his role as the patron of \textbf{scolds} and \textbf{bearserks}. Husband of \textbf{Frie}, and by her father of \textbf{Bolder}. He is also father of \textbf{Thunder} by \textbf{Earth}. Brother of \textbf{Hean} and \textbf{Lother}.

† \textbf{Yivick} (ON. \emph{Gjúki}, OE. \emph{Gifica}, OHG. \emph{Gibicho})
 Founder of the \textbf{Yifking} aught†.



Nations, tribes and clans (of divine and human beings)

† \textbf{Danes} (ON. \emph{danir}, OE. \emph{Dene})
 TO-DO.

† \textbf{Ese} (rhymes with \emph{these}; ON. \emph{æsir}, OE. \emph{ēse})
 A tribe of gods including \textbf{Weden}, \textbf{Thunder}, \textbf{Frie}, \textbf{Hath} and \textbf{Bolder}, word can also refer to all the gods. Singular \textbf{os}. See \textbf{gods}, \textbf{Tues}, \textbf{Wanes}, \textbf{Powers}, \textbf{Dwarves}, \textbf{Elves}, \textbf{Ettins}, \textbf{Thurses}.
 
† \textbf{Geats} (ON. \emph{gautar}, OE. \emph{Gēatas})
 TO-DO.

† \textbf{Huns} (ON. \emph{húnir}, OE. \emph{Hūne}, OHG. \emph{Hūni}, \emph{Hunni})
 TO-DO.
 
† \textbf{Os} (ON. \emph{áss}, OE. \emph{ōs})
 A member of the \textbf{Ese}, see there for attestations.
 
† \textbf{Saxes} (ON. \emph{saxar}, OE. \emph{Seaxan}, \emph{Seaxe})
 TO-DO.

† \textbf{Swedes} (ON. \emph{svíar}, OE. \emph{Swēon})
 TO-DO.

† \textbf{Thurses} (ON. \emph{þursar}, OE. \emph{þyrs}, OS. \emph{thuris}, OHG. \emph{duris})
 Possibly a poetic synonym for \textbf{Ettins}. See also \textbf{Rime-Thurse}
 WSp 8, Shr 31, 35, 36, Hyme 17, Thr 5, 10, 21, 24, 29, 30, Alw 2, I HHb 40, HHw 27.

† \textbf{Yifkings} (Yifk-ing; ON. \emph{svíar}, OE. \emph{Swēon})
 The descendants of \textbf{Yivick}, including \textbf{Guthhere}, \textbf{Guthrun} and \textbf{Hain}.
