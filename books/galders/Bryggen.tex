\bookStart{Galders from Bryggen}

Several galders or magical inscriptions are part of the cache of medieval rune-inscribed objects found at Bryggen in the city of Bergen, Norway.  For simplicity’s sake, they are here listed in ascending order of their runological numbers.

\sectionline

\section{B 257}

\begin{flushright}%
\textbf{Dating:} c. 1335

\textbf{Meter:} \Galdralag
\end{flushright}%

A stick inscribed on four planed sides.  Part of the stick is broken off, which renders the text incomplete.  The inscription is clearly a “love-charm” (that is, a piece of sexually coercive magic), addressed—as shown by the feminine dative \emph{sjalfri þér} ‘thy self’ on side D—to a woman.  The language closely resembles that of \Skirnismal\ 36, in which Shirner, Free’s servant, threatens to carve a runic inscription which will curse the ettin-woman Gird with \emph{ęrgi} ‘queerness, degeneracy’, \emph{ǿði} ‘madness’, and \emph{ó·þoli} ‘restlessness, impatience’ unless she sleep with his master.  It seems that we are here dealing with just such a surviving runic curse, and that \Skirnismal\ 36 is reflecting an  authentic form of Norse “love magic” (for it is unlikely that the present inscription should derive directly from that poem) by which a woman is cursed with sexual restlessness until she succumb to the will of the male curser.

A more distant parallel may be seen in the curse-formula found on the two C7th runic inscriptions from Stentoften and Björketorp (see TODO), wherein the destroyer of the respective monuments is cursed to become \emph{herma-lausaʀ argjú} ‘restless (a different root from \emph{ó·þoli} above!) with queerness’, i.e. ‘incessantly randy’.

Side D ends with a string of fake-Latin gibberish, a clear sign of Christian syncretic influence on the Old Norse-Germanic magical tradition.

\sectionline

\bvg\bva[A]Ríst ek \alst{b}ót-rúnar \hld\ ríst ek \alst{b}jarg-rúnar &
\ind \alst{ei}n-falt við \alst{ǫ}lfum &
\ind \alst{t}ví-falt við \alst{t}rollum &
\ind \alst{þ}rí-falt við \alst{þ}u\emph{rsum}\eva

\bvb I carve cure-runes, I carve rescue-runes: \\
onefold against elves, \\
twofold against trolls, \\
threefold against thurses.\evb\evg


\bvg\bva[B]Við inni \alst{sk}ǿðu \hld\ \alst{sk}ag-val-kyrju &
svá’t \alst{ei} megi \hld\ þó-at \alst{ę́} vili &
\alst{l}ę́-vís kona \hld\ \alst{l}ífi þínu g\emph{randa}.\eva

\bvb Against the scatheful shag-walkirrie, \\
so that she may not—though she always wants to— \\
that guile-wise woman—harm thy life.\evb\evg


\bvg\bva[C]Ek \alst{s}endir þér \hld\ ek \alst{s}é á þér &
\alst{y}lgjar \alst{e}rgi \hld\ ok \alst{ó}·þola; &
á þér hríni \alst{ó}·þoli \hld\ ok \alst{jǫ}tuns móð\emph{r}; &
\alst{s}it-tu aldri, \hld\ \alst{s}op-tu aldri.\eva

\bvb I send to thee, I see on thee \\
a she-wolf’s queerness and restlessness; \\
may restlessness stick on thee, and an ettin’s wrath! \\
Never sit, never sleep!\evb\evg


\bvg\bva[D]Ant mér sem sjalfri þér. &
\edtrans{\textbf{†Beirist rubus rabus et arantabus laus abus rosa gava†}}{...}{\Bfootnote{Latin-like gibberish.}}\eva

\bvb Love me like thy self. \\
...\evb\evg

\sectionline

\section{B 380}

\begin{flushright}%
\textbf{Dating:} ?

\textbf{Meter:} \Galdralag
\end{flushright}%

A short little charm explicitly invoking the two most important Heathen Gods, \inx[P]{Thunder} and \inx[P]{Weden}.  The inscription postdates the official conversion of Norway by over a hundred years, and it is an open question whether the two mentioned gods were still seen in a good light or whether they had already been assimilated into the Catholic system of demons and devils.  This question is important since it determines the context of the letter: was it well-wishing, assuming that the receiver was of like mind to the sender, or did he have more sinister intent than the first line lets on?  Judging from the first line, and from the half-Heathen contents of many other inscriptions found at Bryggen (some from as late as the C14th), I see it as crypto-Heathen.

\sectionline

\bvg\bva[]\edtrans{\alst{H}ęill sé þú \hld\ ok í \alst{h}ugum góðum}{Mayst thou be hale and in good spirits}{\Bfootnote{A formulaic greeting.  The very same line is found in \Hymiskvida\ 41; see note there for parallels.}}; &
\ind \alst{Þ}órr þik \alst{þ}iggi, &
\ind \edtrans{\alst{Ó}ðinn þik \alst{ęi}gi}{may Weden own thee}{\Bfootnote{See note to \Voluspa\ 23.}}.\eva

\bvb Mayst thou be hale and in good spirits; \\
may Thunder receive thee, \\
may Weden own thee.\evb\evg

\sectionline
