\section{The Two Merseburg galders}\chapterStart{}

\begin{flushright}%
\textbf{Dating:} C9th–10th

\textbf{Meter:} \Fornyrdislag, \Galdralag%
\end{flushright}

These two galders, preserved in a manuscript (TODO) are some of the only surviving examples of genuine Heathen galders from the continent.  Both share a common two-part structure, each beginning with an \emph{historiola}—a “historical” account describing the successful effects of the galder in the mythic past—followed by an \emph{imperative} commanding that the willed magic effect take place in the present.

The first galder begins with the historiola describing a group of supernatural women in the midst of a battle, affecting its outcome by fastening or loosening fetters.  The imperative then commands that some fetters in the present be destroyed, so that captive(s) may escape.

The second galder begins with the historiola describing a group of Gods riding through the woods.  Among them is \inx[P]{Balder}, whose young foal sprains its foot.  Three Gods—the otherwise unknown goddess \inx[P]{Sithguth}, the goddess \inx[P]{Sun}, the god \inx[P]{Weden}—in turn chant a healing galder over it.  The imperative—apparently the galder sung by Weden—then commands that a present sprain be healed.

\sectionline

\bvg\bva Ęiris \alst{s}ázun idisi \hld\ \alst{s}ázun hera duo der; &
suma \alst{h}apt \alst{h}ęptidun \hld\ suma \alst{h}ęri lęzidun &
suma \alst{k}lubodun \hld\ umbi \edtrans{\alst{k}uonjo-widi}{chains}{\Bfootnote{A rare word apparently cognate with Gothic \emph{kuna-wida} ‘\emph{Fessel}; \textgreek{ἅλῠσῐς}’ \parencite[76]{Streitberg}, although the first element is not formally identical.}} &
\alst{i}n-sprink hapt-bandun \hld\ \alst{i}n-var vígandun &
\edtext{.H.}{\Bfootnote{The meaning of this letter, which is very clear and written in the same hand as the galders, is uncertain.  To me the most convincing suggestion is that it be read as \emph{.N.}, short for Latin \emph{nomen} ‘name’, presumably the name of the person whom the singer wishes to free from the fetters.}}\eva

\bvb Of yore sat dises, sat here, then there: \\
some fastened fetters, some hindered armies, \\
some cut chains asunder.— \\
Destroy the fetter-bonds, lead the way from the foes! \\
.H.\evb\evg


\bvg\bva \alst{Ph}ol ęnde Wuodan \hld\ \alst{v}uorun zi holza &
dú wart demo Balderes \alst{v}olon \hld\ sín \alst{v}uoz bi·ręnkit &
þú \edtrans{bi·guol en}{begaled him}{\Bfootnote{Sang a \inx[C]{galder} over the horse, the third past singular of \emph{bi·galan} ‘begale’, the transitive of \emph{galan} ‘gale, sing a galder’.  Cf. \Oddrunargratr\ TODO, where a midwife “gales” “bitter galders” over a birthing mother.}} \alst{S}inhtgunt \hld\ \alst{S}unna era swister &
þú bi·guol en \alst{F}rija \hld\ \alst{V}olla era swister &
þú bi·guol en \alst{W}uodan \hld\ só hé \alst{w}ola konda: &
„Só-se \alst{b}ên-ręnkí \hld\ só-se \alst{b}luot-ręnkí \hld\ só-se lidi-ręnkí &
\ind \alst{b}ên zi \alst{b}êna &
\ind \alst{b}luot zi \alst{b}luoda &
\alst{l}id zi ge·\alst{l}iden \hld\ só-se ge·\alst{l}ímida sín!“\eva

\bvb Phol and Weden journeyed in the woods; \\
then was the foot of Balder’s foal sprained. \\
Then \inx[P]{Sithguth} \inx[C]{begale}[begaled] him—\inx[P]{Sun} her sister; \\
then \inx[P]{Frie} begaled him—\inx[P]{Full} her sister; \\
then Weden begaled him, as well he knew: \\
“Like bone-sprain, like blood-sprain, like joint-sprain! \\
\ind Bone to bone, \\
\ind blood to blood, \\
joint to joints, like they were glued together!”\evb\evg

\sectionline
