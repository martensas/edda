\bookStart{The Canterbury Galder}

\begin{flushright}%
Dating: c. 1075

Meter: \Fornyrdislag%
\end{flushright}

This Old Norse galder is found in the Anglo-Saxon manuscript Cotton Caligula A XV.  It runs across the bottom margin of the two facing pages 123v and 124r and is written in very clear runes of the Wiking age long-stave type.  One rune, viz. \textbf{g} in \textbf{vigi} \emph{vegi} ‘smite’ is stung.  The inscription has no word separators.

The galder is of the same type as the two from Sigtuna (U Fv1933;134, U NOR1998;25) and clearly intended for healing; it ends with \emph{viðr áðra-vari} ‘against pus of veins’, and this is probably a declaration of purpose.

\sectionline

\bvg\bva[] Gyrils sár-þvara! &
\alst{F}ar-ðu nú, \hld\ \alst{f}undinn es-tu! &
\alst{Þ}órr vegi \alst{þ}ik \hld\ \alst{þ}ursa dróttinn! &
Jórils sár-þvara. &
Viðr áðra-vari.\eva

\bvb O Gyrel’s wound-TODO! \\
Go thou now, thou art found! \\
May Thunder smite thee, O lord of Thurses! \\
O Erel’s wound-causer. \\
Against pus of veins.\evb\evg
