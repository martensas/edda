\section{Against Swarm (\emph{Wið ymbe})}\chapterStart{}

\begin{flushright}%
\textbf{Dating:} ?

\textbf{Meter:} \Fornyrdislag%para
\end{flushright}%

TODO. That bees are called “victory-wives” is interesting.

\sectionline

\bpg\bpa Wið ymbe nim eorþan, ofer·weorp mid þínre swíþran handa under þínum swíþran fét, and cwet:\epa

\bpb Against a swarm take earth, throw it with thy right hand under thy right foot, and say:\epb\epg


\bvg\bva \alst{F}ó ic under \alst{f}ót, \hld\ \alst{f}unde ic hit. &
Hwæt \alst{eo}rðe mæg \hld\ wið \alst{ea}lra wihta ge·hwilce &
and wið \alst{a}ndan \hld\ and wið \alst{æ}minde &
and wið \edtrans{þá \alst{m}icelan \hld\ \alst{m}annes tungan}{that mighty tongue of man}{\Bfootnote{The tongue is surely here standing in for “speech”, specifically galder; i.e., if the swarming of the bees were caused by an enemy’s cursing, the earth will disarm it.}}.\eva

\bvb I catch under foot, I may have found \emph{it}. \\
How, earth works against everywhich wight \\
and against mischief and against neglect \\
and against that mighty tongue of man.\evb\evg


\bpg\bpa And wiððon \edtrans{for·weorp ofer greót}{throw the grit over}{\Bfootnote{i.e. “throw the earth over the swarm”.}}, þonne hí swirman, and cweð:\epa

\bpb And with that throw the grit over, when they swarm, and say:\epb\epg


\bvg\bva \alst{S}itte gé, \alst{s}ige-wíf, \hld\ \alst{s}ígað to eorþan! &
Næfre gé \alst{w}ilde \hld\ to \alst{w}uda fleogan. &
Beo gé swá ge·\alst{m}indige \hld\ \alst{m}ínes gódes, &
swá bið \alst{m}anna ge·hwilc \hld\ \alst{m}etes and éþeles.\eva

\bvb Sit ye, victory-wives; sink to the earth! \\
Never ye would fly to the woods. \\
Be ye so mindful of \emph{my} good, \\
like is every man of his measure and homestead.\evb\evg

\sectionline
