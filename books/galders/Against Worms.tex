\section{Against wyrms (\emph{Contra vermes})}\chapterStart{}

\begin{flushright}%
\textbf{Dating:} ?

\textbf{Meter:} \Fornyrdislag%para
\end{flushright}%

An Old Saxon manuscript charm against wyrms located in the bone-marrow, probably thought to cause aching.  The galder calls upon a chief worm, Nesse, and its nine offspring, to depart from the patient.  It lays out a path for the worms, who are to leave the sufferer’s body and instead go into an arrow or sharp point (\emph{strála}), probably a ritual implement used to pierce the affect area.

The structure “Go from X to Y, from Y to Z” may be very old, as it is also found in Romani charms collected by \textcite[27,28,95]{Leland1891}  The charm on p. 95 is also against wyrms.  Like in our galder the wyrms (\emph{kirmora}, from Sanskrit \emph{kŕ̥mi}, which is probably related to Germanic \emph{*wurmiz}, although the difference in the initial consonant is unusual—perhaps a taboo formation?) are to leave the body and instead go into the ritual implement, in the Gypsy charm an ointment.  I take me the freedom to reproduce this charm in full, with Leland’s introduction and translation:

“Before sunrise wolf’s milk (Wolfsmilch, rukeskro tçud) is collected, mixed with salt, garlic, and water, put into a pot, and boiled down to a brew. With a part of this the afflicted spot is rubbed, the rest is thrown into a brook, with the words:—

\begin{verse}
\emph{Kirmora jánen ándre tçud} \\
\emph{Andrál tçud, andré sir} \\
\emph{Andrál sir, andré páñi,} \\
\emph{Panensá kiyá dádeske,} \\
\emph{Kiyá Niváseske} \\
\emph{Pçándel tumen shelehá} \\
\emph{Eñávárdesh teñá!}
\end{verse}

\begin{verse}
‘Worms go in the milk, \\
From the milk into the garlic, \\
From the garlic into the water, \\
With the water to (your) father, \\
To the Nivasi, \\
He shall bind you with a rope, \\
Ninety-nine (yards long).’”
\end{verse}

\sectionline

\bvg\bva[]Gang út, \edtrans{\alst{N}esso}{Nesse}{\Bfootnote{The \emph{naming} of the daemon or being which is to be excised is common in ancient magic, including several other galders edited here.  The idea is that knowledge of the name of the entity gives the healer power over it.}}, \hld\ mid \alst{n}igun \alst{n}essi-klínon, &
ut fana þemo marge an þat \alst{b}ên, \hld\ fan þemo \alst{b}êne an þat flęsg, &
ut fan þemo flęsgke an þia \alst{h}úd, \hld\ ut fan þera \alst{h}úd an þesa strála. &
Drohtin, werþe só.\eva

\bvb Go out, O Nesse, with the nine small Nesses! \\
Out from the marrow into the bone, from the bone into the flesh, \\
out from the flesh into the skin, out from the skin into this arrow. \\
Lord, may it be so.\evb\evg

\sectionline
