\bookStart{The Wessobrunner Hymn}

\begin{flushright}%
Dating: late C8th

Meter: \Fornyrdislag%para
\end{flushright}%

This text can be split into two parts, the “poem” and the “prayer”. Following my principle of including sources rather than excluding (TODO: see Introduction), I here present both.

The first part is a short alliterative poem describing the earliest beginning of the world. The poet describes “the greatest of wonders”, namely that the universe began as a void, where neither earth nor heaven existed. In this void was, however, the almighty God, along with his many spirits (presumably the Heavenly Host or the Angels). While the cosmogony expressed is clearly Jewish-Christian rather than Germanic, the poem does contain two word-pairs also found in Norse Heathen stanzas about the creation of the world (see Notes to ll. 2, 3.), which may point toward a repurposing of older Heathen motifs and expressions in the new, Christian context.

The second part is a thoroughly Christian prayer. The author first thanks God for creating the earth and heaven, this is presumably why the poem was included, and for giving good things to mankind. He then asks for faith, strength and wisdom to “withstand devils”, “reproach degeneracy” and “work [God’s] will”.

\sectionline

\bvg
\bva[0]Dat ga·\alst{f}regin ih mit \alst{f}irahim · \alst{f}iri·wizzó męista, &
dat \edtext{\edtext{\alst{e}rdo}{\Afootnote{\emph{ero} ms.}} ni was · noh \alst{ú}f-himil}{\lemma{erdo \dots\ úf-himil ‘earth \dots\ up-heaven’}\Bfootnote{A formulaic merism attested across the Germanic world, expressing the totality of the universe. Cf. especially \Vafthrudnismal\ 21, where the god Weden asks the ettin Webthrithner about the origin of “earth and up-heaven”, and \Voluspa\ 3/3, where it is said, about the time before the World existed, that “earth and up-heaven” never existed.}} &
noh \edtext{\alst{p}aum · noh \alst{p}erek}{\lemma{paum \dots\ perek ‘forest \dots\ mountain’}\Bfootnote{The same word-pair is found in \Grimnismal\ 40, describing the creation of the world from Yimer’s body by the Gods.}} ni was &
ni [...] nohh-ęinig · noh sunna ni skęin &
noh \alst{m}áno ni liuhta · noh der \alst{m}árjo sêo. &
Dó dar ni·\alst{w}iht ni \alst{w}as · ęntjó ni \alst{w}ęntjó, &
ęnti dó was der \alst{ęi}no · \alst{a}l-mahtiko kot, &
\alst{m}anno \alst{m}iltisto, · ęnti dar wárun auh \alst{m}anaké mit inan &
\alst{k}ót-líhhé \alst{g}ęistá, · ęnti \alst{k}ot hęilak.\eva

\bvb I learned among men that greatest of wonders, \\
that earth was not, nor up-heaven, \\
nor a forest, nor a mountain was not, \\
nor any [...]; nor did the sun shine, \\
nor the moon give off light, nor the glittering sea. \\
Then nothing was there, neither of limit nor infinity (TODO: Translation),— \\
and then was the One Almighty God: \\
the mildest of men \ken*{= Christ}, and there were also many with Him: \\
good ghosts, and Holy God.\evb
\evg


\bpg
\bpa Kot al-mahtiko, dú himil ęnti erda ga·worahtós, ęnti dú mannun só manak kót for·gápi,
for·gip mir in dína ga·náda rehta ga·laupa, ęnti kótan willjon; wís-tóm ęnti spáhida, ęnti kraft tiuflun za widar·stantanne, ęnti ark za pi·wísanne, ęnti dínan willjon za ga·wurkhanne.\epa

\bpb O God almighty, thou didst work heaven and earth, and thou didst give men so many good things.
Give me in thy mercy the right belief and good will, wisdom and prophecy, and power to withstand devils and to reproach degeneracy and to work thy will.\epb
\epg
