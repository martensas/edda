Þórr dró Miðgarðsorm. TO-DO: Format as header.

Thunder pulled the Middenyardsworm.\footnotemark[1]
\footnotetext[1]{Clearly not part of the original poem, this is the only title (written with red ink) the poem has in \emph{R}. \emph{748} has the proper title \emph{Hymiskviða} instead.}

Ár valtívar \hld vęiðar nǫ́mu
ok sumblsamir \hld áðr saðir yrði,
hristu tęina \hld ok á hlaut sǫ́u,
fundu þęir at Ægis \hld ørkost hvera.

Of yore the wal-Tues had hunted game\footnotemark[1] and — banqueting before they might eat\footnotemark[1] — they shook the twigs and looked at the leat†\footnotemark[3]; they found at Age’s a great choice of cauldrons.\footnotemark[4]
\footnotetext[1]{Lit. ‘took game’}
\footnotemark[2]{Lit. ‘banquet-some before they might become sated’}
\footnotemark[3]{They sprinkled the leat (sacrificial blood) from the animals and interpreted the pattern; a form of augury. See index entry leat† for more information.}
\footnotemark[4]{A difficult verse. Meaning seems to be thus: The gods had caught many animals, and were ready to eat. They performed a divination ceremony (see note 3) and saw that it was most auspicious to hold the banquet at Age’s estate.}

Sat bergbúi \hld barntęitr fyr,
mjǫk glíkr męgi \hld Miskorblinda,
lęit í augu \hld Yggs barn í þrá:
“þú skalt ǫ́sum \hld opt sumbl gęra!”

Sat the mountain-dweller there, gay like a child, much like the lad of Misherblind; the child of Ug looked into [his] eyes in defiance: “Thou shalt for the Ease oft’ hold banquets!”

Ǫnn fekk jǫtni \hld orðbæginn halr,
hugði at hefndum \hld hann næst við goð,
bað hann Sifjar ver \hld sér fǿra hver,
“þann’s ek ǫllum ǫl \hld yðr of hęita”.

Great toil for the ettin did the word-peevish man cause; thought he of revenge, soon, against the god: he asked the husband of Sif to bring him a cauldron, “that one with which I for you all, ale might brew”.

Né þat mǫ́ttu \hld mærir tívar
ok ginnręgin \hld of geta hvęrgi,
unz af tryggðum \hld Týr Hlórriða
ástráð mikit \hld ęinum sagði:

But that might the renowned Tues and the Gin-Rains†, get ahold of nowhere, — until out of loyalty, a great word of loving advice, Tue to Loride alone did say:
