\bookStart{The Lay of Hallow Harwardson}[Hęlgakviða Hjǫrvarðssonar]

\begin{flushright}%
Dating \parencite{Sapp2022}: early C11th (0.385)–late C11th (0.550)

Meter: \Fornyrdislag%
\end{flushright}%

\sectionline

\section{Regarding Harward and Sighlind (\emph{Frá Hjǫrvarði ok Sigrlinn})}

\bpg\bpa Hjǫrvarðr hét konungr. Hann átti fjórar konur. Ein hét Alfhildr; sonr þeira hét Heðinn. Ǫnnur hét Sę́reiþr; þeira sonr hét Humlungr. In þriðja hét Sinrjóð; þeira sonr hét Hymlingr. Hjǫrvarðr konungr hafði þess heit strengt at eiga þá konu er hann vissi vę́nsta. Hann spurði at Sváfnir konungr átti dóttur allra\footnote{‘vęnallra’ \emph{corr.} \Regius} fegrsta; sú hét Sigrlinn. Iðmundr hét jarl hans; Atli var hans sonr er fór at biðja Sigrlinnar til handa konungi. Hann dvalðisk vetrlangt með Sváfni konungi. Fránmarr hét þar jarl, fóstri Sigrlinnar; dóttir hans hét Álǫf. Jarlinn réð, at meyjar var synjat, ok fór jarlinn heim. Atli jarls sonr stóð einn dag við lund nǫkkurn, en fugl sat í limunum uppi yfir hánum ok hafði heyrt til, at hans menn kǫlluðu vę́nstar konur þę́r, er Hjǫrvarðr konungr átti. Fuglinn kvakaði, en Atli hlýddi, hvat hann sagði. Hann kvað:\epa

\bpb TODO. He quoth:\epb
\epg

\bvg
\bva „Sáttu Sigrlinn, \hld\ Sváfnis dóttur, &
męyna fęgrstu \hld\ ï munarhęimi? &
Þó hagligar \hld\ Hjǫrvarðs konur &
gumnum þykkja \hld\ at Glasislundi.“\eva

\bvb 1\evb
\evg


\bvg
\bva „Mundu við Atla \hld\ Iðmundar son &
fugl fróðhugaðr \hld\ flęira mę́la?“ &
„Mun’k ef mik buðlungr \hld\ blóta vildi &
ok kýs’k þat’s ek vil \hld\ ór konungs garði.“\eva

\bvb 2\evb
\evg


\bvg
\bva 3\eva

\bvb 3\evb
\evg


\bvg
\bva 4\eva

\bvb 4\evb
\evg


\bvg
\bva 5\eva

\bvb 5\evb
\evg


\bvg
\bva 6\eva

\bvb 6\evb
\evg


\bvg
\bva 7\eva

\bvb 7\evb
\evg


\bvg
\bva Sverð vęit’k liggja \hld\ ï Sigarsholmi, &
fjórum fę́ra \hld\ enn fimm tǫgu; &
ęitt es þęira \hld\ ǫllum bętra &
vígnesta bǫl \hld\ ok varið golli.\eva

\bvb Swords I know lying, in Sigharsholm, four less than fifty. One of them is better than all—the \inx[C]{bale} of war-needles\footnoteB{The kenning \emph{vígnest} also appears in} \ken{spears?}—and inlaid with gold.\evb
\evg


\bvg
\bva Hringr ’s ï hjalti, \hld\ hugr ’s ï miðju, &
ógn ’s ï oddi, \hld\ þęim’s ęiga getr; &
liggr með ęggju \hld\ ormr dręyrfáiðr &
en ȧ valbǫstu \hld\ verpr naðr hala.\eva

\bvb A ring is in the hilt; courage is in the middle; fear is in the point, for the one who gets to own it; along the blade lies a serpent painted in blood, but on the walbast\footnoteB{An unclear part of the sword-hilt; see \Sigrdrifumal\ 7.} an adder chases its tail.\evb
\evg
