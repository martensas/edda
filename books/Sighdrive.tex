\bookStart{The Speeches of Syedrive}[Sigrdrífumǫ́l]

\begin{flushright}%
Dating \parencite{Sapp2022}: C10th (0.961)

Meter: \Ljodahattr%
\end{flushright}

% Introduction

The poem and prose under this header follows the order of \Regius. A large count of verses are also cited in \VolsungaMS\ (\VolsungaSaga\ ch. 21).

In \VolsungaSaga\ the present text up to P2 is first paraphrased:

\begin{quote}
  \emph{Brynhildr segir, at tveir konungar bǫrðust. Hét annarr Hjalmgunnarr; hann var gamall ok hinn mesti hermaðr, ok hafði Óðinn honum sigr heitit, en annarr Agnarr eða Auða bróðir. „Ek fellda Hjalmgunnarr í orrostu, en Óðinn stakk mik svefn-þorni í hefnd þess ok kvað mik aldri síðan skyldu sigr hafa ok kvað mik giptast skulu. En ek strengða þess heit þar í mót at giptast engum þeim, er hrę́ðast kynni.“ Sigurðr mę́lti: „Kenn oss ráð til stórra hluta.“ Hun svarar: „Þér munuð betr kunna, en með þǫkkum vil ek kenna yðr, ef þat er nǫkkut, er vér kunnum, þat er yðr mę́tti líka, í rúnum eða ǫðrum hlutum, er liggja til hvers hlutar, ok drekkum bę́ði saman, ok gefi goðin okkr góðan dag, at þér verði nýt ok fregð at mínum vitrleik, ok þú munir eptir þat, er vit réðum.“ Brynhildr fylldi eitt ker ok fę́rði Sigurði ok mę́lti:}

  ‘Byrnhild says that two kings fought. One was called Helmguther; he was old and the greatest warrior, and Weden had promised him victory,
  but the other was called Eyner or Eade’s brother. “I felled Helmguther in battle, but Weden stung me with a sleeping-thorn as revenge for that, and declared that I should never thenceforth have victory, and said that I must marry, but I made a vow in response, to marry no man who could be frightened.” Siward spoke: “Teach us counsels regarding great things.” She answers: “Ye will know better, but with thanks I will teach you, if there is anything which we know that may please you, of runes or other things of importance; and let us both drink together, and may the gods give us two a good day, that thou have use and joy from my wisdom and that thou afterwards recall that which we two speak of.” Byrnhild filled a vessel and brought it to Siward and spoke:’
\end{quote}

After this the present sts. 4–12 and 14–19 are cited uninterrupted, and a paraphrase is given of sts. 20 ff. (TODO: edit these!). While the order of 12–19 (excepting the omission of 13) in \VolsungaMS\ is identical to that of \Regius, and sts. 4–5 likewise come first, the order of the middle sts. 6–11 is very different. The following table shows the relationship between the two ms. for the relevant stanzas:

\begin{longtabu} to \textwidth {|c c c c|}
	\hline
	\multicolumn{2}{|c}{\emph{pres. ed.}} & \Regius & \VolsungaMS \\ [0.5ex]
	\hline\hline\endhead
	\hline\endfoot
	4 & Bjór fǿri’k þér & 4 & 6 \\
	5 & Sig-rúnar skalt rísta & 5 & 7 \\
  6 & Ǫl-rúnar skalt kunna & 6 & 10 \\
  7 & Full skal signa & 6* & 11 \\
  8 & Bjarg-rúnar skalt kunna & 7 & 12 \\
  9 & Brim-rúnar skalt rísta & 8 & 8 \\
  10 & Lim-rúnar skalt kunna & 9 & 13 \\
  11 & Mál-rúnar skalt kunna & 10 & 9 \\
  12 & Hug-rúnar skalt kunna & 11a & 14 \\
  13 & Á bjargi stóð & 11b–12 & − \\
  14 & Á skildi kvað ristnar & 13–14a & 15–17 \\
  15 & Allar vǫ́ru af skafnar & 14b–15 & 18 \\
  16 & Þat eru bókrúnar & 16 & 19 \\
  17 & Nú skalt kjósa & 17 & 20 \\
  18 & Mun’k-a ek flǿja & 18 & 21 \\ [1ex]
	\hline
\end{longtabu}

\sectionline

\bvg
\bva „Lęngi ek \alst{s}vaf, \hld\ lęngi ek \alst{s}ofnuð vas, &
\ind \alst{l}ǫng eru \alst{l}ýða \alst{l}ę́; &
\alst{Ó}ðinn því vęldr \hld\ es \alst{ęi}gi mátta’k &
\ind \alst{b}regða \alst{b}lund-stǫfum.“\eva

\bvb {[Syedrive quoth:]} “Long I slept, long was I asleep, long are the deceits of men. Weden wields it that I could not break the sleeping-staves.”\evb
\evg


\bpg\bpa Sigurðr sęttisk niðr ok spyrr hana nafns. Hón tók þá horn fullt mjaðar ok gaf hǫ́num minnis-vęig.\epa

\bpb Siward set himself down, asking for her name. Then she took a horn full of mead, and gave him a mind-draught:\epb\epg


\bvg
\bva Hęill \alst{D}agr, \hld\ hęilir \alst{D}ags synir, &
\ind hęil \alst{N}ǫ́tt ok \alst{n}ipt! &
\alst{Ó}-ręiðum \alst{au}gum \hld\ lítið \alst{o}kkr þinig &
\ind ok gefið \alst{s}itjǫndum \alst{s}igr!\eva

\bvb “Hail \inx[P]{Day}! Hail the sons of Day!\footnoteB{TODO. Who?} Hail Night and [her] kinswoman \ken*{= Earth}!\footnoteB{According to \Gylfaginning\ 10 Earth is the daughter of Night and \inx[P]{Aner}.} With unwrathful eyes look ye upon us two, and give the sitting ones \ken*{= us} victory.\evb
\evg


\bvg
\bva \edtrans{Hęilir \alst{ę́}sir, \hld\ hęilar \alst{ǫ́}synjur,}{Hail the Ease! Hail the Ossens!}{\Bfootnote{Probably formulaic, subverted by Lock in \Lokasenna\ 11 (see note there for possible ritual use).}} &
\ind hęil sjá in \alst{f}jǫl-nýta \alst{f}old! &
\alst{M}ál ok \alst{m}an-vit \hld\ gefið okkr \alst{m}ę́rum tvęim &
\ind ok \edtrans{\alst{l}ę́knis-hęndr}{healing-hands}{\Bfootnote{Hands with the power to heal (perhaps supernaturally). The singular form \emph{lę́knis-hǫnd} occurs in the semi-Christianized prayer on a c. 1300 stick from Ribe, Denmark (signum DR EM85;493).}} meðan \alst{l}ifum!\eva

\bvb Hail the \inx[G]{Ease}! Hail the \inx[G]{Ossens}! Hail this bountiful fold \ken{earth}! Speech and \inx[C]{manwit} give ye us renowned two, and \inx[C]{healing-hands} while we live.”\evb
\evg


\bpg\bpa Hon nefndisk Sigrdrífa ok var valkyrja. Hon sagði, at tveir konvngar bǫrðusk. Hét annarr Hjalmgunnarr; hann var þá gamall ok inn mesti hermaðr, ok hafði Óðinn hánum sigri heitit.
En \alst{a}nnarr hét \alst{A}gnarr, \hld\ \alst{Au}ðu bróðir // er \alst{v}ę́tr engi \hld\ \alst{v}ildi þiggja.
Sigrdrífa felldi Hjalm-gunnar í orrostunni. En Óðinn stakk hana svefn-þorni í hefnd þess ok kvað hana aldri skyldu síðan sigr vega í orrostu, ok kvað hana giftask skyldu, „en sagða’k hánum at strengða’k heit þar í mót, at giptask øngom þeim manni er hrę́ðask kynni.“ Hann segir ok biðr hana kenna sér speki ef hon vissi tíðendi ór ǫllum heimum. Sigrdrífa kvað:\epa

\bpb She called herself Syedrive and was a walkirrie. She said, that two kings fought. One was called Helmguther; he was then old and the greatest warrior, and Weden had promised him victory.
But the other was called Eyner, Eade’s brother, who in no way wished to surrender.
Syedrive felled Helmguther in the battle, but Weden stung her with a sleeping-thorn as revenge for that, and declared that she should never thenceforth cause victory in battle, and said that she must marry, “but I said to him that I made a vow in response, to marry no man who could be frightened.” He \ken*{= Siward} speaks and asks her to teach him wisdom, if she knew any tidings out of all the \inx[C]{Home}[Homes]. Syedrive quoth:\epb\epg


\bvg
\bva\mssnote{\Regius~32r/18–20, \VolsungaMS~24v/12–14}„\alst{B}jór fǿri’k þér, \hld\ \edtrans{\alst{b}ryn-þings apaldr}{apple-tree of the byrnie-Thing \ken{battle > warrior}}{\Afootnote{\emph{bryn-þinga valdr} ‘wielder of byrnie-Things \ken{battles > warrior}’ \VolsungaMS}}, &
\alst{m}agni blandinn \hld\ ok \alst{m}ęgin-tíri, &
fullr es \alst{l}jóða \hld\ ok \alst{l}íkn-stafa, &
\alst{g}óðra \alst{g}aldra \hld\ ok \edtrans{\alst{g}aman-rúna}{pleasure-runes}{\Afootnote{\emph{gaman-†rędna†} \VolsungaMS}}.\eva

\bvb Beer I bring thee—apple-tree of the byrnie-\inx[C]{Thing} \ken{battle > warrior}!—mixed with might, and might-glory; it is full of \inx[C]{leed}[leeds] and grace-staves, of good \inx[C]{galder}[galders] and pleasure-\inx[C]{rune}[runes].\evb
\evg


\bvg
\bva\mssnote{\Regius~32r/20–22, \VolsungaMS~24v/14–16}\alst{S}ig-rúnar skalt rísta, \hld\ ef vilt \edtrans{\alst{s}igr hafa}{have victory}{\Afootnote{\emph{snotr vera} ‘be clever’ \VolsungaMS}}, &
\ind ok \edtext{rísta}{\Afootnote{\emph{†rist†} \VolsungaMS}} á \alst{h}jalti \alst{h}jǫrs, &
\edtrans{sumar}{some}{\Afootnote{om. \VolsungaMS}} á \edtext{\alst{v}étt-rimum}{\Afootnote{\emph{vétt-†rvnum†} \VolsungaMS}}, \hld\ \edtrans{sumar}{some}{\Afootnote{\emph{ok} ‘and’ \VolsungaMS}} á \edtext{\alst{v}al-bǫstum}{\Afootnote{\emph{val-†bystum†} \VolsungaMS}}, &
\ind ok nęfna \alst{t}ysvar \alst{T}ý.\eva

\bvb Victory-runes shalt thou know, if thou wilt have victory, and carve on the hilt of the sword; some on the weight-rims;\footnoteB{Unclear. TODO.} some on the wal-basts\footnoteB{Possibly the sword-pommel, the word also occurs in \HelgakvidaHjorvardssonar\ 9. TODO.}, and twice name \inx[P]{Tew}.\evb
\evg


\bvg
\bva\mssnote{\Regius~32r/22–24, \VolsungaMS~25r/1–3}\alst{Ǫ}l-rúnar skalt kunna \hld\ ef vilt \edtrans{at}{that}{\Afootnote{emend. from \emph{†a†} \VolsungaMS; om. \Regius}} \alst{a}nnars kvę́n &
\ind \edtext{véli-t þik í \alst{t}ryggð}{\Afootnote{\emph{véli þik eigi tryggð} \VolsungaMS}} ef \alst{t}rúir; &
á \alst{h}orni skal \edtrans{þę́r}{them}{\Afootnote{\emph{þat} ‘it’ \VolsungaMS}} rísta \hld\ ok á \alst{h}andar baki &
\ind ok męrkja á \alst{n}agli \edtrans{\alst{N}auð}{Need}{\Bfootnote{i.e. the n-rune, ᚾ.}}.\eva

\bvb Ale-runes shalt thou know, if thou wilt that another man’s wife not betray thee in troth if thou trustest [in her]. On the horn shall [one] carve them, and on the back of the hand, and mark Need on the nail.\evb
\evg


\bvg
\bva\mssnote{\Regius~32r/24–25, \VolsungaMS~25r/3–4}\edtrans{\alst{F}ull}{The cup}{\Afootnote{\emph{ǫl} ‘The ale’ \VolsungaMS\ breaks alliteration.}} skal signa \hld\ ok við \alst{f}ári séa &
\ind ok verpa \alst{l}auki í \alst{l}ǫg; &
\edtext{\alst{þ}á þat vęit’k, \hld\ at \alst{þ}ér verðr aldri-gi &
\ind \edtext{męini blandinn}{\Afootnote{emend.; \emph{męin-blandinn} \VolsungaMS}} mjǫðr.}{\lemma{þá \dots\ mjǫðr}\Bfootnote{only in \VolsungaMS; om. \Regius}}\eva

\bvb The cup shalt thou sign\footnoteB{Dedicate to the gods with a certain formula. TODO.}, and gaze against the danger, and throw in the liquid a leek. Then I know that it never will be mixed with harm, thy mead.\evb
\evg


\bvg
\bva\mssnote{\Regius~32r/25–26, \VolsungaMS~25r/5–7}\alst{B}jarg-rúnar skalt \edtrans{kunna}{know}{\Afootnote{\emph{nema} ‘learn’ \VolsungaMS}} \hld\ \edtrans{ef \alst{b}jarga vilt}{if thou wilt rescue}{\Afootnote{\emph{ef þú vilt borgit fá} ‘if thou wilt get rescued’ \VolsungaMS}} &
\ind ok lęysa \alst{k}ind frá \alst{k}onum; &
á \alst{l}ófa þę́r skal rísta \hld\ ok of \alst{l}iðu spęnna &
\ind ok biðja \edtrans{þá}{then}{\Afootnote{om. \VolsungaMS}} \alst{d}ísir \alst{d}uga.\eva

\bvb Rescue-runes shalt thou know, if thou wilt rescue and loosen children from women;\footnoteB{i.e. during difficult childbirth.} on the palm shall [one] carve them, and wrap them around the joints, and then bid the dises to avail.\footnoteB{The dises were minor female deities, and as seen by this stanza they were called upon to avail women during childbirth.}\evb
\evg


\bvg
\bva\mssnote{\Regius~32r/27–29, \VolsungaMS~24v/16–19}\alst{B}rim-rúnar skalt \edtrans{rísta}{carve}{\Afootnote{\emph{gjora} ‘make’ \VolsungaMS}} \hld\ ef vilt \alst{b}orgit hafa &
\ind á \alst{s}undi \alst{s}egl-mǫrum; &
á \alst{st}afni \edtrans{skal rísta}{shall [one] carve}{\Afootnote{\emph{skal þę́r rísta} ‘shall [one] carve them’ \VolsungaMS}} \hld\ ok á \alst{st}jórnar blaði &
\ind ok \edtrans{lęggja \alst{ę}ld í \alst{á}r}{lay fire to the oar}{\Bfootnote{i.e. mark it with fire in some way.}};
\edtrans{es-a}{There is not}{\Afootnote{\emph{falla-t} ‘There fall not’ \VolsungaMS}} svá \alst{b}rattr \alst{b}reki \hld\ né svá \alst{b}láar unnir, &
\ind \edtext{þó kømsk-tu \alst{h}ęill af \alst{h}afi}{\lemma{þó ... hafi ‘that ... sea’}\Bfootnote{lit. ‘yet comest thou whole off the sea.’}}.\eva

\bvb Surf-runes shalt thou carve, if thou wilt rescue sail-steeds \ken{ships} on the sound; on the stem shall [one] carve, and on the rudder’s blade, and lay fire to the oar. There is not so steep a breaker nor so blue-black waves, that thou not come whole off the sea.\evb
\evg


\bvg
\bva\mssnote{\Regius~32r/29–31, \VolsungaMS~25r/7–9}\alst{L}im-rúnar skalt kunna \hld\ ef vilt \alst{l}ę́knir vesa &
\ind ok kunna \alst{s}ár at \alst{s}éa; &
á \alst{b}ęrki skal þę́r rísta \hld\ ok á \edtrans{\alst{b}aðmi}{beam}{\Afootnote{\emph{barri} ‘leaf’}} viðar, &
\ind \edtext{þęim’s}{\Afootnote{\emph{þęss es} \VolsungaMS}} \alst{l}úta austr \alst{l}imar.\eva

\bvb Limb-runes shalt thou know, if thou wilt be a leecher, and know how to look at wounds; on a birch shall [one] carve them, and on the beam of the wood: [on] the one whose limbs bow to the east.\footnoteB{Probably referring to a characteristically bent mountain birch bowing to the east.}\evb
\evg


\bvg
\bva\mssnote{\Regius~32r/31—34, \VolsungaMS~24v/19–21}\alst{M}ál-rúnar skalt kunna \hld\ ef \edtext{vilt}{\Afootnote{om. \VolsungaMS}} at \alst{m}ann-gi þér &
\ind \alst{h}ęiptum \edtext{gjaldi}{\Afootnote{\emph{†giallda†} \VolsungaMS}} \alst{h}arm; &
þę́r of \alst{v}indr, \hld\ þę́r of \alst{v}ęfr, &
\ind þę́r of \alst{s}ętr allar \alst{s}aman, &
á \alst{þ}ví \alst{þ}ingi \hld\ es \edtrans{\alst{þ}jóðir}{nations}{\Afootnote{\emph{męnn} \VolsungaMS\ breaks alliteration.}} skulu &
\ind í \alst{f}ulla dóma \alst{f}ara.\eva

\bvb Speech-runes shalt thou know, if thou wilt that no man should repay thy offences with harm; them thou windest, them thou weavest, them thou settest all together, on that Thing as nations shall go to full judgements.\evb
\evg


\bvg
\bva\mssnote{\Regius~32r/34–32v/3, \VolsungaMS~25r/9–10}\alst{H}ug-rúnar skalt \edtrans{kunna}{know}{\Afootnote{\emph{nema} ‘learn’ \VolsungaMS}} \hld\ ef vilt \alst{h}vęrjum vesa &
\ind \edtrans{\alst{g}ęð-svinnari}{sense-swifter}{\Afootnote{\emph{gęð-horskari} ‘sense-sharper’ \VolsungaMS}} \alst{g}uma; &
þę́r of \alst{r}éð, \hld\ þę́r of \alst{r}ęist, &
\ind þę́r of \alst{h}ugði \alst{H}roptr, &
\edtext{af þęim \alst{l}ęgi \hld\ es \alst{l}ekit hafði &
\ind ór \alst{h}ausi \alst{H}ęiðdraupnis &
\ind ok ór \alst{h}orni \alst{H}oddrofnis.}{\lemma{af \dots\ Hoddrofnis ‘from \dots\ Hoardrovner’s [horn].}\Bfootnote{om. \VolsungaMS}}\eva

\bvb Mind-runes shalt thou know, if thou wilt be sense-swifter than every man; them did counsel, them did carve, them did Roft think out, from that liquid which had leaked out of Heathdreepner’s skull and out of Hoardrovner’s horn.\evb
\evg


\bvg
\bva\mssnote{\Regius~32v/3–4}Á \alst{b}jargi stóð \hld\ með \alst{B}rimis ęggjar, &
\ind \alst{h}afði sér á \alst{h}ǫfði \alst{h}jalm; &
\ind þá \alst{m}ę́lti \alst{M}íms hǫfuð &
\ind \alst{f}róðligt it \alst{f}yrsta orð, &
\ind ok \alst{s}agði \alst{s}anna stafi.\eva

\bvb On the barrow [he] stood along Brimer’s edges; had on his head a helmet. Then spoke the Mime’s head, learnedly, the first word, and said true staves:\evb
\evg


\bvg
\bva[14a]\mssnote{\Regius~32v/5–7, \VolsungaMS~25r/11–13}Á \alst{sk}ildi kvað ristnar \hld\ þęim’s stęndr fyr \alst{sk}ínanda goði, &
\edtrans{á \alst{ęy}ra \alst{Á}rvakrs, \hld\ ok á}{on Yorewaker’s ear and on}{\Afootnote{om. \VolsungaMS}} \alst{A}lsvinns hófi, &
\edtext{á}{\Afootnote{\emph{ok á} \VolsungaMS}} því \alst{hv}éli \hld\ es \edtrans{snýsk}{turns}{\Afootnote{\emph{stęndr} ‘stands’ \VolsungaMS}} und ręið \edtrans{\alst{H}rungnis}{Rungner’s}{\Afootnote{emend. based on sense and meter; \emph{Ra\d{v}gnis} \Regius; \emph{Raugnis} \VolsungaMS}}, &
á \alst{S}lęipnis \edtrans{tǫnnum}{teeth}{\Afootnote{\emph{taumum} ‘reins’ \VolsungaMS}} \hld\ ok á \alst{s}lęða fjǫtrum,\eva

\bvb On a shield, [he] declared [there to be] carved [runes]—[on] the one that stands before the shining god\footnoteB{Cf. \Grimnismal\ 39, according to which the sun is covered by a shield, protecting the earth from its heat. Without it, the whole world will burn up.} \ken{sun}; on Yorewaker’s ear and on Allswith’s hoof,\footnoteB{The two horses that pull the sun across the heavens; cf. \Grimnismal\ 38.} on that wheel which turns beneath Rungner’s chariot, on Slopner’s teeth and on the fetters of sleds,\evb
\evg


\bvg
\bva[14b]\mssnote{\Regius~32v/7–9, \VolsungaMS~25r/13–15}á \alst{b}jarnar hrammi \hld\ ok á \alst{B}raga tungu, &
á \alst{u}lfs klóum \hld\ ok á \alst{a}rnar \edtext{nęfi}{\Afootnote{†nefiu† \VolsungaMS}}, &
á \alst{b}lóðgum vę́ngjum \hld\ ok á \alst{b}rúar sporði, &
á \alst{l}ausnar \alst{l}ófa \hld\ ok \edtext{á}{\Afootnote{om. \VolsungaMS}} \alst{l}íknar spori,\eva

\bvb on the bear’s paw and on Bray’s tongue, on the wolf’s claws and on the eagle’s beak, on bloody wings and on the bridge’s supports, on the palm of release and the track of grace,\evb
\evg


\bvg
\bva[14c]\mssnote{\Regius~32v/9–11, \VolsungaMS~25r/15–18}á \alst{g}lęri ok á \alst{g}ulli \hld\ ok á \edtrans{\alst{g}umna hęillum}{men’s luck-charms}{\Afootnote{\emph{góðu silfri} \VolsungaMS}}, &
í \alst{v}íni ok \alst{v}irtri \hld\ ok \edtrans{\alst{v}ili-sessi}{the comfortable seat}{\Afootnote{\emph{vǫlu sessi} ‘a \inx[C]{wallow}’s seat’ \VolsungaMS}\Bfootnote{\emph{í guma holdi} ‘in a man’s flesh’ add. \VolsungaMS\ is clearly an inserted line.}}, &
á \edtrans{\alst{G}ungnis oddi}{Gungner’s point}{\Afootnote{\emph{Gaupnis oddi} ‘Yeapner’s point’ (an elsewhere unknown spear) \VolsungaMS}} \hld\ ok á \edtrans{\alst{G}rana brjósti}{Grane’s chest}{\Afootnote{\emph{gýgjar brjósti} ‘a \inx[C]{gow}’s chest’}}, &
á \alst{n}ornar \alst{n}agli \hld\ ok á \alst{n}ęfi uglu;\eva

\bvb on glass and on gold and on men’s luck-charms, in wine and beerwort and the comfortable seat, on Gungner’s point and on Grane’s chest, on a norn’s nail and on an owl’s beak.\evb
\evg\stepcounter{stanza}


\bvg
\bva\mssnote{\Regius~32v/11–14, \VolsungaMS~25r/18–21}\alst{A}llar vǫ́ru \alst{a}f skafnar, \hld\ þę́r’s vǫ́ru \alst{á} ristnar, &
\ind ok \edtrans{\alst{h}vęrfðar}{turned}{\Afootnote{\emph{†hrędar†} (for \emph{hrǿrðar} ‘stirred’?) \VolsungaMS}} við inn \alst{h}ęlga mjǫð &
\ind ok sęndar á \alst{v}íða \alst{v}ega: &
þę́r ’ru með \edtext{\alst{ǫ́}sum, \hld\ \edtrans{þę́r ’ru}{they are}{\Afootnote{\emph{sumar} ‘some’ \VolsungaMS}} með \alst{ǫ}lfum}{\lemma{ǫ́sum \dots\ ǫlfum ‘Ease \dots\ Elves’}\Afootnote{\emph{ǫlfum \dots\ ǫ́sum} ‘Elves \dots\ Ease’ \VolsungaMS}}, &
\ind \edtrans{sumar}{some}{\Afootnote{\emph{ok} ‘and’ \VolsungaMS}} með \alst{v}ísum \alst{v}ǫnum, &
\ind sumar hafa \alst{m}ęnskir \alst{m}ęnn.\eva

\bvb All were shaven off—those that were carved on—and turned into the holy mead, and sent on wide ways: They are among the Ease, they are among the Elves; some among wise Wanes; some have manly men.\evb
\evg


\bvg
\bva\mssnote{\Regius~32v/14–16, \VolsungaMS~25r/21–25v/3}Þat eru bók-rúnar, \hld\ \edtrans{þat eru}{there are}{\Afootnote{\emph{ok} ‘and’ \VolsungaMS}} bjarg-rúnar &
\ind ok allar ǫl-rúnar &
\ind ok \edtrans{mę́tar}{noble}{\Afootnote{\emph{mę́rar ok} ‘renowned and’ \VolsungaMS}} męgin-rúnar &
hvęim’s þę́r kná ó-villtar \hld\ ok \edtext{ó-spilltar}{\Afootnote{\emph{†of villtar†} \VolsungaMS}} &
\ind sér at hęillum hafa; &
\ind njót-tu ef namt &
\ind unds \edtext{rjúfask}{\Afootnote{\emph{rjúfa} \VolsungaMS}} ręgin!\eva

\bvb There are book-runes, there are rescue-runes, and all ale-runes, and noble might-runes—for whomever knows them unfalsified and uninjured, to use for himself as charms. Benefit if thou learnest, until the Reins are ripped!\evb
\evg

\sectionline

\bvg
\bva\mssnote{\Regius~32v/16–18, \VolsungaMS~25v/3–5}„Nú skalt \alst{k}jósa \hld\ alls þér ’s \alst{k}ostr of boðinn, &
\ind \alst{h}vassa vápna \alst{h}lynr, &
\alst{s}ǫgn eða þǫgn \hld\ haf þér \alst{s}jalfr í hug; &
\ind ǫll eru \alst{m}ęin of \alst{m}etin.“\eva

\bvb {[Syedrive quoth:]} “Now shalt thou choose, as the choice is offered to thee, O maple-tree of sharp weapons \ken{warrior}! Speech or silence have thou in thy own heart; all the harms are measured\footnoteB{i.e. in advance.}!”\evb
\evg


\bvg
\bva\mssnote{\Regius~32v/18–20, \VolsungaMS~25v/5–8}„Mun’k-a ek \alst{f}lǿja \hld\ þótt mik \alst{f}ęigan vitir, &
\ind em’k-a ek \edtrans{með}{with}{\Afootnote{om. \VolsungaMS}} \alst{b}lęyði \alst{b}orinn; &
\alst{á}st-rǫ́ð þín \hld\ ek vil \alst{ǫ}ll hafa &
\ind svá \alst{l}ęngi sem ek \alst{l}ifi.“\eva

\bvb {[Siward quoth:]} “I shall not flee, although thou know me to be fey; I am not born with softness.\footnoteB{TODO: Note about this common heroic expression.} Thy loving counsels all will I have, for as long as I may live.”\evb
\evg


\bvg
\bva\mssnote{\Regius~32v/20–22}„Þat rę́ð’k þér it \alst{f}yrsta \hld\ at við \alst{f}rę́ndr þína &
\ind \alst{v}amma-laust \alst{v}erir; &
\alst{s}íðr þú hęfnir \hld\ þótt þęir \alst{s}akar gøri; &
\ind þat kveða \alst{d}auðum \alst{d}uga.“\eva

\bvb {[Syedrive quoth:]} “That I counsel thee first: that thou against thy kinsmen defend thyself faultlessly. Late oughtst thou to take revenge, although they incur charges; that they say befits the dead.\evb
\evg


\bvg
\bva\mssnote{\Regius~32v/22–24}Þat rę́ð’k þér annat, \hld\ at ęið né svęrir, &
\ind nema þann ’s saðr séi, &
grimmar \edtrans{simar}{strands}{\Bfootnote{i.e. ‘strands of fate’; cf. \HelgakvidaOne\ 3, where the norns are said to twist such strands. Often emended to \emph{limar} ‘ramifications’ in accordance with \Reginsmal\ 4, where that word is used in basically the same context. Such a scribal confusion is easily understood, since \emph{s} in this position was always spelled with long \emph{ſ} in the old mss. The paraphrase (see other note) is not conclusive, since it replaces this word with \emph{hefnd} ‘revenge’.}} \hld\ ganga at tryggð-rofi; &
\ind armr es vára vargr.\eva

\bvb That I counsel thee second: that thou not swear an oath, save for that one which is true. Grim strands come after the troth-breach; wretched is the outlaw of vows.\footnoteB{The punishment is one of torment in the afterlife; see note to \Voluspa\ 39. — The whole verse is paraphrased in \VolsungaSaga\ ch. 21: \emph{Ok sver eigi rangan eið, því at grimm hefnd fylgir griðrofi.} ‘And swear no wrong oath, for grim revenge follows the grith-breach.’}\evb
\evg


\bvg
\bva\mssnote{\Regius~32v/24–25}Þat rę́ð’k þér þriðja \hld\ at þú þingi á &
\ind dęili-t við hęimska hali &
því-at ó-sviðr maðr \hld\ lę́tr oft kveðin &
\ind verri orð an viti.\eva

\bvb That I counsel thee third: that thou on the Thing bandy not with foolish men; for an unwise man often lets be spoken worse words than he ought to know.\evb
\evg


\bvg
\bva\mssnote{\Regius~32v/25–28}Allt er vant \hld\ ef við þęgir; &
\ind þá þikkir þú með blęyði borinn &
\ind eða sǫnnu sagðr; &
\ind hę́ttr es hęimis-kviðr &
\ind nema sér góðan geti. &
Annars dags \hld\ lát hans ǫndu farit &
\ind ok launa svá lýðum lygi.\eva

\bvb All is missing if thou shut up towards it; then thou seemest born with softness, or truthfully accused. Risky is the hometown-verdict, unless one gets himself a good one. At another day let thou destroy his soul, and thus repay the people for the lie.\evb
\evg


\bvg
\bva\mssnote{\Regius~32v/28–30}Þat rę́ð’k þér it fjórða \hld\ ef býr for-dę́ða &
\ind vamma-full á vegi: &
ganga ’s betra \hld\ an gista séi &
\ind þótt þik nǫ́tt of nemi.\eva

\bvb That I counsel thee fourth, if there lives an evil-working woman, full of faults, by the road: to walk is better than to take lodgings, although night overtake thee.\evb
\evg


\bvg
\bva\mssnote{\Regius~32v/30–32}\edtrans{For-njósnar}{looking ahead}{\Bfootnote{Verbal noun to \emph{nýsask fyrir} ‘to look ahead’, as found in \Havamal\ 7.}} augu \hld\ þurfu fira synir &
\ind hvar’s skulu vręiðir vega; &
oft bǫl-vísar konur \hld\ sitja brautu nér; &
\ind þę́r’s dęyfa sverð ok sefa.\eva

\bvb Eyes of looking ahead do the sons of men need, wherever wroth ones should fight; often bale-wise women sit near the highway, those who dull sword and sense.\evb
\evg


\bvg
\bva\mssnote{\Regius~32v/32–34}Þat rę́ð’k þér it fimmta, \hld\ þótt fagrar séir &
\ind brúðir bękkjum á, &
sifja silfr \hld\ lát-a þínum svefni ráða, &
\ind tęygj-at þér at kossi konur.\eva

\bvb That I counsel thee fifth, although thou seest fair brides on the benches, let not kinsmen’s silver rule thy sleep; lure not women to thee for kissing.\evb
\evg


\bvg
\bva\mssnote{\Regius~32v/34}\edtext{Þat rę́ð’k þér it sétta, \hld\ þótt með sęggjum fari}{\lemma{Þat \dots\ fari ‘That \dots\ may grow’}\Bfootnote{With these words 32v of \Regius\ ends and we have the “great lacuna”.}} &
\ind ǫlðrmál til ǫfug: &
drukkinn dęila \hld\ skal-at við dolg-viðu &
\ind margan stelr vín viti.\eva

\bvb That I counsel thee sixth, although among warriors may grow the ale-speaking awry: drunkenly deal shalt thou not with war-trees \ken{warriors}; wine steals wit from many.\evb
\evg
