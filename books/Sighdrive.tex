\bookStart{The Speeches of Syedrive}[Sigrdrífumǫ́l]

\begin{flushright}%
\textbf{Dating} \parencite{Sapp2022}: C10th (0.961)

\textbf{Meter:} \Ljodahattr%
\end{flushright}

% Introduction

% Preservation

\Sigrdrifumal\ is attested in full in \Regius, where it directly proceeds \Fafnismal.  In the manuscript there is no marker of any kind, not even an initial, separating the two “poems”, so that their existence is strictly editorial.

A number of stanzas are also attested in \VolsungaMS, the main ms. of \VolsungaSaga. \VolsungaSaga\ ch. 21 begins:

\begin{quote}
  \emph{Brynhildr segir, at tveir konungar bǫrðust. Hét annarr Hjalmgunnarr; hann var gamall ok hinn mesti hermaðr, ok hafði Óðinn honum sigr heitit, en annarr Agnarr eða Auða bróðir. „Ek fellda Hjalmgunnarr í orrostu, en Óðinn stakk mik svefn-þorni í hefnd þess ok kvað mik aldri síðan skyldu sigr hafa ok kvað mik giptast skulu. En ek strengða þess heit þar í mót at giptast engum þeim, er hrę́ðast kynni.“ Sigurðr mę́lti: „Kenn oss ráð til stórra hluta.“ Hun svarar: „Þér munuð betr kunna, en með þǫkkum vil ek kenna yðr, ef þat er nǫkkut, er vér kunnum, þat er yðr mę́tti líka, í rúnum eða ǫðrum hlutum, er liggja til hvers hlutar, ok drekkum bę́ði saman, ok gefi goðin okkr góðan dag, at þér verði nýt ok fregð at mínum vitrleik, ok þú munir eptir þat, er vit réðum.“ Brynhildr fylldi eitt ker ok fę́rði Sigurði ok mę́lti:}

  ‘Byrnhild says that two kings fought. One was called Helmguther; he was old and the greatest warrior, and Weden had promised him victory. And the other was called Eyner or Eade’s brother. “I felled Helmguther in battle, but Weden stung me with a sleeping-thorn as revenge for that, and declared that I should never thenceforth have victory, and said that I must marry. But in response I made the vow to marry no man who could be frightened.” Siward spoke: “Teach us counsels regarding great things.” She answers: “Ye will know better, but with thanks I will teach you, if there is anything which we know that may please you, of runes or other things of importance; and let us both drink together, and may the gods give us two a good day, that thou have use and joy from my wisdom and that thou afterwards recall that which we two speak of.” Byrnhild filled a vessel and brought it to Siward and spoke:’
\end{quote}

After this the saw cites sts. 5–13 and 15–19 in uninterrupted sequence, and paraphrases sts. 20 ff. (TODO: edit these!).  The order of stanzas in \VolsungaMS\ is rather different from that of \Regius.  Both mss. have sts. 5–6 and 13, 15–19 in the same place, but the order of sts. 7–12 in between is divergent, as seen by the following table:

\begin{longtabu} to \textwidth {|c c c c|}
	\hline
	\multicolumn{2}{|c}{\emph{pres. ed.}} & \Regius & \VolsungaMS \\ [0.5ex]
	\hline\hline\endhead
	\hline\endfoot
	5 & Bjór fǿri’k þér & 5 & 6 \\
	6 & Sig-rúnar skalt rísta & 6 & 7 \\
  7 & Ǫl-rúnar skalt kunna & 7 & 10 \\
  8 & Full skal signa & 7* & 11 \\
  9 & Bjarg-rúnar skalt kunna & 8 & 12 \\
  10 & Brim-rúnar skalt rísta & 9 & 8 \\
  11 & Lim-rúnar skalt kunna & 10 & 13 \\
  12 & Mál-rúnar skalt kunna & 11 & 9 \\
  13 & Hug-rúnar skalt kunna & 12a & 14 \\
  14 & Á bjargi stóð & 12b–13 & − \\
  15 & Á skildi kvað ristnar & 14–15a & 15–17 \\
  16 & Allar vǫ́ru af skafnar & 15b–16 & 18 \\
  17 & Þat eru bókrúnar & 17 & 19 \\
  18 & Nú skalt kjósa & 18 & 20 \\
  19 & Mun’k-a ek flǿja & 19 & 21 \\ [1ex]
	\hline
\end{longtabu}

% Contents

The contents of the poem. TODO

\sectionline

\bpg\bpa Sigurðr reið upp á Hindarfjall ok stefndi suðr til Frakklands. Á fjallinu sá hann ljós mikit svá sem eldr brynni ok ljómaði af til himins. En er hann kom at þá stóð þar skjald-borg ok upp ór merki. Sigurðr gekk í skjald-borgina ok sá at þar lá maðr ok svaf með ǫllum her-vápnum. Hann tók fyrst hjálminn af hǫfði hánum; þá sá hann at þat var kona. Brynjan var fǫst sem hon vę́ri hold-gróin. Þá reist hann með Gram frá hǫfuð-smátt brynjuna í gǫgnum niðr ok svá út í gǫgnum báðar ermar. Þá tók hann brynju af henni en hon vaknaði ok settisk hon upp ok sá Sigurð ok mę́lti:\epa

\bpb Siward rode up on the Hinderfell and stood looking south toward Frankland. On the fell he saw a light as great as if a fire burned, and the rays from it went up to heaven. But when he came there, there was a shield-wall rising up out of the ground. Siward went into the shield-wall and saw that a man lay there, and he was asleep in full gear of war. He first took the helmet off his head; then he saw that it was a woman. The byrnie was as fast as if it were grown out of her flesh. With Gram he then cut the byrnie from the head hole down through it and then out through both sleeves. Then he took the byrnie off her, and she awakened and sat herself up and saw Siward and spoke:\epb\epg


\bvg\bva „Hvat bęit brynju? \hld\ Hví brá’k svefni? &
Hvęrr fęlldi af mér \hld\ fǫlvar nauðir?“ &
„Sigmundar burr, \hld\ slęit fyr skǫmmu &
hrafns \edtext{hrygg-lundir}{\Afootnote{emend.; \emph{hrę́-lundir} \Regius}} \hld\ hjǫrr Sigurðar.“\eva

\bvb “What bit the byrnie? Why did I break my sleep? \\
Who loosened from me these death-pale chains?” \\
\speakernoteb{He answered:}“Syemund’s son did just tear off \\
the raven’s loins, and Siward’s sword.”\evb\evg


\bvg\bva „Lęngi ek \alst{s}vaf, \hld\ lęngi ek \alst{s}ofnuð vas, &
\ind \alst{l}ǫng eru \alst{l}ýða \alst{l}ę́; &
\alst{Ó}ðinn því vęldr \hld\ es \alst{ęi}gi mátta’k &
\ind \alst{b}regða \alst{b}lund-stǫfum.“\eva

\bvb\speakernoteb{[Syedrive quoth:]}%
“Long I slept, long was I asleep, \\
\ind long are the guiles of men. \\
Weden has caused that I could not \\
\ind break the staves of sleep.”\evb\evg


\bpg\bpa Sigurðr sęttisk niðr ok spyrr hana nafns. Hón tók þá horn fullt mjaðar ok gaf hǫ́num minnis-vęig.\epa

\bpb Siward set himself down and asks for her name. Then she took a horn full of mead and gave him a draught of memory:\epb\epg


\bvg\bva%
Hęill \alst{D}agr, \hld\ hęilir \edtrans{\alst{D}ags synir}{Day’s sons}{\Bfootnote{Their identity is uncertain.}}, &
\ind hęil \alst{N}ǫ́tt ok \edtrans{\alst{n}ipt}{the kinswoman \ken*{= Earth}}{\Bfootnote{According to \Gylfaginning\ 10 Earth is the daughter of Night; \emph{nipt} typically refers to a younger female relative.}}! &
\edtrans{\alst{Ó}-ręiðum \alst{au}gum \hld\ lítið \alst{o}kkr þinig}{With unwrathful eyes look ye the way of us two}{\Bfootnote{i.e. “behold us two with friendly gaze”.  An archaic conception; the grace or wrath of the Gods is conveyed by their “face” looking upon the worshipper.  The same thing is found in other ancient literatures, e.g. in the Hebrew Bible, most famously in the “Priestly Blessing” of \emph{Numbers} 6:24–26 (“25 May Yahweh light up His face to you and grant grace to you; / 26 May Yahweh lift up His face to you and give you peace.”)  Other Biblical examples include \emph{Psalms} 4:6 (“Lift up the light of Your face to us, Yahweh) and the chorus of Psalm 80 (“Yahweh God of Armies, bring us back. / Light up Your face, that we may be rescued.”)}} &
\ind ok gefið \alst{s}itjǫndum \alst{s}igr!\eva

\bvb “Hail \inx[P]{Day}! Hail Day’s sons! \\
\ind Hail Night and the kinswoman \ken*{= Earth}! \\
With unwrathful eyes look ye the way of us two, \\
\ind and give the sitters \ken*{= us} victory.\evb\evg


\bvg\bva%
\edtrans{Hęilir \alst{ę́}sir, \hld\ hęilar \alst{ǫ́}synjur}{Hail the Eese! Hail the Ossens!}{\Bfootnote{Probably formulaic, subverted by Lock in \Lokasenna\ 11; see note there for possible ritual use.}}, &
\ind hęil \edtrans{sjá in \alst{f}jǫl-nýta \alst{f}old}{this much-giving Fold}{\Bfootnote{i.e. “the bountiful \inx[P]{Earth}”; an Old Indo-European expression.  In the Norse poetic corpus \emph{fold} elsewhere refers to ‘land, earth’ without mythological associations, the present st. being the only exception.  It is probably a ritual archaism; cf. the Old English \emph{Acreboot}: \emph{Hâl wes þú Folde \hld\ fira módor!} ‘Hail be thou, Fold, mother of men!’ and the Old Indian cognate name \emph{Pr̥thivī} (Mother Earth), found frequently in \Rigveda.  The common Indo-European root is \emph{*pl̥th₂-éwih₂} ‘flat, broad one’; cf. Hfr \emph{Hákdr} 8 (in \Skp\ III), where Earth is the \emph{bręið-lęita brúðr Bálęygs} ‘broad-faced bride of Baleeyed \name{= Weden}’.
For the epithet ‘much-giving’ cf. \emph{Iliad} 3.89: \textgreek{ἐπὶ χθονὶ πουλυ-βοτείρῃ} ‘upon the much-nourishing earth’, where \textgreek{πουλυ-} is cognate with ON \emph{fjǫl-}, both coming from PIE \emph{*pélh₁u-} \char`~\ \emph{*pólh₁u-} ‘much, many’.}}! &
\alst{M}ál ok \alst{m}an-vit \hld\ gefið okkr \alst{m}ę́rum tvęim &
\ind ok \edtrans{\alst{l}ę́knis-hęndr}{a leecher’s hands}{\Bfootnote{The hands of a physician, i.e., hands with healing powers.  The singular \emph{lę́knis-hǫnd} occurs on the Ribe galder stick (DR EM85;493), edited below under Galders.}} meðan \alst{l}ifum!\eva

\bvb Hail the \inx[G]{Eese}! Hail the \inx[G]{Ossens}! \\
\ind Hail this much-giving \inx[P]{Fold}! \\
Speech and \inx[C]{manwit} give ye to us renowned two, \\
\ind and a leecher’s hands, while we live.”\evb\evg


\bpg\bpa Hon nefndisk Sigrdrífa ok var valkyrja. Hon sagði, at tveir konungar bǫrðusk. Hét annarr Hjalmgunnarr; hann var þá gamall ok inn mesti hermaðr, ok hafði Óðinn hánum sigri heitit.
En \alst{a}nnarr hét \alst{A}gnarr, \hld\ \alst{Au}ðu bróðir // er \alst{v}ę́tr engi \hld\ \alst{v}ildi þiggja.
Sigrdrífa felldi Hjalm-gunnar í orrostunni. En Óðinn stakk hana svefn-þorni í hefnd þess ok kvað hana aldri skyldu síðan sigr vega í orrostu, ok kvað hana giftask skyldu, „en sagða’k hánum at strengða’k heit þar í mót, at giptask øngom þeim manni er hrę́ðask kynni.“ Hann segir ok biðr hana kenna sér speki ef hon vissi tíðendi ór ǫllum heimum. Sigrdrífa kvað:\epa

\bpb She called herself Syedrive and was a walkirrie. She said that two kings fought. One was called Helmguther; he was then old and the greatest warrior, and Weden had promised him victory.
And the other was called Eyner, Eade’s brother, who in no way wished to surrender.
Syedrive felled Helmguther in the battle, but Weden stung her with the sleeping-thorn as revenge for that, and declared that she should never thenceforth win victory in battle, and said that she must marry, “but I told him that I in response made a vow to marry no man who could be frightened.” He \ken*{= Siward} speaks and asks her to teach him wisdom; if she knew any tidings out of all the \inx[C]{Home}[Homes]. Syedrive quoth:\epb\epg


\bvg\bva\mssnote{\Regius~32r/18–20, \VolsungaMS~24v/12–14}%
„\alst{B}jór fǿri’k þér, \hld\ \edtrans{\alst{b}ryn-þings apaldr}{apple-tree of the byrnie-Thing \ken{battle > warrior}}{\Afootnote{\emph{bryn-þinga valdr} ‘wielder of byrnie-Things \ken{battles > warrior}’ \VolsungaMS}}, &
\alst{m}agni blandinn \hld\ ok \alst{m}ęgin-tíri, &
fullr es \alst{l}jóða \hld\ ok \alst{l}íkn-stafa, &
\alst{g}óðra \alst{g}aldra \hld\ ok \edtrans{\alst{g}aman-rúna}{pleasure-runes}{\Afootnote{\emph{gaman-†rędna†} \VolsungaMS}}.\eva

\bvb Beer I bring thee, O apple-tree of the byrnie-\inx[C]{Thing} \ken{battle > warrior}! \\
mixed with might and mighty splendour; \\
it is full of \inx[C]{leed}[leeds] and grace-staves, \\
of good \inx[C]{galder}[galders] and pleasure-\inx[C]{rune}[runes].\evb\evg


\bvg\bva\mssnote{\Regius~32r/20–22, \VolsungaMS~24v/14–16}%
\alst{S}ig-rúnar skalt rísta, \hld\ ef vilt \edtrans{\alst{s}igr hafa}{have victory}{\Afootnote{\emph{snotr vera} ‘be clever’ \VolsungaMS}}, &
\ind ok \edtext{rísta}{\Afootnote{\emph{†rist†} \VolsungaMS}} á \alst{h}jalti \alst{h}jǫrs, &
\edtrans{sumar}{some}{\Afootnote{om. \VolsungaMS}} á \edtrans{\alst{v}étt-rimum}{weight-rims}{\Afootnote{\emph{vétt-†rvnum†} \VolsungaMS}\Bfootnote{Unclear. TODO.}}, \hld\ \edtrans{sumar}{some}{\Afootnote{\emph{ok} ‘and’ \VolsungaMS}} á \edtrans{\alst{v}al-bǫstum}{wal-basts}{\Afootnote{\emph{val-†bystum†} \VolsungaMS}\Bfootnote{Possibly the sword-pommel; this word also occurs in \HelgakvidaHjorvardssonar\ 9. TODO.}}, &
\ind ok nęfna \alst{t}ysvar \alst{T}ý.\eva

\bvb Victory-runes shalt thou know, if thou wilt have victory, \\
\ind and carve them on the hilt of the sword; \\
some on the weight-rims, some on the wal-basts, \\
\ind and twice name \inx[P]{Tew}.\evb\evg


\bvg\bva\mssnote{\Regius~32r/22–24, \VolsungaMS~25r/1–3}%
\alst{Ǫ}l-rúnar skalt kunna \hld\ ef vilt \edtrans{at}{that}{\Afootnote{emend. from \emph{†a†} \VolsungaMS; om. \Regius}} \alst{a}nnars kvę́n &
\ind \edtext{véli-t þik í \alst{t}ryggð}{\Afootnote{\emph{véli þik eigi tryggð} \VolsungaMS}} ef \alst{t}rúir; &
á \alst{h}orni skal \edtrans{þę́r}{them}{\Afootnote{\emph{þat} ‘it’ \VolsungaMS}} rísta \hld\ ok á \alst{h}andar baki &
\ind ok męrkja á \alst{n}agli \edtrans{\alst{N}auð}{Need}{\Bfootnote{i.e. the n-rune, ᚾ.}}.\eva

\bvb Ale-runes shalt thou know, if thou wilt that another man’s wife \\
\ind not betray thee in troth if thou trust her. \\
On the horn shall one carve them, and on the back of the hand, \\
\ind and mark Need on the nail.\evb\evg


\bvg\bva\mssnote{\Regius~32r/24–25, \VolsungaMS~25r/3–4}%
\edtrans{\alst{F}ull}{The cup}{\Afootnote{\emph{ǫl} ‘The ale’ \VolsungaMS\ breaks alliteration.}} skal \edtrans{signa}{sign}{\Bfootnote{Dedicating the cup by means of making a certain sign or speech over it. TODO.}} \hld\ ok við \alst{f}ári séa &
\ind ok verpa \alst{l}auki í \alst{l}ǫg; &
\edtext{\alst{þ}á þat vęit’k, \hld\ at \alst{þ}ér verðr aldri-gi &
\ind \edtext{\alst{m}ęini blandinn}{\Afootnote{emend.; \emph{męin-blandinn} \VolsungaMS}} \alst{m}jǫðr.}{\lemma{þá \dots\ mjǫðr}\Bfootnote{only in \VolsungaMS; om. \Regius}}\eva

\bvb The cup shall one sign, and gaze against the danger, \\
\ind and throw in the liquid a leek. \\
Then I know that it will never be \\
\ind mixed with harm, thy mead.\evb\evg


\bvg\bva\mssnote{\Regius~32r/25–26, \VolsungaMS~25r/5–7}%
\alst{B}jarg-rúnar skalt \edtrans{kunna}{know}{\Afootnote{\emph{nema} ‘learn’ \VolsungaMS}} \hld\ \edtrans{ef \alst{b}jarga vilt}{if thou wilt rescue}{\Afootnote{\emph{ef þú vilt borgit fá} ‘if thou wilt have rescued’ \VolsungaMS}} &
\ind ok lęysa \alst{k}ind frá \alst{k}onum; &
á \alst{l}ófa þę́r skal rísta \hld\ ok of \alst{l}iðu spęnna &
\ind ok biðja \edtrans{þá}{then}{\Afootnote{om. \VolsungaMS}} \edtrans{\alst{d}ísir}{dises}{\Bfootnote{Minor goddesses and fates; one of their roles was helping ailing women during childbirth.  Cf. \Fafnismal\ 12 where \emph{nornir} ‘Norns’ is used for the childbirth goddesses.}} \alst{d}uga.\eva

\bvb Rescue-runes shalt thou know, if thou wilt rescue \\
\ind and loosen children from women; \\
on the palm shall one carve them, and wrap them round the joints, \\
\ind and then bid the dises to avail.\evb\evg


\bvg\bva\mssnote{\Regius~32r/27–29, \VolsungaMS~24v/16–19}\alst{B}rim-rúnar skalt \edtrans{rísta}{carve}{\Afootnote{\emph{gjǫra} ‘make’ \VolsungaMS}} \hld\ ef vilt \alst{b}orgit hafa &
\ind á \alst{s}undi \alst{s}egl-mǫrum; &
á \alst{st}afni \edtrans{skal rísta}{shall [one] carve}{\Afootnote{\emph{skal þę́r rísta} ‘shall [one] carve them’ \VolsungaMS}} \hld\ ok á \alst{st}jórnar blaði &
\ind ok \edtrans{lęggja \alst{ę}ld í \alst{á}r}{lay fire into the oar}{\Bfootnote{i.e. mark it with fire in some way.}};
\edtrans{es-a}{There is not}{\Afootnote{\emph{falla-t} ‘There fall not’ \VolsungaMS}} svá \alst{b}rattr \alst{b}reki \hld\ né svá \alst{b}láar unnir, &
\ind \edtext{þó kømsk-tu \alst{h}ęill af \alst{h}afi}{\lemma{þó ... hafi ‘that ... sea’}\Bfootnote{lit. ‘yet comest thou whole off the sea.’}}.\eva

\bvb Surf-runes shalt thou carve, if thou wilt rescue \\
\ind sail-steeds \ken{ships} on the sound; \\
on the stem shall one carve them, and on the rudder’s blade, \\
\ind and lay fire into the oar. \\
There is not so steep a breaker nor so dark blue waves \\
\ind that thou not come whole off the sea.\evb\evg


\bvg\bva\mssnote{\Regius~32r/29–31, \VolsungaMS~25r/7–9}\alst{L}im-rúnar skalt kunna \hld\ ef vilt \alst{l}ę́knir vesa &
\ind ok kunna \alst{s}ár at \alst{s}éa; &
á \alst{b}ęrki skal þę́r rísta \hld\ ok á \edtrans{\alst{b}aðmi}{beam}{\Afootnote{\emph{barri} ‘leaf’}} viðar, &
\ind \edtext{þęim’s}{\Afootnote{\emph{þęss es} \VolsungaMS}} \alst{l}úta austr \alst{l}imar.\eva

\bvb Limb-runes shalt thou know, if thou wilt be a leecher, \\
\ind and know how to look at wounds; \\
on a birch shall one carve them, and on the beam of the wood: \\
\ind on the one whose limbs bow to the east.\footnoteB{Probably referring to a characteristically bent mountain birch bowing to the east.}\evb\evg


\bvg\bva\mssnote{\Regius~32r/31—34, \VolsungaMS~24v/19–21}\alst{M}ál-rúnar skalt kunna \hld\ ef \edtext{vilt}{\Afootnote{om. \VolsungaMS}} at \alst{m}ann-gi þér &
\ind \alst{h}ęiptum \edtext{gjaldi}{\Afootnote{\emph{†giallda†} \VolsungaMS}} \alst{h}arm; &
þę́r of \alst{v}indr, \hld\ þę́r of \alst{v}ęfr, &
\ind þę́r of \alst{s}ętr allar \alst{s}aman, &
á \alst{þ}ví \alst{þ}ingi \hld\ es \edtrans{\alst{þ}jóðir}{nations}{\Afootnote{\emph{męnn} \VolsungaMS\ breaks alliteration.}} skulu &
\ind í \alst{f}ulla dóma \alst{f}ara.\eva

\bvb Speech-runes shalt thou know, if thou wilt that no man \\
\ind should repay thy insults with harm; \\
them dost thou wind, them dost thou weave, \\
\ind them dost thou put all together, \\
on that Thing whereas peoples shall \\
\ind go to full judgements.\evb\evg


\bvg\bva\mssnote{\Regius~32r/34–32v/3, \VolsungaMS~25r/9–10}\alst{H}ug-rúnar skalt \edtrans{kunna}{know}{\Afootnote{\emph{nema} ‘learn’ \VolsungaMS}} \hld\ ef vilt \alst{h}vęrjum vesa &
\ind \edtrans{\alst{g}ęð-svinnari}{sense-swifter}{\Afootnote{\emph{gęð-horskari} ‘sense-sharper’ \VolsungaMS}} \alst{g}uma; &
þę́r of \alst{r}éð, \hld\ þę́r of \alst{r}ęist, &
\ind þę́r of \alst{h}ugði \alst{H}roptr, &
\edtext{af þęim \alst{l}ęgi \hld\ es \alst{l}ekit hafði &
\ind ór \alst{h}ausi \alst{H}ęiðdraupnis &
\ind ok ór \alst{h}orni \alst{H}oddrofnis.}{\lemma{af \dots\ Hoddrofnis ‘from \dots\ Hoardrovner’s [horn].}\Bfootnote{om. \VolsungaMS}}\eva

\bvb Mind-runes shalt thou know, if thou wilt be \\
\ind sense-swifter than every man; \\
them did counsel, them did carve, \\
\ind them did Roft think out, \\
from that liquid which had leaked \\
\ind out of Heathdreepner’s skull \\
\ind and out of Hoardrovner’s horn.\evb\evg


\bvg\bva\mssnote{\Regius~32v/3–4}%
Á \alst{b}jargi stóð \hld\ með \alst{B}rimis ęggjar, &
\ind \alst{h}afði sér á \alst{h}ǫfði \alst{h}jalm; &
\ind þá \alst{m}ę́lti \alst{M}íms hǫfuð &
\ind \alst{f}róðligt it \alst{f}yrsta orð, &
\ind ok \alst{s}agði \alst{s}anna stafi.\eva

\bvb On the barrow he stood along Brimer’s edges; \\
\ind he had on his head a helmet. \\
\ind Then Mime’s head spoke, \\
\ind learnedly, the first word, \\
\ind and said true staves:\evb\evg


\bvg\bva[15a]\mssnote{\Regius~32v/5–7, \VolsungaMS~25r/11–13}%
Á \edtext{\alst{sk}ildi kvað ristnar \hld\ þęim’s stęndr fyr \alst{sk}ínanda goði}{\lemma{skildi \dots\ þęim’s stęndr fyr skínanda goði ‘the shield \dots\ that stands before the shining god’}\Bfootnote{For this notion cf. \Grimnismal\ 39, according to which the Sun is covered by a disc shielding the earth from its heat.  Without it, the whole world would burn up.}}, &
\edtrans{á \alst{ęy}ra \alst{Á}rvakrs, \hld\ ok á}{on Yorewaker’s ear and on}{\Afootnote{om. \VolsungaMS}} \alst{A}lsvinns hófi, &
\edtext{á}{\Afootnote{\emph{ok á} \VolsungaMS}} því \alst{hv}éli \hld\ es \edtrans{snýsk}{turns}{\Afootnote{\emph{stęndr} ‘stands’ \VolsungaMS}} und ręið \edtrans{\alst{H}rungnis}{Rungner’s}{\Afootnote{emend. based on sense and meter; \emph{Ra\d{v}gnis} \Regius; \emph{Raugnis} \VolsungaMS}}, &
á \alst{S}lęipnis \edtrans{tǫnnum}{teeth}{\Afootnote{\emph{taumum} ‘reins’ \VolsungaMS}} \hld\ ok á \alst{s}lęða fjǫtrum,\eva

\bvb On the shield, it said, [runes] were carved—the one that stands before the shining god \ken{sun}; \\
on Yorewaker’s ear and on Allswith’s hoof,\footnoteB{The two horses that pull the sun across the heavens; cf. \Grimnismal\ 38.} \\
on that wheel which turns beneath Rungner’s chariot, \\
on Slapner’s teeth and on the fetters of sleds,\evb\evg


\bvg\bva[15b]\mssnote{\Regius~32v/7–9, \VolsungaMS~25r/13–15}%
á \alst{b}jarnar hrammi \hld\ ok á \alst{B}raga tungu, &
á \alst{u}lfs klóum \hld\ ok á \alst{a}rnar \edtext{nęfi}{\Afootnote{†nefiu† \VolsungaMS}}, &
á \alst{b}lóðgum vę́ngjum \hld\ ok á \alst{b}rúar sporði, &
á \alst{l}ausnar \alst{l}ófa \hld\ \edtext{ok á}{\Afootnote{\emph{ok} \VolsungaMS}} \alst{l}íknar spori,\eva

\bvb on the bear’s paw and on Bray’s tongue, \\
on the wolf’s claws and on the eagle’s beak, \\
on bloody wings and on the bridge’s supports, \\
on the palm of release and the trail of grace,\evb\evg


\bvg\bva[15c]\mssnote{\Regius~32v/9–11, \VolsungaMS~25r/15–18}%
á \alst{g}lęri ok á \alst{g}ulli \hld\ ok á \edtrans{\alst{g}umna hęillum}{men’s luck-charms}{\Afootnote{\emph{góðu silfri} ‘good silver’ \VolsungaMS}}, &
í \alst{v}íni ok \alst{v}irtri \hld\ ok \edtrans{\alst{v}ili-sessi}{the comfortable seat}{\Afootnote{\emph{vǫlu sessi} ‘a \inx[C]{wallow}’s seat’ \VolsungaMS}\Afootnote{\emph{í guma holdi} ‘in a man’s flesh’ add. \VolsungaMS.}}, &
á \edtrans{\alst{G}ungnis oddi}{Gungner’s point}{\Afootnote{\emph{Gaupnis oddi} ‘Yeapner’s point’ (an elsewhere unknown spear) \VolsungaMS}} \hld\ ok á \edtrans{\alst{G}rana brjósti}{Grane’s chest}{\Afootnote{\emph{gýgjar brjósti} ‘a \inx[C]{gow}’s chest’ \VolsungaMS}}, &
á \alst{n}ornar \alst{n}agli \hld\ ok á \alst{n}ęfi uglu;\eva

\bvb on glass and on gold and on men’s luck-charms, \\
in wine and beerwort and the comfortable seat, \\
on Gungner’s point and on Grane’s chest, \\
on a norn’s nail and on an owl’s beak.\evb\evg\stepcounter{stanza}


\bvg\bva\mssnote{\Regius~32v/11–14, \VolsungaMS~25r/18–21}%
\alst{A}llar vǫ́ru \alst{a}f skafnar, \hld\ þę́r’s vǫ́ru \alst{á} ristnar, &
\ind ok \edtrans{\alst{h}vęrfðar}{mixed}{\Afootnote{\emph{†hrędar†} (for \emph{hrǿrðar} ‘stirred’?) \VolsungaMS}} við inn \alst{h}ęlga mjǫð &
\ind ok sęndar á \alst{v}íða \alst{v}ega: &
þę́r ’ru með \edtext{\alst{ǫ́}sum, \hld\ \edtrans{þę́r ’ru}{they are}{\Afootnote{\emph{sumar} ‘some’ \VolsungaMS}} með \alst{ǫ}lfum}{\lemma{ǫ́sum \dots\ ǫlfum ‘Eese \dots\ Elves’}\Afootnote{\emph{ǫlfum \dots\ ǫ́sum} ‘Elves \dots\ Eese’ \VolsungaMS}}, &
\ind \edtrans{sumar}{some}{\Afootnote{\emph{ok} ‘and’ \VolsungaMS}} með \alst{v}ísum \alst{v}ǫnum, &
\ind sumar hafa \alst{m}ęnskir \alst{m}ęnn.\eva

\bvb All were shaven off—those that were carved on— \\
\ind and mixed into the holy mead, \\
\ind and sent on wide ways: \\
they are among the Eese, they are among the Elves, \\
\ind some among the wise Wanes, \\
\ind some have manly men.\evb\evg


\bvg\bva\mssnote{\Regius~32v/14–16, \VolsungaMS~25r/21–25v/3}%
Þat eru \edtrans{\alst{b}ók-rúnar}{book-runes}{\Bfootnote{Or ‘beech-runes’.  The word may also be emended to \emph{bót-rúnar} ‘cure-runes’, since the letters \emph{c} and \emph{t} were, in the TODO miniscule used on Iceland, very similar.  This emendation is favourable for two reasons: (i) it makes more sense, since the semantic pair \emph{bót} ‘cure’ : \emph{bjarg} ‘rescue’ is surely stronger than \emph{bók} ‘book, beech’ : \emph{bjarg} ‘rescue’, and since the present stanza is specifically referring to the practical use of the runes; (ii) the pair \emph{bót-runar} : \emph{bjarg-rúnar} is already found in a runic charm (B 257, edited under Galders from Bryggen).}}, \hld\ \edtrans{þat eru}{those are}{\Afootnote{\emph{ok} ‘and’ \VolsungaMS}} \alst{b}jarg-rúnar &
\ind ok \alst{a}llar \alst{ǫ}l-rúnar &
\ind \edtrans{ok \alst{m}ę́tar}{and noble}{\Afootnote{\emph{ok mę́rar ok} ‘and renowned and’ \VolsungaMS}} \alst{m}ęgin-rúnar &
hvęim’s þę́r kná \alst{ó}·villtar \hld\ ok \edtext{\alst{ó}·spilltar}{\Afootnote{\emph{†of villtar†} \VolsungaMS}} &
\ind sér at \alst{h}ęillum \alst{h}afa; &
\ind \alst{n}jót-tu ef \alst{n}amt &
\ind unds \edtext{\alst{r}júfask}{\Afootnote{\emph{rjúfa} \VolsungaMS}} \alst{r}ęgin!\eva

\bvb They are book-runes, those are rescue-runes, \\
\ind and all ale-runes, \\
\ind and noble might-runes— \\
for whomever knows them unfalsified and uninjured \\
\ind to use for himself as charms. \\
Use them if thou learn them \\
\ind until the Reins are ripped!\evb\evg

\sectionline

\bvg\bva\mssnote{\Regius~32v/16–18, \VolsungaMS~25v/3–5}%
„Nú skalt \alst{k}jósa \hld\ alls þér ’s \alst{k}ostr of boðinn, &
\ind \alst{h}vassa vápna \alst{h}lynr, &
\alst{s}ǫgn eða þǫgn \hld\ haf þér \alst{s}jalfr í hug; &
\ind ǫll eru \alst{m}ęin of \alst{m}etin.“\eva

\bvb {[Syedrive quoth:]} \\
“Now shalt thou choose, as the choice is offered thee, \\
O maple-tree of sharp weapons \ken{warrior}! \\
Speech or silence have for thyself in thy heart; \\
all the harms are measured\footnoteB{i.e. in advance.}!”\evb\evg


\bvg\bva\mssnote{\Regius~32v/18–20, \VolsungaMS~25v/5–8}%
„Mun’k-a ek \alst{f}lǿja \hld\ þótt mik \alst{f}ęigan vitir, &
\ind em’k-a ek \edtrans{með}{with}{\Afootnote{om. \VolsungaMS}} \alst{b}lęyði \alst{b}orinn; &
\alst{á}st-rǫ́ð þín \hld\ ek vil \alst{ǫ}ll hafa &
\ind svá \alst{l}ęngi sem ek \alst{l}ifi.“\eva

\bvb {[Siward quoth:]} “I shall not flee, although thou know me to be \inx[C]{fey}; \\
\ind I was not born with softness.\footnoteB{TODO: Note about this common heroic expression.} \\
Thy loving counsels, all, will I have \\
\ind for as long as I may live.”\evb\evg


\bvg\bva\mssnote{\Regius~32v/20–22}%
„Þat rę́ð’k þér it \alst{f}yrsta \hld\ at við \alst{f}rę́ndr þína &
\ind \alst{v}amma-laust \alst{v}erir; &
\alst{s}íðr þú hęfnir \hld\ þótt þęir \alst{s}akar gøri; &
\ind þat kveða \alst{d}auðum \alst{d}uga.“\eva

\bvb {[Syedrive quoth:]} “This I counsel thee first: that thou against thy kinsmen \\
\ind defend thyself faultlessly. \\
Late oughtst thou to take revenge, although they incur charges; \\
\ind that, they say, befits the dead.\evb\evg


\bvg\bva\mssnote{\Regius~32v/22–24}Þat rę́ð’k þér \alst{a}nnat, \hld\ at \alst{ęi}ð né svęrir, &
\ind nema þann ’s \alst{s}aðr \alst{s}éi, &
\alst{g}rimmar \edtrans{simar}{strands}{\Bfootnote{i.e. ‘strands of fate’; cf. \HelgakvidaOne\ 3, where the norns are said to twist such strands. Often emended to \emph{limar} ‘ramifications’ in accordance with \Reginsmal\ 4, where that word is used in basically the same context. Such a scribal confusion is easily understood, since \emph{s} in this position was always spelled with long \emph{ſ} in the old mss. The paraphrase (see other note) is not conclusive, since it replaces this word with \emph{hefnd} ‘revenge’.}} \hld\ \alst{g}anga at tryggð-rofi; &
\ind armr es \alst{v}ára \alst{v}argr.\eva

\bvb This I counsel thee second: that thou not swear an oath, \\
\ind save for the one which is true. \\
Grim strands follow the troth-breach; \\
\ind wretched is the outlaw of vows.\footnoteB{The punishment is one of torment in the afterlife; see note to \Voluspa\ 39. — The whole stanza is paraphrased in \VolsungaSaga\ ch. 21: \emph{Ok sver eigi rangan eið, því at grimm hefnd fylgir griðrofi.} ‘And swear no wrong oath, for grim revenge follows the grith-breach.’}\evb\evg


\bvg\bva\mssnote{\Regius~32v/24–25}Þat rę́ð’k þér \alst{þ}riðja \hld\ at þú \alst{þ}ingi á &
\ind dęili-t við \alst{h}ęimska \alst{h}ali &
því-at \alst{ó}·sviðr maðr \hld\ lę́tr \alst{o}ft kveðin &
\ind \alst{v}erri orð an \alst{v}iti.\eva

\bvb This I counsel thee third: that thou on the Thing \\
\ind not bandy with foolish men; \\
for an unwise man often lets be spoken \\
\ind worse words than he ought to know.\evb\evg


\bvg\bva\mssnote{\Regius~32v/25–28}Allt es \alst{v}ant \hld\ ef \alst{v}ið þęgir; &
\ind þá þikkir þú með \alst{b}lęyði \alst{b}orinn &
\ind eða \alst{s}ǫnnu \alst{s}agðr; &
\ind \alst{h}ę́ttr es \alst{h}ęimis-kviðr &
\ind nema sér \alst{g}óðan \alst{g}eti. &
\alst{A}nnars dags \hld\ lát hans \edtrans{\alst{ǫ}ndu}{life}{\Bfootnote{lit. ‘breath, spirit’.  Cf. \Voluspa\ 17 where \emph{ǫnd} is Weden’s gift to the first men.}} farit &
\ind ok \alst{l}auna svá \alst{l}ýðum \alst{l}ygi.\eva

\bvb Everything is wrong if thou shut up in reply; \\
\ind then thou seemest born with softness, \\
\ind or truthfully accused. \\
Risky is the hometown-verdict, \\
\ind unless one get himself a good one. \\
On another day destroy his life, \\
\ind and thus repay the people for the lie.\evb\evg


\bvg\bva\mssnote{\Regius~32v/28–30}Þat rę́ð’k þér it \alst{f}jórða \hld\ ef býr \alst{f}or-dę́ða &
\ind \alst{v}amma-full á \alst{v}egi: &
\alst{g}anga ’s betra \hld\ an \alst{g}ista séi &
\ind þótt þik \alst{n}ǫ́tt of \alst{n}emi.\eva

\bvb This I counsel thee fourth: if there lives an evil-working woman, \\
\ind full of faults, by the road, \\
to walk is better than to take lodgings, \\
\ind although night overtake thee.\evb\evg


\bvg\bva\mssnote{\Regius~32v/30–32}\edtrans{\alst{F}or-njósnar}{looking-ahead}{\Bfootnote{Verbal noun to \emph{nýsask fyrir} ‘to look ahead’, as found in \Havamal\ 7.}} augu \hld\ þurfu \alst{f}ira synir &
\ind hvar’s skulu \alst{v}ręiðir \alst{v}ega; &
oft \alst{b}ǫl-vísar konur \hld\ sitja \alst{b}rautu nér; &
\ind þę́r’s dęyfa \alst{s}verð ok \alst{s}efa.\eva

\bvb Eyes of looking-ahead the sons of men need, \\
\ind wherever wroth men should fight; \\
oft bale-wise women sit near the highway, \\
\ind they who dull sword and sense.\evb\evg


\bvg\bva\mssnote{\Regius~32v/32–34}Þat rę́ð’k þér it \alst{f}immta, \hld\ þótt \alst{f}agrar séir &
\ind \alst{b}rúðir bękkjum á, &
\alst{s}ifja \alst{s}ilfr \hld\ lát-a þínum \alst{s}vefni ráða, &
\ind tęygj-at þér at \alst{k}ossi \alst{k}onur.\eva

\bvb This I counsel thee fifth: although thou seest \\
fair brides on the benches, \\
let not kinsmen’s silver rule thy sleep; \\
lure not women to thee for kisses.\evb\evg


\bvg\bva\mssnote{\Regius~32v/34}\edtext{Þat rę́ð’k þér it \alst{s}étta, \hld\ þótt með \alst{s}ęggjum fari}{\lemma{Þat \dots\ fari ‘That \dots\ may grow’}\Bfootnote{With these words fol. 32v of \Regius\ ends, and we have the “great lacuna”.  The rest of the stanzas are supplied from younger paper mss.}} &
\ind \alst{ǫ}lðr-mál til \alst{ǫ}fug: &
drukkinn \alst{d}ęila \hld\ skal-at við \alst{d}olg-viðu &
\ind margan stelr \alst{v}ín \alst{v}iti.\eva

\bvb This I counsel thee sixth: although among warriors may grow \\
the ale-speech too awry, \\
drunkenly deal shalt thou not with war-trees \ken{warriors}; \\
wine steals wit from many.\evb\evg

TODO: More stanzas from paper manuscripts.

\sectionline
