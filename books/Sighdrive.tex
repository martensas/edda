\bookStart{The Speeches of Sighdrive}[Sigrdrífumǫ́l]

\begin{flushright}%
Dating \parencite{Sapp2022}: C10th (0.961)

Meter: \Ljodahattr%
\end{flushright}

% Introduction

Many of the verses are quoted in \VolsungaSaga, but notably the two prayer-verses are missing; possibly an instance of Christian censorship. TODO

\sectionline

\bvg
\bva „Lęngi ek svaf, \hld\ lęngi ek sofnuð vas, &
\ind lǫng eru lýða lę́; &
Óðinn því vęldr \hld\ es ęigi mátta’k &
\ind bregða blundstǫfum.“\eva

\bvb [Sighdrive quoth:] “Long I slept, long was I asleep, long are the deceits of men. TODO.”\evb
\evg


\bpg\bpa Sigurðr sęttisk niðr ok spyrr hana nafns. Hón tók þá horn fullt mjaðar ok gaf hǫ́num minnisvęig.\epa

\bpb Siward set himself down, asking for her name. Then she took a horn full of mead, and gave him a mind-draught:\epb\epg


\bvg
\bva Hęill \alst{D}agr, \hld\ hęilir \alst{D}ags synir, &
\ind hęil \alst{N}ǫ́tt ok \alst{n}ipt! &
\alst{Ó}ręiðum \alst{au}gum \hld\ lítið \alst{o}kkr þinig &
\ind ok gefið \alst{s}itjǫndum \alst{s}igr!\eva

\bvb “Hail \inx[P]{Day}! Hail the sons of Day!\footnoteB{TODO. Who?} Hail Night and [her] kinswoman \ken*{= Earth}!\footnoteB{According to \Gylfaginning\ 10 Earth is the daughter of Night and \inx[P]{Aner}.} With unwrathful eyes look ye upon us two, and give the sitting ones \ken*{= us} victory.\evb
\evg


\bvg
\bva Hęilir \alst{ę́}sir, \hld\ hęilar \alst{ǫ́}synjur, &
\ind hęil sjá in \alst{f}jǫlnýta \alst{f}old! &
\alst{M}ál ok \alst{m}anvit \hld\ gefið okkr \alst{m}ę́rum tvęim &
\ind ok \alst{l}ę́knishęndr meðan \alst{l}ifum!\eva

\bvb Hail the \inx[G]{Ease}! Hail the \inx[G]{Ossens}! Hail this bountiful fold \ken{earth}! Speech and \inx[C]{manwit} give ye us renowned two, and \inx[C]{healing-hands}\footnoteB{Hands with the power to heal (perhaps supernaturally). The singular form \emph{lę́knishǫnd} occurs in the semi-Christianized prayer on a c. 1300 stick from Ribe, Denmark (signum DR EM85;493).} while we live.”\evb
\evg


\bpg\bpa Hon nefndisk Sigrdrífa ok var valkyrja. Hon sagði, at tveir konvngar bǫrðusk. Hét annarr Hjalmgunnarr; hann var þá gamall ok inn mesti hermaðr, ok hafði Óðinn hánum sigri heitit.
En \alst{a}nnarr hét \alst{A}gnarr, \hld\ \alst{Au}ðu bróðir // er \alst{v}ę́tr engi \hld\ \alst{v}ildi þiggja.
Sigrdrífa felldi Hjalmgunnar í orrostunni. En Óðinn stakk hana svefnþorni í hefnd þess ok kvað hana aldri skyldu síðan sigr vega í orrostu, ok kvað hana giftask skyldu, „en sagða’k hánum at strengða’k heit þar í mót, at giptask øngom þeim manni er hrę́ðask kynni.“ Hann segir ok biðr hana kenna sér speki ef hon\footnoteA{\emph{hánom} ms.} vissi tíðendi ór ǫllum heimum. Sigrdrífa kvað:\epa

\bpb She called herself Sighdrive and was a walkirrie. She said that two kings fought. One of them was called Helmguther; he was then old and the greatest harrier, and Weden had promised him victory.
But another one was called Eyner, Eade’s brother, who in no way wished to accept.\footnoteB{i.e. ‘wished to lose’ TODO}
Sighdrive felled Helmguther in the battle, but Weden pierced her with the sleeping-thorn as revenge for that, and said that she would never thenceforth win victory in battle, and said that she must marry, “but I told him that I made a vow against that, to marry no man who could be frightened.” He [= Siward] speaks and asks her to teach him wisdom, if she knew any tidings out of all the \inx[C]{Home}[Homes]. Sighdrive quoth:\epb\epg


\bvg
\bva „Bjór fǿri’k þér, \hld\ brynþings apaldr, &
magni blandinn \hld\ ok męgintíri, &
fullr ’s hann ljóða \hld\ ok líknstafa, &
góðra galdra \hld\ ok gamanrúna.\eva

\bvb Beer I bring thee—apple-tree of the byrnie-\inx[C]{Thing} \ken{battle > warrior}!—mixed with might, and might-glory; it is full of \inx[C]{leed}[leeds], and grace-staves, of good \inx[C]{galder}[galders], and pleasure-\inx[C]{rune}[runes].\evb
\evg


\bvg
\bva Sigrúnar skalt kunna, \hld\ ef vilt sigr hafa, &
\ind ok rísta á hjalti hjǫrs, &
sumar á véttrimum, \hld\ sumar á valbǫstum, &
\ind ok nęfna tysvar Tý.\eva

\bvb Victory-runes shalt thou know, if thou wilt have victory, and carve on the hilt of the sword; some on weight-rims;\footnoteB{Unclear. TODO.} some on walbasts\footnoteB{Possibly the sword-pommel, the word also occurs in \HelgakvidaHjorvardssonar\ 9. TODO.}, and twice name \inx[P]{Tew}.\evb
\evg


\bvg
\bva Ǫlrúnar skalt kunna \hld\ ef vilt annars kvę́n &
\ind véli-t þik í tryggð ef trúir; &
á horni skal þér rísta \hld\ ok á handar baki &
\ind ok męrkja á nagli nauð.\eva

\bvb Ale-runes shalt thou know, if thou wilt [have] another man’s wife; she will not betray thee in troth if thou trustest.\evb
\evg


\bvg
\bva Full skal signa \hld\ ok við fári séa &
\ind ok verpa lauki í lǫg; &
\edtext{þá þat vęit’k, \hld\ at þér verðr aldrigi &
męini blandinn mjǫðr.}{\lemma{þá \dots\ mjǫðr}\Bfootnote{so \VolsungaSaga; om. \Regius}}\eva

\bvb The cup shalt thou sign\footnoteB{Dedicate to the gods with a certain formula. TODO.}, and gaze against the danger, and throw in the liquid a leek! Then I know that it will never be, thy mead mixed with harm.\evb
\evg

...


\bvg
\bva Á bjargi stóð \hld\ með Brimis ęggjar, &
\ind hafði sér á hǫfði hjalm; &
þá mę́lti \hld\ Míms hǫfuð &
\ind fróðligt it fyrsta orð, &
\ind ok sagði sanna stafi.\eva

\bvb On the barrow he stood along Brimer’s edges; had on his head a helmet. Then spoke the Mime’s head, learnedly, the first word, and said true staves:\evb
\evg


\bvg
\bva Á skildi kvað ristnar \hld\ þęim’s stęndr fyr skínanda goði, &
á ęyra Árvakrs, \hld\ ok á Alsvinns hófi, &
á því hvéli es snýz \hld\ undir ręið Hrungnis, &
á Slęipnis tǫnnum \hld\ ok á slęða fjǫtrum, &
á bjarnar hrammi \hld\ ok á Braga tungu, &
á ulfs klóm \hld\ ok á arnar nęfi, &
á blóðgum vę́ngjum \hld\ ok á brúar sporði, &
á lausnar lófa \hld\ ok á líknar spori, &
á glęri ok á gulli \hld\ ok á gumna hęillum, &
í víni ok virtri \hld\ ok vilisessi. &
Á Gungnis oddi \hld\ ok á Grana brjósti, &
á nornar nagli \hld\ ok á nęfi uglu;\eva

\bvb On a shield, it declared [there to be] carved [runes]—[on the shield] that stands before the shining god\footnoteB{According to \Grimnismal\ 39 the sun is covered by a shield, protecting the earth from its heat. Without it, the whole world will burn up.} \ken{sun}—[also] on Yorewaker’s ear, on Allswith’s hoof,\footnoteB{The two horses that pull the sun across the heavens; cf. \Grimnismal\ 38.} on that wheel which turns beneath Rungner’s chariot, on Slopner’s teeth, and on the fetters of sleds, on the bear’s paw, and on Bray’s tongue, on the wolf’s claws, and on the eagle’s beak, on bloody wings, and on the bridge’s supports, on the palm of release, and the track of grace, on glass and on gold, and on the luck-charms of men, in wine and beerwort, and on the comfortable seat, on Gungner’s point, and on Grane’s breast, on a norn’s nail, and on an owl’s beak.\evb
\evg


\bvg
\bva Allar vǫ́ru af skafnar, \hld\ þę́r’s vǫ́ru á ristnar, &
\ind ok hvęrfðar við inn hęlga mjǫð &
\ind ok sęndar á víða vega.\eva

\bvb All were shaven off—those that were carved on—and turned into the holy mead, and sent on wide ways:\evb
\evg


\bvg
\bva Þę́r ’ru með ǫ́sum, \hld\ þę́r ’ru með ǫlfum, &
\ind sumar með vísum vǫnum, &
\ind sumar hafa męnskir męnn.\eva

\bvb They are among the Ease, they are among the Elves; some among wise Wanes; some have manly men.\evb
\evg

...

\sectionline

\bvg
\bva „Nú skalt kjósa \hld\ allz þér ’s kostr of boðinn, &
\ind hvassa vápna hlynr, &
sǫgn eða þǫgn \hld\ haf þér sjalfr í hug, &
\ind ǫll eru męin of metin.“\eva

\bvb [Sighdrive quoth:] “Now shalt thou choose, as the choice is offered to thee, maple-tree of sharp weapons \ken{warrior}! Speech or silence have thou in thy own heart; all the harms are measured\footnoteB{i.e. in advance.}.”\evb
\evg


\bvg
\bva „Mun’k-a ek flǿja \hld\ þótt mik fęigan vitir, &
em’k-a ek með blęyði borinn; &
ástrǫ́ð þín \hld\ ek vil ǫll hafa &
svá lęngi sem ek lifi.“\eva

\bvb [Siward quoth:] “I shall not flee, although thou know me to be fey; I am not born with softness.\footnoteB{TODO: Note about this common heroic expression.} Thy loving counsels all will I have, for as long as I may live.”\evb
\evg


\bvg
\bva „Þat rę́ð’k þér it fyrsta \hld\ at við frę́ndr þína &
\ind vammalaust verir; &
síðr þú hęfnir \hld\ þótt þęir sakar gøri; &
\ind þat kveða dauðum duga.“\eva

\bvb [Sighdrive quoth:] “That I counsel thee first: that thou against thy kinsmen defend thyself faultlessly. Late oughtst thou to take revenge, although they incur charges; that they say befits the dead.\evb
\evg


\bvg
\bva Þat rę́ð’k þér annat, \hld\ at ęið né svęrir, &
\ind nema þann ’s saðr séi, &
grimmar \edtrans{simar}{strands}{\Bfootnote{i.e. ‘strands of fate’; cf. \HelgakvidaOne\ 3, where the norns are said to twist such strands. Often emended to \emph{limar} ‘ramifications’ in accordance with \Reginsmal\ 4, where that word is used in basically the same context. Such a scribal confusion is easily understood, since \emph{s} in this position was always spelled with long \emph{ſ} in the old mss. The paraphrase (see other note) is not conclusive, since it replaces this word with \emph{hefnd} ‘revenge’.}} \hld\ ganga at tryggðrofi; &
\ind armr es vára vargr.\eva

\bvb That I counsel thee second: that thou not swear an oath, save for that one which is true. Grim strands come after the troth-breach; wretched is the outlaw of vows.\footnoteB{The punishment is one of torment in the afterlife; see note to \Voluspa\ 39. — The whole verse is paraphrased in \VolsungaSaga\ ch. 21: \emph{Ok sver eigi rangan eið, því at grimm hefnd fylgir griðrofi.} ‘And swear no wrong oath, for grim revenge follows the grith-breach.’}\evb
\evg


\bvg
\bva ...\eva

\bvb That I counsel thee third: that thou on the Thing bandy not with foolish men; for an unwise man often lets be spoken worse words than he ought to know.\evb
\evg


\bvg
\bva ...\eva

\bvb All is missing if thou shut up towards it; then thou seemest born with softness, or truthfully accused. Risky is the verdict of neighbours, unless one gets himself a good one.\evb
\evg


\bvg
\bva ...\eva

\bvb At another day make his breath go away, and thus repay the people for the lie.\evb
\evg
