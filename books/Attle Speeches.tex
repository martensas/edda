\bookStart{Greenlendish Speeches of Attle}[Atlamǫ́l in grǿnlęndsku]
\def\thisBookCode{Atlamal}

\begin{flushright}%
\textbf{Dating} \parencite{Sapp2022}: late C11th (0.472)

\textbf{Meter:} \Malahattr
\end{flushright}%

\section{Introduction}

The \textbf{Greenlendish Speeches of Attle} are only preserved in \Regius.  The poem is composed in \Malahattr\ throughout.  Unlike the preceding \Atlakvida, it seems actually to have been composed in one of the Norse settlements on Greenland, for in st. 18 the poet makes reference to a “white bear”.  The polar bear (\emph{Ursus maritimus}) is indeed found on Greenland, but not on Iceland or the Scandinavian peninsula.  To what inhospitable northern wastes the Norse had brought the legends about Attle (\emph{Attila})!

The language of the poem is noticably younger than its predecessor; most notably the sound change \emph{vr-} > \emph{r-} is consistently applied.

\sectionline

\section{The Greenlendish Speeches of Attle}

\bvg\bva Frétt hęfir \alst{ǫ}ld \alst{ȯ}-fǫ́ \hld\ þá’s \alst{ę}ndr um gǫrðu &
\alst{s}ęggir \alst{s}am-kundu, \hld\ \alst{s}ú vas nýt fę́stum; &
\alst{ǿ}xtu \alst{ęi}n-mę́li, \hld\ \alst{y}ggt vas þęim síðan &
ok it \alst{s}ama \alst{s}onum Gjúka \hld\ es vǫ́ru \alst{s}ann-ráðnir.\eva

\bvb Unfew [many] people have learned when... TODO.\evb\evg

TODO: More stanzas!

\sectionline
