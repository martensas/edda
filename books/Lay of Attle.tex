\bookStart{The Lay of Attle}[Atlakviða]

BPG %TODO prose formatting
Dauði Atla.

Guðrún Gjúkadóttir hefndi brǿðra sinna, svá sem frę́gt er orðit. Hon drap fyrst sonu Atla, en eptir drap hon Atla ok brendi hǫllina ok hirðina alla; um þetta er sjá kviða ort.

The Death of Attle

Guthrun Yivicksdaughter avenged her brothers, as has become famous. She first killed the sons of Attle, and after that she killed Attle, and burned the hall and the whole hird. Regarding that this lay is wrought.

\bvg
\bva Atli sęndi \hld\ ár til Gunnars &
kunnan sęgg at ríða, \hld\ Knéfrøðr vas sá hęitinn; &
at gǫrðum kom hann Gjúka \hld\ ok at Gunnars hǫllu, &
bękkjum aringręypum \hld\ ok at bjóri svǫ́sum.\eva

\bvb Attle sent early to Guther a well-known messenger to ride; Kneefred that one was called. To the estates of Yivick he came, and to the hall of Guther; to the hearth-surrounding benches, and to the lovely beer.\evb
\evg


\bvg
\bva Drukku þar dróttmęgir \hld\ —ęn \edtext{dyljęndr}{\lemma{dyljęndr ‘concealed ones’}\Bfootnote{\textcite{FinnurEdda} reasonably interprets this as referring to Attle’s spies at Guther’s court.}} þǫgðu— &
vín í \edtext{valhǫllu}{\lemma{valhǫllu ‘the walhall’}\Bfootnote{The interpretation of this compound is difficult in context. The first element \emph{val-} could be (1) \emph{valr} ‘falcon’, referring to the aristocratic hunting practice; (2) \emph{valr} ‘\inx[G]{Wales}[Wale]’, cognate with ‘Welsh’ but in ON referring to the French or Romans, stressing the southern location or appearance of the hall; or (3) \emph{valr} ‘(collective) the battle-slain’, foreshadowing the inevitable death (\inx[C]{feyness}) of the \inx[G]{Yivickings}. In this case it is linguistically identical to \inx[L]{Walhall}, Weden’s hall, whither the battle-slain go.}}, \hld\ vręiði sǫ́usk þęir Húna; &
kallaði þá Knéfrøðr \hld\ kaldri rǫddu, &
sęggr inn suðrǿni \hld\ sat hann á bękk hǫ́m:\eva

\bvb There the dright-lads drank—but the concealed ones were silent—wine in the walhall; wary were they of the wrath of the Huns. Then Kneefred, the southern man, called with cold voice; he sat on a high bench:\evb
\evg


\bvg
\bva “Atli mik hingat sęndi \hld\ ríða øręndi, &
mar inum mélgręypa, \hld\ Myrkvið inn ókunna &
at biðja yðr, Gunnarr, \hld\ at it á bękk kǿmið &
með hjǫlmum aringręypum \hld\ at sǿkja hęim Atla.\eva

\bvb “Attle me hither sent to ride an errand, with the bit-champing horse through the uncharted Mirkwood, to ask you, Guther, that ye two on the bench might come, with hearth-surrounding helmets, to seek the home of Attle.\evb
\evg


\bvg
\bva Skjǫldu kneguð þar vęlja \hld\ ok skafna aska, &
hjalma gullroðna \hld\ ok Húna męngi, &
silfrgyllt sǫðulklę́ði, \hld\ sęrki valrauða, &
dafar, darraða, \hld\ drǫsla mélgręypa.\eva

\bvb There ye might choose shields, and smooth ash-spears, helmets gold-reddened, and the multitude of the Huns, silver-gilt saddle-cloth, walred serks, dafs, standards, bit-champing steeds.\evb
\evg


\bvg
\bva Vǫll lézk ykkr ok myndu gefa \hld\ víðrar Gnitahęiðar &
af gęiri gjallanda \hld\ ok af gylltum stǫfnum, &
stórar męiðmar \hld\ ok staði Danpar, &
hrís þat it mę́ra \hld\ es meðr Myrkvið kalla.\eva

\bvb GAGAGA\evb
\evg


\bvg
\bva Hǫfði vatt þá Gunnarr \hld\ ok Hǫgna til sagði: &
Hvat rę́ðr þú okkr, sęggr inn ǿri, \hld\ allz vit slíkt hęyrum? &
Gull vissa ek ekki \hld\ á Gnitahęiði, &
þat es vit ę́ttim-a \hld\ annat slíkt.\eva

\bvb His head turned Guther then, and to Hain said: “What counselest thou we two do, younger man, as we such things hear? I knew of no gold on the Gnitheath, that we did not own as much of.\evb
\evg


\bvg
\bva Sjau ęigu vit salhús \hld\ sverða full, &
hvęrju eru þęira \hld\ hjǫlt ór gulli; &
mínn vęit ek mar bęztan \hld\ ęn mę́ki hvassastan, &
boga bękksǿma \hld\ ęn brynjur ór gulli.\eva

\bvb We own seven hallhouses, filled with swords—on each of them is a golden hilt; I know my horse to be the best, and my sword the sharpest; my bow bench-fit, and my byrnies of gold.\evb
\evg


\bvg
\bva Hjalm ok skjǫld hvítastan, \hld\ kominn ór hǫll Kjárs; &
ęinn es mínn bętri \hld\ ęn sé allra Húna.\eva

\bvb A helmet and the whitest shield, taken out of the hall of Chear; alone is mine better, than [those] of all of the Huns.”\evb
\evg


\bvg
\bva Hvat hyggr þú brúði bęndu \hld\ þá es hón okkr baug sęndi, &
varinn váðum hęiðingja? \hld\ Hykk at hón vǫrnuð byði! &
Hár fann ek hęiðingja \hld\ riðit í hring rauðum; &
ylfskr es vegr okkarr \hld\ at ríða øręndi.\eva

\bvb “What does thou think the bride meant, when she us two an armlet sent, wrapped with the cloth of a heath-dweller \ken{wolf}? I think that she bid us a warning! I found the hair of a heath-dweller wrapped round the red ring; wolven is our way, to ride that errand.”\evb
\evg


\bvg
\bva Niðjar-gi hvǫttu Gunnar \hld\ né náungr annarr, &
rýnęndr né ráðęndr, \hld\ né þęir es ríkir vǫ́ru; &
kvaddi þá Gunnarr \hld\ sęm konungr skyldi, &
mę́rr í mjǫðranni \hld\ af móði stórum:\eva

\bvb No kinsmen urged Guther, nor any other close one, nor counselors nor advisors, nor those who mighty were. Guther then announced—as a king should, renowned in the mead-house—out of great courage:\evb
\evg


\bvg
\bva Rís-tu nú, Fjǫrnir, \hld\ lát-tu á flęt vaða &
gręppa gullskálir \hld\ með gumna hǫndum!\eva

\bvb “Rise now, Ferner; let on the floorboards wade forth the golden bowls of warriors, along the hands of men!\evb
\evg


\bvg
\bva Ulfr mun ráða \hld\ arfi Niflunga, &
gamlir granvarðir, \hld\ ef Gunnars missir, &
birnir blakkfjallir \hld\ bíta þreftǫnnum, &
gamna gręystóði, \hld\ ef Gunnarr né kømr-at.\eva

\bvb The wolf will rule the inheritance of the Niflings: the old grey guardians, if Guther is missing. Bears black-furred bite with wrangling teeth, amusing the pack of bitches, if Guther comes not.”\evb
\evg


\bvg
\bva Lęiddu landrǫgni \hld\ lýðar ónęisir, &
grátęndr, gunnhvatan, \hld\ ór garði Húna; &
þá kvað þat inn ǿri \hld\ ęrfivǫrðr Hǫgna: &
Hęilir farið nú ok horskir \hld\ hvar’s ykkr hugr tęygir!\eva

\bvb GAGAGA\evb
\evg


\bvg
\bva Fetum létu frǿknir \hld\ um fjǫll at þyrja &
marina mélgręypu, \hld\ Myrkvið inn ókunna; &
hristisk ǫll Húnmǫrk \hld\ þar es harðmóðgir fóru, &
vrǫ́ku þęir vannstyggva \hld\ vǫllu algrǿna.\eva

\bvb GAGAGA\evb
\evg


\bvg
\bva Land sǫ́u þęir Atla \hld\ ok liðskjalfar djúpar &
Bikka greppar standa \hld\ á borg inni há &
sal of suðrþjóðum, \hld\ slęginn sessmęiðum, &
bundnum rǫndum, \hld\ blęikum skjǫldum,\eva

\bvb The land of Attle saw they, TODO\evb
\evg


\bvg
\bva dafar, darraða; \hld\ ęn þar drakk Atli &
vín í valhǫllu; \hld\ vęrðir sǫ́tu úti &
at varða þęim Gunnari \hld\ ef þęir hér vitja kǿmi &
með gęiri gjallanda \hld\ at vękja gram hildi.\eva

\bvb but there drank Attle wine in the wale-hall\footnoteB{TODO: this is not Weden’s hall, rather ‘the Roman hall’.} \dots\ \evb
\evg


\bvg
\bva Systir fann þęira snemmst \hld\ at þęir í sal kvǫ́mu, &
brǿðr hęnnar báðir, \hld\ bjóri var hón lítt drukkin: &
Ráðinn ert-u nú, Gunnarr, \hld\ hvat munt-u, ríkr, vinna &
við Húna harmbrǫgðum? \hld\ Hǫll gakk þú ór snemma!\eva

\bvb Their sister found earliest they they had come into the hall, both of her brothers—on beer was she lightly drunk—“Betrayed art thou now, Guther; why wilt thou, mighty one, struggle against Hunnish harm-tricks? Go early out of the hall!\footnoteB{Before anything evil might happen.}”\evb
\evg


\bvg
\bva Bętr hęfðir þú, bróðir, \hld\ at þú í brynju fǿrir, &
sęm hjǫlmum aringręypum \hld\ at séa hęim Atla; &
sę́tir þú í sǫðlum \hld\ sólhęiða daga, &
nái nauðfǫlva \hld\ létir nornir gráta.\eva

\bvb Better hadst thou, brother, if thou in byrnie travelled, and with hearth-surrounding helmets, to see the home of Attle.\evb
\evg


\bvg
\bva Húna skjaldmęyjar \hld\ hęrfi kanna &
ęn Atla sjalfan \hld\ létir þú í ormgarð koma; &
nú es sá ormgarðr \hld\ ykkr of folginn.\eva

\bvb GAGAGA\evb
\evg


\bvg
\bva Sęinað es nú, systir, \hld\ at samna Niflungum, &
langt es at lęita \hld\ lýða sinnis til, &
of rosmufjǫll Rínar, \hld\ rekka ónęissa.\eva

\bvb GAGAGA\evb
\evg


\bvg
\bva Fengu þęir Gunnar \hld\ ok í fjǫtur sęttu, &
vinir Borgunda, \hld\ ok bundu fastla; &
sjau hjó Hǫgni \hld\ sverði hvǫssu &
ęn inum átta hratt hann \hld\ í ęld hęitan.\eva

\bvb Caught they Guther, and in fetters set him—the friends of the Burgends—and bound them tightly. Seven Hain hewed down with sharp sword, and the eighth one threw he into the hot fire.\evb
\evg


\bvg
\bva \edtext{Svá skal frǿkn \hld\ fjándum vęrjask;}{\lemma{Svá \dots\ vęrjask}\Bfootnote{Line moved from the last verse to this one since it seems to connect semantically with the immediately following line, and also creates a regular line distribution of 4-4 instead of 5-3.}} &
Hǫgni varði \hld\ hęndr Gunnars. &
frǫ́gu frǿknan \hld\ ef fjǫr vildi &
Gotna þjóðann \hld\ gulli kaupa.\eva

\bvb Thus shall the bold against fiends ward himself; Hain warded the hands of Guther. They asked the bold one if to buy he wished—the ruler of the Gots—his life with gold.\footnoteB{The Huns ask Guther (it is clear that “ruler of the Gots” refers to him, cf. 1, 3, 10) if he wishes to ransom Hain. He instead responds with the following:}\evb
\evg


\bvg {\small [Guther quoth:]}
\bva “Hjarta skal mér Hǫgna \hld\ í hęndi liggja &
blóðugt, ór brjósti \hld\ skorit baldriða, &
saxi slíðrbęitu, \hld\ syni þjóðans.”\eva

\bvb “The heart of Hain shall lie me in the hands: bloody from the breast—cut from the bold rider with a slide-biting sax\footnoteB{i.e. a short-sword with a blade so sharp that it draws blood when one slides the finger across it.}—of the son of the sovereign.”\evb
\evg


\bvg
\bva Skǫ́ru þęir hjarta \hld\ Hjalla ór brjósti &
blóðugt ok á bjóð lǫgðu \hld\ ok bǫ́ru þat fyr Gunnar.\eva

\bvb They cut the heart of Helle out of the breast; bloody on a platter they laid it, and carried it before Guther.\evb
\evg


\bvg
\bva Þá kvað þat Gunnarr, \hld\ gumna dróttinn: &
Hér hęfi ek hjarta \hld\ Hjalla ins blauða, &
ólíkt hjarta \hld\ Hǫgna ins frǿkna, &
es mjǫk bifask \hld\ es á bjóði liggr; &
bifðisk hǫlfu męirr \hld\ es í brjósti lá!\eva

\bvb Then quoth that Guther, the lord of men: “Here have I the heart of Helle the soft—unlike the heart of Hain the bold!—which much trembles, when on the platter it lies; it trembled twice as much, when in the breast it lay.”\evb
\evg


\bvg
\bva Hló þá Hǫgni \hld\ es til hjarta skǫ́ru &
kvikvan kumblasmið \hld\ kløkkva hann sízt hugði; &
blóðugt þat á bjóð lǫgðu \hld\ ok bǫ́ru fyr Gunnar.\eva

\bvb Hain laughed then, when to the heart they cut on the living wound-smith \ken{warrior}; he thought least of sobbing. Bloody on a platter they laid it, and carried it before Guther.\evb
\evg


\bvg
\bva Mę́rr kvað þat Gunnarr, \hld\ Gęir-Niflungr: &
Hér hęfi ek hjarta \hld\ Hǫgna ins frǿkna, &
ólíkt hjarta \hld\ Hjalla ins blauða, &
es lítt bifask \hld\ es á bjóði liggr; &
bifðisk svági mjǫk \hld\ þá’s í brjósti lá!\eva

\bvb Renowned quoth that Guther, the Gore-Nifling: “Here have I the heart of Hain the bold—unlike the heart of Helle the soft!—which little trembles, when on the platter it lies; it trembled not as much, when in the breast it lay.\evb
\evg


\bvg
\bva Svá skaltu, Atli, \hld\ augum fjarri &
sęm munt \hld\ męnjum verða; &
es und ęinum mér \hld\ ǫll of folgin &
hodd Niflunga: \hld\ Lifir-a nú Hǫgni!\eva

\bvb Thus shalt thou, Attle, be as far from the eyes, as thou wilt from the neck-rings. ’Tis by me alone all concealed, the hoard of the Niflings—now Hain lives not!\evb
\evg


\bvg
\bva Ęy vas mér týja \hld\ meðan vit tvęir lifðum, &
nú es mér ęngi \hld\ es ęinn lifi’k; &
Rín skal ráða \hld\ rógmalmi skatna, &
svinn, ǫ́skunna \hld\ arfi Niflunga.\eva

\bvb I was ever in doubt when we two lived; now I am not when alone I live. The Rhine shall rule the strife-ore of princes \ken{gold}, swift, the os-born inheritance of the Niflings.\evb
\evg


\bvg
\bva Í veltanda vatni \hld\ lýsask valbaugar &
hęldr an á hǫndum gull \hld\ skíni Húna bǫrnum.\eva

\bvb In tumbling water the Welsh bighs gleam, rather than gold might shine on the hands of the children of Huns.”\evb
\evg

...

\bvg
\bva Ęldi gaf hón alla \hld\ es inni vǫ́ru &
ok frá morði þęira Gunnars \hld\ komnir vǫ́ru ór Myrkhęimi; &
forn timbr fellu, \hld\ fjarghús ruku, &
bǿr Buðlunga, \hld\ brunnu ok skjaldmęyjar, &
inni aldrstamar, \hld\ hnigu í ęld hęitan.\eva

\bvb To the fire she gave all those who were inside, who from their murder of Guther were come out of Mirkham. Ancient timbers fell, great houses smoked—the settlement of the Buthlungs—burned the shield–maidens likewise; inside aged trunks bowed into hot fire.\evb
\evg


\bvg
\bva Fullrǿtt’s umb þetta; \hld\ fęrr ęngi svá síðan &
brúðr í brynju \hld\ brǿðra at hęfna; &
hón hęfir þriggja \hld\ þjóðkonunga &
banorð borið, \hld\ bjǫrt, áðr sylti.\eva

\bvb ’Tis fully told of this; none hence fares so, a bride in byrnie, her brothers to avenge. She has of three great kings borne the bane-word,\footnoteB{i.e. ‘She has slain three great kings.’ This expression and its Germanic and Indo-European relatives is discussed in detail in \textcite{Watkins1995}[417--422].} bright woman, before she may die.\evb
\evg


\bvg
\bva Enn segir gleggra í Atlamálum inum grǿnlenskum.\eva

\bvb Yet this is told more clearly in the Greenlendish Speeches of Attle.\evb
\evg
