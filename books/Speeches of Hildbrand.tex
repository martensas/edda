\book{The Speeches of Hildbrand}\bookStart

% Introduction

{\small For the text of original poem, I do not present the manuscript text, but rather a standardized text of my own. I have however aimed to generally follow the dialect of the manuscript, rather than present a standardized Old High German or Old Saxon. The rules of normalization have been as follows:
Vowels:
> Ms. \emph{ae}, \emph{ei} and \emph{e}, where etymologically from \emph{ai}, have been normalized as \emph{ei}.
> Ms. \emph{o} and \emph{ao}, where etymologically from \emph{au}, have been normalized as \emph{ao}. This may be somewhat controversial.
> \emph{ostar}, \emph{Otachre} > \emph{aostar}, \emph{Aotachre}).
> Ms. \emph{uo} and \emph{o}, where etymologically from long \emph{ō}, have been normalized as \emph{ō}.
Consonants:
> Ms. \emph{r} and \emph{w}, where etymologically from \emph{hw} and \emph{hr}, have been thus normalized. That this was the case in the original poem is obvious; such words never alliterate with \emph{w} or \emph{r}, but only with \emph{r}, as can be most definitively seen in lines 56 (ms.: \alst{h}eremo ... \alst{h}rusti) and 66 (ms.: \alst{h}ewun \alst{h}armlicco \alst{h}uitte). If this were not enough, the retention in the ms. of the \emph{h} at previously given places is yet further support.
> Ms. \emph{tt}, where etymologically from \emph{t}, has been thus normalized.
> Ms. \emph{ƿ} (wynn), \emph{u} and \emph{uu}, where representing \emph{w}, have been thus normalized.

The pronoun which exclusively appears in the ms. as \emph{her} ‘he’ has been so kept, rather than normalized to the standard OHG \emph{er}.
The punctuation of the original (entirely consisting of interpuncts) has not been retained.}



\bvg
\bva[0] Ik gihōrta dat seggen &
dat sih urhettun \hld einon mōtīn &
Hiltibrant enti Hadubrant \hld untar heriun tweim &
sunufatarungo \hld iro saro rihtun &
garutun \edtext{sie}{\Afootnote{se \HildMS}} iro gūdhamun \hld gurtun sih iro swert ana &
helidos ubar \edtext{hringa}{\Afootnote{ringa \HildMS}} \hld dō sie to dero hiltiu ritun\eva

\bvb[0] I heard it said, that two contenders alone did meet: Hildbrand and Hathbrand, under two hosts. Son and father ordered their armour, readied their war-cloth, girded their swords on, the heroes over the mail, when to that battle they rode.\evb
\evg


\bvg
\bva[0] Hiltibrant gimahalta Heribrantes sunu \hld her was hērōro man &
ferahes frōtōro \hld her frāgēn gistōnt &
fōhēm wortum \hld \edtext{hwer}{\Afootnote{wer \HildMS}} sin fater wāri &
fireo in folche \hld {[...]} &
{[...]} \hld eddo \edtext{hwelīhhes}{\Afootnote{welihhes \HildMS}} cnōsles dū sīs &
ibu dū mī ēnan sagēs \hld ik mī de odre wēt &
chind in \edtext{chunincrīche}{\Afootnote{chunnincriche \HildMS}} \hld chūd ist mīn al irmindeot\eva

\bvb[0] Hildbrand spoke, Harbrand's son — he was the hoarier man, more learned in life, — he began to ask with few words, who his father might be, of men in the folk, [...] “or of which lineage thou be; if thou me one say, I the others will know; child, in the kingdom, known to me are all great men.”\evb
\evg


\bvg
\bva[0] Hadubrant gimahalta \hld Hiltibrantes sunu &
dat sagetun mī ūsere liuti &
alte enti frōte \hld dea ērhina wārun &
dat Hiltibrant hēti min fater \hld ih heitu hadubrant &
forn her aostar giweit \hld flaoh her Aotachres nīd &
hina miti Deotrihhe \hld enti sīnero degano filu &
her furlēt in lante \hld luttila sitten &
brūt in būre \hld barn unwahsan &
arbeolaosa \hld her reit aostar hina &
des sid Deotrihhe \hld darba gistōntum &
\edtext{fateres}{\Afootnote{fatereres \HildMS}} mīnes \hld dat was sō friuntlaos man &
her was Aotachre \hld ummet tirri &
degano dechisto \hld unti \edtext{Deotrihhe}{\Afootnote{\emph{add.} darba gistontun \HildMS}} &
her was eo folches at ente \hld imo was eo \edtext{fehta}{\Afootnote{peheta \HildMS}} ti leob &
chūd was her \hld chōnēm mannum &
ni wāniu ih iu līb habbe\eva

\bvb[0] Hathbrand spoke, Hildbrand's son: “It told me our people, the old and learned, those who earlier lived, that Hildbrand was called my father — I am called Hathbrand, — he previously hurried east; he fled Edwaker's hate, thither with Thedrich, and his multitude of thanes. He left in the land a little one to stay, a bride in the bower, a bairn ungrown, without inheritance; he rode east thither, as Thedrich was in great need of my father — that was such a friendless man. He was to Edwaker exceptionally hostile, the dearest of thanes under Thedrich. He was ever at the front of the troop; ever did the fight gladden him; known was he among keen men. — I ween not that he have life.”\evb
\evg


\bvg
\bva[0] weitu irmingot {\small [quad hiltibrant]} \hld obana ab hebane &
dat dū neo dana halt mit sus sibban man &
dinc ni gileitōs &
want her dō ar arme \hld wuntane baoga &
cheisuringu gitān \hld so imo sie der chuning gab &
huneo truhtin \hld dat ih dir it nū bī huldī gibu\eva

\bvb[0] “I call on God as witness, [quoth Hildbrand], above in heaven, that thou never with such a close man once more lead dispute.” Unwound he then from his arm some twisted bighs, made from imperial coin, which the king once gave him, the lord of the Huns: — “This I now give thee as pledge.”\evb
\evg


\bvg
\bva[0] Hadubrant gimahalta \hld Hiltibrantes sunu &
mit gēru scal man \hld geba infāhan &
ort widar orte \hld [...] &
dū bist dir altēr hun \hld ummet spāhēr &
spenis mih mit dīnem wortum \hld wili mih dinu speru werpan &
bist alsō gialtēt man \hld sō dū ēwīn inwit fōrtōs &
dat sagetun mi \hld sēolīdante &
westar ubar wentilsēo \hld dat man wīc furnam &
tōt ist Hiltibrant \hld Heribrantes sunu\eva

\bvb[0] Hathbrand spoke, Hildbrand's son: “With spear shall one earn gifts, point against point! Thou art, old Hun, exceptionally clever; thou lurest me with thy words, wilt thou at me hurl thy spear! Thou art thus old, though thou ever deceit hast worked. — It told me seafarers, heading west o’er the Wendle-sea <= Mediterranean>, that war took that man: — dead is Hildbrand, Harbrand's son!”\evb
\evg


\bvg
\bva[0] Hiltibrant gimahalta \hld Heribrantes sunu &
wela gisihu ih in dīnēm hrustim &
dat dū habēs heime \hld hērron gōten &
dat dū noh bī desemo rīche \hld reccheo ni wurti\eva

\bvb[0] Hildbrand spoke, Harbrand's son: “I see well on thy equipment, that thou hast a good lord at home, that thou yet in his reign art not become an exile.\evb
\evg


\bvg
\bva[0] welaga nu waltant got {\small [quad hiltibrant]} \hld weiwurt skihit &
ih wallōta sumaro enti wintro \hld sehstic ur lante &
dar man mih eo scerita \hld in folc sceotantero &
sō man mir at burc einīgeru \hld banun ni gifasta &
nu scal mih swāsat chind \hld swertu haowan &
bretōn mit sīnu billiu \hld eddo ih imo ti banin werdan &
doh maht dū nū aodlīhho \hld ibu dir dīn ellen taoc &
in sus hēremo man \hld hrusti giwinnan &
raoba birahanen \hld ibu du dar einīg reht habēs\eva

\bvb[0] Well now, wielding God, [quoth Hildbrand], woeful Weird passes. I wallowed for summers and winters sixty, out of the land, where one ever placed me in the troop of shooters; thus one at no fortress my bane did inflict. Now shall my own child hew at me with sword; beat down with blade, or I become his bane; — yet canst thou now easily, if thy courage avail thee, from such a hoary man win the equipment, bear away the booty, if thou thereto have any right.\evb
\evg


\bvg
\bva[0] der sī doh nu argōsto {\small [quad hiltibrant]} aostarliuto &
der dir nū wīges warne \hld nū dih et sō wel lustit &
gudea gimeinun \hld niuse der mōti &
hwedar sih \edtext{hiutu dēro}{\Afootnote{dēro hiuti \HildMS}} hregilo \hld hrōmen mōti &
eddo desero brunnōno \hld beidero waltan\eva

\bvb[0] Yet now he may be the weakest, [quoth Hildbrand], of the eastern peoples, who would refuse thee the fight, when thou so greatly cravest to struggle together. Try he who might, which one today of his arms may boast, or both of these byrnies wield!”\evb
\evg


\bvg
\bva[0] dō lietun sie aerist \hld askim scrītan &
scarpēn scūrim \hld dat in dem sciltim stōnt &
dō stōptun tosamane \hld staimbort hlūdun &
hewun harmlīcco \hld hwīte scilti &
unti imo iro lintūn \hld luttilo wurtun &
giwigan miti wābnum \hld [...]\eva

\bvb[0] Then they first let their ash-spears glide, in a harsh torrent, that they stuck in the shields. Then charged they into each other — the war-boards [SHIELDS] resounded — struck they bitterly the white shields, until their linden-planks [SHIELDS] became little, worn down by the weapons, [...]\evb
\evg
