\book{The Speeches of Hildbrand.}\bookStart

% Introduction

{\small For the text of original poem I generally present the manuscript text. I found it very difficult to produce a normalization without too heavily distorting the received text, being as it is, a blend of several dialects. I have, however, added acute accents to signify long vowels, capitalized proper names, consistently replaced \emph{ƿ} (wynn) and \emph{uu} with \emph{w}, and made minor corrections where the manuscript is clearly in error—these are noted in the critical apparatus. The punctuation of the original, entirely consisting of interpuncts, at times representing line breaks and caesurae and at others sporadically placed, has not been retained. The words \emph{quad Hiltibrant} “Hildbrand quoth” (found in the ms. at the caesurae of ll, 30, 49, and 58), without doubt late hypermetrical additions, have been removed from the German text, but are presented in small font in the English translation.}

\vspace{3em}

\bvg
\bva[0]Ik gihórta dat seggen &
dat sih urhettun \hld aenon muotín &
Hiltibrant enti Hadubrant \hld untar heriun twém &
sunufatarungo \hld iro saro rihtun &
garutun se iro gúdhamun \hld gurtun sih iro swert ana &
helidos ubar \edtext{hringa}{\Afootnote{ringa \HildMS}} \hld dó sie to dero hiltiu ritun\eva

\bvb[0] I heard it said, that two contenders alone did meet: Hildbrand and Hathbrand, under two hosts. Son and father ordered their armour, readied their war-cloth, girded their swords on, the heroes over the mail, when to that battle they rode.\evb
\evg


\bvg\setlinenum{6}
\bva[0]Hiltibrant \edtext{gimahalta}{\Afootnote{\emph{add.} heribrantes sunu “Harbrand’s son” \HildMS}} \hld her was héróro man &
ferahes frótóro \hld her frágén gistuont &
fóhém wortum \hld \edtext{hwer}{\Afootnote{wer \HildMS}} sín fater wári &
fireo in folche \hld {[...]} &
{[...]} \hld eddo \edtext{hwelíhhes}{\Afootnote{welihhes \HildMS}} cnuosles dú sís &
ibu dú mí énan sagés \hld ik mí de odre wét &
chind in \edtext{chunincríche}{\Afootnote{chunnincriche \HildMS}} \hld chúd ist mín al irmindeot\eva

\bvb[0] Hildbrand spoke — he was the hoarier man, more learned in life — he began to ask, with few words, who his father might be, of men in the troop, [...] “or of which lineage thou be; if thou me one say, I the others will know; child, in the kingdom, known to me are all great men.”\evb
\evg


\bvg\setlinenum{13}
\bva[0]Hadubrant gimahalta \hld Hiltibrantes sunu &
\edtext{dat sagetun mí \hld úsere liuti}{\lemma{dat ... liuti}\Bfootnote{this l. breaks no rhythmic rules (cf. l. 42), but the needed alliteration is missing.}} &
alte anti fróte \hld dea érhina wárun &
dat Hiltibrant haetti mín fater \hld ih heittu Hadubrant &
forn her óstar \edtext{giweit}{\Afootnote{gihueit \HildMS}} \hld flóh her Ótachres níd &
hina miti Theotríhhe \hld enti sínero degano filu &
her furlaet in lante \hld luttila sitten &
\edtext{brút}{\Afootnote{prut \HildMS}} in búre \hld barn unwahsan &
arbeolaosa \hld \edtext{her raet}{\Afootnote{heraet \HildMS}} óstar hina &
det síd Detríhhe \hld darba gistuontum &
\edtext{fateres}{\Afootnote{fatereres \HildMS}} mínes \hld dat was só friuntlaos man &
her was Ótachre \hld ummet tirri &
degano dechisto \hld unti \edtext{Deotríchhe}{\Afootnote{\emph{add.} darba gistontun \HildMS}} &
her was eo folches at ente \hld imo was eo \edtext{fehta}{\Afootnote{peheta \HildMS}} ti leop &
chúd was her \hld \edtext{chóném}{\Afootnote{chonnem \HildMS}} mannum &
ni wániu ih iu líb habbe\eva

\bvb[0] Hathbrand spoke, Hildbrand’s son: “It told me our people, the old and learned, those who earlier lived, that Hildbrand was called my father — I am called Hathbrand. Long ago he hurried east — he fled Edwaker’s hate — thither with Thedrich, and his great many thanes. He left in the land a little one to stay, a bride in the bower, a bairn ungrown, without inheritance; he rode east thither, as Thedrich was in great need of my father; — that was so friendless a man. He was to Edwaker exceptionally hostile, the dearest of thanes under Thedrich. He was ever at the front of the troop, ever did the fight gladden him, known was he among keen men; I ween not that he have life.”\evb
\evg


\bvg\setlinenum{29}
\bva[0] wettu irmingot \hld obana ab \edtext{hebane}{\Afootnote{heuane \HildMS}} &
dat dú neo dana halt mit sus sippan man &
dinc ni gileitós &
want her dó ar arme \hld wuntane bauga &
cheisuringu gitán \hld so imo sie der chuning gap &
huneo truhtin \hld dat ih dir it nú bí huldí gibu\eva

\bvb[0] “I call on Ermin-god as witness, {\small [quoth Hildbrand]}, above in heaven, that thou never with such a close man once more lead dispute.” Unwound he then from his arm some twisted bighs\footnoteA{Armlets used as currency during the Migration Period; ON \emph{baugr}, OE \emph{béag}. — The giving of rings and armlets in exchange for loyalty was common across all of Germanic Europe, as seen in the many ruler-kennings of the type “breaker of rings” (like \emph{béaga brytta} “the breaker of bighs” \Beowulf\ ll. 35, 352, 1487.) This is also connected with the oath-ring, and the famous ring-swords. TODO? reference some literature on this.}, made from imperial coin, which the king once gave him, the lord of the Huns—“This I now give thee as pledge.”\evb
\evg


\bvg\setlinenum{35}
\bva[0]Hadubrant gimahalta \hld Hiltibrantes sunu &
mit géru scal man \hld geba infáhan &
ort widar orte \hld [...] &
dú bist dir altér hun \hld ummet spáhér &
spenis mih mit díném wortun \hld wili mih dínu speru werpan &
\edtext{bist}{\Afootnote{pist \HildMS}} alsó gialtét man \hld só dú éwín inwit fórtós &
dat sagetun mí \hld séolídante &
westar ubar Wentilséo \hld dat man wíc furnam &
tót ist Hiltibrant \hld Heribrantes suno\eva

\bvb[0] Hathbrand spoke, Hildbrand’s son: “With spear shall one earn gifts, point against point! Thou art, old Hun, exceptionally clever; thou lurest me with thy words, wilt thou at me thy spear hurl! Thou art thus old, though thou ever deceit didst work. — It told me seafarers, heading west o’er the Wendle-sea\footnoteA{The Mediterranean, referring to the Vandals in North Africa.}, that war took that man: — dead is Hildbrand, Harbrand’s son!”\evb
\evg


\bvg\setlinenum{44}
\bva[0]Hiltibrant gimahalta \hld Heribrantes suno &
wela gisihu ih \hld in díném hrustim &
dat dú habés héme \hld hérron góten &
dat dú noh bí desemo ríche \hld reccheo ni wurti\eva

\bvb[0] Hildbrand spoke, Harbrand’s son: “I see well on thy equipment, that thou hast a good lord at home, that thou still in this reign didst not become an exile.”\evb
\evg


\bvg\setlinenum{48}
\bva[0] welaga nú waltant got \hld wéwurt skihit &
ih wallóta sumaro enti wintro \hld sehstic ur lante &
dar man mih eo scerita \hld in folc sceotantero &
só man mir at burc énigeru \hld banun ni gifasta &
nú scal mih swásat chind \hld swertu hauwan &
bretón mit sínu billiu \hld eddo ih imo ti banin werdan &
doh maht dú nú aodlíhho \hld ibu dir dín ellen taoc &
in sus héremo man \hld hrusti giwinnan &
rauba \edtext{birahanen}{\Afootnote{bihrahanen \HildMS}} \hld ibu dú dar éníg reht habés\eva

\bvb[0] “Well now, wielding god, {\small [quoth Hildbrand]}, woeful Weird\footnoteA{The personification of fate, in this case most likely just a noun. OE \emph{Wyrd} (\Beowulf\ 455: \emph{Gǽð á Wyrd swá hío scel} “Ever goes Weird as she must”), ON \emph{Urðr} ‘one of the norns’.} comes to pass. I wallowed for summers and winters sixty out of the land, where one ever set me in the troop of shooters; thus one at no fortress my bane did inflict. Now shall my own child hew at me with sword; beat down with his blade, or I his bane become. Yet canst thou now easily, if thy courage avail thee, from such a hoary man win the equipment, bear away the booty, if thou thereto have any right.”\evb
\evg


\bvg\setlinenum{57}
\bva[0] der sí doh nú argósto \hld óstarliuto &
der dir nú wíges warne \hld nú dih es só wel lustit &
gúdea gimeinun \hld niuse de motti &
\edtext{hwedar}{\Afootnote{werdar \HildMS}} sih \edtext{hiutu déro}{\Afootnote{dero hiutu \HildMS}} hregilo \hld \edtext{hruomen}{\Afootnote{hrumen \HildMS}} muotti &
\edtext{eddo}{\Afootnote{erdo \HildMS}} desero brunnóno \hld bédero waltan\eva

\bvb[0] “He be now the weakest {\small [quoth Hildbrand]} of the eastern peoples, who refuse thee the fight, when thou so greatly cravest to struggle together; — try he who might, which of us today of these garments may boast, or both of these byrnies wield!”\evb
\evg


\bvg\setlinenum{62}
\bva[0]dó lettun se aerist \hld asckim scrítan &
scarpén scúrim \hld dat in dem sciltim stónt &
dó stóptun tosamane \hld staimbort \edtext{hlúdun}{\Afootnote{chludun \HildMS}} &
hewun harmlícco \hld hwítte scilti &
unti imo iro lintún \hld luttilo wurtun &
giwigan miti wábnum \hld [...]\eva

\bvb[0] Then let they first their ash-spears glide, in harsh torrents, that in the shields they stuck. Then charged they into each other — the war-boards \ken{shields} resounded — struck they bitterly the white shields, until their lindens \ken{shields} became little, worn down by the weapons, [...]\evb
\evg
