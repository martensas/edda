\bookStart{Deer}[Deor]

\begin{flushright}%
\textbf{Dating:} TODO

\textbf{Meter:} \Fornyrdislag%para
\end{flushright}%

\section{Introduction}

A dirge from the Exeter Book.  The poem briefly summarizes the tragic lives of five figures from Germanic heroic legend, each ending with the refrain \emph{Þæs ofer-eode \hld\ þisses swá mæg} ‘That passed over; this may likewise.’  After this the poet reflects on fate, and finally tells his own story as an outcast.

The five legends mentioned are:

\begin{enumerate}
  \item Wayland the Smith, who was captured by the tyrant Nithad and forced to make jewelry for him and his family. He took revenge by raping
  \item Nithad’s daughter, Beadhild. The child born from this act was Woody (OE \emph{Wudga}), an obscure hero.
  \item Mathild, the protagonist of a poorly attested love tragedy.
  \item Thedric the Great, who ruled over the Gots.
  \item Erminric, who succeeded Thedric, and was eventually slain.
\end{enumerate}

The name \emph{Déor}, first revealed in line 37, is the ancestor of modern English “deer”, and it can mean this in Old English as well, but it can also betoken ‘beast, animal’ more generally.  It is not otherwise known as a personal name and is clearly fictional; we may perhaps compare \Fafnismal\ 2, where the young hero \inx[P]{Siward} calls himself \emph{gǫfugt dýr} ‘noble beast/deer’.

\section{Deer}

\bvg\bva[1]%
\alst{W}elund him be \alst{w}urman \hld\ \alst{w}ræces cunnade, &
\alst{â}n-hýdig \alst{eo}rl \hld\ \alst{ea}rfoþa dréag, &
hæfde him tó ge·\alst{s}iþþe \hld\ \alst{s}orge ǫnd lǫngaþ, &
\alst{w}inter-cealde \alst{w}ræce; \hld\ \alst{w}éan oft ǫn·fǫnd, &
siþþan hine \alst{N}íðhad ǫn \hld\ \alst{n}éde lęgde, &
\alst{s}wǫncre \alst{s}eono-bende \hld\ ǫn \alst{s}yllan mǫnn. &
\alst{Þ}æs ofer-eode, \hld\ \alst{þ}isses swá mæg!\eva

\bvb \inx[P]{Wayland}[{\huge W}ayland] with worms his exile experienced; \\
the one-minded earl hardship did suffer; \\
had him for companions sorrow and longing, \\
winter-cold exile; woes he often found, \\
since \inx[P]{Nithad} on him fetters did lay; \\
heavy sinew-bonds on the better man. \\
\emph{That} passed over; \emph{this} may likewise.\evb\evg


\bvg\bva[2][8]%
\alst{B}eadohilde ne wæs \hld\ hyre \alst{b}róþra déaþ &
on \alst{s}efan swá \alst{s}âr \hld\ swá hyre \alst{s}ylfre þing, &
þæt heo \alst{g}earo-líce \hld\ on·\alst{g}ieten hæfde &
þæt heo \alst{é}acen wæs; \hld\ \alst{æ̂}fre ne meahte &
\alst{þ}riste ge·\alst{þ}ęncan, \hld\ hú ymb \alst{þ}æt sceolde. &
\alst{Þ}æs ofer-eode, \hld\ \alst{þ}isses swá mæg!\eva

\bvb For \inx[P]{Beadhild} was not her brothers’ deaths \\
on her heart so sore, as her own thing, \\
that she clearly had understood, \\
that she was pregnant.  Never could she \\
bravely think out what about \emph{that} she should do. \\
\emph{That} passed over; \emph{this} may likewise.\evb\evg


\bvg\bva[3][14]%
Wé þæt \alst{M}æðhilde \hld\ \alst{m}ǫnge ge·frugnon &
wurdon \alst{g}rund-léase \hld\ \alst{G}eates frige, &
þæt hi seo \alst{s}org-lufu \hld\ \alst{s}lǽp ealle bi·nǫm. &
\alst{Þ}æs ofer-eode, \hld\ \alst{þ}isses swá mæg!\eva

\bvb That for Mathild many, we have heard, \\
bottomless [troubles] arose, for Geat’s beloved, \\
that the sorrowful love her of sleep all deprived. \\
\emph{That} passed over; \emph{this} may likewise.\evb\evg


\bvg\bva[4][18]%
\alst{Þ}eodríc áhte \hld\ \alst{þ}rítig wintra &
\alst{M}ǽringa burg; \hld\ þæt wæs \alst{m}ǫnegum cu̇þ. &
\alst{Þ}æs ofer-eode, \hld\ \alst{þ}isses swá mæg!\eva

\bvb \inx[P]{Thedric} owned for thirty winters \\
the fort of the Meerings; that was to many known. \\
\emph{That} passed over; \emph{this} may likewise.\evb\evg


\bvg\bva[5][21]%
Wé ge·\alst{a}scodan \hld\ \alst{Eo}rmanríces &
\alst{w}ylfenne ge·þȯht; \hld\ áhte \alst{w}íde folc &
\alst{G}otena ríces. \hld\ \edtrans{Þæt wæs \alst{g}rim cyning!}{that was a grim king!}{\Bfootnote{Formulaic; cf. \Beowulf\ 11b: \emph{Þæt wæs gód cyning!} ‘That was a good king!’}} &
\alst{S}æt \alst{s}ęcg mǫnig \hld\ \alst{s}orgum ge·bunden, &
\alst{w}éan on \alst{w}énan, \hld\ \alst{w}ýscte ge·neahhe &
þæt þæs \alst{c}yne-ríces \hld\ ofer-\alst{c}umen wǽre. &
\alst{Þ}æs ofer-eode, \hld\ \alst{þ}isses swá mæg!\eva

\bvb We have learned of \inx[P]{Erminric}’s \\
wolven nature; he wielded widely the folk \\
of the realm of the Gots—that was a grim king! \\
Sat many a man by sorrows bound, \\
woes in his thoughts; wished aplenty \\
that the kingdom might be overcome. \\
\emph{That} passed over; \emph{this} may likewise.\evb\evg


\bvg\bva[6][28]%
\alst{S}iteð \alst{s}org-céarig, \hld\ \alst{s}ǽlum bi·dæ̂led, &
on \alst{s}efan \alst{s}weorceð, \hld\ \alst{s}ylfum þinceð &
þæt sý \alst{ę}nde-léas \hld\ \alst{ea}rfoda dæ̂l. &
Mæg \alst{þ}ǫnne ge·\alst{þ}ęncan, \hld\ þæt geond \alst{þ}ás woruld &
\alst{w}itig dryhten \hld\ \alst{w}ęndeþ ge·neahhe, &
\alst{eo}rle mǫnegum \hld\ \alst{â}re ge·sceawað, &
\alst{w}ís-licne blǽd, \hld\ sumum \alst{w}éana dæ̂l.\eva

\bvb One sits grieved with sorrow, of blessings bereft; \\
his heart darkens; to himself he thinks \\
that endless must be his share of hardships. \\
He may then think that throughout this world \\
the Wise Lord turns coat aplenty. \\
To many an earl honour he shows, \\
sure success—to another a share of woes.\evb\evg


\bvg\bva[7][35]%
Þæt ic bi mé \alst{s}ylfum \hld\ \alst{s}ęcgan wille, &
þæt ic \alst{h}wile wæs \hld\ \alst{H}eodeninga scóp, &
\alst{d}ryhtne \alst{d}ýre— \hld\ mé wæs \alst{D}eor nǫma. &
Áhte ic \alst{f}ela wintra \hld\ \alst{f}olgað tilne, &
\alst{h}oldne \alst{h}laford, \hld\ oþþæt \alst{H}eorrenda nú, &
\alst{l}éoð-cræftig mǫnn \hld\ \alst{l}ǫnd-ryht ge·þáh, &
þæt me \alst{eo}rla hléo \hld\ \alst{æ̂}r ge·sealde. &
\alst{Þ}æs ofer-eode, \hld\ \alst{þ}isses swá mæg!\eva

\bvb This of myself I wish to say, \\
that for a while I was the Heedenings’s shop, \\
dear to their lord—Deer was my name. \\
I had a multitude of winters a good retinue, \\
a \inx[C]{hold} bread-giver, until Harrend now, \\
the lay-crafty man has won the land-right \\
which to \emph{me} the shelter of earls once did grant. \\
\emph{That} passed over; \emph{this} may likewise.\evb\evg

\sectionline
