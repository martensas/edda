\bookStart{The Speeches of Webthrithner}[Vafþrúðnismǫ́l]

\begin{flushright}%
\textbf{Dating} \parencite{Sapp2022}: 800s (0.105)–900s (0.894)

\textbf{Meter:} \Ljodahattr%
\end{flushright}%

A wisdom contest poem, known by the author of \Gylfaginning.

Far from being a loose collection of pieces of mythic information, the poem is tightly structured.

Weden first asks his wife, Frie, for counsel, as he is curious about the ancient wisdom which the ettin Webthrithner might possess (1). Frie expresses worry, as she considers Webthrithner wiser than all other ettins (2), but Weden says that he has travelled far and wide, and wishes to know what Webthrithner’s hall is like (3). Frie wishes Weden good luck against the ettin (4) and he departs, to challenge Webthrithner’s \emph{orð-spęki} ‘word-wisdom’ (5). He arrives at hall of Webthrithner (6), who promptly declares that Weden will not come out of the hall unless he be wiser than him (7). Weden introduces himself as Gainred, saying that he has travelled far in need of Webthrithner’s hospitality (8). Webthrithner invites Weden to sit down (9), but he instead utters a gnomic stanza (10) not unlike those of the first section of \Havamal.

Webthrithner then begins to ask questions relating to the mythology, each answered by Weden in turn. The questions concern which horses pull the day (11–12) and night (13–14), the river which divides the gods and ettins (15–16) and the plain where Surt and the gods will fight (17–18).

Webthrithner calls the god learned, invites him to sit on the bench, and declares that the loser of the contest must give his head (19). The roles are now reversed, and Weden asks the ettin about the origins of earth and heaven (20–21), of sun and moon (22–23), of day, night, and the phases of the moon (24–25), and of winter and summer (26–27); then about the earliest being, namely the ettin \inx[P]{Earyelmer} (28–29), his origins (30–31) and how he reproduced asexually (32–33). Weden continues by asking what Webthrithner himself first remembers (34–35), about the origin of the wind (36–37), the god \inx[P]{Nearth} (38–39), Walhall and the Oneharriers (40–41), and where Webthrithner learned these esoteric pieces of wisdom (42–43).

After this the structure and tone of the questions change; each one begins with the same first half as that of st. 3, and they concern the end-times. Weden asks which humans will survive after the Fimble-winter (44–45), how the sun will rise after Fenrer has destroyed it (46–47), about some obscure maidens (48–49; see discussion there), which Eese will survive after the flame of Surt goes out (50–51) and how Weden will die (52–53). Finally, he asks what Weden spoke in the ear of Balder before he was burned on the pyre (54). Webthrithner at last understands the identity of his challenger, since only Weden himself could know the answer to that question. He laconically accepts his imminent death and the futility of his word-wisdom (55); the poem ends with his admission that Weden will always be the wisest (56).

\sectionline

\bvg\bva\speakernote{[Óðinn kvað:]}\mssnote{\Regius~7v/9}„Ráð mér nú \alst{F}rigg \hld\ alls mik \alst{f}ara tíðir &
\ind at \alst{v}itja \alst{V}af-þrúðnis; &
\alst{f}or-vitni mikla \hld\ kveð’k mér á \alst{f}ornum stǫfum &
\ind við þann hinn \alst{a}l-svinna \alst{jǫ}tun.“\eva

\bvb\speakernoteb{[\inx[P]{Weden} quoth:]}
“Counsel me now, \inx[P]{Frie}, as I desire to journey \\
to visit \inx[P]{Webthrithner}; \\
Very curious am I of ancient staves \\
from that all-wise \inx[G]{Ettins}[ettin].\footnoteB{i.e. ‘I am very curious to learn his ancient wisdom.’ Cf. st. 55.}”\evb\evg


\bvg\bva\speakernote{[Frigg kvað:]}\mssnote{\Regius~7v/12}„\alst{H}ęima lętja \hld\ mynda’k \alst{H}ęrja-fǫðr &
\ind í \alst{g}ǫrðum \alst{g}oða; &
því-at \alst{ę}ngi \alst{jǫ}tun \hld\ hugða’k \alst{ja}fn-ramman &
\ind sęm \alst{V}af-þrúðni \alst{v}esa.“\eva

\bvb\speakernoteb{[Frie quoth:]}
“At home would I keep the Father of Hosts \ken*{= Weden}, \\
in the yards of the Gods— \\
for no ettin have I judged to be \\
as strong as Webthrithner.”\evb\evg


\bvg\bva\speakernote{[Óðinn kvað:]}\mssnote{\Regius~7v/13}„\alst{F}jǫlð ek \alst{f}ór, \hld\ \alst{f}jǫlð \alst{f}ręistaða’k, &
\ind fjǫlð ek \alst{r}ęynda \alst{r}ęgin; &
hitt \alst{v}il’k \alst{v}ita, \hld\ hvé \alst{V}af-þrúðnis &
\ind \alst{s}ala-kynni \alst{s}éi.“\eva

\bvb\speakernoteb{[Weden quoth:]}
“Much I journeyed, much I tried, \\
much I tested the \inx[G]{Reins}. \\
This I wish to know: how Webthrithner’s \\
halls might be.”\evb\evg


\bvg\bva\speakernote{[Frigg kvað:]}\mssnote{\Regius~7v/15}„\alst{H}ęill þú farir, \hld\ \alst{h}ęill þú aptr komir, &
\ind hęill á \alst{s}innum \alst{s}éir; &
\alst{ǿ}ði þér dugi \hld\ hvar’s skalt, \alst{A}lda-fǫðr, &
\ind \alst{o}rðum mę́la \alst{jǫ}tun.“\eva

\bvb\speakernoteb{[Frie quoth:]}
“Whole journey thou, whole come thou back, \\
whole be thou on thy paths! \\
May thy wisdom avail thee, where thou shalt, O \inx[P]{Eldfather} \name{= Weden}, \\
with words address the ettin!”\evb\evg


\bvg\bva\mssnote{\Regius~7v/17}%
\alst{F}ór þá Óðinn \hld\ at \alst{f}ręista orð-spęki &
\ind þess hins \alst{a}l-svinna \alst{jǫ}tuns; &
at \alst{h}ǫllu kom, \hld\ \edtext{es}{\Afootnote{ok \Regius}} átti \edtext{\alst{H}ymis}{\lemma{Hymis}\Afootnote{\emph{metr. emend. after} \textcite{FinnurEdda}; Íms \Regius}} faðir; &
\ind \alst{i}nn gekk \alst{Y}ggr þegar.\eva

\bvb Then journeyed Weden to test the word-wisdom \\
of that all-wise ettin. \\
To the hall he came, which the father of \inx[P]{Hymer} \ken*{= Webthrithner} owned; \\
shortly walked \inx[P]{Ug} \name{= Weden} inside.\evb\evg


\bvg\bva\speakernote{[Óðinn kvað:]}\mssnote{\Regius~7v/18}%
„\alst{H}ęill þú nú, Vaf-þrúðnir, \hld\ nú em’k í \alst{h}ǫll kominn &
\ind á þik \alst{s}jalfan \alst{s}éa; &
hitt vil’k \alst{f}yrst vita, \hld\ ef \alst{f}róðr séir &
\ind eða \alst{a}l-sviðr, \alst{jǫ}tunn.“\eva

\bvb\speakernoteb{[Weden quoth:]}%
“Hail thee now, O Webthrithner; now am I come into the hall, \\
to see thy very self! \\
This I wish first to know, if learned thou be, \\
or all-wise, O ettin.”\evb\evg


\bvg\bva\speakernote{Vafþrúðnir:}\mssnote{\Regius~7v/20}%
„Hvat ’s þat \alst{m}anna, \hld\ es í \alst{m}ínum sal &
\ind verpumk \alst{o}rði \alst{á}? &
\alst{ú}t þú né kømr \hld\ \alst{ó}rum hǫllum frá, &
\ind nema þú inn \alst{s}notrari \alst{s}éir.“\eva

\bvb\speakernoteb{[Webthrithner quoth:]}%
“What sort of man is that, who in \emph{my} hall \\
throws words at me? \\
Out comest thou not from \emph{our} halls, \\
unless thou be the smarter man.”\evb\evg


\bvg\bva\speakernote{Óðinn kvað:}\mssnote{\Regius~7v/22}%
„\edtext{\alst{G}agnráðr}{\Afootnote{Gang-ráðr ‘Gangred; Journey-adviser’ \GylfMS\ (paraphrased).}} hęiti’k, \hld\ nú em’k af \alst{g}ǫngu kominn, &
\ind \alst{þ}yrstr til \alst{þ}inna sala; &
\alst{l}aðar þurfi \hld\ hęf’k \alst{l}ęngi farit &
\ind ok þinna \alst{a}nd-fanga, \alst{jǫ}tunn.“\eva

\bvb\speakernoteb{[Weden quoth:]}%
“\inx[P]{Gainred} I am called; now am I come from walking, \\
thirsty, to thy halls. \\
In need of a welcome have I journeyed for long; \\
and of thy reception, ettin!”\evb\evg


\bvg\bva\speakernote{Vafþrúðnir:}\mssnote{\Regius~7v/24}%
„Hví þú þá, \alst{G}agnráðr, \hld\ mę́lisk af \alst{g}olfi fyrir? &
\ind far þú í \alst{s}ess í \alst{s}al; &
þá skal \alst{f}ręista, \hld\ hvárr \alst{f}lęira viti, &
\ind \alst{g}ęstr eða hinn \alst{g}amli þulr.“\eva

\bvb\speakernoteb{[Webthrithner quoth:]}%
“Why then, Gainred, speakest thou from the floor before me? \\
Take a seat in the hall! \\
Then it shall be tried, which of the two might know more: \\
the guest, or the old \inx[C]{thyle} \ken*{I}.”\evb\evg


\bvg\bva\speakernote{[Óðinn kvað:]}\mssnote{\Regius~7v/26}%
„\alst{Ó}-auðigr maðr, \hld\ es til \alst{au}ðigs kømr, &
\ind \edtrans{mę́li \alst{þ}arft eða \alst{þ}ęgi}{ought to speak the needful or shut up}{\Bfootnote{Formulaic, this line occurs identically in \Havamal\ 19.}}; &
\alst{o}fr-mę́lgi mikil \hld\ hygg’k at \alst{i}lla geti &
\ind hvęim’s við \edtrans{\alst{k}ald-rifjaðan}{cold-ribbed}{\Bfootnote{i.e. ‘cold-hearted, cunning’.}} \alst{k}ømr.“\eva

\bvb\speakernoteb{[Weden quoth:]}%
“The unwealthy man who comes to a wealthy one \\
ought to speak the needful or shut up. \\
Great over-speaking, I judge, will bring evil \\
for whomever comes by a cold-ribbed one.”\evb\evg


\bvg\bva\speakernote{Vafþrúðnir:}\mssnote{\Regius~7v/28}%
„Sęg mér, \alst{G}agnráðr, \hld\ alls á \alst{g}olfi vill &
\ind þíns of \alst{f}ręista \alst{f}rama, &
hvé \alst{h}ęstr \alst{h}ęitir, \hld\ sá’s \alst{h}vęrjan dręgr &
\ind \alst{d}ag of \alst{d}rótt-mǫgu.“\eva

\bvb\speakernoteb{[Webthrithner quoth:]}%
“Say to me, Gainred, since on the floor thou wilt \\
tempt thy furtherance: \\
What is the horse called which pulls every \\
day over the lads of the retinue \ken{men}?”\evb\evg


\bvg\bva\speakernote{[Óðinn kvað:]}\mssnote{\Regius~7v/30}%
„\alst{Sk}in-faxi hęitir, \hld\ es hinn \alst{sk}íra dręgr &
\ind \alst{d}ag of \alst{d}rótt-mǫgu; &
\alst{h}ęsta batstr \hld\ þykkir með \alst{H}ręið-gotum; &
\ind ęy lýsir \alst{m}ǫn af \alst{m}ari.“\eva

\bvb\speakernoteb{[Weden quoth:]}%
“\inx[P]{Shinefax} is he called who pulls the bright \\
day over the lads of the retinue. \\
The best of horses he seems among the \inx[G]{Reth-Gots}; \\
ever shines that stallion’s mane.”\evb\evg


\bvg\bva\speakernote{[Vafþrúðnir:]}\mssnote{\Regius~7v/32}%
„Sęg þat, \alst{G}agn-ráðr, \hld\ alls á \alst{g}olfi vill &
\ind þíns of \alst{f}ręista \alst{f}rama, &
hvé \alst{jó}r hęitir, \hld\ sá’s \alst{au}stan dręgr &
\ind \alst{n}ǫ́tt of \alst{n}ýt ręgin.“\eva

\bvb\speakernoteb{[Webthrithner quoth:]}%
“Say this, Gainred, since on the floor thou wilt \\
tempt thy furtherance: \\
What the steed is called which pulls from the east \\
the night over the useful \inx[G]{Reins}?”\evb\evg


\bvg\bva\speakernote{[Óðinn kvað:]}\mssnote{\Regius~7v/33}„\alst{H}rím-faxi \alst{h}ęitir, \hld\ es \alst{h}vęrja dręgr &
\ind \alst{n}ǫ́tt of \alst{n}ýt ręgin; &
\alst{m}él-dropa fęllir \hld\ \alst{m}orgin hvęrjan; &
\ind þaðan kømr \alst{d}ǫgg of \alst{d}ala.“\eva

\bvb\speakernoteb{[Weden quoth:]}%
“\inx[P]{Rimefax}\ is he called who pulls every \\
night over the useful Reins. \\
Each morning he does drool from his bit; \\
thence comes dew in the dales.\footnoteB{For another explanation of the origin of dew, see \Voluspa\ TODO.}”\evb\evg


\bvg\bva\speakernote{[Vafþrúðnir:]}\mssnote{\Regius~8r/1}„Sęg þat, \alst{G}agnráðr, \hld\ alls á \alst{g}olfi vill &
\ind \edtrans{þíns of \alst{f}ręista \alst{f}rama}{tempt thy furtherance}{\Bfootnote{i.e. try his luck, see how far he gets.  Formulaic; cf. \Havamal\ 2.}}, &
hvé \alst{ǫ́} hęitir, \hld\ sú’s dęilir með \alst{jǫ}tna sonum &
\ind \alst{g}rund, ok með \alst{g}oðum.“\eva

\bvb\speakernoteb{[Webthrithner quoth:]}%
“Say this, Gainred, since on the floor thou wilt \\
tempt thy furtherance: \\
What the river is called which divides the ground \\
between the sons of ettins and gods?”\evb\evg


\bvg\bva\speakernote{[Óðinn kvað:]}\mssnote{\Regius~8r/2}„\alst{Í}fing hęitir \alst{ǫ́}, \hld\ es dęilir með \alst{jǫ}tna sonum &
\ind \alst{g}rund, ok með \alst{g}oðum; &
\alst{o}pin rinna \hld\ hón skal umb \alst{a}ldr-daga; &
\ind verðr-at \alst{í}ss á \alst{ǫ́}u.“\eva

\bvb\speakernoteb{[Weden quoth:]}%
“\inx[L]{Iving} is the river called which divides the ground \\
between the sons of ettins and gods. \\
Open shall she through her life-days flow; \\
there forms no ice on the river.”\evb\evg


\bvg\bva\speakernote{[Vafþrúðnir:]}\mssnote{\Regius~8r/3}„Sęg þat, \alst{G}agnráðr, \hld\ alls á \alst{g}olfi vill &
\ind þíns of \alst{f}ręista \alst{f}rama, &
hvé \alst{v}ǫllr hęitir, \hld\ es finnask \alst{v}igi at &
\ind \alst{S}urtr ok hin \alst{s}vǫ́su goð.“\eva

\bvb\speakernoteb{[Webthrithner quoth:]}%
“Say this, Gainred, since on the floor thou wilt \\
tempt thy furtherance: \\
What that plain is called where they find each other at war, \\
\inx[P]{Surt} and the excellent Gods?”\evb\evg


\bvg\bva\speakernote{Óðinn:}\mssnote{\Regius~8r/4}„\alst{V}ígríðr hęitir \alst{v}ǫllr, \hld\ es finnask \alst{v}ígi at &
\ind \alst{S}urtr ok hin \alst{s}vǫ́su goð; &
\alst{h}undrað rasta \hld\ hann ’s á \alst{h}vęrjan veg; &
\ind sá ’s þęim \alst{v}ǫllr \alst{v}itaðr.“\eva

\bvb\speakernoteb{Weden:}%
“\inx[L]{Wighride} is the plain called where they find each other at war, \\
Surt and the excellent gods. \\
A hundred \inx[C]{rest}[rests] it goes in every way; \\
for them that plain is marked out.”\evb\evg


\bvg\bva\speakernote{Vafþrúðnir:}\mssnote{\Regius~8r/6}„\alst{F}róðr est nú gęstr, \hld\ \alst{f}ar á bękk jǫtuns, &
\ind ok mę́lumk í \alst{s}essi \alst{s}aman; &
\alst{h}ǫfði vęðja \hld\ vit skulum \alst{h}ǫllu í &
\ind \alst{g}ęstr, of \alst{g}oð-spęki.“\eva

\bvb\speakernoteb{Webthrithner:}%
“Learned art thou now, guest, come onto the ettin’s bench \\
and let us speak on the seat together. \\
Wager a head, shall we two in the hall, \\
O guest, over god-wisdom!”\evb\evg

\sectionline

\bvg\bva\speakernote{Óðinn:}\mssnote{\Regius~8r/9, \AM~3r/1}„Sęg þat hit \alst{ęi}na, \hld\ ef þitt \edtext{\alst{ǿ}ði}{\lemma{ǿði}\Bfootnote{The first word on fol. 3r. of \AM; from this point we have the poem in both manuscripts.}} dugir &
\ind ok þú \alst{V}af-þrúðnir \alst{v}itir, &
hvaðan \alst{jǫ}rð of kom, \hld\ eða \alst{u}pp-himinn &
\ind \alst{f}yrst, hinn \alst{f}róði jǫtunn?“\eva

\bvb\speakernoteb{Weden:}%
“Say the one, if thy wisdom avails, \\
and thou, Webthrithner, mightst know: \\
Whence Earth did come, or \inx[L]{Up-heaven}, \\
first, O learned ettin?”\evb\evg


\bvg\bva\speakernote{Vafþrúðnir:}\mssnote{\Regius~8r/10, \AM~3r/2}„Ór \alst{Y}mis holdi \hld\ vas \alst{jǫ}rð of skǫpuð, &
\ind en ór \alst{b}ęinum \alst{b}jǫrg, &
\alst{h}iminn ór \alst{h}ausi \hld\ hins \alst{h}rím-kalda jǫtuns, &
\ind en ór \edtrans{\alst{s}vęita}{blood}{\Bfootnote{In poetry \emph{svęiti}, although cognate with ModEngl. ‘sweat’, almost always means ‘blood’. This is also the case with the OE \emph{swât}, as seen e.g. in \Beowulf\ 1286a: \emph{sweord} swâte \emph{fâh} ‘sword stained with \emph{sweat}’, 2689b–2690: \emph{hé ge-blódegod wearð // sâwul-dríore;} \hld\ swât \emph{ýðum wéoll.} ‘he was bloodied in soul-gore; the \emph{sweat} gushed in waves’.}} \alst{s}ę́r.“\eva

\bvb\speakernoteb{Webthrithner:}%
“Out of \inx[P]{Yimer}’s flesh was the earth shaped, \\
but out of his bones the mountains; \\
heaven out of the skull of the rime-cold ettin, \\
but out of his blood the sea.\footnoteB{The present st. very closely resembles \Grimnismal\ 41; lines 1 and 4 here are identical to lines 1–2 there, and lines 2 and 3a here are very similar to line 3a and 4 there. Although the stanzas are clearly related, they are still distinct enough that the one cannot be a direct scribal copy of the other. I think that the relation is most likely to be oral, and that the two stanzas were both composed in the same, most likely West Norwegian, community of poets, or perhaps even by the same individual.}”\evb\evg


\bvg\bva\speakernote{Óðinn:}\mssnote{\Regius~8r/12, \AM~3r/3}„Sęg þat \alst{a}nnat, \hld\ ef þitt \alst{ǿ}ði dugir &
\ind ok þú \alst{V}af-þrúðnir \alst{v}itir, &
hvaðan \alst{M}áni of kom, \hld\ svá’t fęrr \alst{m}ęnn yfir, &
\ind eða \alst{S}ól hit sama.“\eva

\bvb\speakernoteb{Weden:}%
“Say the other, if thy wisdom avails, \\
and thou, Webthrithner, mightst know: \\
Whence Moon did come, he that journeys over men, \\
or Sun likewise?”\evb\evg


\bvg\bva\speakernote{Vafþrúðnir:}\mssnote{\Regius~8r/13, \AM~3r/4}„\alst{M}undil-fari hęitir, \hld\ hann’s \alst{M}ána faðir &
\ind ok svá \alst{S}ólar hit \alst{s}ama; &
\alst{h}imin \alst{h}verfa \hld\ þau skulu \alst{h}vęrjan dag &
\ind \edtrans{\alst{ǫ}ldum at \alst{á}r-tali}{for the year-tally of mankind}{\Bfootnote{Cf. \Voluspa\ 6, where the Reins gave names to night, the moon-phases, morning, midday, afternoon, and evening \emph{ǫ́rum at tęlja} ‘the years for to tally’.}}.“\eva

\bvb\speakernoteb{Webthrithner:}%
“\inx[P]{Mundlefare} is he called; he is Moon’s father, \\
and so of Sun likewise. \\
Turn round in heaven shall they, every day, \\
for the year-tally of mankind.”\evb\evg


\bvg\bva\speakernote{Óðinn:}\mssnote{\Regius~8r/15, \AM~3r/6}„\alst{S}ęg þat þriðja, \hld\ alls þik \alst{s}vinnan kveða &
\ind ok þú \alst{V}af-þrúðnir \alst{v}itir, &
hvaðan \alst{D}agr of kom, \hld\ sá’s fęrr \alst{d}rótt yfir, &
\ind eða \alst{N}ǫ́tt með \alst{n}iðum.“\eva

\bvb\speakernoteb{Weden:}%
“Say the third, as they call thee wise, \\
and thou, Webthrithner, mightst know: \\
Whence Day came, he that journeys over the retinue, \\
or Night with the moon-phases?”\evb\evg


\bvg\bva\speakernote{Vafþrúðnir:}\mssnote{\Regius~8r/17, \AM~3r/8}„\alst{D}ęllingr hęitir, \hld\ hann’s \alst{D}ags faðir, &
\ind en \alst{N}ǫ́tt vas \alst{N}ǫrvi borin; &
\edtrans{\alst{n}ý ok \alst{n}ið}{The waxing and waning}{\Bfootnote{i.e. “the moon-phases”.  Cf. \Voluspa\ 6.}} \hld\ skópu \alst{n}ýt ręgin &
\ind \alst{ǫ}ldum at \alst{á}r-tali.“\eva

\bvb\speakernoteb{Webthrithner:}%
“\inx[P]{Delling} is he called; he is \inx[P]{Day}’s father, \\
but \inx[P]{Night} was born to \inx[P]{Narrow}. \\
The waxing and waning did the useful Reins create \\
for the year-tally of mankind.”\evb\evg


\bvg\bva\speakernote{Óðinn kvað:}\mssnote{\Regius~8r/18, \AM~3r/9}„Sęg þat \alst{f}jórða, \hld\ alls þik \alst{f}róðan kveða, &
\ind ok þú \alst{V}af-þrúðnir \alst{v}itir, &
hvaðan \alst{v}etr of kom \hld\ eða \alst{v}armt sumar &
\ind \alst{f}yrst með \alst{f}róð ręgin.“\eva

\bvb\speakernoteb{Weden quoth:}%
“Say the fourth, as they call thee learned, \\
and thou, Webthrithner, mightst know: \\
Whence winter did come, or warm summer, \\
first, among the learned Reins?”\evb\evg


\bvg\bva\speakernote{Vafþrúðnir:}\mssnote{\Regius~8r/20, \AM~3r/10}\edtext{„\alst{V}ind-svalr hęitir, \hld\ hann’s \alst{V}etrar faðir, &
\ind en \alst{S}vǫ́suðr \alst{S}umars.“}{\lemma{Vind-svalr \dots\ Sumars}\Bfootnote{The second half of the st. seems to be missing.}}\eva

\bvb\speakernoteb{Webthrithner:}%
“\inx[P]{Windswoll}\ is he called; he is \inx[P]{Winter}’s father; \\
but \inx[P]{Sosuth}\ [is] \inx[P]{Summer}’s.”\evb\evg


\bvg\bva\speakernote{Óðinn kvað:}\mssnote{\Regius~8r/21, \AM~3r/11}„Sęg þat \alst{f}imta, \hld\ alls þik \alst{f}róðan kveða, &
\ind ok þú \alst{V}af-þrúðnir \alst{v}itir, &
hvęrr \alst{á}sa \alst{ę}ldstr \hld\ eða \alst{Y}mis niðja &
\ind \alst{y}rði í \alst{á}r-daga.“\eva

\bvb\speakernoteb{Weden quoth:}%
“Say the fifth, as they call thee learned, \\
and thou, Webthrithner, mightst know: \\
Who of the \inx[G]{Eese}, or of Yimer’s kinsmen \ken{ettins}, \\
in days of yore might have become eldest?\footnoteB{i.e. ‘which being arose first of all?’ Cf. the question on the C9th Malt Stone (DR NOR1988;5): \textbf{huaʀisi : alistiąsa}, perhaps \emph{Hvaʀ es inn ęlisti ása?} ‘Who is the eldest of the Eese?’}”\evb\evg


\bvg\bva\speakernote{Vafþrúðnir:}\mssnote{\Regius~8r/22, \AM~3r/12}„\alst{Ø}r-ófi vetra \hld\ áðr vę́ri \alst{jǫ}rð of skǫpuð, &
\ind þá vas \alst{B}er-gęlmir \alst{b}orinn, &
\alst{Þ}rúð-gęlmir \hld\ vas \alst{þ}ess faðir, &
\ind en \alst{Au}r-gęlmir \alst{a}fi.“\eva

\bvb\speakernoteb{Webthrithner:}%
“Uncountable winters before the Earth was created, \\
then was \inx[P]{Bareyelmer} born. \\
\inx[P]{Thrithyelmer} was that one’s father, \\
and \inx[P]{Earyelmer} the grandfather.”\evb\evg


\bvg\bva\speakernote{Óðinn kvað:}\mssnote{\Regius~8r/23, \AM~3r/14}„\alst{S}ęg þat \alst{s}étta, \hld\ alls þik \alst{s}vinnan kveða, &
\ind ok þú \alst{V}af-þrúðnir \alst{v}itir, &
hvaðan \alst{Au}r-gęlmir kom \hld\ með \alst{jǫ}tna sonum &
\ind \alst{f}yrst, hinn \alst{f}róði jǫtunn.“\eva

\bvb\speakernoteb{Weden quoth:}%
“Say the sixth, as they call thee wise, \\
and thou, Webthrithner, mightst know: \\
Whence Earyelmer came among the sons of ettins, \\
first, O learned ettin?”\evb\evg


\bvg\bva\speakernote{Vafþrúðnir:}\mssnote{\Regius~8r/25, \AM~3r/15, \GylfMS}„\edtext{\alst{Ó}r \alst{É}li-vǫ́gum \hld\ stukku \alst{ęi}tr-dropar, &
\ind svá \alst{ó}x unds ór varð \alst{jǫ}tunn; &
\edtext{þar \alst{ó}rar \alst{ę́}ttir \hld\ kómu \alst{a}llar saman; &
\ind því’s \edtrans{þat}{it}{\Bfootnote{i.e. the ettin race.}} \alst{ę́} \alst{a}lt til \alst{a}talt.“}{\lemma{órar \dots\ atalt ‘Our \dots\ fierce’}\Bfootnote{so \GylfMS; om. \Regius\AM.}}}{\lemma{ALL}\Bfootnote{Over æons the splashing venom-drops combined until they formed a sentient being: this was Earyelmer, whom \Gylfaginning\ identifies with \inx[P]{Yimer}.  This stanza is cited in support of the lengthy and embellished creation narrative found in \Gylfaginning, but it is not certain that this is what our poet had in mind.

The Ilewaves are probably a reflex of the chaotic primeval Waters found in many West Eurasian mythologies, including Genesis 1:1–3 and \Rigveda\ 10.129.  Of these two foundational religious sources the latter is closer to the present stanza, and probably holds the more archaic conception.  Where we find in the Jewish narrative a proper \emph{creation}; at the very beginning of time God’s spirit is on the Waters and He makes the light shine over them, we find in these two Indo-European texts a \emph{spontaneous emergence} of a single primeval entity long before the Gods are born—here from the violent splashing of venom, in \Rigveda\ 10.129.3 from “the power of heat” (\emph{tápasaḥ mahinā́}).  This entity in turn asexually begets sexual beings—here through rubbing his limbs together, in \Rigveda\ 10.129.4 simply giving rise to “desire” (\emph{kā́ma}) which serves as the “primal seed of thought” (\emph{mánasaḥ rétaḥ prathamám})—and it is from these that the world is populated.}}\eva

\bvb\speakernoteb{Webthrithner:}%
“From the \inx[L]{Ilewaves} splashed venom-drops; \\
so it grew until it formed an ettin. \\
Our lineages came there all together, \\
thus it is ever all too fierce.”\evb\evg


\bvg\bva\speakernote{Óðinn kvað:}\mssnote{\Regius~8r/26, \AM~3r/16}%
„\alst{S}ęg þat \alst{s}jaunda, \hld\ alls þik \alst{s}vinnan kveða, &
\ind ok þú \alst{V}af-þrúðnir \alst{v}itir, &
hvé sá \alst{b}ǫrn gat \hld\ hinn \edtrans{\alst{b}aldni}{stubborn}{\Afootnote{so \AM; \emph{aldni} ‘the aged, old’ \Regius\ breaks alliteration}} jǫtunn, &
\ind es hann hafði-t \alst{g}ýgjar \alst{g}aman.“\eva

\bvb\speakernoteb{Weden quoth:}%
“Say the seventh, as they call thee wise, \\
and thou, Webthrithner, mightst know: \\
How that one begot bairns, the stubborn ettin, \\
when he knew not a troll-woman’s pleasure?”\evb\evg


\bvg\bva\speakernote{Vafþrúðnir kvað:}\mssnote{\Regius~8r/27, \AM~3r/17}%
„\edtext{Und \alst{h}ęndi vaxa \hld\ kvǫ́ðu \alst{h}rím-þursi &
\ind \alst{m}ęy ok \alst{m}ǫg saman; &
\alst{f}ótr við \alst{f}ǿti}{\lemma{Und hęndi \dots\ fótr við fǿti ‘Within the hand \dots\ Foot against foot’}\Bfootnote{The image is masturbatory.  The stanza is paraphrased in \Gylfaginning\ 5: \emph{En svá er sagt, at þá er hann svaf, fekk hann sveita. Þá óx undir vinstri hendi honum maðr ok kona, ok annarr fótr hans gat son við ǫðrum, en þaðan af kómu ę́ttir.} ‘But so is said, that when he slept he began to sweat.  Then grew within his left hand a man and a woman, and one foot of his begat a son by the other, and thereof come the lineages [of Ettins].’}} \hld\ gat hins \alst{f}róða jǫtuns &
\ind \alst{s}ex-hǫfðaðan \alst{s}on.“\eva

\bvb\speakernoteb{Webthrithner quoth:}%
“Within the hand of the \inx[G]{Rime-Thurses}[rime-thurse], they said, did grow \\
a maiden and lad together. \\
Foot by a foot begat for the learned ettin \\
a six-headed son.”\evb\evg


\bvg\bva\speakernote{Óðinn kvað:}\mssnote{\Regius~8r/29, \AM~3r/18}„Sęg þat ǫ́ttunda, \hld\ alls þik fróðan kveða, &
\ind ok þú \alst{V}af-þrúðnir \alst{v}itir, &
hvat \alst{f}yrst of mant \hld\ eða \alst{f}ręmst of vęitst, &
\ind þú est \alst{a}l-sviðr \alst{jǫ}tunn.“\eva

\bvb\speakernoteb{Weden quoth:}%
“Say the eigth, as they call thee learned, \\
and thou, Webthrithner, mightst know: \\
What recallest thou first, or knowest foremost? \\
Thou art all-wise, ettin!”\evb\evg


\bvg\bva\speakernote{Vafþrúðnir kvað:}\mssnote{\Regius~8r/30, \AM~3r/19, \GylfMS}„\alst{Ø}r-ófi vetra \hld\ áðr vę́ri \alst{jǫ}rð of skǫpuð, &
\ind þá vas \alst{B}er-gęlmir \alst{b}orinn; &
þat \alst{f}yrst of man’k, \hld\ es hinn \alst{f}róði jǫtunn &
\ind á vas \alst{l}úðr of \alst{l}agiðr.“\eva

\bvb\speakernoteb{Webthrithner quoth:}%
“Uncountable winters before the Earth was created, \\
then was Bareyelmer born. \\
It I first remember, when the learned ettin \\
on the tree-trunk was laid.\footnoteB{An obscure mythological reference.  According to the prose of \Gylfaginning, after the sons of \inx[P]{Byre} (that is, \inx[P]{Weden}, \inx[P]{Will} and \inx[P]{Wigh}) slew Yimer, so much blood flew from his wounds that all the race of Ettins were drowned, save for Bareyelmer and his family, who survived by getting up on his \emph{lúðr}.  This is clearly a variant of the flood myth, but it may be of Biblical origin.

In regular prose, \emph{lúðr} usually means ‘trumpet, blowing horn’, less commonly ‘flour-bin’; the underlying sense seems to be ‘hollowed-out wood’.  Considering the transitive nature of Bareyelmer being laid (\emph{of lagiðr}) upon it, the stanza might instead be referring a ship burial, so that the first thing Webthrithner remembers is Bareyelmer’s funeral.}”\evb\evg


\bvg\bva\speakernote{Óðinn kvað:}\mssnote{\Regius~8r/32, \AM~3r/21}„\alst{S}ęg þat níunda, \hld\ alls þik \alst{s}vinnan kveða, &
\ind ok þú \alst{V}af-þrúðnir \alst{v}itir, &
hvaðan \alst{v}indr of kømr \hld\ svá’t fęrr \alst{v}ág yfir, &
\ind \edtrans{ę́ męnn hann \alst{s}jalfan of \alst{s}éa}{men always see his very self}{\Bfootnote{Most likely a negative clitic \emph{-t} has been lost from the verb \emph{séa} ‘see’, which would have given the proper reading: “men \emph{never} see his very self”.}}.“\eva

\bvb\speakernoteb{Weden quoth:}%
“Say the ninth, as they call thee wise, \\
and thou, Webthrithner, mightst know: \\
Whence the wind comes which fares over the wave; \\
men always see his very self?”\evb\evg


\bvg\bva\speakernote{Vafþrúðnir:}\mssnote{\Regius~8r/34, \AM~3r/22}„\alst{H}rę́-svęlgr \alst{h}ęitir, \hld\ es sitr á \alst{h}imins ęnda, &
\ind \alst{jǫ}tunn í \alst{a}rnar ham; &
af hans \alst{v}ę́ngjum \hld\ kveða \alst{v}ind koma &
\ind \alst{a}lla męnn \alst{y}fir.“\eva

\bvb\speakernoteb{Webthrithner:}%
“\inx[P]{Rawswallower} is he called who sits at heaven’s end; \\
an ettin in an eagle’s \inx[C]{hame}. \\
From his wings they say that the wind comes \\
over all men.”\evb\evg


\bvg\bva\speakernote{[Óðinn kvað:]}\mssnote{\Regius~8v/1, \AM~3r/24}„Sęg þat \alst{t}íunda, \hld\ alls þú \alst{t}íva rǫk &
\ind ǫll \alst{V}afþrúðnir \alst{v}itir, &
hvaðan Njǫrðr of kom \hld\ með ása sonum; &
\edtrans{\alst{h}ofum ok \alst{h}ǫrgum}{hoves and harrows}{\Bfootnote{A formulaic merism, see note to \Voluspa\ 7 for other occurrences. This stanza seems to be referring to the large count of cultic places named after Nearth in Norway (TODO: source this); cf. here \Grimnismal\ 16, where it is said that Nearth \emph{rę́ðr hǫ́-timbruðum hǫrgi} ‘rules a high-timbered harrow’. Also of interest is \Lokasenna\ 51, where a goddess speaks about her \emph{véum ok vǫngum} ‘wighs and wongs’, other cultic names. All of these examples suggest something about the Heathen view of shrines.}} \hld\ rę́ðr \alst{h}und-mǫrgum &
\ind ok varð-at \alst{ǫ́}sum \alst{a}linn.“\eva

\bvb\speakernoteb{[Weden quoth:]}%
“Say the tenth, since thou of the \inx[P]{Rakes of the Tews} \\
all, O Webthrithner, mightst know: \\
Whence \inx[P]{Nearth} did come among the sons of the \inx[G]{Eese}? \\
\inx[C]{hove}[Hoves] and \inx[C]{harrow}[harrows] he rules hound-many, \\
and he was not to the Eese begotten.”\evb\evg


\bvg\bva\speakernote{[Vafþrúðnir kvað:]}\mssnote{\Regius~8v/3, \AM~3r/26}„Í \alst{V}ana-hęimi \hld\ skópu hann \alst{v}ís ręgin &
\ind ok sęldu at \alst{g}íslingu \alst{g}oðum, &
í \alst{a}ldar rǫk \hld\ hann mun \alst{a}ptr koma &
\ind hęim með \alst{v}ísum \alst{v}ǫnum.“\eva

\bvb\speakernoteb{[Webthrithner quoth:]}%
“In \inx[L]{Waneham} the wise \inx[G]{Reins}\footnoteB{While \emph{ręgin} ‘Reins’ is usually just a synonym of \emph{goð} ‘gods’, it seems here to refer specifically to the Wanes, in contrast with the \inx[G]{Eese}.} created him, \\
and sold him as a hostage to/for the gods. \\
In the Rakes of Mankind\footnoteB{i.e. the \inx[P]{Rakes of the Reins}.} he will come back \\
home among the wise \inx[G]{Wanes}.”\evb\evg

\sectionline

{\small The two following stanzas are damaged in both \Regius\ and \AM; \Regius\ has only st. 40, but splits it in two, while \AM\ has 40/1 (abbreviated in the ms.: \emph{S. þ. e. XI}) and then jumps to the answer. The present two stanzas are reconstructed. TODO: explain better.}

\sectionline

\bvg\bva\speakernote{[Óðinn kvað:]}\mssnote{\Regius~8v/5, \AM~3r/28}„Sęg þat \alst{ę}llipta, \hld\ hvar \alst{ý}tar túnum í &
\ind \alst{h}ǫggvask \alst{h}vęrjan dag; &
\edtext{\alst{v}al þęir kjósa}{\lemma{val þęir kjósa ‘the slain they choose’}\Bfootnote{It is from this verbal phrase that the agent noun \emph{val-kyrja} ‘\inx[G]{walkirries}[walkirrie]’ is derived; yet those are all women (as the very gender of the word shows), while the Oneharriers are male.}} \hld\ ok ríða \alst{v}ígi frá, &
\ind \alst{s}itja męir of \alst{s}áttir \alst{s}aman.“\eva

\bvb\speakernoteb{[Weden quoth:]}%
“Say the eleventh: Where men in yards \\
cut each other down every day? \\
The slain they choose and from the battle ride; \\
{[then]} they sit more at peace together.”\evb\evg


\bvg\bva\speakernote{[Vafþrúðnir kvað:]}\mssnote{\AM~3r/28}„\alst{A}llir \alst{ęi}n-hęrjar \hld\ \alst{Ó}ðins túnum í &
\ind \alst{h}ǫggvask \alst{h}vęrjan dag, &
\alst{v}al þęir kjósa \hld\ ok ríða \alst{v}ígi frá, &
\ind \alst{s}itja męir of \alst{s}áttir \alst{s}aman.“\eva

\bvb\speakernoteb{[Webthrithner quoth:]}%
“All the \inx[G]{Oneharriers} in Weden’s yards \\
cut each other down every day. \\
The slain they choose and from the battle ride; \\
{[then]} they sit more at peace together.”\evb\evg


\bvg\bva\speakernote{[Óðinn kvað:]}\mssnote{\Regius~8v/6, \AM~3v/1}„Sęg þat \alst{t}olpta, \hld\ hví þú \alst{t}íva rǫk &
\ind ǫll \alst{V}af-þrúðnir \alst{v}itir, &
frá \alst{jǫ}tna rúnum \hld\ ok \alst{a}llra goða &
\ind þú hit \alst{s}annasta \alst{s}ęgir, &
\ind hinn \alst{a}l-svinni \alst{jǫ}tunn.“\eva

\bvb\speakernoteb{[Weden quoth:]}%
“Say the twelfth: Why thou the rakes of the Tews \\
all, Webthrithner, mightst know? \\
From the \inx[C]{rune}[runes] of the ettins and of all the gods \\
sayest thou the truest, \\
O all-wise ettin!”\evb\evg


\bvg\bva\speakernote{[Vafþrúðnir kvað:]}\mssnote{\Regius~8v/8, \AM~3v/2}„Frá \alst{jǫ}tna rúnum \hld\ ok \alst{a}llra goða &
\ind ek kann \alst{s}ęgja \alst{s}att, &
\ind því-at \alst{h}vęrn hęf’k \alst{h}ęim of komit, &
\alst{n}íu kom’k hęima \hld\ fyr \alst{n}ifl-hęl neðan; &
\ind \alst{h}inig dęyja ór \alst{h}ęlju \alst{h}alir.“\eva

\bvb\speakernoteb{[Webthrithner quoth:]}%
“From the runes of the ettins and of all the gods \\
I can speak truly, \\
for I have come into each \inx[C]{Home}. \\
Into nine Homes I came beneath \inx[L]{Nivelhell}; \\
that way die men out of \inx[L]{Hell}.\footnoteB{Presumably lower underworlds, more severe than the ‘normal’ one. \textcite{FinnurEdda}\ considers \emph{ór hęlju} ‘out of Hell’ a later interpolation, presumably for metric reasons, but there is no textual support for it.}”\evb\evg

\sectionline

\bvg\bva\speakernote{[Óðinn kvað:]}\mssnote{\Regius~8v/11, \AM~3v/4}„\alst{F}jǫlð ek \alst{f}ór, \hld\ \alst{f}jǫlð \alst{f}ręistaða’k, &
\ind fjǫlð ek \alst{r}ęynda \alst{r}ęgin; &
hvat lifir \alst{m}anna, \hld\ þá’s hinn \alst{m}ę́ra líðr &
\ind \alst{f}imbul-vetr með \alst{f}irum?“\eva

\bvb\speakernoteb{[Weden quoth:]}%
“Much I journeyed, much I tried, \\
much I tested the Reins.\footnoteB{Cf. v. 3.} \\
What remains of men, when the renowned \inx[L]{Fimble-winter} \\
passes among people?”\evb\evg


\bvg\bva\speakernote{[Vafþrúðnir kvað:]}\mssnote{\Regius~8v/13, \AM~3v/6}„\alst{L}íf ok \alst{L}ífþrasir, \hld\ en þau \alst{l}ęynask munu &
\ind í \alst{h}olti \alst{H}odd-mímis; &
\alst{m}orgin-dǫggvar \hld\ þau sér at \alst{m}at hafa; &
\ind þaðan af \alst{a}ldir \alst{a}lask.“\eva

\bvb\speakernoteb{[Webthrithner quoth:]}%
“\inx[P]{Life} and \inx[P]{Lifethrasher}, but they will hide themselves \\
in \inx[P]{Hoardmimer}’s wood.\footnoteB{Perhaps in the hollowed-out Uggdrassle.} \\
Morning-dew [will] they have as food; \\
thence [will] generations be bred.”\evb\evg


\bvg\bva\speakernote{[Óðinn kvað:]}\mssnote{\Regius~8v/15, \AM~3v/8}„\alst{F}jǫlð ek \alst{f}ór, \hld\ \alst{f}jǫlð \alst{f}ręistaða’k, &
\ind fjǫlð ek \alst{r}ęynda \alst{r}ęgin; &
hvaðan kømr \alst{s}ól \hld\ á hinn \alst{s}létta himin, &
\ind \edtrans{es þessa hęfr \alst{F}ęnrir \alst{f}arit?}{when Fenrer has this one slain.}{\Bfootnote{Cf. \Voluspa\ TODO. Here it is Fenrer himself who will swallow the sun unless it there be taken as a poetic synonym for ‘wolf’ (which undoubtedly is its original meaning). TODO}}\eva

\bvb\speakernoteb{[Weden quoth:]}%
“Much I journeyed, much I tried, \\
much I tested the Reins. \\
Whence comes Sun onto the smooth heaven, \\
when \inx[P]{Fenrer} has this one\footnoteB{The current incarnation of the sun, as explained in the next st.} slain?”\evb\evg


\bvg\bva\speakernote{[Vafþrúðnir kvað:]}\mssnote{\Regius~8v/16, \AM~3v/9}„\alst{Ęi}na dóttur \hld\ berr \alst{a}lf-rǫðull, &
\ind áðr hana \alst{F}ęnrir \alst{f}ari; &
sú skal \alst{r}íða, \hld\ þá’s \alst{r}ęgin dęyja, &
\ind \alst{m}óður brautir \alst{m}ę́r.“\eva

\bvb\speakernoteb{[Webthrithner quoth:]}%
“A lone daughter the elf-wheel \ken*{= Sun} bears \\
before Fenrer might slay her. \\
She shall ride—when the Reins die— \\
the maiden, her mother’s paths.”\evb\evg


\bvg\bva\speakernote{[Óðinn kvað:]}\mssnote{\Regius~8v/18, \AM~3v/10}\alst{F}jǫlð ek \alst{f}ór, \hld\ \alst{f}jǫlð \alst{f}ręistaða’k, &
\ind fjǫlð ek \alst{r}ęynda \alst{r}ęgin; &
hvęrjar ’ru \alst{m}ęyjar, \hld\ es líða \alst{m}ar yfir, &
\ind \alst{f}róð-gęðjaðar \alst{f}ara.\eva

\bvb\speakernoteb{[Weden quoth:]}%
“Much I journeyed, much I tried, \\
much I tested the Reins. \\
Which are the maidens that pass over the ocean; \\
wise-minded they go?”\evb\evg


\bvg\bva\speakernote{[Vafþrúðnir kvað:]}\mssnote{\Regius~8v/19, \AM~3v/11}\alst{Þ}ríar \alst{þ}jóð-áar \hld\ falla \alst{þ}orp yfir &
\ind \alst{m}ęyja \alst{M}ǫg-þrasis; &
\alst{h}amingjur ęinar \hld\ þę́r’s í \alst{h}ęimi eru, &
\ind þó þę́r með \alst{jǫ}tnum \alst{a}lask.\eva

\bvb\speakernoteb{[Webthrithner quoth:]}%
“Three great rivers fall over the settlement \\
of the maidens of Maythrasher; \\
the only Hamings are they in the Home,\footnoteB{In Ettinham, or in the entire world?} \\
though they are among ettins begotten.”\evb\evg


\bvg\bva\speakernote{[Óðinn kvað:]}\mssnote{\Regius~8v/21, \AM~3v/13}„\alst{F}jǫlð ek \alst{f}ór, \hld\ \alst{f}jǫlð \alst{f}ręistaða’k, &
\ind fjǫlð ek \alst{r}ęynda \alst{r}ęgin; &
hvęrir ráða \alst{ę́}sir \hld\ \alst{ęi}gnum goða, &
\ind þá’s \alst{s}loknar \alst{S}urta-logi?“\eva

\bvb\speakernoteb{[Weden quoth:]}%
“Much I journeyed, much I tried, \\
much I tested the Reins. \\
Which Eese rule the ownings of the gods \\
when the flame of \inx[P]{Surt} goes out?”\evb\evg


\bvg\bva\speakernote{[Vafþrúðnir kvað:]}\mssnote{\Regius~8v/22, \AM~3v/14}„\alst{V}íðarr ok \alst{V}áli \hld\ byggva \alst{v}é goða, &
\ind þá’s \alst{s}loknar \alst{S}urta-logi; &
\alst{M}óði ok \alst{M}agni \hld\ skulu \alst{M}jǫllni hafa &
\ind \alst{V}ingnis at \alst{v}íg-þroti.“\eva

\bvb\speakernoteb{[Webthrithner quoth:]}%
“\inx[P]{Wider} and \inx[P]{Wonnel} settle the \inx[C]{wigh}[wighs] of the gods \\
when the flame of Surt goes out. \\
\inx[P]{Mood} and \inx[P]{Main} shall own \inx[P]{Millner} \\
after \inx[P]{Wingner}’s fight-exhaustion \ken{death}.\footnoteB{ie. ‘when Thunder dies’.}”\evb\evg


\bvg\bva\speakernote{[Óðinn kvað:]}\mssnote{\Regius~8v/24, \AM~3v/16}„\alst{F}jǫlð ek \alst{f}ór, \hld\ \alst{f}jǫlð \alst{f}ręistaða’k, &
\ind fjǫlð ek \alst{r}ęynda \alst{r}ęgin; &
hvat verðr \alst{Ó}ðni \hld\ at \alst{a}ldr-lagi, &
\ind þá’s \alst{r}júfask \alst{r}ęgin?“\eva

\bvb\speakernoteb{[Weden quoth:]}%
“Much I journeyed, much I tried, \\
much I tested the Reins. \\
What brings Weden’s life to an end, \\
when the Reins are ripped?\footnoteB{Formulaic; see note to \Baldrsdraumar\ TODO.}”\evb\evg


\bvg\bva\speakernote{[Vafþrúðnir kvað:]}\mssnote{\Regius~8v/25, \AM~3v/17}„\alst{U}lfr glęypa \hld\ mun \alst{A}lda-fǫðr, &
\ind þess mun \alst{V}íðarr \alst{v}reka; &
\alst{k}alda \alst{k}japta \hld\ hann \alst{k}lyfja mun &
\ind \alst{v}itnis \alst{v}ígi at.“\eva

\bvb\speakernoteb{[Webthrithner quoth:]}%
“The wolf will devour \inx[P]{Eldfather} \name{= Weden}: \\
that will Wider avenge. \\
The cold jaws he will cleave \\
of the Wolf at the battle.”\evb\evg


\bvg\bva\speakernote{[Óðinn kvað:]}\mssnote{\Regius~8v/27, \AM~3v/19}„\alst{F}jǫlð ek \alst{f}ór, \hld\ \alst{f}jǫlð \alst{f}ręistaða’k, &
\ind fjǫlð ek \alst{r}ęynda \alst{r}ęgin; &
hvat mę́lti Óðinn, \hld\ áðr á bál stigi, &
\ind \alst{s}jalfr í ęyra \alst{s}yni?“\eva

\bvb\speakernoteb{[Weden quoth:]}%
“Much I journeyed, much I tempted, \\
much I tested the Reins. \\
What spoke Weden, before he would mount the pyre,\footnoteB{The phrase \emph{stíga á} ‘step onto, mount’ is also used to refer to one stepping aboard a ship or mounting a horse (see \CV: \emph{stíga} for citations).  Its use for a person being borne onto the pyre seems formulaic and has been compared with \Beowulf\ 1118b: \emph{gu̇ð-rinc á·stáh} ‘the war-champion mounted [his pyre]’, although the interpretation of that line is controversial.  \textcite{KlaeberBeowulf}[186] follow Grundtvig in emending \emph{gu̇ð-rinc} to \emph{gu̇ð-réc} ‘war-smoke’ and relate it to \Beowulf\ 3144b (\emph{wudu-réc á·stáh} ‘wood-smoke rose up’, also in a description of a cremation.  According to them \Grimnismal\ 54 “almost certainly refers not to Baldr but to Óðinn, probably imagined to mount the pyre in order to set fire to it.”} \\
himself into the son’s \ken*{= Balder’s} ear?”\evb\evg


\bvg\bva\speakernote{[Vafþrúðnir kvað:]}\mssnote{\Regius~8v/28, \AM~3v/19}„\alst{Ęy} \edtext{mann-gi}{\Afootnote{\emph{manni} dat. sg. \Regius\AM\ is impossible; a subject is needed.}} vęit, \hld\ hvat þú í \alst{á}r-daga &
\ind \alst{s}agðir í ęyra \alst{s}yni; &
\edtrans{\alst{f}ęigum}{fey}{\Bfootnote{A word with strong fatalistic connections. Webthrithner realises that he was bound to die from the moment he proposed the wager (v. 19), as no being can outwit Weden.}} munni \hld\ mę́lta’k mína \alst{f}orna stafi &
\ind ok of \alst{r}agna \alst{r}ǫk.
Nú við \alst{Ó}ðin \hld\ dęilda’k mína \edtrans{\alst{o}rð-spęki}{word-wisdom}{\Bfootnote{The same word-wisdom Weden in st. 5 set out to try.}}; &
\ind þú est ę́ \alst{v}ísastr \edtrans{\alst{v}era}{of men}{\Bfootnote{\emph{verr} means ‘husband, man’ and is here used for reasons of alliteration; it does not imply that Weden is not a God.}}.“\eva

\bvb\speakernoteb{[Webthrithner quoth:]}%
“No man ever knows what thou in days of yore \\
saidst into the ear of the son. \\
With a \inx[C]{fey} mouth have I spoken my ancient \inx[C]{stave}[staves], \\
and about the Rakes of the Reins. \\
Now with Weden have I shared my word-wisdom; \\
thou art ever wisest of men!”\evb\evg

\sectionline
