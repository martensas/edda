\part{Ancient Germanic Charms and Spells}

I have here gathered sundry charms spells; galders and leeds, assembled from sources across the ancient Germanic world. I have generally only included those with clear Heathen elements or contexts, though a few are of Christian origin. The Old Saxon baptismal vow, while explicitly anti-pagan, has also been included due to its mention of Germanic Heathen deities.


\chapter{Continental Germanic spells}

\section{The two Merseburg charms}

\bvg
\bva Eiris \alst{s}ázun idísi \hld\ \alst{s}ázun hera dóder; &
suma \alst{h}apt \alst{h}eptidun \hld\ suma \alst{h}eri lezidun &
suma \alst{c}lubodun \hld\ umbi \alst{c}óniowidi &
\alst{i}nsprinc haptbandun \hld\ \alst{i}nfar fígandun .H.\eva

\bvb Of yore stayed dises, stayed here and there: some fastened fetters, some hindered hosts, some cleaved shackles.—Break the fetter-bonds, flee the fiends! .H.\footnoteB{TODO: note about this strange mark in the ms.}\evb
\evg


\bvg
\bva \edtext{\alst{F}ol}{\lemma{Fol}\Afootnote{\emph{Phol} ms.}} ende Wódan \hld\ \alst{f}órun zi holza &
dú wart demo Balderes \alst{f}olon \hld\ sín \alst{f}óz birenkit &
thú bigól en \alst{S}inthgunt \hld\ \alst{S}unna era swister &
thú bigól en \alst{F}rija \hld\ \alst{F}olla era swister &
thú bigól en \alst{W}ódan \hld\ só hé \alst{w}ola conda &
sóse \alst{b}énrenkí \hld\ sóse \alst{b}lótrenkí &
\ind sóse lidirenkí &
\ind \alst{b}én zi \alst{b}éna &
\ind \alst{b}lót zi \alst{b}lóda &
\alst{l}id zi ge\alst{l}iden \hld\ sóse ge\alst{l}imida sín.\eva

\bvb Phol and Weden journeyed to the woods; then was the foot of Balder’s foal sprained. Then \inx[C]{begale}[begaled] him \inx[P]{Sithguth}, \inx[P]{Sun} her sister; then begaled him \inx[P]{Frie}, \inx[P]{Full} her sister; then begaled him Weden, as he well knew: “Like bone-sprain, like blood-sprain, like joint-sprain! Bone to bone, blood to blood, joint to joints, like were they glued together!”\evb
\evg


\section{Against worms (Contra vermes)}

\bvg
\bva Gang út, \alst{n}esso, \hld\ mid \alst{n}igun \alst{n}essiklínon, &
ut fana themo marge an that bén, &
fan themo béne an that flesg, &
ut fan themo flesgke an thia húd, &
ut fan thera húd an thesa strála. &
Drohtin, werthe só.\eva

\bvb Go out, Nesse, with nine small Nesses! Out from the marrow onto the bone, from this bone onto the flesh, out from the flesh onto the skin, out from the skin onto these arrows. Lord, may it be so.\evb
\evg


\section{The Old Saxon Baptismal vow}

\bpg
\bpa „Forsachistu diobolæ?“ \emph{et respondeat:} „ec forsacho diabolæ“\epa

\bpb “Forsakest thou the Devil?” and he should respond: “I forsake the Devil.”\epb
\epg


\bpg
\bpa „end allum diobol geldę?“ \emph{respondeat:} „end ec forsacho allum
diobol geldæ.“\epa

\bpb “And all Devil-yields?” he should respond: “I forsake all devil-yields.”\epb
\epg


\bpg
\bpa „End allum dioboles wercum?“ \emph{respondeat} „end ec forsacho allum dioboles wercum and wordum, Thunær ende Wóden ende Saxnóte ende allëm them unholdum the hira genótas sint.“\epa

\bpb “And all the works of the Devil?” he should respond: “and I forsake all the works and words of the Devil; Thunder and Weden and Saxneet and all those unhold ones who are their fellows.”\epb
\epg


\bpg
\bpa „Gelóbistu in got alamehtigun fadær?“ „Ec gelóbo in got alamehtigun fadær.“\epa

\bpb “Believest thou in God, the almighty father?” “I believe in God, the almighty father.”\epb
\epg


\bpg
\bpa „Gelóbistu in Crist godes suno?“ „Ec gelóbo in Crist gotes suno.“\epa

\bpb “Believest thou in Christ, God’s son?” “I believe in Christ, God’s son.”\epb
\epg


\bpg
\bpa „Gelóbistu in hálogan gast?“ „Ec gelóbo in hálogan gast.“\epa

\bpb “Believest thou in the Holy Ghost?” “I believe in the Holy Ghost.”\epb
\epg


\chapter{Old English spells}

\section{Against a dwarf}


\section{Wið fǽrstice}

Attested in \Lacnunga.

\bvg
\bva[0]Hlúde wǽran hý, lá, hlúde, \hld\ ðá hý ofer þone hlǽw ridan, &
wǽran ánmóde, \hld\ ðá hý ofer land ridan. &
Scyld ðú ðé nú, þú ðysne níð \hld\ genesan móte. &
Út, lýtel spere, \hld\ gif hér inne síe! &
Stód under linde, \hld\ under leohtum scylde, &
þær ðá mihtigan wíf \hld\ hýra mægen berǽddon &
and hý gyllende \hld\ gáras sændan; &
ic him óðerne \hld\ eft wille sændan, &
fléogende fláne \hld\ forane tógéanes. &
Ut, lytel spere, \hld\ gif hit her inne sy! &
Sæt smið, \hld\ sloh seax &
lytel iserna, \hld\ wund swiðe. &
Ut, lytel spere, \hld\ gif her inne sy! &
Syx smiðas sætan, \hld\ wælspera worhtan. &
Ut, spere, \hld\ næs in, spere! &
Gif her inne sy \hld\ isenes dæl, &
hægtessan geweorc, \hld\ hit sceal gemyltan. &
Gif ðu wære on fell scoten \hld\ oððe wære on flæsc scoten &
oððe wære on blod scoten \hld\ [...] &
oððe wære on lið scoten, \hld\ næfre ne sy ðin lif atæsed; &
gif hit wære esa gescot \hld\ oððe hit wære ylfa gescot &
oððe hit wære hægtessan gescot, \hld\ nu ic wille ðin helpan. &
þis ðe to bote esa gescotes, \hld\ ðis ðe to bote ylfa gescotes, &
ðis ðe to bote hægtessan gescotes; \hld\ ic ðin wille helpan. &
Fleo þær on \hld\ fyrgen-hæfde &
Hal westu, \hld\ helpe ðin drihten. &
Nim þonne þæt seax, ado on wætan.\eva

\bvb[0]Loud were they, lo, loud, when they rode over that mound.\evb
\evg

\section{Nine herbs charm}

\bvg
\bva[0]Gemyne ðú mugwyrt \hld\ hwæt þú ámeldodest &
hwæt þu renadest \hld\ æt Regenmelde?\eva

\bvb Rememberest thou, Mugwort, what thou madest known; what thou arrangedest at Reinmeld?\evb
\evg


\bvg\setlinenum{2}
\bva[0]Una þú hattest \hld\ yldost wyrta &
þú miht wið III \hld\ and wið XXX &
þú miht wiþ attre \hld\ and wið onflyge &
þú miht wiþ þám láþan \hld\ ðe geond lond færð\eva

\bvb thou availest against three and against thirty; thou availest against the venom and against the onflier; thou availest against the loathsome one that goes through the lands.\evb
\evg


\bvg\setlinenum{6}
\bva[0]+ Ond þú wegbráde \hld\ wyrta módor &
éast[a]n op[e]ne \hld\ inn[a]n mihtigu &
ofer ðy cræte curran \hld\ ofer ðy cwéne réodan &
\ind ofer ðy brýde brýodedon &
\ind ofer ðy fearras fnærdon.\eva

\bvb And thou, Waybroad, mother of worts, open from the east, mighty from within. Over thee TODO.\evb
\evg


\bvg\setlinenum{6}
\bva[0]Eallum þu þon wiðstóde \hld\ and wiðstunedest &
swá ðú wiðstonde attre \hld\ and onflyge &
and þǽm láðan \hld\ þe geond lond fereð.\eva

\bvb Them all withstoodest thou then, and stoppedst; so may thou withstand the venom and the onflier, and the loathsome one that goes through the lands.\evb
\evg


\bvg\setlinenum{6}
\bva[0]Stune hætte þéos wyrt, \hld\ héo on stáne geweox &
stond héo wið attre, \hld\ stunað héo wærce &
Stiðe héo hatte, \hld\ wiðstunað héo attre &
wreceð héo wráðan, \hld\ weorpeð út attor\eva

\bvb Ston is this wort called; she grew on stone; she withstands venom, she stops aches. Stithe is she called; she stops venom; she drives away the wroth one; she casts out the venom.\evb
\evg


\bvg\setlinenum{6}
\bva[0]+ Þis is séo wyrt \hld\ séo wiþ wyrm gefeaht &
þéos mæg wið attre, \hld\ héo mæg wið onflyge &
héo mæg wið ðám láþan \hld\ ðe geond lond fereþ\eva

\bvb This is the wort which fought against the worm; this one avails against the venom; she avails against the onflier; she avails against the loathsome one that goes through the lands.\evb
\evg


\bvg\setlinenum{6}
\bva[0]Fleoh þú nú attorláðe, \hld\ séo lǽsse ðá máran &
séo máre þá lǽssan, \hld\ oððæt him beigra bót sý\eva

\bvb TODO\evb
\evg


\bvg\setlinenum{6}
\bva[0]Gemyne þú, mægðe,\hld\ hwæt þú ámeldodest &
hwæt ðú geændadest \hld\ æt Alorforda &
þæt nǽfre for gefloge \hld\ feorh ne gesealde &
syþðan him mon mægðan \hld\ tú mete gegyrede\eva

\bvb TODO\evb
\evg


\bvg\setlinenum{6}
\bva[0]Þis is séo wyrt \hld\ ðe wergulu hatte &
ðás onsænde seolh \hld\ ofer sǽs hrygc &
ondan attres \hld\ óþres tó bóte\eva

\bvb TODO\evb
\evg


\bvg\setlinenum{6}
\bva[0]Ðás VIIII magon \hld\ wið nygon attrum.\eva

\bvb TODO\evb
\evg


\bvg\setlinenum{6}
\bva[0]+ Wyrm cóm snícan, \hld\ toslát hé man &
ðá genam Wóden \hld\ VIIII wuldortánas &
slóh ðá þá nǽddran \hld\ þæt héo on VIIII tófléah &
Þǽr geændade æppel \hld\ and attor &
þæt héo nǽfre ne wolde \hld\ on hús búgan\eva

\bvb A \inx[C]{Worm} came crawling; he tore apart a man. Then took Weden nine glory-twigs; slew then that adder, that it TODO into nine [parts]. There ended apple and venom, that he would never come into a house.\evb
\evg


\bvg\setlinenum{6}
\bva[0]+ Fille and finule, \hld\ felamihtigu twá &
þá wyrte gesceop \hld\ wítig drihten &
hálig on heofonum, \hld\ þá hé hongode &
sette and sænde \hld\ on VII worulde &
earmum and éadigum \hld\ eallum tó bóte\eva

\bvb Fill and Fennel, many-mighty two; those worts shaped the wise lord, holy on heaven, when he hung. He set and sent them onto seven worlds; to the wretched and the wealthy, to all for healing.\evb
\evg


\bvg\setlinenum{6}
\bva[0]Stond héo wið wærce, \hld\ stunað héo wið attre &
séo mæg \edtext{wið III \hld\ \emph{and} wið XXX}{\lemma{wið III and wið XXX ‘against three and against thirty’}\Bfootnote{Formulaic; an uncountable amount; “snakes” are probably understood. This oral formula appears in many folk ballads, viz. (Child) 4EFG, 18B, 20C, 30, 53BCDEIKM, 63EFH, 73I, 97AC, 100AG, 110BGH, 156G, 185A, 187A, 187C, 190A, 192A, 193B, 203C, 211A, 217GHLN, 244A, 268A, 269C, 281ABC. Things described include horses, heads of cattle, warriors, days, years, winters.}} &
wið [féondes] hond \hld\ and wið fǽrbregde &
wið malscrunge \hld\ manra wihta\eva

\bvb against three and against thirty\evb
\evg


\bvg\setlinenum{6}
\bva[0]+ Nu magon þás VIIII wyrta \hld\ wið nygon wuldorgeflogenum &
wið VIIII attrum \hld\ and wið nygon onflygnum &
wið ðý réadan attre, \hld\ wið ðý runlan attre &
wið ðý hwitan attre, \hld\ wið ðý [hæwe]nan attre &
wið ðý geolwan attre, \hld\ wið ðý grénan attre &
wið ðý wonnan attre, \hld\ wið ðý wedenan attre &
wið ðý brúnan attre, \hld\ wið ðý basewan attre &
wið wyrmgeblæd, \hld\ wið wætergeblæd &
wið þorngeblæd, \hld\ wið þystelgeblæd &
wið ýsgeblæd, \hld\ wið attorgeblæd\eva

\bvb Now these nine worts avail against glory-onfliers: against nine venoms and against nine onfliers; against the red venom; against the TODO venom; against the white venom; against the TODO venom; against the yellow venom; against the green venom; against the TODO venom; against the TODo venom; against the brown venom; against the TODO venom; against worm-TODO; against water-TODO; against thorn-TODO; against thistle-TODO; against ice-TODO; against venom-TODO.\evb
\evg


\bvg\setlinenum{6}
\bva[0]Gif ænig attor cume \hld\ éastan fleógan &
oððe ǽnig norðan cume &
oððe ǽnig westan \hld\ ofer werðeóde\eva

\bvb If any venom come from the east, flying; or any come from the north; or any from the west, over man-kind.\evb
\evg


\bvg\setlinenum{6}
\bva[0]+ Críst stód ofer ádle \hld\ ǽngan cundes &
Ic ána wát \hld\ ea rinnende &
þǽr þá nygon nǽdran \hld\ néan behealdað\eva

\bvb TODO\evb
\evg


\bvg\setlinenum{6}
\bva[0]Motan ealle wéoda \hld\ nu wyrtum áspringan &
sǽs tóslúpan, \hld\ eal sealt wæter &
ðonne ic þis attor \hld\ of ðé gebláwe\eva

\bvb TODO\evb
\evg


PROSE SECTION.
Mucgwyrt, wegbrade þe eastan open sy, lombescyrse, attorlaðan, mageðan, netelan, wudusuræppel, fille \& finul, ealde sapan. Gewyrc ða wyrta to duste, mængc wiþ þa sapan and wiþ þæs æpples gor.

wyrc slypan of wætere and of axsan, genim finol, wyl on þære slyppan and beþe mid æggemongc, þonne he þa sealfe on do, ge ær ge æfter.


* Sing þæt galdor on æcre þara wyrta, :III: ær he hy wyrce and on þone æppel ealswa; ond singe þon men in þone muð and in þa earan buta and on ða wunde þæt ilce gealdor, ær he þa sealfe on do :.



\chapter{Old Norse spells}

\section{Charms from Bryggen}

These charms have been found at Bryggen, Bergen, Norway.

\bvg
\bva[B380] \alst{H}ęill sé þú \hld\ ok í \alst{h}ugum góðum; &
\ind \alst{Þ}órr þik \alst{þ}iggi, &
\ind \edtrans{\alst{Ó}ðinn þik \alst{ęi}gi}{‘may Weden own thee’}{\Bfootnote{See note to \Voluspa\ 23.}}.\eva

\bvb Be thou hale, and in good spirits;\footnoteB{A formula also attested in \Hymiskvida\ 41; see there for parallels.} may Thunder receive thee, may Weden own thee.\evb
\evg


\section{Runic plates}
