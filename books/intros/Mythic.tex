\bookStart{Introduction to Mythic Poetry}

This section encompasses all Norse Eddaic narrative poetry concerning the pre-Christian Germanic gods.  That this poetry is exclusively in Old Norse is a matter of preservation, for the Old Norse language is the only Germanic language for which any poetry of this type survives.

\section{Manuscripts}

\subsection{Codex Regius (R)}

By far the most important manuscript is GKS 2365 4to (siglum \Regius), the so-called Codex Regius.  It dates to around 1270 and consists of 45 surviving foll. containing 29 poems.  The ms. itself is divided into two parts or sections; the first (on foll. 1–20, containing 11 poems) dealing mostly with mythology, the second (on foll. 20–45, containing 18 poems) dealing with heroic legend from the Walsing cycle.  Scribal characteristics show that these two parts have been copied from separate source manuscripts, and they are each introduced with a particularly large initial letter. (TODO: cite)

\Regius\ is not a mere anthology of poems, but shows substantial editorial input as well.  Short prose sections tie a group of the mythological poems together into a loose narrative, though it is clear from their meter, style, and language that these poems are separate works composed by various poets over time.  When it comes to the heroic poems long prose segments occur both within and between them, creating a \inx[C]{saw}-like prosimetrical form where the prose sometimes comes to dominate the poetry.  A manuscript closely related to the heroic half of \Regius\ has clearly served as the main source for large swathes of the younger \VolsungaSaga.

A large gap famously occurs in the heroic half; between foll. 32 and 33 one quire has gone missing.  Its contents are mostly unknown, but it would have included the end of \Sigrdrifumal\ and the beginning of the Fragmentary Lay of Siward (TODO).  Some of the stanzas probably contained in it may be restored from the \VolsungaSaga, and these are edited in \emph{Fragments from the Saw of the Walsings} below.  For further literature on \Regius\ see TODO.

\subsection{AM 748 I a 4to (A)}

Second in importance stands AM 748 I a 4to (siglum \AM).  It dates to around 1300 and is in fragmentary state, consisting of just 6 foll.  The beginning and end are absent, and between foll. 2 and 3 there is a lacuna, so that at least 3 (but probably more) foll. are missing.

\AM\ contains seven poems.  On 1r–2v are found in succession the latter half of \Harbardsljod, the full \Baldrsdraumar, and the first half of \Skirnismal.  There is then the lacuna—Finnur Jónsson guesses that just one fol. is missing—and on 3r–6v are found in succession most of \Vafthrudnismal, all of \Grimnismal\ and \Hymiskvida, and the introductory prose to \Volundarkvida.  Among medieval mss., \Baldrsdraumar\ is only attested in \AM, while the other six poems are also found in the first, mythological, part of \Regius. The order of the poems varies drastically between \AM\ and \Regius.

\AM\ has no trace of a frame narrative tying together \Hymiskvida\ and \Lokasenna\ (and indeed the latter poem has left no trace in it), but otherwise \AM\ and \Regius\ do share a substantial amount of prose.  The two mss. generally agree very closely in both prose and poetr, a fact which proves beyond any doubt that the two stem from a common manuscript archetype, rather than being independent witnesses of oral tradition.

The edition of \AM\ here consulted is \textcite{Finnur1896}.

\subsection{Manuscripts of Snorre’s Edda}

The first two sections of Snorre’s Edda—\Gylfaginning\ and \Skaldskaparmal—contain quotations from several mythological Eddic poems.  Snorre reproduces stanzas from (TODO) \Voluspa, \Vafthrudnismal, \Grimnismal, and a variant of \Lokasenna\ (see introduction to that poem) in \Gylfaginning; in additional, the heroic \Grottasongr\ is attested in full in \Skaldskaparmal.  Apart from these known works, Snorre also reprodues a few otherwise unknown stanzas in Eddic meters, which are edited at the end of this section under the heading \emph{Fragments from Snorre’s Edda}.

The four main mss. for the Prose Edda are:%TODO: use table like in the Heliand introduction

\begin{enumerate}
  \item Codex Regius of the Prose Edda (GKS 2367 4to, siglum \RegiusProse), dating to 1300-1350.
  \item Codex Trajectinus (Traj 1374, siglum \Trajectinus), a c. 1595 paper copy of a ms. closely related to \RegiusProse.
  \item Codex Wormianus (AM 242 fol., siglum \Wormianus), dating to 1340–70. \Wormianus\ also contains the \Rigsthula.
  \item Codex Upsaliensis (DG 11, siglum \Upsaliensis), dating to 1300–25.  This mss. is a heavily abbreviated and very poorly done copy of an early ms., which makes its frequent errors even more outrageous.
\end{enumerate}

When all four mss. agree on a reading, the abbreviation \GylfMS\ is used synonymously with \RegiusProse\Trajectinus\Wormianus\Upsaliensis.  For discussion on their internal stemmatics and origins I refer to \textcite{Haukur2017}.

\subsection{Other manuscripts}

A few other Eddic-style poems from various sources are also included in the present edition.  TODO (Svipdagsmál and \Grougaldr) are found only in post-reformation Icelandic paper mss., namely TODO.  While I have not consulted such paper mss. for poems attested in medieval mss., I have had to rely on them for these poems.  About these poems in particular it has to be said that late first \emph{attestation} does not neccessary imply early \emph{composition}.  A good proof of this is \Baldrsdraumar, which is first attested in the fragmentary medieval \AM, and then (with some interpolated stanzas) in much later paper mss.  We cannot exclude that some of these poems would have existed in other lost medieval mss., perhaps even on the now-lost pages of \Regius\ or \AM.
