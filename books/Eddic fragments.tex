\bookStart{Eddic fragments from Snorre’s Edda}

TODO: Discussion on the fragments.

Numerous Eddic verses are quoted in Snorre’s Edda. Most of them come from Eddic poems preserved in other manuscripts, but a few do not. One is attributed to a lost poem (Homedall’s Galder), while the rest are quoted in the context of longer narrative prose sections.

\sectionline

The tone and context of this verse is highly reminding of mythic wisdom contests, especially that of \Vafthrudnismal. It is quoted in \Gylfaginning\ 2, being the second Eddic verse in the text, following \Havamal\ 1 in the same chapter, which is uttered by Yilfer himself when he enters the hall of the Ease (who in \Gylfaginning\ are presented as a group of deceitful sorcerers, rather than gods).

\bpg\bpa Hann sá þrjú hásę́ti ok hvert upp frá ǫðru, ok sátu þrír menn sinn í hverju. Þá spurði hann, hvert nafn hǫfðingja þeira vę́ri. Sá svarar, er hann leiddi inn, at sá, er í inu neðsta hásę́ti sat, var konungr — „ok heitir Hárr, en þar nę́st sá, er heitir Jafnhárr, en sá ofast, er Þriði heitir.“Þá spyrr Hárr komandann, hvárt fleira er erendi hans, en heimill er matr ok drykkr honum sem ǫllum þar í Háva hǫll. Hann segir, at fyrst vill hann spyrja, ef nokkurr er fróðr maðr inni. Hárr segir, at hann komi eigi heill út, nema hann sé fróðari,\epa

\bpb He [= Yilfer] saw three high-seats and each one higher than the other, and sat there three men, one in each seat. Then he asked what the names of those chieftains were. Then High asks the one who is come, whether  \epb\epg

\bvg
\bva ok statt-u \alst{f}ramm \hld\ meðan þú \alst{f}regn &
\ind sitja skal sá es sęgir.\eva

\bvb “and stand forth while thou askest; sit shall he who speaks!”\evb
\evg

\sectionline

\section{Homedall’s Galder (Hęimdallargaldr)}

This mysterious stanza is quoted in \Gylfaginning\ 27, the chapter describing Homedall. The poem is mentioned but not quoted in \Skaldskaparmal\ 15: \emph{Heimdallar hǫfuð heitir sverð. Svá er sagt, at hann var lostinn mannshǫfði í gegnum. Um þat er kveðit í Heimdallar galdri, ok er síðan kallat hǫfuð mjǫtuðr Heimdallar} ‘The sword is called Homedall’s head. So it is said, that he was pierced by a man’s head. Regarding that was sung in Homedall’s galder, and thereafter the head is called Homedall’s bane.’

\bvg
\bva „Níu em’k \alst{m}ǿðra \alst{m}ǫgr, &
níu em’k \alst{s}ystra \alst{s}onr.“\eva

\bvb “I am nine mothers’ lad; I am nine sisters’ son.”\evb
\evg

\sectionline

This passage is closely paralleled in Saxo (TODO). See \textcite{Hopkins2021}.

\bpg\bpa Inn þriði áss er sá, er kallaðr er Njǫrðr. Hann býr á himni, þar sem heitir Nóatún. Hann rę́ðr fyrir gǫngu vinds ok stillir sjá ok eld. Á hann skal heita til sę́fara ok til veiða. Hann er svá auðigr ok fésę́ll, at hann má gefa þeim auð landa eða lausafjár. Á hann skal til þess heita. Eigi er Njǫrðr ása ę́ttar. Hann var upp fę́ddr í Vanaheimi, en Vanir gísluðu hann goðunum ok tóku í mót at gíslingu þann, er Hę́nir heitir. Hann varð at sę́tt með goðum ok Vǫnum. Njǫrðr á þá konu, er Skaði heitir, dóttir Þjaza jǫtuns. Skaði vill hafa bústað þann, er átt hafði faðir hennar, þat er á fjǫllum nǫkkurum, þar sem heitir Þrymheimr, en Njǫrðr vill vera nę́r sę́. Þau sę́ttust á þat, at þau skyldu vera níu nę́tr í Þrymheimi, en þá aðrar níu at Nóatúnum. En er Njǫrðr kom aftr til Nóatúna af fjallinu, þá kvað hann þetta:\epa

\bpb The third Os is that one who is called Nearth. He lives in heaven, there as is called Nowetowns. He rules the motion of the wind and calms sea and fire. Upon him shall one call for sea-faring and for hunting. He is so wealthy and blessed with cattle that he may give them a wealth of lands or loose cattle. Upon him shall one call for that. Nearth is not of the lineage of the Ease. He was brought up in Wanehome, but the Wanes gave him as a hostage towards the gods and received as a hostage that one who is called Heener. He was used for reconciling the gods and the Wanes. Nearth has that woman who is called Shede, the daughter of the ettin Thedse. Shede wishes to have the dwelling place which her father had owned, which lies on some certain fells in the place called Thrimham, but Nearth wishes to be near the sea. They agreed to it that they would be for nine nights in Thrimham, but the other nine at Nowetowns. But when Nearth came back to the Nowetowns from the fell, then he quoth this:\epb\epg

\bvg
\bva „\alst{L}ęið erumk fjǫll, \hld\ vas’k-a \alst{l}ęngi á, &
\ind \alst{n}ę́tr ęinar \alst{n}íu; &
\alst{u}lfa þytr \hld\ mér þótti \alst{i}llr vesa &
\ind hjá \alst{s}ǫngvi \alst{s}vana.“\eva

\bvb “The fells are loathsome to me; I was not long on them—only for nine nights. The howling of the wolves thought me bad, held against the song of the swans.”\evb
\evg

\bpg\bpa Þá kvað Skaði þetta:\epa

\bpb Then Shede quoth this:\epb\epg

\bvg
\bva „\alst{S}ofa né mát’k-a’k \hld\ \alst{s}ę́var bęðjum á &
\ind \alst{f}ugls jarmi \alst{f}yrir; &
sá mik \alst{v}ękr \hld\ es af \alst{v}íði kømr &
\ind \alst{m}orgun hvęrjan \alst{m}ár.“\eva

\bvb “I could not sleep on the beds of the sea due to the bleating of the bird. That one wakes me when from the wide sea it comes, every morning, the mew.”\evb
\evg

\bpg\bpa Þá fór Skaði upp á fjall ok byggði í Þrymheimi, ok ferr hon mjǫk á skíðum ok með boga ok skýtr dýr. Hon heitir ǫndurgoð eða ǫndurdís.\epa

\bpb Then Shede went up to the fells and dwelled in Thrimham, and she often goes on skis with her bow and shoots beasts. She is called ski-god or ski-dise.\epb\epg

\sectionline

\bpg\bpa Þá fór Þórr til ár þeirar, er Vimur heitir, allra á mest. Þá spennti hann sik megingjǫrðum ok studdi forstreymis Gríðarvǫl, en Loki helt undir megingjarðar. Ok þá er Þórr kom á miðja ána, þá óx svá mjǫk áin, at uppi braut á ǫxl honum. Þá kvað Þórr þetta:\epa

\bpb Then Thunder journeyed to that river which is called Wimbre, the greatest of all rivers. then he fastened his strength-girdle and leaned upon Grith’s stave against the stream, and Lock held the strength-girdle. And when Thunder came to the middle of the river, then it grew so great that it came up unto his shoulders. Then Thunder quoth this:\epb\epg

\bvg
\bva „\alst{V}ax-at-tu nú, \alst{V}imur, \hld\ alls mik þik \alst{v}aða tíðir &
\ind \alst{jǫ}tna garða \alst{í}; &
\alst{v}ęizt, ef þú \alst{v}ęx \hld\ at þá \alst{v}ęx mér ǫ́smęgin &
\ind jafn\alst{h}átt upp sem \alst{h}iminn.“\eva

\bvb “Grow thou not now, Wimbre, as I wish to wade through thee into the yards of the ettins; know that if thou growest, that my os-might then grows as high as heaven.”\evb
\evg

\bpg\bpa Þá sér Þórr uppi í gljúfrum nǫkkurum, at Gjálp, dóttir Geirrǫðar stóð þar tveim megin árinnar, ok gerði hon árvǫxtinn. Þá tók Þórr upp ór ánni stein mikinn ok kastaði at henni ok mę́lti svá: „At ósi skal á stemma.“ Eigi missti hann, þar er hann kastaði til. Ok í því bili bar hann at landi ok fekk tekit reynirunn nǫkkurn ok steig svá ór ánni. Því er þat orðtak haft, at reynir er bjǫrg Þórs.\epa

\bpb Then Thunder sees above in some gorges, that Yelp, daughter of Garfrith stood there on either side of the river, and she caused it to grow. Then Thunder took up out of the river a great stone, and threw it at her and spoke thus: “At its source shall a river be dammed!” He did not miss his target. And in that moment he came on land and grasped ahold of a certain rowan-branch and thus stepped out of the river. Thus it is a saying that the rowan is Thunder’s deliverance.\epb\epg

\sectionline

This additional verse is only found in \Upsaliensis, but seems in all regards like an old Eddic verse and has thus been included.

\bvg
\bva „\alst{Ęi}nu \edtrans{\emph{sinni}}{time}{\Bfootnote{emend.; om. \Upsaliensis}} \hld\ nęytta’k \alst{a}lls męgins &
\ind \alst{jǫ}tna gǫrðum \alst{í} &
þá’s \alst{G}jǫlp ok \alst{G}ręip, \hld\ dǿtr \alst{G}ęirraðar, &
\ind vildu \alst{h}ęfja mik til \alst{h}imins“\eva

\bvb “A single time I used all [my] strength in the yards of the ettins: When Yelp and Grope, Garfrith’s daughters, wished to lift me [up] to heaven.”\evb
\evg
