\bookStart{Eddic fragments from Snorre’s Edda}
%TODO: Further discussion on the fragments.

A number of Eddic lines, stanzas and groups of stanzas are quoted in Snorre’s Edda.  The majority of them are taken from longer Eddic poems preserved in full in other manuscripts (primarily \Regius\ and \AM), but a few are found nowhere else.  These fragments will be edited in the present section.

The fragments have some things in common: they are generally pieces of spoken dialogue quoted in the context of longer narrative prose sections, and are, with one exception (Homedal’s galder, see below), not introduced by reference to their source but rather with phrases like \emph{þá kvað hann} ‘then he quoth’.

\sectionline

\section{A lost riddle-poem}

This half-stanza is quoted in \Gylfaginning\ 2, being the second Eddic verse in the text, following \Havamal\ 1 in the same chapter, which is uttered by Yilfer himself when he enters the hall of the Eese. The whole section is clearly referencing other Eddic mythic wisdom contests and particularly reminiscent of \Vafthrudnismal.

\bpg\bpa Hann sá þrjú há-sę́ti ok hvert upp frá ǫðru, ok sátu þrír menn sinn í hverju. Þá spurði hann, hvert nafn hǫfðingja þeira vę́ri. Sá svarar, er hann leiddi inn, at sá, er í inu neðsta hásę́ti sat, var konungr, ok heitir Hárr, en þar nę́st sá, er heitir Jafnhárr, en sá ofast, er Þriði heitir. Þá spyrr Hárr komandann, hvárt fleira er erendi hans, en heimill er matr ok drykkr honum sem ǫllum þar í Háva hǫll. Hann segir, at fyrst vill hann spyrja, ef nǫkkurr er fróðr maðr inni. Hárr segir, at hann komi eigi heill út, nema hann sé fróðari,\epa

\bpb He [= Yilfer] saw three high-seats and each higher than the other, and three men sat there, each in his own seat. Then he asked what the names of those chieftains were. He who led him in answers that the one who sat in the lowest high-seat was a king called High, and next to him he who is called Evenhigh, and uppermost he who is called Third. Then High asks the guest whether he has any other errands, but food and drink will be freely offered him, like all men there in the High One’s hall. He [= Yilfer] asks whether anyone within is a learned man.  High says that he will not come out whole unless he be more learned [than he],\epb\epg

\bvg\bva „ok statt-u \alst{f}ramm \hld\ meðan þú \alst{f}regn &
\ind \alst{s}itja skal \alst{s}á es \alst{s}ęgir.“\eva

\bvb “and stand forth while thou askest; \\
\ind sit shall he who speaks!”\evb\evg

\sectionline

\section{Nearth and Shede}

The following passage is almost the whole of \Gylfaginning\ 23, excepting at the very end \emph{svá er sagt} ‘so it is said’, after which is quoted \Grimnismal\ 11.
Notably, the two stanzas cited here are also found translated in \textcite{Saxo}[1.8.18--19], where they are said to have been spoken by Hadding and Rainhild, respectively.  For discussion \textcite{Hopkins2021}.

\sectionline

\bpg\bpa Inn þriði áss er sá, er kallaðr er Njǫrðr. Hann býr á himni, þar sem heitir Nóatún. Hann rę́ðr fyrir gǫngu vinds ok stillir sjá ok eld. Á hann skal heita til sę́-fara ok til veiða. Hann er svá auðigr ok fé-sę́ll, at hann má gefa þeim auð, landa eða lausa-fjár. Á hann skal til þess heita. Eigi er Njǫrðr ása ę́ttar. Hann var upp fǿddr í Vana-heimi, en Vanir gísluðu hann goðunum ok tóku í mót at gíslingu þann, er Hǿnir heitir. Hann varð at sę́tt með goðum ok Vǫnum. Njǫrðr á þá konu, er Skaði heitir, dóttir Þjatsa jǫtuns. Skaði vill hafa bú-stað þann, er átt hafði faðir hennar, þat er á fjǫllum nǫkkurum, þar sem heitir Þrym-heimr, en Njǫrðr vill vera nę́r sę́. Þau sę́ttust á þat, at þau skyldu vera níu nę́tr í Þrym-heimi, en þá aðrar níu at Nóa-túnum. En er Njǫrðr kom aftr til Nóatúna af fjallinu, þá kvað hann þetta:\epa

\bpb The third Os is that one who is called Nearth. He lives in the heaven in the place called Nowetowns. He rules the course of the wind, and stills sea and fire. On him shall one call for sea-faring and for hunting. He is so wealthy and blessed with money that he may give them a wealth of lands or loose property; on him shall one call for that sake. Nearth is not of the lineage of the Eese. He was brought up in Wanehome, but the Wanes gave him as a hostage to the gods, and in return got as hostage that one who is called Heener. He was used to reconcile the gods and the Wanes. Nearth has that woman who is called Shede, the daughter of the ettin Thedse. Shede wishes to have the dwelling which her father had owned, which lies on some fells in the place called Thrimham—but Nearth wishes to live by the sea. They agreed with each other that they would live for nine nights in Thrimham, but the other nine at Nowetowns. But when Nearth came back to the Nowetowns from the fell, he quoth this:\epb\epg

\bvg\bva „\alst{L}ęið erumk fjǫll, \hld\ vas’k-a \alst{l}ęngi á, &
\ind \alst{n}ę́tr ęinar \alst{n}íu; &
\alst{u}lfa þytr \hld\ mér þótti \alst{i}llr vesa &
\ind hjá \alst{s}ǫngvi \alst{s}vana.“\eva

\bvb “Loathsome are the fells for me; I was not long thereon— \\
\ind but for nine nights. \\
The wolves’ howl seemed me evil \\
\ind next to the song of swans.”\evb\evg

\bpg\bpa Þá kvað Skaði þetta:\epa

\bpb Then Shede quoth this:\epb\epg

\bvg\bva „\alst{S}ofa né mát’k-a’k \hld\ \alst{s}ę́var bęðjum á &
\ind \alst{f}ugls jarmi \alst{f}yrir; &
sá mik \alst{v}ękr \hld\ es af \alst{v}íði kømr &
\ind \alst{m}orgun hvęrjan \alst{m}ár.“\eva

\bvb “I could not sleep on the beds of the sea \\
\ind for the bleating of the bird. \\
He awakes me, when from the wide sea he comes, \\
\ind every morning, the mew.”\evb\evg

\bpg\bpa Þá fór Skaði upp á fjall ok byggði í Þrym-heimi, ok ferr hon mjǫk á skíðum ok með boga ok skýtr dýr. Hon heitir ǫndur-goð eða ǫndur-dís.\epa

\bpb Then Shede went up to the fells and dwelled in Thrimham, and she often goes on skis with her bow and shoots beasts. She is called ski-god or ski-dise.\epb\epg

\sectionline

\section{Homedal’s Galder (\emph{Hęimdallargaldr})}

This mysterious fragment is quoted in \Gylfaginning\ 27, the chapter describing Homedal, which is here reproduced in full. The fragment consists of two c-lines and appears to be the end of a stanza in the fitting meter \Galdralag.

The same poem is mentioned again in \Skaldskaparmal\ 15: \emph{Heimdallar hǫfuð heitir sverð. Svá er sagt, at hann var lostinn manns hǫfði í gegnum. Um þat er kveðit í Heimdallar-galdri, ok er síðan kallat hǫfuð mjǫtuðr Heimdallar} ‘A sword is called Homedal’s head. So is said that he was run through with a man’s head.  About that it is sung in Homedal’s galder, and henceforth the head is called Homedal’s bane.’

\sectionline

\bpg\bpa Heimdallr heitir einn. Hann er kallaðr hvíti áss; hann er mikill ok heilagr. Hann báru at syni meyjar níu ok allar systr; hann heitir ok Hallinskíði ok Gullintanni; tennr hans váru af gulli. Hestr hans heitir Gulltoppr. Hann býr þar er heitir Himinbjǫrg við Bifrǫst; hann er vǫrðr goða ok sitr þar við himins enda at gę́ta brúarinnar fyrir berg-risum. Hann þarf minna svefn en fugl. Hann sér jafnt nótt sem dag hundrað rasta frá sér; hann heyrir ok þat, er gras vex á jǫrðu eða ull á sauðum, ok allt þat er hę́ra lę́tr. Hann hefir lúðr þann er Gjallar-horn heitir, ok heyrir blástr hans í alla heima. Heimdallar sverð er kallat hǫfuð manns. Hér er svá sagt:\epa

\bpb Homedal one is named.  He is called the White Os; he is great and holy.  He was born as the son of nine maidens, sisters all.  He is also named Haldenshid and Goldentooth; his tooth were of gold.  His horse is called Goldtop.  He lives at the place called the Heavenbarrows near Bivrest.  He is the Watchman of the Gods and sits there at Heaven’s end to guard the bridge against barrow-risers.  He needs less sleep than a bird.  Both night and day he sees a hundred rests away from him; he also hear when grass grows on the ground or wool on sheep, and everything which sounds louder. He has the basoon called the Horn of Yell, and his blowing can be heard in all realms.  Homedal’s sword is called a man’s head.  Here it says so:\epb\epg

\sectionline

(Here the text cites \Grimnismal\ 13; see there.)

\sectionline

\bpg\bpa Ok enn segir hann sjalfr í Heimdallar-galdri:\epa

\bpb And further he himself says in Homedal’s Galder:\epb\epg


\bvg\bva „Níu em’k \edtrans{\alst{m}ǿðra}{mothers}{\Afootnote{so \RegiusProse\Trajectinus\Wormianus; \emph{męyja} ‘maidens’ \Upsaliensis}} \alst{m}ǫgr, &
níu em’k \alst{s}ystra \edtrans{\alst{s}onr}{son}{\Afootnote{om. \Trajectinus}}.“\eva

\bvb “Of nine mothers I’m the lad, \\
of nine sisters I’m the son.”\evb\evg

\sectionline

\section{Gna and the Wanes}

The following passage is from \Gylfaginning\ 35, which lists the \inx[G]{Ossens}.

\sectionline

\bpg\bpa Fjórtánda Gná, hana sendir Frigg í ymsa heima at ørindum sínum. Hon á þann hest, er renn lopt ok lǫg, er heitir Hóf-varpnir. Þat var eitt sinn, er hon reið, at vanir nǫkkvǫrir sá reið hennar í loptinu. Þa mę́lti einn:\epa

\bpb The fourteenth is Gna; Frie sends her into every home to do her errands. She owns the horse who runs through air and sea, and is called Hoofwarpner. It was one time when she rode that some Wanes saw her riding in the air. Then one spoke:\epb\epg

\bvg\bva „Hvat þar \alst{f}lýgr, \hld\ hvat þar \alst{f}ęrr, &
\ind eða at \alst{l}opti \alst{l}íðr?“\eva

\bvb “What flies there, what fares there, \\
\ind or passes through the air?”\evb\evg


\bpg\bpa Hon svarar:\epa

\bpb She answers:\epb\epg


\bvg\bva „Né ek \alst{f}lýg, \hld\ þó ek \alst{f}ęr &
\ind ok at \alst{l}opti \alst{l}ið’k &
á \alst{H}óf-varpni, \hld\ þęim’s \alst{H}am-skęrpir &
\ind \alst{g}at við \alst{G}arð-rofu.“\eva

\bvb “I fly not, though I fare, \\
\ind and pass through the air, \\
on Hoofwarpner, whom Hamsherper \\
\ind begot with Yardrove.”\evb\evg


\bpg\bpa Af Gnár nafni er svá kallat, at þat gnę́far, er hátt ferr:\epa

\bpb From Gna’s name it is so called that something which fares high up \emph{protrudes}.\epb\epg

\sectionline

\section{Balder’s Death}

\Gylfaginning\ 49 contains the narrative of Balder’s death, beginning with his ominous dreams, and ending with the Eese failing to “weep him out of Hell” (for a summary and discussion of the myth and its attestations, see the introduction to \Voluspa\ 31–33). At the end of the chapter, a single \Ljodahattr\ speech-stanza is quoted.

\sectionline

\bpg\bpa Því nę́st sendu ę́sir um allan heim ørind-reka at biðja, at Baldr vę́ri grátinn ór Helju, en allir gerðu þat, menninir ok kykvendin ok jǫrðin ok steinarnir ok tré ok allr málmr, svá sem þú munt sét hafa, at þessir lutir gráta, þá er þeir koma ór frosti ok í hita. Þá er sendi-menn fóru heim ok hǫfðu vel rekit sín ørindi, finna þeir í helli nǫkkvǫrum, hvar gýgr sat; hon nefndist Þǫkk. Þeir biðja hana gráta Baldr ór helju, hon segir:\epa

\bpb Next after that the Eese sent an errand-runner through all the \inx[C]{Home}, to ask that Balder be wept out of hell. And all did that, the men and the beasts and the earth and the stones and trees and all bedrock, as thou must have seen, that these things weep when they come out of cold and into heat. When the messengers journeyed home, and had ran their errand well, they find in a certain cave that a \inx[C]{gow} sat there; she called herself Thanks. They ask her to weep Balder out of hell. She says:\epb\epg


\bvg\bva „\alst{Þ}ǫkk mun gráta \hld\ \alst{þ}urrum tǫ́rum &
\ind \alst{B}aldrs \alst{b}ál-farar; &
\alst{k}yks né dauðs \hld\ naut’k-a \alst{K}arls sonar &
\ind \alst{h}afi \alst{H}ęl því’s \alst{h}ęfir.“\eva

\bvb “Thanks will weep–with dry tears \\
\ind for Balder’s pyre-journey \ken{death}. \\
Neither alive nor dead did I benefit from Churl’s son \ken*{= Balder}; \\
\ind let Hell have what she has!”\evb\evg


\bpg\bpa En þess geta menn, at þar hafi verit Loki Laufeyjarson, er flest hefir illt gørt með ásum.\epa

\bpb But men guess that this must have been Lock, Leafy’s son, who has done the most evil among the Eese.\epb\epg

\sectionline

\section{Thunder’s Journey to Garfrith}

\Skaldskaparmal\ 26, here edited in part, is the only surviving retelling of Thunder’s journey to the ettin Garfrith, and his following fight with, and slaying of, him and his two daughters, Yelp and Grope. This was apparently a well-known story, and is also mentioned in Vetrl Lv 1/1b (quoted in \Skaldskaparmal\ 11, which lists kennings for Thunder): \emph{stétt of Gjǫlp dauða} ‘thou didst step over the dead Yelp’.
The prose of \Skaldskaparmal\ 26 seems to be based on an earlier, now-lost poem in \Ljodahattr, from which it quotes two stanzas. The first is found in all four main manuscripts, while the second is found only in \Upsaliensis. Both are spoken by Thunder and closely resemble each other stylistically, which is why they most likely come from the same poem.

\sectionline

\bpg\bpa Þá fór Þórr til ár þeirar, er Vimur heitir, allra á mest. Þá spennti hann sik megin-gjǫrðum ok studdi for-streymis Gríðar-vǫl, en Loki helt undir megin-gjarðar. Ok þá er Þórr kom á miðja ána, þá óx svá mjǫk áin, at uppi braut á ǫxl honum. Þá kvað Þórr þetta:\epa

\bpb Then Thunder journeyed to that river which is called Wimbre, greatest of all rivers. Then he wrapped his might-girdle around himself and leaned upon Grith’s stave against the stream, and Lock held up the might-girdle. And when Thunder came to the middle of the river, then it waxed so great that it broke over his shoulders. Then Thunder quoth this:\epb\epg


\bvg\bva „\alst{V}ax-at-tu nú, \alst{V}imur, \hld\ alls mik þik \alst{v}aða tíðir &
\ind \alst{jǫ}tna garða \alst{í}; &
\alst{v}ęitst, ef þú \alst{v}ęx \hld\ at þá \alst{v}ęx mér ǫ́s-męgin &
\ind jafn-\alst{h}átt upp sem \alst{h}iminn.“\eva

\bvb “Wax not now, O Wimbre, as I wish to wade through thee \\
\ind into the yards of the ettins. \\
Thou knowest, if thou waxest, then my os-might waxes \\
\ind up as high as the heaven.”\evb\evg


\bpg\bpa Þá sér Þórr uppi í gljúfrum nǫkkurum, at Gjálp, dóttir Geirrøðar \edtrans{stóð þar tveim megin árinnar, ok gerði hon ár-vǫxtinn.}{stood on both sides of the river, and she caused the river’s growth}{\Bfootnote{She stood with her legs spread and befouled the river.}} Þá tók Þórr upp ór ánni stein mikinn ok kastaði at henni ok mę́lti svá: „At ósi skal á stemma.“ Eigi missti hann, þar er hann kastaði til, ok í því bili bar hann at landi ok fekk tekit reyni-runn nǫkkurn ok steig svá ór ánni. Því er þat orð-tak haft, at reynir er bjǫrg Þórs.\epa

\bpb Then Thunder sees that up in some certain gorges Yelp, daughter of Garfrith, stood on both sides of the river, and she caused the river’s growth. Then Thunder took up from the river a great stone and threw it at her and spoke so: “At its source shall the river be dammed.” He did not miss his target, and in that moment he threw himself towards land and got hold of a certain rowan shrub, and thus stepped out of the river. From this comes the saying that the rowan is Thunder’s deliverance.\epb\epg


\bpg\bpa En er Þórr kom til Geirrøðar, þá var þeim fé-lǫgum vísat fyrst í geita-hús til her-bergis, ok var þar einn stóll til sę́tis, ok sat Þórr þar. Þá varð hann þess varr, at stóllinn fór undir honum upp at rę́fri. Hann stakk Gríðar-veli upp í raftana ok lét sígast fast á stólinn. Varð þá brestr mikill, ok fylgði skrę́kr. Þar hǫfðu verit undir stólinum dǿtr Geirrøðar, Gjálp ok Greip, ok hafði hann brotit hrygginn í báðum. Þa kvað Þórr:\epa

\bpb And when Thunder came to Garfrith’s home the fellows were first shown into a goathouse for lodgings, and therein one chair was for sitting, and Thunder sat down on it. Then he noticed that the chair beneath him was moving up toward the roof. He thrusted Grith’s stave up against the rafters and made it push firm onto the chair. Then there was a great crack, followed by a shriek; there beneath the chair had been the daughters of Garfrith, Yelp and Grope, and he had broken both their backs. Then Thunder quoth:\epb\epg

\bvg\bva „\alst{Ęi}nu \edtrans{\emph{sinni}}{time}{\Bfootnote{metr. and sens. emend.; om. \Upsaliensis}} \hld\ nęytta’k \alst{a}lls męgins &
\ind \alst{jǫ}tna gǫrðum \alst{í} &
þá’s \alst{G}jǫlp ok \alst{G}ręip, \hld\ dǿtr \alst{G}ęir-raðar, &
\ind vildu \alst{h}ęfja mik til \alst{h}imins.“\eva

\bvb “Only one time I used all my might \\
\ind in the yards of the ettins, \\
when Yelp and Grope, daughters of Garfrith, \\
\ind would lift me to the heaven.”\evb\evg

\sectionline

\section{On the Making of Glapner}

The following stanza about the making of Glapner, the fetter used to bind the Fenrerswolf, is found in the short work on kennings today called the \emph{Little Scalda} (\emph{Lítla skálda}), which text was probably used as a source by Snorre; see further \textcite[129--47]{Males2020}.  A variant of this stanza is transparently paraphrased in \Gylfaginning\ 28: \emph{Hann var gǫrr af sex hlutum: af dyn kattarins ok af skeggi konunnar ok af rótum bjargsins ok af sinum bjarnarins ok af anda fisksins ok af fogls hráka.} ‘It [Glapner] was made of six things: of the cat’s din and of the woman’s beard and of the mountain’s root and of the bear’s sinews and of the fish’s breath and of the fowl’s spittle.’  The two differences—\emph{hráka} ‘spittle’ for \emph{mjǫlk} ‘milk’, and the inverted order of lines 2 and 3—suggest that Snorre had access to a somewhat different version.  It is not attributed to any named poem.

\sectionline

\bvg\bva Ór \alst{k}attar dyn \hld\ ok ór \alst{k}onu skeggi, &
ór \alst{f}isks anda \hld\ ok ór \alst{f}ugla mjǫlk, &
ór \alst{b}ergs rótum \hld\ ok \alst{b}jarnar sinum, &
\ind ór því vas hann \alst{G}leipnir \alst{g}ǫrr.\eva

\bvb “From cat’s din and from woman’s beard; \\
from fish’s breath and from fowls’ milk; \\
from mountain’s roots and bear’s sinews; \\
\ind from this was Glapner made.”\evb\evg

\sectionline
