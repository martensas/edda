\bookStart{Walder}[Waldhere]

\begin{flushright}%
\textbf{Dating:} TODO

\textbf{Meter:} \Fornyrdislag%para
\end{flushright}%

A heroic poem preserved in two fragments.  The flyting between the heroes Walder and Guther in fragment 2 is very reminiscent of the dialogue in \Hildebrandslied.

For the manuscript I have inspected the digital facsimile at https://digipal.eu/digipal/page/1072/.

\sectionline

\bvg\bva hyrde hyne georne:
„Huru Welande... \hld\ worc ne geswiceð
monna ænigum \hld\ ðara ðe Mimming can
heardne gehealdan. \hld\ Oft æt hilde gedreas
swatfag and sweordwund \hld\ secg æfter oðrum.
ætlan ordwyga, \hld\ ne læt ðin ellen nu gyt
gedreosan to dæge, \hld\ dryhtscipe
[nú] is se dæg cumen
þæt ðu scealt aninga \hld\ oðer twega,
lif forleosan \hld\ oððe langne dóm
âgan mid ęldum, \hld\ Ælf-hęres sunu!
Nalles ic ðé, wine mín, \hld\ wordum cide,
ðy ic ðé ge·sáwe \hld\ æt ðam sweord-plegan
ðurh edwit-scype \hld\ æniges mǫnnes
wíg for·bugan \hld\ oððe on weal fleon,
líce beorgan, \hld\ ðeah þe lâðra fela
ðinne byrn-hǫmon \hld\ billum heowun,
ac ðu symle furðor \hld\ feohtan sóhtest,
mǽl ofer mearce; \hld\ ðy ic ðe metod on·dréd,
þæt ðu to fyren-líce \hld\ feohtan sóhtest
æt ðam æt-stealle \hld\ oðres monnes,
wíg-rǽdenne. \hld\ Weorða ðe selfne
gódum dǽdum, \hld\ ðenden ðin god ręcce.
Ne murn ðu for ði méce; \hld\ ðe wearð mâðma cyst
gifeðe to geoce, \hld\ mid ðy ðú Gu̇ðhęre scealt
beot for·bigan, \hld\ ðæs ðe he ðas beaduwe on·gan
...d unryhte \hld\ ǽrest sécan.
For-sóc he ðam swurde \hld\ and ðam sync-fatum,
béaga mænigo, \hld\ nu sceal béaga-léas
hworfan from ðisse hilde, \hld\ hlâfurd sécan
ealdne éðel \hld\ oððe hér ǽr swefan,
gif he ða [...]“\eva

\bvb TODO.\evb\evg

\sectionline

\bvg\bva „...ce bæteran
b·úton ðam ânum \hld\ ðe ic eac hafa
on stân-fate \hld\ stille ge·hided.
Ic wât þæt hit ðóhte \hld\ Ðeodric Widian
selfum on·sendon, \hld\ and eac sinc micel
mâðma mid ði méce, \hld\ monig oðres mid him
golde ge·girwan \hld\ (iulean ge·nam),
þæs ðe hine of nearwum \hld\ Níðhades mǽg,
Welandes bearn, \hld\ Widia ut forlet;
ðurh fifela geweald \hld\ forð on·ętte.“
Waldere maðelode, \hld\ wíga ęllen-rof,
hæfde him on handa \hld\ hilde-frófre,
gu̇ð-billa gripe, \hld\ gyddode wordum:
„Hwæt, ðu húru wéndest, \hld\ wine Burgenda,
þæt me Hagenan hand \hld\ hilde ge·fremede
and getwæmde ...ðewigges. \hld\ Feta, gyf ðu dyrre,
æt ðus heaðu-węrigan \hld\ hâre byrnan.
Standeð me hér on eaxelum \hld\ Ælfheres lâf,
gód and géap-neb, \hld\ golde ge·weorðod,
ealles un-scende \hld\ æðelinges réaf
to habbanne, \hld\ þonne hand węreð
feorh-hord feondum. \hld\ Ne bið fah wið mé,
þonne ...... un-mǽgas \hld\ ęft on·gynnað,
mécum ge·metað, \hld\ swá gé mé dydon.
Ðeah mæg sige syllan \hld\ se ðe symle byð
recon and rǽd-fęst \hld\ ryh... ...a ge·hwilces.
Se ðe him to ðam hâlgan \hld\ helpe ge·lifeð,
to gode gioce, \hld\ hé þær gearo findeð
gif ða earnunga \hld\ ǽr ge·ðenceð.
Þonne moten wlance \hld\ welan britnian,
æhtum wealdan, \hld\ þæt is [...]“\eva

\bvb TODO.\evb\evg

\sectionline
