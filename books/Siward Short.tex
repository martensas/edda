\bookStart{Short Lay of Siward}[Sigurðarkviða in skǫmmu]

\begin{flushright}%
\textbf{Dating} \parencite{Sapp2022}: early C11th (0.876)

\textbf{Meter:} \Fornyrdislag%
\end{flushright}

\section{Introduction}

Despite its title it is one of the longer poems, having approximately 300 long-lines.

\sectionline

\section{Short Lay of Siward}

\bvg\bva%
Ár vas þat’s Sigurðr \hld\ sótti Gjúka &
vǫlsungr ungi \hld\ es vegit hafði &
tók við tryggðum \hld\ tvęggja brǿðra &
sęldusk ęiða \hld\ ęljun-frǿknir.\eva

\bvb TODO: Translation.\evb\evg


\bvg\bva%
Męy buðu hǫ́num \hld\ ok męiðma fjǫlð &
Guðrúnu ungu \hld\ Gjúka dóttur &
drukku ok dǿmðu \hld\ dǿgr mart saman &
Sigurðr ungi \hld\ ok synir Gjúka.\eva

\bvb TODO: Translation.\evb\evg


\bvg\bva%
Unds þęir Brynhildar \hld\ biðja fóru &
svá’t þęim Sigurðr \hld\ ręið ï sinni &
vǫlsungr ungi \hld\ ok vega kunni; &
hann of ę́tti \hld\ ef hann ęiga knę́tti.\eva

\bvb TODO: Translation.\evb\evg


\bvg\bva%
Sęggr inn suðr-ǿni \hld\ lagði sverð nøkkvit &
mę́ki mál-fáan \hld\ ȧ meðal þęira &
né han konu \hld\ kyssa gęrði &
né húnskr konungr \hld\ hęfja sér af armi &
męy frum-unga \hld\ fal hann męgi Gjúka.\eva

\bvb TODO: Translation.\evb\evg


\bvg\bva%
Hǫ́n sér at lifi \hld\ lǫst ne vissi &
ok at aldr-lagi \hld\ ekki grand &
vamm þat’s vę́ri \hld\ eða vesa hygði; &
gengu þess ȧ milli \hld\ grimmar urðir.\eva

\bvb TODO: Translation.\evb\evg


\bvg\bva%
Ęin sat hon úti \hld\ aptan dags, &
nam hǫ́n svá bęrt \hld\ um at mę́lask: &
„Hafa skal’k Sigurð, \hld\ — eða þó svelti!— &
mǫg frum-ungan, \hld\ mér ȧ armi.\eva

\bvb TODO: Translation.\evb\evg


\bvg\bva%
\alst{O}rð mę́lta’k nú, \hld\ iðrumk \alst{ę}ptir þess, &
kvǫ́n ’s hans \alst{G}uðrún \hld\ en ek \alst{G}unnars, &
\alst{l}jótar nornir \hld\ skópu oss \alst{l}anga þrǫ́.\eva

\bvb Words I now spoke; I regret them afterwards. \\
His wife is Guthrun, but I am Guther’s; \\
ugly norns shaped for us a long yearning.\evb\evg


\bvg\bva%
STANZATEXT\eva

\bvb TODO: Translation.\evb\evg


\bvg\bva%
STANZATEXT\eva

\bvb TODO: Translation.\evb\evg


TODO: More stanzas

\sectionline
