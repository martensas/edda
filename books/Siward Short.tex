\bookStart{Short Lay of Siward}[Sig·urðar kviða in skǫmmu]
\setBookCode{Sigurdskamma}

\begin{flushright}%
\textbf{Dating} \parencite{Sapp2022}: early C11th (0.876)

\textbf{Meter:} \Fornyrdislag%
\end{flushright}

\section{Introduction}

Despite its title it is one of the longer poems, having approximately 300 long-lines.

\sectionline

\section{Short Lay of Siward}

\bvg\bva%
Ár vas þat’s \alst{S}ig·urðr \hld\ \alst{s}ótti Gjúka &
\alst{v}ǫlsungr ungi \hld\ es \alst{v}egit hafði; &
\alst{t}ók við \alst{t}ryggðum \hld\ \alst{t}vęggja brǿðra &
sęldusk \alst{ęi}ða \hld\ \alst{ę}ljun-frǿknir.\eva

\bvb It was of yore when Siward sought out Yivick, \\
the young Walsing who had fought. \\
He got the truces of two brothers; \\
oaths they exchanged, men brave of zeal.\evb\evg


\bvg\bva%
\alst{M}ęy buðu hǫ́num \hld\ ok \alst{m}ęiðma fjǫlð, &
\alst{G}uð·ru̇nu ungu \hld\ \alst{G}júka dóttur; &
\alst{d}rukku ok \alst{d}ǿmðu \hld\ \alst{d}ǿgr mart saman &
\alst{S}ig·urðr ungi \hld\ ok \alst{s}ynir Gjúka.\eva

\bvb They offered him a maiden and a multitude of treasures: \\
young Guthrun, Yivick’s daughter. \\
They drank and discussed many a day and night together, \\
young Siward and the sons of Yivick.\evb\evg


\bvg\bva%
Und’s þęir \alst{B}ryn·hildar \hld\ \alst{b}iðja fóru &
\alst{s}vá’t þęim \alst{S}ig·urðr \hld\ ręið ï \alst{s}inni &
\alst{v}ǫlsungr ungi \hld\ ok \alst{v}ega kunni; &
hann of \alst{ę́}tti \hld\ ef hann \alst{ęi}ga knę́tti.\eva

\bvb TODO: Translation.\evb\evg


\bvg\bva%
\alst{S}ęggr inn \alst{s}uðr-ǿni \hld\ lagði \alst{s}verð nøkkvit &
\alst{m}ę́ki \alst{m}ál-fáan \hld\ ȧ \alst{m}eðal þęira &
né hann \alst{k}onu \hld\ \alst{k}yssa gęrði &
né \alst{h}únskr konungr \hld\ \alst{h}ęfja sér af armi &
\alst{m}ęy frum-unga \hld\ fal hann \alst{m}ęgi Gjúka.\eva

\bvb TODO: Translation.\evb\evg


\bvg\bva%
Hǫ́n sér at \alst{l}ífi \hld\ \alst{l}ǫst né vissi &
ok at \alst{a}ldr-lagi \hld\ \alst{ę}kki grand &
\alst{v}amm þat’s \alst{v}ę́ri \hld\ eða \alst{v}esa hygði; &
\alst{g}engu þess ȧ milli \hld\ \alst{g}rimmar urðir.\eva

\bvb TODO: Translation.\evb\evg


\bvg\bva%
\alst{Ęi}n sat hon \alst{ú}ti \hld\ \alst{a}ptan dags, &
\edtext{nam hǫ́n svá bęrt \hld\ umb at mę́lask:}{\Bfootnote{No alliteration can be found in this line.}} &
„Hafa skal’k \alst{S}ig·urð, \hld\ —eða þó \alst{s}velti!— &
\alst{m}ǫg frum-ungan, \hld\ \alst{m}ér ȧ armi.\eva

\bvb TODO: Translation.\evb\evg


\bvg\bva%
\alst{O}rð mę́lta’k nú, \hld\ iðrumk \alst{ę}ptir þess, &
kvǫ́n ’s hans \alst{G}uð·ru̇n \hld\ en ek \alst{G}unnars, &
\alst{l}jótar nornir \hld\ skópu oss \alst{l}anga þrǫ́.\eva

\bvb Words I now spoke; I regret them afterwards. \\
His wife is Guthrun, but I am Guther’s; \\
ugly norns shaped for us a long yearning.\evb\evg


\bvg\bva%
Opt gęngr hȯn innan, \hld\ ills of fylld, &
ísa ok jǫkla, \hld\ aptan hvęrn, &
es þau Guð·ru̇n \hld\ ganga á bęð &
ok hana Sig·urðr \hld\ svęipr ï ripti, &
konungr inn hu̇nski, \hld\ kván frjá sïna.\eva

\bvb TODO: Translation.\evb\evg


\bvg\bva%
Vǫn gęng ek vilja, \hld\ vers ok beggja, &
verð’k mik gǿla \hld\ af grimmum hug.\eva

\bvb TODO: Translation.\evb\evg


\bvg\bva%
Nam af þęim hęiptum \hld\ hvętjask at vígi: &
„Þú skalt, Gunnarr, \hld\ gørst of láta &
mïnu landi \hld\ ok mér sjalfri; &
mun’k una aldri \hld\ með ǫðlingi.\eva

\bvb TODO: Translation.\evb\evg


\bvg\bva%
Mun’k aptr fara \hld\ þar’s áðan vas’k &
með ná-bornum \hld\ niðjum mïnum, &
þar mun’k sitja \hld\ ok sofa lífi &
nema þú Sig·urð \hld\ svelta látir &
ok jǫfur ǫðrum \hld\ ǿðri verðir.\eva

\bvb TODO: Translation.\evb\evg


\bvg\bva%
Lǫ́tum son fara \hld\ feðr í sinni! &
Skalat úlf ala \hld\ ungan lengi. &
Hveim verðr hǫlða \hld\ hefnd léttari &
síðan til sátta \hld\ at sonr lifi.“\eva

\bvb TODO: Translation.\evb\evg


\bvg\bva%
Reiðr varð Gunnarr \hld\ ok hnipnaði, &
sveip sínum hug, \hld\ sat um allan dag; &
hann vissi þat \hld\ vil-gi gǫrla &
hvat hánum vę́ri \hld\ vinna sę́mst &
eða hǫ́num vę́ri \hld\ vinna bezt, &
allz sik Vǫlsung \hld\ vissi firrðan &
ok at Sig·urð \hld\ sǫknuð mikinn.\eva

\bvb TODO: Translation.\evb\evg


\bvg\bva%
Ýmist hann hugði \hld\ jafn-langa stund, &
þat var eigi \hld\ árar títt &
at frá konungdóm \hld\ kvánir gengi; &
nam hann sér Hǫgna \hld\ heita at rúnum, &
þar átti hann \hld\ allz fulltrúa.\eva

\bvb TODO: Translation.\evb\evg


\bvg\bva%
Ein er mér Bryn·hildr \hld\ ǫllum betri, &
um borin Buðla, \hld\ hon er bragr kvenna; &
fyrr skal ek mínu \hld\ fjǫrvi láta &
en þeirar meyjar \hld\ meiðmum týna.\eva

\bvb TODO: Translation.\evb\evg


\bvg\bva%
Vill þú okkr fylki \hld\ til fjár véla? &
Gott er at ráða \hld\ Rínar málmi. &
ok unandi \hld\ auði stýra &
ok sitjandi \hld\ sę́lu njóta.\eva

\bvb TODO: Translation.\evb\evg


\bvg\bva%
Einu því Hǫgni \hld\ annsvǫr veitti: &
„Samir eigi okkr \hld\ slíkt at vinna, &
sverði rofna \hld\ svarna eiða, &
eiða svarna, \hld\ unnar tryggðir.\eva

\bvb TODO: Translation.\evb\evg


\bvg\bva%
Vitum-a vit ȧ moldu \hld\ męnn in sę́lli &
meðan fjórir vér \hld\ fólki ráðum &
ok sá inn húnski \hld\ her-baldr lifir, &
né in mę́tri \hld\ mę́gð á moldu &
ef vér fimm sonu \hld\ fǿðum lengi &
ǫ́ttum góða \hld\ ǿxla knę́ttim.\eva

\bvb TODO: Translation.\evb\evg


\bvg\bva%
Ek veit gǫrla \hld\ hvaðan vegir standa: &
Eru Bryn·hildar \hld\ brek of mikil.“\eva

\bvb TODO: Translation.\evb\evg


\bvg\bva%
„Við skulum Guð·þorm \hld\ gørva at vígi, &
yngra bróður, \hld\ ófróðara; &
hann var fyr útan \hld\ eiða svarna, &
eiða svarna, \hld\ unnar tryggðir.“\eva

\bvb TODO: Translation.\evb\evg


\bvg\bva%
Dę́lt var at eggja \hld\ ȯ·bil-gjarnan, &
stóð til hjarta \hld\ hjǫrr Sig·urði.\eva

\bvb TODO: Translation.\evb\evg


\bvg\bva%
Réð til hefnda \hld\ her-gjarn í sal &
ok eptir varp \hld\ ȯ·bil-gjǫrnum; &
fló til Guð·þorms \hld\ Grams ramm-liga &
kyn-birt járn \hld\ ór konungs hendi.\eva

\bvb TODO: Translation.\evb\evg


\bvg\bva%
Hné hans um dólgr \hld\ til hluta tveggja; &
hendr ok hǫfuð \hld\ hné á annan veg &
en fótahlutr \hld\ fell aftr í stað.\eva

\bvb TODO: Translation.\evb\evg


\bvg\bva%
Sofnuð var Guð·ru̇n \hld\ ï sę́ingu &
sorga-laus \hld\ hjá Sig·urði; &
en hon vaknaði \hld\ vilja firrð &
es hon Fręys vinar \hld\ flaut ï dręyra.\eva

\bvb TODO: Translation.\evb\evg


\bvg\bva%
Svá sló hon svárar \hld\ sínar hendr &
at ramm-hugaðr \hld\ reis upp við beð: &
„Grát-a þú, Guð·ru̇n, \hld\ svá grimm-liga, &
brúðr frum-unga, \hld\ þér brǿðr lifa.\eva

\bvb TODO: Translation.\evb\evg


\bvg\bva%
Á’k til ungan \hld\ erfi-nytja, &
kann-at hann firrast \hld\ ór fjánd-garði; &
þeir sér hafa \hld\ svárt ok dátt &
en nę́r numið \hld\ nýlig ráð.\eva

\bvb TODO: Translation.\evb\evg


\bvg\bva%
Ríðr-a þeim síðan \hld\ þótt sjau alir, &
systur sonr \hld\ slíkr at þingi; &
ek veit gǫrla \hld\ hví gegnir nú: &
Ein veldr Bryn·hildr \hld\ ǫllu bǫlvi.\eva

\bvb TODO: Translation.\evb\evg


\bvg\bva%
Mér unni mę́r \hld\ fyr mann hvern &
en við Gunnar \hld\ grand ekki vann’k; &
þyrmða’k sifjum, \hld\ svǫrnum eiðum, &
síðr vę́ra’k heitinn \hld\ hans kvánar vinr.“\eva

\bvb TODO: Translation.\evb\evg


\bvg\bva%
Kona varp ǫndu \hld\ en konungr fjǫrvi, &
svá sló hon sváran \hld\ sinni hęndi &
at kvǫ́ðu við \hld\ kálkar ï vǫ́ &
ok gullu við \hld\ gę̇ss ï tu̇ni.\eva

\bvb TODO: Translation.\evb\evg


\bvg\bva%
Hló þá Bryn·hildr, \hld\ Buðla dóttir, &
einu sinni \hld\ af ǫllum hug &
er hon til hvílu \hld\ heyra knátti &
gjallan grát \hld\ Gjúka dóttur.\eva

\bvb TODO: Translation.\evb\evg


\bvg\bva%
Hitt kvað þá Gunnarr, \hld\ gramr hauk-stalda: &
„Hlę́r-a þú af því, \hld\ heipt-gjǫrn kona, &
glǫð á gólfi, \hld\ at þér góðs viti. &
Hví hafnar þú \hld\ inum hvíta lit, &
feikna fǿðir? \hld\ Hygg at feig sér.\eva

\bvb TODO: Translation.\evb\evg


\bvg\bva%
Þú vę́rir þess \hld\ verðust kvenna &
at fyr augum þér \hld\ Atla hjǫggim, &
sę́ir brǿðr þínum \hld\ blóðugt sár, &
undir dreyrgar, \hld\ knę́ttir yfir binda.“\eva

\bvb TODO: Translation.\evb\evg


\bvg\bva%
„Frýr-a maðr þér engi, Gunnarr, \hld\ hefir full-vegit! &
Lítt sésk Atli \hld\ ófu þína; &
hann mun ykkar \hld\ ǫnd síðarri &
ok ę́ vera \hld\ afl it meira.\eva

\bvb TODO: Translation.\evb\evg


\bvg\bva%
Segja mun’k þér, Gunnarr, \hld\ —sjalfr veitst gǫrla— &
hvé ér yðr snemma \hld\ til saka réðuð; &
varð’k-at ek til ung \hld\ né of·þrungin &
full-gǿdd féi \hld\ á fleti bróður.\eva

\bvb TODO: Translation.\evb\evg


\bvg\bva%
Né ek vilda þat \hld\ at mik verr ę́tti &
áðr þér Gjúkungar \hld\ riðuð at garði &
þrír á hestum, \hld\ þjóð-konungar, &
en þeira fǫr \hld\ þǫrf-gi vę́ri.\eva

\bvb TODO: Translation.\evb\evg


\bvg\bva%
Þeim hétumk þá
er með gulli sat \hld\ á Grana bógum; &
vas-at hann í augu \hld\ yðr of líkr, &
né á engi hlut \hld\ at ȧ·litum; &
þó þikkisk ér \hld\ þjóð-konungar.\eva

\bvb TODO: Translation.\evb\evg


\bvg\bva%
Ok mér Atli þat \hld\ einni sagði &
at hvárki lézt \hld\ hǫfn um deila, &
gull né jarðir, \hld\ nema ek gefast létak; &
ok engi hlut \hld\ auðins féar, &
þá er mér jóðungri \hld\ eig\emph{u} sęldi &
ok mér jóðungri \hld\ a\emph{u}ra talði.\eva

\bvb TODO: Translation.\evb\evg


\bvg\bva%
Þá var á hvǫrfun \hld\ hugr minn um þat &
hvárt ek skylda vega \hld\ eða val fella, &
bǫll í brynju, \hld\ um bróður sǫk. &
Þat myndi þá \hld\ þjóð-kunnt vera, &
mǫrgum manni \hld\ at munar stríði.\eva

\bvb TODO: Translation.\evb\evg


\bvg\bva%
Létum síga \hld\ sátt-mǫ́l okkur; &
lék mér meirr í mun \hld\ meiðmar þiggja, &
bauga rauða, \hld\ burar Sig·mundar, &
né ek annars mannz \hld\ aura vildak,\eva

\bvb TODO: Translation.\evb\evg


\bvg\bva%
unna einum \hld\ né ýmissum, &
bjó-at um hverfan \hld\ hug men-Skǫgul; &
allt mun þat Atli \hld\ eptir finna &
er hann mína spyrr \hld\ morð-fǫr gǫrva,\eva

\bvb TODO: Translation.\evb\evg


\bvg\bva%
at þeygi skal \hld\ þunn-geð kona &
annarrar ver \hld\ aldri leiða; &
þá mun á hefndum \hld\ harma minna.“\eva

\bvb TODO: Translation.\evb\evg


\bvg\bva%
Upp reis Gunnarr, \hld\ gramr verðungar, &
ok umb hals konu \hld\ hendr of lagði; &
gengu allir, \hld\ ok þó ýmsir, &
af heilum hug, \hld\ hana at letja.\eva

\bvb TODO: Translation.\evb\evg


\bvg\bva%
Hratt af hálsi \hld\ hveim þar sér; &
lét-a mann sik letja \hld\ langrar gǫngu.\eva

\bvb TODO: Translation.\evb\evg


\bvg\bva%
Nam hann sér Hǫgna \hld\ hvętja at ru̇num: &
„Sęggi vil’k alla \hld\ ï sal ganga, &
þïna með mïnum \hld\ —nú ’s þǫrf mikil— &
vita ef męini \hld\ morð-fǫr konu &
und’s af méli \hld\ ęnn męin komi, &
þȧ lǫ́tum því \hld\ þarfar ráða.“\eva

\bvb TODO: Translation.\evb\evg


\bvg\bva%
Ęinu því Hǫgni \hld\ and·svǫr vęitti: &
„Lętj-a maðr hana \hld\ langrar gǫngu &
þar’s hon aptr·borin \hld\ aldri verði! &
Hȯn krǫng of komsk \hld\ fyr kné móður, &
hȯn ę́ borin \hld\ ȯ·vilja til, &
mǫrgum manni \hld\ at móð-trega.“\eva

\bvb TODO: Translation.\evb\evg


\bvg\bva%
Hvarf sér ȯ·hróðugr \hld\ and·spilli frȧ &
þar’s mǫrk męnja \hld\ męiðmum deildi.\eva

\bvb TODO: Translation.\evb\evg


\bvg\bva%
Lęit hȯn umb alla \hld\ ęigu sïna, &
soltnar þýjar \hld\ ok sal-konur; &
gull-brynju smó \hld\ —vas-a gótt ï hug— &
áðr sik miðlaði \hld\ mę́kis ęggjum.\eva

\bvb TODO: Translation.\evb\evg


\bvg\bva%
Hné við bólstri \hld\ hȯn ȧ annan veg &
ok hjǫr-unduð \hld\ hugði at rǫ́ðum:\eva

\bvb TODO: Translation.\evb\evg


\bvg\bva%
„Nú skulu ganga \hld\ þęir’s gull vili &
ok minna því \hld\ at mér þiggja; &
ek gef hvęrri \hld\ of hroðit sigli, &
bók ok blę́ju, \hld\ bjartar váðir.“\eva

\bvb TODO: Translation.\evb\evg


\bvg\bva%
Þǫgðu allir, \hld\ hugðu at rǫ́ðum, &
ok allir sęnn \hld\ ann·svǫr vęittu: &
„Ǿrnar soltnar \hld\ munum ęnn lifa, &
verða sal-konur \hld\ sǿmð at vinna.“\eva

\bvb TODO: Translation.\evb\evg


\bvg\bva%
Und’s af hyggjandi \hld\ hǫr-skrýdd kona, &
ung at aldri, \hld\ orð viðr of kvað: &
„Vil’k-at ek mann trauðan \hld\ né tor-bǿnan &
um ȯra sǫk \hld\ aldri týna.\eva

\bvb TODO: Translation.\evb\evg


\bvg\bva%
Þó mun á beinum \hld\ brenna yðrum &
fę́ri eyrir \hld\ þȧ’s ér framm komið, &
neitt Menju góð, \hld\ mïn at vitja.\eva

\bvb TODO: Translation.\evb\evg


\bvg\bva%
Seztu niðr, Gunnarr, \hld\ mun’k segja þér &
lífs ør·vę̇na \hld\ ljósa brúði; &
mun-a yðvart far \hld\ allt í sundi &
þótt ek hafa \hld\ ǫndu látit.\eva

\bvb TODO: Translation.\evb\evg


\bvg\bva%
Sǫ́tt munuð it Guð·ru̇n, \hld\ snemmr an hyggir; &
hefir kunn kona \hld\ við konung\emph{i} &
daprar minjar \hld\ at dauðan ver.\eva

\bvb TODO: Translation.\evb\evg


\bvg\bva%
Þar ’s mę́r borin, \hld\ móðir fǿðir; &
sú mun hvítari \hld\ an inn heiði dagr, &
Svan·hildr, vesa, \hld\ sólar geisla.\eva

\bvb TODO: Translation.\evb\evg


\bvg\bva%
Gefa munt Guð·ru̇nu \hld\ góðra nǫkkurum, &
skeyti skǿða \hld\ skatna mengi; &
munat at vilja \hld\ ver-sę́l gefin, &
hana mun Atli \hld\ eiga ganga, &
of borinn Buðla, \hld\ bróðir minn.\eva

\bvb TODO: Translation.\evb\evg


\bvg\bva%
Margs á’k minnaz \hld\ hvé við mik fóru &
þȧ’s mik sára \hld\ svikna hǫfðuð; &
vaðin at vilja \hld\ vas’k meðan lifða’k.\eva

\bvb TODO: Translation.\evb\evg


\bvg\bva%
Munt Odd·rúnu \hld\ eiga vilja &
en þik Atli mun \hld\ eigi láta; &
it munuð lúta \hld\ á laun saman, &
hon mun þér unna \hld\ sem ek skyldak, &
ef okkr góð um skǫp \hld\ gerði verða.\eva

\bvb TODO: Translation.\evb\evg


\bvg\bva%
Þik mun Atli \hld\ illu beita, &
mundu ï øngan \hld\ orm-garð lagiðr.\eva

\bvb TODO: Translation.\evb\evg


\bvg\bva%
Þat mun ok verða \hld\ þvígit lengra &
at Atli mun \hld\ ǫndu týna, &
sę́lu sinni, \hld\ ok sofa lífi. &
Því’t hǫ́num Guð·ru̇n \hld\ grýmir á beð &
snǫrpum eggjum \hld\ af sárum hug.\eva

\bvb TODO: Translation.\evb\evg


\bvg\bva%
Sǿmri vę́ri Guð·ru̇n, \hld\ systir okkur &
frum-ver sínum \hld\ [...] &
ef hen\emph{n}i gę́fi \hld\ góðra ráð &
eða ę́tti hon hug \hld\ oss um líkan.\eva

\bvb TODO: Translation.\evb\evg


\bvg\bva%
Ȯ·ǫrt mę́li ek nú \hld\ en hȯn eigi mun &
of ȯra sǫk \hld\ aldri týna; &
hana munu hefja \hld\ hǫ́var bǫ́rur &
til Jónakrs \hld\ óðal-torfu.\eva

\bvb TODO: Translation.\evb\evg


\bvg\bva%
{[...]} eru í varúðum \hld\ Jónakrs sonum; &
mun hon Svanhildi \hld\ senda af landi, &
sína mey \hld\ ok Sig·urðar.\eva

\bvb TODO: Translation.\evb\evg


\bvg\bva%
Hana munu bíta \hld\ Bikka rǫ́ð &
því’t Jǫrmun·rekkr \hld\ ȯ·þarft lifir; &
þȧ’s ǫll farin \hld\ ę́tt Sig·urðar, &
eru Guð·ru̇nar \hld\ grǿti at fleiri.\eva

\bvb TODO: Translation.\evb\evg


\bvg\bva%
Biðja mun’k þik \hld\ bǿnar einnar, &
sú mun í heimi \hld\ hinzt bǿn vera: &
Lát-tu svá breiða \hld\ borg á velli &
at undir oss ǫllum \hld\ jafn-rúmt sé, &
þeim es sultu \hld\ með Sig·urði.\eva

\bvb TODO: Translation.\evb\evg


\bvg\bva%
Tjaldi þar um þá borg \hld\ tjǫldum ok skjǫldum, &
vala-ript vel fǫ́ð \hld\ ok vala mengi; &
brenni mér inn húnska \hld\ ȧ hlið aðra.\eva

\bvb TODO: Translation.\evb\evg


\bvg\bva%
Brenni inum húnska \hld\ ȧ hlið aðra &
mïna þjȯna, \hld\ menjum gǫfga, &
tvȧ at hǫfðum \hld\ ok tveir haukar; &
þá er ǫllu skipt \hld\ til jafnaðar.\eva

\bvb TODO: Translation.\evb\evg


\bvg\bva%
Liggi okkar enn í milli \hld\ malmr hring-variðr, &
egg-hvasst járn, \hld\ svá endr lagit &
þȧ’s vit bę́ði \hld\ beð einn stigum &
ok hétum þȧ \hld\ hjóna nafni.\eva

\bvb TODO: Translation.\evb\evg


\bvg\bva%
Hrynja hǫ́num þȧ \hld\ ȧ hę́l þeygi &
hlunn-blik hallar, \hld\ hringa litkuð &
ef hǫ́num fylgir \hld\ ferð mïn heðan; &
þeygi mun vǫ́r fǫr \hld\ aum-lig vera.\eva

\bvb TODO: Translation.\evb\evg


\bvg\bva%
Því’t hánum fylgja \hld\ fimm ambóttir, &
átta þjónar, \hld\ eðlum góðir, &
fóstr-man mitt \hld\ ok faðerni &
þat’s Buðli gaf \hld\ barni sínu.\eva

\bvb TODO: Translation.\evb\evg


\bvg\bva%
Margt sagða ek, \hld\ mynda’k fleira &
er mér meirr mjǫtuðr \hld\ mál-rúm gę́fi; &
ȯ·mun þverr, \hld\ undir svella, &
satt eitt sagða’k— \hld\ svá mun ek láta!“\eva

\bvb TODO: Translation.\evb\evg

\sectionline
