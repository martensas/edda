\bookStart{The Spae of the Wallow}[Vǫluspǫ́]

\begin{flushright}%
Dating \parencite{Sapp2022}: C10th (0.865)–early C11th (0.121)

Meter: \Fornyrdislag%
\end{flushright}

The \textbf{Spae of the Wallow} is the most comprehensive mythological text surviving from Heathen times.

The poem is attested in full in two independent recensions. The first is \Regius\ (GKS 2365 4to; 1270s), where it is the first poem, found on folios 1r–3r. Second is \Hauksbok\ (AM 544 4to; 1300–75), where it is found at 20r–21r in the middle of a large collection of saws and Catholics works. Many verses are also cited in \Gylfaginning. For its constituent manuscripts see the General Introduction. %TODO: elaborate?

As seen from the title, the poem is a \inx[C]{spae} (\emph{spǫ́} ‘prophecy’) in the form of a monologue spoken by a \inx[C]{wallow} (\emph{vǫlva} ‘seeress, sibyl, prophetess’), summoned by Weden in order to relate mythological knowledge.

The motif of Weden journeying to ask beings (typically ettins or wallows) is also seen in the poems \Baldrsdraumar, wherein Weden summons a wallow out of her grave in \inx[L]{Hell} in order to understand why the god \inx[P]{Balder} is having ominous nightmares, and \Vafthrudnismal, wherein Weden challenges (and defeats) the wise ettin \inx[P]{Webthrithner} to a wisdom contest.

In its being a sort of mythic catalogue it also resembles the latter part of \Havamal, \Grimnismal, \Sigrdrifumal\ and \Allvismal, though it differs from them in a key way: it gives a (mostly?) complete chronological overview of the important events of the mythology. That is not to say that the events described are clear. They are related in a highly allusive fashion—certainly presupposing that the audience already be familiar with them—and there may also be gaps and later inserts that obscure our understanding.

\sectionline

The poem begins with a bid for silence (1), and the wallow recalling her earliest memories (2). She then recounts the ordering of the world by the gods (3–6) and the golden age of peace and plenty (7–8), which is, however, interrupted by the intrusion of three unidentified ettin-maidens (8, and see note there). After this follow two verses about the shaping of the dwarfs (9–10), and then several originally separate \emph{dwarf-tallies} (11–15), which are without doubt later inserts. Returning to the main narrative thread is described the creation and endowment of the first man and woman (16–17), the Ash of Ugdrassle (18), and the three \inx[G]{norns} living under it (19).

This is where the two full recensions of the poem diverge. Because of its older age and larger count of verses I have here followed the order of \Regius: the wallow recalls how a woman named Goldwey was sacrificed and reborn three times (20), and how she, under the name Heath, practiced sorcery and witchcraft (21). She then recalls the first war in the world, between the Ease and Wanes (22–23), and alludes to the slaying of the smith, who according to \Gylfaginning\ 42 was promised \inx[P]{Frow} and the sun and moon in exchange for building the wall of Osyard (24-25). This is followed by a cryptic verse describing Homedall’s hidden silence or hearing (26).

In \Hauksbok\ the structure is quite different. After the description of the norns (19), the Ease go to decide what action to take regarding the promising of Frow to the ettin (my 24-25), and Homedall’s hearing is described (26). Then follows the two verses about the old hag in Ironwood who raises the wolves that will swallow the sun and moon (40-41). After this come verses 20–23 in the same order as \Regius\ (see above).

To illustrate the differences between mss., and which verses are attested in which, I have prepared the following table showing the order of verses by manuscript, compared to the present edition. As most verses in \GylfMS\ are quoted on their own, and have little relation to the original order, these are simply marked with plus signs. When verses are quoted in a series, they are preceded by an alphabetically incrementing letter denoting which series they belong to. When there is a major difference in a ms. relative to the ed., such as in v. 10 where \GylfMS\ omits the first two lines, it is then marked with a star. The verses beginning with \emph{Þȧ gingu ręgin ǫll} ‘Then went the Reins all’ are represented by the following sentence.

\begin{longtabu} to \textwidth {|c c c c c c|}
	\hline
	\multicolumn{2}{|c}{\emph{Present ed.}} & \Regius & \Hauksbok & \RegiusProse\Trajectinus\Wormianus & \Upsaliensis \\ [0.5ex]
	\hline\hline\endhead
	\hline\endfoot
	1 & Hljóðs bið’k allar & 1 & 1 & − & − \\
	2 & Ek man jǫtna & 2 & 2 & − & − \\
	3 & Ár vas alda & 3 & 3 & + & + \\
	4 & Áðr Burs synir & 4 & 4 & − & − \\
	5 & Sól varp sunnan & 5 & 5 & +* & +* \\
	6 & \dots\ nǫ́tt ok niðjum & 6 & 6 & − & − \\
	7 & Hittusk ę̇sir & 7 & 7 & − & − \\
	8 & Tęflðu í tu̇ni & 8 & 8 & − & − \\
	9 & \dots\ hvęrr skyldi dverga & 9 & 9 & B1 & B1 \\
	10 & Þar vas Móðsognir & 10 & 10 & B2* & B2* \\
	11–15 & \emph{Dwarf-tallies} & 11–15 & 11–16 & + & + \\
	16 & Unz þrír kvǫ̇mu & 16 & 17 & − & − \\
	17 & Ǫnd þau né ǫ́ttu & 17 & 18 & − & − \\
	18 & Ask vęit’k standa & 18 & 19 & + & + \\
	19 & Þaðan koma męyjar & 19–20 & 20–21 & − & − \\
	20 & Þat man hǫ̇n folkvíg & 21–22 & 27 & − & − \\
	21 & Hęiði hétu & 23 & 28 & − & − \\
	22 & \dots\ hvárt skyldu ę̇sir & 24 & 29 & − & − \\
	23 & Flęygði Óðinn & 25 & 30 & − & − \\
	24 & \dots\ hvęrr hęfði lopt alt & 26 & 22 & C1 & C1 \\
	25 & Þȯrr ęinn þar vá & 27 & 23 & C2* & C2* \\
	26 & Vęit hǫ̇n Hęimdallar & 28 & 24 & − & − \\
	27 & Ęin sat hǫ̇n úti & 29 & − & − & − \\
	28 & Alt vęit’k, Óðinn & 29 & − & + & + \\
	29 & Valði hęnni Hęrfǫðr & 30 & − & − & − \\
	30 & Sá hǫ̇n valkyrjur & 31 & − & − & − \\
	31 & Ek sá Baldri & 32 & − & − & − \\
	32 & Varð af męiði & 33 & − & − & − \\
	33 & Þó hann ę́va hęndr & 34 & − & − & − \\
	34 & Þȧ kná Váli & − & 31 & − & − \\
	35 & Hapt sá hǫ̇n liggja & 35 & 32* & − & − \\
	36 & Ǫ́ fęllr austan & 36 & − & − & − \\
	37 & Stóð fyr norðan & 36 & − & − & − \\
	38 & Sal sá hǫ̇n standa & 37 & 36 & E1 & E1 \\
	39 & Sér hǫ̇n þar vaða & 38 & 37 & E2* & E2* \\
	40 & Austr býr hin aldna & 39 & 25 & A1 & A1 \\
	41 & Fyllisk fjǫrvi & 40 & 26 & A2 & A2 \\
	42 & Sat þar ȧ haugi & 41 & 34 & − & − \\
	43 & Gól of ǫ̇sum & 42 & 35 & − & − \\
	44, 49, 57 & Gęyr Garmr mjǫk & 43, 46, 55 & 33, 38, 43, 48, 51 & − & − \\
	45 & Brǿðr munu bęrjask & 44 & 39 & − & − \\
	46 & Lęika Míms synir & 45 & 40 & D1* & D1* \\
	47 & Skęlfr Yggdrasils & 45* & 41 & D1* & D1* \\
	48 & Hvat ’s með ǫ̇sum? & 49 & 42 & D2 & D2* \\
	50 & Hrymr ękr austan & 47 & 44 & D3 & − \\
	51 & Kjóll fęrr austan & 48 & 45 & D4 & − \\
	52 & Surtr fęrr sunnan & 50 & 46 & +, D5 & + \\
	53 & Þȧ kømr Hlínar & 51 & 47 & D6 & − \\
	54 & Þȧ kømr hinn mikli & 52 & − & D7 & − \\
	55 & Gínn lopt yfir & − & 48 & — & − \\
	56 & Þȧ kømr hinn mę́ri & 53* & 49* & C8 & − \\
	57 & Sól tér sortna & 54 & 50 & C9 & − \\
	59 & Sér hǫ̇n upp koma & 56 & 52 & − & − \\
	60 & Finnask ę̇sir & 57* & 53 & − & − \\
	61 & Þar munu ęptir & 58 & 54 & − & − \\
	62 & Munu ȯsánir & 59 & 55 & − & − \\
	63 & Þȧ kná Hø̇nir & 60 & 56 & − & − \\
	64 & Sal sér hǫ̇n standa & 61 & 57 & + & + \\
	65 & Þar kømr hinn dimmi & 62 & 59 & − & − \\
	X & Þȧ kømr hinn ríki & − & 58 & − & − \\ [1ex]
	\hline
\end{longtabu}

\sectionline

\bvg
\bva\mssnote{\Regius~1r/2, \Hauksbok~20r/1}„\alst{H}ljóðs bið’k allar \hld\ \edtext{\alst{h}ęlgar}{\lemma{hęlgar}\Afootnote{om. \Regius}} kindir, &
\edtext{\alst{m}ęiri ok \alst{m}inni}{\lemma{męiri ok minni ‘greater and lesser’}\Bfootnote{It is unclear what is being modified here. It may either be ‘greater and lesser holy kindreds’, in which case it may be equivalent to the phrase \inx[F]{Ease and Elves} (both earthly and heavenly supernatural beings; see Encyclopedia for occurences.) or ‘the greater and lesser sons of Homedall \ken{men}’, in which case it refers to all social classes. It seems rather out of character for such a high ranking person in Norse society as the poet must have been to invoke an ancestral relationship between human social classes, considering how biologically such distinctions were otherwise regarded (cf. my introduction to the \Rigsthula), but on the other hand this may be part of the likely liminal nature of the performance.
In any case, the wallow is clearly asking all intelligent beings that may be present for silence, and the expression is a merism of the type ‘gods and men’; see \textcite{West2007}[99-100].}} \hld\ \alst{m}ǫgu Hęimdallar; &
\alst{v}ilt at, \alst{V}alfǫðr, \hld\ \alst{v}ęl fram tęlja’k &
\alst{f}orn spjǫll \alst{f}ira, \hld\ þau’s \alst{f}ręmst of man?\eva

\bvb “For hearing I ask all holy kindreds, greater and lesser, sons of Homedall\footnoteB{Cf. \Rigsthula, wherein Righ, identified by the prose as Homedall, sires three castes of men (namely earls, churls and thralls).} \ken{men}! Wilt thou, Walfather \name{= Weden}, that I well count forth the ancient tidings of men, those which I foremost recall?\footnoteB{Cf. \Vafthrudnismal\ 34, 35 with very similar phrasing. The whole introductory formula is positively Indo-European, see \textcite{West2007}[63,92-93,312].}\evb
\evg


\bvg
\bva\mssnote{\Regius~1r/4, \Hauksbok~20r/2}\alst{E}k man \alst{jǫ}tna \hld\ \alst{á}r of borna, &
þȧ’s \alst{f}orðum mik \hld\ \alst{f}ǿdda hǫfðu; &
\alst{n}íu man’k hęima, \hld\ \alst{n}íu \edtext{íviðjur}{\Afootnote{thus \Regius\Hauksbok. \Regius\ was previously read \emph{íviði}, but this was disproved by an x-ray scan undertaken by \textcite{StefanKarlsson1979}.}}, &
\alst{m}jǫtvið \alst{m}ę́ran \hld\ fyr \alst{m}old neðan.\eva

\bvb I recall \inx[G]{Ettins}, born of yore, they who formerly had nourished me. Nine \inx[C]{Home}[Homes] I recall; nine \inx[G]{Inwithies}; the renowned \inx[P]{Metwood} beneath the soil.\footnoteB{Certainly \inx[P]{Ugdrassle}, “beneath the soil” likely referring to it still being a seed.}\evb
\evg


\bvg
\bva\mssnote{\Regius~1r/6, \Hauksbok~20r/4, \GylfMS}\alst{Á}r vas \alst{a}lda \hld\ \edtrans{þar’s \alst{Y}mir byggði}{there where Yimer dwelled}{\Afootnote{\emph{þat’s ękki vas} ‘that when nothing was’ \GylfMS}}, &
vas-a sandr né sę́r, \hld\ né svalar unnir; &
jǫrð fannsk ę́va \hld\ né upphiminn; &
gap vas ginnunga, \hld\ ęn gras \edtext{hvęrgi}{\Afootnote{\emph{ekki} \Hauksbok}}.\eva

\bvb ’Twas the beginning of \inx[C]{eld}[elds], there as \inx[P]{Yimer} dwelled; was there not sand nor sea, nor cool waves. Earth was never found, nor \inx[L]{Up-heaven}; a gap ’twas of ginnings, but grass nowhere.\footnoteB{According to \Gylfaginning\ 4–5 the world first consisted of two extremities: Nivelham in the north, from which the freezing venom-rivers called the \inx[L]{Ilewaves} ran until they froze to ice; and Muspellsham in the south, from which sparking lava flowed. The ice and lava met in the \inx[L]{Gap of Ginnings} (\emph{Ginnungagap}; see Encyclopedia), “which was as calm as windless air”, and there combined to form the first being, \inx[P]{Yimer}, who was the ancestor of the ettins. This is also told in }\evb
\evg


\bvg
\bva\mssnote{\Regius~1r/8, \Hauksbok~20r/5}Áðr Burs synir \hld\ bjǫðum of ypðu, &
þęir es Miðgarð \hld\ mę́ran skópu; &
sól skęin sunnan \hld\ ȧ salar stęina; &
þȧ vas grund gróin \hld\ grø̇num lauki.\eva

\bvb Before the sons of \inx[P]{Byre} raised up the flatlands, they who shaped the renowned \inx[L]{Middenyard}. Sun shone from the south on the stones of the hall; then was the ground grown with green leek.\footnoteB{The sons of Byre (according to \Gylfaginning\ 6: Weden, Will and Wigh) lift the lands out of the primordial chasm.}\evb
\evg


\bvg
\bva[5a]\mssnote{\Regius~1r/11, \Hauksbok~20r/7}Sól varp sunnan, \hld\ \edtext{sinni Mȧna}{\lemma{sinni Mȧna ‘the companion of Moon’}\Bfootnote{At times translated as ‘its moon’. This cannot be correct, as \emph{mȧni} ‘moon’ is masculine, while \emph{sinni}, dative singular of \emph{sínn} ‘its (reflexive)’ is feminine.}}, &
hęndi hinni hǿgri \hld\ \edtext{of himinjǫður}{\lemma{over heaven’s rim}\Afootnote{\emph{†vm himin iodyr†} ‘over the heaven-horse-beast(?)’ \Regius; \emph{of ioður} ‘over the rim’ \Hauksbok}};\eva

\bvb[5a]Sun cast from the south—the companion of \inx[P]{Moon}—her right hand over heaven’s rim;\footnoteB{The sun heaved herself up over the horizon and rose for the first time.}\evb
\evg


\bvg
\bva[5b]\mssnote{\Regius~1r/12, \Hauksbok~20r/7, \GylfMS}sól þat né vissi, \hld\ hvar hǫ̇n sali átti; &
\edtext{stjǫrnur þat né vissu, \hld\ hvar þę́r staði ǫ́ttu}{\lemma{stjǫrnur \dots\ ǫ́ttu}\Bfootnote{In \GylfMS\ this line follows 5, so that the order is sun, moon, stars.}}; &
mȧni þat né vissi, \hld\ hvat hann męgins átti.\eva

\bvb[5b]Sun knew not where halls she owned; stars knew not where steads they owned; Moon knew not what sort of might he owned.\evb
\evg\stepcounter{stanza}


\bvg
\bva\mssnote{\Regius~1r/13, \Hauksbok~20r/9}Þȧ gingu ręgin ǫll \hld\ ȧ rǫkstóla, &
ginnhęilǫg goð, \hld\ ok umb þat gę́ttusk: &
Nǫ́tt ok niðjum \hld\ nǫfn of gǫ́fu, &
morgin hétu \hld\ ok miðjan dag, &
undurn ok aptan, \hld\ ǫ́rum at tęlja.\eva

\bvb Then went the Reins all onto the rake-seats:\footnoteB{Presumably their thrones by the \inx[L]{Ash of Ugdrassle}; first element \emph{rǫk} defined by \CV\ as ‘reason, ground, origin’.} the gin-holy gods, and from each other took counsel about that.\footnoteB{10, 23, 25 (TODO) would suggest two lines be missing here.} To night and the moon-phases names did they give; morning they called, and middle day; afternoon and evening, the years for to tally.\footnoteB{Cf. \Vafthrudnismal\ 23, 25.}\evb
\evg


\bvg
\bva\mssnote{\Regius~1r/16, \Hauksbok~20r/10}Hittusk ę̇sir \hld\ ȧ Iðavęlli, &
\edtext{þęir’s hǫrg ok hof \hld\ hǫ́ timbruðu}{\lemma{þęir’s \dots\ timbruðu ‘they who \dots\ timbered’}\Afootnote{\emph{afls kostuðu \hld\ allz freistuðu} ‘[their] strength they tried; everything they tempted’ \Hauksbok}}; &
afla lǫgðu, \hld\ auð smíðuðu, &
tangir skópu \hld\ ok tól gęrðu.\eva

\bvb The Ease found each other on \inx[L]{Idewold}, they who \inx[C]{harrow} and \inx[C]{hove} high timbered; hearths they laid, wealth they smithed, tongs they shaped and tools they made.\evb
\evg


\bvg
\bva\mssnote{\Regius~1r/18, \Hauksbok~20r/12}Tęflðu í tu̇ni, \hld\ tęitir vǫ́ru, &
vas þęim véttugis \hld\ vant ór golli, &
unz þríar kvǫ̇mu \hld\ þursa męyjar, &
ȧmátkar mjǫk, \hld\ ór Jǫtunhęimum.\eva

\bvb They played \inx[C]{Tavel} in the yards; merry were they: for them was nothing golden wanting\footnoteB{Indeed, even the gaming bricks were made out of gold; cf. v. 59.}—until three came, maidens of \inx[G]{Thurses}, very loathsome out of \inx[L]{Ettinham}.\footnoteB{These are immediately forgotten and not again mentioned (unless they are taken to be the norns in v. 21, but they would then be introduced twice).—There seems to be something missing between here, perhaps giving further information of the three thurse-maidens, or detailing the reason for the creation of dwarfs?}\evb
\evg

\sectionline

\bvg
\bva\mssnote{\Regius~1r/20, \Hauksbok~20r/14, \GylfMS}Þȧ gingu ręgin ǫll \hld\ ȧ rǫkstóla, &
ginnhęilǫg goð, \hld\ ok umb þat gę́ttusk: &
\edtext{Hvęrr skyldi dverga}{\lemma{hvęrr skyldi dverga ‘Who would \dots\ of dwarfs’}\Afootnote{thus \Regius\Wormianus\Upsaliensis; \emph{at skyldi dverga} ‘That they would \dots\ of dwarfs’ \RegiusProse\Trajectinus; \emph{hverir skyldu dvergar} ‘Which dwarfs would [shape the retinues]’ \Hauksbok}} \hld\ \edtext{drótt}{\lemma{drótt ‘retinue’}\Afootnote{thus \GylfMS; \emph{drotin} \Regius\ with late definite is wo. doubt not original; \emph{dróttir} ‘the retinues’ \Hauksbok}} \edtext{of skępja}{\lemma{of skępja ‘shape’}\Afootnote{\emph{spekia} ‘soothe’ \Upsaliensis}} &
\edtext{ór \edtext{brimi blóðgu}{\lemma{brimi blóðgu ‘bloody surf’}\Afootnote{thus \Hauksbok\RegiusProse\Wormianus\Upsaliensis; \emph{Brimis blóði} ‘the blood of Brimmer’ \Regius\Trajectinus}} \hld\ ok ór \edtext{blǫ́um}{\lemma{blǫ́um ‘blue-black’}\Afootnote{metr. emend.; \emph{blám} \Regius; \emph{Bláins} ‘Blown’s’ \Hauksbok\Wormianus; \emph{Bláms} \RegiusProse\Trajectinus\Upsaliensis\ is prob. a corrupt form of \emph{Bláins}}} lęggjum?}{\lemma{ór brimi \dots\ lęggjum ‘out of the bloody ... legs’}\Bfootnote{I think that the poem simply telling of “the bloody surf” and “the blue-black legs” fits better with its general allusive style, but the resulting composite reading may be somewhat controversial.

According to \Gylfaginning\ 14 the dwarfs first originated as maggots in the corpse of Yimer, whose bones are described in \Grimnismal\ TODO and \Vafthrudnismal\ TODO as being used to make rocks. Dwarfs dwell in the rocks and earth; cf. for instance \Ynglingatal\ 2, where the Swedish king Swayther (\emph{Svęigðir} disappears into a rock in pursuit of a dwarf. More difficult to explain is the creation of dwarfs out of Yimer’s blood (which according to \Grimnismal\ TODO and \Vafthrudnismal\ TODO is the sea), since dwarfs are never said to dwell in water. — If one chooses the reading \emph{Bláinn} ‘Blown’ (named in the \inx[C]{thule}[thules] as a dwarf) instead of \emph{blǫ́um} ‘blue-black’, then following Gurevich (\emph{Skp} 2017, p. 693) one may see a kenning “the legs of Blown \name{dwarf} \ken{stone}”. Blown has otherwise been read as a poetic name for Yimer, but that is never attested elsewhere.}}\eva

\bvb Then went the Reins all onto the rake-seats: the gin-holy gods, and from each other took counsel about that: Who would shape the retinue of \inx[G]{Dwarfs}, out of the bloody surf, and out of the blue-black legs?\evb
\evg


\bvg
\bva\mssnote{\Regius~1r/21, \Hauksbok~20r/15, \GylfMS}\edtext{\edtext{Þar vas Móðsognir}{\Afootnote{thus \Hauksbok; \emph{Þar †mótſognir vitnir†} ‘there Mootsowner wolf’ \Regius. The prose of \Gylfaginning\ 14 confirms that the correct form of the name is \emph{Móðsognir}, not \emph{Mótsognir}.}} \hld\ mę́ztr of orðinn &
dverga allra, \hld\ ęn Durinn annarr;}{\lemma{Þar \dots\ annarr ‘There \dots\ second’}\Bfootnote{om. \GylfMS, but the author must have had access to the full verse, since he paraphrases it in the following way: \emph{Móðsognir var ę́ðstr ok annarr Durinn} ‘Moodsowner was the highest in rank, and Dorn the second.’}} &
\edtext{\edtext{þęir manlíkun \hld\ mǫrg of gęrðu,}{\lemma{þęir \dots\ gęrðu ‘They \dots\ many’}\Afootnote{thus \Regius\Hauksbok\Upsaliensis; \emph{þar manlíkun \hld\ mǫrg of gęrðusk} ‘There man-likenesses many were made’ \RegiusProse\Trajectinus\Wormianus}} &
dvergar \edtext{ór}{\lemma{ór ‘out of’}\Afootnote{thus \Regius; \emph{í} ‘in’ \GylfMS\Hauksbok}} jǫrðu, \hld\ \edtext{sęm Durinn sagði}{\lemma{sęm Durinn sagði ‘as Dorn said’}\Afootnote{thus \Regius\Hauksbok\RegiusProse\Wormianus; \emph{sem †dur menn† sagdi} ‘as door-men(?) said’ \Trajectinus; \emph{sem †þeim dyrinn kendi†} ‘as the animals(?) taught them’ \Upsaliensis}}.}{\lemma{þęir \dots\ sagði ‘They \dots\ said.’}\Bfootnote{There are two conflicting forms of the verse. Either the dwarfs were created on their own; this is supported by the prose of \Gylfaginning\ (see note to last v.) and by the \GylfMS\ containing this verse. On the other hand, both \Regius\ and \Hauksbok\ have the “worthiest” dwarfs Moodsowner and Dorn shaping “man-likenesses” out of soil. I have chosen the latter reading, but both should be considered.}}\eva

\bvb There was Moodsowner made the worthiest of all dwarfs, but Dorn [was] second. They man-likenesses many did make: dwarfs out of the earth, as Dorn said.\evb
\evg

\sectionline

{\small Two lists of dwarfs. That both belonged to the original poem is impossible, since several names (Oakenshield, Great-grandfather) appear in both. The three following verses seem to belong together, since there is no repetition of names. From the last line of the middle one, it seems that it should have been placed at the end of the group.}

\bvg %TODO: move these verses to appendix.
\bva\mssnote{\Regius~1r/23, \Hauksbok~20r/17, \GylfMS}Nýi ok Niði, \hld\ Norðri, Suðri, &
Austri, Vestri, \hld\ Alþjófr, Dvalinn, &
Bívurr, Bávurr, \hld\ Bǫmburr, Nóri, &
Ȧnn ok Ȧnarr, \hld\ Ái, Mjǫðvitnir.\eva

\bvb New and Nithe, Norther and Suther, Easter and Wester, Allthief, Dwollen, Bewer, Bower, Bamber, Noor, Own and Owner, Great-grandfather, Meadwitner.\evb
\evg


\bvg
\bva\mssnote{\Regius~1r/25, \Hauksbok~20r/18, \GylfMS}Vęigr ok Gandalfr, \hld\ Vindalfr, Þráinn, &
Þękkr ok Þorinn, \hld\ Þrór, Vitr ok Litr, &
Nár ok Nýráðr, \hld\ nú hęf’k dverga, &
Ręginn ok Ráðsviðr, \hld\ rétt of talða.\eva

\bvb Wey and Gandelf, Windelf, Thrown, Thetch and Thorn, Throo, Wit and Lit, Nee and Newred—now have I the dwarfs—Rain and Redswith—rightly tallied.\evb
\evg


\bvg
\bva\mssnote{\Regius~1r/28, \Hauksbok~20r/20, \GylfMS}Fíli, Kíli, \hld\ Fundinn, Náli, &
Hępti, Víli, \hld\ Hannarr, Svíurr, &
Frár, Hornbori, \hld\ Frę́gr ok Lȯni, &
Aurvangr, Jari, \hld\ Ęikinskjaldi.\eva

\bvb Filer, Chiler, Found and Needler, Hefter, Wiler, Hanner, Swigher, Fraw, Hornborer, Fray and Looner, Earwong, Earer, Oakenshield.\evb
\evg

\sectionline

\bvg
\bva\mssnote{\Regius~1r/30, \Hauksbok~20r/22, \GylfMS}Mál es dverga \hld\ í Dvalins liði &
ljȯna kindum \hld\ til Lofars tęlja, &
\edtext{þęir}{\Afootnote{\emph{þeim} \Hauksbok}} es sóttu \hld\ frȧ salar stęini &
aurvanga sjǫt \hld\ til Jǫruvalla.\eva

\bvb ’Tis time to tally the dwarfs in Dwollen’s retinue [back] to Loffer for the kindreds of men;\footnoteB{A standard genealogical introduction (compare \Haleygjatal\ 1). The (patrinlineal) line of dwarfs is to be counted back to their progenitor, Loffer. This possibly disagrees with v. 10, where Moodsowner is said to be the foremost (and presumably the oldest) of the dwarfs, and Loffer is not mentioned.} they who sought, from the stone of the hall, the abode of \inx[L]{Earwongs} to the \inx[L]{Erwolds}.\footnoteB{Cf. \Gylfaginning\ 14: “But these came from Swornshigh (\emph{Svarinshaugr}) to the Earwongs on the Erwolds, and thence Lofer is come; these are their names: Sherper (\emph{Skirpir}), Werper (\emph{Virpir}), Showfind, Great-grandfather, Elf and Ing (\emph{Ingi}), Oakenshield, Fale (\emph{Falr}), Frost, Finn, Ginner.”}\evb
\evg


\bvg
\bva\mssnote{\Regius~1r/32, \Hauksbok~20r/24, \GylfMS}Þar vas Draupnir \hld\ ok Dolgþrasir, &
Hár, Haugspori, \hld\ Hlévangr, Glói, &
Skirfir, Virfir, \hld\ Skáfiðr, Ái, &
Alfr ok Yngvi, \hld\ Ęikinskjaldi, &
Fjalarr ok Frosti, \hld\ Finnr ok Ginnarr; &
Þat mun \edtext{ę́}{\Afootnote{om. \Regius}} uppi, \hld\ meðan ǫld lifir, &
langniðjatal \hld\ \edtext{til}{\Afootnote{om. \Hauksbok}} Lofars hafat.\eva

\bvb There was Dreepner and Dollowthrasher, High, Highspurer, Leewong, Glower, Sherver, Werver, Showfind, Great-grandfather, Elf and Ing, Oakenshield, Feller and Frost, Finn and Ginner: That will ever be remembered while the age lives,\footnoteB{Two archaic formulæ. The first literally ‘that will ever [be] up above’, cf. \HervararSaga\ TODO: “We two are cursed, brother, thy bane am I become! That will ever be remembered (\emph{þat mun ę́ uppi}, but both mss. \emph{þat mun enn uppi}), evil is the doom of the norns!” The second is found in a runic inscription, U 323 (980–1015): “Ever will lie—while the age lives (\textbf{meþ + altr + lifiʀ} \emph{með aldr lifir})—the hard-hammered bridge, broad, after a good man.” An especially close parallel is found in Þstf \emph{Stuttdr} (v. 5, Kari Ellen Gade ed. in \Skp\ II): \emph{Ęy mun uppi \hld\ Ęndils, meðan stęndr // sólborgar salr, \hld\ svǫrgǿðis fǫr.} ‘Always will be remembered—while the hall of the sun’s stronghold \ken{sky/heaven > earth} stands—the journey of the fattener of Andle’s bird-fattener \ken{raven/eagle > warrior}.’} the tally of descendants lifted to Lofer.\evb
\evg

\sectionline

\bvg
\bva\mssnote{\Regius~1v/1, \Hauksbok~20r/26}Unz \edtext{þrír}{\Afootnote{gramm. emend. \emph{þrjár} \Regius\Hauksbok}} kvǫ̇mu \hld\ \edtext{ór því liði}{\Afootnote{\emph{þussa brúðir} ‘brides of thurses’ \Hauksbok\ is wo. doubt corrupt.}} &
\edtext{ǫflgir ok ȧstkir}{\lemma{ǫflgir ok ȧstkir ‘strong and lovely’}\Afootnote{\emph{ȧstkir ok ǫflgir} ‘lovely and strong’ \Hauksbok}} \hld\ ę̇sir \edtext{at húsi}{\lemma{at húsi ‘along the house’}\Bfootnote{i.e. ‘along the settlement’.}}; &
fundu ȧ landi \hld\ lítt męgandi &
Ask ok Emblu \hld\ ørlǫglausa.\eva

\bvb Until three came out of that host: strong and lovely Ease along the house; they found on land the little availing Ash and Emble, \inx[C]{orlay}-less.\footnoteB{According to \Gylfaginning\ 9 the sons of Byre (cf. v. 4) were walking along the sea-shore, when they found two logs which they picked up and shaped into humans. That the two were logs seems to be supported by their names; Ash is easily identified with the same-named wood species (\emph{Fraxinus excelsior}). Humans are also very commonly kenned with tree-names in Scoldish poetry (for a short discussion see \textciteshorttitle{SkP} I, p. lxxv ff.), and while this is rarer in the Eddic corpus it occurs e.g. in \Sigrdrifumal\ 4: \emph{brynþings apaldr} ‘apple-tree of the byrnie-\inx[C]{Thing} \ken{battle > warrior}’.}\evb
\evg


\bvg
\bva\mssnote{\Regius~1v/3, \Hauksbok~20r/27}Ǫnd þau né ǫ́ttu, \hld\ óð þau né hǫfðu, &
lǫ́ né lę́ti \hld\ né litu góða; &
ǫnd gaf Óðinn, \hld\ óð gaf Hø̇nir, &
lǫ́ gaf Lóðurr \hld\ ok litu góða.\eva

\bvb Breath they owned not, \inx[C]{wode} they had not, not craft nor sound nor good countenance. Breath gave Weden, wode gave Heener, craft gave Lother, and good countenance.\evb
\evg


\bvg
\bva\mssnote{\Regius~1v/5, \Hauksbok~20r/29, \GylfMS}Ask vęit’k \edtext{standa}{\lemma{standa ‘standing’}\Afootnote{thus \Regius\Hauksbok\Upsaliensis; \emph{ausinn} ‘poured, sprinkled’ \RegiusProse\Trajectinus\Wormianus}}, \hld\ hęitir \edtext{Yggdrasill}{\Afootnote{Yggdrasils \RegiusProse}}, &
hǫ́r \edtext{baðmr}{\lemma{baðmr ‘beam’}\Afootnote{\emph{borinn} ‘born’ \Upsaliensis\ is wo. doubt corrupt.}}, \edtext{ausinn}{\lemma{ausinn ‘poured’}\Afootnote{\emph{hęilagr} ‘holy’ \GylfMS}} \hld\ hvíta auri; &
þaðan koma dǫggvar \hld\ \edtext{þę́r’s}{\Afootnote{\emph{es} ‘which’ \RegiusProse\Trajectinus}} í dala falla; &
stęndr \edtext{ę́}{\Afootnote{\emph{om.} \Upsaliensis}} yfir \edtext{grø̇nn}{\Afootnote{\emph{†grvnn†} \RegiusProse; \emph{†grein†} \Upsaliensis}} \hld\ Urðar brunni.\eva

\bvb An ash I know stand[ing], \inx[L]{Ugdrassle} ’tis called; a high beam \ken{tree}, poured with white mud.\footnoteB{i.e. ‘white mud is (or has been) poured upon it.’ Cf. perhaps the Indian ritual pouring of beverages onto the \emph{lingam}—For the whole passage cf. v. 26.} Thence come the dew-drops which in the dales fall; it stands ever green over the \inx[L]{Well of Weird}.\evb
\evg


\bvg
\bva\mssnote{\Regius~1v/8, \Hauksbok~20r/31}Þaðan koma męyjar \hld\ margs vitandi &
þríar ór þęim \edtext{sę́}{\lemma{sę́ ‘lake’}\Afootnote{\emph{sal} ‘hall’ \Hauksbok}}, \hld\ es \edtext{und}{\lemma{und ‘under’}\Afootnote{\emph{ȧ} ‘on’ \Hauksbok}} þolli stęndr; &
Urð hétu ęina, \hld\ aðra Verðandi, &
skǫ́ru ȧ skíði, \hld\ Skuld hina þriðju &
þę́r lǫg lǫgðu, \hld\ þę́r líf køru, &
alda bǫrnum, \hld\ ørlǫg \edtext{sęggja}{\lemma{sęggja ‘of men’}\Afootnote{\emph{at sęgja} ‘to say’ \Hauksbok}}.\eva

\bvb Thence come maidens, much knowing: three out of that lake, which stands under the pine\footnoteB{But here simply meaning ‘tree’; perhaps the same applies for “ash” earlier.}: Weird they called one, the other Werthing—carved they on boards—Shild the third. Laws they laid, lives they chose: for the children of mortals, the \inx[C]{orlay} of men.\footnoteB{i.e. ‘they have laid laws, they have chosen lives’. It is well known that in Old Norse as in other old Germanic languages the simple past is often used interchangably in both the perfective and imperfective sense.}\evb
\evg


\bvg
\bva\mssnote{\Regius~1v/11, \Hauksbok~20v/5}Þat man hǫ̇n folkvíg \hld\ fyrst í hęimi, &
es Gollvęigu \hld\ gęirum studdu &
ok í hǫll Háars \hld\ hȧna bręnndu, &
\edtext{þrysvar bręnndu}{\Afootnote{\emph{†þrysvar brendv þrysvar brendv†} \Hauksbok}} \hld\ þrysvar borna, &
opt ȯsjaldan, \hld\ þó ęnn lifir.\eva

\bvb That troop-conflict\footnoteB{While appealing to read \emph{folk-víg} ‘troop-conflict’ as meaning ‘ethnic conflict’ (between the Ease and Wanes), I more cautiously see the first element \emph{folk} carrying its earlier meaning of ‘troop, group of warriors’.} \ken{war} she recalls, the first in the \inx[C]{Home}, as Goldwey with spears they goaded, and in the hall of \inx[P]{Higher} \name{= Weden} \ken*{= Walhall} burned her: thrice they burned the thrice born; often unseldom, though she yet lives.\footnoteB{Very cryptic. TODO: double check Snorri. Goldwey was apparently sacrificed, cremated and reborn three times (in short succession?) by the Ease.}\evb
\evg


\bvg
\bva\mssnote{\Regius~1v/13, \Hauksbok~20v/7}Hęiði hétu, \hld\ hvar’s til húsa kom, &
\edtext{vǫlu}{\Afootnote{\emph{ok vǫlu} \Hauksbok}} vęlspáa, \hld\ vitti ganda; &
sęið \edtrans{hvar’s kunni}{where she could}{\Afootnote{\emph{hon kvnni} ‘she could’ \Regius; \emph{hon hvars hvn kunni} ‘she soth where she could’ \Hauksbok}}, \hld\ sęið \edtrans{hug lęikinn}{deluded minds}{\Afootnote{\emph{hon leikinn} \Regius; \emph{hon hugleikin} \Hauksbok}}; &
ę́ vas angan \hld\ illrar brúðar.\eva

\bvb Heath they called—where to houses she came—the well-spaeing\footnoteB{Gifted at soothsaying.} \inx[C]{wallow}; she bewitched \inx[C]{gand}[gands]. She soth\footnoteB{Past tense of \inx[C]{sithe} (ON \emph{síða}) ‘to enchant, bewitch’.)} where she could, she soth deluded minds; ever was she the love of any evil bride.\evb
\evg


\bvg
\bva\mssnote{\Regius~1v/16, \Hauksbok~20v/9}Þȧ gingu ręgin ǫll \hld\ ȧ rǫkstóla, &
ginnhęilǫg goð, \hld\ ok umb þat gę́ttusk: &
Hvárt skyldu ę̇sir \hld\ afráð gjalda, &
eða skyldu goð ǫll \hld\ gildi ęiga?\eva

\bvb Then went the Reins all onto the rake-seats: the gin-holy gods, and from each other took counsel about that: whether the Ease should tribute yield, or should the gods all a banquet hold?\evb
\evg


\bvg
\bva\mssnote{\Regius~1v/17, \Hauksbok~20v/11}Flęygði Óðinn \hld\ ok í folk of skaut; &
þat vas ęnn folkvíg \hld\ \edtrans{fyrr}{earlier}{\Afootnote{thus \Hauksbok; \emph{fyrst} ‘first’ \Regius. The \Regius\ reading is certainly due to the close relation with 20/1, but it cannot be correct as this verse is describing a different war, and thus not the first!}} í hęimi; &
brotinn vas borðvęggr \hld\ borgar ȧsa, &
knǫ́ttu vanir vígspǫ́u \hld\ vǫllu sporna.\eva

\bvb Weden hurled, and into the opposing troop did shoot;\footnoteB{The object, a spear, is understood. This seems to reference a ritual, well-attested in the literature, wherein a war-chief would dedicate an opposing army as a human sacrifice to Weden by throwing a spear over them, typically with the incantation \emph{Óðinn á yðr alla} ‘Weden owns you all!’; he would then own the battle-slain in that they joined him as \inx[G]{Ownharriers} in \inx[L]{Walhall}. Weden is also described as “owning” dead men in \Harbardsljod\ 24 (namely slain nobles, contrasted with \inx[P]{Thunder} who is insultingly said to “own the kin of thralls”) and in runic inscription \emph{N B380}, here edited under Charms and Spells, a sort of greeting wherein the receiver is wished to be owned by Weden (and “received” by Thunder). For further literature see \textciteshorttitle{PCRN-HS} II:24, p. 560, II:25, p. 617, and especially III:42, p. 1166ff.} that was yet a troop-conflict \ken{war} earlier in the \inx[L]{Home}. Broken was the board-wall\footnoteB{Wall made of planks.} of the fortification of the Ease; the Wanes did by a conflict-\inx[C]{spae} tread the fields.\footnoteB{The Wanes used a magic spell to invade the Ease.}\evb
\evg


\bvg
\bva\mssnote{\Regius~1v/19, \Hauksbok~20r/34, \GylfMS}Þȧ gingu ręgin ǫll \hld\ ȧ rǫkstóla, &
ginnhęilǫg goð, \hld\ ok umb þat gę́ttusk: &
Hvęrr hęfði lopt alt \hld\ lę́vi blandit &
eða ę́tt jǫtuns \hld\ Óðs męy gefna.\eva

\bvb Then went the Reins all onto the rake-seats: the gin-holy gods, and from each other took counsel about that: Who had the air all with treason blended, or to the ettin’s \inx[C]{aught} given \inx[P]{Wode}’s maiden \ken*{= Frow}?\footnoteB{That is, promised Frow to the ettin NAME. TODO: relate with what Snorri writes about the building of the wall.}\evb
\evg


\bvg
\bva\mssnote{\Regius~1v/20, \Hauksbok~20r/36, \GylfMS}\edtext{Þȯrr ęinn \edtext{þar vá}{\lemma{þar vá ‘fought there’}\Afootnote{thus \Hauksbok\Trajectinus\Upsaliensis; \emph{þar var} ‘was there’ \Regius; \emph{þat vann} ‘did, accomplished it’ \RegiusProse; \emph{þat vá} ‘fought it’ \Wormianus}} \hld\ þrunginn móði, &
hann sjaldan sitr, \hld\ es slíkt of fregn; &
\edtext{ȧ gingusk ęiðar, \hld\ orð ok sǿri, &
mǫ́l ǫll męginlig, \hld\ es ȧ meðal \edtext{fóru}{\lemma{fóru ‘had gone’}\Afootnote{\emph{vǫ́ru} ‘had been’ \Hauksbok\Trajectinus}}.}{\lemma{ȧ \dots\ fóru.}\Afootnote{om. \Wormianus}}}{\lemma{Þȯrr \dots\ fóru.}\Bfootnote{The order followed is that of \Regius\Hauksbok; in \GylfMS\ the two helmings (\emph{Þȯrr \dots\ fregn;} \emph{ȧ \dots\ fóru}) come in reverse order.}}\eva

\bvb Thunder alone fought there, pressed by wrath; he seldom sits, when of such\footnoteB{Oath-breaking, lies and deception.} he learns. Trampled were oaths, speeches and vows; the mighty treaties all, which between them had gone.\evb
\evg

\sectionline

\bvg
\bva\mssnote{\Regius~1v/23, \Hauksbok~20v/1}Vęit hǫ̇n Hęimdallar \hld\ hljóð of folgit &
und hęiðvǫnum \hld\ hęlgum baðmi; &
ȧ sér ausask \hld\ aurgum forsi &
af veði Valfǫðrs. \hld\ Vituð ér ęnn eða hvat?\eva

\bvb Knows she the sound of Homedall \ken{= Horn of Yell?} hidden, ’neath a shady\footnoteB{\emph{hęiðvanr}, literally ‘clear-, bright-less’.}, hallowed beam \ken*{the Ash of Ugdrassle}. On [it] she sees being poured a muddy torrent\footnoteB{lit. ‘on she sees being poured with a muddy torrent’, which should be the same mud as in v. 19. However, if ms. \emph{á} is read as \emph{ǫ́} ‘river’, it would mean “A river she sees being fed by a muddy waterfall, from ...”}, from the pledge of Walfather\footnoteB{Presumably referring to Weden’s sacrifice of an eye at Mimer’s well.} \name{= Weden} \ken*{Mimer’s well?}—know ye yet, or what?\footnoteB{“Do you (Weden) know enough now, or what?”—repeated in 28, 33, 34, 38, 40, 47, 60, 61.}”\evb
\evg

\sectionline

{\small The following two verses are written together as one in \Regius.}

\bvg
\bva\mssnote{\Regius~1v/25}Ęin sat hǫ̇n úti, \hld\ þȧ’s hinn aldni kom &
yggjungr ȧsa \hld\ ok í augu lęit; &
hvęrs fregnið mik? \hld\ hví fręistið mín?\eva

\bvb Lone sat she outside, when the old one came: the Terrifier of the Ease \ken*{= Weden}, and into [her] eyes looked. [The Wallow:] “Why inquirest thou me? Why triest thou me?\footnoteB{\emph{fręista} has a sense of testing someone, especially intellectually. Cf. \Havamal\ 2, 26, 142, \Vafthrudnismal\ 3, 5.}\evb
\evg

\bvg
\bva\mssnote{\Regius~1v/26, \GylfMS}Alt vęit’k, Óðinn, \hld\ hvar auga falt &
\edtext{í hinum mę́ra}{\Afootnote{thus \Wormianus; \emph{þitt} (corr.) \emph{i enom męra} ‘id.’ \Regius; \emph{í þęim hinum meira} ‘id.’ (norm.) \Trajectinus\Upsaliensis; \emph{vr þeim envm mę́ra} ‘out of the renowned’ \RegiusProse}} \hld\ Mímis brunni; &
drekkr mjǫð Mímir \hld\ morgin hvęrjan &
af \edtext{veði}{\lemma{veði ‘pledge’}\Afootnote{†veiþi† ‘hunting’ \RegiusProse}} Valfǫðrs. \hld\ Vituð ér ęnn eða hvat?\eva

\bvb I know it all, Weden; where thy eye thou hidst: in the renowned \inx[L]{Well of Mime}, [there] drinks Mime mead every morning, from the pledge of Walfather\footnoteB{See note to v. 26.} \name{= Weden} \ken*{Mimer’s well?}—know ye yet, or what?”\evb
\evg


\bvg
\bva\mssnote{\Regius~1v/29}Valði hęnni Hęrfǫðr \hld\ hringa ok męn, &
\edtext{fekk spjǫll spaklig}{\lemma{fekk spjǫll spaklig ‘received wise tidings’}\Bfootnote{fé, spjǫll spaklig ‘wealth, wise tidings’ \Regius\ is metrically deficient, since alliteration would need to fall on the strongly stressed noun \emph{fé}. The emended text also works better in context since it parallels v. 1, where the wallow likewise says that she will relate \emph{spjǫll} ‘tidings, sayings’ (cf. English \emph{gospel} lit. ‘good news’, translating Greek \textgreek{εὐαγγέλιον}). See \textcite[51--53]{Haukur2020}, \textcite[16]{Males2023} for discussion.}} \hld\ ok \edtext{spáganda}{\lemma{spáganda ‘spae-gands’}\Bfootnote{Spirits sent out in order to secretly gather information. See relevant Encyclopedia entries.}}; &
sá vítt ok umb vítt \hld\ of verǫld hvęrja.\eva

\bvb Host-father \name{= Weden} chose for her rings and necklaces; [he] received wise tidings and \inx[C]{spae}-\inx[C]{gands}; she looked widely and widely about, o’er every world.\evb
\evg


\bvg
\bva\mssnote{\Regius~1v/30}Sá hǫ̇n valkyrjur \hld\ vítt of komnar, &
gǫrvar at ríða \hld\ til goðþjóðar: &
\edtext{Skuld hélt skildi, \hld\ ęn Skǫgul ǫnnur, &
Gunnr, Hildr, Gǫndul \hld\ ok Gęirskǫgul; &
\alst{n}ú eru talðar \hld\ \edtrans{\alst{N}ǫnnur Hęrjans}{Nans \name{maidens} of Harn \name{= Weden} \ken{walkirries}}{\Bfootnote{\emph{Nanna} ‘\inx[P]{Nan}’ was the wife of \inx[P]{Balder}, but her name is here in the plural certainly being used to mean ‘maidens, goddesses’. The walkirries are also referred to as Weden’s maidens in two thules, namely TODO.}}, &
gǫrvar at ríða \hld\ grund valkyrjur.}{\lemma{Skuld \dots\ valkyrjur}\Bfootnote{These four lines, especially from the out-of-place ending (\emph{nú eru talðar}), seem to be a latter insert from a \emph{thule} counting the walkirries.}}\eva

\bvb She saw \inx[G]{Walkirries}, widely come, ready to ride to \inx[L]{Godthede}: Shild held a shield and Shagle another; Guth, Hild, Gandle and Goreshagle; now are tallied the Nannies of Harn \name{= Weden} \ken{walkirries}; walkirries ready to ride the ground.\evb
\evg

\sectionline

{\small Told allusively in 31–33 is the death of Balder at the hands of his blind brother Hath; it is prophesized with very similar language in \Baldrsdraumar\ 9–11 and described in detail in \Gylfaginning\ 49.}

\bvg
\bva\mssnote{\Regius~2r/2}Ek sá Baldri, \hld\ blóðgum \edtext{tívur}{\lemma{tívur ‘tue’}\Bfootnote{Dative}}, &
Óðins barni, \hld\ ørlǫg folgin; &
stóð of vaxinn \hld\ vǫllum hę́ri &
mjór ok mjǫk fagr \hld\ mistiltęinn.\eva

\bvb I saw Balder’s—the bloody \inx[G]{Tues}[tue]’s, Weden’s child’s—\inx[C]{orlay} sealed;\footnoteB{Or ‘hidden’. The verb \emph{fela} ‘hide, conceal’ is used in poetry to describe burial in mounds, as in \Ynglingatal\ 24 (“[...] And afterwards the victory-havers hid (\emph{fǫ́lu}) the ruler on Borrey.”) or the C10th Karlevi stone (“Hidden (\textbf{fulkin} \emph{folginn}) in this mound lies he whom the greatest deeds followed; [...]”)} grown did stand—higher than the plains—a slender and very fair mistletoe.\evb
\evg


\bvg
\bva\mssnote{\Regius~2r/4}Varð af męiði, \hld\ þęim’s mę́r sýndisk, &
harmflaug hę́ttlig, \hld\ Hǫðr nam skjóta. &
Baldrs bróðir vas \hld\ of borinn snimma, &
sá nam, Óðins sonr, \hld\ ęinnę́ttr vega.\eva

\bvb Became of that beam, which meager looked, a baneful harm-flier—Hath took to shoot. Balder’s brother \ken*{= Hath} was born early; that one took—Weden’s son, one night old—to fight.\evb
\evg


\bvg
\bva\mssnote{\Regius~2r/6}Þó hann ę́va hęndr \hld\ né hǫfuð kęmbði, &
áðr ȧ bál of bar \hld\ Baldrs andskota. &
Ęn Frigg of grét \hld\ í Fęnsǫlum &
vǫ́ Valhallar. \hld\ Vituð ér ęnn eða hvat?\eva

\bvb Washed he never hands, nor head combed, before onto the pyre he did bear Balder’s opponent. But Frie did lament, in the Fenhalls, the woe of Walhall\footnoteB{i.e. Balder’s death.}—know ye yet, or what?\evb
\evg


\bvg
\bva\mssnote{\Hauksbok~20v/12}\edtext{Þȧ kná Váli \hld\ vígbǫnd snúa &
hęldr vǫ́ru harðgǫr \hld\ hǫpt ór þǫrmum.}{\lemma{Þȧ \dots\ þǫrmum.}\Bfootnote{Only attested in \Hauksbok, where it is combined with the last two lines of the next v. (\emph{þar \dots\ hvat?}).}}\eva

\bvb Then did \inx[C]{Wonnel} the war-bonds turn; were they rather sturdy, fetters made out of intestines.\footnoteB{According to \Gylfaginning\ 50 the Ease captured Lock’s two sons, Wonnel and Narve (or Nare). They turned Wonnel into a wolf and had him kill his brother Narve, whose intestines were then taken and used to bind Lock so that he lay on top of three pointed stones; one digging into his shoulder-blades, one digging into his loins and one digging into his houghs. The intestine-fetters then turned into iron.}\evb
\evg


\bvg
\bva[35a]\mssnote{\Regius~2r/8}Hapt sá hǫ̇n liggja \hld\ und Hveralundi &
lę́gjarnlíki \hld\ Loka ȧþękkjan;\eva

\bvb[35a]A captive she saw lying, ’neath Wharlund: the guileful form of similar Lock.\evb
\evg


\bvg
\bva[35b]\mssnote{\Regius~2r/9, \Hauksbok~20v/13}þar sitr Sigyn \hld\ þęygi of sínum &
veri vęlglýjuð. \hld\ Vitud ér ęnn eða hvat?\eva

\bvb[35b]There sits Sighyn, not at all cheerful, o’er her husband\footnoteB{See \FraLoka.}—know ye yet, or what?\evb
\evg\stepcounter{stanza}

\sectionline

\bvg
\bva\mssnote{\Regius~2r/10}Ǫ́ fęllr austan \hld\ of ęitrdala &
sǫxum ok sverðum, \hld\ Slíðr hęitir sú.\eva

\bvb A river falls from the east, above the venom-dales, with saxes and swords; Slide is that one called.\evb
\evg


\bvg
\bva\mssnote{\Regius~2r/11}Stóð fyr norðan \hld\ ȧ Niðavǫllum &
salr ór golli \hld\ Sindra ę́ttar, &
ęn annarr stóð \hld\ ȧ Ȯkólni, &
bjórsalr jǫtuns, \hld\ ęn sá Brimir hęitir.\eva

\bvb Stood to the north, on the Nithewolds, a hall out of gold, of the lineage of Sinder \ken{dwarves}; but another one stood, on Uncoalner, the beer-hall of an ettin, but Brimmer is that one called.\evb
\evg


\bvg
\bva\mssnote{\Regius~2r/13, \Hauksbok~20v/19, \GylfMS}Sal sá hǫ̇n standa \hld\ sólu fjarri &
Nástrǫndu ȧ, \hld\ norðr horfa dyrr; &
falla ęitrdropar \hld\ inn umb ljóra, &
sá ’s undinn salr \hld\ orma hryggjum.\eva

\bvb A hall she saw standing, far from the sun, on Nawstrand; north face the doors;—fall venom-drops in through the smoke-vent, that hall is wound by the spines of snakes.\evb
\evg


\bvg
\bva\mssnote{\Regius~2r/15, \Hauksbok~20v/21, \GylfMS}\edtext{Sá hǫ̇n}{\lemma{Sá hǫ̇n ‘she saw’}\Afootnote{\emph{thus} \Regius; ser hon ‘she sees’ \Hauksbok; skulu ‘shall [be]’ \GylfMS}} þar vaða \hld\ þunga strauma &
męnn męinsvara \hld\ ok morðvarga &
ok þann’s annars glępr \hld\ ęyraru̇nu. &
Þar \edtext{saug}{\lemma{saug ‘sucked’}\Afootnote{\emph{thus} \Hauksbok; súg (\emph{corrupt}) \Regius; kvęlr ‘torments’}} Níðhǫggr \hld\ nái framgingna; &
slęit vargr vera. \hld\ Vituð ér ęnn eða hvat?\eva

\bvb There she saw wading, through heavy streams, oath-breaking men and murder-wargs, and the one who beguiles another’s ear-whisperer \ken{wife}. There sucked \inx[P]{Nithehewer} from corpses passed-on; the warg tore men asunder—know ye yet, or what?\footnoteB{Uniquely in this verse is described punishment in the Heathen afterlife. The crimes are what one might expect from the Germanic worldview: breaking oaths, committing a murder and evading punishment, and seducing a married woman.}\evb
\evg

\sectionline

\bvg
\bva\mssnote{\Regius~2r/17, \Hauksbok~20v/2, \GylfMS}Austr \edtext{býr}{\lemma{býr ‘dwells’}\Afootnote{\emph{thus} \Hauksbok\GylfMS; sat ’stayed’ \Regius}} hin \edtext{aldna}{\lemma{aldna ‘old’}\Afootnote{arma ‘wretched’ \Upsaliensis}} \hld\ í \edtext{Éarnviði}{\lemma{Éarnviði ‘Ironwood’}\Afootnote{\emph{metr. emend.}; Járnviði \Regius\Hauksbok\RegiusProse\Wormianus\Upsaliensis; Járnviðjum ‘Ironwoods’ \Trajectinus}} &
ok \edtext{fǿðir}{\Afootnote{\emph{thus} \Hauksbok\GylfMS; fǿddi ‘nourished’ \Regius}} þar \hld\ Fęnris kindir; &
verðr \edtext{af}{\Afootnote{ór \Trajectinus\RegiusProse}} þęim ǫllum \hld\ ęinna nøkkurr &
tungls \edtext{tjúgari}{\lemma{tjúgari ‘seizer’}\Afootnote{†tuigan† \Trajectinus; tregari ‘griever’ \Upsaliensis}\Bfootnote{As the young agentive suffix \emph{-ari} is found nowhere else in the poem it is possible that this word is corrupt. If it is, it must have occurred early in the transmission as reflexes of \emph{*tjúgari} are found in all surviving mss.}} \hld\ í trolls hami.\eva

\bvb In the east dwells the old woman, in \inx[L]{Ironwood}, and nourishes there the kindreds of \inx[P]{Fenrer} \ken{wolves}; from them all becomes one most particular: a seizer of the moon, in the \inx[C]{hame} of a troll.\footnoteB{The old hag raises the cubs of the wolf Fenrer, of which a particularly fierce one will swallow the moon. According to \Grimnismal\ 40 the sun is chased by a wolf called Skoll, while another wolf, Hate Rothswitner’s son, runs in front of her. This is elaborated upon in \Gylfaginning\ 12, where it is said that Skoll swallows the moon, while Hate swallows the sun. High then explains that “A lone troll-woman (\emph{gýgr}) lives to the east of Middenyard in that forest called Ironwood”, and “feeds the sons of many ettins, all in the likenesses of wolves, and thereof these wolves (i.e. Skoll and Hate) come. And it is also said that from that lineage a single one becomes the mightiest, and he is called \inx[P]{Moongarm}. He fills himself with the life of all those men who die and he swallows the moon and stains heaven and all the air with blood. Thereof the sun loses its rays and the winds are violent and moan hither and thither, and thus it says in the Spae of the Wallow: [...]” after which this and the following verse are quoted. This seems very much like a composite from several sources—probably \Voluspa\ 40–41 and \Grimnismal\ 40—but becomes contradictory when it states that two wolves swallow the moon.
Assuming that this is only a confusion on the part of the author of \Gylfaginning, this verse and the next must be describing Skoll, but it is of course not impossible that there was confusion about the exact details of these events among the Heathen poets. In favour of this seems to speak \Vafthrudnismal\ 46–47, where the sun is said to be swallowed by Fenrer (but see note there).}\evb
\evg


\bvg
\bva\mssnote{\Regius~2r/19, \Hauksbok~20v/4, \GylfMS}Fyllisk fjǫrvi \hld\ fęigra manna, &
rýðr ragna sjǫt \hld\ rauðum dręyra, &
svǫrt verða sólskin \hld\ umb sumur ęptir, &
veðr ǫll válynd. \hld\ Vituð ér ęnn eða hvat?\eva

\bvb It \ken*{= the wolf} fills itself with the life of \inx[C]{fey} men; it reddens the abode of the \inx[G]{Reins} with red gore. Black becomes the sunshine about the summers afterwards;\footnoteB{After the air is filled with blood the sun can no longer shine clearly.} the winds all woeful—know ye yet, or what?\evb
\evg


\bvg
\bva\mssnote{\Regius~2r/21, \Hauksbok~20v/16}Sat þar ȧ haugi \hld\ ok sló hǫrpu &
gýgjar hirðir, \hld\ glaðr Ęggþér; &
gól of hǫ̇num \hld\ í Gaglviði &
fagrrauðr hani, \hld\ sá’s Fjalarr hęitir.\eva

\bvb Sat there on the \inx[C]{howe} and struck the harp, the troll-woman’s herdsman,\footnoteB{He herded the flock of monstrous wolves, as it were.} glad \inx[P]{Edgethew}. Above him crowed, in Galewood\footnoteB{\emph{gagl} ‘wild goose’, maybe here referring to carrion-eating ravens? Possibly the same as Ironwood.}, a fair-red cock, that one who Feller is called.\evb
\evg


\bvg
\bva\mssnote{\Regius~2r/23, \Hauksbok~20v/18}Gól of ǫ̇sum \hld\ Gollinkambi, &
sá vękr hǫlða \hld\ at Hęrjafǫðrs, &
ęn annarr gęlr \hld\ fyr jǫrð neðan &
sótrauðr hani \hld\ at sǫlum Hęljar.\eva

\bvb Above the Ease crowed Goldencombe: he wakes men at the Father of Hosts’s [estate]; but another one crows beneath the earth: a soot-red cock, at the halls of Hell.\evb
\evg


\bvg
\bva\mssnote{\Regius~2r/25}Gęyr Garmr mjǫk \hld\ fyr Gnipahęlli, &
fęstr mun slitna, \hld\ ęn Freki rinna; &
fjǫlð vęit hǫ̇n frǿða, \hld\ framm sé’k lęngra &
of ragna rǫk, \hld\ rǫmm sigtíva.\eva

\bvb Barks Garm loudly before the Gnip-caverns; the rope will tear, and Freck run. Much she knows of learning, forth I see yet further; about the mighty Rakes of the Reins, of the victory-tues.\evb
\evg


\bvg
\bva\mssnote{\Regius~2r/28, \Hauksbok~20v/24, \GylfMS}Brǿðr munu bęrjask \hld\ ok at bǫnum verðask, &
munu \edtext{systrungar}{\lemma{systrungar ‘sister’s sons’}\Afootnote{†stystrungar† \Trajectinus}} \hld\ sifjum spilla; &
hart ’s \edtext{í hęimi}{\lemma{í hęimi ‘in the Home’}\Afootnote{\emph{thus} \Regius\Hauksbok\Upsaliensis; með hǫlðum ‘among men’ \RegiusProse\Trajectinus\Wormianus}}, \hld\ hórdȯmr mikill, &
skęggǫld, skalmǫld, \hld\ \edtext{skildir}{\lemma{skildir ‘shields’}\Afootnote{\emph{add.} ’ru ‘are’ \Regius}} \edtext{klofnir}{\lemma{klofnir ‘cloven’}\Afootnote{klofna ‘become cloven’ \Upsaliensis}}, &
\edtext{vindǫld}{\lemma{vindǫld ‘wind-eld’}\Bfootnote{In \Hauksbok\ capitalized, marking the beginning of a new verse.}}, vargǫld, \hld\ \edtext{áðr}{\lemma{áðr ‘before’}\Afootnote{unz (\emph{norm.}) ‘until’ \Upsaliensis}} verǫld \edtext{stęypisk}{\lemma{stęypisk ‘tumbles down’}\Afootnote{grundir gjalla \hld\ gífr fljúgandi (\emph{norm.}) ‘foundations shrill, fiends flying’ \emph{add. after this line} \Hauksbok}} &
\edtext{mun \edtext{ęngi}{\Afootnote{†enn† \Upsaliensis}} maðr \hld\ ǫðrum þyrma.}{\lemma{mun \dots\ þyrma ‘before \dots\ spare’}\Bfootnote{\emph{om.} \RegiusProse\Trajectinus\Wormianus}}\eva

\bvb Brothers will fight, and become each other’s slayers; sister’s sons will spill their kinship.\footnoteB{Whether through incest or treachery. TODO: literary evidence of the phrase \emph{spilla sifjum}.} ’Tis hard in the Home, whoredom great: axe-eld, sword-eld—shields are rent—wind-eld, warg-eld; before the world\footnoteB{\emph{ver-ǫld} ‘world’ is literally ‘man-eld’, ‘the eld of man’.} tumbles down, no man will another spare.\evb
\evg


\bvg
\bva\mssnote{\Regius~2r/32, \Hauksbok~20v/27, \GylfMS}\edtext{Lęika Míms synir, \hld\ ęn mjǫtuðr kyndisk &
at hinu galla \hld\ Gjallarhorni; &
hǫ́tt blę́ss Hęimdallr, \hld\ horn ’s ȧ lopti; &
\edtext{mę́lir}{\lemma{mę́lir ‘speaks’}\Afootnote{†mey† \RegiusProse; †nie† \Trajectinus}} Óðinn \hld\ við Míms hǫfuð.}{\lemma{Lęika \dots\ hǫfuð.}\Bfootnote{In \GylfMS\ ll. 1–2 (\emph{Lęika \dots\ Gjallarhorni;} ‘Play \dots\ Horn of Yell.’) are missing, and ll. 3–4 (\emph{hǫ́tt \dots\ hǫfuð.} ‘High \dots\ head [of Mime.]’) are instead paired with the first two lines of the next v. (\emph{Skęlfr \dots\ losnar;})}}\eva

\bvb Play the sons of Mime, and the Metted is kindled, at [the sounding of] the shrill Horn of Yell. Loudly blows Homedall; the horn is aloft; Weden speaks with the head of Mime.\evb
\evg


\bvg
\bva\mssnote{\Regius~2v/3, \Hauksbok~20v/28, \GylfMS}\edtext{Skęlfr Yggdrasils \hld\ askr standandi, &
ymr it aldna tré, \hld\ ęn jǫtunn losnar;}{\lemma{Skęlfr \dots\ losnar ‘Quakes \dots\ loosens’}\Bfootnote{thus \Hauksbok\GylfMS; in \Regius\ the two lines are reversed.}} &
\edtext{hrę́ðask allir \hld\ ȧ hęlvegum &
áðr Surtar þann \hld\ sefi of glęypir.}{\lemma{hrę́ðask allir \dots\ glęypir ‘All are frightened \dots\ devour [it.]’}\Bfootnote{Only in \Hauksbok.}} \eva

\bvb Quakes the Ash of Ugdrassle, standing; groans the old tree, and the ettin loosens. All are frightened on the Hell-ways, before Surt’s kinsman does devour it.\evb
\evg


\bvg
\bva\mssnote{\Regius~2v/8, \Hauksbok~20v/30, \GylfMS}Hvat ’s með ǫ̇sum? \hld\ hvat ’s með \edtext{ǫlfum}{\lemma{ǫlfum ‘Elves’}\Afootnote{asynivm ‘Ossens’ \Upsaliensis}}? &
\edtext{gnýr allr Jǫtunhęimr, \hld\ ę̇sir ’ru ȧ þingi,}{\lemma{gnýr \dots\ þingi}\Afootnote{\emph{om.} \Upsaliensis}} &
stynja dvergar \hld\ fyr \edtext{stęindurum}{\Afootnote{steins \Upsaliensis — -dyrum \Hauksbok\Wormianus\Upsaliensis}} &
\edtext{\edtext{vęggbergs}{\lemma{vęggbergs ‘wedge-rock’}\Afootnote{vegbergs ‘way-rock’ \Hauksbok\Trajectinus\Wormianus}} vísir}{\Afootnote{\emph{om.} \Upsaliensis}} — \hld\ vituð ér ęnn eða hvat?\eva

\bvb What is with the Ease? What is with the Elves? Roars all Ettinham, the Ease are at the Thing. Dwarfs groan before gates of stone, the princes of the wedge-rock—know ye yet, or what?\evb
\evg


\bvg
\bva\mssnote{\Regius~2v/4, \Hauksbok~20v/32}Gęyr nú Garmr mjǫk \hld\ fyr Gnipahęlli, &
fęstr mun slitna, \hld\ ęn Freki rinna; &
fjǫlð vęit hǫ̇n frǿða, \hld\ framm sé’k lęngra &
of ragna rǫk, \hld\ rǫmm sigtíva.\eva

\bvb Barks now Garm loudly before the Gnip-caverns; the rope will tear, and Freck run. Much she knows of learning, forth I see yet further; about the mighty Rakes of the Reins, of the victory-tues.\evb
\evg


\bvg
\bva\mssnote{\Regius~2v/4, \Hauksbok~20v/32, \RegiusProse\Trajectinus\Wormianus}Hrymr ękr austan, \hld\ hęfsk lind fyrir, &
snýsk Jǫrmungandr \hld\ í jǫtunmóði; &
ormr knýr unnir, \hld\ \edtext{ęn ari hlakkar}{\lemma{ęn ari hlakkar ‘but the eagle screams’}\Afootnote{ǫrn mun hlakka ‘the eagle will scream’ \RegiusProse\Trajectinus}}, &
slítr nái neffǫlr; \hld\ Naglfar losnar.\eva

\bvb Rim drives from the east, holding his shield before himself; Ermingand writhes about in ettin’s wrath. The worm propels the waves, but the eagle screams: the pale-beak tears corpses; Nailfare loosens.\evb
\evg


\bvg
\bva\mssnote{\Regius~2v/6, \Hauksbok~20v/34, \RegiusProse\Trajectinus\Wormianus}Kjóll fęrr austan \hld\ koma munu Múspells &
of lǫg lýðir, \hld\ ęn Loki stýrir; &
fara fíflmęgir \hld\ með Freka allir, &
þęim es bróðir \hld\ Býlęists í fǫr.\eva

\bvb A ship travels from the east—come will Muspell’s subjects by sea—but Lock steers it. Travel the warlocks all with Freck; with them comes the brother of Bylest \ken*{= Lock} along.\evb
\evg


\bvg
\bva\mssnote{\Regius~2v/10, \Hauksbok~20v/36, \GylfMS}\edtext{Surtr}{\Afootnote{Svartr \Upsaliensis}} fęrr sunnan \hld\ með sviga lę́vi, &
skínn af sverði \hld\ sól valtíva; &
grjótbjǫrg gnata, \hld\ ęn \edtext{gífr rata}{\Afootnote{guðar hrata ‘[but] the gods stagger’ \Upsaliensis \emph{is wo. doubt corrupt, the anachronistic masc. pl. of} guð \emph{is proof enough}}}, &
troða halir hęlveg, \hld\ ęn himinn klofnar.\eva

\bvb Surt comes from the south with the betrayer of the stick \ken{fire}; from the sword shines the sun of the slain-Tues. Boulders clash, but the fiends reel; men march on the \inx[L]{Hell-ways}, but heaven is cloven.\evb
\evg


\bvg
\bva\mssnote{\Regius~2v/13, \Hauksbok~20v/37, \RegiusProse\Trajectinus\Wormianus}Þȧ kømr Hlínar \hld\ harmr annarr framm, &
es Óðinn fęrr \hld\ við ulf vega, &
—ęn bani Bęlja \hld\ bjartr at Surti— &
þȧ mun Friggjar \hld\ falla \edtext{angan}{\Afootnote{angantyr \Regius}}.\eva

\bvb Then comes \inx[P]{Line}’s second sorrow to pass,\footnoteB{That the first sorrow was the death of Balder (see vv. 31–33) is unanimously understood. Line is described in \Gylfaginning\ 35 as a minor goddess \emph{sett til gæzlu yfir þeim mönnum, er Frigg vill forða við háska nökkurum} ‘placed to watch over those men which Frie wishes to protect against any particular danger’. In spite of this, almost all translators and commentors have understood Line as here referring to Frie, or questioned whether her existence as a separate goddess is not a misunderstanding on the part of the author of \Gylfaginning. \textcite{Hopkins2017} argues excellently that this need not be the case; as a subordinate goddess of Frie, Line’s two sorrows would be her failing to protect Balder and Weden (the son and husband of her mistress, respectively) from harm.} as Weden goes to strike against the wolf—but the bane of \inx[P]{Bellow} \ken*{= Free}, bright, [goes] against Surt—then will Frie’s beloved \ken*{= Weden} fall.\evb
\evg


\bvg
\bva\mssnote{\Regius~2v/15, \RegiusProse\Trajectinus\Wormianus}\edtext{Þȧ kømr hinn mikli \hld\ mǫgr Sigfǫður}{\lemma{Þȧ kømr \dots\ Sigfǫður ‘Then \dots\ Sighfather’}\Afootnote{Gęngr Óðins sonr \hld\ við ulf vega ‘Goes Weden’s son against the wolf to fight’ \GylfMS}}, &
Víðarr \edtext{vega}{\Afootnote{of veg \GylfMS}} \hld\ at valdýri; &
lę́tr męgi Hveðrungs \hld\ mund of standa &
hjǫr til hjarta; \hld\ þȧ ’s hefnt fǫður.\eva

\bvb Then comes the great lad of \inx[P]{Sighfather} \name{= Weden}: Wider, to strike at the murderous beast. He lets his hand plunge the sword into the heart of \inx[P]{Whethring}’s \name{= Lock} lad \ken*{= Wolf}; then is the father \ken*{= Weden} avenged.\evb
\evg


\bvg
\bva\mssnote{\Hauksbok~20v/39}\edtext{Gínn lopt yfir \hld\ lindi jarðar, &
gapa ýgs kjaptar \hld\ orms í hę́ðum; &
mun Óðins son \hld\ \edtext{ęitri}{\lemma{ęitri ‘venom’}\Afootnote{emend.; \emph{ormi} ‘the worm’ \Hauksbok. It seems likely that the author of \Gylfaginning\ had access to this verse. Cf. \Gylfaginning\ 51: “Thunder bears the bane-word from the Middenyardsworm and thence strides away nine paces. Then he falls dead to the earth due to the venom (\emph{ęitri}) which the Worm blows on him.”}} mǿta &
vargs at \edtext{dauða}{\Afootnote{da... \Hauksbok}} \hld\ Víðars niðja.}{\lemma{Gínn \dots\ niðja.}\Bfootnote{The final part of this verse is almost completely illegible. For the present edition I have relied on the reading of \textcite[13,44\psqq]{JonHelgason1971}.}}\eva

\bvb Yawns over the air the girdle of the earth \ken*{= Middenyardsworm}; gape the jaws of the fierce worm in the heights. The venom of the beast will meet Weden’s son \ken*{= Thunder}, after the deaths of Wider’s kinsmen \ken*{= the Ease}.\evb
\evg


\bvg
\bva\mssnote{\Regius~2v/17, \Hauksbok~20v/41, \RegiusProse\Trajectinus\Wormianus}\edtext{Þȧ kømr}{\lemma{þȧ kømr ‘then comes’}\Afootnote{\emph{Gęngr} ‘goes’ \GylfMS}} hinn mę́ri \hld\ mǫgr Hlǫðynjar &
\edtext{gęngr Óðins sonr \hld\ við orm vega.}{\lemma{gęngr \dots\ vega}\Afootnote{Only in \Regius}} &
\edtext{Drepr af móði \hld\ Miðgarðs véurr; &
munu halir allir \hld\ hęimstǫð ryðja; &
gęngr fet níu \hld\ Fjǫrgynjar burr &
nęppr frȧ naðri, \hld\ níðs ȯkvíðnum.}{\lemma{Drepr \dots\ ȯkviðnum ‘Middenyard’s \dots\ adder’}\Afootnote{\emph{neppr af naðri / niðs ȯkvíðnum / munu halir allir / hęimstǫð ryðja, / es af móði drepr / Miðgarðs véurr} ‘[Goes the renowned lad of Lathyn,] pained, away from the loathsome adder. All men will clear their homesteads, when Middenyard’s wigh-ward strikes out of wrath.’ \GylfMS}}\eva

\bvb Then comes the renowned lad of Lathyn \ken*{= Thunder}: the son of Weden goes the \inx[C]{worm} to meet. Middenyard’s Wigh-ward strikes out of wrath; all men will clear their homesteads.\footnoteB{It seems likely that the order found in \Gylfaginning\ is original. After Thunder (appropriately kenned ‘Middenyard’s wigh-ward’) is slain, the Ettins take over the lands and make farming impossible. Cf. \Thrymskvida\ 18: “Shortly the Ettins will settle Osyard, unless thou thy hammer for thyself dost fetch!”} The son of Firgyn goes nine paces, pained, away from the loathsome adder \ken*{= Middenyardsworm}.\footnoteB{Thunder, mortally wounded, struggles nine steps away from the Worm before he falls. See note to previous verse.}\evb
\evg


\bvg
\bva\mssnote{\Regius~2v/20, \Hauksbok 21r/1, \GylfMS}Sól tér sortna, \hld\ \edtext{søkkr fold í mar}{\lemma{søkkr \dots\ mar}\Bfootnote{This line is very similar to a line of v. 24 in Arnthur ‘earl-scold’ Thurthson’s Drape of Thurfinn (\Skp: Arn \emph{Þorfdr} 24\textsuperscript{II}): \emph{søkkr fold í mar døkkvan} ‘sinks the fold into the dark sea’. For this reason, \emph{søkkr} ‘sinks’ \RegiusProse\Trajectinus\Wormianus\ has been chosen over \emph{sígr} ‘descends’ \Regius\Hauksbok\Upsaliensis.}}, &
hverfa af himni \hld\ hęiðar stjǫrnur; &
gęisar ęimi \hld\ við aldrnara; &
lęikr hǫ́r hiti \hld\ við himin sjalfan.\eva

\bvb The sun does blacken, sinks the fold \ken{earth} into the sea; disappear off heaven the clear stars. Rages smoke from the nourisher of life \ken*{fire}; licks the high heat heaven itself.\evb
\evg


\bvg
\bva\mssnote{\Regius~2v/22, \Hauksbok 21r/2}Gęyr nú Garmr mjǫk \hld\ fyr Gnipahęlli, &
fęstr mun slitna, \hld\ ęn Freki rinna; &
fjǫlð vęit hǫ̇n frǿða, \hld\ framm sé’k lęngra &
of ragna rǫk, \hld\ rǫmm sigtíva.\eva

\bvb Barks now Garm loudly before the Gnip-caverns; the rope will tear, and Freck run. Much she knows of learning, forth I see yet further; about the mighty Rakes of the Reins, of the victory-tues.\evb
\evg


\bvg
\bva\mssnote{\Regius~2v/23, \Hauksbok 21r/4}Sér hǫ̇n upp koma \hld\ ǫðru sinni &
jǫrð ór ę́gi \hld\ iðjagrø̇na; &
falla forsar, \hld\ flýgr ǫrn yfir, &
sá’s ȧ fjalli \hld\ fiska vęiðir.\eva

\bvb Up she sees coming, another time: the earth out of the ocean, ever green anew. Fall torrents; flies an eagle above, he who on the fells fish does catch.\evb
\evg


\bvg
\bva\mssnote{\Regius~2v/24, \Hauksbok 21r/5}\edtext{Finnask}{\lemma{finnask ‘find each other’}\Afootnote{\emph{hittask} \Hauksbok\ provides closer parallelism with v. 7.}} ę̇sir \hld\ ȧ Iðavęlli &
ok umb moldþinur \hld\ mǫ́tkan dø̇ma, &
\edtext{ok minnask þar \hld\ ȧ męgindȯma}{\lemma{ok minnask \dots\ męgindȯma ‘and remember \dots\ mighty judgements’}\Afootnote{om. \Regius}} &
ok ȧ Fimbultýs \hld\ fornar ru̇nar.\eva

\bvb The Ease find each other on Idewold, and about the mighty earth-strip \ken*{the Middenyardsworm} converse, and there look back on mighty verdicts, and on Fimbletue’s \name{Weden’s} ancient runes.\evb
\evg

\bvg
\bva\mssnote{\Regius~2v/26, \Hauksbok 21r/7}Þar munu ęptir \hld\ undrsamligar &
gollnar tǫflur \hld\ í grasi finnask, &
þę́r’s í árdaga \hld\ áttar hǫfðu.\eva

\bvb There will afterwards wondrous golden Tavel-bricks in the grass be found: those which in days of yore they had owned.\footnoteB{Cf. v. 9. The rediscovering of the golden game pieces symbolizes a new golden age.}\evb
\evg


\bvg
\bva\mssnote{\Regius~2v/28, \Hauksbok 21r/9}Munu ȯsánir \hld\ akrar vaxa; &
bǫls mun alls batna \hld\ mun Baldr koma; &
búa Hǫðr ok Baldr \hld\ Hropts sigtoptir, &
vęl valtívar. \hld\ Vituð ér ęnn eða hvat?\eva

\bvb Unsown will fields grow; the bale will all be bettered; Balder will come. Hath and Balder bedwell the victory-plots of Roft \name{= Weden}—well, the slain-Tues—know ye yet, or what?\footnoteB{The evil of Hath’s slaying Balder will be forgotten as the two peacefully live together.}\evb
\evg


\bvg
\bva\mssnote{\Regius~2v/30, \Hauksbok 21r/11}Þȧ kná Hø̇nir \hld\ hlautvið kjósa &
ok burir byggva \hld\ \edtext{brǿðra tvęggja}{\lemma{brǿðra tvęggja ‘of two brothers’}\Bfootnote{Alternatively \emph{brǿðra Tvęggja} ‘the brothers of Tway \name{= Weden}’, attested in \Gylfaginning\ 6 as \inx[P]{Will} and \inx[P]{Wigh}, but they are never attested as having children, and it is thus more natural to read \emph{tvęggja} as the gen. pl. of \emph{tvęir} ‘two’.}} &
vindhęim víðan. \hld\ Vituð ér ęnn eða hvat?\eva

\bvb Then does Heener choose the \inx[C]{leat}-wood,\footnoteB{Restore the bloot and practice divination.} and the sons of two brothers \ken*{= Hath and Balder} settle the wide wind-home \ken{heaven}—know ye yet, or what?\evb
\evg


\bvg
\bva\mssnote{\Regius~2v/31, \Hauksbok 21r/12, \GylfMS}Sal \edtext{sér hǫ̇n}{\lemma{sér hǫ̇n ‘she sees’}\Afootnote{\emph{vęit’k} ‘I know’ \GylfMS}} standa \hld\ sólu fęgra, &
\edtext{golli þakðan}{\lemma{golli þakðan ‘thatched with gold’}\Afootnote{\emph{golli bętra} ‘better than gold’ \RegiusProse\Trajectinus}}, \hld\ ȧ \edtext{Gimléi}{\Afootnote{metr. emend.; \emph{Gimlé} \Regius\Hauksbok\GylfMS}}; &
\edtext{þar}{\lemma{þar ‘there’}\Afootnote{\emph{þann} ‘[in] that [hall]’ \Trajectinus\Wormianus}} skulu dyggvar \hld\ dróttir byggva &
ok umb aldrdaga \hld\ ynðis njóta.\eva

\bvb A hall she sees standing, fairer than the sun: thatched with gold, on Gemlee; there dutiful men shall dwell, and during their life-days enjoy delight.\evb
\evg


\bvg
\bva\mssnote{\Regius~3r/2, \Hauksbok 21r/15}Þar kømr hinn dimmi \hld\ dręki fljúgandi, &
naðr frȧnn neðan \hld\ frȧ Niðafjǫllum; &
berr sér í fjǫðrum \hld\ —flýgr vǫll yfir— &
Níðhǫggr nái; \hld\ nú mun hǫ̇n søkkvask.“\eva

\bvb Then comes the shadowy dragon flying; the gleaming adder down below from the \inx[L]{Nithefells}. Nithehewer in his feathers—flying over the field—carries corpses.” — Now she will sink!\footnoteB{The wallow, referring to herself in third person, descends back down into her grave, whence Weden woke her. See Introduction.}”\evb
\evg


\bvg
\bva[X]\mssnote{\Hauksbok 21r/14}\edtext{Þȧ kømr hinn ríki \hld\ at ręgindȯmi &
ǫflugr ofan \hld\ sá’s ǫllu rę́ðr.}{\lemma{Þȧ \dots\ rę́ðr.}\Bfootnote{This verse is found only in \Hauksbok, in between the last two vv. It is without doubt a late, Christian addition.}}\eva

\bvb[X] — Then comes the mighty one, for the great judgement; strong from above, the one who over all things wields.\evb
\evg
