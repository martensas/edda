\bookStart{The Spae of the Wallow}[Vǫluspǫ́]

% Introduction.

The \textbf{\inx[C]{spae}[Spae] of the \inx[C]{wallow}[Wallow]} is the most comprehensive mythological text surviving from Heathen times. It takes the form of the monologue of a wallow, summoned by Weden in order to relate mythological knowledge. In this it fits closely with \Vafthrudnismal, \Grimnismal, \Sigrdrifumal\ and \Allvismal, but differs from them in several ways: there is no format of a dialogue (this it shares with \Grimnismal) or competition; the meter is in \Fornyrdislag; and it gives an overview of the mythological chronology in an otherwise unparalleled way.

Events are related in a very allusive fashion, and not all of them are clear. There are also some likely gaps, possibly the result of misplaced verses. The poem begins with a bid for silence (v. 1), and the wallow reckoning her earliest memories (v. 2). She then recounts the ordering of the cosmos by the gods (vv. 3–6) and the earliest golden age (vv. 7–8), which however is interrupted by the intrusion of three unidentified ettin maidens (v. 8, and see note there). After this follow two verses about the shaping of the dwarfs (9–10), and then several independent \emph{dwarf-tallies} (vv. 11–15), which are undoubtedly later inserts. We then return to the gods, specifically the creation of man (vv. 16–17). Judging from the end of verse 8 and the beginning of verse 16, it seems likely that these various dwarf-related verses have taken the place of some other verse. After this we get a description of the great tree Ugdrassle (v. 18), and the three norns living under it (v. 19).

This is where our two full recensions diverge. We have here followed the order of \Regius, but whether it is the original is hard to say. In \Regius\ the wallow recounts the earliest war in the world %TODO

The poem is attested in full in two independent recensions. The first is \Regius\ (GKS 2365 4to; 1270s), where it is the first poem, found on folios 1r–3r. Second is Hawksbook, \Hauksbok\ (AM 544 4to; 1300–75), where it is found at 20r–21r in the middle of a large collection of saws and Catholics works. Many verses are also cited in \Gylfaginning, which here has the general siglum \GylfMS—to avoid confusion, it is only used when all employed witness mss. agree. See further the General Introduction. %TODO: elaborate?

Order of verses by manuscript, compared to this edition. As most verses in \GylfMS\ are quoted on their own, and have little relation to the original order, these are simply marked with plus signs. When verses are quoted in a series, they are preceded by an alphabetically incrementing letter denoting which series they belong to. When there is a major difference in a ms. relative to the ed., such as in v. 10 where \GylfMS\ omits the first two lines, it is then marked with a star. The verses beginning with \emph{Þȧ gingu ręgin ǫll ...} are represented by the following sentence.
\begin{longtable}{|c c c c c c|}
	\hline
	\multicolumn{2}{|c}{\emph{Current ed.}} & \Regius & \Hauksbok & \RegiusProse\Trajectinus\Wormianus & \Upsaliensis \\ [0.5ex]
	\hline\hline
	1 & Hljóðs bið’k allar hęlgar kindir & 1 & 1 & − & − \\
	2 & Ek man jǫtna ár of borna & 2 & 2 & − & − \\
	3 & Ár vas alda þar’s Ymir byggði & 3 & 3 & + & + \\
	4 & Áðr Burs synir bjǫðum of ypðu & 4 & 4 & − & − \\
	5 & Sól varp sunnan sinni mȧna & 5 & 5 & +* & +* \\
	6 & ... nǫ́tt ok niðjum nǫfn of gǫ́fu & 6 & 6 & − & − \\
	7 & Hittusk ę̇sir ȧ Iðavęlli & 7 & 7 & − & − \\
	8 & Tęflðu í túni, tęitir vǫ́ru & 8 & 8 & − & − \\
	9 & ... hvęrr skyldi dverga drótt of skępja & 9 & 9 & B1 & B1 \\
	10 & Þar vas Móðsognir mę́ztr of orðinn & 10 & 10 & B2* & B2* \\
	− & \emph{Dwarf-tallies} & 11–15 & 11–16 & + & + \\
	16 & Unz þrír kvǫ̇mu ór því liði & 16 & 17 & − & − \\
	17 & Ǫnd þau né ǫ́ttu, óð þau né hǫfðu & 17 & 18 & − & − \\
	18 & Ask vęit’k standa hęitir Yggdrasill & 18 & 19 & + & + \\
	19 & Þaðan koma męyjar margs vitandi & 19–20 & 20–21 & − & − \\
	20 & Þat man hǫ̇n folkvíg fyrst í hęimi & 21–22 & 27 & − & − \\
	21 & Hęiði hétu, hvar’s til húsa kom & 23 & 28 & − & − \\
	22 & ... hvárt skyldu ę̇sir afráð gjalda & 24 & 29 & − & − \\
	23 & Flęygði Óðinn ok í folk of skaut; & 25 & 30 & − & − \\
	24 & ... hvęrr hęfði lopt alt lę́vi blandit  & 26 & 22 & C1 & C1 \\
	25 & Þȯrr ęinn þar vá þrunginn móði & 27 & 23 & C2* & C2* \\
	26 & Vęit hǫ̇n Hęimdallar hljóð of folgit & 28 & 24 & − & − \\
	27 & Ęin sat hǫ̇n úti, þȧ’s hinn aldni kom & 29 & − & − & − \\
	28 & Alt vęit’k, Óðinn, hvar auga falt & 29 & − & + & + \\
	29 & Valði hęnni Hęrfǫðr hringa ok męn & 30 & − & − & − \\
	30 & Sá hǫ̇n valkyrjur vítt of komnar & 31 & − & − & − \\
	31 & Ek sá Baldri, blóðgum tívi & 32 & − & − & − \\
	32 & Varð af męiði, þęim’s mę́r sýndisk & 33 & − & − & − \\
	33 & Þó hann ę́va hęndr né hǫfuð kęmbði & 34 & − & − & − \\
	34 & Þȧ kná Váli vígbǫnd snúa & − & 31 & − & − \\
	35 & Hapt sá hǫ̇n liggja und Hveralundi & 35 & 32* & − & − \\
	36 & Ǫ́ fęllr austan of ęitrdala & 36 & − & − & − \\
	37 & Stóð fyr norðan ȧ Niðavǫllum & 36 & − & − & − \\
	38 & Sal sá hǫ̇n standa sólu fjarri & 37 & 36 & E1 & E1 \\
	39 & Sér hǫ̇n þar vaða þunga strauma & 38 & 37 & E2* & E2* \\
	40 & Austr býr hin aldna í Járnviði & 39 & 25 & A1 & A1 \\
	41 & Fyllisk fjǫrvi fęigra manna & 40 & 26 & A2 & A2 \\
	42 & Sat þar ȧ haugi ok sló hǫrpu & 41 & 34 & − & − \\
	43 & Gól of ǫ̇sum Gollinkambi & 42 & 35 & − & − \\
	44, 49, 57 & Gęyr Garmr mjǫk fyr Gnipahęlli & 43, 46, 55 & 33, 38, 43, 48, 51 & − & − \\
	45 & Brǿðr munu bęrjask ok at bǫnum verðask, & 44 & 39 & − & − \\
	46 & Lęika Míms synir, ęn mjǫtuðr kyndisk & 45 & 40 & D1* & D1* \\
	47 & Skęlfr Yggdrasils askr standandi & 45* & 41 & D1* & D1* \\
	48 & Hvat ’s með ǫ̇sum? hvat ’s með ǫlfum? & 49 & 42 & D2 & D2* \\
	50 & Hrymr ękr austan, hęfsk lind fyrir & 47 & 44 & D3 & − \\
	51 & Kjóll fęrr austan koma munu Múspells & 48 & 45 & D4 & − \\
	52 & Surtr fęrr sunnan með sviga lę́vi & 50 & 46 & +, D5 & + \\
	53 & Þȧ kømr Hlínar harmr annarr framm & 51 & 47 & D6 & − \\
	54 & Þȧ kømr hinn mikli mǫgr Sigfǫður & 52 & − & D7 & − \\
	55 & Gínn lopt yfir lindi jarðar & − & 48 & — & − \\
	56 & Þȧ kømr hinn mę́ri mǫgr Hlǫðynjar & 53* & 49* & C8 & − \\
	57 & Sól tér sortna, søkkr fold í mar & 54 & 50 & C9 & − \\
	59 & Sér hǫ̇n upp koma ǫðru sinni & 56 & 52 & − & − \\
	60 & Finnask ę̇sir ȧ Iðavęlli & 57* & 53 & − & − \\
	61 & Þar munu ęptir undrsamligar & 58 & 54 & − & − \\
	62 & Munu ȯsánir akrar vaxa & 59 & 55 & − & − \\
	63 & Þȧ kná Hø̇nir hlautvið kjósa & 60 & 56 & − & − \\
	64 & Sal sér hǫ̇n standa sólu fęgra & 61 & 57 & + & + \\
	65 & Þar kømr hinn dimmi dręki fljúgandi & 62 & 59 & − & − \\
	X & Þȧ kømr hinn ríki at ręgindȯmi & − & 58 & − & − \\ [1ex]
	\hline
\end{longtable}

\bvg {\small Greeting to the audience, bidding of Weden.}
\bva\ledleftnote{\Regius\Hauksbok}\alst{H}ljóðs bið’k allar \hld\ \edtext{\alst{h}ęlgar}{\lemma{hęlgar}\Afootnote{\emph{om.} \Regius}} kindir, &
\alst{m}ęiri ok \alst{m}inni \hld\ \alst{m}ǫgu Hęimdallar; &
\alst{v}ildu at, \alst{V}alfǫðr, \hld\ \alst{v}ęl fram tęlja’k &
\alst{f}orn spjǫll \alst{f}ira, \hld\ þau’s \alst{f}ręmst of man?\eva

\bvb For hearing I ask all holy kindreds, greater and lesser, sons of Homedall\footnoteB{Cf. \Rigsthula, wherein Righ, identified by the prose as Homedall, sires three castes of men (namely earls, churls and thralls). — \emph{męiri ok minni} “greater and lesser” may be understood in two ways. It either modifies “holy kindreds”, in which case it could be equivalent to a phrase like “\inx[G]{Ease and Elves}” (i.e. both earthly and heavenly supernatural beings; see Index for occurences.), or “the sons of Homedall”, in which case it refers to all social classes. In any case she is asking all intelligent beings that may be present for silence; the expression is a merism, see West (2007), 99–100.} \ken{men}! Wilt thou, Walfather \name{= Weden}, that I well count forth the ancient sayings of men, those which I foremost recall?\footnoteB{Cf. \Vafthrudnismal\ 34, 35 with very similar phrasing. The whole introductory formula is positively Indo-European, see West (2007), 63, 92–93, 312.}\evb
\evg


\bvg {\small Wallow reckons what she recalls; the creation and ordering of the world.}
\bva\ledleftnote{\Regius\Hauksbok}Ek man jǫtna \hld\ ár of borna, &
þȧ es forðum \hld\ mik fǿdda hǫfðu; &
níu man’k hęima, \hld\ níu \edtext{íviðjur}{\Afootnote{\emph{Previously read} íviði, \emph{but closer study of} \Regius\ \emph{has disproven this. See Stefán Karlsson 1979.}}}, &
mjǫtvið mę́ran \hld\ fyr mold neðan.\eva

\bvb I recall \inx[G]{Ettins}, born of yore, those who anciently had nourished me. Nine \inx{Homes} I recall, nine \inx[G]{Inwithies}; the renowned \inx[P]{Metwood} beneath the soil.\footnoteB{Certainly \inx[P]{Ugdrassle}, “beneath the soil” likely referring to it still being a seed.}\evb
\evg


\bvg
\bva\ledleftnote{\Regius\Hauksbok\GylfMS}Ár vas alda \hld\ \edtext{þar’s Ymir byggði}{\lemma{þar’s ... byggði “there ... dwelled”}\Afootnote{þat’s ekki vas “that which nothing was” \GylfMS}}, &
vas-a sandr né sę́r, \hld\ né svalar unnir; &
jǫrð fansk ę́va \hld\ né upphiminn; &
gap vas ginnunga, \hld\ ęn gras \edtext{hvęrgi}{\Afootnote{ekki \Hauksbok}}.\eva

\bvb It was the beginning of \inx[C]{eld}[elds], there where \inx[P]{Yimer} dwelled; was there not sand nor sea, nor cool waves. Earth was never found, nor \inx[L]{Up-heaven}; a \inx[L]{gap of ginnings}[gap was of ginnings],\footnoteB{\emph{ginnungr} (of which \emph{ginnunga} would be the genitive plural) means ‘hawk’ in the Scoldish poetry, but that meaning hardly makes sense here, unless it is taken as an obscure sky-kenning. In any case it refers to the primeval void.} but grass nowhere.\evb
\evg


\bvg
\bva\ledleftnote{\Regius\Hauksbok}Áðr Burs synir \hld\ bjǫðum of ypðu, &
þęir es Miðgarð \hld\ mę́ran skópu; &
sól skęin sunnan \hld\ ȧ salar stęina; &
þȧ vas grund gróin \hld\ grø̇num lauki.\eva

\bvb Before the sons of \inx[P]{Byre} the flatlands did upwards lift, they who shaped the renowned \inx[L]{Middenyard}. Sun shone from the south on the stones of the hall; then was the ground grown with green leek.\footnoteB{The sons of Byre, that is Weden, Will and Wigh (cf. \Gylfaginning\ TODO), lift the lands out of the primordial chaos.}\evb
\evg


\bvg
\bva\ledleftnote{\Regius\Hauksbok\GylfMS}\edtext{Sól varp sunnan, \hld\ sinni mȧna, &
hęndi hinni hǿgri \hld\ \edtext{of himinjǫður}{\Afootnote{vm himin iodyr \Regius\ of ioður \Hauksbok}}}{\lemma{Sól ... himinjǫður}\Afootnote{\emph{om.} \GylfMS}}; &
sól þat né vissi, \hld\ hvar hǫ̇n sali átti; &
\edtext{stjǫrnur þat né vissu, \hld\ hvar þę́r staði ǫ́ttu}{\lemma{stjǫrnur ... ǫ́ttu}\Bfootnote{In \GylfMS\ follows 5, so that order is sun, moon, stars.}}; &
mȧni þat né vissi, \hld\ hvat hann męgins átti.\eva

\bvb Sun cast from the south—the companion of \inx[P]{Moon}\footnoteB{At times translated as “its moon”; this cannot be correct, as \emph{mȧni} ‘moon’ is masculine, while \emph{sinni}, dative singular of \emph{sínn} ‘its (reflexive)’ is feminine.}—her right hand over heaven’s rim;\footnoteB{The sun heaved herself up over the horizon and rose for the first time.} Sun knew not, where halls she owned; stars knew not, where steads they owned; Moon knew not, what sort of might he owned.\evb
\evg


\bvg
\bva\ledleftnote{\Regius\Hauksbok}Þȧ gingu ręgin ǫll \hld\ ȧ rǫkstóla, &
ginnhęilǫg goð, \hld\ ok umb þat gę́ttusk. &
Nǫ́tt ok niðjum \hld\ nǫfn of gǫ́fu, &
morgin hétu \hld\ ok miðjan dag, &
undurn ok aptan, \hld\ ǫ́rum at tęlja.\eva

\bvb Then went the Powers all onto the rake-seats\footnoteB{Judgment-seats; first element \emph{rǫk} defined by \CV\ as ‘reason, ground, origin’.}: the gin-holy gods, and from each other took counsel about that.\footnoteB{10, 23, 25 (TODO) would suggest two lines be missing here.}—To night and the moon-phases names did they give; morning they called, and middle day; afternoon and evening, the years for to tally.\footnoteB{Cf. \emph{Web} 23, 25.}\evb
\evg


\bvg
\bva\ledleftnote{\Regius\Hauksbok}Hittusk ę̇sir \hld\ ȧ Iðavęlli, &
\edtext{þęir’s hǫrg ok hof \hld\ hǫ́ timbruðu}{\lemma{þęir’s ... timbruðu “they ... timbered”}\Afootnote{afls kostuðu \hld\ allz freistuðu “[their] strength they tried; all they tempted” \Hauksbok}}; &
afla lǫgðu, \hld\ auð smíðuðu, &
tangir skópu \hld\ ok tól gęrðu.\eva

\bvb The Ease found each other on the \inx[L]{Idewolds}, they who \inx[C]{harrow}[harrows] and \inx[L]{hove}[hoves] high timbered; hearths they laid, wealth they smithed; tongs they shaped, and tools they made.\evb
\evg


\bvg
\bva\ledleftnote{\Regius\Hauksbok}Tęflðu í túni, \hld\ tęitir vǫ́ru, &
vas þęim véttugis \hld\ vant ór golli, &
unz þríar kvǫ̇mu \hld\ þursa męyjar, &
ȧmátkar mjǫk, \hld\ ór Jǫtunhęimum.\eva

\bvb They played \inx[C]{Tavel} in the yards, joyous were they: was for them no lack of gold\footnoteB{Cf. v. 59.}—until three came, maidens of \inx[G]{Thurses}, greatly loathsome out of \inx[L]{Ettinham}.\footnoteB{These are immediately forgotten and not again mentioned (unless they are taken to be the norns in v. 21, but they would then be introduced twice).—There seems to be something missing between here, perhaps giving further information of the three thurse-maidens, or detailing the reason for the creation of dwarfs?}\evb
\evg


\bvg {\small Creation of dwarfs.}
\bva\ledleftnote{\Regius\Hauksbok\GylfMS}Þȧ gingu ręgin ǫll \hld\ ȧ rǫkstóla, &
ginnhęilǫg goð, \hld\ ok umb þat gę́ttusk: &
\edtext{hvęrr skyldi dverga}{\lemma{hvęrr skyldi dverga “Who would ... of dwarfs”}\Afootnote{\emph{thus} \Regius\Wormianus\Upsaliensis; at skyldi dverga “That they would ... of dwarfs” \RegiusProse\Trajectinus; hverir skyldu dvergar “Which dwarfs would [shape the people]” \Hauksbok}} \hld\ \edtext{drótt of}{\Afootnote{\emph{thus} \GylfMS; drotin (\emph{late definite wo. doubt not original}) \Regius; dróttir “the people” \Hauksbok}} \edtext{skępja}{\Afootnote{spekia “soothe [the troop]” \Upsaliensis}} &
\edtext{ór \edtext{brimi blóðgu}{\lemma{brimi blóðgu “bloody surf”}\Afootnote{\emph{thus} \Hauksbok\RegiusProse\Wormianus\Upsaliensis; Brimis blóði “the blood of Brimmer” \Regius\Trajectinus}} \hld\ ok ór \edtext{blǫ́um lęggjum}{\lemma{blǫ́um lęggjum “blue-black legs”}\Afootnote{\emph{metr. emend}; ‘blám leggiom’ \Regius; Bláins lęggjum “the legs of Blown” \Hauksbok\Wormianus; Bláms lęggjum (\emph{wo. doubt corrupt form of former}) \RegiusProse\Trajectinus\Upsaliensis}}}{\lemma{ór brimi ... lęggjum}\Bfootnote{I think that the poem simply telling of “the bloody surf” and “the blue-black legs” fits better with its general allusive style, but this choice may be somewhat controversial.}}?\eva

\bvb — Then went the Powers all onto the rake-seats: the gin-holy gods, and from each other took counsel about that: Who would shape the troops of \inx[G]{Dwarfs}, out of the bloody surf, and out of the blue-black legs?\footnoteB{According to \Grimnismal\ TODO and \Vafthrudnismal\ TODO crags were made out of Yimer’s legs, and according to \Gylfaginning\ TODO the dwarfs first originated as maggots in Yimer’s rotting corpse. Since dwarfs were considered to dwell in rocks this is not strange. If one reads Blown instead of blue-black, then following Gurevich (\emph{Skp} 2017, p. 693) one may see a kenning “the legs of Blown \name{dwarf} \ken{stone}”, but if there is a being named Blown here, then that should be Yimer, although such name is never attested for him. More difficult to explain is the creation of dwarfs out of Yimer’s blood (which is the sea), since dwarfs are not otherwise known to dwell in water.}\evb
\evg


\bvg
\bva\ledleftnote{\Regius\Hauksbok\GylfMS}\edtext{\edtext{Þar vas Móðsognir}{\Afootnote{\emph{thus} \Hauksbok; ‘Þar mótſognir vitnir’ “there Mootsowner wolf” (\emph{wo. doubt corrupt}) \Regius\ — \emph{The prose of} \Gylfaginning\ \emph{confirms reading Móðsognir.}}} \hld\ mę́ztr of orðinn &
dverga allra, \hld\ ęn Durinn annarr;}{\lemma{Þar ... annarr “There ... second”}\Afootnote{\emph{om.} \GylfMS}} &
\edtext{\edtext{þęir manlíkun \hld\ mǫrg of gęrðu,}{\lemma{þęir ... gęrðu “They ... many”}\Afootnote{\emph{thus} \Regius\Hauksbok\Upsaliensis; þar manlíkun / mǫrg of gęrðusk (\emph{norm.}) “There man-likenesses many were made” \RegiusProse\Trajectinus\Wormianus}} &
dvergar \edtext{í}{\lemma{ór “out of”}\Afootnote{\emph{thus} \Regius\ í “in” \RegiusProse\Trajectinus\Wormianus\Upsaliensis\Hauksbok}} jǫrðu, \hld\ \edtext{sęm Durinn sagði}{\lemma{sęm Durinn sagði “as Dorn said”}\Afootnote{\emph{thus} \Regius\Hauksbok\RegiusProse\Wormianus; sem dur menn sagdi “as door-men said” \Trajectinus; sem þeim dyrinn kendi “as the animals taught them” \Upsaliensis}}.}{\lemma{þęir ... sagði “They ... said.”}\Bfootnote{There are two conflicting forms of the verse. Either the dwarfs were created on their own; this is supported by the prose of \Gylfaginning\ (see note to last v.) and by the form of its verse. On the other hand, both \Regius\ and \Hauksbok\ have the “worthiest” dwarfs Moodsowner and Dorn shaping “man-likenesses” out of soil. I have gone with the latter reading, but both should be considered.}}\eva

\bvb There was Moodsowner become the worthiest of all dwarfs, but Dorn [was] second. They made man-likenesses many; dwarfs out of the earth, as Dorn said.\evb
\evg

%TODO: move these verses to appendix.
\bvg {\small Two lists of dwarfs. That both belonged to the original poem is impossible, since several names (Oakenshield, Great-grandfather) appear in both. The three following verses seem to belong together, since there is no repetition of names. From the last line of the middle one, it seems that it should have been placed at the end of the group.}
\bva\ledleftnote{\Regius\Hauksbok\GylfMS}Nýi ok Niði, \hld\ Norðri, Suðri, &
Austri, Vestri, \hld\ Alþjófr, Dvalinn, &
Bívurr, Bávurr, \hld\ Bǫmburr, Nóri, &
Ȧnn ok Ȧnarr, \hld\ Ái, Mjǫðvitnir.\eva

\bvb — New and Nithe, Norther and Suther, Easter and Wester, Allthief, Dwollen, Bewer, Bower, Bamber, Noor, Own and Owner, Great-grandfather, Meadwitner.\evb
\evg


\bvg
\bva\ledleftnote{\Regius\Hauksbok\GylfMS}Vęigr ok Gandalfr, \hld\ Vindalfr, Þráinn, &
Þękkr ok Þorinn, \hld\ Þrór, Vitr ok Litr, &
Nár ok Nýráðr, \hld\ nú hęf’k dverga, &
Ręginn ok Ráðsviðr, \hld\ rétt of talða.\eva

\bvb Wey and Gandelf, Windelf, Thrown, Thetch and Thorn, Throo, Wit and Lit, Nee and Newred—now have I the dwarfs—Rain and Redswith—rightly tallied.\evb
\evg


\bvg {\small Second list.}
\bva\ledleftnote{\Regius\Hauksbok\GylfMS}Fíli, Kíli, \hld\ Fundinn, Náli, &
Hępti, Víli, \hld\ Hannarr, Svíurr, &
Frár, Hornbori, \hld\ Frę́gr ok Lȯni, &
Aurvangr, Jari, \hld\ Ęikinskjaldi.\eva

\bvb Filer, Chiler, Found and Needler, Hefter, Wiler, Hanner, Swigher, Fraw, Hornborer, Fray and Looner, Earwong, Earer, Oakenshield.\evb
\evg


\bvg
\bva\ledleftnote{\Regius\Hauksbok\GylfMS}Mál es dverga \hld\ í Dvalins liði &
ljȯna kindum \hld\ til Lofars tęlja, &
\edtext{þęir}{\Afootnote{þeim \Hauksbok}} es sóttu \hld\ frȧ salar stęini &
aurvanga sjǫt \hld\ til Jǫruvalla.\eva

\bvb — ’Tis time to tally the dwarfs in Dwollen’s host [back] to Loffer, for the kindreds of men;\footnoteB{A standard genealogical introduction (compare \Haleygjatal\ 1). The line of dwarfs is to be counted to their progenitor, Loffer. This possibly disagrees with the earlier introduction (“There was ...”), where Moodsown is said to be the foremost of the dwarfs, and Loffer is not mentioned.} they who sought, from the stone of the hall, the abode of \inx{Earwongs}\footnoteB{\CV\ \emph{aurvangr} ‘a loamy field’, and indeed this fits etymologically.} to the \inx{Erwolds}.\footnoteB{\Gylfaginning\ (TODO): “But these came from Swornshigh (\emph{Svarinshaugr}) to the Earwongs on the Erwolds, and thence Lofer is come; these are their names: Sherper (\emph{Skirpir}), Werper (\emph{Virpir}), Showfind, Great-grandfather, Elf and Ing (\emph{Ingi}), Oakenshield, Fale (\emph{Falr}), Frost, Finn, Ginner.”}\evb
\evg


\bvg
\bva\ledleftnote{\Regius\Hauksbok\GylfMS}Þar vas Draupnir \hld\ ok Dolgþrasir, &
Hár, Haugspori, \hld\ Hlévangr, Glói, &
Skirfir, Virfir, \hld\ Skáfiðr, Ái, &
Alfr ok Yngvi, \hld\ Ęikinskjaldi, &
Fjalarr ok Frosti, \hld\ Finnr ok Ginnarr; &
Þat mun \edtext{ę́}{\Afootnote{\emph{om.} \Regius}} uppi, \hld\ meðan ǫld lifir, &
langniðja-tal \hld\ \edtext{til}{\Afootnote{\emph{om.} \Hauksbok}} Lofars hafat.\eva

\bvb There was Dreepen and Dollowthrasher, High, Highspurer, Leewong, Glower, Sherver, Werver, Showfind, Great-grandfather, Elf and Ing, Oakenshield, Feller and Frost, Finn and Ginner: That will ever be remembered, while the \inx{eld} lives\footnoteB{Two archaic formulae. The first literally “that will ever up above”, cf. \HervararSaga\ TODO: “We two are cursed, brother, thy bane am I become! That will ever be remembered (\emph{þat mun ę́ uppi}, but both mss. \emph{þat mun enn uppi}), evil is the doom of the norns!”. The second is found in a runic inscription, U 323 (980–1015): “Ever will lie, while the eld lives (\textbf{meþ + altr + lifiʀ} \emph{með aldr lifir}), the hard-hammered bridge, broad, after a good man.”}, the tally of descendants, heaved to Lofer.\evb
\evg


\bvg {\small Creation of first men.}
\bva\ledleftnote{\Regius\Hauksbok}Unz \edtext{þrír}{\Afootnote{\emph{gramm. emend.} þrjár (\emph{norm.}) \Regius\Hauksbok}} kvǫ̇mu \hld\ \edtext{ór því liði}{\Afootnote{þussa brúðir “brides of thurses” (\emph{wo. doubt corrupt}) \Hauksbok}} &
\edtext{ǫflgir ok ȧstkir}{\Afootnote{ȧstkir ok ǫflgir \Hauksbok}} \hld\ ę̇sir at húsi; &
fundu ȧ landi \hld\ lítt męgandi &
Ask ok Emblu \hld\ ørlǫglausa.\eva

\bvb — Until three came out of that host: strong and lovely Ease along the houses; they found on land the little availing Ash and Emble, lacking \inx[C]{orlay}.\footnoteB{For, according to \Gylfaginning\ (TODO: reference), they were pieces of driftwood.}\evb
\evg


\bvg
\bva\ledleftnote{\Regius\Hauksbok}Ǫnd þau né ǫ́ttu, \hld\ óð þau né hǫfðu, &
lǫ́ né lę́ti \hld\ né litu góða; &
ǫnd gaf Óðinn, \hld\ óð gaf Hø̇nir, &
lǫ́ gaf Lóðurr \hld\ ok litu góða.\eva

\bvb Breath they owned not, \inx[C]{wode} they had not, not craft nor sound, nor good complexion. Breath gave Weden, wode gave Heen, craft gave Lother, and good complexion.\evb
\evg


\bvg {\small The ash of Ugdrassle and its three norns.}
\bva\ledleftnote{\Regius\Hauksbok\GylfMS}Ask vęit’k \edtext{standa}{\lemma{standa “stand[ing]”}\Afootnote{\emph{thus} \Regius\Hauksbok\Upsaliensis; ausinn “[is] poured” \RegiusProse\Trajectinus\Wormianus}}, \hld\ hęitir \edtext{Yggdrasill}{\Afootnote{Yggdrasils \RegiusProse}}, &
hǫ́r \edtext{baðmr}{\lemma{baðmr “beam”}\Afootnote{borinn “born” (\emph{wo. doubt corrupt}) \Upsaliensis}}, \edtext{ausinn}{\lemma{ausinn “poured”}\Afootnote{hęilagr (\emph{norm.}) “holy” \GylfMS}} \hld\ hvíta auri; &
þaðan koma dǫggvar \hld\ \edtext{þę́r’s}{\Afootnote{er “which” \RegiusProse\Trajectinus}} í dala falla; &
\edtext{stęndr}{\Afootnote{\emph{add.} hann \RegiusProse\Trajectinus}} \edtext{ę́}{\Afootnote{\emph{om.} \Upsaliensis}} yfir \edtext{grø̇nn}{\Afootnote{‘grvnn’ \RegiusProse; ‘grein’ \Upsaliensis}} \hld\ Urðar brunni.\eva

\bvb — An ash I know standing, \inx{Ugdrassle} ’tis called: a high beam\footnoteB{Tree.}, poured with white mud\footnoteB{Compare perhaps with the Indian ritual pouring of beverages onto the \emph{lingam}.—For the whole passage compare 27.}. Thence come the dew-drops which in the dales fall; it stands ever green over the \inx{Well of Weird}.\evb
\evg


\bvg
\bva\ledleftnote{\Regius\Hauksbok}Þaðan koma męyjar \hld\ margs vitandi &
þríar ór þęim \edtext{sę́}{\lemma{sę́ “lake”}\Afootnote{sal “hall” \Hauksbok}}, \hld\ es \edtext{und}{\lemma{und “beneath”}\Afootnote{ȧ “on” \Hauksbok}} þolli stęndr; &
Urð hétu ęina, \hld\ aðra Verðandi, &
skǫ́ru ȧ skíði, \hld\ Skuld hina þriðju &
þę́r lǫg lǫgðu, \hld\ þę́r líf køru, &
alda bǫrnum, \hld\ ørlǫg \edtext{sęggja}{\lemma{sęggja “of men”}\Afootnote{at segia “to say” \Hauksbok}}.\eva

\bvb Thence come maidens, much knowing: three out of that lake, which stands beneath the pine\footnoteB{But here simply meaning ‘tree’; perhaps the same applies for “ash” earlier.}: Weird they called one, the other Worthing—carved they on boards—Shild the third. Laws they laid, lives they chose: for the children of mortals, the \inx[C]{orlay} of men.\evb
\evg


\bvg {\small The origin of the Wallow.}
\bva\ledleftnote{\Regius\Hauksbok}Þat man hǫ̇n folkvíg \hld\ fyrst í hęimi, &
es Gollvęigu \hld\ gęirum studdu &
ok í hǫll Háars \hld\ hȧna bręnndu, &
\edtext{þrysvar bręnndu}{\Afootnote{‘þrysvar brendv þrysvar brendv’ \Hauksbok}} \hld\ þrysvar borna, &
opt ȯsjaldan, \hld\ þó hǫ̇n ęnn lifir.\eva

\bvb — That troop-conflict\footnoteB{While appealing to read \emph{folk-víg} ‘troop-conflict’ as meaning ‘ethnic conflict’ (between the Ease and Wanes), I more cautiously see the first element \emph{folk} carrying its earlier meaning of ‘troop, group of warriors’.} she recalls, the first in the \inx[C]{Home}, as Goldwey with spears they goaded, and in the hall of \inx[P]{Higher} \name{= Weden} \ken[L]{Walhall} burned her: thrice they burned the thrice born; often unseldom, though she yet lives.\footnoteB{Very cryptic. TODO: double check Snorri. Goldwey was apparently burned three times “often unseldom” (in short succession?) by the Ease, which yet did not kill her?}\evb
\evg


\bvg
\bva\ledleftnote{\Regius\Hauksbok}Hęiði hétu, \hld\ hvar’s til húsa kom, &
\edtext{vǫlu}{\Afootnote{ok vǫlu \Hauksbok}} \edtext{velspáa}{\Afootnote{\emph{metr. emend.}; ‘uel spá’ \Regius; ‘vel spa’ \Hauksbok}}, \hld\ vitti hǫ̇n ganda; &
sęið \edtext{hvar’s kunni}{\Afootnote{hon kvnni \Regius; hon hvars hvn kunni \Hauksbok}}, \hld\ sęið \edtext{hug lęikinn}{\Afootnote{hon leikinn \Regius; hon hugleikin \Hauksbok}}; &
ę́ vas hǫ̇n angan \hld\ illrar brúðar.\eva

\bvb Heath they called her, where to houses she came: a well-spaeing\footnoteB{Gifted at soothsaying.} \inx[C]{wallow}, she bewitched \inx[C]{gand}[gands]. She soth\footnoteB{Past tense of \inx[C]{sithe} (ON. \emph{síða}) ‘to enchant, bewitch’.)} where she could, she soth deluded minds; ever was she the love of an evil bride.\evb
\evg


\bvg {\small War between Ease and Wanes.}
\bva\ledleftnote{\Regius\Hauksbok}Þȧ gingu ręgin ǫll \hld\ ȧ rǫkstóla, &
ginnhęilǫg goð, \hld\ ok umb þat gę́ttusk: &
hvárt skyldu ę̇sir \hld\ afráð gjalda, &
eða skyldu goð ǫll \hld\ gildi ęiga?\eva

\bvb Then went the Powers all onto the rake-seats: the gin-holy gods, and from each other took counsel about that: whether the Ease should tribute yield, or should the gods all a banquet hold?\evb
\evg


\bvg
\bva\ledleftnote{\Regius\Hauksbok}Flęygði Óðinn \hld\ ok í folk of skaut; &
þat vas ęnn folkvíg \hld\ fyrst í hęimi; &
brotinn vas borðvęggr \hld\ borgar ȧsa, &
knǫ́ttu vanir vígspǫ́ \hld\ vǫllu sporna.\eva

\bvb Weden flung [a spear], and into the opposing army did shoot; that was yet the first troop-conflict\footnoteB{See note to v. 20.} in the \inx{Home}. Broken was the board-wall\footnoteB{Wall made of planks.} of the fortification of the Ease; the Wanes did by a conflict-\inx[C]{spae} tread the fields.\footnoteB{The Wanes used magic spells to defeat the Ease.}\evb
\evg


\bvg {\small Building of the wall by the ettin.}
\bva\ledleftnote{\Regius\Hauksbok\GylfMS}Þȧ gingu ręgin ǫll \hld\ ȧ rǫkstóla, &
ginnhęilǫg goð, \hld\ ok umb þat gę́ttusk: &
hvęrr hęfði lopt alt \hld\ lę́vi blandit &
eða ę́tt jǫtuns \hld\ Óðs męy gefna.\eva

\bvb Then went the Powers all onto the rake-seats: the gin-holy gods, and from each other took counsel about that: Who had the air all with treason blended, or to the ettin’s \inx{aught} given \inx{Wode}’s maiden\footnoteB{That is, promised Frie to the ettin NAME. TODO: relate with what Snorri writes about the building of the wall.}?\evb
\evg


\bvg {\small Thunder slays him.}
\bva\ledleftnote{\Regius\Hauksbok\GylfMS}\edtext{Þȯrr ęinn \edtext{þar vá}{\lemma{þar vá “fought there”}\Afootnote{\emph{thus} \Hauksbok\Trajectinus\Upsaliensis; þar var “was there” \Regius; þat vann “performed it" \RegiusProse; þat ua “fought it” \Wormianus}} \hld\ þrunginn móði, &
hann sjaldan sitr, \hld\ es slíkt of fregn; &
\edtext{ȧ gingusk ęiðar, \hld\ orð ok sǿri, &
mǫ́l ǫll męginlig, \hld\ es ȧ meðal \edtext{fóru}{\Afootnote{voru “[between them] were” \Hauksbok\Trajectinus}}.}{\lemma{ȧ ... fóru.}\Afootnote{\emph{om.} \Wormianus}}}{\lemma{Þȯrr ... fóru.}\Bfootnote{In \GylfMS\ the two helmings (\emph{Þȯrr ... fregn;} \emph{ȧ ... fóru}) come in reverse order of \Regius\Hauksbok, which is here followed.}}\eva

\bvb Thunder alone fought there, pressed by wrath; he seldom sits, when of such\footnoteB{Oath-breaking, lies and deception.} he learns. Trampled were oaths, speeches and vows; the mighty treaties all, which between them had gone.\evb
\evg


\bvg {\small Homedall’s hearing hidden beneath Ugdrassle.}
\bva\ledleftnote{\Regius\Hauksbok}Vęit hǫ̇n Hęimdallar \hld\ hljóð of folgit &
und hęiðvǫnum \hld\ hęlgum baðmi; &
ȧ sér hǫ̇n ausask \hld\ aurgum forsi &
af veði Valfǫðrs. \hld\ Vituð ér ęnn eða hvat?\eva

\bvb — Knows she the hearing of Homedall hidden, ’neath a shady\footnoteB{\emph{hęiðvanr}, literally ‘clear-, bright-less’.}, hallowed beam\footnoteB{The tree must be Ugdrassle.}. On it she sees being poured a muddy torrent\footnoteB{Literally “on she sees being poured with a muddy torrent”, which should be the same mud as in v. 19. However, if ms. \emph{ȧ} is read as \emph{ǫ́} ‘river’, it would mean “A river she sees being fed by a muddy waterfall, from ...”}, from the pledge of Walfather\footnoteB{Presumably referring to Weden’s sacrifice of an eye at Mimer’s well.} \name{= Weden} \ken*{Mimer’s well?}—know ye yet, or what?\footnoteB{“Do ye (Weden) know enough now, or what?”—repeated in 28, 33, 34, 38, 40, 47, 60, 61.}”\evb
\evg


\bvg {\small Weden sought out the wallow.—The following two verses are written together as one in \Regius.}
\bva\ledleftnote{\Regius}Ęin sat hǫ̇n úti, \hld\ þȧ’s hinn aldni kom &
yggjungr ȧsa \hld\ ok í augu lęit; &
hvęrs fregnið mik? \hld\ hví fręistið mín?\eva

\bvb — Lone sat she outside, when the old one came: the Terrifier of the Ease\footnoteB{Weden.}, and into [her] eyes looked. “Why inquirest thou me? Why temptest thou me?\footnoteB{The Wallow speaks.}\evb
\evg

\bvg
\bva\ledleftnote{\Regius\GylfMS}Alt vęit’k, Óðinn, \hld\ hvar auga falt &
\edtext{í hinum mę́ra}{\Afootnote{\emph{thus} \Wormianus; þitt (\emph{with points marking as error}) i enom męra \Regius í þęim hinum meira (“id.”) (\emph{norm.}) \Trajectinus\Upsaliensis; vr þeim envm mę́ra “out of the renowned” \RegiusProse}} \hld\ Mímis brunni; &
drekkr mjǫð Mímir \hld\ morgin hvęrjan &
af \edtext{veði}{\lemma{veði “pledge”}\Afootnote{veiþi “hunting”}} Valfǫðrs. \hld\ Vituð ér ęnn eða hvat?\eva

\bvb I know it all, Weden; where thine eye thou hidst: in the renowned \inx{Well of Mime}, [there] drinks Mime mead every morning, from the pledge of Walfather\footnoteB{See note to v. 26.} \name{= Weden} \ken*{Mimer’s well?}—know ye yet, or what?”\evb
\evg


\bvg
\bva\ledleftnote{\Regius}Valði hęnni Hęrfǫðr \hld\ hringa ok męn; &
\edtext{féspjǫll spaklig}{\lemma{“wise wealth-spells”}\Bfootnote{By some authors (see Haukur 2020, p. 51 ff.) emended to \emph{fekk spjǫll spaklig} “he (= Weden) received wise tidings”}} \hld\ ok spáganda; &
sá hǫ̇n vítt ok umb vítt \hld\ of verǫld hvęrja.\eva

\bvb Host-father chose for her, rings and necklaces, wise wealth-spells, and spae-gands\footnoteB{The meaning of a \emph{gand} not fully clear. In this verse perhaps staffs used in ritual?}; saw she widely and widely about, o’er every world.\evb
\evg


\bvg {\small The Walkirries.}
\bva\ledleftnote{\Regius}Sá hǫ̇n valkyrjur \hld\ vítt of komnar, &
gǫrvar at ríða \hld\ til goðþjóðar. &
\edtext{Skuld hęlt skildi, \hld\ ęn Skǫgul ǫnnur, &
Gunnr, Hildr, Gǫndul \hld\ ok Gęirskǫgul; &
nú eru talðar \hld\ nǫnnur Hęrjans, &
gǫrvar at ríða \hld\ grund valkyrjur.}{\lemma{Skuld ... valkyrjur}\Bfootnote{These four lines, especially from the out-of-place ending (\emph{nú eru talðar}), seem to be a latter insert from a \emph{thule} counting the walkirries.}}\eva

\bvb Saw she walkirries, widely come, ready to ride to \inx{Godthede}. Shild held a shield, and Shagle another; Guth, Hild, Gandle, and Goreshagle; now are tallied the women of the Lord of Hosts: \inx{walkirries} ready to ride the ground.\evb
\evg


\bvg {\small The fate of Balder.}
\bva\ledleftnote{\Regius}Ek sá Baldri, \hld\ blóðgum tívi, &
Óðins barni, \hld\ ørlǫg folgin; &
stóð of vaxinn \hld\ vǫllum hę́ri &
mjór ok mjǫk fagr \hld\ mistiltęinn.\eva

\bvb — I saw Balder’s, the bloody tue’s, the child of Weden’s, \inx{orlay} sealed\footnoteB{Notably, \emph{fela} ‘hide, conceal’ is used to describe burial in mounds, as in \Ynglingatal\ 24, Öl 1 (900s): “hidden (\textbf{fulkin} \emph{folginn}) in this mound lies he whom the greatest deeds followed...”}; grown did stand, higher than the fields, slender and greatly fair, the mistletoe.\footnoteB{Told allusively in the following three verses is the death of Balder at the hands of his blind brother Hath. \Gylfaginning\ TODO}\evb
\evg


\bvg
\bva\ledleftnote{\Regius}Varð af męiði, \hld\ þęim’s mę́r sýndisk, &
harmflaug hę́ttlig, \hld\ Hǫðr nam skjóta. &
Baldrs bróðir vas \hld\ of borinn snimma, &
sá nam, Óðins sonr, \hld\ ęinnę́ttr vega;\eva

\bvb Became of that beam, which meager seemed, a baneful harm-flier; Hath began to shoot. Balder’s brother was born early; that one began, Weden’s son, one night old, to fight.\evb
\evg


\bvg
\bva\ledleftnote{\Regius}þó hann ę́va hęndr \hld\ né hǫfuð kęmbði, &
áðr ȧ bál of bar \hld\ Baldrs andskota. &
Ęn Frigg of grét \hld\ í Fęnsǫlum &
vǫ́ Valhallar. \hld\ Vituð ér ęnn eða hvat?\eva

\bvb Washed he never hands, nor head combed, before onto the pyre he did bear Balder’s opponent. But Frie did lament, in the Fenhalls, the woe of Walhall—know ye yet, or what?\evb
\evg


\bvg
\bva\ledleftnote{\Hauksbok}\edtext{Þȧ kná Váli \hld\ vígbǫnd snúa &
hęldr vǫ́ru harðgǫr \hld\ hǫpt ór þǫrmum.}{\lemma{Þȧ ... þǫrmum.}\Bfootnote{Only attested in \Hauksbok\, where it is combined with the last two lines of the next v. (\emph{þar ... hvat?}).}}\eva

\bvb Then did \inx[C]{Wonnel} the war-bonds turn; were they rather sturdy, fetters made out of intestines.\evb
\evg


\bvg {\small The imprisoned Locke.}
\bva\ledleftnote{\Regius\Hauksbok}\edtext{Hapt sá hǫ̇n liggja \hld\ und Hveralundi &
lę́gjarnlíki \hld\ Loka ȧþękkjan;}{\lemma{Hapt ... ȧþękkjan}\Afootnote{\emph{om.} \Hauksbok}} &
þar sitr Sigyn \hld\ þęygi of sínum &
veri vęlglýjuð. \hld\ Vitud ér ęnn eða hvat?\eva

\bvb A captive she saw lying, ’neath Wharlund: the guileful form of similar Locke. There sits Sighyn, not at all cheerful, above her husband;\footnoteB{See \FraLoka.}—know ye yet, or what?\evb
\evg


\bvg
\bva\ledleftnote{\Regius}Ǫ́ fęllr austan \hld\ of ęitrdala &
sǫxum ok sverðum, \hld\ Slíðr hęitir sú.\eva

\bvb A river falls from the east, above the venom-dales, with saxes and swords; Slide is that one called.\evb
\evg


\bvg {\small Two halls.}
\bva\ledleftnote{\Regius}Stóð fyr norðan \hld\ ȧ Niðavǫllum &
salr ór golli \hld\ Sindra ę́ttar, &
ęn annarr stóð \hld\ ȧ Ȯkólni, &
bjórsalr jǫtuns, \hld\ ęn sá Brimir hęitir.\eva

\bvb Stood to the north, on the Nithewolds, a hall out of gold, of the \inx{aught} of Sinder; but another one stood, on Uncoalner, the beer-hall of an ettin, and Brimmer ’tis called.\evb
\evg


\bvg {\small The worst hall.}
\bva\ledleftnote{\Regius\Hauksbok\GylfMS}Sal sá hǫ̇n standa \hld\ sólu fjarri &
Nástrǫndu ȧ, \hld\ norðr horfa dyrr; &
falla ęitrdropar \hld\ inn umb ljóra, &
sá ’s undinn salr \hld\ orma hryggjum.\eva

\bvb A hall she saw standing, far from the sun, on Nawstrand, north face the doors; fall venom-drops in through the smoke-vent, that hall is wound by the spines of snakes.\evb
\evg


\bvg
\bva\ledleftnote{\Regius\Hauksbok\GylfMS}\edtext{Sá hǫ̇n}{\lemma{Sá hǫ̇n “she saw”}\Afootnote{\emph{thus} \Regius; ser hon “she sees” \Hauksbok; skulu “shall” \GylfMS}} þar vaða \hld\ þunga strauma &
męnn męinsvara \hld\ ok morðvarga &
ok þann’s annars glępr \hld\ ęyrarúnu. &
Þar \edtext{saug}{\lemma{saug “sucked”}\Afootnote{\emph{thus} \Hauksbok; súg (\emph{corrupt form of} saug) \Regius; kvęlr “torments”}} Níðhǫggr \hld\ nái framgingna; &
slęit vargr vera. \hld\ Vituð ér ęnn eða hvat?\eva

\bvb There she saw wade, through heavy streams, oath-breaking men and murderwargs, and the one who confounds another’s understanding\footnoteB{Literally “who confounds another’s ear-rune;” false counsellors.}. There sucked Nithehew from corpses passed-on; the warg tore men asunder—know ye yet, or what?\evb
\evg


\bvg {\small The hag nourishes the destroyers in Ironwood.}
\bva\ledleftnote{\Regius\Hauksbok\GylfMS}Austr \edtext{býr}{\Afootnote{\emph{Thus} \Hauksbok\GylfMS\ sat “stayed [the old]” \Regius}} hin \edtext{aldna}{\Afootnote{arma “the wretched woman” \Upsaliensis}} \hld\ í \edtext{Járnviði}{\Afootnote{jarnuidiom “[in] Ironwoods” \Trajectinus}} &
ok \edtext{fǿðir}{\Afootnote{\emph{Thus} \Hauksbok\GylfMS; fǿddi “nourished” \Regius}} þar \hld\ Fęnris kindir; &
verðr \edtext{af}{\Afootnote{ór “out of [them] \Trajectinus\RegiusProse}} þęim ǫllum \hld\ ęinna nøkkurr &
tungls \edtext{tjúgari}{\lemma{tjúgari}\Afootnote{tuigan \Trajectinus\ \emph{wo. doubt corrupt}; tregari “griever [of the moon]” \Upsaliensis\ — As the young agentive suffix \emph{-ari} is found only here in the poem, it is possible that this word is corrupt. In that case, it must have occurred quite early in the transmission, as reflexes of \emph{*tiugari} are found in all surviving mss.}} \hld\ í trolls hami.\eva

\bvb In the east dwells the old woman, in \inx{Ironwood}, and nourishes there the kindreds of \inx{Fenner}; from them all becomes one most particular: a seizer of the moon, in the \inx{hame} of a troll.\footnoteB{The old hag raises the offspring of the wolf Fenner, of which one will swallow the moon (and according to \Gylfaginning\ TODO the other the sun). See note to the next v.}\evb
\evg


\bvg
\bva\ledleftnote{\Regius\Hauksbok\GylfMS}Fyllisk fjǫrvi \hld\ fęigra manna, &
rýðr ragna sjǫt \hld\ rauðum dręyra, &
svǫrt verða sólskin \hld\ umb sumur ęptir, &
veðr ǫll válynd. \hld\ Vituð ér ęnn eða hvat?\eva

\bvb He\footnoteB{The wolf.} fills himself with the life of \inx[C]{fey} men; he reddens the abode of the \inx{Powers} with red gore. Black becomes the sunshine about the summers afterwards\footnoteB{After the sun is swallowed. But since the wallow does not tell us that this is a different wolf (it seems rather it be one and the same), it may reflect an earlier version of the myth, where one son of Fenner swallowed both the sun and moon. Yet, according to \Vafthrudnismal\ 36-37 it is Fenner himself who will swallow the sun (and thus likely the moon as well,) unless it there be taken as a general \inx{hote} for ‘wolf’ (which undoubtedly is its original meaning). TODO}; the storms all woeful—know ye yet, or what?\evb
\evg


\bvg {\small Edgethew struck harp; a fair-red cock crowed.}
\bva\ledleftnote{\Regius\Hauksbok}Sat þar ȧ haugi \hld\ ok sló hǫrpu &
gýgjar hirðir, \hld\ glaðr Ęggþér; &
gól of hǫ̇num \hld\ í Gaglviði &
fagrrauðr hani, \hld\ sá’s Fjalarr hęitir.\eva

\bvb Sat there on the \inx{high} and struck the harp, the troll-woman’s herdsman, glad \inx{Edgethew}. Above him crowed, in Galewood\footnoteB{\emph{gagl} ‘wild goose’, maybe here referring to carrion-eating ravens? Possibly the same as Ironwood.}, a fair-red cock, that one who Feller is called.\evb
\evg


\bvg {\small A golden cock crowed in Osyard; a soot-red in Hell.}
\bva\ledleftnote{\Regius\Hauksbok}Gól of ǫ̇sum \hld\ Gollinkambi, &
sá vękr hǫlða \hld\ at Hęrjafǫðrs, &
ęn annarr gęlr \hld\ fyr jǫrð neðan &
sótrauðr hani \hld\ at sǫlum Hęljar.\eva

\bvb Above the Ease crowed Goldencombe: he wakes men at the Father of Hosts’s [estate]; but another one crows beneath the earth: a soot-red cock, at the halls of Hell.\evb
\evg


\bvg
\bva\ledleftnote{\Regius\Hauksbok}Gęyr Garmr mjǫk \hld\ fyr Gnipahęlli, &
fęstr mun slitna, \hld\ ęn Freki rinna; &
fjǫlð vęit hǫ̇n frǿða, \hld\ framm sé’k lęngra &
of ragna rǫk, \hld\ rǫmm sigtíva.\eva

\bvb Barks Garm loudly before the Gnip-caverns; the rope will tear, and Freck run. Much she knows of learning, forth I see yet further; about the mighty Rakes of the Powers, of the victory-tues.\evb
\evg


\bvg {\small Degeneration of man.}
\bva\ledleftnote{\Regius\Hauksbok\GylfMS}Brǿðr munu bęrjask \hld\ ok at bǫnum verðask, &
munu \edtext{systrungar}{\lemma{systrungar “sister’s sons”}\Afootnote{stystrungar (\emph{wo. doubt corrupt}) \Trajectinus}} \hld\ sifjum spilla; &
hart ’s \edtext{í hęimi}{\lemma{í hęimi “in the home”}\Afootnote{\emph{thus} \Regius\Hauksbok\Upsaliensis; með hǫlðum “among men” \RegiusProse\Trajectinus\Wormianus}}, \hld\ hórdȯmr mikill, &
skęggǫld, skalmǫld, \hld\ \edtext{skildir}{\lemma{skildir “shields”}\Afootnote{\emph{add.} ró “are” \Regius}} \edtext{klofnir}{\lemma{klofnir “cloven”}\Afootnote{klofna “become cloven" \Upsaliensis}}, &
\edtext{vindǫld}{\lemma{vindǫld “wind-eld”}\Bfootnote{In \Hauksbok\ capitalized, marking as new verse.}}, vargǫld, \hld\ \edtext{áðr}{\lemma{áðr “before”}\Afootnote{unz (\emph{norm.}) “until” \Upsaliensis}} verǫld \edtext{stęypisk}{\lemma{stęypisk “tumbles down”}\Bfootnote{After this word \Hauksbok\ has a line not found in \Regius\ or \GylfMS: \emph{grundir gjalla / gífr fljúgandi} (\emph{norm.}) “foundations shrill, fiends flying”}} &
\edtext{mun \edtext{ęngi}{\Afootnote{enn (\emph{wo. doubt corrupt}) \Upsaliensis}} maðr \hld\ ǫðrum þyrma.}{\lemma{mun ... þyrma “before ... spare.”}\Bfootnote{\emph{om.} \RegiusProse\Trajectinus\Wormianus}}\eva

\bvb Brothers will fight, and become each other’s slayers; sister’s sons will spill their kinship.\footnoteB{Whether through incest or treachery. TODO: literary evidence of the phrase \emph{spilla sifjum}.} ’Tis hard in the Home, whoredom great: axe-eld, sword-eld—shields are rent—wind-eld, warg-eld; before the world\footnoteB{\emph{ver-ǫld} ‘world’ is literally ‘man-eld’, ‘the eld of man’.} tumbles down, no man will another spare.\evb
\evg


\bvg {\small Prophesied events come to pass.}
\bva\ledleftnote{\Regius\Hauksbok\GylfMS}\edtext{Lęika Míms synir, \hld\ ęn mjǫtuðr kyndisk &
at hinu galla \hld\ Gjallarhorni; &
hǫ́tt blę́ss Hęimdallr, \hld\ horn ’s ȧ lopti; &
\edtext{mę́lir}{\lemma{mę́lir “speaks”}\Afootnote{mey \RegiusProse; nie \Trajectinus\ \emph{both wo. doubt corrupt}}} Óðinn \hld\ við Míms hǫfuð.}{\lemma{Lęika ... hǫfuð.}\Bfootnote{In \GylfMS\ ll. 1–2 (\emph{Lęika ... Gjallarhorni;} “Play ... Horn of Yell.”) are missing, and ll. 3–4 (\emph{hǫ́tt ... hǫfuð.} “High ... head [of Mime.]”) are instead paired with the first two lines of the next v. (Skęlfr ... losnar;)}}\eva

\bvb Play the sons of Mime, and the Metted is kindled, at [the sounding of] the shrill Horn of Yell. Loudly blows Homedall; the horn is aloft; Weden speaks with the head of Mime.\evb
\evg


\bvg
\bva\ledleftnote{\Regius\Hauksbok\GylfMS}\edtext{Skęlfr Yggdrasils \hld\ askr standandi, &
ymr it aldna tré, \hld\ ęn jǫtunn losnar;}{\lemma{Skęlfr ... losnar “Quakes ... loosens.”}\Bfootnote{thus \Hauksbok\GylfMS; in \Regius\ the two lines are reversed.}} &
\edtext{hrę́ðask allir \hld\ ȧ hęlvegum &
áðr Surtar þann \hld\ sefi of glęypir.}{\lemma{hrę́ðask ... glęypir “[All] are frightened ... devour [it.]”}\Bfootnote{only in \Hauksbok}} \eva

\bvb Quakes the ash of Ugdrassle, standing; groans the old tree, and the ettin loosens. All are frightened on the Hell-ways, before Surt’s kinsman does devour it.\evb
\evg


\bvg
\bva\ledleftnote{\Regius\Hauksbok\GylfMS}Hvat ’s með ǫ̇sum? \hld\ hvat ’s með \edtext{ǫlfum}{\lemma{ǫlfum “Elves”}\Afootnote{asynivm “Ossens” \Upsaliensis}}? &
\edtext{gnýr allr Jǫtunhęimr, \hld\ ę̇sir ’ro ȧ þingi,}{\lemma{gnýr ... þingi}\Afootnote{\emph{om.} \Upsaliensis}} &
stynja dvergar \hld\ fyr \edtext{stęindurum}{\Afootnote{steins \Upsaliensis — -dyrum \Hauksbok\Wormianus\Upsaliensis}} &
\edtext{\edtext{vęggbergs}{\lemma{vęggbergs “wedge-rock”}\Afootnote{vegbergs “way-rock” \Hauksbok\Trajectinus\Wormianus}} vísir}{\Afootnote{\emph{om.} \Upsaliensis}} — \hld\ vituð ér ęnn eða hvat?\eva

\bvb — What is with the Ease? What is with the Elves? Roars all Ettinham, the Ease are at the Thing. Dwarfs groan before gates of stone, the princes of the wedge-rock—know ye yet, or what?\evb
\evg


\bvg
\bva\ledleftnote{\Regius\Hauksbok}Gęyr nú Garmr mjǫk \hld\ fyr Gnipahęlli, &
fęstr mun slitna, \hld\ ęn Freki rinna; &
fjǫlð vęit hǫ̇n frǿða, \hld\ framm sé’k lęngra &
of ragna rǫk, \hld\ rǫmm sigtíva.\eva

\bvb Barks now Garm loudly before the Gnip-caverns; the rope will tear, and Freck run. Much she knows of learning, forth I see yet further; about the mighty Rakes of the Powers, of the victory-tues.\evb
\evg


\bvg {\small The enemies of the gods assemble.}
\bva\ledleftnote{\Regius\Hauksbok\RegiusProse\Trajectinus\Wormianus}Hrymr ękr austan, \hld\ hęfsk lind fyrir, &
snýsk Jǫrmungandr \hld\ í jǫtunmóði; &
ormr knýr unnir, \hld\ \edtext{ęn ari hlakkar}{\lemma{ęn ari hlakkar “but the eagle screams”}\Afootnote{ǫrn mun hlakka “the eagle will scream” \RegiusProse\Trajectinus}}, &
slítr nái neffǫlr; \hld\ Naglfar losnar.\eva

\bvb Rim drives from the east, holding his shield before himself; Ermingand writhes about in ettin’s wrath. The worm propels the waves, but the eagle screams: the pale-beak tears corpses; Nailfare loosens.\evb
\evg


\bvg
\bva\ledleftnote{\Regius\Hauksbok\RegiusProse\Trajectinus\Wormianus}Kjóll fęrr austan \hld\ koma munu Múspells &
of lǫg lýðir, \hld\ ęn Loki stýrir; &
fara fíflmęgir \hld\ með Freka allir, &
þęim es bróðir \hld\ Býlęists í fǫr.\eva

\bvb A ship travels from the east—come will Muspell’s subjects by sea—but Locke steers it. Travel the warlocks all with Freck; with them comes the brother of Bylest \ken{Locke}[1] along.\evb
\evg


\bvg {\small Surt comes; the final battle begins.}
\bva\ledleftnote{\Regius\Hauksbok\GylfMS}\edtext{Surtr}{\Afootnote{Svartr \Upsaliensis}} fęrr sunnan \hld\ með sviga lę́vi, &
skínn af sverði \hld\ sól valtíva; &
grjótbjǫrg gnata, \hld\ ęn \edtext{gífr rata}{\Afootnote{guðar hrata “[but] the gods stagger” (\emph{wo. doubt corrupt, young masc. pl. is proof enough.}) \Upsaliensis}}, &
troða halir hęlveg, \hld\ ęn himinn klofnar.\eva

\bvb Surt comes from the south, with the betrayer of the stick \ken{fire}; from the sword shines the sun of the slain-tues; boulders clash, but the fiends reel; men march on the \inx{Hell-ways}, but heaven is sundered.\evb
\evg


\bvg {\small Weden falls to the Wolf and Free to Surt.}
\bva\ledleftnote{\Regius\Hauksbok\RegiusProse\Trajectinus\Wormianus}Þȧ kømr Hlínar \hld\ harmr annarr framm, &
es Óðinn fęrr \hld\ við ulf vega, &
ęn bani Bęlja \hld\ bjartr at Surti; &
þȧ mun Friggjar \hld\ falla \edtext{angan}{\Afootnote{angantyr \Regius}}.\eva

\bvb Then comes \inx{Line}’s second sorrow to pass, as Weden goes to strike against the wolf; but the bane of \inx{Bellow}\footnoteB{\inx{Free}.}, bright, [goes] against Surt; then will Frie’s beloved\footnoteB{Weden, her husband.} fall.\evb
\evg


\bvg {\small Wider avenges Weden and slays the Wolf.}
\bva\ledleftnote{\Regius\RegiusProse\Trajectinus\Wormianus}\edtext{Þȧ kømr hinn mikli \hld\ mǫgr Sigfǫður}{\lemma{Þȧ kømr ... Sigfǫður “Then ... Sighfather”}\Afootnote{Gęngr Óðins sonr / við ulf vega “Goes Weden’s son against the wolf to fight” \GylfMS}}, &
Víðarr \edtext{vega}{\Afootnote{of veg \GylfMS}} \hld\ at valdýri; &
lę́tr hann męgi Hveðrungs \hld\ mund of standa &
hjǫr til hjarta; \hld\ þȧ ’s hefnt fǫður.\eva

\bvb Then comes the great lad of \inx{Sighfather}, Wider, to strike at the murderous beast; he lets his hand plunge the sword into the heart of \inx{Whethring}’s lad\footnoteB{The son of Locke; the wolf.}; then is the father avenged.\evb
\evg


\bvg
\bva\ledleftnote{\Hauksbok}\edtext{Gínn lopt yfir \hld\ lindi jarðar, &
gapa ýgs kjaptar \hld\ orms í hę́ðum; &
mun Óðins son \hld\ \edtext{ęitri}{\lemma{ęitri “venom”}\Afootnote{ormi “the worm” \Hauksbok, \emph{cf. the prose of} \Gylfaginning: \emph{“Thunder bears the bane-word from the Middenyardsworm and thence strides away nine paces. Then he falls dead to the earth by the \textbf{venom} \emph{(ęitri)} which the Worm blows on him.”}}} mǿta &
vargs at \edtext{dauða}{\Afootnote{da... \Hauksbok}} \hld\ Víðars niðja.}{\lemma{Gínn ... niðja.}\Bfootnote{Reading taken from Jón Helgason 1971, pp. 13, 44ff.}}\eva

\bvb Yawns over the air the girdle of the earth \ken{the Middenyardsworm}[1]; gape the jaws of the fierce worm in the heights. The venom of the beast will meet Weden’s son \ken{Thunder}[1], after the deaths of Wider’s kinsmen \ken{the Ease}[1].\evb
\evg


\bvg {\small Thunder and the Worm kill each other.}
\bva\ledleftnote{\Regius\Hauksbok\RegiusProse\Trajectinus\Wormianus}\edtext{Þȧ kømr}{\Afootnote{Gęngr \GylfMS}} hinn mę́ri \hld\ mǫgr Hlǫðynjar &
\edtext{gęngr Óðins sonr \hld\ við orm vega.}{\lemma{gęngr ... vega}\Afootnote{\emph{Only in} \Regius}} &
\edtext{Drepr af móði \hld\ Miðgarðs véurr; &
munu halir allir \hld\ hęimstǫð ryðja; &
gęngr fet níu \hld\ Fjǫrgynjar burr &
nęppr frȧ naðri, \hld\ níðs ȯkvíðnum.}{\lemma{Drepr ... ȯkviðnum}\Afootnote{neppr af naðri / niðs ȯkvíðnum / munu halir allir / hęimstǫð ryðja, / es af móði drepr / Miðgarðs véurr “[Goes the renowned lad of Lathyn,] pained, away from the loathsome adder. All men will empty their homesteads, when Middenyard’s wigh-ward strikes out of wrath.” \GylfMS}}\eva

\bvb Then comes the renowned lad of Lathyn: the son of Weden goes the \inx{worm} to meet. Middenyard’s wigh-ward strikes out of wrath; all men will their homesteads empty.\footnoteB{It seems likely that the order found in \Gylfaginning\ is original. After Thunder dies, farming becomes impossible, and thus men must leave their homes.} The son of Firgyn goes nine paces, pained, away from the loathsome adder.\footnoteB{Thunder, mortally wounded, struggles nine steps away from the Worm before he falls. See note to previous verse.}\evb
\evg


\bvg {\small Culmination.}
\bva\ledleftnote{\Regius\Hauksbok\GylfMS}Sól tér sortna, \hld\ \edtext{søkkr fold í mar}{\lemma{søkkr ... mar}\Bfootnote{This line is very similar to a line of v. 24 in Arnthur ‘earl-scold’ Thurthson’s Drape of Thurfinn (\Skp: Arn \emph{Þorfdr} 24\textsuperscript{II}): \emph{søkkr fold í mar døkkvan} “sinks the fold into the dark sea”. For this reason, \emph{søkkr} ‘sinks’ \RegiusProse\Trajectinus\Wormianus has been chosen over \emph{sígr} ‘descends’ \Regius\Hauksbok\Upsaliensis.}}, &
hverfa af himni \hld\ hęiðar stjǫrnur; &
gęisar ęimi \hld\ við aldrnara; &
lęikr hǫ́r hiti \hld\ við himin sjalfan.\eva

\bvb The sun does blacken, sinks the fold into the sea; disappear off heaven the clear stars. Rages smoke from the nourisher of life\footnoteB{Fire.}; licks the high heat heaven itself.\evb
\evg


\bvg
\bva\ledleftnote{\Regius\Hauksbok}Gęyr nú Garmr mjǫk \hld\ fyr Gnipahęlli, &
fęstr mun slitna, \hld\ ęn Freki rinna; &
fjǫlð vęit hǫ̇n frǿða, \hld\ framm sé’k lęngra &
of ragna rǫk, \hld\ rǫmm sigtíva.\eva

\bvb Barks now Garm loudly before the Gnip-caverns; the rope will tear, and Freck run. Much she knows of learning, forth I see yet further; about the mighty Rakes of the Powers, of the victory-tues.\evb
\evg


\bvg {\small The world is reborn.}
\bva\ledleftnote{\Regius\Hauksbok}Sér hǫ̇n upp koma \hld\ ǫðru sinni &
jǫrð ór ę́gi \hld\ iðjagrø̇na; &
falla forsar, \hld\ flýgr ǫrn yfir, &
sá’s ȧ fjalli \hld\ fiska vęiðir.\eva

\bvb Sees she come up, a second time: the earth out of the sea, ever green anew. Torrents fall; flies an eagle above, the one who on the fells fish does catch.\evb
\evg


\bvg
\bva\ledleftnote{\Regius\Hauksbok}Finnask ę̇sir \hld\ ȧ Iðavęlli &
ok umb moldþinur \hld\ mǫ́tkan dø̇ma, &
ok minnask þar \hld\ ȧ męgindȯma &
ok ȧ Fimbultýs \hld\ fornar rúnar.\eva

\bvb The Ease find each other on the Idewolds, and about the mighty earth-strip\footnoteB{The Middenyardsworm.} converse, and remember there mighty judgements, and Fimbletue’s <= Weden’s> ancient runes.\evb
\evg

\bvg {\small A new golden age.}
\bva\ledleftnote{\Regius\Hauksbok}Þar munu ęptir \hld\ undrsamligar &
gollnar tǫflur \hld\ í grasi finnask, &
þę́r’s í árdaga \hld\ áttar hǫfðu.\eva

\bvb There will afterwards wondrous golden Tavel-bricks in the grass be found: those which in days of yore they had owned.\footnoteB{Cf. v. 9. The rediscovering of the golden game pieces symbolizes a new golden age.}\evb
\evg


\bvg
\bva\ledleftnote{\Regius\Hauksbok}Munu ȯsánir \hld\ akrar vaxa; &
bǫls mun alls batna \hld\ mun Baldr koma; &
búa Hǫðr ok Baldr \hld\ Hropts sigtoptir, &
vęl valtívar. \hld\ Vituð ér ęnn eða hvat?\eva

\bvb Unsown will fields grow: evil will all be bettered: Balder will come. Bedwell Hath and Balder the victory-plots of Roft <= Weden>, well, the slain Tues—know ye yet, or what?\evb
\evg


\bvg
\bva\ledleftnote{\Regius\Hauksbok}Þȧ kná Hø̇nir \hld\ hlautvið kjósa &
ok burir byggva \hld\ brǿðra Tvęggja &
vindhęim víðan. \hld\ Vituð ér ęnn eða hvat?\eva

\bvb Then does Heen choose the \inx{leat}-wood\footnoteB{Restore the bloot and practice divination.}, and the sons of the brothers of Tway <= Weden> settle the wide wind-home\ken{Sky.}\footnoteB{Will and Wigh? Who their sons are is unknown.}—know ye yet, or what?\evb
\evg


\bvg
\bva\ledleftnote{\Regius\Hauksbok\GylfMS}Sal \edtext{sér hǫ̇n}{\lemma{sér hǫ̇n “she sees”}\Afootnote{vęit’k (\emph{norm.}) “I know” \GylfMS}} standa \hld\ sólu fęgra, &
golli \edtext{þakðan}{\lemma{þakðan “thatched”}\Afootnote{betra “better [than gold]” \RegiusProse\Trajectinus}}, \hld\ ȧ \edtext{Gimléi}{\Afootnote{\emph{metr. emend.} Gimlé (\emph{norm.}) \Regius\Hauksbok\GylfMS}}; &
\edtext{þar}{\lemma{þar “there”}\Afootnote{þann “it [shall dutiful men bedwell]” \Trajectinus\Wormianus}} skulu dyggvar \hld\ dróttir byggva &
ok umb aldrdaga \hld\ ynðis njóta.\eva

\bvb A hall she sees standing, fairer than the sun: thatched with gold, on Gemlee; there dutiful men shall dwell, and in their life-days delights enjoy.\evb
\evg


\bvg {\small The dragon still lives; the wallow descends.}
\bva\ledleftnote{\Regius\Hauksbok}Þar kømr hinn dimmi \hld\ dręki fljúgandi, &
naðr frȧnn neðan \hld\ frȧ Niðafjǫllum; &
berr sér í fjǫðrum \hld\ —flýgr vǫll yfir— &
Níðhǫggr nái; \hld\ nú mun hǫ̇n søkkvask.\eva

\bvb — Then comes the shadowy dragon flying; the gleaming adder down below from the \inx{Nithfells}. Nithehew bears in his feathers—flying over the field—corpses.” — Now she will sink!\footnoteB{The wallow, referring to herself in third person, descends back down into her grave, whence Weden woke her.}\evb
\evg


\bvg {\small Spurious verse from \Hauksbok.}
\bva[X]\ledleftnote{\Hauksbok}\edtext{Þȧ kømr hinn ríki \hld\ at ręgindȯmi &
ǫflugr ofan \hld\ sá’s ǫllu rę́ðr.}{\lemma{Þȧ ... rę́ðr.}\Bfootnote{This verse is found only in \Hauksbok, in between the last two vv. It is without doubt a late, Christian addition.}}\eva

\bvb[X] — Then comes the mighty one, for the great judgement; strong from above, the one who over all things wields.\evb
\evg
