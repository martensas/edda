Under this section I have gathered sundry \emph{galders} (charms and spells) attested in Old Germanic languages. I have generally only included those with clear Heathen or traditional elements. The Old Saxon baptismal vow, while explicitly anti-Heathen, has also been included due to its mention of Germanic Heathen deities.


\chapter{Continental Germanic spells}

\bookStart{The Two Merseburg Galders}

\begin{flushright}%
Dating: TODO.

Meter: \Fornyrdislag, \Galdralag%
\end{flushright}

These two galders, preserved in a manuscript (TODO) are some of the only surviving examples of genuine Heathen galders from the continent. The two share a common two-part structure, each beginning with an \emph{historiola} (a pseudo-historical account describing the successful effects of the galder in the mythic past), followed by an \emph{imperative}, commanding that the willed effects take place in the present.

The first galder begins with an historiola describing a group of supernatural women in the midst of a battle who placed soldiers in fetters, hindering an army. The imperative then commands that some fetters in the present be destroyed so that captive(s) can escape.

The second galder begins with an historiola describing a group of Gods riding through the woods. Among them is Balder, whose horse sprains its foot. Three Gods are said to have sung (see Note to \emph{bi·guol} below) a healing-galder each over the horse in order to heal it. First sang the goddess Sithguth, then the goddess Sun, and finally the god Weden. The imperative (apparently the same as was sung over Balder’s horse) then commands that a sprain in the present be healed.

\sectionline

\bvg
\bva Ęiris \alst{s}ázun idisi \hld\ \alst{s}ázun hera duo der; &
suma \alst{h}apt \alst{h}ęptidun \hld\ suma \alst{h}ęri lęzidun &
suma \alst{k}lubodun \hld\ umbi \alst{k}uonjo-widi &
\alst{i}n·sprink hapt-bandun \hld\ \alst{i}n·far fígandun &
\edtext{.H.}{\Bfootnote{The meaning of this letter, which is very clear and written in the same hand as the galders, is uncertain. To me, the most convincing suggestion is that it be read as \emph{.N.}, short for Latin \emph{nomen} ‘name’, presumably the name for the person whom the singer wishes to free from the fetters.}}\eva

\bvb Of yore sat dises, sat here, then there: \\
some fastened fetters, some hindered armies, \\
some cleaved shackles (TODO!).— \\
Destroy the fetter-bonds, flee the fiends! \\
.H.\evb
\evg


\bvg
\bva \edtext{\alst{F}ol}{\Afootnote{\emph{Phol} ms.}} ęnde Wódan \hld\ \alst{f}uorun zi holza &
dú wart demo Balderes \alst{f}olon \hld\ sín \alst{f}uoz bi·ręnkit &
þú \edtrans{bi·guol}{begale}{\Bfootnote{third past singular of \emph{bi·galan} ‘begale’, transitive of \emph{galan} ‘gale, sing a galder’. This verb is important as it is the origin of the verbal noun “galder” (literally ‘something galed’), which is thus shown to describe the charm.}} en \edtext{\alst{S}inthgunt}{\lemma{Sinthgunt}\Afootnote{\emph{Sinhtgunt} ms.}} \hld\ \alst{S}unna era swister &
þú bi·guol en \alst{F}rija \hld\ \alst{F}olla era swister &
þú bi·guol en \alst{W}ódan \hld\ só hé \alst{w}ola konda &
só-se \alst{b}èn-ręnkí \hld\ só-se \alst{b}luot-ręnkí \hld\ só-se lidi-ręnkí &
\ind \alst{b}èn zi \alst{b}èna &
\ind \alst{b}luot zi \alst{b}luoda &
\alst{l}id zi ge·\alst{l}iden \hld\ só-se ge·\alst{l}imida sín!\eva

\bvb Phol and Weden journeyed in the woods; \\
then was the foot of Balder’s foal sprained. \\
Then \inx[C]{begale}[begaled] him \inx[P]{Sithguth}, \inx[P]{Sun} her sister; \\
then begaled him \inx[P]{Frie}, \inx[P]{Full} her sister; \\
then begaled him Weden, as he knew well: \\
Like bone-sprain, like blood-sprain, like joint-sprain! \\
Bone to bone, \\
blood to blood, \\
joint to joints, like were they glued together!\evb
\evg



\section{Against worms (Contra vermes)}

\bvg
\bva Gang út, \alst{n}esso, \hld\ mid \alst{n}igun \alst{n}essi-klínon, &
ut fana þemo marge an þat \alst{b}èn, \hld\ fan þemo \alst{b}ène an þat flesg, &
ut fan þemo flesgke an þia \alst{h}úd, \hld\ ut fan þera \alst{h}úd an þesa strála. &
Drohtin, werthe só.\eva

\bvb Go out, Nesse, with nine small Nesses! Out from the marrow onto the bone, from this bone onto the flesh, out from the flesh onto the skin, out from the skin onto these arrows. Lord, may it be so.\evb
\evg


\section{The Old Saxon Baptismal vow}

\bpg
\bpa „For·sachistu diobole?“ \emph{et respondeat:} „ec for·sacho diabole“\epa

\bpb “Forsakest thou the Devil?” \emph{and he should respond:} “I forsake the Devil.”\epb
\epg


\bpg
\bpa „end allum diobol-gelde?“ \emph{respondeat:} „end ec for·sacho allum diobol-gelde.“\epa

\bpb “And all devil-yields?” \emph{he should respond:} “I forsake all devil-yields.”\epb
\epg


\bpg
\bpa „End allum dioboles wercum?“ \emph{respondeat} „end ec for·sacho allum dioboles wercum and wordum, Thuner ende Wóden ende Sax-nòte ende allem them un·holdum the hira ge·nòtas sint.“\epa

\bpb “And all the Devil’s works” \emph{he should respond:} “and I forsake all the works and words of the Devil; Thunder and Weden and Saxneet and all those unhold ones who are their fellows.”\epb
\epg


\bpg
\bpa „Ge·lòbistu in Got ala-męhtigun fader?“ „Ec ge·lòbo in Got ala-męhtigun fader.“\epa

\bpb “Believest thou in God, the almighty father?” “I believe in God, the almighty father.”\epb
\epg


\bpg
\bpa „Ge·lòbistu in Crist Godes suno?“ „Ec ge·lòbo in Crist Gotes suno.“\epa

\bpb “Believest thou in Christ, God’s son?” “I believe in Christ, God’s son.”\epb
\epg


\bpg
\bpa „Ge·lòbistu in hàlogan gàst?“ „Ec ge·lòbo in hàlogan gàst.“\epa

\bpb “Believest thou in the Holy Ghost?” “I believe in the Holy Ghost.”\epb
\epg


\chapter{Old English spells}

%\section{Against a Dwarf (\emph{Wið dweorh})}\chapterStart{}
\setBookCode{WidDweorh}

\begin{flushright}%
\textbf{Dating:} TODO

\textbf{Meter:} \Fornyrdislag%
\end{flushright}

\subsection{Introduction}

TODO: Introduction.

\sectionline

\subsection{Text}

\bpg\bpa Mann sceal niman \emph{seofon} lytle of-lætan swylce mann mid ofrað, ond wrítan þás naman on ælcre oflætan: Maximianus, Malchus, Johannes, Martinianus, Dionisius, Constantinus, Serafion.  Þænne eft þæt galdor þæt hér æfter cweð[eð] mann sceal singan, ærest on þæt wynstre éare, þænne on þæt swíðre éare, þænne búfan þæs mannes moldan; ond gá þænne ân mæden-mann tó, ond hó hit ǫn his sweoran, ond dó mann swá þrý dagas.  Him bið sóna sél.\epa

\bpb One shall take seven small wafers, such as one offers [during the Mass], and write these names on each wafer: Maximianus, Malchus, Johannes, Martinianus, Dionysius, Constantinus, Seraphion.  After that shall one sing this galder which is henceforth said; first into the left ear, then into the right ear, then over the man’s head; and thereafter a maiden go forth, and hang it on his neck; and one do so for three days.  He will soon be well.\epb\epg


\bvg\bva%
Hér cóm \alst{i}n·gangan \hld\ \alst{i}n·spiden wiht, &
hæfde him his \alst{h}aman ǫn \alst{h}anda; \hld\ cwæð þæt þú his \alst{h}æncgest wǽre, &
\alst{l}ęgeþe þé his téage \emph{ǫ}n sweoran; \hld\ ǫn·gunnan him ǫf þæm \alst{l}ande líðan. &
Sóna swá hý ǫf þæm \alst{l}ande cóman \hld\ þá ǫn·gunnan him þá \emph{\alst{l}eomu} cólian.— &
Þá cóm in·gangan \hld\ déores sweostar; &
þá ge·\alst{æ}ndode héo \hld\ ond \alst{â}ðas swór, &
þæt næfre þis þæm \alst{a}dlegan \hld\ \emph{\alst{e}gl}ian ne móste &
né þæm þe þis \alst{g}aldor \hld\ be·\alst{g}ýtan mihte &
oððe þe þis \alst{g}aldor \hld\ on·\alst{g}alan cu̇ðe. &
Amen fiað.\eva

\bvb Here an inspiden wight came walking in, \\
had his harness in his hands, said that thou wert his horse, \\
laid his reins on thy neck; they began to ride away from the land. \\
As soon as they came away from the land then they began to cool limbs. \\
Then the beast’s sister came walking in; \\
then she made an end to it and swore oaths \\
that this never should torment the ailing man, \\
nor him who this galder might get, \\
nor whomever this galder could gale. \\
Amen, let it be.\evb\evg

\sectionline


\section{Against a Sudden Stitch (\emph{Wið fǽr-stice})}\chapterStart{}

\begin{flushright}%
\textbf{Dating:} ?

\textbf{Meter:} \Fornyrdislag%para
\end{flushright}%

Attested in \Lacnunga.

\sectionline

\bvg\bva \alst{H}lúde wǽran hý, lá, \alst{h}lúde, \hld\ þá hý ofer þone \alst{h}lǽw ridan, &
wǽran \alst{â}n-móde, \hld\ þá hý \alst{o}fer land ridan. &
Scyld þú þé nú, þú þysne \alst{n}íð \hld\ ge·\alst{n}esan móte. &
\alst{Ú}t, lýtel spere, \hld\ gif hér \alst{i}nne síe!\eva

\bvb Loud were they, lo, loud, when they rode over that mound; \\
they were steadfast, when they rode over land. \\
Shield thyself now; thou mayst escape this evil! \\
Out little spear, if here within it be!\evb\evg


\bvg\bva Stód under \alst{l}inde, \hld\ under \alst{l}eohtum scylde, &
þær þá \alst{m}ihtigan wíf \hld\ hýra \alst{m}ægen be·rǽddon &
and hý \alst{g}yllende \hld\ \alst{g}âras sændan; &
ic him \alst{ó}ðerne \hld\ \alst{e}ft wille sændan, &
\alst{f}léogende \alst{f}lâne \hld\ \alst{f}orane tó·géanes. &
\alst{Ú}t, lytel spere, \hld\ gif hit her \alst{i}nne sý!\eva

\bvb Stood under the linden \ken{shield}—under the light shield— \\
where those mighty wives their might arrayed, \\
and they yelling spears did send. \\
To them another [projectile] will I send back: \\
a flying arrow, aimed against [them]. \\
Out little spear, if here within it be!\evb\evg


\bvg\bva \alst{S}æt \alst{s}mið, \hld\ \alst{s}loh seax, &
lytel \alst{í}serna, \hld\ \alst{w}und swíðe. &
\alst{Ú}t, lytel spere, \hld\ gif her \alst{i}nne sý!\eva

\bvb Sat the smith, struck the sax: \\
a little iron-thing—a great wound. \\
Out little spear, if here within it be!\evb\evg


\bvg\bva \alst{S}yx \alst{s}miðas \alst{s}ætan, &
\alst{w}æl-spera \alst{w}orhtan. &
\alst{Ú}t, spere, \hld\ næs \alst{i}n, spere! &
Gif her \alst{i}nne sý \hld\ \alst{í}senes dǽl, &
\alst{h}æg-tessan ge·weorc, \hld\ \alst{h}it sceal ge·myltan.\eva

\bvb Six smiths sat, \\
wrought slaughter-spears. \\
Out, spear! Be not in, spear! \\
If here within be a part of iron, \\
the work of a \inx[C]{hag-tess}—\emph{it} shall melt!\evb\evg


\bvg\bva Gif þú wǽre on \alst{f}ell scoten \hld\ oððe wǽre on \alst{f}læsc scoten &
oððe wǽre on blód scoten \hld\ [...] &
oððe wǽre on \alst{l}ið scoten, \hld\ næfre ne sý þín \alst{l}íf atæsed;\eva

\bvb If thou wert shot in the skin, or wert shot in the flesh, \\
or wert shot in the blood, [...], \\
or wert shot in the limb—never be thy life injured.\evb\evg


\bvg\bva gif hit wǽre \alst{ė}sa ge·scot \hld\ oððe hit wǽre \alst{y}lfa ge·scot &
oððe hit wǽre \alst{h}æg-tessan ge·scot, \hld\ nú ic wille þín \alst{h}elpan: &
þis þé tó bóte \alst{ė}sa ge·scotes, \hld\ þis þé tó bóte \alst{y}lfa ge·scotes, &
þis þé tó bóte \alst{h}æg-tessan ge·scotes; \hld\ ic þín wille \alst{h}elpan.\eva

\bvb If it were Eese-shot, or it were Elf-shot,\footnoteB{Formulaic; see \inx[F]{Eese and Elves}. That they are held in the same category as the hag-tess—a witch—indicates Christian influence. Among the Germanic peoples the elves and Eese were originally beneficial, as seen by numerous names like Alfred (OE \emph{Ęlf-réd} ‘Elf-counsel’), Oswald (OE \emph{Ós-weald} ‘Os-power’), Elfwin (Lomb. \emph{Alb-oin} ‘Elf-friend’), Oshelm (Lomb. \emph{Anselm} ‘Os-helmet’).}  \\
or it were Hag-tess-shot—now I will help thee! \\
This for thee as cure against Eese-shot; this for thee as cure against Elf-shot;  \\
this for thee as cure against Hag-tess-shot—I will help thee!\evb\evg


\bvg\bva \alst{F}leo þær on \hld\ \alst{f}yrgen-hæfde! &
\alst{H}âl wes-tu, \hld\ \alst{h}elpe þín drihten! &
Nim þonne þæt seax, \hld\ ado on wætan.\eva

\bvb TODO. \\
Be thou hale, may the Lord help thee.\evb\evg

\sectionline


\section{The Nine Herbs galder}\chapterStart{}
\setBookCode{NineHerbs}

\begin{flushright}%
\textbf{Dating:} ?

\textbf{Meter:} \Fornyrdislag%para
\end{flushright}%

\subsection{Introduction}

TODO: introduction

\sectionline

\subsection{Text}

\bvg\bva%
Ge·\alst{m}yne ðú \alst{m}ug-wyrt \hld\ hwæt þú á·\alst{m}eldodest &
hwæt þu \alst{r}enadest \hld\ æt \alst{R}egen-melde?\eva

\bvb Rememberest thou, Mugwort, what thou didst declare, \\
what thou didst arrange at Reinmeld?\evb\evg


\bvg
\bva Una þú hâttest \hld\ yldost wyrta &
þú miht \edtext{wið III \hld\ and wið XXX}{\lemma{wið III and wið XXX ‘against three and against thirty’}\Bfootnote{I.e. ‘against a great number of foemen’.  ‘Three and thirty’ is formulaic and appears in many later English and Scottish folk-songs, viz. Child 4EFG, 18B, 20C, 30, 53BCDEIKM, 63EFH, 73I, 97AC, 100AG, 110BGH, 156G, 185A, 187A, 187C, 190A, 192A, 193B, 203C, 211A, 217GHLN, 244A, 268A, 269C, 281ABC.  Things described include horses, heads of cattle, warriors, days, years, winters.}} &
þú miht wiþ attre \hld\ and wið on·flyge &
þú miht wiþ þâm lâþan \hld\ ðe geond lond færð\eva

\bvb Un thou art called, the oldest of worts; \\
thou availest against three and against thirty; \\
thou availest against the venom and against the onflier; \\
thou availest against the loathsome one that journeys through the lands.\evb\evg


\bvg
\bva + Ond þú weg·bráde \hld\ wyrta módor &
éastan opene \hld\ innan mihtigu &
ofer ðy cræte curran \hld\ ofer ðy cwéne reodan &
\ind ofer ðy brýde brýodedon &
\ind ofer ðy fearras fnærdon.\eva

\bvb And thou, Waybroad, mother of worts, \\
open from the east, mighty from within. \\
Over thee TODO.\evb\evg


\bvg
\bva Eallum þu þon wið·stóde \hld\ and wið·stunedest &
swá ðú wið·stonde attre \hld\ and on·flyge &
and þǽm lâðan \hld\ þe geond lond fereð.\eva

\bvb Them all didst thou then withstand, and didst stop; \\
so mayst thou withstand the venom and the onflier, \\
and the loathsome one that journeys through the lands.\evb\evg


\bvg
\bva Stune hætte þéos wyrt, \hld\ héo on stâne ge·weox &
stond héo wið attre, \hld\ stunað héo wærce &
Stiðe héo hatte, \hld\ wið·stunað héo attre &
wreceð héo wrâðan, \hld\ weorpeð út attor.\eva

\bvb Stun is this wort called, she grew on stone; \\
she withstands venom, she stops aches. \\
Stithe is she called, she stops the venom; \\
she drives away the wroth one, casts out the venom.\evb\evg


\bvg
\bva + Þis is séo wyrt \hld\ séo wiþ wyrm ge·feaht &
þéos mæg wið attre, \hld\ héo mæg wið on·flyge; &
héo mæg wið ðâm lâþan \hld\ ðe geond lond fereþ.\eva

\bvb This is the wort that fought against the Wyrm; \\
this one avails against the venom, she avails against the onflier; \\
she avails against the loathsome one that journeys through the lands.\evb\evg


\bvg
\bva Fleoh þú nú attor-lâðe, \hld\ séo lǽsse ðá mâran &
séo mâre þá lǽssan, \hld\ oððæt him beigra bót sý!\eva

\bvb TODO\evb\evg


\bvg
\bva Ge·myne þú, mægðe,\hld\ hwæt þú á·meldodest &
hwæt ðú ge·ændadest \hld\ æt Alor-forda &
þæt nǽfre for ge·floge \hld\ feorh ne ge·sealde &
syþðan him mǫn mægðan \hld\ tú mete ge·gyrede\eva

\bvb TODO\evb\evg


\bvg
\bva Þis is séo wyrt \hld\ ðe wer-gulu hatte &
ðás on·sænde seolh \hld\ ofer sǽs hrygc &
ondan attres \hld\ óþres tó bóte\eva

\bvb TODO\evb\evg


\bvg
\bva Ðás VIIII magon \hld\ wið nygon attrum.\eva

\bvb These nine avail against nine venoms.\evb\evg


\bvg
\bva + Wyrm cóm snícan, \hld\ to·slât hé man &
ðá ge·nam Wóden \hld\ VIIII wuldor-tânas &
slóh ðá þá nǽddran \hld\ þæt héo on VIIII tó·fléah &
Þǽr ge·ændade æppel \hld\ and attor &
þæt héo nǽfre ne wolde \hld\ on hús búgan.\eva

\bvb A \inx[C]{Wyrm} came crawling; he tore apart a man. \\
Then took Weden nine glory-twigs, \\
slew then that adder, that it sprung into nine [parts]. \\
There ended apple and venom, \\
that she would never wish to enter a house.\evb\evg


\bvg
\bva + Fille and finule, \hld\ fela-mihtigu twá &
þá wyrte ge·sceop \hld\ wítig drihten &
hâlig on heofonum, \hld\ þá hé hongode; &
sętte and sęnde \hld\ on VII worulde &
earmum and éadigum \hld\ eallum tó bóte\eva

\bvb Fill and Fennel, the many-mighty two; \\
those worts the wise lord shaped, \\
holy in heaven when he hung. \\
He set and sent them into seven worlds, \\
for wretched men and for wealthy, for all men as a cure.\evb\evg


\bvg
\bva \alst{St}ǫnd héo wið wærce, \hld\ \alst{st}unað héo wið attre &
séo mæg wið III \hld\ \emph{and} wið XXX &
wið \emph{\alst{f}éondes} hǫnd \hld\ and wið \alst{f}ǽr-bregde &
wið \alst{m}alscrunge \hld\ \alst{m}ânra wihta\eva

\bvb She stands against ache, she stands against venom;
she avails against three and against thirty;
against \evb\evg


\bvg
\bva + Nu magon þás \emph{nygon} \alst{w}yrta \hld\ wið nygon \alst{w}uldor-ge·flogenum &
wið \emph{nygon} \alst{a}ttrum \hld\ and wið nygon \alst{o}n·flygnum &
wið ðý \alst{r}éadan attre, \hld\ wið ðý \alst{r}unlan attre &
wið ðý \alst{h}witan attre, \hld\ wið ðý \emph{\alst{h}æwe}nan attre &
wið ðý \alst{g}eolwan attre, \hld\ wið ðý \alst{g}rénan attre &
wið ðý \alst{w}onnan attre, \hld\ wið ðý \alst{w}edenan attre &
wið ðý \alst{b}rúnan attre, \hld\ wið ðý \alst{b}asewan attre &
wið \alst{w}yrm-ge·blæd, \hld\ wið \alst{w}æter-ge·blæd &
wið \alst{þ}orn-ge·blæd, \hld\ wið \alst{þ}ystel-ge·blæd &
wið \alst{ý}s-ge·blæd, \hld\ wið \alst{a}ttor-ge·blæd\eva

\bvb Now these nine worts avail against glory-onfliers: \\
against nine venoms and against nine onfliers; \\
against the red venom; against the TODO venom; \\
against the white venom; against the TODO venom; \\
against the yellow venom; against the green venom; \\
against the TODO venom; against the TODO venom; \\
against the brown venom; against the TODO venom; \\
against worm-TODO; against water-TODO; \\
against thorn-TODO; against thistle-TODO; \\
against ice-TODO; against venom-TODO.\evb\evg


\bvg
\bva Gif ænig \alst{a}ttor cume \hld\ \alst{éa}stan fleógan &
oððe ǽnig norðan cume &
oððe ǽnig \alst{w}estan \hld\ ofer \alst{w}er-þeóde\eva

\bvb If any venom should come flying from the east; \\
or any come from the north; \\
or any from the west, over mankind.\evb\evg


\bvg
\bva + Críst stód ofer \alst{á}dle \hld\ \alst{ǽ}ngan cundes &
Ic \alst{â}na wât \hld\ éa rinnende &
þǽr þá \alst{n}ygon \alst{n}ǽdran \hld\ \alst{n}éan be·healdað\eva

\bvb Christ stood over TODO; \\
I know one river running; \\
there the nine adders TODO.\evb\evg


\bvg
\bva Motan ealle \alst{w}éoda \hld\ nu \alst{w}yrtum á·springan &
\alst{s}ǽs tó·\alst{s}lúpan, \hld\ eal \alst{s}ealt wæter &
ðonne ic \alst{þ}is attor \hld\ of \alst{þ}é ge·bláwe.\eva

\bvb TODO\evb\evg


\bpg\bpa Mucgwyrt, weg-brade þe eastan open sy, lombes-cyrse, attor-laðan, mageðan, netelan, wudu-sur-æppel, fille and finul, ealde sapan. Ge·wyrc ða wyrta to duste, mængc wiþ þa sapan and wiþ þæs æpples gor. Wyrc slypan of wætere and of axsan, ge·nim finol, wyl on þære slyppan and beþe mid æggemongc, þonne he þa sealfe on do, ge ær ge æfter. Sing þæt galdor on æcre þara wyrta, :III: ær he hy wyrce and on þone æppel eal-swa; ond singe þon męn in þǫne mu̇ð and in þá éaran búta and on ðá wunde þæt ilce gealdor, ær he þá sealfe on dó.\epa

\bpb TODO.\epb\epg

\sectionline



\chapter{Old Norse spells}

\section{Ribe rune charm}

\bvg
\bva[]\alst{Jo}rð bið ak varðę \hld\ ok \alst{u}p-himęn &
\alst{s}ól ok \alst{s}antę María \hld\ ok \alst{s}alfęn Guð dróttęn &
þęt hamn \alst{l}ę́ mik \alst{l}ę́knęs-hand \hld\ ok \alst{l}yf-tungę &
at lyfę \alst{b}ifjandę \hld\ þęr \alst{b}ótę þarf. &
\ind Ór \alst{b}ak ok ór \alst{b}ryst
\ind ór \alst{l}íkę ok ór \alst{l}im &
\ind ór \alst{ǿ}fęn ok ór \alst{ǿ}ręn &
\ind ór \alst{a}llę þé þęr \alst{i}llt kann í \alst{a}t-kumę. &
Svart hètęr \alst{st}ènn \hld\ han \alst{st}ę́r í hafę útę, &
\ind þęr liggęr á þé \alst{n}í\emph{u} \alst{n}auðęr; &
\ind þęr skulę hvęrki \alst{s}ǿtęn \alst{s}ofę; &
\ind ęð \alst{v}armęn \alst{v}akę; &
førr ęn þú þęssa bót biðęr, þęr ak orð at-kvę́ðę ronti.\eva

\bvb I bid earth to ward, and up-heaven, \\
sun and saint Mary—and the very lord God, \\
that he lend me a healing-hand and medicine-tongue, \\
as medicine for the trembling one who needs a cure. \\
Out of back and out of breast; \\
out of body and out of limb; \\
out of eyes and out of ears; \\
out of everything where evil which might come in! \\
Swart is called a stone, he stands out in the ocean: \\
there lie on it nine needs; \\
they will not [let thee] sleep sweetly \\
nor wake warmly— \\
until thou prayest this cure, where I tried the words of the charms.\evb
\evg

\section{Charms from Bryggen}

These charms are found inscribed on medieval pieces of wood found at Bryggen in the city of Bergen, Norway.

\sectionline

A stick with four sides, dated to c. 1335. It is clearly a love-charm and—as seen by the feminine dative adjective \emph{sjalfri} ‘self’ on side C—addressed to a woman. The language is very close to that of \Skirnismal\ 36, wherein Shirner threatens to curse the ettin-woman Gird with \emph{ęrgi} ‘degeneracy’ and \emph{ǿði} ‘madness’ and \emph{óþoli} ‘impatience’ unless she sleep with his master, Free. A crucial difference is of course that this charm is not an Eddic narrative poem; it must have been expected to work. Both of these share a root with the curse-formula seen on the two C7th runic inscriptions from Stentoften and Björketorp (see TODO), wherein the destroyer of the respective monuments will be \emph{hermalausaʀ argjú} ‘restless with degeneracy’, i.e. ‘incessantly randy’. As it would be absurd to think that the poet of \Skirnismal\ should have learned this type of magic from one of the rune-stones, and then passed this onto the carver of the present inscription, we must rather be dealing with a common form of curse magic, wherein the victim is cursed with incessant randiness leading to sexual perversion.

\bvg
\bva[A]\mssnote{\textbf{B257}}Ríst ek \alst{b}ót-rúnar \hld\ ríst ek \alst{b}jarg-rúnar &
\ind \alst{ei}n-falt við \alst{ǫ}lfum &
\ind \alst{t}ví-falt við \alst{t}rollum &
\ind \alst{þ}rí-falt við \alst{þ}u\emph{rsum}\eva

\bvb I carve healing-runes; I carve saving-runes; onefold against elves; twofold against trolls; threefold against thurses.\evb
\evg


\bvg
\bva[B]Við inni \alst{sk}ǿðu \hld\ \alst{sk}ag-val-kyrju &
svá’t \alst{ei} megi \hld\ þó-at \alst{ę́} vili &
\alst{l}ę́-vís kona \hld\ \alst{l}ífi þínu g\emph{randa}.\eva

\bvb Against the scatheful shag-walkirrie, so that she may not—although she ever wishes to, that guile-wise woman—harm thy life.\evb
\evg


\bvg
\bva[C]Ek \alst{s}endir þér \hld\ ek \alst{s}é á þér &
\alst{y}lgjar \alst{e}rgi \hld\ ok \alst{ó}þola; &
á þér hríni \alst{ó}þoli \hld\ ok \alst{jǫ}tuns móð\emph{r}; &
\alst{s}it-tu aldri, \hld\ \alst{s}op-tu aldri.\eva

\bvb I send to thee—I see on thee—a she-wolf’s degeneracy and impatience; on thee stick impatience, and an ettin’s wrath! Sit thou never, sleep thou never!\evb
\evg


\bvg
\bva[D]Ant mér sem sjalfri þér. Beirist rubus rabus et arantabus laus abus rosa gava\eva

\bvb Love me like thy self.\evb
\evg

\sectionline

\bvg
\bva[]\mssnote{\textbf{B380}}\alst{H}ęill sé þú \hld\ ok í \alst{h}ugum góðum; &
\ind \alst{Þ}órr þik \alst{þ}iggi, &
\ind \edtrans{\alst{Ó}ðinn þik \alst{ęi}gi}{‘may Weden own thee’}{\Bfootnote{See note to \Voluspa\ 23.}}.\eva

\bvb Be thou hale, and in good spirits;\footnoteB{A formula also attested in \Hymiskvida\ 41; see there for parallels.} may Thunder receive thee, may Weden own thee.\evb
\evg


\section{Runic plates}
