\part{Ancient Germanic Charms and Spells}

I have here gathered sundry charms spells; galders and leeds, assembled from sources across the ancient Germanic world. I have generally only included those with clear Heathen elements or contexts, though a few are of Christian origin. The Old Saxon baptismal vow, while explicitly anti-pagan, has also been included due to its mention of Germanic Heathen deities.


\chapter{Continental Germanic spells}

\section{The two Merseburg charms}

\bvg
\bva Eiris \alst{s}ázun idísi \hld\ \alst{s}ázun hera dóder; &
suma \alst{h}apt \alst{h}ęptidun \hld\ suma \alst{h}eri lęzidun &
suma \alst{c}lubodun \hld\ umbi \alst{c}ónio-widi &
\alst{i}n·sprinc hapt-bandun \hld\ \alst{i}n·far fígandun .H.\eva

\bvb Of yore stayed dises, stayed here and there: some fastened fetters, some hindered hosts, some cleaved shackles.—Break the fetter-bonds, flee the fiends! .H.\footnoteB{TODO: note about this strange mark in the ms.}\evb
\evg


\bvg
\bva \edtext{\alst{F}ol}{\lemma{Fol}\Afootnote{\emph{Phol} ms.}} ende Wódan \hld\ \alst{f}órun zi holza &
dú wart demo Balderes \alst{f}olon \hld\ sín \alst{f}óz bi·ręnkit &
þú bi·gól en \edtext{\alst{S}inthgunt}{\lemma{Sinthgunt}\Afootnote{\emph{Sinhtgunt} ms.}} \hld\ \alst{S}unna era swister &
þú bi·gól en \alst{F}rija \hld\ \alst{F}olla era swister &
þú bi·gól en \alst{W}ódan \hld\ só hé \alst{w}ola conda &
sóse \alst{b}èn-ręnkí \hld\ sóse \alst{b}lót-ręnkí &
\ind sóse lidi-ręnkí &
\ind \alst{b}èn zi \alst{b}èna &
\ind \alst{b}lót zi \alst{b}lóda &
\alst{l}id zi ge·\alst{l}iden \hld\ só-se ge·\alst{l}imida sín.\eva

\bvb Phol and Weden journeyed to the woods; then was the foot of Balder’s foal sprained. Then \inx[C]{begale}[begaled] him \inx[P]{Sithguth}, \inx[P]{Sun} her sister; then begaled him \inx[P]{Frie}, \inx[P]{Full} her sister; then begaled him Weden, as he well knew: “Like bone-sprain, like blood-sprain, like joint-sprain! Bone to bone, blood to blood, joint to joints, like were they glued together!”\evb
\evg


\section{Against worms (Contra vermes)}

\bvg
\bva Gang út, \alst{n}esso, \hld\ mid \alst{n}igun \alst{n}essi-klínon, &
ut fana þemo marge an þat \alst{b}èn, \hld\ fan þemo \alst{b}ène an þat flesg, &
ut fan þemo flesgke an þia \alst{h}úd, \hld\ ut fan þera \alst{h}úd an þesa strála. &
Drohtin, werthe só.\eva

\bvb Go out, Nesse, with nine small Nesses! Out from the marrow onto the bone, from this bone onto the flesh, out from the flesh onto the skin, out from the skin onto these arrows. Lord, may it be so.\evb
\evg


\section{The Old Saxon Baptismal vow}

\bpg
\bpa „For·sachistu diobole?“ \emph{et respondeat:} „ec for·sacho diabole“\epa

\bpb “Forsakest thou the Devil?” and he should respond: “I forsake the Devil.”\epb
\epg


\bpg
\bpa „end allum diobol-gelde?“ \emph{respondeat:} „end ec for·sacho allum diobol-gelde.“\epa

\bpb “And all Devil-yields?” he should respond: “I forsake all devil-yields.”\epb
\epg


\bpg
\bpa „End allum dioboles wercum?“ \emph{respondeat} „end ec for·sacho allum dioboles wercum and wordum, Thuner ende Wóden ende Sax-nòte ende allem them un·holdum the hira ge·nòtas sint.“\epa

\bpb “And all the works of the Devil?” he should respond: “and I forsake all the works and words of the Devil; Thunder and Weden and Saxneet and all those unhold ones who are their fellows.”\epb
\epg


\bpg
\bpa „Ge·lòbistu in got ala-męhtigun fader?“ „Ec ge·lòbo in got ala-męhtigun fader.“\epa

\bpb “Believest thou in God, the almighty father?” “I believe in God, the almighty father.”\epb
\epg


\bpg
\bpa „Ge·lòbistu in Crist godes suno?“ „Ec ge·lòbo in Crist gotes suno.“\epa

\bpb “Believest thou in Christ, God’s son?” “I believe in Christ, God’s son.”\epb
\epg


\bpg
\bpa „Ge·lòbistu in hàlogan gàst?“ „Ec ge·lòbo in hàlogan gàst.“\epa

\bpb “Believest thou in the Holy Ghost?” “I believe in the Holy Ghost.”\epb
\epg


\chapter{Old English spells}

\section{Against a dwarf}


\section{Wið fę́r-stice}

Attested in \Lacnunga.

\bvg
\bva[0]Hlúde wę́ran hý, lá, hlúde, \hld\ ðá hý ofer þone hlę́w ridan, &
wę́ran àn-móde, \hld\ ðá hý ofer land ridan. &
Scyld ðú ðé nú, þú ðysne níð \hld\ ge·nesan móte. &
Út, lýtel spere, \hld\ gif hér inne síe!\eva

\bvb[0]Loud were they, lo, loud, when they rode over that mound; they were steadfast, when they rode over land. Shield thyself now; thou mayst escape this evil! Out little spear, if here within it be!\evb
\evg


\bvg
\bva[0]Stód under linde, \hld\ under leohtum scylde, &
þęr ðá mihtigan wíf \hld\ hýra męgen be·rę́ddon &
and hý gyllende \hld\ gàras sęndan; &
ic him óðerne \hld\ eft wille sęndan, &
fléogende flàne \hld\ forane tó·géanes. &
Út, lytel spere, \hld\ gif hit her inne sý!\eva

\bvb[0]Stood under the linden \ken{shield}—under the light shield—where those mighty wives their might arrayed, and they yelling spears did send. I to them another will afterwards send: a flying arrow, back against [them]. Out little spear, if here within it be!\evb
\evg


\bvg
\bva[0]Sęt smið, \hld\ sloh seax &
lytel íserna, \hld\ wund swíðe. &
Ut, lytel spere, \hld\ gif her inne sý!\eva

\bvb[0]Sat the smith, struck the sax; a little iron-thing; a wound severe. Out little spear, if here within it be!\evb
\evg


\bvg
\bva[0]Syx smiðas sętan, \hld\ węl-spera worhtan. &
Út, spere, \hld\ nęs in, spere! &
Gif her inne sy \hld\ ísenes dęl, &
hęg-tessan geweorc, \hld\ hit sceal ge·myltan.\eva

\bvb[0]Six smiths sat, wrought slaughter-spears; out, spear; be not in, spear! If here within be a part of iron, a work of a \inx[C]{hag-tess}—it shall melt.\evb
\evg


\bvg
\bva[0]Gif ðu węre on fell scoten \hld\ oððe węre on flęsc scoten &
oððe węre on blod scoten \hld\ [...] &
oððe węre on lið scoten, \hld\ nęfre ne sy ðin lif atęsed;\eva

\bvb[0]If thou wert shot in the skin, or wert shot in the flesh, \\
or wert shot in the blood, [or wert shot in bone], \\
or wert shot in the limb—never be thy life injured.\evb
\evg


\bvg
\bva[0]gif hit wę́re ésa ge·scot \hld\ oððe hit wę́re ylfa ge·scot &
oððe hit wę́re hęg-tessan ge·scot, \hld\ nu ic wille ðín helpan: &
þis ðe to bóte ésa ge·scotes, \hld\ ðis ðe to bóte ylfa ge·scotes, &
ðis ðe to bóte hęg-tessan ge·scotes; \hld\ ic ðin wille helpan.\eva

\bvb[0]If it were Ease-shot, or it were elf-shot,\footnoteB{Formulaic; see \inx[F]{Ease and Elves}. That they are held in the same category as the hag-tess—a witch—indicates Christian influence. Among the Germanic peoples the elves and Ease were originally beneficial, as seen by numerous names like Alfred (OE \emph{Ęlfréd} ‘Elf-counsel’), Oswald (OE \emph{Ósweald} ‘Os-power’), Elfwin (Lomb. \emph{Alboin} ‘Elf-friend’), Oshelm (Lomb. \emph{Anselm} ‘Os-helmet’).}  \\
or it were hag-tess-shot—now I wish to help thee! \\
This for thee as cure against Ease-shot; this for thee as cure against elf-shot;  \\
this for thee as cure against hag-tess-shot—I wish to help thee!\evb
\evg


\bvg
\bva[0]Fleo þęr on \hld\ fyrgen-hęfde, &
hàl wes-tu, \hld\ helpe ðín drihten, &
nim þonne þęt seax, \hld\ ado on wętan.\eva

\bvb[0]TODO.\evb
\evg

\section{Nine herbs charm}

\bvg
\bva[0]Ge·myne ðú mug-wyrt \hld\ hwęt þú á·meldodest &
hwęt þu renadest \hld\ ęt Regen-melde?\eva

\bvb Rememberest thou, Mugwort, what thou madest known,  \\
what thou arrangedest at Reinmeld?\evb
\evg


\bvg\setlinenum{2}
\bva[0]Una þú hàttest \hld\ yldost wyrta &
þú miht wið III \hld\ and wið XXX &
þú miht wiþ attre \hld\ and wið on·flyge &
þú miht wiþ þàm làþan \hld\ ðe geond lond fęrð\eva

\bvb Un art thou called, oldest of worts; \\
thou availest against three and against thirty; \\
thou availest against the venom and against the onflier; \\
thou availest against the loathsome one that journeys through the lands.\evb
\evg


\bvg\setlinenum{6}
\bva[0]+ Ond þú weg·bráde \hld\ wyrta módor &
éastan opene \hld\ innan mihtigu &
ofer ðy cręte curran \hld\ ofer ðy cwéne réodan &
\ind ofer ðy brýde brýodedon &
\ind ofer ðy fearras fnęrdon.\eva

\bvb And thou, Waybroad, mother of worts, open from the east, mighty from within. Over thee TODO.\evb
\evg


\bvg\setlinenum{6}
\bva[0]Eallum þu þon wið·stóde \hld\ and wið·stunedest &
swá ðú wið·stonde attre \hld\ and on·flyge &
and þę́m làðan \hld\ þe geond lond fereð.\eva

\bvb Them all withstoodest thou then, and stoppedst; \\
so may thou withstand the venom and the onflier, \\
and the loathsome one that journeys through the lands.\evb
\evg


\bvg\setlinenum{6}
\bva[0]Stune hętte þéos wyrt, \hld\ héo on stàne ge·weox &
stond héo wið attre, \hld\ stunað héo węrce &
Stiðe héo hatte, \hld\ wið·stunað héo attre &
wreceð héo wràðan, \hld\ weorpeð út attor.\eva

\bvb Stun is this wort called, she grew on stone; \\
she withstands venom, she stops aches. \\
Stithe is she called, she stops the venom; \\
she drives away the wroth one, she casts out the venom.\evb
\evg


\bvg\setlinenum{6}
\bva[0]+ Þis is séo wyrt \hld\ séo wiþ wyrm ge·feaht &
þéos męg wið attre, \hld\ héo męg wið on·flyge; &
héo męg wið ðàm làþan \hld\ ðe geond lond fereþ.\eva

\bvb This is the wort that fought against the Worm; \\
this one avails against the venom, she avails against the onflier; \\
she avails against the loathsome one that journeys through the lands.\evb
\evg


\bvg\setlinenum{6}
\bva[0]Fleoh þú nú attor-làðe, \hld\ séo lę́sse ðá màran &
séo màre þá lę́ssan, \hld\ oððęt him beigra bót sý!\eva

\bvb TODO\evb
\evg


\bvg\setlinenum{6}
\bva[0]Ge·myne þú, męgðe,\hld\ hwęt þú á·meldodest &
hwęt ðú ge·ęndadest \hld\ ęt Alor-forda &
þęt nę́fre for ge·floge \hld\ feorh ne ge·sealde &
syþðan him mǫn męgðan \hld\ tú mete ge·gyrede\eva

\bvb TODO\evb
\evg


\bvg\setlinenum{6}
\bva[0]Þis is séo wyrt \hld\ ðe wer-gulu hatte &
ðás on·sęnde seolh \hld\ ofer sę́s hrygc &
ondan attres \hld\ óþres tó bóte\eva

\bvb TODO\evb
\evg


\bvg\setlinenum{6}
\bva[0]Ðás VIIII magon \hld\ wið nygon attrum.\eva

\bvb These nine avail against nine venoms.\evb
\evg


\bvg\setlinenum{6}
\bva[0]+ Wyrm cóm snícan, \hld\ to·slàt hé man &
ðá ge·nam Wóden \hld\ VIIII wuldor-tànas &
slóh ðá þá nę́ddran \hld\ þęt héo on VIIII tó·fléah &
Þę́r ge·ęndade ęppel \hld\ and attor &
þęt héo nę́fre ne wolde \hld\ on hús búgan.\eva

\bvb A \inx[C]{Worm} came crawling; he tore apart a man. \\
Then took Weden nine glory-twigs, \\
slew then that adder, that it sprung into nine [parts]. \\
There ended apple and venom, \\
that she would never wish to enter a house.\evb
\evg


\bvg\setlinenum{6}
\bva[0]+ Fille and finule, \hld\ fela-mihtigu twá &
þá wyrte ge·sceop \hld\ wítig drihten &
hàlig on heofonum, \hld\ þá hé hongode &
sette and sęnde \hld\ on VII worulde &
earmum and éadigum \hld\ eallum tó bóte\eva

\bvb Fill and Fennel, the many-mighty two; \\
those worts shaped the wise lord, \\
holy in heaven, when he hung. \\
He set and sent them into seven worlds, \\
for wretched men and for wealthy, for all men as a cure.\evb
\evg


\bvg\setlinenum{6}
\bva[0]Stond héo wið węrce, \hld\ stunað héo wið attre &
séo męg \edtext{wið III \hld\ \emph{and} wið XXX}{\lemma{wið III and wið XXX ‘against three and against thirty’}\Bfootnote{Formulaic; an uncountable amount; “snakes” are probably understood. This oral formula appears in many folk ballads, viz. (Child) 4EFG, 18B, 20C, 30, 53BCDEIKM, 63EFH, 73I, 97AC, 100AG, 110BGH, 156G, 185A, 187A, 187C, 190A, 192A, 193B, 203C, 211A, 217GHLN, 244A, 268A, 269C, 281ABC. Things described include horses, heads of cattle, warriors, days, years, winters.}} &
wið [féondes] hond \hld\ and wið fę́r-bregde &
wið malscrunge \hld\ manra wihta\eva

\bvb She stands against ache, she stands against venom;
she avails against three and against thirty;
against \evb
\evg


\bvg\setlinenum{6}
\bva[0]+ Nu magon þás VIIII wyrta \hld\ wið nygon wuldor-ge·flogenum &
wið VIIII attrum \hld\ and wið nygon on·flygnum &
wið ðý réadan attre, \hld\ wið ðý runlan attre &
wið ðý hwitan attre, \hld\ wið ðý [hęwe]nan attre &
wið ðý geolwan attre, \hld\ wið ðý grénan attre &
wið ðý wonnan attre, \hld\ wið ðý wedenan attre &
wið ðý brúnan attre, \hld\ wið ðý basewan attre &
wið wyrm-ge·blęd, \hld\ wið węter-ge·blęd &
wið þorn-ge·blęd, \hld\ wið þystel-ge·blęd &
wið ýs-ge·blęd, \hld\ wið attor-ge·blęd\eva

\bvb Now these nine worts avail against glory-onfliers: \\
against nine venoms and against nine onfliers; \\
against the red venom; against the TODO venom; \\
against the white venom; against the TODO venom; \\
against the yellow venom; against the green venom; \\
against the TODO venom; against the TODO venom; \\
against the brown venom; against the TODO venom; \\
against worm-TODO; against water-TODO; \\
against thorn-TODO; against thistle-TODO; \\
against ice-TODO; against venom-TODO.\evb
\evg


\bvg\setlinenum{6}
\bva[0]Gif ęnig attor cume \hld\ éastan fleógan &
oððe ę́nig norðan cume &
oððe ę́nig westan \hld\ ofer wer-ðeóde\eva

\bvb If any venom should come flying from the east; \\
or any come from the north; \\
or any from the west, over mankind.\evb
\evg


\bvg\setlinenum{6}
\bva[0]+ Críst stód ofer ádle \hld\ ę́ngan cundes &
Ic àna wàt \hld\ éa rinnende &
þę́r þá nygon nę́dran \hld\ néan be·healdað\eva

\bvb Christ stood over TODO; \\
I know one river running, \\
there the nine adders TODO.\evb
\evg


\bvg\setlinenum{6}
\bva[0]Motan ealle wéoda \hld\ nu wyrtum á·springan &
sę́s tó·slúpan, \hld\ eal sealt węter &
ðonne ic þis attor \hld\ of ðé ge·bláwe\eva

\bvb TODO\evb
\evg


\bpg\bpa Mucgwyrt, weg-brade þe eastan open sy, lombes-cyrse, attor-laðan, mageðan, netelan, wudu-sur-ęppel, fille and finul, ealde sapan. Ge·wyrc ða wyrta to duste, męngc wiþ þa sapan and wiþ þęs ępples gor.\epa

\bpb TODO.\epb\epg


\bpg\bpa Wyrc slypan of wętere and of axsan, ge·nim finol, wyl on þęre slyppan and beþe mid ęggemongc, þonne he þa sealfe on do, ge ęr ge ęfter.\epa

\bpb TODO.\epb\epg


\bpg\bpa Sing þęt galdor on ęcre þara wyrta, :III: ęr he hy wyrce and on þone ęppel eal-swa; ond singe þon men in þone muð and in þa earan buta and on ða wunde þęt ilce gealdor, ęr he þa sealfe on do :.\epa

\bpb TODO.\epb\epg



\chapter{Old Norse spells}

\section{Ribe rune charm}

\bvg
\bva[]\alst{Jo}rð bið ak varðę \hld\ ok \alst{u}p-himęn &
\alst{s}ól ok \alst{s}antę María \hld\ ok \alst{s}alfęn Guð dróttęn &
þęt hamn \alst{l}ę́ mik \alst{l}ę́knęs-hand \hld\ ok \alst{l}yf-tungę &
at lyfę \alst{b}ifjandę \hld\ þęr \alst{b}ótę þarf. &
\ind Ór \alst{b}ak ok ór \alst{b}ryst
\ind ór \alst{l}íkę ok ór \alst{l}im &
\ind ór \alst{ǿ}fęn ok ór \alst{ǿ}ręn &
\ind ór \alst{a}llę þé þęr \alst{i}llt kann í \alst{a}t-kumę. &
Svart hètęr \alst{st}ènn \hld\ han \alst{st}ę́r í hafę útę, &
\ind þęr liggęr á þé \alst{n}í\emph{u} \alst{n}auðęr; &
\ind þęr skulę hvęrki \alst{s}ǿtęn \alst{s}ofę; &
\ind ęð \alst{v}armęn \alst{v}akę; &
førr ęn þú þęssa bót biðęr, þęr ak orð at-kvę́ðę ronti.\eva

\bvb I bid earth to ward, and up-heaven, \\
sun and saint Mary—and the very lord God, \\
that he lend me a healing-hand and medicine-tongue, \\
as medicine for the trembling one who needs a cure. \\
Out of back and out of breast; \\
out of body and out of limb; \\
out of eyes and out of ears; \\
out of everything where evil which might come in! \\
Swart is called a stone, he stands out in the ocean: \\
there lie on it nine needs; \\
they will not [let thee] sleep sweetly \\
nor wake warmly— \\
until thou prayest this cure, where I tried the words of the charms.\evb
\evg

\section{Charms from Bryggen}

These charms are found inscribed on medieval pieces of wood found at Bryggen in the city of Bergen, Norway.

\sectionline

A stick with four sides, dated to c. 1335. It is clearly a love-charm and—as seen by the feminine dative adjective \emph{sjalfri} ‘self’ on side C—addressed to a woman. The language is very close to that of \Skirnismal\ 36, wherein Shirner threatens to curse the ettin-woman Gird with \emph{ęrgi} ‘degeneracy’ and \emph{ǿði} ‘madness’ and \emph{óþoli} ‘impatience’ unless she sleep with his master, Free. A crucial difference is of course that this charm is not an Eddic narrative poem; it must have been expected to work. Both of these share a root with the curse-formula seen on the two C7th runic inscriptions from Stentoften and Björketorp (see TODO), wherein the destroyer of the respective monuments will be \emph{hermalausaʀ argjú} ‘restless with degeneracy’, i.e. ‘incessantly randy’. As it would be absurd to think that the poet of \Skirnismal\ should have learned this type of magic from one of the rune-stones, and then passed this onto the carver of the present inscription, we must rather be dealing with a common form of curse magic, wherein the victim is cursed with incessant randiness leading to sexual perversion.

\bvg
\bva[A]\mssnote{\textbf{B257}}Ríst ek \alst{b}ót-rúnar \hld\ ríst ek \alst{b}jarg-rúnar &
\ind \alst{ei}n-falt við \alst{ǫ}lfum &
\ind \alst{t}ví-falt við \alst{t}rollum &
\ind \alst{þ}rí-falt við \alst{þ}u\emph{rsum}\eva

\bvb I carve healing-runes; I carve saving-runes; onefold against elves; twofold against trolls; threefold against thurses.\evb
\evg


\bvg
\bva[B]Við inni \alst{sk}ǿðu \hld\ \alst{sk}ag-val-kyrju &
svá’t \alst{ei} megi \hld\ þó-at \alst{ę́} vili &
\alst{l}ę́-vís kona \hld\ \alst{l}ífi þínu g\emph{randa}.\eva

\bvb Against the scatheful shag-walkirrie, so that she may not—although she ever wishes to, that guile-wise woman—harm thy life.\evb
\evg


\bvg
\bva[C]Ek \alst{s}endir þér \hld\ ek \alst{s}é á þér &
\alst{y}lgjar \alst{e}rgi \hld\ ok \alst{ó}þola; &
á þér hríni \alst{ó}þoli \hld\ ok \alst{jǫ}tuns móð\emph{r}; &
\alst{s}it-tu aldri, \hld\ \alst{s}op-tu aldri.\eva

\bvb I send to thee—I see on thee—a she-wolf’s degeneracy and impatience; on thee stick impatience, and an ettin’s wrath! Sit thou never, sleep thou never!\evb
\evg


\bvg
\bva[D]Ant mér sem sjalfri þér. Beirist rubus rabus et arantabus laus abus rosa gava\eva

\bvb Love me like thy self.\evb
\evg

\sectionline

\bvg
\bva[]\mssnote{\textbf{B380}}\alst{H}ęill sé þú \hld\ ok í \alst{h}ugum góðum; &
\ind \alst{Þ}órr þik \alst{þ}iggi, &
\ind \edtrans{\alst{Ó}ðinn þik \alst{ęi}gi}{‘may Weden own thee’}{\Bfootnote{See note to \Voluspa\ 23.}}.\eva

\bvb Be thou hale, and in good spirits;\footnoteB{A formula also attested in \Hymiskvida\ 41; see there for parallels.} may Thunder receive thee, may Weden own thee.\evb
\evg


\section{Runic plates}
