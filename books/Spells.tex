\part{Ancient Germanic Charms and Spells}

I have here gathered sundry charms spells; galders and leeds, assembled from sources across the ancient Germanic world. I have generally only included those with clear Heathen elements or contexts, though a few are of Christian origin. The Old Saxon baptismal vow, while explicitly anti-pagan, has also been included due to its mention of Germanic Heathen deities.


\chapter{Continental Germanic spells}

\section{The two Merseburg charms}

\bvg
\bva Eiris \alst{s}ázun idísi \hld\ \alst{s}ázun hera dóder; &
suma \alst{h}apt \alst{h}eptidun \hld\ suma \alst{h}eri lezidun &
suma \alst{c}lubodun \hld\ umbi \alst{c}óniowidi &
\alst{i}nsprinc haptbandun \hld\ \alst{i}nfar fígandun .H.\eva

\bvb Of yore stayed dises, stayed here and there: some fastened fetters, some hindered hosts, some cleaved shackles.—Break the fetter-bonds, flee the fiends! .H.\footnoteB{TODO: note about this strange mark in the ms.}\evb
\evg


\bvg
\bva \edtext{\alst{F}ol}{\lemma{Fol}\Afootnote{\emph{Phol} ms.}} ende Wódan \hld\ \alst{f}órun zi holza &
dú wart demo Balderes \alst{f}olon \hld\ sín \alst{f}óz birenkit &
thú bigól en \edtext{\alst{S}inthgunt}{\lemma{Sinthgunt}\Afootnote{\emph{Sinhtgunt} ms.}} \hld\ \alst{S}unna era swister &
thú bigól en \alst{F}rija \hld\ \alst{F}olla era swister &
thú bigól en \alst{W}ódan \hld\ só hé \alst{w}ola conda &
sóse \alst{b}énrenkí \hld\ sóse \alst{b}lótrenkí &
\ind sóse lidirenkí &
\ind \alst{b}én zi \alst{b}éna &
\ind \alst{b}lót zi \alst{b}lóda &
\alst{l}id zi ge\alst{l}iden \hld\ sóse ge\alst{l}imida sín.\eva

\bvb Phol and Weden journeyed to the woods; then was the foot of Balder’s foal sprained. Then \inx[C]{begale}[begaled] him \inx[P]{Sithguth}, \inx[P]{Sun} her sister; then begaled him \inx[P]{Frie}, \inx[P]{Full} her sister; then begaled him Weden, as he well knew: “Like bone-sprain, like blood-sprain, like joint-sprain! Bone to bone, blood to blood, joint to joints, like were they glued together!”\evb
\evg


\section{Against worms (Contra vermes)}

\bvg
\bva Gang út, \alst{n}esso, \hld\ mid \alst{n}igun \alst{n}essiklínon, &
ut fana themo marge an that bén, &
fan themo béne an that flesg, &
ut fan themo flesgke an thia húd, &
ut fan thera húd an thesa strála. &
Drohtin, werthe só.\eva

\bvb Go out, Nesse, with nine small Nesses! Out from the marrow onto the bone, from this bone onto the flesh, out from the flesh onto the skin, out from the skin onto these arrows. Lord, may it be so.\evb
\evg


\section{The Old Saxon Baptismal vow}

\bpg
\bpa „Forsachistu diobolę?“ \emph{et respondeat:} „ec forsacho diabolę“\epa

\bpb “Forsakest thou the Devil?” and he should respond: “I forsake the Devil.”\epb
\epg


\bpg
\bpa „end allum diobol geldę?“ \emph{respondeat:} „end ec forsacho allum
diobol geldę.“\epa

\bpb “And all Devil-yields?” he should respond: “I forsake all devil-yields.”\epb
\epg


\bpg
\bpa „End allum dioboles wercum?“ \emph{respondeat} „end ec forsacho allum dioboles wercum and wordum, Thunęr ende Wóden ende Saxnóte ende allëm them unholdum the hira genótas sint.“\epa

\bpb “And all the works of the Devil?” he should respond: “and I forsake all the works and words of the Devil; Thunder and Weden and Saxneet and all those unhold ones who are their fellows.”\epb
\epg


\bpg
\bpa „Gelóbistu in got alamehtigun fadęr?“ „Ec gelóbo in got alamehtigun fadęr.“\epa

\bpb “Believest thou in God, the almighty father?” “I believe in God, the almighty father.”\epb
\epg


\bpg
\bpa „Gelóbistu in Crist godes suno?“ „Ec gelóbo in Crist gotes suno.“\epa

\bpb “Believest thou in Christ, God’s son?” “I believe in Christ, God’s son.”\epb
\epg


\bpg
\bpa „Gelóbistu in hálogan gást?“ „Ec gelóbo in hálogan gást.“\epa

\bpb “Believest thou in the Holy Ghost?” “I believe in the Holy Ghost.”\epb
\epg


\chapter{Old English spells}

\section{Against a dwarf}


\section{Wið fǽrstice}

Attested in \Lacnunga.

\bvg
\bva[0]Hlúde wǽran hý, lá, hlúde, \hld\ ðá hý ofer þone hlǽw ridan, &
wǽran ánmóde, \hld\ ðá hý ofer land ridan. &
Scyld ðú ðé nú, þú ðysne níð \hld\ genesan móte. &
Út, lýtel spere, \hld\ gif hér inne síe!\eva

\bvb[0]Loud were they, lo, loud, when they rode over that mound; they were steadfast, when they rode over land. Shield thyself now; thou mayst escape this evil! Out little spear, if here within it be!\evb
\evg


\bvg
\bva[0]Stód under linde, \hld\ under leohtum scylde, &
þęr ðá mihtigan wíf \hld\ hýra męgen berę́ddon &
and hý gyllende \hld\ gáras sęndan; &
ic him óðerne \hld\ eft wille sęndan, &
fléogende fláne \hld\ forane tógéanes. &
Ut, lytel spere, \hld\ gif hit her inne sy!\eva

\bvb[0]Stood under the linden \ken{shield}—under the light shield—where those mighty wives their might arrayed, and they yelling spears did send. I to them another will afterwards send: a flying arrow, back against [them]. Out little spear, if here within it be!\evb
\evg


\bvg
\bva[0]Sęt smið, \hld\ sloh seax &
lytel iserna, \hld\ wund swiðe. &
Ut, lytel spere, \hld\ gif her inne sy!\eva

\bvb[0]Sat the smith, struck the sax; a little iron-thing; a wound severe. Out little spear, if here within it be!\evb
\evg


\bvg
\bva[0]Syx smiðas sętan, \hld\ węlspera worhtan. &
Ut, spere, \hld\ nęs in, spere! &
Gif her inne sy \hld\ isenes dęl, &
hęgtessan geweorc, \hld\ hit sceal gemyltan.\eva

\bvb[0]Six smiths sat, wrought slaughter-spears; out, spear; be not in, spear! If here within be a part of iron, a work of a \inx[C]{hag-tess}—it shall melt.\evb
\evg


\bvg
\bva[0]Gif ðu węre on fell scoten \hld\ oððe węre on flęsc scoten &
oððe węre on blod scoten \hld\ [...] &
oððe węre on lið scoten, \hld\ nęfre ne sy ðin lif atęsed;\eva

\bvb[0]If thou wert shot in the skin, or wert shot in the flesh, or wert shot in the blood, [or wert shot in bone], or wert shot in the limb—never be thy life injured.\evb
\evg


\bvg
\bva[0]gif hit węre esa gescot \hld\ oððe hit węre ylfa gescot &
oððe hit węre hęgtessan gescot, \hld\ nu ic wille ðin helpan: &
þis ðe to bote esa gescotes, \hld\ ðis ðe to bote ylfa gescotes, &
ðis ðe to bote hęgtessan gescotes; \hld\ ic ðin wille helpan.\eva

\bvb[0]If it were the shot of Ease, or it were the shot of elves,\footnoteB{Formulaic; see \inx[F]{Ease and Elves}. That they are held in the same category as the hag-tess—a witch—indicates Christian influence. Among the Germanic peoples the elves and Ease were originally beneficial, something shown by numerous names like Alfred (OE \emph{Ęlfréd} ‘Elf-counsel’), Oswald (OE \emph{Ósweald} ‘Os-power’), Elfwin (Lomb. \emph{Alboin} ‘Elf-friend’), Oshelm (Lomb. \emph{Anselm} ‘Os-helmet’).} or it were the shot of a hag-tess—now I will help thee. This for thee as remedy to the shot of Ease; this for thee as remedy to the shot of elves; this for thee as remedy to the shot of a hag-tess—I will help thee.\evb
\evg


\bvg
\bva[0]Fleo þęr on \hld\ fyrgen-hęfde, &
hal westu, \hld\ helpe ðin drihten, &
nim þonne þęt seax, \hld\ ado on wętan.\eva

\bvb[0]TODO.\evb
\evg

\section{Nine herbs charm}

\bvg
\bva[0]Gemyne ðú mugwyrt \hld\ hwęt þú ámeldodest &
hwęt þu renadest \hld\ ęt Regenmelde?\eva

\bvb Rememberest thou, Mugwort, what thou madest known; what thou arrangedest at Reinmeld?\evb
\evg


\bvg\setlinenum{2}
\bva[0]Una þú hattest \hld\ yldost wyrta &
þú miht wið III \hld\ and wið XXX &
þú miht wiþ attre \hld\ and wið onflyge &
þú miht wiþ þám láþan \hld\ ðe geond lond fęrð\eva

\bvb thou availest against three and against thirty; thou availest against the venom and against the onflier; thou availest against the loathsome one that goes through the lands.\evb
\evg


\bvg\setlinenum{6}
\bva[0]+ Ond þú wegbráde \hld\ wyrta módor &
éast[a]n op[e]ne \hld\ inn[a]n mihtigu &
ofer ðy cręte curran \hld\ ofer ðy cwéne réodan &
\ind ofer ðy brýde brýodedon &
\ind ofer ðy fearras fnęrdon.\eva

\bvb And thou, Waybroad, mother of worts, open from the east, mighty from within. Over thee TODO.\evb
\evg


\bvg\setlinenum{6}
\bva[0]Eallum þu þon wiðstóde \hld\ and wiðstunedest &
swá ðú wiðstonde attre \hld\ and onflyge &
and þǽm láðan \hld\ þe geond lond fereð.\eva

\bvb Them all withstoodest thou then, and stoppedst; so may thou withstand the venom and the onflier, and the loathsome one that goes through the lands.\evb
\evg


\bvg\setlinenum{6}
\bva[0]Stune hętte þéos wyrt, \hld\ héo on stáne geweox &
stond héo wið attre, \hld\ stunað héo węrce &
Stiðe héo hatte, \hld\ wiðstunað héo attre &
wreceð héo wráðan, \hld\ weorpeð út attor\eva

\bvb Ston is this wort called; she grew on stone; she withstands venom, she stops aches. Stithe is she called; she stops venom; she drives away the wroth one; she casts out the venom.\evb
\evg


\bvg\setlinenum{6}
\bva[0]+ Þis is séo wyrt \hld\ séo wiþ wyrm gefeaht &
þéos męg wið attre, \hld\ héo męg wið onflyge &
héo męg wið ðám láþan \hld\ ðe geond lond fereþ\eva

\bvb This is the wort which fought against the worm; this one avails against the venom; she avails against the onflier; she avails against the loathsome one that goes through the lands.\evb
\evg


\bvg\setlinenum{6}
\bva[0]Fleoh þú nú attorláðe, \hld\ séo lǽsse ðá máran &
séo máre þá lǽssan, \hld\ oððęt him beigra bót sý\eva

\bvb TODO\evb
\evg


\bvg\setlinenum{6}
\bva[0]Gemyne þú, męgðe,\hld\ hwęt þú ámeldodest &
hwęt ðú geęndadest \hld\ ęt Alorforda &
þęt nǽfre for gefloge \hld\ feorh ne gesealde &
syþðan him mon męgðan \hld\ tú mete gegyrede\eva

\bvb TODO\evb
\evg


\bvg\setlinenum{6}
\bva[0]Þis is séo wyrt \hld\ ðe wergulu hatte &
ðás onsęnde seolh \hld\ ofer sǽs hrygc &
ondan attres \hld\ óþres tó bóte\eva

\bvb TODO\evb
\evg


\bvg\setlinenum{6}
\bva[0]Ðás VIIII magon \hld\ wið nygon attrum.\eva

\bvb TODO\evb
\evg


\bvg\setlinenum{6}
\bva[0]+ Wyrm cóm snícan, \hld\ toslát hé man &
ðá genam Wóden \hld\ VIIII wuldortánas &
slóh ðá þá nǽddran \hld\ þęt héo on VIIII tófléah &
Þǽr geęndade ęppel \hld\ and attor &
þęt héo nǽfre ne wolde \hld\ on hús búgan\eva

\bvb A \inx[C]{Worm} came crawling; he tore apart a man. Then took Weden nine glory-twigs; slew then that adder, that it TODO into nine [parts]. There ended apple and venom, that he would never come into a house.\evb
\evg


\bvg\setlinenum{6}
\bva[0]+ Fille and finule, \hld\ felamihtigu twá &
þá wyrte gesceop \hld\ wítig drihten &
hálig on heofonum, \hld\ þá hé hongode &
sette and sęnde \hld\ on VII worulde &
earmum and éadigum \hld\ eallum tó bóte\eva

\bvb Fill and Fennel, many-mighty two; those worts shaped the wise lord, holy on heaven, when he hung. He set and sent them onto seven worlds; to the wretched and the wealthy, to all for healing.\evb
\evg


\bvg\setlinenum{6}
\bva[0]Stond héo wið węrce, \hld\ stunað héo wið attre &
séo męg \edtext{wið III \hld\ \emph{and} wið XXX}{\lemma{wið III and wið XXX ‘against three and against thirty’}\Bfootnote{Formulaic; an uncountable amount; “snakes” are probably understood. This oral formula appears in many folk ballads, viz. (Child) 4EFG, 18B, 20C, 30, 53BCDEIKM, 63EFH, 73I, 97AC, 100AG, 110BGH, 156G, 185A, 187A, 187C, 190A, 192A, 193B, 203C, 211A, 217GHLN, 244A, 268A, 269C, 281ABC. Things described include horses, heads of cattle, warriors, days, years, winters.}} &
wið [féondes] hond \hld\ and wið fǽrbregde &
wið malscrunge \hld\ manra wihta\eva

\bvb against three and against thirty\evb
\evg


\bvg\setlinenum{6}
\bva[0]+ Nu magon þás VIIII wyrta \hld\ wið nygon wuldorgeflogenum &
wið VIIII attrum \hld\ and wið nygon onflygnum &
wið ðý réadan attre, \hld\ wið ðý runlan attre &
wið ðý hwitan attre, \hld\ wið ðý [hęwe]nan attre &
wið ðý geolwan attre, \hld\ wið ðý grénan attre &
wið ðý wonnan attre, \hld\ wið ðý wedenan attre &
wið ðý brúnan attre, \hld\ wið ðý basewan attre &
wið wyrmgeblęd, \hld\ wið wętergeblęd &
wið þorngeblęd, \hld\ wið þystelgeblęd &
wið ýsgeblęd, \hld\ wið attorgeblęd\eva

\bvb Now these nine worts avail against glory-onfliers: against nine venoms and against nine onfliers; against the red venom; against the TODO venom; against the white venom; against the TODO venom; against the yellow venom; against the green venom; against the TODO venom; against the TODo venom; against the brown venom; against the TODO venom; against worm-TODO; against water-TODO; against thorn-TODO; against thistle-TODO; against ice-TODO; against venom-TODO.\evb
\evg


\bvg\setlinenum{6}
\bva[0]Gif ęnig attor cume \hld\ éastan fleógan &
oððe ǽnig norðan cume &
oððe ǽnig westan \hld\ ofer werðeóde\eva

\bvb If any venom come from the east, flying; or any come from the north; or any from the west, over man-kind.\evb
\evg


\bvg\setlinenum{6}
\bva[0]+ Críst stód ofer ádle \hld\ ǽngan cundes &
Ic ána wát \hld\ ea rinnende &
þǽr þá nygon nǽdran \hld\ néan behealdað\eva

\bvb TODO\evb
\evg


\bvg\setlinenum{6}
\bva[0]Motan ealle wéoda \hld\ nu wyrtum áspringan &
sǽs tóslúpan, \hld\ eal sealt węter &
ðonne ic þis attor \hld\ of ðé gebláwe\eva

\bvb TODO\evb
\evg


PROSE SECTION.
Mucgwyrt, wegbrade þe eastan open sy, lombescyrse, attorlaðan, mageðan, netelan, wudusuręppel, fille \& finul, ealde sapan. Gewyrc ða wyrta to duste, męngc wiþ þa sapan and wiþ þęs ępples gor.

wyrc slypan of wętere and of axsan, genim finol, wyl on þęre slyppan and beþe mid ęggemongc, þonne he þa sealfe on do, ge ęr ge ęfter.


* Sing þęt galdor on ęcre þara wyrta, :III: ęr he hy wyrce and on þone ęppel ealswa; ond singe þon men in þone muð and in þa earan buta and on ða wunde þęt ilce gealdor, ęr he þa sealfe on do :.



\chapter{Old Norse spells}

\section{Ribe rune charm}

\bvg
\bva[]\alst{Jo}rð bið ak varðę \hld\ ok \alst{u}phimęn &
\alst{s}ól ok \alst{s}antę María \hld\ ok \alst{s}alfęn Guð dróttęn &
þęt han \alst{l}ę́ mik \alst{l}ę́knęshand \hld\ ok \alst{l}yftungę &
at lyfę \alst{b}ifjandę \hld\ þęr \alst{b}ótę þarf. &
\ind Ór \alst{b}ak ok ór \alst{b}ryst
\ind ór \alst{l}íkę ok ór \alst{l}im &
\ind ór \alst{ǿ}fęn ok ór \alst{ǿ}ręn &
\ind ór \alst{a}llę þé þęr \alst{i}llt kann í \alst{a}tkumę. &
Svart hétęr \alst{st}énn \hld\ han \alst{st}ę́r í hafę útę, &
\ind þęr liggęr á þé \alst{n}í\emph{u} \alst{n}auðęr; &
\ind þęr skulę hvęrki \alst{s}ǿtęn \alst{s}ofę; &
\ind ęð \alst{v}armęn \alst{v}akę; &
førr ęn þú þęssa bót biðęr, þęr ak orð atkvę́ðę ronti.\eva

\bvb I ask earth to ward, and up-heaven, sun and saint Mary—and lord God himself, that he lend me a healing-hand and curing tongue, to cure the trembling one who needs remedy. Out of back and out of breast; out of body and out of limb; out of eyes and out of ears; out of everything where evil which might come in! Swart is called a stone—he stands out in the ocean—there lie on it nine needs; they will not [let thee] sleep sweetly nor wake warmly—until thou prayest this remedy, where I tried the words of the charms.\evb
\evg

\section{Charms from Bryggen}

These charms are found inscribed on medieval pieces of wood found at Bryggen in the city of Bergen, Norway.

\sectionline

A stick with four sides, dated to c. 1335. It is clearly a love-charm and—as seen by the feminine dative adjective \emph{sjalfri} ‘self’ on side C—addressed to a woman. The language is very close to that of \Skirnismal\ 36, wherein Shirner threatens to curse the ettin-woman Gird with \emph{ęrgi} ‘degeneracy’ and \emph{ǿði} ‘madness’ and \emph{óþoli} ‘impatience’ unless she sleep with his master, Free. A crucial difference is of course that this charm is not an Eddic narrative poem; it must have been expected to work. Both of these share a root with the curse-formula seen on the two C7th runic inscriptions from Stentoften and Björketorp (see TODO), wherein the destroyer of the respective monuments will be \emph{hermalausaʀ argjú} ‘restless with degeneracy’, i.e. ‘incessantly randy’. As it would be absurd to think that the poet of \Skirnismal\ should have learned this type of magic from one of the rune-stones, and then passed this onto the carver of the present inscription, we must rather be dealing with a common form of curse magic, wherein the victim is cursed with incessant randiness leading to sexual perversion.

\bvg
\bva[A]\mssnote{\textbf{B257}}Ríst ek \alst{b}ótrúnar \hld\ ríst ek \alst{b}jargrúnar &
\ind \alst{ei}nfalt við \alst{ǫ}lfum &
\ind \alst{t}vífalt við \alst{t}rollum &
\ind \alst{þ}rífalt við \alst{þ}u\emph{rsum}\eva

\bvb I carve healing-runes; I carve saving-runes; onefold against elves; twofold against trolls; threefold against thurses.\evb
\evg


\bvg
\bva[B]Við inni \alst{sk}ǿðu \hld\ \alst{sk}ag-valkyrju &
svá’t \alst{ei} megi \hld\ þó-at \alst{ę́} vili &
\alst{l}ę́vís kona \hld\ \alst{l}ífi þínu g\emph{randa}.\eva

\bvb Against the scatheful shag-walkirrie, so that she may not—although she ever wishes to, that guile-wise woman—harm thy life.\evb
\evg


\bvg
\bva[C]Ek \alst{s}endir þér \hld\ ek \alst{s}é á þér &
\alst{y}lgjar \alst{e}rgi \hld\ ok \alst{ó}þola; &
á þér hríni \alst{ó}þoli \hld\ ok \alst{jǫ}tuns móð\emph{r}; &
\alst{s}it-tu aldri, \hld\ \alst{s}op-tu aldri.\eva

\bvb I send to thee—I see on thee—a she-wolf’s degeneracy and impatience; on thee stick impatience, and an ettin’s wrath! Sit thou never, sleep thou never!\evb
\evg


\bvg
\bva[D]Ant mér sem sjalfri þér. Beirist rubus rabus et arantabus laus abus rosa gava\eva

\bvb Love me like thy self.\evb
\evg

\sectionline

\bvg
\bva[]\mssnote{\textbf{B380}}\alst{H}ęill sé þú \hld\ ok í \alst{h}ugum góðum; &
\ind \alst{Þ}órr þik \alst{þ}iggi, &
\ind \edtrans{\alst{Ó}ðinn þik \alst{ęi}gi}{‘may Weden own thee’}{\Bfootnote{See note to \Voluspa\ 23.}}.\eva

\bvb Be thou hale, and in good spirits;\footnoteB{A formula also attested in \Hymiskvida\ 41; see there for parallels.} may Thunder receive thee, may Weden own thee.\evb
\evg


\section{Runic plates}
