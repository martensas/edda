\bookStart{The Speeches of Shirner}[Skírnismǫ́l]

\begin{flushright}%
Dating \parencite{Sapp2022}: C10th (0.897)

Meter: \Ljodahattr, \Galdralag\ (TODO)%
\end{flushright}

% Introduction

The whole poem is attested in both \Regius\ and \AM. The name \emph{Skírnismǫ́l} ‘\textbf{Speeches of Shirner}’ comes from \AM; \Regius\ has the header \emph{Fǫr Skírnis} ‘Shirner’s journey’.

The same myth is told in \Gylfaginning\ 37. A single verse of the present poem is quoted there, namely the last one (42), with some minor differences in wording that would seem to stem from oral tradition (see Note there). One could speculate that the author of \Gylfaginning\ did not have a copy of this poem in front of him, but rather knew of the story through an oral tradition which included only the last verse. This seems unlikely for the chief reason that this paraphrase does not add a single detail not already in the present poem, but on the other hand condenses and abbreviates that which is already written here. Thus Shirner’s journey and curse (roughly vv. 10–38 here) is simply summarized in the following manner: “Then Shirner journeyed and requested the woman [i.e. Gerd] for him [i.e. Free], and received her promise, that nine nights later she would come to the place which is called Barrey, and have a wedding with Free.”

On the other hand, the paragraph in \Gylfaginning\ 37 that corresponds to what is here P1 is much more detailed. It goes: “Gymer was a man called, and his woman Earbode; she was of the lineage of mountain-risers. Their daughter is Gerd, who is fairest of all women. It was one day as Free had gone to Lithshelf and looked about all the Homes. And when he looked to the north he saw on a farm a large and fair house, and into that house walked a woman. And when she brought out her hands and closed the doors before her, then light shone off her hands—both into the air and onto the waters—and all the homes were brightened by her. That beauty, when he had set himself in that holy seat, harmed him so that he walked away filled with pain. And when he came home he spoke nothing. Nothing slept he, nothing drank he. Nobody dared to ask him to speak. Then Nearth had Shirner, Free’s shoe-swain, called unto him, and asked him to go to Free and ask him to speak, [...]”

It seems to me that this circumstance, where the part corresponding to the poem is a short paraphrase, but the part corresponding to the prose passage is much more detailed, can only have arisen if the former already had a fixed form, whereas the latter was freer and could vary with each retelling. For this, see further TODO.

\sectionline

\bpg
\bpa\mssnote{\Regius~11r/10, \AM~2r/11}Freyr, sonr Njarðar, hafði einn dag setsk í Hliðskjálf ok sá um heima alla; hann sá í Jǫtunheima ok sá þar mey fagra, þá er hon gekk frá skála fǫður síns til skemmu; þar af fekk hann hugsóttir miklar. Skírnir hét skósveinn Freys. Njǫrðr bað hann kveðja Frey máls. Þá mę́lti Skaði:\epa

\bpb \inx[P]{Free}, son of \inx[P]{Nearth}, had one day sat himself down in \inx[L]{Lithshelf} and looked about all the \inx[C]{Homes}. He looked into the \inx[L]{Ettinhomes} and saw there a fair maiden as she walked from her father’s hall to her bower; thereof he got great heart-aches. \inx[P]{Shirner} was called the shoe-swain of Free. Nearth asked him to speak with Free. Then \inx[P]{Shede} spoke:\epb
\epg


\bvg
\bva\mssnote{\Regius~11r/14, \AM~2r/15}„\edtext{Rís-tu nú Skírnir \hld\ ok gakk at bęiða}{\lemma{rís \dots\ bęiða ‘rise \dots\ speak’}\Bfootnote{Alliteration is missing here. A simple solution would be to replace \emph{gakk} ‘go’ with a synonym like \emph{rinn} ‘run’ or \emph{ráð} ‘resolve’, but this breaks the mirroring in 2/2.}} &
\ind okkarn mála mǫg, &
ok þess at fregna \hld\ hvęim hinn fróði séi &
\ind ofvręiði \edtext{afi}{\lemma{afi ‘man’}\Bfootnote{While this word usually means ‘father’ or ‘grandfather’, it must here certainly mean ‘man’ without a connotation of old age. See further \CV.}}.“\eva

\bvb “Rise thou now, O Shirner, and go to ask our lad \ken*{= Free} to speak; and to learn at whom the learned man \ken*{= Free} might be cross.”\evb
\evg


\bvg {\small Shirner quoth:}
\bva\mssnote{\Regius~11r/15, \AM~2r/17}„Illra orða \hld\ es mér ón at ykkrum syni, &
\ind ef ek gęng at mę́la við mǫg, &
ok þess at fregna, \hld\ hvęim hinn fróði séi &
\ind ofvręiði afi.“\eva

\bvb “Bad words I expect from your son \ken*{= Free}, if I go with the lad to speak; and to learn at whom the wise man might be cross.”\evb
\evg


\bvg {\small Shirner quoth:}
\bva\mssnote{\Regius~11r/17, \AM~2r/18}„Sęg þat Fręyr, \hld\ folkvaldi goða, &
\ind ok ek vilja vita, &
hví þú ęinn sitr \hld\ ęndlanga sali &
\ind minn dróttinn of daga.“\eva

\bvb “Tell it, O Free, troop-wielder of the gods, I too would want to know: why thou alone stayest in the endlong halls, my lord, during the days?”\evb
\evg


\bvg {\small Free quoth:}
\bva\mssnote{\Regius~11r/19, \AM~2r/20}„Hví of sęgja’k þér, \hld\ sęggr hinn ungi, &
\ind mikinn móðtrega? &
því’t alfrǫðull \hld\ lýsir of alla daga &
\ind ok þęygi at mínum munum.“\eva

\bvb “Why should I tell thee, O young youth, about [my] great mood-grief? For the elf-wheel \ken{sun} shines during all days, and naught to my liking.”\evb
\evg


\bvg {\small Shirner quoth:}
\bva\mssnote{\Regius~11r/20, \AM~2r/21}„Muni þína \hld\ hykk-a svá mikla vesa, &
\ind at þú mér \edtext{sęggr}{\lemma{sęggr ‘man’}\Bfootnote{usually means simply ‘man’, its original meaning was ‘messenger’ and it seems to have some connotation with youth, something also seen in \Volundarkvida\ 23 where it’s used in reference to the young sons of king Nithad. It’s here used to mirror Free’s addressing Shirner as \emph{sęggr hinn ungi} ‘the youth; Shirner points out that the two are of equal age, so Free is as much of a young man as he.}} né sęgir; &
ungir saman \hld\ vǫ́rum í árdaga, &
\ind vęl mę́ttim tvęir trúask.“\eva

\bvb “Thy liking I do not think so large, that thou, O youth, oughtst not to me tell it. Young together were we in days of yore; we two might well trust each other.”\evb
\evg


\bvg {\small Free quoth:}
\bva\mssnote{\Regius~11r/22, \AM~2r/23}„Í Gymis gǫrðum \hld\ ek ganga sá &
\ind mér tíða męy; &
armar lýstu, \hld\ en af þaðan &
\ind allt lopt ok lǫgr.\eva

\bvb “In Gymer’s yards I saw walking a maiden, dear to me. The arms shone, but thereof all the air and sea.\evb
\evg


\bvg
\bva\mssnote{\Regius~11r/24, \AM~2r/24}Mę́r es mér tíðari \hld\ an manna hvęim &
\ind ungum í árdaga; &
ása ok alfa \hld\ þat vill ęngi maðr, &
\ind at vit sátt séim.“\eva

\bvb “The maiden is dearer to me than to any young man in days of yore. Of the \inx[F]{Ease and Elves} no man\footnoteB{For other examples of gods being called men see TODO.} wants that we two be reconciled.”\evb
\evg


\bvg {\small Shirner quoth:}
\bva\mssnote{\Regius~11r/25, \AM~2r/25}„Mar gef mér þá, \hld\ es mik of myrkvan beri &
\ind vísan vafrloga, &
ok þat sverð, \hld\ es sjalft vegisk &
\ind við jǫtna ę́tt.“\eva

\bvb “Then give me the steed, which might bear me over the dark, wise wavering-flame; and that sword, which by itself might strike against the lineage of the \inx[G]{Ettins}.”\evb
\evg


\bvg {\small Free quoth:}
\bva\mssnote{\Regius~11r/27, \AM~2r/27}„Mar þér þann gef’k, \hld\ es þik of myrkvan \edtext{berr &
\ind vísan vafrloga,
ok þat sverð, \hld\ es sjalft mun vegask, &
\ind ef sá ’s horskr es hęfr.“}{\lemma{berr ‘bears’; mun vegask, ef sá ’s horskr es hęfr ‘will strike, if he is wise who owns it’}\Bfootnote{Responding, Free replaces the subjunctive verb forms (\emph{beri} ‘might bear’ \emph{vegisk} ‘might strike’), giving a sense of certainity and authority. The steed and sword are faultless, and if Shirner fails on the mission, it would be only due to his own fault (“if he is sharp who owns it.”).}}\eva
%TODO? Change the line numbering from 1–4 to 1, 3–4.

\bvb “That steed I give thee, which bears thee over the dark, wise wavering-flame; and that sword, which by itself will strike, if he is sharp who owns it.”\evb
\evg


\bvg {\small Shirner spoke with the horse:}
\bva\mssnote{\Regius~11r/29, \AM~2r/28}„Myrkt es úti, \hld\ mál kveð’k okkr fara &
\ind úrig fjǫll yfir &
\ind þursa þjóð yfir; &
báðir vit komumk \hld\ eða okkr báða tękr
\ind sá hinn \edtext{ámátki jǫtunn}{\lemma{ámátki jǫtunn ‘terrifying ettin’}\Bfootnote{Formulaic. \emph{ámáttigr} ‘terrifying’ seems to have a supernatural connotation, and only occurs in four other places in the Poetic Edda: in \Voluspa\ 8, \Grimnismal\ 11 and \HelgakvidaHjorvardssonar\ 17 it is paired with \emph{jǫtunn} ‘ettin’, while in \HelgakvidaHjorvardssonar\ 14 it describes a man with clearly supernatural attributes.}}.“\eva

\bvb “’Tis dark outside; I call it time for us two to journey: over the drizzling mountains, over the people of the \inx[G]{Thurses}. Both two we come, or us both that terrifying ettin takes.\footnoteB{Shirner declares his intention not to abandon his horse.}”\evb
\evg


\bpg
\bpa\mssnote{\Regius~11r/31, \AM~2v/1}Skírnir reið i Jǫtunheima til Gymis garða; þar váru hundar ólmir ok bundnir fyrir skíðgarðs hliði þess, er um sal Gerðar var. Hann reið at þar, er féhirðir sat á haugi, ok kvaddi hann: \epa

\bpb Shirner rode into the Ettinhomes, to Gymer’s yards. There were hounds, fierce and bound in front of the slope of that wooden fence which surrounded Gird’s\footnote{It is first now that we are informed of the maiden’s name.} hall. He rode to where a shepherd sat on a mound, and greeted him:\epb
\epg


\bvg
\bva\mssnote{\Regius~11v/2, \AM~2v/4}„Sęg þat hirðir, \hld\ es á haugi sitr &
\ind ok varðar alla vega: &
hvé ek at andspilli \hld\ komumk hins unga mans &
\ind fyr gręyjum Gymis.“\eva

\bvb “Say it, O herdsman, who sittest on the mound, and guardest all ways: How I to discourse might come with the young maiden, past Gymer’s greyhounds?”\evb
\evg


\bvg {\small [The herdsman quoth:]}
\bva\mssnote{\Regius~11v/4, \AM~2v/5}„Hvárt ert fęigr, \hld\ eða ert \edtext{fram ginginn}{\lemma{fram ginginn ‘passed-on’}\Bfootnote{i.e. ‘dead’.}} &
\ind [...]; &
andspillis vanr \hld\ þú skalt ę́ vesa &
\ind góðrar męyjar Gymis.“\eva

\bvb “Whether thou art fey, or passed-on; [...]? . Lacking discourse shalt thou ever be, with Gymer’s good maiden.”\evb
\evg


\bvg {\small [Shirner quoth:]}
\bva\mssnote{\Regius~11v/6, \AM~2v/7}„\edtext{Kostir}{\lemma{kostir ‘choices’}\Bfootnote{i.e. ‘alternative choices, other ways’.}} ’ru bętri \hld\ \edtext{an}{\Afootnote{thus \AM; \emph{hęldr an at} ‘rather than to [be]’ \Regius}} kløkkva sé &
\ind hvęim es fúss es fara, &
ęinu dǿgri \hld\ mér vas aldr of skapaðr &
\ind ok alt líf of lagit.“\eva

\bvb “Choices are better than sobbing, for whomever is eager to journey. On a single day was my age shaped, and all my life laid [in place].\footnoteB{The Germanic fatalistic worldview, wherein one’s course of life was predetermined at birth, are here clearly seen. Presumably after uttering these words Shirner rides through the fire surrounding the fortress. — The causative \emph{lęgja} ‘to lay (down, in place)’ is closely connected to fate; the expression is formulaic. Cf. \Lokasenna\ 48: \emph{í árdaga vas þér hit ljóta líf of lagit} ‘in days of yore was thy ugly life laid [in place]’ and \Voluspa\ 19: \emph{þę́r lǫg lǫgðu} ‘they [= the Norns] laid laws [in place]’.}”\evb
\evg


\bvg {\small [Gird quoth:]}
\bva\mssnote{\Regius~11v/7, \AM~2v/8}„Hvat ’s þat hlym hlymja \hld\ es hlymja hęyri’k nú til &
\ind ossum rǫnnum í? &
jǫrð bifask, \hld\ en allir fyr &
\ind skjalfa garðar Gymis.“\eva

\bvb “What is that din of dins, which I of dins now hear in our houses? The earth trembles, and before [me] all the yards of Gymer quake.”\evb
\evg


\bvg {\small A servant-woman quoth:}
\bva\mssnote{\Regius~11v/9, \AM~2v/10}„Maðr ’s hér úti, \hld\ stiginn af mars baki, &
\ind jó lę́tr til jarðar taka.“\eva

\bvb “A man is here outside, stepped down off a horse’s back; he lets take his steed to the ground.\footnoteB{According to \textcite{FinnurEdda} a still-known Icelandic expression; Shirner lets his horse graze.} (TODO: translation)”\evb
\evg


\bvg {\small [Gird quoth:]}
\bva\mssnote{\Regius~11v/10, \AM~2v/11}„Inn bið þú hann ganga \hld\ í okkarn sal &
\ind ok drekka hinn mę́ra mjǫð, &
þó ek hitt óumk, \hld\ at hér úti séi &
\ind minn bróðurbani.“\eva

\bvb “Bid thou him to go in into our hall, and to drink the renowned mead; though I fear that here outside might be my brother’s bane-man.”\evb
\evg


\bvg {\small [Gird quoth:]}
\bva\mssnote{\Regius~11v/12, \AM~2v/13}„Hvat ’s þat alfa \hld\ né ása sona, &
\ind né víssa vana; &
hví ęinn of komt \hld\ ęikinn fúr yfir &
\ind ór salkynni at séa?“\eva

\bvb “What sort is that, not of Elves, nor of sons of the Ease, nor of the wise Wanes? Why camest thou alone over the raging fire, to see the state of our hall?”\evb
\evg


\bvg {\small [Shirner quoth:]}
\bva\mssnote{\Regius~11v/14}„Em’k-at alfa \hld\ né ása sona &
\ind né víssa vana, &
þó ęinn of kom’k \hld\ ęikinn fúr yfir &
\ind yður salkynni at séa.\eva

\bvb “I am not of the Elves, nor of sons of the Ease, nor of the wise Wanes—yet I came alone over the raging fire, to see the state of your hall.\evb
\evg


\bvg
\bva\mssnote{\Regius~11v/15, \AM~2v/14}Ępli ęllifu \hld\ hér hef’k algollin, &
\ind þau mun’k þér Gęrðr gefa, &
frið at kaupa, \hld\ at þú þér Fręy kveðir &
\ind ólęiðastan at lifa.“\eva

\bvb Elven apples have I here, all-golden; those I will to thee, O Gird, give to purchase [thy] love, that thou callest Free for thee most unloathsome \ken{loveliest} in life.\footnoteB{\emph{at lifa} seems to mean ‘in life’ here rather than the typical infinitive construction ‘to live’. This is an archaism from its origin as a verbal noun meaning ‘living’.}”\evb
\evg


\bvg {\small [Gird quoth:]}
\bva\mssnote{\Regius~11v/17, \AM~2v/15}„Ępli ęllifu \hld\ ek þigg aldrigi &
\ind at mans-kis munum, &
né vit Fręyr, \hld\ meðan okkart fjǫr lifir, &
\ind byggum bę́ði saman.“\eva

\bvb “Eleven apples [will] I never accept, to any man’s liking; nor [will] I and Free—while our lives remain\footnoteB{lit. ‘while our life-force lives’}—dwell both together.”\evb
\evg


\bvg {\small [Shirner quoth:]}
\bva\mssnote{\Regius~11v/19, \AM~2v/17 (ll. 1–2)}„Baug þér þá gef’k, \hld\ þann’s bręndr of vas &
\ind með ungum Óðins syni, &
\edtext{átta ’ru jafnhǫfgir, \hld\ es af drjúpa &
\ind hina níundu hvęrja nǫ́tt.“}{\lemma{Baug ... nǫ́tt ‘The bigh ... night.’}\Bfootnote{In \AM\ these lines and 22:1–2 are missing. Instead 1–2 here and 22:3–4 are combined into one.}}\eva

\bvb “The \inx[C]{bigh} I then give thee, that one which was burned with Weden’s young son \ken*{= Balder}. Eight are even-heavy, which from it drip, every ninth night.\footnoteB{The bigh, while not named, is clearly Dreepner as known from \Gylfaginning\ 49, describing Balder’s funeral: “Weden laid on the pyre that gold ring which is called Dreepner. Its nature was such that every ninth night, eight even-heavy golden rings dripped from it.” When \inx[P]{Harmod} later comes to \inx[L]{Hell} to try to bring Balder back, Balder tells him to bring the ring back to Weden, as a token of memory.}”\evb
\evg


\bvg {\small [Gird quoth:]}
\bva\mssnote{\Regius~11v/21, \AM~2v/18 (ll. 3–4)}„Baug þikk-a’k, \hld\ þótt bręndr séi, &
\ind með ungum Óðins syni; &
es-a mér golls vant \hld\ í gǫrðum Gymis &
\ind at dęila fé fǫður.“\eva

\bvb “The bigh I accept not, though it may have been burned with Weden’s young son \ken*{= Balder}; I have no want of gold in Gymer’s yards, in sharing the \inx[C]{fee} of my father.”\evb
\evg


\bvg {\small [Shirner quoth:]}
\bva\mssnote{\Regius~11v/23, \AM~2v/19}„Sér þú mę́ki, mę́r, \hld\ mjóvan, málfáan, &
\ind es hęf’k í hendi hér? &
hǫfuð hǫggva \hld\ mun’k þér halsi af, &
\ind nema mér sę́tt sęgir.“\eva

\bvb “Seest thou, O maiden, this sword—slender, pictured-painted\footnoteB{The sword is inlaid with metal forming a pattern. For examples see TODO.}—which I have here in my hand? Off thy neck will I hew thy head, unless thou agree with me.\footnoteB{lit. ‘unless thou to me sayest an agreement/settlement.’}”\evb
\evg


\bvg {\small [Gird quoth:]}
\bva\mssnote{\Regius~11v/25, \AM~2v/20}„Ánauð þola \hld\ vil’k aldrigi &
\ind at \edtrans{manskis}{‘any man’s (lit. ‘no man’s)’}{\Afootnote{\emph{mannz ænskis} \AM}} munum, &
þó hins get’k, \hld\ ef it Gymir finnizk &
vígs ótrauðir \hld\ at ykkr vega tíði.“\eva

\bvb “Suffer coercion will I never, to any man’s liking; though I suppose, if thou and Gymer meet—men unreluctant of conflict—that ye two will wish to fight.\footnoteB{Gird says that she will let herself be forced to marry Free, even if this means that Shirner and Gymer will fight over her.}”\evb
\evg


\bvg {\small [Shirner quoth:]}
\bva\mssnote{\Regius~11v/27, \AM~2v/22}„Sér þú mę́ki, mę́r, \hld\ mjóvan, málfáan, &
\ind es hęf’k í hendi hér? &
fyr þessum ęggjum \hld\ hnígr sá hinn aldni jǫtunn, &
\ind verðr þinn fęigr faðir.\eva

\bvb “Seest thou, O maiden, this sword—slender, pictured-painted—which I have here in my hand? Before these edges the aged ettin \ken*{= Gymer} sinks down; \inx[C]{fey} becomes thy father.\evb
\evg


\bvg
\bva\mssnote{\Regius~11v/28, \AM~2v/24}\edtext{Tamsvęndi}{\lemma{tamsvęndi ‘taming-wand’}\Bfootnote{Has been interpreted as a sword, TODO.}} þik drep’k, \hld\ ęn þik tęmja mun’k, &
\ind mę́r, at mínum munum, &
þar skalt ganga \hld\ es þik gumna synir &
\ind síðan ę́va séi.\eva

\bvb With the taming-wand I strike thee, but thee I will tame, O maiden, to my liking. There shalt thou go, where thee the sons of men never since may see.\evb
\evg


\bvg
\bva\mssnote{\Regius~11v/30, \AM~2v/26}\edtext{Ara þúfu á \hld\ skalt ár sitja}{\lemma{ara þúfu á \hld\ skalt ár sitja ‘on an eagle’s hill shalt thou sit in early morning’}\Afootnote{\emph{ár skalt sitja \hld\ ara þúfu á} ‘early shalt thou sit on an eagle’s hill’ \AM}}, &
\ind \edtext{horfa hęimi ór; &
\ind snugga hęljar til}{\lemma{horfa hęimi ór; snugga hęljar til ‘turn out of the world; hanker after Hell’}\Afootnote{horfa ok snugga hęljar til ‘turn and hanker to hell’ \AM}}; &
matr sé þér męir lęiðr \hld\ an manna hvęim &
\ind hinn fráni ormr með firum.\eva

\bvb On an eagle’s hill shalt thou sit in early morning; turn out of the world; hanker after \inx[L]{Hell}. Food will be thee more loathsome, than to any man the gleaming serpent \ken*{the Middenyardsworm} among firs \ken{men}.\footnoteB{Presumably her food will be as disgusting as the Middenyardsworm (for its disgusting nature see Note to \Hymiskvida\ 22). The threat seems to be that Gird will be forced to sit alone on an eagle’s nest, deprived of food and longing for death.}\evb
\evg


\bvg
\bva\mssnote{\Regius~11v/32}At undrsjónum verðir \hld\ es út of kømr, &
\ind á þik Hrímnir hari &
\ind á þik hotvetna stari, &
víðkunnari verðir \hld\ an vǫrðr með goðum, &
\ind gapi þú grindum frá.\eva

\bvb A wondrous sight [wilt] thou become, when out thou comest; at thee [will] Rimner ogle; at thee [will] anyone stare. More widely known [wilt] thou become than the ward among the Gods \ken*{= Homedall}; thou [wilt] gape from the gates.\evb
\evg


\bvg
\bva\mssnote{\Regius~12r/2}Tópi ok ópi, \hld\ tjǫsull ok óþoli, &
\ind vaxi þér tǫ́r með trega; &
sęzk þú niðr \hld\ en mun’k sęgja þér &
\ind sváran súsbreka, &
\ind ok tvinnan trega.\eva

\bvb Toop and oop, tessle and impatience; may thy tear grow with grief! Sit thyself down, and I will tell thee a severe roaring-breaker, and a twined grief.\evb
\evg


\bvg
\bva\mssnote{\Regius~12r/3}Tramar gnęypa \hld\ þik skulu gęrstan dag &
\ind jǫtna gǫrðum í, &
til hrímþursa hallar \hld\ þú skalt hvęrjan dag &
\ind kranga kostalaus; &
\ind kranga kostavǫn; &
grát at gamni \hld\ skalt í gǫgn hafa &
\ind ok lęiða með tǫ́rum trega.\eva

\bvb Thee shall fiends torment at the dismal day, in the yards of the Ettins. To the halls of the Rime-thurses shalt thou every day creep choiceless; creep choice-lacking. Weeping for joy shalt thou have in exchange, and nurse grief with tears.\evb
\evg


\bvg
\bva\mssnote{\Regius~12r/7}Með þursi þríhǫfðuðum \hld\ þú skalt ę́ nara &
\ind eða verlaus vesa, &
\ind þitt geð grípi; &
\ind þik morn morni &
ves þú sem þistill, \hld\ sá’s þrunginn vas &
\ind í ofanverða ǫ́nn.\eva

\bvb With a three-headed thurse shalt thou ever live, or be husband-less. Thy senses grasp; murrain mourn thee; be thou like the thistle that was pressed in the uppermost working season!\evb
\evg


\bvg
\bva\mssnote{\Regius~12r/9}Til holts ek gekk \hld\ ok til hrás viðar &
\ind gambantęin at geta &
\ind gambantęin ek gat.\eva

\bvb To the wood I went, and to the young tree, the \inx[C]{gombentoe} for to get; the gombentoe I got.\footnoteB{Presumably the “taming-wand” in 26.}\evb
\evg


\bvg
\bva\mssnote{\Regius~12r/10}Ręiðr ’s þér Óðinn, \hld\ ręiðr ’s þér Ásabragr, &
\ind þik skal Fręyr fíask, &
hin firinilla mę́r, \hld\ en fingit hęfr &
\ind gambanręiði goða.\eva

\bvb Wroth with thee is Weden; wroth with thee is Bray of the Ease \name*{= Thunder?}; thee shall Free come to hate, O horrible maiden, if thou hast earned the gomben-wrath of the gods.\evb
\evg


\bvg
\bva\mssnote{\Regius~12r/12}Hęyri jǫtnar, \hld\ hęyri hrímþursar, &
synir Suttunga, \hld\ sjalfir ásliðar, &
hvé fyrir býð’k, \hld\ hvé fyrir banna’k &
\ind manna glaum mani, &
\ind manna nyt mani.\eva

\bvb Hear may Ettins, hear may Rime-thurses, sons of Sutting \ken{ettins}, the os-retinues \ken*{= Ease} themselves: how I forbid, how I forban the company of men from the maiden; the use of men from the maiden.\evb
\evg


\bvg
\bva\mssnote{\Regius~12r/14}Hrímgrímnir hęitir þurs, \hld\ es þik hafa skal &
\ind fyr nágrindr neðan, &
þar þér vílmęgir \hld\ á viðarrótum &
\ind gęitahland gefi; &
ǿðri drykkju \hld\ fá þú aldrigi, &
\ind mę́r, af þínum munum, &
\ind mę́r, at mínum munum.\eva

\bvb Rimegrimner is called the thurse, who shall have thee, down beneath Nawgrind—where the lads of toil \ken{thralls}, on the roots of the tree, goat-piss [will] give thee. A better drink mayst thou never get, O maiden, of thy liking; O maiden, to my liking!\evb
\evg


\bvg
\bva\mssnote{\Regius~12r/16}\edtext{Þurs}{\lemma{þurs ‘thurse’}\Bfootnote{Thurse is the name of the \textbf{þ}-rune (ᚦ); it is carved as part of the curse.}} ríst’k þér \hld\ ok \edtrans{þría stafi}{three staves}{\Bfootnote{Three runic letters, possibly representing each of the three following words (\emph{ęrgi} ‘degeneracy’ etc.). This expression also appears on the C7th Gummarp stone: \textbf{h\textsc{a}þuwol\textsc{a}fʀ s\textsc{a}te st\textsc{a}b\textsc{a} þri\textsc{a} fff} ‘Hathwolf placed three staves: fff’, where the \textbf{f}-rune (ᚠ) is standing for its name, \inx[C]{fee} (i.e. wealth, cattle).}}, &
\ind \edtrans{ęrgi ok ǿði ok óþola}{degeneracy and madness and impatience}{\Bfootnote{Both \emph{ęrgi} ‘degeneracy’ and \emph{óþoli} ‘impatience’ (here probably with a sexual connotation), are found in the love magic charm on the rune stick B257 from Bryggen, here edited under Charms and Spells. \emph{ęrgi} is also found in the curse-formula on the C7th Proto-Norse runestones from Stentoften and Björketorp. See further introduction to B257.}}, &
svá ek þat af ríst \hld\ sem ek þat á ręist, &
\ind ef gęrvask þarfar þess.“\eva

\bvb \inx[G]{Thurses}[Thurse] I carve for thee, and three staves: \inx[C]{degeneracy} and madness and impatience. So I carve it off as I carved it on, if need arises of that.\footnoteB{Shirner has carved the curse (which will realize all the threats from 26–35), but tells Gird that he will scrape it off if she will accept his demands. She then responds:}”\evb
\evg


\bvg {\small [Gird quoth:]}
\bva\mssnote{\Regius~12r/19}„Hęill ves þú hęldr, svęinn, \hld\ ok tak við hrímkálki &
\ind fullum forns mjaðar, &
þó hafða’k ę́tlat, \hld\ at mynda’k aldrigi &
\ind unna \edtext{vaningja}{\lemma{vaningja ‘Waning’}\Bfootnote{A rare word, lit. ‘descendant of the \inx[G]{Wanes}’, it only occurs at one other place in the corpus, namely in the \inx[C]{thule} of boar-names. Boars were sacred to Free, TODO.}} vęl.“\eva

\bvb “Be thou rather hale, O swain, and receive the rime-chalice, full of ancient mead\footnoteB{Occurs identically in \Lokasenna\ 52.}—although I had intended that I never would love the Waning \ken*{= Free} well.”\evb
\evg


\bvg {\small [Shirner quoth:]}
\bva\mssnote{\Regius~12r/21}„Ørendi mín \hld\ vil’k ǫll vita, &
\ind áðr ríða’k hęim heðan, &
nę́r á þingi \hld\ munt hinum þroska &
\ind nęnna Njarðar syni.“\eva

\bvb “My errands all I wish to know, before I might ride home hence; when on the \inx[C]{Thing} thou wilt with the vigorous son of Nearth \ken*{= Free} be joined?”\evb
\evg


\bvg {\small [Gird quoth:]}
\bva\mssnote{\Regius~12r/23}„Barri hęitir, \hld\ es vit bę́ði vitum, &
\ind lundr lognfara, &
en ępt nę́tr níu, \hld\ þar mun Njarðar syni &
\ind Gęrðr unna gamans.“\eva

\bvb “Barrey is called—as we both know—a grove of calm rushes, and after nine nights there will to the son of Nearth \ken*{= Free} Gird her pleasure grant.”\evb
\evg


\bpg
\bpa\mssnote{\Regius~12r/24}Þá reið Skírnir heim. Freyr stóð úti ok kvaddi hann ok spurði tíðenda:\epa

\bpb Then Shirner rode home. Free stood outside and greeted him and asked him for the tidings:\epb
\epg


\bvg
\bva\mssnote{\Regius~12r/25}„Sęg mér, Skírnir, \hld\ áðr verpir sǫðli af mar &
\ind ok stígir feti framarr, &
hvat árnaðir \hld\ í Jǫtunhęima &
\ind þíns eða míns munar?“\eva

\bvb “Say me, O Shirner, before thou throwest the saddle off the steed, and takest a step further: what thou earnedst in the \inx[L]{Ettinhomes}, to thy or my liking?”\evb
\evg


\bvg {\small [Shirner quoth:]}
\bva\mssnote{\Regius~12r/27}„Barri hęitir, \hld\ es vit báðir vitum, &
\ind lundr lognfara, &
en ępt nę́tr níu, \hld\ þar mun Njarðar syni &
\ind Gęrðr unna gamans.“\eva

\bvb “Barrey is called—as we both know—a grove of calm rushes, and after nine nights there will to the son of Nearth \ken*{= Free} Gird her pleasure grant.”\evb
\evg


\bvg {\small [Free quoth:]}
\bva\mssnote{\Regius~12r/28, \GylfMS}Lǫng es nǫ́tt, \hld\ \edtrans{langar ’u tvę́r}{long are two}{\Bfootnote{thus \Regius; \emph{lǫng es ǫnnur} ‘long is another’ \GylfMS}}, &
\ind hvé of þręyja’k þríar? &
opt mér mánaðr \hld\ minni þótti &
\ind an sjá hǫlf hýnǫ́tt.\eva

\bvb Long is a night; long are two; how can I yearn for three? Oft a month to me seemed less, than this half wedding-night.\footnoteB{The wedding-night (TODO: it's a hapax so explain the etymology?) is presumably half in that it is not consumated.}\evb
\evg
