\bookStart{The Speeches of Shirner}[Skírnismǫ́l]

% Introduction

The \textbf{Speeches of Shirner}


Fǫr Skírnis

Shirner’s Journey


BPG
BPAFreyr, sonr Njarðar, hafði einn dag setsk í Hliðskjálf ok sá um heima alla; hann sá í Jǫtunheima ok sá þar mey fagra, þá er hon gekk frá skála fǫður síns til skemmu; þar af fekk hann hugsóttir miklar. Skírnir hét skósveinn Freys. Njǫrðr bað hann kveðja Frey máls. Þá mę́lti Skaði:EPA

BPB \inx[P]{Free}, son of \inx[P]{Nearth}, had one day sat himself down in \inx[L]{Lithshelf} and looked about all the \inx[C]{Homes}. He looked into the \inx[L]{Ettinhomes} and saw there a fair maiden as she walked from her father’s hall to her bower; thereof he got great heart-aches. \inx[P]{Shirner} was called the shoe-swain of Free. Nearth asked him to speak with Free. Then \inx[P]{Scathe} spoke: EPB
EPG


\bvg
\bva „Rís-tu nú Skírnir \hld\ ok gakk at bęiða &
\ind okkarn mála mǫg, &
ok þess at fregna \hld\ hvęim hinn fróði séi &
\ind ofręiði \edtext{afi}{\lemma{afi ‘man’}\Bfootnote{While this word usually means ‘father’ or ‘grandfather’, it must here certainly mean ‘man’ without a connotation of old age. See further \CV.}}.“\eva

\bvb “Rise thou now, Shirner, and go to ask our lad \ken*{= Free} to speak; and to learn at whom the learned man \ken*{= Free} might be cross.”\evb
\evg


\bvg {\small Shirner quoth:}
\bva „Illra orða \hld\ es mér ón at ykrum syni, &
\ind ef ek gęng at mę́la við mǫg, &
ok þess at fregna, \hld\ hvęim hinn fróði séi &
\ind ofręiði afi.“\eva

\bvb “Bad words I expect from your son, if I go with the lad to speak; and to learn at whom the wise man might be cross.”\evb
\evg


\bvg {\small Shirner quoth:}
\bva „Sęg þat Fręyr, \hld\ folkvaldi goða, &
\ind ok ek vilja vita, &
hví þú ęinn sitr \hld\ ęndlanga sali &
\ind minn dróttinn of daga.“\eva

\bvb — “Tell thou it, Free, troop-wielder of the gods, I too would want to know: Why thou alone stayest in the endlong halls, my lord, during the days?”\evb
\evg


\bvg {\small Free quoth:}
\bva „Hví of sęgja’k þér, \hld\ sęggr hinn ungi, &
\ind mikinn móðtrega? &
því’t alfrǫðull \hld\ lýsir of alla daga &
\ind ok þęygi at mínum munum.“\eva

\bvb — “Why ought I to tell thee, young man, about great mood-grief? For the elf-wheel \ken{sun} shines during all days, and naught to my delight.”\evb
\evg


\bvg {\small Shirner quoth:}
\bva „Muni þína \hld\ hykk-a svá mikla vesa, &
\ind at þú mér \edtext{sęggr}{\lemma{sęggr ‘man’, originally ‘messenger’}\Bfootnote{Here used in reference to Free’s addressing Shirner as \emph{sęggr hinn ungi} ‘the young man’. Shirner points out that the two are of equal age, so Free is as much of a young man as he.}} né sęgir; &
ungir saman \hld\ vǫ́rum í árdaga, &
\ind vęl mę́ttim tvęir trúask.“\eva

\bvb “Thy delights I do not think so large, that thou to me, man, ought not to say them. Young together were we in days of yore; we two might well trust each other.”\evb
\evg


\bvg {\small Free quoth:}
\bva „Í Gymis gǫrðum \hld\ ek ganga sá &
\ind mér tíða męy; &
armar lýstu, \hld\ ęn af þaðan &
\ind allt lopt ok lǫgr.“\eva

\bvb “In Gymer’s yards I saw walking a maiden, dear to me. The arms shone, but thereof all the air and sea.”\evb
\evg


\bvg
\bva „Mę́r es mér tíðari \hld\ an manna hvęim &
\ind ungum í árdaga; &
ása ok alfa \hld\ þat vill ęngi maðr, &
\ind at vit sátt séim.“\eva

\bvb “The maiden is dearer to me than to any young man in days of yore. Of the \inx[G]{Ease and Elves} no man\footnoteB{For other examples of gods being called men see TODO.} wants that we two be reconciled.”\evb
\evg


It is likely that a verse is missing here, where Free asks Shirner to go to fetch the maiden for him.


\bvg {\small Shirner quoth:}
\bva „Mar gef mér þá, \hld\ es mik of myrkvan beri &
\ind vísan vafrloga, &
ok þat sverð, \hld\ es sjalft vegisk &
\ind við jǫtna ę́tt.“\eva

\bvb “Then give me the steed, which might bear me over the dark, wise wavering-flame; and that sword, which by itself might strike against the \inx[C]{aught} of the \inx[G]{Ettins}.”\evb
\evg


\bvg {\small Free quoth:}
\bva „Mar þér þann gef’k, \hld\ es þik of myrkvan \edtext{berr &
\ind vísan vafrloga,
ok þat sverð, \hld\ es sjalft mun vegask, &
\ind ef sá ’s horskr es hęfr.“}{\lemma{berr ‘bears’; mun vegask, ef sá ’s horskr es hęfr ‘will strike, if he is wise who owns it’}\Bfootnote{Responding, Free switches out the subjunctive verb forms (“might bear [...] might strike”), giving a sense of certainity and authority. The steed and sword are faultless, and if Shirner fails on the mission, it would be only due to his own fault.}}\eva
%TODO? Change the line numbering from 1–4 to 1, 3–4.

\bvb “That steed I give thee, which bears thee over the dark, wise wavering-flame; and that sword, which by itself will strike, if he is wise who owns it.”\evb
\evg


\bvg {\small Shirner spoke with the horse:}
\bva „Myrkt es úti, \hld\ mál kveð’k okr fara &
\ind úrig fjǫll yfir &
\ind þursa þjóð yfir; &
báðir vit komumk \hld\ eða okr báða tękr
\ind sá hinn \edtext{ámátki jǫtunn}{\lemma{ámátki jǫtunn ‘terrifying ettin’}\Bfootnote{Formulaic. \emph{ámáttigr} ‘terrifying’ seems to have a supernatural connotation, and only occurs in four other places in the Poetic Edda: in \Voluspa\ 8, \Grimnismal\ 11 and \HelgakvidaHjorvardssonar\ 17 it is paired with \emph{jǫtunn} ‘ettin’, while in \HelgakvidaHjorvardssonar\ 14 it describes a man with clearly supernatural attributes.}}.“\eva

\bvb “’Tis dark outside; I call it time for us two to journey: over the drizzling mountains, over the people of the \inx[G]{Thurses}. Both two we come, or us both that terrifying ettin takes.\footnoteB{Shirner declares his intention not to abandon his horse.}”\evb
\evg


BPG
BPASkírnir reið i Jǫtunheima til Gymis garða; þar váru hundar ólmir ok bundnir fyrir skíðgarðs hliði þess, er um sal Gerðar var. Hann reið at þar, er féhirðir sat á haugi, ok kvaddi hann: EPA

BPBShirner rode into the Ettinhomes to Gymer’s yards. There were hounds, fierce and bound in front of the slope of that wooden fence which surrounded Gird’s\footnote{Rather strangely, it is first now that we are informed of the maiden’s name.} hall. He rode to where a shepherd sat on a mound, and greeted him: EPB
EPG


\bvg
\bva „Sęg þat hirðir, \hld\ es á haugi sitr &
\ind ok varðar alla vega: &
hvé ek at andspilli \hld\ komumk hins unga mans &
\ind fyr gręyjum Gymis.“\eva

\bvb “Say it, herdsman, who sittest on the mound, and guardest all ways: How I to discourse might come with the young maiden, past Gymer’s greyhounds?”\evb
\evg


\bvg {\small [The herdsman quoth:]}
\bva „Hvárt est fęigr, \hld\ eða est \edtext{fram ginginn}{\lemma{fram ginginn ‘passed-on’}\Bfootnote{i.e. ‘dead’.}} &
\ind [...]; &
andspillis vanr \hld\ þú skalt ę́ vesa &
\ind góðrar męyjar Gymis.“\eva

\bvb “Either art thou fey, or passed-on; [...]. Lacking discourse shalt thou ever be, with Gymer’s good maiden.”\evb
\evg


\bvg {\small [Shirner quoth:]}
\bva „\edtext{Kostir}{\lemma{kostir ‘choices’}\Bfootnote{i.e. ‘alternative choices, other ways’.}} ’ró bętri \hld\ hęldr an at kløkkva séi &
\ind hvęim es fúss es fara, &
ęinu dǿgri \hld\ mér vas aldr of skapaðr &
ok alt líf of lagit.“\eva

\bvb “Choices are better, rather than sobbing, for whomever is eager to depart. On a single day was my lifetime shaped, and all my life was laid.\footnoteB{The ancient fatalistic beliefs, wherein one’s course of life was predetermined at birth, are here fully evident. Cf. TODO.}”\evb
\evg


\bvg {\small [Gird quoth:]}
\bva „Hvat ’s hlym hlymja \hld\ es hlymja hęyri’k nú til &
\ind ossum rǫnnum í? &
jǫrð bifask, \hld\ ęn allir fyr &
\ind skjalfa garðar Gymis.“\eva

\bvb “What is the din of dins, which I of dins now hear in our houses? The earth trembles, and in front, all the yards of Gymer quake.”\evb
\evg


\bvg {\small A servant-woman quoth:}
\bva „Maðr er hér úti, \hld\ stiginn af mars baki, &
\ind jó lę́tr til jarðar taka.“\eva

\bvb “A man is here outside, stepped down off a horse’s back; he lets take his steed to the ground.\footnoteB{According to \textcite{FinnurEdda} a still-known Icelandic expression; Shirner lets his horse graze.} (TODO: translation)”\evb
\evg


\bvg {\small [Gird quoth:]}
\bva „Inn bið þú hann ganga \hld\ í okkarn sal &
\ind ok drekka hinn mę́ra mjǫð, &
þó ek hitt óumk, \hld\ at hér úti séi &
\ind minn bróðurbani.“\eva

\bvb “Bid thou him to go in into our hall, and to drink the renowned mead; though I fear that here outside might be my brother’s bane-man.”\evb
\evg


\bvg {\small [Gird quoth:]}
\bva „Hvat ’s þat alfa \hld\ né ása sona, &
\ind né víssa vana? &
hví ęinn of komt \hld\ ęikinn fúr yfir &
\ind ór salkynni at séa.“\eva

\bvb “What sort is that, not of Elves, nor of sons of the Ease, nor of wise Wanes? Why camest thou alone over the raging fire, to see the state of our hall?”\evb
\evg


\bvg {\small [Shirner quoth:]}
\bva „Emkat alfa \hld\ né ása sona &
\ind né víssa vana, &
þó ęinn of kom’k \hld\ ęikinn fúr yfir &
\ind yður salkynni at séa.\eva

\bvb “I am not of the Elves, nor of sons of the Ease, nor of wise Wanes; although I came alone over the raging fire, to see the state of our hall.\evb
\evg


\bvg
\bva Ępli ęllifu \hld\ hér hef’k algollin, &
\ind þau mun’k þér Gęrðr gefa, &
frið at kaupa, \hld\ at þú þér Fręy kveðir &
\ind ólęiðastan at lifa.“\eva

\bvb Apples eleven I have here, all-golden; those I will to thee, Gird, give; to purchase the friendship, that thou callest Free with thee dearest\footnoteB{lit. ‘most unloathsome’} to live.\footnoteB{i.e. that Gird}”\evb
\evg


\bvg {\small [Gird quoth:]}
\bva „Ępli ęllifu \hld\ ek þigg aldrigi &
\ind at manskis munum, &
né vit Fręyr, \hld\ meðan okkart fjǫr lifir, &
\ind byggum bę́ði saman.“\eva

\bvb “Apples eleven I never accept, to any man’s delights; nor do I and Free—while our lives remain—dwell both together.”\evb
\evg


\bvg {\small [Shirner quoth:]}
\bva „Baug þér þá gef’k, \hld\ þann’s bręndr of vas &
\ind með ungum Óðins syni, &
átta ’ró jafnhǫfgir, \hld\ es af drjúpa &
\ind hina níundu hvęrja nótt.“\eva

\bvb “The \inx[C]{bigh} I then give thee, that one which was burned with Weden’s young son\footnoteB{The bigh (armlet) that burned on the funeral pyre together with \inx[P]{Balder}. It is notable that it was thought to have been recovered.} \ken*{= Balder}. Eight are even-heavy, which from it drip, every ninth night.\footnoteB{The bigh is apparently capable of reproducing itself.}”\evb
\evg


\bvg {\small [Gird quoth:]}
\bva „Baug þikkak, \hld\ þótt bręndr séi, &
\ind með ungum Óðins syni; &
esa mér golls vant \hld\ í gǫrðum Gymis &
\ind at dęila fé fǫður.“\eva

\bvb “The bigh I accept not, although it be burned with Weden’s young son \ken*{= Balder}; there is for me no want of gold in Gymer’s yards, sharing the \inx[C]{fee} of my father.”\evb
\evg


\bvg {\small [Shirner quoth:]}
\bva „Sér þú mę́ki, mę́r, \hld\ mjóvan, málfáan, &
\ind es hęf’k í hendi hér? &
hǫfuð hǫggva \hld\ mun’k þér halsi af, &
\ind nema mér sę́tt sęgir.“\eva

\bvb “Seest thou this sword—slender, pictured-painted\footnoteB{The sword is inlaid with metal forming a pattern. For examples see TODO.}—which I have here in my hand? Off thy neck will I hew thy head, unless thou agree with me.\footnoteB{lit. ‘unless thou to me sayest an agreement/settlement.’}”\evb
\evg


\bvg {\small [Gird quoth:]}
\bva „Ánauð þola \hld\ vil’k aldrigi &
\ind at manskis munum, &
þó hins get’k, \hld\ ef it Gymir finnizk &
\ind vígs ótrauðir at vegizk.“\eva

\bvb “Suffer coercion will I never, to any man’s delights; though I mean, if thou and Gymer meet, that ye two unreluctant of conflict may fight.”\evb
\evg


\bvg {\small [Shirner quoth:]}
\bva „Sér þú mę́ki, mę́r, \hld\ mjóvan, málfáan, &
\ind es hęf’k í hendi hér? &
fyr þessum ęggjum \hld\ hnígr sá hinn aldni jǫtunn, &
\ind verðr þinn fęigr faðir.\eva

\bvb “Seest thou this sword—slender, pictured-painted—which I have here in my hand? By these edges the aged ettin \ken*{= Gymer} reclines; \inx[C]{fey} becomes thy father.\evb
\evg


\bvg
\bva Tamsvęndi þik drep’k, \hld\ ęn þik tęmja mun’k, &
\ind mę́r, at mínum munum, &
þar skalt ganga \hld\ es þik gumna synir &
\ind síðan ę́va séi. \eva

\bvb With the taming-wand I strike thee, but I will tame thee, maiden, to my delights. There shalt thou go, where the sons of men never since may see thee.\evb
\evg


\bvg
\bva EDITION \eva

\bvb TRANSLATION\evb
\evg


\bvg
\bva EDITION \eva

\bvb TRANSLATION\evb
\evg


\bvg
\bva EDITION \eva

\bvb TRANSLATION\evb
\evg


\bvg
\bva EDITION \eva

\bvb TRANSLATION\evb
\evg


\bvg
\bva EDITION \eva

\bvb TRANSLATION\evb
\evg


\bvg
\bva EDITION \eva

\bvb TRANSLATION\evb
\evg



....




\bvg {\small [Free quoth:]}
\bva Lǫng es nótt, \hld\ langar ’ró tvę́r, &
\ind hvé of þręyja’k þríar? &
opt mér mánaðr \hld\ minni þótti &
\ind an sjá hǫlf hýnótt.\eva

\bvb Long is a night; long are two; how can I yearn for three? Often a month seemed to me less, than this half wedding-night.\footnoteB{The wedding-night (TODO: it's a hapax so explain the interpretation) is half in that it is not consumated.}\evb
\evg
