\bookStart{Flyting of Lock}[Lokasęnna]

\begin{flushright}%
\textbf{Dating} \parencite{Sapp2022}: C10th (0.965)

\textbf{Meter:} \Ljodahattr%
\end{flushright}

\section{Introduction}

The \textbf{Flyting of Lock} (\Lokasenna) is only preserved in \Regius, where it follows \Hymiskvida.  The two poems are tied together into a single narrative by the prose passage “From Eagre and the Gods”, but they are certainly distinct compositions.  The differences in style between the two are drastic, and in \AM\ \Hymiskvida\ stands alone.

The poem has been interpreted as blasphemous (TODO: elaborate), but there is nothing in the language to suggest a late dating.

\sectionline

\section{From Eagre and the Gods (\emph{Frá Ę́gi ok goðum})}

\bpg\bpa Ę́gir, er ǫðru nafni hét Gymir, hann hafði búit ásum ǫl þá er hann hafði fengit ketil inn mikla sem nú er sagt. Til þeirar veitslu kom Óðinn ok Frigg kona hans. Þórr kom eigi því at hann var í austr-vegi. Sif var þar, kona Þórs; Bragi, ok Iðunn kona hans. Týr var þar, hann var ein-hendr; Fenrisulfr sleit hǫnd af hánum, þá er hann var bundinn. Þar var Njǫrðr ok kona hans Skaði; Freyr ok Freyja; Víðarr son Óðins. Loki var þar, ok þjónustu-menn Freys, Byggvir ok Beyla. Mart var þar ása ok alfa.\epa

\bpb \inx[P]{Eagre}, who by another name is called \inx[P]{Gymer}, had prepared an ale-feast for the Eese when he had got the great kettle as is now told.\footnoteB{See the immediately preceding \Hymiskvida.} To that gathering came \inx[P]{Weden} and \inx[P]{Frie}, his woman. \inx[P]{Thunder} came not, for he was on the \inx[L]{Eastern Way}. Sib was there, Thunder’s woman; \inx[P]{Bray} and \inx[P]{Idun}, his woman. \inx[P]{Tew} was there, he was one-handed. The \inx[P]{Fenrerswolf} tore his hand off when it was bound.\footnoteB{This detail is probably brought up to chronologically date the events of the poem as happening after the binding of Fenrer in the mythology.} There was \inx[P]{Nearth}, and his woman \inx[P]{Shede}; \inx[P]{Free} and \inx[L]{Frow}; \inx[P]{Wider}, the son of \inx[P]{Weden}. \inx[P]{Lock} was there, and the servants of Free: \inx[P]{Bew} and \inx[P]{Beal}. There was a great many of the \inx[G]{Eese} and \inx[G]{Elves}\footnoteB{A formulaic expression, see \inx[F]{Eese and Elves}.}.\epb\epg


\bpg\bpa Ę́gir átti tvá þjónustu-menn; Fimafengr ok Eldir. Þar var lýsi-gull haft fyr elds-ljós; sjalft barsk þar ǫl. Þar var griða-stadr mikill. Menn lofuðu mjǫk hversu góðir þjónustu-menn Ę́gis vóru. Loki mátti eigi heyra þat, ok drap hann Fimafeng. Þá skóku ę́sir skjǫldu sína ok ǿptu at Loka, ok eltu hann braut til skógar, en þeir fóru at drekka. Loki hvarf aptr ok hitti úti Eldi; Loki kvaddi hann:\epa

\bpb Eagre had two servants: \inx[P]{Femfinger} and \inx[P]{Elder}. There glowing gold was used instead of fire; the ale there poured itself. That place was a great \inx[C]{grith-stead}.\footnoteB{A place wherein all violence was forbidden, see Encyclopedia.} Men greatly praised how good the servants of Eagre were. Lock could not stand to hear that, and he slew Femfinger. Then the Eese shook their shields and screamed at Lock,\footnoteB{Some sort of ancient war dance. Cf. the Old Swedish Heathen Law: “He screams three nithing-screams TODO”.} and chased him away to the forest—but they went to drink. Lock turned back around and met Elder outside. Lock greeted him:\epb\epg

\sectionline

\section{The Flyting of Lock}

\bvg\bva „Sęg þú þat, \alst{E}ldir, \hld\ \edtext{svá’t \alst{ęi}nu-gi &
\ind \alst{f}eti gangir \alst{f}ramarr}{\lemma{svá’t \dots\ framarr ‘so that \dots\ further’}\Bfootnote{Cf. \Havamal\ 38: \emph{feti ganga framarr} ‘take one step further’.}}, &
hvat hér \alst{i}nni \hld\ hafa at \alst{ǫ}l-mǫ́lum &
\ind \alst{s}ig-tíva \alst{s}ynir.“\eva

\bvb “Tell this, O Elder, so that thou not \\
\ind take one step further: \\
What here within for their ale-speeches have \\
\ind the sons of the victory-Tews \ken{gods}?\footnoteB{i.e. ‘what do they speak about over the ale?’}”\evb\evg


\bvg\bva\speakernote{Ęldir:}%
„Of \alst{v}ǫ́pn sín dǿma \hld\ ok of \alst{v}íg-risni sína &
\ind \alst{s}ig-tíva \alst{s}ynir; &
\alst{á}sa ok \alst{a}lfa, \hld\ es hér \alst{i}nni eru, &
\ind \edtext{mann-gi ’s þér í orði vinr.}{\lemma{mann-gi \dots\ vinr ‘none \dots\ words.’}\Bfootnote{i.e. “none of them say anything good about you.” — The (lack of) alliteration here is very notable, and also occurs in st. 10 (between \emph{Víðarr} and \emph{ulfs}, see note there). It could simply be explained by the line being corrupt, but as there are no signs of that we ought to look for other explanations. I see two, namely that (a) the semi-vowel \emph{v} (\textipa{/w/}) is participating in vowel-alliteration with \emph{o}. Such an alliteration between \emph{v} and true vowels is never encountered in Scaldic poetry, but it might have been existed in the simpler Eddic styles; or that (2) the poem (or at least the relevant lines) is of such old age that it was composed before the North Germanic loss of \emph{v} before rounded vowels. This is supported by the fact that in both the present st. and st. 10 the words beginning with vowels (\emph{orð} ‘word’, \emph{ulfr} ‘wolf’) have cognates in other Germanic languages that begin with \emph{w}, and in the case of the word \emph{ulfr} this consonant is also attested in several old Scandinavian runic inscriptions. For metrical reasons the lines must postdate syncope, but on the basis of three clearly related C7th runestones from Blekinge (from Stentoften, Gummarp, and Istaby; DR 357–359) the loss of \emph{w} before rounded vowels is shown also to have occurred after some syncope (so DR 359 \textbf{h\textsc{a}þuwulafʀ} \emph{Haþuwulᵃfʀ}). Of course, even if the alliteration indeed is on \emph{v}, this does not require dating the whole poem to the late Proto-Norse period (indeed, according to the analysis done by \textcite{Sapp2022}, it is not even the linguistically oldest poem preserved); the older forms could simply be an archaism.

A C7th Proto-Norse form of the c-line might be: \emph{*mannagí ’s þéʀ in worðé winiʀ}.}}“\eva

\bvb\speakernoteb{Elder quoth:}%
“Of their weapons they speak, and of their fight-valiance, \\
\ind the sons of the victory-Tews \ken{gods}; \\
of the Eese and Elves which are here within \\
\ind none is thee a friend in words.”\evb\evg


\bvg\bva\speakernote{Loki kvað:}%
„\alst{I}nn skal ganga \hld\ \alst{Ę́}gis hallir í &
\ind á þat \alst{s}umbl at \alst{s}éa, &
\edtrans{\alst{jǫ}ll ok \alst{ǫ́}fu}{scorn and hatred}{\Bfootnote{\emph{ioll oc áfo} \Regius. These two interesting words have been interpreted in a variety of ways: \CV\ sees the first word as \emph{jóll} ‘wild angelica’, whereas the second is taken to be an error for \emph{áfr} ‘a beverage [...] translated by Magnaeus by \emph{sorbitio avenacea}, a sort of common ale brewed of oats’. TODO: What do other editors say? Esp. Kommentar.}} \hld\ fǿri’k \alst{á}sa sonum &
\ind ok \edtext{blęnd’k þęim svá \alst{m}ęini \alst{m}jǫð}{\lemma{blęnd’k \dots\ męini mjǫð ‘I mix \dots\ the mead with harm’}\Bfootnote{Formulaic, cf. \Sigrdrifumal\ 8 (and others TODO).}}.“\eva

\bvb\speakernoteb{Lock quoth:}%
“In shall I go into Eagre’s halls, \\
\ind on that \inx[C]{simble} for to see; \\
scorn and hatred I bring the sons of the Eese, \\
\ind and I mix for them so the mead with harm.”\evb\evg


\bvg\bva\speakernote{Ęldir kvað:}%
„Vęitst, ef \alst{i}nn gęngr \hld\ \alst{Ę́}gis hallir í &
\ind á þat \alst{s}umbl at \alst{s}éa, &
\alst{h}rópi ok rógi \hld\ ef ęyss á \alst{h}oll ręgin, &
\ind á \alst{þ}ér munu þau \alst{þ}ęrra \alst{þ}at.“\eva

\bvb\speakernoteb{Elder quoth:}%
“Know, if in thou goest into Eagre’s halls, \\
\ind for to see that simble: \\
if slander and strife thou dost pour on the \inx[C]{hold} \inx[G]{Reins}, \\
\ind on \emph{thee} will they dry it off.”\evb\evg


\bvg\bva\speakernote{Loki kvað:}%
„Vęitst þat \alst{Ę}ldir, \hld\ ef \alst{ęi}nir skulum &
\ind \alst{s}ár-yrðum \alst{s}akask, &
\alst{au}ðigr verða \hld\ mun’k í \alst{a}nd-svǫrum, &
\ind \edtrans{ef þú \alst{m}ę́lir til \alst{m}art!}{if thou speak too much!}{\Bfootnote{Formulaic; cf. \Havamal\ 27.}}“\eva

\bvb\speakernoteb{Lock quoth:}%
“Know that, O Elder, if alone we [two] shall \\
\ind banter with wounding words: \\
wealthy will I in my answers become, \\
\ind if thou speak too much!”\evb\evg


\bpg\bpa Síðan gekk Loki inn í hǫllina; en er þeir sá, er fyrir váru, hverr inn var kominn, þǫgnuðu þeir allir.\epa

\bpb Thereafter Lock went into the hall, but when those who were there before him saw who was come inside, they all turned silent.\epb\epg


\bvg\bva\speakernote{Loki kvað:}%
„\alst{Þ}yrstr ek kom \hld\ \alst{þ}ęssar hallar til &
\ind \alst{L}optr of \alst{l}angan veg, &
\alst{ǫ́}su at biðja, \hld\ at mér \alst{ęi}nn gefi &
\ind \edtrans{\alst{m}ę́ran drykk \alst{m}jaðar.}{renowned drink of mead}{\Bfootnote{Formulaic language for describing mead; cf. \Havamal\ 105, 140, \Skirnismal\ 16. TODO: more parallels.}}\eva

\bvb\speakernoteb{Lock quoth:}%
“Thirsty to these halls came I, \\
\ind Loft \name{= Lock}, over a long way, \\
to ask the Eese that they give me one \\
\ind renowned drink of mead.\evb\evg


\bvg\bva Hví \alst{þ}ęgið ér svá \hld\ \alst{þ}rungin goð, &
\ind at \alst{m}ę́la né \alst{m}ęguð; &
\alst{s}essa ok staði \hld\ vęlið mér \alst{s}umbli at, &
\ind eða \alst{h}ęitið mik \alst{h}eðan!“\eva

\bvb Why shut ye up, O pressed Gods, so \\
\ind that ye cannot speak? \\
Choose seats and places for me at the simble, \\
\ind or call me hence [away]!\footnoteB{i.e. “Cease your ambiguity; give me a seat or tell me to leave!”}”\evb\evg


\bvg\bva\speakernote{Bragi:}%
„\alst{S}essa ok staði \hld\ vęlja þér \alst{s}umbli at &
\ind \alst{ę́}sir \alst{a}ldri-gi; &
því-at \alst{ę́}sir vitu \hld\ hvęim \alst{a}lda skulu &
\ind \alst{g}amban-sumbl of \alst{g}eta.“\eva

\bvb\speakernoteb{Bray [quoth]:}%
“Choose seats and places for thee at the simble \\
\ind the Eese will never do, \\
for the Eese know for which man they shall \\
\ind prepare the gomben-simble.”\evb\evg


\bvg\bva\speakernote{[Loki:]}%
„Mant þat \alst{Ó}ðinn, \hld\ es vit í \alst{á}r-daga &
\ind \alst{b}lendum \alst{b}lóði saman? &
\alst{ǫ}lvi bęrgja \hld\ létsk \alst{ęi}gi mundu, &
\ind nema okkr vę́ri \alst{b}ǫ́ðum \alst{b}orit.“\eva

\bvb\speakernoteb{[Lock quoth:]}%
“Recallest thou, Weden, when we two in days of yore \\
\ind blended our blood together? \\
Thou saidst that thou wouldst never taste ale, \\
\ind unless it were for us both borne forth!”\evb\evg


\bvg\bva\speakernote{[Óðinn:]}%
\edtext{„Rís þú Víðarr \hld\ ok lát ulfs fǫður}{\lemma{Rís \dots\ fǫður ‘Rise \dots\ father’}\Bfootnote{For the alliteration see note to st. 2.  A C7th Proto-Norse form of the line might be: \emph{*Rís þú Wíðarʀ · auk lát wulfs faður}.}} &
\ind \alst{s}itja \alst{s}umbli at, &
síðr oss \alst{L}oki \hld\ kvęði \alst{l}asta-stǫfum &
\ind \alst{Ę́}gis hǫllu \alst{í}.“\eva

\bvb\speakernoteb{[Weden quoth:]}%
“Rise thou, Wider, and let the Wolf’s father \ken*{= Lock} \\
\ind sit at the simble, \\
lest Lock should greet us with words of vice \\
\ind in Eagre’s hall.”\evb\evg


\bpg\bpa Þá stóð Víðarr upp ok skenkti Loka, en áðr hann drykki, kvaddi hann ásuna:\epa

\bpb Then Wider stood up and poured a drink to Lock, but before he \ken*{= Lock} drank, he greeted the Eese:\epb\epg


\bvg\bva „Hęilir \alst{ę́}sir, \hld\ hęilar \alst{ǫ́}synjur &
\ind ok ǫll \alst{g}inn-hęilǫg \alst{g}oð, &
nema sá \alst{ęi}nn \alst{ǫ́}ss \hld\ es \alst{i}nnar sitr &
\ind \alst{B}ragi \alst{b}ękkjum á.“\eva

\bvb “Hail the \inx[G]{Eese}! Hail the \inx[G]{Ossens}, \\
\ind and all \inx[C]{yin-holy} Gods!\footnoteB{The first two half-lines are identical to the prayer \Sigrdrifumal\ 3–4.  The prayer formula may actually have been used in Heathen toasts, where the second half of the stanza was used to ask for a boon.  Lock subverts it by instead insulting one of the gods present, which would have come off as blasphemous to the Heathen audience.} \\
Save for that one \inx[G]{Eese}[os] who sits further within: \\
\ind Bray, on the benches.”\evb\evg


\bvg\bva\speakernote{[Bragi] kvað:}%
„\edtrans{\alst{M}ar ok \alst{m}ę́ki}{Steed and sword}{\Bfootnote{Formulaic, also occuring in \Skirnismal\ TODO.}} \hld\ gef’k þér \alst{m}íns féar &
\ind ok \alst{b}ǿtir þér svá \alst{b}augi \alst{B}ragi, &
síðr þú \alst{ǫ́}sum \hld\ \alst{ǫ}fund of gjaldir— &
\ind \alst{g}ręm þú ęigi \alst{g}oð at þér!“\eva

\bvb\speakernoteb{{[Bray]} quoth:}%
“Steed and sword I give thee of my own wealth, \\
\ind and so restores thee Bray with a \inx[C]{bigh}, \\
lest thou shouldst yield envy to the Eese— \\
\ind anger not the Gods against thee!”\evb\evg


\bvg\bva\speakernote{[Loki] kvað:}%
„\alst{Jó}s ok \alst{a}rm-bauga \hld\ munt \alst{ę́} vesa &
\ind \alst{b}ęggja vanr \alst{B}ragi, &
\alst{á}sa ok \alst{a}lfa, \hld\ es hér \alst{i}nni eru, &
\ind þú est við \alst{v}íg \alst{v}arastr, &
\ind ok \alst{sk}jarrastr við \alst{sk}ot.“\eva

\bvb\speakernoteb{{[Lock]} quoth:}%
“Of both steed and arm-bighs wilt thou ever \\
\ind be, O Bray, lacking! \\
Of the Eese and Elves which are here within, \\
\ind \emph{thou} art with war wariest \\
\ind and shiest with shot.”\evb\evg


\bvg\bva\speakernote{[Bragi] kvað:}%
„Vęit’k, ef fyr \alst{ú}tan vę́ra’k, \hld\ svá sem fyr \alst{i}nnan em’k, &
\ind \alst{Ę́}gis hǫll \alst{o}f kominn, &
\alst{h}ǫfuð þitt \hld\ bę́ra’k í \alst{h}ęndi mér; &
\ind\edtext{\alst{l}ít’k þér þat fyr \alst{l}ygi}{\Bfootnote{\emph{‘litt ec þer þat fyr lygi’} \Regius.  A variety of emendations have been proposed for this line. Simplest would be \emph{lítt es þér þat fyr lygi} ‘that is little [punishment] for thee for lying’. Based on the similarity of \emph{ꞇ̇} (= \emph{tt}) and \emph{c} \textcite{FinnurEdda} gives \emph{lykak þér þat fyr lygi} ‘so I would bring to thee for thy lie’.}}.“\eva

\bvb\speakernoteb{{[Bray]} quoth:}%
“I know if outside I were as inside I am \\
\ind come into Eagre’s hall,\footnoteB{As explicitly said in P1, the rule of \inx[C]{grith} (a truce of non-violence, even between enemies; see Encyclopedia) applied inside the hall. Being bound to it, Bray (or the other gods) cannot injure Lock.} \\
the head of thine would I bear in my hands; \\
\ind this I see for thy lie.”\evb\evg


\bvg\bva\speakernote{[Loki] kvað:}%
„\alst{S}njallr est í \alst{s}essi, \hld\ skal-at-tu \alst{s}vá gęra, &
\ind \alst{B}ragi \alst{b}ękk-skrautuðr; &
\alst{v}ega þú gakk \hld\ ef \alst{v}ręiðr séir; &
\ind \alst{h}yggsk vę́tr \alst{h}vatr fyrir.“\eva

\bvb\speakernoteb{{[Lock]} quoth:}%
“Valiant art thou in the seat; thou shalt not do thus, \\
\ind O Bray the bench-adorner! \\
Go thou to fight if thou art wroth; \\
\ind the bold thinks not in advance.\footnoteB{Lock attacks Bray’s invoking of the rule of grith; a truly brave man would not care about such a thing.}”\evb\evg


\bvg\bva\speakernote{[Iðunn] kvað:}%
„\alst{B}ið ek, \alst{B}ragi, \hld\ \alst{b}arna sifjar duga &
\ind ok allra \alst{ó}sk-maga, &
at þú \alst{L}oka \hld\ kveðir-a \alst{l}asta-stǫfum &
\ind \alst{Ę́}gis hǫllu \alst{í}.“\eva

\bvb\speakernoteb{{[Idun]} quoth:}%
“I bid thee, O Bray, to respect the bond of children, \\
\ind and of all the beloved sons, \\
that thou not greet Lock with words of vice \\
\ind in Eagre’s hall.”\evb\evg


\bvg\bva\speakernote{[Loki] kvað:}%
„Þęgi þú, \alst{I}ðunn, \hld\ þik kveð’k \alst{a}llra kvinna &
\ind \alst{v}er-gjarnasta \alst{v}esa &
síðst þú \alst{a}rma þína \hld\ lagðir \alst{í}tr-þvęgna &
\ind umb þinn \alst{b}róður-\alst{b}ana.“\eva

\bvb\speakernoteb{{[Lock]} quoth:}%
“Shut up thou, Idun! Thee I declare, of all women, \\
\ind most man-eager to be, \\
since thy nobly washed arms thou cast \\
\ind about thy brother’s bane.”\evb\evg


\bvg\bva\speakernote{[Iðunn] kvað:}%
„\alst{L}oka ek kveð’k-a \hld\ \alst{l}asta-stǫfum &
\ind \alst{Ę́}gis hǫllu \alst{í}; &
\alst{B}raga ek kyrri \hld\ \alst{b}jór-ręifan, &
\ind \alst{v}il’k-at at it \alst{v}ręiðir \alst{v}egisk.“\eva

\bvb\speakernoteb{{[Idun]} quoth:}%
“I greet not Lock with words of vice, \\
\ind in Eagre’s hall. \\
Bray I calm, made rowdy from beer— \\
\ind I wish not that ye two wroth ones should fight.”\evb\evg


\bvg\bva\speakernote{[Gefjun] kvað:}%
„Hví it \alst{ę́}sir tvęir \hld\ skuluð \alst{i}nni hér &
\ind \alst{s}ár-yrðum \alst{s}akask? &
\alst{L}opts-ki þat vęit \hld\ at hann \alst{l}ęikinn es &
\ind ok hann \alst{f}jǫrg-vall \alst{f}ría.”\eva

\bvb\speakernoteb{{[Giben]} quoth:}
“Why shall ye two Eese here within, \\
\ind with wound-words each other blame? \\
Loft \name{= Lock} knows not that he is being played, \\
\ind and him TODO.”\evb\evg


\bvg\bva\speakernote{[Loki] kvað:}%
„\alst{Þ}ęgi þú, Gęfjun, \hld\ \alst{þ}ęss mun’k nú geta &
\ind es þik \alst{g}lapði at \alst{g}ęði: &
\alst{s}vęinn inn hvíti \hld\ es þér \alst{s}igli gaf &
\ind ok þú \alst{l}agðir \alst{l}ę́r yfir.“\eva

\bvb\speakernoteb{{[Lock]} quoth:}%
“Shut up thou, Giben! Of \emph{him} will I now speak, \\
\ind who seduced thy senses: \\
the white swain who gave thee a necklace, \\
\ind and thou cast o’er him thy leg!”\evb\evg


\bvg\bva\speakernote{[Óðinn kvað] þat:}%
„\edtext{\alst{Ǿ}rr est, Loki, \hld\ ok \alst{ø}r-viti}{\lemma{Ǿrr \dots\ ok ør-viti ‘Mad \dots\ and out of wits’}\Bfootnote{Formulaic, occurs at two other places (TODO), and is probably alluded to in st. TODO of the present poem.}} &
\ind es þú fę́r þér \alst{G}ęfjun at \alst{g}ręmi &
því-at \alst{a}ldar \alst{ø}r-lǫg \hld\ hygg at \alst{ǫ}ll of viti &
\ind \alst{ja}fn-gǫrla sem \alst{e}k.“\eva

\bvb\speakernoteb{{[Weden quoth]} this:}%
“Mad art thou, Lock, and out of wits, \\
\ind as thou earnest Giben’s anger against thee, \\
for all orlays of people I ween that she should know, \\
\ind just as clearly as I.”\evb\evg


\bvg\bva\speakernote{[Loki] kvað:}%
„Þęgi þú, \alst{Ó}ðinn, \hld\ þú kunnir \alst{a}ldri-gi &
\ind dęila \alst{v}íg með \alst{v}erum; &
opt þú \alst{g}aft \hld\ þęim’s \alst{g}efa skyldir-a, &
\ind inum \alst{s}lę́vurum, \alst{s}igr.“\eva

\bvb\speakernoteb{{[Lock]} quoth:}%
“Shut up thou, Weden! Thou couldst never \\
\ind deal out war midst men— \\
oft hast thou given them thou shouldst not have given, \\
\ind the slower men, victory.”\evb\evg


\bvg\bva\speakernote{[Óðinn] kvað:}%
„Vęitst ef ek \alst{g}af \hld\ þęim’s \alst{g}efa né skylda, &
\ind inum \alst{s}lę́vurum, \alst{s}igr, &
\alst{á}tta vetr \hld\ vast fyr \alst{jǫ}rð neðan &
\ind \edtrans{\alst{k}ýr mólkandi}{a milch cow}{\Bfootnote{May also be read as “milking cows”, the nom. sg. \emph{kýr} being identical to the nom./acc. pl. \emph{kýr}, and \emph{mólka} meaning both ‘to milk’ and ‘to give milk’.  “Milch cow” is preferable for two reasons, viz. (i) that the phrase is followed by \emph{ok kona} ‘and a woman’ rather than \emph{sem kona} ‘as a woman’ or similar, and (ii) that it agrees with another instance where Lock is gives birth in the form of a female animal (cows, of course, only giving milk after calving), namely the episode of the building of the wall around Osyard as told in \Gylfaginning\ 42.}} ok \alst{k}ona &
\ind ok hęfir þar \alst{b}ǫrn of \alst{b}orit &
\ind ok hugða’k þat \alst{a}rgs \alst{a}ðal.“\eva

\bvb\speakernoteb{{[Weden]} quoth:}%
“Thou knowest, that if I have given them I should not have given, \\
\ind the slower men, victory; \\
for eight winters wast thou beneath the earth \\
\ind a milch cow and a woman, \\
\ind and thou hast there borne children, \\
\ind and I’ve judged that a \inx[C]{queer}’s nature.”\evb\evg


\bvg\bva\speakernote{[Loki] kvað:}„En þik \alst{s}íga kóðu \hld\ \alst{S}ámsęyju í &
\ind ok drapt á \alst{v}ett sem \alst{v}ǫlur, &
\alst{v}itka líki \hld\ fórt \alst{v}er-þjóð yfir, &
\ind ok hugða’k þat \alst{a}rgs \alst{a}ðal.“\eva

\bvb\speakernoteb{{[Lock]} quoth:}%
“But thou, they said, didst sink down into Samsy, \\
\ind and didst beatst the drum like do wallows. \\
In a warlock’s likeness thou didst journey through mankind, \\
\ind and I’ve judged \emph{that} a queer’s nature.”\evb\evg


\bvg\bva\speakernote{[Frigg kvað:]}%
„\alst{Ø}r-lǫgum \alst{y}kkrum \hld\ skylið \alst{a}ldri-gi &
\ind \alst{s}ęgja \alst{s}ęggjum frá, &
hvat it \alst{ę́}sir tvęir \hld\ drýgðuð í \alst{á}r-daga; &
\ind \alst{f}irrisk ę́ \alst{f}orn rǫk \alst{f}irar.“\eva

\bvb\speakernoteb{[Frie quoth:]}%
“Of your orlays should ye two never \\
\ind speak to the youths; \\
whatever which ye two Eese did in days of yore, \\
\ind let ancient fates be ever shunned by folk.”\evb\evg


\bvg\bva\speakernote{[Loki kvað:]}%
„Þęgi þú, \alst{F}rigg, \hld\ þú est \alst{F}jǫrgyns mę́r &
\ind ok hęfir ę́ \alst{v}er-gjǫrn \alst{v}esit, &
es þá \alst{V}éa ok Vilja \hld\ létst þér, \alst{V}iðris kvę́n, &
\ind \alst{b}áða í \alst{b}aðm of tękit.“\eva

\bvb\speakernoteb{[Lock quoth:]}%
“Shut up thou, Frie! Thou art Firgyn’s maiden, \\
\ind and has always been man-eager: \\
as [when] Wigh and Will, thou hadst, O Withrer’s wife, \\
\ind both in thy bosom taken.”\evb\evg


\bvg\bva\speakernote{[Frigg kvað:]}%
„Vęitst ef \alst{i}nni \alst{ę́}tta’k \hld\ \alst{Ę́}gis hǫllum \alst{í} &
\ind \alst{B}aldri líkan \alst{b}ur &
\alst{ú}t né kvę́mir \hld\ frá \alst{á}sa sonum &
\ind ok vę́ri þá at þér \alst{v}ręiðum \alst{v}egit.“\eva

\bvb\speakernoteb{[Frie quoth:]}%
“Thou knowest, if within I owned, in Eagre’s halls, \\
\ind a boy alike to Balder: \\
out came thou not from the sons of the Eese, \\
\ind and thou wouldst be fought with wrath.”\evb\evg


\bvg\bva\speakernote{[Loki kvað:]}%
„Ęnn vill þú, \alst{F}rigg, \hld\ at ek \alst{f}lęiri tęlja &
\ind \alst{m}ína \alst{m}ęin-stafi: &
ek því \alst{r}éð \hld\ es þú \alst{r}íða sér-at &
\ind \alst{s}íðan Baldr at \alst{s}ǫlum.“\eva

\bvb\speakernoteb{[Lock quoth:]}
“Still wilt thou, Frie, that I recount more \\
\ind of my harmful deeds: \\
I did plan that thou shouldst not see Balder \\
\ind riding to the halls henceforth.”\evb\evg


\bvg\bva\speakernote{[Fręyja kvað:]}%
„\alst{Ǿ}rr est, Loki, \hld\ es þú \alst{y}ðra tęlr &
\ind \alst{l}jóta \alst{l}ęið-stafi; &
\alst{ø}r-lǫg Frigg \hld\ hygg at \alst{ǫ}ll viti &
\ind þótt hón \alst{s}jǫlf-gi \alst{s}ęgi.“\eva

\bvb\speakernoteb{[Frow quoth:]}
“Mad art thou, Lock, as thou dost count \\
\ind your ugly, loathsome deeds: \\
all orlays I ween that Frie might know, \\
\ind though she herself says them not.”\evb\evg


\bvg\bva\speakernote{[Loki kvað:]}%
„Þęgi þú, \alst{F}ręyja, \hld\ þik kann’k \alst{f}ull-gørva; &
\ind es-a þér \edtrans{\alst{v}amma \alst{v}ant}{free of blemishes}{\Bfootnote{Formulaic, cf. \Havamal\ 22: \emph{hann es-a vamma vanr} ‘he is not free of blemishes’.}}: &
\alst{á}sa ok \alst{a}lfa, \hld\ es hér \alst{i}nni eru, &
\ind \alst{h}vęrr \alst{h}ęfir þinn \alst{h}ór vesit.“\eva

\bvb\speakernoteb{[Lock quoth:]}
“Shut up thou, Frow! I know thee full well— \\
\ind thou art not free of blemishes: \\
of the Eese and Elves which are here within \\
\ind has each one been thy lover!”\evb\evg


\bvg\bva\speakernote{[Fręyja kvað:]}%
\edtext{„\alst{F}lǫ́ ’s þér tunga, \hld\ hygg at þér \alst{f}ręmr myni &
\ind ó·\alst{g}ótt of \alst{g}ala;}{\lemma{Flǫ́ ... gala; ‘False ... thee’}\Bfootnote{The language is again strikingly similar to \Havamal, particularly 29/3–4: “A quick-spoken tongue—unless it be held in place—oft sings evil [into being] for itself (\emph{opt sér ó·gótt of gęlr}).” and 116/3–4: “a false-counseling tongue (\emph{flá-rǫ́ð tunga}) brought his life to its end, and in no way over a truthful charge.”}} &
vręiðir ’ru þér \alst{ę́}sir \hld\ ok \alst{ǫ́}synjur, &
\ind \edtrans{\alst{h}ryggr munt \alst{h}ęim fara}{grieved wilt thou journey home}{\Bfootnote{Frow here shows her ability to foresee the future.  Lock will come to regret his insults.}}.“\eva

\bvb\speakernoteb{[Frow quoth:]}
“False is thy tongue, I ween that it henceforth will \\
\ind sing evil [into being] for thee. \\
Wroth with thee are the Eese and Ossens: \\
\ind grieved wilt thou journey home.”\evb\evg


\bvg\bva\speakernote{Loki:}%
„Þęgi þú, \alst{F}ręyja, \hld\ þú est \alst{f}or-dę́ða &
\ind ok \alst{m}ęini blandin \alst{m}jǫk, &
síðst-u at \alst{b}rǿðr þínum \hld\ siðu \alst{b}líð ręgin &
\ind ok myndir þá, \alst{F}ręyja, \alst{f}rata.“\eva

\bvb\speakernoteb{Lock [quoth]:}
“Shut up thou, Frow! Thou art an evil-working woman, \\
\ind and much mixed with harm, \\
since against thy brother the blithe Reins bewitched thee, \\
\ind and thou wouldst then, O Frow, fart.”\evb\evg


\bvg\bva\speakernote{Njǫrðr:}%
„Þat ’s \alst{v}á-lítit \hld\ þótt sér \alst{v}arðir \alst{v}ers fái, &
\ind \alst{h}ós eða \alst{h}várs; &
hitt ’s \alst{u}ndr, es \alst{á}ss ragr \hld\ es hér \alst{i}nn of kominn &
\ind ok hęfir sá \alst{b}ǫrn of \alst{b}orit.“\eva

\bvb\speakernoteb{Nearth [quoth]:}
“It is little woe that women should get themselves a man, \\
\ind a lover or whomever else. \\
This is a wonder, that a queer os is come here within, \\
\ind and that man has born children!”\evb\evg


\bvg\bva\speakernote{Loki:}%
„\alst{Þ}ęgi þú, Njǫrðr, \hld\ \alst{þ}ú vast austr heðan &
\ind \alst{g}ísl of sęndr at \alst{g}oðum; &
\alst{H}ymis meyjar \hld\ hǫfðu þik at \alst{h}land-trogi &
\ind ok þér í \alst{m}unn \alst{m}igu.“\eva

\bvb\speakernoteb{Lock [quoth]:}%
“Shut up thou, Nearth! Thou wast east hence \\
\ind sent as hostage for the Gods. \\
Hymer’s maidens had thee for a lant-trough, \\
\ind and pissed thee in the mouth!”\evb\evg


\bvg\bva\speakernote{Njǫrðr:}%
„Sú esumk \alst{l}íkn \hld\ es vas’k \alst{l}angt heðan &
\ind \alst{g}ísl of sęndr at \alst{g}oðum: &
þá ek \edtext{\alst{m}ǫg gat \hld\ þann’s \alst{m}ann-gi fíar}{\lemma{mǫg \dots\ þann’s mann-gi fíar ‘the lad whom no man hates’}\Bfootnote{Free.}}, &
\ind ok þikkir sá \alst{á}sa \alst{ja}ðarr.“\eva

\bvb\speakernoteb{Nearth [quoth]:}%
“This is my relief, as I was far-away hence \\
\ind sent as hostage for the Gods: \\
I afterwards begot the lad whom no man hates, \\
\ind and he seems the peak of the Eese.”\evb\evg


\bvg\bva\speakernote{Loki:}%
„\alst{H}ę́tt-u nú, Njǫrðr, \hld\ haf á \alst{h}ófi þik; &
\ind mun’k-a því \alst{l}ęyna \alst{l}ęngr: &
við \alst{s}ystur þinni \hld\ gatst \alst{s}líkan mǫg, &
\ind ok es-a þó \alst{ó}nu \alst{v}err.“\eva

\bvb\speakernoteb{Lock [quoth]:}
“Stop now, Nearth; restrain thyself! \\
\ind I will no longer hide it: \\
by thy sister didst thou beget such a lad, \\
\ind and there can be expected nothing worse.”\evb\evg


\bvg\bva\speakernote{Týr:}%
„Fręyr ’s \alst{b}ętstr \hld\ allra \alst{b}all-riða &
\ind \alst{á}sa gǫrðum \alst{í}; &
\alst{m}ęy né grǿtir \hld\ né \alst{m}anns konu, &
\ind ok lęysir ór \alst{h}ǫptum \alst{h}vęrn.“\eva

\bvb\speakernoteb{Tew [quoth]:}%
“Free is the best of all bold riders \\
\ind in the yards of the Eese; \\
he makes no maiden cry, nor any man’s woman, \\
\ind and loosens anyone from his bonds!”\evb\evg


\bvg\bva\speakernote{Loki:}%
„\alst{Þ}ęgi þú, Týr, \hld\ \alst{þ}ú kunnir aldri-gi &
\ind \edtrans{bera \alst{t}ilt með \alst{t}vęim}{settle strife among two}{\Bfootnote{Uncertain. TODO.}}; &
\alst{h}andar ennar \alst{h}ǿgri \hld\ mun’k \alst{h}innar geta &
\ind es þér slęit \alst{F}ęnrir \alst{f}rá.“\eva

\bvb\speakernoteb{Lock [quoth]:}%
“Shut up thou, Tew! \emph{Thou} couldst never \\
\ind settle strife among two; \\
of the right hand I next will speak, \\
\ind which from thee Fenrer tore.”\evb\evg


\bvg\bva\speakernote{Týr:}%
„\alst{H}andar em’k vanr \hld\ en þú \alst{H}róðrs-vitnis; &
\ind \alst{b}ǫl es \alst{b}ęggja þráa; &
ulf-gi hęfir ok vel \hld\ es í bǫndum skal &
\ind bíða \alst{r}agna \alst{r}økrs.“\eva

\bvb\speakernoteb{Tew [quoth]:}%
“A hand am I lacking, but thou Rothwitner; \\
\ind both yearnings are a bale! \\
Nor does the Wolf have it well, who in bonds shall \\
\ind await the Twilight of the Reins.”\evb\evg


\bvg\bva\speakernote{Loki:}%
„\alst{Þ}ęgi þú, Týr, \hld\ \alst{þ}at varð þinni konu &
\ind at hon átti \alst{m}ǫg við \alst{m}ér! &
\edtrans{\alst{Ǫ}ln}{ell}{\Bfootnote{Wool, measured in ells, was often used for barter in Iceland and Norway.}} né pęnning \hld\ hafðir þess \alst{a}ldri-gi &
\ind \alst{v}an-réttis, \alst{v}ę-sall.“\eva

\bvb\speakernoteb{Lock [quoth]:}%
“Shut up thou, Tew! It happened to thy woman, \\
\ind that she had a lad by me! \\
Neither ell nor penny hadst thou ever for that \\
\ind injustice, O wretch!”\evb\evg


\bvg\bva\speakernote{Fręyr:}%
„\alst{U}lf sé’k liggja \hld\ \alst{á}ar-ósi fyr &
\ind unds \alst{r}júfask \alst{r}ęgin; &
því munt \alst{n}ę́st, \hld\ nema \alst{n}ú þęgir, &
\ind \alst{b}undinn, \alst{b}ǫlva smiðr!“\eva

\bvb\speakernoteb{Free [quoth]:}%
“The Wolf I see lying before the river-mouth, \\
\ind until the Reins are ripped; \\
therefore wilt thou next—unless thou now shut up— \\
\ind be bound, O smith of bales!”\evb\evg


\bvg\bva\speakernote{Loki:}„\alst{G}ulli kęypta \hld\ létst \alst{G}ymis dóttur &
\ind ok \alst{s}ęldir þitt \alst{s}vá \alst{s}verð, &
en es \alst{M}úspells synir \hld\ ríða \alst{M}yrk-við yfir &
\ind \alst{v}ęitst-a þá, \alst{v}ę-sall, hvé \alst{v}egr!“\eva

\bvb\speakernoteb{Lock [quoth]:}%
“Bought with gold hadst thou Gymer’s daughter \ken*{= Gird}, \\
\ind and didst so sell thy sword— \\
but when Muspell’s sons ride over Mirkwood \\
\ind knowest thou not, O wretch, how to fight!”\evb\evg


\bvg\bva\speakernote{Byggvir:}%
„Vęitst ef \alst{ø}ðli \alst{ę́}tta’k \hld\ sem \alst{I}ngunar-Fręyr, &
\ind ok \alst{s}vá \alst{s}ę́l-ligt \alst{s}etr: &
\alst{m}ęrgi smę́ra \hld\ \alst{m}ølða’k þá \alst{m}ęin-krǫ́ku &
\ind ok \alst{l}ęmða alla í \alst{l}iðu.“\eva

\bvb\speakernoteb{Bewe [quoth]:}%
“Thou knowest, if a pedigree I had like Ingwin-Free, \\
\ind and such blessed pasture— \\
smaller than marrow would I mill this harm-crow, \\
\ind and beat all his limbs lame!”\evb\evg


\bvg\bva\speakernote{Loki:}%
„Hvat ’s þat it \alst{l}itla \hld\ es þat \alst{l}ǫggra sé’k &
\ind ok \alst{s}nap-víst \alst{s}napir? &
At \alst{ęy}rum Fręys \hld\ munt \alst{ę́} vesa &
\ind ok und \alst{k}vęrnum \alst{k}laka.“\eva

\bvb\speakernoteb{Lock [quoth]:}%
“What is this little thing which I see crawling, \\
\ind and snap-wisely snapping? \\
At the ears of Free wilt thou ever be, \\
\ind and chirping under mills!”\evb\evg


\bvg\bva\speakernote{[Byggvir kvað:]}%
„\alst{B}yggvir ek hęiti, \hld\ en mik \alst{b}ráðan kveða &
\ind \alst{g}oð ǫll ok \alst{g}umar; &
því em’k \alst{h}ér \alst{h}róðugr \hld\ at drekka \alst{H}ropts męgir &
\ind \alst{a}llir \alst{ǫ}l saman.“\eva

\bvb\speakernoteb{[Bewe quoth:]}%
“Bewe I am called, and hurried do call me \\
\ind all Gods and men; \\
therefore I am here honoured when Roft’s lads \ken{eese} drink \\
\ind ale all together.”\evb\evg


\bvg\bva\speakernote{[Loki kvað:]}%
„\alst{Þ}ęgi þú, Byggvir, \hld\ \alst{þ}ú kunnir aldri-gi &
\ind dęila með \alst{m}ǫnnum \alst{m}at; &
ok þik í \alst{f}lęts strá \hld\ \alst{f}inna né mǫ́ttu &
\ind þá’s \alst{v}ǫ́gu \alst{v}erar.“\eva

\bvb\speakernoteb{[Lock quoth:]}
“Shut up thou, Bewe! \emph{Thou} couldst never \\
\ind deal out food midst men, \\
and in the bench-straw they could not find thee, \\
\ind whenever men did fight.”\evb\evg


\bvg\bva\speakernote{[Hęimdallr kvað:]}%
„\alst{Ǫ}lr est, Loki \hld\ svá’t es \alst{ø}r-viti, &
\ind hví né \alst{l}ętsk-a þú, \alst{L}oki? &
því-at \alst{o}f-drykkja \hld\ vęldr \alst{a}lda hvęim &
\ind es sína \alst{m}ę́lgi né \alst{m}an-at.“\eva

\bvb\speakernoteb{[Homedal quoth:]}%
“Drunk art thou, Lock, so that thou art out of wits; \\
\ind why holdest thou not back, O Lock? \\
For over-drinking causes for every man \\
\ind that he no more recalls his speech.”\evb\evg


\bvg\bva\speakernote{[Loki kvað:]}%
„\alst{Þ}ęgi þú, Hęimdallr, \hld\ \alst{þ}ér vas í ár-daga &
\ind it \alst{l}jóta \edtrans{\alst{l}íf of \alst{l}agit}{life laid [in place]}{\Bfootnote{i.e., his fate was decided.  Formulaic; see TODO.}}; &
\alst{ǫ}rgu baki \hld\ munt \alst{ę́} vesa &
\ind ok \alst{v}aka \edtrans{\alst{v}ǫrðr goða}{Watchman of the Gods}{\Bfootnote{Formulaic epithet of Homedal, who had to guard the rainbow bridge of the Gods against their enemies.  See note to \Grimnismal\ 13.}}.“\eva

\bvb\speakernoteb{[Lock quoth:]}%
“Shut up thou, Homedal! For \emph{thee} was in days of yore \\
\ind thy ugly life laid [in place]; \\
with a stiff back wilt thou ever be \\
\ind and waking, O Watchman of the Gods.”\evb\evg


\bvg\bva\speakernote{[Skaði kvað:]}%
„\alst{L}étt ’s þér, Loki; \hld\ mun-at-tu \alst{l}ęngi svá &
\ind \alst{l}ęika \alst{l}ausum hala, &
því at þik á \alst{h}jǫrvi skulu \hld\ ins \alst{h}rím-kalda magar &
\ind \alst{g}ǫrnum binda \alst{g}oð.“\eva

\bvb\speakernoteb{[Shede quoth:]}%
“’Tis light for thee, Lock—thou wilt not for long \\
\ind play with loose tail so, \\
for on a sword shall, with thy rime-cold lad’s \\
\ind guts, the Gods bind thee.”\evb\evg


\bvg\bva\speakernote{[Loki kvað:]}%
„Vęitst ef mik á \alst{h}jǫrvi skulu \hld\ ins \alst{h}rím-kalda magar &
\ind \alst{g}ǫrnum binda \alst{g}oð, &
\alst{f}yrstr ok øfstr \hld\ vas’k at \alst{f}jǫr-lagi &
\ind \alst{þ}ar’s vér á \alst{Þ}jatsa \alst{þ}rifum.“\eva

\bvb\speakernoteb{[Lock quoth:]}%
“Know, if on a sword shall, with my rime-cold lad’s \\
\ind guts, the Gods bind me: \\
first and highest was I in life-taking \\
\ind when we laid hands on Thedse.”\evb\evg


\bvg\bva\speakernote{[Skaði kvað:]}%
„Vęitst ef \alst{f}yrstr ok øfstr \hld\ vast at \alst{f}jǫr-lagi &
\ind \alst{þ}á’s ér á \alst{Þ}jatsa \alst{þ}rifuð, &
frá mínum \alst{v}éum \hld\ ok \alst{v}ǫngum skulu &
\ind þér ę́ \alst{k}ǫld rǫ́ð \alst{k}oma.“\eva

\bvb\speakernoteb{[Shede quoth:]}%
“Thou knowest, if first and highest thou wast in life-taking \\
\ind when ye laid hands on Thedse: \\
from my wighs and wongs shall for thee \\
\ind ever cold counsels come.”\evb\evg


\bvg\bva\speakernote{[Loki kvað:]}%
„\alst{L}éttari í mǫ́lum \hld\ vast við \alst{L}aufęyjar son &
\ind þá’s létsk mér á \alst{b}ęð þinn \alst{b}oðit; &
\alst{g}etit verðr oss slíks \hld\ ef vér \alst{g}ǫrva skulum &
\ind tęlja \alst{v}ǫmmin \alst{v}ǫ́r.“\eva

\bvb\speakernoteb{[Lock quoth:]}%
“Lighter in speech wast thou with Leafie’s son \ken*{= Lock = me} \\
\ind when thou hadst me bid to thy bed; \\
such will be said of us, if we clearly shall \\
\ind recount our blemishes.\evb\evg


\bpg\bpa Þá gekk Sif fram ok byrlaði Loka í hrím-kálki mjǫð ok mę́lti:\epa

\bpb Then Sib walked forth and poured for Lock mead in a rime-chalice, and spoke:\epb\epg


\bvg\bva „\alst{H}ęill ves þú nú, Loki, \hld\ ok tak við \alst{h}rím-kálki &
\ind \alst{f}ullum \alst{f}orns mjaðar, &
hęldr þú hana \alst{ęi}na \hld\ látir með \alst{á}sa sonum &
\ind \alst{v}amma-lausa \alst{v}esa.“\eva

\bvb “Hale be thou now, O Lock, and receive this rime-chalice, \\
\ind full of ancient mead, \\
that thou rather let her alone among the sons of the Eese \\
\ind remain blemish-less.\footnoteB{Sib attempts to bribe Lock with drink, so that she alone will remain unaccused among the gods.}”\evb\evg


\bpg\bpa Hann tók við horni ok drakk af:\epa

\bpb He received the horn and drank from it:\epb\epg


\bvg\bva „\alst{Ęi}n þú vę́rir \hld\ \alst{e}f þú svá vę́rir, &
\ind \alst{v}ǫr ok grǫm at \alst{v}eri; &
ęinn ek \alst{v}ęit, \hld\ svá’t ek \alst{v}ita þikkjumk, &
\ind \alst{h}ór ok af \alst{H}lórriða, &
\ind ok vas þat sá inn \edtrans{\alst{l}ę́-vísi \alst{L}oki}{guile-wise Lock}{\Bfootnote{Formulaic, also occuring in \Hymiskvida\ 37. Cf. also \Voluspa\ 35 where Lock is called \emph{lę́-gjarn} ‘guile-eager’ and note to \Voluspa\ 17 where Lother (possibly to be identified with Lock) gives men \emph{lǫ́}, which may be an accusative form of \emph{lę́}.}}.“\eva

\bvb “Alone wert thou, if thou so wert \\
\ind wary and wroth against man. \\
I know one—whom I think myself to know— \\
\ind adulterer behind even \inx[P]{Loride}’s back, \\
\ind and that was the guile-wise Lock!”\evb\evg


\bvg\bva\speakernote{[Bęyla kvað:]}%
„\edtrans{\alst{F}jǫll ǫll skjalfa}{The fells all quake}{\Bfootnote{The movement of gods, especially Thunder, is often signalled by cosmic disturbances.  See note to \Thrymskvida\ 21.}}, \hld\ hygg á \alst{f}ǫr vesa &
\ind \alst{h}ęiman \alst{H}lórriða; &
hann \alst{r}ę́ðr \alst{r}ó \hld\ þeim’s \alst{r}ǿgir hér &
\ind \alst{g}oð ǫll ok \alst{g}uma!“\eva

\bvb\speakernoteb{[Beal quoth:]}%
“The fells all quake—I think on the journey \\
\ind from home Loride to be. \\
He brings to rest him who here maligns \\
\ind all Gods and men!”\evb\evg


\bvg\bva\speakernote{[Loki kvað:]}%
„Þęgi þú, \alst{B}ęyla, \hld\ þú est \alst{B}yggvis kvę́n &
\ind ok \alst{m}ęini blandin \alst{m}jǫk; &
\alst{ó}-kynjan męira \hld\ kom-a með \alst{á}sa sonum; &
\ind ǫll est, \alst{d}ęigja, \alst{d}ritin.“\eva

\bvb\speakernoteb{[Lock quoth:]}%
“Shut up thou, Beal! Thou art Bewe’s wife, \\
\ind and much mixed with harm; \\
a greater disgrace came not among the sons of the Eese; \\
\ind thou art all, O kneaderess, shitty!”\evb\evg


\bpg\bpa Þá kom Þórr at ok kvað:\epa

\bpb Then Thunder arrived and quoth:\epb\epg


\bvg\bva%
„\alst{Þ}ęgi þú, rǫg vę́ttr, \hld\ \alst{þ}ér skal minn \alst{þ}rúð-hamarr, &
\ind \alst{M}jǫllnir, \alst{m}ál fyr-nema! &
\alst{H}ęrða klett \hld\ drep’k þér \alst{h}alsi af, &
\ind ok verðr þá þínu \alst{f}jǫrvi of \alst{f}arit.“\eva

\bvb “Shut up thou, queer wight! Thee shall my thrith-hammer \\
\ind Millner, deprive of speech! \\
The shoulder-rock \ken{head} I strike off thy neck, \\
\ind and then is thy lifeblood spilled!”\evb\evg


\bvg\bva\speakernote{[Loki kvað:]}%
„\alst{Ja}rðar burr \hld\ es hér nú \alst{i}nn kominn; &
\ind hví \alst{þ}rasir þú svá, \alst{Þ}órr? &
En þá þorir \alst{ę}kki \hld\ es skalt við \alst{u}lfinn vega &
\ind ok \alst{s}velgr hann allan \alst{S}ig-fǫður.“\eva

\bvb\speakernoteb{[Lock quoth:]}%
“The son of Earth is now here come inside, \\
\ind why dost thou thrash so, O Thunder? \\
But then darest thou not, when with the Wolf thou shalt fight, \\
\ind and he swallows Syefather \name{= Weden} whole.”\evb\evg


\bvg\bva\speakernote{[Þórr kvað:]}%
„\alst{Þ}ęgi þú, rǫg vę́ttr, \hld\ \alst{þ}ér skal minn \alst{þ}rúð-hamarr, &
\ind \alst{M}jǫllnir, \alst{m}ál fyr-nema! &
\alst{U}pp ek þér verp \hld\ ok á \alst{au}str-vega &
\ind \alst{s}íðan þik mann-gi \alst{s}ér.“\eva

\bvb\speakernoteb{[Thunder quoth:]}%
“Shut up thou, queer wight! Thee shall my thrith-hammer \\
\ind Millner, deprive of speech! \\
Up I throw thee, and onto the eastern ways; \\
\ind thereafter no man sees thee!”\evb\evg


\bvg\bva\speakernote{[Loki kvað:]}%
„\alst{Au}str-fǫrum þínum \hld\ skalt \alst{a}ldri-gi &
\ind \alst{s}ęgja \alst{s}ęggjum frá &
síðst \edtrans{í \alst{h}anska þumlungi \hld\ \alst{h}núkðir þú}{in the thumb of a glove thou didst crawl}{\Bfootnote{A reference to Thunder’s encounter with the ettin Shrymer.  The story is told in full in \Gylfaginning.  A related story is also hinted at in \Harbardsljod\ TODO, although the ettin there is called Feller.}}, Ęin-hęri, &
\ind ok \alst{þ}óttisk-a \alst{þ}á \alst{Þ}órr vesa!“\eva

\bvb\speakernoteb{[Lock quoth:]}%
“Of thy eastern journeys shalt thou never \\
\ind speak to the youths, \\
since in the thumb of a glove thou didst crawl, Oneharrier, \\
\ind and didst not seem to be Thunder then!”\evb\evg


\bvg\bva\speakernote{[Þórr kvað:]}%
„\alst{Þ}ęgi þú, rǫg vę́ttr, \hld\ \alst{þ}ér skal minn \alst{þ}rúð-hamarr, &
\ind \alst{M}jǫllnir, \alst{m}ál fyr-nema! &
\alst{h}ęndi inni \alst{h}ǿgri \hld\ drep’k þik \alst{H}rungnis bana, &
\ind svá’t þér \alst{b}rotnar \alst{b}ęina hvat.“\eva

\bvb\speakernoteb{[Thunder quoth:]}%
“Shut up thou, queer wight! Thee shall my thrith-hammer \\
\ind Millner, deprive of speech! \\
With the right hand I strike thee with Rungner’s bane, \\
\ind so that every bone in thee breaks.”\evb\evg


\bvg\bva\speakernote{[Loki kvað:]}%
„\alst{L}ifa ę́tla’k mér \hld\ \alst{l}angan aldr &
\ind þótt \alst{h}ǿtir \alst{h}amri mér; &
\alst{sk}arpar álar \hld\ þóttu þér \alst{Sk}rymis vesa &
\ind ok máttir-a þá \alst{n}ęsti \alst{n}áa &
\ind ok svaltsk þá \alst{h}ungri \alst{h}ęill.“\eva

\bvb {[Lock quoth:]}
“For myself I intend to live a long life, \\
\ind although thou dost threaten me with the hammer. \\
Sharp seemed the straps of Shrymer to thee, \\
\ind and then couldst thou not reach thy provisions, \\
\ind and then wast thou dying, healthy, of hunger.”\evb\evg


\bvg\bva\speakernote{[Þórr kvað:]}„\alst{Þ}ęgi þú, rǫg vę́ttr, \hld\ \alst{þ}ér skal minn \alst{þ}rúð-hamarr, &
\ind \alst{M}jǫllnir, \alst{m}ál fyr-nema! &
\alst{H}rungnis bani \hld\ mun þér í \alst{h}ęl koma &
\ind fyr \alst{N}á-grindr \alst{n}eðan.“\eva

\bvb\speakernoteb{[Thunder quoth:]}%
“Shut up thou, queer wight! Thee shall my thrith-hammer \\
\ind Millner, deprive of speech! \\
Rungner’s bane will take thee to hell, \\
\ind down beneath Neegrind!”\evb\evg


\bvg\bva\speakernote{[Loki kvað:]}%
„Kvað’k fyr \alst{ǫ́}sum, \hld\ kvað’k fyr \alst{á}sa sonum, &
\ind þat’s mik \alst{h}vatti \alst{h}ugr, &
en fyr þér \alst{ęi}num \hld\ mun’k \alst{ú}t ganga &
\ind því-at ek \alst{v}ęit at þú \alst{v}egr.\eva

\bvb\speakernoteb{[Lock quoth:]}
“I spoke before the Eese; I spoke before the sons of the Eese, \\
\ind whatever my heart did goad me. \\
but before thee alone will I walk out, \\
\ind for I know that thou dost strike.\evb\evg


\bvg\bva%
\alst{Ǫ}l gørðir þú, \alst{Ę́}gir, \hld\ en þú \alst{a}ldri munt &
\ind \alst{s}íðan \alst{s}umbl of gøra; &
\alst{ęi}ga þín \alst{ǫ}ll, \hld\ es hér \alst{i}nni es, &
\ind \alst{l}ęiki yfir \alst{l}ogi &
\ind ok \alst{b}renni þér á \alst{b}aki.“\eva

\bvb Ale hast thou made, Eagre, but thou wilt never \\
\ind since make a simble! \\
All thy estate which is here within— \\
\ind may flame play over it, \\
\ind and burn thee on the back!”\evb\evg

\sectionline

\section{From Lock (\emph{Frá Loka})}

The binding of Lock is known from two other places. Closest at hand are sts. H1 and 34 of the \Voluspa, but they offer no full narrative.

\Gylfaginning\ 50 has a longer account, somewhat different from the present prose. There the Eese captured Lock’s two sons, Wonnel and “Nare or Narve”. They turned Wonnel into a wolf (\emph{vargr}, which also means ‘outlaw’) and had him tear his brother Narve apart. Narve’s intestines were then taken and used to bind Lock on top of three pointed stones, with one digging into his shoulder-blades, the other digging into his loins, and the third digging into his houghs. The intestines then turned into iron.

Since the author of \Gylfaginning\ knew \Voluspa, it is possible that he combined a text similar to \FraLoka\ with st. H1, interpreting \emph{Vála víg-bǫnd} as ‘Wonnel’s war-bonds’. Wonnel is otherwise only known as the son of Weden, and there is no reason as to why he could not have bound Lock.

\sectionline

\bpg\bpa En eptir þetta falst Loki í Fránangrs-forsi í lax líki. Þar tóku ę́sir hann. Hann var bundinn með þǫrmum sonar Nara; en Narfi, sonr hans, varð at vargi. Skaði tók eitr-orm ok festi upp yfir and-lit Loka; draup þar ór eitr. Sigyn, kona Loka, sat þar ok helt munn-laug undir eitrit. En er munn-laugin var full bar hon út eitrit, en meðan draup eitrit á Loka. Þá kipptist hann svá hart við, at þaðan af skalf jǫrð ǫll; þat eru nú kallaðir land-skjálftar.\epa

\bpb And after this Lock hid himself in the Freenangersforce in the form of a salmon. There the Eese took him. He was bound with the intestines of his son Nare, but his son Narve was made a wolf/outlaw. Shede took a venomous serpent and fastened it up above Lock’s face; from it ran venom. Syein, Lock’s wife, sat there and held a basin under the venom. And when the basin was full she carried out the venom, but meanwhile the venom ran onto Lock. Then he struggled so hard that thereof all the earth quaked; that is now called earth-quakes.\epb\epg

\sectionline
