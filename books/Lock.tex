\bookStart{Flyting of Lock}[Lokasęnna]
\def\thisBookCode{Lokasenna}

\begin{flushright}%
\textbf{Dating} \parencite{Sapp2022}: C10th (0.965)

\textbf{Meter:} \Ljodahattr%
\end{flushright}

\section{Introduction}

{\small The \textbf{Flyting of Lock} (\Lokasenna) is only preserved in \Regius, where it follows \Hymiskvida\ and comes before \Thrymskvida.  In \Regius\ it is tied together into a continuous narrative with \Hymiskvida\ by the prose passage “From Eagre and the Gods”, but the two poems are certainly distinct compositions, for they are drastically different in style.  In \AM, \Hymiskvida\ stands alone with no trace of a frame narrative.

A stanza that appears to belong to \Lokasenna\ is found in \Gylfaginning\ 20; it is edited below following the end of the poem.

The poem has often (TODO) been interpreted as a blasphemous composition belonging to the period after conversion, with the reasoning that no pious pagan would have written a poem insulting his own gods.  On the other hand its archaic language and the breadth of mythological knowledge point to the pagan period, nor is the attack on the gods something the poet neccessarily agrees with; after all, Lock is punished by the most popular god of the Wiking Age, Thunder.}

\section{From Eagre and the Gods (\emph{Frá Ę́gi ok goðum})}

\bpg\bpa Ę́gir, er ǫðru nafni hét Gymir, hann hafði búit ásum ǫl þá er hann hafði fengit ketil inn mikla \edtrans{sem nú er sagt}{as was just told}{\Bfootnote{In immediately preceding \Hymiskvida.}}. Til þeirar veitslu kom Óðinn ok Frigg kona hans. Þórr kom eigi því at hann var í austr-vegi. Sif var þar, kona Þórs; Bragi, ok Iðunn kona hans. Týr var þar, hann var ein-hendr; Fenrisulfr sleit hǫnd af hánum, þá er hann var bundinn. Þar var Njǫrðr ok kona hans Skaði; Freyr ok Freyja; Víðarr son Óðins. Loki var þar, ok þjónustu-menn Freys, Byggvir ok Beyla. Mart var þar ása ok alfa.\epa

\bpb \inx[P]{Eagre}[{\huge E}agre], who by another name was called \inx[P]{Gymer}, he had prepared an ale-feast for the Eese when he had got the great kettle as was just told. To that gathering came \inx[P]{Weden} and \inx[P]{Frie} his wife. \inx[P]{Thunder} came not, for he was on the \inx[L]{Eastern Way}. Sib was there, Thunder’s wife; \inx[P]{Bray} and \inx[P]{Idun} his wife. \inx[P]{Tew} was there; he was one-handed; the \inx[P]{Fenrerswolf} tore his hand off when it was bound.\footnoteB{This detail is probably brought up to chronologically date the events of the poem as happening after the binding of Fenrer.} \inx[P]{Nearth} was there and his wife \inx[P]{Shede}; \inx[P]{Free} and \inx[L]{Frow}; \inx[P]{Wider} the son of \inx[P]{Weden}. \inx[P]{Lock} was there, and the servants of Free, \inx[P]{Bew} and \inx[P]{Beal}. A multitude of \inx[G]{Eese} and \inx[G]{Elves}\footnoteB{A formulaic expression, see \inx[F]{Eese and Elves}.} was there.\epb\epg


\bpg\bpa Ę́gir átti tvá þjónustu-menn, Fimafengr ok Eldir. Þar var lýsi-gull haft fyr elds-ljós; sjalft barsk þar ǫl. Þar var griða-stadr mikill. Menn lofuðu mjǫk hversu góðir þjónustu-menn Ę́gis vóru. Loki mátti eigi heyra þat, ok drap hann Fimafeng. Þá skóku ę́sir skjǫldu sína ok ǿptu at Loka, ok eltu hann braut til skógar, en þeir fóru at drekka. Loki hvarf aptr ok hitti úti Eldi; Loki kvaddi hann:\epa

\bpb Eagre had two servants, \inx[P]{Femfinger} and \inx[P]{Elder}. There glowing gold was used instead of fire; the ale there carried itself. It was a great \inx[C]{grith}-place.\footnoteB{A place wherein all violence was forbidden, see Index.} Men greatly praised how good the servants of Eagre were; Lock could not stand to hear it, and he slew Femfinger. Then the Eese shook their shields and screamed at Lock,\footnoteB{Some sort of ancient war dance. Cf. the Old Swedish Heathen Law: “He screams three nithing-screams TODO”.} and chased him away to the woods—but they went [back] to drink. Lock turned back and met Elder outside. Lock greeted him:\epb\epg

\sectionline

\section{The Flyting of Lock}

\bvg\bva%
„Sęg þú þat, \alst{Ę}ldir, \hld\ \edtext{svá’t \alst{ęi}nu-gi &
\ind \alst{f}eti gangir \alst{f}ramarr}{\lemma{svá’t \dots\ framarr ‘so that \dots\ further’}\Bfootnote{Shared with \Havamal\ 38.}}, &
hvat hér \alst{i}nni \hld\ \edtrans{hafa at \alst{ǫ}l-mǫ́lum}{they say over the ale}{\Bfootnote{Lit. “they have for their ale-speeches”.}} &
\ind \alst{s}ig-tíva \alst{s}ynir.“\eva

\bvb “{\huge T}ell this, Elder, so that thou not \\
\ind take one step further: \\
What here within they say over the ale, \\
\ind the sons of the victory-Tews \ken{gods}?”\evb\evg


\bvg\bva\speakernote{Ęldir:}%
„Of \alst{v}ǫ́pn sïn dǿma \hld\ ok of \alst{v}íg-risni sïna &
\ind \alst{s}ig-tíva \alst{s}ynir; &
\alst{ȧ}sa ok \alst{a}lfa, \hld\ es hér \alst{i}nni eru, &
\ind \edtrans{mann-gi ’s þér ï \alst{o}rði \alst{v}inr.}{none is thee a friend in words.}{\Bfootnote{I.e., “nobody says anything good about you.”

The alliteration here is notable, and also occurs in st. 10 (\emph{Víðarr} : \emph{ulfs}, see note there).  There are no signs of corruption, and so there are two possible explanations.  Either (1) the semi-vowel \emph{v} (\textipa{/w/}) is participating in vowel-alliteration with \emph{o}— such alliteration between \emph{v} and true vowels is never encountered in Scaldic poetry, but there are some examples from Eddic styles—or (2) the poem (or the relevant lines) was composed before the North Germanic loss of \emph{v} before rounded vowels.  (2) finds support in the notable fact that in both the present st. and st. 10 the words \emph{orð} ‘word’ and \emph{ulfr} ‘wolf’ originally began with \emph{v}; in the case of the word \emph{ulfr} this consonant is attested in old Scandinavian runic inscriptions.  For metrical reasons the lines must postdate the syncope of most unstressed short vowels, but on the basis of the three closely related C7th runestones from Blekinge (DR 357–359, from Stentoften, Gummarp, and Istaby) the loss of \emph{w} before rounded vowels is shown to have occurred later; so DR 359 \textbf{h\textsc{a}þuwulafʀ} \emph{Haþuwulᵃfʀ}.  If the alliteration indeed should fall on \emph{v}, this would not require dating the whole \Lokasenna\ to the late Proto-Norse period (indeed, according to the analysis done by \textcite{Sapp2022}, it is not even the linguistically oldest poem preserved); the older forms could, for instance, reflect archaic poetic formulae.

A C7th Proto-Norse form of this c-line might be: \emph{*mann-gí ’s þéʀ in worðé winiʀ}.}}“\eva

\bvb\speakernoteb{Elder quoth:}%
“Of their weapons they speak and of their battle-prowess, \\
\ind the sons of the victory-Tews \ken{gods}. \\
Of the Eese and Elves which are here within \\
\ind none is thee a friend in words.”\evb\evg


\bvg\bva\speakernote{Loki kvað:}%
„\alst{I}nn skal ganga \hld\ \alst{Ę́}gis hallir ï &
\ind ȧ þat \edtrans{\alst{s}umbl}{simble}{\Bfootnote{The Germanic word for “feast, banquet”.}} at \alst{s}éa, &
\edtrans{\alst{jǫ}ll ok \alst{ǫ́}fu}{scorn and hatred}{\Bfootnote{Two rare words to which the present translation hardly does justice.  The former occurs nowhere else, while the latter only otherwise occurs in \Sigurdskamma\ 33.  They have been interpreted in a variety of ways: \CV\ sees the first word as \emph{jóll} ‘wild angelica’, whereas the second is taken to be an error for \emph{áfr} (“a beverage [...] translated by Magnaeus by \emph{sorbitio avenacea}, a sort of common ale brewed of oats”).  TODO: What do other editors say? Esp. Kommentar.}} \hld\ fǿri’k \alst{ȧ}sa sonum &
\ind ok \edtext{blęnd’k þęim svá \alst{m}ęini \alst{m}jǫð}{\lemma{blęnd’k \dots\ męini mjǫð ‘I mix \dots\ the mead with harm’}\Bfootnote{Formulaic, cf. \Sigrdrifumal\ 8 (and others TODO).}}.“\eva

\bvb\speakernoteb{Lock quoth:}%
“I shall go into Eagre’s halls, \\
\ind on that \inx[C]{simble} for to see. \\
Scorn and hatred I bring the sons of the Eese, \\
\ind and I mix for them so the mead with harm.”\evb\evg


\bvg\bva\speakernote{Ęldir kvað:}%
„Vęitst, ef \alst{i}nn gęngr \hld\ \alst{Ę́}gis hallir ï &
\ind ȧ þat \alst{s}umbl at \alst{s}éa, &
\alst{h}rópi ok rógi \hld\ ef ęyss ȧ \alst{h}oll ręgin, &
\ind ȧ \alst{þ}ér munu þau \alst{þ}ęrra \alst{þ}at.“\eva

\bvb\speakernoteb{Elder quoth:}%
“Thou knowest if thou goest into Eagre’s halls, \\
\ind on that simble for to see— \\
if slander and strife thou pourest on the \inx[C]{hold} \inx[G]{Reins}, \\
\ind on \emph{thee} will they dry it off!”\evb\evg


\bvg\bva\speakernote{Loki kvað:}%
„Vęitst þat \alst{Ę}ldir, \hld\ ef \alst{ęi}nir skulum &
\ind \alst{s}ár-yrðum \alst{s}akask, &
\alst{au}ðigr verða \hld\ mun’k ï \alst{a}nd-svǫrum, &
\ind \edtrans{ef þú \alst{m}ę́lir til \alst{m}art!}{if thou speak too much!}{\Bfootnote{Formulaic; cf. \Havamal\ 27.}}“\eva

\bvb\speakernoteb{Lock quoth:}%
“Thou knowest that, Elder, if one-on-one we shall \\
\ind banter with wounding words, \\
wealthy will I grow in answers, \\
\ind if thou speak too much!”\evb\evg


\bpg\bpa Síðan gekk Loki inn í hǫllina; en er þeir sá, er fyrir váru, hverr inn var kominn, þǫgnuðu þeir allir.\epa

\bpb Thereafter Lock went into the hall, but when those who were there before him saw who was come inside, they all turned silent.\epb\epg


\bvg\bva\speakernote{Loki kvað:}%
„\alst{Þ}yrstr ek kom \hld\ \alst{þ}ęssar hallar til &
\ind \alst{L}optr of \alst{l}angan veg, &
\alst{ǫ̇}su at biðja, \hld\ at mér \alst{ęi}nn gefi &
\ind \edtrans{\alst{m}ę́ran drykk \alst{m}jaðar}{renowned drink of mead}{\Bfootnote{Formulaic language for describing mead; cf. \Havamal\ 105, 140, \Skirnismal\ 16. TODO: more parallels.}}.\eva

\bvb\speakernoteb{Lock quoth:}%
“Thirsty I came unto these halls, \\
\ind Loft \name{= Lock}, over a long way, \\
to bid the Eese that they give me but one \\
\ind renowned drink of mead.\evb\evg


\bvg\bva%
Hví \alst{þ}ęgið ér svá \hld\ \alst{þ}rungin goð, &
\ind at \alst{m}ę́la né \alst{m}ęguð; &
\edtext{\alst{s}essa ok staði \hld\ vęlið mér \alst{s}umbli at, &
\ind eða \alst{h}ęitið mik \alst{h}eðan!}{\lemma{sessa \dots\ heðan! ‘Choose \dots\ hence!’}\Bfootnote{That is, “Cease your dallying; give me a seat or tell me to leave!”}}“\eva

\bvb Why shut ye up so, ye pressed Gods, \\
\ind that ye cannot speak? \\
Choose seats and places for me at the simble, \\
\ind or call me away hence!”\evb\evg


\bvg\bva\speakernote{Bragi:}%
„\alst{S}essa ok staði \hld\ vęlja þér \alst{s}umbli at &
\ind \alst{ę̇}sir \alst{a}ldri-gi; &
því-at \alst{ę̇}sir vitu \hld\ \edtrans{hvęim \alst{a}lda}{which man}{\Bfootnote{Here “person, being”.  See note to \Vafthrudnismal\ 55/6.}} skulu &
\ind \edtrans{\alst{g}amban-sumbl}{gomben-simble}{\Bfootnote{\emph{gamban} ‘gomben’ being an obscure prefix which only occurs in \Lokasenna, \Skirnismal\ and \Harbardsljod.  \CV\ suggest it means something like “costly”.}} of \alst{g}eta.“\eva

\bvb\speakernoteb{Bray [quoth]:}%
“Choose seats and places for thee at the simble \\
\ind the Eese will never do, \\
for the Eese know for which man they shall \\
\ind prepare the gomben-simble.”\evb\evg


\bvg\bva\speakernote{[Loki:]}%
\edtext{„Mant þat \alst{Ó}ðinn, \hld\ es vit ï \alst{á}r-daga &
\ind \alst{b}lendum \alst{b}lóði saman? &
\alst{ǫ}lvi bęrgja \hld\ létsk \alst{ęi}gi mundu, &
\ind nema okkr vę́ri \alst{b}ǫ́ðum \alst{b}orit.“}{\lemma{All}\Bfootnote{Lock turns to Weden, chief of the Eese, and reminds him of an oath of blood-brotherhood the two had undertaken in the early days of the world.  The circumstances of the oath between them are otherwise entirely unknown.}}\eva

\bvb\speakernoteb{[Lock quoth:]}%
“Recallest thou, Weden, when we two in days of yore \\
\ind blended our blood together? \\
Taste ale wouldst thou never do, \\
\ind unless it were for us both borne forth!”\evb\evg


\bvg\bva\speakernote{[Óðinn:]}%
„\edtrans{Rís þȧ \alst{V}íðarr \hld\ ok lát \alst{u}lfs fǫður}{Rise thou, Wider, and let the Wolf’s father \ken*{= Lock}}{\Bfootnote{For the alliteration see note to st. 2.  A C7th Proto-Norse form of the line might be: \emph{*Rís þan Wíðarʀ · auk lát wulfs fǫður}.}} &
\ind \alst{s}itja \alst{s}umbli at, &
síðr oss \alst{L}oki \hld\ kvęði \alst{l}asta-stǫfum &
\ind \alst{Ę́}gis hǫllu \alst{ï}.“\eva

\bvb\speakernoteb{[Weden quoth:]}%
“Then rise, O Wider, and let the Wolf’s father \ken*{= Lock} \\
\ind sit at the simble, \\
lest Lock should greet us with words of vice \\
\ind in Eagre’s hall.”\evb\evg


\bpg\bpa Þá stóð Víðarr upp ok skenkti Loka, en áðr hann drykki, kvaddi hann ásuna:\epa

\bpb Then Wider stood up and poured a drink to Lock, but before he \ken*{= Lock} drank, he greeted the Eese:\epb\epg


\bvg\bva%
„Hęilir \alst{ę̇}sir, \hld\ hęilar \alst{ǫ̇}synjur &
\ind ok ǫll \alst{g}inn-hęilǫg \alst{g}oð, &
nema sá \alst{ęi}nn \alst{ǫ̇}ss \hld\ es \alst{i}nnar sitr &
\ind \alst{B}ragi \alst{b}ękkjum ȧ.“\eva

\bvb “Hail the \inx[G]{Eese}! Hail the \inx[G]{Ossens}, \\
\ind and all \inx[C]{yin-holy} Gods!\footnoteB{The first two half-lines are identical to the prayer in \Sigrdrifumal\ 3–4.  The prayer formula may actually have been used in Heathen toasts, where the second half of the stanza was used to ask for a boon.  Lock subverts it by instead insulting one of the gods present, which would have come off as blasphemous to the Heathen audience.} \\
Save for that one \inx[G]{Eese}[os] who sits further within: \\
\ind Bray, on the benches.”\evb\evg


\bvg\bva\speakernote{[Bragi] kvað:}%
„\edtrans{\alst{M}ar ok \alst{m}ę́ki}{Steed and sword}{\Bfootnote{Formulaic pair; see \Havamal\ 83/2.}} \hld\ gef’k þér \alst{m}ïns féar &
\ind ok \alst{b}ǿtir þér svá \alst{b}augi \alst{B}ragi, &
síðr þú \alst{ǫ̇}sum \hld\ \alst{ǫ}fund of gjaldir; &
\ind \alst{g}ręm þú ęigi \alst{g}oð at þér!“\eva

\bvb\speakernoteb{{[Bray]} quoth:}%
“Steed and sword I give thee of my own wealth, \\
\ind and so restores thee Bray with a \inx[C]{bigh}, \\
lest thou repay the Eese with envy; \\
\ind anger not the Gods against thee!”\evb\evg


\bvg\bva\speakernote{[Loki] kvað:}%
„\alst{Jó}s ok \alst{a}rm-bauga \hld\ munt \alst{ę́} vesa &
\ind \alst{b}ęggja vanr \alst{B}ragi, &
\alst{ȧ}sa ok \alst{a}lfa, \hld\ es hér \alst{i}nni eru, &
\ind þú est við \alst{v}íg \alst{v}arastr, &
\ind ok \alst{sk}jarrastr við \alst{sk}ot.“\eva

\bvb\speakernoteb{{[Lock]} quoth:}%
“Of steed and arm-bighs both wilt thou always be \\
\ind lacking both, O Bray! \\
Of the Eese and Elves which are here within, \\
\ind thou art with war wariest \\
\ind and shiest with shot.”\evb\evg


\bvg\bva\speakernote{[Bragi] kvað:}%
„Vęit’k, ef fyr \alst{ú}tan vę́ra’k, \hld\ svá sem fyr \alst{i}nnan em’k, &
\ind \alst{Ę́}gis hǫll \alst{o}f kominn, &
\alst{h}ǫfuð þitt \hld\ bę́ra’k ï \alst{h}ęndi mér; &
\ind\edtext{\alst{l}ít’k þér þat fyr \alst{l}ygi}{\Bfootnote{\emph{‘litt ec þer þat fyr lygi’} \Regius.  A variety of emendations have been proposed for this line. Simplest would be \emph{lítt es þér þat fyr lygi} ‘that is little [punishment] for thee for lying’. Based on the similarity of \emph{ꞇ̇} (= \emph{tt}) and \emph{c} \textcite{FinnurEdda} gives \emph{lykak þér þat fyr lygi} ‘so I would bring to thee for thy lie’.}}.“\eva

\bvb\speakernoteb{{[Bray]} quoth:}%
“I know if outside I were as inside I am \\
\ind come into Eagre’s hall,\footnoteB{As said in P1, the rule of \inx[C]{grith} (a truce of non-violence, even between enemies; see Index) applied inside the hall.  Bray and the other gods are thus bound not to injure Lock.} \\
that head on thee would I bear in my hands; \\
\ind this I see for thy lie.”\evb\evg


\bvg\bva\speakernote{[Loki] kvað:}%
„\alst{S}njallr est ï \alst{s}essi, \hld\ skal-at-tu \alst{s}vá gęra, &
\ind \alst{B}ragi \alst{b}ękk-skrautuðr; &
\alst{v}ega þú gakk \hld\ ef \alst{v}ręiðr séir; &
\ind \alst{h}yggsk vę́tr \alst{h}vatr fyrir.“\eva

\bvb\speakernoteb{{[Lock]} quoth:}%
“Valiant art thou in the seat; thou shalt not do so, \\
\ind O Bray the bench-adorner! \\
Go to fight if thou art wroth; \\
\ind the bold thinks not ahead.\footnoteB{Lock attacks Bray’s excuse; a true brave would fight regardless of the grith.}”\evb\evg


\bvg\bva\speakernote{[Iðunn] kvað:}%
„\alst{B}ið ek, \alst{B}ragi, \hld\ \alst{b}arna sifjar duga &
\ind ok allra \alst{ȯ}sk-maga, &
at þú \alst{L}oka \hld\ kveðir-a \alst{l}asta-stǫfum &
\ind \alst{Ę́}gis hǫllu \alst{ï}.“\eva

\bvb\speakernoteb{{[Idun]} quoth:}%
“I bid thee, Bray, to respect the bond of children \\
\ind and all beloved sons, \\
that thou not greet Lock with words of vice \\
\ind in Eagre’s hall.”\evb\evg


\bvg\bva\speakernote{[Loki] kvað:}%
„Þęgi þú, \alst{I}ðunn, \hld\ þik kveð’k \alst{a}llra kvinna &
\ind \alst{v}er-gjarnasta \alst{v}esa &
síðst þú \alst{a}rma þïna \hld\ lagðir \alst{í}tr-þvęgna &
\ind umb þinn \alst{b}róður-\alst{b}ana.“\eva

\bvb\speakernoteb{{[Lock]} quoth:}%
“Shut thou up, Idun! Thee I call of all women \\
\ind the most man-eager, \\
since thy brightly washed arms thou didst cast \\
\ind about thy brother’s bane.”\evb\evg


\bvg\bva\speakernote{[Iðunn] kvað:}%
„\alst{L}oka ek kveð’k-a \hld\ \alst{l}asta-stǫfum &
\ind \alst{Ę́}gis hǫllu \alst{ï}; &
\alst{B}raga ek kyrri \hld\ \alst{b}jór-ręifan, &
\ind \alst{v}il’k-at at it \alst{v}ręiðir \alst{v}egisk.“\eva

\bvb\speakernoteb{{[Idun]} quoth:}%
“I greet not Lock with words of vice, \\
\ind in Eagre’s hall. \\
Bray I calm, made rowdy from beer— \\
\ind I wish not that ye two wroth ones should fight.”\evb\evg


\bvg\bva\speakernote{[Gęfjun] kvað:}%
„Hví it \alst{ę̇}sir tvęir \hld\ skuluð \alst{i}nni hér &
\ind \alst{s}ár-yrðum \alst{s}akask? &
\alst{L}opts-ki þat vęit \hld\ at hann \alst{l}ęikinn es &
\ind ok hann \alst{f}jǫrg-vall \alst{f}ría.”\eva

\bvb\speakernoteb{{[Giben]} quoth:}
“Why shall ye two Eese here within, \\
\ind with wound-words each other blame? \\
Loft \name{= Lock} knows not that he is being played, \\
\ind and him TODO.”\evb\evg


\bvg\bva\speakernote{[Loki] kvað:}%
„\alst{Þ}ęgi þú, Gęfjun, \hld\ \alst{þ}ęss mun’k nú geta &
\ind es þik \alst{g}lapði at \alst{g}ęði: &
\alst{s}vęinn inn hvíti \hld\ es þér \alst{s}igli gaf &
\ind ok þú \alst{l}agðir \alst{l}ę́r yfir.“\eva

\bvb\speakernoteb{{[Lock]} quoth:}%
“Shut thou up, Giben! Of \emph{him} will I now speak, \\
\ind who seduced thy senses: \\
the white swain who gave thee a necklace, \\
\ind and thou cast o’er him thy leg!”\evb\evg


\bvg\bva\speakernote{[Óðinn kvað] þat:}%
„\edtext{\alst{Ǿ}rr est, Loki, \hld\ ok \alst{ø}r-viti,}{\lemma{Ǿrr \dots\ ok ør-viti ‘Mad \dots\ and out of wits’}\Bfootnote{Formulaic, occurs at two other places (TODO). Cf. also st. 47 below.}} &
\ind es þú fę̇r þér \alst{G}ęfjun at \alst{g}ręmi &
því-at \alst{a}ldar \alst{ø}r-lǫg \hld\ hygg at \alst{ǫ}ll of viti &
\ind \alst{ja}fn-gǫrla sem \alst{e}k.“\eva

\bvb\speakernoteb{{[Weden quoth]} this:}%
“Mad art thou, Lock, and out of wits, \\
\ind as thou earnest Giben’s anger against thee, \\
for all the orlays of men I think she knows, \\
\ind just as clearly as I.”\evb\evg


\bvg\bva\speakernote{[Loki] kvað:}%
„Þęgi þú, \alst{Ó}ðinn, \hld\ þú kunnir \alst{a}ldri-gi &
\ind dęila \alst{v}íg með \alst{v}erum; &
opt þú \alst{g}aft \hld\ þęim’s \alst{g}efa skyldir-a, &
\ind inum \alst{s}lę́vurum, \alst{s}igr.“\eva

\bvb\speakernoteb{{[Lock]} quoth:}%
“Shut thou up, Weden! Thou couldst never \\
\ind deal out war amidst men— \\
oft hast thou given them thou shouldst not have given, \\
\ind the slower men, victory.”\evb\evg


\bvg\bva\speakernote{[Óðinn] kvað:}%
„Vęitst ef ek \alst{g}af \hld\ þęim’s \alst{g}efa né skylda, &
\ind inum \alst{s}lę́vurum, \alst{s}igr, &
\alst{á}tta vetr \hld\ vast fyr \alst{jǫ}rð neðan &
\ind \edtrans{\alst{k}ýr mólkandi}{a milch cow}{\Bfootnote{May also be read as “milking cows”, the nom. sg. \emph{kýr} being identical to the nom./acc. pl. \emph{kýr}, and \emph{mólka} meaning both ‘to milk’ and ‘to give milk’.  “Milch cow” is preferable for two reasons, viz. (i) that the phrase is followed by \emph{ok kona} ‘and a woman’ rather than \emph{sem kona} ‘as a woman’ or similar, and (ii) that it agrees with another instance where Lock is gives birth in the form of a female animal (cows, of course, only giving milk after calving), namely the episode of the building of the wall around Osyard as told in \Gylfaginning\ 42.}} ok \alst{k}ona &
\ind ok hęfir þar \alst{b}ǫrn of \alst{b}orit &
\ind ok hugða’k þat \alst{a}rgs \alst{a}ðal.“\eva

\bvb\speakernoteb{{[Weden]} quoth:}%
“Thou knowest, that if I have given them I should not have given, \\
\ind the slower men, victory; \\
for eight winters wast thou beneath the earth \\
\ind a milch cow and a woman, \\
\ind and thou hast there borne children, \\
\ind and I’ve judged that a \inx[C]{queer}’s nature.”\evb\evg


\bvg\bva\speakernote{[Loki] kvað:}%
„En þik \alst{s}íga kóðu \hld\ \alst{S}ȧms-ęyju ï &
\ind ok drapt ȧ \alst{v}ett sem \alst{v}ǫlur, &
\alst{v}itka líki \hld\ fórt \alst{v}er-þjóð yfir, &
\ind ok hugða’k þat \alst{a}rgs \alst{a}ðal.“\eva

\bvb\speakernoteb{{[Lock]} quoth:}%
“But thou, they said, didst sink down in Samsey, \\
\ind and beatest the drum like do wallows. \\
In a warlock’s likeness thou didst journey through mankind, \\
\ind and I’ve judged \emph{that} a queer’s nature.”\evb\evg


\bvg\bva\speakernote{[Frigg kvað:]}%
„\alst{Ø}r-lǫgum \alst{y}kkrum \hld\ skylið \alst{a}ldri-gi &
\ind \alst{s}ęgja \alst{s}ęggjum frȧ, &
hvat it \alst{ę̇}sir tvęir \hld\ drýgðuð ï \alst{á}r-daga; &
\ind \alst{f}irrisk ę́ \alst{f}orn rǫk \alst{f}irar.“\eva

\bvb\speakernoteb{[Frie quoth:]}%
“Of your orlays should ye two never \\
\ind speak to the youths. \\
Whatever ye two Eese did in days of yore, \\
\ind let ancient fates be ever shunned by folk.”\evb\evg


\bvg\bva\speakernote{[Loki kvað:]}%
„Þęgi þú, \alst{F}rigg, \hld\ þú est \alst{F}jǫrgyns mę́r &
\ind ok hęfir ę́ \alst{v}er-gjǫrn \alst{v}esit, &
es þȧ \alst{V}éa ok Vilja \hld\ létst þér, \alst{V}iðris kvę̇n, &
\ind \alst{b}áða ï \alst{b}aðm of tękit.“\eva

\bvb\speakernoteb{[Lock quoth:]}%
“Shut thou up, Frie! Thou art Firgyn’s maiden, \\
\ind and has always been man-eager: \\
as [when] Wigh and Will, thou hadst, O Withrer’s wife, \\
\ind both in thy bosom taken.”\evb\evg


\bvg\bva\speakernote{[Frigg kvað:]}%
„Vęitst ef \alst{i}nni \alst{ę́}tta’k \hld\ \alst{Ę́}gis hǫllum \alst{ï} &
\ind \alst{B}aldri líkan \alst{b}ur &
\alst{ú}t né kvę̇mir \hld\ frȧ \alst{ȧ}sa sonum &
\ind ok vę́ri þȧ at þér \alst{v}ręiðum \alst{v}egit.“\eva

\bvb\speakernoteb{[Frie quoth:]}%
“Thou knowest, if within I owned, in Eagre’s halls, \\
\ind a boy alike to Balder: \\
out came thou not from the sons of the Eese, \\
\ind and thou wouldst be fought with wrath.”\evb\evg


\bvg\bva\speakernote{[Loki kvað:]}%
„Ęnn vill þú, \alst{F}rigg, \hld\ at ek \alst{f}lęiri tęlja &
\ind \alst{m}ïna \alst{m}ęin-stafi: &
ek því \alst{r}éð \hld\ es þú \alst{r}íða sér-at &
\ind \alst{s}íðan Baldr at \alst{s}ǫlum.“\eva

\bvb\speakernoteb{[Lock quoth:]}
“Still wilt thou, Frie, that I count more \\
\ind of my harmful deeds: \\
I did plan that thou shouldst not see Balder \\
\ind riding to the halls henceforth.”\evb\evg


\bvg\bva\speakernote{[Fręyja kvað:]}%
„\alst{Ǿ}rr est, Loki, \hld\ es þú \alst{y}ðra tęlr &
\ind \alst{l}jóta \alst{l}ęið-stafi; &
\alst{ø}r-lǫg Frigg \hld\ hygg at \alst{ǫ}ll viti &
\ind þótt hǫ̇n \alst{s}jǫlf-gi \alst{s}ęgi.“\eva

\bvb\speakernoteb{[Frow quoth:]}
“Mad art thou, Lock, when thou dost count \\
\ind your ugly, loathsome deeds: \\
all orlays I think that Frie might know, \\
\ind though she tell them not herself.”\evb\evg


\bvg\bva\speakernote{[Loki kvað:]}%
„Þęgi þú, \alst{F}ręyja, \hld\ þik kann’k \alst{f}ull-gørva; &
\ind es-a þér \edtrans{\alst{v}amma \alst{v}ant}{free of blemishes}{\Bfootnote{Formulaic, cf. \Havamal\ 22/4: \emph{hann es-a vamma vanr} ‘he is not free of blemishes’.}}: &
\alst{ȧ}sa ok \alst{a}lfa, \hld\ es hér \alst{i}nni eru, &
\ind \alst{h}vęrr \alst{h}ęfir þinn \alst{h}ór vesit.“\eva

\bvb\speakernoteb{[Lock quoth:]}
“Shut thou up, Frow! I know thee full well— \\
\ind thou art not free of blemishes: \\
of the Eese and Elves which are here within \\
\ind has each one been thy lover!”\evb\evg


\bvg\bva\speakernote{[Fręyja kvað:]}%
\edtext{„\alst{F}lǫ́ ’s þér tunga, \hld\ hygg at þér \alst{f}ręmr myni &
\ind ȯ·\alst{g}ótt of \alst{g}ala;}{\lemma{Flǫ́ \dots\ gala ‘False \dots\ thee’}\Bfootnote{The language is again strikingly similar to \Havamal, particularly 29/3–4 and 116/3–4.}} &
vręiðir ’ru þér \alst{ę̇}sir \hld\ ok \alst{ǫ̇}synjur, &
\ind \edtrans{\alst{h}ryggr munt \alst{h}ęim fara}{grieved wilt thou journey home}{\Bfootnote{Frow here shows her ability to foresee the future.  Lock will come to regret his insults.}}.“\eva

\bvb\speakernoteb{[Frow quoth:]}
“False is thy tongue, I ween that it henceforth will \\
\ind sing evil [into being] for thee. \\
Wroth with thee are the Eese and Ossens: \\
\ind grieved wilt thou journey home.”\evb\evg


\bvg\bva\speakernote{Loki:}%
„Þęgi þú, \alst{F}ręyja, \hld\ þú est \alst{f}or-dę́ða &
\ind ok \alst{m}ęini blandin \alst{m}jǫk, &
síðst-u at \alst{b}rǿðr þïnum \hld\ siðu \alst{b}líð ręgin &
\ind ok myndir þȧ, \alst{F}ręyja, \alst{f}rata.“\eva

\bvb\speakernoteb{Lock [quoth]:}
“Shut thou up, Frow! Thou art an evil-working woman, \\
\ind and much mixed with harm, \\
since against thy brother the blithe Reins bewitched thee, \\
\ind and thou wouldst then, O Frow, fart.”\evb\evg


\bvg\bva\speakernote{Njǫrðr:}%
„Þat ’s \alst{v}á-lítit \hld\ þótt sér \alst{v}arðir \alst{v}ers fȧi, &
\ind \alst{h}ós eða \alst{h}várs; &
hitt ’s \alst{u}ndr, es \alst{ȧ}ss ragr \hld\ es hér \alst{i}nn of kominn &
\ind ok hęfir sá \alst{b}ǫrn of \alst{b}orit.“\eva

\bvb\speakernoteb{Nearth [quoth]:}
“It is little woe that women should get themselves a man, \\
\ind a lover or whomever else. \\
This is a wonder, that a queer os is come here within, \\
\ind and that man has born children!”\evb\evg


\bvg\bva\speakernote{Loki:}%
„\alst{Þ}ęgi þú, Njǫrðr, \hld\ \alst{þ}ú vast austr heðan &
\ind \alst{g}ísl of sęndr at \alst{g}oðum; &
\alst{H}ymis meyjar \hld\ hǫfðu þik at \alst{h}land-trogi &
\ind ok þér ï \alst{m}unn \alst{m}igu.“\eva

\bvb\speakernoteb{Lock [quoth]:}%
“Shut thou up, Nearth! Thou wast east hence \\
\ind sent as hostage for the Gods. \\
Hymer’s maidens had thee for a lant-trough, \\
\ind and pissed thee in the mouth!”\evb\evg


\bvg\bva\speakernote{Njǫrðr:}%
„Sú esumk \alst{l}íkn \hld\ es vas’k \alst{l}angt heðan &
\ind \alst{g}ísl of sęndr at \alst{g}oðum: &
þȧ ek \edtext{\alst{m}ǫg gat \hld\ þann’s \alst{m}ann-gi fíar}{\lemma{mǫg \dots\ þann’s mann-gi fíar ‘the lad whom no man hates’}\Bfootnote{Free.}}, &
\ind ok þikkir sá \alst{ȧ}sa \alst{ja}ðarr.“\eva

\bvb\speakernoteb{Nearth [quoth]:}%
“This is my relief, as I was far-away hence \\
\ind sent as hostage for the Gods, \\
when I begot the lad whom no man hates \\
\ind and he seems the peak of the Eese.”\evb\evg


\bvg\bva\speakernote{Loki:}%
„\alst{H}ę́tt-u nú, Njǫrðr, \hld\ haf ȧ \alst{h}ófi þik; &
\ind mun’k-a því \alst{l}ęyna \alst{l}ęngr: &
við \alst{s}ystur þinni \hld\ gatst \alst{s}líkan mǫg, &
\ind ok es-a þó \alst{ȯ}nu \alst{v}err.“\eva

\bvb\speakernoteb{Lock [quoth]:}
“Stop now, Nearth; restrain thyself! \\
\ind I will no longer hide it: \\
by thy sister didst thou beget such a lad, \\
\ind and there can be expected nothing worse.”\evb\evg


\bvg\bva\speakernote{Týr:}%
„Fręyr ’s \alst{b}ętstr \hld\ allra \alst{b}all-riða &
\ind \alst{ȧ}sa gǫrðum \alst{ï}; &
\alst{m}ęy né grǿtir \hld\ né \alst{m}anns konu, &
\ind ok lęysir ór \alst{h}ǫptum \alst{h}vęrn.“\eva

\bvb\speakernoteb{Tew [quoth]:}%
“Free is the best of all bold riders \\
\ind in the yards of the Eese; \\
he makes no maiden cry, nor any man’s woman, \\
\ind and loosens anyone from his bonds!”\evb\evg


\bvg\bva\speakernote{Loki:}%
„\alst{Þ}ęgi þú, Týr, \hld\ \alst{þ}ú kunnir aldri-gi &
\ind \edtrans{bera \alst{t}ilt með \alst{t}vęim}{settle strife among two}{\Bfootnote{Uncertain. TODO.}}; &
\alst{h}andar ennar \alst{h}ǿgri \hld\ mun’k \alst{h}innar geta &
\ind es þér slęit \alst{F}ęnrir \alst{f}rȧ.“\eva

\bvb\speakernoteb{Lock [quoth]:}%
“Shut thou up, Tew! \emph{Thou} couldst never \\
\ind settle strife among two; \\
of the right hand I next will speak, \\
\ind which from thee Fenrer tore.”\evb\evg


\bvg\bva\speakernote{Týr:}%
„\alst{H}andar em’k vanr \hld\ en þú \alst{h}róðrs vitnis; &
\ind \alst{b}ǫl es \alst{b}ęggja þráa; &
ulf-gi hęfir ok vel \hld\ es ï bǫndum skal &
\ind bíða \alst{r}agna \alst{r}økrs.“\eva

\bvb\speakernoteb{Tew [quoth]:}%
“A hand am I lacking, but thou the Famous Wolf; \\
\ind both yearnings are a bale! \\
Nor does the Wolf have it well, who in bonds shall \\
\ind await the Twilight of the Reins.”\evb\evg


\bvg\bva\speakernote{Loki:}%
„\alst{Þ}ęgi þú, Týr, \hld\ \alst{þ}at varð þinni konu &
\ind at hon átti \alst{m}ǫg við \alst{m}ér! &
\edtrans{\alst{Ǫ}ln}{ell}{\Bfootnote{Wool, measured in ells, was often used for barter in Iceland and Norway.}} né pęnning \hld\ hafðir þess \alst{a}ldri-gi &
\ind \alst{v}an-réttis, \alst{v}ę-sall.“\eva

\bvb\speakernoteb{Lock [quoth]:}%
“Shut thou up, Tew! It happened to thy woman, \\
\ind that she had a lad by me! \\
Neither ell nor penny hadst thou ever for that \\
\ind injustice, O wretch!”\evb\evg


\bvg\bva\speakernote{Fręyr:}%
„\alst{U}lf sé’k liggja \hld\ \alst{á}ar ósi fyr &
\ind unds \alst{r}júfask \alst{r}ęgin; &
því munt \alst{n}ę́st, \hld\ nema \alst{n}ú þęgir, &
\ind \alst{b}undinn, \alst{b}ǫlva smiðr!“\eva

\bvb\speakernoteb{Free [quoth]:}%
“The Wolf I see lying before a river-mouth, \\
\ind until the Reins are ripped; \\
therefore wilt thou next—unless thou now shut up— \\
\ind be bound, O smith of bales!”\evb\evg


\bvg\bva\speakernote{Loki:}%
„\alst{G}ulli kęypta \hld\ létst \alst{G}ymis dóttur &
\ind ok \alst{s}ęldir þitt \alst{s}vá \alst{s}verð, &
en es \alst{M}úspells synir \hld\ ríða \alst{M}yrk-við yfir &
\ind \alst{v}ęitst-a þȧ, \alst{v}ę-sall, hvé \alst{v}egr!“\eva

\bvb\speakernoteb{Lock [quoth]:}%
“Bought with gold thou hadst Gymer’s daughter \ken*{= Gird}, \\
\ind and didst so sell thy sword, \\
but when Muspell’s sons ride over Mirkwood \\
\ind knowest thou not, O wretch, how to fight!”\evb\evg


\bvg\bva\speakernote{Byggvir:}%
„Vęitst ef \alst{ø}ðli \alst{ę́}tta’k \hld\ sem \alst{I}ngunar-Fręyr, &
\ind ok \alst{s}vá \alst{s}ę́l-ligt \alst{s}etr: &
\alst{m}ęrgi smę́ra \hld\ \alst{m}ølða’k þȧ \alst{m}ęin-krǫ́ku &
\ind ok \alst{l}ęmða alla ï \alst{l}iðu.“\eva

\bvb\speakernoteb{Bewer [quoth]:}%
“Thou knowest, if I had a pedigree like Ingwin-Free, \\
\ind and such blessed pasture— \\
smaller than bone meal would I mill this harm-crow, \\
\ind and beat all his limbs lame!”\evb\evg


\bvg\bva\speakernote{Loki:}%
„Hvat ’s þat it \alst{l}itla \hld\ es þat \alst{l}ǫggra sé’k &
\ind ok \alst{s}nap-víst \alst{s}napir? &
At \alst{ęy}rum Fręys \hld\ munt \alst{ę́} vesa &
\ind ok und \alst{k}vęrnum \alst{k}laka.“\eva

\bvb\speakernoteb{Lock [quoth]:}%
“What is this little thing I see crawling \\
\ind and snap-wisely snapping? \\
At the ears of Free wilt thou ever be, \\
\ind and chirping under mills!”\evb\evg


\bvg\bva\speakernote{[Byggvir kvað:]}%
„\alst{B}yggvir ek hęiti, \hld\ en mik \alst{b}ráðan kveða &
\ind \edtext{\alst{g}oð ǫll ok \alst{g}umar}{\lemma{goð \dots\ ok gumar ‘Gods and men’}\Bfootnote{This pairing also occurs in \Lokasenna\ 55/4 and \Reginsmal\ 19.}}; &
því em’k \alst{h}ér \alst{h}róðugr \hld\ at drekka \alst{H}ropts męgir &
\ind \alst{a}llir \alst{ǫ}l saman.“\eva

\bvb\speakernoteb{[Bewer quoth:]}%
“Bewer I am called, and hurried do call me \\
\ind all the Gods and men; \\
therefore I am here honoured that Roft’s lads \ken*{the \textsc{eese}} drink \\
\ind ale all together.”\evb\evg


\bvg\bva\speakernote{[Loki kvað:]}%
„\alst{Þ}ęgi þú, Byggvir, \hld\ \alst{þ}ú kunnir aldri-gi &
\ind dęila með \alst{m}ǫnnum \alst{m}at; &
ok þik ï \alst{f}lęts strá \hld\ \alst{f}inna né mǫ́ttu &
\ind þȧ’s \alst{v}ǫ́gu \alst{v}erar.“\eva

\bvb\speakernoteb{[Lock quoth:]}
“Shut thou up, Bewer! Thou couldst never \\
\ind deal out food amidst men, \\
and in the bench-straw they could not find thee, \\
\ind whenever men did fight.”\evb\evg


\bvg\bva\speakernote{[Hęimdallr kvað:]}%
„\alst{Ǫ}lr est, Loki \hld\ svá’t es \alst{ø}r-viti, &
\ind hví né \alst{l}ętsk-a þú, \alst{L}oki? &
því-at \alst{o}f-drykkja \hld\ vęldr \alst{a}lda hvęim &
\ind es sïna \alst{m}ę́lgi né \alst{m}an-at.“\eva

\bvb\speakernoteb{[Homedal quoth:]}%
“Drunk art thou, Lock, so that thou art out of wits; \\
\ind why holdest thou not back, Lock? \\
For over-drinking makes every man \\
\ind no more recall his speech.”\evb\evg


\bvg\bva\speakernote{[Loki kvað:]}%
„\alst{Þ}ęgi þú, Hęimdallr, \hld\ \alst{þ}ér vas ï ár-daga &
\ind it \alst{l}jóta \edtrans{\alst{l}íf of \alst{l}agit}{life laid [down]}{\Bfootnote{His course of life was decreed (by the Norns).  Formulaic; see TODO.}}; &
\alst{ǫ}rgu baki \hld\ munt \alst{ę́} vesa &
\ind ok \alst{v}aka \edtrans{\alst{v}ǫrðr goða}{Watchman of the Gods}{\Bfootnote{Formulaic epithet of Homedal, who had to guard the rainbow bridge of the Gods against their enemies.  See note to \Grimnismal\ 13.}}.“\eva

\bvb\speakernoteb{[Lock quoth:]}%
“Shut thou up, Homedal! For \emph{thee} in days of yore \\
\ind thy ugly life was laid [down]. \\
With a stiff back wilt thou ever be \\
\ind and waking, O Watchman of the Gods.”\evb\evg


\bvg\bva\edtext{\speakernote{[Skaði kvað:]}}{\lemma{[Skaði kvað:] ‘[Shede quoth:]’}\Bfootnote{The speaker of sts. 49 and 51 is not indicated anywhere, but is almost certainly Shede for both.  Lock’s mention of Thedse’s slaying in 50 (see Note) is only effective if it relates personally to whomever he is attacking, and this is only the case for Shede.  This also explains her answer in 51.  Further, since Shede is explicitly mentioned in P1, she should be expected to have a speaking role in the poem.}}%
„\alst{L}ėtt ’s þér, Loki; \hld\ mun-at-tu \alst{l}ęngi svá &
\ind \alst{l}ęika \alst{l}ausum hala, &
\edtext{því at þik ȧ \alst{h}jǫrvi skulu \hld\ ins \alst{h}rïm-kalda magar &
\ind \alst{g}ǫrnum binda \alst{g}oð.}{\lemma{því at þik ȧ hjǫrvi skulu \hld\ ins hrïm-kalda magar / gǫrnum binda goð. ‘for on a sword with thy rime-cold lad’s / guts the Gods shall bind thee’}\Bfootnote{See \FraLoka\ below.}}“\eva

\bvb\speakernoteb{[Shede quoth:]}%
“Thou takest it lightly, Lock—thou wilt not so for long \\
\ind play with a loose tail, \\
for on a sword with thy rime-cold lad’s \\
\ind guts the Gods shall bind thee.”\evb\evg


\bvg\bva\speakernote{[Loki kvað:]}%
„Vęitst ef mik ȧ \alst{h}jǫrvi skulu \hld\ ins \alst{h}rïm-kalda magar &
\ind \alst{g}ǫrnum binda \alst{g}oð, &
\alst{f}yrstr ok øfstr \hld\ vas’k at \alst{f}jǫr-lagi &
\ind \edtrans{\alst{þ}ar’s vér ȧ \alst{Þ}jatsa \alst{þ}rifum}{where we laid hands on Thedse}{\Bfootnote{A reference to a longwinded myth told most fully in \Skaldskaparmal\ 2–4 and \Haustlong\ 2–13.  After Thedse abducted \inx[P]{Idun} the Eese made Lock recover her, which he set out to do by flying to Thedse’s farm in the shape of a hawk.  When he found Idun he turned her into a nut, took her in his claws, and turned back to Osyard.  Thedse quickly spotted him, set chase in the form of an eagle, and was soon closing the distance.  The Eese within Osyard saw this and hurriedly threw wood shavings on the ground; just as Lock had passed above them they set fire to the shavings; the fire rose and burned the wings of Thedse, who fell down to the ground and was soon killed.  After this, Shede, Thedse’s daughter, came to Osyard to avenge her father, but the gods convinced her to a settlement, after which she married Nearth and became one of them.  It is most sensible that Lock brings this myth up in order to insult Shede.}}.“\eva

\bvb\speakernoteb{[Lock quoth:]}%
“Thou knowest, if on a sword with my rime-cold lad’s \\
\ind guts the Gods shall bind me, \\
first and highest was I in life-taking \\
\ind where we laid hands on \inx[P]{Thedse}.”\evb\evg


\bvg\bva\speakernote{[Skaði kvað:]}%
„Vęitst ef \alst{f}yrstr ok øfstr \hld\ vast at \alst{f}jǫr-lagi &
\ind \alst{þ}ȧ’s ér ȧ \alst{Þ}jatsa \alst{þ}rifuð, &
frȧ mïnum \alst{v}éum \hld\ ok \alst{v}ǫngum skulu &
\ind þér ę́ \alst{k}ǫld rǫ́ð \alst{k}oma.“\eva

\bvb\speakernoteb{[Shede quoth:]}%
“Thou knowest, if first and highest thou wast in life-taking \\
\ind where ye laid hands on Thedse: \\
from my wighs and wongs shall for thee \\
\ind ever cold counsels come.”\evb\evg


\bvg\bva\speakernote{[Loki kvað:]}%
„\alst{L}ėttari ï mǫ́lum \hld\ vast við \alst{L}aufęyjar son &
\ind þȧ’s létsk mér ȧ \alst{b}ęð þinn \alst{b}oðit; &
\alst{g}etit verðr oss slíks \hld\ ef vér \alst{g}ǫrva skulum &
\ind tęlja \alst{v}ǫmmin \alst{v}ǫ́r.“\eva

\bvb\speakernoteb{[Lock quoth:]}%
“Lighter in speech wast thou with Leafie’s son \ken*{= Lock = me} \\
\ind when thou hadst me bid to thy bed; \\
such will be said of us, if we clearly shall \\
\ind recount our blemishes.\evb\evg


\bpg\bpa Þá gekk Sif fram ok byrlaði Loka í hrím-kalki mjǫð ok mę́lti:\epa

\bpb Then Sib walked forth and poured for Lock mead in a \inx[C]{rime-chalice}, and spoke:\epb\epg


\bvg\bva%
\edtext{„\alst{H}ęill ves þú nú, Loki, \hld\ ok tak við \alst{h}rïm-kalki &
\ind \alst{f}ullum \alst{f}orns mjaðar}{\lemma{Hęill \dots\ mjaðar ‘Hale \dots\ mead’}\Bfootnote{Formulaic; repeated identically in \Skirnismal\ 37/1–2.}}, &
hęldr þú hana \alst{ęi}na \hld\ látir með \alst{ȧ}sa sonum &
\ind \alst{v}amma-lausa \alst{v}esa.“\eva

\bvb “Hale be thou now, O Lock, and receive this rime-chalice, \\
\ind full of ancient mead! \\
Rather oughtst thou to let me alone among the sons of the Eese \\
\ind remain blemish-less.”\evb\evg


\bpg\bpa Hann tók við horni ok drakk af:\epa

\bpb He received the horn and drank from it:\epb\epg


\bvg\bva%
„\alst{Ęi}n þú vę́rir \hld\ \alst{e}f þú svá vę́rir, &
\ind \alst{v}ǫr ok grǫm at \alst{v}eri; &
ęinn ek \alst{v}ęit, \hld\ svá’t ek \alst{v}ita þikkjumk, &
\ind \alst{h}ór ok af \alst{H}lórriða, &
\ind ok vas þat sá inn \edtrans{\alst{l}ę́-vísi \alst{L}oki}{guile-wise Lock}{\Bfootnote{Formulaic, also occuring in \Hymiskvida\ 37.  Cf. also \Voluspa\ 35 where Lock is called \emph{lę́-gjarn} ‘guile-eager’ and note to \Voluspa\ 17 where Lother (possibly to be identified with Lock) gives men \emph{lǫ́}, which may be an accusative form of \emph{lę́}.}}.“\eva

\bvb “Alone wouldst thou be, if thou so wert \\
\ind wary and wroth against man. \\
I know one—whom I think me to know— \\
\ind adulterer behind even \inx[P]{Loride}’s back, \\
\ind and that was the guile-wise Lock!”\evb\evg


\bvg\bva\speakernote{[Bęyla kvað:]}%
„\edtrans{\alst{F}jǫll ǫll skjalfa}{The fells all quake}{\Bfootnote{The movement of gods, especially Thunder, is often signalled by cosmic disturbance.  See note to \Thrymskvida\ 21.}}, \hld\ hygg ȧ \alst{f}ǫr vesa &
\ind \alst{h}ęiman \alst{H}lórriða; &
hann \alst{r}ę́ðr \alst{r}ó \hld\ þeim’s \alst{r}ǿgir hér &
\ind \alst{g}oð ǫll ok \alst{g}uma!“\eva

\bvb\speakernoteb{[Beal quoth:]}%
“The fells all quake—I think on the journey \\
\ind from home Loride to be. \\
He brings to rest him who here maligns \\
\ind all the Gods and men!”\evb\evg


\bvg\bva\speakernote{[Loki kvað:]}%
„Þęgi þú, \alst{B}ęyla, \hld\ þú est \alst{B}yggvis kvę̇n &
\ind ok \alst{m}ęini blandin \alst{m}jǫk; &
\alst{ȯ}-kynja’n męira \hld\ kom-a með \alst{ȧ}sa sonum; &
\ind \edtrans{ǫll est, \alst{d}ęigja, \alst{d}ritin}{thou art all, dough-girl, dungy}{\Bfootnote{\emph{dęigja} ‘dough-girl’ is a derivative of \emph{dęigr} ‘dough’ and refers to a young girl at a farm who kneads dough, milks the cows and such.  The insult here is that she is still dirtied with the dung of milch cows.}}.“\eva

\bvb\speakernoteb{[Lock quoth:]}%
“Shut thou up, Beal! Thou art Bewer’s wife, \\
\ind and much mixed with harm. \\
A greater disgrace came not among the sons of the Eese; \\
\ind thou art all, dough-girl, dungy!”\evb\evg


\bpg\bpa Þá kom Þórr at ok kvað:\epa

\bpb Then Thunder arrived and quoth:\epb\epg


\bvg\bva%
„\alst{Þ}ęgi þú, rǫg vę́ttr, \hld\ þér skal mïnn \edtrans{\alst{þ}rúð-hamarr}{thrith-hammer}{\Bfootnote{“Strength-hammer”, \emph{þrúðr} ‘thrith’ being an obsolete word for strength used only in connection with Thunder or ettins.  \emph{Þrúðr} ‘\inx[P]{Thrith}’ is also the name of Thunder’s daughter.}}, &
\ind \alst{M}jǫllnir, \alst{m}ál fyr-nema! &
\alst{H}ęrða klett \hld\ drep’k þér \alst{h}alsi af, &
\ind ok verðr þȧ þïnu \alst{f}jǫrvi of \alst{f}arit.“\eva

\bvb “Shut thou up, \inx[C]{queer} wight! Thee shall my thrith-hammer \\
\ind Millner, deprive of speech! \\
The shoulder-rock \ken{head} I strike off thy neck, \\
\ind and then is thy life destroyed!”\evb\evg


\bvg\bva\speakernote{[Loki kvað:]}%
„\alst{Ja}rðar burr \hld\ es hér nú \alst{i}nn kominn; &
\ind hví \alst{þ}rasir þú svá, \alst{Þ}ȯrr? &
En þȧ þorir \alst{ę}kki \hld\ \edtext{es skalt við \alst{u}lf’inn vega &
\ind ok \alst{s}velgr hann allan \alst{S}ig-fǫður}{\lemma{es skalt við ulfinn vega / ok svelgr hann allan Sig-fǫður ‘when thou shalt fight the Wolf / and he swallows Syefather \name{= Weden} whole.’}\Bfootnote{A reference to the Rakes of the Reins, where \inx[P]{Weden} is slain by the Wolf and then avenged by his son \inx[P]{Wider}.  Thunder, meanwhile, dies while slaying the Wyrm; see \Voluspa\ 51–53, \Vafthrudnismal\ 53.}}.“\eva

\bvb\speakernoteb{[Lock quoth:]}%
“Earth’s Son is here now come inside, \\
\ind why thrashest thou so, Thunder? \\
But thou wilt nowise dare when thou shalt fight the Wolf \\
\ind and he swallows Syefather \name{= Weden} whole.”\evb\evg


\bvg\bva\speakernote{[Þȯrr kvað:]}%
„\alst{Þ}ęgi þú, rǫg vę́ttr, \hld\ þér skal mïnn \alst{þ}rúð-hamarr, &
\ind \alst{M}jǫllnir, \alst{m}ál fyr-nema! &
\alst{U}pp ek þér verp \hld\ ok ȧ \alst{au}str-vega &
\ind \alst{s}íðan þik mann-gi \alst{s}ér.“\eva

\bvb\speakernoteb{[Thunder quoth:]}%
“Shut thou up, queer wight! Thee shall my thrith-hammer \\
\ind Millner, deprive of speech! \\
Up I throw thee, and onto the eastern ways; \\
\ind thereafter no man may see thee!”\evb\evg


\bvg\bva\speakernote{[Loki kvað:]}%
„\alst{Au}str-fǫrum þïnum \hld\ skalt \alst{a}ldri-gi &
\ind \alst{s}ęgja \alst{s}ęggjum frȧ &
síðst \edtrans{ï \alst{h}anska þumlungi \hld\ \alst{h}núkðir þú}{in the thumb of a glove thou didst crawl}{\Bfootnote{This stanza and 62 below refer to Thunder’s encounter with the ettin Shrimer, which is retold in \Gylfaginning\ 45.  A related narrative is mentioned in \Harbardsljod\ TODO, although the ettin there is called Feller.}}, Ęin-hęri, &
\ind ok \alst{þ}ȯttisk-a \alst{þ}ȧ \alst{Þ}ȯrr vesa!“\eva

\bvb\speakernoteb{[Lock quoth:]}%
“From thy eastern journeys shalt thou never \\
\ind speak to the youths, \\
since in the thumb of a glove thou crawledest, Oneharrier, \\
\ind and didst not seem to be Thunder then!”\evb\evg


\bvg\bva\speakernote{[Þȯrr kvað:]}%
„\alst{Þ}ęgi þú, rǫg vę́ttr, \hld\ þér skal mïnn \alst{þ}rúð-hamarr, &
\ind \alst{M}jǫllnir, \alst{m}ál fyr-nema! &
\alst{h}ęndi inni \alst{h}ǿgri \hld\ drep’k þik \alst{H}rungnis bana, &
\ind svá’t þér \alst{b}rotnar \alst{b}ęina hvat.“\eva

\bvb\speakernoteb{[Thunder quoth:]}%
“Shut thou up, queer wight! Thee shall my thrith-hammer \\
\ind Millner, deprive of speech! \\
With the right hand I strike thee with Rungner’s bane \ken*{= Millner}, \\
\ind so that every bone in thee breaks.”\evb\evg


\bvg\bva\speakernote{[Loki kvað:]}%
„\alst{L}ifa ę́tla’k mér \hld\ \alst{l}angan aldr &
\ind þótt \alst{h}ǿtir \alst{h}amri mér; &
\alst{sk}arpar ȧlar \hld\ þȯttu þér \alst{Sk}rymis vesa &
\ind ok máttir-a þȧ \alst{n}ęsti \alst{n}áa &
\ind ok svaltsk þȧ \alst{h}ungri \alst{h}ęill.“\eva

\bvb\speakernoteb{[Lock quoth:]}%
“To live a long life I intend for myself, \\
\ind though thou mighst threaten me with the hammer. \\
Sharp seemed Shrimer’s straps to thee, \\
\ind and then couldst thou not reach thy provisions, \\
\ind and then wast thou dying, healthy, of hunger.”\evb\evg


\bvg\bva\speakernote{[Þȯrr kvað:]}%
„\alst{Þ}ęgi þú, rǫg vę́ttr, \hld\ þér skal mïnn \alst{þ}rúð-hamarr, &
\ind \alst{M}jǫllnir, \alst{m}ál fyr-nema! &
\alst{H}rungnis bani \hld\ mun þér ï \alst{h}ęl koma &
\ind fyr \alst{N}á-grindr \alst{n}eðan.“\eva

\bvb\speakernoteb{[Thunder quoth:]}%
“Shut thou up, queer wight! Thee shall my thrith-hammer \\
\ind Millner, deprive of speech! \\
Rungner’s bane will take thee to hell, \\
\ind down beneath Neegrind!”\evb\evg


\bvg\bva\speakernote{[Loki kvað:]}%
„Kvað’k fyr \alst{ǫ̇}sum, \hld\ kvað’k fyr \alst{ȧ}sa sonum, &
\ind þat’s mik \alst{h}vatti \alst{h}ugr, &
en fyr þér \alst{ęi}num \hld\ mun’k \alst{ú}t ganga &
\ind því-at ek \alst{v}ęit at þú \alst{v}egr.\eva

\bvb\speakernoteb{[Lock quoth:]}
“I spoke before the Eese; I spoke before the sons of the Eese \\
\ind whatever my heart did goad me, \\
but for thee alone will I walk out, \\
\ind for I know that thou strikest.\evb\evg


\bvg\bva%
\alst{Ǫ}l gørðir þú, \alst{Ę́}gir, \hld\ en þú \alst{a}ldri munt &
\ind \alst{s}íðan \alst{s}umbl of gøra; &
\alst{ęi}ga þïn \alst{ǫ}ll, \hld\ es hér \alst{i}nni es, &
\ind \alst{l}ęiki yfir \alst{l}ogi &
\ind ok \alst{b}renni þér ȧ \alst{b}aki.“\eva

\bvb Ale hast thou made, Eagre, but thou wilt never \\
\ind henceforth make a simble! \\
All thy estate which is here within— \\
\ind let flame play over it, \\
\ind and burn thee in the back!”\evb\evg

\sectionline

\section{From Lock (\emph{Frá Loka})}

The binding of Lock is known from two other places. Closest at hand is \Voluspa\ 34, but it offers no full narrative.

\Gylfaginning\ 50 has a longer account, somewhat different from the present prose. There the Eese captured Lock’s two sons, Wonnel and “Nare or Narve”. They turned Wonnel into a wolf (\emph{vargr}, which also means ‘outlaw’) and had him tear his brother Narve apart. Narve’s intestines were then taken and used to bind Lock on top of three pointed stones, with one digging into his shoulder-blades, the other digging into his loins, and the third digging into his houghs. At last the intestines turned into iron and Lock was bound.

Since the author of \Gylfaginning\ knew \Voluspa, it is possible that he combined a text similar to \FraLoka\ with \Voluspa\ H1, interpreting \emph{Vȧla víg-bǫnd} as ‘Wonnel’s war-bonds’. Wonnel is otherwise only known as the son of Weden, and there is no reason as to why he could not have bound Lock.

\sectionline

\bpg\bpa En eptir þetta falst Loki í Fránangrs-forsi í lax líki. Þar tóku ę́sir hann. Hann var bundinn með þǫrmum sonar Nara; en Narfi, sonr hans, varð at vargi. Skaði tók eitr-orm ok festi upp yfir and-lit Loka; draup þar ór eitr. Sigyn, kona Loka, sat þar ok helt munn-laug undir eitrit. En er munn-laugin var full bar hon út eitrit, en meðan draup eitrit á Loka. Þá kipptist hann svá hart við, at þaðan af skalf jǫrð ǫll; þat eru nú kallaðir land-skjálftar.\epa

\bpb And after this Lock hid himself in the Freenangersforce in the form of a salmon. There the Eese took him. He was bound with the intestines of his son Nare, but his son Narve was made a wolf/outlaw. Shede took a venomous serpent and fastened it up above Lock’s face; from it ran venom. Syein, Lock’s wife, sat there and held a basin under the venom. And when the basin was full she carried out the venom, but meanwhile the venom ran onto Lock. Then he struggled so hard that thereof all the earth quaked; that is now called earth-quakes.\epb\epg

\sectionline

\section{Stanza from \Gylfaginning}

In \Gylfaginning\ 20 the following stanza is cited as proof of Frie’s foresight regarding the orlays of men.  It is introduced by the words \emph{svá sem hér er sagt, at Óðinn mę́lti sjalfr við þann ás, er Loki heitir} ‘just as it is said here, that Weden himself spoke to that Os who is called Lock’.

The text looks like an amalgamation of several \Lokasenna\ stanzas (which is why it has been placed here, rather than among the Fragments From Snorre’s Edda); l. 1 corresponds to st. 21/1 (spoken by Weden), l. 2 to st. 47/2 (spoken by Homedal), and ll. 3–4 to st. 29/3–4 (spoken by Frow).  It is possible that it derives from an alternate version of \Lokasenna, but it could also have been formed due to Snorre’s misremembering the rest of the stanza after the first line, which is also attributed to Weden in st. 21.

\sectionline

\bvg\bva[]%
„\alst{Ǿ}rr est, Loki, \hld\ ok \alst{ø}r-viti, &
\ind hví né \alst{l}ętsk-a þú, \alst{L}oki? &
\alst{ø}r-lǫg Frigg \hld\ hygg at \alst{ǫ}ll viti &
\ind þótt hǫ̇n \alst{s}jǫlf-gi \alst{s}ęgi.“\eva

\bvb “Mad art thou, Lock, and out of wits, \\
\ind why holdest thou not back, O Lock? \\
All orlays I think that Frie might know, \\
\ind though she tell them not herself.”\evb\evg

\sectionline
