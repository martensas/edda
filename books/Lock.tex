\bookStart{The Flyting of Lock}[Lokasęnna]

\begin{flushright}%
Dating \parencite{Sapp2022}: C10th (0.965)

Meter: \Ljodahattr%
\end{flushright}

Preserved in \Regius, directly following \Hymiskvida, though the poems without doubt were originally separate; the stylistic differences are drastical.

The poem has been interpreted as blasphemous (TODO: elaborate), but shows no linguistic signs of being particularly late.

\sectionline

\section{From Eagre and the gods (\emph{Frá Ę́gi ok goðum})}

\bpg\bpa Ę́gir, er ǫðru nafni hét Gymir, hann hafði búit ásum ǫl þá er hann hafði fengit ketil inn mikla sem nú er sagt. Til þeirar veizlu kom Óðinn ok Frigg kona hans. Þórr kom eigi þvíat hann var í austrvegi. Sif var þar, kona Þórs; Bragi, ok Iðunn kona hans. Týr var þar, hann var einhendr; Fenrisulfr sleit hǫnd af hánum, þá er hann var bundinn. Þar var Njǫrðr ok kona hans Skaði; Freyr ok Freyja; Víðarr son Óðins. Loki var þar, ok þjónustumenn Freys, Byggvir ok Beyla. Mart var þar ása ok alfa. Ę́gir átti tvá þjónustumenn; Fimafengr ok Eldir. Þar var lýsigull haft fyr eldsljós; sjalft barsk þar ǫl. Þar var griðastadr mikill. Menn lofuðu mjǫk hversu góðir þjónustumenn Ę́gis vóru. Loki mátti eigi heyra þat, ok drap hann Fimafeng. Þá skóku ę́sir skjǫldu sína ok ǿptu at Loka, ok eltu hann braut til skógar, en þeir fóru at drekka. Loki hvarf aptr ok hitti úti Eldi; Loki kvaddi hann:\epa

\bpb \inx[P]{Eagre}, who by another name is called \inx[P]{Gymer}, had prepared an ale-feast for the Ease when he had got the great kettle as now is told.\footnoteB{See the immediately preceding \Hymiskvida.}

To that gathering came \inx[P]{Weden} and \inx[P]{Frie}, his woman. \inx[P]{Thunder} came not, for he was on the \inx[L]{Eastern Way}. Sib was there, Thunder’s woman; \inx[P]{Bray} and \inx[P]{Idun}, his woman. \inx[P]{Tue} was there, he was one-handed. The \inx[P]{Fenrerswolf} tore his hand off when it was bound.\footnoteB{This detail is probably brought up to chronologically date the events of the poem as happening after the binding of Fenrer in the mythology.} There was \inx[P]{Nearth}, and his woman \inx[P]{Shede}; \inx[P]{Free} and \inx[L]{Frow}; \inx[P]{Wider}, the son of \inx[P]{Weden}. \inx[P]{Lock} was there, and the servants of Free: \inx[P]{Bew} and \inx[P]{Beal}. There was a great many of the \inx[G]{Ease} and \inx[G]{Elves}\footnoteB{A formulaic expression, see \inx[F]{Ease and Elves}.}.

Eagre had two servants: \inx[P]{Femfinger} and \inx[P]{Elder}. There was glowing gold used instead of fire; the ale there poured itself. There was a great \inx[C]{grith-stead}.\footnoteB{A place wherein all violence was forbidden, see Encyclopedia.} Men greatly praised how good the servants of Eagre were. Lock could not stand that, and he slew Femfinger.

Then the Ease shook their shields and screamed at Lock,\footnoteB{Some sort of ancient war dance. Cf. the Old Swedish Heathen Law: “He screams three nithing-screams TODO”.} and chased him away to the forest, but then they went to drink. Lock came back and found Elder outside; Lock greeted him:\epb\epg

\sectionline

\bvg
\bva „Seg þú þat, Eldir, \hld\ \edtext{svá’t ęinugi &
\ind feti gangir framarr}{\lemma{svá’t \dots\ framarr ‘so that \dots\ further’}\Bfootnote{Cf. \Havamal\ 38: \emph{feti ganga framarr} ‘take one step further’.}}, &
hvat hér inni \hld\ hafa at ǫlmǫ́lum &
\ind sigtíva synir.“\eva

\bvb “Say thou it, Elder, so that thou take not one step further: what here within they bring up over the ale,\footnoteB{lit. ‘have for their ale-speeches’} the sons of the victory-Tues \ken{gods}.”\evb
\evg


\bvg {\small Elder quoth:}
\bva „Of vǫ́pn sín dǿma \hld\ ok of vígrisni sína &
\ind sigtíva synir; &
ása ok alfa, \hld\ es hér inni eru, &
\ind \edtext{manngi ’s þér í orði vinr.}{\lemma{manngi \dots\ vinr “none \dots\ words.”}\Bfootnote{i.e. “none of them say anything good about you.” — The (lack of) alliteration here is very notable, and also occurs in a c-line of v. 10 (see note there). Both of the two lines are otherwise perfect, and so it seems that \emph{v} (\textipa{/w/}) is participating in vowel-alliteration. Such is never encountered in scoldic poetry, it could have been delegated to the simpler Eddic styles. Alternatively the poem is of such age that it was composed before the North Germanic loss of \emph{v} before rounded vowels. This is supported by the fact that in both this stanza and st. 10 the words starting with vowels have cognates in other Germanic languages that begin with \emph{w}; in the case of \emph{ulfr} in v. 10 this consonant is well attested in old runic inscriptions.

If the alliteration indeed is on \emph{v}, this does not require dating the whole poem to the Proto-Norse period; perhaps the poet was aware of the change which had taken place a few generations before him, and employed the older form as an archaism. For metrical reasons the poem must certainly post-date the syncope period (in the C6th), but we know from the transitional C7th Blekinge runestones from Stentoften (DR 357), Gummarp (DR 358) and Istaby (DR 359) that the loss of \textipa{/w-/} occured after syncope anyway.

A C7th Proto-Norse form of the c-line might be: \emph{mannagí ’s þéʀ in worðé winʀ}.}}“\eva

\bvb “Of their weapons they converse, and of their fight-valiance, the sons of the victory-Tues \ken{gods}; of the Ease and Elves which are here within, none is thee a friend in words.”\evb
\evg


\bvg {\small Lock quoth:}
\bva „Inn skal ganga \hld\ Ę́gis hallir í &
\ind á þat sumbl at séa, &
\edtext{jǫll ok ǫ́fu}{\lemma{jǫll ok ǫ́fu “scorn and spite”}\Bfootnote{ioll oc áfo \Regius\. These two interesting words have been interpreted in a variety of ways: \CV\ sees the first word as \emph{jóll} ‘wild angelica’, whereas the second is taken to be an error for \emph{áfr} ‘a beverage [...] translated by Magnaeus by \emph{sorbitio avenacea}, a sort of common ale brewed of oats’.}} \hld\ fǿri’k ása sonum &
\ind ok blęnd’k þęim svá męini mjǫð.“\eva

\bvb “In shall I go into Eagre’s halls, for to see that \inx[C]{simble}; scorn and strife I bring to the sons of the Ease, and I mix for them so the mead with harm.”\evb
\evg


\bvg {\small Elder quoth:}
\bva „Vęizt, ef inn gęngr \hld\ Ę́gis hallir í &
\ind á þat sumbl at séa, &
hrópi ok rógi \hld\ ef ęyss á holl ręgin, &
\ind á þér munu þau þęrra þat.“\eva

\bvb “Know, if in thou goest into Eagre’s halls, for to see that simble: if slander and strife thou pourest onto the \inx[C]{hold} \inx[G]{Reins}, they will dry it off on thee.”\evb
\evg


\bvg {\small Lock quoth:}
\bva „Vęizt þat Ęldir, \hld\ ef ęinir skulum &
\ind sáryrðum sakask, &
auðigr verða \hld\ mun’k í andsvǫrum, &
\ind ef þú mę́lir til mart.“\eva

\bvb “Know it, Elder, if alone we two shall banter with wound-words: I will become wealthy in my answers, if thou speak too much.\footnoteB{Cf. \Havamal\ TODO mę́la til mart.}”\evb
\evg


\bpg
\bpa Síðan gekk Loki inn í hǫllina; en er þeir sá, er fyrir váru, hverr inn var kominn, þǫgnuðu þeir allir.\epa

\bpb Thereafter Lock walked into the hall, but when those who were there before him saw who was come inside, they all turned silent.\epb
\epg


\bvg {\small Lock quoth:}
\bva „Þyrstr ek kom \hld\ þessar hallar til &
\ind Loptr of langan veg, &
ǫ́su at biðja, \hld\ at mér ęinn gefi &
\ind mę́ran drykk mjaðar.\eva

\bvb “Thirsty I, Loft \name{= Lock}, came to these halls over a long way, to ask the Ease that they to me give a single renowned drink of mead.\evb
\evg


\bvg
\bva Hví þęgið ér svá \hld\ þrungin goð, &
\ind at mę́la né męguð; &
sessa ok staði \hld\ vęlið mér sumbli at, &
\ind eða hęitið mik heðan.“\eva

\bvb Why shut ye up so, pressed gods, that ye may not speak? Seats and places choose for me at the simble, or call me [away] hence.\footnoteB{i.e. “Cease your ambiguity; give me a seat or tell me to leave!”}”\evb
\evg


\bvg {\small Bray quoth:}
\bva „Sessa ok staði \hld\ vęlja þér sumbli at &
\ind ę́sir aldrigi; &
því’t ę́sir vitu \hld\ hvęim þęir alda skulu &
\ind gambansumbl of geta.“\eva

\bvb “Seats and places choose the Ease never for thee at the simble; for the Ease know which men they shall bid to the gomben-simble.”\evb
\evg


\bvg {\small [Lock quoth:]}
\bva „Mant þat Óðinn, \hld\ es vit í árdaga &
\ind blendum blóði saman? &
ǫlvi bęrgja \hld\ lézk ęigi mundu, &
\ind nema okkr vę́ri bǫ́ðum borit.“\eva

\bvb “Recallest thou, Weden, as we two in days of yore blended our blood together? Thou saidst thou wouldst not taste ale, unless it were for us both brought forth.”\evb
\evg


\bvg {\small [Weden quoth:]}
\bva \edtext{„Rís þú Víðarr \hld\ ok lát ulfs fǫður}{\lemma{Rís \dots\ fǫður ‘Rise \dots\ wolf’}\Bfootnote{For the alliteration see note to v. 2. A C7th Proto-Norse form of the c-line might be: \emph{Rís þú Wíðarʀ · auk lát wulfs faður}.}}
\ind sitja sumbli at,
síðr oss Loki \hld\ kveði lastastǫfum
\ind Ę́gis hǫllu í.“\eva

\bvb “Rise thou, Wider, and let the father of the wolf \ken*{= Lock} sit at the simble, lest Lock accuse us of fault in the hall of Eagre.”\evb
\evg


\bpg
\bpa Þá stóð Víðarr upp ok skenkti Loka, en áðr hann drykki, kvaddi hann ásuna:\epa

\bpb Then Wider stood up and poured to Lock, but before he [= Lock] drunk, he greeted the Ease:\epb
\epg


\bvg
\bva „Hęilir ę́sir, \hld\ hęilar ǫ́synjur &
\ind ok ǫll ginnhęilǫg goð, &
nema sá ęinn ǫ́ss \hld\ es innar sitr &
\ind Bragi bękkjum á.“\eva

\bvb “Hail the \inx[G]{Ease}! Hail the \inx[G]{Ossens}, and all the \inx[C]{gin-holy} gods!\footnoteB{The first two half-lines prayer formula are identical to \Sigrdrifumal\ 2–3, for which reason it is possibly of authentic Heathen origin. To the original audience Lock’s parody of it would then have been seen as highly offensive and blasphemous.} Save for that one \inx[G]{Ease}[os], who sits further within: Bray, on the benches.”\evb
\evg


\bvg {\small [Bray] quoth:}
\bva „Mar ok mę́ki \hld\ gef’k þér míns féar &
\ind ok bǿtir þér svá baugi Bragi, &
síðr þú ǫ́sum \hld\ ǫfund of gjaldir, &
\ind gręmjat goð at þér.“\eva

\bvb “Steed and sword I give thee of my own wealth, and so recompenses thee Bray with a \inx[C]{bigh}, since thou repayest the Ease with envy; do not anger the gods towards thee.”\evb
\evg


\bvg {\small [Lock] quoth:}
\bva „Jós ok armbauga \hld\ munt ę́ vesa &
\ind bęggja vanr Bragi, &
ása ok alfa, \hld\ es hér inni eru, &
\ind þú ert við víg varastr,
\ind ok skjarrastr við skot.“\eva

\bvb “Of both steed and arm-bighs wilt thou ever be, Bray, lacking; of the Ease and Elves which are here within, art thou the wariest of war, and the shyest of shot.”\evb
\evg


\bvg {\small [Bray] quoth:}
\bva „Vęit’k, ef fyr útan vę́ra’k, \hld\ sem fyr innan em’k, &
\ind Ę́gis hǫll of kominn, &
hǫfuð þitt \hld\ bę́ra’k í hęndi mér; &
\ind\edtext{lít’k þér þat fyr lygi}{\Bfootnote{‘litt ec þer þat fyr lygi’ \Regius. A variety of emendations have been proposed for this line. Simplest would be \emph{lítt es þér þat fyr lygi} ‘that is little [punishment] for thee for lying’. Based on the similarity of \emph{c} and \emph{ꞇ̇} (= \emph{tt}) \textcite{FinnurEdda} gives \emph{lykak þér þat fyr lygi} ‘so I would bring to thee for thy lie’.}}.“\eva

\bvb “I know if outside I were, as inside I am come into the hall of Eagre: thy head I would bear in my hands; this I see for thy lie.”\evb
\evg


\bvg {\small [Lock] quoth:}
\bva „Snjallr ert í sessi, \hld\ skalattu svá gęra, &
Bragi bękkskrautuðr; &
vega þú gakk \hld\ ef vręiðr séir; &
hyggsk vę́tr hvatr fyrir.“\eva

\bvb “Valiant art thou in the seat; thou shalt not do thus, Bray the bench-ornamenter! Go to strike if thou art wroth; the bold does not think in advance.\footnoteB{Cf. \Havamal\ nýsisk fróðra TODO, really the opposite sentiment.}”\evb
\evg


\bvg {\small [Idun] quoth:}
\bva „Bið’k, Bragi, \hld\ barna sifjar duga &
\ind ok allra óskmaga, &
at þú Loka \hld\ kveðir-a lastastǫfum &
\ind Ę́gis hǫllu í.“\eva

\bvb “I bid thee, O Bray, to respect the TODO, that thou not accuse Lock of fault in the hall of Eagre.”\evb
\evg


\bvg {\small [Lock] quoth:}
\bva „Þęgi þú, Iðunn, \hld\ þik kveð’k allra kvinna &
\ind vergjarnasta vesa &
síz þú arma þína \hld\ lagðir ítrþvęgna &
\ind umb þinn bróðurbana.“\eva

\bvb “Shut up thou, Idun: thee I say of all women to be the most man-eager, since thou laid thy beautifully washed arms around thy brother’s bane.”\evb
\evg


\bvg {\small [Idun] quoth:}
\bva „Loka ek kveð’k-a \hld\ lastastǫfum &
\ind Ę́gis hǫllu í; &
Braga ek kyrri \hld\ bjórręifan, &
\ind vil’k-at ek at it vręiðir vegisk.“\eva

\bvb “I do not accuse Lock of fault in the hall of Eagre. Bray I calm, cheerful from beer—I do not wish that ye two wroth ones may fight.”\evb
\evg


\bvg {\small [Giben] quoth:}
\bva „Hví it ę́sir tvęir \hld\ skuluð inni hér &
\ind sáryrðum sakask? &
Lofts-ki þat vęit \hld\ at hann lęikinn es &
\ind ok hann fjǫrgvall frjá.”\eva

\bvb “TODO”\evb
\evg


\bvg {\small [Lock] quoth:}
\bva „Þęgi þú, Gefjun, \hld\ þęss mun’k nú geta &
\ind es þik glapði at gęði: &
svęinn inn hvíti \hld\ es þér sigli gaf &
\ind ok þú lagðir lę́r yfir.“\eva

\bvb “Shut up thou, o Giben! Of him I will now speak, who confounded thy senses: the white swain, who gave thee a necklace, and thou laidest thy leg over [him].”\evb
\evg


\bvg {\small [Weden] quoth that:}
\bva „Ǿrr ert, Loki, \hld\ ok ørviti &
es þú fę́r þér Gęfjun at gręmi &
því’t aldar ørlǫg \hld\ hygg at hón ǫll of viti &
jafngǫrla sem ek.“\eva

\bvb “Mad art thou, o Lock, and out of wits, as thou incurrest the wrath of Giben; for, all orlays of people I judge that she might know, just as clearly as I.”\evb
\evg


\bvg {\small [Lock] quoth:}
\bva „Þęgi þú, Óðinn, \hld\ þú kunnir aldrigi &
\ind dęila víg með verum; &
opt þú gaft \hld\ þęim’s þú gefa skyldir-a, &
\ind inum slę́vurum, sigr.“\eva

\bvb “Shut up thou, o Weden: thou couldst never deal out war amongst men—often thou gavest to the ones thou shouldst not have given, to the slower men victory.”\evb
\evg


\bvg {\small [Weden] quoth:}
\bva „Vęizt ef ek gaf \hld\ þęim’s ek gefa né skylda, &
\ind inum slę́vurum, sigr, &
átta vetr \hld\ vast fyr jǫrð neðan &
\ind kýr mólkandi ok kona &
\ind ok hęfir þú þar bǫrn of borit &
\ind ok hugða’k þat args aðal.“\eva

\bvb “Know that if I gave to the ones I should not have given, to the slower men victory: for eight nights wast thou beneath the earth, milking cows and a woman, and there hast thou borne children, and I’ve judged that a degenerate’s nature.”\evb
\evg


\bvg {\small [Lock] quoth:}
\bva „En þik síga kóðu \hld\ Sámsęyju í &
\ind ok drapt á vett sem vǫlur, &
vitka líki \hld\ fórt verþjóð yfir, &
\ind ok hugða’k þat args aðal.“\eva

\bvb “But thou, they said, didst sink down upon Samsy, and thou beatst the drum like wallows [do]. In the likeness of a sorcerer thou journeyedst among the nations of men, and I’ve judged that a degenerate’s nature.”\evb
\evg


\bvg {\small [Frie] quoth:}
\bva „Ørlǫgum ykkrum \hld\ skylið aldrigi &
\ind sęgja sęggjum frá, &
hvat it ę́sir tvęir drýgðuð í árdaga; &
\ind firrisk ę́ forn rǫk firar.“\eva

\bvb “Regarding your two’s orlays should ye never speak to youths; that which ye two Ease did in days of yore—always may ancient rakes be shunned by men.”\evb
\evg


\bvg {\small [Lock] quoth:}
\bva „Þęgi þú, Frigg, \hld\ þú ert Fjǫrgyns mę́r &
\ind ok hęfir ę́ vergjǫrn verit, &
es þá Véa ok Vilja \hld\ lézt þér, Viðris kvę́n, &
\ind báða í baðm of tękit.“\eva

\bvb “Shut up thou, o Frie: thou art Firgyn’s maiden, and has always been man-eager—when Wigh and Will, thou letst, o Withrer’s wife, both in thy bosom take.”\evb
\evg


\bvg {\small [Frie] quoth:}
\bva „Vęizt ef inni ę́tta’k \hld\ Ę́gis hǫllum í &
\ind Baldri líkan bur &
út þú né kvę́mir \hld\ frá ása sonum &
\ind ok vę́ri þá at þér vręiðum vegit.“\eva

\bvb “Know, that if here inside I owned, in Eagre’s halls, a son alike to Balder: out came thou not, away from the sons of the Ease, and thou would be fought with wrath.”\evb
\evg


\bvg {\small [Lock] quoth:}
\bva „En vill þú, Frigg, \hld\ at ek flęiri tęlja &
\ind mína męinstafi: &
ek því réð \hld\ es þú ríða sér-at &
\ind síðan Baldr at sǫlum.“\eva

\bvb “Yet wilt thou, o Frie, that I count more of my harmful deeds: I caused it, that thou dost not hence see Balder riding toward the halls.”\evb
\evg


\bvg {\small [Frow] quoth:}
\bva „Ǿrr ert, Loki, \hld\ es þú yðra tęlr &
\ind ljóta lęiðstafi; &
ørlǫg Frigg \hld\ hygg at ǫll viti &
\ind þótt hón sjǫlf-gi sęgi.“\eva

\bvb “Mad art thou, o Lock, as thou countest your ugly loathsome deeds: all orlays I judge that Frie might know, although she says them not herself.”\evb
\evg


\bvg {\small [Lock] quoth:}
\bva „VERSE“\eva

\bvb “TRANSLATION”\evb
\evg


\bvg {\small [Frow] quoth:}
\bva „VERSE“\eva

\bvb “TRANSLATION”\evb
\evg


\bvg {\small [Lock] quoth:}
\bva „VERSE“\eva

\bvb “TRANSLATION”\evb
\evg


\bvg {\small [Nearth] quoth:}
\bva „VERSE“\eva

\bvb “TRANSLATION”\evb
\evg


\bvg {\small [Lock] quoth:}
\bva „VERSE“\eva

\bvb “TRANSLATION”\evb
\evg


\bvg {\small [Nearth] quoth:}
\bva „VERSE“\eva

\bvb “TRANSLATION”\evb
\evg


\bvg {\small [Lock] quoth:}
\bva „VERSE“\eva

\bvb “TRANSLATION”\evb
\evg


\bvg {\small [Tue] quoth:}
\bva „VERSE“\eva

\bvb “TRANSLATION”\evb
\evg


\bvg {\small [Lock] quoth:}
\bva „VERSE“\eva

\bvb “TRANSLATION”\evb
\evg


\bvg {\small [Tue] quoth:}
\bva „VERSE“\eva

\bvb “TRANSLATION”\evb
\evg


\bvg {\small [Lock] quoth:}
\bva „VERSE“\eva

\bvb “TRANSLATION”\evb
\evg


\bvg {\small [Free] quoth:}
\bva „VERSE“\eva

\bvb “TRANSLATION”\evb
\evg


\bvg {\small [Lock] quoth:}
\bva „VERSE“\eva

\bvb “TRANSLATION”\evb
\evg


\bvg {\small [Bew] quoth:}
\bva „VERSE“\eva

\bvb “TRANSLATION”\evb
\evg


\bvg {\small [Lock] quoth:}
\bva „VERSE“\eva

\bvb “TRANSLATION”\evb
\evg


\bvg {\small [Bew] quoth:}
\bva „VERSE“\eva

\bvb “TRANSLATION”\evb
\evg


\bvg {\small [Lock] quoth:}
\bva „VERSE“\eva

\bvb “TRANSLATION”\evb
\evg


\bvg {\small [Homedall] quoth:}
\bva „VERSE“\eva

\bvb “TRANSLATION”\evb
\evg


\bvg {\small [Lock] quoth:}
\bva „VERSE“\eva

\bvb “TRANSLATION”\evb
\evg
