\bookStart{The Speeches of the High One}[Hávamǫ́l]

%Introduction.

The \textbf{Speeches of the High One} is the second poem of \Regius, which is also the only ancient manuscript in which it is attested. Several verses are however cited in other places, such as Eyv \emph{Hák} (TODO: formatting) 21 and \FostrbroedhraSaga\ TODO.

The poem as it currently comes down to us hardly seems like a single composition, much rather like a grab bag of traditional verses and poems associated with the god Weden. It combines two separate advice-poems with verses concerning Weden’s love adventures, runes and spells. Little unites these various strands other than their speaker.

Following previous authors, I identify several such strands, excepting various lone insert-verses. In the present edition each of them is given a separate, short introduction:

\begin{itemize}
  \item 1–79 The Guest-strand, containing practical life advice placed within a frame narrative of a guest arriving at a homestead.
  \item 81–89 Other verses of advice, mostly composed in \Fornyrdislag.
  \item 90–109 Weden’s love adventures, advice for love and seduction.
  \item 110–135 The Speeches of Loddfathomer (\emph{Loddfáfnismǫ́l}), advice given to Loddfathomer.
  \item 136–144 The Rune-tally (\emph{Rúnatal}), various verses relating to runes.
  \item 145–163 The Leed-tally (\emph{Ljóðatal}), Weden’s listing of 18 spells.
  \item 164 Final verse, composed when the poem as we have it was assembled.
\end{itemize}

Whatever their origins, it is clear from the final verse that they have been thought of as a single work, but it is notable that this verse, which also contains the title \emph{Hávamǫ́l} ‘Speeches of the High One’, is highly metrically irregular. It has likely been composed by the person who assembled the disparate elements listed above into one text.

\sectionline

\section{The Guest-strand}

The Guest-Strand (Old Norse: \emph{Gęstaþáttr}) is possibly the finest work in Norse poetry. Sadly, its structure has been obscured by various inserted and possibly displaced verses. My hope is to shed some light on the original vision behind the poem, while as usual not changing the order of verses as they appear in the only surviving witness manuscript.

The poem moves through many elements of life, but in a poetically almost seamless way. To move from one topic to another, the poet often employs transitions where a verse recalls the structure of the previous one, but with a new subject. This is particularly evident in verses 4–5 and 10–11.

The strand begins with a verse encouraging travellers to be wary of entering strange houses without first spying out who is inside (1), after which a voice inside of a farmstead (possibly Weden?) announces that a guest is waiting to be let in (2). The same speaker then lists several things which the newly arrived guest needs from the host, namely: fire, food and clothes (3), water, a towel, a great welcome, a good reception, an opportunity to speak and silence in return (4).

After this focus shifts to the conduct of the wanderer, with an introductory verse explaining that he needs wit (specifically \inx[C]{manwit} (\emph{manvit}); see Encyclopedia), lest he become a laughing-stock (5). He should be silent but attentive, and choose his words carefully (6–7). He should be confident in himself and his own decisions, and not rely too much on the opinions of others (8–9), since there is nothing better one may bring along on the journey than much manwit (10).

Here the advice moves to the subject alcohol. Where the best thing one may bring along on the journey is manwit, the worst is too much ale (11). It is not as good as men call it (12) since it “robs [them] of their senses”; it is even personified as a “heron of forgetfulness” (13). A drinking round is best when the participants do not drink too much, but rather regain their senses afterwards (14).

Verse 15 contains some general advice; a royal child should be silent, thoughtful and bold in battle, and all men should stay happy, until they die.

TODO.

\sectionline

\bvg
\bva \alst{G}áttir allar \hld\ áðr \alst{g}angi framm &
\ind \edtext{of \alst{sk}oðask \alst{sk}yli,}{\lemma{of skoðask skyli}\Bfootnote{\emph{om.} \GylfMS}} &
\ind of \alst{sk}yggnask \alst{sk}yli; &
því’t ó\alst{v}íst ’s at \alst{v}ita, \hld\ hvar ó\alst{v}inir &
\ind sitja á \alst{f}lęti \alst{f}yrir.\eva

\bvb All doorways—before one might go forth—should be watched, should be spied at; for uncertain ’tis to know, where enemies sit on the benches inside.\evb
\evg


\bvg
\bva \alst{G}efęndr hęilir, \hld\ \alst{g}ęstr ’s inn kominn, &
\ind hvar skal \alst{s}itja \alst{s}já? &
mjǫk es \alst{b}ráðr \hld\ sá’s á \alst{b}rǫndum skal &
\ind síns of \alst{f}ręista \alst{f}rama.\eva

\bvb Hail the givers,\footnoteB{The hosts.} a guest is come in! Where shall this one sit? Very impatient is he, who on the fires shall try his distinction.\footnoteB{Possibly referring a Norwegian folk custom, wherein a guest would sit down on the wood-pile outside of the door, waiting until being let in. See further TODO SOME ARTICLE on this custom. The speaker thus announces to the hosts that a frozen, wet and tired guest has arrived and currently sits impatiently on the wood-pile, and ought to be taken in.}\evb
\evg


\bvg
\bva \alst{Ę}lds es þǫrf \hld\ þęim’s \alst{i}nn es kominn &
\ind ok á \alst{k}néi \alst{k}alinn, &
\alst{m}atar ok váða \hld\ es \alst{m}anni þǫrf, &
\ind þęim’s hęfr of \alst{f}jall \alst{f}arit.\eva

\bvb Of fire is there need for the one who is come in, and cold about the knees; of food and of clothing is there need for the one who over the fell has fared.\evb
\evg


\bvg
\bva \alst{V}ats es þǫrf \hld\ þęim’s til \alst{v}erðar kømr, &
\ind \alst{þ}ęrru ok \alst{þ}jóðlaðar, &
\alst{g}óðs of ǿðis, \hld\ —ef sér \alst{g}eta mę́tti— &
\ind \alst{o}rðs ok \alst{ę}ndrþǫgu.\eva

\bvb Of water is there need for the one who comes for a meal; of a towel and of a great welcome; of a good reception—if he might get one—of speech, and of silence in return.\footnoteB{There is a well thought-out linear progression throughout this verse. The guest must first wash himself, then dry himself with a towel, then be welcomed to sit and eat at the table and speak with the host. The host has done his part, and now it is the guest’s turn. This nicely leads the transition to the following verses, where the proper conduct of the guest (first in speech, and then in various other areas) is discussed.}\evb
\evg


\bvg
\bva \alst{V}its es þǫrf \hld\ þęim’s \alst{v}íða ratar; &
\ind dę́lt es \alst{h}ęima \alst{h}vat; &
at \alst{au}gabragði \hld\ verðr sá’s \alst{ę}kki kann &
\ind ok með \alst{s}notrum \alst{s}itr.\eva

\bvb Of wit is there need for the one who widely roams; everything is easy at home. A laughing-stock\footnoteB{An idiom, \emph{augabragð} lit. ‘twinkling of an eye, moment’.} becomes he who nothing knows, and among the clever sits.\evb
\evg


\bvg
\bva At \alst{h}yggjandi sinni \hld\ skyli-t maðr \alst{h}rǿsinn vesa, &
\ind hęldr \alst{g}ę́tinn at \alst{g}ęði, &
þá’s \alst{h}orskr ok þǫgull \hld\ kømr \alst{h}ęimisgarða til, &
\ind sjaldan verðr \alst{v}íti \alst{v}ǫrum. &
\edtext{því’t \alst{ó}brigðra vin \hld\ fę́r \alst{a}ldrigi, &
\ind an \alst{m}anvit \alst{m}ikit.}{\lemma{því \dots\ mikit}\Bfootnote{The shift in person from third to second, along with the abnormal verse length (six lines instead of four), indicates that this is an insertion.}}\eva

\bvb Of his thinking should man not be boastful; rather guarding of his senses, when sharp and silent he comes to a homestead; sudden injury seldom strikes the wary, (for thou gettest never an unfickler friend, than much \inx[C]{manwit}.)\evb
\evg


\bvg
\bva Hinn \alst{v}ari gęstr, \hld\ es til \alst{v}erðar kømr, &
\ind \alst{þ}unnu hljóði \alst{þ}ęgir; &
\alst{ęy}rum hlýðir, \hld\ ęn \alst{au}gum skoðar, &
\ind svá nýsisk \alst{f}róðra hvęrr \alst{f}yrir.\eva

\bvb The wary guest—when for a meal he comes—with thin heed shuts up.\footnoteB{i.e. is in attentive silence.} With ears he heeds, but with eyes observes; so pries each learned man about.\evb
\evg


\bvg
\bva Hinn es \alst{s}ę́ll, \hld\ es \alst{s}ér of getr &
\ind \alst{l}of ok \alst{l}íknstafi; &
\alst{ó}dę́lla es við þat, \hld\ es \alst{ęi}ga skal &
\ind \alst{a}nnars brjóstum \alst{í}.\eva

\bvb The one is blessed, who for himself gets praise and staves of grace. ’Tis uneasy regarding that which one shall own in another’s breast.\evb
\evg


\bvg
\bva \alst{S}á es \alst{s}ę́ll, \hld\ es \alst{s}jalfr of á &
\ind \alst{l}of ok vit meðan \alst{l}ifir; &
því’t \alst{i}ll rǫ́ð \hld\ hęfr maðr \alst{o}pt þęgit &
\ind \alst{a}nnars brjóstum \alst{ó}r.\eva

\bvb That one is blessed, whose self owns praise and wits while he lives; for ill counsels has man oft taken out of another’s breast.\evb
\evg


\bvg
\bva \alst{B}yrði \alst{b}ętri \hld\ berr-at maðr \alst{b}rautu at, &
\ind an sé \alst{m}anvit \alst{m}ikit; &
\alst{au}ði bętra \hld\ þykkir þat í \alst{ó}kunnum stað; &
\ind slíkt es \alst{v}álaðs \alst{v}era.\eva

\bvb A better burden bears man not on the road than much manwit. In an unknown place it seems better than wealth; such is the shelter of the impoverished.\evb
\evg

% TODO: NEW SECTION (Alcohol)

\bvg
\bva \alst{B}yrði \alst{b}ętri \hld\ berr-at maðr \alst{b}rautu at, &
\ind an sé \alst{m}anvit \alst{m}ikit; &
\alst{v}egnest \alst{v}erra \hld\ \alst{v}egr-a \alst{v}ęlli at, &
\ind an sé \alst{o}fdrykkja \alst{ǫ}ls.\eva

\bvb A better burden bears man not on the road than much manwit. Worse way-provision he drags not along in the field\footnoteB{\emph{vǫllr} ‘plain, (uncultivated) field’ is repeated in vv. 38 and 49. It is easily understood that the heaths and plains of Iron Age Norway were particularly unsafe places, where a traveller needed to keep his wits with him, lest he fall victim to robbers or murderers.} than a too great drink of ale.\evb
\evg


\bvg
\bva Es-a svá \alst{g}ótt, \hld\ sęm \alst{g}ótt kveða, &
\ind \alst{ǫ}l \alst{a}lda sonum; &
því’t \alst{f}ę́ra vęit, \hld\ es \alst{f}lęira drekkr, &
\ind síns til \alst{g}ęðs \alst{g}umi.\eva

\bvb ’Tis not so good, as good they say, ale for the sons of men; for the less he knows, as the more he drinks, man of his own senses.\evb
\evg


\bvg
\bva \alst{Ó}minnishegri hęitir, \hld\ sá’s yfir \alst{ǫ}lðrum þrumir, &
\ind hann stelr \alst{g}ęði \alst{g}uma; &
þess \alst{f}ogls \alst{f}jǫðrum \hld\ ek \alst{f}jǫtraðr vas’k &
\ind í \alst{g}arði \alst{G}unnlaðar.\eva

\bvb The heron of forgetfulness is called he who above ale-feasts hovers; he robs men of their senses.\footnoteB{Here drunkenness is personified as a bird, a “heron of forgetfulness”.} With that bird’s feathers I was fettered in the yards of \inx[P]{Guthlathe}.\evb
\evg


\bvg
\bva \alst{Ǫ}lr ek varð, \hld\ varð \alst{o}frǫlvi, &
\ind at hins \alst{f}róða \alst{F}jalars; &
því es \alst{ǫ}lðr bazt, \hld\ at \alst{a}ptr of hęimtir &
\ind hvęrr sitt \alst{g}ęð \alst{g}umi.\eva

\bvb Drunk I became—I became the drunkest by far—at the learned Fealer’s [home]. Thus is an ale-feast best, as each man takes his senses back home.\evb
\evg

% TODO: NEW SECTION (War)

\bvg
\bva \alst{Þ}agalt ok hugalt \hld\ skyli \alst{þ}jóðans barn &
\ind ok \alst{v}ígdjarft \alst{v}esa; &
\alst{g}laðr ok ręifr \hld\ skyli \alst{g}umna hvęrr, &
\ind unz sinn \alst{b}íðr \alst{b}ana.\eva

\bvb Silent and thoughtful should the ruler’s child be, and battle-bold. Glad and cheerful should each man be, until he suffer his bane.\evb
\evg


\bvg
\bva \alst{Ó}snjallr maðr \hld\ hyggsk munu \alst{ę}y lifa, &
\ind ef við \alst{v}íg \alst{v}arask; &
ęn \alst{ę}lli gefr hǫ́num \hld\ \alst{ę}ngi frið, &
\ind þótt hǫ́num \alst{g}ęirar \alst{g}efi.\eva

\bvb The unvalorous man thinks he will forever live, if he of war is wary; but old age gives him no peace, although spears might give him.\footnoteB{He might have been spared by the spears, but death will still find him. The underlying meaning seems to be that since death is unavoidable it is better to live bravely, even if one risks dying in battle, than to live cowardly and die of old age. This verse connects well to the ancient view of the ‘straw-death’.}\evb
\evg


\bvg
\bva \alst{K}ópir afglapi, \hld\ es til \alst{k}ynnis kømr, &
\ind \alst{þ}ylsk hann umb eða \alst{þ}rumir; &
alt es \alst{s}ęnn, \hld\ ef \alst{s}ylg of getr, &
\ind uppi es þá \alst{g}ęð \alst{g}uma.\eva

\bvb Gapes the oaf when to visit he comes; he mumbles about or loiters. All at once—if a sip he gets—are the senses of the man exposed.\evb
\evg


\bvg
\bva Sá ęinn \alst{v}ęit, \hld\ es \alst{v}íða ratar &
\ind ok hęfr \alst{f}jǫlð of \alst{f}arit, &
hvęrju \alst{g}ęði \hld\ stýrir \alst{g}umna hvęrr, &
\ind sá es \alst{v}itandi ’s \alst{v}its.\eva

\bvb He alone knows, who widely roams, and has travelled much: his own senses does each man control, who is aware of his wits.\evb
\evg


\bvg
\bva \alst{H}aldi-t maðr á kęri, \hld\ drekki þó at \alst{h}ófi mjǫð, &
\ind mę́li \alst{þ}arft eða \alst{þ}ęgi; &
\alst{ó}kynnis þess \hld\ váar þik \alst{ę}ngi maðr, &
\ind at gangir \alst{s}nimma at \alst{s}ofa.\eva

\bvb Man ought not to hold onto the cask, yet drink a fitting serving of mead; he ought to speak the needful or shut up.\footnoteB{Identical to a certain verse in \Vafthrudnismal\ TODO: which one} For that uncouthness will no man blame thee, that thou go early to sleep.\evb
\evg


\bvg
\bva \alst{G}rǫ́ðugr halr, \hld\ nema \alst{g}ęðs viti, &
\ind \alst{e}tr sér \alst{a}ldrtrega; &
opt fę́r \alst{h}lǿgis, \hld\ es með \alst{h}orskum kømr, &
\ind \alst{m}anni hęimskum \alst{m}agi.\eva

\bvb The gluttonous man—unless he know his sense—eats himself a life-sorrow. Oft the belly—when among the sharp he comes—brings a foolish man ridicule.\evb
\evg


\bvg
\bva \alst{H}jarðir þat vitu, \hld\ nę́r \alst{h}ęim skulu, &
\ind ok \alst{g}anga þá af \alst{g}rasi; &
ęn \alst{ó}sviðr maðr \hld\ kann \alst{ę́}vagi &
\ind síns of \alst{m}ál \alst{m}aga.\eva

\bvb Herds know when homewards they shall [turn], and then part from the grass; but an unwise man never knows the measure of his own belly.\evb
\evg


\bvg
\bva \alst{V}esall maðr \hld\ ok \alst{i}lla skapi &
\ind \alst{h}lę́r at \alst{h}vívetna; &
hitki hann \alst{v}ęit, \hld\ es \alst{v}ita þyrpti, &
\ind at hann es-a \alst{v}amma \alst{v}anr.\eva

\bvb The wretched man, and the ill-spirited, laughs at whatever. This he knows not, which he might need to know: he is not free of blemishes.\evb
\evg


\bvg
\bva \alst{Ó}sviðr maðr \hld\ vakir umb \alst{a}llar nę́tr &
\ind ok \alst{h}yggr at \alst{h}vívetna; &
þá es \alst{m}óðr, \hld\ es at \alst{m}orni kømr; &
\ind alt es \alst{v}íl sęm \alst{v}as.\eva

\bvb The unwise man is awake for all nights, and thinks of whatever. Then he is weary when the morning comes; [his] trouble is all as it was.\evb
\evg


\bvg
\bva \alst{Ó}snotr maðr \hld\ hyggr sér \alst{a}lla vesa &
\ind \alst{v}iðrhlę́jęndr \alst{v}ini; &
hit-ki hann \alst{f}iðr, \hld\ þótt þęir of hann \alst{f}ár lesi, &
\ind ef með \alst{s}notrum \alst{s}itr.\eva

\bvb The unclever man thinks all who laugh with him friends. This he finds not, that they find flaws in him, if among the clever he sits.\evb
\evg


\bvg
\bva \alst{Ó}snotr maðr \hld\ hyggr sér \alst{a}lla vesa &
\ind \alst{v}iðhlę́jęndr \alst{v}ini; &
\alst{þ}á þat fiðr \hld\ es at \alst{þ}ingi kømr, &
\ind at á \alst{f}ormę́lęndr \alst{f}áa.\eva

\bvb The unclever man thinks all who laugh with him friends. Then he finds—when to the \inx[C]{Thing} he comes—that he has spokesmen few.\footnoteB{Repeated in v. 62. He has few who are ready to take his side and speak up for him; the sense is that true friends are proven in conflict, not in talking. The Thing (see Encyclopedia) was the old Germanic legal assembly, and so the specific reference here is legal disputes, but it should be kept in mind that they could easily turn into deadly feuds.}\evb
\evg


\bvg
\bva \alst{Ó}snotr maðr \hld\ þykkisk \alst{a}lt vita, &
\ind ef á sér i \alst{v}ǫ́ \alst{v}eru; &
hitki hann \alst{v}ęit, \hld\ hvat hann skal \alst{v}ið kveða, &
\ind ef hans \alst{f}ręista \alst{f}irar.\eva

\bvb The unclever man seems to know everything if he takes shelter in a nook. This he knows not, what he shall say in return if men test him.\evb
\evg


\bvg
\bva \alst{Ó}snotr maðr, \hld\ es með \alst{a}ldir kømr, &
\ind \alst{þ}at ’s bazt at hann \alst{þ}ęgi; &
\alst{ę}ngi þat vęit, \hld\ at hann \alst{ę}kki kann, &
\ind nema hann \alst{m}ę́li til \alst{m}art. &
\alst{v}ęit-a maðr, \hld\ hinn’s \alst{v}ę́tki vęit, &
\ind þótt hann \alst{m}ę́li til \alst{m}art.\eva

\bvb The unclever man, when among people he comes, ’tis best that he shut up. None knows that he nothing knows, unless he speak too much. Man knows not, who nothing knows, although he speak too much.\footnoteB{That is, mindless speech will not make him any wiser.}\evb
\evg


\bvg
\bva \alst{F}róðr sá þykkisk, \hld\ es \alst{f}regna kann, &
\ind ok \alst{s}ęgja hit \alst{s}ama, &
\alst{ęy}vitu lęyna \hld\ męgu \alst{ý}ta synir &
\ind því es \alst{g}ęngr of \alst{g}uma.\eva

\bvb Learned seems he who can ask and answer the same. Naught may the sons of men conceal of that\footnoteB{Rumours and gossip.} which goes about a man.\evb
\evg


\bvg
\bva \alst{Ǿ}rna mę́lir, \hld\ sá’s \alst{ę́}va þęgir, &
\ind \alst{st}aðlausu \alst{st}afi; &
\alst{h}raðmę́lt tunga, \hld\ nema \alst{h}aldęndr ęigi, &
\ind opt sér ó\alst{g}ótt of \alst{g}ęlr.\eva

\bvb Quite enough speaks he—who never shuts up—utterings of absurdity. A quick-spoken tongue—unless it be held in place\footnoteB{lit. ‘unless holders own it’ or ‘unless it own holders’. The ‘holders’ may perhaps refer to the teeth holding the tongue in places.}—oft sings evil [into being] for itself.\evb
\evg


\bvg
\bva At \alst{au}gabragði \hld\ skal-a maðr \alst{a}nnan hafa, &
\ind \edtext{þótt}{\lemma{þótt “although”}\Bfootnote{Perhaps an error? \emph{es} ‘when’ would surely work better in context.}} til \alst{k}ynnis \alst{k}omi; &
margr \alst{f}róðr þykkisk, \hld\ ef \alst{f}reginn es-at &
\ind ok nái \alst{þ}urrfjallr \alst{þ}ruma.\eva

\bvb As a laughing-stock shall man not have another, although he come to visit. Many a one seems learned if he is not asked, and manages to loiter about dry-skinned.\footnoteB{This sense of \emph{fjall} is apparently almost non-existent in Old Norse literature, but compare Swedish \emph{fjäll} ‘scale (on fish and reptiles)’. The meaning is in any case figurative, equivalent to the English “get one’s feet wet”.}\evb
\evg


\bvg
\bva \alst{F}róðr þykkisk \hld\ sá’s \edtrans{\alst{f}lótta}{flee}{\Bfootnote{Emended to \emph{flátta} ‘mock’ by \textcite{Athugasemdir1929}}} tękr &
\ind \alst{g}ęstr at \alst{g}ęst hę́ðinn; &
\alst{v}ęit-a gǫrla \hld\ sá’s of \alst{v}erði glissir, &
\ind þótt með \alst{g}rǫmum \alst{g}lami.\eva

\bvb Learned seems he who takes to flee\footnoteB{Probably not literally, rather ‘pulls back, does not take part’.} when a guest at a guest is scoffing. He knows not clearly, who grins above the food, that he with fiends be prattling.\evb
\evg


\bvg
\bva \alst{G}umnar margir \hld\ erusk \alst{g}agnhollir, &
\ind ęn at \alst{v}irði \alst{v}rekask; &
\alst{a}ldar róg \hld\ þat mun \alst{ę́} vesa; &
\ind órir \alst{g}ęstr við \alst{g}ęst.\eva

\bvb Many men are \inx[C]{hold} to each other, but over a meal drive each other away. The strife of mankind will that ever be; guest raves against guest.\evb
\evg


\bvg
\bva \alst{Á}rliga verðar \hld\ skyli maðr \alst{o}pt fáa, &
\ind nema til \alst{k}ynnis \alst{k}omi; &
\alst{s}itr ok \alst{s}nópir, \hld\ lę́tr sęm \alst{s}olginn sé, &
\ind ok kann \alst{f}regna at \alst{f}ǫ́u.\eva

\bvb An early meal should man oft get, unless he come to visit: he sits and idles haplessly, makes as if starved, and can ask about little.\evb
\evg


\bvg
\bva \alst{A}fhvarf mikit \hld\ es til \alst{i}lls vinar, &
\ind þótt á \alst{b}rautu \alst{b}úi, &
ęn til \alst{g}óðs vinar \hld\ liggja \alst{g}agnvegir, &
\ind þótt hann sé \alst{f}irr \alst{f}arinn.\eva

\bvb A great detour ’tis to a wicked friend, although he on the highway live; but to a good friend lie the shortest ways, although he far gone be.\evb
\evg


\bvg
\bva \alst{G}anga skal, \hld\ skal-a \alst{g}ęstr vesa &
\ind \alst{ęy} í \alst{ęi}num stað; &
\edtext{\alst{l}júfr verðr \alst{l}ęiðr}{\lemma{ljúfr verðr lęiðr ‘the loved becomes loathed’}\Bfootnote{}}, \hld\ ef \alst{l}ęngi sitr &
\ind \alst{a}nnars flętjum \alst{á}.\eva

\bvb One shall go; shall not be a guest forever in one place. The loved becomes loathed if for long he sits on another’s benches.\evb
\evg


\bvg
\bva \alst{B}ú es \alst{b}ętra, \hld\ þótt lítit sé, &
\ind \alst{h}alr es \alst{h}ęima \alst{h}vęrr; &
þótt \alst{t}vę́r gęitr ęigi \hld\ ok \alst{t}augręptan sal, &
\ind þat es þó \alst{b}ętra an \alst{b}ǿn.\eva

\bvb A dwelling is better, though small it be: each is a warrior at home. Though two goats he own, and a cord-roofed hall, that is yet better than begging.\evb
\evg


\bvg
\bva \alst{B}ú es \alst{b}ętra, \hld\ þótt lítit sé, &
\ind \alst{h}alr es \alst{h}ęima \alst{h}vęrr; &
\alst{b}lóðugt es hjarta \hld\ þęim’s \alst{b}iðja skal &
\ind sér í \alst{m}ál hvęrt \alst{m}atar.\eva

\bvb A dwelling is better, though small it be: each is a warrior at home. Bloody is the heart of the one who shall beg for himself each meal of food.\evb
\evg


\bvg
\bva \alst{V}ǫ́pnum sínum \hld\ skal-a maðr \alst{v}ęlli á &
\ind \edtext{\alst{f}eti ganga \alst{f}ramarr}{\lemma{feti ganga framarr ‘take one step further’}\Bfootnote{Cf. \Lokasenna\ 1: \emph{svát ęinugi feti gangir framarr,} ‘so that thou not take one step further’.}}; &
því’t ó\alst{v}íst ’s at \alst{v}ita, \hld\ nę́r verðr á \alst{v}egum úti &
\ind \alst{g}ęirs of þǫrf \alst{g}uma.\eva

\bvb From his weapons shall man in the field not take one step further; for uncertain ’tis to know, when on the ways outside, man comes in need of a spear.\evb
\evg


\bvg
\bva Fann’k-a \alst{m}ildan mann \hld\ eða svá \edtext{\alst{m}atar góðan}{\lemma{matar góðan ‘good of meat’}\Bfootnote{A Viking Age expression; see Encyclopedia.}}, &
\ind at vę́ri-t \alst{þ}iggja \alst{þ}ęgit; &
eða \alst{s}íns féar \hld\ \alst{s}vági \edtext{[...]}{\Bfootnote{It is doubtless that a word has been lost here; the meter and sense require it. \textcite{FinnurEdda}\ suggests \emph{gløggvan} ‘miserly, stingy’, giving a litotes ‘so not stingy’, i.e., ‘so generous’.}}, &
\ind at \alst{l}ęið sé \alst{l}aun, ef þegi.\eva

\bvb I found not a generous man, or one so \inx[C]{good of meat}, that a gift were not accepted; or one of his \inx[C]{fee} so not [...], that the rewards were loathed, if he accepted [them].\footnoteB{No man is so generous that he would refuse a gift presented to him, nor loathe receiving a favour as thanks for his generosity.}\evb
\evg


\bvg
\bva \alst{F}éar síns, \hld\ es \alst{f}ęngit hęfr, &
\ind skyli-t maðr \alst{þ}ǫrf \alst{þ}ola; &
opt sparir \alst{l}ęiðum \hld\ þat’s hęfr \alst{l}júfum hugat; &
\ind mart gęngr \alst{v}err an \alst{v}arir.\eva

\bvb Of his own \inx[C]{fee}, which he has earned, should man not suffer need. Oft one saves for the loathed what was meant for the loved; many a thing goes worse than one expects.\evb
\evg


\bvg
\bva \alst{V}ǫ́pnum ok \alst{v}ǫ́ðum \hld\ skulu \alst{v}inir glęðjask; &
\ind þat ’s á \alst{s}jǫlfum \alst{s}ýnst; &
\alst{v}iðrgefęndr ok ęndrgefęndr \hld\ erusk \alst{v}inir lęngst, &
\ind ef þat bíðr at \alst{v}erða \alst{v}ęl.\eva

\bvb With weapons and garments shall friends gladden each other; that is most seen on oneself.\footnoteB{i.e. in one’s own lived experience.} Mutual givers and return-givers are friends for the longest, if it\footnoteB{The friendship.} is to last long.\evb
\evg


\bvg
\bva \alst{V}in sínum \hld\ skal maðr \alst{v}inr \alst{v}esa, &
\ind ok \alst{g}jalda \alst{g}jǫf við \alst{g}jǫf; &
\alst{h}látr við \alst{h}látri \hld\ skyli \alst{h}ǫlðar taka, &
\ind ęn \alst{l}ausung við \alst{l}ygi.\eva

\bvb With his friend shall man be a friend, and reward gift against gift; laughter against laughter should men take, but duplicity against lie.\evb
\evg


\bvg
\bva \alst{V}in sínum \hld\ skal maðr \alst{v}inr vesa, &
\ind \alst{þ}ęim ok \alst{þ}ess vin; &
ęn \alst{ó}vinar síns \hld\ skyli \alst{ę}ngi maðr &
\ind \alst{v}inar \alst{v}inr \alst{v}esa.\eva

\bvb With his friend shall man be a friend, with him and his friend; but with his enemy’s, should no man, friend’s friend be.\evb
\evg


\bvg
\bva \alst{V}ęizt, ef \alst{v}in átt, \hld\ þann’s \alst{v}ęl trúir &
\ind ok vilt af hǫ́num \alst{g}ótt \alst{g}eta, &
\alst{g}ęði skalt við þann \hld\ ok \alst{g}jǫfum skipta, &
\ind \alst{f}ara at \alst{f}inna opt.\eva

\bvb Know, if thou have a friend, one on which thou well trust, and wilt receive good from: mind and gifts shalt thou share with him; journey to find him oft.\footnoteB{This verse is closely related to 117, which seems like an abridged version of this one.}\evb
\evg


\bvg
\bva Ef þú \alst{á}tt \alst{a}nnan, \hld\ þann’s þú \alst{i}lla trúir, &
\ind vilt af hǫ́num þó \alst{g}ótt \alst{g}eta, &
\alst{f}agrt skalt mę́la, \hld\ ęn \alst{f}látt hyggja &
\ind ok gjalda \alst{l}ausung við \alst{l}ygi.\eva

\bvb If thou have another, one on which thou badly trust, and wilt yet receive good from: fairly shalt thou speak, but falsely think, and pay duplicity against lie.\evb
\evg


\bvg
\bva Þat ’s \alst{ę}nn umb þann, \hld\ es þú \alst{i}lla trúir &
\ind ok þér es \alst{g}runr at \alst{g}ęði, &
\alst{h}lę́ja skalt við þęim \hld\ ok of \alst{h}ug mę́la; &
\ind \alst{g}lík skulu \alst{g}jǫld \alst{g}jǫfum.\eva

\bvb ’Tis yet regarding that one, on which thou badly trustest, and who causes thy senses doubt:\footnoteB{lit. “and for thee is doubt in senses”.} laugh shalt thou with him, and speak with care; rewards shall be equal to gifts.\footnoteB{Equivalent to the last line of the previous v. (“reward duplicity against lie”).}\evb
\evg


\bvg
\bva Ungr vas’k \alst{f}orðum, \hld\ \alst{f}ór’k ęinn saman, &
\ind þá varð’k \alst{v}illr \alst{v}ega; &
\alst{au}ðigr þóttumk, \hld\ es \alst{a}nnan fann’k, &
\ind \alst{m}aðr es \alst{m}anns gaman.\eva

\bvb Young was I once, I travelled alone; then I became lost about the ways. Wealthy I thought myself when another one I found; man is the pleasure of man.\evb
\evg


\bvg
\bva \alst{M}ildir frǿknir \hld\ \alst{m}ęnn bazt lifa, &
\ind \alst{s}jaldan \alst{s}út ala; &
\alst{ó}snjallr maðr \hld\ \alst{u}ggir hvatvetna, &
\ind sýtir ę́ \alst{g}løggr við \alst{g}jǫfum.\eva

\bvb Generous, bold men live the best; seldom they nourish grief. The unvalorous man is frightened by whatever; ever the stingy man grieves a gifts.\footnoteB{Refer back to v. 39; after receiving a gift, one was culturally obliged to give something back.}\evb
\evg


\bvg
\bva \alst{V}áðir mínar \hld\ gaf’k \alst{v}ęlli at &
\ind \alst{t}vęim \alst{t}rémǫnnum; &
\alst{r}ekkar þat þóttusk, \hld\ es \alst{r}ipt hǫfðu; &
\ind \alst{n}ęiss es \alst{n}ǫkkviðr halr.\eva

\bvb My garments I gave in the field, to two tree-men. Champions they seemed when cloaks they had; shameful is the naked warrior.\footnoteB{One of the hardest verses in the poem. After much thought I consider the probable sense to be that the clothes make the warrior; under expensive gear a thin tree-man might be hiding, and likewise even a strong man (I see the choice of the word \emph{halr} ‘warrior’ rather than the more neutral \emph{maðr} ‘man, person’ as intentional) when naked and facing a heavily armoured opponent becomes as vulnerable as the ‘tree-man’ on a plain.}\evb
\evg


\bvg
\bva Hrørnar \alst{þ}ǫll, \hld\ sú’s stęndr \alst{þ}orpi á, &
\ind hlýrat hęnni \alst{b}ǫrkr né \alst{b}arr; &
svá es \alst{m}aðr, \hld\ sá’s \alst{m}anngi ann; &
\ind hvat skal hann \alst{l}ęngi \alst{l}ifa?\eva

\bvb Wilters the pine that stands on the yard; shields her not bark nor needle. So is the man who loves none; for what shall he live for long?\evb
\evg


\bvg
\bva \alst{Ę}ldi hęitari \hld\ brinnr með \alst{i}llum vinum &
\ind \alst{f}riðr \alst{f}imm daga, &
ęn þá \alst{sl}oknar, \hld\ es hinn \alst{s}étti kømr, &
\ind ok \alst{v}ersnar allr \alst{v}inskapr.\eva

\bvb Hotter than fire burns peace among bad friends, for \inx[C]{five days};\footnoteB{A reference to the five-day week (see also v. 74); the number is symbolic. See further Encyclopedia.} but then goes out when the sixth one comes, and all the friendship worsens.\evb
\evg


\bvg
\bva \alst{M}ikit ęitt \hld\ skal-a \alst{m}anni gefa; &
\ind opt kaupir sér í \alst{l}ítlu \alst{l}of, &
með \alst{h}ǫlfum \alst{h}lęif \hld\ ok með \alst{h}ǫllu kęri &
\ind \alst{f}ekk ek mér \alst{f}élaga.\eva

\bvb Much at once shall one not give a man; oft one buys oneself praise for little. With half a loaf and an awry cask, I got me a companion.\evb
\evg


\bvg
\bva \alst{L}ítilla sanda, \hld\ \alst{l}ítilla sę́va, &
\ind lítil eru \alst{g}ęð \alst{g}uma; &
því’t \alst{a}llir męnn \hld\ \alst{u}rðu-t jafnspakir; &
\ind \alst{h}ǫlf es ǫld \alst{h}var.\eva

\bvb Of small sands, of small seas; small are the senses of man. For all have not become evenly knowing; half is every man.\footnoteB{The genitive “of small sands, of small seas” is probably a partitive; man’s horizons are small, the universe is far greater than he, and always will be. On the meaning of the second half of the verse I find that of \textcite{Athugasemdir1929} most convincing, namely that everybody has both strengths and weaknesses. As nobody can excel at everything, nobody is complete; every person is half. This fits particularly closely with v. 71 and 131.}\evb
\evg


\bvg
\bva \alst{M}eðalsnotr \hld\ skyli \alst{m}anna hvęrr, &
\ind ę́va til \alst{s}notr \alst{s}é; &
þęim es \alst{f}yrða \hld\ \alst{f}ęgrst at lifa, &
\ind es \alst{v}ęl mart \alst{v}itu.\eva

\bvb Middle-clever should each man be; never too clever. For those men ’tis fairest to live, who know well enough.\evb
\evg


\bvg
\bva \alst{M}eðalsnotr \hld\ skyli \alst{m}anna hvęrr, &
\ind ę́va til \alst{s}notr \alst{s}é; &
\alst{s}notrs manns hjarta \hld\ verðr \alst{s}jaldan glatt, &
\ind ef sá ’s \alst{a}lsnotr es \alst{á}.\eva

\bvb Middle-clever should each man be; never too clever. The clever man’s heart is seldom gladdened, if he is all-clever that owns [it].\evb
\evg


\bvg
\bva \alst{M}eðalsnotr \hld\ skyli \alst{m}anna hvęrr, &
\ind ę́va til \alst{s}notr \alst{s}é; &
\alst{ø}rlǫg sín \hld\ viti \alst{ę}ngi fyr; &
\ind þęim es \alst{s}orgalausastr \alst{s}efi.\eva

\bvb Middle-clever should each man be; never too clever. His own \inx[C]{orlay} ought none to know ahead; his is the most sorrowless mind.\footnoteB{Who knows not his fate. It is fitting that Weden would say this, having knowledge of the inevitable destruction of the world and hisself.}\evb
\evg


\bvg
\bva \alst{B}randr af \alst{b}randi \hld\ \alst{b}rinnr unz \alst{b}runninn es, &
\ind \alst{f}uni kvęykisk af \alst{f}una; &
\alst{m}aðr af \alst{m}anni \hld\ verðr at \alst{m}áli kuðr; &
\ind ęn til \alst{d}ǿlskr af \alst{d}ul.\eva

\bvb Fire by fire burns until it burnt is; flame is kindled from flame. Man by man becomes known for speech, but the too dull by his delusion.\evb
\evg


\bvg
\bva \alst{Á}r skal rísa, \hld\ sá’s \alst{a}nnars vill &
\ind \alst{f}é eða \alst{f}jǫr hafa; &
sjaldan \alst{l}iggjandi ulfr \hld\ \alst{l}ę́r of getr, &
\ind né \alst{s}ofandi maðr \alst{s}igr.\eva

\bvb Early shall rise he who another’s \inx[C]{fee} or life will have. Seldom does the lying wolf get a thigh, or the sleeping man victory.\evb
\evg


\bvg
\bva \alst{Á}r skal rísa, \hld\ sá’s á \alst{y}rkjęndr fáa, &
\ind ok ganga síns \alst{v}erka á \alst{v}it; &
\alst{m}art of dvęlr \hld\ þann’s umb \alst{m}orgin sefr, &
\ind \alst{h}alfr es auðr und \alst{h}vǫtum.\eva

\bvb Early shall rise he who owns workers few, and go his work to meet. Much is kept back from him who in the morning sleeps; half the wealth is due to the brisk.\footnoteB{Half of a man’s wealth is due to his briskness.}\evb
\evg


\bvg
\bva \alst{Þ}urra skíða \hld\ ok \alst{þ}akinna nę́fra, &
\ind þess kann \alst{m}aðr \alst{m}jǫt, &
ok þess \alst{v}iðar, \hld\ es \alst{v}innask męgi &
\ind \alst{m}ál ok \alst{m}issęri.\eva

\bvb Of dry planks and of thatching birch bark: thereof man knows the measure—and of that firewood which may be used for a season and half-year.\footnoteB{Over the winter.}\evb
\evg


\bvg
\bva \alst{Þ}vęginn ok męttr \hld\ ríði maðr \alst{þ}ingi at, &
\ind þótt hann sé-t \alst{v}ę́ddr til \alst{v}ęl; &
\alst{sk}úa ok bróka \hld\ \alst{sk}ammisk ęngi maðr &
\ind né \alst{h}ęsts in \alst{h}ęldr, &
\ind \edtext{þótt hann \alst{h}afi’t góðan}{\lemma{þótt \dots\ góðan “although \dots\ good one”}\Bfootnote{As \textcite{FinnurEdda}\ points out, surely a late insertion. Whoever made it was not aware of the rules of the \Ljodahattr, interpreting the c-line as a \Fornyrdislag\ a-line, and then insreting the supposed b-line.}}.\eva

\bvb Washed and filled ought man to ride to the Thing, although he might not be dressed too well; of his shoes and breeches ought no man to be ashamed, nor indeed of his horse, (although he might not have a good one.)\evb
\evg


\bvg
\bva \alst{S}napir ok gnapir, \hld\ es til \alst{s}ę́var kømr, &
\ind \alst{ǫ}rn á \alst{a}ldinn mar; &
svá es \alst{m}aðr, \hld\ es með \alst{m}ǫrgum kømr &
\ind ok á \alst{f}ormę́lęndr \alst{f}áa.\eva

\bvb Shuffles and stoops—when to the sea it comes—the eagle on the aged ocean. So is the man, as among the many comes, and has spokesmen few.\footnoteB{Cf. v. 25.}\evb
\evg


\bvg
\bva \alst{F}regna ok sęgja \hld\ skal \alst{f}róðra hvęrr, &
\ind sá’s vill \alst{h}ęitinn \alst{h}orskr; &
\alst{ęi}nn vita \hld\ né \alst{a}nnarr skal, &
\ind \alst{þ}jóð vęit ef \alst{þ}rír ’ró.\eva

\bvb Ask and speak shall each learned man, who wishes to be called sharp; one shall know, but not another: thirty\footnoteB{\emph{þjóð} lit. ‘people, nation’; cf. \Skaldskaparmal\ (TODO): \emph{þjóð eru þrír tigir} “thirty are a \emph{people}”.} know if there are three.\evb
\evg


\bvg
\bva \alst{R}íki sitt \hld\ skyli \alst{r}áðsnotra &
\ind hvęrr í \alst{h}ófi \alst{h}afa; &
þá hann þat \alst{f}innr, \hld\ es með \alst{f}rǿknum kømr, &
\ind at \alst{ę}ngi es \alst{ęi}nna hvatastr.\eva

\bvb His power should each counsel-clever man use in moderation; then he finds it—when among the bold he comes—that none is the briskest of all.\footnoteB{i.e., every man has his match. For the expression compare particularly \VolsungaSaga\ TODO \emph{þviat hverr sa, er med maurgum kemr, ma þat finna eitthvert sinn, at einge er einna hvataztr} “for each one who comes among the many must at some point find that none is the briskest of all.”}\evb
\evg


\bvg
\bva \alst{O}rða þęira, \hld\ es maðr \alst{ǫ}ðrum sęgir, &
\ind opt hann \alst{g}jǫld of \alst{g}etr.\eva

\bvb For those words which man to another says, he oft gets recompense.\evb
\evg


\bvg
\bva \alst{M}ikilsti snimma \hld\ kom’k í \alst{m}arga staði, &
\ind ęn til \alst{s}íð í \alst{s}uma; &
\alst{ǫ}l vas drukkit, \hld\ sumt vas \alst{ó}lagat; &
\ind sjaldan hittir \alst{l}ęiðr í \alst{l}ið.\eva

\bvb Much too early I came to many places, and too late to some. The ale was drunk, at other times yet unbrewed;\footnoteB{lit. “some [of it] was unbrewed”} seldom finds the loathsome man his place.\evb
\evg


\bvg
\bva \alst{H}ér ok \alst{h}var \hld\ myndi mér \alst{h}ęim of boðit, &
\ind ef þyrpta’k at \alst{m}ǫ́lungi \alst{m}at, &
eða \alst{t}vau lę́r hęngi \hld\ at hins \alst{t}ryggva vinar, &
\ind þar’s ek hafða \alst{ęi}tt \alst{e}tit.\eva

\bvb Here and there would I to a home be invited, if at no meal-time I needed food; or [if] two hams would hang at the trusty friend’s [home], where I one had eaten.\footnoteB{Not everyone is hospitable, especially with regards to food, which was valuable and had to be closely counted among subsistence farmers. The poet notes that even a “trusty friend” (might be sarcastic) would invite him to eat at his house more often if he brought more food than he ate.}\evb
\evg


\bvg
\bva \alst{Ę}ldr es baztr \hld\ með \alst{ý}ta sonum &
\ind ok \alst{s}ólar \alst{s}ýn, &
\alst{h}ęilyndi sitt, \hld\ ef \alst{h}afa náir, &
\ind án við \alst{l}ǫst at \alst{l}ifa.\eva

\bvb Fire is best among the sons of men, and the sight of the sun; one’s good health—if thou manage to keep it—and living without vice.\evb
\evg


\bvg
\bva Es-at maðr \alst{a}lls vesall, \hld\ þótt sé \alst{i}lla hęill, &
\ind \alst{s}umr es af \alst{s}onum \alst{s}ę́ll, &
\alst{s}umr af frę́ndum, \hld\ \alst{s}umr af fé ǿrnu, &
\ind sumr af \alst{v}erkum \alst{v}ęl.\eva

\bvb Man is not all wretched, though he of poor health be: someone is blessed by sons, someone by kinsmen, someone by ample \inx[C]{fee}, someone by works done well.\evb
\evg


\bvg
\bva Bętra ’s \alst{l}ifðum, \hld\ ok sę́l\alst{l}ifðum, &
\ind ęy getr \alst{k}vikr \alst{k}ú; &
\alst{ę}ld sá’k \alst{u}pp brinna \hld\ \alst{au}ðgum manni fyr, &
\ind ęn úti vas \alst{d}auðr fyr \alst{d}urum.\eva

\bvb ’Tis better with the living, and the blessed living: ever gets the quick\footnoteB{i.e. the living.} a cow.\footnoteB{A reference to the cattle-based economy (see also v. 76), the cow being used as a metonym. The meaning is that new opportunities always present themselves.} A fire\footnoteB{His funeral-pyre.} I saw burning high for a wealthy man, but outside he was dead before the door.\evb
\evg


\bvg
\bva \alst{H}altr ríðr \alst{h}rossi, \hld\ \alst{h}jǫrð rekr \alst{h}andarvanr, &
\ind daufr \alst{v}egr ok \alst{d}ugir; &
\alst{b}lindr es \alst{b}ętri, \hld\ an \alst{b}ręndr séi; &
\ind \alst{n}ýtr manngi \alst{n}ás.\eva

\bvb A halt man rides a horse; a handless drives a herd; a deaf fights and avails. Blind is better than be burnt; no man has use for a corpse.\evb
\evg


\bvg
\bva \alst{S}onr es bętri, \hld\ þótt sé \alst{s}íð of alinn &
\ind ęptir \alst{g}inginn \alst{g}uma; &
sjaldan \alst{b}autarstęinar \hld\ standa \alst{b}rautu nę́r, &
\ind nema ręisi \alst{n}iðr at \alst{n}ið.\eva

\bvb A son is better, although he late be born after a passed-on man\footnoteB{i.e. after the father is dead.}; seldom beat-stones\footnoteB{Large menhirs raised as memorial stones, later and especially in Upland decorated with Runic inscriptions.} near the highway stand, save by kinsman after kinsman raised.\evb
\evg


\bvg
\bva \edtext{\alst{T}vęir ’ru ęins hęrjar, \hld\ \alst{t}unga es hǫfuðs bani; &
mér ’s í \alst{h}eðin \alst{h}vęrn \hld\ \alst{h}andar vę́ni.}{\lemma{Tvęir \dots\ vę́ni}\Bfootnote{Whole v. undoubtedly a later insertion, the divergent meter is proof enough.}}\eva

\bvb Two are of one host;\footnoteB{\emph{hęrjar} gen. sg. of \emph{hęrr} ‘host, army’ may alternatively be read as the nom. pl. meaning ‘harriers, raiders,’ present in \emph{ęinhęrjar} (\inx[G]{Ownharriers}). Thus ‘two are the destroyers of one (i.e. the person)’.} the tongue is the head’s bane;\footnoteB{The tongue and the head are part of the same body and need each other, yet the former often leads to the demise of the latter. — For this phrase cf. especially the Old Swedish Heathen Law \parencite{Läffler1879}: \emph{Faldr þan orð havr giuit · Glöpr orða værstr · Tunga houuðbani · Liggi i vgildum acri} “Falls the one who has given the word—wickedness is the worst of words; the tongue the head’s bane-man—may he lie in an unpaid field (i.e. no weregild will be paid for him).”} in every cloak I expect a hand.\evb
\evg


\bvg
\bva \alst{N}ǫ́tt verðr fęginn, \hld\ sá’s \alst{n}esti trúir, &
\ind \alst{sk}ammar ’ru \alst{sk}ips ráar, &
\ind \alst{h}verf es \alst{h}austgríma; &
\alst{f}jǫlð of viðrir \hld\ á \alst{f}imm dǫgum, &
\ind ęn \alst{m}ęir á \alst{m}ánaði.\eva

\bvb At night he rejoices, who trusts on his provisions; short are the ship’s sailyards;\footnoteB{TODO: Write about the varying interpretations (Finnur, Cleasby, Skp) of this line.} ever-shifting is the autumn night. The weather shifts much in \inx[C]{five days},\footnoteB{See note to v. 51 and Encyclopedia.} but more in a month.\evb
\evg


\bvg
\bva \alst{V}ęit-a hinn, \hld\ es \alst{v}ę́tki \alst{v}ęit, &
\ind margr verðr \edtext{af \alst{au}rum}{\lemma{af aurum}\Afootnote{‘aflꜹðrom’ \emph{ms.}}} \alst{a}pi; &
maðr es \alst{au}ðigr, \hld\ annarr \alst{ó}auðigr, &
\ind skyli-t þann \alst{v}ítka \alst{v}áar.\eva

\bvb The one knows not, who nothing knows: many a man becomes by treasures the fool.\footnoteB{For \emph{api}, here “fool”, see \inx[C]{ape}.} A man is wealthy, another not wealthy; one oughtn’t to curse him for his woe.\evb
\evg


\bvg
\bva \alst{D}ęyr fé, \hld\ \alst{d}ęyja frę́ndr, &
\ind dęyr \alst{s}jalfr hit \alst{s}ama; &
ęn \alst{o}rðstírr \hld\ dęyr \alst{a}ldrigi &
\ind hvęim’s sér \alst{g}óðan \alst{g}etr.\eva

\bvb \inx[C]{fee}[Fee] dies, kinsmen die, oneself dies the same;\footnoteB{The power of this succinct merism may be less clear to the modern reader. In Germanic Iron Age society a man’s wealth was reckoned by how many heads of cattle (for which compare particularly English \emph{chattel} ‘tangible, movable property’ and the etymology of \emph{capital}) he owned, and his social power by the number of able male relatives ready to side with him in conflict. The meaning is thus: all your power will pass away, and so too must you. — For poetic analogues, see \textcite[99\psqq]{West2007}.} but a word-glory never dies, for whomever gets himself a good one.\evb
\evg


\bvg
\bva \alst{D}ęyr fé, \hld\ \alst{d}ęyja frę́ndr, &
\ind dęyr \alst{s}jalfr hit \alst{s}ama; &
\alst{e}k vęit \alst{ęi}nn \hld\ at \alst{a}ldrigi dęyr: &
\ind \alst{d}ómr of \alst{d}auðan hvęrn.\eva

\bvb Fee dies, kinsmen die, oneself dies the same. I know one that never dies: the \inx[C]{Doom} o’er each man dead.\evb
\evg


\bvg
\bva \alst{F}ullar grindr \hld\ sá’k fyr \alst{F}itjungs sonum, &
\ind nú bera þęir \alst{v}ánar \alst{v}ǫl; &
svá es \alst{au}ðr \hld\ sęm \alst{au}gabragð, &
\ind hann es \alst{v}altastr \alst{v}ina.\eva

\bvb Full pens I saw for the sons of Fitting; now they carry the staff of hope.\footnoteB{A beggar’s staff.} So is wealth like the twinkling of an eye; it is the ficklest of friends.\evb
\evg


\bvg
\bva \alst{Ó}snotr maðr, \hld\ es \alst{ęi}gnask getr &
\ind \alst{f}é eða \alst{f}ljóðs munuð; &
\alst{m}etnaðr hǫ́num þróask, \hld\ ęn \alst{m}anvit aldrigi; &
\ind framm gęngr hann \alst{d}rjúgt í \alst{d}ul.\eva

\bvb The unclever man, if he gets to own fee or a girl’s grace: his conceit flourishes, but never his manwit; far he goes forth in delusion.\evb
\evg

\sectionline

\bvg
\bva Þat es þá \alst{r}ęynt, \hld\ es þú at \alst{r}únum spyrr \hld\ \edtext{hinum \alst{r}ęginkunnum}{\lemma{hinum ręginkunnum ‘the ones born of the Reins’}\Bfootnote{This expression also appears on the C4th–6th Noleby stone; see Encyclopedia \inx[C]{rune}.}}, &
\ind þęim’s \alst{g}ęrðu \alst{g}innręgin &
\ind ok \alst{f}áði \alst{f}imbulþulr; &
\ind \alst{þ}á hęfr hann bazt, ef hann \alst{þ}ęgir.\eva

\bvb Then that is proven of which thou inquires the runes, the ones born of the Reins, those which the \inx[G]{gin-Reins} made, and the Fimblethyle \name{= Weden} painted. (Then he has it best, if he shuts up.)\footnoteB{This verse, dealing with runic magic, hardly fits into the previous or following section. It would on the other hand fit very well in the much later Rune-Tally. The last verse with its shift in person is likely to be an insert.}\evb
\evg

\sectionline

\section{Verses of practical advice, mostly in \Fornyrdislag.}

\bvg
\bva At \alst{k}veldi skal dag lęyfa, \hld\ \alst{k}onu es bręnnd es, &
\alst{m}ę́ki es ręyndr es, \hld\ \alst{m}ęy es gefin es, &
\alst{í}s es \alst{y}fir kømr, \hld\ \alst{ǫ}l es drukkit es.\eva

\bvb At evening shall one praise day, a woman when she is burned, a sword when it is tried, a maiden when she is given,\footnoteB{i.e. in marriage.} ice when one crosses over, ale when it is drunk.\evb
\evg


\bvg
\bva Í \alst{v}indi skal \alst{v}ið hǫggva, \hld\ \alst{v}eðri á sę́ róa, &
\alst{m}yrkri við \alst{m}an spjalla, \hld\ \alst{m}ǫrg eru dags augu, &
á \alst{sk}ip skal \alst{sk}riðar orka, \hld\ ęn á \alst{sk}jǫld til hlífar, &
\alst{m}ę́ki til hǫggs, \hld\ ęn \alst{m}ęy til kossa.\eva

\bvb In wind shall one cut wood, in storm row on the sea, in darkness meet with a maiden; many are the eyes of day. A ship shall one have for its speed, but a shield for shelter; a sword for striking, but a maiden for her kisses.\evb
\evg


\bvg
\bva Við \alst{ę}ld skal \alst{ǫ}l drekka, \hld\ ęn á \alst{í}si skríða, &
\alst{m}agran \alst{m}ar kaupa, \hld\ ęn \alst{m}ę́ki saurgan, &
\alst{h}ęima \alst{h}ęst fęita, \hld\ ęn \alst{h}und á búi.\eva

\bvb By fire shall one drink ale, and on the ice skate; buy a meager stallion, and a rusty sword; fatten the horse at home, and the hound in the household.\evb
\evg


\bvg
\bva \alst{M}ęyjar orðum \hld\ skyli \alst{m}anngi trúa, &
\ind né því’s \alst{k}veðr \alst{k}ona; &
\edtext{\edtext{þvít}{\Afootnote{\emph{om.} \FostrbroedhraSaga}} á \alst{h}verfanda \alst{h}véli \hld\ \edtext{vǫ́ru}{\Afootnote{er \FostrbroedhraSaga}} þęim \edtext{\alst{h}jǫrtu skǫpuð}{\lemma{hjǫrtu skǫpuð}\Afootnote{hjarta skapat \FostrbroedhraSaga}}, &
\ind \edtext{\alst{b}rigð}{\lemma{brigð}\Afootnote{ok brigð \FostrbroedhraSaga}} í \alst{b}rjóst of \edtext{lagið}{\Afootnote{‘laginn’ \FostrbroedhraSaga}}.}{\lemma{þvít \dots\ lagið}\Bfootnote{Quoted in slightly divergent form in \FostrbroedhraSaga\ (Thott 1768 4°\textsuperscript{x}, fol. 210r): \emph{“And then he remembered the ditty which had been composed about loose women: [...]”}}}\eva

\bvb The words of a maiden should no man believe, nor that which a woman sings. For on a spinning wheel were their hearts shaped; fickleness in their breasts was laid.\evb
\evg


\bvg
\bva \alst{B}restanda \alst{b}oga, \hld\ \alst{b}rinnanda loga, &
\alst{g}ínanda ulfi, \hld\ \alst{g}alandi krǫ́ku, &
\alst{r}ýtanda svíni, \hld\ \alst{r}ótlausum viði, &
\alst{v}axanda \alst{v}ági, \hld\ \alst{v}ellanda katli,\eva

\bvb The bursting bow, the burning flame, the gaping wolf, the crowing crow, the roaring swine, the rootless tree, the waxing wave, the swelling kettle,\evb
\evg


\bvg
\bva \alst{f}ljúganda \alst{f}lęini, \hld\ \alst{f}allandi bǫ́ru, &
\alst{í}si \alst{ęi}nnę́ttum, \hld\ \alst{o}rmi hringlęgnum, &
\alst{b}rúðar \alst{b}ęðmǫ́lum \hld\ eða \alst{b}rotnu sverði, &
\alst{b}jarnar lęiki \hld\ eða \alst{b}arni konungs, &
\alst{s}júkum kalfi, \hld\ \alst{s}jalfráða þrę́li, &
\alst{v}ǫlu \alst{v}ilmę́li, \hld\ \alst{v}al nýfęldum.\eva

\bvb the flying spear, the falling billow, the one-night old ice, the coiled-up serpent, the bed-speeches of a bride, or the broken sword, the play of a bear, or the child of a king, the sick calf, the freed slave, the pleasing speech of a wallow, newly felled corpses,\evb
\evg

In \Regius the following two verses come in the opposite order, but it is clear that 88 should conclude the old list of things not to trust. It is clear from its meter that 87 is a separate composition; it was probably inserted in between 86 and 88 by an inattentive scribe.

\bvg
\bva[88]\alst{b}róðurbana sínum \hld\ þótt á \alst{b}rautu mǿti, &
\alst{h}úsi \alst{h}alfbrunnu, \hld\ \alst{h}ęsti alskjótum, &
þá ’s \alst{jó}r \alst{ó}nýtr, \hld\ ef \alst{ęi}nn fótr brotnar; &
verðr-it maðr svá \alst{t}ryggr \hld\ at þessu \alst{t}rúi ǫllu.\eva

\bvb his brother’s bane-man—though on the highway they meet—a half-burned house, an all-fleet horse: then is the steed useless, if one foot breaks. There may be no man so trusting, that he trust in all this.\evb
\evg\stepcounter{stanza}


\bvg
\bva[87]\alst{A}kri \alst{á}rsǫ́num \hld\ trúi \alst{ę}ngi maðr, &
\ind né til \alst{s}nimma \alst{s}yni; &
\alst{v}eðr rę́ðr akri, \hld\ ęn \alst{v}it syni; &
\ind \alst{h}ę́tt es þęira \alst{h}várt.\eva

\bvb An early sown field ought no man to trust, nor too early\footnoteB{i.e. in life.} a son. The weather rules the field, but the wits the son; there is risk to both of them.\evb
\evg\stepcounter{stanza}

\sectionline

\section{Advice on love and Weden’s failed seduction of Billing’s maiden.}

\bvg
\bva Svá ’s \alst{f}riðr kvinna \hld\ þęira’s \alst{f}látt hyggja, &
sęm aki \alst{jó} \alst{ó}bryddum \hld\ á \alst{í}si hǫ́lum &
\alst{t}ęitum, \alst{t}vévetrum \hld\ ok sé \alst{t}amr illa, &
eða í \alst{b}yr óðum \hld\ \alst{b}ęiti stjórnlausu, &
eða skyli \alst{h}altr \alst{h}ęnda \hld\ \alst{h}ręin í þáfjalli.\eva

\bvb So is the love of women—those who falsely think—like one rode an unshod horse on slippery ice—a merry one, two winters old, and badly tamed—or in mad wind tacked a rudderless [ship], or [as] should a halt man catch a reindeer on a thawing mountain.\evb
\evg


\bvg
\bva \alst{B}ert nú mę́li’k, \hld\ því-at \alst{b}ę́ði vęit’k, &
\ind brigðr es \alst{k}arla hugr \alst{k}onum, &
þá \alst{f}ęgrst mę́lum, \hld\ es \alst{f}lást hyggjum; &
\ind þat tę́lir \alst{h}orska \alst{h}ugi.\eva

\bvb Plainly I now speak, for I know both [sides]: fickle is men’s thought towards women. We then speak the most fairly, when the most falsely we think; that entices sharp minds.\evb
\evg


\bvg
\bva \alst{F}agrt skal mę́la \hld\ ok \alst{f}é bjóða, &
\ind sá’s vill \alst{f}ljóðs ǫ́st \alst{f}áa, &
\alst{l}íki \alst{l}ęyfa \hld\ hins \alst{l}jósa mans, &
\ind sá \alst{f}ę́r, es \alst{f}ríar.\eva

\bvb Fairly shall speak, and offer \inx[C]{fee}, he who will earn a girl’s love; [he shall] praise the body of the light maiden; he gets, who woos.\footnoteB{i.e., ‘he who woos her gets her’.}\evb
\evg


\bvg
\bva \alst{Á}star firna \hld\ skyli \alst{ę}ngi maðr &
\ind \alst{a}nnan \alst{a}ldrigi; &
opt fáa á \alst{h}orskan, \hld\ es á \alst{h}ęimskan né fáa, &
\ind \alst{l}ostfagrir \alst{l}itir.\eva

\bvb For [his] love should no man ever blame another; oft they seize the sharp one, when they seize not the foolish one, lust-fair looks.\footnoteB{Looks so fair}\evb
\evg


\bvg
\bva \alst{Ęy}vitar firna, \hld\ es maðr \alst{a}nnan skal, &
\ind þess es of margan \alst{g}ęngr \alst{g}uma; &
\alst{h}ęimska ór \alst{h}orskum \hld\ gęrir \alst{h}ǫlða sonu &
\ind sá hinn \alst{m}átki \alst{m}unr.\eva

\bvb For nothing shall man ever blame another, which happens to many a man; fools out of sharp ones makes—among the sons of men—that mighty delight \ken{love}.\evb
\evg


\bvg
\bva \alst{H}ugr ęinn þat vęit, \hld\ es býr \alst{h}jarta nę́r, &
\ind ęinn es hann \alst{s}ér of \alst{s}efa; &
øng es \alst{s}ótt verri \hld\ hvęim \alst{s}notrum manni &
\ind an sér \alst{ø}ngu at \alst{u}na.\eva

\bvb The thought alone knows what dwells close to the heart; he is alone with his mind. No ailment is worse for any clever man, than to be content with nothing.\evb
\evg


\bvg
\bva Þat þá \alst{r}ęyndak, \hld\ es í \alst{r}ęyri sat’k, &
\ind ok vę́tta’k \alst{m}íns \alst{m}unar, &
\alst{h}old ok \alst{h}jarta \hld\ vas mér hin \alst{h}orska mę́r, &
\ind þęygi hana at \alst{h}ęldr \alst{h}ęf’k.\eva

\bvb That I then discovered, as I sat in the reed, and awaited my pleasure. My flesh and heart that sharp maiden was; I have her none the more.\evb
\evg


\bvg
\bva \alst{B}illings męy \hld\ ek fann \alst{b}ęðjum á &
\ind \alst{s}ólhvíta \alst{s}ofa; &
\alst{ja}rls \alst{y}nði \hld\ þótti mér \alst{ę}kki vesa &
\ind nema við þat \alst{l}ík at \alst{l}ifa.\eva

\bvb Billing’s maiden I found on the beds, sun-white, sleeping. An earl’s pleasure seemed me naught to be, save for living alongside that body.\evb
\evg


\bvg {\small [Billing’s maiden:]}
\bva „\alst{Au}k nę́r \alst{a}ptni \hld\ skalt-u \alst{Ó}ðinn koma, &
\ind ef vilt þér \alst{m}ę́la \alst{m}an, &
\alst{a}lt eru \alst{ó}skǫp, \hld\ nema \alst{ęi}n vitim &
\ind \alst{s}likan lǫst \alst{s}aman.“\eva

\bvb “And by evening shalt thou, Weden, come, if thou wilt for thee have the maiden \ken*{= me}; all is misshapen, if we might not know one such vice together.”\evb
\evg


\bvg
\bva \alst{A}ptr ek hvarf \hld\ ok \alst{u}nna þóttumk &
\ind \alst{v}ísum \alst{v}ilja frá; &
\alst{h}itt ek \alst{h}ugða, \hld\ at \alst{h}afa mynda’k &
\ind \alst{g}ęð hęnnar alt ok \alst{g}aman.\eva

\bvb Back I turned—and thought myself to love [her]—away from my wise will; this I thought, that I would own her senses all and pleasure.\evb
\evg


\bvg
\bva Svá kom’k \alst{n}ę́st, \hld\ at hin \alst{n}ýta vas &
\ind \alst{v}ígdrótt ǫll of \alst{v}akin; &
með \alst{b}rinnǫndum ljósum \hld\ ok \alst{b}ornum viði, &
\ind svá vas mér \alst{v}ílstígr of \alst{v}itaðr.\eva

\bvb So I came next, as was the useful\footnoteB{Sarcastic.} battle-people all awake; with burnings lights and carried wood;\footnoteB{They were presumably armed with sticks.} so was for me a miserable path\footnoteB{Ambiguous whether it refers to the beating he would have received at the hands of the men had he entered, or to his walk of shame away from the hall.} marked out.\evb
\evg


\bvg
\bva \alst{Au}k nę́r morni, \hld\ es vas’k \alst{ę}nn of kominn, &
\ind þá vas \alst{s}aldrótt of \alst{s}ofin; &
\alst{g}ręy ęitt þá fann’k \hld\ hinnar \alst{g}óðu konu &
\ind \alst{b}undit \alst{b}ęðjum á.\eva

\bvb And by morning, when I was come again, then was the hall-people asleep. A lone bitch I then found, owned by the good woman, bound on the beds.\evb
\evg


\bvg
\bva Mǫrg es \alst{g}óð mę́r, \hld\ ef \alst{g}ǫrva kannar, &
\ind \alst{h}ugbrigð við \alst{h}ali; &
þá þat \alst{r}ęynda’k, \hld\ es hit \alst{r}áðspaka &
\ind tęygða’k á \alst{f}lę́rðir \alst{f}ljóð. &
\alst{h}ǫ́ðungar \alst{h}vęrrar \hld\ lęitaði mér hit \alst{h}orska man &
\ind ok hafða’k þess \alst{v}ę́tki \alst{v}ífs.\eva

\bvb Many a good maiden—if one knows her clearly—is heart-fickle towards men; that I learned when into sins I lured that counsel-clever woman. All sorts of disgraces that sharp girl sought out for me, and I had naught of that wife.\evb
\evg

\sectionline

\section{Weden’s obtaining of the mead of poetry}

This story is told in \Gylfaginning. Weden under the name Baleworker used a drill named \inx[P]{Rate} in order to drill into the mountains. TODO.

\bvg
\bva Hęima \alst{g}laðr \alst{g}umi \hld\ ok við \alst{g}ęsti ręifr, &
\ind \alst{s}viðr skal of \alst{s}ik vesa; &
\alst{m}innigr ok \alst{m}ǫ́lugr, \hld\ ef vill \alst{m}argfróðr vesa; &
\ind opt skal \alst{g}óðs \alst{g}eta; &
\alst{f}imbul\alst{f}ambi hęitir, \hld\ sá’s \alst{f}átt kann sęgja; &
\ind þat es \alst{ó}snotrs \alst{a}ðal.\eva

\bvb At home shall man be glad, and cheerful towards a guest; wise about himself. Remembering and speaking, if he wishes to be many-learned; oft shall he speak of good. A fimble-fool is called he who can say little; that is an unclever man’s nature.\evb
\evg


\bvg
\bva Hinn \alst{a}ldna \alst{jǫ}tun sóttak, \hld\ nú em’k \alst{a}ptr of kominn; &
\ind fátt gat’k \alst{þ}ęgjandi \alst{þ}ar; &
\alst{m}ǫrgum orðum \hld\ \alst{m}ę́lta’k í minn frama &
\ind í \alst{S}uttungs \alst{s}ǫlum.\eva

\bvb The old ettin I sought, now am I come back; I got little silence there. Many words I spoke to my furtherance, in the halls of Sutting.\evb
\evg


\bvg
\bva \alst{G}unnlǫð mér of \alst{g}af \hld\ \alst{g}ollnum stóli á &
\ind \alst{d}rykk hins \alst{d}ýra mjaðar; &
\alst{i}ll \alst{i}ðgjǫld \hld\ lét’k hana \alst{ę}ptir hafa &
\ind síns hins \alst{h}ęila \alst{h}ugar. &
\ind (síns hins \alst{s}vára \alst{s}efa).\eva

\bvb \inx[P]{Guthlathe} did give me, on the golden chair, a drink of the dear mead; evil recompense I let her have afterwards, for her whole heart; for her severe affection.\evb
\evg


\bvg
\bva \alst{R}ata munn \hld\ létumk \alst{r}úms of fáa &
\ind ok of \alst{g}rjót \alst{g}naga; &
\alst{y}fir ok \alst{u}ndir \hld\ stóðumk \alst{jǫ}tna vegir, &
\ind svá \alst{h}ę́tta’k \alst{h}ǫfði til.\eva

\bvb Rate’s mouth I let bring me room, and gnaw away at the rubble. Over and under me stood the roads of the ettins \ken{mountains}; so I risked my head.\evb
\evg


\bvg
\bva \alst{V}ęl kęypts hlutar \hld\ hęf’k \alst{v}ęl notit; &
\ind \alst{f}ás es \alst{f}róðum vant; &
því’t \alst{Ó}ðrerir \hld\ nú \alst{u}pp ’s kominn &
\ind á \alst{a}lda vés \alst{ja}rðar.\eva

\bvb The well purchased thing \ken{mead of poetry} I have used well; little is lacking for the learned—for Woderearer is now come up onto the earths of the \inx[C]{wigh} of men \ken*{Middenyard}.\footnoteB{Weden says that he has made good use of the mead of poetry, since it can now be tapped and served by wise humans.}\evb
\evg


\bvg
\bva \alst{I}fi es mér á, \hld\ at vę́ra’k \alst{ę}nn kominn &
\ind \alst{jǫ}tna gǫrðum \alst{ó}r, &
ef \alst{G}unnlaðar né nyta’k, \hld\ hinnar \alst{g}óðu konu, &
\ind es lǫgðumk \alst{a}rm \alst{y}fir.\eva

\bvb There is doubt in me, that I were still come out of the yards of the Ettins if Guthlathe I had not used: that good woman, whom I laid my arm over.\evb
\evg


\bvg
\bva Hins \alst{h}indra dags \hld\ gingu \alst{h}rímþursar &
\ind \alst{H}áva ráðs at fregna, &
\ind (\alst{H}áva \alst{h}ǫllu í,) &
at \alst{B}ǫlverki spurðu, \hld\ ef vę́ri með \alst{b}ǫndum kominn &
\ind eða hęfði hǫ́num \alst{S}uttungr of \alst{s}óit.\eva

\bvb The other day went the Rime-Thurses to ask for the counsel of the High One; in the hall of High One. About Baleworker \name{= Weden} \ken*{me} they asked, if he \ken*{I} were come among the bonds \name{gods}, or if Suttung had slain him.\evb
\evg


\bvg
\bva Baugęið \alst{Ó}ðinn \hld\ hygg at \alst{u}nnit hafi, &
\ind hvat skal hans \alst{t}ryggðum \alst{t}rúa? &
\alst{S}uttung \alst{s}vikvinn \hld\ hann lét \alst{s}umbli frá &
\ind ok \alst{g}rǿtta \alst{G}unnlǫðu.\eva

\bvb A \inx[C]{bigh-oath} I ween that Weden has sworn; how shall one trust his truces? He let Sutting walk betrayed from the simble, and Guthlathe made to weep.\evb
\evg

\sectionline

\section{The Speeches of Loddfathomer}

\emph{Loddfáfnismǫ́l}. Advice given to Loddfathomer. In \Regius\ this section is marked out with a large initial, like the beginnings of separate poems.

\sectionline

\bvg
\bva Mál ’s at \alst{þ}ylja \hld\ \alst{þ}ular stóli á; &
\ind \alst{U}rðar brunni \alst{a}t &
\alst{s}á’k ok þagða’k, \hld\ \alst{s}á’k ok hugða’k, &
\ind hlýdda’k á \alst{m}anna \alst{m}ál; &
of \alst{r}únar hęyrða’k dǿma, \hld\ né umb \alst{r}ǫ́ðum þǫgðu &
\ind \alst{H}áva \alst{h}ǫllu at, &
\ind \alst{H}áva \alst{h}ǫllu í &
\ind hęyrða’k \alst{s}ęgja \alst{s}vá:\eva

\bvb ’Tis time to \inx[C]{thill}, upon the chair of the \inx[C]{thyle}. At the well of Weird, I saw and I shut up: I saw and I thought: I heeded the matters of men. Of runes I heard them speak, nor about counsels were they silent, at the hall of the High One \name{= Weden} \ken*{= Walhall}, in the hall of the High One, I heard [them] say thus:\footnoteB{The speaker, describing himself as a thyle (\emph{þulr} ‘sage, chanter of memorized poetry’), says that he will relate what he has heard said at the hall of the High One \name{= Weden} \ken*{= Walhall}. Considering the location, it seems almost certain that the giver of this advice was \inx[P]{Weden}. The receiver of the advice, \inx[P]{Loddfathomer} (see Encyclopedia for etymologies), is otherwise unknown.}\evb
\evg


\bvg
\bva \alst{R}ǫ́ðumk þér Loddfáfnir, \hld\ at þú \alst{r}ǫ́ð nemir, &
\ind \alst{n}jóta munt ef \alst{n}emr, &
\ind þér munu \alst{g}óð ef \alst{g}etr: &
\alst{n}ǫ́tt þú rís-at, \hld\ nema á \alst{n}jósn séir, &
\ind eða lęitir þér \alst{i}nnan \alst{ú}t staðar.\eva

\bvb I counsel thee Loddfathomer, that thou learn the counsels; thou wilt benefit if thou learnest; they will be good for thee if thou gettest: At night thou rise not, unless at scouting thou be, or thou art forced out from within a place.\footnoteB{Very difficult phrase. Possibly a euphemism for needing to relieve oneself?}\evb
\evg


\bvg
\bva \alst{R}ǫ́ðumk þér Loddfáfnir, \hld\ at þú \alst{r}ǫ́ð nemir, &
\ind \alst{n}jóta munt ef \alst{n}emr, &
\ind þér munu \alst{g}óð ef \alst{g}etr: &
\alst{f}jǫlkunnigri konu \hld\ skal-at-tu í \alst{f}aðmi sofa, &
\ind svá’t hon \alst{l}yki þik \alst{l}iðum. &
Hón svá \alst{g}ęrir \hld\ at þú \alst{g}áir ęigi &
\ind \alst{þ}ings né \alst{þ}jóðans máls; &
\alst{m}at þú vill-at \hld\ né \alst{m}anskis gaman &
\ind fęrr þú \alst{s}orgafullr at \alst{s}ofa.\eva

\bvb I counsel thee Loddfathomer, that thou learn the counsels; thou wilt benefit if thou learnest; they will be good for thee if thou gettest: In the bosom of a \inx[C]{feal-cunning} woman shalt thou never sleep, so that she might lock you in [her?] limbs. She makes it so that thou heed not the \inx[C]{Thing}, nor the ruler’s speech; food wilt thou not [have], nor any man’s pleasure; thou farest sorrowful to sleep.\evb
\evg


\bvg
\bva \alst{R}ǫ́ðumk þér Loddfáfnir, \hld\ at þú \alst{r}ǫ́ð nemir, &
\ind \alst{n}jóta munt ef \alst{n}emr, &
\ind þér munu \alst{g}óð ef \alst{g}etr: &
\alst{a}nnars konu \hld\ tęyg þér \alst{a}ldrigi &
\ind \alst{ęy}rarúnu \alst{a}t.\eva

\bvb I counsel thee Loddfathomer, that thou learn the counsels; thou wilt benefit if thou learnest; they will be good for thee if thou gettest: Never lure another man’s woman into [becoming] thy ear-whisperer \ken{lover}.\evb
\evg


\bvg
\bva \alst{R}ǫ́ðumk þér Loddfáfnir, \hld\ at þú \alst{r}ǫ́ð nemir, &
\ind \alst{n}jóta munt ef \alst{n}emr, &
\ind þér munu \alst{g}óð ef \alst{g}etr: &
á \alst{f}jalli eða \alst{f}irði, \hld\ ef þik \alst{f}ara tíðir, &
\ind fásk-tu at \alst{v}irði \alst{v}ęl.\eva

\bvb I counsel thee Loddfathomer, that thou learn the counsels; thou wilt benefit if thou learnest; they will be good for thee if thou gettest: on the fell or firth—if thou desire to travel—get thyself a good meal.\evb
\evg


\bvg
\bva \alst{R}ǫ́ðumk þér Loddfáfnir, \hld\ at þú \alst{r}ǫ́ð nemir, &
\ind \alst{n}jóta munt ef \alst{n}emr, &
\ind þér munu \alst{g}óð ef \alst{g}etr: &
\alst{i}llan mann \hld\ lát \alst{a}ldrigi &
\ind \edtext{\alst{ó}hǫpp at þér \alst{v}ita}{\Bfootnote{Excluding some corrpution (but there hardly seems to be any) this line is probably one the few undisputed cases of \emph{v-} alliterating with a vowel.}}. &
af \alst{i}llum manni \hld\ fę́r \alst{a}ldrigi &
\ind \alst{g}jǫld hins \alst{g}óða hugar.\eva

\bvb I counsel thee Loddfathomer, that thou learn the counsels; thou wilt benefit if thou learnest; they will be good for thee if thou gettest: An evil man let thou never know of thy misfortunes. From an evil man receivest thou never recompense for thy good heart.\evb
\evg


\bvg
\bva \alst{O}farla bíta \hld\ sá’k \alst{ęi}num hal &
\ind \alst{o}rð \alst{i}llrar konu, &
\alst{f}lárǫ́ð tunga \hld\ varð hǫ́num at \alst{f}jǫrlagi &
\ind ok þęygi of \alst{s}anna \alst{s}ǫk.\eva

\bvb Biting I saw, high up on one man, the words of an evil woman; a deceit-counseling tongue brought his life to end, and in no way over a truthful charge.\evb
\evg


\bvg
\bva \alst{R}ǫ́ðumk þér Loddfáfnir, \hld\ at þú \alst{r}ǫ́ð nemir, &
\ind \alst{n}jóta munt ef \alst{n}emr, &
\ind þér munu \alst{g}óð ef \alst{g}etr: &
\alst{v}ęizt, ef \alst{v}in átt, \hld\ þann’s \alst{v}ęl trúir, &
\ind \alst{f}ar þú at \alst{f}inna opt; &
\edtext{því’t \alst{h}rísi vęx \hld\ ok \alst{h}ǫ́u grasi}{\lemma{hrísi vęx ok hǫ́u grasi ‘with brushwood and with tall grass grows’}\Bfootnote{Identical with \Grimnismal\ 17/1.}} &
\ind \alst{v}egr, es \alst{v}ę́tki trøðr,\eva

\bvb I counsel thee Loddfathomer, that thou learn the counsels; thou wilt benefit if thou learnest; they will be good for thee if thou gettest: Know, if thou have a friend, one on which thou well trust, journey to find him oft; for with brushwood and tall grass grows the way which no man treads.\evb
\evg


\bvg
\bva \alst{R}ǫ́ðumk þér Loddfáfnir, \hld\ at þú \alst{r}ǫ́ð nemir, &
\ind \alst{n}jóta munt ef \alst{n}emr, &
\ind þér munu \alst{g}óð ef \alst{g}etr: &
\alst{g}óðan mann \hld\ tęyg þér at \alst{g}amanrúnum &
\ind ok nem \alst{l}íknargaldr meðan \alst{l}ifir.\eva

\bvb I counsel thee Loddfathomer, that thou learn the counsels; thou wilt benefit if thou learnest; they will be good for thee if thou gettest: Lure a good man to thee through pleasure-runes,\footnoteB{Pleasurable conversation. Cf. 128.} and learn healing-galders while thou livest.\evb
\evg


\bvg
\bva \alst{R}ǫ́ðumk þér Loddfáfnir, \hld\ at þú \alst{r}ǫ́ð nemir, &
\ind \alst{n}jóta munt ef \alst{n}emr, &
\ind þér munu \alst{g}óð ef \alst{g}etr: &
\alst{v}in þínum \hld\ \alst{v}es aldrigi &
\ind \alst{f}yrri at \alst{f}laumslitum. &
\alst{s}org etr hjarta, \hld\ ef þú \alst{s}ęgja né náir &
\ind \alst{ęi}nhvęrjum \alst{a}llan hug.\eva

\bvb I counsel thee Loddfathomer, that thou learn the counsels; thou wilt benefit if thou learnest; they will be good for thee if thou gettest: With thy friend be thou never the first to tear apart the company. Sorrow eats thy heart if thou cannot speak to anyone thy whole mind.\footnoteB{cf. v. 122.}\evb
\evg


\bvg
\bva \alst{R}ǫ́ðumk þér Loddfáfnir, \hld\ at þú \alst{r}ǫ́ð nemir, &
\ind \alst{n}jóta munt ef \alst{n}emr, &
\ind þér munu \alst{g}óð ef \alst{g}etr: &
orðum \alst{sk}ipta \hld\ \alst{sk}alt aldrigi &
\ind við \alst{ó}svinna \alst{a}pa.\eva

\bvb I counsel thee Loddfathomer, that thou learn the counsels; thou wilt benefit if thou learnest; they will be good for thee if thou gettest: Words shalt thou never exchange with unwise apes.\evb
\evg


\bvg
\bva Því’t af illum \alst{m}anni \hld\ \alst{m}unt aldrigi &
\ind \alst{g}óðs laun of \alst{g}eta, &
ęn \alst{g}óðr maðr \hld\ mun þik \alst{g}ęrva męga &
\ind \edtext{\alst{l}íknfastan}{\lemma{líknfastan ‘health-firm’}\Bfootnote{A cpd. from \emph{líkn} \ONP: ‘mercy, compassion, relief, comfort, help’ and \emph{fastr} ‘fast, firm’. \textcite{LaFargeGlossary} give a tentative ‘assured of favour’, while \CV\ gives ‘fast in goodwill, beloved’. I read it as literally as possible, since the word \emph{líkn} has some connections with healing.}} at \alst{l}ofi.\eva

\bvb For from an evil man wilt thou never get a reward for thy goodness, but a good man will know make thee health-firm by [his] praise.\evb
\evg


\bvg
\bva \alst{S}ifjum ’s þá blandit \hld\ hvęrr es \alst{s}ęgja rę́ðr &
\ind \alst{ęi}num \alst{a}llan hug; &
alt es \alst{b}ętra \hld\ an sé \alst{b}rigðum at vesa: &
\ind es-a sá \alst{v}inr es \alst{v}ilt ęitt sęgir.\eva

\bvb Kinship is then blended,\footnoteB{cf. v. 44.} when any man decides to speak to one man his whole mind. Everything is better than to be among the fickle; he is no friend, who speaks that which is wanted alone.\evb
\evg


\bvg
\bva \alst{R}ǫ́ðumk þér Loddfáfnir, \hld\ at þú \alst{r}ǫ́ð nemir, &
\ind \alst{n}jóta munt ef \alst{n}emr, &
\ind þér munu \alst{g}óð ef \alst{g}etr: &
þrimr orðum sęnna \hld\ skal-at-tu þér við verra mann, &
\ind opt hinn \alst{b}ętri \alst{b}ilar. &
\ind þá’s hinn \alst{v}erri \alst{v}egr.\eva

\bvb I counsel thee Loddfathomer, that thou learn the counsels; thou wilt benefit if thou learnest; they will be good for thee if thou gettest: With three words shalt thou not flyte with a worse man;\footnoteB{i.e. ‘not even with three words’.} oft the better one breaks when the worse one strikes.\evb
\evg


\bvg
\bva \alst{R}ǫ́ðumk þér Loddfáfnir, \hld\ at þú \alst{r}ǫ́ð nemir, &
\ind \alst{n}jóta munt ef \alst{n}emr, &
\ind þér munu \alst{g}óð ef \alst{g}etr: &
\alst{sk}ósmiðr þú verir \hld\ né \alst{sk}ęptismiðr, &
\ind nema \alst{s}jǫlfum þér \alst{s}éir. &
\alst{Sk}ór ’s \alst{sk}apaðr illa\hld\ eða \alst{sk}apt sé vrangt, &
\ind þá ’s þér \alst{b}ǫls \alst{b}eðit.\eva

\bvb I counsel thee Loddfathomer, that thou learn the counsels; thou wilt benefit if thou learnest; they will be good for thee if thou gettest: Thou ought not to be a shoe-maker nor shaft-maker, unless thou be one for thyself. [If] the shoe is shaped badly or the shaft be crooked, then for thee a \inx[C]{bale} is bidden.\footnoteB{i.e. ‘the customer will put a curse you’.}\evb
\evg


\bvg
\bva \alst{R}ǫ́ðumk þér Loddfáfnir, \hld\ at þú \alst{r}ǫ́ð nemir, &
\ind \alst{n}jóta munt ef \alst{n}emr, &
\ind þér munu \alst{g}óð ef \alst{g}etr: &
hvars þú \alst{b}ǫl kant, \hld\ kveð þér \alst{b}ǫlvi at &
\ind ok gefat þínum \alst{f}jǫ́ndum \alst{f}rið.\eva

\bvb I counsel thee Loddfathomer, that thou learn the counsels; thou wilt benefit if thou learnest; they will be good for thee if thou gettest: Where thou a bale knowest, declare it to be a bale, and give not thy enemies peace.\footnoteB{i.e. ‘if somebody puts a curse on you, do not ignore it, but respond forcefully’, though it should be noted that the verse has often been interpreted as a command to call out evil, even when done towards somebody else, and there is nothing in it that goes against that reading.}\evb
\evg


\bvg
\bva \alst{R}ǫ́ðumk þér Loddfáfnir, \hld\ at þú \alst{r}ǫ́ð nemir, &
\ind \alst{n}jóta munt ef \alst{n}emr, &
\ind þér munu \alst{g}óð ef \alst{g}etr: &
\alst{i}llu fęginn \hld\ ves þú \alst{a}ldrigi, &
\ind \edtext{ęn lát þér at \alst{g}óðu \alst{g}etit}{\lemma{ęn lát þér at góðu getit ‘but rather let thyself be pleased by good’}\Bfootnote{This construction is equivalent to the sense ACC. A. IV. in \CV.}}.\eva

\bvb I counsel thee Loddfathomer, that thou learn the counsels; thou wilt benefit if thou learnest; they will be good for thee if thou gettest: Gladdened by evil be thou never, but let thyself be pleased by good.\evb
\evg


\bvg
\bva \alst{R}ǫ́ðumk þér Loddfáfnir, \hld\ at þú \alst{r}ǫ́ð nemir, &
\ind \alst{n}jóta munt ef \alst{n}emr, &
\ind þér munu \alst{g}óð ef \alst{g}etr: &
\alst{u}pp líta \hld\ skal-at-tu í \alst{o}rrostu; &
\alst{g}jalti \alst{g}líkir \hld\ verða \alst{g}umna synir &
\ind síðr þitt of \alst{h}ęilli \alst{h}alir.\eva

\bvb I counsel thee Loddfathomer, that thou learn the counsels; thou wilt benefit if thou learnest; they will be good for thee if thou gettest: Up shalt thou not look in battle—alike to a madman become the sons of men—lest men bewitch thy [sense/life/face].\footnoteB{A very difficult verse. \CV\ explains \emph{gjalti} as an old dative of \emph{gǫltr} ‘boar, hog’, and thus sees the closely related phrase \emph{verða at gjalti} as “‘to be turned into a hog’, i.e. ‘to turn mad with terror’, esp. in a fight”. The vowel breaking is however unexpected here, since \emph{gǫltr} (< Proto-Norse \emph{*galtuʀ}) is an u-stem, which makes the stem-vowel in the dat. sg. \emph{gęlti} (< \emph{*galtiu}, cf. \textbf{kunimudiu}, dat. sg. of \emph{*Kunimunduʀ}, on the Tjurkö 1 bracteate) the result of i-umlaut rather than an original short \emph{*e}.

\textcite{LaFargeGlossary} instead explains the word as a borrowing from Old Irish \emph{geilt} ‘insane, mad’. \textcite{PettitEdda} follows this, and arguess that the whole theme of the verse probably be of Celtic origin, giving several examples from Celtic literature of warriors going mad upon looking up into the sky during battle. In this case the men (\emph{halir}, which word seems to have an association with warriors; cf. 36–37, 49) would be to quote Pettit some sort of “supernatural sky warriors”, in my opinion most likely the \inx[G]{Ownharriers}.}\evb
\evg


\bvg
\bva \alst{R}ǫ́ðumk þér Loddfáfnir, \hld\ at þú \alst{r}ǫ́ð nemir, &
\ind \alst{n}jóta munt ef \alst{n}emr, &
\ind þér munu \alst{g}óð ef \alst{g}etr: &
Ef vilt þér góða \alst{k}onu \hld\ \alst{k}vęðja at \edtext{gamanrúnum}{\lemma{gamanrúnum ‘pleasure-runes’}\Bfootnote{While easily interpreted as ‘intercourse’, the word is used in 118 with a decidedly non-sexual meaning. It probably just means ‘good, light-hearted conversation’.}} &
\ind ok \alst{f}á \alst{f}ǫgnuð af, &
\alst{f}ǫgru skalt hęita \hld\ ok láta \alst{f}ast vesa; &
\ind lęiðisk manngi \alst{g}ótt ef \alst{g}etr.\eva

\bvb I counsel thee Loddfathomer, that thou learn the counsels; thou wilt benefit if thou learnest; they will be good for thee if thou gettest: If thou wilt for thee welcome a good woman to pleasure-runes, and receive good cheer from [her]; fair things shalt thou promise, and let it be fast; none loathes a good thing if one gets it.\evb
\evg


\bvg
\bva \alst{R}ǫ́ðumk þér Loddfáfnir, \hld\ at þú \alst{r}ǫ́ð nemir, &
\ind \alst{n}jóta munt ef \alst{n}emr, &
\ind þér munu \alst{g}óð ef \alst{g}etr: &
\alst{v}aran bið’k þik \alst{v}esa \hld\ ok ęigi of\alst{v}aran, &
ves þú við \alst{ǫ}l varastr, \hld\ ok við \alst{a}nnars konu &
ok við \alst{þ}at hit \alst{þ}riðja, \hld\ at \alst{þ}jófar né lęiki.\eva

\bvb I counsel thee Loddfathomer, that thou learn the counsels; thou wilt benefit if thou learnest; they will be good for thee if thou gettest: Wary I ask thee to be, and not over-wary; be wariest with ale, and with another man’s woman, and with the third, that thieves do not outplay [thee].\evb
\evg


\bvg
\bva \alst{R}ǫ́ðumk þér Loddfáfnir, \hld\ at þú \alst{r}ǫ́ð nemir, &
\ind \alst{n}jóta munt ef \alst{n}emr, &
\ind þér munu \alst{g}óð ef \alst{g}etr: &
at \alst{h}áði né \alst{h}látri \hld\ \alst{h}af aldrigi &
\ind \alst{g}ęst né \alst{g}anganda.\eva

\bvb I counsel thee Loddfathomer, that thou learn the counsels; thou wilt benefit if thou learnest; they will be good for thee if thou gettest: In mockery or laughter have thou never a guest nor wanderer.\evb
\evg


\bvg
\bva \alst{O}pt vitu \alst{ó}gǫrla, \hld\ þęir’s sitja \alst{i}nni fyr, &
\ind hvęrs þęir ’ru \alst{k}yns es \alst{k}oma; &
es-at maðr svá \alst{g}óðr \hld\ at \alst{g}alli né fylgi, &
\ind né svá \alst{i}llr at \alst{ęi}nu-gi dugi.\eva

\bvb They oft hardly know, who sit inside, of what sort those men are who come; no man is so good that no flaw follows him, nor so evil that he to nothing avails.\evb
\evg


\bvg
\bva \alst{R}ǫ́ðumk þér Loddfáfnir, \hld\ at þú \alst{r}ǫ́ð nemir, &
\ind \alst{n}jóta munt ef \alst{n}emr, &
\ind þér munu \alst{g}óð ef \alst{g}etr: &
at \alst{h}ǫ́rum þul \hld\ \alst{h}lę́ aldrigi, &
\ind opt ’s \alst{g}ótt þat’s \alst{g}amlir kveða, &
opt ór \alst{sk}ǫrpum bęlg \hld\ \alst{sk}ilin orð koma &
\ind þęim’s \alst{h}angir með \alst{h}ǫ́um &
\ind ok \alst{sk}ollir með \alst{sk}rǫ́um, &
\ind ok \alst{v}áfir með \alst{v}ílmǫgum.\eva

\bvb I counsel thee Loddfathomer, that thou learn the counsels; thou wilt benefit if thou learnest; they will be good for thee if thou gettest: At a hoary thyle laugh thou never; oft ’tis good, that which the old sing. Oft out of a scorched leather discerning words come; out of that one that hangs with hides, and dangles with dry skins, and sways among lads of toil \ken{thralls}.\footnoteB{TODO: Some note on this. \emph{vilmǫgum} meaning ‘veal-stomachs’? Cf. Crawford’s video on this.}\evb
\evg


\bvg
\bva \alst{R}ǫ́ðumk þér Loddfáfnir, \hld\ at þú \alst{r}ǫ́ð nemir, &
\ind \alst{n}jóta munt ef \alst{n}emr, &
\ind þér munu \alst{g}óð ef \alst{g}etr: &
\alst{g}ęst þú né \alst{g}ęyj-a \hld\ né á \alst{g}rind hrę́kir; &
\ind get þú \alst{v}ǫ́luðum \alst{v}ęl.\eva

\bvb I counsel thee Loddfathomer, that thou learn the counsels; thou wilt benefit if thou learnest; they will be good for thee if thou gettest: Bark not at a guest, nor spit at the gate;\footnoteB{Behind which the guest stands, waiting for the farmer to open.} furnish the impoverished well.\evb
\evg


\bvg
\bva \alst{R}amt es þat tré, \hld\ es \alst{r}íða skal &
\ind \alst{ǫ}llum at \alst{u}pploki; &
\alst{b}aug þú gef \hld\ eða þat \alst{b}iðja mun &
\ind þér \alst{l}ę́s hvęrs á \alst{l}iðu.\eva

\bvb Strong is that wood which shall swing to open for all;\footnoteB{i.e. the beam of the gate in front of the farm.} give a bigh, or it will bid thee every kind of deceit onto thy limbs.\evb
\evg


\bvg
\bva \alst{R}ǫ́ðumk þér Loddfáfnir, \hld\ at þú \alst{r}ǫ́ð nemir, &
\ind \alst{n}jóta munt ef \alst{n}emr, &
\ind þér munu \alst{g}óð ef \alst{g}etr: &
hvar’s \alst{ǫ}l drekkir \hld\ kjós þér \alst{ja}rðar męgin, &
því’t \alst{jǫ}rð tękr við \alst{ǫ}lðri, \hld\ ęn \alst{ę}ldr við sóttum, &
\alst{ęi}k við \alst{a}bbindi, \hld\ \alst{a}x við fjǫlkyngi, &
\alst{h}ǫll við \alst{h}ýrógi; \hld\ \alst{h}ęiptum skal mána kvęðja, &
\alst{b}ęiti við \alst{b}itsóttum, \hld\ ęn við \alst{b}ǫlvi rúnar; &
\ind \alst{f}old skal við \alst{f}lóði taka.\eva

\bvb I counsel thee Loddfathomer, that thou learn the counsels; thou wilt benefit if thou learnest; they will be good for thee if thou gettest: Wherever thou ale drinkest, choose for thee the might of the earth; for earth takes against drunkenness, but fire against sickness; oak against dysentery, the ear [of wheat] against sorcery, bearded rye against hernia—in conflicts shall one invoke Moon\footnoteB{According to \Voluspa\ 5, the moon has some sort of power, and based on \Lokasenna\ P3 \emph{kvęðja} ‘greet, call’ seems to be the word used for invoking in prayer.}—heather against bite-sicknesses; but \inx[C]{rune}[runes] against \inx[C]{bale};\footnoteB{cf. v. 124, 149.} the fold \ken{earth} must take against the flood.\evb
\evg

\sectionline

\section{The Rune-Tally}

These scattered verses have the header \emph{Rúnatals þáttr} ‘Strand of the Rune-Tally’ in younger Eddic paper manuscripts. They give an archaic, mystic impression; it is as if they were drawn from the lips of an Odinic priest.

\bvg
\bva\alst{V}ęit’k at ek hekk \hld\ \alst{v}indga męiði á &
\ind \alst{n}ę́tr allar \alst{n}íu, &
\alst{g}ęiri undaðr \hld\ ok \alst{g}efinn Óðni, &
\ind \alst{s}jalfr \alst{s}jǫlfum mér, &
á þęim \alst{m}ęiði, \hld\ es \alst{m}anngi vęit, &
\ind hvęrs af \alst{r}ótum \alst{r}innr.\eva

\bvb I know that I hung on the windy beam, for all of nine nights; wounded by spear and given to Weden—myself to myself—on that beam, which no man knows, of whose roots it runs.\evb
\evg


\bvg
\bva Við \alst{h}lęifi mik sę́ldu-t \hld\ né við \alst{h}orni-gi; &
\alst{n}ýsta’k \alst{n}iðr, \hld\ \alst{n}am’k upp rúnar, &
\alst{ǿ}pandi nam, \hld\ fell’k \alst{a}ptr þaðan.\eva

\bvb With loaf they gladdened me not, nor with any horn. I peered down, I took up the runes, screaming I took; I fell back thence.\evb
\evg


\bvg
\bva \alst{F}imbulljóð níu \hld\ nam’k af hinum \alst{f}rę́gja syni &
\ind \alst{B}ǫlþorns, \alst{B}ęstlu fǫður, &
ok ek \alst{d}rykk of gat \hld\ hins \alst{d}ýra mjaðar &
\ind \alst{au}sinn \alst{Ó}ðreri.\eva

\bvb Nine \inx[C]{fimble-leeds} I learned from the famous son of \inx[P]{Balethorn}, the father of \inx[P]{Bestle}—and a drink I got, of that dear mead poured to \inx[P]{Woderearer}.\footnoteB{This verse fits poorly here and seems like an insert. It mentions \emph{ljóð} ‘leeds; (magical) songs, incantations’ rather than runes, and has nothing to do with Weden’s hanging on the tree. Bestle was Weden’s mother and Balethorn his maternal grandfather. The famous son of Balethorn would then be his maternal uncle. The custom of sending sons away to be fostered by their maternal uncles or grandfathers (which seems to be what is going on here) was quite common in Germanic society, cf. TODO.}\evb
\evg


\bvg
\bva Þá nam’k \alst{f}rę́vask \hld\ ok \alst{f}róðr vesa &
\ind ok \alst{v}axa ok \alst{v}ęl hafask; &
\alst{o}rð mér af \alst{o}rði \hld\ \alst{o}rðs lęitaði &
\ind \alst{v}erk mér af \alst{v}erki \alst{v}erks.\eva

\bvb Then I took to thrive, and be learned, and grow and have myself well. A word for me of a word a word sought out; a work for me of a work a work.\footnoteB{Each good word and deed was followed by another.}\evb
\evg


\bvg
\bva \alst{R}únar munt finna \hld\ ok \alst{r}áðna stafi, &
\ind mjǫk \alst{st}óra \alst{st}afi, &
\ind mjǫk \alst{st}inna \alst{st}afi, &
\ind es \alst{f}áði \alst{f}imbulþulr &
\ind ok \alst{g}ęrðu \alst{g}innręgin &
\ind ok \alst{r}ęist Hroptr \edtext{\alst{r}agna}{\lemma{ragna ‘of the Reins’}\Afootnote{‘rǫgna’ \Regius}}.\eva

\bvb \inx[C]{rune}[Runes] wilt thou find, and interpreted staves: very large staves, very stiff staves, which \inx[P]{Fimblethyle} \name{= Weden} painted, and the \inx[G]{gin-Reins} made, and Roft \name{= Weden} of the Reins carved.\evb
\evg


\bvg
\bva \alst{Ó}ðinn með \alst{ǫ́}sum, \hld\ ęn fyr \alst{ǫ}lfum Dáinn, &
\ind \alst{D}valinn \alst{d}vergum fyr, &
\ind \alst{Á}sviðr \alst{jǫ}tnum fyr, &
\ind ek ręist \alst{s}jalfr \alst{s}umar.\eva

\bvb \inx[P]{Weden} among the \inx[G]{Ease}, but for the \inx[G]{Elves} \inx[P]{Dowen}; \inx[P]{Dwollen} for the \inx[G]{Dwarfs}; \inx[P]{Onswith} for the Ettins; I myself carved some.\footnoteB{The identity of the speaker is not clear.}\evb
\evg


\bvg
\bva Vęizt, hvé \alst{r}ísta skal? \hld\ Vęizt, hvé \alst{r}áða skal? &
Vęizt, hvé \alst{f}áa skal? \hld\ Vęizt, hvé \alst{f}ręista skal? &
Vęizt, hvé \alst{b}iðja skal? \hld\ Vęizt, hvé \alst{b}lóta skal? &
Vęizt, hvé \alst{s}ęnda skal? \hld\ Vęizt, hvé \alst{s}óa skal?\eva

\bvb Knowest thou how one shall carve? Knowest thou how one shall read? Knowest thou how one shall paint? Knowest thou how one shall try? Knowest thou how one shall bid? Knowest thou how one shall \inx[C]{bloot}? Knowest thou one shall send? Knowest thou how one shall \inx[C]{soo}?\footnoteB{A symmetric structure would be attained if the first four verbs refer to \inx[C]{rune}[runes]—carving, interpreting, painting (with blood?), and divining—while the latter four refer to sacrifice—praying, sacrificing, sending (the sacrifice or the prayer; making sure the gods receive it), and slaying the victim. See further relevant Encyclopedia entries. The meter of the v. is unusual, but bears some resemblance to Vg 216 (the Högstena galder). TODO: Elaborate.}\evb
\evg


\bvg
\bva \alst{B}ętra ’s ó\alst{b}eðit \hld\ an sé of\alst{b}lótit, &
\ind ęy sér til \alst{g}ildis \alst{g}jǫf; &
bętra ’s ó\alst{s}ęnt \hld\ an sé of\alst{s}óit; &
\edtext{[...]}{\Bfootnote{Last line probably missing here; the meter and sense require it.}}\eva

\bvb ’Tis better unbid than over\inx[C]{bloot}[blooted]; a gift always sees recompense. ’Tis better unsent than over\inx[C]{soo}[sooed]; [...].\footnoteB{Identical wording (\emph{biðja} ‘to bid; to pray’ : \emph{blóta} ‘to bloot; to sacrifice’; \emph{senda} ‘to send’ : \emph{sóa} ‘to soo; to slay’) suggests a close relation to the previous verse. — The sense seems to be that it is better not to sacrifice at all than to sacrifice in excess, since even a small gift (to the gods) will be rewarded. This mechanistic system of gifts and rewards between man and the gods is also seen in other Indo-European pagan literatures. Compare the Sanskrit \emph{Dehí me, dádāmi te} ‘Give to me; I give to thee’ or Latin \emph{dō ut dēs} ‘I give that thou might give’.}\evb
\evg


\bvg
\bva Svá \alst{Þ}undr of ręist \hld\ fyr \alst{þ}jóða rǫk &
þar’s \alst{u}pp of ręis, \hld\ es \alst{a}ptr of kom.\eva

\bvb Thus \inx[P]{Thound} \name{= Weden} carved for the rakes of nations, where up he rose as back he came.\footnoteB{A very cryptic v. TODO.}\evb
\evg

\sectionline

\section{The Leed-Tally}

This final section of the poem has fittingly been called the Leed-Tally (\emph{Ljóðatal}). The speaker (certainly Weden) recounts eighteen spells, aristocratic and Odinic in character; they deal with such things as healing (2, 12), battle (3, 4, 5, 8, 11, 13), countering sorcery (6, 10), stilling the elements (7, 9), and seduction (16, 17).

In particular the fourth spell bears a strong likeness to the first Merseburg charm.


\bvg
\bva Ljóð \alst{þ}au kann’k, \hld\ es kann-at \alst{þ}jóðans kona &
\ind ok \alst{m}anskis \alst{m}ǫgr. &
\alst{H}jǫlp hęitir ęitt, \hld\ þat þér \alst{h}jalpa mun &
\ind við \alst{s}orgum ok \alst{s}ǫkum, \hld\ ok \alst{s}útum gǫrvǫllum.\eva

\bvb Those \inx[C]{leed}[leeds] I know, as knows not the ruler’s woman, and no man’s lad. Help is called one, it will help thee against sorrows and sakes,\footnoteB{Legal proceedings.} and all kinds of griefs.\footnoteB{TODO: elaborate on translatioon}\evb
\evg


\bvg
\bva Þat kann’k \alst{a}nnat, \hld\ es þurfu \alst{ý}ta synir, &
\ind þęir’s vilja \alst{l}ę́knar \alst{l}ifa.\eva

\bvb I know another, which the sons of men need;\footnoteB{Identical wording to 164/2.} they who wish to live as healers.\evb
\evg


\bvg
\bva Þat kann’k \alst{þ}riðja, \hld\ ef mér verðr \alst{þ}ǫrf mikil &
\ind \alst{h}apts við mína \alst{h}ęiptmǫgu, &
\alst{ę}ggjar dęyfi’k \hld\ minna \alst{a}ndskota, &
\ind bíta-t þęim \alst{v}ǫ́pn né \alst{v}élir.\eva

\bvb I know the third, if I come in great need of hindrance against my conflict-lads \ken{enemies}; I dull the edges of my opponents; for them neither weapons nor wiles bite.\evb
\evg


\bvg
\bva Þat kann’k \alst{f}jórða, \hld\ ef mér \alst{f}yrðar bera &
\ind \alst{b}ǫnd at \alst{b}óglimum, &
svá ek \alst{g}ęl, \hld\ at \alst{g}anga má’k, &
\ind sprettr mér af \alst{f}ótum \alst{f}jǫturr. &
\ind ęn af \alst{h}ǫndum \alst{h}apt.\eva

\bvb I know the fourth, if men bear bonds onto my shoulder-limbs \ken{arms}: so I gale that walk I may; springs from my feet the fetter, but from my hands the bond.\evb
\evg


\bvg
\bva Þat kann’k \alst{f}imta, \hld\ ef sé’k af \alst{f}ári skotinn &
\ind \alst{f}lęin í \alst{f}olki vaða, &
flýgr-a svá \alst{st}int, \hld\ at \alst{st}ǫðvi’g-a’k, &
\ind ef hann \alst{s}jónum of \alst{s}é’k.\eva

\bvb I know the fifth, if I see a dangerous arrow wading in the troop; it flies not so stiffly that I may not hinder it, if I see it with my sights.\evb
\evg


\bvg
\bva Þat kann’k \alst{s}étta, \hld\ ef mik \alst{s}ę́rir þegn &
\ind á \alst{r}ótum \alst{r}ás viðar. &
þann \alst{h}al, \hld\ es mik \alst{h}ęipta kvęðr, &
\ind þann eta \alst{m}ęin hęldr an \alst{m}ik.\eva

\bvb I know the sixth, if a thane injures me on the roots of a green tree;\footnoteB{i.e., he carves harmful magic runes into the roots.} that man who sings hatred against me, him the harms eat rather than me.\evb
\evg


\bvg
\bva Þat kann’k \alst{s}jaunda, \hld\ ef \alst{s}é’k hǫ́van loga &
\ind \alst{s}al of \alst{s}essmǫgum, &
\alst{b}rinnr-at svá \alst{b}ręitt, \hld\ at hǫ́num \alst{b}jargi’g-a’k; &
\ind þann kann’k \alst{g}aldr at \alst{g}ala.\eva

\bvb I know the seventh, if I see a high hall burning above seat-lads \ken{warriors}: it burns not so broadly that I do not save it [= the hall]\footnoteB{i.e. he can reduce the fire so that the hall is not destroyed (and presumably so that the trapped warriors survive).}—that galder I can gale.\evb
\evg


\bvg
\bva Þat kann’k \alst{á}tta, \hld\ es \alst{ǫ}llum es &
\ind \alst{n}ytsamligt at \alst{n}ema, &
\alst{h}var’s \alst{h}atr vęx \hld\ með \alst{h}ildings sonum, &
\ind þat má’k \alst{b}ǿta \alst{b}rátt.\eva

\bvb I know the eighth, which for all is useful to learn: wherever hatred grows among the sons of princes, it I may shortly mend.\evb
\evg


\bvg
\bva Þat kann’k \alst{n}íunda, \hld\ ef mik \alst{n}auðr of stęndr &
\ind at bjarga \alst{f}ari mínu á \alst{f}loti, &
\alst{v}ind ek kyrri \hld\ \alst{v}ági á &
\ind ok \alst{s}vę́fi’k allan \alst{s}ę́.\eva

\bvb I know the ninth, if need requires me to save my friend on a floater \ken{ship}: the wind I calm on the wave, and put all the sea asleep.\evb
\evg


\bvg
\bva Þat kann’k \alst{t}íunda, \hld\ ef sé’k \alst{t}únriður &
\ind \alst{l}ęika \alst{l}opti á, &
ek svá \alst{v}inn’k, \hld\ at \edtrans{þę́r \alst{v}illar fara}{they (\emph{feminine}) journey lost}{\Bfootnote{emend.; \emph{þęir villir fara} ‘they (\emph{masculine}) journey lost’ \Regius}} &
\ind sinna \alst{h}ęim-\alst{h}ama &
\ind sinna \alst{h}ęim-\alst{h}uga.\eva

\bvb I know the tenth, if I see \inx[G]{town-riders} playing aloft: I accomplish it so that they journey lost of their home-\inx[C]{hame}[hames]; of their home-minds.\footnoteB{The \emph{riður} ‘(female) riders’ were witches who were thought to leave their hames (\emph{hamir} ‘skins, shapes’) in a form of astral projection in order to fly around in the air, tormenting villagers. Their original bodies would of course be lying in a comatose state, and with the bodies their original minds; their humanness. Weden was through his second sight able to see these riders, and could use his superior magical abilities in order to confuse them so that they were not able to return to their original hames or minds; a cruel fate. — Weden likewise brags about tricking \emph{riders} in \Harbardsljod\ 20.}\evb
\evg


\bvg
\bva Þat kann’k \alst{ę}llipta, \hld\ ef skal’k til \alst{o}rrostu &
\ind \alst{l}ęiða \alst{l}angvini, &
und \alst{r}andir gęl’k, \hld\ ęn þęir með \alst{r}íki fara, &
\ind \alst{h}ęilir \alst{h}ildar til, &
\ind \alst{h}ęilir \alst{h}ildi frá, &
\ind koma þęir \alst{h}ęilir \alst{h}vaðan.\eva

\bvb I know the eleventh, if I shall lead old friends into battle: beneath the shields I gale, and they go powerfully, healthy to the conflict; healthy from the conflict; they return healthy from wherever.\evb
\evg


\bvg
\bva Þat kann’k \alst{t}olpta, \hld\ ef sé’k á \alst{t}ré uppi &
\ind \alst{v}áfa \alst{v}irgilná, &
svá ek \alst{r}íst \hld\ ok í \alst{r}únum fá’k, &
\ind at sá \alst{g}ęngr \alst{g}umi. &
\ind ok \alst{m}ę́lir við \alst{m}ik.\eva

\bvb I know the twelfth, if I see high up on a tree a gallow-corpse waving: so I carve, and paint in the runes, that that man walks and speaks with me.\evb
\evg


\bvg
\bva Þat kann’k \alst{þ}rettánda \hld\ ef skal’k \alst{þ}egn ungan &
\ind \alst{v}erpa \alst{v}atni á, &
mun-at hann \alst{f}alla, \hld\ þótt í \alst{f}olk komi, &
\ind \alst{h}nígr-a sá \alst{h}alr fyr \alst{h}jǫrum.\eva

\bvb I know the thirteenth, if I shall upon a young thane throw water:\footnoteB{Describing the pagan ritual of pouring water on a newborn child. Cf. \Rigsthula\ 7, 21, 34.} he will not fall, although he comes into battle; that man sinks not down before swords.\evb
\evg


\bvg
\bva Þat kann’k \alst{f}jǫgurtánda, \hld\ ef skal’k \alst{f}yrða liði &
\ind \alst{t}ęlja \alst{t}íva fyr, &
\alst{á}sa ok \alst{a}lfa \hld\ ek kann \alst{a}llra skil, &
\ind fár kann ó\alst{s}notr \alst{s}vá.\eva

\bvb I know the fourteenth, if I shall count the Tues before the retinue of men: of all the Ease and Elves I know the discernments;\footnoteB{Cf. \Hymiskvida\ 38, where the corresponding verb \emph{skilja} is used in the context of god-knowledge.} few unwise men can do so.\evb
\evg


\bvg
\bva \alst{Þ}at kann’k fimtánda, \hld\ es gól \alst{Þ}jóðrørir &
\ind \alst{d}vergr fyr \alst{D}ęllings \alst{d}urum, &
\alst{a}fl gól \alst{ǫ́}sum, \hld\ ęn \alst{ǫ}lfum frama, &
\ind \alst{h}yggju \alst{H}roptatý.\eva

\bvb I know the fifteenth, which Thedrearer galed, the dwarf before Delling’s doors. Power he galed for the Ease, but for the Elves fame; thought for Roft-Tue \name{= Weden}.\evb
\evg


\bvg
\bva Þat kann’k \alst{s}extánda, \hld\ ef vil’k hins \alst{s}vinna mans &
\ind hafa \alst{g}ęð alt ok \alst{g}aman, &
\alst{h}ugi \alst{h}vęrfi’k \hld\ \alst{h}vitarmri konu &
\ind ok \alst{s}ný’k hęnnar ǫllum \alst{s}efa.\eva

\bvb I know the sixteenth, if I will from the wise girl have her whole sense and pleasure; the heart I change of the white-armed woman, and I turn her whole affection.\evb
\evg


\bvg
\bva Þat kann’k \alst{s}jautjánda \hld\ at mik \alst{s}ęint mun firrask &
\ind hit \alst{m}anunga \alst{m}an.\eva

\bvb I know the seventeenth, that the girl-young girl will lately shun me.\evb
\evg


\bvg
\bva \alst{L}jóða þessa \hld\ munt \alst{L}oddfáfnir &
\ind lengi \alst{v}anr \alst{v}esa; &
\ind þó sé þér \alst{g}óð ef \alst{g}etr, &
\ind \alst{n}ýt ef \alst{n}emr, &
\ind \alst{þ}ǫrf ef \alst{þ}iggr.\eva

\bvb Of these leeds wilt thou, Loddfathomer, long be deprived, although they might be good for thee if thou gettest, beneficial if thou learnest, needful if thou acceptest.\evb
\evg


\bvg
\bva Þat kann’k \alst{á}tjánda, \hld\ es \alst{ę́}va kęnni’k &
\ind \alst{m}ęy né \alst{m}anns konu, &
—\alst{a}lt es bętra \hld\ es \alst{ęi}nn of kann, &
\ind þat fylgir \alst{l}jóða \alst{l}okum— &
nema þęiri \alst{ęi}nni, \hld\ es mik \alst{a}rmi vęrr, &
\ind eða mín \alst{s}ystir \alst{s}é.\eva

\bvb I know the eighteenth, which I will never teach a maiden nor man’s woman—everything is better when one alone can do it; that follows the end of the leeds—save for her alone who wraps me in her arm,\footnoteB{This interesting expression is also used \Volundarkvida\ 2. — The one who wraps Weden in her arm may be his wife, Frie. He has no known sister.} or who my sister is.\evb
\evg


\bvg
\bva Nú eru \alst{H}áva mǫ́l kveðin \hld\ \alst{H}áva \alst{h}ǫllu í; &
\ind \alst{a}llþǫrf \alst{ý}ta sonum, &
\ind \alst{ó}þǫrf \edtext{\alst{jǫ}tna}{\lemma{jǫtna}\Afootnote{ýta \emph{corrected in margin} \Regius}} sonum; &
hęill sá’s \alst{k}vað, \hld\ hęill sá’s \alst{k}ann, &
\ind \alst{n}jóti sá’s \alst{n}am, &
\ind \alst{h}ęilir þęir’s \alst{h}lýddu.\eva

\bvb Now are the speeches of the High One sung, in the hall of the High One; of great need for the sons of men, of harm for the sons of ettins! Hail he who sang [them]; hail he who knows [them]; may he benefit who learned [them]; hail those who heeded [them]!\evb
\evg
