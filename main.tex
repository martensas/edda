% This file should be compiled with XeLaTeX.

\documentclass[openany]{memoir}

% Font
\usepackage{fontspec}
\setmainfont{Junicode}[
	Extension=.ttf,
	BoldFont=*-Bold,
	ItalicFont=*-Italic]

% Underline that does not skip descender
% Should be called \nsunderline
 
% Packages
\usepackage{xparse}
\usepackage{geometry} %For margins.
\usepackage{longtable} %Long tables.

% Formatting
\usepackage{reledmac}

% Define verse counters
\newcounter{versea}
\newcounter{verseb}

\begin{document}

% Book and chapter commands
	\NewDocumentCommand{\chapterStart}{o O{Chap}}{% Command at the start of chapter
		\setcounter{versea}{0}%
		\setcounter{verseb}{0}%
		\stepcounter{chapter}%
		\IfNoValueF{#1}{%
			\begin{center}%
			\textbf{#2. \arabic{chapter}} \\
			{#1}\end{center}%
		}%
	}

	\newcommand{\bookStart}{% Command at the start of book
		\setcounter{chapter}{0} % Set chapter count to zero.
		\chapterStart{}%
	}

% Verse format commands
	\NewDocumentCommand{\bvg}{o}{% Begin verse group
		\begin{ledgroup}%
		\beginnumbering%
	}

	\NewDocumentCommand{\bva}{o}{% Begin verse a
		\stepcounter{versea}%
		\begin{large}\begin{stanza}% Begin stanza
		% Add verse number
		\IfNoValueT{#1}{%
			\newdimen\width% Create dimension width
			\setbox0=\hbox{\textbf{\arabic{versea} }}% Create hbox with the verse label
			\width=\wd0 \advance\width by \dp0% Set width to the width of the box
			\hspace{-\the\width}% Add a hspace = -(width)
			\textbf{\arabic{versea} }% Add the verse number
		}
	}
	\NewDocumentCommand{\eva}{o}{% End verse a
		\& \end{stanza}\end{large}\endnumbering% End reledmac numbering and stanza
		\vspace{1.5mm}% Vertical space
	}
	
	\NewDocumentCommand{\bvb}{o}{% Begin verse b
		\stepcounter{verseb}%
		\IfNoValueT{#1}{%
			\textbf{\arabic{verseb} }%
		}%
	}
	\NewDocumentCommand{\evb}{o}{% End verse b
		% Nothing (for now)
	}
	
	\NewDocumentCommand{\evg}{o}{% End verse group
		\end{ledgroup}%
		\vspace{1cm}%
	}
	
% Note formatting
	% Side note margin
	\setlength{\ledlsnotesep}{2 \ledlsnotesep}
	
	% Make foot notes paragraphs
	\Xarrangement[A]{paragraph}
	
% Poem formatting
	% First line number at 3
	\firstlinenum{2}
	\linenumincrement{2}
	
	% Stanza indentation (required for \astanza to work)
	\setstanzaindents{5, 2, 2}
	\setcounter{stanzaindentsrepetition}{2}

	% Mark cæsura.
	\newcommand{\hld}{\hspace{5mm} }%\leavevmode\unskip\quad\ignorespaces}

	% Indent lines (in Ljóðaháttr or Galdralag).
	\newcommand{\ind}{%
		\hspace{1.5em}%
	}

	% Mark alliteration. This might not be present in the final version.
	\NewDocumentCommand{\alst}{m}{%
		\underline{#1}%
	}
	
	% Mark kennings.
	\NewDocumentCommand{\ken}{m}{%
		\textsc{[{#1}]}%
	}

% Index link command
%	{#1}\textsuperscript{†}%	Dagger at the end
%	\nsunderline{#1}%	Underline

	\NewDocumentCommand{\inx}{m}{%
		{#1}%\textsuperscript{†}%
	}

% Sigla
	% Authors
	\newcommand{\Finnur}{%
		Finnur%
	}
	\newcommand{\Snorri}{%
		Snorri%
	}
	\newcommand{\CV}{%
		Cleasby-Vigfússon%
	}
	\newcommand{\Skp}{% Skaldic Poetry of the Scandinavian Middle Ages
		\emph{Skp}%
	}
	
	%Modern books
	\newcommand{\FaulkesEdda}{%
		\emph{SnE} 2005%
	}
	
	% Manuscripts
	\newcommand{\Regius}{% Codex Regius (of the poetic edda)
		\emph{R}%
	}
	\newcommand{\Hauksbok}{% Hauksbok
		\emph{H}%
	}
	\newcommand{\GylfMS}{% For referring to Gylfaginning manuscripts when verses are attested there.
		\emph{G}%
	}
	\newcommand{\RegiusProse}{% Codex Regius of the Prose Edda
		\emph{S}%
	}
	\newcommand{\Trajectinus}{% Codex Trajectinus
		\emph{T}%
	}
	\newcommand{\Wormianus}{% Codex Wormianus
		\emph{W}%
	}
	\newcommand{\Upsaliensis}{% Codex Upsaliensis
		\emph{U}%
	}
	\newcommand{\HildMS}{% For referring to the Hildebrandslied manuscript.
		\emph{ms.}%
	}
	\newcommand{\Hickes}{% George Hickes
		\emph{Hickes}%
	}
	
	% Texts
	\newcommand{\Hildebrandslied}{% Speeches of Hildbrand
		\emph{Hild}%
	}
	\newcommand{\Beowulf}{% Beewolf
		\emph{Bee}%
	}
	\newcommand{\Ynglingatal}{% Tally of the Inglings
		\emph{Ing}%
	}
	\newcommand{\Hervarar}{% Saw of Harware
		\emph{HarS}%
	}
	\newcommand{\Gylfaginning}{% For referring to Gylfaginning as a text
		\emph{Yilf}% F
	}
	\newcommand{\Haleygjatal}{% Tally of the Hallowlendings
		\emph{Hal}%
	}
	\newcommand{\Rigsthula}{% Thule of Righ
		\emph{Righ}%
	}
	\newcommand{\Vafthrudnismal}{% Speeches of Webthrithner
		\emph{Web}%
	}
	\newcommand{\FraLoka}{% From Locke
		\emph{FrL}%
	}

% Books

% Theology, mythology, independent order
	\book{\emph{Vǫluspǫ́} — The Spae of the Wallow}\bookStart

% Introduction.

{\small The “\inx{Spae} of the \inx{Wallow}” is attested in full in two principal versions. The earlier is the Codex Regius of the Poetic Edda, \Regius\ (GKS 2365 4to; 1270s), where it is the first poem, found on folios 1r–3r. The later is Hawksbook, \Hauksbok\ (AM 544 4to; 1300–75), where it is found at 20r–21r, in the middle of a large collection of saws and Catholics works. Many verses are also cited in \Gylfaginning, which here has the general siglum \GylfMS; to avoid confusion, it is only used when all the witness \Gylfaginning\ mss. agree. For it, I have here relied on Finnur Jónsson’s 1931 critical edition (Finnur 1931), while discarding his emendations. The noted mss. of it are:\begin{enumerate}
	\item The Codex Upsaliensis \Upsaliensis\ (DG 11; 1300–25)
	\item The Codex Upsaliensis \Upsaliensis\ (DG 11; 1300–25)
	\item The Codex Upsaliensis \Upsaliensis\ (DG 11; 1300–25)
	\item The Codex Upsaliensis \Upsaliensis\ (DG 11; 1300–25)
\end{enumerate}}


\bvg {\small Greeting to the audience, bidding of Weden.}
\bva\ledleftnote{\emph{RH}}\alst{H}ljóðs bið’k allar \hld \edtext{\alst{h}ęlgar}{\lemma{hęlgar}\Afootnote{\emph{om.} \Regius}} kindir, &%nvl
\alst{m}ęiri ok \alst{m}inni \hld \alst{m}ǫgu Hęimdallar; &%nvl
\alst{v}ildu at, \alst{V}alfǫðr, \hld \alst{v}ęl fram tęlja’k &%nvl
\alst{f}orn spjǫll \alst{f}ira, \hld þau’s \alst{f}ręmst of man?\eva

\bvb Of hearing I bid all holy kinds, the greater and lesser sons of \inx{Homedall}\footnoteA{Cf. \Rigsthula, wherein Homedall, under the name Righ, sires the three castes (earls, churls and thralls).}. Wilt thou, Leader of the Slain, that I well tell forth the ancient sayings of firs\footnoteA{Men.}, those I foremost recall?\footnoteA{Cf. \Vafthrudnismal\ 34, 35 with very similar phrasing.}\evb
\evg


\bvg {\small Wallow reckons what she recalls; the creation and ordering of the world.}
\bva\ledleftnote{\emph{RH}}Ek man jǫtna \hld ár of borna, &%nvl
þá es forðum \hld mik fǿdda hǫfðu; &%nvl
níu man’k hęima, \hld níu \edtext{íviðjur}{\Afootnote{\emph{Previously read} íviði, \emph{but closer study of} \Regius\ \emph{has disproven this. See Gripla 3, pp. 227–28.}}}, &%nvl
mjǫtvið mæran \hld fyr mold neðan.\eva

\bvb I recall \inx{ettins}, born of yore, those who anciently had nourished me. Nine \inx{homes} I recall, nine \inx{inwithies}, the renowned \inx{Metwood} beneath the earth.\footnoteA{Certainly Ugdrassle, “beneath the earth” likely referring to it still being a seed.}\evb
\evg


\bvg
\bva\ledleftnote{\emph{RHG}}Ár vas alda \hld \edtext{þar’s Ymir byggði}{\Afootnote{þat’s ekki vas “[of elds], that when nothing was” \GylfMS}}, &%nvl
vas-a sandr né sær, \hld né svalar unnir; &%nvl
jǫrð fansk æva \hld né upphiminn; &%nvl
gap vas ginnunga, \hld ęn gras \edtext{hvęrgi}{\Afootnote{ekki \Hauksbok}}.\eva

\bvb It was the beginning of \inx{elds}, there where \inx{Yime} dwelled; was there not sand nor sea, nor cool waves. The earth was never found, nor \inx{up-heaven}; a gap was of ginnings\footnoteA{See index Gap of Ginnings. \emph{ginnungr} means ‘hawk’ in the Scoldish poetry, but that meaning is strange here, unless it be an obscure sky-kenning (referring to the void).}, but grass nowhere.\evb
\evg


\bvg
\bva Áðr Burs synir \hld bjǫðum of ypðu, &%nvl
þęir es Miðgarð \hld mæran skópu; &%nvl 
sól skein sunnan \hld á salar stęina; &%nvl
þá vas grund gróin \hld grǿnum lauki.\eva

\bvb Before the sons of Bur the flatlands did upwards lift, they who the renowned Middenyard shaped. Sun shone from the south on the stones of the hall; then was the ground grown with green leek.\footnoteA{The sons of Bur, that is Weden, Will and Wigh (cf. \Gylfaginning\ TODO), lift the lands out of the primordial chaos (the Gap of Ginnings).}\evb
\evg


\bvg
\bva Sól varp sunnan, \hld sinni mána, &%nvl
hęndi hinni hǿgri \hld \edtext{umb himinjǫður}{\Afootnote{vm himin iodyr \Regius\ of ioður \Hauksbok}}; &%nvl
sól þat né vissi, \hld hvar hón sali átti; &%nvl
stjǫrnur þat né vissu, \hld hvar þær staði ǫ́ttu; &%nvl
máni þat né vissi, \hld hvat hann męgins átti.\eva

\bvb Sun cast from the south, — the companion of Moon\footnoteA{At times translated as “its moon”; this cannot be correct, as \emph{máni} ‘moon’ is masculine, while \emph{sinni}, dative singular of \emph{sínn} ‘its (reflexive)’ is feminine.}, — her right hand about heaven’s rim;\footnoteA{The sun heaved herself up over the horizon and rose for the first time?} Sun knew not, where halls she owned; stars knew not, where steads they owned; Moon knew not, what of might he owned.\evb
\evg


\bvg
\bva Þá gingu ręgin ǫll \hld á rǫkstóla, &%nvl
ginnhęilǫg goð, \hld ok gættusk umb þat.\eva

\bvb Then went the Powers all onto the rake-seats\footnoteA{Judgment-seats; first element \emph{rǫk} defined by \CV\ as ‘reason, ground, origin’.}: the gin-holy gods, and from each other took counsel about that.\footnoteA{10, 23, 25 would suggest two \inx{long-lines} be missing here.}\evb
\evg


\bvg
\bva Nótt ok niðjum \hld nǫfn of gǫ́fu, &%nvl
morgin hétu \hld ok miðjan dag, &%nvl
undurn ok aptan, \hld ǫ́rum at tęlja.\eva

\bvb To night and the moon’s phases, names did they give; morning they called, and middle day; afternoon and evening, the years for to tally.\footnoteA{Cf. \emph{Web} 23, 25.}
\evb
\evg


\bvg
\bva Hittusk æsir \hld á Iðavęlli, &%nvl
\edtext{þęir’s hǫrg ok hof \hld hǫ́ timbruðu}{\lemma{þęir’s ... timbruðu}\Afootnote{afls kostuðu / allz freistuðu. “[their] strength they tried; everything they tempted.” \Hauksbok}}; &%nvl
afla lǫgðu, \hld auð smíðuðu, &%nvl
tangir skópu \hld ok tól gęrðu.\eva

\bvb The Ease found each other on the \inx{Idewolds}, they who \inx{harrows} and \inx{hoves} high timbered: hearths they laid, wealth they smithed, tongs they shaped, and tools they made.\evb
\evg


\bvg
\bva Tęflðu í túni, \hld tęitir vǫ́ru, &%nvl
vas þeim véttugis \hld vant ór golli, &%nvl
unz þríar kvǫ́mu \hld þursa męyjar, &%nvl
ámátkar mjǫk, \hld ór Jǫtunhęimum.\eva

\bvb They played \inx{Tavel} in the yards, joyous were they: was for them no lack of gold\footnoteA{Cf. v. 59.}, until three came, maidens of \inx{thurses}, greatly terrifying, out of \inx{Ettinham}.\footnoteA{These are immediately forgotten and not again mentioned (unless they are taken to be the norns in v. 21, but they would then be introduced twice). — Clearly, there is something missing between this verse and the next, detailing the reason for creation of dwarves.}\evb
\evg


\bvg {\small Creation of dwarves.} %TODO: add critical from Gylfaginning
\bva — Þá gingu ręgin ǫll \hld á rǫkstóla, &%nvl
ginnhęilǫg goð, \hld ok gættusk umb þat: &%nvl
\edtext{hvęrr skyldi dverga}{\Afootnote{\emph{thus} \Regius\ huerer skylldu duergar \Hauksbok}} \hld \edtext{dróttir}{\Afootnote{drotin \Regius}} skępja &
ór \edtext{Brimis blóði}{\Afootnote{\emph{thus} \Regius\ brimi bloðgv “out of the bloody surf” \Hauksbok}} \hld ok ór Bláins lęggjum?\eva

\bvb Then went the Powers all onto the rake-seats: the gin-holy gods, and from each other took counsel about that: Who would shape the multitudes of \inx{dwarves}, out of the blood of \inx{Brime}, and out of the legs of \inx{Blown}?\evb
\evg


\bvg
\bva Þar vas \edtext{Móðsognir}{\Afootnote{\emph{thus} \Hauksbok\ Mótsognir \Regius}} \hld mæztr of orðinn &%nvl
dverga allra, \hld ęn Durinn annarr; &%nvl
þęir manlíkun \hld mǫrg of gęrðu, &%nvl
dvergar í jǫrðu, \hld sęm Durinn sagði.\eva

\bvb There was Moodsown become the worthiest of all dwarves, but Dorn [was] second. They made men-likenesses many; dwarves out of the earth, as Dorn said.\footnoteA{A cryptic verse; \emph{manlíkan} ‘man-likeness’ only appears here. Were the lower dwarves shaped out of soil, by the higher dwarves, Moodsown and Dorn, themselves made by the gods from Yime’s flesh and blood? “as Dorn said” would imply that Dorn did not shape the dwarves by own hands; did he and Moodsown shape the first lower dwarves out of stone, and then command these to finish the creation?}\evb
\evg


\bvg {\small Two lists of dwarves. That both belonged to the original poem is impossible, since several names (Oakenshield, Great-grandfather) appear twice. The three following verses seem to belong together, since there is no repetition of names. From the last verse of the middle one, it seems that it should have been placed at the end of the group.}
\bva — Nýi ok Niði, \hld Norðri, Suðri, &%nvl
Austri, Vestri, \hld Alþjófr, Dvalinn, &%nvl
Bívurr, Bávurr, \hld Bǫmburr, Nóri, &%nvl
Ánn ok Ánarr, \hld Ái, Mjǫðvitnir.\eva

\bvb New and Nithe, Norther and Suther, Easter and Wester, Allthief, Dwollen, Bewer, Bower, Bamber, Noor, Own and Owner, Great-grandfather, Meadwitten.\evb
\evg


\bvg
\bva Vęigr ok Gandalfr, \hld Vindalfr, Þráinn, &%nvl
Þękkr ok Þorinn, \hld Þrór, Vitr ok Litr, &%nvl
Nár ok Nýráðr, \hld nú hęf’k dverga, &%nvl
Ręginn ok Ráðsviðr, \hld rétt of talða.\eva

\bvb Wey and Gandelf, Windelf, Thrown, Thetch and Thorn, Throo, Wit and Lit, Nee and Newred; — now have I the dwarves, — Rain and Redswith, — rightly tallied.\evb
\evg


\bvg
\bva Fíli, Kíli, \hld Fundinn, Náli, &%nvl
Hępti, Víli, \hld Hannarr, Svíurr, &%nvl
Frár, Hornbori, \hld Frægr ok Lóni, &%nvl
Aurvangr, Jari, \hld Ęikinskjaldi.\eva

\bvb Filer, Chiler, Founden and Needler, Heft, Wiler, Hanner, Swigher, Frew, Hornborer, Fray and Looner, Earwong, Erer, Oakenshield.\evb
\evg


\bvg
\bva — Mál es dverga \hld í Dvalins liði &%nvl
ljóna kindum \hld til Lofars tęlja, &%nvl
\edtext{þęir}{\Afootnote{þeim \Hauksbok}} es sóttu \hld frá salar stęini &%nvl
aurvanga sjǫt \hld til Jǫruvalla.\eva

\bvb It is time to tally the dwarves in Dwollen’s host [back] to Loffer, unto the kins of men\footnoteA{A standard genealogical introduction (compare \Haleygjatal\ 1). The line of dwarves is to be counted to their progenitor, Loffer. This possibly disagrees with the earlier introduction (“There was ...”), where Moodsown is said to be the foremost of the dwarves, and Loffer is not mentioned.}; they who sought, from the stone of the hall, the abode of \inx{Earwongs}\footnoteA{\CV\ \emph{aurvangr} ‘a loamy field’, and indeed this fits etymologically.} to the \inx{Erwolds}.\footnoteA{\Gylfaginning\ (TODO): “But these came from Swornshigh (\emph{Svarinshaugr}) to the Earwongs on the Erwolds, and thence Lofer is come; these are their names: Sherper (\emph{Skirpir}), Werper (\emph{Virpir}), Showfind, Great-grandfather, Elf and Ing (\emph{Ingi}), Oakenshield, Fale (\emph{Falr}), Frost, Finn, Ginner.”}\evb
\evg


\bvg
\bva Þar vas Draupnir \hld ok Dolgþrasir, &%nvl
Hár, Haugspori, \hld Hlévangr, Glói, &%nvl
Skirfir, Virfir, \hld Skáfiðr, Ái, &%nvl
Alfr ok Yngvi, \hld Ęikinskjaldi, &%nvl
Fjalarr ok Frosti, \hld Finnr ok Ginnarr; &%nvl
Þat mun \edtext{æ}{\Afootnote{\emph{om.} \Regius}} uppi, \hld meðan ǫld lifir, &%nvl
langniðja-tal \hld \edtext{til}{\Afootnote{\emph{om.} \Hauksbok}} Lofars hafat.\eva

\bvb TODO: Render all names. — Sherver, Werver, Showfind, Great-grandfather, Elf and Ing, Oakenshield, Feller and Frost, Finn and Ginner: That will ever be remembered, while the \inx{eld} lives\footnoteA{Two archaic formulae. The first literally “that will ever up above”, cf. \Hervarar\ TODO: “We two are cursed, brother, thy bane am I become! That will ever be remembered (\emph{þat mun æ uppi}, but both mss. \emph{þat mun enn uppi}), evil is the doom of the norns!”. The second is found in a runic inscription, U 323 (980–1015): “Ever will lie, while the eld lives (\textbf{meþ + altr + lifiʀ} \emph{með aldr lifir}), the hard-hammered bridge, broad, after a good man.”}, the tally of descendants, heaved to Lofer.\evb
\evg


\bvg {\small Creation of first men.}
\bva Unz þrír kvǫ́mu \hld \edtext{ór því liði}{\Afootnote{þussa brúðir (\emph{wo. doubt in error}) \Hauksbok}} &%nvl
\edtext{ǫflgir ok ástkir}{\Afootnote{ástkir ok ǫflgir \Hauksbok}} \hld æsir at húsi; &%nvl
fundu á landi \hld lítt męgandi &%nvl
Ask ok Emblu \hld ørlǫglausa.\eva

\bvb — Until three came out of that host: mighty and loving Ease along the houses; they found on land the little availing \inx{Ash} and \inx{Emble}, \inx{orlay}-less.\footnoteA{For, according to \Gylfaginning\ (TODO: reference), they were pieces of driftwood.}\evb
\evg


\bvg
\bva Ǫnd þau né ǫ́ttu, \hld óð þau né hǫfðu, &%nvl
lǫ́ né læti \hld né litu góða; &%nvl
ǫnd gaf Óðinn, \hld óð gaf Hǿnir, &%nvl
lǫ́ gaf Lóðurr \hld ok litu góða.\eva

\bvb Breath they owned not, \inx{wode} they had not, not craft nor sound, nor good complexion. Breath gave \inx{Weden}, wode gave \inx{Heen}, craft gave \inx{Lother}, and good complexion.\evb
\evg


\bvg {\small The ash of Ugdrassle and its three norns.}
\bva Ask vęit’k standa, \hld hęitir Yggdrasill, &%nvl
hǫ́r baðmr, ausinn \hld hvíta auri; &%nvl
þaðan koma dǫggvar \hld þær’s í dala falla; &%nvl
stęndr æ yfir grǿnn \hld Urðar brunni.\eva

\bvb — I know an ash does stand, called \inx{Ugdrassle}: a high beam\footnoteA{Tree.}, poured with white mud\footnoteA{Compare perhaps with the Indian ritual pouring of beverages onto the \emph{lingam}. — For the whole passage compare 27.}. Thence come the dew-drops which in the dales fall; it stands ever green over the \inx{Well of Weird}.\evb
\evg


\bvg
\bva Þaðan koma męyjar \hld margs vitandi &%nvl
þríar ór þeim \edtext{sæ}{\Afootnote{sal (“[out of that] hall”) \Hauksbok}}, \hld es \edtext{und}{\Afootnote{á (“on [the pine]”) \Hauksbok}} þolli stendr; &%nvl
Urð hétu ęina, \hld aðra Verðandi, &%nvl
skǫ́ru á skíði, \hld Skuld hina þriðju &%nvl
þær lǫg lǫgðu, \hld þær líf køru, &%nvl
alda bǫrnum, \hld ørlǫg \edtext{sęggja}{\Afootnote{at segia (“[the orlay] to say.”) \Hauksbok}}.\eva

\bvb Thence come maidens, much knowing: three out of that lake, which stands beneath the pine\footnoteA{But here simply meaning ‘tree’; perhaps the same applies for “ash” earlier.}: Weird they called one, the other Werthing—they carved onto planks—Shild the third. Laws they laid; lives they chose: for the children of mortals, the \inx{orlay} of men.\evb
\evg


\bvg {\small The origin of the Wallow.}
\bva Þat man hón folkvíg \hld fyrst í hęimi, &
es Gollvęigu \hld gęirum studdu &
ok í hǫll Háars \hld hána bręnndu, &
\edtext{þrysvar bręnndu}{\Afootnote{\emph{repeated twice} \Hauksbok}} \hld þrysvar borna, &
opt ósjaldan, \hld þó hón ęnn lifir.\eva

\bvb — That troop-conflict she recalls\footnoteA{While appealing to read \emph{folk-víg} ‘folk-conflict’ as meaning ‘ethnic conflict’, thus describing the war between the Ease and Wanes, \emph{folk} almost certainly here carries its earlier meaning of ‘troop, group of warriors’.}, the first in the \inx{home}, as Goldwey with spears they goaded, and in the hall of \inx{Higher}\footnoteA{The hall of Weden; Walhall.} burned: thrice they burned the thrice born; often unseldom, though she yet lives.\footnoteA{Very cryptic. TODO: double check Snorri. Goldwey was apparently burned three times “often unseldom” (in short succession?) by the Ease, which yet did not kill her?}\evb
\evg


\bvg
\bva \edtext{Hęiði}{\Afootnote{\emph{metr. emend.} Héidi hána \Regius\ Heiði hana \Hauksbok}} hétu, \hld hvar’s til húsa kom, &%nvl
\edtext{vǫlu}{\Afootnote{ok vǫlu \Hauksbok}} \edtext{velspáa}{\Afootnote{\emph{metr. emend.} uel spá \Regius\ vel spa \Hauksbok}}, \hld vitti hón ganda; &%nvl
sęið \edtext{hvar’s kunni}{\Afootnote{hon kvnni \Regius\ hon hvars hvn kunni \Hauksbok}}, \hld sęið \edtext{hug lęikinn}{\Afootnote{hon leikinn \Regius\ hon hugleikin \Hauksbok}}; &%nvl
æ vas hón angan \hld illrar brúðar.\eva

\bvb Heath they called her, where to houses she came: a well-spaeing\footnoteA{Gifted at soothsaying.} \inx{wallow}, she bewitched \inx{gands}. She soth\footnoteA{Past tense of sithe (ON \emph{síða}) ‘to enchant, bewitch’.} where she could, she soth deluded minds; ever was she the love of the evil bride.\evb
\evg


\bvg {\small War between Ease and Wanes.}
\bva Þá gingu ręgin ǫll \hld á rǫkstóla, &%nvl
ginnhęilǫg goð, \hld ok gættusk umb þat: &%nvl
hvárt skyldi æsir \hld afráð gjalda, &%nvl
eða skyldi goð ǫll \hld gildi ęiga?\eva


\bvb Then went the Powers all onto the rake-seats: the gin-holy gods, and from each other took counsel about that: whether the Ease should tribute yield, or should the gods all a banquet hold?\evb
\evg

\bvg
\bva Flęygði Óðinn \hld ok í folk of skaut; &%nvl
þat vas ęnn folkvíg \hld fyrst í hęimi; &%nvl
brotinn vas borðvęggr \hld borgar ása, &%nvl
knǫ́ttu vanir vígspǫ́ \hld vǫllu sporna.\eva

\bvb Weden flung [a spear], and into the opposing army did shoot; that was yet the first folk-war\footnoteA{\emph{folk} probably in its earlier sense, ‘troop’, though reading it as ‘people, folk’ is attractive, since it would give \emph{folkvíg} the meaning ‘ethnic conflict’.} in the \inx{home}. Broken was the board-wall\footnoteA{Wall made of planks.} of the fortification of the Ease; the Wanes did by \inx{wigh-spae} tread the fields.\footnoteA{The Wanes used magic spells to defeat the Ease.}\evb
\evg


\bvg {\small Building of the wall by the ettin.}
\bva Þá gingu ręgin ǫll \hld á rǫkstóla, &%nvl
ginnhęilǫg goð, \hld ok gættusk umb þat: &%nvl
hvęrr hęfði lopt alt \hld lævi blandit &%nvl
eða ætt jǫtuns \hld Óðs męy gefna.\eva

\bvb Then went the Powers all onto the rake-seats: the gin-holy gods, and from each other took counsel about that: Who had the air all with treason blended, or, to the ettin’s \inx{aught}, given \inx{Wode}’s maiden\footnoteA{That is, promised Frie to the ettin NAME. TODO: relate with what Snorri writes about the building of the wall.}?\evb
\evg


\bvg {\small Thunder slays him.}
\bva Þórr ęinn þar vá \hld þrunginn móði, &%nvl
hann sjaldan sitr, \hld es slíkt of fregn; &%nvl
á gingusk ęiðar, \hld orð ok sǿri, &%nvl
mǫ́l ǫll męginlig, \hld es á meðal \edtext{fóru}{\Afootnote{voru (“[between them] were.”) \Hauksbok}}.\eva

\bvb Thunder alone fought there, pressed by wrath; he seldom sits, when of such he learns. Trampled were oaths, vows and sworn words; the mighty treaties all, which between them had gone.\evb
\evg


\bvg {\small Homedall’s hearing hidden beneath Ugdrassle.}
\bva Vęit hón Hęimdallar \hld hljóð of folgit &%nvl
und hęiðvǫnum \hld hęlgum baðmi; &%nvl
á sér hón ausask \hld aurgum forsi &%nvl
af veði Valfǫðrs. \hld Vituð ér ęnn eða hvat?\eva

\bvb — Knows she the hearing of Homedall hidden, ’neath a shady\footnoteA{\emph{hęiðvanr}, literally ‘clear-, bright-less’.}, hallowed beam\footnoteA{The tree must be Ugdrassle.}. On it she sees being poured a muddy torrent\footnoteA{Literally “on she sees being poured with a muddy torrent”, which should be the same mud as in v. 19. However, if ms. \emph{á} is read as \emph{ǫ́} ‘river’, it would mean “A river she sees being fed by a muddy waterfall, from ...”}, from the pledge of the \inx{Father of the Slain}; — Know ye yet, or what?\footnoteA{“Do ye (Weden) know enough now, or what?” — repeated in 28, 33, 34, 38, 40, 47, 60, 61.}”\evb
\evg


\bvg {\small Weden sought out the wallow.}
\bva Ęin sat hón úti, \hld þá’s hinn aldni kom &%nvl
yggjungr ása \hld ok í augu lęit; &%nvl
hvęrs fregnið mik? \hld hví fręistið mín? &%nvl
Alt vęit’k, Óðinn, \hld hvar auga falt &%nvl
í hinum mæra \hld Mímis brunni; &%nvl
drekkr mjǫð Mímir \hld morgin hvęrjan &%nvl
af veði Valfǫðrs. \hld Vituð ér ęnn eða hvat?\eva

\bvb — Lone sat she outside, when the old one came: the Terrifier of the Ease, and into [her] eyes looked. “Why inquirest thou me? Why temptest thou me? I know it all, Weden; where thine eye thou hidst: in the renowned \inx{Well of Mime}, [there] drinks Mime mead every morning, from the pledge of the \inx{Father of the Slain}; — Know ye yet, or what?”\evb
\evg


\bvg
\bva Valði hęnni Hęrfǫðr \hld hringa ok męn; &%nvl
\edtext{féspjǫll spaklig}{\Afootnote{\emph{By some authors (see Haukur Þorgeirsson 2020, p. 51 ff.) emended to} fekk spjǫll spaklig “[and necklaces;] he received wise tidings”}} \hld ok spáganda; &%nvl
sá hón vítt ok of vítt \hld of verǫld hvęrja.\eva

\bvb Host-father chose for her, rings and necklaces, wise wealth-spells, and spae-gands\footnoteA{The meaning of a \emph{gand} not fully clear. In this verse perhaps staffs used in ritual?}; saw she widely and yet wider, o’er every world.\evb
\evg


\bvg {\small The Walkirries.}
\bva Sá hón valkyrjur \hld vítt of komnar, &%nvl
gǫrvar at ríða \hld til goðþjóðar. &%nvl
Skuld hęlt skildi, \hld ęn Skǫgul ǫnnur, &%nvl
Gunnr, Hildr, Gǫndul \hld ok Gęirskǫgul; &%nvl
nú eru talðar \hld nǫnnur Hęrjans, &%nvl
gǫrvar at ríða \hld grund valkyrjur.\eva

\bvb Saw she walkirries, widely come, ready to ride to \inx{Godthede}. Shild held a shield, and Shagle another; Guth, Hild, Gandle, and Goreshagle; now are tallied the women of Harn: ready to ride the ground, \inx{walkirries}.\evb
\evg


\bvg {\small The fate of Bolder.}
\bva Ek sá Baldri, \hld blóðgum tívi, &%nvl
Óðins barni, \hld ørlǫg folgin; &%nvl
stóð of vaxinn \hld vǫllum hæri &%nvl
mjór ok mjǫk fagr \hld mistiltęinn.\eva

\bvb — I saw Bolder’s, the bloody tue’s, the child of Weden’s, \inx{orlay} sealed\footnoteA{Notably, \emph{fela} ‘hide, conceal’ is used to describe burial in mounds, as in \Ynglingatal\ 24, Öl 1 (900s): “hidden (\textbf{fulkin} \emph{folginn}) in this mound lies he whom the greatest deeds followed...”}; grown did stand, higher than the fields, slender and greatly fair, the mistletoe.\footnoteA{Told allusively in the following three verses is the death of Bolder at the hands of his blind brother Hath. \Gylfaginning\ TODO}\evb
\evg


\bvg
\bva Varð af męiði, \hld þęim’s mær sýndisk, &%nvl
harmflaug hættlig, \hld Hǫðr nam skjóta. &%nvl
Baldrs bróðir vas \hld of borinn snimma, &%nvl
sá nam, Óðins sonr, \hld ęinnættr vega;\eva

\bvb Became of that beam, which meager seemed, a baneful harm-flier; Hath began to shoot. Bolder’s brother was born early; that one began, Weden’s son, one night old, to slay.\evb
\evg


\bvg
\bva þó hann æva hęndr \hld né hǫfuð kęmbði, &%nvl
áðr á bál of bar \hld Baldrs andskota. &%nvl
Ęn Frigg of grét \hld í Fęnsǫlum &%nvl
vǫ́ Valhallar. \hld Vituð ér ęnn eða hvat?\eva

\bvb Washed he never hands, nor head combed, before onto the pyre, he did bear Bolder’s opponent. But Frie did lament, in the Fenhalls, the woe of Walhall; — Know ye yet, or what?\evb
\evg


\bvg {\small The imprisoned Lock.}
\bva Hapt sá hón liggja \hld und Hveralundi &%nvl
lægjarns líki \hld Loka áþękkjan; &%nvl
Þa kná Váli \hld vígbǫnd snúa &
hęldr vǫ́ru harðgǫr \hld hǫpt ór þǫrmum &
þar sitr Sigyn \hld þęygi of sínum &%nvl
veri vęl glýjuð. \hld Vitud ér ęnn eða hvat?\eva

\bvb A captive she saw lying, beneath Wharlund: a bodily likeness of guileful Lock. Then did Woal the war-bonds turn; very were they sturdy, fetters made of intestines. There sits Sighyn, not at all cheerful, above her husband;\footnoteA{See \FraLoka.} — Know ye yet, or what?\evb
\evg


\bvg
\bva Ǫ́ fęllr austan \hld of ęitrdala &%nvl
sǫxum ok sverðum, \hld Slíðr heitir sú.\eva

\bvb A river falls from the east, above the venom-dales, with saxes and swords; Slide is that one called.\evb
\evg


\bvg {\small Two halls.}
\bva Stóð fyr norðan \hld á Niðavǫllum &%nvl
salr ór golli \hld Sindra ættar, &%nvl
ęn annarr stóð \hld á Ókólni, &%nvl
bjórsalr jǫtuns, \hld ęn sá Brimir hęitir.\eva

\bvb Stood to the north on the Nithewolds a hall out of gold, owned by the \inx{aught} of Sinder; but another one stood, on Uncoalen, the beer-hall of an ettin, and that one is called Brimmer.\evb
\evg


\bvg {\small The worst hall.}
\bva Sal sá hón standa \hld sólu fjarri &%nvl
Nástrǫndu á, \hld norðr horfa dyrr; &%nvl
falla ęitrdropar \hld inn um ljóra, &%nvl
sá ’s undinn salr \hld orma hryggjum.\eva

\bvb A hall saw she standing, far from the sun, on Neestrand, north face the doors; fall venom-drops in through the smoke-vent, that hall is wound by the spines of snakes.\evb
\evg


\bvg
\bva Sér hón þar vaða \hld þunga strauma &%nvl
męnn męinsvara \hld ok morðvarga &%nvl
ok þann’s annars glępr \hld ęyrarúnu. &%nvl
Þar sýgr Níðhǫggr \hld nái framgingna; &%nvl
slítr vargr vera. \hld Vituð ér ęnn eða hvat?\eva

\bvb There she sees wade, through heavy streams, oath-breaking men and murderwargs, and the one who confounds another’s understanding\footnoteA{Literally “who confounds another’s ear-rune,” probably referring to false counsellors.}. There sucks Nithehew from corpses passed-on; the warg tears men; — Know ye yet, or what?\evb
\evg


\bvg {\small The hag nourishes the destroyers in Ironwood.}
\bva Austr sat hin aldna \hld í Járnviði &%nvl
ok fǿddi þar \hld Fęnris kindir; &%nvl
verðr af þeim ǫllum \hld ęinna nøkkurr &%nvl
tungls tjúgari \hld í trolls hami.\eva

\bvb In the east sat the old woman, in \inx{Ironwood}, and nourished there the kinds of \inx{Fenner}; becomes out of them all a single one, the pitch-forker of the moon, in the \inx{hame} of a troll.\footnoteA{The old hag raises the offspring of the wolf Fenner, of which one will swallow the moon (and according to \Snorri\ TODO the other the sun). See note to the next v.}\evb
\evg


\bvg
\bva Fyllisk fjǫrvi \hld fęigra manna, &%nvl
rýðr ragna sjǫt \hld rauðum dręyra, &%nvl
svǫrt var þá sólskin \hld umb sumur ęptir, &%nvl
veðr ǫll válynd. \hld Vituð ér ęnn eða hvat?\eva

\bvb He\footnoteA{The wolf.} fills himself with the life of \inx{fey} men; he reddens the abode of the \inx{Powers} with red gore. Black becomes the sunshine about the summers afterwards\footnoteA{After the sun is swallowed. But since the wallow does not tell us that this is a different wolf (it seems rather it be one and the same), it may reflect an earlier version of the myth, where one son of Fenner swallowed both the sun and moon. Yet, according to \Vafthrudnismal\ 36-37 it is Fenner himself who will swallow the sun (and thus likely the moon as well,) unless it there be taken as a \inx{hote} for ‘wolf’, which it undoubtedly originally is. See \inx{Moon}. TODO}; the storms all woeful; — Know ye yet, or what?\evb
\evg


\bvg {\small Edgethew struck harp; a fair-red cock crowed.}
\bva Sat þar á haugi \hld ok sló hǫrpu &%nvl
gýgjar hirðir, \hld glaðr Ęggþér; &%nvl
gól of hǫ́num \hld í gaglviði &%nvl
fagrrauðr hani, \hld sá’s Fjalarr hęitir.\eva

\bvb Sat there on the \inx{high} and struck the harp, the troll-woman’s keeper, glad \inx{Edgethew}. Above him crowed, in Gallowwood\footnoteA{Probably the same as Ironwood.}, a fair-red cock, that one who Feller is called.\evb
\evg


\bvg {\small A golden cock crowed in Osyard; a soot-red in Hell.}
\bva Gól of ǫ́sum \hld Gollinkambi, &%nvl
sá vękr hǫlða \hld at Hęrjafǫðrs, &%nvl
ęn annarr gęlr \hld fyr jǫrð neðan &%nvl
sótrauðr hani \hld at sǫlum Hęljar.\eva

\bvb Above the Ease crowed Goldencombe: he wakes men at the Father of Hosts’s [estate]; but another one crows below the earth: a soot-red cock, at the halls of Hell.\evb
\evg


\bvg
\bva Gęyr Garmr mjǫk \hld fyr Gnipahęlli, &%nvl
fęstr mun slitna, \hld ęn Freki rinna; &%nvl
fjǫlð vęit hón frǿða, \hld framm sé’k lęngra &%nvl
of ragna rǫk, \hld rǫmm sigtíva.\eva

\bvb Barks Garm loudly before the Gnip-caverns; the rope will tear, and Freck run. Much she knows of wisdom, forth I see yet further; about the rakes of the Powers: the mighty [fates] of the victory-tues.\evb
\evg


\bvg {\small Degeneration of man.}
\bva Brǿðr munu bęrjask \hld ok at bǫnum verða, &%nvl
munu systrungar \hld sifjum spilla, &%nvl
hart ’s í hęimi, \hld hórdómr mikill, &%nvl
skęggǫld, skalmǫld, \hld skildir ’ró klofnir, &%nvl
vindǫld, vargǫld, \hld áðr verǫld stęypisk, &%nvl
mun ęngi maðr \hld ǫðrum þyrma.\eva

\bvb Brothers will fight one another, and become slayers; sister’s sons will spill their kinship.\footnoteA{Whether through incest or treachery. TODO: literary evidence of the phrase \emph{spilla sifjum}.} ’Tis hard in the \inx{home}, great whoredom: halberd-eld, short-sword-eld; shields are split! Wind-eld, warg-eld; before the world\footnoteA{\emph{ver-ǫld} ‘world’ might perhaps be better translated as ‘man-eld’, ‘the eld of man’ with the other elds preceding it.} is overthrown, will no man another spare.\evb
\evg


\bvg {\small Prophesied events come to pass.}
\bva Lęika Míms synir, \hld ęn mjǫtuðr kyndisk &%nvl
at hinu galla \hld Gjallarhorni; &%nvl
hǫ́tt blæss Hęimdallr, \hld horn ’s á lopti; &%nvl
mælir Óðinn \hld við Míms hǫfuð.\eva

\bvb The sons of Mime play, and the Metted is kindled, at [the sounding of] the shrill Horn of Yell. Loud blows Homedall, the horn is aloft; speaks Weden with the head of Mime.\evb
\evg


\bvg
\bva \edtext{Skęlfr Yggdrasils \hld askr standandi, &
ymr it aldna tré, \hld ęn jǫtunn losnar;}{\lemma{Skęlfr ... losnar} \Afootnote{\emph{thus} \Hauksbok \emph{In} \Regius\ \emph{the two long-lines are reversed.}}} &
\edtext{hræðask allir \hld á hęlvegum &
áðr Surtar þann \hld sefi of glęypir.}{\lemma{hræðask ... glęypir} \Afootnote{\emph{om.} \Regius.}} \eva

\bvb Quakes the ash of Ugdrassle, standing; groans the old tree, and the ettin loosens. All are frightened on the Hell-ways, before Surt’s kinsman \emph{it} does devour.\evb
\evg


\bvg
\bva Hvat ’s með ǫ́sum? \hld hvat ’s með ǫlfum? &
gnýr allr Jǫtunhęimr, \hld æsir ’ro á þingi, &
stynja dvergar \hld fyr stęindurum &
vęggbergs vísir — \hld vituð ér ęnn eða hvat?\footnoteB{In \Regius\ this v. follows v. 50 (Kjóll fęrr austan ...)}\eva

\bvb — What is with Ease? What is with Elves? Roars all Ettinham, Ease are at the Thing. Dwarves groan before gates of stone, the princes of the mountain-wall; — Know ye yet, or what?\evb
\evg


\bvg
\bva Gęyr nú Garmr mjǫk \hld fyr Gnipahęlli, &%nvl
fęstr mun slitna, \hld ęn Freki rinna; &%nvl
fjǫlð vęit hón frǿða, \hld framm sé’k lęngra &%nvl
of ragna rǫk, \hld rǫmm sigtíva.\eva

\bvb Barks now Garm loudly before the Gnip-caverns; the rope will tear, and Freck run. Much she knows of wisdom, forth I see yet further; about the rakes of the Powers: the mighty [fates] of the victory-tues.\evb
\evg


\bvg {\small The enemies of the gods assemble.}
\bva Hrymr ękr austan, \hld hęfsk lind fyrir, &%nvl
snýsk Jǫrmungandr \hld í jǫtunmóði; &%nvl
ormr knýr unnir, \hld ęn ari hlakkar, &%nvl
slítr nái neffǫlr; \hld Naglfar losnar.\eva

\bvb Rim drives from the east, holding the shield before himself. Ermingand writhes about himself in ettin-wrath: the worm propels the waves, but the eagle screams: the pale-beak tears corpses; Nailfare loosens.\evb
\evg


\bvg
\bva Kjóll fęrr austan \hld koma munu Múspells &%nvl
of lǫg lýðir, \hld ęn Loki stýrir; &%nvl
fara fíflmęgir \hld með Freka allir, &%nvl
þęim es bróðir \hld Býlęists í fǫr.\eva

\bvb A keel travels from the east—come will Muspell’s subjects by sea—but Lock steers it. Travel the warlocks all with Freck; with them fares the brother of Bylest along.\evb
\evg


\bvg {\small Surt comes; the final battle begins.}
\bva \edtext{Surtr}{\Afootnote{\emph{Svartr} \Gylfaginning}} fęrr sunnan \hld með sviga lævi, &%nvl
skínn af sverði \hld sól valtíva; &%nvl
grjótbjǫrg gnata, \hld ęn \edtext{gífr rata}{\Afootnote{guðar hrata (“[but] the gods stagger”) \Upsaliensis \emph{is wo. doubt corrupted, the young masc. pl. is proof enough.}}}, &%nvl
troða halir hęlveg, \hld ęn himinn klofnar.\eva

\bvb Surt comes from the south, with the switch-bane\footnoteA{According to \CV\ ‘fire’.}; from the sword shines the sun of the slain-tues; boulders clash, but the fiends reel; men march on the \inx{Hell-ways}, but heaven is sundered.\evb
\evg


\bvg {\small Weden falls to the Wolf and Free to Surt.}
\bva Þá kømr Hlínar \hld harmr annarr framm, &%nvl
es Óðinn fęrr \hld við ulf vega, &%nvl
ęn bani Bęlja \hld bjartr at Surti; &%nvl
þá mun Friggjar \hld falla \edtext{angan}{\Afootnote{\emph{thus} \Hauksbok\ angantyr \Regius}}.\eva

\bvb Then comes \inx{Line}’s second sorrow to pass, as Weden goes to strike against the wolf; but the bane of \inx{Bellow}\footnoteA{\inx{Free}.}, bright, [goes] against Surt; then will Frie’s beloved\footnoteA{Weden, her husband.} fall.\evb
\evg


\bvg {\small Wither avenges Weden and slays the Wolf.}
\bva \edtext{Þá kømr hinn mikli \hld mǫgr Sigfǫður}{\lemma{Þá kømr ... Sigfǫður}\Afootnote{Gęngr Óðins sonr / við ulf vega \Gylfaginning}}, &%nvl
Víðarr \edtext{vega}{\Afootnote{of veg \Gylfaginning}} \hld at valdýri; &%nvl
lætr hann męgi Hveðrungs \hld mund of standa &%nvl
hjǫr til hjarta; \hld þá ’s hefnt fǫður.\eva

\bvb Then comes the great lad of \inx{Sighfather}, Wither, to strike at the murderous beast; he lets his hand plunge the sword into the heart of \inx{Whethring}’s lad\footnoteA{The son of Lock; the wolf.}; then is the father avenged.\evb
\evg


\bvg {\small Thunder and the Worm kill each other.}
\bva \edtext{Þá kømr}{\Afootnote{Gęngr \Gylfaginning}} hinn mæri \hld mǫgr Hlǫðynjar &%nvl
\edtext{gęngr Óðins sonr \hld ormi mǿta.}{\lemma{gęngr ... mǿta}\Afootnote{\emph{om.} \Gylfaginning}} &%nvl
\edtext{Drepr af móði \hld Miðgarðs véurr; &
munu halir allir \hld hęimstǫð ryðja; &
gęngr fet níu \hld Fjǫrgynjar burr &
nęppr frá naðri, \hld níðs ókvíðinn.}{\lemma{Drepr ... ókviðinn} \Afootnote{neppr af naðri / niðs ókvíðnum / munu halir allir / heimstǫð ryðja, / er af móði drepr / Miðgarðs véurr (“[Goes the renowned lad of Lathyn,] pained, away from the loathsome adder. All men will empty their homesteads, when Middenyard’s wigh-ward strikes out of wrath.”) \Gylfaginning}}\eva

\bvb Then comes the renowned lad of Lathyn: the son of Weden goes the \inx{worm} to meet. Middenyard’s wigh-ward strikes out of wrath; all men will their homesteads empty.\footnoteA{It seems likely that the order found in \Gylfaginning\ is original. After Thunder dies, farming becomes impossible, and thus they must leave their homes.} The son of Firgyn goes nine paces, pained, away from the loathsome adder.\footnoteA{Thunder, mortally wounded, struggles nine steps away from the serpent before he falls. TODO: Snorri’s account}\evb
\evg


\bvg {\small Culmination.}
\bva Sól tér sortna, \hld sígr fold í mar, &%nvl
hverfa af himni \hld hęiðar stjǫrnur; &%nvl
gęisar ęimi \hld við aldrnara; &%nvl
lęikr hǫ́r hiti \hld við himin sjalfan.\eva

\bvb The sun does blacken, descends the fold into the sea; disappear from heaven the clear stars. Rages smoke from the nourisher of life\footnoteA{Fire.}; licks the high heat heaven itself.\evb
\evg


\bvg
\bva Gęyr nú Garmr mjǫk \hld fyr Gnipahęlli, &%nvl
fęstr mun slitna, \hld ęn Freki rinna; &%nvl
fjǫlð vęit hón frǿða, \hld framm sé’k lęngra &%nvl
of ragna rǫk, \hld rǫmm sigtíva.\eva

\bvb Barks now Garm loudly before the Gnip-caverns; the rope will tear, and Freck run. Much she knows of wisdom, forth I see yet further; about the rakes of the Powers: the mighty [fates] of the victory-tues.\evb
\evg


\bvg {\small The world is reborn.}
\bva Sér hón upp koma \hld ǫðru sinni &%nvl
jǫrð ór ægi \hld iðjagrǿna —; &%nvl
falla forsar, \hld flýgr ǫrn yfir, &%nvl
sá’s á fjalli \hld fiska vęiðir.\eva

\bvb Sees she come up, a second time: the earth out of the sea, ever green anew. Torrents fall, flies an eagle above, the one who on the fells fish does catch.\evb
\evg


\bvg
\bva Finnask æsir \hld á Iðavęlli &%nvl
ok of moldþinur \hld mǫ́tkan dǿma, &%nvl
ok minnask þar \hld á męgindóma &%nvl
ok á Fimbultýs \hld fornar rúnar.\eva

\bvb The Ease find each other on the Idewolds, and about the mighty earth-strip\footnoteA{The Middenyardsworm.} converse, and remember there mighty judgements, and Fimbletue’s ancient runes.\evb
\evg

\bvg {\small A new golden age.}
\bva Þar munu ęptir \hld undrsamligar &%nvl
gollnar tǫflur \hld í grasi finnask, &%nvl
þær’s í árdaga \hld áttar hǫfðu.\eva

\bvb There will afterwards wondrous golden Tavel-bricks in the grass be found: those which in days of yore they had owned.\footnoteA{Cf. v. 9. The rediscovering of the golden game pieces symbolizes a new golden age.}\evb
\evg


\bvg
\bva Munu ósánir \hld akrar vaxa; &%nvl
bǫls mun alls batna \hld mun Baldr koma; &%nvl
búa Hǫðr ok Baldr \hld Hropts sigtoptir &%nvl
(vęl valtívar, \hld Vituð ér ęnn eða hvat?)\eva

\bvb Will unsown fields grow: evil will all be bettered: Bolder will come. Hath and Bolder dwell in the building-plots of Roft: happily, the slain-Tues; — Know ye yet, or what?\evb
\evg


\bvg
\bva Þá kná Hǿnir \hld hlautvið kjósa &%nvl
ok burir byggva \hld brǿðra Tvęggja &%nvl
vindhęim víðan. \hld Vituð ér ęnn eða hvat?\eva

\bvb Then does Heen choose the \inx{leat}-wood, and the sons of the brothers of Tway build in the wide wind-home\footnoteA{TODO. What does Snorri write about the wind-home?}; — Know ye yet, or what?\evb
\evg


\bvg
\bva Sal sér hón standa \hld sólu fęgra, &%nvl
golli þakðan, \hld á \edtext{Gimléi}{\Afootnote{\emph{metr. emend.} Gimlé \Regius\ Gimle \Hauksbok}}; &%nvl
þar munu dyggvar \hld dróttir byggva &%nvl
ok um aldrdaga \hld ynðis njóta.\eva

\bvb A hall she sees standing, fairer than the sun, thatched with gold, on Gemlee; there the dutiful \inx{drights} will dwell, and in their days of life delights enjoy.\evb
\evg


\bvg {\small The dragon still lives; the wallow descends.}
\bva Þar kømr hinn dimmi \hld dręki fljúgandi, &%nvl
naðr fránn neðan \hld frá Niðafjǫllum; &%nvl
berr sér í fjǫðrum \hld — flýgr vǫll yfir — &%nvl
Níðhǫggr nái; \hld nú mun hón søkkvask.\eva

\bvb — Then comes the shadowy dragon flying; the gleaming serpent down below from the \inx{Nithfells}. Nithehew bears in his feathers—flying over the field—corpses.” — Now she will sink!\footnoteA{The wallow, referring to herself in third person, descends back down into her grave, whence Weden woke her.}\evb
\evg
% — General introduction
%	\book{The Speeches of Webthrithner. (Vafþrúðnismǫ́l)}\bookStart

\bvg {\small (Óðinn kvað:)}
\bva Ráð mér nú \alst{F}rigg \hld\ alls mik \alst{f}ara tíðir &
\ind at \alst{v}itja \alst{V}afþrúðnis; &
\alst{f}orvitni mikla \hld\ kveð'k mér á \alst{f}ornum stǫfum &
\ind við þann hinn \alst{a}lsvinna \alst{jǫ}tun.\eva

\bvb \inx{Weden} quoth: “Counsel me now, \inx{Frie}, as I desire to travel to visit \inx{Webthrithner}; greatly curious am I of ancient staves\footnotemark[1] by that all-wise \inx{ettin}."\evb
\footnotetext[1]{Ancient (pieces of) lore; cf. v. 55. — Meaning (from \emph{great} onwards) is clear, but form is very confused.}
\evg


\bvg {\small (Frigg kvað:)}
\bva \alst{H}ęima lętja \hld\ mynda'k \alst{H}ęrjafǫðr &
\ind í \alst{g}ǫrðum \alst{g}oða; &
\alst{ę}ngi \alst{jǫ}tun \hld\ hugða'k \alst{ja}fnramman &
\ind sęm \alst{V}afþrúðni \alst{v}esa.\eva

\bvb Frie quoth: “I would encourage the \inx{Leader of Armies} to [stay at] home in the yards of the gods, for I've judged no ettin be as strong as\footnotemark[3] Webthrithner."\evb
\footnotetext[3]{Lit. ‘equal-strong'.}
\evg


\bvg {\small (Óðinn kvað:)}
\bva Fjǫlð ek fór, \hld\ fjǫlð fręistaða'k, &
\ind fjǫlð ek ręynda ręgin; &
hitt vil'k vita, \hld\ hvé Vafþrúðnis &
\ind salakynni séi.\eva

\bvb Weden quoth: “Much I travelled, much I tried, much I tested the \inx{Reins}\footnotemark[4]. \emph{This} I want to know, how the condition of the halls of Webthrithner might be?"\evb
\footnotetext[4]{The gods.}
\evg


\bvg {\small (Frigg kvað:)}
\bva Hęill þú farir, \hld\ hęill þú aptr komir, &
\ind hęill á sinnum séir; &
ǿði þér dugi \hld\ hvar's skalt, Aldafǫðr, &
\ind orðum mæla jǫtun.\eva

\bvb Frie quoth: “Whole may thou travel, whole may thou return, whole may thou be on thy paths! May thy wisdom suffice, \inx{Leader of Men}, when thou go to exchange words with the ettin."\evb
\evg


\bvg
\bva Fór þá Óðinn \hld\ at fręista orðspęki &
\ind þess hins alsvinna jǫtuns; &
at hǫllu hann kom, \hld\ es\footnotemark[1] átti Íms faðir; &
\ind inn gekk Yggr þegar.\eva
\footnotetext[1]{Ms. \emph{ok} corrected to \emph{es}. Alliteration is lacking in this line, for which reason FJ emends \emph{Íms} to \emph{Hymis}.}

\bvb Then went Weden, to try the word-wisdom of that all-wise ettin. To the hall he came, which the father of \inx{Ime}\footnotemark[5] owned; shortly the \inx{Frightener}\footnotemark[6] walked in.\evb
\footnotetext[5]{Webthrithner.}
\footnotetext[6]{Weden.}
\evg


\bvg {\small (Óðinn kvað:)}
\bva Hęill þú nú, Vafþrúðnir, \hld\ nú em'k í hǫll kominn &
\ind á þik sjalfan séa; &
hitt vilk fyrst vita, \hld\ ef fróðr séir &
\ind eða alsviðr, jǫtunn.\eva

\bvb Weden quoth: “Hail thee now, Webthrithner; now I have come into the hall, to see thee thyself. \emph{This} I want to know first, if knowing thou might be, or all-wise, ettin!"\evb
\evg


\bvg {\small (Vafþrúðnir kvað:)}
\bva Hvat's þat manna, \hld\ es í mínum sal &
\ind verpumk orði á? &
út þú né kømr \hld\ órum hǫllum frá. &
\ind nema þú inn snotrari séir.\eva

\bvb Webthrithner quoth: “What is that of men\footnotemark[10], that in \emph{my} hall throws words at me? Thou will not come \emph{out}, from \emph{our}\footnotemark[11] halls, unless thou be the wiser [of us two]."\evb
\footnotetext[10]{Ie., ‘what man is that'. The use of the neuter pronoun \emph{hvat} by Web-str. may be seen as an insult or a way of belittling the guest.}
\footnotetext[11]{Prob. again meaning ‘my', unless Web-str. has allies present in the hall, but no such indication is given.}
\evg


\bvg {\small (Óðinn kvað:)}
\bva Gagnráðr\footnotemark[5] hęiti'k, \hld\ nú em'k af gǫngu kominn, &
\ind þyrstr til þinna sala; &
laðar þurfi \hld\ hęf'k lęngi farit &
\ind ok þinna andfanga, jǫtunn.\eva
\footnotetext[5]{R's \emph{Gagnráðr} ‘Gainred', is attested as Gangráðr ‘Journey-adviser' in \emph{Gylf}.}

\bvb Weden quoth: “\inx{Gainred} I am called, I am come from the journey, thirsty to thy halls. I have travelled for a long time in need of hospitality, and of thy reception, ettin!"\evb
\evg


\bvg \bvg {\small (Vafþrúðnir kvað:)}
\bva Hví þú þá, Gagnráðr, \hld\ mælisk af golfi fyrir? &
\ind far þú í sess í sal; &
þá skal fręista, \hld\ hvárr flęira viti, &
\ind gęstr eða hinn gamli þulr.\eva

\bvb Webthrithner quoth: “Why then, Gainred, art thou speaking from the floor before [me]? Take a seat in the hall! Then it shall be proven, which of the two might know more; the guest, or the old \inx{thyle}."\evb
\evg


\bvg {\small (Gagnráðr kvað:)}
\bva Óauðigr maðr, \hld\ es til auðigs kømr, &
\ind mæli þarft eða þęgi; &
ofrmælgi mikil \hld\ hygg at illa geti &
\ind hvęim's við kaldrifjaðan kømr.\eva

\bvb Gainred quoth: “An unwealthy man, who comes to a wealthy [one], ought to speak what is needed, or be silent.\footnotemark[14] Much over-speaking\footnotemark[15], I judge, will be bad for the one who comes to a cold-ribbed\footnotemark[16] [man]."\evb
\footnotetext[14]{Line identical to \emph{High} 18/2. The whole verse strongly reminds of verses from the \emph{Guest-thread} portion of said poem.}
\footnotetext[15]{“Speaking too much".}
\footnotetext[16]{That is, ‘cold-hearted', ‘cunning'.}
\evg


\bvg \bvg {\small (Vafþrúðnir kvað:)}
\bva Sęg mér, Gagnráðr, \hld\ alls á golfi vill &
\ind þíns of fręista frama, &
hvé hęstr hęitir, \hld\ sá's hvęrjan dręgr &
\ind dag of dróttmǫgu.\eva

\bvb Webthrithner quoth: “Say to me, Gainred, since on the floor I will to try thy fame: What is the horse called, which pulls each \emph{day} above the sons of the retinue \ken{Men}?"\evb
\evg


\bvg {\small (Gagnráðr kvað:)}
\bva Skinfaxi hęitir, \hld\ es hinn skíra dręgr &
\ind dag of dróttmǫgu; &
hęsta baztr \hld\ þykkir með Hręiðgotum; &
\ind ęy lýsir mǫn af mari.\eva

\bvb Gainred quoth: “\inx{Shining-fax} [that one] is called, who pulls the bright day above the sons of the retinue. The best of horses he seems among the \inx{Rode-goths}; the mane of that stallion ever shines."\evb
\evg


\bvg (Vafþrúðnir kvað:) &
\bva Sęg þat, Gagnráðr, \hld\ alls á golfi vill &
\ind þíns of fręista frama, &
hvé jór hęitir, \hld\ sá's austan dręgr &
\ind nótt of nýt ręgin.\eva

\bvb Webthrithner quoth: “Say this, Gainred, since on the floor I will to try thy fame: What is the horse called, which from the east pulls night above the useful \inx{Reins}?"\evb
\evg


\bvg {\small (Gagnráðr kvað:)}
\bva Hrímfaxi hęitir, \hld\ es hvęrja dręgr &
\ind nótt of nýt ręgin; &
méldropa fęllir \hld\ morgin hvęrjan; &
\ind þaðan kømr dǫgg of dala.\eva

\bvb Gainred quoth: “\inx{Frost-fax} [that one] is called, who pulls each night above the useful Reins. Every morning he lets foam fall from his bit\footnotemark[26]; thence comes dew in the valleys."\evb
\footnotetext[26]{Lit. “he fells bit-drops".}\evg


\bvg {\small (Vafþrúðnir kvað:)}
\bva Sęg þat, Gagnráðr, \hld\ alls á golfi vill &
\ind þíns of fręista frama, &
hvé ǫ́ hęitir, \hld\ sú's dęilir með jǫtna sonum &
\ind grund ok með goðum.\eva

\bvb Webthrithner quoth: “Say this, Gainred, since on the floor I will to try thy fame; How the river is called, which divides the ground between the sons of ettins and the gods?"\evb
\evg


\bvg {\small (Gagnráðr kvað:)}
\bva Ífing hęitir ǫ́, \hld\ es dęilir með jǫtna sonum &
\ind grund ok með goðum; &
opin rinna \hld\ hón skal um aldrdaga; &
\ind verðr-at íss á ǫ́.\eva

\bvb Gainred quoth: “\inx{Iving} the river is called, which divides the ground between the sons of ettins and the gods. Throughout [her] life-days she shall flow open; ice forms not on the river."\evb
\evg


\bvg {\small (Vafþrúðnir kvað:)}
\bva Sęg þat, Gagnráðr, \hld\ alls á golfi vill &
\ind þíns of fręista frama, &
hvé vǫllr hęitir, \hld\ es finnask vigi at &
\ind Surtr ok hin svǫ́su goð.\eva

\bvb Webthrithner quoth: “Say this, Gainred, since on the floor I will to try thy fame: How that valley is called, where \inx{Surt} and the excellent gods find each other at war?"\evb
\evg


\bvg {\small (Gagnráðr kvað:)}
\bva Vígríðr hęitir vǫllr, \hld\ es finnask vígi at &
\ind Surtr ok hin svǫ́su goð; &
hundrað rasta \hld\ hann's á hvęrjan veg; &
\ind sá's þęim vǫllr vitaðr.\eva

\bvb Gainred quoth: “\inx{Battle-rider} is the valley called, where Surt and the cheerful gods find each other at war. A hundred rests\footnotemark[30], he stretches in each direction; that valley is known for them.\footnotemark[31]"\evb
\footnotetext[30]{An old unit of length, from its name prob. the length a horse could travel before resting.}
\footnotetext[31]{That is, known for its great size.}\evg


\bvg {\small (Vafþrúðnir kvað:)}
\bva Fróðr est nú gęstr, \hld\ far á bękk jǫtuns, &
\ind mælumk í sessi saman; &
hǫfði vęðja \hld\ vit skulum hǫllu í &
\ind gęstr, of gęðspęki.\eva

\bvb Webthrithner quoth: “Knowing art thou now, guest, sit down on the ettin's bench; let us speak while sitting together. In the hall we shall wager a head, guest, over [our] mind-wisdom."\evb
\evg


\bvg {\small (Gagnráðr kvað:)}
\bva Sęg þat hit ęina, \hld\ ef þitt ǿði\footnotemark[10] dugir &
\ind ok þú Vafþrúðnir vitir, &
hvaðan jǫrð of kom \hld\ eða upphiminn &
\ind fyrst, hinn fróði jǫtunn.\eva
\footnotetext[10]{Starting with \emph{ǿði}, the poem is also preserved in 748.}

\bvb Gainred quoth: “Say the first\footnotemark[32], if thy wisdom suffices, and thou, Webthrithner, might know: Whence, O knowing ettin, the earth first came, or \inx{up-heaven}?"\evb
\footnotemark[32]{Lit. ‘one'.}\evg


\bvg {\small (Vafþrúðnir kvað:)}
\bva Ór Ymis holdi \hld\ vas jǫrð of skǫpuð, &
\ind ęn ór bęinum bjǫrg, &
himinn ór hausi \hld\ hins hrimkalda jǫtuns, &
\ind ęn ór svęita sær.\eva

\bvb Webthrithner quoth: “Out of \inx{Yime's} hull\footnotemark[35], the earth was created, but the mountains out of his bones. Heaven out of the skull of the frost-cold ettin, but the sea out of his sweat.\footnotemark[36]"\evb
\footnotetext[35]{His body.}
\footnotetext[36]{\emph{svęiti} ‘sweat' is a common kenning for blood. — This v. closely resembles \emph{Grím} 40.}\evg


\bvg {\small (Gagnráðr kvað:)}
\bva Sęg þat annat, \hld\ ef þitt ǿði dugir &
\ind ok þú Vafþrúðnir vitir, &
hvaðan máni of kom, \hld\ svát fęrr menn yfir, &
\ind eða sól hit sama.\eva

\bvb Gainred quoth: “Say the second, if thy wisdom suffices, and thou, Webthrithner, might know: Whence the moon came, so that it travels over men, or likewise the sun?"\evb
\evg


\bvg {\small (Vafþrúðnir kvað:)}
\bva Mundilfari hęitir, \hld\ hann's Mána faðir &
\ind ok svá Solar hit sama; &
himin hverfa \hld\ þau skulu hvęrjan dag &
\ind ǫldum at ártali.\eva

\bvb Webthrithner quoth: “\inx{Moundelfare} [that one] is called, he is the father of the Moon, and likewise of the Sun. They shall circle in the heavens every day, for men to reckon time\footnotemark[40]."\evb
\footnotetext[40]{Lit. “for men to year-reckoning".}\evg


\bvg {\small (Gagnráðr kvað:)}
\bva Sęg þat þriðja, \hld\ alls þik svinnan kveða &
\ind ok þú Vafþrúðnir vitir, &
hvaðan dagr of kom, \hld\ sá's fęrr drótt yfir, &
\ind eða nótt með niðum.\eva

\bvb Gainred quoth: “Say the third, since [they] call thee wise, and thou, Webthrithner, might know: Whence the day came, the one that travels over the rettinue, or night with the moon-phases?"\evb
\evg


\bvg {\small (Vafþrúðnir kvað:)}
\bva Dęllingr hęitir, \hld\ hann's Dags faðir, &
\ind ęn Nótt vas Nǫrvi borin; &
ný ok nið \hld\ skópu nýt ręgin &
\ind ǫldum at ártali.\eva

\bvb Webthrithner quoth: “\inx{Delling} [that one] is called, he is the father of \inx{Day}, but \inx{Night} was born to \inx{Nare}. The waxing and waning [of the moon], the useful Reins created, for men to reckon time."\evb
\evg


\bvg {\small (Gagnráðr kvað:)}
\bva Sęg þat fjórða, \hld\ alls þik fróðan kveða, &
\ind ok þú Vafþrúðnir vitir, &
hvaðan vetr of kom \hld\ eða varmt sumar &
\ind fyrst með fróð ręgin.\eva

\bvb Gainred quoth: “Say the fourth, since [they] call thee knowing, and thou, Webthrithner, might know: Whence winter first came, or the warm summer, among the knowing Reins?"\evb
\evg


\bvg {\small (Vafþrúðnir kvað:)}
\bva Vindsvalr hęitir, \hld\ hann's Vetrar faðir, &
\ind ęn Svǫ́suðr Sumars.\footnotemark[15]\eva
\footnotetext[15]{Second half of the v. seems missing.}

\bvb Webthrithner quoth: “\inx{Wind-cool} [that one] is called, he is the father of \inx{Winter}, but \inx{Delightful} of \inx{Summer}."\evb
\evg


\bvg {\small (Gagnráðr kvað:)}
\bva Sęg þat fimta, \hld\ alls þik fróðan kveða, &
\ind ok þú Vafþrúðnir vitir, &
hvęrr ása ęlztr \hld\ eða Ymis niðja &
\ind yrði í árdaga.\eva

\bvb Gainred quoth: “Say the fifth, since [they] call thee knowing, and thou, Webthrithner, might know: Who, in days of yore became the eldest of the \inx{Anses}, or of the descendants of Yime?"\evb
\evg


\bvg {\small (Vafþrúðnir kvað:)}
\bva Ørófi vetra \hld\ áðr væri jǫrð skǫpuð, &
\ind þá vas Bergęlmir borinn, &
Þrúðgęlmir \hld\ vas þess faðir, &
\ind ęn Aurgęlmir afi.\eva

\bvb Webthrithner quoth: “Uncountable winters before the earth would be created, then \inx{Bear-yeller} was born. \inx{Strength-yeller} was \emph{that one's} father, and \inx{Mud-yeller} the grandfather."\evb
\evg


\bvg {\small (Gagnráðr kvað:)}
\bva Sęg þat sétta, \hld\ alls þik svinnan kveða, &
\ind ok þú Vafþrúðnir vitir, &
hvaðan Aurgęlmir kom \hld\ með jǫtna sonum &
\ind fyrst, hinn fróði jǫtunn.\eva

\bvb Gainred quoth: “Say the sixth, since [they] call thee wise, and thou, Webthrithner, might know: Whence, O knowing ettin, Mud-yeller first came among the sons of ettins?"\evb
\evg


\bvg {\small (Vafþrúðnir kvað:)}
\bva Ór Élivǫ́gum \hld\ stukku ęitrdropar, &
\ind svá óx unz ór varð jǫtunn; &
órar ættir \hld\ kómu þar allar saman; &
\ind því's þat æ alt til atalt.\footnotemark[20]\eva
\footnotetext[20]{Lines 3–4 missing in R and 748, but quoted in \emph{Gylf}.}

\bvb Webthrithner quoth: “From the \inx{Ell-waves}, poison-drops splashed; thus [it] grew until an ettin emerged. \emph{Our} family lines all together originated there, therefore our race\footnotemark[45] is forever fierce against all.\footnotemark[46]"\evb
\footnotetext[45]{Lit. ‘it' or ‘that'.}
\footnotetext[46]{Somewhat strange phrasing, but the line does not appear damaged. It is clearly an explanation of the fierce and maleficent nature of the ettins, as their first ancestors were created from poison.}\evg


\bvg {\small (Gagnráðr kvað:)}
\bva Sęg þat sjaunda, \hld\ alls þik svinnan kveða, &
\ind ok þú Vafþrúðnir vitir, &
hvé sá bǫrn gat \hld\ hinn baldni\footnotemark[25] jǫtunn, &
\ind es hann hafði-t gýgjar gaman.\eva
\footnotetext[25]{R has \emph{aldni}, ‘aged, old'. This breaks alliteration; \emph{baldni} ‘bold, defiant' has been substituted from 748.}

\bvb Gainred quoth: “Say the seventh, since [they] call thee wise, and thou, Webthrithner, might know: How did that one, the defiant ettin, beget children, when he did not enjoy the [carnal] pleasure of a troll-woman?"\evb
\evg


\bvg {\small (Vafþrúðnir kvað:)}
\bva Und hęndi vaxa \hld\ kvǫ́ðu hrímþursi &
\ind męy ok mǫg saman; &
fótr við fǿti \hld\ gat hins fróða jǫtuns &
\ind sexhǫfðaðan son.\eva

\bvb Webthrithner quoth: “Neath the hand\footnotemark[50] on the \inx{frost-thurse}, [they] said that a maiden and lad grew together. A foot against a foot begot, for the knowing ettin, a six-headed son."\evb
\footnotetext[50]{\emph{hęndi} (dative of \emph{hǫnd}) means ‘hand', but might here be a poetic circumlocution for ‘arm'.}\evg


\bvg {\small (Gagnráðr kvað:)}
\bva Sęg þat áttunda, \hld\ alls þik fróðan kveða, &
\ind ok þú Vafþrúðnir vitir, &
hvat fyrst of mant \hld\ eða fręmst of vęizt, &
\ind þú est alsviðr jǫtunn.\eva

\bvb Gainred quoth: “Say the eigth, since [they] call thee knowing, and thou, Webthrithner, might know: What dost thou first remember, or earliest know?\footnotemark[55] Thou art all-wise, ettin."\evb
\footnotetext[55]{Cf. Vsp 1.}\evg


\bvg {\small (Vafþrúðnir kvað:)}
\bva Ørófi vetra \hld\ áðr væri jǫrð of skǫpuð, &
\ind þá vas Bergęlmir borinn; &
þat fyrst um man'k, \hld\ es hinn fróði jǫtunn &
\ind á vas lúðr of lagiðr.\footnotemark[30]\eva
\footnotetext[30]{This verse is quoted in \emph{Gylf}.}

\bvb Webthrithner quoth: “Uncountable winters before the earth would be created, then Bear-yeller was born. \emph{That} I first remember, when the knowing ettin\footnotemark[60] was laid down on the funeral-bed\footnotemark[61]."\evb
\footnotetext[60]{That is, Bear-yeller. Cf. v. 29.}
\footnotetext[61]{\emph{lúðr}, a tricky word.}\evg


\bvg {\small (Gagnráðr kvað:)}
\bva Sęg þat níunda, \hld\ alls þik svinnan kveða, &
\ind ok þú Vafþrúðnir vitir, &
hvaðan vindr of kømr \hld\ svát fęrr vág yfir, &
\ind æ męnn hann sjalfan of séa.\eva

\bvb Gainred quoth: “Say the ninth, since [they] call thee wise, and thou, Webthrithner, might know: Whence the wind comes, so that he travels over the wave; forever men see him himself.\footnotemark[65]"\evb
\footnotetext[65]{Perhaps a negation has been lost here; the wind is never seen by men.}\evg


\bvg {\small (Vafþrúðnir kvað:)}
\bva Hræsvęlgr hęitir, \hld\ es sitr á himins ęnda, &
\ind jǫtunn í arnar ham; &
af hans vængjum \hld\ kveða vind koma &
\ind alla męnn yfir.\eva

\bvb Webthrithner quoth: “\inx{Corpse-swallower} [that one] is called, which sits at the end of the heavens, an ettin in the shape of an eagle. From his wings, they say [that] the wind comes over all men."\evb
\evg


\bvg {\small (Gagnráðr kvað:)}
\bva Sęg þat tíunda, \hld\ alls þú tíva rǫk &
\ind ǫll Vafþrúðnir vitir, &
hvaðan Njǫrðr of kom \hld\ með niðjum ása. &
Hófum ok hǫrgum \hld\ hann ræðr hundmǫrgum &
\ind ok varð-at hann ǫ́sum alinn.\eva

\bvb Gainred quoth: “Say the \emph{tenth}, since thou, Webthrithner, of all the fates of the \inx{Tues} might know: Whence \inx{Nearth} came into the company of the kinsmen of the \inx{Anses}? He rules an immense number\footnotemark[68] of \inx{hoves} and \inx{heargs}, and he was not begotten among the Anses."\evb
\footnotetext[68]{Lit. “he rules hound-many".}\evg


\bvg {\small (Vafþrúðnir kvað:)}
\bva Í Vanahęimi \hld\ skópu hann vís ręgin &
\ind ok sęldu at gíslingu goðum, &
í aldar rǫk \hld\ hann mun aptr koma &
\ind hęim með vísum vǫnum.\eva

\bvb Webthrithner quoth: “In \inx{Wane-Home}, the wise \inx{Reins}\footnotemark[69] created him, and sold him as a hostage to the gods. In the fate of the age, he will come back, home among the wise \inx{Wanes}."\evb
\footnotetext[69]{Though \emph{ręgin} usually serves as a direct synonym of \emph{goð} 'god(s)', it here seems to refer specifically to the Wanes, in contrast with the \inx{Eses} or gods.}\evg


\bvg {\small (Gagnráðr kvað:)}
\bva Sęg þat ęllipta, \hld\ hvar ýtar túnum í &
\ind hǫggvask hvęrjan dag; &
val þęir kjósa \hld\ ok ríða vígi frá, &
\ind sitja męir of sáttir saman.\footnotemark[35]\eva
\footnotetext[35]{This and the next v. are damaged in both R and 748; R has only this verse, but splits it in two (the 2nd starting with \emph{val}), while 748 has 40:1 (Ms.: \emph{S. þ. e. XI}) and then jumps to the answer v. 41. They have here been reconstructed, but it is possible some lines are still missing.}

\bvb Gainred quoth: “Say the eleventh, Where men in yards, hew away at each other every day? They choose those destined to die in war, and ride [away] from battle; [then] they sit more content together."\evb
\evg


\bvg {\small (Vafþrúðnir kvað:)}
\bva Allir ęinhęrjar \hld\ Óðins túnum í &
\ind hǫggvask hvęrjan dag, &
val þeir kjósa \hld\ ok ríða vígi frá, &
\ind sitja męir of sáttir saman.\eva

\bvb Webthrithner quoth: “In Weden's yards, all the \inx{Lone Warriors} hew away at each other every day. They choose those destined to die in war, and ride [away] from battle; [then] they sit more content together."\evb
\evg


\bvg {\small (Gagnráðr kvað:)}
\bva Sęg þat tolpta, \hld\ hví þú tíva rǫk &
\ind ǫll Vafþrúðnir vitir, &
frá jǫtna rúnum \hld\ ok allra goða &
\ind þú hit sannasta sęgir, &
\ind hinn alsvinni jǫtunn.\eva

\bvb Gainred quoth: “Say the twelfth, Why thou, Webthrithner, shouldst know all the fates of the \inx{Tues}\footnotemark[73]? From the \inx{runes} of the ettins and of all the gods, thou, the all-wise ettin, speakest most truly."\evb
\footnotetext[73]{The gods. Formation identical to \emph{ragna rǫk} ‘the fates of the Reins'.}\evg


\bvg {\small (Vafþrúðnir kvað:)}
\bva Frá jǫtna rúnum \hld\ ok allra goða &
\ind ek kann sęgja satt, &
þvíat hvęrn hęf'k \hld\ heim of komit, &
níu kom'k hęima \hld\ fyr niflhęl neðan; &
\ind hinig dęyja ór hęlju halir.\eva

\bvb Webthrithner quoth: “From the runes of the ettins and of all the gods I can speak truly, for I have been about each \inx{Home}. I was about nine Homes beneath Nivelhell; this way men die out of Hell\footnotemark[1]."\evb
\footnotetext[1]{A difficult verse. Finnur considers \emph{ór hęlju} “out of Hell” a later interpolation.}\evg


\bvg {\small (Gagnráðr kvað:)}
\bva Fjǫlð ek fór, \hld\ fjǫlð fręistaða'k, &
\ind fjǫlð ek ręynda ręgin; &
hvat lifir manna, \hld\ þá's hinn mæra líðr &
\ind fimbulvetr með firum?\eva

\bvb Gainred quoth: “Much I travelled, much I tried, much I tested the \inx{Reins}.\footnotemark[80] What remains\footnotemark[79] of men, when the famous \inx{fimbol-winter} passes among firs\footnotemark[81]?”\evb
\footnotetext[79]{Lit. “lives".}
\footnotetext[80]{Here begins the repetition of the same “mantra" used in v. 3, which continues until the final question (v. 54).}
\footnotetext[81]{Among men.}\evg


\bvg {\small (Vafþrúðnir kvað:)}
\bva Líf ok Lífþrasir, \hld\ ęn þau lęynask munu &
\ind í holti Hoddmímis; &
morgindǫggvar \hld\ þau sér at mat hafa; &
\ind þaðan af aldir alask.\eva

\bvb Webthrithner quoth: “Life and Lifethrasher, and they will hide themselves in the wood of Hoard-Mime\footnotemark[85]. Morning-dew they [will] have as food; thereof generations [will] be bred.”\evb
\footnotetext[85]{Prob. the same as Uggdrassle.}}\evg


\bvg {\small (Gagnráðr kvað:)}
\bva Fjǫlð ek fór, \hld\ fjǫlð fręistaða'k, &
\ind fjǫlð ek ręynda ręgin; &
hvaðan kømr sól \hld\ á hinn slétta himin, &
\ind es þessa hęfr Fęnrir farit?\eva

\bvb Gainred quoth: “Much I travelled, much I tried, much I tested the Reins. Whence comes sun onto the smooth heaven, when \inx{Fenner} has killed this one\footnotemark[1]?"\evb
\footnotetext[1]{That is, the current incarnation of the sun, as explained in the next v.}\evg


\bvg {\small (Vafþrúðnir kvað:)}
\bva Ęina dóttur \hld\ berr alfrǫðull, &
\ind áðr hana Fęnrir fari; &
sú skal ríða, \hld\ þá's ręgin dęyja, &
\ind móður brautir mær.\eva

\bvb Webthrithner quoth: “One daughter the elf-wheel <= Sun> bears, before \inx{Fenner} might kill her. When the Reins die, that one, the maiden, shall ride the paths of the mother.”\evb
\evg


\bvg {\small (Gagnráðr kvað:)}
\bva Fjǫlð ek fór, \hld\ fjǫlð fręistaða'k, &
\ind fjǫlð ek ręynda ręgin; &
hvęrjar 'ro męyjar, \hld\ es líða mar yfir, &
\ind fróðgęðjaðar fara.\eva

\bvb Weden quoth: “Much I travelled, much I tried, much I tested the Reins. Which are the maidens that pass over the ocean; wise-minded they go?”\evb
\evg


\bvg {\small (Vafþrúðnir kvað:)}
\bva Þríar þjóðár \hld\ falla þorp yfir &
\ind męyja Mǫgþrasis; &
hamingjur ęinar \hld\ þær’s í hęimi eru, &
\ind þó þær með jǫtnum alask.\eva

\bvb Webthrithner quoth: “Three great rivers fall over the settlement of the maidens of Maythrasher; the only Hamings that are in the Home,\footnotemark[1] though they are raised among the ettins\footnotemark[2]."\evb
\footnotetext[1]{Either in Ettinhome, or in the entire world.}
\footnotetext[2]{See index entry Maythrasher.}\evg


\bvg {\small (Gagnráðr kvað:)}
\bva Fjǫlð ek fór, \hld\ fjǫlð fręistaða'k, &
\ind fjǫlð ek ręynda ręgin; &
hvęrir ráða æsir \hld\ ęignum goða, &
\ind þá's sloknar Surtalogi?\eva

\bvb Gainred quoth: “Much I travelled, much I tried, much I tested the Reins. Which properties of the gods will the \inx{Anses} [still] rule\footnotemark[105], when the flame of \inx{Surt} burns out?"\evb
\footnotetext[105]{Or ‘control’.}
\evg


\bvg {\small (Vafþrúðnir kvað:)}
\bva Víðarr ok Váli \hld\ byggva vé goða, &
\ind þá's sloknar Surtalogi; &
Móði ok Magni \hld\ skulu Mjǫlni hafa &
\ind Vingnis at vígþroti.\eva

\bvb Webthrithner quoth: “\inx{Wider} and \inx{Weel} [will] inhabit the sanctuaries of the gods, when the \inx{flame of Surt} burns out. \inx{Mood} and \inx{Main} will own \inx{Meldner}, when \inx{Wingner} can no longer fight\footnotemark[110]."\evb
\footnotetext[110]{Lit. “at Wingner's fight-exhaustion", referring to his death.}
\evg


\bvg {\small (Gagnráðr kvað:)}
\bva Fjǫlð ek fór, \hld\ fjǫlð fręistaða'k, &
\ind fjǫlð ek ręynda ręgin; &
hvat verðr Óðni \hld\ at aldrlagi, &
\ind þá's rjúfask ręgin?\eva

\bvb Gainred quoth: “Much I travelled, much I tried, much I tested the Reins. What brings Weden’s life to an end, when the Reins are broken?"\evb
\evg


\bvg {\small (Vafþrúðnir kvað:)}
\bva Ulfr glęypa \hld\ mun Aldafǫðr, &
\ind þess mun Víðarr vreka; &
kalda kjapta \hld\ hann klyfja mun &
\ind vitnis vígi at.\eva

\bvb Webthrithner quoth: “The wolf will swallow \inx{Eldfather}; Wider will avenge that. He will cleave the cold jaws of the wolf at the battle."\evb
\evg


\bvg {\small (Gagnráðr kvað:) &
\bva Fjǫlð ek fór, \hld\ fjǫlð fręistaða'k, &
\ind fjǫlð ek ręynda ręgin; &
hvat mælti Óðinn, \hld\ áðr á bál stigi, &
\ind sjalfr í ęyra syni?\eva

\bvb Gainred quoth: “Much I travelled, much I tried, much I tested the Reins. What spoke Weden himself, before [he]\footnotemark[115] would step onto the funeral pyre, into the ear of the son?"\evb
\footnotetext[115]{Prob. Weden's son, that is \inx{Balder}.}\evg


\bvg {\small (Vafþrúðnir kvað:)}
\bva Ęy \edtext{manngi}{\Afootnote{manni \Regius\AM\ \emph{is impossible; we need a nominative here.}}} vęit, \hld\ hvat þú í árdaga &
\ind sagðir í ęyra syni; &
fęigum munni \hld\ mælta’k mína forna stafi &
\ind ok of ragna rǫk. &
Nú við Óðin \hld\ dęilda’k mína orðspęki; &
\ind þú est æ vísastr vera.\eva

\bvb Webthrithner quoth: “Ever no man knows, what thou in days of yore saidst in the ear of the son. With a death-doomed\footnoteB{Webthrithner here realizes that he was bound to die from the moment (v. 19) he proposed the wager, as no being can outwit Weden.} mouth I spoke my ancient utterings, and of the \inx{Rakes of the Reins}. Now with Weden I shared my word-wisdom\footnoteB{The same word-wisdom Weden in v. 5 set out to try.}; thou art ever wisest of beings\footnoteB{Word used is \emph{verr} ‘husband, man’. Perhaps in the broader sense of ‘male being’.}."\evb
\evg
% — Weeden
%	\book{Speeches of the High One. (\emph{Hávamǫ́l})}\bookStart

Introduction.
\small{\emph{Hávamǫ́l}, or “the Speeches of the High One”, is the second poem of \Regius.}

\bvg Advice to wanderers.
\bva \alst{G}áttir allar \hld áðr \alst{g}angi framm &
\ind \edtext{of \alst{sk}oðask \alst{sk}yli}{\lemma{of skoðask skyli}\Bfootnote{\emph{om.} \GylfMS}} &
\ind of \alst{sk}ygnask \alst{sk}yli; &
því’t ó\alst{v}íst's at \alst{v}ita, \hld hvar ó\alst{v}inir &
\ind sitja á \alst{f}lęti \alst{f}yrir.\eva

\bvb All doorways—before one might go forth—should be watched, should be spied at; for uncertain ’tis to know, where enemies sit on the benches inside.\evb
\evg


\bvg
\bva \alst{G}efęndr hęilir, \hld \alst{g}ęstr’s inn kominn, &
\ind hvar skal \alst{s}itja \alst{s}já? &
mjǫk es \alst{b}ráðr \hld sá's á \alst{b}rǫndum skal &
\ind síns of \alst{f}ręista \alst{f}rama.\eva

\bvb Hail the givers\footnoteB{The hosts.}! A guest is come in, where shall that one sit? Greatly hurried is he, who on the fires shall try his fame.\footnoteB{According to \Finnur\ referring a Norwegian folk custom, wherein a guest would sit down on the wood-pile, waiting until being called in. See further TODO.}\evb
\evg


\bvg
\bva \alst{Ę}lds es þǫrf \hld þęim's \alst{i}nn es kominn &
\ind ok á \alst{k}néi \alst{k}alinn, &
\alst{m}atar ok váða \hld es \alst{m}anni þǫrf, &
\ind þęim's hęfr of \alst{f}jall \alst{f}arit.\eva

\bvb Of fire is need, for the one who inside is come, and cold about the knees; of food and clothing is need, for the man who over the fell has fared.\evb
\evg


\bvg
\bva \alst{V}ats es þǫrf \hld þęim's til \alst{v}erðar kømr, &
\ind \alst{þ}ęrru ok \alst{þ}jóðlaðar, &
\alst{g}óðs of ǿðis, \hld —ef sér \alst{g}eta mætti— &
\ind \alst{o}rðs ok \alst{ę}ndrþǫgu.\eva

\bvb Of water is need, for the one who comes for a meal, a towel and a friendly welcome; a good reception—if he might get one—speech, and silence in return.\evb
\evg


\bvg
\bva \alst{V}its es þǫrf \hld þęim's \alst{v}íða ratar; &
\ind dælt es \alst{h}ęima \alst{h}vat; &
at \alst{au}gabragði \hld verðr sá's \alst{ę}kki kann &
\ind ok með \alst{s}notrum \alst{s}itr.\eva

\bvb Of wits is need, for the one who roams widely; all is familiar at home. A laughing stock\Bfootnote{An idiom, \emph{augabragð} lit. ‘twinkling of the eye, moment’.} becomes he who nothing knows, and among the clever sits.\evb
\evg


\bvg
\bva At \alst{h}yggjandi sinni \hld skyli-t maðr \alst{h}rǿsinn vesa, &
\ind hęldr \alst{g}ætinn at \alst{g}ęði, &
þá’s \alst{h}orskr ok þǫgull \hld kømr \alst{h}ęimisgarða til, &
\ind sjaldan verðr \alst{v}íti \alst{v}ǫrum. &
\edtext{því’t óbrigðra vin \hld fær þú aldrigi, &
\ind an \alst{m}anvit \alst{m}ikit.}{\lemma{því ... mikit}\Bfootnote{The shift in person from 3rd to 2nd, along with the abnormal verse length (6 lines instead of 4), strongly suggests that this is an insert.}}\eva

\bvb Of his thinking should man not be boastful; rather guarding of his senses, when sharp and silent, he comes to a homestead; sudden injury seldom strikes the wary. (For thou getst never a more steadfast\footnoteB{Lit. ‘more unfickle’.} friend, than much \inx{manwit}{C}.)\evb
\evg


\bvg
\bva Hinn \alst{v}ari gęstr, \hld es til \alst{v}erðar kømr, &
\ind \alst{þ}unnu hljóði \alst{þ}ęgir; &
\alst{ęy}rum hlýðir, \hld ęn \alst{au}gum skoðar, &
\ind svá nýsisk \alst{f}róðra hvęrr \alst{f}yrir.\eva

\bvb The wary guest, when he comes to a host, is in attentive silence\footnotemark[13]. With ears he listens, but with eyes observes; thus every experienced man looks out in advance.\evb
\footnotetext[13]{Lit. 'with thin heed is silent'}
\evg


\bvg
\bva Hinn es \alst{s}æll, \hld es \alst{s}ér of getr &
\ind \alst{l}of ok \alst{l}íknstafi; &
\alst{ó}dælla es við þat, \hld es \alst{ęi}ga skal &
\ind \alst{a}nnars brjóstum \alst{í}.\eva

\bvb The one is fortunate, who gets praise and high esteem for \emph{himself}. It is not easy regarding that, which [he] must own in another's breast.\evb
\evg


\bvg
\bva \alst{S}á es \alst{s}æll, \hld es \alst{s}jalfr of á &
\ind \alst{l}of ok vit meðan \alst{l}ifir; &
því’t \alst{i}ll rǫ́ð \hld hęfr maðr \alst{o}pt þęgit &
\ind \alst{a}nnars brjóstum \alst{ó}r.\eva

\bvb That one is fortunate, who \emph{himself} owns praise and wits while he lives, for man has oft received ill counsel from another's breast.\evb
\evg


\bvg
\bva \alst{B}yrði \alst{b}ętri \hld berr-at maðr \alst{b}rautu at, &
\ind an sé \alst{m}anvit \alst{m}ikit; &
\alst{au}ði bętra \hld þykkir þat í \alst{ó}kunnum stað; &
\ind slíkt es \alst{v}álaðs \alst{v}era.\eva

\bvb No man carries a better burden on the road than much manwit; in a foreign place, it seems better than wealth; such is the refuge of the wretched.\evb
\evg


\bvg
\bva \alst{B}yrði \alst{b}ętri \hld bęrr-at maðr \alst{b}rautu at, &
\ind an sé \alst{m}anvit \alst{m}ikit; &
\alst{v}egnest \alst{v}erra \hld \alst{v}egr-a \alst{v}ęlli at, &
\ind an sé \alst{o}fdrykkja \alst{ǫ}ls.\eva

\bvb No man carries a better burden on the road than much manwit; on the field he can drag no worse provision along than too much ale drunk.\evb
\evg


\bvg
\bva Es-a svá \alst{g}ótt, \hld sęm \alst{g}ótt kveða, &
\ind \alst{ǫ}l \alst{a}lda sonum; &
því’t \alst{f}æra vęit, \hld es \alst{f}lęira drekkr, &
\ind síns til \alst{g}ęðs \alst{g}umi.\eva

\bvb Is not so good, as good they call it, ale for the sons of men. For less he knows, as more he drinks, man of his own senses.\evb
\evg


\bvg
\bva \alst{Ó}minnishegri hęitir, \hld sá’s of \alst{ǫ}lðrum þrumir, &
\ind hann stelr \alst{g}ęði \alst{g}uma; &
þess \alst{f}ogls \alst{f}jǫðrum \hld ek \alst{f}jǫtraðr vask &
\ind í \alst{g}arði \alst{G}unnlaðar.\eva

\bvb A heron of forgetfulness is called he who loiters about the ale; he robs men of their senses. With that bird’s feathers I fettered was, in the estate of Guthlath.\evb
\evg


\bvg
\bva \alst{Ǫ}lr ek varð, \hld varð \alst{o}frǫlvi, &
\ind at hins \alst{f}róða \alst{F}jalars; &
því es \alst{ǫ}lðr bazt, \hld at \alst{a}ptr of hęimtir &
\ind hvęrr sitt \alst{g}ęð \alst{g}umi.\eva

\bvb Drunk I became, I became the far drunkest, at the knowing Fealer’s. Thus with drinking it is best, that each man come back to his senses.\evb
\evg


\bvg
\bva \alst{Þ}agalt ok hugalt \hld skyli \alst{þ}jóðans barn &
\ind ok \alst{v}ígdjarft \alst{v}esa; &
\alst{g}laðr ok ręifr \hld skyli \alst{g}umna hvęrr, &
\ind unz sinn \alst{b}íðr \alst{b}ana.\eva

\bvb Taciturn and thoughtful should the ruler’s child be, and bold in battle. Glad and happy, should each man be, until he suffer his bane.\evb
\evg


\bvg
\bva \alst{Ó}snjallr maðr \hld hyggsk munu \alst{ę}y lifa, &
\ind ef við \alst{v}íg \alst{v}arask; &
ęn \alst{ę}lli gefr \hld hǫ́num \alst{ę}ngi frið, &
\ind þótt hǫ́num \alst{g}ęirar \alst{g}efi.\eva

\bvb The unvalorous man thinks he will ever live, if he of conflict is wary; but old age gives him no peace, although spears would give him.\evb
\evg


\bvg
\bva \alst{K}ópir afglapi, \hld es til \alst{k}ynnis \alst{k}ømr, &
\ind \alst{þ}ylsk hann umb eða \alst{þ}rumir; &
alt es \alst{s}ęnn, \hld ef \alst{s}ylg of getr, &
\ind uppi es þá \alst{g}ęð \alst{g}uma.\eva

\bvb The oaf gapes, when to a visit he comes; he mumbles about or loiters; all at once—if a sip he gets—are the senses of the man exposed.\evb
\evg


\bvg
\bva Sá ęinn \alst{v}ęit, \hld es \alst{v}íða ratar &
\ind ok hęfr \alst{f}jǫlð of \alst{f}arit, &
hvęrju \alst{g}ęði \hld stýrir \alst{g}umna hvęrr, &
\ind sá es \alst{v}itandi’s \alst{v}its.\eva

\bvb He alone knows, who widely wanders, and has travelled much: his senses does each man control, who is aware of his wits.\evb
\evg


\bvg
\bva \alst{H}aldi-t maðr á kęri, \hld drekki þó at \alst{h}ófi mjǫð, &
\ind mæli \alst{þ}arft eða \alst{þ}ęgi; &
ókynnis \alst{þ}ess \hld váar \alst{þ}ik ęngi maðr, &
\ind at gangir \alst{s}nimma at \alst{s}ofa.\eva

\bvb A man ought not hold onto the cask, rather drink a fitting serving of mead; he ought to speak what is needed or be silent\footnoteB{Identical to a certain verse in \Vafthrudnismal\ TODO: which one}. For that uncouthness will no man blame thee, that thou go early to sleep.\evb
\evg


\bvg
\bva \alst{G}rǫ́ðugr halr, \hld nema \alst{g}ęðs viti, &
\ind \alst{e}tr sér \alst{a}ldrtrega; &
opt fær \alst{h}lǿgis, \hld es með \alst{h}orskum kømr, &
\ind \alst{m}anni hęimskum \alst{m}agi.\eva

\bvb The gluttonous man, unless he know his sense, eats himself a life-sorrow; often the belly, when among sharp ones he comes, brings a foolish man ridicule.\evb
\evg


\bvg
\bva \alst{H}jarðir þat vitu, \hld nær \alst{h}ęim skulu, &
\ind ok \alst{g}anga þá af \alst{g}rasi; &
ęn \alst{ó}sviðr maðr \hld kann \alst{æ}vagi &
\ind síns of \alst{m}ál \alst{m}aga.\eva

\bvb Herds know, when home they must, and then part from the grass; but an unwise man never knows the measure of his belly.\evb
\evg


\bvg
\bva \alst{V}esall maðr \hld ok \alst{i}lla skapi &
\ind \alst{h}lær at \alst{h}vívetna; &
hitki hann \alst{v}ęit, \hld es \alst{v}ita þyrpti, &
\ind at hann es-a \alst{v}amma \alst{v}anr.\eva

\bvb The wretched man, and ill-spirited, laughs at whatever; he knows it not, which he needs to know: that he is not free of blemishes.\evb
\evg


\bvg
\bva \alst{Ó}sviðr maðr \hld vakir of \alst{a}llar nætr &
\ind ok \alst{h}yggr at \alst{h}vívetna; &
þá es \alst{m}óðr, \hld es at \alst{m}orni kømr; &
\ind alt es \alst{v}íl sęm \alst{v}as.\eva

\bvb The unwise man wakes for all nights, and thinks of whatever; then he is weary when the morning comes: his trouble is all as it was.\evb
\evg


\bvg
\bva \alst{Ó}snotr maðr \hld hyggr sér \alst{a}lla vesa &
\ind \alst{v}iðhlæjęndr \alst{v}ini; &
hitki hann \alst{f}iðr, \hld þótt þęir of hann \alst{f}ár lesi, &
\ind ef með \alst{s}notrum \alst{s}itr.\eva

\bvb The unclever man thinks all who laugh at him his friends; it he notices not, though they speak poorly about him, if he sits among the clever.\evb
\evg


\bvg
\bva \alst{Ó}snotr maðr \hld hyggr sér \alst{a}lla vesa &
\ind \alst{v}iðhlæjendr \alst{v}ini; &
\alst{þ}á þat fiðr \hld es at \alst{þ}ingi kømr, &
\ind at á \alst{f}ormælęndr \alst{f}áa.\eva

\bvb The unclever man thinks all who laugh at him his friends; then he finds, when to the \inx{Thing}{C} he comes, that he has spokesmen\Bfootnote{Men ready to take his side.} few.\evb
\evg


\bvg
\bva \alst{Ó}snotr maðr \hld þykkisk \alst{a}lt vita, &
\ind ef á sér i \alst{v}ǫ́ \alst{v}eru; &
hitki hann \alst{v}ęit, \hld hvat hann skal \alst{v}ið kveða, &
\ind ef hans \alst{f}ręista \alst{f}irar.\eva

\bvb The unclever man seems to know everything, if he shelters himself in a nook; it he knows not, what he shall say in return, if men test him.\evb
\evg


\bvg
\bva \alst{Ó}snotr maðr, \hld es með \alst{a}ldir kømr, &
\ind \alst{þ}at’s bazt at hann \alst{þ}ęgi; &
\alst{ę}ngi þat vęit, \hld at hann \alst{ę}kki kann, &
\ind nema hann \alst{m}æli til \alst{m}art. &
\alst{v}ęit-a maðr, \hld hinn’s \alst{v}ætki vęit, &
\ind þótt hann \alst{m}æli til \alst{m}art.\eva

\bvb The unclever man, when among people he comes, it is best that he be silent. None knows that he nothing knows, unless he speak too much. (A man knows not, who nothing knows, although he speak too much.)\evb
\evg


\bvg
\bva \alst{F}róðr sá þykkisk, \hld es \alst{f}regna kann, &
\ind ok \alst{s}ęgja hit \alst{s}ama, &
\alst{ęy}vitu lęyna \hld męgu \alst{ý}ta synir &
\ind því es \alst{g}ęngr of \alst{g}uma.\eva

\bvb Learned seems he who can ask, and answer the same; naught conceal may the sons of men, of that\Bfootnote{Rumours and gossip.} which goes about a man.\evb
\evg


\bvg
\bva \alst{Ǿ}rna mælir, \hld sá’s \alst{æ}va þęgir, &
\ind \alst{st}aðlausu \alst{st}afi; &
\alst{h}raðmælt tunga, \hld nema \alst{h}aldęndr ęigi, &
\ind opt sér ó\alst{g}ótt of \alst{g}ęlr.\eva

\bvb Quite enough does he speak, who is never silent, utterings of absurdity; a quick-spoken tongue—unless it be held in place\Bfootnote{Literally ‘unless holders own it’ or ‘unless it own holders’.}—often sings evil [into existence] for itself.\evb
\evg


\bvg
\bva At \alst{au}gabragði \hld skal-a maðr \alst{a}nnan hafa, &
\ind \edtext{þótt}{\lemma{þótt “although”}\Bfootnote{\emph{Perhaps an error?} es \emph{‘when’ would surely work better in context.}} til \alst{k}ynnis \alst{k}omi; &
margr \alst{f}róðr þykkisk, \hld ef hann \alst{f}reginn es-at &
\ind ok nái hann \alst{þ}urrfjallr \alst{þ}ruma.\eva

\bvb As a laughing stock shall man not have another, although he come to visit; many a man seems learned if he is not asked, and manages to loiter about dry-skinned\footnoteB{This sense of \emph{fjall} is apparently almost non-existent in Old Norse literature, but compare Swedish \emph{fjäll} ‘scale (on fish and reptiles)’. The meaning is in any case figurative; compare the English saying \emph{get your boots wet}.}.\evb
\evg


\bvg
\bva \alst{F}róðr þykkisk \hld sá’s \alst{f}lótta tękr &
\ind \alst{g}ęstr at \alst{g}ęst hæðinn; &
\alst{v}ęit-a gǫrla \hld sá’s of \alst{v}erði glissir, &
\ind þótt með \alst{g}rǫmum \alst{g}lami.\eva

\bvb Learned seems he, who takes to flight, when a guest at a guest is scoffing. He knows not fully, who grins above the food, that he with fiends be prattling.\evb
\evg


\bvg
\bva \alst{G}umnar margir \hld erusk \alst{g}agnhollir, &
\ind ęn at \alst{v}irði \alst{v}rekask; &
\alst{a}ldar róg \hld þat mun \alst{æ} vesa; &
\ind órir \alst{g}ęstr við \alst{g}ęst.\eva

\bvb Many men are loyal to each other, but over a meal drive each other away. The strife of mankind will that ever be; guest raves against guest.\evb
\evg


\bvg
\bva \alst{Á}rliga verðar \hld skyli maðr \alst{o}pt fáa, &
\ind nema til \alst{k}ynnis \alst{k}omi; &
\alst{s}itr ok \alst{s}nópir, \hld lætr sęm \alst{s}olginn sé, &
\ind ok kann \alst{f}regna at \alst{f}ǫ́u.\eva

\bvb An early meal should man often get, unless he is come to visit; he sits and idles haplessly about, makes as if he is starved, and knows to ask about little.\evb
\evg


\bvg
\bva \alst{A}fhvarf mikit \hld es til \alst{i}lls vinar, &
\ind þótt á \alst{b}rautu \alst{b}úi, &
ęn til \alst{g}óðs vinar \hld liggja \alst{g}agnvegir, &
\ind þótt hann sé \alst{f}irr \alst{f}arinn.\eva

\bvb A great detour is it to an ill friend, though on the highway he live; but to a good friend lie the shortest ways, though he be far gone.\evb
\evg


\bvg
\bva \alst{G}anga skal, \hld skal-a \alst{g}ęstr vesa &
\ind \alst{ęy} í \alst{ęi}num stað; &
\alst{l}júfr verðr \alst{l}ęiðr, \hld ef \alst{l}ęngi sitr &
\ind \alst{a}nnars flętjum \alst{á}.\eva

\bvb One shall go, he shall not be a guest ever in one place; lovely becomes the path, if long one sits on another’s bench.\evb
\evg


\bvg
\bva \alst{B}ú es \alst{b}ętra, \hld þótt lítit sé, &
\ind \alst{h}alr es \alst{h}ęima \alst{h}vęrr; &
þótt \alst{t}vær gęitr ęigi \hld ok \alst{t}augręptan sal, &
\ind þat es þó \alst{b}ętra an \alst{b}ǿn.\eva

\bvb A home is better, though little it be: each is a man at home; though two goats he own, and a cord-roofed hall, that is yet better than begging.\evb
\evg


\bvg
\bva \alst{B}ú es \alst{b}ętra, \hld þótt lítit sé, &
\ind \alst{h}alr es \alst{h}ęima \alst{h}vęrr; &
\alst{b}lóðugt es hjarta \hld þęim's \alst{b}iðja skal &
\ind sér í \alst{m}ál hvęrt \alst{m}atar.\eva

\bvb A home is better, though little it be: each is a man at home; bloody is the heart of the one who must beg for each meal of food.\evb
\evg


\bvg
\bva \alst{V}ǫ́pnum sínum \hld skal-a maðr \alst{v}ęlli á &
\ind \alst{f}eti ganga \alst{f}ramar; &
því’t ó\alst{v}íst's at \alst{v}ita, \hld nær verðr á \alst{v}egum úti &
\ind \alst{g}ęirs of þǫrf \alst{g}uma.\eva

\bvb From his weapons shall no man on the field take a step away; for uncertain ’tis to know, when on the ways outside, man comes in need of a spear.\evb
\evg


\bvg
\bva Fann-k-a \alst{m}ildan mann \hld eða svá \alst{m}atar góðan, &
\ind at væri-t \alst{þ}iggja \alst{þ}ęgit; &
eða \alst{s}íns féar \hld \alst{s}vági gløggvan, &
\ind at \alst{l}ęið sé \alst{l}aun, ef þegi.\eva

\bvb I found not a mild man, nor one so good of meat, that a gift was not received; nor of his wealth so frugal, that the rewards be loathed if he took them.\evb
\evg


\bvg
\bva \alst{F}éar síns, \hld es \alst{f}ęngit hęfr, &
\ind skyli-t maðr \alst{þ}ǫrf \alst{þ}ola; &
opt sparir \alst{l}ęiðum \hld þat's hęfr \alst{l}júfum hugat; &
\ind mart gęngr \alst{v}err an \alst{v}arir.\eva

\bvb 40\evb
\evg


\bvg
\bva \alst{V}ǫ́pnum ok \alst{v}ǫ́ðum \hld skulu \alst{v}inir glęðjask; &
\ind þat's á \alst{s}jǫlfum \alst{s}ýnst; &
\alst{v}iðrgefęndr \hld erusk \alst{v}inir lęngst, &
\ind ef þat bíðr at \alst{v}erða \alst{v}ęl.\eva

\bvb With weapons and garments shall friends gladden each other; that is most seen on oneself; mutual givers are friends for the longest, if [the friendship] takes to last long.\evb
\evg


\bvg
\bva \alst{V}in sínum \hld skal maðr \alst{v}inr \alst{v}esa, &
\ind ok \alst{g}jalda \alst{g}jǫf við \alst{g}jǫf; &
\alst{h}látr við \alst{h}látri \hld skyli \alst{h}ǫlðar taka, &
\ind en \alst{l}ausung við \alst{l}ygi.\eva

\bvb With his friend shall a man a friend be, and repay gift against gift; laughter against laughter should men take, but falsehood against lie.\evb
\evg


\bvg
\bva \alst{V}in sínum \hld skal maðr \alst{v}inr vesa, &
\ind \alst{þ}ęim ok \alst{þ}ęss vin; &
en \alst{ó}vinar síns \hld skyli \alst{ę}ngi maðr &
\ind \alst{v}inar \alst{v}inr \alst{v}esa.\eva

\bvb With his friend shall a man a friend be, with him and with his friend; but with his enemy’s, shall no man, friend’s friend be.\evb
\evg


\bvg
\bva \alst{V}ęizt, ef þú \alst{v}in átt, \hld þann's þú \alst{v}ęl trúir &
\ind ok vilt af hǫ́num \alst{g}ótt \alst{g}eta, &
\alst{g}ęði skalt við þann \hld ok \alst{g}jǫfum skipta, &
\ind \alst{f}ara at \alst{f}inna opt.\eva

\bvb Know, if thou hast a friend, which thou trust well, and wilt of him receive good: mind shalt thou with exchange him, and gifts; travel to see him oft.\evb
\evg


\bvg
\bva Ef þú \alst{á}tt \alst{a}nnan, \hld þann's þú \alst{i}lla trúir, &
\ind vildu af hǫ́num þó \alst{g}ótt \alst{g}eta, &
\alst{f}agrt skalt mæla, \hld ęn \alst{f}látt hyggja &
\ind ok gjalda \alst{l}ausung við \alst{l}ygi.\eva

\bvb If thou hast another, which thou trust little, thou wilt of him receive good; fairly shalt thou speak, but deceitfully think, and repay duplicity with lies.\evb
\evg


\bvg
\bva Þat's \alst{ę}nn of þann, \hld es þú \alst{i}lla trúir &
\ind ok þér es \alst{g}runr at \alst{g}ęði, &
\alst{h}læja skalt við þęim \hld ok of \alst{h}ug mæla; &
\ind glík skulu \alst{g}jǫld \alst{g}jǫfum.\eva

\bvb 46\evb
\evg


\bvg
\bva Ungr vas'k \alst{f}orðum, \hld \alst{f}ór'k ęinn saman, &
\ind þá varð'k \alst{v}illr \alst{v}ega; &
\alst{au}ðigr þóttumk, \hld es \alst{a}nnan fann'k, &
\ind \alst{m}aðr es \alst{m}anns gaman.\eva

\bvb Young I was once, I travelled all alone; then I became astray of the paths. Wealthy I thought myself when I another found; man is the joy of man.\evb
\evg


\bvg
\bva \alst{M}ildir frǿknir \hld \alst{m}ęnn bazt lifa, &
\ind \alst{s}jaldan \alst{s}út ala; &
\alst{ó}snjallr maðr \hld \alst{u}ggir hvatvetna, &
\ind sýtir æ \alst{g}løggr við \alst{g}jǫfum.\eva

\bvb 48\evb
\evg


\bvg
\bva \alst{V}áðir mínar \hld gaf'k \alst{v}ęlli at &
\ind \alst{t}vęim \alst{t}rémǫnnum; &
\alst{r}ekkar þat þóttusk, \hld es \alst{r}ipt hǫfðu; &
\ind \alst{n}ęiss es \alst{n}ǫkkviðr halr.\eva

\bvb My garments I gave at the field, to two tree-men. They seemed to be champions, when they cloaks had; ashamed is the naked man.\evb
\evg


\bvg
\bva Hrørnar \alst{þ}ǫll, \hld sú's stęndr \alst{þ}orpi á, &
\ind hlýrat hęnni \alst{b}ǫrkr né \alst{b}arr; &
svá es \alst{m}aðr, \hld sá’s \alst{m}anngi ann; &
\ind hvat skal hann \alst{l}ęngi \alst{l}ifa?\eva

\bvb Wilters the pine, which on the field stands; shield her neither bark nor needle; so is the man, who loves no man; for what shall he live long?\evb
\evg


\bvg
\bva \alst{Ę}ldi hęitari \hld brinnr með \alst{i}llum vinum &
\ind \alst{f}riðr \alst{f}imm daga, &
ęn þá \alst{sl}oknar, \hld es hinn \alst{s}étti kømr, &
\ind ok \alst{v}ersnar allr \alst{v}inskapr.\eva

\bvb Peace burns with ill friends for five days\footnoteB{As \Finnur\ points out, a reference to the five-day week; the number is symbolic.} hotter than fire; but it then goes out, when the sixth comes, and all the friendship worsens.\evb
\evg


\bvg
\bva \alst{M}ikit ęitt \hld skal-a \alst{m}anni gefa; &
\ind opt kaupir sér í \alst{l}ítlu \alst{l}of, &
með \alst{h}ǫlfum \alst{h}lęif \hld ok með \alst{h}ǫllu kęri &
\ind \alst{f}ekk ek mér \alst{f}élaga.\eva

\bvb Much alone, should one not give to a man; often little buys praise. With a half loaf [of bread] and an awry vat, I got me a fellow.\evb
\footnotetext[x]{Lit. “often [one] buys oneself in little praise"}
\evg


\bvg
\bva \alst{L}ítilla sanda, \hld \alst{l}ítilla sæva, &
\ind lítil eru \alst{g}ęð \alst{g}uma; &
því’t \alst{a}llir męnn \hld \alst{u}rðu-t jafnspakir; &
\ind \alst{h}ǫlf es ǫld \alst{h}var.\eva

\bvb Of small sands, of small seas; small are the senses of man. For all have not become evenly wise; half is each man.\footnoteB{Where shores are small, seas are small. Compared to the power of the natural forces, a man is only a grain of sand in the desert, a drop of water in the sea; his wisdom will always be incomplete.}\evb
\evg


\bvg
\bva \alst{M}eðalsnotr \hld skyli \alst{m}anna hvęrr, &
\ind æva til \alst{s}notr \alst{s}é; &
þęim es \alst{f}yrða \hld \alst{f}ęgrst at lifa, &
\ind es \alst{v}ęl mart \alst{v}itu.\eva

\bvb Middle-clever should each man be; never too clever. For those men is it fairest to live, who know well enough.\evb
\evg


\bvg
\bva \alst{M}eðalsnotr \hld skyli \alst{m}anna hvęrr, &
\ind æva til \alst{s}notr \alst{s}é; &
\alst{s}notrs manns hjarta \hld verðr \alst{s}jaldan glatt, &
\ind ef sá’s \alst{a}lsnotr es \alst{á}.\eva

\bvb Middle-clever should each man be; never too clever. The clever man’s heart seldom turns glad, if he is all-clever that owns it.\evb
\evg


\bvg
\bva \alst{M}eðalsnotr \hld skyli \alst{m}anna hvęrr, &
\ind æva til \alst{s}notr \alst{s}é; &
\alst{ø}rlǫg sin \hld viti \alst{ę}ngi fyrir; &
\ind þęim es \alst{s}orgalausastr \alst{s}efi.\eva

\bvb Middle-clever should each man be; never too clever. May no man know his \inx{orlay}{C} ahead; his is the most sorrowless mind\footnoteB{Who knows not his fate. One may contrast Weden who has knowledge of his own inevitable doom.}.\evb
\evg


\bvg
\bva \alst{B}randr af \alst{b}randi \hld \alst{b}rinnr unz \alst{b}runninn es, &
\ind \alst{f}uni kvęykisk af \alst{f}una; &
\alst{m}aðr af \alst{m}anni \hld verðr at \alst{m}áli kuðr; &
\ind ęn til \alst{d}ǿlskr af \alst{d}ul.\eva

\bvb Fire by fire burns until it is burnt; flame is kindled by flame. Man by man becomes known by speech, but the too dull by his conceit.\evb
\evg


\bvg
\bva \alst{Á}r skal rísa, \hld sá’s \alst{a}nnars vill &
\ind \alst{f}é eða \alst{f}jǫr hafa; &
sjaldan \alst{l}iggjandi ulfr \hld \alst{l}ær of getr, &
\ind né \alst{s}ofandi maðr \alst{s}igr.\eva

\bvb Early shall rise, he who another’s cattle or life will have; seldom does the lying wolf get a thigh, or the sleeping man victory.\evb
\evg


\bvg
\bva \alst{Á}r skal rísa, \hld sá's á \alst{y}rkjęndr fáa, &
\ind ok ganga síns \alst{v}erka á \alst{v}it; &
\alst{m}art of dvęlr \hld þann’s of \alst{m}orgin sefr, &
\ind \alst{h}alfr es auðr und \alst{h}vǫtum.\eva

\bvb Early shall rise, he who owns workers few, and go his work to meet; much is kept back from him who in the morning sleeps; half the wealth is due to the bold\Bfootnote{Half of one’s success comes from boldness.}.\evb
\evg


\bvg
\bva \alst{Þ}urra skíða \hld ok \alst{þ}akinna næfra, &
\ind þess kann \alst{m}aðr \alst{m}jǫt, &
ok þess \alst{v}iðar, \hld es \alst{v}innask męgi &
\ind \alst{m}ál ok \alst{m}issęri.\eva

\bvb Dry planks and thatching birch bark, of this man knows the use; and of that firewood, which may be used for a season and half-year\footnoteB{Over the winter.}.\evb
\evg


\bvg
\bva Þvęginn ok \alst{m}ęttr \hld ríði \alst{m}aðr þingi at, &
\ind þótt hann sé-t \alst{v}æddr til \alst{v}ęl; &
\alst{sk}úa ok bróka \hld \alst{sk}ammisk ęngi maðr &
\ind né \alst{h}ęsts in \alst{h}ęldr, \hld \edtext{þótt hann \alst{h}afi't góðan}{\lemma{þótt ... góðan “though ... good one”}\Afootnote{As \Finnur\ points out, surely a later addition.}}.\eva

\bvb Washed and filled may a man ride to the Thing, though he be not dressed too well; of his shoes and breeches ought no man be ashamed, nor indeed of his horse. (Though he have not a good one).\evb
\evg


\bvg
\bva \alst{S}napir ok gnapir, \hld es til \alst{s}ævar kømr, &
\ind \alst{ǫ}rn á \alst{a}ldinn mar; &
svá es \alst{m}aðr, \hld es með \alst{m}ǫrgum kømr &
\ind ok á \alst{f}ormælęndr \alst{f}áa.\eva

\bvb It shuffles and stoops, when to the sea it comes, the eagle on the aged ocean; so is the man who among the many comes, and has spokesmen few.\evb
\evg


\bvg
\bva \alst{F}regna ok sęgja \hld skal \alst{f}róðra hvęrr, &
\ind sá’s vill \alst{h}ęitinn \alst{h}orskr; &
\alst{ęi}nn vita \hld né \alst{a}nnarr skal, &
\ind \alst{þ}jóð vęit ef \alst{þ}rír eru.\eva

\bvb Ask and speak shall every learned man, who will be called sharp; one shall know, but not another: the people knows if there are three.\evb
\evg


\bvg
\bva \alst{R}íki sitt \hld skyli \alst{r}áðsnotra &
\ind \alst{h}vęrr í \alst{h}ófi \alst{h}afa; &
þá hann þat \alst{f}innr, \hld es með \alst{f}rǿknum kømr, &
\ind at \alst{ę}ngi es \alst{ęi}nna hvatastr. &
\alst{O}rða þęira, \hld es maðr \alst{ǫ}ðrum sęgir, &
\ind opt hann \alst{g}jǫld of \alst{g}etr.\eva

\bvb 64\evb
\evg


\bvg
\bva \alst{M}ikilsti snimma \hld kom’k í \alst{m}arga staði, &
\ind ęn til \alst{s}íð í \alst{s}uma; &
\alst{ǫ}l vas drukkit, \hld sumt vas \alst{ó}lagat; &
\ind sjaldan hittir \alst{l}ęiðr í \alst{l}ið.\eva

\bvb 65\evb
\evg


\bvg
\bva \alst{H}ér ok \alst{h}var \hld myndi mér \alst{h}ęim of boðit, &
\ind ef þyrpta'k at \alst{m}ǫ́lungi \alst{m}at, &
eða \alst{t}vau lær hęngi \hld at hins \alst{t}ryggva vinar, &
\ind þar's ek hafða \alst{ęi}tt \alst{e}tit.\eva

\bvb 66\evb
\evg


\bvg
\bva \alst{Ę}ldr es baztr \hld með \alst{ý}ta sonum &
\ind ok \alst{s}ólar \alst{s}ýn, &
\alst{h}ęilyndi sitt, \hld ef \alst{h}afa náir, &
\ind án við \alst{l}ǫst at \alst{l}ifa.\eva

\bvb Fire is best among the sons of men, and the sight of the sun; one’s good health, if one manage to have it, without living with vice.\evb
\evg


\bvg
\bva Es-at maðr \alst{a}lls vesall, \hld þótt sé \alst{i}lla hęill, &
\ind \alst{s}umr es af \alst{s}onum \alst{s}æll, &
\alst{s}umr af frændum, \hld \alst{s}umr af fé ǿrnu, &
\ind sumr af \alst{v}erkum \alst{v}ęl.\eva

\bvb No man is all wretched, though he of poor health be: Some find joy in sons,
some in friends, some in ample \inx{fee}{C}, some in works done well.\evb
\evg


\bvg
\bva Bętra es \alst{l}ifðum, \hld ok sæl\alst{l}ifðum, &
\ind ęy getr \alst{k}vikr \alst{k}ú; &
\alst{ę}ld sá’k \alst{u}pp brinna \hld \alst{au}ðgum manni fyr, &
\ind ęn úti vas \alst{d}auðr fyr \alst{d}urum.\eva

\bvb ’Tis better with the living, and the joyfully living; ever gets the quick\footnoteB{Living.} a cow\footnoteB{A reference to the cattle-based economy, the cow being used as a metonym. The meaning is that new opportunities always present themselves.}. A fire\footnoteB{His funeral-pyre.} I saw burning on high for a wealthy man, but outside he was dead before the door.\evb
\evg


\bvg
\bva \alst{H}altr ríðr \alst{h}rossi, \hld \alst{h}jǫrð rekr \alst{h}andarvanr, &
\ind daufr \alst{v}egr ok \alst{d}ugir; &
\alst{b}lindr es \alst{b}ętri, \hld an \alst{b}ręndr séi; &
\ind \alst{n}ýtr mangi \alst{n}ás.\eva

\bvb The halt man rides a horse, the handless drives a herd, the deaf fights and avails; blind is better than burnèd be: no man has use for a corpse.\evb
\evg


\bvg
\bva \alst{S}onr es bętri, \hld þótt sé \alst{s}íð of alinn &
\ind ęptir \alst{g}inginn \alst{g}uma; &
sjaldan \alst{b}autarstęinar \hld standa \alst{b}rautu nær, &
\ind nema ręisi \alst{n}iðr at \alst{n}ið.\eva

\bvb A son is better, though he late be born, after a passed-on man: seldom beat-stones near the highway stand, unless by kinsman for kinsman raised.\evb
\evg


\bvg
\bva \alst{T}vęir 'ro ęins hęrjar, \hld \alst{t}unga es hǫfuðs bani; &
\ind mér's í \alst{h}eðin \alst{h}vęrn \hld \alst{h}andar væni.\eva

\bvb Two are the harriers of one: the tongue is the head’s bane; in every cloak I expect a hand.\evb
\evg


\bvg
\bva \alst{N}ótt verðr fęginn, \hld sá's \alst{n}esti trúir, &
\ind \alst{sk}ammar 'ro \alst{sk}ips ráar. &
– – – – &
\ind \alst{H}verf es \alst{h}austgríma, &
\alst{f}jǫlð of viðrir \hld á \alst{f}imm dǫgum, &
\ind en \alst{m}ęir á \alst{m}ánaði.\eva

\bvb 73\evb
\evg


\bvg
\bva \alst{V}ęit-a hinn, \hld es \alst{v}ætki \alst{v}ęit, &
\ind margr verðr \edtext{af \alst{au}rum}{\lemma{af aurum}\footnoteA{ ‘aflꜹðrom’ \emph{ms.}}} \alst{a}pi; &
maðr es \alst{au}ðigr, \hld annarr \alst{ó}auðigr, &
\ind skyli-t þann \alst{v}ítka \alst{v}áar.\eva

\bvb He knows not, who nothing knows: treasures make many a man a fool; a man is wealthy, another not wealthy, one oughtn’t curse him for his woe.\evb
\evg


\bvg
\bva \alst{D}ęyr fé, \hld \alst{d}ęyja frændr, &
\ind dęyr \alst{s}jalfr hit \alst{s}ama; &
ęn \alst{o}rðstírr \hld dęyr \alst{a}ldrigi &
\ind hvęim's sér \alst{g}óðan \alst{g}etr.\eva

\bvb \inx{Fee}{C} dies, kinsmen die\Bfootnote{The power of this succinct expression may be less clear to the modern reader. In Germanic Iron Age society wealth was measured by how many heads of cattle a man owned, but he would only be able to keep it if he had a strong clan of male relatives ready to support and protect him.}, oneself dies the same; but a word-glory never dies, for whomever gets himself a good one.\evb
\evg


\bvg
\bva \alst{D}ęyr fé, \hld \alst{d}ęyja frændr, &
\ind dęyr \alst{s}jalfr hit \alst{s}ama; &
\alst{e}k vęit \alst{ęi}nn \hld at \alst{a}ldri dęyr: &
\ind \alst{d}ómr of \alst{d}auðan hvęrn.\eva

\bvb \inx{Fee}{C} dies, kinsmen die, oneself dies the same; I know but one, that never dies: the \inx{Doom}{C} over each man dead.\evb
\evg


\bvg
\bva \alst{F}ullar grindr \hld sá'k fyr \alst{F}itjungs sonum, &
\ind nú bera þęir \alst{v}ánar \alst{v}ǫl; &
svá es \alst{au}ðr \hld sęm \alst{au}gabragð, &
\ind hann es \alst{v}altastr \alst{v}ina. —\eva

\bvb 77\evb
\evg


\bvg
\bva \alst{Ó}snotr maðr, \hld es \alst{ęi}gnask getr &
\ind \alst{f}é eða \alst{f}ljóðs munuð; &
\alst{m}etnaðr hǫ́num þróask, \hld ęn \alst{m}anvit aldrigi; &
\ind framm gęngr hann \alst{d}rjúgt í \alst{d}ul. —\eva

\bvb 78\evb
\evg


\bvg
\bva Þat es þá \alst{r}ęynt, \hld es þú at \alst{r}únum spyrr \hld hinum \alst{r}ęginkunnum, &
\ind þęim's \alst{g}ęrðu \alst{g}innręgin &
\ind ok \alst{f}áði \alst{f}imbulþulr; &
\ind þá hęfr hann bazt, ef þęgir. —\eva

\bvb Then that is proven of which thou inquires the runes, the ones born of the Powers, those which the yin-Powers made, and the fimbol-thyle painted; then he has it best, if he shuts up.\evb
\evg


\bvg
\bva At \alst{k}veldi skal dag lęyfa, \hld \alst{k}onu es bręnd es, &
\alst{m}æki es ręyndr es, \hld \alst{m}ęy es gefin es, &
\alst{í}s es \alst{y}fir kømr, \hld \alst{ǫ}l es drukkit es. —\eva

\bvb At evening [one] shall praise day, a woman when [she] is burned, a sword when [it] is tried, a maiden when [she] is given. Ice when [one] comes over, ale when [it] is drunk.\evb
\evg


\bvg
\bva Í \alst{v}indi skal \alst{v}ið hǫggva, \hld \alst{v}eðri á sæ róa, &
\alst{m}yrkri við \alst{m}an spjalla, \hld \alst{m}ǫrg eru dags augu, &
á \alst{sk}ip skal \alst{sk}riðar orka, \hld ęn á \alst{sk}jǫld til hlífar, &
\alst{m}æki til hǫggs, \hld ęn \alst{m}ęy til kossa. —\eva

\bvb In wind [one] shall cut wood, [in] storm row on the sea, [in] darkness meet with a maid; many are the eyes of day. A ship [one] shall use for its speed, a shield for shelter, a sword for striking, but a maiden for [her] kisses.\evb
\evg


\bvg
\bva Við \alst{e}ld skal \alst{ǫ}l drekka. \hld ęn á \alst{í}si skríða, &
\alst{m}agran \alst{m}ar kaupa, \hld ęn \alst{m}æki saurgan, &
\alst{h}ęima \alst{h}ęst fęita, \hld ęn \alst{h}und á búi. \eva

\bvb 82\evb
\evg


\bvg Regarding the love of women, and Woden's failed love-adventures.
\bva \alst{M}ęyjar orðum \hld skyli \alst{m}angi trúa, &
\ind né því's \alst{k}veðr \alst{k}ona; &
á \alst{h}verfanda \alst{h}véli \hld vǫ́ru þęim \alst{h}jǫrtu skǫpuð, &
\ind \alst{b}rigð í \alst{b}rjóst of lagið\footnotemark[29]. \eva
\footnotetext[29]{The second half of this st. is quoted in \emph{Fbr}, Thott 1768 4°\textsuperscript{x}, ff.: “And he then remembered the short verse, which had been composed about unfaithful women:"}

\bvb The words of a maiden should no man believe, nor that which a woman says. On a spinning wheel were their hearts for them shaped; betrayal laid in their breasts.\evb
\evg


\bvg
\bva \alst{B}restanda \alst{b}oga, \hld \alst{b}rinnanda loga, &
\alst{g}ínanda ulfi, \hld \alst{g}alandi krǫ́ku, &
\alst{r}ýtanda svíni, \hld \alst{r}ótlausum viði, &
\alst{v}axanda \alst{v}ági, \hld \alst{v}ellanda katli,\eva

\bvb Bursting bow, burning flame, gaping wolf, crowing crow, roaring swine, rootless tree, waxing wave, swelling kettle,\evb
\evg


\bvg
\bva \alst{f}ljúganda \alst{f}lęini, \hld \alst{f}allandi bǫ́ru, &
\alst{í}si \alst{ęi}nnættum, \hld \alst{o}rmi hringlęgnum, &
\alst{b}rúðar \alst{b}ęðmǫ́lum \hld eða \alst{b}rotnu sverði, &
\alst{b}jarnar lęiki \hld eða \alst{b}arni konungs,
\alst{s}júkum kalfi, \hld \alst{s}jalfráða þræli, &
\alst{v}ǫlu \alst{v}ilmæli, \hld \alst{v}al nýfęldum. —\eva

\bvb 85\evb
\evg


\bvg
\bva \alst{A}kri \alst{á}rsǫ́num \hld trúi \alst{ę}ngi maðr, &
\ind né til \alst{s}nimma \alst{s}yni; &
\alst{v}eðr ræðr akri, \hld ęn \alst{v}it syni; &
\ind \alst{h}ætt es þęira \alst{h}várt. —\eva

\bvb 86\evb
\evg


\bvg
\bva \alst{B}róður\alst{b}ana sínum \hld þótt á \alst{b}rautu mǿti, &
\alst{h}úsi \alst{h}alfbrunnu, \hld \alst{h}ęsti alskjótum, &
þá's \alst{jó}r \alst{ó}nýtr, \hld ef \alst{ęi}nn fótr brotnar; &
verðit maðr svá \alst{t}ryggr \hld at þessu \alst{t}rúi ǫllu. —\eva

\bvb 87\evb
\evg


\bvg
\bva Svá's \alst{f}riðr kvinna \hld þęira's \alst{f}látt hyggja, &
sęm aki \alst{jó} óbryddum \hld á \alst{í}si hǫ́lum &
(\alst{t}ęitum, \alst{t}vévetrum \hld ok sé \alst{t}amr illa), &
eða í \alst{b}yr óðum \hld \alst{b}ęiti stjórnlausu, &
eða skyli \alst{h}altr \alst{h}ęnda \hld \alst{h}ręin í þáfjalli. —\eva

\bvb 88\evb
\evg


\bvg
\bva \alst{B}ert nú mæli'k, \hld því-at \alst{b}æði vęit'k, &
\ind brigðr es \alst{k}arla hugr \alst{k}onum, &
þá \alst{f}ęgrst mælum, \hld es \alst{f}lást hyggjum; &
\ind þat tælir \alst{h}orska \alst{h}ugi.\eva

\bvb 89\evb
\evg


\bvg
\bva \alst{F}agrt skal mæla \hld ok \alst{f}é bjóða, &
\ind sá's vill \alst{f}ljóðs ǫ́st \alst{f}áa, &
\alst{l}íki \alst{l}ęyfa \hld hins \alst{l}jósa mans, &
\ind sá \alst{f}ær, es \alst{f}ríar.\eva

\bvb 90\evb
\evg


\bvg
\bva \alst{Á}star firna \hld skyli \alst{ę}ngi maðr &
\ind \alst{a}nnan \alst{a}ldrigi; &
opt fáa á \alst{h}orskan, \hld es á \alst{h}ęimskan né fáa, &
\ind \alst{l}ostfagrir \alst{l}itir.\eva

\bvb 91\evb
\evg


\bvg
\bva \alst{Ęy}vitar firna, \hld es maðr \alst{a}nnan skal, &
\ind þess's of margan \alst{g}ęngr \alst{g}uma; &
\alst{h}ęimska ór \alst{h}orskum \hld gęrir \alst{h}ǫlða sonu &
\ind sá hinn \alst{m}átki \alst{m}unr.\eva

\bvb 92\evb
\evg


\bvg
\bva \alst{H}ugr ęinn þat vęit, \hld es býr \alst{h}jarta nær, &
\ind ęinn es hann \alst{s}ér of \alst{s}efa; &
øng es \alst{s}ótt verri \hld hvęim \alst{s}notrum manni &
\ind an sér \alst{ø}ngu at \alst{u}na.\eva

\bvb 93\evb
\evg


\bvg
\bva Þat þá \alst{r}ęyndak, \hld es í \alst{r}ęyri sat'k, &
\ind ok vætta'k \alst{m}íns \alst{m}unar, &
\alst{h}old ok \alst{h}jarta \hld vas mér hin \alst{h}orska mær, &
\ind þęygi hana at \alst{h}ęldr \alst{h}ęf'k.\eva

\bvb 94\evb
\evg


\bvg
\bva \alst{B}illings męy \hld ek fann \alst{b}ęðjum á &
\ind \alst{s}ólhvíta \alst{s}ofa; &
\alst{ja}rls \alst{y}nði \hld þótti mér \alst{ę}kki vesa &
\ind nema við þat \alst{l}ík at \alst{l}ifa.\eva

\bvb 95\evb
\evg


\bvg
\bva “\alst{Au}k nær \alst{a}ptni \hld skaltu \alst{Ó}ðinn koma, &
\ind ef vilt þér \alst{m}æla \alst{m}an, &
\alst{a}lt eru \alst{ó}skǫp, \hld nema \alst{ęi}n vitim &
\ind \alst{s}likan lǫst \alst{s}aman.”\eva

\bvb 96\evb
\evg


\bvg
\bva \alst{A}ptr ek hvarf \hld ok \alst{u}nna þóttumk &
\ind \alst{v}ísum \alst{v}ilja frá; &
\alst{h}itt ek \alst{h}ugða, \hld at \alst{h}afa mynda'k &
\ind \alst{g}ęð hęnnar alt ok \alst{g}aman.\eva

\bvb 97\evb
\evg


\bvg
\bva Svá kom'k \alst{n}æst, \hld at hin \alst{n}ýta vas &
\ind \alst{v}ígdrótt ǫll of \alst{v}akin; &
með \alst{b}rinnǫndum ljósum \hld ok \alst{b}ornum viði, &
\ind svá vas mér \alst{v}ílstígr of \alst{v}itaðr.\eva

\bvb 98\evb
\evg


\bvg
\bva \alst{Au}k nær morni, \hld es vas'k \alst{ę}nn of kominn, &
\ind þá vas \alst{s}aldrótt of \alst{s}ofin; &
\alst{g}ręy ęitt þá fan'k \hld hinnar \alst{g}óðu konu &
\ind \alst{b}undit \alst{b}ęðjum á.\eva

\bvb 99\evb
\evg


\bvg
\bva Mǫrg es \alst{g}óð mær, \hld ef \alst{g}ǫrva kannar, &
\ind \alst{h}ugbrigð við \alst{h}ali; &
þá þat \alst{r}ęynda'k, \hld es hit \alst{r}áðspaka &
\ind tęygða'k á \alst{f}lærðir \alst{f}ljóð. &
\alst{h}ǫ́ðungar \alst{h}vęrrar \hld lęitaði mér hit \alst{h}orska man &
\ind ok hafða'k þess \alst{v}ætki \alst{v}ífs.\eva

\bvb 100\evb
\evg


\bvg Side-composition to the previous poem, starting with a general maxim.
\bva Hęima \alst{g}laðr \hld ok við \alst{g}ęsti ręifr, &
\ind \alst{s}viðr skal of \alst{s}ik vesa; &
\alst{m}innigr ok \alst{m}ǫ́lugr, \hld ef vill \alst{m}argfróðr vesa; &
\ind opt skal \alst{g}óðs \alst{g}eta; &
\alst{f}imbul\alst{f}ambi hęitir, \hld sás \alst{f}átt kann sęgja; &
\ind þat es \alst{ó}snotrs \alst{a}ðal.\eva

\bvb 101\evb
\evg


\bvg
\bva Hinn \alst{a}ldna \alst{jǫ}tun sóttak, \hld nú em'k \alst{a}ptr of kominn; &
\ind fátt gat'k \alst{þ}ęgjandi \alst{þ}ar; &
\alst{m}ǫrgum orðum \hld \alst{m}ælta'k í minn frama &
\ind í \alst{S}uttungs \alst{s}ǫlum.\eva

\bvb The old ettin I sought, now am I come back; I got little silence there. Many words I spoke to my fame, in the halls of Suttung.\evb
\evg


\bvg
\bva \alst{G}unnlǫð mér of \alst{g}af \hld \alst{g}ollnum stóli á &
\ind \alst{d}rykk hins \alst{d}ýra mjaðar; &
\alst{i}ll \alst{i}ðgjǫld \hld lét'k hana \alst{ę}ptir hafa &
\ind síns hins \alst{h}ęila \alst{h}ugar. &
\ind (síns hins \alst{s}vára \alst{s}efa).\eva

\bvb 103\evb
\evg


\bvg
\bva \alst{R}ata munn \hld létumk \alst{r}úms of fáa &
\ind ok of \alst{g}rjót \alst{g}naga; &
\alst{y}fir ok \alst{u}ndir \hld stóðumk \alst{jǫ}tna vegir, &
\ind svá \alst{h}ættak \alst{h}ǫfði til.\eva

\bvb 104\evb
\evg


\bvg
\bva \alst{V}ęl kęypts hlutar \hld hęf'k \alst{v}ęl notit; &
\ind \alst{f}ás es \alst{f}róðum vant; &
\alst{Ó}ðrerir \hld nú \alst{u}pp's kominn &
\ind á \alst{a}lda vé \alst{ja}ðars.\eva

\bvb 105\evb
\evg


\bvg
\bva \alst{I}fi es mér á, \hld at væra'k \alst{ę}nn kominn &
\ind \alst{jǫ}tna gǫrðum \alst{ó}r, &
ef \alst{G}unnlaðar né nyta'k, \hld hinnar \alst{g}óðu konu, &
\ind es lǫgðumk \alst{a}rm \alst{y}fir.\eva

\bvb 106\evb
\evg


\bvg
\bva Hins \alst{h}indra dags \hld gingu \alst{h}rímþursar &
\ind (Háva ráðs at fregna) &
\ind \alst{H}áva \alst{h}ǫllu í, &
at \alst{B}ǫlverki spurðu, \hld ef væri með \alst{b}ǫndum kominn &
\ind eða hęfði hǫ́num \alst{S}uttungr of \alst{s}óit.\eva

\bvb 107\evb
\evg


\bvg
\bva Baugęið \alst{Ó}ðinn \hld hygg at \alst{u}nnit hafi, &
\ind hvat skal hans \alst{t}ryggðum \alst{t}rúa? &
\alst{S}uttung \alst{s}vikvinn \hld hann lét \alst{s}umbli frá &
\ind ok \alst{g}rǿtta \alst{G}unnlǫðu.\eva

\bvb 108\evb
\evg


Advice of the Fimble-Thyle, given to Loddfáfnir.


\bvg
\bva Mál’s at \alst{þ}ylja \hld \alst{þ}ular stóli á; &
\ind \alst{U}rðar brunni \alst{a}t &
\alst{s}á’k ok þagða’k, \hld \alst{s}á’k ok hugða’k, &
\ind hlýdda’k á \alst{m}anna \alst{m}ál; &
of \alst{r}únar hęyrða’k dǿma, \hld né um \alst{r}ǫ́ðumk þǫgðu &
\ind \alst{H}áva \alst{h}ǫllu at, &
\ind \alst{H}áva \alst{h}ǫllu í &
\ind hęyrða'k \alst{s}ęgja \alst{s}vá:\eva

\bvb It is time to \inx{thilly}{C}, upon the chair of the \inx{thyle}{C}. At the well of Weird I saw and I was silent; I saw and I pondered; I heeded the matters of men. Of runes I heard them speak, nor about counsels were they silent; at the hall of the High One; in the hall of the High One, I heard them say thus:\evb
\evg


\bvg
\bva \alst{R}ǫ́ðumk þér Loddfáfnir, \hld at þú \alst{r}ǫ́ð nemir, &
\ind \alst{n}jǫta munt ef \alst{n}emr, &
\ind þér munu \alst{g}óð ef \alst{g}etr: &
\alst{n}ótt þú rís-at, \hld nema á \alst{n}jósn séir, &
\ind eða lęitir þér \alst{i}nnan \alst{ú}t staðar.\eva

\bvb I counsel thee Loddfathomer, that thou take the counsels; thou wilt benefit if thou take them; they will be good for thee if thou get them: At night thou rise not, unless at scouting thou be, or TODO\evb
\evg


\bvg
\bva \alst{R}ǫ́ðumk þér Loddfáfnir, \hld at þú \alst{r}ǫ́ð nemir, &
\ind \alst{n}jóta munt ef \alst{n}emr, &
\ind þér munu \alst{g}óð ef \alst{g}etr: &
\alst{f}jǫlkunnigri konu \hld skal-at-tu í \alst{f}aðmi sofa, &
\ind svá at hon \alst{l}yki þik \alst{l}iðum. &
Hón svá \alst{g}ęrir \hld at þú \alst{g}áir ęigi &
\ind \alst{þ}ings né \alst{þ}jóðans máls &
\alst{m}at þú vill-at \hld né \alst{m}anskis gaman &
\ind fęrr þú \alst{s}orgafullr at \alst{s}ofa.\eva

\bvb 111\evb
\evg


\bvg
\bva \alst{R}ǫ́ðumk þér Loddfáfnir, \hld at þú \alst{r}ǫ́ð nemir, &
\ind \alst{n}jóta munt ef \alst{n}emr, &
\ind þér munu \alst{g}óð ef \alst{g}etr: &
\alst{a}nnars konu \hld tęyg þér \alst{a}ldrigi &
\ind \alst{ęy}rarúnu \alst{a}t.\eva

\bvb 112\evb
\evg


\bvg
\bva \alst{R}ǫ́ðumk þér Loddfáfnir, \hld at þú \alst{r}ǫ́ð nemir, &
\ind \alst{n}jóta munt ef \alst{n}emr, &
\ind þér munu \alst{g}óð ef \alst{g}etr: &
á \alst{f}jalli eða \alst{f}irði, \hld ef þik \alst{f}ara tíðir, &
\ind fásk-tu at \alst{v}irði \alst{v}ęl.\eva

\bvb 113\evb
\evg


\bvg
\bva \alst{R}ǫ́ðumk þér Loddfáfnir, \hld at þú \alst{r}ǫ́ð nemir, &
\ind \alst{n}jóta munt ef \alst{n}emr, &
\ind þér munu \alst{g}óð ef \alst{g}etr: &
\alst{i}llan mann \hld lát \alst{a}ldrigi &
\ind óhǫpp at þér vita. &
af \alst{i}llum manni \hld fær þú \alst{a}ldrigi &
\ind \alst{g}jǫld hins \alst{g}óða hugar.\eva

\bvb 114\evb
\evg


\bvg
\bva \alst{O}farla bíta \hld sá'k \alst{ęi}num hal &
\ind \alst{o}rð \alst{i}llrar konu, &
\alst{f}lárǫ́ð tunga \hld varð hǫ́num at \alst{f}jǫrlagi &
\ind ok þęygi of \alst{s}anna \alst{s}ǫk.\eva

\bvb 115\evb
\evg


\bvg
\bva \alst{R}ǫ́ðumk þér Loddfáfnir, \hld at þú \alst{r}ǫ́ð nemir, &
\ind \alst{n}jóta munt ef \alst{n}emr, &
\ind þér munu \alst{g}óð ef \alst{g}etr: &
\alst{v}ęizt ef \alst{v}in átt, \hld þann's \alst{v}el trúir, &
\ind \alst{f}ar þú at \alst{f}inna opt. &
því’t \alst{h}rísi vęx \hld ok \alst{h}óu grasi &
\ind \alst{v}egr, es \alst{v}ætki trøðr,\eva

\bvb 116\evb
\evg


\bvg
\bva \alst{R}ǫ́ðumk þér Loddfáfnir, \hld at þú \alst{r}ǫ́ð nemir, &
\ind \alst{n}jóta munt ef \alst{n}emr, &
\ind þér munu \alst{g}óð ef \alst{g}etr: &
\alst{v}in þínum \hld \alst{v}es þú aldrigi &
\ind \alst{f}yrri at \alst{f}laumslitum. &
\alst{s}org etr hjarta, \hld ef þú \alst{s}ęgja né náir &
\ind \alst{ęi}nhvęrjum \alst{a}llan hug.\eva

\bvb 117\evb
\evg


\bvg
\bva \alst{R}ǫ́ðumk þér Loddfáfnir, \hld at þú \alst{r}ǫ́ð nemir, &
\ind \alst{n}jóta munt ef \alst{n}emr, &
\ind þér munu \alst{g}óð ef \alst{g}etr: &
\alst{g}óðan mann \hld tęyg þér at \alst{g}amanrúnum &
\ind ok nem \alst{l}íknargaldr meðan \alst{l}ifir.\eva

\bvb 118\evb
\evg


\bvg
\bva \alst{R}ǫ́ðumk þér Loddfáfnir, \hld at þú \alst{r}ǫ́ð nemir, &
\ind \alst{n}jóta munt ef \alst{n}emr, &
\ind þér munu \alst{g}óð ef \alst{g}etr: &
orðum \alst{sk}ipta \hld þú \alst{sk}alt aldrigi &
\ind við \alst{ó}svinna \alst{a}pa.\eva

\bvb 119\evb
\evg


\bvg
\bva Af illum \alst{m}anni \hld \alst{m}undu aldrigi &
\ind \alst{g}óðs laun of \alst{g}eta, &
en \alst{g}óðr maðr \hld mun þik \alst{g}ęrva męga &
\ind \alst{l}íknfastan at \alst{l}ofi.\eva

\bvb 120\evb
\evg


\bvg
\bva \alst{S}ifjum es þá blandit \hld hvęrr es \alst{s}ęgja ræðr &
\ind \alst{ęi}num \alst{a}llan hug; &
alt es \alst{b}ętra \hld an sé \alst{b}rigðum at vesa: &
\ind es-a sá \alst{v}inr es \alst{v}ilt ęitt sęgir.\eva

\bvb 121\evb
\evg


\bvg
\bva \alst{R}ǫ́ðumk þér Loddfáfnir, \hld at þú \alst{r}ǫ́ð nemir, &
\ind \alst{n}jóta munt ef \alst{n}emr, &
\ind þér munu \alst{g}óð ef \alst{g}etr: &
þrimr orðum sęnna \hld skal-at-tu þér við verra mann, &
\ind opt hinn \alst{b}ętri \alst{b}ilar. &
\ind þás hinn \alst{v}erri \alst{v}egr.\eva

\bvb 122\evb
\evg


\bvg
\bva \alst{R}ǫ́ðumk þér Loddfáfnir, \hld at þú \alst{r}ǫ́ð nemir, &
\ind \alst{n}jóta munt ef \alst{n}emr, &
\ind þér munu \alst{g}óð ef \alst{g}etr: &
\alst{sk}ósmiðr þú verir\hld né \alst{sk}ęptismiðr, &
\ind nema \alst{s}jǫlfum þér \alst{s}éir. &
\alst{Sk}ór's \alst{sk}apaðr illa\hld eða \alst{sk}apt sé rangt, &
\ind þá's þér \alst{b}ǫls \alst{b}eðit.\eva

\bvb 123\evb
\evg


\bvg
\bva \alst{R}ǫ́ðumk þér Loddfáfnir, \hld at þú \alst{r}ǫ́ð nemir, &
\ind \alst{n}jóta munt ef \alst{n}emr, &
\ind þér munu \alst{g}óð ef \alst{g}etr: &
hvars þú \alst{b}ǫl kant, \hld kveð þér \alst{b}ǫlvi at &
\ind ok gefat þínum \alst{f}jǫ́ndum \alst{f}rið.\eva

\bvb 124\evb
\evg


\bvg
\bva \alst{R}ǫ́ðumk þér Loddfáfnir, \hld at þú \alst{r}ǫ́ð nemir, &
\ind \alst{n}jóta munt ef \alst{n}emr, &
\ind þér munu \alst{g}óð ef \alst{g}etr: &
\alst{i}llu fęginn \hld ves þú \alst{a}ldrigi, &
\ind ęn lát þér at \alst{g}óðu \alst{g}etit.\eva

\bvb 125\evb
\evg


\bvg
\bva \alst{R}ǫ́ðumk þér Loddfáfnir, \hld at þú \alst{r}ǫ́ð nemir, &
\ind \alst{n}jóta munt ef \alst{n}emr, &
\ind þér munu \alst{g}óð ef \alst{g}etr: &
\alst{u}pp líta \hld skalattu í \alst{o}rrostu &
\alst{g}jalti \alst{g}líkir \hld verða \alst{g}umna synir &
\ind síðr þitt of \alst{h}ęilli \alst{h}alir.\eva

\bvb 126\evb
\evg


\bvg
\bva \alst{R}ǫ́ðumk þér Loddfáfnir, \hld at þú \alst{r}ǫ́ð nemir, &
\ind \alst{n}jóta munt ef \alst{n}emr, &
\ind þér munu \alst{g}óð ef \alst{g}etr: &
Ef vilt þér góða \alst{k}onu \hld \alst{k}vęðja at gamanrúnum &
\ind ok \alst{f}á \alst{f}ǫgnuð af, &
\alst{f}ǫgru skaldu heita \hld ok láta \alst{f}ast vesa; &
\ind lęiðisk mangi \alst{g}ótt ef \alst{g}etr.\eva

\bvb 127\evb
\evg


\bvg
\bva \alst{R}ǫ́ðumk þér Loddfáfnir, \hld at þú \alst{r}ǫ́ð nemir, &
\ind \alst{n}jóta munt ef \alst{n}emr, &
\ind þér munu \alst{g}óð ef \alst{g}etr: &
\ind \alst{v}aran bið'k þik \alst{v}esa &
\ind ok \alst{ęi}gi \alst{o}fvaran, &
ves þú við \alst{ǫ}l varastr. \hld ok við \alst{a}nnars konu &
ok við \alst{þ}at hit \alst{þ}riðja, \hld at \alst{þ}jófar né lęiki.\eva

\bvb 128\evb
\evg


\bvg
\bva \alst{R}ǫ́ðumk þér Loddfáfnir, \hld at þú \alst{r}ǫ́ð nemir, &
\ind \alst{n}jóta munt ef \alst{n}emr, &
\ind þér munu \alst{g}óð ef \alst{g}etr: &
at \alst{h}áði né \alst{h}látri \hld \alst{h}af þú aldrigi &
\ind \alst{g}ęst né \alst{g}anganda.\eva

\bvb 129\evb
\evg


\bvg
\bva \alst{O}pt vitu \alst{ó}gǫrla, \hld þęir's sitja \alst{i}nni fyrir, &
\ind hvęrs þęir 'ro \alst{k}yns es \alst{k}oma; &
esat maðr svá \alst{g}óðr \hld at \alst{g}alli né fylgi, &
\ind né svá \alst{i}llr at \alst{ęi}nugi dugi.\eva

\bvb 130\evb
\evg


\bvg
\bva \alst{R}ǫ́ðumk þér Loddfáfnir, \hld at þú \alst{r}ǫ́ð nemir, &
\ind \alst{n}jóta munt ef \alst{n}emr, &
\ind þér munu \alst{g}óð ef \alst{g}etr: &
at \alst{h}ǫ́rum þul \hld \alst{h}læ þú aldrigi, &
\ind opt es \alst{g}ótt þats \alst{g}amlir kveða, &
opt ór \alst{sk}ǫrpum bęlg \hld \alst{sk}ilin orð koma &
\ind þęims \alst{h}angir með \alst{h}ǫ́m &
\ind ok \alst{sk}ollir með \alst{sk}rǫ́m, &
\ind ok \alst{v}áfir með \alst{v}ilmǫgum.\eva

\bvb 131\evb
\evg


\bvg
\bva \alst{R}ǫ́ðumk þér Loddfáfnir, \hld at þú \alst{r}ǫ́ð nemir, &
\ind \alst{n}jóta munt ef \alst{n}emr, &
\ind þér munu \alst{g}óð ef \alst{g}etr: &
\alst{g}ęst né \alst{g}ęyja \hld né á \alst{g}rind hrękir; &
\ind get þú \alst{v}ǫ́luðum \alst{v}el.\eva

\bvb 132\evb
\evg


\bvg
\bva \alst{R}amt es þat tré, \hld es \alst{r}íða skal &
\ind \alst{ǫ}llum at \alst{u}pploki; &
\alst{b}aug þú gef \hld eða þat \alst{b}iðja mun &
\ind þér \alst{l}æs hvęrs á \alst{l}iðu.\eva

\bvb 133\evb
\evg


\bvg
\bva \alst{R}ǫ́ðumk þér Loddfáfnir, \hld at þú \alst{r}ǫ́ð nemir, &
\ind \alst{n}jóta munt ef \alst{n}emr, &
\ind þér munu \alst{g}óð ef \alst{g}etr: &
hvars \alst{ǫ}l drekkir \hld kjós þér \alst{ja}rðar męgin, &
því’t \alst{jǫ}rð tękr við \alst{ǫ}ldri, \hld ęn \alst{ę}ldr við sóttum, &
\alst{ęi}k við \alst{a}bbindi, \hld \alst{a}x við fjǫlkyngi, &
\alst{h}ǫll við \alst{h}ýrógi; \hld \alst{h}ęiptum skal mána kvęðja, &
\alst{b}ęiti við \alst{b}itsóttum, \hld ęn við \alst{b}ǫlvi rúnar; &
\ind \alst{f}old skal við \alst{f}lóði taka.\eva

\bvb 134\evb
\evg


Of Woden's taking of the runes.
It is clear that these verses have very little to do with the rest of the poem, but instead are separate. It is for this reason that they are labelled as \emph{Rúnatals þáttr} (The strand of the Runecount) in younger Eddic paper manuscripts. Many give an archaic, pagan impression. It is as if they were drawn from the lips of an Odinic priest.


\bvg
\bva \footnotemark[2] \alst{V}ęit'k at ek hekk \hld \alst{v}indga męiði á &
\ind \alst{n}ætr allar \alst{n}íu, &
\alst{g}ęiri undaðr \hld ok \alst{g}efinn Óðni, &
\ind \alst{s}jalfr \alst{s}jǫlfum mér, &
á þęim \alst{m}ęiði, \hld es \alst{m}angi vęit, &
\ind hvęrs af \alst{r}ótum \alst{r}innr\footnotemark[3].\eva
\footnotetext[3]{This v. clearly belongs together with 138 and 140 and may have been part of a greater now-lost work.}
\footnotetext[3]{Note that \emph{rótum} has lost initial \emph{v-}. Cf. 150/2, which retains it. Unlike the specifically WN. loss of \emph{v} before \emph{r} (cf. 32/2), in this word it occurred in all NGmc. dialects, cf. Swe. \emph{rot}. This may make it less reliable to use for dating the stanza.}

\bvb I know that I hung upon a windy beam [for] nine nights all, gored by spear, and given to Woden, myself to myself, on that tree, which no man knows whereof his roots run.\evb
\evg


\bvg
\bva Við \alst{h}lęifi mik sældu \hld né við \alst{h}ornigi; &
\alst{n}ýsta'k \alst{n}iðr, \hld \alst{n}am'k upp rúnar, &
\alst{ǿ}pandi nam, \hld fell'k \alst{a}ptr þaðan.\eva

\bvb With a loaf [of bread] they gladdened me not, nor with horn's drink. I peered down, I took up the runes, screaming [I] took; then I fell back thence.\evb
\evg


\bvg
\bva \alst{F}imbulljóð níu \hld nam'k af hinum \alst{f}rægja syni &
\ind \alst{B}ǫlþorns, \alst{B}ęstlu fǫður, &
ok ek \alst{d}rykk of gat \hld hins \alst{d}ýra mjaðar &
\ind \alst{au}sinn \alst{Ó}ðreri.\footnotemark[1] —\eva
\footnotetext[1]{It has been noted (FJ) that this verse fits better in the next section of the poem. It is awkwardly placed here, since it mentions \emph{ljóð} '(magical) songs', rather than runes.}

\bvb Nine fimbol-songs, I got from the famous son of \textbf{Balethorn}, the father of \textbf{Bestel}. And a drink I got, of the expensive mead poured to \textbf{Woderearer}.\evb
\evg


\bvg
\bva Þá nam'k \alst{f}rævask \hld ok \alst{f}róðr vesa &
\ind ok \alst{v}axa ok \alst{v}ęl hafask; &
\alst{o}rð mér af \alst{o}rði \hld \alst{o}rðs lęitaði &
\ind \alst{v}erk mér af \alst{v}erki \alst{v}erks.\eva

\bvb Then I began to thrive, and be learned, and grow and have it well. A word for me of a word a word sought out; a work for me of a work a work\footnoteB{Each good word and deed was followed by another.}.\evb
\evg


\bvg
\bva \alst{R}únar munt finna \hld ok \alst{r}áðna stafi, &
\ind mjǫk \alst{st}óra \alst{st}afi, &
\ind mjǫk \alst{st}inna \alst{st}afi, &
\ind es \alst{f}áði \alst{f}imbulþulr &
\ind ok \alst{g}ęrðu \alst{g}innręgin &
\ind ok \alst{r}ęist Hroptr \alst{r}agna\footnotemark[5]. —\eva
\footnotetext[5]{Corrected from \emph{rǫgna}. Cf. Eskál \emph{Vell} 31/2 in SkP I, p. 322.}

\bvb \textbf{Runes} thou wilt find, and interpreted staves; much large staves, much stiff staves, which the \textbf{fimbol-thyle} painted, and the \textbf{ginn}-\textbf{Powers} made, and the \textbf{Roft} of the Powers carved.\evb
\evg


\bvg
\bva \alst{Ó}ðinn með \alst{ǫ́}sum, \hld ęn fyr \alst{ǫ}lfum Dáinn, &
\ind \alst{D}valinn \alst{d}vergum fyrir, &
\ind \alst{Á}sviðr \alst{jǫ}tnum fyrir, &
ek\footnotemark[10] ręist \alst{s}jalfr \alst{s}umar. —\eva
\footnotetext[10]{Since Woden has already been mentioned, it is unclear who the \emph{I} here may be.}

\bvb \textbf{Woden} among the \textbf{Ease}, but before the \textbf{Elves} \textbf{Dowen}, \textbf{Dwollen} before the \textbf{Dwarves}, \textbf{Aswood} before the etins; I myself carved some.\evb
\evg


\bvg
\bva Vęiztu, hvé \alst{r}ísta skal? \hld vęiztu, hvé \alst{r}áða skal? &
vęiztu, hvé \alst{f}áa skal? \hld vęiztu, hvé \alst{f}ręista skal? &
vęiztu, hvé \alst{b}iðja skal? \hld vęiztu, hvé \alst{b}lóta skal? &
vęiztu, hvé \alst{s}ęnda skal? \hld vęiztu, hvé \alst{s}óa skal? —\footnotemark[5]\eva
\footnotetext[5]{This v. bears strong resemblance with Vg 216 (Högstena gealdor). TODO: Elaborate.}

\bvb Knowest thou how to carve? Knowest thou how to interpret? Knowest thou how to paint? Knowest thou how to try? Knowest thou how to bid? Knowest thou how to bloot? Knowest thou how to send? Knowest thou how to \textbf{soo}?\evb
\evg


\bvg
\bva \alst{B}ętra's ó\alst{b}eðit \hld an sé of\alst{b}lótit, &
\ind ęy sér til \alst{g}ildis \alst{g}jǫf; &
bętra's ó\alst{s}ęnt \hld an sé of\alst{s}óit.\footnotemark[6]\eva
\footnotetext[6]{A final line is likely missing here. — Identical word-pairing (\emph{biðja} – \emph{blóta}, \emph{senda} – \emph{sóa}) may reveal this v.'s relation with the previous one.}

\bvb Better is unbid than be excessively blooted; a gift always looks to a tribute. Better is unsent than be excessively sooed.\evb
\evg


\bvg
\bva Svá \alst{Þ}undr of ręist \hld fyr \alst{þ}jóða rǫk &
þar's \alst{u}pp of ręis, \hld es \alst{a}ptr of kom.\eva

\bvb Thus \textbf{Thound} did carve for the fate of the nations; there did rise up, who came back.\footnotemark[8]\evb
\footnotetext[8]{A most cryptic v.}
\evg


\bvg Woden's recounting of his Songs.
\bva Ljóð \alst{þ}au kan'k, \hld es kann-at \alst{þ}jóðans kona &
\ind ok \alst{m}anskis \alst{m}ǫgr. &
\alst{H}jǫlp hęitir ęitt, \hld þat þér \alst{h}jalpa mun &
\ind við \alst{s}orgum ok \alst{s}ǫkum, \hld ok \alst{s}útum gǫrvǫllum.\eva

\bvb I know those \textbf{leeds}, which the prince's woman does not know, and no man's lad. Help one is called, it will help thee, against sorrows and sakes\footnotemark[9], and all kinds of misfortunes.\evb
\footnotetext[9]{Legal proceedings.}
\evg


\bvg
\bva Þat kan'k \alst{a}nnat, \hld es þurfu \alst{ý}ta synir,\footnotemark[10] &
\ind þęir's vilja \alst{l}æknar \alst{l}ifa.\eva
\footnotetext[10]{Identical wording to 163/2.}

\bvb 145\evb
\evg


\bvg
\bva \alst{Þ}at kan'k \alst{þ}riðja, \hld ef mér verðr \alst{þ}ǫrf mikil &
\ind \alst{h}apts við mina \alst{h}ęiptmǫgu, &
\alst{ę}ggjar dęyfi'k \hld minna \alst{a}ndskota, &
\ind bítat þęim \alst{v}ǫ́pn né \alst{v}ęlir.\eva

\bvb 146\evb
\evg


\bvg
\bva Þat kan'k \alst{f}jórða, \hld ef mér \alst{f}yrðar bera &
\ind \alst{b}ǫnd at \alst{b}oglimum, &
svá ek \alst{g}ęl, \hld at \alst{g}anga má'k, &
\ind sprettr mér af \alst{f}ótum \alst{f}jǫturr. &
\ind ęn af \alst{h}ǫndum \alst{h}apt.\eva

\bvb 147\evb
\evg


\bvg
\bva Þat kan'k \alst{f}imta, \hld ef sé'k af \alst{f}ári skotinn &
\ind \alst{f}lęin í \alst{f}olki vaða, &
flýgr-a svá \alst{st}int, \hld at \alst{st}ǫðvigak, &
\ind ef hann \alst{s}jónum of \alst{s}é'k.\eva

\bvb 148\evb
\evg


\bvg
\bva Þat kan'k \alst{s}étta, \hld ef mik \alst{s}ærir þegn &
\ind á \alst{v}rótum hrás \alst{v}iðar. &
þann \alst{h}al, \hld es mik \alst{h}ęipta kvęðr, &
\ind þann eta \alst{m}ęin hęldr an \alst{m}ik.\eva

\bvb 149\evb
\evg


\bvg
\bva Þat kan'k \alst{s}jaunda, \hld ef \alst{s}é'k hóvan loga &
\ind \alst{s}al of \alst{s}essmǫgum, &
\alst{b}rinnrat svá \alst{b}ręitt, \hld at hǫ́num \alst{b}jargigak; &
\ind þann kan'k \alst{g}aldr at \alst{g}ala.\eva

\bvb 150\evb
\evg


\bvg
\bva Þat kan'k \alst{á}tta, \hld es \alst{ǫ}llum es &
\ind \alst{n}ytsamligt at \alst{n}ema, &
\alst{h}var's \alst{h}atr vęx \hld með \alst{h}ildings sonum, &
\ind þat má'k \alst{b}ǿta \alst{b}rátt.\eva

\bvb 151\evb
\evg


\bvg
\bva Þat kan'k \alst{n}íunda, \hld ef mik \alst{n}auðr of stęndr &
\ind at bjarga \alst{f}ari á \alst{f}loti, &
\alst{v}ind ek kyrri \hld \alst{v}ági á &
\ind ok \alst{s}væfi'k allan \alst{s}æ.\eva

\bvb 152\evb
\evg


\bvg
\bva Þat kan'k \alst{t}íunda, \hld ef sé'k \alst{t}únriður &
\ind \alst{l}ęika \alst{l}opti á, &
ek svá \alst{v}in'k, \hld at þær \alst{v}illar fara &
\ind sinna \alst{h}ęim-\alst{h}ama &
\ind sinna \alst{h}ęim-\alst{h}uga.\eva

\bvb 153\evb
\evg


\bvg
\bva Þat kan'k \alst{ę}llipta, \hld ef skal'k til \alst{o}rrostu &
\ind \alst{l}ęiða \alst{l}angvini, &
und \alst{r}andir gęlk, \hld ęn þęir með \alst{r}íki fara, &
\ind \alst{h}ęilir \alst{h}ildar til, &
\ind \alst{h}ęilir \alst{h}ildi frá, &
\ind koma þęir \alst{h}ęilir \alst{h}vaðan.\eva

\bvb 154\evb
\evg


\bvg
\bva Þat kan'k \alst{t}olpta, \hld ef sé'k á \alst{t}ré uppi &
\ind \alst{v}áfa \alst{v}irgilná, &
svá ek \alst{r}íst \hld ok í \alst{r}únum fá'k, &
\ind at sá \alst{g}ęngr \alst{g}umi. &
\ind ok \alst{m}ælir við \alst{m}ik.\eva

\bvb 155\evb
\evg


\bvg
\bva \alst{Þ}at kan'k \alst{þ}rettánda \hld ef skal'k \alst{þ}egn ungan &
\ind \alst{v}erpa \alst{v}atni á,\footnotemark[49] &
munat hann \alst{f}alla, \hld þótt í \alst{f}olk komi, &
\ind \alst{h}nígr-a sá \alst{h}alr fyr \alst{h}jǫrum.\eva
\footnotetext[49]{Describing the pagan ritual of pouring water on a newborn child. Cf. \emph{Rþ} 7/1b, 21/1b, 34/2a.}

\bvb 156\evb
\evg


\bvg
\bva Þat kan'k \alst{f}jogurtánda, \hld ef skal'k \alst{f}yrða liði &
\ind \alst{t}ęlja \alst{t}íva fyrir, &
\alst{á}sa ok \alst{a}lfa \hld ek kann \alst{a}llra skil, &
\ind fár kann ó\alst{s}notr \alst{s}vá.\eva

\bvb 157\evb
\evg


\bvg
\bva \alst{Þ}at kan'k fimtánda, \hld es gól {Þ}jóðrørir &
\ind \alst{d}vergr fyr \alst{D}ęllings \alst{d}urum, &
\alst{a}fl gól \alst{ǫ́}sum, \hld ęn \alst{ǫ}lfum frama, &
\ind \alst{h}yggju \alst{H}roptatý.\eva

\bvb 158\evb
\evg


\bvg
\bva Þat kan'k sextánda, \hld ef vil'k hins svinna mans &
\ind hafa gęð alt ok gaman, &
\alst{h}ugi \alst{h}vęrfi'k \hld \alst{h}vitarmri konu &
\ind ok \alst{s}ný'k hęnnar ǫllum \alst{s}efa.\eva

\bvb 159\evb
\evg


\bvg
\bva Þat kan'k \alst{s}jautjánda \hld at mik \alst{s}ęint mun firrask &
\ind hit \alst{m}anunga \alst{m}an.\eva

\bvb 160\evb
\evg


\bvg
\bva Þat kan'k \alst{á}tjánda, \hld es \alst{æ}va kęnni'k &
\ind \alst{m}ęy né \alst{m}anns konu, &
\alst{a}lt es bętra \hld es \alst{ęi}nn of kann, &
\ind þat fylgir \alst{l}jóða \alst{l}okum, &
nema þęiri \alst{ęi}nni, \hld es mik \alst{a}rmi vęrr, &
\ind eða mín \alst{s}ystir \alst{s}é.\eva

\bvb 161\evb
\evg


\bvg
\bva Nú eru \alst{H}áva mál \hld kveðin \alst{H}áva\alst{h}ǫllu í &
\ind \alst{a}llþǫrf \alst{ý}ta sonum, &
\ind \alst{ó}þǫrf \alst{jǫ}tna sonum; &
hęill sás kvað, \hld hęill sás kann, &
\ind \alst{n}jóti sás \alst{n}am, &
\ind \alst{h}ęilir þęirs \alst{h}lýddu.\footnotemark[50]\eva
\footnotetext[50]{A highly metrically irregular verse; clearly younger than the rest of the poem collection. See the Introduction, p. TODO.}

\bvb 162\evb
\evg
% — Weeden
%	\include{books/04 From the Sons of king Reeding.tex}% — Weeden
%	\include{books/05 Speeches of Grimner.tex}% — Weeden
%	Þórr fór ór austrvegi ok kom at sundi einu. Ǫðrum megum sundsins var ferjukarlinn með skipit. Þórr kallaði:

Thunder travelled out of the eastern ways and came to a sound. At the other side of the sound was the ferryman with the ship. Thunder called out:

“Hvęrr ’s sá svęinn svęina \hld es stęndr fyr sundit handan?”

“Who is that swain of swains, that stands across the sound?”

Hann svaraði:
“Hvęrr ’s sá karl karla \hld es kallar of váginn?”

He answered:
“Who is that churl of churls, that calls out over the wave?”

Fęr þú mik of sundit, \hld fǿði’k þik á morgun;
męis hęfi’k á baki, \hld verðr-a matrinn bętri. 

If thou take me over the sound, I feed thee in the morning; a basket I have on my back, food does not get better [than that].

Át ek í hvíld \hld áðr ek hęiman fór,
síldr ok hafra; \hld saðr em’k ęnn þęss.

I ate in rest, before I travelled from home, herring and hegoats; I am still full from that.

Árligum verkum
hrósar þú verðinum;
veizt at u fyr görla,
döpr eru þín heimkynni,
dauð hygg ek að þín móðir sé. 

...

Þórr kvað:
“Skammt mun nú mál okkat vesa, \hld allz þú mér skǿtingu ęinni svarar;
launa mun ek þér farsynjun \hld ef vit finnumk í sinn annat!
Farþú nú þar’s þik hafi allan gramir!”

Thunder quoth:
“Now our speech will be short, as thou but answers me with taunts; I will reward thee for this ferry-refusal, if we meet another time! Now go, whither the fiends may have all of thee!”
% — Weeden + Thunder
%	\bookStart{The Lay of Hymer}[Hymiskviða]

% Introduction.
Attested in two manuscripts, \Regius\ and \AM. The two are surprisingly consistent.

Þórr dró Miðgarðsorm. % TODO: Format as header.

Thunder pulled up the Middenyardsworm.\footnotetext{This is the only title the poem has in \Regius. \AM\ has the proper title \emph{Hymiskviða} instead.}


\bvg
\bva Ár valtívar \hld\ vęiðar nǫ́mu &
ok sumblsamir \hld\ áðr saðir yrði, &
hristu tęina \hld\ ok á hlaut sǫ́u, &
fundu þęir at Ę́gis \hld\ ørkost hvera.\eva

\bvb Of yore the slaughter-Tues had caught game\footnoteB{Lit. ‘took game’}, and banqueting before they might eat\footnoteB{Lit. ‘might become sated’}, they shook the twigs and looked at the \inx{leat}; they found at Eagre’s a great choice of cauldrons.\footnoteB{The gods sprinkled the leat (sacrificial blood) of the beasts and interpreted the pattern; they found it most auspicious to feast at Eagre’s.}\evb
\evg


\bvg
\bva Sat bergbúi \hld\ barntęitr fyrir, &
mjǫk glíkr męgi \hld\ Miskorblinda, &
lęit í augu \hld\ Yggs barn í þrá: &
„þú skalt ǫ́sum \hld\ opt sumbl \edtext{gęra}{\lemma{gęra “host”}\Afootnote{gefa “give” \AM}}!“\eva

\bvb — Sat the mountain-dweller \ken*{= Eagre} there, joyous like a child, much like the lad of Misherblind\footnoteB{A reference to a lost myth? Unless Misherblind is an alternative name for Firneet, Eagre’s father.}; into his eyes looked the child of Ug \name{= Weden} \ken*{= Thunder} in defiance: “Thou shalt for the Ease oft’ host banquets!”\footnoteB{Having seen that Eagre has a great store of cauldrons, Thunder orders him to host future banquets for the Ease.}\evb
\evg


\bvg
\bva Ǫnn fekk jǫtni \hld\ orðbę́ginn halr, &
hugði at hefndum \hld\ hann nę́st við goð, &
bað hann Sifjar ver \hld\ sér fǿra hver, &
„þann’s ek ǫllum ǫl \hld\ yðr of hęita.“\eva

\bvb Great toil for the ettin the word-peevish man \ken*{= Thunder} caused; thought he of revenge, soon, against the god: asked he Sib’s husband to bring him a cauldron, “that one with which I for you all ale might brew.\footnoteB{Eagre asks Thunder to find a single cauldron which can hold enough ale to supply all the Ease.}”\evb
\evg


\bvg
\bva Né þat mǫ́ttu \hld\ mę́rir tívar &
ok ginnręgin \hld\ of geta hvęrgi, &
unz af tryggðum \hld\ Týr Hlórriða &
ástráð mikit \hld\ ęinum sagði:\eva

\bvb But that might the renowned Tues and the \inx{Gin-Reins} nowhere get ahold of, until out of loyalty, a great word of loving advice Tue to Loride \name{= Thunder} alone did say:\evb
\evg


\bvg
\bva „Býr fyr austan \hld\ Élivága &
hundvíss Hymir \hld\ at himins ęnda, &
á minn faðir \hld\ móðugr kętil, &
\edtext{rúmbrugðinn}{\Afootnote{‘rumbrygðan’ \AM}} hver \hld\ rastar djúpan.“\eva

\bvb “Lives to the east of the Ilewaves the houndwise Hymer, at the end of heaven. Owns my father\footnoteB{Hymer being Tue’s father.}, fierce, a kettle; a size-renowned cauldron one \inx{rest} deep.”\evb
\evg


\bvg
\bva „Veiztu, ef þiggjum \hld\ þann lǫgvelli?“ &
„Ef, vinr, vélar \hld\ vit gørvum til!“\eva

\bvb “Knowest thou if we will receive that ale-boiler?” — “If, friend, we two make use of wiles!”\footnoteB{The speakers are not indicated, but it is most sensible that Thunder asks and Tue answers.}\evb
\evg

\bvg
\bva Fóru drjúgum \hld\ \edtext{dag þann framan}{\lemma{dag þann framan “from the beginning of the day”}\Afootnote{\emph{Emendation from Finnur 1932}; dag þann fram “on that day forth” \Regius; dag fráliga “swiftly at day” \AM}} &
Ásgarði frá \hld\ unz til \edtext{Ęgils}{\lemma{Ęgils “Agle’s”}\Afootnote{\emph{thus} \Regius; Ę́gis “Eagre’s” \AM; — \AM\ \emph{reading possibly from confusion with Eagre described earlier in the poem, but or the shepherd did share his name.}}} kvǫ́mu. &
Hirði hann hafra \hld\ horngǫfgasta; &
hurfu at hǫllu \hld\ es Hymir átti.\eva

\bvb — They travelled with great strides from the beginning of the day, from Osyard, until to Agle’s they came—he herded bucks with the noblest of horns—they turned to the hall which Hymer owned.\evb
\evg


\bvg
\bva Mǫgr fann ǫmmu, \hld\ mjǫk lęiða sér, &
hafði hǫfða \hld\ hundruð níu. &
ęn ǫnnur gekk \hld\ algollin framm &
brúnhvít bera \hld\ bjórvęig syni.\eva

\bvb The lad found his grandmother greatly loathsome; she had of heads nine hundred. But another woman, all-golden, stepped forth: white-browed, she carried a beer-draught for the son \ken*{= Tue}.\evb
\evg


\bvg
\bva „Áttniðr jǫtna \hld\ ek vilja’k ykr &
hugfulla tvá \hld\ und hvera sętja; &
es mínn \edtext{fríi}{\lemma{fríi “lover”}\Afootnote{\emph{thus} \Regius; faðir “father” \AM}} \hld\ mǫrgu sinni &
gløggr við gęsti \hld\ gǫrr ills hugar.“\eva

\bvb “Kinsman of ettins! I would wish to set you high-mettled two under the cauldrons; my lover has many a time been stingy against guests, quick to ill temper.”\footnoteB{Tue’s mother (the all-golden woman in previous v.) wishes to hide him and Thunder, lest her husband (Hymer) find them.}\evb
\evg


\bvg
\bva Ęn váskapaðr \hld\ varð \edtext{síðbúinn}{\Afootnote{\emph{om.} \AM}}, &
harðráðr Hymir, \hld\ hęim af vęiðum; &
gekk inn í sal, \hld\ glumðu jǫklar, &
vas karls, es kom, \hld\ kinnskógr frørinn.\eva

\bvb But the misshapen one was come late—the hard-minded Hymer—home from the hunt. He entered the hall—icicles clattered—frozen was the cheek-forest \ken{beard} of the churl who came.\evb
\evg


\bvg
\bva „Ves þú hęill, Hymir, \hld\ í hugum góðum! &
Nú ’s sonr kominn \hld\ til sala þinna, &
sá’s vit vę́ttum \hld\ af vęgi lǫngum; &
fylgir hǫ́num \hld\ Hróðrs andskoti, &
vinr verliða; \hld\ Véurr hęitir sá.\eva

\bvb “Be thou hale, Hymer, in good spirits!\footnoteB{Formula identically mirrored in runic inscription N B380: \emph{Heill sé þú / ok í hugum góðum. / Þórr þik þiggi, / Óðinn þik eigi.} “May thou be hale, and in good spirits! May Thunder receive thee, may Weden own thee.” Cf. also \Beowulf\ l. 407: \emph{Wæs þú Hróðgár hál!} “Be thou, Rothgar, hale!”} Now the son is come to thy halls, the one whom we two have been expecting, from a long way off. Follows him the opponent of Rooder <ettin> \ken*{= Thunder}, the friend of manly retinues \ken*{= Thunder}; Wighward \name{= Thunder} he is called.\evb
\evg


\bvg
\bva Sé þú hvar sitja \hld\ und salar gafli, &
svá \edtext{forða sér}{\Afootnote{forðask \AM}}, \hld\ stęndr \edtext{súl}{\Afootnote{‘sol’ \AM}} fyrir.“ &
Sundr stǫkk súla \hld\ fyr sjón jǫtuns, &
ęn \edtext{allr}{\Afootnote{áðr \Regius\AM TODO: elaborate, mention Finnur}} í tvau \hld\ áss brotnaði.\eva

\bvb See where they sit, ’neath the hall’s gable: thus they hide themselves—a pillar stands before them!” The pillars sprang asunder before the sight of the ettin, but all in two the beam was broken.\evb
\evg


\bvg
\bva Stukku átta, \hld\ ęn ęinn af þęim &
hverr harðslęginn \hld\ hęill af þolli; &
framm gingu þęir, \hld\ ęn forn jǫtunn &
sjónum lęiddi \hld\ sinn andskota.\eva

\bvb Eight\footnoteB{Eight kettles.} sprung apart, but one of them, a hard-forged kettle, [came] whole off its peg\footnoteB{Presumably the one in which Tue and Thunder were hiding.}. Forth went they, but the ancient ettin with his sight beheld\footnoteB{Literally “led with his sight”.} his opponent.\evb
\evg


\bvg
\bva Sagðit hǫ́num \hld\ hugr vęl þá’s sá &
gýgjar \edtext{grǿti}{\lemma{grǿti “distresser”}\Afootnote{gę́ti “keeper, warder” \AM}} \hld\ á golf kominn, &
þar vǫ́ru þjórar \hld\ þrír of tęknir, &
bað \edtext{sęnn}{\Afootnote{‘sun’ \AM}} jǫtunn \hld\ sjóða ganga.\eva

\bvb His heart was not pleased then, when he saw the distresser of troll-women \ken*{= Thunder} come on the floor. There were three bulls taken: the ettin at once bade them be cooked.\evb
\evg


\bvg
\bva Hvęrn létu þęir \hld\ hǫfði skęmra &
ok á sęyði \hld\ síðan bǫ́ru, &
át Sifjar verr \hld\ áðr sofa gingi, &
ęinn með ǫllu \hld\ øxn tvá Hymis.\eva

\bvb Each one they let shorten by a head, and onto the fire-pit then carried: ate the husband of Sib \ken*{= Thunder}, before he might go to sleep, alone all together two of Hymer’s oxen.\evb
\evg


\bvg
\bva Þótti hǫ́rum \hld\ Hrungnis spjalla &
verðr Hlórriða \hld\ vęl fullmikill, &
„munum at aptni \hld\ ǫðrum verða &
við vęiðimat \hld\ vér þrír lifa.“\eva

\bvb To the hoary friend of Rungner \name{ettin} \ken*{= Hymer} seemed Loride’s meal far too large; “next evening will we three by game-meat have to live.\footnoteB{Hymer’s stinginess (he refuses to share more of his own food, forcing his guests to go hunt) illustrates the otherness of the Ettins. See introduction to the poem.}”\evb
\evg


\bvg
\bva Véurr kvaðzk vilja \hld\ á vág róa, &
ef ballr jǫtunn \hld\ bęitur gę́fi. &
„Hverf þú til \edtext{hjarðar}{\Afootnote{hallar \emph{(corr.)} \AM}}, \hld\ ef hug trúir, &
brjótr berg-Dana, \hld\ bęitur sǿkja.\eva

\bvb Wighward \name{= Thunder} called himself willing to row on the wave, if the baleful ettin might give pieces of bait. “Turn to the herd, if thou trust in thy heart—breaker of boulder-Danes \ken*{\textsc{ettins} > = Thunder}!—to seek pieces of bait.\evb
\evg


\bvg
\bva Þess \edtext{vę́ntir mik}{\Afootnote{vę́nti ek \Regius}}, \hld\ at þér \edtext{mynit}{\lemma{mynit “will not”}\Afootnote{myni ”will” \Regius}} &
ǫgn at oxa \hld\ auðfeng vesa.“ &
Svęinn sýsliga \hld\ svęif til skógar, &
þar’s oxi stóð \hld\ alsvartr fyrir.\eva

\bvb I expect that the oxen for bait will not be easily caught by thee.”—The swain \name{= Thunder} sharply turned to the woods, there where an ox stood, all-black, before [him].\evb
\evg


\bvg
\bva Braut af þjóri \hld\ þurs ráðbani &
hǫ́tún ofan \hld\ horna tveggja. &
„Verk þikkja þín \hld\ verri myklu &
kjóla valdi \hld\ an kyrr sitir.“\eva

\bvb From the bull broke the treacherous slayer of the thurse \ken*{= Thunder} off the high meadow of the two horns \ken{head}, from above.—“Thy works seem far worse to the wielder of keels \ken*{= Hymer = me}, than if thou didst sit calm.\footnoteB{Hymer snidely belittles Thunder’s deed of pulling off the head of the ox (presumably by the horns).}”\evb
\evg

\bvg
\bva Bað hlunngota \hld\ hafra dróttinn &
\edtext{áttrunn}{\Afootnote{‘atrænn’ \AM}} apa \hld\ útar fǿra, &
ęn sá jǫtunn \hld\ sína \edtext{talði}{\Afootnote{‘milldi’ \emph{(corr.)} \AM}}, &
lítla fýsi \hld\ lęngra at róa.\eva

\bvb The lord of he-goats \ken*{= Thunder} bade the kinsman of the \inx{ape}[C]\footnoteB{The specific meaning of \emph{api} (here rendered as “ape”) is highly uncertain. It seems to generally refer to a fool, but see Index.}\ [\textsc] to push the launching-steed \ken{ship} further out; but that ettin told of his scarce wish to row any longer.\footnoteB{The parallelism is notable, as Hymer, who just mocked Thunder, is now forced to do his willing by rowing.}\evb
\evg


\bvg
\bva Dró \edtext{mę́rr}{\Afootnote{\emph{thus} \Regius; ‘mæirr’ \AM}} Hymir \hld\ móðugr hvala &
ęinn á ǫngli \hld\ upp sęnn tváa, &
ęn aptr í skut \hld\ Óðni sifjaðr &
Véurr við vélar \hld\ vað gęrði sér.\eva

\bvb Angered, the renowned Hymer pulled whales: one on the hook, soon up two; but back in the stern the one related to Weden \ken*{= Thunder}, Wighward \name{= Thunder}, cleverly made himself a fishing-line.\evb
\evg


\bvg
\bva Ęgnði á ǫngul \hld\ sá’s ǫldum bergr, &
orms ęinbani \hld\ oxa hǫfði; &
gęin við \edtext{agni}{\lemma{agni “bait”}\Bfootnote{\emph{thus} \AM; ǫngli ‘hook’ \Regius}}, \hld\ sú’s goð fía, &
umbgjǫrð neðan \hld\ allra landa.\eva

\bvb On the hook fastened he who saves men \ken*{= Thunder}—the lone slayer of the Worm \ken*{= Thunder}—the head of the ox. At the bait snapped the one whom the gods hate \ken*{= Middenyardsworm}—the surrounder of all lands \ken*{= Middenyardsworm}—from below.\evb
\evg


\bvg
\bva Dró djarfliga \hld\ dáðrakkr Þórr &
orm ęitrfáan \hld\ upp at borði; &
hamri kníði \hld\ hǫ́fjall skarar &
ofljótt ofan \hld\ ulfs hnitbróður.\eva

\bvb Daringly pulled deed-bold Thunder the venom-glistening Worm up on the gunwale; with the hammer he struck the high mountain of hair\footnoteB{A rather unfitting kenning, since serpent do not have hair.} \ken{head}—greatly hideous, from above—on the clash-brother of the Wolf \ken*{= Middenyardsworm}.\evb
\evg


\bvg
\bva \edtext{Hraungǫlkn}{\lemma{hraungǫlkn}\Afootnote{hręingǫlkn \Regius\AM}} \edtext{hrutu}{\Afootnote{\emph{thus} \AM; hlumðu \Regius}, \hld\ ęn hǫlkn þutu, &
fór hin forna \hld\ fold ǫll saman; &
søkkðisk síðan \hld\ sá fiskr í mar.\eva

\bvb The wilderness-monsters bounded, but the bedrock resounded; moved the ancient earth all at once; sank thereafter that fish \ken*{= Middenyardsworm} into the sea.\evb
\evg


\bvg
\bva Ótęitr jǫtunn, \hld\ es aptr røru, &
\skipnumbering\edtext{[...]}{\Bfootnote{There is without doubt a line missing here, the grammar and sense require it.}}
svá’t \edtext{ár}{\lemma{ár “in the early morning”}\Afootnote{\Finnur\ \emph{suggests}} svá’t at ǫ́r “so that by the oar.”} Hymir \hld\ ękki mælti, &
vęifði rǿði \hld\ veðrs annars til.\eva

\bvb The not joyous ettin, as they rowed back, [...], so that in the early morning\footnoteB{Assuming this is the correct reading, it would seem like the fishers have spent the whole night at sea, presumably with Hymer rowing all the way.} Hymer spoke nothing; he pulled the oar around, against the storm.\evb
\evg


\bvg
\bva „Mundu of vinna \hld\ verk halft við mik,
at hęim hvala \hld\ haf til bǿjar
eða flotbrúsa \hld\ fęstir okkarn.“\eva

\bvb \evb
\evg


\bvg
\bva Gekk Hlórriði,\eva

\bvb \evb
\evg


\bvg
\bva Ok ęnn jǫtunn\eva

\bvb \evb
\evg


\bvg
\bva Ęn Hlórriði,\eva

\bvb \evb
\evg


\bvg
\bva Unz þat hin fríða\eva

\bvb \evb
\evg


\bvg
\bva Harðr ręis á kné\eva

\bvb \evb
\evg


\bvg
\bva »Mǫrg vęitk mæti\eva

\bvb \evb
\evg


\bvg
\bva Þat ’s til kostar,\eva

\bvb \evb
\evg


\bvg
\bva Faðir Móða\eva

\bvb \evb
\evg


\bvg
\bva Fórut lęngi,\eva

\bvb \evb
\evg


\bvg
\bva Hóf sér af hęrðum\eva

\bvb \evb
\evg


\bvg
\bva Fórut lęngi,\eva

\bvb \evb
\evg


\bvg
\bva Ęn ér hęyrt hafið,\eva

\bvb \evb
\evg


\bvg
\bva Þróttǫflugr kom\eva

\bvb \evb
\evg
% — Thunder
%	Vręiðr vas þá Ving-Þórr \hld es hann vaknaði
ok síns hamars \hld of saknaði,
skegg nam at hrista, \hld skǫr nam at dýja,
réð Jarðar burr \hld um at þreifask.

Wroth was then Wing-Thunder when he woke, and of his hammer was bereaved; his beard he began to shake, his locks he began to pull; the son of Earth decided to look around.

Ok hann þat orða \hld alls fyrst of kvað:
“Hęyrðu nú, Loki, \hld hvat ek nú mæli
es ęigi vęit \hld jarðar hvęrgi
né upphimins: \hld áss es stolinn hamri!”

And he that word, first of all did say: “Hear thou now, Lock, what I now speak, which nowhere is known, not of earth nor up-heaven: the os has been robbed of his hammer!”

Gengu þęir fagra \hld Fręyju túna
ok hann þat orða \hld allz fyrst of kvað:
“Muntu mér, Fręyja, \hld fjaðrhams ljá
ef ek mínn hamar \hld mætta’k hitta?”

Went they to the fair towns† of Frow, and he that word, first of all did say: “Wilt thou me, Frow, the feather-hame† lend, if I my hammer might find?”

Fręyja kvað:
“Þó mynda’k gefa þér \hld þótt ór gulli væri
ok þó sęlja \hld at væri ór silfri.”
Fló þá Loki, \hld fjaðrhamr dunði,
unz fyr útan kom \hld ása garða
ok fyr innan kom \hld jǫtna hęima. 

Frow quoth:
“I would yet give it to thee, though it were out of gold, and yet offer\footnotemark[1] it to thee, if it were out of silver.”\footnotemark[2] Then Lock flew — the feather-hame rustled — until outside he came of the yards of the Ease, and inside he came of the homes of the ettins.
\footnotetext[1]{\emph{sęlja} ‘sell’ here has its earlier meaning, cf. Gothic \emph{saljan} ‘\emph{opfern}; θύειν’ (Streitberg 1910, p. 116).}
\footnotetext[2]{Regaining the hammer is of such importance to the gods (cf. v. 17; without it the Ease stand powerless against an ettin invasion) that Frow would lend the feather-hame to the greedy and unreliable Lock, even if it were made out of solid gold or silver.}

Þrymr sat á haugi, \hld þursa dróttinn,
gręyjum sínum \hld gullbǫnd snøri
ok mǫrum sínum \hld mǫn jafnaði. 

Thrim sat on the high†, the lord of thurses; on his greyhounds the golden leashes he twisted, and on his mares the manes he cut even.
% — Thunder
%	“Vaki mær męyja, \hld vaki mín vina,
Hyndla systir, \hld es í hęlli býr;
nú ’s røkr røkra, \hld ríða vit skulum
til Valhallar \hld ok til vés hęilags.

Frow quoth:
“Wake maiden of maidens, wake my friend, sister Hindle, who lives in the rock-face. Now is the twilight of twilights, we two shall ride to Walhall, and to the holy wigh†!

Biðjum Hęrjafǫðr \hld í hugum sitja,
hann gęldr ok gefr \hld gull verðugum\footnotemark[1],
gaf hann Hęrmóði \hld hjalm ok brynju,
ęn Sigmundi \hld sverð at þiggja.
\footnotetext[1] Emended by Finnur and Guðni to \emph{verðungu}, 'to the retinue'.

Let us bid the Father of Hosts <= Weden> to be in his favour; he rewards and gives gold to the worthy. He gave Heremood helmet and byrnie, but Sighmund a sword to receive.

Gefr hann sigr sumum\footnotetext[1], \hld ęn sumum\footnotetext[2] aura,
mælsku mǫrgum \hld ok manvit firum,
byri gefr brǫgnum, \hld ęn brag skǫldum,
gefr hann mannsęmi \hld mǫrgum rekki.
\footnotetext[1] ms. \emph{sonum}
\footnotetext[2] ms. \emph{suinnum}

He gives victory to some, but to some silver\footnotemark[1]; speech to many, and manwit to men. Fair wind he gives to noble ones, and bray [POETRY] to scolds†; he gives valour to many a champion.
\footnotemark[1] Lit. "ounces".

Þór munk blóta, \hld þess munk biðja,
at hann æ við þik \hld einart láti;
þó ’s hǫ́num ótítt \hld við jǫtuns brúðir.

To Thunder I will bloot†, and bid for this, that he always might show friendliness to thee, although he is prejudiced against the brides of the ettins\footnotemark[1].
\footnotetext[1] Lit. “though [it] is to him infrequent with ettin's brides”.

Nú taktu ulf þinn \hld ęinn af stalli,
lát hann rinna \hld með runa mínum.”
[Hyndla kvað:] Sęinn es gǫltr þinn \hld goðveg troða,
vilkat mar minn \hld mætan hlǿða.

Now take thy single wolf from the stable; let him run with my boar.” [Hindle quoth:] “Slow is thy boar, to tread the Godways; I wish not lade my dear steed.”

Flǫ́ est Fręyja, \hld es fręistar mín,
visar þú augum \hld á oss þannig,
es hafir ver þinn \hld í valsinni
Óttar unga \hld Innsteins bur.”

Thou art deicitful, Frow, as thou tempts me; thou shows thy eyes on us this way, since thou hast thy man on the Walways: the young Oughthere, Instone's offspring.”

[Fręyja kvað:]
“Dulið est Hyndla, \hld draums ætlak þér,
es kveðr ver minn \hld í valsinni.

Frow quoth:
Thou art foolish, Hindle, I think thee dreamy, saying my man [is] on the Walways.

Þar’s gǫltr glóar \hld Gullinbursti,
Hildisvíni, \hld es mér hagir gęrðu,
dvergar tvęir \hld Dáinn ok Nabbi.

There the boar glows, Goldenbristle, the hildswine\footnotemark[1], which the skillful for me made: the two dwarves Dowen and Nab.
\footnotemark[1] \emph{Hildisvíni} 'battle-swine', in this case probably an alternative name for Goldenbristle.

Sęnn í sǫðlum \hld sitja vit skulum
ok of jǫfra \hld ættir dǿma,
(gumna þęira, \hld es frá goðum kómu).

Soon in the saddles we two shall sit, and judge about the aughts† of princes, of those men who came from the gods.

Þęir hafa vęðjat \hld Vala malmi
Óttarr ungi \hld ok Angantýr;
skylt ’s at vęita, \hld svát skati hinn ungi
fǫðurlęifð hafi \hld ępt frændr sína.

They have wagered on the Welsh ore [GOLD], Oughthere and Ongenthew; it is required to grant, so that the young prince might have the fatherly inheritance left behind by his kinsmen.\footnotemark[1]
\footnotemark[1] Lit. 'the father-remains after his kinsmen'. — Happening seems to be that Oughthere and Ongenthew each lay claim the inheritance. In order to settle the matter (in Oughthere's favour) Hindle must (\emph{skylt es} “it is required, obligated”) divulge (\emph{vęita} ‘to grant, to give away’) what she knows about his lineage.

Hǫrg hann mér gęrði \hld hlaðinn stęinum;
nú ’s grjót þat \hld at glęri orðit;
rauð hann í nýju \hld nauta blóði;
æ trúði Óttarr \hld á ǫ́synjur.\footnotemark[1]
\footnotemark[1] Frow argues yet further in favour of Oughthere, bringing up his piety shown towards the godesses.

A harrow† he made for me, loaded with stones; now that stone-pile is become into glass. He reddened [it] in fresh blood of oxen; Oughthere ever trusted on the osennies†.

Nú lát-tu forna \hld niðja talða
ok uppbornar \hld ættir manna
hvat ’s Skjǫldunga, \hld hvat ’s Skilfinga,
hvat ’s Ǫðlinga \hld hvat ’s Ylfinga
hvat ’s hǫldborit, \hld hvat ’s hęrsborit
męst manna val \hld und Miðgarði?

Now let be recounted the ancient lines of kinsmen, and the upborn\footnote[1] aughts† of men: What is of the Shieldings? What is of the Shilvings? What is of the Athlings? What is of the Wolvings? What is born of hero? What is born of chief, the mightiest choice of men in Midyard?”
\footnote[1] Noble.

Þú est Óttarr \hld borinn Innstęini,
ęn Innstęinn vas \hld Alfi inum gamla,
Alfr vas Ulfi, \hld Ulfr Sæfara,
ęn Sæfari \hld Svan inum rauða.

Hindle quoth:
“Thou\footnote[1] art, Oughthere, born to Instone, but Instone was born to Elf the old, Elf to Wolf, Wolf to Seafare, but Seafare to Swan the red.
\footnote[1] Hindle, apparently in a trance-like state, speaks straight to Oughthere.

Móður átti faðir þinn \hld męnjum gǫfga,
hygg at héti \hld Hlédís gyðja,
Fróði vas faðir þęirar, \hld ęn Fríund\footnotemark[1] móðir;
ǫll þótti ætt sú \hld með yfirmǫnnum.
\footnotemark[1] Emended from the meaningless ms. reading \emph{friaut}.

Thy father had thy mother, beautiful with neck-rings, I think that she was called Leedise yidde†. Frood was her father, but Friend her mother; all her aught seemed to be among overmen.

Auði vas áðr \hld ǫflgastr manna,
Halfdanr fyrri \hld hæstr Skjǫldunga,
fræg vǫ́ru folkvíg, \hld þaus framir gęrðu,
hvarfla þóttu verk \hld með himins skautum.
    
Ed was before [that] the most powerful of men, Halfdane earlier the highest of Shieldings. Renowned were the troop-battles which the famous ones performed; his <= Halfdane's> works seemed to travel around the corners of heaven.

Ęflðisk við Ęymund \hld ǿztan manna
ęn vá Sigtrygg \hld með svǫlum ęggjum,
ęiga gekk Almvęig, \hld ǿzta kvinna,
ólu þau ok ǫ́ttu \hld átján sonu.
    
He <= Halfdane> became the in-law of Iemund\footnotemark[1], the noblest of men, but he slew Sightrue with cool edges. He went on to have Elmwey, the noblest of women; they begot and had eighteen sons.
\footnotemark[1] Lit. "[he] was strengthened by". Parallelism of "noblest of men/women" makes the meaning yet clearer. Elmwey was Iemund's daughter or sister.

Þaðan eru Skjǫldungar, \hld þaðan eru Skilfingar,
þaðan eru Ǫðlingar, \hld þaðan eru Ynglingar,
þaðan es hǫldborit, \hld þaðan es hęrsborit,
mest mannaval \hld und Miðgarði;
alt ’s þat ætt þín, \hld Óttarr heimski.

Thereof are the Shieldings! Thereof are the Shilvings! Thereof are the Inglings!\footnotemark[1] Thereof is born of hero! Thereof is born of chief, the mightiest choice of men in Midyard! That is all thy aught†, foolish Oughthere!”
\footnotemark[1] Note the contradiction with v. 12. Since the Inglings have already been mentioned (under the name Shilvings, of the difference between the two see the index), it seems likely that Wolvings is the original reading.

Vas Hildigunnr \hld hęnnar móðir,
Svǫ́fu barn \hld ok sækonungs;
alt ’s þat ætt þín, \hld Óttarr hęimski.
varðar\footnote[1] at viti svá, \hld viltu ęnn lęngra?
\footnote[1] Emended from ms. \emph{varði}.

Hildguth was her mother, the child of Swabe and Seaking; that is all thy aught†, foolish Oughthere!—It is meaningful that one might know thus; wilt thou [go] yet further?

Dagr átti Þóru \hld dręngjamóður,
ólusk í ætt þar \hld ǿztir kappar,
Fraðmarr ok Gyrðr \hld ok Frekar báðir,
Ámr ok Jǫsurmarr, \hld Alfr hinn gamli.
varðar at viti svá, \hld viltu ęnn lęngra?

Day had Thure, the mother of valiant men; in that aught were begotten the noblest champions: Fradmer and Yird, and both Frecks; Ame and Essirmer; Elf the old.—It is meaningful that one might know thus; wilt thou [go] yet further?

Kętill hét vinr þęira \hld Klypps arfþęgi,
vas hann móðurfaðir \hld móður þinnar;
þar vas Fróði \hld fyrr ęnn Kári,
ęn Hildi vas \hld Hóalfr of getinn.

Kettle, the inheritor of Clip, was their friend; he was the father of thy mother's mother. There was Frood, yet earlier Keer, but Highelf was by Hild begotten.
% — Frow
%	\bookStart{The Lay of Wayland}[Vǫlundarkviða]

BPG % TODO We need a "bpg", begin prose group!
BPA Níðuðr hét konungr í Svíþjóð.
BPA Hann átti tvá sonu ok eina dóttur. Hon hét Böðvildr.
BPA Brę́ðr váru þrír, synir Finnakonungs.
BPA Hét einn Slagfiðr, annarr Egill, þriði Völundr.
BPA Þeir skriðu ok veiddu dýr. Þeir kómu í Úlfdali ok gerðu sér þar hús.
BPA Þar er vatn, er heitir Úlfsjár.
BPA Snemma of morgin fundu þeir á vatnsströndu konur þrjár, ok spunnu lín.
BPA Þar váru hjá þeim álftarhamir þeira. Þat váru valkyrjur.
BPA Þar váru tvę́r dę́tr Hlöðvés konungs, Hlaðguðr svanhvít ok Hervör alvitr, in þriðja var Ölrún Kjársdóttir af Vallandi.
BPA Þeir höfðu þę́r heim til skála með sér. Fekk Egill Ölrúnar, en Slagfiðr Svanhvítrar, en Völundr Alvitrar.
BPA Þau bjuggu sjau vetr. Þá flugu þę́r at vitja víga ok kómu eigi aftr.
BPA Þá skreið Egill at leita Ölrúnar, en Slagfiðr leitaði Svanhvítrar, en Völundr sat í Úlfdölum.
BPA Hann var hagastr maðr, svá at menn viti, í fornum sögum.
BPA Níðuðr konungr lét hann höndum taka, svá sem hér er um kveðit: EPA

BPB Nithad was named a king in Sweden.
BPB He owned two sons and one daughter, she was called Beadhild.
BPB There were three brothers, the sons of a Finnish king.
BPB The first was called Slayfinn, the second Agle, the third Wayland.
BPB They travelled on skis and hunted wild animals. They came into Wolfdale and made for themselves houses there.
BPB There is a water there, called Wolfsea.
BPB Early in the morning they found on the lake-shore three women, and they were spinning linen.
BPB By them were their swan-\inx{hames}[C]; those were Walkirries.
BPB Two of them were the daughters of king Lathwy: Lathguth Swanwhite and Harware Allwit, the third was Alerune, daughter of Kear of \inx{Walland}[P]\footnoteB{The Roman emperor. See Index for more.}.
BPB They brought them with them to their halls. Agle got Alerune, but Beatfinn Swanwhite, but Wayland Allwit.
BPB They lived there for seven winters, then they left to attend battles, and did not return.
BPB Then Agle left on skis to seek out Alerune, but Beatfinn sought out Swanwhite, but Wayland stayed in Wolfdale.
BPB He was the most skillful man, which men have known in ancient tales.
BPB King Nithad had him taken, about which this has been sung:
EPG


\bvg
\bva Męyjar flugu sunnan \hld\ Myrkvið í gǫgnum &
alvitr ungar, \hld\ ørlǫg drýgja; &
þę́r á sę́varstrǫnd \hld\ sęttusk at hvílask &
drósir suðrǿnar, \hld\ dýrt lín spunnu.\eva

\bvb Maidens flew from the south through Mirkwood—young allwits\footnoteB{Maybe look at what this means. TODO.}—to fulfill \inx{orlay}[C]. They on the lake-shore set down to rest; the southern ladies span expensive linen.\evb
\evg


\bvg
\bva Ęin nam þęira \hld\ Ęgil at vęrja &
fǫgr mę́r fira \hld\ faðmi ljósum. &
Ǫnnur vas Svanhvít, \hld\ svanfjaðrar dró, &
\edtext{[...]}{\Bfootnote{Wo. doubt a line has gone missing here, mentioning the name of Slayfinn.}} &
ęn hin þriðja \hld\ þęira systir &
varði hvítan \hld\ hals Vǫlundar.\eva

\bvb One of them began—the fair maiden of men—to ward Agle with her light bosom. The second was Swanwhite, her swan-feathers she pulled—but the third of the sisters warded the white throat of Wayland.\evb
\evg


\bvg
\bva Sǫ́tu síðan \hld\ sjau vetr at þat, &
ęn hinn átta \hld\ allan þrǫ́ðu, &
ęn hinn níunda \hld\ nauðr of skilði, &
męyjar fýstusk \hld\ á myrkvan við, &
alvitr ungar \hld\ ørlǫg drýgja.\eva

\bvb Then they stayed for seven winters after that, and all the eighth they yearned, and on the ninth did need separate them: the maidens longed for the Mirky wood: the young allwits to fulfill orlay.\footnoteB{Mirkwood being the war-ravaged lands of the Gots and Huns; as walkirries their \emph{orlay} is to judge over battles.}\evb
\evg


\bvg
\bva Kom þar af vęiði \hld\ veðręygr skyti &
Vǫlundr líðandi \hld\ of langan veg, &
Slagfiðr ok Ęgill, \hld\ sali fundu auða, &
gingu út ok inn \hld\ ok umb sǫ́usk.\eva

\bvb Came there from the hunt the weather-eyed shooter: Wayland passing from a long way. Slayfinn and Agle found the halls deserted, they walked out and in, and looked about.\evb
\evg


\bvg
\bva Austr skręið Ęgill \hld\ at Ǫlrúnu, &
ęn suðr Slagfiðr \hld\ at Svanhvítu, &
ęn ęinn Vǫlundr \hld\ sat í Ulfdǫlum.\eva

\bvb East skied Agle for Alerune, but south Slayfinn for Swanwhite; but alone Wayland stayed in the Wolfdales.\evb
\evg


\bvg
\bva Hann sló goll rautt \hld\ við gim fastan, &
lukði hann alla \hld\ linnbaugum vel; &
svá bęið hann \hld\ sinnar ljóssar &
kvánar, ef hǫ́num \hld\ of koma gęrði.\eva

\bvb He struck the red gold by fastened gemstone, enclosed he all the serpent-\inx{bighs}\footnoteB{i.e. armlets shaped like serpents, perhaps even literally; compare the Viking age armlet found in a hoard in Undrom, Ångermanland, northern Sweden. Museum ID 108822 HST. TODO: Maybe include photo?} well; thus awaited he his bright wife, if to him she might come.\evb
\evg


\bvg
\bva Þat spyrr Níðuðr, \hld\ Níara dróttinn, &
at ęinn Vǫlundr \hld\ sat í Ulfdǫlum; &
nóttum fóru sęggir, \hld\ nęglðar vǫ́ru brynjur, &
skildir bliku þęira \hld\ við hinn skarða mána.\eva

\bvb It Nithad learns—lord of the Nears—that alone Wayland stayed in the Wolfdales. By night travelled warriors—nailed\footnoteB{i.e. plated.} were their byrnies; their shields gleamed by the waning moon.\evb
\evg


\bvg
\bva Stigu ór sǫðlum \hld\ at salar gafli, &
gingu inn þaðan \hld\ ęndlangan sal, &
sǫ́u þęir á bast \hld\ bauga dręgna, &
sjau hundruð allra, \hld\ es sá sęggr átti.\eva

\bvb They stepped out of the saddles towards the hall’s gables; went inside thence through the length of the hall. Saw they on a bast-rope bighs drawn: seven hundred in all, which that man owned.\evb
\evg


\bvg
\bva Ok þęir af tóku \hld\ ok þęir á létu &
fyr ęinn útan, \hld\ es af létu; &
kom þar af vęiði \hld\ veðręygr skyti &
Vǫlundr líðandi \hld\ of langan veg.\eva

\bvb And they took off and they put back on; but for one, which away they put.\footnoteB{That this is the bigh mentioned by itself in vv. 17 and 26 seems likely. \Finnur\ writes: “The ring which Nithad kept must have had special properties, and distinguished itself before others. There is no doubt that the ring is a flight ring; whether this was clear to the poet is however questionable. This much is certain, that Wayland seems to be able to fly away only after he has got back the ring; that is, the one which Beadhild brings him.” \emph{(My translation from the Danish.)}—The reader may for himself judge the plausibility of this, but it seems to me that Wayland, being an exceptionally skilled craftsman, may just as well have crafted wings for himself without need for magical rings. This would be closer to the Daedalus myth, for the influence of which see the Introduction to this poem.}—Came there from the hunt the weather-eyed shooter: Wayland passing from a long way.\evb
\evg


\bvg
\bva Gekk brúnni \hld\ beru hold stęikja, &
ár brann hrísi \hld\ allþurru fura, &
viðr hinn vindþurri, \hld\ fyr Vǫlundi.\eva

\bvb He went the brown she-bear’s hull to roast; early burned the twigs of all-dry pine—the wind-dry wood—before Wayland.\evb
\evg


\bvg
\bva Sat á berfjalli, \hld\ bauga talði, &
alfa ljóði \hld\ ęins saknaði. &
hugði at hęfði \hld\ Hlǫðvés dóttir, &
Alvitr unga, \hld\ vę́ri aptr komin.\eva

\bvb Sat he on the bear-skin, his bighs he counted—the prince of elves was missing one. Thought he that Ladwigh’s daughter might have it; that the young Allwit might be come back.\evb
\evg


\bvg
\bva Sat hann svá lęngi, \hld\ at hann sofnaði, &
ok hann vaknaði \hld\ viljalauss; &
vissi sér á hǫndum \hld\ hǫfgar nauðir, &
ęn á fótum \hld\ fjǫtur of spęntan.\eva

\bvb He sat so long that asleep he fell, and he awoke, powerless. He knew on his hands tortuous restraints, and on his feet tightened fetters.\evb
\evg


\bvg
(Vǫlundr kvað)
\bva Hvęrir ’ró jǫfrar \hld\ þęir’s á lǫgðu &
bęstisíma \hld\ ok bundu mik?\eva

\bvb Wayland quoth: \\
“Who are those princes, that laid on thick bast-ropes, and bound me?”\evb
\evg


\bvg
\bva Kallaði nú Níðuðr, \hld\ Níara dróttinn: &
„Hvar gazt Vǫlundr, \hld\ vísi alfa, &
óra aura, \hld\ í Ulfdǫlum? &
Goll vas þar ęigi \hld\ á Grana lęiðu, &
fjarri hugða’k várt land \hld\ fjǫllum Rínar.“\eva

\bvb Nithad now called, lord of the Nears: “Where gottest thou, Wayland, leader of Elves, \emph{our} ounces in the Wolfdales? Gold was there not on Grane’s path; far I thought our land from the mountains of the Rhine.\footnoteB{Grane was the horse of the legendary hero Siward, who slew the dragon Fathomer. These events were set in continental Germany. The sense of this sarcastic statement is thus “Where did you get that gold? A dragon’s hoard?”. (This interpretation I first encountered from \Finnur, but I cannot see any likelier.)}”\evb
\evg


\bvg (Vǫlundr kvað)
\bva „Man’k at męiri \hld\ mę́ti ǫ́ttum, &
es vér hęil hjú \hld\ hęima vǫ́rum. &
Hlaðguðr ok Hervǫr \hld\ borin vas Hlǫðvé, &
kunn vas Ǫlrún \hld\ Kíars dóttir.“\eva

Wayland quoth: \\
\bvb “I remember that we owned a more precious thing, when we a healthy household were at home: Ladguth and Harware were born to Ladwigh; known was Alerune, Keer’s daughter.”\evb
\evg


\bvg
\bva Úti stóð kunnig \hld\ kvǫ́n Níðaðar, &
hón inn of gekk \hld\ ęndlangan sal, &
stóð á golfi, \hld\ stilti rǫddu: &
„es-a sá nú hýrr, \hld\ es ór holti fęrr.\eva

\bvb Outside stood the cunning wife of Nithad; she walked inside across the length of the hall, stood on the floor, steered her voice: That one\footnoteB{The abducted Wayland.} is now not cheery, who comes out of the wood.\evb
\evg


\bvg
\bva Tęnn hǫ́num tęygjask \hld\ es hǫ́num’s tét sverð &
ok hann Bǫðvildar \hld\ baug of þękkir. &
Ǫ́mun eru augu \hld\ ormi hinum frána, &
sníðið ér hann \hld\ sina magni, &
ok sętið hann síðan \hld\ í Sę́varstǫð.“\eva

\bvb His teeth are bared when he is shown the sword; and he recognizes Beadhild’s bigh. Reminiscent are the eyes to the gleaming snake’s.—Cut ye from him the might of his sinews, and place him thereafter on Seastead!”\evb
\evg


\bvg
\bva[P] Svá var gǫrt, at skornar váru sinar í knésfótum ok settr í holm einn, er þar var fyrir landi, er hét Sę́varstaðr. Þar smíðaði hann konungi allskyns gǫrsimar; engi maðr þorði at fara til hans, nema konungr einn. Vǫlundr kvað:\eva

\bvb[P] Thus was done, that the sinews in his houghs were cut, and he was placed on a lonely islet, which there lay before the land, called Seastead. There he smithed for the king all manner of jewels. No man dared travel to him, but the king alone. Wayland quoth:\evb
\evg


\bvg
\bva „Sé’k Níðaði \hld\ sverð á linda, &
þat’s ek hvęsta \hld\ sęm hagast kunna’k &
ok ek hęrða’k \hld\ sęm hǿgst þótti; &
sá ’s mér fránn mę́kir \hld\ ę́ fjarri borinn. &
sé’kk-a þann Vǫlundi \hld\ til smiðju borinn.\eva

\bvb I see a sword on Nithad’s belt, the one I sharpened as most handily I knew, and hardened as most pleasingly seemed. Now that gleaming sword is ever far-away carried; I see it not to Wayland’s smithy carried.\evb
\evg


\bvg
\bva Nú berr Bǫðvildr \hld\ brúðar minnar, &
bíð’k-a þess bót, \hld\ bauga rauða.“\eva

\bvb Now Beadhild bears my bride’s—I get no recompense for that—red bighs.
\evg


\bvg
\bva Sat hann né svaf ávalt \hld\ ok sló hamri; &
vél gęrði hęldr \hld\ hvatt Níðaðí; &
drifu ungir tvęir \hld\ á dýr séa &
synir Níðaðar \hld\ í Sę́varstǫð.\eva

\bvb He sat nor slept always, and struck the hammer; rather he boldly planned wiles for Nithad.—Two young ones hurried to look at precious things: Nithad’s sons, to Seastead.\evb
\evg


\bvg
\bva Kvǫ́mu til kistu, \hld\ krǫfðu lukla, &
opin vas illúð, \hld\ es í sǫ́u, &
fjǫlð vas þar męina, \hld\ es mǫgum sýndisk &
at vę́ri goll rautt \hld\ ok gǫrsimar.\eva

\bvb They came to the chest, demanded the keys; open was the evil, when inside they looked. A multitude was there of harm, which to the lads seemed, like were it red gold and jewels.\evb
\evg


\bvg
\bva „Komið ęinir tvęir, \hld\ komið annars dags; &
ykkr lę́t’k þat goll \hld\ of gefit verða; &
sęgið-a męyjum \hld\ né salþjóðum, &
manni ęngum, \hld\ at mik fyndið.“\eva

\bvb “Come alone ye two, come another day; to you I will let that gold be given! Tell not maidens, nor the people of the hall, nor any man, that ye saw me.”\evb
\evg


\bvg
\bva Snimma kallaði \hld\ sęggr á annan, &
bróðir á bróður: \hld\ „gǫngum baug séa!“ &
Kómu til kistu, \hld\ krǫfðu lukla, &
opin vas illúð \hld\ es í litu.\eva

\bvb Early called one man to another, brother to brother: “Let us go see the bighs!”. They came to the chest, demanded the keys; open was the evil, when inside they looked.\evb
\evg


\bvg
\bva Snęið af hǫfuð \hld\ húna þęira &
ok und fęn fjǫturs \hld\ fǿtr of lagði, &
ęn þę́r skálar, \hld\ es und skǫrum vǫ́ru, &
svęip útan silfri, \hld\ sęldi Níðaði.\eva

\bvb He sliced off the heads of those bear-cubs\footnoteB{An affectionate term for the young boys. TODO: Relate to Bearserks.}, and under the fetter’s fen their feet did lay; but the bowls\footnoteB{Their skulls.}, which were under their locks, he coated with silver and gave to Nithad.\evb
\evg


\bvg
\bva Ęn ór augum \hld\ jarknastęina &
sęndi kunnigri \hld\ kvǫ́n Níðaðar; &
ęn ór tǫnnum \hld\ tvęggja þęira &
sló brjóstkringlur, \hld\ sęndi Bǫðvildi.\eva

\bvb But out of the eyes, earkenstones he sent to the cunning wife of Nithad; but out of the teeth of the two, he struck breast-brooches, sent to Beadhild.\evb
\evg


\bvg
\bva Þá nam Bǫðvildr \hld\ baugi at hrósa &
\edtext{[...]}{\Bfootnote{The meter requires a half-line here, likely containing a more specific description of the bigh.}}\ \hld\ es brotit hafði, &
„þori’k-a’k sęgja, \hld\ nema þér ęinum.“\eva

\bvb Then Beadhild began to praise the ring,\footnoteB{Clearly the verse is incomplete, but the story can be gleaned: Beadhild breaks the bigh she has been given by her parents (previously mentioned in vv. 10—see the note there—and 17), but dares not tell anybody but Wayland.} [...] which she had broken, “I dare not tell, but to thee alone.”\evb
\evg


\bvg
\bva „Ek bǿti svá \hld\ brest á golli, &
at fęðr þínum \hld\ fęgri þykkir, &
ok mǿðr þinni \hld\ miklu bętri, &
ok sjalfri þér \hld\ at sama hófi.“\eva

\bvb “I mend such the crack on the gold, that to thy father it fairer seems, and to thy mother far better, and to thyself of the same rank.”\evb
\evg


\bvg
\bva Bar hann hána bjóri, \hld\ þvíat hann bętr kunni, &
svát hón í sessi \hld\ of sofnaði. &
„Nú hęfk hęfnt \hld\ harma minna &
allra nema ęinna \hld\ íviðgjǫrnum.“\eva

\bvb He overcame her with beer—for he was more cunning—so that she in the seat asleep did fall. “Now have I avenged my harms—all but one—on the insidious ones.\footnoteB{King Nithad and his wife.}”\evb
\evg


\bvg
\bva „Vęl ek, kvað Vǫlundr, \hld\ verða’k á fitjum, &
þęim’s mik Níðaðar \hld\ nǫ́mu rekkar.“ &
Hlę́jandi Vǫlundr \hld\ hófsk at lopti, &
grátandi Bǫðvildr \hld\ gekk ór ęyju. &
tregði fǫr friðils \hld\ ok fǫður vreiði.\eva

\bvb “Well I”, quoth Wayland, “fall on my paddles\footnoteB{\emph{C-V}: \emph{fit} ‘the webbed foot of water-birds’, the reader may picture for himself. Wayland has crafted wings in stead of his feet, of which use Nithad’s men deprived him.}, those which Nithad’s men bereaved me of!” Laughing Wayland threw himself in the air; weeping Beadhild went from the island: she grieved the lover’s flight, and the father’s fury.\evb
\evg


\bvg
\bva Úti stóð kunnig \hld\ kvǫ́n Níðaðar, &
ok hón inn of gekk \hld\ ęndlangan sal, &
— ęn hann á salgarð \hld\ sęttisk at hvílask —, &
„Vakir þú Níðuðr, \hld\ Níara dróttinn?“\eva

\bvb Outside stood the cunning wife of Nithad, she walked inside across the length of the hall—but he, on the courtyard, set down to rest. “Art thou awake, Nithad, lord of the Nears?”\evb
\evg


\bvg
\bva „Vaki’k ávalt \hld\ viljalauss, &
sofna’k minst, \hld\ síz sonu dauða, &
kęll mik í hǫfuð, \hld\ kǫld erumk rǫ́ð þín, &
vilnumk þess nú, \hld\ at við Vǫlund dǿma’k.“\eva

\bvb “I am always awake, powerless; I sleep the least, since the death of my sons. My head freezes, cold are thy counsels; I wish now but that: to speak with Wayland.”\evb
\evg


\bvg
\bva „Sęg mér þat Vǫlundr, \hld\ vísi alfa, &
af hęilum hvat varð \hld\ húnum mínum?“\eva

\bvb “Say it to me, Wayland, leader of Elves: what became of my healthy bear-cubs?”\evb
\evg


\bvg
\bva „Ęiða skalt mér áðr \hld\ alla vinna, &
at skips borði \hld\ ok at skjaldar rǫnd, &
at mars bǿgi \hld\ ok at mę́kis ęgg &
at þú kvęlj-at \hld\ kvǫ́n Vǫlundar, &
né brúði minni \hld\ at bana verðir, &
þótt kvǫ́n ęigim, \hld\ þá’s ér kunnið, &
eða jóð ęigim \hld\ innan hallar.\eva

\bvb “Before that shalt thou swear me all oaths:—by the deck of the ship and the rim of the shield, by the bough of the steed and the edge of the sword—that thou wilt not torment the wife of Wayland, nor of my bride become the bane, though we a wife might own, which ye know, or a babe might own inside the hall.\footnoteB{Wayland has Nithad swear an oath that he will not harm Beadhild, nor their (yet unborn) child.}\evb
\evg


\bvg
\bva Gakk til smiðju, \hld\ es gęrðir þú, &
þar fiðr þú bęlgi \hld\ blóði stokna, &
snęið’k af hǫfuð \hld\ húna þinna &
ok und fęn fjǫturs \hld\ fǿtr of lagða’k.\eva

\bvb Go to the smithy, which thou made; there thou wilt find bellows, with blood sprinkled. I sliced off the heads of thy bear-cubs, and under the fetter’s fen their feet did I lay.\evb
\evg


\bvg
\bva Ęn þę́r skálar, \hld\ es und skǫrum vǫ́ru, &
svęip’k útan silfri, \hld\ sęlda’k Níðaði, &
ęn ór augum \hld\ jarknastęina, &
sęnda'k kunnigri \hld\ kvǫ́n Níðaðar.\eva

\bvb But the bowls, which were under their locks, I coated with silver and gave to Nithad. But out of the eyes, earkenstones I sent to the cunning wife of Nithad.\evb
\evg


\bvg
\bva Ęn ór tǫnnum \hld\ tvęggja þęira &
sló’k brjóstkringlur, \hld\ sęnda’k Bǫðvildi; &
nú gęngr Bǫðvildr \hld\ barni aukin, &
ęingadóttir \hld\ ykkur bęggja.“\eva

\bvb But out of the teeth of the two, I struck breast-brooches, sent to Beadhild. Now walks Beadhild, swollen with child; the only daughter of you both.”\evb
\evg


\bvg
\bva „Mę́ltir-a þú þat mál, \hld\ es mik męir tregi, &
né þik vilja’k Vǫlundr \hld\ verr of níta; &
es-at svá maðr hǫ́r, \hld\ at þik af hęsti taki, &
né svá ǫflugr, \hld\ at þik neðan skjóti. &
þar’s þú skollir \hld\ við ský uppi.“\eva

\bvb “Thou spokest not that speech which might grieve me more; nor could I worse wish, Wayland, to deny thee. There is no man so high that he from horse might take thee, nor so mighty that he might shoot thee down, there where thou jeerest, by the clouds above!”\evb
\evg


\bvg
\bva Hlę́jandi Vǫlundr \hld\ hófsk at lopti, &
ęn ókátr Níðuðr \hld\ þá ęptir sat.\eva

\bvb Laughing Wayland threw himself in the air, but gloomy Nithad thereafter stayed.\evb
\evg


\bvg
\bva „Upp rís Þakkráðr, \hld\ þrę́ll minn bazti, &
bið Bǫðvildi, \hld\ męy hina bráhvítu, &
gangi fagrvarið \hld\ við fǫður rǿða.“\eva

\bvb “Rise up Thankred, my best thrall; ask Beadhild—the brow-white maiden—to go fair-clothed, with her father to counsel.”\evb
\evg


\bvg
\bva „Es þat satt Bǫðvildr, \hld\ es sǫgðu mér, &
sǫ́tuð it Vǫlundr \hld\ saman í holmi?“\eva

\bvb “Is it true, Beadhild, as they said to me: stayed thou and Wayland together on the island?”\evb
\evg


\bvg
\bva „Satt ’s þat Níðuðr \hld\ es sagði þér: &
sǫ́tum vit Vǫlundr \hld\ saman í holmi &
ęina ǫgurstund, \hld\ ę́va skyldi; &
ek vę́tr hǫ́num \hld\ vinna kunna’k, &
ek vę́tr hǫ́num \hld\ vinna mátta’k.“\eva

\bvb “It is true, Nithad, as \emph{he} said\footnoteB{Beadhild, knowing that the only one who is aware of what happened is Wayland, makes the subtle change in the conjugation, from her father’s general plural (“what \emph{they} said”), to the specific singular (“what \emph{he} said”).} to thee: I and Wayland stayed together on the island, for one grave moment—it should never have been. I knew nought struggle against him, I could nought struggle against him.\footnoteB{She was both mentally (\CV: \emph{kunna} ‘know, understand’) and physically (\CV: \emph{mega} ‘to have strength to do, avail’) incapable of struggling against him. As Finnur comments, a potent final verse.}”\evb
\evg
% — Heroic/mythological, transition
%	Svá sęgja menn í fornum sǫgum, at ęinnhvęrr af ǫ́sum, sá es Heimdallr hét, fór fęrðar sinnar ok framm með sjóvarstrǫndu nǫkkurri, kom at ęinum húsabǿ ok nefndisk Rígr; ęptir þęiri sǫgu es kvæði þetta.

Thus men say in ancient saws†, that one of the Ease†, he who was called Homedall, went on his journey and forth along some lake-shore, came upon a lone homestead and called himself Righ. After that saw is this poem:

Ár kvǫ́ðu ganga \hld grǿnar brautir
ǫflgan ok aldinn \hld ǫ́s kunnigan,
ramman ok rǫskvan \hld Ríg stíganda. 

Of yore they said [did] walk the green paths, a powerful and aged os†, cunning; the strong and quick Righ, striding.

Gekk hann męir at þat \hld miðrar brautar,
kom hann at húsi, \hld hurð var á gætti;
inn nam at ganga, \hld eldr var á golfi,
hjón sǫ́tu þar \hld hǫ́r at arni,
Ái ok Edda \hld aldinfalda. 

Went he further at that, on the middle of the road; came he to a house, the door was wide open. H e began to walk inside, fire was on the floor. A couple sat there, hoary by the hearth: Great Grandfather and Great Grandmother, old-fashioned.
% — Society
%
% Heroic poems, order of the Codex Regius
%	\bookStart{First Lay of Hallow Hundingsbane}[Helgakviða Hundingsbana fyrsta]

\bvg
\bva Ár vas alda \hld\ þat’s arar gullu &
hnigu hęilǫg vǫtn \hld\ af Himinfjǫllum; &
þá hafði Hęlga \hld\ inn hugumstóra &
Borghildr borit \hld\ í Brálundi.\eva

\bvb It was the beginning of \inx[C]{eld}[elds], as eagles shrieked; holy waters poured down from the Heavenfells; then Burhild in Browlund gave birth to Hallow the Great-hearted.\evb
\evg


\bvg
\bva Nótt varð í bǿ, \hld\ nornir kvǫ́mu, &
þę́r’s ǫðlingi \hld\ aldr of skópu; &
þann bǫ́ðu fylki \hld\ frę́gstan verða &
ok buðlunga \hld\ bęztan þykkja.\eva

\bvb Night came in the settlement; norns came, those who did shape the prince’s life; that marshaller \name{= Hallow} they declared would become most renowned, and of kings seem the foremost.\evb
\evg


\bvg
\bva Sneru þę́r af afli \hld\ ørlǫgþǫ́ttu &
þá’s borgir braut \hld\ í Brálundi; &
þę́r um gręiddu \hld\ gullinsímu &
ok und mána sal \hld\ miðjan fęstu.\eva

\bvb They turned with their might the strands of \inx[C]{orlay}, as he broke cities in Browlund; they arranged golden bands, and under the moon's hall fastened [them in] the middle.\evb
\evg

%	Helgi fekk Sigrúnar ok áttu þau sonu; var Helgi eigi gamall. Dagr Hǫgna sonr blótaði Óðin til fǫðurhefnda. Óðinn léði Dag geirs síns. Dagr fann Helga, mág sinn, þar sem heitir at Fjǫturlundi. Hann lagði í gǫgnum Helga með geirnum. Þar fell Helgi en Dagr reið til fjalla ok sagði Sigrúnu tíðindi: 

Hallow got Sighrun, and they owned sons; Hallow was not old. Day, son of Hain, blooted† to Weden to gain revenge for his father. Weden lent his spear to Day. Day found Hallow, his brother-in-law, on the place called Fetterlund. He pierced Hallow with the spear. There Hallow fell, but Day rode to the mountains and told Sighrun the news:

“Trauðr em ek, systir, \hld trega þér at sęgja
þvíat ek hęfi nauðigr \hld nipti grætta:
Fell í morgun \hld und Fjǫturlundi
buðlungr sá es vas \hld bęztr í hęimi
ok hildingum \hld á hálsi stóð.”

“Regretful am I, sister, to pain thee by saying — for I have been forced to cause my kinswoman to cry: This morning fell 'neath Fetterlund, the prince who was the best in the world, and on the necks of rulers stood.”
f
%	\bookStart{The Lay of Hallow Harwardson}[Hęlgakviða Hjǫrvarðssonar]

fra hiorvarþi oc sigrlinn.

hiorvarþr het konvngr hann atti iiii. konor einn het alfhildr. sonr þeira het heðinn. onnor het sę́reiþr. þeira sonr het hvmlvngr. in þriþia het sinríoþ. þeira sonr het hymlingr. Hiorvarþr konvngr hafði þess heit strengt át eiga þa kono er hann vissi vę́nsta. H ann spvrþi at svafnir konvngr. atti dottvr vę́nallra fegrsta sv het sigrlinn. Jþmvndr het iarl hans atli var hans. sonr er for at biþia sigrlinnar til handa konvngi. hann dvalþiz vetr langt meþ svafni konvngi. Fránmaʀ het þar iarl fostri sigrlinnar. dottir hans het alóf. Jarlinn réþ at meyiar var syniat oc fór iarlinn. heim. atli iarls sonr stoþ einn dag viþ lvnd noccorn enn fvgl sat i limonom vppi yfir hanom oc hafdi heyrt til at hans. menn. kǫlloþo vę́nstar kónor þę́r er hiorvarþr konvngr. atti. fvglinn qvacaþi enn atli lyddi. hvat hann sagdi. hann qvaþ.


\bvg
\bva 1\eva

\bvb 1\evb
\evg


\bvg
\bva 2\eva

\bvb 2\evb
\evg


\bvg
\bva 3\eva

\bvb 3\evb
\evg


\bvg
\bva 4\eva

\bvb 4\evb
\evg


\bvg
\bva 5\eva

\bvb 5\evb
\evg


\bvg
\bva 6\eva

\bvb 6\evb
\evg


\bvg
\bva 7\eva

\bvb 7\evb
\evg


\bvg
\bva 8\eva

\bvb 8\evb
\evg


\bvg
\bva Sverð vęit’k liggja \hld\ î Sigarsholmi, &
fjórum fę́ra \hld\ enn fimm tǫgu; &
ęitt es þęira \hld\ ǫllum bętra &
vígnesta bǫl \hld\ ok varið golli.\eva

\bvb Swords I know lying, in Sigharsholm, four less than fifty. One of them is better than all—the bale of war-needles\footnoteB{The kenning \emph{vígnest} also appears in} \ken{spears?}—and inlaid with gold.\evb
\evg


\bvg
\bva Hringr ’s î hjalti, \hld\ hugr ’s î miðju, &
ógn ’s î oddi, \hld\ þęim’s ęiga getr; &
liggr með ęggju \hld\ ormr dręyrfáiðr &
en ȧ valbǫstu \hld\ verpr naðr hala.\eva

\bvb A ring is in the hilt; courage is in the middle; fear is in the point, for the one who gets to own it; along the blade lies a serpent painted in blood, but on the walbast an adder chases its tail.\evb
\evg

%	\bookStart{The Lay of Attle}[Atlakviða]

BPG %TODO prose formatting
Dauði Atla.

Guðrún Gjúkadóttir hefndi brǿðra sinna, svá sem frę́gt er orðit. Hon drap fyrst sonu Atla, en eptir drap hon Atla ok brendi hǫllina ok hirðina alla; um þetta er sjá kviða ort.

The Death of Attle

Guthrun Yivicksdaughter avenged her brothers, as has become famous. She first killed the sons of Attle, and after that she killed Attle, and burned the hall and the whole hird. Regarding that this lay is wrought.

\bvg
\bva Atli sęndi \hld\ ár til Gunnars &
kunnan sęgg at ríða, \hld\ Knéfrøðr vas sá hęitinn; &
at gǫrðum kom hann Gjúka \hld\ ok at Gunnars hǫllu, &
bękkjum aringręypum \hld\ ok at bjóri svǫ́sum.\eva

\bvb Attle sent early to Guther a well-known messenger to ride; Kneefred that one was called. To the estates of Yivick he came, and to the hall of Guther; to the hearth-surrounding benches, and to the lovely beer.\evb
\evg


\bvg
\bva Drukku þar dróttmęgir \hld\ —ęn \edtext{dyljęndr}{\lemma{dyljęndr ‘concealed ones’}\Bfootnote{\textcite{FinnurEdda} reasonably interprets this as referring to Attle’s spies at Guther’s court.}} þǫgðu— &
vín í \edtext{valhǫllu}{\lemma{valhǫllu ‘the walhall’}\Bfootnote{The interpretation of this compound is difficult in context. The first element \emph{val-} could be (1) \emph{valr} ‘falcon’, referring to the aristocratic hunting practice; (2) \emph{valr} ‘\inx[G]{Wales}[Wale]’, cognate with ‘Welsh’ but in ON referring to the French or Romans, stressing the southern location or appearance of the hall; or (3) \emph{valr} ‘(collective) the battle-slain’, foreshadowing the inevitable death (\inx[C]{feyness}) of the \inx[G]{Yivickings}. In this case it is linguistically identical to \inx[L]{Walhall}, Weden’s hall, whither the battle-slain go.}}, \hld\ vręiði sǫ́usk þęir Húna; &
kallaði þá Knéfrøðr \hld\ kaldri rǫddu, &
sęggr inn suðrǿni \hld\ sat hann á bękk hǫ́m:\eva

\bvb There the dright-lads drank—but the concealed ones were silent—wine in the walhall; wary were they of the wrath of the Huns. Then Kneefred, the southern man, called with cold voice; he sat on a high bench:\evb
\evg


\bvg
\bva “Atli mik hingat sęndi \hld\ ríða øręndi, &
mar inum mélgręypa, \hld\ Myrkvið inn ókunna &
at biðja yðr, Gunnarr, \hld\ at it á bękk kǿmið &
með hjǫlmum aringręypum \hld\ at sǿkja hęim Atla.\eva

\bvb “Attle me hither sent to ride an errand, with the bit-champing horse through the uncharted Mirkwood, to ask you, Guther, that ye two on the bench might come, with hearth-surrounding helmets, to seek the home of Attle.\evb
\evg


\bvg
\bva Skjǫldu kneguð þar vęlja \hld\ ok skafna aska, &
hjalma gullroðna \hld\ ok Húna męngi, &
silfrgyllt sǫðulklę́ði, \hld\ sęrki valrauða, &
dafar, darraða, \hld\ drǫsla mélgręypa.\eva

\bvb There ye might choose shields, and smooth ash-spears, helmets gold-reddened, and the multitude of the Huns, silver-gilt saddle-cloth, walred serks, dafs, standards, bit-champing steeds.\evb
\evg


\bvg
\bva Vǫll lézk ykkr ok myndu gefa \hld\ víðrar Gnitahęiðar &
af gęiri gjallanda \hld\ ok af gylltum stǫfnum, &
stórar męiðmar \hld\ ok staði Danpar, &
hrís þat it mę́ra \hld\ es meðr Myrkvið kalla.\eva

\bvb GAGAGA\evb
\evg


\bvg
\bva Hǫfði vatt þá Gunnarr \hld\ ok Hǫgna til sagði: &
Hvat rę́ðr þú okkr, sęggr inn ǿri, \hld\ allz vit slíkt hęyrum? &
Gull vissa ek ekki \hld\ á Gnitahęiði, &
þat es vit ę́ttim-a \hld\ annat slíkt.\eva

\bvb His head turned Guther then, and to Hain said: “What counselest thou we two do, younger man, as we such things hear? I knew of no gold on the Gnitheath, that we did not own as much of.\evb
\evg


\bvg
\bva Sjau ęigu vit salhús \hld\ sverða full, &
hvęrju eru þęira \hld\ hjǫlt ór gulli; &
mínn vęit ek mar bęztan \hld\ ęn mę́ki hvassastan, &
boga bękksǿma \hld\ ęn brynjur ór gulli.\eva

\bvb We own seven hallhouses, filled with swords—on each of them is a golden hilt; I know my horse to be the best, and my sword the sharpest; my bow bench-fit, and my byrnies of gold.\evb
\evg


\bvg
\bva Hjalm ok skjǫld hvítastan, \hld\ kominn ór hǫll Kjárs; &
ęinn es mínn bętri \hld\ ęn sé allra Húna.\eva

\bvb A helmet and the whitest shield, taken out of the hall of Chear; alone is mine better, than that of all of the Huns.”\evb
\evg


\bvg
\bva Hvat hyggr þú brúði bęndu \hld\ þá es hón okkr baug sęndi, &
varinn váðum hęiðingja? \hld\ Hykk at hón vǫrnuð byði! &
Hár fann ek hęiðingja \hld\ riðit í hring rauðum; &
ylfskr es vegr okkarr \hld\ at ríða øręndi.\eva

\bvb “What does thou think the bride meant, when she us two an armlet sent, wrapped with the cloth of a heath-dweller \ken{wolf}? I think that she bid us a warning! I found the hair of a heath-dweller wrapped round the red ring; wolven is our way, to ride that errand.”\evb
\evg


\bvg
\bva Niðjar-gi hvǫttu Gunnar \hld\ né náungr annarr, &
rýnęndr né ráðęndr, \hld\ né þęir es ríkir vǫ́ru; &
kvaddi þá Gunnarr \hld\ sęm konungr skyldi, &
mę́rr í mjǫðranni \hld\ af móði stórum:\eva

\bvb No kinsmen urged Guther, nor any other close one, nor counselors nor advisors, nor those who mighty were. Guther then announced—as a king should, renowned in the mead-house—out of great courage:\evb
\evg


\bvg
\bva Rís-tu nú, Fjǫrnir, \hld\ lát-tu á flęt vaða &
gręppa gullskálir \hld\ með gumna hǫndum!\eva

\bvb “Rise now, Ferner; let on the floorboards wade forth the golden bowls of warriors, along the hands of men!\evb
\evg


\bvg
\bva Ulfr mun ráða \hld\ arfi Niflunga, &
gamlir granvarðir, \hld\ ef Gunnars missir, &
birnir blakkfjallir \hld\ bíta þreftǫnnum, &
gamna gręystóði, \hld\ ef Gunnarr né kømr-at.\eva

\bvb The wolf will rule the inheritance of the Niflings: the old grey guardians, if Guther is missing. Bears black-furred bite with wrangling teeth, amusing the pack of bitches, if Guther comes not.”\evb
\evg


\bvg
\bva Lęiddu landrǫgni \hld\ lýðar ónęisir, &
grátęndr, gunnhvatan, \hld\ ór garði Húna; &
þá kvað þat inn ǿri \hld\ ęrfivǫrðr Hǫgna: &
Hęilir farið nú ok horskir \hld\ hvar’s ykkr hugr tęygir!\eva

\bvb GAGAGA\evb
\evg


\bvg
\bva Fetum létu frǿknir \hld\ um fjǫll at þyrja &
marina mélgręypu, \hld\ Myrkvið inn ókunna; &
hristisk ǫll Húnmǫrk \hld\ þar es harðmóðgir fóru, &
vrǫ́ku þęir vannstyggva \hld\ vǫllu algrǿna.\eva

\bvb GAGAGA\evb
\evg


\bvg
\bva Land sǫ́u þęir Atla \hld\ ok liðskjalfar djúpar &
Bikka greppar standa \hld\ á borg inni há &
sal of suðrþjóðum, \hld\ slęginn sessmęiðum, &
bundnum rǫndum, \hld\ blęikum skjǫldum,\eva

\bvb The land of Attle saw they, TODO\evb
\evg


\bvg
\bva dafar, darraða; \hld\ ęn þar drakk Atli &
vín í valhǫllu; \hld\ vęrðir sǫ́tu úti &
at varða þęim Gunnari \hld\ ef þęir hér vitja kǿmi &
með gęiri gjallanda \hld\ at vękja gram hildi.\eva

\bvb but there drank Attle wine in the wale-hall\footnoteB{TODO: this is not Weden’s hall, rather ‘the Roman hall’.} ... \evb
\evg


\bvg
\bva Systir fann þęira snemmst \hld\ at þęir í sal kvǫ́mu, &
brǿðr hęnnar báðir, \hld\ bjóri var hón lítt drukkin: &
Ráðinn ert-u nú, Gunnarr, \hld\ hvat munt-u, ríkr, vinna &
við Húna harmbrǫgðum? \hld\ Hǫll gakk þú ór snemma!\eva

\bvb Their sister found earliest they they had come into the hall, both of her brothers—on beer was she lightly drunk—“Betrayed art thou now, Guther; why wilt thou, mighty one, struggle against Hunnish harm-tricks? Go early out of the hall!\footnoteB{Before anything evil might happen.}”\evb
\evg


\bvg
\bva Bętr hęfðir þú, bróðir, \hld\ at þú í brynju fǿrir, &
sęm hjǫlmum aringręypum \hld\ at séa hęim Atla; &
sę́tir þú í sǫðlum \hld\ sólhęiða daga, &
nái nauðfǫlva \hld\ létir nornir gráta.\eva

\bvb Better hadst thou, brother, if thou in byrnie travelled, and with hearth-surrounding helmets, to see the home of Attle.\evb
\evg


\bvg
\bva Húna skjaldmęyjar \hld\ hęrfi kanna &
ęn Atla sjalfan \hld\ létir þú í ormgarð koma; &
nú es sá ormgarðr \hld\ ykkr of folginn.\eva

\bvb GAGAGA\evb
\evg


\bvg
\bva Seinað es nú, systir, \hld\ at samna Niflungum, &
langt es at lęita \hld\ lýða sinnis til, &
of rosmufjǫll Rínar, \hld\ rekka ónęissa.\eva

\bvb GAGAGA\evb
\evg


\bvg
\bva Fengu þęir Gunnar \hld\ ok í fjǫtur sęttu, &
vinir Borgunda, \hld\ ok bundu fastla; &
sjau hjó Hǫgni \hld\ sverði hvǫssu &
ęn inum átta hratt hann \hld\ í ęld hęitan.\eva

\bvb Caught they Guther, and in fetters set him—the friends of the Burgends—and bound them tightly. Seven Hain hewed down with sharp sword, and the eighth one threw he into the hot fire.\evb
\evg


\bvg
\bva \edtext{Svá skal frǿkn \hld\ fjándum vęrjask;}{\lemma{Svá ... vęrjask}\Bfootnote{Line moved from the last verse to this one since it seems to connect semantically with the immediately following line, and also creates a regular line distribution of 4-4 instead of 5-3.}} &
Hǫgni varði \hld\ hęndr Gunnars. &
frǫ́gu frǿknan \hld\ ef fjǫr vildi &
Gotna þjóðann \hld\ gulli kaupa.\eva

\bvb Thus shall the bold against fiends ward himself; Hain warded the hands of Guther. They asked the bold one if to buy he wished—the ruler of the Gots—his life with gold.\footnoteB{The Huns ask Guther (it is clear that “ruler of the Gots” refers to him, cf. 1, 3, 10) if he wishes to ransom Hain. He instead responds with the following:}\evb
\evg


\bvg
\bva “Hjarta skal mér Hǫgna \hld\ í hęndi liggja &
blóðugt, ór brjósti \hld\ skorit baldriða, &
saxi slíðrbęitu, \hld\ syni þjóðans.”\eva

\bvb (Guther quoth:) \\ “The heart of Hain shall lie me in the hands: bloody from the breast—cut from the bold rider with a slide-biting sax\footnoteB{i.e. a short-sword with a blade so sharp that it draws blood when one slides the finger across it.}—of the son of the sovereign.”\evb
\evg


\bvg
\bva Skǫ́ru þęir hjarta \hld\ Hjalla ór brjósti &
blóðugt ok á bjóð lǫgðu \hld\ ok bǫ́ru þat fyr Gunnar.\eva

\bvb They cut the heart of Helle out of the breast; bloody on a platter they laid it, and carried it before Guther.\evb
\evg


\bvg
\bva Þá kvað þat Gunnarr, \hld\ gumna dróttinn: &
Hér hęfi ek hjarta \hld\ Hjalla ins blauða, &
ólíkt hjarta \hld\ Hǫgna ins frǿkna, &
es mjǫk bifask \hld\ es á bjóði liggr; &
bifðisk hǫlfu męirr \hld\ es í brjósti lá!\eva

\bvb Then quoth that Guther, the lord of men: “Here have I the heart of Helle the soft—unlike the heart of Hain the bold!—which much trembles, when on the platter it lies; it trembled twice as much, when in the breast it lay.”\evb
\evg


\bvg
\bva Hló þá Hǫgni \hld\ es til hjarta skǫ́ru &
kvikvan kumblasmið \hld\ kløkkva hann sízt hugði; &
blóðugt þat á bjóð lǫgðu \hld\ ok bǫ́ru fyr Gunnar.\eva

\bvb Hain laughed then, when to the heart they cut on the living wound-smith \ken{warrior}; he thought least of sobbing. Bloody on a platter they laid it, and carried it before Guther.\evb
\evg


\bvg
\bva Mę́rr kvað þat Gunnarr, \hld\ Gęir-Niflungr: &
Hér hęfi ek hjarta \hld\ Hǫgna ins frǿkna, &
ólíkt hjarta \hld\ Hjalla ins blauða, &
es lítt bifask \hld\ es á bjóði liggr; &
bifðisk svági mjǫk \hld\ þá’s í brjósti lá!\eva

\bvb Renowned quoth that Guther, the Gore-Nifling: “Here have I the heart of Hain the bold—unlike the heart of Helle the soft!—which little trembles, when on the platter it lies; it trembled not as much, when in the breast it lay.\evb
\evg


\bvg
\bva Svá skaltu, Atli, \hld\ augum fjarri &
sęm munt \hld\ męnjum verða; &
es und ęinum mér \hld\ ǫll of folgin &
hodd Niflunga: \hld\ Lifir-a nú Hǫgni!\eva

\bvb Thus shalt thou, Attle, be as far from the eyes, as thou wilt from the neck-rings. ’Tis by me alone all concealed, the hoard of the Niflings—now Hain lives not!\evb
\evg


\bvg
\bva Ęy vas mér týja \hld\ meðan vit tvęir lifðum, &
nú es mér ęngi \hld\ es ęinn lifi’k; &
Rín skal ráða \hld\ rógmalmi skatna, &
svinn, ǫ́skunna \hld\ arfi Niflunga.\eva

\bvb I was ever in doubt when we two lived; now I am not when alone I live. The Rhine shall rule the strife-ore of princes \ken{gold}, swift, the os-born inheritance of the Niflings.\evb
\evg


\bvg
\bva Í veltanda vatni \hld\ lýsask valbaugar &
hęldr an á hǫndum gull \hld\ skíni Húna bǫrnum.\eva

\bvb In tumbling water the Welsh bighs gleam, rather than gold might shine on the hands of the children of Huns.”\evb
\evg

...

\bvg
\bva Ęldi gaf hón alla \hld\ es inni vǫ́ru &
ok frá morði þęira Gunnars \hld\ komnir vǫ́ru ór Myrkhęimi; &
forn timbr fellu, \hld\ fjarghús ruku, &
bǿr Buðlunga, \hld\ brunnu ok skjaldmęyjar, &
inni aldrstamar, \hld\ hnigu í ęld hęitan.\eva

\bvb To the fire she gave all those who were inside, who from their murder of Guther were come out of Mirkham. Ancient timbers fell, great houses smoked—the settlement of the Buthlungs—burned the shield–maidens likewise; inside aged trunks bowed into hot fire.\evb
\evg


\bvg
\bva Fullrǿtt’s umb þetta; \hld\ fęrr ęngi svá síðan &
brúðr í brynju \hld\ brǿðra at hęfna; &
hón hęfir þriggja \hld\ þjóðkonunga &
banorð borið, \hld\ bjǫrt, áðr sylti.\eva

\bvb ’Tis fully told of this; none hence fares so, a bride in byrnie, her brothers to avenge. She has of three great kings borne the bane-word,\footnoteB{i.e. ‘She has slain three great kings.’ This expression and its Germanic and Indo-European relatives is discussed in detail in \textcite{Watkins1995}[417--422].} bright woman, before she may die.\evb
\evg


\bvg
\bva Enn segir gleggra í Atlamálum inum grǿnlenskum.\eva

\bvb Yet this is told more clearly in the Greenlendish Speeches of Attle.\evb
\evg

%	\book{The Third Lay of Guthrun. (\emph{Guðrúnarkviða III})}\bookStart

Herkja hét ambǫ́tt Atla; hón hafði verit frilla hans. Hón sagði Atla at hón hefði sét Þjóðrek ok Guðrúnu bæði saman. Atli var þá allókátr. Þá kvað Guðrún: 

Hark was named the female thrall of Attle; she had been his concubine. She told Attle that she had seen Thederick and Guthrun both together. Attle was then wholly displeased. Then Guthrun quoth:

\bva “Hvat es þér, Atli? \hld Æ, Buðla sonr, \\%M
es þér hryggt í hug; \hld hví hlær þú æva? \\%M
Hitt myndi ǿðra \hld jǫrlum þykkja \\%M
at við menn mæltir \hld ok mik sæir.”\\%E

What is with thee, Attle? Always, son of Buthel, thou art sad in mind; why dost thou never laugh? TO-DO

\bva “Tregr mik þat, Guðrún, \hld Gjúka dóttir: \\%M
Mér í hǫllu \hld Hęrkja sagði \\%M
at þit Þjóðrekr \hld undir þaki svæfið \\%M
ok léttliga \hld líni vęrðið.”\\%E

\bvb This troubles me, Guthrun, daughter of Yivick: To me in the hall, Hark has said, that thou and Thederick slept 'neath thatched roof, and ye lightly warded the linen.\footnote[1]
\footnote[1] That is, they threw off their clothes and slept together.

\bva “Þér mun’k alls þęss \hld ęiða vinna \\%M
at inum hvíta \hld helga stęini. \\%M
at ek við Þjóðmar \hld þat-ki átta’k \\%M
es vǫrðr né verr \hld vinna knátti.\\%E

Nema ek halsaða \hld hęrja stilli,
jǫfur óneisinn, \hld ęinu sinni;
aðrar vǫ́ru \hld okkrar spękjur
es við hǫrmug tvau \hld hnigum at rúnum.

\bva Hér kom Þjóðrekr \hld með þrjá tǫgu,
lifa þęir né ęinir, \hld þriggja tega manna;
hrinktu mik at brǿðrum \hld ok at brynjuðum,
hrinktu mik at ǫllum \hld á hǫfuðniðjum.

\bva Sęntu at Saxa, \hld sunnmanna gram;
hann kann hęlga \hld hver vellanda;”
sjau hundruð manna \hld í sal gengu
áðr kvæn konungs \hld í kętil tǿki.

\bvb Send for Saxe, the prince of southmen; he knows how to hallow a swelling cauldron!” — Seven hundred men went into the hall, before the wife of the king might touch the kettle.

\bva “Kęmr-a nú Gunnarr, \hld kalli’k-a Hǫgna,
sé’k-a síðan \hld svása brǿðr;
sverði myndi Hǫgni \hld slíks harms reka,
nú verð’k sjǫlf fyr mik \hld synja lýta.”

\bvb “Now Guthhere comes not, I call not on Hain; I see not hence [my] sweet brothers. With sword Hain would drive away such an affronts; now I will for myself disprove the slanders.”

\bva Brá hón til botns \hld bjǫrtum lófa \\%M
ok hón upp of tók \hld jarknastęina: \\%M
Sé nú sęggir \hld sykn em ek orðin \\%M
hęilagliga— \hld hvé sjá hverr velli.\\%E

\bvb She brought her bright palms to the bottom, and she up did take the gemstones: “May men now behold—I am proven innocent through holy means—how this cauldron boils.”

\bva Hló þá Atla \hld hugr í brjósti \\%M
es hann hęilar sá \hld hęndr Guðrúnar: \\%M
Nú skal Hęrkja \hld til hvers ganga, \\%M
sú er Guðrúnu \hld grandi vænti. \\%E

\bvb Then the heart of Ettle laughed in his breast, when he saw the hands of Guthrun unscathed: “Now Hark shall go to the cauldron, she who hoped to cause injury to Guthrun.”

\bva Sá-at maðr armligt, \hld hvęrr es þat sá at, \\%M
hvé þar á Hęrkju \hld hęndr sviðnuðu; \\%M
lęiddu þá męy \hld í mýri fúla, \\%M
svá þá Guðrún \hld sinna harma.\\%E

\bvb No man saw something so pitiful, of each that saw that: how there on Hark the hands were scorched. Then they led the maiden into the foul bog; thus Guðrún was reconstituted for her affronts.

%	\bookStart{The Speeches of Sighdrive}[Sigrdrífumǫ́l]

% Introduction

Many of the verses are quoted in \VolsungaSaga, but notably the two prayer-verses are missing; possibly an instance of Christian censorship. TODO

\bvg {\small [Sighdrive quoth:]}
\bva „Lęngi ek svaf, \hld\ lęngi ek sofnuð vas, &
\ind lǫng eru lýða lę́; &
Óðinn því vęldr \hld\ es ęigi mátta’k &
\ind bregða blundstǫfum.“\eva

\bvb “Long I slept, long was I asleep, long are the deceits”\evb
\evg

BPG
BPA Sigurðr sęttisk niðr ok spyrr hana nafns. Hón tók þá horn fullt mjaðar ok gaf hǫ́num minnisvęig.EPA

BPB Siward set himself down, asking for her name. Then she took a horn full of mead, and gave him a mind-draught:EPB
EPG


\bvg
\bva Hęill Dagr, \hld\ hęilir Dags synir, &
\ind heil Nǫ́tt ok nipt! &
Óręiðum augum \hld\ lítið okkr þinig &
\ind ok gefið sitjǫndum sigr!\eva

\bvb “Hail \inx[P]{Day}! Hail the sons of Day!\footnoteB{TODO. Who?} Hail Night and [her] kinswoman \ken*{= Earth}!\footnoteB{According to \Gylfaginning\ TODO, Earth is the daughter of Night.} With unwrathful eyes look ye upon us two, and give the sitting ones \ken*{= us} victory.\evb
\evg


\bvg
\bva Hęilir ę́sir, \hld\ hęilar ǫ́synjur, &
\ind hęil sjá in fjǫlnýta fold! &
Mál ok manvit \hld\ gefið okkr mę́rum tvęim &
\ind ok lę́knishęndr meðan lifum!\eva

\bvb Hail the \inx[G]{Ease}! Hail the \inx[G]{Ossens}! Hail this bountiful fold \name{= Earth}! Speech and \inx[C]{manwit} give ye us renowned two, and \inx[C]{healing-hands}\footnoteB{Hands with the power to heal (perhaps supernaturally). This word also occurs in the semi-Christianized prayer on a c. 1300 stick from Ribe, Denmark (signum DR EM85;493).} while we live.”\evb
\evg


BPG
BPA Hon nefndisk Sigrdrífa ok var valkyrja. Hon sagði, at tveir konvngar bǫrðusk. Hét annarr Hjalmgunnarr; hann var þá gamall ok inn mesti hermaðr, ok hafði Óðinn hánum sigri heitit. En annarr hét Agnarr, · Auðu bróðir · er vę́tr engi · vildi þiggja. Sigrdrífa felldi Hjalmgunnar í orrostunni. En Óðinn stakk hana svefnþorni í hefnd þess ok kvað hana aldri skyldu síðan sigr vega í orrostu, ok kvað hana giftask skyldu, „en sagða’k hánum at strengða’k heit þar í mót, at giptask øngom þeim manni er hrę́ðask kynni.“ Hann segir ok biðr hana kenna sér speki ef hon\footnoteA{hánom ms.} vissi tíðendi ór ǫllum heimum. Sigrdrífa kvað:EPA

BPB She called herself Sighdrive and was a walkirrie. She said that two kings fought. One of them was called Helmguther; he was then old and the greatest harrier, and Weden had promised him victory. But another one was called Eyner, Eade’s brother, who in no way wished to accept.\footnoteB{i.e. ‘wished to lose’ TODO} Sighdrive felled Helmguther in the battle, but Weden pierced her with the sleeping-thorn as revenge for that, and said that she would never thenceforth win victory in battle, and said that she must marry, “but I told him that I made a vow against that, to marry no man who could be frightened.” He [i.e. Siward] speaks and asks her to teach him wisdom, if she knew any tidings out of all the \inx[C]{Home}[Homes]. Sighdrive quoth: EPB
EPG


\bvg
\bva „Bjór fǿri’k þér, \hld\ brynþings apaldr, &
magni blandinn \hld\ ok męgintíri, &
fullr ’s hann ljóða \hld\ ok líknstafa, &
góðra galdra \hld\ ok gamanrúna.\eva

\bvb Beer I bring thee—apple-tree of the byrnie-\inx[C]{Thing} \ken{battle > warrior}—mixed with might, and might-glory; it is filled with \inx[C]{leed}[leeds], and grace-staves; good \inx[C]{galder}[galders], and pleasure-\inx[C]{rune}[runes].\evb
\evg


\bvg
\bva Sigrúnar skalt kunna, \hld\ ef vilt sigr hafa, &
\ind ok rísta á hjalti hjǫrs, &
sumar á véttrimum, \hld\ sumar á valbǫstum, &
\ind ok nęfna tysvar Tý.\eva

\bvb Victory-runes shalt thou know, if thou wilt have victory, and carve on the hilt of the sword; some on weight-rims;\footnoteB{Unclear.} some on walbasts\footnoteB{Possibly the sword-pommel, the word also occurs in \HelgakvidaHjorvardssonar\ 9.}, and name \inx[P]{Tue} twice.\evb
\evg


\bvg
\bva Ǫlrúnar skalt kunna \hld\ ef þv vill aɴarſ qvęn &
\ind vęli t þic i trꝩgd ef þv trvir. &
a horni ſcal þęr riſta \hld\ oc a handar baki &
\ind oc merkia a nagli nꜹþ.\eva

\bvb Ale-runes shalt thou know, if TODO\evb
\evg


\bvg
\bva Full skal signa \hld\ ok við fári sjá &
\ind ok verpa lauki í lǫg; &
\edtext{þá þat vęitk, \hld\ at þér verðr aldri &
męini blandinn mjǫðr.}{\lemma{þá ... mjǫðr}\Bfootnote{\emph{thus} \VolsungaSaga, \emph{om.} \Regius}}\eva

\bvb TODO\evb
\evg

...


\bvg
\bva Þá mę́lti \hld\ Míms hǫfuð &
\ind fróðligt it fyrsta orð, &
\ind ok sagði sanna stafi.\eva

\bvb Then spoke the head of Mime learnedly the first word, and said true staves:\evb
\evg


\bvg
\bva Á skildi kvað ristnar \hld\ þęim’s stęndr fyr skínanda goði, &
á ęyra Árvakrs, \hld\ ok á Alsvinnz hófi, &
á því hvéli es snýz \hld\ undir ręið Hrungnis, &
á Slęipnis tǫnnum \hld\ ok á slęða fjǫtrum, &
á bjarnar hrammi \hld\ ok á Braga tungu, &
á ulfs klóm \hld\ ok á arnar nęfi, &
á blóðgum vę́ngjum \hld\ ok á brúar sporði, &
á lausnar lófa \hld\ ok á líknar spori, &
á glęri ok á gulli \hld\ ok á gumna hęillum, &
í víni ok virtri \hld\ ok vilisessi.\eva

\bvb On a shield it said were carved [runes]—[the shield] that stands before the shining god\footnoteB{According to \Grimnismal\ 39 the sun is covered by a shield, protecting the earth from its heat. Without it, the whole world would burn up.} \ken{sun}—[also] on the ear of Yorewaker, on the hoof of Allswith,\footnoteB{The two horses that pull the sun across the heavens; cf. \Grimnismal\ 38.} on that wheel which turns beneath the chariot of Rungner, on the teeth of Slopner, and on the fetters of sleds, on the paw of the bear, and on the tongue of Bray, on the claws of the wolf, and on the beak of the eagle, on bloody wings, and on the supports of the bridge, on the palm of release, and the track of grace, on glass and on gold, and on the good healths of men, in wine and beerwort, and on the comfortable seat.\evb
\evg


\bvg
\bva Á Gungnis oddi \hld\ ok á Grana brjósti, &
á nornar nagli \hld\ ok á nęfi uglu; &
allar vǫ́ru af skafnar, \hld\ þę́r es vǫ́ru á ristnar, &
\ind ok hvęrfðar við inn hęlga mjǫð &
\ind ok sęndar á víða vega.\eva

\bvb On the point of Gungner, and on the breast of Grane, on the nail of a norn, and on the beak of an owl;—all were shaven off—those that were carved on—and thrown into the holy mead, and sent on wide ways:\evb
\evg


\bvg
\bva Þę́r ’ró með ǫ́sum, \hld\ þę́r ’ró með ǫlfum, &
sumar með vísum vǫnum, \hld\ sumar hafa męnskir męnn.\eva

\bvb They are among Ease, they are among Elves; some among wise Wanes, some are had by manly men.\evb
\evg


%	\book{The Speeches of Hildbrand}\bookStart

% Introduction

{\small For the text of original poem, I do not present the manuscript text, but rather a standardized text of my own. I have however aimed to generally follow the dialect of the manuscript, rather than present a standardized Old High German or Old Saxon. The rules of normalization have been as follows:
Vowels:
> Ms. \emph{ae}, \emph{ei} and \emph{e}, where etymologically from \emph{ai}, have been normalized as \emph{ei}.
> Ms. \emph{o} and \emph{ao}, where etymologically from \emph{au}, have been normalized as \emph{ao}. This may be somewhat controversial.
> \emph{ostar}, \emph{Otachre} > \emph{aostar}, \emph{Aotachre}).
> Ms. \emph{uo} and \emph{o}, where etymologically from long \emph{ō}, have been normalized as \emph{ō}.
Consonants:
> Ms. \emph{r} and \emph{w}, where etymologically from \emph{hw} and \emph{hr}, have been thus normalized. That this was the case in the original poem is obvious; such words never alliterate with \emph{w} or \emph{r}, but only with \emph{r}, as can be most definitively seen in lines 56 (ms.: \alst{h}eremo ... \alst{h}rusti) and 66 (ms.: \alst{h}ewun \alst{h}armlicco \alst{h}uitte). If this were not enough, the retention in the ms. of the \emph{h} at previously given places is yet further support.
> Ms. \emph{tt}, where etymologically from \emph{t}, has been thus normalized.
> Ms. \emph{ƿ} (wynn), \emph{u} and \emph{uu}, where representing \emph{w}, have been thus normalized.

The pronoun which exclusively appears in the ms. as \emph{her} ‘he’ has been so kept, rather than normalized to the standard OHG \emph{er}.
The punctuation of the original (entirely consisting of interpuncts) has not been retained.}



\bvg
\bva[0] Ik gihōrta dat seggen &
dat sih urhettun \hld einon mōtīn &
Hiltibrant enti Hadubrant \hld untar heriun tweim &
sunufatarungo \hld iro saro rihtun &
garutun \edtext{sie}{\Afootnote{se \HildMS}} iro gūdhamun \hld gurtun sih iro swert ana &
helidos ubar \edtext{hringa}{\Afootnote{ringa \HildMS}} \hld dō sie to dero hiltiu ritun\eva

\bvb[0] I heard it said, that two contenders alone did meet: Hildbrand and Hathbrand, under two hosts. Son and father ordered their armour, readied their war-cloth, girded their swords on, the heroes over the mail, when to that battle they rode.\evb
\evg


\bvg
\bva[0] Hiltibrant gimahalta Heribrantes sunu \hld her was hērōro man &
ferahes frōtōro \hld her frāgēn gistōnt &
fōhēm wortum \hld \edtext{hwer}{\Afootnote{wer \HildMS}} sin fater wāri &
fireo in folche \hld {[...]} &
{[...]} \hld eddo \edtext{hwelīhhes}{\Afootnote{welihhes \HildMS}} cnōsles dū sīs &
ibu dū mī ēnan sagēs \hld ik mī de odre wēt &
chind in \edtext{chunincrīche}{\Afootnote{chunnincriche \HildMS}} \hld chūd ist mīn al irmindeot\eva

\bvb[0] Hildbrand spoke, Harbrand's son — he was the hoarier man, more learned in life, — he began to ask with few words, who his father might be, of men in the folk, [...] “or of which lineage thou be; if thou me one say, I the others will know; child, in the kingdom, known to me are all great men.”\evb
\evg


\bvg
\bva[0] Hadubrant gimahalta \hld Hiltibrantes sunu &
dat sagetun mī ūsere liuti &
alte enti frōte \hld dea ērhina wārun &
dat Hiltibrant hēti min fater \hld ih heitu hadubrant &
forn her aostar giweit \hld flaoh her Aotachres nīd &
hina miti Deotrihhe \hld enti sīnero degano filu &
her furlēt in lante \hld luttila sitten &
brūt in būre \hld barn unwahsan &
arbeolaosa \hld her reit aostar hina &
des sid Deotrihhe \hld darba gistōntum &
\edtext{fateres}{\Afootnote{fatereres \HildMS}} mīnes \hld dat was sō friuntlaos man &
her was Aotachre \hld ummet tirri &
degano dechisto \hld unti \edtext{Deotrihhe}{\Afootnote{\emph{add.} darba gistontun \HildMS}} &
her was eo folches at ente \hld imo was eo \edtext{fehta}{\Afootnote{peheta \HildMS}} ti leob &
chūd was her \hld chōnēm mannum &
ni wāniu ih iu līb habbe\eva

\bvb[0] Hathbrand spoke, Hildbrand's son: “It told me our people, the old and learned, those who earlier lived, that Hildbrand was called my father — I am called Hathbrand, — he previously hurried east; he fled Edwaker's hate, thither with Thedrich, and his multitude of thanes. He left in the land a little one to stay, a bride in the bower, a bairn ungrown, without inheritance; he rode east thither, as Thedrich was in great need of my father — that was such a friendless man. He was to Edwaker exceptionally hostile, the dearest of thanes under Thedrich. He was ever at the front of the troop; ever did the fight gladden him; known was he among keen men. — I ween not that he have life.”\evb
\evg


\bvg
\bva[0] weitu irmingot {\small [quad hiltibrant]} \hld obana ab hebane &
dat dū neo dana halt mit sus sibban man &
dinc ni gileitōs &
want her dō ar arme \hld wuntane baoga &
cheisuringu gitān \hld so imo sie der chuning gab &
huneo truhtin \hld dat ih dir it nū bī huldī gibu\eva

\bvb[0] “I call on God as witness, [quoth Hildbrand], above in heaven, that thou never with such a close man once more lead dispute.” Unwound he then from his arm some twisted bighs, made from imperial coin, which the king once gave him, the lord of the Huns: — “This I now give thee as pledge.”\evb
\evg


\bvg
\bva[0] Hadubrant gimahalta \hld Hiltibrantes sunu &
mit gēru scal man \hld geba infāhan &
ort widar orte \hld [...] &
dū bist dir altēr hun \hld ummet spāhēr &
spenis mih mit dīnem wortum \hld wili mih dinu speru werpan &
bist alsō gialtēt man \hld sō dū ēwīn inwit fōrtōs &
dat sagetun mi \hld sēolīdante &
westar ubar wentilsēo \hld dat man wīc furnam &
tōt ist Hiltibrant \hld Heribrantes sunu\eva

\bvb[0] Hathbrand spoke, Hildbrand's son: “With spear shall one earn gifts, point against point! Thou art, old Hun, exceptionally clever; thou lurest me with thy words, wilt thou at me hurl thy spear! Thou art thus old, though thou ever deceit hast worked. — It told me seafarers, heading west o’er the Wendle-sea <= Mediterranean>, that war took that man: — dead is Hildbrand, Harbrand's son!”\evb
\evg


\bvg
\bva[0] Hiltibrant gimahalta \hld Heribrantes sunu &
wela gisihu ih in dīnēm hrustim &
dat dū habēs heime \hld hērron gōten &
dat dū noh bī desemo rīche \hld reccheo ni wurti\eva

\bvb[0] Hildbrand spoke, Harbrand's son: “I see well on thy equipment, that thou hast a good lord at home, that thou yet in his reign art not become an exile.\evb
\evg


\bvg
\bva[0] welaga nu waltant got {\small [quad hiltibrant]} \hld weiwurt skihit &
ih wallōta sumaro enti wintro \hld sehstic ur lante &
dar man mih eo scerita \hld in folc sceotantero &
sō man mir at burc einīgeru \hld banun ni gifasta &
nu scal mih swāsat chind \hld swertu haowan &
bretōn mit sīnu billiu \hld eddo ih imo ti banin werdan &
doh maht dū nū aodlīhho \hld ibu dir dīn ellen taoc &
in sus hēremo man \hld hrusti giwinnan &
raoba birahanen \hld ibu du dar einīg reht habēs\eva

\bvb[0] Well now, wielding God, [quoth Hildbrand], woeful Weird passes. I wallowed for summers and winters sixty, out of the land, where one ever placed me in the troop of shooters; thus one at no fortress my bane did inflict. Now shall my own child hew at me with sword; beat down with blade, or I become his bane; — yet canst thou now easily, if thy courage avail thee, from such a hoary man win the equipment, bear away the booty, if thou thereto have any right.\evb
\evg


\bvg
\bva[0] der sī doh nu argōsto {\small [quad hiltibrant]} aostarliuto &
der dir nū wīges warne \hld nū dih et sō wel lustit &
gudea gimeinun \hld niuse der mōti &
hwedar sih \edtext{hiutu dēro}{\Afootnote{dēro hiuti \HildMS}} hregilo \hld hrōmen mōti &
eddo desero brunnōno \hld beidero waltan\eva

\bvb[0] Yet now he may be the weakest, [quoth Hildbrand], of the eastern peoples, who would refuse thee the fight, when thou so greatly cravest to struggle together. Try he who might, which one today of his arms may boast, or both of these byrnies wield!”\evb
\evg


\bvg
\bva[0] dō lietun sie aerist \hld askim scrītan &
scarpēn scūrim \hld dat in dem sciltim stōnt &
dō stōptun tosamane \hld staimbort hlūdun &
hewun harmlīcco \hld hwīte scilti &
unti imo iro lintūn \hld luttilo wurtun &
giwigan miti wābnum \hld [...]\eva

\bvb[0] Then they first let their ash-spears glide, in a harsh torrent, that they stuck in the shields. Then charged they into each other — the war-boards [SHIELDS] resounded — struck they bitterly the white shields, until their linden-planks [SHIELDS] became little, worn down by the weapons, [...]\evb
\evg

%	\include{books/Fight at Finnsbury.tex}
\end{document}
