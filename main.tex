% This file should be compiled with XeLaTeX.

\documentclass[openany]{memoir}

% Font
\usepackage{fontspec}
\setmainfont{Junicode}[
	Extension=.ttf,
	BoldFont=*-Bold,
	ItalicFont=*-Italic]

% Underline that does not skip descender
% Should be called \nsunderline
 
% Packages
\usepackage{xparse}
\usepackage{geometry} %For margins.

% Formatting
\usepackage{reledmac}

% Define verse counters
\newcounter{versea}
\newcounter{verseb}

\begin{document}

% Book and chapter commands
	\NewDocumentCommand{\chapterStart}{o O{Chap}}{% Command at the start of chapter
		\setcounter{versea}{0}%
		\setcounter{verseb}{0}%
		\stepcounter{chapter}%
		\IfNoValueF{#1}{%
			\begin{center}%
			\textbf{#2. \arabic{chapter}} \\
			{#1}\end{center}%
		}%
	}

	\newcommand{\bookStart}{% Command at the start of book
		\setcounter{chapter}{0} % Set chapter count to zero.
		\chapterStart{}%
	}

% Verse format commands
	\NewDocumentCommand{\bvg}{o}{% Begin verse group
		\begin{ledgroup}%
		\beginnumbering%
	}

	\NewDocumentCommand{\bva}{o}{% Begin verse a
		\stepcounter{versea}%
		\begin{large}\begin{stanza}% Begin stanza
		% Add verse number
		\IfNoValueT{#1}{%
			\newdimen\width% Create dimension width
			\setbox0=\hbox{\textbf{\arabic{versea} }}% Create hbox with the verse label
			\width=\wd0 \advance\width by \dp0% Set width to the width of the box
			\hspace{-\the\width}% Add a hspace = -(width)
			\textbf{\arabic{versea} }% Add the verse number
		}
	}
	\NewDocumentCommand{\eva}{o}{% End verse a
		\& \end{stanza}\end{large}\endnumbering% End reledmac numbering and stanza
		\vspace{1.5mm}% Vertical space
	}
	
	\NewDocumentCommand{\bvb}{o}{% Begin verse b
		\stepcounter{verseb}%
		\IfNoValueT{#1}{%
			\textbf{\arabic{verseb} }%
		}%
	}
	\NewDocumentCommand{\evb}{o}{% End verse b
		% Nothing (for now)
	}
	
	\NewDocumentCommand{\evg}{o}{% End verse group
		\end{ledgroup}%
		\vspace{1cm}%
	}
	
% Note formatting
	% Side note margin
	\setlength{\ledlsnotesep}{2 \ledlsnotesep}
	
	% Make foot notes paragraphs
	\Xarrangement[A]{paragraph}
	
% Poem formatting
	% First line number at 3
	\firstlinenum{2}
	\linenumincrement{2}
	
	% Stanza indentation (required for \astanza to work)
	\setstanzaindents{5, 2, 2}
	\setcounter{stanzaindentsrepetition}{2}

	% Mark cæsura.
	\newcommand{\hld}{\hspace{5mm} }%\leavevmode\unskip\quad\ignorespaces}

	% Indent lines (in Ljóðaháttr or Galdralag).
	\newcommand{\ind}{%
		\hspace{1.5em}%
	}

	% Mark alliteration. This might not be present in the final version.
	\NewDocumentCommand{\alst}{m}{%
		\underline{#1}%
	}

% Index link command
%	{#1}\textsuperscript{†}%	Dagger at the end
%	\nsunderline{#1}%	Underline

	\NewDocumentCommand{\inx}{m}{%
		{#1}%\textsuperscript{†}%
	}

% Sigla
	% Authors
	\newcommand{\Finnur}{%
		Finnur%
	}
	\newcommand{\Snorri}{%
		Snorri%
	}
	\newcommand{\CV}{%
		Cleasby-Vigfússon%
	}
	
	%Modern books
	\newcommand{\FaulkesEdda}{%
		\emph{SnE} 2005%
	}
	
	% Manuscripts
	\newcommand{\Regius}{% Codex Regius (of the poetic edda)
		\emph{R}%
	}
	\newcommand{\Hauksbok}{% Hauksbok
		\emph{H}%
	}
	\newcommand{\GylfMS}{% For referring to Gylfaginning manuscripts when verses are attested there.
		\emph{G}%
	}
	\newcommand{\RegiusProse}{% Codex Regius of the Prose Edda
		\emph{R\textsubscript{2}}%
	}
	\newcommand{\Trajectinus}{% Codex Trajectinus
		\emph{T}%
	}
	\newcommand{\Wormianus}{% Codex Wormianus
		\emph{W}%
	}
	\newcommand{\Upsaliensis}{% Codex Upsaliensis
		\emph{U}%
	}
	\newcommand{\HildMS}{% For referring to the Hildebrandslied manuscript.
		\emph{ms.}%
	}
	
	% Texts
	\newcommand{\Hildebrandslied}{% Speeches of Hildbrand
		\emph{Hild}%
	}
	\newcommand{\Beowulf}{% Beewolf
		\emph{Bee}%
	}
	\newcommand{\Ynglingatal}{% Tally of the Inglings
		\emph{Ing}%
	}
	\newcommand{\Hervarar}{% Saw of Harware
		\emph{HarS}%
	}
	\newcommand{\Gylfaginning}{% For referring to Gylfaginning as a text
		\emph{Yilf}% F
	}
	\newcommand{\Haleygjatal}{% Tally of the Hallowlendings
		\emph{Hal}%
	}
	\newcommand{\Rigsthula}{% Thule of Righ
		\emph{Righ}%
	}
	\newcommand{\Vafthrudnismal}{% Speeches of Webthrithner
		\emph{Web}%
	}
	\newcommand{\FraLoka}{% From Locke
		\emph{FrL}%
	}

% Books
%	\book{The Spae of the Wallow. (Vǫluspǫ́)}\bookStart

% Introduction.

\small{\emph{Vǫluspǫ́}, or the "Spae of the Wallow\footnotemark[1]", is the first poem of the R. manuscript.}
\footnotetext[1]{The Eng. equivalent of ON. \emph{vǫlva} 'seeress'; 'prophetess'. See index.}

% Greeting to the sons of Homedale, asking of Weden.

\bva Hljóðs bið'k allar \hld hęlgar kindir, \\%M
męiri ok minni \hld mǫgu Hęimdallar; \\%M
vildu at, Valfǫðr, \hld vęl fram tęlja'k \\%M
forn spjǫll fira, \hld þau's fręmst of man?

\bvb Of silence I bid all holy kins\footnotemark[1], [the] greater and lesser sons of Homedale\footnotemark[2]. Wilt thou, Leader of the Slain}\footnotemark[3], that I well tell forth the ancient sayings of firs\footnotemark[4], those I foremost recall?\footnotemark[5] \\
\footnotetext[1]{The 'Holy kins', according to FJ referring to the gods, but it might "simply" be an allusion to the divine origin of men.}
\footnotetext[2]{Cf. with \emph{Rþ}, wherein Rig (Homedale) sires the \emph{greater and lesser} human races (\emph{earls}, \emph{churls} and \emph{thralls}). The Wallow (speaking through the poet) addresses not just the human audience, but also the gods.}
\footnotetext[3]{FJ believes the name (for Weden) is chosen intentionally, as it refers to the final fight at the \textbf{Twilight of the Powers.}}
\footnotetext[4]{"Of men".}
\footnotetext[5]{Cf. \emph{Web} 34, 35 with very similar phrasing.}\\%E

\bva Ek man jǫtna \hld ár of borna, \\%M
þá es forðum \hld mik fǿdda hǫfðu; \\%M
níu man'k hęima, \hld níu íviðjur\footnotemark[4], \\%M
mjǫtvið mæran \hld fyr mold neðan.
\footnotetext[5]{Previously read \emph{íviði}, but closer study of R has disproven this. See \emph{Gripla} 3, pp. 227–28.}\\%E

\bvb I recall ettins, born of yore, those who earlier had nourished me. Nine Homes I recall, nine Inwithies; the famous Metwood, beneath the earth.

\bva Ár vas alda \hld þar’s Ymir byggði,\footnotemark[6] \\%M
vas-a sandr né sær, \hld né svalar unnir; \\%M
jǫrð fansk æva \hld né upphiminn; \\%M
gap vas ginnunga, \hld ęn gras hvęrgi.
\footnotetext[6]{\emph{Gylf} has \emph{þat’s ekki vas}.}\\%E

\bvb It was beginning of elds, where Yime dwelled\footnotemark[8]; there was not sand nor sea, nor cool waves. The earth was never found, nor up-heaven; a gap was of ginnings, but grass nowhere. 
\footnotemark[8]{Gylf. “that nothing was”.}

\bva Áðr Burs synir \hld bjǫðum of ypðu, \\%M
þęir es Miðgarð \hld mæran skópu; \\%M
sól skein sunnan \hld á salar stęina; \\%M
þá vas grund gróin \hld grǿnum lauki.\\%E

\bvb Ere Bur's sons did lift the flatlands, they who shaped the renowned Midyard; sun shone from the south on the stones of the hall, then the ground was grown with green leek.

\bva Sól varp sunnan, \hld sinni mána,\footnotemark[10] \\%M
hęndi hinni hǿgri \hld um himinjǫður; \\%M
sól þat né vissi, \hld hvar hon sali átti; \\%M
(stjǫrnur þat né vissu, \hld hvar þær staði ǫ́ttu); \\%M
máni þat né vissi, \hld hvat hann męgins átti.
\footnotetext[10]{At times translated as "its moon". This cannot be correct, as \emph{máni} 'moon' is masculine, while \emph{sinni}, dative singular of \emph{sínn} 'its (reflexive)' is feminine.}\\%E

\bvb Sun, the companion of Moon, cast from the south her right hand about heaven's rim; Sun knew not where she had halls, stars knew not what places they had, Moon knew not what he of power had.

\bva Þá gingu ręgin ǫll \hld á rǫkstóla, \\%M
ginnhęilǫg goð, \hld ok gættusk of þat.\footnotemark[20] \\%M
\footnotetext[20]{Cf. 9:1–4, 23:1–4, 25:1–4; two long-lines containing a question seem to be missing here.}\\%E

\bvb Then the Powers† all went onto the rake-seats†, the gin-holy† gods, and together took counsel of this:

\bva Nótt ok niðjum \hld nǫfn of gǫ́fu, \\%M
morgin hétu \hld ok miðjan dag, \\%M
undurn ok aptan, \hld ǫ́rum at tęlja.\footnotemark[22]
\footnotetext[22]{Cf. \emph{Web} 23, 25.}\\%E

\bvb To night and [her?] descendants they gave names; morning they named, and mid-day, undern (= mid-afternoon) and evening, for to reckon the years.

\bva Hittusk æsir \hld á Iðavęlli, \\%M
þęir's hǫrg ok hof \hld hótimbruðu; \\%M
afla lǫgðu, \hld auð smíðuðu, \\%M
tangir skópu \hld ok tól gęrðu.\\%E

\bvb The \textbf{Eses} met on the \textbf{Idewald}, they who harrows and hofs timbered up high; forges [they] laid, wealth forged, tongs shaped, and tools made.

\bva Tęflðu í túni, \hld tęitir vǫ́ru, \\%M
vas þeim véttugis \hld vant ór golli, \\%M
unz þríar kvǫ́mu \hld þursa męyjar, \\%M
ámátkar mjǫk, \hld ór Jǫtunhęimum.

\bvb They played \textbf{tables} in the yards, joyous were they, for them was no lack of gold. Until three\footnotemark[21] came, maidens of \emph{thurses}, much terrifying, out of \textbf{Ettinhome}.
\footnotetext[21]{These three \emph{thurse-maidens} are immediately forgotten and never again mentioned (unless they are taken to be the norns in v. 21 — but they would then be introduced twice). — Clearly there is something missing between this verse and the next, detailing the creation of dwarves.}\\%E

\bva Þá gingu ręgin ǫll \hld á rǫkstóla, \\%M
ginnhęilǫg goð, \hld ok gættusk of þat, \\%M
hverr skyldi dverga \hld dróttir skępja \\%M
ór Brimis blóði \hld ok ór Bláins lęggjum.\footnotemark[22]
\footnotetext[22]{The final two long-line vary substantially. \emph{R} has \emph{hverr scyldi duerga drotin scepia or brimis bloði oc or blám leggiom.} \emph{H} has \emph{huerer skylldu duergar drottir skepia or brimi bloðgv ok or Blains leggivm.}}\\%E

\bvb Then the Powers† all went onto the rake-seats†, the gin-holy† gods, and together took counsel of this: Who would shape the multitudes\footnotemark[23] of dwarves, out of the blood of Brime, and out of the legs of Blown?
\footnotetext[23]{alt. "the lord"}

\bva Þar vas Móðsognir\footnotemark[25] \hld mæztr of orðinn \\%M
dverga allra, \hld en Durinn annarr; \\%M
þęir manlíkun \hld mǫrg of gęrðu, \\%M
dvergar í jǫrðu, \hld sęm Durinn sagði.\\%E
\footnotetext[25]{R. \emph{mótsognir}, H. \emph{móðsognir}}

\bvb There did Moodsown become the worthiest of all dwarves, but Dorn [was] second; they made men-likenesses many, dwarves out of the earth, as Dorn said.\footnotemark[25]
\footnotetext[25]{A cryptic verse; \emph{manlíkan} 'man-likeness' is a hapax. It seems to imply that the lower dwarves were shaped out of soil or stone, by the mightiest dwarves, Moodsown and Dorn, themselves shaped from the blood of Brime and the legs of Blown (probably alternative names for \textbf{Yime}). \emph{sęm Durinn sagði} 'as Dorn said' implies that Dorn did not shape the dwarves himself; perhaps, he and Moodsown shaped the first lower dwarves out of stone, and then commanded these to finish the creation?}\\%E

\bva Nýi ok Niði, \hld Norðri, Suðri, \\%M
Austri, Vestri, \hld Alþjófr, Dvalinn, \\%M
Bívurr, Bávurr, \hld Bǫmburr, Nóri, \\%M
Ánn ok Ánarr, \hld Ái, Mjǫðvitnir.\\%E\footnotemark[28]
\footnotetext[28]{The three following verses seem to belong together, since there is no repetition of names. From the last verse of the middle one, it seems that it should have been placed at the end of the list.}

\bva Vęigr ok Gandalfr, \hld Vindalfr, Þráinn, \\%M
Þękkr ok Þorinn, \hld Þrór, Vitr ok Litr, \\%M
Nár ok Nýráðr, \hld nú hęf'k dverga, \\%M
Ręginn ok Ráðsviðr, \hld rétt of talða.\\%E

\bva Fíli, Kíli, \hld Fundinn, Náli, \\%M
Hęptifíli, \hld Hannarr, Svíurr, \\%M
Frár, Hornbori, \hld Frægr ok Lóni, \\%M
Aurvangr, Jari, \hld Ęikinskjaldi.\\%E

\bva Mál es dverga \hld í Dvalins liði \\%M
ljóna kindum \hld til Lofars tęlja, \\%M
þęir es sóttu \hld frá salar stęini \\%M
Aurvanga sjǫt \hld til Jǫruvalla.\\%E\footnotemark[30]
\footnotemark[30]{From the repeated names (Ęikinskjaldi, Ái), and the out-of-place introduction (\emph{mál es dverga...} 'a speech is of dwarves...'), it is clear that this verse and the following are originally separate from the previous three, and are a late (and redundant) addition to the \emph{Wale's Spae}.}

\bva Þar vas Draupnir \hld ok Dolgþrasir, \\%M
Hár, Haugspori, \hld Hlévangr, Glói, \\%M
Skirfir, Virfir, \hld Skáfiðr, Ái, \\%M
Alfr ok Yngvi, \hld Ęikinskjaldi, \\%M
Fjalarr ok Frosti, \hld Finnr ok Ginnarr; \\%M
Þat mun æ uppi, \hld meðan ǫld lifir, \\%M
langniðja-tal \hld til Lofars hafat.\\%E

\bva Unz þrír kvǫ́mu \hld ór því liði \\%M
ǫflgir ok ástkir \hld æsir at húsi, \\%M
fundu á landi \hld lítt męgandi \\%M
Ask ok Emblu \hld ørlǫglausa.\\%E

Until three came out of that host: the mighty and loving Ease at house. They found on land the little availing Ash and Emble, lacking orlay†.

\bva Ǫnd þau né ǫ́ttu, \hld óð þau né hǫfðu, \\%M
lǫ́ né læti \hld né litu góða; \\%M
ǫnd gaf Óðinn, \hld óð gaf Hǿnir, \\%M
lǫ́ gaf Lóðurr \hld ok litu góða.\\%E

\bva Breath they owned not, wode† they had not; [neither] craft nor sound, nor good complexion. Breath gave Weden, wode gave Hean, craft gave Lother, and good complexion.

\bva Ask veit'k standa, \hld hęitir Yggdrasill, \\%M
hǫ́r baðmr, ausinn \hld hvíta auri; \\%M
þaðan koma dǫggvar \hld þær's í dala falla; \\%M
stęndr æ yfir grǿnn \hld Urðar brunni.\\%E

I know an ash stands, called Ugdrassle, a high tree, sprinkled with white mud; thence come the dew-drops which fall in the dales; it stands evergreen over the well of Weird.

\bva Þaðan koma męyjar \hld margs vitandi \\%M
þríar ór þeim sæ, \hld es und þolli stendr; \\%M
Urð hétu ęina, \hld aðra Verðandi, \\%M
skǫ́ru á skíði, \hld Skuld hina þriðju \\%M
þær lǫg lǫgðu, \hld þær líf køru, \\%M
alda bǫrnum, \hld ørlǫg sęggja.\\%E

Thence come maidens, much knowing, three out of the lake which stands beneath the tree\footnote[1]: Weird they call one, the other Werthing—they carved on wooden boards—Shild the third. They laid laws, they chose lives, for the children of men, the orlay† of mortals.
\footnote[1] Lit. “pine”

\bva Þat man hon folkvíg \hld fyrst í hęimi, \\%M
es Gollvęigu \hld gęirum studdu \\%M
ok í hǫll Háars \hld hána bręndu, \\%M
þrysvar bręndu \hld þrysvar borna, \\%M
(opt ósjaldan, \hld þó hon ęnn lifir).\\%E

\bva Hęiði hétu, \hld hvar's til húsa kom, \\%M
vǫlu vęlspáa, \hld vitti hon ganda; \\%M
sęið, hvars kunni, \hld sęið hug lęikinn; \\%M
æ vas hon angan \hld illrar brúðar.\\%E

\bva Þá gingu ręgin ǫll \hld á rǫkstóla, \\%M
ginnhęilǫg goð, \hld ok gættusk of þat, \\%M
hvárt skyldi æsir \hld afráð gjalda, \\%M
eða skyldi goð ǫll \hld gildi ęiga.\\%E

\bvb Then the Powers† all went onto the rake-seats†, the gin-holy† gods, and together took counsel of this:

\bva Flęygði Óðinn \hld ok í folk of skaut; \\%M
þat vas ęnn folkvíg \hld fyrst í hęimi; \\%M
brotinn vas borðvęggr \hld borgar ása, \\%M
knǫ́ttu vanir vígspǫ́ \hld vǫllu sporna.\\%E

\bva Þá gingu ręgin ǫll \hld á rǫkstóla, \\%M
ginnhęilǫg goð, \hld ok gættusk um þat, \\%M
hvęrr hęfði lopt alt \hld lævi blandit \\%M
eða ætt jǫtuns \hld Óðs męy gefna.\\%E

\bvb Then the Powers† all went onto the rake-seats†, the gin-holy† gods, and together took counsel of this:

\bva Þórr ęinn þar vá \hld þrunginn móði, \\%M
hann sjaldan sitr, \hld es slíkt of fregn; \\%M
á gingusk ęiðar, \hld orð ok sǿri, \\%M
mǫ́l ǫll męginlig, \hld es á meðal fóru.\\%E

\bva Vęit hon Hęimdallar \hld hljóð of folgit \\%M
und hęiðvǫnum \hld hęlgum baðmi; \\%M
á sér hon ausask \hld aurgum forsi \\%M
af veði Valfǫðrs. \hld Vituð ér ęnn eða hvat?\\%E

\bva Ęin sat hon úti, \hld þá's hinn aldni kom \\%M
yggjungr ása \hld ok í augu lęit — \\%M
»hvęrs fregnið mik? \hld hví fręistið mín? \\%M
Alt vęit'k, Óðinn, \hld hvar auga falt \\%M
í hinum mæra \hld Mímis brunni;« \\%M
drekkr mjǫð Mímir \hld morgin hvęrjan \\%M
af veði Valfǫðrs. \hld Vituð ér ęnn eða hvat?\\%E

\bvb Lone she sat outside, when the old one came, the \textbf{Ose of Terror}, and looked into [her] eyes. "Why inquirest me? Why temptest me? All I know, Weden, where thine eye thou hidst: in the renowned \textbf{well of Mime}. Mime drinks mead every morning, from the pledge of the \textbf{Leader of the Slain}. Know ye yet, or what?"

\bva Valði hęnni Hęrfǫðr \hld hringa ok męn; \\%M
fekk spjǫll spaklig \hld ok spáganda; \\%M
sá vítt ok of vítt \hld of verǫld hvęrja.\\%E

\bva Sá hon valkyrjur \hld vítt of komnar, \\%M
gǫrvar at ríða \hld til goðþjóðar. \\%M
Skuld hęlt skildi, \hld ęn Skǫgul ǫnnur, \\%M
Gunnr, Hildr, Gǫndul \hld ok Gęirskǫgul; \\%M
nú eru talðar \hld nǫnnur Hęrjans, \\%M
gǫrvar at ríða \hld grund valkyrjur.\\%E

\bva Ek sá Baldri, \hld blóðgum tívur, \\%M
Óðins barni, \hld ørlǫg folgin; \\%M
stóð of vaxinn \hld vǫllum hæri \\%M
mjór ok mjǫk fagr \hld mistiltęinn.\\%E

\bvb I saw \textbf{Balder}'s, the bloody \textbf{tue}'s, the child of Weden's hidden \textbf{orlay}; grown did stand, higher than the meadows, the slender and much fair mistletoe.

\bva Varð af męiði, \hld þęim's mær sýndisk, \\%M
harmflaug hættlig, \hld Hǫðr nam skjóta. \\%M
Baldrs bróðir vas \hld of borinn snimma, \\%M
sá nam, Óðins sonr, \hld ęinnættr vega;\\%E

\bvb Of that tree, which looked slender, became a dangerous harm-flier; Had began to shoot. Balder's brother was born early; that son of Weden, one night old, began to kill.

\bva þó hann æva hęndr \hld né hǫfuð kęmbði, \\%M
áðr á bál of bar \hld Baldrs andskota. \\%M
Ęn Frigg of grét \hld í Fęnsǫlum \\%M
vǫ́ Valhallar. \hld Vituð ér ęnn eða hvat?\\%E

\bvb Hands he never washed, nor head combed, before onto the pyre he did bear the opponent of Balder. But Frigg did lament, in the Fenhalls, the woe of Walhall; know ye yet, or what?

\bva Hapt sá hon liggja \hld und Hveralundi \\%M
lægjarns líki \hld Loka áþękkjan; \\%M
þar sitr Sigyn \hld þęygi of sínum \\%M
veri vęl glýjuð. \hld Vitud ér ęnn eða hvat?\\%E

\bva Ǫ́ fęllr austan \hld of ęitrdala \\%M
sǫxum ok sverðum, \hld Slíðr heitir sú.\\%E

\bva Stóð fyr norðan \hld á Niðavǫllum \\%M
salr ór golli \hld Sindra ættar, \\%M
ęn annarr stóð \hld á Ókólni, \\%M
bjórsalr jǫtuns, \hld ęn sá Brimir hęitir.\\%E

\bva Sal sá hon standa \hld sólu fjarri \\%M
Nástrǫndu á, \hld norðr horfa dyrr; \\%M
falla ęitrdropar \hld inn um ljóra, \\%M
sá ’s undinn salr \hld orma hryggjum.\\%E

\bvb A hall she saw stand, far from the sun, on Corpsestrand; the doors face to the north. Drops of venom fell in through the smoke-vent, that hall is wound by the spines of snakes.

\bva Sér hon þar vaða \hld þunga strauma \\%M
męnn męinsvara \hld ok morðvarga \\%M
ok þanns annars glępr \hld ęyrarúnu. \\%M
Þar sýgr Níðhǫggr \hld nái framgingna; \\%M
slítr vargr vera. \hld Vituð ér ęnn eða hvat?\\%E

\bvb There she sees wade through heavy streams, oath-breaking men and murderwargs, TODO. There sucks Nithehew from corpses passed-on; the warg pulls weres [MEN] apart. Know ye yet, or what?

\bva Austr sat hin aldna \hld í Járnviði \\%M
ok fǿddi þar \hld Fęnris kindir; \\%M
verðr af þeim ǫllum \hld ęinna nøkkurr \\%M
tungls tjúgari \hld í trolls hami.\\%E

\bvb East sat the old woman, in Ironwood, and there nourished the kin of Fenner; TODO

\bva Fyllisk fjǫrvi \hld fęigra manna, \\%M
rýðr ragna sjǫt \hld rauðum dręyra, \\%M
svǫrt var þá sólskin \hld um sumur ęptir, \\%M
veðr ǫll válynd. \hld Vituð ér ęnn eða hvat?\\%E

\bvb It fills itself with the life-force of fey men; reddens the seat of the Powers with red gore. Black become the sunrays in the summers afterwards; the weather all hostile. Know ye yet, or what?

\bva Sat þar á haugi \hld ok sló hǫrpu \\%M
gýgjar hirðir, \hld glaðr Ęggþér; \\%M
gól of hǫ́num \hld í gaglviði \\%M
fagrrauðr hani, \hld sá's Fjalarr hęitir.\\%E

\bvb Sat there on the mound, and struck the harp, the troll-woman's keeper, glad Edgethew; by him crowed, in Gallowwood, a fair-red cock, he who is called Fealer.

\bva Gól of ǫ́sum \hld Gollinkambi, \\%M
sá vękr hǫlða \hld at Hęrjafǫðrs, \\%M
ęn annarr gęlr \hld fyr jǫrð neðan \\%M
sótrauðr hani \hld at sǫlum Hęljar.\\%E

\bvb By the Eses crowed Goldencombe, he who wakes men, at the \textbf{Father of Armies}', but another crows for below the earth, a soot-red cock, at the halls of Hell.

\bva Gęyr Garmr mjǫk \hld fyr Gnipahęlli, \\%M
fęstr mun slitna, \hld ęn freki rinna, \\%M
fjǫlð vęit'k frǿða, \hld framm sé'k lęngra \\%M
of ragna rǫk, \hld rǫmm sigtíva.\\%E

\bvb TODO. Much she knows of wisdom, forth I see yet further; about the fates of the Powers, the mighty ones of the victory-tues.

\bva Brǿðr munu bęrjask \hld ok at bǫnum verða, \\%M
munu systrungar \hld sifjum spilla, \\%M
hart ’s í hęimi, \hld hórdómr mikill, \\%M
skęggǫld, skalmǫld, \hld skildir 'ro klofnir, \\%M
vindǫld, vargǫld, \hld áðr verǫld stęypisk, \\%M
mun ęngi maðr \hld ǫðrum þyrma.\\%E

\bvb Brothers will fight one another, and become [one another's] bane; sister's sons will waste their in-laws. 'Tis hard in the Home, great whoredom: halberd-\textbf{eld}, short-sword-eld; shields are split. Wind-eld, outlaw-eld, before the world\footnotetext[4] is overthrown, no man will spare another.
\footnotetext[4]{\emph{ver-ǫld} 'world' might perhaps be better translated as 'man-eld', 'the eld of man' with the other elds preceding it.}

\bva Lęika Míms synir, \hld ęn mjǫtuðr kyndisk \\%M
at hinu galla \hld Gjallarhorni \\%M
hótt blæss Hęimdallr, \hld horn ’s á lopti; \\%M
mælir Óðinn \hld við Míms hǫfuð.\\%E

\bvb The \textbf{sons of Mime} play, but the \textbf{Metted} is kindled, at [the sound of] the shrill \textbf{Yeller-horn}. Homedall blows loudly; the horn is in the air. Weden speaks with the head of Mime.

\bva Skęlfr Yggdrasils \hld askr standandi, \\%M
ymr aldit tré, \hld ęn jǫtunn losnar; \\%M
hræðask allir \hld á hęlvegum \\%M
áðr Surtar þann \hld sevi of glęypir.\\%E

\bvb The ash of Ugdrassel shakes standing; the old tree groans, and the ettin is loosened. All are frightened on the \textbf{Hell-ways}, before \textbf{Surt's kinsman} does devour it.

\bva Hvat ’s með ǫ́sum? \hld hvat ’s með ǫlfum? \\%M
gnýr allr Jǫtunhęimr, \hld æsir ’ro á þingi, \\%M
stynja dvergar \hld fyr stęindurum \\%M
vęggbergs vísir — \hld vituð ér ęnn eða hvat?\\%E

\bvb What is with Eses? What is with Elves? All Ettinhome roars, Eses are at the Thing. Dwarves groan before gates of stone, the princes of the mountain-walls. Know ye yet, or what?

\bva Gęyr nú Garmr mjǫk \hld fyr Gnipahęlli, \\%M
fęstr mun slitna, \hld ęn freki rinna, \\%M
fjǫlð vęit'k frǿða, \hld framm sé'k lęngra \\%M
of ragna rǫk, \hld rǫmm sigtíva.\\%E

\bva Hrymr ękr austan, \hld hęfsk lind fyrir, \\%M
snýsk Jǫrmungandr \hld í jǫtunmóði; \\%M
ormr knýr unnir, \hld ęn ari hlakkar, \\%M
slítr nái niðfǫlr; \hld Naglfar losnar.\\%E

\bva Kjóll fęrr austan \hld koma munu Múspells \\%M
of lǫg lýðir, \hld ęn Loki stýrir; \\%M
fara fíflmęgir \hld með freka allir, \\%M
þęim es bróðir \hld Býlęists í fǫr.\\%E

\bva Surtr\footnotemark[19] fęrr sunnan \hld með sviga lævi, \\%M
skínn af sverði \hld sól valtíva; \\%M
grjótbjǫrg gnata, \hld ęn gífr\footnotemark[20] rata, \\%M
troða halir hęlveg, \hld ęn himinn klofnar.
\footnotetext[19]{SnE: \emph{Svartr}}
\footnotetext[20]{SnE: \emph{guðar} 'gods'.}\\%E

\bva Þá kømr Hlínar \hld harmr annarr framm, \\%M
es Óðinn fęrr \hld við ulf vega, \\%M
ęn bani Bęlja \hld bjartr at Surti; \\%M
þá mun Friggjar \hld falla angan.\\%E

\bva Þá kømr hinn mikli \hld mǫgr Sigfǫður, \\%M
Víðarr vega \hld at valdýri; \\%M
lætr hann męgi Hveðrungs \hld mund of standa \\%M
hjǫr til hjarta; \hld þá ’s hefnt fǫður.\\%E

\bva Þá kømr hinn mæri \hld mǫgr Hlǫðynjar \\%M
gęngr Óðins sonr \hld ormi mǿta. \\%M
Drepr af móði \hld Miðgarðs véurr; \\%M
munu halir allir \hld hęimstǫð ryðja; \\%M
gęngr fet níu \hld Fjǫrgynjar burr \\%M
nęppr frá naðri, \hld níðs ókvíðinn.\\%E

\bva Sól tér sortna, \hld søkkr fold í mar, \\%M
hverfa af himni \hld hęiðar stjǫrnur; \\%M
gęisar ęimi \hld við aldrnara; \\%M
lęikr hór hiti \hld við himin sjalfan.\\%E

\bva Gęyr Garmr mjǫk \hld fyr Gnipahęlli, \\%M
fęstr mun slitna, \hld ęn freki rinna, \\%M
fjǫlð vęit'k frǿða, \hld framm sé'k lęngra \\%M
of ragna rǫk, \hld rǫmm sigtíva.\\%E

\bva Sér hon upp koma \hld ǫðru sinni \\%M
jǫrð ór ægi \hld iðjagrǿna —; \\%M
falla forsar, \hld flýgr ǫrn yfir, \\%M
sás á fjalli \hld fiska vęiðir.\\%E

\bva Finnask æsir \hld á Iðavęlli \\%M
ok of moldþinur \hld mǫ́tkan dǿma, \\%M
ok minnask þar \hld á męgindóma \\%M
ok á Fimbultýs \hld fornar rúnar\\%E

\bvb The Ease are found on the Idewald

\bva Þar munu ęptir \hld undrsamligar \\%M
gollnar tǫflur \hld í grasi finnask, \\%M
þærs í árdaga \hld áttar hǫfðu.\\%E

\bva Munu ósánir \hld akrar vaxa; \\%M
bǫls mun alls batna \hld mun Baldr koma; \\%M
búa Hǫðr ok Baldr \hld Hropts sigtoptir \\%M
(vęl valtívar, \hld Vituð ér ęnn eða hvat?)\\%E

\bvb Fields will grow unsown, all evil be bettered, Balder will come. The \textbf{wal-tues}, Had and Balder, will well inhabit the building-plots of Roft. Know ye yet or what?

\bva Þá kná Hǿnir \hld hlautvið kjósa \\%M
ok burir byggva \hld brǿðra Tvęggja \\%M
vindhęim víðan. \hld Vituð ér ęnn eða hvat?\\%E

\bva Sal sér hon standa \hld sólu fęgra, \\%M
golli þakðan, \hld á Gimléi; \\%M
þar munu dyggvar \hld dróttir byggva \\%M
ok of aldrdaga \hld ynðis njóta.[3]\\%E

\bvb A hall she sees stand, fairer than the sun, thatched with gold, on Gimlee; there the dutiful \texbf{drights} will dwell, and in their \textbf{alder}-days enjoy delight.

\bva Þar kømr hinn dimmi \hld dręki fljúgandi, \\%M
naðr fránn neðan \hld frá Niðafjǫllum; \\%M
berr sér í fjǫðrum \hld — flýgr vǫll yfir — \\%M
Níðhǫggr nái; \hld nú mun hón søkkvask.

\bvb Then comes the shadowy dragon flying; the gleaming serpent down below from the \textbf{Nithfells}. He, Nithehew, carries in his feathers—flying over the plain—corpses." Now she will be sunk!\footnote[1]
\footnote[1] The Wale, referring to herself in third person, sinks back down into her grave, whence Weden woke her.

	For the text of original poem, I do not present the manuscript text, but rather a standardized text of my own. I have however aimed to generally follow the dialect of the manuscript, rather than present a standardized Old High German or Old Saxon. The rules of normalization have been as follows:
Vowels:
> Ms. \emph{ae}, \emph{ei} and \emph{e}, where etymologically from \emph{ai}, have been normalized as \emph{ei}.
> Ms. \emph{o} and \emph{ao}, where etymologically from \emph{au}, have been normalized as \emph{ao}. This may be somewhat controversial.
> \emph{ostar}, \emph{Otachre} > \emph{aostar}, \emph{Aotachre}).
> Ms. \emph{uo} and \emph{o}, where etymologically from long \emph{ō}, have been normalized as \emph{ō}.
Consonants:
> Ms. \emph{r} and \emph{w}, where etymologically from \emph{hw} and \emph{hr}, have been thus normalized. That this was the case in the original poem is obvious; such words never alliterate with \emph{w} or \emph{r}, but only with \emph{r}, as can be most definitively seen in lines 56 (ms.: \alst{h}eremo ... \alst{h}rusti) and 66 (ms.: \alst{h}ewun \alst{h}armlicco \alst{h}uitte). If this were not enough, the retention in the ms. of the \emph{h} at previously given places is yet further support.
> Ms. \emph{tt}, where etymologically from \emph{t}, has been thus normalized.
> Ms. \emph{ƿ} (wynn), \emph{u} and \emph{uu}, where representing \emph{w}, have been thus normalized.

The pronoun which exclusively appears in the ms. as \emph{her} ‘he’ has been so kept, rather than normalized to the standard OHG \emph{er}.
The punctuation of the original (entirely consisting of interpuncts) has not been retained.

----

\bva Ik gihōrta dat seggen &
dat sih urhettun \hld einon mōtīn &
Hiltibrant enti Hadubrant \hld untar heriun tweim &
sunufatarungo \hld iro saro rihtun &
garutun \edtext{sie}{se \HildMS} iro gūdhamun \hld gurtun sih iro swert ana &
helidos ubar \edtext{hringa}{ringa \HildMS} \hld dō sie to dero hiltiu ritun\eva

\bvb I heard it said, that two contenders alone did meet: Hildbrand and Hathbrand, under two hosts. Son and father ordered their armour, readied their war-cloth, girded their swords on, the heroes over the mail, when to that battle they rode.\evb

\bva Hiltibrant gimahalta Heribrantes sunu \hld her was hērōro man &
ferahes frōtōro \hld her frāgēn gistōnt &
fōhēm wortum \hld \edtext{hwer}{wer \HildMS} sin fater wāri &
fireo in folche \hld [...] &
[...] \hld eddo \edtext{hwelīhhes}{welihhes \HildMS} cnōsles dū sīs &
ibu dū mī ēnan sagēs \hld ik mī de odre wēt &
chind in \edtext{chunincrīche}{chunnincriche \HildMS} \hld chūd ist mīn al irmindeot\eva

\bvb Hildbrand spoke, Harbrand's son — he was the hoarier man, more learned in life, — he began to ask with few words, who his father might be, of men in the folk, [...] “or of which lineage thou be; if thou me one say, I the others will know; child, in the kingdom, known to me are all great men.”\evb

\bva Hadubrant gimahalta \hld Hiltibrantes sunu &
dat sagetun mī ūsere liuti &
alte enti frōte \hld dea ērhina wārun &
dat Hiltibrant hēti min fater \hld ih heitu hadubrant &
forn her aostar giweit \hld  flauh her Aotachres nīd &
hina miti Deotrihhe \hld enti sīnero degano filu &
her furlēt in lante \hld luttila sitten &
brūt in būre \hld barn unwahsan &
arbeolaosa \hld her reit aostar hina &
des sid Deotrihhe \hld darba gistōntum &
\edtext{fateres}{fatereres \HildMS} mīnes \hld dat was sō friuntlaos man &
her was Aotachre \hld ummet tirri &
degano dechisto \hld unti \edtext{Deotrihhe}{\emph{add.} darba gistontun \HildMS} &
her was eo folches at ente \hld imo was eo \edtext{fehta}{peheta \HildMS} ti leob &
chūd was her \hld chōnēm mannum &
ni wāniu ih iu līb habbe\eva

\bvb Hathbrand spoke, Hildbrand's son: “It told me our people, the old and learned, those who earlier lived, that Hildbrand was called my father — I am called Hathbrand, — he previously hurried east; he fled Edwaker's hate, thither with Thedrich, and his multitude of thanes. He left in the land a little one to stay, a bride in the bower, a bairn ungrown, without inheritance; he rode east thither, as Thedrich was in great need of my father — that was such a friendless man. He was to Edwaker exceptionally hostile, the dearest of thanes under Thedrich. He was ever at the front of the troop; ever did the fight gladden him; known was he among keen men. — I ween not that he have life.”\evb

\bva weitu irmingot {\small [quad hiltibrant]} \hld obana ab hebane &
dat dū neo dana halt mit sus sibban man &
dinc ni gileitōs &
want her dō ar arme \hld wuntane baoga &
cheisuringu gitān \hld so imo sie der chuning gab &
huneo truhtin \hld dat ih dir it nū bī huldī gibu\eva

\bvb “I call on God as witness, [quoth Hildbrand], above in heaven, that thou never with such a close man once more lead dispute.” Unwound he then from his arm some twisted bighs, made from imperial coins, which the king once gave him, the lord of the Huns: — “This I now give thee as pledge.”\evb

\bva Hadubrant gimahalta \hld Hiltibrantes sunu &
mit gēru scal man \hld geba infāhan &
ort widar orte \hld [...] &
dū bist dir altēr hun \hld ummet spāhēr &
spenis mih mit dīnem wortum \hld wili mih dinu speru werpan &
bist alsō gialtēt man \hld sō dū ēwīn inwit fōrtōs &
dat sagetun mi \hld sēolīdante &
westar ubar wentilsēo \hld dat man wīc furnam &
tōt ist Hiltibrant \hld Heribrantes sunu\eva

\bvb Hathbrand spoke, Hildbrand's son: “With spear shall one earn gifts, point against point! Thou art, old hun, exceptionally clever; thou lurest me with thy words, wilt thou at me hurl thy spear! Thou art thus old, though thou ever deceit hast worked. — It told me seafarers, heading west o’er the Wendle-sea <= Mediterranean>, that war took that man: — dead is Hildbrand, Harbrand's son!”\evb

\bva Hiltibrant gimahalta \hld Heribrantes sunu &
wela gisihu ih in dīnēm hrustim &
dat dū habēs heime \hld hērron gōten &
dat dū noh bī desemo rīche \hld reccheo ni wurti\eva

\bvb Hildbrand spoke, Harbrand's son: “I see well on thy equipment, that thou hast a good lord at home, that thou yet in his reign art not become an exile.\evb

\bva welaga nu waltant got {\small [quad hiltibrant]} \hld weiwurt skihit &
ih wallōta sumaro enti wintro \hld sehstic ur lante &
dar man mih eo scerita \hld in folc sceotantero &
sō man mir at burc einīgeru \hld banun ni gifasta &
nu scal mih swāsat chind \hld swertu haowan &
bretōn mit sīnu billiu \hld eddo ih imo ti banin werdan &
doh maht dū nū aodlīhho \hld ibu dir dīn ellen taoc &
in sus hēremo man \hld hrusti giwinnan &
raoba birahanen \hld ibu du dar einīg reht habēs\eva

\bvb Well now, wielding God, [quoth Hildbrand], woeful Weird passes. I wallowed for summers and winters sixty, out of the land, where one ever placed me in the troop of shooters; thus one at no fortress my bane did inflict. Now shall my own child hew at me with sword; beat down with blade, or I become his bane; — yet canst thou now easily, if thy courage avail thee, from such a hoary man win the equipment, bear away the booty, if thou thereto have any right.\evb

\eva der sī doh nu argōsto {\small [quad hiltibrant]} aostarliuto &
der dir nū wīges warne \hld nū dih et sō wel lustit &
gudea gimeinun \hld niuse der mōti &
hwedar sih \edtext{hiutu dēro}{dēro hiuti \HildMS} hregilo \hld hrōmen mōti &
eddo desero brunnōno \hld beidero waltan\eva

\bvb Yet now he may be the weakest, [quoth Hildbrand], of the eastern peoples, who would refuse thee the fight, when thou so greatly cravest to struggle together. Try he who might, which one today of his arms may boast, or both of these byrnies wield!”\evb

\bva dō lietun sie aerist \hld askim scrītan &
scarpēn scūrim \hld dat in dem sciltim stōnt &
dō stōptun tosamane \hld staimbort hlūdun &
hewun harmlīcco \hld hwīte scilti &
unti imo iro lintūn \hld luttilo wurtun &
giwigan miti wābnum \hld [...]\eva

\bvb Then they first let their ash-spears glide, in a harsh torrent, that they stuck in the shields. Then charged they into each other — the war-boards [SHIELDS] resounded — struck they bitterly the white shields, until their linden-planks [SHIELDS] became little, worn down by the weapons, [...]\evb


\end{document}
