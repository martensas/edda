% This file should be compiled with XeLaTeX.

\documentclass[b5paper]{memoir}

% Font and typesetting
\usepackage[final]{microtype}

\usepackage{fontspec}
\setmainfont{Junicode}[
	Extension=.ttf,
	BoldFont=*-Bold,
	ItalicFont=*-Italic,
	BoldItalicFont=*-BoldItalic]
\newfontfamily{\greekfont}{Times New Roman}
\usepackage{polyglossia}
\setdefaultlanguage{english}

% TODO: Underline that does not skip descender?
% Should be called \nsunderline

% Bibliography
\usepackage[style=apa]{biblatex}
\addbibresource{bibliography.bib}
\DefineBibliographyStrings{english}{%
  sequens = {f\adddot},
  sequentes = {ff\adddot},
}

\newbibmacro*{textciteshorttitle}{% Cite short titles. From Stack Exchange: https://tex.stackexchange.com/questions/489928/biblatex-apa-textcite-with-shorttitle
  \ifbool{cbx:parens}
    {\bibcloseparen\global\boolfalse{cbx:parens}}
    {}%
  \setunit{\compcitedelim}%
  \printfield[bibhyperref]{shorttitle}%
  \iffieldundef{postnote}
    {}
    {\ifnumequal{\value{citecount}}{\value{citetotal}}
       {\printunit{\global\booltrue{cbx:parens}\addspace\bibopenparen}}
       {}}}

\DeclareCiteCommand{\textciteshorttitle}
  {\usebibmacro{cite:init}%
   \usebibmacro{prenote}}
  {\usebibmacro{citeindex}%
   \usebibmacro{textciteshorttitle}}
  {}
  {\usebibmacro{textcite:postnote}%
   \usebibmacro{cite:post}}

% Packages
\usepackage{xparse}% For better document commands
\usepackage{graphicx}% For rotating characters (used when citing runic inscriptions)
\usepackage{hyperref}% For index links
\usepackage{longtable}% Long tables.
\usepackage{tipa}% IPA

% Critical edition
\usepackage{reledmac}

% Headers
\pagestyle{myheadings}
\makeoddhead{myheadings}{\thepage}{book}{}
\makeevenhead{myheadings}{}{The Ancient Germanic Poetry}{\thepage}

% Define verse counters
\newcounter{versea}
\newcounter{verseb}
\newcounter{prosea}
\newcounter{proseb}

\begin{document}

% Book and chapter commands
  \setsecnumdepth{none}% Disable numbering of sections
  \maxsecnumdepth{none}% Disable numbering of sections

	\NewDocumentCommand{\chapterStart}{o O{Chap}}{% Command at the start of chapter
		\setcounter{versea}{0}%
		\setcounter{verseb}{0}%
		\stepcounter{section}%
		\IfNoValueF{#1}{%
			\begin{center}%
			\textbf{#2. \arabic{section}} \\
			{#1}\end{center}%
		}%
	}

	\NewDocumentCommand{\bookStart}{m o}{% Command at the start of book
		% arg 1 (mandatory): English title
		% arg 2 (optional):  Original title
	  \IfValueTF{#2}{%
      \chapter*{#1 \emph{(#2)}}%
      \def\booktitle{#1 \emph{(#2)}}%
    }{%
      \chapter*{#1}%
      \def\booktitle{#1}%
    }%
    \addcontentsline{toc}{chapter}{\booktitle}
    \setcounter{section}{0}% Set chapter count to zero.
    \chapterStart{}%
	}

% Verse format commands
	\NewDocumentCommand{\bvg}{o}{% Begin verse group
		\begin{ledgroup}%
		\beginnumbering%
		\setcounter{footnoteB}{0}%
	}

	\NewDocumentCommand{\bva}{o}{% Begin verse a
		\begin{large}\begin{stanza}% Begin stanza
		\IfNoValueTF{#1}{% Add verse number according to counter
			\stepcounter{versea}% Step verse counter
			\flagstanza{\textbf{\arabic{versea}}}%
		}{\IfEq{#1}{0}{}{% If optional verse number is NOT 0 we show it.
			\flagstanza{\textbf{#1}}%
		}}%
	}
	\NewDocumentCommand{\eva}{o}{% End verse a
		\& \end{stanza}\end{large}% End reledmac stanza
		\vspace{1.5mm}% Vertical space
	}

	\NewDocumentCommand{\bvb}{o}{% Begin verse b (see above)
		\IfNoValueTF{#1}{%
			\stepcounter{verseb}%
%			\textbf{\arabic{verseb} }%
		}{\IfEq{#1}{0}{}{%
%			\textbf{\textbf{#1}}
		}}%
		\noindent%
	}
	\NewDocumentCommand{\evb}{o}{% End verse b
		% Nothing (for now?)
	}

	\NewDocumentCommand{\evg}{o}{% End verse group
		\endnumbering\end{ledgroup}% End numbering and ledgroup
		\vspace{1cm}%
	}

% Prose format commands
	\NewDocumentCommand{\bpg}{o}{% Begin verse group
  	\begin{ledgroup}%
  	\beginnumbering%
  	\setcounter{footnoteB}{0}%
  }

  \NewDocumentCommand{\bpa}{o}{% Begin verse a
		\begin{large}% Begin stanza
		\IfNoValueTF{#1}{% Add verse number according to counter
			\stepcounter{prosea}% Step verse counter
			\flagstanza{\textbf{P\arabic{prosea}}}%
		}{\IfEq{#1}{0}{}{% If optional verse number is NOT 0 we show it.
			\flagstanza{\textbf{P#1}}%
		}}%
	}
	\NewDocumentCommand{\epa}{o}{% End verse a
		\& \end{large}% End large
		\vspace{1.5mm}% Vertical space
	}

	\NewDocumentCommand{\bpb}{o}{% Begin verse b (see above)
		\IfNoValueTF{#1}{%
			\stepcounter{proseb}%
%			\textbf{\arabic{verseb} }%
		}{\IfEq{#1}{0}{}{%
%			\textbf{\textbf{#1}}
		}}%
		\noindent%
	}
	\NewDocumentCommand{\epb}{o}{% End verse b
		% Nothing (for now?)
	}

	\NewDocumentCommand{\epg}{o}{% End verse group
		\endnumbering\end{ledgroup}% End numbering and ledgroup
		\vspace{1cm}%
	}

% Note formatting
	% Sidenote margin
	\setlength{\ledlsnotesep}{2 \ledlsnotesep}

	% Make A footnotes paragraphs
	\Xarrangement[A]{paragraph}

	% Make B footnotes roman
	\renewcommand*{\thefootnoteB}{\alph{footnoteB}}

% Poem formatting
	% First line number at 3
	\firstlinenum{2}
	\linenumincrement{2}

	% Stanza indentation (required by reledmac)
	\setstanzaindents{5, 2, 2}
	\setcounter{stanzaindentsrepetition}{2}

	% Line numbers directly under verse number (kind of a hack)
	\setlength{\linenumsep}{-1.62pc}

	% Mark cæsura.
	\newcommand{\hld}{ · }%

	% Indent lines (in Ljóðaháttr or Galdralag).
	\newcommand{\ind}{%
		\hspace{1.5em}%
	}

	% Mark alliteration. This might not be present in the final version.
	\NewDocumentCommand{\alst}{m}{%
		\underline{#1}%
	}

	% Mark kennings.
	\NewDocumentCommand{\ken}{sm}{%
		% No small caps; text inside the brackets.
		\IfBooleanTF{#1}{% No small caps
			{[{#2}]}% EXAMPLE: [= Wooden]
		}{%
			\textsc{[{#2}]}% EXAMPLE: [ETTIN]
		}%
	}%

	% Mark names with angular brackets.
	\NewDocumentCommand{\name}{m}{%
	% Text inside the brackets.
		<{#1}>% EXAMPLE: <ettin>
	}%

% Index commands
	\NewDocumentCommand{\inx}{o m O{#2}}{% Index link (type, link, optional alt display)
		\IfNoValueTF{#1}{% TODO: phase this out
			ERRORERRORERROR%
		}{% Proper noun
			\hyperref[#1:#2]{#3\textsuperscript{#1}}%
		}%
	}

	\NewDocumentCommand{\inxitem}{o m}{% Index label (type, word)
		\item[\textbf{#2}]%
		\phantomsection\label{#1:#2}%
	}

% Sigla
  \subsection{Languages}
\begin{itemize}%
	\item Eng. = Modern English
	\item Ger. = Modern German
	\item Got. = Gotnish (or Gothic)
	\item Lomb. = Lombardic
	\item MHG = Middle High German
	\item OE = Old English
	\item OF = Old Frisian
	\item OHG = Old High German
	\item ON = Old Norse
	\item OS = Old Saxon
	\item OSwe. = Old Swedish
	\item PGmc. = Proto-Germanic
	\item PN = Proto-Norse
	\item PNWGmc. = Proto-North-West Germanic
\end{itemize}

\subsection{Grammar}
\begin{itemize}%
	\item 1st = first-person
	\item 2nd = second-person
	\item 3rd = third-person
	\item acc. = accusative case
	\item cpd = compound
	\item dat. = dative case
	\item gen. = genitive case
	\item imper. = imperative mood
	\item ind. = indicative mood
	\item instr. = instrumental case
	\item nom. = nominative case
	\item pl. = plural number
	\item sg. = singular number
	\item subj. = subjunctive mood
\end{itemize}

\subsection{Other abbreviations}
\begin{itemize}%
	\item cert. = certainly
	\item c. = circa
	\item cf. = \emph{confere}; compare
	\item corr. = corrected in the ms.
	\item e. = excerpt (not the whole stanza)
	\item ed. = edition, edited (by)
	\item e.g. = \emph{exemplio gratia}; for instance
	\item emend. = emendation, emended (by)
	\item fol., foll. = folio, folios
	\item i.e. = \emph{id est}; that is
	\item l., ll. = line, lines
	\item lit. = literally
	\item metr. emend. = emended based on (secure) metrical criteria
	\item ms., mss. = manuscript, manuscripts
	\item norm. = normalised from the ms. spelling
	\item om. = omitted by
	\item p., pp. = page, pages
	\item tr. = translation, translated (by)
	\item sens. emend. = emended based on sense
	\item st., sts. = stanza, stanzas
	\item viz. = \emph{vidēlicet}; namely, to wit
	\item wo. = without
	\item wrt. = with regard to
\end{itemize}

% Old texts, primary sources
% The command codes must be as close to the original language titles as possible.
\subsection{Primary sources}

\newcommand{\AitareyaBrahmana}{%
	\emph{AB}%
}
\newcommand{\Alvissmal}{% Speeches of Allwise
	\emph{Alv}%
}
\newcommand{\Atlakvida}{% Lay of Attle
	\emph{Akv}%
}
\newcommand{\Atlamal}{% Speeches of Attle
	\emph{Am}%
}
\newcommand{\Baldrsdraumar}{% The Dreams of Balder
	\emph{Bdr}%
}
\newcommand{\Beowulf}{% Beewolf
	\emph{Beow}%
}
\newcommand{\Brot}{% Fragment of a Lay of Siward
	\emph{Brot}%
}
\newcommand{\Deor}{% Deer
	\emph{Deer}%
}
\newcommand{\EyrbyggjaSaga}{% Saw of Harware and Heathric
	\emph{Eb}%
}
\newcommand{\EgilsSaga}{% Saw of Harware and Heathric
	\emph{Eg}%
}
\newcommand{\Fafnismal}{% Speeches of Fathomer
	\emph{Fáfn}%
}
\newcommand{\FostrbroedhraSaga}{% Saw of the Foster-brothers
	\emph{FbrS}%
}

\newcommand{\FraLoka}{% From Lock
	\emph{From Lock}%
}%TODO: remove this

\newcommand{\Grettissaga}{% Saw of Gretter
	\emph{GrettS}%
}
\newcommand{\Grimnismal}{% Speeches of Grimner
	\emph{Grm}%
}
\newcommand{\Gripisspa}{% Spae of Griper
	\emph{Gríp}%
}
\newcommand{\Grottasongr}{% Song of Grotte
	\emph{Grotta}%
}
\newcommand{\Grougaldr}{% Galder of Growe
	\emph{Grg}%
}
\newcommand{\Gudrunarhvot}{% Guthrun’s Instigation
	\emph{Ghv}%
}
\newcommand{\GudrunOne}{% Guthrun’s First Lay
	\emph{Guðr I}%
}
\newcommand{\GudrunTwo}{% Guthrun’s Second Lay
	\emph{Guðr II}%
}
\newcommand{\GudrunThree}{% Guthrun’s Third Lay
	\emph{Guðr III}%
}
\newcommand{\Gulatingslog}{% Law of the Gole-thing
	\emph{Gula}%
}
\newcommand{\Gylfaginning}{% The Guiling of Yilver; for referring to Gylfaginning as a text
	\emph{Gylf}%
}
\newcommand{\Hakonarmal}{% Speeches of Hathkin
	\emph{Hákm}%
}

\newcommand{\HakonarSaga}{% Saw of Hathkin the good
	\emph{HákGóð}%
}
\newcommand{\Haleygjatal}{% Tally of the Hallowlendings
	\emph{HalT}%
}%TODO: Add

\newcommand{\Hamdismal}{% Speeches of Hamthew
	\emph{Hamð}%
}
\newcommand{\Harbardsljod}{% Leed of Hoarbeard
	\emph{Hárb}%
}
\newcommand{\Haustlong}{% Harvest-long
	\emph{Haustl}
}
\newcommand{\Havamal}{% Speeches of the High One
	\emph{Háv}%
}
\newcommand{\HelgakvidaHjorvardssonar}{% Lay of Hallow Harwardson
	\emph{HHj}%
}
\newcommand{\HelgakvidaOne}{% First Lay of Hallow Hundingsbane
	\emph{HHund I}%
}
\newcommand{\HelgakvidaTwo}{% Second Lay of Hallow Hundingsbane
	\emph{HHund II}%
}
\newcommand{\Heliand}{%
  \emph{Heli}%
}
\newcommand{\Helreid}{% Byrnhild’s Hell-ride
	\emph{Helr}%
}
\newcommand{\HervararSaga}{% Saw of Harware and Heathric
	\emph{HarS}%
}
\newcommand{\Hildebrandslied}{% Speeches of Hildbrand
	\emph{Hildebrand}%
}
\newcommand{\Hymiskvida}{% Lay of Hymer
	\emph{Hym}%
}
\newcommand{\Hyndluljod}{% Leed of Hindle
	\emph{Hdl}%
}

\newcommand{\Lacnunga}{% Leekning
	\emph{Lacning}%
}%TODO: add this

\newcommand{\Lokasenna}{% Flyting of Lock
	\emph{Lok}%
}

\newcommand{\Malshattakvadi}{
	\emph{Mhkv}
}%TODO: add this

\newcommand{\Mahabharata}{
	\emph{MBʰ}
}
\newcommand{\MerseburgOne}{% First Merseburg charm
	\emph{Mers I}%
}
\newcommand{\MerseburgTwo}{% Second Merseburg charm
	\emph{Mers II}%
}
\newcommand{\Muspilli}{% Muspell
	\emph{Muspilli}%
}
\newcommand{\Oddrunargratr}{% Weeping of Ordrun
	\emph{Oddrgr}%
}
\newcommand{\Reginsmal}{% The Speeches of Rein
	\emph{Reg}%
}
\newcommand{\Rigsthula}{% Thule of Righ
	\emph{Rþ}%
}
\newcommand{\Rigveda}{%
	\emph{R̥V}%
}
\newcommand{\SaxonGenesis}{% Old Saxon Genesis
	\emph{OSGen}%
}
\newcommand{\Sigurdskamma}{% Short Lay of Siward
	\emph{Sigsk}%
}
\newcommand{\Sigrdrifumal}{% Speeches of Syedrive
	\emph{Sigrdr}%
}
\newcommand{\Skaldskaparmal}{% The Matter of Scoldship
	\emph{Skm}%
}
\newcommand{\Skirnismal}{% Speeches of Shirner
	\emph{Skm}%
}

\newcommand{\Sogubrot}{
	\emph{AncKings}
}%TODO: add
\newcommand{\Solarljod}{
	\emph{Sun}
}%TODO: add
\newcommand{\Sonatorrek}{
	\emph{Sont}
}%TODO: add
\newcommand{\Sorlathattr}{% Strand of Sarle
	\emph{Sarle}%
}%TODO: add
\newcommand{\ThidreksSaga}{% Saw of Thedrich
	\emph{ThidS}%
}%TODO: add

\newcommand{\Thorsdrapa}{% Drape of Thunder
	\emph{Þdr}%
}
\newcommand{\Thrymskvida}{% Lay of Thrim
	\emph{Þrk}%
}
\newcommand{\Vafthrudnismal}{% Speeches of Webthrithner
	\emph{Vafþ}%
}
\newcommand{\Volsathattr}{% Strand of Walse
	\emph{Vǫlsþ}%
}
\newcommand{\VolsungaSaga}{% Saw of the Walsings
	\emph{VǫlsS}%
}
\newcommand{\Volundarkvida}{% Lay of Wayland
	\emph{Vkv}%
}
\newcommand{\Voluspa}{% Spae of the Wallow
	\emph{Vsp}%
}

\newcommand{\Waldere}{% Walder
	\emph{Walder}%
}%TODO: add
\newcommand{\YnglingaSaga}{% Saw of the Inglings
	\emph{IngS}%
}%TODO: add
\newcommand{\Ynglingatal}{% Tally of the Inglings
	\emph{IngT}%
}%TODO: add

\begin{itemize}%
	\item \AitareyaBrahmana\ = \emph{Aitareyá Brā́hmaṇa}
	\item \Alvissmal\ = \emph{Alvíssmǫ́l} (Speeches of Allwise)
	\item \Atlakvida\ = \emph{Atlakviða} (Lay of Attle)
	\item \Atlamal\ = \emph{Atlamǫ́l} (Speeches of Attle)
	\item \Baldrsdraumar\ = \emph{Baldrs draumar} (Dreams of Balder)
	\item \Beowulf\ = \emph{Beowulf}
	\item \Brot\ = \emph{Brot af Sigurðarkviða} (Fragment of a Lay of Siward)
	\item \Deor\ = \emph{Déor} (Deer)
	\item \EyrbyggjaSaga\ = \emph{Eyrbyggja saga} (Saw of the Ere-dwellers)
	\item \Fafnismal\ = \emph{Fáfnismǫ́l} (Speeches of Fathomer)
	\item \FostrbroedhraSaga\ = \emph{Fóstrbrǿðra saga} (Saw of the Fosterbrothers)
	\item \Grettissaga\ = \emph{Grettis saga} (Saw of Gretter)
	\item \Grimnismal\ = \emph{Grímnis mǫ́l} (Speeches of Grimner)
	\item \Gripisspa\ = \emph{Grípisspǫ́} (Spae of Griper)
	\item \Grottasongr\ = \emph{Grottasǫngr} (Song of Grotte)
	\item \Grougaldr\ = \emph{Gróugaldr} (Galder of Growe)
	\item \Gudrunarhvot\ = \emph{Guðrúnarhvǫt} (Goading of Guthrun)
	\item \GudrunOne\ = \emph{Guðrúnarkviða I} (First Lay of Guthrun)
	\item \GudrunTwo\ = \emph{Guðrúnarkviða II} (Second Lay of Guthrun)
	\item \GudrunThree\ = \emph{Guðrúnarkviða III} (Third Lay of Guthrun)
	\item \Gulatingslog\ = \emph{Gulaþingslǫg} (Law of the Gole‑Thing)
	\item \Gylfaginning\ = \emph{Gylfaginning} (Beguiling of Yilver)
	\item \Hakonarmal\ = \emph{Hǫ́konarmǫ́l} (Speeches of Hathkin)
	\item \HakonarSaga\ = \emph{Hǫ́konar saga góða} (Saw of Hathkin the good)
	\item \Hamdismal\ = \emph{Hamðismǫ́l} (Speeches of Hamthew)
	\item \Harbardsljod\ = \emph{Hárbarðljóð} (Leeds of Hoarbeard)
	\item \Haustlong\ = \emph{Haustlǫng} (Harvest‑long)
	\item \Havamal\ = \emph{Hávamǫ́l} (Speeches of the High One)
	\item \HelgakvidaHjorvardssonar\ = \emph{Helgakviða Hjǫrvarðssonar} (Lay of Hallow Harwardson)
	\item \HelgakvidaOne\ = \emph{Helgakviða Hundingsbana I} (First Lay of Hallow Hundingsbane)
	\item \HelgakvidaTwo\ = \emph{Helgakviða Hundingsbana II} (Second Lay of Hallow Hundingsbane)
	\item \Heliand\ = \emph{Heliand}
	\item \Helreid\ = \emph{Helreið Brynhildar} (Hell‑ride of Byrnhild)
	\item \HervararSaga\ = \emph{Hervarar saga} (Saw of Harware and Heathric)
	\item \Hildebrandslied\ = \emph{Hildebrandslied}
	\item \Hymiskvida\ = \emph{Hymiskviða} (Lay of Hymer)
	\item \Hyndluljod\ = \emph{Hyndluljóð} (Leeds of Hindle)
	\item \Lokasenna\ = \emph{Lokasenna} (Flyting of Lock)
	\item \Mahabharata\ = \emph{Mahā́bʰārata}
	\item \MerseburgOne\ = Merseburg galder I
	\item \MerseburgTwo\ = Merseburg galder II
	\item \Oddrunargratr\ = \emph{Oddrúnargrátr} (Weeping of Ordrun)
	\item \Reginsmal\ = \emph{Ręginsmǫ́l} (Speeches of Rein)
	\item \Rigsthula\ = \emph{Rigsþula} (Thule of Righ)
	\item \Rigveda\ = \emph{R̥g-vedá}, with translations from Jamison‑Brereton unless otherwise specified.
	\item \SaxonGenesis\ = \emph{Old Saxon Genesis}
	\item \Sigurdskamma\ = \emph{Sigurðarkviða skamma} (Short Lay of Siward)
	\item \Sigrdrifumal\ = \emph{Sigrdrífumǫ́l} (Speeches of Syedrive)
	\item \Skaldskaparmal\ = \emph{Skaldskaparmǫ́l} (Matter of Scoldship)
	\item \Skirnismal\ = \emph{Skírnismǫ́l} (Speeches of Shirner)
	\item \Thorsdrapa\ = \emph{Þórsdrápa} (Drape of Thunder)
	\item \Thrymskvida\ = \emph{Þrymskviða} (Lay of Thrim)
	\item \Vafthrudnismal\ = \emph{Vafþrúðnismǫ́l} (Speeches of Webthrithner)
	\item \Volsathattr\ = \emph{Vǫlsaþáttr} (Strand of Walse)
	\item \VolsungaSaga\ = \emph{Vǫlsunga saga} (Saw of the Walsings)
	\item \Volundarkvida\ = \emph{Vǫlundarkviða} (Lay of Wayland)
	\item \Voluspa\ = \emph{Vǫluspǫ́} (Spae of the Wallow)
\end{itemize}%


% Manuscripts
\newcommand{\AM}{% AM 748 I a 4to (https://handrit.is/manuscript/view/da/AM04-0748-I-a, https://books.google.se/books?id=L-MOAAAAQAAJ)
	\textbf{A}%
}
\newcommand{\AMb}{% AM 748 I b 4to (https://handrit.is/manuscript/view/is/AM04-0748-Ib)
	\textbf{A\textsubscript{b}}%
}
\newcommand{\EddaBms}{% AM 757 a 4° (https://handrit.is/manuscript/view/is/AM04-0757a)
	\textbf{B}%
}
\newcommand{\FlatMS}{% Flateyjarbok
	\textbf{F}%
}
\newcommand{\GylfMS}{% For referring to Gylfaginning manuscripts when stanzas are attested there.
	\textbf{G}%
}
\newcommand{\Hauksbok}{% Hauksbok
	\textbf{H}%
}
\newcommand{\VolsungaMS}{% NKS 1824 b 4° (https://skaldic.ku.dk/q?p=skp/mss/ms/512 and https://onp.ku.dk/onp/onp.php?b2195-53)
	\textbf{N}%
}
\newcommand{\Regius}{% Codex Regius (of the poetic edda)
	\textbf{R}%
}
\newcommand{\RegiusProse}{% Codex Regius of the Prose Edda
	\textbf{S}%
}
\newcommand{\Trajectinus}{% Codex Trajectinus
	\textbf{T}%
}
\newcommand{\Wormianus}{% Codex Wormianus (https://clarino.uib.no/menota/text/menota/AM-242-fol)
	\textbf{W}%
}
\newcommand{\Upsaliensis}{% Codex Upsaliensis
	\textbf{U}%
}
\newcommand{\HildMS}{% For referring to the Hildebrandslied manuscript.
	ms.%
}

\subsection{Manuscripts}
\begin{itemize}%
	\item \AM\ = AM 748 I a 4° (https://handrit.is/manuscript/view/da/AM04-0748-I-a)
	\item \AMb\ = AM 748 I b 4° (https://handrit.is/manuscript/view/is/AM04-0748-Ib)
	\item \EddaBms\ = AM 757 a 4° (https://handrit.is/manuscript/view/is/AM04-0757a)
	\item \FlatMS\ = Flatsęyjarbók, GKS 1005 fol. (https://handrit.is/manuscript/view/is/GKS02-1005)
	\item \GylfMS\ = all manuscripts of \Gylfaginning; equivalent to \RegiusProse\Trajectinus\Upsaliensis\Wormianus
	\item \Hauksbok\ = Hauksbók, AM 544 4° (https://handrit.is/manuscript/view/en/AM04-0544)
	\item \VolsungaMS\ = NKS 1824 b 4° (https://onp.ku.dk/onp/onp.php?m9641)
	\item \Regius\ = Codex Regius of the Poetic Edda, GKS 2365 4° (https://eae.ku.dk/q?p=eae/vols/text/1)
	\item \RegiusProse\ = Codex Regius of the Prose Edda, GKS 2367 4° (https://handrit.is/manuscript/view/is/GKS04-2367)
	\item \Trajectinus\ = Codex Trajectinus, Traj 1374ˣ
	\item \Upsaliensis\ = Codex Upsaliensis, DG 11
	\item \Wormianus\ = Codex Wormianus, AM 242 fol. (https://clarino.uib.no/menota/text/menota/AM-242-fol)
\end{itemize}

% Meters
\newcommand{\Drottkvett}{% Court-recited
	\emph{Court-recited meter}%
}
\newcommand{\Fornyrdislag}{% Law of Ancient Speeches
	\emph{Ancient-words-law}%
}
\newcommand{\Galdralag}{% Meter of Speeches
	\emph{Galders-law}%
}
\newcommand{\Ljodahattr}{% Meter of Leeds
	\emph{Leeds-meter}%
}
\newcommand{\Kviduhattr}{%
	\emph{Lay-meter}%
}
\newcommand{\Malahattr}{% Meter of Speeches
	\emph{Speeches-meter}%
}

%Modern books and editions (TODO: move these to bibliography)
\newcommand{\CV}{% Cleasby-Vigfússon dictionary of Old Norse
	\textciteshorttitle{CleasbyVigfusson}% \emph{C-V}%
}
\newcommand{\FGT}{% First Grammatical Treatise
	\textciteshorttitle{FGTHaugen}%
}
\newcommand{\ONP}{% Dictionary of Old Norse Prose
	\emph{ONP}%
}
\newcommand{\Skp}{% Skaldic Poetry of the Scandinavian Middle Ages
	\textciteshorttitle{SkP}%
}
%

% Render books
  % This file only contains the various books included. This is done so that it can be edited separately from main.tex, which contains the formatting code.

% Introduction, bibliography and abbreviations
\frontmatter%

Introduction to Eddic poetry
  Don't go too indepth on individual poems! Each one will have its own introduction.
  Metrics and conventions
    Alliteration
    Kennings
  How can we know the age of the Eddic poems?
    Linguistic criteria
    Archeological evidence
    Comparison with known Christian texts (Sólarljóð, Hugsvinnsmál)
    Snorri thought they were old
    Saxo had access to them
    Many of them clearly describe non-Icelandic surroundings
      Especially Hávamál is clearly Norwegian

Ancient Germanic cult(ure)
  Honour, personal integrity
  Notes on the terms \emph{argr} and \emph{ergi}
  Religious conceptions
    Cosmic cycles
    Reincarnation
    Analogies with other Indo-European traditions

Notes to translation
  Why Anglish names?
  Point about literal translation for use by scholars of comparative mythology

Notes to critical edition (TODO: move from introduction to \Voluspa)
  Relevant manuscripts and which poems in each
    R = GKS 2365 4to
    A = AM 748 I a 4to
    Prose Edda group:
    S
    T
    W
    U
    Paper manuscripts? (Fjǫlsvinnsmál, Baldrs draumar)
  
Bibliography and sigla
  TODO: Probably move this from main.tex

Abbreviations
  wo. = without
%

\mainmatter%

\part{Mythic poetry}% Theology, mythology, generally independent order
	\bookStart{Spae of the Wallow}[Vǫluspǫ́]

\begin{flushright}%
\textbf{Dating} \parencite{Sapp2022}: C10th (0.865)–early C11th (0.121)

\textbf{Meter:} \Fornyrdislag%
\end{flushright}

\section{Introduction}

The \textbf{Spae of the Wallow} (\Voluspa) is the most comprehensive mythological text surviving from Heathen times.  The poem is a \inx[C]{spae} (\emph{spǫ́} ‘prophecy’) in the form of a monologue spoken by a \inx[C]{wallow} (\emph{vǫlva} ‘seeress, sibyl, prophetess’) summoned by the god Weden in order to relate mythological knowledge.  Weden’s frequent journeys to question various beings about mythological lore should be seen in the light of his incessant lust for knowledge and wisdom.  The most similar instance is \Baldrsdraumar, wherein Weden summons another wallow out of her grave in \inx[L]{Hell} in order to find out why the god \inx[P]{Balder} is having ominous nightmares.  There is also \Vafthrudnismal, wherein Weden challenges the wise ettin \inx[P]{Webthrithner} to a wisdom contest and defeats him.  These journeys are further alluded to in \Harbardsljod\ TODO.

In its being a mythic catalogue \Voluspa\ also resembles (parts of) poems like \Havamal, \Grimnismal, \Sigrdrifumal, and \Allvismal, but it differs from them all in a key way: instead of being a motley collection of scattered mythological lore, \Voluspa\ offers a chronological overview of the whole Norse mythic timeline, from the creation of the world to its demise and rebirth.  That is not to say that the events in it clearly described; they are related in a highly allusive fashion that presupposes that the audience is already familiar with them.  There may also be some later omissions and inserts that make the poem more difficult to read.

\Voluspa\ is attested in full in two independent recensions.  The first and most important is \Regius, where it is the first poem and found on foll. 1r–3r; the other is \Hauksbok, where it is found in the middle of a large collection of saws and Catholics works at 20r–21r.

Many stanzas from the poem are also cited or paraphrased in \Gylfaginning, for which \Voluspa\ was clearly one of the main sources.  These paraphrases are still of critical value, e.g. in st. 19, where \emph{sal} ‘hall’ in the paraphrase agrees with \Hauksbok\ against \Regius\ \emph{sę́} ‘lake’.  For the four mss. of \Gylfaginning—\RegiusProse, \Trajectinus, \Wormianus, and \Upsaliensis—see the General Introduction.

For the differences between the mss. the reader may consult the following table prepared by the editor.  The several stanzas in \Gylfaginning, which are quoted independently and with little relation to the order of the original poem, are marked with plus signs.  The sequences containg uninterrupted quotations of several stanzas are marked with an incrementing alphabetic symbol, so that \emph{B1} is the first stanza in the second sequence, and so on.  When a stanza found in a ms. is strongly divergent (e.g. st. 10, where \Gylfaginning\ omits the first two half-lines), its number is followed by a star.  The stanzas beginning with \emph{Þȧ gingu ręgin ǫll} ‘Then went the Reins all’ are represented by the half-line immediately following.

\begin{longtabu} to \textwidth {|c c c c c c|}
	\hline
	\multicolumn{2}{|c}{\emph{pres. ed.}} & \Regius & \Hauksbok & \RegiusProse\Trajectinus\Wormianus & \Upsaliensis \\ [0.5ex]
	\hline\hline\endhead
	\hline\endfoot
	1 & Hljóðs bið’k allar & 1 & 1 & − & − \\
	2 & Ek man jǫtna & 2 & 2 & − & − \\
	3 & Ár vas alda & 3 & 3 & + & + \\
	4 & áðr Burs synir & 4 & 4 & − & − \\
	5 & Sól varp sunnan & 5 & 5 & +* & +* \\
	6 & \dots\ nǫ́tt ok niðjum & 6 & 6 & − & − \\
	7 & Hittusk ę̇sir & 7 & 7 & − & − \\
	8 & Tęflðu ï tu̇ni & 8 & 8 & − & − \\
	9 & \dots\ hvęrr skyldi dverga & 9 & 9 & B1 & B1 \\
	10 & Þar vas Móðsognir & 10 & 10 & B2* & B2* \\
	11–15 & \emph{Dwarf-tallies} & 11–15 & 11–16 & + & + \\
	16 & Unds þrír kvǫ̇mu & 16 & 17 & − & − \\
	17 & Ǫnd þau né ǫ́ttu & 17 & 18 & − & − \\
	18 & Ask vęit’k standa & 18 & 19 & + & + \\
	19 & Þaðan koma męyjar & 19–20 & 20–21 & − & − \\
	20 & Þat man hǫ̇n folk-víg & 21–22 & 27 & − & − \\
	21 & Hęiði hétu & 23 & 28 & − & − \\
	22 & \dots\ hvárt skyldu ę̇sir & 24 & 29 & − & − \\
	23 & Flęygði Óðinn & 25 & 30 & − & − \\
	24 & \dots\ hvęrr hęfði lopt alt & 26 & 22 & C1 & C1 \\
	25 & Þȯrr ęinn þar vá & 27 & 23 & C2* & C2* \\
	26 & Vęit hǫ̇n Hęimdalar & 28 & 24 & − & − \\
	27 & Ęin sat hǫ̇n úti & 29 & − & − & − \\
	28 & Alt vęit’k, Óðinn & 29 & − & + & + \\
	29 & Valði hęnni Hęr-fǫðr & 30 & − & − & − \\
	30 & Sá hǫ̇n val-kyrjur & 31 & − & − & − \\
	31 & Ek sá Baldri & 32 & − & − & − \\
	32 & Varð af męiði & 33 & − & − & − \\
	33 & Þó hann ę́va hęndr & 34 & − & − & − \\
	H1 & Þȧ kná Váli & − & 31 & − & − \\
	34a & Hapt sá hǫ̇n liggja & 35a & − & − & − \\
	34b & þar sitr Sigyn & 35b & 32 & − & − \\
	35 & Ǫ́ fęllr austan & 36 & − & − & − \\
	36 & Stóð fyr norðan & 36 & − & − & − \\
	37 & Sal sá hǫ̇n standa & 37 & 36 & E1 & E1 \\
	38 & Sér hǫ̇n þar vaða & 38 & 37 & E2* & E2* \\
	39 & Austr býr hin aldna & 39 & 25 & A1 & A1 \\
	40 & Fyllisk fjǫrvi & 40 & 26 & A2 & A2 \\
	41 & Sat þar ȧ haugi & 41 & 34 & − & − \\
	42 & Gól of ǫ̇sum & 42 & 35 & − & − \\
	43, 48, 56 & Gęyr (nú) Garmr mjǫk & 43, 46, 55 & 33, 38, 43, 48, 51 & − & − \\
	44 & Brǿðr munu bęrjask & 44 & 39 & − & − \\
	45 & Lęika Mïms synir & 45 & 40 & D1* & D1* \\
	H2 & Hrę́ðask allir & − & 41 & − & − \\
	46 & Hvat ’s með ǫ̇sum? & 49 & 42 & D2 & D2* \\
	48 & Hrymr ękr austan & 47 & 44 & D3 & − \\
	49 & Kjóll fęrr austan & 48 & 45 & D4 & − \\
	50 & Surtr fęrr sunnan & 50 & 46 & +, D5 (cited twice) & + \\
	51 & Þȧ kømr Hlïnar & 51 & 47 & D6 & − \\
	52 & Þȧ kømr hinn mikli & 52 & − & D7 & − \\
	H3 & Gïnn lopt yfir & − & 48 & — & − \\
	53 & Þȧ kømr hinn mę́ri & 53* & 49* & D8 & − \\
	54 & Sól tér sortna & 54 & 50 & D9 & − \\
	56 & Sér hǫ̇n upp koma & 56 & 52 & − & − \\
	57 & Finnask ę̇sir & 57* & 53 & − & − \\
	58 & Þar munu ęptir & 58 & 54 & − & − \\
	59 & Munu ȯ·sánir & 59 & 55 & − & − \\
	60 & Þȧ kná Hø̇nir & 60 & 56 & − & − \\
	61 & Sal sér hǫ̇n standa & 61 & 57 & + & + \\
	H4 & Þȧ kømr hinn ríki & − & 58 & − & − \\
	62 & Þar kømr hinn dimmi & 62 & 59 & − & − \\ [1ex]
	\hline
\end{longtabu}


\sectionline

The poem begins with a bid for silence (1), and the wallow recalling her earliest memories (2). She then recounts the ordering of the world by the gods (3–6) and the golden age of peace and plenty (7–8), which is, however, interrupted by the intrusion of three unidentified ettin-maidens (8, and see note there). After this follow two verses about the shaping of the dwarfs (9–10), and then several originally separate \emph{dwarf-tallies} (11–15), which are without doubt later inserts. Returning to the main narrative thread is described the creation and endowment of the first man and woman (16–17), Ugdrassle’s Ash (18), and the three \inx[G]{norns} living under it (19).

This is where the two full recensions of the poem diverge. Because of its older age and larger count of verses I have here followed the order of \Regius: the wallow recalls how a woman named Goldwey was sacrificed and reborn three times (20), and how she, under the name Heath, practiced sorcery and witchcraft (21). She then recalls the first war in the world, between the Eese and Wanes (22–23), and alludes to the slaying of the smith, who according to \Gylfaginning\ 42 was promised \inx[P]{Frow} and the sun and moon in exchange for building the wall of Osyard (24-25). This is followed by a cryptic verse describing Homedal’s hidden silence or hearing (26).

In \Hauksbok\ the structure is quite different. After the description of the norns (19), the Eese go to decide what action to take regarding the promising of Frow to the ettin (my 24-25), and Homedal’s hearing is described (26). Then follows the two verses about the old hag in Ironwood who raises the wolves that will swallow the sun and moon (40-41). After this come verses 20–23 in the same order as \Regius\ (see above).

\sectionline

\section{The Spae of the Wallow}

\bvg\bva\mssnote{\Regius~1r/2, \Hauksbok~20r/1}%
\edtext{„\edtrans{\alst{H}ljóðs bið’k}{For hearing I ask}{\Bfootnote{The same introductory expression is found in st. 2 of Eyel’s Headransom (Egill \emph{Hfl} in \Skp\ 5): \emph{hljóðs biðjum hann} ‘for hearing we [I] ask him’.}} allar \hld\ \edtext{\alst{h}ęlgar}{\lemma{hęlgar}\Afootnote{om. \Regius}} kindir, &
\edtrans{\alst{m}ęiri ok \alst{m}inni}{greater and lesser}{\Bfootnote{It is ambiguous to which phrase these adjectives belong.  It may either be (a) ‘holy kindreds greater and lesser’, which could be equivalent to the phrase \inx[F]{Eese and Elves} (both earthly and heavenly supernatural beings; see Index for occurrences); or (b) ‘greater and lesser lads of Homedal’.  (b) is probably to be preferred as the more natural reading, in which case ‘greater or lesser’ may refer literally to physical size (the younger and older members of the audience) or more figuratively to the various social classes.}} \hld\ \edtrans{\alst{m}ǫgu Hęimdalar;}{lads of Homedal \ken{men}}{\Bfootnote{Homedal sired the three castes of men, as told in \Rigsthula.}} &
\alst{v}ilt at, \alst{V}al-fǫðr, \hld\ \alst{v}ęl fram tęlja’k &
\alst{f}orn spjǫll \alst{f}ira, \hld\ \edtrans{þau’s \alst{f}ręmst of man}{which I foremost recall}{\Bfootnote{Cf. \Vafthrudnismal\ 34–35 with similar phrasing.}}?}{\lemma{ALL}\Bfootnote{The wallow begins by asking for the silence of both gods and men, a meristic expression \parencite[99--100]{West2007}.  The whole introductory formula has Indo-European parallels; see \textcite{West2007}[63,92-93,312].}}\eva

\bvb “For hearing I ask all holy races \ken{gods}, \\
greater and lesser lads of Homedal \ken{men}! \\
Wilt thou, Walfather \name{= Weden}, that I well tell forth \\
the ancient sayings of men which I foremost recall?\evb\evg


\bvg\bva\mssnote{\Regius~1r/4, \Hauksbok~20r/2}%
\alst{E}k man \alst{jǫ}tna \hld\ \alst{á}r of borna, &
þȧ’s \alst{f}orðum mik \hld\ \alst{f}ǿdda hǫfðu; &
\alst{n}íu man’k hęima, \hld\ \alst{n}íu \edtext{ïviðjur}{\Afootnote{so \Regius\Hauksbok.  \Regius\ has previously been as read \emph{iviði}, but this was disproven by an x-ray scan undertaken by \textcite{StefanKarlsson1979}.}}, &
\edtrans{\alst{m}jǫt-við \alst{m}ę́ran \hld\ fyr \alst{m}old neðan.}{the renowned Metwood beneath the soil.}{\Bfootnote{Probably \inx[L]{Ugdrassle’s Ash}, being still a seed.}}\eva

\bvb I recall \inx[G]{Ettins} born of yore, \\
they who formerly had nourished me. \\
Nine \inx[C]{Home}[Homes] I recall, nine \inx[G]{Inwithies}; \\
the renowned \inx[P]{Metwood} beneath the soil.\evb\evg


\bvg\bva\mssnote{\Regius~1r/6, \Hauksbok~20r/4, \GylfMS}%
\alst{Á}r vas \alst{a}lda \hld\ \edtrans{þar’s \alst{Y}mir byggði}{where Yimer dwelled}{\Afootnote{\emph{þat’s ękki vas} ‘that when nothing was’ \GylfMS}}, &
vas-a \alst{s}andr né \alst{s}ę́r, \hld\ né \alst{s}valar unnir; &
\edtext{\alst{jǫ}rð fannsk \alst{ę́}va \hld\ né \alst{u}pp-himinn}{\lemma{jǫrð \dots\ né upp-himinn ‘Earth \dots\ nor Up-heaven’}\Bfootnote{A well-attested formulaic cosmological word-pair found in all four Old Germanic languages with poetic traditions (ON, OE, OS, OHG), especially in concern the creation and destruction of the world. See \inx[F]{Earth and Upheaven}.}}; &
\edtrans{\alst{g}ap vas \alst{g}innunga}{there was the Gap of Ginnings}{\Bfootnote{The “Gap of Hawks”, that is, the midspace between Earth and Upheaven.  See Index.}}, \hld\ en \alst{g}ras \edtrans{hvęrgi}{nowhere}{\Afootnote{\emph{ękki} ‘not’ \Hauksbok}};\eva

\bvb It was early of ages where \inx[P]{Yimer} dwelled; \\
there was not sand nor sea nor cool waves. \\
\inx[L]{Earth} was never found, nor \inx[L]{Up-heaven}; \\
there was the \inx[L]{Gap of Ginnings}, but grass nowhere,\footnoteB{A more extensive creation narrative is found in \Gylfaginning\ 4–5, according to which the world first consisted of two extremities: the frozen Nivelham in the north and scorching Muspellsham in the south. From Nivelham the freezing venom-rivers called the \inx[L]{Ilewaves} ran until they froze to ice, while burning lava flowed from Muspellsham. The ice and lava met in the Gap of Ginnings, “which was as calm as windless air”, and there combined to form the first being, \inx[P]{Yimer}, who was the ancestor of the ettins.}\evb\evg


\bvg\bva\mssnote{\Regius~1r/8, \Hauksbok~20r/5}%
áðr \edtrans{\alst{B}urs synir}{the Sons of Byre}{\Bfootnote{In \Gylfaginning\ 6 identified as Weden, Will and Wigh, who sacrificed Yimer and shaped the cosmos out of his body. For this see also \Vafthrudnismal\ 20–21 and \Grimnismal\ 41–42.}} \hld\ \alst{b}jǫðum of ypðu, &
þęir es \alst{M}ið-garð \hld\ \alst{m}ę́ran skópu; &
\alst{s}ól skęin \alst{s}unnan \hld\ ȧ \alst{s}alar stęina; &
þȧ vas \alst{g}rund \alst{g}róin \hld\ \edtrans{\alst{g}rø̇num lauki}{green leek}{\Bfootnote{A sign of the golden age, since the leek was believed to be the noblest plant and had important cultural significance.  This is seen from \GudrunTwo\ 2, where \inx[P]{Siward}’s superiority to the \inx[P]{Yivickings} is compared to a stag among wild beasts, gold among silver, and a green leek in grass. The leek was valued in folk magic, as seen already on gold bracteates from the C5th and C6th, where it appears as a charm word in the form {ᛚᚨᚢᚲᚨᛉ} \emph{laukaʀ}, in one inscription paired with {ᛚᛁᚾᚨ} \emph{lína} ‘linen’.  Classical Norse attestations of magic use include \Sigrdrifumal\ 8, where the leek is thrown into mead against poison; and the \Volsathattr, where a horse penis is said to be \emph{\alst{l}íni gǿddr \hld\ en \alst{l}aukum studdr} ‘endowed with linen and supported by leeks’ in a poetic line.  The leek was particularly associated with women and domestic life, as seen by its pairing with “linen”.  Kennings for women frequently have the leek as a determinant (TODO: Meissner reference?), and Anon \emph{Sveinfl} 1 (\Skp\ I TODO.) sarcastically states that a battle was not \emph{sem manni \hld\ mę́r lauk eða ǫl bę́ri} ‘as if a maiden brought a man leek or ale’.}}.\eva

\bvb before the \inx[P]{Sons of Byre} uplifted the flatlands, \\
they who shaped renowned \inx[L]{Middenyard}. \\
Sun shone from the south on the stones of the hall; \\
then was the ground grown with green leek.\evb\evg


\bvg\bva\mssnote{\Regius~1r/11, \Hauksbok~20r/7, \GylfMS}%
\edtext{\alst{S}ól varp \alst{s}unnan, \hld\ \edtrans{\alst{s}inni Mȧna}{Moon’s companion}{\Bfootnote{At times translated as ‘its moon’. This cannot be correct, as \emph{mȧni} ‘moon’ is masculine, while \emph{sinni}, dat. sg. of \emph{sïnn} ‘its (reflexive)’ is feminine.}}, &
\alst{h}ęndi hinni \alst{h}ǿgri \hld\ of \edtrans{\alst{h}imin-jǫður}{heaven’s rim}{\Afootnote{composite; \emph{himin †iodyr†} \Regius; \emph{ioður} \Hauksbok.}\Bfootnote{Some recent editors have taken it upon themselves to normalize the reading of \Regius\ as \emph{himin-jó-dýr} ‘heaven-horse-beast’, which is not just nonsensical but also unmetrical due the stress pattern.  On the other hand the reading of \Hauksbok, normalized to \emph{jǫður} ‘rim, edge’, is clearly deficient since it lacks the neccessary alliteration on \emph{h}.  If we see \emph{iodyr} \Regius\ as corrupted from \emph{*iodur} we can restore \emph{himin-jǫður}, as done here.}}}{\lemma{Sól \dots\ himin-jǫður ‘Sun \dots\ heaven’s rim’}\Afootnote{om. \GylfMS.}\Bfootnote{Probably a poetic description of the dawn; the Sun lifted herself up over the horizon and rose for the first time.}}; &
\alst{S}ól þat né vissi, \hld\ hvar hǫ̇n \alst{s}ali átti; &
\edtext{\alst{st}jǫrnur þat né vissu, \hld\ hvar þę́r \alst{st}aði ǫ́ttu}{\lemma{stjǫrnur \dots\ ǫ́ttu}\Afootnote{In \GylfMS\ this line comes last, so that the order is sun, moon, stars.}}; &
\edtext{\alst{M}ȧni þat né vissi, \hld\ hvat hann \alst{m}ęgins átti.}{\lemma{Mȧni \dots\ átti ‘Moon \dots\ had’}\Bfootnote{The moon was believed to have supernatural powers and could be invoked in conflict (cf. \Havamal\ 137/7.)}}
\eva

\bvb The Sun cast from the south—the Moon’s companion— \\
her right hand over heaven’s rim. \\
The Sun knew not where halls she had; \\
the stars knew not where seats they had; \\
the Moon knew not what sort of might he had.\evb\evg


\bvg\bva\mssnote{\Regius~1r/13, \Hauksbok~20r/9}%
\edtext{Þȧ gingu \alst{r}ęgin ǫll \hld\ ȧ \edtrans{\alst{r}ǫk-stóla}{rake-seats}{\Bfootnote{Their seats of judgment at the \inx[L]{Thing of the Gods}.}}, &
\alst{g}inn-hęilǫg \alst{g}oð, \hld\ ok umb þat \alst{g}ę́ttusk}{\lemma{Þȧ \dots\ gę́ttusk ‘Then \dots\ of this.’}\Bfootnote{A formulaic expression for the convening of the \inx[L]{Thing of the Gods}, identically repeated below in sts. 9/1–2, 22/1–2, and 24/1–2.  Cf. also the formula shared between \Baldrsdraumar\ 1/1–3 and \Thrymskvida\ 14/1–3, which follows the structure of the present formula very closely: \emph{Sęnn vǫ́ru ę̇sir \hld\ allir ȧ þingi // ok ǫ̇synjur \hld\ allar ȧ máli, // ok umb þat réðu \hld\ ríkir tívar.} ‘Soon were the \inx[G]{Eese} all at the \inx[C]{Thing}, // and the \inx[G]{Ossens} all at speech, // and of this counseled the mighty \inx[G]{Tews}.’

In the five occurrences of these two formulae outside of the present stanza, the demonstrative pronoun \emph{þat} ‘this’ clearly refers to an immediately following question introduced by a \emph{hv}-word (e.g. \Thrymskvida\ 14/4: \emph{hvé þęir Hlórriða \hld\ hamar of sǿtti?} ‘how they Loride’s \name{= Thunder’s} hammer would find?’)  Following this pattern we would expect to find such a question after \emph{umb þat gę́ttusk} ‘took counsel of this’ in the present stanza, and it seems most likely to presume that they have been lost in transmission.}}. &
\edtext{\alst{N}ǫ́tt ok \alst{n}iðjum \hld\ \alst{n}ǫfn of gǫ́fu, &
\alst{m}orgin hétu \hld\ ok \alst{m}iðjan dag, &
\alst{u}ndurn ok \alst{a}ptan, \hld\ \alst{ǫ́}rum at tęlja.}{\lemma{Nǫ́tt \dots\ tęlja ‘To night \dots\ tally’}\Bfootnote{Cf. \Vafthrudnismal\ 23, where it is said that the sun and moon turn round in heaven \emph{ǫldum at ár-tali} ‘for mankind’s tally of years’, and 25, where it is said that the Reins created the moon-phases for the same purpose.}}\eva

\bvb Then went the Reins all onto the rake-seats: \\
the Yin-holy Gods, and from each other took counsel of this. \\
To night and the moon-phases names they gave; \\
morning they named, and middle day, \\
afternoon and evening, the years for to tally.\evb\evg


\bvg\bva\mssnote{\Regius~1r/16, \Hauksbok~20r/10}%
Hittusk \alst{ę̇}sir \hld\ ȧ \alst{I}ða-vęlli, &
\edtext{þęir’s \alst{h}ǫrg ok \alst{h}of \hld\ \alst{h}ǫ́-timbruðu}{\lemma{þęir’s \dots\ hǫ́-timbruðu ‘they who \dots\ timbered on high’}\Afootnote{\emph{afls kostuðu \hld\ alls freistuðu} ‘[their] strength they tried; everything they tempted’ \Hauksbok}\Bfootnote{Two formulæ. \emph{hǫrgr ok hof} ‘harrow and hove’ is a merism, i.e. ritual structures made of stone and wood; cf. \Vafthrudnismal\ 38 and \HelgakvidaHjorvardssonar\ TODO, as well as the Norwegian Christian laws that impose ‘the burning of hoves and the breaking of harrows’ (\emph{brenna hof ok brjóta hǫrga}).  \emph{hǫ́-timbra} ‘high timber, timber on high’ is a rare compound and only occurs at one other place in the ON corpus, viz. in \Grimnismal\ 16, where it describes a harrow ruled by Nearth.

This line has often been wondered at; why would the Gods themselves make cultic buildings?  Yet they partake in ritual slaughter of beasts, divination, and feasting (e.g. \Voluspa\ 61, \Hymiskvida\ 1, 39, \Lokasenna, \Haustlong\ 2), and their deeds form the precedent for upright human behaviour.}}; &
\alst{a}fla lǫgðu, \hld\ \alst{au}ð smíðuðu, &
\alst{t}angir skópu \hld\ ok \alst{t}ól gęrðu.\eva

\bvb The Eese found each other on the \inx[L]{Idewolds}, \\
they who \inx[C]{harrow} and \inx[C]{hove} timbered on high; \\
hearths they laid, wealth they smithed, \\
tongs they shaped and tools they made.\evb\evg


\bvg\bva\mssnote{\Regius~1r/18, \Hauksbok~20r/12}%
\edtext{\edtrans{\alst{T}ęflðu}{played Tables}{\Bfootnote{A verb derived from \emph{tafl} ‘board game’, an old borrowing from Latin \emph{tabula}.  “Tables” is used as a cognate translation; the exact type of board game referred to is unimportant.}} ï \alst{t}u̇ni, \hld\ \alst{t}ęitir vǫ́ru, &
\edtrans{\alst{v}as þęim \alst{v}étter-gis \hld\ \alst{v}ant ór gulli}{for them was nothing golden wanting}{\Bfootnote{Indeed even the bricks they played with were of gold. See st. 58.}}, &
unds \edtext{\alst{þ}ríar kvǫ̇mu \hld\ \alst{þ}ursa męyjar}{\lemma{þríar \dots\ þursa męyjar ‘three maidens of Thurses’}\Bfootnote{These three maidens are never mentioned again (unless they are taken to be the three norns in st. 19, but they would then be introduced twice). It is possible that an additional stanza giving further information about them has been lost. If it originally existed, it was already absent from the version employed by the author of \Gylfaginning, who gives no new information.}}, &
\edtrans{\alst{ȧ}m-átkar}{uncanny}{\Bfootnote{The word \emph{ám-áttigr} has a clear association with supernatural beings; trolls and ettins. It occurs in four other places in \Regius. In \Grimnismal\ 11, \Skirnismal\ 10 and \HelgakvidaHjorvardssonar\ 17 it modifies \emph{jǫtunn} ‘ettin’ in a \Ljodahattr\ c-line. In \HelgakvidaHjorvardssonar\ 14 it is used by the daughter of an ettin to refer to a human hero.}} mjǫk, \hld\ ór \alst{Jǫ}tun-hęimum.}{\lemma{ALL}\Bfootnote{The whole stanza is paraphrased in \Gylfaginning\ ch. 14:
\begin{quote}\emph{Ok því nę́st smíðuðu þeir málm ok stein ok tré ok svá gnóg-liga þann málm, er gull heitir, at ǫll bús-gǫgn ok ǫll reiði-gǫgn hǫfðu þeir af gulli, ok er sú ǫld kǫlluð gull-aldr, áðr en spilltist af til-kvámu kvinnanna; þę́r kómu ór Jǫtun-heimum.}

‘And after this they smithed ore and stone and wood, and so abundantly [did they smith] that ore which is called gold, that all their house tools and riding tools were golden. And that age is called the golden age, before it was spoiled by the arrival of the women; they came from Ettinham.’\end{quote}
after which he describes the creation of the dwarfs (see next stanza).}}\eva

\bvb They played \inx[C]{Tables} in the yard; merry were they; \\
for them was nothing golden wanting— \\
until three maidens of \inx[G]{Thurses} came, \\
most uncanny, from \inx[L]{Ettinham}.\evb\evg

\sectionline

\bvg\bva\mssnote{\Regius~1r/20, \Hauksbok~20r/14, \GylfMS}%
Þȧ gingu \alst{r}ęgin ǫll \hld\ ȧ \alst{r}ǫk-stóla, &
\alst{g}inn-hęilǫg \alst{g}oð, \hld\ ok umb þat \alst{g}ę́ttusk: &
\edtrans{Hvęrr skyldi \alst{d}verga}{Who would \dots\ of dwarfs’}{\Afootnote{so \Regius\Wormianus\Upsaliensis; \emph{at skyldi dverga} ‘That they would \dots\ of dwarfs’ \RegiusProse\Trajectinus; \emph{hverir skyldu dvergar} ‘Which dwarfs would [shape the retinues]’ \Hauksbok}} \hld\ \edtrans{\alst{d}rótt}{the retinue}{\Afootnote{so \GylfMS; \emph{drotin} ‘the lord’ \Regius; \emph{dróttir} ‘the retinues’ \Hauksbok}} \edtrans{of skępja}{shape}{\Afootnote{\emph{spekia} ‘soothe’ \Upsaliensis}} &
\edtext{ór \edtrans{\alst{b}rimi \alst{b}lóðgu}{bloody surf}{\Afootnote{so \Hauksbok\RegiusProse\Wormianus\Upsaliensis; \emph{Brimis blóði} ‘the blood of Brimmer’ \Regius\Trajectinus}} \hld\ ok ór \edtrans{\alst{b}lǫ́um}{blue-black}{\Afootnote{metr. emend. from \emph{blám} \Regius; \emph{Bláins} ‘Blown’s’ \Hauksbok\Wormianus; \emph{Bláms} \RegiusProse\Trajectinus\Upsaliensis\ is prob. a corrupt form of \emph{Bláins}}} lęggjum?}{\lemma{ór brimi \dots\ lęggjum ‘out of the bloody \dots\ legs’}\Bfootnote{I think that the poem simply telling of “the bloody surf” and “the blue-black legs” fits better with its general allusive style, but the resulting composite reading may be somewhat controversial.

According to \Gylfaginning\ 14 the dwarfs first originated as maggots in the corpse of Yimer, out of whose bones the rocks were made (\Grimnismal\ 41, \Vafthrudnismal\ 21).  Dwarfs dwell in the rocks and earth; cf. for instance \Ynglingatal\ 2, where the Swedish king Swayther (\emph{Svęigðir} disappears into a rock in pursuit of a dwarf.  More difficult to explain is the creation of dwarfs out of Yimer’s blood (from which was made the sea, \Grimnismal\ 41, \Vafthrudnismal\ 21), since dwarfs are never said to dwell in water. — If one chooses the reading \emph{Bláinn} ‘Blown’ (named in the \inx[C]{thule}[thules] as a dwarf) instead of \emph{blǫ́um} ‘blue-black’, then following Gurevich (\emph{Skp} 2017, p. 693) one may see a kenning “the legs of Blown \name{dwarf} \ken{stone}”. Blown has otherwise been read as a poetic name for Yimer, but that is never attested elsewhere.}}\eva

\bvb Then went the Reins all onto the rake-seats: \\
the Yin-holy Gods, and from each other took counsel of this: \\
Who would shape the retinue of \inx[G]{Dwarfs}, \\
from the bloody surf and from the blue-black legs?\evb\evg


\bvg\bva\mssnote{\Regius~1r/21, \Hauksbok~20r/15, \GylfMS}%
\edtext{\edtext{Þar vas \alst{M}óðsognir}{\Afootnote{so \Hauksbok; \emph{Þar †mótſognir vitnir†} ‘there Mootsowner wolf(?)’ \Regius. The prose of \Gylfaginning\ 14 agrees with \Hauksbok\ that the correct form of the name is \emph{Móðsognir}, not \emph{Mótsognir}.}} \hld\ \alst{m}ę́tstr of orðinn &
\alst{d}verga allra, \hld\ en \alst{D}urinn annarr;}{\lemma{Þar \dots\ annarr ‘There \dots\ second’}\Bfootnote{om. \GylfMS, but the author must have had the full verse, since he paraphrases these lines in the following way: \emph{Móðsognir var ę́ðstr ok annarr Durinn.} ‘Moodsowner was the highest in rank, and Dorn the second.’ before citing}} &
\edtext{\edtext{þęir \alst{m}an-líkun \hld\ \alst{m}ǫrg of gęrðu,}{\lemma{þęir \dots\ gęrðu ‘They \dots\ did make’}\Afootnote{so \Regius\Hauksbok\Upsaliensis; \emph{þar man-líkun \hld\ mǫrg of gęrðusk} ‘There man-likenesses many were made’ \RegiusProse\Trajectinus\Wormianus}} &
\alst{d}vergar \edtrans{ï}{in}{\Afootnote{so \GylfMS\Hauksbok; \emph{ór} ‘out of’ \Regius}} jǫrðu, \hld\ \edtrans{sęm \alst{D}urinn sagði}{as Dorn said}{\Afootnote{so \Regius\Hauksbok\RegiusProse\Wormianus; \emph{sem †dur menn† sagði} ‘as door-men(?) said’ \Trajectinus; \emph{sem †þeim dyrinn kendi†} ‘as the beasts(?) taught them’ \Upsaliensis}}.}{\lemma{þęir \dots\ sagði ‘They \dots\ said.’}\Bfootnote{There are two conflicting interpretations of the creation of the dwarfs. Either they arose on their own; this is supported by the prose of \Gylfaginning\ (see note to previous st.) and by the form of the stanza quoted there (but it may have been changed to correspond to the author’s vision). On the other hand, both \Regius\ and \Hauksbok\ have the dwarfs Moodsowner and Dorn shaping “man-likenesses” out of soil. The present edition follows the second version.}}\eva

\bvb There was Moodsowner made the worthiest \\
of all dwarfs, but Dorn [was] second. \\
They man-likenesses many did make: \\
dwarfs in the earth, as Dorn said.\evb\evg

\sectionline

{\small Sts. 11–15 contain two originally distinct lists of dwarf-names; part of them are almost certainly later inserts.  There is a repetition of names (Oakenshield, Great-grandfather), and more than one formulaic conclusion.

Sts. 11–13, having no repeated names, seem to belong together. If they do, st. 12, which contains the formulaic conclusion to the list, should probably switch places with 13.

Sts. 14–15 form the second group, having an introduction and a conclusion which both mention the dwarf Loffer.}

%TODO: move these stanzas to appendix?
\bvg\bva\mssnote{\Regius~1r/23, \Hauksbok~20r/17, \GylfMS}%
\alst{N}ýi ok \alst{N}iði, \hld\ \alst{N}orðri, Suðri, &
\alst{Au}stri, Vestri, \hld\ \alst{A}l-þjófr, Dvalinn, &
\alst{B}ívurr, \alst{B}ávurr, \hld\ \alst{B}ǫmburr, Nóri, &
\alst{Ȧ}nn ok \alst{Ȧ}narr, \hld\ \alst{Á}i, Mjǫð-vitnir.\eva

\bvb New and Nithe, Norther and Souther, \\
Easter and Wester, Allthief, Dwollen, \\
Bewer, Bower, Bamber, Noor, \\
Own and Owner, Great-grandfather, Meadwitner.\evb\evg


\bvg\bva\mssnote{\Regius~1r/25, \Hauksbok~20r/18, \GylfMS}%
\alst{V}ęigr ok Gand-alfr, \hld\ \alst{V}ind-alfr, Þráinn, &
\alst{Þ}ękkr ok \alst{Þ}orinn, \hld\ \alst{Þ}rór, Vitr ok Litr, &
\alst{N}ár ok \alst{N}ý-ráðr— \hld\ \alst{n}ú hęf’k dverga &
—\alst{R}ęginn ok \alst{R}áð-sviðr— \hld\ \alst{r}étt of talða.\eva

\bvb Wey and Gandelf, Windelf, Thrown, \\
Thetch and Thorn, Threw, Wit and Lit, \\
Nee and Newred—now have I the dwarfs— \\
Rain and Redswith—rightly tallied.\evb\evg


\bvg\bva\mssnote{\Regius~1r/28, \Hauksbok~20r/20, \GylfMS}%
\alst{F}íli, Kíli, \hld\ \alst{F}undinn, Náli, &
\alst{H}ępti, Víli, \hld\ \alst{H}annarr, Svíurr, &
\alst{F}rár, Horn-bori, \hld\ \alst{F}rę́gr ok Lȯni, &
\alst{Au}r-vangr, \alst{Ja}ri, \hld\ \alst{Ęi}kin-skjaldi.\eva

\bvb Filer, Chiler, Found and Needler, \\
Hefter, Wiler, Hanner, Swigher, \\
Fraw, Hornborer, Fray and Looner, \\
Earwong, Earer, Oakenshield.\evb\evg


\bvg\bva\mssnote{\Regius~1r/30, \Hauksbok~20r/22, \GylfMS}%
Mál es \alst{d}verga \hld\ ï \alst{D}valins liði &
\alst{l}jȯna kindum \hld\ til \alst{L}ofars tęlja, &
\edtext{þęir}{\Afootnote{\emph{þeim} \Hauksbok}} es \alst{s}óttu \hld\ frȧ \alst{s}alar stęini &
\alst{Au}r-vanga sjǫt \hld\ til \alst{Jǫ}ru-valla.\eva

\bvb ’Tis time to tally the dwarfs in Dwollen’s troops \\
{[back]} to Loffer for the races of men;\footnoteB{A standard genealogical introduction (cf. \Haleygjatal\ 1: \emph{meðan hans ę́tt \dots\ til goða tęljum} ‘while we tally his line \dots\ [back] to the gods’).  The (patrilineal) line of dwarfs is to be counted back to their progenitor, Loffer.  This possibly disagrees with st. 10, where Moodsowner is said to be the foremost (and presumably the oldest) of the dwarfs, and Loffer is not mentioned, but such details were probably not very important.} \\
they who sought, from the stone of the hall, \\
the abode of the \inx[L]{Earwongs} to the \inx[L]{Erwolds}.\footnoteB{Cf. \Gylfaginning\ 14: “But these came from Swornshigh (\emph{Svarinshaugr}) to the Earwongs on the Erwolds, and thence Lofer is come; these are their names: Sherper (\emph{Skirpir}), Werper (\emph{Virpir}), Showfind, Great-grandfather, Elf and Ing (\emph{Ingi}), Oakenshield, Fale (\emph{Falr}), Frost, Finn, Ginner.”}\evb\evg


\bvg\bva\mssnote{\Regius~1r/32, \Hauksbok~20r/24, \GylfMS}%
Þar vas \alst{D}raupnir \hld\ ok \alst{D}olg-þrasir, &
\alst{H}ár, \alst{H}aug-spori, \hld\ \alst{H}lé-vangr, Glói, &
\alst{Sk}irfir, Virfir, \hld\ \alst{Sk}áfiðr, Ái, &
\alst{A}lfr ok \alst{Y}ngvi, \hld\ \alst{Ęi}kin-skjaldi, &
\alst{F}jalarr ok \alst{F}rosti, \hld\ \alst{F}innr ok Ginnarr; &
Þat mun \edtext{\alst{ę́}}{\Afootnote{om. \Regius}} \alst{u}ppi, \hld\ meðan \alst{ǫ}ld lifir, &
\alst{l}ang-niðja-tal \hld\ \edtext{til}{\Afootnote{om. \Hauksbok}} \alst{L}ofars hafat.\eva

\bvb There was Dreepner and Dollowthrasher, \\
High, Highspurer, Leewong, Glower, \\
Sherver, Werver, Showfind, Great-grandfather, \\
Elf and Ing, Oakenshield, \\
Feller and Frost, Finn and Ginner.— \\
It will ever be remembered while the age lives,\footnoteB{Two archaic formulæ. The first literally ‘that will ever [be] up above’, cf. \HervararSaga\ TODO: “We two are cursed, brother, thy bane am I become! That will ever be remembered (\emph{þat mun ę́ uppi}, but both mss. \emph{þat mun enn uppi}), evil is the doom of the norns!” The second is found in a runic inscription, U 323 (980–1015): “Ever will lie—while the age lives (\textbf{meþ + altr + lifiʀ} \emph{með aldr lifir})—the hard-hammered bridge, broad, after a good man.” An especially close parallel is found in Þstf \emph{Stuttdr} (st. 5, Kari Ellen Gade ed. in \Skp\ II): \emph{Ęy mun uppi \hld\ Ęndils, meðan stęndr // sól-borgar salr, \hld\ svǫr-gǿðis fǫr.} ‘Always will be remembered—while the hall of the sun’s stronghold \ken{sky/heaven > earth} stands—the journey of the fattener of Andle’s bird \ken{raven/eagle > warrior}.’} \\
the tally of kinsmen lifted to Lofer.\evb\evg

\sectionline

\bvg\bva\mssnote{\Regius~1v/1, \Hauksbok~20r/26}%
\edtrans{Unds}{Until}{\Bfootnote{We seem to be missing a preceding sentence here, probably being contained in a now-lost stanza.  What this st. would have contained is of course impossible to know, but it may have given a reason for the creation of men.}} \edtext{\alst{þ}rír}{\Afootnote{gramm. emend.; \emph{þrjár} \Regius\Hauksbok}} kvǫ̇mu \hld\ \edtext{ór \alst{þ}ví liði}{\Afootnote{\emph{þussa brúðir} ‘brides of thurses’ \Hauksbok\ is probably corrupt due to the influence of st. 8; the adjectives in l. 2 are in the masculine.}} &
\edtrans{\alst{ǫ}flgir ok \alst{ȧ}stkir}{strong and lovely}{\Afootnote{\emph{ȧstkir ok ǫflgir} (norm.) ‘lovely and strong’ \Hauksbok}} \hld\ \alst{ę̇}sir \edtrans{at húsi}{along the settlement}{\Bfootnote{An adverbial, lit. ‘along the house’; the gods were not walking in the wilderness.}}; &
fundu ȧ \alst{l}andi \hld\ \alst{l}ítt męgandi &
\alst{A}sk ok \alst{E}mblu \hld\ \alst{ø}r-lǫg-lausa.\eva

\bvb Until three came out of that host: \\
strong and lovely Eese along the settlement; \\
they found on land the little availing \\
Ash and Emble, \inx[C]{orlay}-less.\footnoteB{This verse is paraphrased in \Gylfaginning\ 9: \emph{Þá er þeir gengu með sę́var-strǫndu Bors synir, fundu þeir tré tvau ok tóku upp trén ok skǫpuðu af menn. Gaf inn fyrsti ǫnd ok líf, annarr vit ok hrę́ring, þriði á-sjónu, mál ok heyrn ok sjón, gáfu þeim klę́ði ok nǫfn. Hét karl-maðrinn Askr, en konan Embla, ok ólst þaðan af mann-kindin, sú er byggðin var gefinn undir Mið-garði.} ‘When the sons of Byre (cf. st. 4) walked along the sea-shore they found two trees and they took up the trees and shaped men from them. The first one gave breath (\emph{ǫnd}) and life, the second wit and movement, the third sight, speech, appearance and sight; they gave them clothes and names. The male was called Ash, and the woman Emble, and from them mankind was begotten, to whom were given the dwelling within Middenyard.’

The ON cognate of tree, \emph{tré}, can also mean ‘pieces of wood’, and it is traditionally seen as referring to pieces of driftwood. Yet as pointed out by \textcite{Hultgård2006} the comparative evidence suggests that the two were in fact living, growing trees (they would thus be part of the foliage described in st. 4) and there is nothing in the sources that speaks against this.

While Ash is easily identified with the same-named wood species (\emph{Fraxinus excelsior}), the etymology of Emble is much more difficult. The shaping of men from trees is used by poets in various kennings for men and women, especially in Scaldic poetry (for a short discussion see \textciteshorttitle{SkP} I, p. lxxv ff.). While this is rarer in the Eddic corpus it does occur, e.g. in \Sigrdrifumal\ 5: \emph{bryn-þings apaldr} ‘apple-tree of the byrnie-\inx[C]{Thing} \ken{battle > warrior}’.}\evb\evg


\bvg\bva\mssnote{\Regius~1v/3, \Hauksbok~20r/27}%
\alst{Ǫ}nd þau né \alst{ǫ́}ttu, \hld\ \alst{ó}ð þau né hǫfðu, &
\alst{l}ǫ́ né \alst{l}ę́ti \hld\ né \alst{l}itu góða; &
\alst{ǫ}nd gaf \alst{Ó}ðinn, \hld\ \alst{ó}ð gaf Hø̇nir, &
\alst{l}ǫ́ gaf \alst{L}óðurr \hld\ ok \alst{l}itu góða.\eva

\bvb Breath they owned not, \inx[C]{wode} they had not, \\
not craft nor sound nor good countenance. \\
Breath gave Weden, wode gave Heener, \\
craft gave Lother, and good countenance.\evb\evg

\sectionline

\bvg\bva\mssnote{\Regius~1v/5, \Hauksbok~20r/29, \GylfMS}%
\alst{A}sk vęit’k \edtext{standa}{\lemma{standa ‘standing’}\Afootnote{so \Regius\Hauksbok\Upsaliensis; \emph{ausinn} ‘poured, sprinkled’ \RegiusProse\Trajectinus\Wormianus}}, \hld\ hęitir \edtext{\alst{Y}gg-drasill}{\Afootnote{\emph{Ygg-drasils} \RegiusProse}}, &
\alst{h}ǫ́r \edtrans{baðmr}{beam}{\Afootnote{\emph{borinn} ‘born’ \Upsaliensis\ is wo. doubt corrupt.}}, \edtrans{ausinn}{poured}{\Afootnote{\emph{hęilagr} ‘holy’ \GylfMS}} \hld\ \alst{h}víta auri; &
þaðan koma \alst{d}ǫggvar \hld\ \edtext{þę́r’s}{\Afootnote{\emph{es} \RegiusProse\Trajectinus}} ï \alst{d}ala falla; &
stęndr \edtext{\alst{ę́}}{\Afootnote{\emph{om.} \Upsaliensis}} \alst{y}fir \edtext{grø̇nn}{\Afootnote{\emph{†grvnn†} \RegiusProse; \emph{†grein†} \Upsaliensis}} \hld\ \alst{U}rðar brunni.\eva

\bvb An ash I know standing, ’tis called \inx[L]{Ugdrassle}; \\
a high beam \ken{tree}, poured with white mud.\footnoteB{i.e. ‘white mud is (or has been) poured upon it.’ Possibly relevant is the Indian ritual pouring of beverages onto the phallic \emph{lingam} (though the good Nikhil S. Dwibhashyam denies that this goes back to the Vedic period, and so it may be unrelated). For the whole passage cf. st. 26.} \\
Thence come the dew-drops which fall in the dales; \\
it stands ever green over \inx[L]{Weird’s Well}.\evb\evg


\bvg\bva\mssnote{\Regius~1v/8, \Hauksbok~20r/31}%
Þaðan koma \alst{m}ęyjar \hld\ \alst{m}args vitandi &
\alst{þ}ríar ór þęim \edtrans{sal}{hall}{\Afootnote{so \Hauksbok, \GylfMS\ (paraphrase); \emph{sę́} ‘lake’ \Regius}} \hld\ es \edtrans{und}{under}{\Afootnote{\emph{ȧ} ‘on’ \Hauksbok}} \edtrans{\alst{þ}olli}{tree}{\Bfootnote{Literally ‘fir’, but the word is only used for the alliteration.  The same may perhaps apply to \emph{askr} ‘ash’ above, the species being indeterminate.}} stęndr; &
\alst{U}rð hétu \alst{ęi}na, \hld\ \alst{a}ðra Verðandi, &
—\edtrans{\alst{sk}ǫ́ru ȧ \alst{sk}íði}{they scored billets}{\Bfootnote{Unclear; perhaps they carve markings for the number of years each man has to live.}}— \hld\ \alst{Sk}uld hina þriðju &
þę́r \alst{l}ǫg \alst{l}ǫgðu, \hld\ þę́r \alst{l}íf køru, &
\alst{a}lda bǫrnum, \hld\ \alst{ø}r-lǫg \edtrans{sęggja}{of youths}{\Afootnote{\emph{at sęgja} ‘to say’ \Hauksbok}}.\eva

\bvb Thence come maidens, much knowing: \\
three from the hall which stands under the tree. \\
Weird they called one, the other Werthing \\
—they scored billets—Shild the third. \\
Laws they laid, lives they chose \\
for the children of mankind, the \inx[C]{orlay} of youths.\footnoteB{i.e. ‘they have carved on boards, they have laid laws, they have chosen lives’. It is well known that in Old Norse as in other old Germanic languages the simple past can have both perfective and imperfective sense. — This st. is paraphrased in \Gylfaginning\ 15: \emph{Þar stendr salr einn fagr undir askinum við brunninn, ok ór þeim sal koma þrjár meyjar, þę́r er svá heita: Urðr, Verðandi, Skuld. Þessar meyjar skapa mǫnnum aldr; þę́r kǫllum vér nornir.} ‘There is a single fair hall beneath the ash-tree by the well, and from that hall come three maidens, who are called thus: Weird, Werthing, Shild. These maidens shape the ages of men (formulaic! TODO.); we call them norns.’}\evb\evg

\sectionline

\bvg\bva\mssnote{\Regius~1v/11, \Hauksbok~20v/5}%
Þat man hǫ̇n \edtrans{\alst{f}olk-víg}{troop-conflict}{\Bfootnote{\emph{folk} here carries its older meaning ‘troop, band’, as seen in the Slavic borrowing exemplified by Russian \textrussian{полк} ‘regiment, host, army’.}} \hld\ \alst{f}yrst ï hęimi, &
es \alst{G}ull-vęigu \hld\ \alst{g}ęirum studdu &
ok ï \alst{h}ǫll \alst{H}áars \hld\ \alst{h}ȧna bręnndu, &
\edtext{\alst{þ}rysvar bręnndu}{\Afootnote{\emph{†þrysvar brendv þrysvar brendv†} \Hauksbok}} \hld\ \alst{þ}rysvar borna, &
\alst{o}pt, \alst{ȯ}-sjaldan, \hld\ þó hǫ̇n \alst{ę}nn lifir.\eva

\bvb That troop-conflict she recalls, the first in the \inx[C]{Home}, \\
when Goldwey with spears they goaded, \\
and in the hall of \inx[P]{Higher} \name{= Weden} \ken*{= Walhall} burned her; \\
thrice they burned the thrice born, \\
often, unseldom, though she still lives.\footnoteB{Very cryptic. TODO: double check Snorri. Goldwey was apparently slain, cremated and reborn three times (in short succession?) by the Eese.}\evb\evg


\bvg\bva\mssnote{\Regius~1v/13, \Hauksbok~20v/7}%
\alst{H}ęiði \alst{h}étu, \hld\ hvar’s til \alst{h}úsa kom, &
\edtext{\alst{v}ǫlu}{\Afootnote{\emph{ok vǫlu} \Hauksbok}} \alst{v}ęl-spáa, \hld\ \alst{v}itti ganda; &
\alst{s}ęið hǫ́n \edtrans{hvar’s hǫ́n kunni}{where she could}{\Afootnote{so \Hauksbok; \emph{hǫ́n kunni} ‘she knew’ \Regius}}, \hld\ \alst{s}ęið hǫ́n \edtrans{hug lęikinn}{deluded minds}{\Afootnote{so \Hauksbok; \emph{leikinn} \Regius}}; &
\alst{ę́} vas hǫ̇n \alst{a}ngan \hld\ \alst{i}llrar brúðar.\eva

\bvb Heath they called—where to houses she came— \\
a well-spaeing \inx[C]{wallow}; she bewitched \inx[C]{gand}[gands]. \\
She sorcered where she could; she sorcered deluded minds; \\
she was always the love of any evil bride.\evb\evg

\sectionline

\bvg\bva\mssnote{\Regius~1v/16, \Hauksbok~20v/9}%
Þȧ gingu \alst{r}ęgin ǫll \hld\ ȧ \alst{r}ǫk-stóla, &
\alst{g}inn-hęilǫg goð, \hld\ ok umb þat \alst{g}ę́ttusk: &
Hvárt skyldu \alst{ę̇}sir \hld\ \alst{a}f-ráð gjalda, &
eða skyldu \edtrans{\alst{g}oð’in}{the Gods}{\Bfootnote{The clitic definite is very rare in older Norse poetry; this is its only occurence in \Voluspa.}} ǫll \hld\ \alst{g}ildi ęiga?\eva

\bvb Then went the Reins all onto the rake-seats: \\
the Yin-holy Gods, and from each other took counsel of this: \\
Whether the Eese should yield tribute, \\
or should the Gods all hold a banquet?\evb\evg


\bvg\bva\mssnote{\Regius~1v/17, \Hauksbok~20v/11}%
\alst{F}lęygði Óðinn \hld\ ok ï \alst{f}olk of skaut; &
þat vas ęnn \alst{f}olk-víg \hld\ \edtrans{\alst{f}yrr}{earlier}{\Afootnote{so \Hauksbok; \emph{fyrst} ‘first’ \Regius. The \Regius\ reading cannot be correct as this st. is describing a different war, and thus not the first. It has probably arisen due to the similarity with st. 20/1.}} ï hęimi; &
\alst{b}rotinn vas \alst{b}orð-vęggr \hld\ \alst{b}orgar ȧsa, &
knǫ́ttu \alst{v}anir \alst{v}íg-spǫ́ \hld\ \alst{v}ǫllu sporna.\eva

\bvb Weden hurled and shot into the troop;\footnoteB{The object, a spear, is understood. This seems to reference a ritual, well-attested in the literature, wherein a war-chief would dedicate an opposing army as a human sacrifice to Weden by throwing a spear over them, typically with the incantation \emph{Óðinn á yðr alla} ‘Weden owns you all!’; he would then own the battle-slain in that they joined him as \inx[G]{Oneharriers} in \inx[L]{Walhall}. Weden is also described as “owning” dead men in \Harbardsljod\ 24 (namely slain nobles, contrasted with \inx[P]{Thunder} who is insultingly said to “own the kin of thralls”) and in runic inscription \emph{N B380} (edited below under Galders), a sort of greeting wherein the receiver is wished to be owned by Weden (and “received” by Thunder). For further literature see \textciteshorttitle{PCRN-HS} II:24, p. 560, II:25, p. 617, and especially III:42, p. 1166ff.} \\
that was yet a troop-conflict earlier in the \inx[L]{Home}. \\
Broken was the plank-wall of the stronghold of the Eese; \\
the Wanes by a war-\inx[C]{spae} did tread the fields.\footnoteB{The Wanes used magic spells to win the battle.}\evb\evg

\sectionline

\bvg\bva\mssnote{\Regius~1v/19, \Hauksbok~20r/34, \GylfMS}%
Þȧ gingu \alst{r}ęgin ǫll \hld\ ȧ \alst{r}ǫk-stóla, &
\alst{g}inn-hęilǫg \alst{g}oð, \hld\ ok umb þat \alst{g}ę́ttusk: &
Hvęrr hęfði \alst{l}opt alt \hld\ \alst{l}ę́vi blandit &
eða \alst{ę́}tt \alst{jǫ}tuns \hld\ \alst{Ó}ðs męy gefna?\eva

\bvb Then went the Reins all onto the rake-seats: \\
the Yin-holy Gods, and from each other took counsel of this: \\
Who might have blended all the air with deceit, \\
or to the ettin’s lineage given \inx[P]{Wode}’s maiden \ken*{= Frow}?\footnoteB{That is, promised Frow to the wall-builder.  Cf. \Gylfaginning\ 42.  TODO: elaborate.}\evb\evg


\bvg\bva\mssnote{\Regius~1v/20, \Hauksbok~20r/36, \GylfMS}%
\edtext{\alst{Þ}ȯrr ęinn \edtrans{\alst{þ}ar vá}{fought there}{\Afootnote{so \Hauksbok\Trajectinus\Upsaliensis; \emph{þar var} ‘was there’ \Regius; \emph{þat vann} ‘accomplished it’ \RegiusProse; \emph{þat vá} ‘fought it’ \Wormianus}} \hld\ \alst{þ}runginn móði, &
\edtrans{hann \alst{s}jaldan \alst{s}itr \hld\ es \alst{s}líkt of fregn;}{he seldom sits when of such he learns}{\Bfootnote{Namely ettins encroaching on the gods.  Thunder is the defender of the gods (\Thrymskvida\ 18) and is willing to break certain laws of frith for this purpose (\Lokasenna\ 57–64).}} &
\edtext{\alst{ȧ} gingusk \alst{ęi}ðar, \hld\ \alst{o}rð ok sǿri, &
\alst{m}ǫ́l ǫll \alst{m}ęgin-lig, \hld\ es ȧ \alst{m}eðal \edtext{fóru}{\lemma{fóru ‘had gone’}\Afootnote{\emph{vǫ́ru} ‘had been’ \Hauksbok\Trajectinus}}.}{\lemma{ȧ \dots\ fóru.}\Afootnote{om. \Wormianus}}}{\lemma{ALL}\Afootnote{The order of the lines is that of \Regius\Hauksbok; in \GylfMS\ the two helmings (\emph{Þȯrr \dots\ fregn;} and \emph{ȧ \dots\ fóru.}) are reversed.}}\eva

\bvb Thunder alone fought there, pressed by wrath; \\
he seldom sits when of such he learns. \\
Trampled were oaths, speeches and vows, \\
the mighty treaties all which had gone between them.\evb\evg

\sectionline

\bvg\bva\mssnote{\Regius~1v/23, \Hauksbok~20v/1}%
Vęit hǫ̇n \alst{H}ęimdalar \hld\ \alst{h}ljóð of folgit &
und \edtrans{\alst{h}ęið-vǫnum}{shady}{\Bfootnote{Literally ‘light-less’, \emph{hęiðr} referring especially to the light of a clear sky.}} \hld\ \alst{h}ęlgum baðmi; &
\alst{ǫ́} sér hǫ̇n \alst{au}sask \hld\ \edtrans{\alst{au}rgum}{muddy}{\Bfootnote{Which should be the same mud (\emph{aurr}) as in st. 19, there said of Weird’s Well.}} forsi &
af \edtrans{\alst{v}eði \alst{V}al-fǫðrs}{Walfather’s pledge}{\Bfootnote{Weden placed his eye in Mimer’s well, which gives wisdom to any man who drinks from it.  So \Gylfaginning\ 15: \emph{Þar kom Alfǫðr ok beiddisk eins drykkjar af brunninum, en hann fekk eigi, fyrr en hann lagði auga sitt at veði.} ‘There came Allfather and asked for a single drink from the well, but he did not get it before he laid down his eye as a pledge.’}}. \hld\ \edtrans{\alst{V}ituð ér ęnn eða hvat?}{Know ye yet, or what?}{\Bfootnote{“Do you, Weden, know enough now, or what?”, repeated in 28, 33, 34, 38, 40, 47, 60, and 61.  Similar refrains are found in \Baldrsdraumar\ and \Hyndluljod.}}\eva

\bvb She knows Homedal’s sound \ken*{= Horn of Yell?} hidden \\
beneath the shady, hallowed beam \ken*{= Ugdrassle’s Ash?}. \\
A river she sees being fed by a muddy torrent \\
from Walfather’s pledge \ken*{= Mimer’s well}.—Know ye yet, or what?”\evb\evg

\sectionline

\bvg\bva\mssnote{\Regius~1v/25}%
\edtrans{\alst{Ęi}n sat hǫ̇n \alst{ú}ti}{Alone sat she outside}{\Bfootnote{To \emph{sitja úti} ‘sit outside’ has a cultural connotation of meditation in order to connect or communicate with the otherworld; cf. the noun \emph{úti-seta}.  This line is directly repeated in \Sigurdskamma\ 6/1a.}}, \hld\ þȧ’s hinn \alst{a}ldni kom &
\alst{y}ggjungr \alst{ȧ}sa \hld\ ok ï \alst{au}gu lęit: &
‚hvęrs \alst{f}regnið mik? \hld\ hví \alst{f}ręistið mïn?\eva

\bvb Alone sat she outside when the old one came, \\
the Terrifier of the Eese \ken*{= Weden}, and looked into her eyes. \\
\speakernoteb{[The Wallow:]}%
‘Of what askest thou me? Why tempest thou me?\footnoteB{\emph{fręista} has a sense of testing someone, especially intellectually. Cf. \Havamal\ 2, 26, \Vafthrudnismal\ 3, 5.}\evb\evg


\bvg\bva\mssnote{\Regius~1v/26, \GylfMS}%
\alst{A}lt vęit’k, \alst{Ó}ðinn, \hld\ hvar \alst{au}ga falt &
\edtrans{ï hinum \alst{m}ę́ra}{in the renowned}{\Afootnote{so \Wormianus; \emph{þitt} (corr.) \emph{i enom męra} ‘id.’ \Regius; \emph{j þeim enom meira} ‘in the greater’ \Trajectinus; \emph{i þeim envm mæra} ‘in the renowned’ \Upsaliensis; \emph{vr þeim envm mę́ra} ‘out of the renowned’ \RegiusProse}} \hld\ \alst{M}ímis brunni; &
drekkr \alst{m}jǫð \alst{M}ímir \hld\ \alst{m}orgin hvęrjan &
af \edtrans{\alst{v}eði}{pledge}{\Afootnote{\emph{†veiði†} \RegiusProse}} \alst{V}al-fǫðrs.‘ \hld\ \alst{V}ituð ér ęnn eða hvat?\eva

\bvb I know it all, Weden, where thine eye thou hidst: \\
in the renowned \inx[L]{Mimer’s Well} \\
drinks Mimer mead every morning \\
from Walfather’s pledge.’—Know ye yet, or what?\evb\evg


\bvg\bva\mssnote{\Regius~1v/29}%
Valði hęnni \alst{H}ęr-fǫðr \hld\ \alst{h}ringa ok męn, &
\edtrans{fekk \alst{sp}jǫll \alst{sp}ak-lig}{got foresighted tidings}{\Afootnote{emend.; \emph{fe spioll spaclig} \Regius}\Bfootnote{The reading of \Regius\ may be interpreted either as (1): \emph{fé-spjǫll spak-lig} ‘foresighted wealth-spells’ or (2) \emph{fé, spjǫll spak-lig} ‘wealth, foresighted tidings’; both are metrically deficient.  In (1) a second element in a cpd. like \emph{fé-spjǫll} cannot carry alliteration, and (2) has three strongly stressed nominals; in both cases \emph{fé} which stands first would be expected to carry the alliteration.  The word \emph{fé} ‘wealth, cattle’ also makes little sense in context, since Weden is the one giving her expensive jewellery.

The emendation places the verb \emph{fekk} ‘got, received’ for \emph{fé}.  Verbs carry less stress than verbs, and the line is thus metrically equivalent to 28/3b \emph{drekkr mjǫð Mímir}.  The line parallels st. 1, where the wallow likewise says that she will relate \emph{spjǫll} ‘tidings, sayings’ (cf. English \emph{gospel} lit. ‘good news’ which originally translates the Greek \textgreek{εὐαγγέλιον}).  For discussion on this reading see \textcite[51--53]{Haukur2020}, \textcite[16]{Males2023}.}} \hld\ ok \edtrans{\alst{sp}á-ganda}{spae-gands}{\Bfootnote{Spirits sent out in order to gather hidden wisdom and spaes.  See relevant Index entries.}}; &
sá \alst{v}ítt ok umb \alst{v}ítt \hld\ of \alst{v}er-ǫld hvęrja.\eva

\bvb Host-father \name{= Weden} chose for her rings and a necklace, \\
he got foresighted tidings and \inx[C]{spae}-\inx[C]{gands}— \\
she saw widely and more widely, o’er every world.\evb\evg


\bvg\bva\mssnote{\Regius~1v/30}%
Sá hǫ̇n \alst{v}al-kyrjur \hld\ \alst{v}ítt of komnar, &
\alst{g}ǫrvar at ríða \hld\ til \edtrans{\alst{g}oð-þjóðar}{land of the Gots}{\Bfootnote{Ambiguous; ON \emph{goð-þjóð} may mean either (1) ‘land of the Gots’ or (2) ‘land of the Gods’, for the difficult cluster \emph{tþ} in \emph{Got-þjóð} ‘land of the Gots’ was at some point changed to \emph{ðþ}.  Sense (1) is preferred since it is attested in three other places in \Regius, viz. \Helreid\ TODO and \Gudrunarhvot\ TODO and TODO; (2) is entirely unattested.  One may note that ON \emph{Got-þjóð} reflects the attested Gotnish self-name, \emph{Gut-þiuda}, found in the October 29 entry of the Gotnish calender (TODO: reference).

The Walkirries have a particular association with the Gots, who fought the greatest battles of the Migration Period; cf. note to \Volundarkvida\ 1/1b.}}: &
\edtext{\alst{Sk}uld hélt \alst{sk}ildi, \hld\ en \alst{Sk}ǫgul ǫnnur, &
\alst{G}unnr, Hildr, \alst{G}ǫndul \hld\ ok \alst{G}ęir-skǫgul; &
\alst{n}ú eru talðar \hld\ \edtrans{\alst{N}ǫnnur Hęrjans}{Nans of Harn \name{= Weden}}{\Bfootnote{\emph{Nanna} ‘\inx[P]{Nan}’ (the name itself is a nursing word) was the wife of \inx[P]{Balder}, but the word is here certainly being used to refer generically to ‘maidens, women’.  Cf. Þul \emph{Ásynja} (\Skp\ 3), where the walkirries are kenned \emph{Óðins męyjar} ‘Weden’s maidens’.}}, &
\alst{g}ǫrvar at ríða \hld\ \alst{g}rund, val-kyrjur.}{\lemma{Skuld \dots\ val-kyrjur. ‘Shild \dots\ walkirries.’}\Bfootnote{Judging especially by the out-of-place phrase \emph{nú eru talðar} ‘now are tallied’, these four lines seem to be a later insert from a \inx[C]{thule} counting the walkirries.}}\eva

\bvb She saw \inx[G]{Walkirries} come from afar, \\
ready to ride to the land of the \inx[C]{Gots}. \\
Shild held a shield and Shagle another, \\
Guth, Hild, Gandle and Goreshagle— \\
now are tallied the Nans of Harn \name{= Weden}, \\
ready to ride the ground, the walkirries.\evb\evg

\sectionline

{\small Told allusively in \Voluspa\ 31–33 is the myth about Balder’s death.  Balder, the son of Weden and Frie, was slain with an arrow shot by his blind half-brother Hath, whose hand was guided by Lock.  Weden could not slay Hath, who was his son, and so he seduced the woman Rind, apparently through love-magic (Cormac Awmundson’s TODO: \emph{sęið Yggr til rindar} ‘Ug won Rind through sorcery’).  Rind gave birth to Wonnel, who grew very fast; after just one day he was big enough to kill Hath, which he also did, avenging Balder’s death.  The other important sources for this myth are \Baldrsdraumar\ 8–11, \Gylfaginning\ 49, and \textcite{Saxo} 3.4.1–8.

The language of \Baldrsdraumar\ is so similar to the present sts. that they must be of common origin; \Baldrsdraumar\ 11/2–4 is near-identical to \Voluspa\ 32/4–33/2.  The biggest narrative difference is that \Baldrsdraumar\ mentions Rind, who is not found in \Voluspa.

The most elaborate narrative is found in \Gylfaginning\ 49, which may be shortly summarised as follows: Balder has terrible nightmares about his own death, and so his mother Frie makes all sorts of things (fire, water, venom, metals, stones, trees, diseases, beasts, et. c.) swear oaths not to harm him.  After this the Eese make sport of shooting and striking at him, since he cannot be harmed.  Lock is annoyed by this and approaches Frie while disguised as a woman.  He finds out from her that there is one thing that did not swear the oath—the mistletoe, which was thought too young.  Lock takes a mistletoe and a bow and gives it to the blind god Hath, showing him where to shoot.  Hath does so, and kills Balder.  After this \Gylfaginning\ describes Balder’s funeral (treated poetically in Wolf Ugson’s fragmentary \emph{House-drape}, ÚlfrU \emph{Húsdrp} in \Skp\ III) and how the gods attempted to “weep Balder out of hell”, which failed (see Eddic Fragments in the present ed.)  \Gylfaginning\ 50 goes on to describe how the Eese punished Lock (see st. 34 below.)

It is notable that \Gylfaginning\ 49–50 fails to mention Wonnel.  This part of the myth may have been left out for moral reasons, but was certainly known to the author of the Prose Edda; cf. \Gylfaginning\ 30: \emph{Áli eða Váli heitir einn, sonr Óðins ok Rindar. Hann er djarfr í orrostum ok mjǫk happ-skęytr} ‘Onnel or Wonnel one is called, the son of Weden and Rind. He is brave in battles and a very lucky shot’ and \Skaldskaparmal\ 19: \emph{Hvernig skal kenna Vála? Svá, at kalla hann son Óðins ok Rindar, [\dots] hefni-ás Baldrs, dólg Haðar ok bana hans, [\dots]} ‘How shall one ken Wonnel? Namely by calling him the son of Weden and Rind, [\dots] avenging \inx[C]{os} of Balder, the foe of Hath and his bane, [\dots].’

The last source is \textcite{Saxo} 3.4.1–8, who retells the revenge narrative in typical euhemerized form; his versions of Hath and Balder are distinctly human generals and rulers. It may be summarized as follows: Weden takes counsel from a group of seers; one of them, Horsethief the Finn, foretells that Rind, daughter of the Russian king, will bear him another son to avenge Balder.  Weden soon enlists in the king’s army and leads it to great victories, but is continually spurned by the daughter.  He tries various other disguises but is still refused.  At last he disguises himself as an old woman and becomes her physician.  When she turns sick, he binds her, supposedly in order to give her a certain foul potion—he instead rapes her, apparently with her father’s consent.  Their son, Bo, grows up to become a fierce raider.  One day Weden summons him and reminds him of his duty to avenge his brother, Balder.  Bo slays Hath in a duel, but soon perishes from his wounds.}%TODO: add Saxo’s Latin names in parenthesis

\sectionline

\bvg\bva\mssnote{\Regius~2r/2}%
Ek sá \alst{B}aldri, \hld\ \alst{b}lóðgum \edtrans{tífur}{victim’s}{\Bfootnote{This word is rather difficult and possibly corrupt.  It may be connected with \emph{týr} ‘tew, god’, but the dat. sg. of \emph{týr} is \emph{tívi} and the intrusive \emph{r} is unexplained.  A better explanation is given by \CV, who connect it with OE \emph{tiber, tifer} ‘victim, hostage’, but this also has some problems.  \emph{blóðgum} ‘bloody’ is masc. dat. sg., but OE \emph{tiber} is neuter.  If we are dealing with a masc. noun \emph{*tífurr} with the same declension as \emph{jǫfurr}, we would expect dat. sg. \emph{*tífri}, not \emph{tífur} (which would however be the expected acc. sg.).}}, &
\alst{Ó}ðins barni, \hld\ \alst{ø}r-lǫg \edtrans{folgin}{sealed}{\Bfootnote{Or “hidden”.  The verb \emph{fela} ‘hide, conceal’ is used in poetry to describe burial in mounds, as in \Ynglingatal\ 24 (“[...] And afterwards the victory-havers hid (\emph{fǫ́lu}) the ruler on Borrey.”) or the C10th Karlevi stone (“Hidden (\textbf{fulkin} \emph{folginn}) in this mound lies he whom the greatest deeds followed; [...]”)}}; &
stóð of \alst{v}axinn \hld\ \alst{v}ǫllum hę́ri &
\alst{m}jór ok \alst{m}jǫk fagr \hld\ \alst{m}istil-tęinn.\eva

\bvb I saw Balder’s—the bloody victim’s, \\
Weden’s child’s—\inx[C]{orlay} sealed: \\
there stood grown—higher than the plains, \\
slender and most fair—the mistletoe.\evb\evg


\bvg\bva\mssnote{\Regius~2r/4}%
Varð af \alst{m}ęiði, \hld\ þęim’s \alst{m}ę́r sýndisk, &
\alst{h}arm-flaug \alst{h}ę́ttlig, \hld\ \alst{H}ǫðr nam skjóta. &
\alst{B}aldrs \alst{b}róðir vas \hld\ of \alst{b}orinn snimma, &
sá nam, \alst{Ó}ðins sonr, \hld\ \alst{ęi}n-nę́ttr vega.\eva

\bvb Of the tree which slender seemed \\
became a baneful harm-flier—Hath took to shoot. \\
Balder’s brother \ken*{= Wonnel} was born early; \\
he took, Weden’s son, one night old, to fight.\evb\evg


\bvg\bva\mssnote{\Regius~2r/6}%
\edtext{Þó}{\lemma{Þó \dots\ kęmbði ‘washed \dots\ combed’}\Bfootnote{A collocation, see note to \Havamal\ 61 for discussion and other examples. Wonnel, being oathbound and on the mission to avenge his brother, could not engage in such acts of personal vanity.}} ę́va \alst{h}ęndr \hld\ né \alst{h}ǫfuð kęmbði, &
áðr ȧ \alst{b}ál of \alst{b}ar \hld\ \alst{B}aldrs and-skota; &
en \alst{F}rigg of grét \hld\ í \alst{F}ęn-sǫlum &
\edtrans{\alst{v}ǫ́ \alst{V}al-hallar}{the woe of Walhall}{\Bfootnote{The deaths of two sons; Balder and Hath.}}. \hld\ \alst{V}ituð ér ęnn eða hvat?\eva

\bvb He washed ne’er his hands nor combed his head, \\
before onto the pyre he bore Balder’s opponent \ken*{= Hath}, \\
and Frie lamented in the Fenhalls \\
the woe of Walhall.—Know ye yet, or what?\evb\evg

\sectionline

{\small After Balder was avenged the Eese went to catch lock.  They bound him up with his son’s intestines.  A snake was then placed over his face to drip venom onto it.  His wife, Syein, sat over him and caught the venom in a small basin; when she had to empty it he writhed so greatly that the earth shook.  This myth is found in \FraLoka\ (the prose at the end of \Lokasenna) and \Gylfaginning\ 50.}

\sectionline

\bvg\bva\mssnote{\Regius~2r/8, \Hauksbok~20v/13}%
\edtext{\alst{H}apt sá hǫ̇n liggja \hld\ und \alst{H}vera-lundi &
\edtrans{\alst{l}ę́-gjarns}{guile-eager}{\Bfootnote{A formulaic epithet of Lock. See note to TODO for other examples and discussion.}} líki \hld\ \alst{L}oka ȧ-þękkjan;}{\lemma{Hapt \dots ȧ-þękkjan ‘A captive \dots\ to Lock,’}\Afootnote{Replaced with H1 \Hauksbok.}} &
\alst{þ}ar sitr Sigyn \hld\ \alst{þ}ęygi of sínum &
\alst{v}eri \alst{v}ęl-glýjuð. \hld\ \alst{V}ituð ér ęnn eða hvat?\eva

\bvb A captive \ken{= Lock} she saw lying beneath Wharlund: \\
a guile-eager man’s form, alike to Lock,
There sits Syein not at all cheerful, \\
o’er her husband.—Know ye yet, or what?\evb\evg

\sectionline

{\small The following sts. are paraphrased in \Gylfaginning\ 52:

\begin{quote}
	\emph{Þá mę́lti Gangleri: „Hvat verðr þá eptir, er brenndr er himinn ok jǫrð ok heimr allr, ok dauð goðin ǫll ok allir Einherjar ok alt mann-folk, ok hafið ér áðr sagt, at hverr maðr skal lifa í nǫkkvǫrum heimi um allar aldir?“}

	\emph{Þá svarar Þriði: „Margar eru þá vistir góðar ok margar illar; batst er þá at vera á Gimléi á himni, ok all-gótt er til góðs drykkjar þeim, er þat þykkir gaman, í þeim sal, er Brimir heitir; hann stendr ok á himni. Sá er ok góðr salr, er stendr á Niða-fjǫllum, gørr af rauðu gulli; sá heitir Sindri. Í þessum sǫlum skulu byggja góðir menn ok sið-látir.}

	\emph{Á Ná-strǫndum er mikill salr ok illr ok horfa norðr dyrr; hann er ok ofinn allr orma-hryggjum sem vanda-hús, en orma hǫfuð ǫll vitu inn í húsit ok blása eitri, svá at eptir salnum renna eitr-ár, ok vaða þę́r ár eið-rofar ok morð-vargar, svá sem hér segir:“}
\end{quote}

\begin{quote}
	‘Then spoke Gangler: “What will then remain, when heaven and earth and the whole world is burned, and gods are dead and all the Oneharriers and all man-kind—and [still] ye have said earlier, that each man will live in some world for all ages?”

	Then answers Third: “Many good dwellings are there then, and many ill: it is then best to be in Gimlee in the heaven, and it is very good of good drink for those who find joy in that, in the hall which is called Brimmer; it also stands in heaven. Another good hall is the one which stands on the Nithfells, made from red gold; it is called Sinder. In these halls good and well-mannered men will dwell.

	On Neestrand is a great and bad hall, and its doors face north. It is all woven with the spines of serpents like a wicker-house, but the heads of the serpents all look into the house and blow venom, so that through the hall rivers of venom run, and in those rivers wade oath-breakers and murder-wargs, as is said here:”’
\end{quote}

after which are quoted sts. 37 and 38/1–2, followed by the prose: \emph{En í Hver-gelmi er verst} ‘But in Wharyelmer is is worst’ and 38/4.}

\sectionline

\bvg\bva\mssnote{\Regius~2r/10}%
\alst{Ǫ́} fęllr \alst{au}stan \hld\ of \alst{ęi}tr-dala &
\alst{s}ǫxum ok \alst{s}verðum, \hld\ \edtrans{\alst{S}líðr}{Slide}{\Bfootnote{i.e. ‘very sharp’. Cf. \Atlakvida\ 23: \emph{sax slíðr-bęitt} ‘slide-biting sax’.}} hęitir sú.\eva

\bvb A river falls from the east, above the venom-dales; \\
{[a river]} of saxes and swords, Slide is that one called.\footnoteB{TODO. There are other examples of such a river.}\evb\evg


\bvg\bva\mssnote{\Regius~2r/11}%
Stóð fyr \alst{n}orðan \hld\ ȧ \edtrans{\alst{N}iða-vǫllum}{Nithwolds}{\Afootnote{\emph{Niða-fjǫllum} ‘Nithfells’ \Regius\Wormianus\ (paraphrase); \emph{fjǫllom nǫkkurum} ‘some certain fells’ \Trajectinus}} &
\alst{s}alr ór gulli \hld\ \alst{S}indra ę́ttar; &
en \alst{a}nnarr stóð \hld\ ȧ \alst{Ȯ}kólni, &
\alst{b}jór-salr jǫtuns, \hld\ \edtrans{en sá \alst{B}rimir hęitir}{and it is called Brimmer}{\Bfootnote{It is not clear if this is the name of the ettin or the hall itself. The author of \Gylfaginning\ considered it the name of the hall.}}.\eva

\bvb Stood to the north on the Nithwolds, \\
a hall of gold, of Sinder’s lineage \ken{dwarfs}. \\
But another one stood on Uncolner, \\
an ettin’s beer-hall, and it is called Brimmer.\evb\evg


\bvg\bva\mssnote{\Regius~2r/13, \Hauksbok~20v/19, \GylfMS}%
\alst{S}al \edtrans{sá hǫ̇n}{she saw}{\Afootnote{\emph{vęit’k} ‘I know’ \GylfMS; cf. st. 61.}} standa \hld\ \alst{s}ólu fjarri &
\alst{N}á-strǫndu ȧ, \hld\ \alst{n}orðr horfa dyrr; &
falla \alst{ęi}tr-dropar \hld\ \alst{i}nn umb ljóra, &
sá ’s \alst{u}ndinn salr \hld\ \alst{o}rma hryggjum.\eva

\bvb A hall she saw standing, far from the sun, \\
on Neestrand; north face its doors. \\
Venom-drops fall in through the smoke-vent; \\
that hall is wound with the spines of snakes.\evb\evg


\bvg\bva\mssnote{\Regius~2r/15, \Hauksbok~20v/21, \GylfMS}%
\edtrans{Sá hǫ̇n}{she saw}{\Afootnote{so \Regius; \emph{ser hon} ‘she sees’ \Hauksbok; \emph{skulu} ‘shall [be]’ \GylfMS}} \alst{þ}ar vaða \hld\ \alst{þ}unga strauma &
\alst{m}ęnn \alst{m}ęin-svara \hld\ ok \edtrans{\alst{m}orð-varga}{murder-wargs}{\Bfootnote{Murderous outlaws.}} &
ok þann’s \alst{a}nnars glępr \hld\ \alst{ęy}ra-ru̇nu. &
Þar \edtrans{saug}{sucked}{\Afootnote{so \Hauksbok; \emph{†súg†} \Regius; \emph{kvęlr} ‘torments’ \GylfMS}} \alst{N}íð-hǫggr \hld\ \alst{n}ái fram-gingna; &
slęit \alst{v}argr \alst{v}era. \hld\ \alst{V}ituð ér ęnn eða hvat?\eva

\bvb She saw there wading through heavy streams \\
false-swearing men and murder-wargs, \\
and the one who beguiles another’s ear-whisperer \ken{wife}. \\
There sucked \inx[P]{Nithehewer} from corpses passed-on; \\
the warg tore at men.—Know ye yet, or what?\footnoteB{In this st. is clearly described watery punishment in the Heathen afterlife, also seen in \Reginsmal\ 3–4 and possibly in \Grimnismal\ 21. The crimes are what one might expect from the Germanic worldview: perjury, shameful murder, and adultery with a married woman. In Anglo-Saxon and Nordic laws the committer of such crimes gained the title of \inx[C]{nithing}, that is, one afflicted with \inx[C]{nithe} (severe shame). It is not surprising then that such nithings would be tortured by a creature named Nithehewer ‘Nithe-striker’. The practice of burying in bogs and flood-marks (or generally outside of settlements) is well attested in sources about Germanic culture from Tacitī Germania onwards—I consider it likely that the heavy streams in this stanza and others represent such graves. This is further elaborated on in \textcite{GermanicGems2}.}\evb\evg


\bvg\bva\mssnote{\Regius~2r/17, \Hauksbok~20v/2, \GylfMS}%
\edtrans{\alst{Au}str}{In the east}{\Bfootnote{The cardinal direction associated with ettins and other monsters.}} \edtrans{býr}{dwells}{\Afootnote{so \Hauksbok\GylfMS; \emph{sat} ’sat/stayed’ \Regius}} hin \edtrans{\alst{a}ldna}{old}{\Afootnote{\emph{arma} ‘wretched’ \Upsaliensis}} \hld\ í \edtrans{\alst{Éa}rn-viði}{Ironwood}{\Afootnote{metr. emend.; \emph{Járnviði} \Regius\Hauksbok\RegiusProse\Wormianus\Upsaliensis; \emph{Járn-viðjum} ‘Ironwoods’ \Trajectinus}} &
ok \edtrans{\alst{f}ǿðir}{nourishes}{\Afootnote{so \Hauksbok\GylfMS; \emph{fǿddi} ‘nourished’ \Regius}} þar \hld\ \alst{F}ęnris kindir; &
verðr \edtext{af}{\Afootnote{\emph{ór} \Trajectinus\RegiusProse}} þęim \alst{ǫ}llum \hld\ \alst{ęi}nna nøkkurr &
\alst{t}ungls \edtrans{\alst{t}júgari}{seizer}{\Afootnote{\emph{†tuigan†} \Trajectinus; \emph{tregari} ‘griever’ \Upsaliensis. As the young agentive suffix \emph{-ari} is found nowhere else in the poem it is possible that this word is corrupt. If it is, it must have occurred early in the transmission, as reflexes of \emph{tjúgari} are found in all surviving mss.}} \hld\ í \alst{t}rolls hami.\eva

\bvb In the east dwells the old woman, in \inx[L]{Ironwood}, \\
and nourishes there the kindreds of \inx[P]{Fenrer} \ken{wolves}; \\
from them all comes one most certain: \\
a seizer of the Moon in a troll’s \inx[C]{hame}.\footnoteB{The old hag raises the cubs of the wolf Fenrer, of which a particularly fierce one will swallow the moon. According to \Grimnismal\ 40 the sun is chased by a wolf called Skoll, while another wolf, Hate Rothswitner’s son, runs in front of her. This is elaborated upon in \Gylfaginning\ 12, where it is said that Skoll swallows the moon, while Hate swallows the sun. High then explains that “A lone troll-woman (\emph{gýgr}) lives to the east of Middenyard in that forest called Ironwood”, and “feeds the sons of many ettins, all in the likenesses of wolves, and thereof these wolves (i.e. Skoll and Hate) come. And it is also said that from that lineage a single one becomes the mightiest, and he is called \inx[P]{Moongarm}. He fills himself with the life of all those men who die and he swallows the moon and stains heaven and all the air with blood. Thereof the sun loses its rays and the winds are violent and moan hither and thither, and thus it says in the Spae of the Wallow: [...]” after which this and the following st. are quoted. This seems very much like a composite from several sources—probably \Voluspa\ 40–41 and \Grimnismal\ 40—but becomes contradictory when it states that two wolves swallow the moon.
Assuming that this is only a confusion on the part of the author of \Gylfaginning, this st. and the next must be describing Skoll, but it is of course not impossible that there was confusion about the exact details of these events among the Heathen poets. In favour of that seems to speak \Vafthrudnismal\ 46–47, where the sun is said to be swallowed by Fenrer (but see note there).}\evb\evg


\bvg\bva\mssnote{\Regius~2r/19, \Hauksbok~20v/4, \GylfMS}%
\alst{F}yllisk \alst{f}jǫrvi \hld\ \alst{f}ęigra manna, &
\alst{r}ýðr \alst{r}agna sjǫt \hld\ \alst{r}auðum dręyra, &
\alst{s}vǫrt verða \alst{s}ól-skin \hld\ of \alst{s}umur ęptir, &
\alst{v}eðr ǫll \alst{v}á-lynd. \hld\ \alst{V}ituð ér ęnn eða hvat?\eva

\bvb He fills himself with the lifeblood of \inx[C]{fey} men; \\
he reddens the abode of the \inx[G]{Reins} with red gore. \\
Black turn the sun’s rays in summers thereafter; \\
the winds all woeful.—Know ye yet, or what?\evb\evg


\bvg\bva\mssnote{\Regius~2r/21, \Hauksbok~20v/16}%
\edtrans{\alst{S}at þar ȧ haugi}{There sat on the mound}{\Bfootnote{The motif of ettins sitting on burial mounds is also found in \Thrymskvida\ 6 and \Skirnismal\ P2.  The significance of this is uncertain,.}} \hld\ ok \alst{s}ló hǫrpu &
\alst{g}ýgjar hirðir, \hld\ \alst{g}laðr Ęggþér; &
\alst{g}ól of hǫ̇num \hld\ í \edtrans{\alst{G}agl-viði}{Galewood}{\Bfootnote{An otherwise unknown location; the first element is \emph{gagl} ‘wild goose’.  Galewood is perhaps the same as Ironwood.}} &
\alst{f}agr-rauðr hani, \hld\ sá’s \alst{F}jalarr hęitir.\eva

\bvb There sat on the mound and struck the harp \\
the gow’s herdsman, glad \inx[P]{Edgethew}.\footnoteB{Edgethew “herds” the flock of monstrous wolves for the old woman in st. 39.} \\
Over him crowed in \inx[L]{Galewood} \\
a fair-red cock, he who is called Feller.\evb\evg


\bvg\bva\mssnote{\Regius~2r/23, \Hauksbok~20v/18}%
\alst{G}ól of ǫ̇sum \hld\ \alst{G}ullin-kambi, &
sá vękr \alst{h}ǫlða \hld\ at \alst{H}ęrja-fǫðrs, &
en \alst{a}nnarr gęlr \hld\ fyr \alst{jǫ}rð neðan &
\alst{s}ót-rauðr hani \hld\ at \alst{s}ǫlum Hęljar.\eva

\bvb Over the Eese crowed Goldencomb; \\
he wakes men at the Father of Hosts’s \name{= Weden’s} [hall]— \\
but another one crows beneath the earth: \\
a soot-red cock at the halls of Hell.\evb\evg

\sectionline

{\small With the crowing of these three cocks (the first in Ettinham, the second in Walhall, the third in Hell) the destruction of the world begins, and immediately afterwards we get the first occurrence of the refrain stanza (ON \emph{stęf}).}

\sectionline

\bvg\bva\mssnote{\Regius~2r/25}%
\alst{G}ęyr \alst{G}armr mjǫk \hld\ fyr \alst{G}nipa-hęlli, &
\alst{f}ęstr mun slitna, \hld\ en \alst{F}reki rinna; &
\alst{f}jǫlð vęit hǫ̇n \alst{f}rǿða, \hld\ \alst{f}ramm sé’k lęngra &
of \alst{r}agna \alst{r}ǫk, \hld\ \alst{r}ǫmm sig-tíva.\eva

\bvb Garm barks much before the Gnip-halls; \\
the rope will tear and the Wolf run. \\
She knows much wisdom; I foresee further \\
about the mighty \inx[L]{Rakes of the Reins}, of the victory-Tews \ken{gods}.\evb\evg


\bvg\bva\mssnote{\Regius~2r/28, \Hauksbok~20v/24, \GylfMS}%
\alst{B}rǿðr munu \alst{b}ęrjask \hld\ ok at \alst{b}ǫnum verðask, &
munu \edtrans{\alst{s}ystrungar}{the children of sisters}{\Afootnote{\emph{†stystrungar†} \Trajectinus}} \hld\ \edtrans{\alst{s}ifjum spilla}{defile the kinship}{\Bfootnote{i.e. ‘commit incest’, probably referring to marriages between first cousins.  Compare related words found in laws, e.g. \emph{frę́nd-semis spell} ‘incest’ and especially \emph{sifja spell} ‘id.’
The idea of incest as a sign of the end times is also found in \Rigveda\ 10.10.10a–b (norm. and tr., Nikhil S. Dwibhashyam. (2023, oct. 28). \emph{Véda quote 6}. https://nikhilsd.com/dvq/6/): \emph{Ā́ ghā tā́ gachān \hld\ úttarā yugā́ni, // yátra jāmáyaḥ \hld\ kr̥ṇávann ájāmi} ‘There shall come indeed those later ages when relatives shall do (acts) not (fit for) relatives.’}}; &
\alst{h}art ’s \edtrans{í \alst{h}ęimi}{in the Home}{\Afootnote{so \Regius\Hauksbok\Upsaliensis; \emph{með hǫlðum} ‘among men’ \RegiusProse\Trajectinus\Wormianus}}, \hld\ \alst{h}ór-dȯmr mikill, &
\alst{sk}ęggj-ǫld, \alst{sk}alm-ǫld, \hld\ \edtrans{\alst{sk}ildir}{shields}{\Afootnote{\emph{’ru} ‘are’ add. \Regius}} \edtrans{klofnir}{split}{\Afootnote{\emph{klofna} ‘become split’ \Upsaliensis}}, &
\edtrans{\alst{v}ind-ǫld}{wind-age}{\Bfootnote{In \Hauksbok\ the \emph{v} is capitalized, marking the beginning of a new stanza.}}, \alst{v}arg-ǫld, \hld\ \edtrans{áðr}{before}{\Afootnote{\emph{unz} (norm.) ‘until’ \Upsaliensis}} \edtrans{\alst{v}er-ǫld}{man-age}{\Bfootnote{Translated as such since it stands next to various other compounds ending in \emph{ǫld} ‘age’.  ON \emph{ver-ǫld} is cognate with English “world”, but in ON that sense is usually expressed with \emph{hęimr} (e.g. l. 3 of the present stanza).}} \edtrans{stęypisk}{tumbles down}{\Bfootnote{\emph{grundir gjalla \hld\ gífr fljúgandi} (norm.) ‘foundations shrill, fiends flying’ add. after this l. \Hauksbok}} &
\edtext{mun \edtext{\alst{ę}ngi}{\Afootnote{\emph{†enn†} \Upsaliensis}} maðr \hld\ \alst{ǫ}ðrum þyrma.}{\lemma{mun \dots\ þyrma ‘before \dots\ spare’}\Bfootnote{om. \RegiusProse\Trajectinus\Wormianus}}\eva

\bvb Brothers will fight and become each other’s slayers; \\
the children of sisters will defile the kinship. \\
’Tis hard in the Home; whoredom is great: \\
axe-age, sword-age—shields are split— \\
wind-age, warg-age! Before the man-age tumbles down, \\
no man will another spare.\evb\evg

\sectionline

{\small Sts. 45–54 (with the omission of the refrain-stanza 47) are cited in sequence in \Gylfaginning\ 51.}

\sectionline

\bvg\bva\mssnote{\Regius~2r/32, \Hauksbok~20v/27, \GylfMS}%
\edtext{Lęika \alst{M}íms synir, \hld\ en \alst{m}jǫtuðr kyndisk &
at hinu \alst{g}alla \hld\ \alst{G}jallar-horni;}{\lemma{Lęika \dots\ Gjallar-horni; ‘Mime’s \dots Yell.’}\Bfootnote{om. \GylfMS}} &
\alst{h}ǫ́tt blę́ss \alst{H}ęimdallr, \hld\ \alst{h}orn ’s ȧ lopti; &
\edtrans{\alst{m}ę́lir}{speaks}{\Afootnote{\emph{†mey†} \RegiusProse; \emph{†nie†} \Trajectinus}} Óðinn \hld\ við \alst{M}íms hǫfuð; &
\edtext{skęlfr \alst{Y}ggdrasils \hld\ \alst{a}skr standandi, &
\alst{y}mr it \alst{a}ldna tré, \hld\ en \alst{jǫ}tunn losnar.}{\lemma{Skęlfr \dots\ losnar ‘Ugdrassle’s \dots\ loosens’}\Bfootnote{so \Hauksbok\GylfMS; in \Regius\ the two lines are reversed.}}
\eva

\bvb Mime’s sons play and the Metted is kindled \\
at [the sound of] the shrill Horn of Yell. \\
High blows Homedal; the horn is aloft; \\
Weden speaks with the head of Mime. \\
Ugdrassle’s Ash trembles, standing: \\
the old tree creaks and the ettin loosens.\evb\evg


\bvg\bva\mssnote{\Regius~2v/8, \Hauksbok~20v/30, \GylfMS}%
Hvat ’s með \alst{ǫ̇}sum? \hld\ hvat ’s með \edtrans{\alst{ǫ}lfum}{Elves}{\Afootnote{\emph{ǫ́synjum} ‘Ossens’ \Upsaliensis}}? &
\edtext{gnýr \alst{a}llr \alst{Jǫ}tun-hęimr, \hld\ \alst{ę̇}sir ’ru ȧ þingi,}{\lemma{gnýr \dots\ þingi}\Afootnote{om. \Upsaliensis}} &
\alst{st}ynja dvergar \hld\ fyr \edtext{\alst{st}ęin-durum}{\Afootnote{\emph{stęins} \Upsaliensis; \emph{stęin-dyrum} \Hauksbok\Wormianus\Upsaliensis}} &
\edtext{\edtext{\alst{v}ęgg-bergs}{\Afootnote{\emph{veg-bergs} \Hauksbok\Trajectinus\Wormianus}} \alst{v}ísir}{\Afootnote{om. \Upsaliensis}}. \hld\ \alst{V}ituð ér ęnn eða hvat?\eva

\bvb What is with the Eese? What is with the Elves? \\
All Ettinham roars; the Eese are at the Thing. \\
Dwarfs groan before gates of stone, \\
the hillside’s princes.—Know ye yet, or what?\evb\evg


\bvg\bva\mssnote{\Regius~2v/4, \Hauksbok~20v/32}%
\alst{G}ęyr nú \alst{G}armr mjǫk \hld\ fyr \alst{G}nipa-hęlli, &
\alst{f}ęstr mun slitna, \hld\ en \alst{f}reki rinna; &
\alst{f}jǫlð vęit hǫ̇n \alst{f}rǿða, \hld\ \alst{f}ramm sé’k lęngra &
of \alst{r}agna \alst{r}ǫk \hld\ \alst{r}ǫmm sig-tíva.\eva

\bvb Now Garm barks much before the Gnip-halls; \\
the rope will tear and the Wolf run. \\
She knows much wisdom; I foresee further \\
about the mighty Rakes of the Reins, of the victory-Tews \ken{gods}.\evb\evg


\bvg\bva\mssnote{\Regius~2v/4, \Hauksbok~20v/32, \RegiusProse\Trajectinus\Wormianus}%
\alst{H}rymr ękr austan, \hld\ \alst{h}ęfsk lind fyrir, &
snýsk \alst{Jǫ}rmun-gandr \hld\ í \alst{jǫ}tun-móði, &
\alst{o}rmr knýr \alst{u}nnir, \hld\ \edtrans{en \alst{a}ri hlakkar}{and the eagle screams}{\Afootnote{\emph{ǫrn mun hlakka} ‘the eagle will scream’ \RegiusProse\Trajectinus}}, &
slítr \alst{n}ái \alst{n}ef-fǫlr; \hld\ \alst{N}agl-far losnar.\eva

\bvb Rim drives from the east, holding his shield before him; \\
Ermingand writhes about in ettin-wrath. \\
The Wyrm propels the waves and the eagle screams: \\
the pale-beak tears at corpses; Nailfare loosens.\evb\evg


\bvg\bva\mssnote{\Regius~2v/6, \Hauksbok~20v/34, \RegiusProse\Trajectinus\Wormianus}%
\alst{K}jóll fęrr austan \hld\ \alst{k}oma munu Múspells &
of \alst{l}ǫg \alst{l}ýðir, \hld\ en \alst{L}oki stýrir; &
\alst{f}ara \alst{f}ífl-męgir \hld\ með \alst{f}reka allir, &
þęim es \alst{b}róðir \hld\ \alst{B}ýlęists í fǫr.\eva

\bvb A ship fares from the east—come will Muspell’s \\
subjects o’er the sea—and Lock steers it. \\
The devil-lads journey all with the Wolf; \\
with them comes the brother of Bylest \ken*{= Lock} along.\evb\evg


\bvg\bva\mssnote{\Regius~2v/10, \Hauksbok~20v/36, \GylfMS}%
\edtext{\alst{S}urtr}{\Afootnote{\emph{Svartr} \Upsaliensis}} fęrr \alst{s}unnan \hld\ með \alst{s}viga lę́vi, &
skínn af \alst{s}verði \hld\ \edtrans{\alst{s}ól val-tíva}{sun of the slain-Tew}{\Bfootnote{\emph{val-tíva} is here taken as gen. sg. of \emph{val-tívar} ‘slain-Tews’, for which cf. st. 59 below, but the sense of this is obscure.  Perhaps it means that Surt’s sword shines as bright as the heavenly Gods?  The word may also (so \CV) be read as gen. sg. of unattested \emph{*val-tívi} ‘tew of the slain’, referring to Surt, but this is tautological: “Surt comes from the south with fire; from his sword shines the sun of Surt”.}}; &
\alst{g}rjót-bjǫrg \alst{g}nata, \hld\ en \edtrans{\alst{g}ífr rata}{fiends reel}{\Afootnote{\emph{guðar hrata} ‘[but] the gods stagger’ \Upsaliensis}\Bfootnote{The reading of \Upsaliensis\ is wo. doubt corrupt; the anachronistic masc. pl. ending \emph{-ar} is proof enough, for the word \emph{goð} \char`~\ \emph{guð} ‘gods’ was always neuter in heathen times.}}, &
troða \alst{h}alir \edtrans{\alst{h}ęl-veg}{Hellway}{\Bfootnote{The road on which one has to travel after death to reach his final resting place.  Cf. \Helreid.}}, \hld\ en \alst{h}iminn klofnar.\eva

\bvb Surt comes from the south with the twig’s betrayer \ken{fire}; \\
from the sword shines the sun of the slain-Tews. \\
Boulders clash and the fiends reel; \\
men tread the \inx[L]{Hellway} and heaven is split.\evb\evg

\sectionline

{\small The following two sts. describe how Weden fights the Wolf and dies, and how he is avenged by the Wolf.  This fight is also mentioned in \Vafthrudnismal\ 53.}

\sectionline

\bvg\bva\mssnote{\Regius~2v/13, \Hauksbok~20v/37, \RegiusProse\Trajectinus\Wormianus}%
Þȧ kømr \edtrans{\alst{H}línar \hld\ \alst{h}armr annarr}{Line’s second sorrow}{\Bfootnote{The first sorrow being the death of Balder.  Line is described in \Gylfaginning\ 35 as a minor goddess \emph{sett til gę́zlu yfir þeim mǫnnum, er Frigg vill forða við háska nǫkkurum} ‘placed to watch over those men which Frie wishes to protect against any particular danger’. In spite of this almost all translators and editors have understood Line as synonymous with Frie, or even asked whether her existence as a distinct goddess is not something invented by the author of \Gylfaginning.  \textcite{Hopkins2017} argues that this need not be the case; as a maidservant of Frie, Line’s two sorrows would consist in her failure to protect both the son and husband of her mistress.}} framm, &
es \alst{Ó}ðinn fęrr \hld\ við \alst{u}lf vega, &
—en \edtrans{\alst{b}ani \alst{B}ęlja}{the bane of Bellower \ken*{= Free}}{\Bfootnote{Bellower (ON \emph{Bęli}) was slain by Free in an obscure duel; see Index.}} \hld\ \alst{b}jartr at Surti— &
þȧ mun \alst{F}riggjar \hld\ \alst{f}alla \edtext{angan}{\Afootnote{so \Hauksbok\GylfMS; \emph{angantyr} \Regius}}.\eva

\bvb Then comes \inx[P]{Line}’s second sorrow to pass, \\
when Weden goes to fight the Wolf \\
—but the bane of Bellower \ken*{= Free}, bright, [goes] against Surt— \\
then will Frie’s beloved \ken*{= Weden} fall.\evb\evg


\bvg\bva\mssnote{\Regius~2v/15, \RegiusProse\Trajectinus\Wormianus}%
\edtrans{Þȧ kømr hinn \alst{m}ikli \hld\ \alst{m}ǫgr Sig-fǫður}{Then comes the great lad of Syefather}{\Afootnote{\emph{Gęngr Óðins sonr \hld\ við ulf vega} ‘Weden’s son goes the Wolf to fight’ \GylfMS.}}, &
\alst{V}íðarr \edtext{\alst{v}ega}{\Afootnote{\emph{of veg} \GylfMS}} \hld\ at \alst{v}al-dýri; &
lę́tr \alst{m}ęgi \edtrans{Hveðrungs}{Whethring}{\Bfootnote{An obscure name for \inx[P]{Lock}, whose son is the Wolf.}} \hld\ \alst{m}und of standa &
\alst{h}jǫr til \alst{h}jarta; \hld\ þȧ ’s \alst{h}efnt fǫður.\eva

\bvb Then comes the great lad of \inx[P]{Syefather}, \\
Wider, to fight that slaughter-beast. \\
He lets his hand through \inx[P]{Whethring}’s lad \ken*{= the Wolf} \\
drive the sword to the heart—then the father is avenged!\evb\evg


\bvg\bva\mssnote{\Regius~2v/17, \Hauksbok~20v/41, \RegiusProse\Trajectinus\Wormianus}%
\edtext{\edtrans{\edtrans{Þȧ kømr}{Then comes}{\Afootnote{\emph{Gęngr} ‘Goes’ \GylfMS}} hinn \alst{m}ę́ri \hld\ \alst{m}ǫgr \edtrans{Hlǫðynjar}{Lathyn}{\Afootnote{add. \emph{gęngr Óðins sonr \hld\ við orm vega.} ‘Weden’s son goes the Wyrm to fight.’ \Regius.}},}{Then comes the renowned lad of Lathyn}{\Afootnote{om. \Hauksbok.}} &
\edtrans{gęngr \alst{f}et níu \hld\ \alst{F}jǫrgynjar burr}{Firgyn’s son goes nine paces}{\Afootnote{om. \GylfMS.}} &
\alst{n}ęppr frȧ \alst{n}aðri, \hld\ \alst{n}íðs ȯ-kvíðnum; &
\edtrans{munu \alst{h}alir allir \hld\ \alst{h}ęim-stǫð ryðja}{All men will clear their homesteads}{\Bfootnote{After the Thunder is slain the Earth is no longer habitable.  Cf. \Harbardsljod\ TODO, \Thrymskvida\ 18.}} &
es af \alst{m}óði drepr \hld\ \edtrans{\alst{M}ið-garðs véurr}{Middenyard’s Wighward}{\Bfootnote{“The Guardian of the Sanctuaries of Middenyard”; a fitting kenning.}}.}{\lemma{ALL}\Bfootnote{The present version of the stanza is an amalgamation of all three sources (\Regius, \Hauksbok\ and \GylfMS), based most closely on the latter two, which have the last 3 lines in the same order.  \Regius\ has the lines in the following order, using the numbering of the pres. ed.: 1, 5, 4, 2, 3.  It also inserts another line between 1 and 5.}}\eva

\bvb Then comes the renowned lad of Lathyn \name{= Earth} \ken*{= Thunder}; \\
Firgyn’s son goes nine paces \\
pained, away from the loathsome adder \ken*{= Middenyardswyrm}. \\
All men will clear their homesteads \\
when Middenyard’s Wigh-ward strikes out of wrath.\evb\evg


\bvg\bva\mssnote{\Regius~2v/20, \Hauksbok~21r/1, \GylfMS}%
\alst{S}ól tér \alst{s}ortna, \hld\ \edtext{\edtrans{\alst{s}økkr}{sinks}{\Afootnote{so \RegiusProse\Trajectinus\Wormianus; \emph{sígr} ‘descends’ \Regius\Hauksbok\Upsaliensis}} fold í mar}{\lemma{søkkr \dots\ mar ‘sinks \dots\ the sea’}\Bfootnote{The reading \emph{søkkr} ‘sinks’ is supported by Arn \emph{Þorfdr} 24 (\Skp\ II), which is probably based on the present line: \emph{Bjǫrt verðr sól at svartri; \hld\ søkkr fold í mar døkkvan;} ‘The bright sun turns to black; the fold sinks into the dark sea’.}}, &
\alst{h}verfa af \alst{h}imni \hld\ \alst{h}ęiðar stjǫrnur; &
gęisar \alst{ęi}mi \hld\ við \alst{a}ldr-nara; &
lęikr \alst{h}ǫ́r \alst{h}iti \hld\ við \alst{h}imin sjalfan.\eva

\bvb Sun starts to blacken; the fold \ken{earth} sinks into the sea; \\
from heaven fade the shining stars. \\
Smoke rages from the life-nourisher \ken{fire}; \\
the high heat licks the very heaven.\evb\evg


\bvg\bva\mssnote{\Regius~2v/22, \Hauksbok~21r/2}%
\alst{G}ęyr nú \alst{G}armr mjǫk \hld\ fyr \alst{G}nipa-hęlli, &
\alst{f}ęstr mun slitna, \hld\ en \alst{f}reki rinna; &
\alst{f}jǫlð vęit hǫ̇n \alst{f}rǿða, \hld\ \alst{f}ramm sé’k lęngra &
of \alst{r}agna \alst{r}ǫk, \hld\ \alst{r}ǫmm sig-tíva.\eva

\bvb Now Garm barks much before the Gnip-halls; \\
the rope will tear and the Wolf run. \\
She knows much wisdom; I foresee further \\
about the mighty Rakes of the Reins, of the Victory-Tews \ken{gods}.\evb\evg

%TODO: Insert image?

\sectionline

{\small With the last repetition of the refrain stanza the destruction has reached its apex.  Sts. 56–59 are paraphrased in \Gylfaginning\ ch. 53:

\begin{quote}
	\emph{Þá mę́lti Gangleri: „Hvárt lifa nǫkkur goðin þá, eða er þá nǫkkur jǫrð eða himinn?“ Hárr segir: „Upp skýtr jǫrðunni þá ór sę́num, ok er þá grǿn ok fǫgr. Vaxa þá akrar ó·sánir. Víðarr ok Váli lifa, svá at eigi hefir sę́rinn ok Surta-logi grandat þeim, ok byggja þeir á Iða-velli, þar sem fyrr var Ás-garðr, ok þar koma þá synir Þórs, Móði ok Magni, ok hafa þar Mjǫllni. Því nę́st koma þar Baldr ok Hǫðr frá Heljar, setjast þá allir samt, ok talast við, ok minnast á rúnar sínar, ok rǿða of tíðendi þau, er fyrrum hǫfðu verit, of Mið-garðs-orm ok um Fenris-úlf. Þá finna þeir í grasinu gull-tǫflur þę́r, er ę́sirnir hǫfðu átt. Svá er sagt:“}
\end{quote}

\begin{quote}
	‘Then spoke Gangler: “Do any of the gods then live, or is there then any earth or heaven?”  High says: “The earth then shoots up from the seas, and it is then green and fair.  Then grow acres unsown.  Wider and Wonnel live, for the sea and Surt’s flame have not harmed them, and they settle on the Idewolds where there earlier was Osyard; and then the sons of Thunder, Mood and Main, come there, and there they have Millner.  Next come Balder and Hath from Hell; then they all make peace with each other and discuss and think back on their runes, and speak about the tidings which had been in antiquity, about the Middenyardswyrm and about the Fenrerswolf.  Then they find in the grass those golden game-bricks which the Eese had owned. So it is said:”’
\end{quote}

after which is quoted \Vafthrudnismal\ 51.}

\sectionline

\bvg\bva\mssnote{\Regius~2v/23, \Hauksbok~21r/4}%
Sér hǫ̇n \alst{u}pp koma \hld\ \edtrans{\alst{ǫ}ðru sinni}{a second time}{\Bfootnote{The first time probably being the lifting of the Earth in st. 4.}} &
\alst{jǫ}rð ór \alst{ę́}gi \hld\ \alst{i}ðja-grø̇na; &
\alst{f}alla \alst{f}orsar, \hld\ \alst{f}lýgr ǫrn yfir, &
sá’s ȧ \alst{f}jalli \hld\ \alst{f}iska vęiðir.\eva

\bvb She sees coming up a second time \\
Earth from the ocean, ever green anew. \\
Torrents fall, flies the eagle above, \\
which on the fells catches fish.\evb\evg


\bvg\bva\mssnote{\Regius~2v/24, \Hauksbok~21r/5}%
\edtrans{Finnask}{find each other}{\Afootnote{\emph{hittask} \Hauksbok\ provides closer parallelism with st. 7, but for the same reason it may also have replaced earlier \emph{finnask}.}} \alst{ę̇}sir \hld\ ȧ \alst{I}ða-vęlli &
ok umb \alst{m}old-þinur \hld\ \alst{m}ǫ́tkan dø̇ma, &
\edtrans{ok \alst{m}innask þar \hld\ ȧ \alst{m}ęgin-dȯma}{and there think back on mighty verdicts}{\Afootnote{om. \Regius}} &
ok ȧ \alst{F}imbul-týs \hld\ \alst{f}ornar ru̇nar.\eva

\bvb The Eese find each other on the Idewolds, \\
and of the mighty Earth-strip \ken*{= the Middenyardswyrm} judge, \\
and there think back on mighty verdicts, \\
and on Fimble-Tew’s \name{= Weden’s} ancient runes.\evb\evg


\bvg\bva\mssnote{\Regius~2v/26, \Hauksbok~21r/7}%
Þar munu \alst{ę}ptir \hld\ \edtext{\alst{u}ndr-samligar &
\alst{g}ullnar tǫflur}{\lemma{undr-samligar gullnar tǫflur ‘wondersome golden game-bricks’}\Bfootnote{A fine literary device.  In st. 8 the golden age of the Eese, exemplified by their playing board games, was spoiled by the three ettin-women.  The rediscovering of the golden board game then betokens a new golden age.}} \hld\ í \alst{g}rasi finnask, &
þę́r’s í \alst{á}r-daga \hld\ \alst{á}ttar hǫfðu.\eva

\bvb There will afterwards wondersome \\
golden game-bricks in the grass be found, \\
those which in days of yore they had owned.\evb\evg


\bvg\bva\mssnote{\Regius~2v/28, \Hauksbok~21r/9}%
Munu \alst{ȯ}-sánir \hld\ \alst{a}krar vaxa, &
\alst{b}ǫls mun alls \alst{b}atna, \hld\ mun \alst{B}aldr koma; &
búa \alst{H}ǫðr ok Baldr \hld\ \alst{H}ropts sig-toptir, &
\alst{v}ęl \alst{v}al-tívar. \hld\ \alst{V}ituð ér ęnn eða hvat?\eva

\bvb Unsown will acres grow; \\
the bale will all be bettered; Balder will come. \\
Hath and Balder bedwell Roft’s \name{= Weden’s} victory-plots \\
well, the slain-Tews.—Know ye yet, or what?\footnoteB{The evil of Hath’s slaying Balder will be forgotten as the two live together in peace.}\evb\evg


\bvg\bva\mssnote{\Regius~2v/30, \Hauksbok~21r/11}%
Þȧ kná \alst{H}ø̇nir \hld\ \edtrans{\alst{h}laut-við kjósa}{choose the leat-wood}{\Bfootnote{Foresee the future by the means of twigs drenched in the blood of slaughtered beasts.  See \Hymiskvida\ 1 and the encyclopedia entry for “leat”.}} &
ok \alst{b}urir \alst{b}yggva \hld\ \edtrans{\alst{b}rǿðra tvęggja}{the two brothers}{\Bfootnote{The present translation understands \emph{tvęggja} as the gen. pl. of \emph{tvęir} ‘two’; the two brothers are presumably Hath and Balder, mentioned in the previous stanza.
Since the original ms. does not capitalize proper nouns one could also read \emph{brǿðra Tvęggja} ‘the brothers of Tway \name{= Weden}’.  Weden’s brothers are attested in \Gylfaginning\ 6 as \inx[P]{Will} and \inx[P]{Wigh}; they are never said to have children.}} &
\alst{v}ind-hęim \alst{v}íðan. \hld\ \alst{V}ituð ér ęnn eða hvat?\eva

\bvb Then does Heener choose the \inx[C]{leat}-wood, \\
and the sons of the two brothers settle \\
the wide wind-home \ken{sky/heaven}.—Know ye yet, or what?\evb\evg


\bvg\bva\mssnote{\Regius~2v/31, \Hauksbok~21r/12, \GylfMS}%
\alst{S}al \edtrans{sér hǫ̇n}{she sees}{\Afootnote{\emph{vęit’k} ‘I know’ \GylfMS}} standa \hld\ \alst{s}ólu fęgra, &
\edtrans{\alst{g}ulli þakðan}{thatched with gold}{\Afootnote{\emph{gulli bętra} ‘better than gold’ \RegiusProse\Trajectinus}}, \hld\ ȧ \edtext{\alst{G}imléi}{\Afootnote{metr. emend.; \emph{Gimlé} \Regius\Hauksbok\GylfMS}}; &
\edtrans{þar}{there}{\Afootnote{\emph{þann} ‘[in] that [hall]’ \Trajectinus\Wormianus}} skulu \alst{d}yggvar \hld\ \alst{d}róttir byggva &
ok umb \alst{a}ldr-daga \hld\ \alst{y}nðis njóta.\eva

\bvb A hall she sees standing, fairer than the sun, \\
thatched with gold, on Gemlee; \\
there shall faithful folk settle, \\
and in their days of life enjoy delight.\evb\evg


\bvg\bva\mssnote{\Regius~3r/2, \Hauksbok~21r/15}%
Þar kømr hinn \alst{d}immi \hld\ \alst{d}ręki fljúgandi, &
\alst{n}aðr frȧnn \alst{n}eðan \hld\ frȧ \alst{N}iða-fjǫllum; &
berr sér í \alst{f}jǫðrum \hld\ —\alst{f}lýgr vǫll yfir— &
\alst{N}íð-hǫggr \alst{n}ái; \hld\ \edtrans{\alst{n}ú mun hǫ̇n søkkvask}{Now she will sink!}{\Bfootnote{The wallow, referring to herself in third person, descends back down into her grave, whence Weden woke her.  Cf. the very last half-line of \Helreid: \emph{søkkst-u, gýgjar-kyn} ‘sink, thou gow’s kin!’}}.\eva

\bvb Then comes the gloomy dragon flying, \\
the gleaming adder down below from the \inx[L]{Nithfells}. \\
He carries in his feathers—he flies over the field— \\
Nithehewer, corpses.—Now she will sink!”\evb\evg

\sectionline

\section{Stanzas from \Hauksbok}

\Hauksbok\ has a few substantial insertions and differences.  Their style strongly suggests that they are later compositions.

\sectionline

{\small Ll. 1–2 of st. 34 are replaced by the following.}

\bvg\bva[H1]\mssnote{\Hauksbok~20v/12}%
\edtext{Þȧ kná \edtrans{\alst{V}áli}{Wonnel}{\Afootnote{emend.; \emph{Vála} \Hauksbok}} \hld\ \alst{v}íg-bǫnd snúa &
\alst{h}ęldr vǫ́ru \alst{h}arð-gǫr \hld\ \alst{h}ǫpt ór þǫrmum.}{\lemma{Þȧ \dots\ þǫrmum.}\Bfootnote{Only attested in \Hauksbok, where it replaces ll. 1–2 of 34.}}\eva

\bvb Then did \inx[C]{Wonnel} the war-bonds twist: \\
the most sturdy fetters were made from intestines.\evb\evg

\sectionline

{\small Ll. 5–6 of 45 are followed by the following lines, forming another four-line stanza.}

\bvg\bva[H2]\mssnote{\Hauksbok~20v/28}%
\alst{H}rę́ðask allir \hld\ ȧ \alst{h}ęl-vegum &
áðr \alst{S}urtar þann \hld\ \alst{s}efi of glęypir.\eva

\bvb All are frightened on the Hell-ways, \\
before Surt’s kinsman does devour it.\evb\evg

\sectionline

{\small The following stanza appears between 52 and 53.}

\bvg\bva[H3]\mssnote{\Hauksbok~20v/39}%
\edtext{Gïnn \alst{l}opt yfir \hld\ \alst{l}indi jarðar, &
gapa \alst{ý}gs kjaptar \hld\ \alst{o}rms í hę́ðum; &
mun \alst{Ó}ðins son \hld\ \edtrans{\alst{ęi}tri}{venom}{\Afootnote{emend.; \emph{ormi} ‘Wyrm’ \Hauksbok.}\Bfootnote{Cf. \Gylfaginning\ 51: “Thunder bears the bane-word from the Middenyardswyrm and strides nine paces away from it. Then he falls dead to the earth for the venom (\emph{ęitri}) which the Wyrm blows on him.”}} mǿta &
\alst{v}args at \edtext{da\emph{uða}}{\Afootnote{da... \Hauksbok}} \hld\ \alst{V}íðars niðja.}{\lemma{Gïnn \dots\ niðja.}\Bfootnote{The final part of this stanza is almost completely illegible.  I have relied on the reading of \textcite[13,44\psqq]{JonHelgason1971}.}}\eva

\bvb Over the air yawns the Girdle of the Earth \ken*{= Middenyardswyrm}; \\
the jaws of the fierce Wyrm gape in the heights. \\
Weden’s son \ken*{= Thunder} will meet the venom \\
of the Warg, after the deaths of Wider’s kinsmen \ken*{= the Eese}.\evb\evg

\sectionline

{\small The following half-stanza appears between 61 and 62; it is generally held to be evidence of late Christian influence.}

\bvg\bva[H4]\mssnote{\Hauksbok~21r/14}%
Þȧ kømr hinn \alst{r}íki \hld\ at \alst{r}ęgin-dȯmi &
\alst{ǫ}flugr \alst{o}fan \hld\ sá’s \alst{ǫ}llu rę́ðr.\eva

\bvb Then comes the mighty one to the great judgement, \\
strong from above, he who rules everything.\evb\evg

\sectionline
% Weden, all gods
	\bookStart{The Speeches of Webthrithner}[Vafþrúðnismǫ́l]

\begin{flushright}%
Dating \parencite{Sapp2022}: C9th (0.105)–C10th (0.894)

Meter: \Ljodahattr%
\end{flushright}%

A wisdom contest poem, known by the author of \Gylfaginning.

Weden first asks his wife, Frie, for counsel, as he is curious about the ancient wisdom which the ettin Webthrithner might possess (1). Frie expresses worry, as she considers Webthrithner wiser than all other ettins (2), but Weden says that he has travelled far and wide, and wishes to know what Webthrithner’s hall is like (3). Frie wishes Weden good luck against the ettin (4) and he departs, to challenge Webthrithner’s \emph{orðspęki} ‘word-wisdom’ (5). He arrives at hall of Webthrithner (6), who promptly declares that Weden will not come out of the hall unless he be wiser than him (7). Weden introduces himself as Gainred, saying that he has travelled far in need of Webthrithner’s hospitality (8). Webthrithner invites Weden to sit down (9), but he instead utters a gnomic verse not unlike those of the first section of \Havamal\ (10).

Webthrithner then begins to ask questions relating to the mythology, each answered by Weden in turn. The questions concern which horses pull the day (11–12) and night (13–14), the river which divides the gods and ettins (15–16) and the plain where Surt and the gods will fight (17–18).

Webthrithner calls the god learned, invites him to sit on the bench, and declares that the loser of the contest must give his head (19). It is now Weden’s turn to ask and the ettin’s to answer, namely about the origins of earth and heaven (20–21), of sun and moon (22–23), of day, night, and the phases of the moon (24–25), and of winter and summer (26–27); then about the earliest god or ettin, namely \inx[P]{Earyelmer} (28–29), his origins (30–31) and how he reproduced asexually (32–33). He continues by asking what Webthrithner first remembers (34–35), about the origin of the wind (36–37) and of the god \inx[P]{Nearth} (38–39), then about Walhall (40–41) and where Webthrithner learned these esoteric pieces of wisdom (42–43).

After this the structure and tone of the questions change; each one begins with the same first half as that of verse 3, and they concern the end-times. Weden asks about the humans who will survive after the Fimble-winter (44–45), how the sun will rise after Fenrer has destroyed the current one (46–47), about some obscure ettin-maidens (48–49; see there), which Ease will survive after the flame of Surt goes out (50–51) and how Weden will die (52–53). Finally, the god asks what he spoke in the ear of Balder before he burned on the pyre (54). Webthrithner finally realizes the identity of his guest, and says that no man may ever know what he spoke in the ear of his son. He laconically accepts his imminent death, and the futility of his wisdom (55); the poem ends with his admission that Weden is ever the wisest of beings (56).

\sectionline

\bvg
\bva\mssnote{\Regius~7v/9}Ráð mér nú \alst{F}rigg \hld\ alls mik \alst{f}ara tíðir &
\ind at \alst{v}itja \alst{V}afþrúðnis; &
\alst{f}orvitni mikla \hld\ kveð’k mér á \alst{f}ornum stǫfum &
\ind við þann hinn \alst{a}lsvinna \alst{jǫ}tun.\eva

\bvb\ [\inx[P]{Weden} quoth:] “Counsel me now, \inx[P]{Frie}, as I desire to journey to visit \inx[P]{Webthrithner}; great curiosity I have of ancient staves by that all-wise \inx[G]{Ettins}[ettin].\footnoteB{i.e. ‘I am greatly curious of the all-wise ettin’s ancient pieces of wisdom.’ Cf. v. 55.}”\evb
\evg


\bvg
\bva\mssnote{\Regius~7v/12}\alst{H}ęima lętja \hld\ mynda’k \alst{H}ęrjafǫðr &
\ind í \alst{g}ǫrðum \alst{g}oða; &
því’t \alst{ę}ngi \alst{jǫ}tun \hld\ hugða’k \alst{ja}fnramman &
\ind sęm \alst{V}afþrúðni \alst{v}esa.\eva

\bvb {[Frie quoth:]} “At home I would wish to keep the Father of Hosts \ken*{= Weden}, in the yards of the gods—for no ettin have I judged to be even-strong with Webthrithner.”\evb
\evg


\bvg
\bva\mssnote{\Regius~7v/13}\alst{F}jǫlð ek \alst{f}ór, \hld\ \alst{f}jǫlð \alst{f}ręistaða’k, &
\ind fjǫlð ek \alst{r}ęynda \alst{r}ęgin; &
hitt \alst{v}il’k \alst{v}ita, \hld\ hvé \alst{V}afþrúðnis &
\ind \alst{s}alakynni \alst{s}éi.\eva

\bvb {[Weden quoth:]} “Much I journeyed, much I tried, much I tested the \inx[G]{Reins}. This I wish to know: how the condition of the halls of Webthrithner might be.”\evb
\evg


\bvg
\bva\mssnote{\Regius~7v/15}\alst{H}ęill þú farir, \hld\ \alst{h}ęill þú aptr komir, &
\ind hęill á \alst{s}innum \alst{s}éir; &
\alst{ǿ}ði þér dugi \hld\ hvar’s skalt, \alst{A}ldafǫðr, &
\ind \alst{o}rðum mę́la \alst{jǫ}tun.\eva

\bvb {[Frie quoth:]} “Whole journey thou, whole come thou back, whole be thou on thy paths! Thy wisdom avail thee, where thou shalt, O \inx[P]{Eldfather} \name{= Weden}, address with words the ettin.”\evb
\evg


\bvg
\bva\mssnote{\Regius~7v/17}\alst{F}ór þá Óðinn \hld\ at \alst{f}ręista orðspęki &
\ind þess hins \alst{a}lsvinna \alst{jǫ}tuns; &
at \alst{h}ǫllu kom, \hld\ \edtext{es}{\Afootnote{ok \Regius}} átti \edtext{\alst{H}ymis}{\lemma{Hymis}\Afootnote{\emph{metr. emend. after} \textcite{FinnurEdda}; Íms \Regius}} faðir; &
\ind \alst{i}nn gekk \alst{Y}ggr þegar.\eva

\bvb Then journeyed Weden to test the word-wisdom of that all-wise ettin. To a hall he came, which the father of \inx[P]{Hymer} \ken*{= Webthrithner} owned; shortly \inx[P]{Ug} \name{= Weden} walked in.\evb
\evg


\bvg
\bva\mssnote{\Regius~7v/18}\alst{H}ęill þú nú, Vafþrúðnir, \hld\ nú em’k í \alst{h}ǫll kominn &
\ind á þik \alst{s}jalfan \alst{s}éa; &
hitt vil’k \alst{f}yrst vita, \hld\ ef \alst{f}róðr séir &
\ind eða \alst{a}lsviðr, \alst{jǫ}tunn.\eva

\bvb {[Weden quoth:]} “Hail thee now, O Webthrithner; now am I come into the hall, to look upon thy self! This I wish first to know, if learned thou be, or all-wise, O ettin.”\evb
\evg


\bvg
\bva\mssnote{\Regius~7v/20}Hvat ’s þat \alst{m}anna, \hld\ es í \alst{m}ínum sal &
\ind verpumk \alst{o}rði \alst{á}? &
\alst{ú}t þú né kømr \hld\ \alst{ó}rum hǫllum frá. &
\ind nema þú inn \alst{s}notrari \alst{s}éir.\eva

\bvb {[Webthrithner quoth:]} “What sort of man is that, who in \emph{my} hall throws words at me? Out comest thou not from \emph{our} halls, unless thou be the cleverer.”\evb
\evg


\bvg
\bva\mssnote{\Regius~7v/22}\edtext{\alst{G}agnráðr}{\Bfootnote{Gangráðr ‘Journey-adviser’ \GylfMS}} hęiti’k, \hld\ nú em’k af \alst{g}ǫngu kominn, &
\ind \alst{þ}yrstr til \alst{þ}inna sala; &
\alst{l}aðar þurfi \hld\ hęf’k \alst{l}ęngi farit &
\ind ok þinna \alst{a}ndfanga, \alst{jǫ}tunn.\eva

\bvb {[Weden quoth:]} “\inx[P]{Gainred} I am called, now am I come from walking, thirsty, to thy halls. In need of a welcoming have I journeyed for long; and [in need] of thy reception, ettin!”\evb
\evg


\bvg
\bva\mssnote{\Regius~7v/24}Hví þú þá, \alst{G}agnráðr, \hld\ mę́lisk af \alst{g}olfi fyrir? &
\ind far þú í \alst{s}ess í \alst{s}al; &
þá skal \alst{f}ręista, \hld\ hvárr \alst{f}lęira viti, &
\ind \alst{g}ęstr eða hinn \alst{g}amli þulr.\eva

\bvb {[Webthrithner quoth:]} “Why then, Gainred, speakest thou from the floor before me? Take a seat in the hall! Then it shall be tried, which of the two might know more; the guest, or the old \inx[C]{thyle} \ken*{I}.”\evb
\evg


\bvg
\bva\mssnote{\Regius~7v/26}\alst{Ó}auðigr maðr, \hld\ es til \alst{au}ðigs kømr, &
\ind mę́li \alst{þ}arft eða \alst{þ}ęgi; &
\alst{o}frmę́lgi mikil \hld\ hygg’k at \alst{i}lla geti &
\ind hvęim’s við \alst{k}aldrifjaðan \alst{k}ømr.\eva

\bvb {[Weden quoth:]} “An unwealthy man, who to a wealthy one comes, ought to speak the needful or be silent.\footnoteB{Last line identical to \Havamal\ 18. The language of the whole stanza bears close resemblance to that poem.} Great over-speaking, I judge, will bring evil for whomever to a cold-ribbed\footnoteB{i.e. ‘cold-hearted, cunning’.} man comes.”\evb
\evg


\bvg
\bva\mssnote{\Regius~7v/28}Sęg mér, \alst{G}agnráðr, \hld\ alls á \alst{g}olfi vill &
\ind þíns of \alst{f}ręista \alst{f}rama, &
hvé \alst{h}ęstr \alst{h}ęitir, \hld\ sá’s \alst{h}vęrjan dręgr &
\ind \alst{d}ag of \alst{d}róttmǫgu.\eva

\bvb {[Webthrithner quoth:]} “Say to me, Gainred, since on the floor thou wilt try thy fame: What is the horse called, which pulls each day above the lads of the retinue \ken{men}?”\evb
\evg


\bvg
\bva\mssnote{\Regius~7v/30}\alst{Sk}infaxi hęitir, \hld\ es hinn \alst{sk}íra dręgr &
\ind \alst{d}ag of \alst{d}róttmǫgu; &
\alst{h}ęsta baztr \hld\ þykkir með \alst{H}ręiðgotum; &
\ind ęy lýsir \alst{m}ǫn af \alst{m}ari.\eva

\bvb {[Weden quoth:]} “\inx[P]{Shinefax} is called he who pulls the bright day above the lads of the retinue. The best of horses he seems among the \inx[G]{Reth-Gots}; the mane of that stallion ever shines.”\evb
\evg


\bvg
\bva\mssnote{\Regius~7v/32}Sęg þat, \alst{G}agnráðr, \hld\ alls á \alst{g}olfi vill &
\ind þíns of \alst{f}ręista \alst{f}rama, &
hvé \alst{jó}r hęitir, \hld\ sá’s \alst{au}stan dręgr &
\ind \alst{n}ǫ́tt of \alst{n}ýt ręgin.\eva

\bvb {[Webthrithner quoth:]} “Say this, Gainred, since on the floor thou wilt try thy fame: What is the steed called, which from the east pulls night above the useful \inx[G]{Reins}?”\evb
\evg


\bvg
\bva\mssnote{\Regius~7v/33}\alst{H}rímfaxi \alst{h}ęitir, \hld\ es \alst{h}vęrja dręgr &
\ind \alst{n}ǫ́tt of \alst{n}ýt ręgin; &
\alst{m}éldropa fęllir \hld\ \alst{m}orgin hvęrjan; &
\ind þaðan kømr \alst{d}ǫgg of \alst{d}ala.\eva

\bvb {[Weden quoth:]} “\inx[P]{Rimefax}\ he is called, who pulls each night above the useful Reins. Every morning he lets foam fall from his bit\footnoteB{lit. “he fells bit-drops”.}; thence comes dew in the dales.\footnoteB{For another explanation of the origin of dew, see}”\evb
\evg


\bvg
\bva\mssnote{\Regius~8r/1}Sęg þat, \alst{G}agnráðr, \hld\ alls á \alst{g}olfi vill &
\ind þíns of \alst{f}ręista \alst{f}rama, &
hvé \alst{ǫ́} hęitir, \hld\ sú’s dęilir með \alst{jǫ}tna sonum &
\ind \alst{g}rund, ok með \alst{g}oðum.\eva

\bvb {[Webthrithner quoth:]} “Say this, Gainred, since on the floor thou wilt try thy fame: How the river is called, which divides the ground between the sons of ettins and the gods?”\evb
\evg


\bvg
\bva\mssnote{\Regius~8r/2}\alst{Í}fing hęitir \alst{ǫ́}, \hld\ es dęilir með \alst{jǫ}tna sonum &
\ind \alst{g}rund, ok með \alst{g}oðum; &
\alst{o}pin rinna \hld\ hón skal umb \alst{a}ldrdaga; &
\ind verðr-at \alst{í}ss á \alst{ǫ́}.\eva

\bvb {[Weden quoth:]} “\inx[L]{Iving} the river is called, which divides the ground between the sons of ettins and the gods. Throughout [her] life-days she shall flow open; ice forms not on the river.”\evb
\evg


\bvg
\bva\mssnote{\Regius~8r/3}Sęg þat, \alst{G}agnráðr, \hld\ alls á \alst{g}olfi vill &
\ind þíns of \alst{f}ręista \alst{f}rama, &
hvé \alst{v}ǫllr hęitir, \hld\ es finnask \alst{v}igi at &
\ind \alst{S}urtr ok hin \alst{s}vǫ́su goð.\eva

\bvb {[Webthrithner quoth:]} “Say this, Gainred, since on the floor thou wilt try thy fame: How that plain is called, where \inx[P]{Surt} and the excellent gods find each other at war?”\evb
\evg


\bvg {\small Óðinn:}
\bva\mssnote{\Regius~8r/4}\alst{V}ígríðr hęitir \alst{v}ǫllr, \hld\ es finnask \alst{v}ígi at &
\ind \alst{S}urtr ok hin \alst{s}vǫ́su goð; &
\alst{h}undrað rasta \hld\ hann’s á \alst{h}vęrjan veg; &
\ind sá ’s þęim \alst{v}ǫllr \alst{v}itaðr.\eva

\bvb Weden [quoth]: “\inx[L]{Wighride} is the plain called, where Surt and the cheerful gods find each other at war. A hundred \inx[C]{rest}[rests] it stretches in each direction; for them that plain is marked out.”\evb
\evg


\bvg {\small Vafþrúðnir:}
\bva\mssnote{\Regius~8r/6}\alst{F}róðr est nú gęstr, \hld\ \alst{f}ar á bękk jǫtuns, &
\ind ok mę́lumk í \alst{s}essi \alst{s}aman; &
\alst{h}ǫfði vęðja \hld\ vit skulum \alst{h}ǫllu í &
\ind \alst{g}ęstr, of \alst{g}oðspęki.\eva

\bvb Webthrithner [quoth]: “Learned art thou now, O guest, come onto the ettin’s bench, and let us speak on the seat together. Wager a head, shall we two in the hall, O guest, over god-wisdom.”\evb
\evg

\sectionline

\bvg {\small Óðinn:}
\bva\mssnote{\Regius~8r/9, \AM~3r/1}Sęg þat hit \alst{ęi}na, \hld\ ef þitt \edtext{\alst{ǿ}ði}{\lemma{ǿði}\Bfootnote{The first word on fol. 3r. of \AM; from this point we have the poem in both manuscripts.}} dugir &
\ind ok þú \alst{V}afþrúðnir \alst{v}itir, &
hvaðan \alst{jǫ}rð of kom \hld\ eða \alst{u}pphiminn &
\ind \alst{f}yrst, hinn \alst{f}róði jǫtunn.\eva

\bvb Weden [quoth]: “Say the one, if thy wisdom suffices, and thou, Webthrithner, knowest: Whence Earth did come, or \inx[L]{Up-heaven}, first, O learned ettin.”\evb
\evg


\bvg {\small Vafþrúðnir:}
\bva\mssnote{\Regius~8r/10, \AM~3r/2}Ór \alst{Y}mis holdi \hld\ vas \alst{jǫ}rð of skǫpuð, &
\ind en ór \alst{b}ęinum \alst{b}jǫrg, &
\alst{h}iminn ór \alst{h}ausi \hld\ hins \alst{h}rimkalda jǫtuns, &
\ind en ór \alst{s}vęita \alst{s}ę́r.\eva

\bvb Webthrithner [quoth]: “Out of \inx[P]{Yimer}’s hull was the earth created, but out of his bones the crags; heaven out of the skull of the rime-cold ettin, but out of his blood\footnoteB{\emph{svęiti} ‘sweat’ is often used to refer to blood.} the sea.\footnoteB{This st. very closely resembles \Grimnismal\ 40–41 TODO.}”\evb
\evg


\bvg {\small Óðinn:}
\bva\mssnote{\Regius~8r/12, \AM~3r/3}Sęg þat \alst{a}nnat, \hld\ ef þitt \alst{ǿ}ði dugir &
\ind ok þú \alst{V}afþrúðnir \alst{v}itir, &
hvaðan \alst{M}áni of kom, \hld\ svá’t fęrr \alst{m}ęnn yfir, &
\ind eða \alst{S}ól hit sama.\eva

\bvb Weden [quoth]: “Say the other, if thy wisdom suffices, and thou, Webthrithner, knowest: Whence Moon did come, he that travels over men, or Sun likewise?”\evb
\evg


\bvg {\small Vafþrúðnir:}
\bva\mssnote{\Regius~8r/13, \AM~3r/4}\alst{M}undilfari hęitir, \hld\ hann’s \alst{M}ána faðir &
\ind ok svá \alst{S}olar hit \alst{s}ama; &
\alst{h}imin \alst{h}verfa \hld\ þau skulu \alst{h}vęrjan dag &
\ind \alst{ǫ}ldum at \alst{á}rtali.\eva

\bvb Webthrithner [quoth]: “\inx[P]{Mundelfare} is he called; he is the father of the Moon, and likewise of the Sun. Circle in the heaven shall they every day, for people to tally years.”\evb
\evg


\bvg {\small Óðinn:}
\bva\mssnote{\Regius~8r/15, \AM~3r/6}\alst{S}ęg þat þriðja, \hld\ alls þik \alst{s}vinnan kveða &
\ind ok þú \alst{V}afþrúðnir \alst{v}itir, &
hvaðan \alst{d}agr of kom, \hld\ sá’s fęrr \alst{d}rótt yfir, &
\ind eða \alst{n}ǫ́tt með \alst{n}iðum.\eva

\bvb Weden [quoth]: “Say the third, as they call thee wise, and thou, Webthrithner, knowest: Whence the day came, the one that travels over the retinue, or night with the moon-phases?”\evb
\evg


\bvg {\small Vafþrúðnir:}
\bva\mssnote{\Regius~8r/17, \AM~3r/8}\alst{D}ęllingr hęitir, \hld\ hann’s \alst{D}ags faðir, &
\ind en \alst{N}ǫ́tt vas \alst{N}ǫrvi borin; &
\alst{n}ý ok \alst{n}ið \hld\ skópu \alst{n}ýt ręgin &
\ind \alst{ǫ}ldum at \alst{á}rtali.\eva

\bvb Webthrithner [quoth]: “\inx[P]{Delling} is called; he is the father of \inx[P]{Day}, but \inx[P]{Night} was born to \inx[P]{Narrow}. The waxing and waning,\footnoteB{i.e. the phases of the moon.} did the useful Reins create, for people to tally years.”\evb
\evg


\bvg {\small Óðinn kvað:}
\bva\mssnote{\Regius~8r/18, \AM~3r/9}Sęg þat \alst{f}jórða, \hld\ alls þik \alst{f}róðan kveða, &
\ind ok þú \alst{V}afþrúðnir \alst{v}itir, &
hvaðan \alst{v}etr of kom \hld\ eða \alst{v}armt sumar &
\ind \alst{f}yrst með \alst{f}róð ręgin.\eva

\bvb Weden quoth: “Say the fourth, as they call thee learned, and thou, Webthrithner, knowest: Whence winter did come, or the warm summer, first among the learned Reins?”\evb
\evg


\bvg {\small Vafþrúðnir:}
\bva\mssnote{\Regius~8r/20, \AM~3r/10}\edtext{\alst{V}indsvalr hęitir, \hld\ hann’s \alst{V}etrar faðir, &
\ind en \alst{S}vǫ́suðr \alst{S}umars.}{\lemma{Vindsvalr \dots\ Sumars}\Bfootnote{The second half of the st. seems to be missing.}}\eva

\bvb Webthrithner [quoth]: “\inx[P]{Windswoll}\ he is called, he is the father of \inx[P]{Winter}; but \inx[P]{Sosuth}\ of \inx[P]{Summer}.”\evb
\evg


\bvg {\small Óðinn kvað:}
\bva\mssnote{\Regius~8r/21, \AM~3r/11}Sęg þat \alst{f}imta, \hld\ alls þik \alst{f}róðan kveða, &
\ind ok þú \alst{V}afþrúðnir \alst{v}itir, &
hvęrr \alst{á}sa \alst{ę}lztr \hld\ eða \alst{Y}mis niðja &
\ind \alst{y}rði í \alst{á}rdaga.\eva

\bvb Weden quoth: “Say the fifth, as they call thee learned, and thou, Webthrithner, knowest: Who in days of yore became the eldest of the \inx[G]{Ease}, or of the kinsmen of Yimer \ken{ettins}?\footnoteB{Cf. the question on the C9th Malt Stone (DR NOR1988;5): \textbf{huaʀisi : alistiąsa}, perhaps \emph{Hvaʀ es inn ęlisti ása?} ‘Who is the eldest of the Ease?’}”\evb
\evg


\bvg {\small Vafþrúðnir:}
\bva\mssnote{\Regius~8r/22, \AM~3r/12}\alst{Ø}rófi vetra \hld\ áðr vę́ri \alst{jǫ}rð of skǫpuð, &
\ind þá vas \alst{B}ergęlmir \alst{b}orinn, &
\alst{Þ}rúðgęlmir \hld\ vas \alst{þ}ess faðir, &
\ind en \alst{Au}rgęlmir \alst{a}fi.\eva

\bvb Webthrithner [quoth]: “Uncountable winters before the earth would be created, then \inx[P]{Bearyelmer} was born. \inx[P]{Thrithyelmer} was that one’s father, but \inx[P]{Earyelmer} the grandfather.”\evb
\evg


\bvg {\small Óðinn kvað:}
\bva\mssnote{\Regius~8r/23, \AM~3r/14}\alst{S}ęg þat \alst{s}étta, \hld\ alls þik \alst{s}vinnan kveða, &
\ind ok þú \alst{V}afþrúðnir \alst{v}itir, &
hvaðan \alst{Au}rgęlmir kom \hld\ með \alst{jǫ}tna sonum &
\ind \alst{f}yrst, hinn \alst{f}róði jǫtunn.\eva

\bvb Weden quoth: “Say the sixth, as they call thee wise, and thou, Webthrithner, knowest: Whence Earyelmer came among the sons of ettins, first, O learned ettin?”\evb
\evg


\bvg {\small Vafþrúðnir:}
\bva\mssnote{\Regius~8r/25, \AM~3r/15}\edtext{\alst{Ó}r \alst{É}livǫ́gum \hld\ stukku \alst{ęi}trdropar, &
\ind svá \alst{ó}x unz ór varð \alst{jǫ}tunn; &
\edtext{órar ę́ttir \hld\ kómu þar allar saman; &
\ind því’s þat ę́ alt til atalt.}{\lemma{órar \dots\ atalt ‘our ... fierce’}\Bfootnote{so \GylfMS; om. \Regius\AM}}}{\lemma{Ór \dots\ atalt ‘Out of ... fierce’}\Bfootnote{quoted in \GylfMS}}\eva

\bvb Webthrithner [quoth]: “Out of the \inx[L]{Ilewaves} splashed venom-drops; thus grew until an ettin emerged. Our lineages came there all together, therefore they are ever wholly fierce.\footnoteB{Over aeons splashing venom-drops combined into a sentient being, Yimer, the ancestor of all Ettins. The present poem’s account of the creation is not nearly as detailed as that of \Gylfaginning.}”\evb\evg


\bvg {\small Óðinn kvað:}
\bva\mssnote{\Regius~8r/26, \AM~3r/16}\alst{S}ęg þat \alst{s}jaunda, \hld\ alls þik \alst{s}vinnan kveða, &
\ind ok þú \alst{V}afþrúðnir \alst{v}itir, &
hvé sá \alst{b}ǫrn gat \hld\ hinn \edtext{\alst{b}aldni}{\lemma{baldni}\Afootnote{so \AM; \emph{aldni} ‘the aged, old’ \Regius\ breaks alliteration}} jǫtunn, &
\ind es hann hafði-t \alst{g}ýgjar \alst{g}aman.\eva

\bvb Weden quoth: “Say the seventh, as they call thee wise, and thou, Webthrithner, knowest: How did that one, the defiant ettin, beget children, when he did not enjoy the pleasure of a troll-woman?”\evb
\evg


\bvg {\small Vafþrúðnir kvað:}
\bva\mssnote{\Regius~8r/27, \AM~3r/17}Und \alst{h}ęndi vaxa \hld\ kvǫ́ðu \alst{h}rímþursi &
\ind \alst{m}ęy ok \alst{m}ǫg saman; &
\alst{f}ótr við \alst{f}ǿti \hld\ gat hins \alst{f}róða jǫtuns &
\ind \alst{s}exhǫfðaðan \alst{s}on.\eva

\bvb Webthrithner quoth: “Neath the arm\footnoteB{lit. ‘hand’.} on the \inx[G]{Rime-Thurses}[rime-thurse], they said that a maiden and lad grew together. A foot against a foot begot, of the learned ettin, a six-headed son.”\evb
\evg


\bvg {\small Óðinn kvað:}
\bva\mssnote{\Regius~8r/29, \AM~3r/18}Sęg þat ǫ́ttunda, \hld\ alls þik fróðan kveða, &
\ind ok þú \alst{V}afþrúðnir \alst{v}itir, &
hvat \alst{f}yrst of mant \hld\ eða \alst{f}ręmst of vęizt, &
\ind þú est \alst{a}lsviðr \alst{jǫ}tunn.\eva

\bvb Weden quoth: “Say the eigth, as they call thee learned, and thou, Webthrithner, knowest: What thou first rememberest, or foremost knowest? Thou art all-wise, ettin.”\evb
\evg


\bvg {\small Vafþrúðnir kvað:}
\bva\mssnote{\Regius~8r/30, \AM~3r/19}\edtext{\alst{Ø}rófi vetra \hld\ áðr vę́ri \alst{jǫ}rð of skǫpuð, &
\ind þá vas \alst{B}ergęlmir \alst{b}orinn; &
þat \alst{f}yrst of man’k, \hld\ es hinn \alst{f}róði jǫtunn &
\ind á vas \alst{l}úðr of \alst{l}agiðr.}{\lemma{Ørófi \dots\ lagiðr}\Bfootnote{The whole verse is quoted in \Gylfaginning.}}\eva

\bvb Webthrithner quoth: “Uncountable winters before the earth would be created, then Bearyelmer was born. That I first remember, when the learned ettin on the tree-trunk was laid.\footnoteB{The reference here is obscure. According to the prose of \Gylfaginning, after the sons of \inx[P]{Byre} (that is, \inx[P]{Weden}, \inx[P]{Will} and \inx[P]{Wigh}) slew Yimer, so much blood flew from his wounds that all the race of Ettins were drowned, save for  Bearyelmer and his family, who survived by getting up on his \emph{lúðr}. In regular prose, \emph{lúðr} usually means ‘trumpet’, but it can also refer to a hollow tree-trunk. Considering the transitive nature of Bearyelmer being laid (\emph{of lagiðr}) on it, it could rather be interpreted as describing a boat burial, in which case the first thing Webthrithner remembers would be Bearyelmer’s funeral.}”\evb
\evg


\bvg {\small Óðinn kvað:}
\bva\mssnote{\Regius~8r/32, \AM~3r/21}\alst{S}ęg þat níunda, \hld\ alls þik \alst{s}vinnan kveða, &
\ind ok þú \alst{V}afþrúðnir \alst{v}itir, &
hvaðan \alst{v}indr of kømr \hld\ svá’t fęrr \alst{v}ág yfir, &
\ind ę́ męnn hann \alst{s}jalfan of \alst{s}éa.\eva

\bvb Weden quoth: “Say the ninth, as they call thee wise, and thou, Webthrithner, knowest: Whence the wind comes, he that travels over the wave; ever men see his self.\footnoteB{Almost certainly a negation has been lost here, men can of course not see the wind.}”\evb
\evg


\bvg {\small Vafþrúðnir:}
\bva\mssnote{\Regius~8r/34, \AM~3r/22}\alst{H}rę́svęlgr \alst{h}ęitir, \hld\ es sitr á \alst{h}imins ęnda, &
\ind \alst{jǫ}tunn í \alst{a}rnar ham; &
af hans \alst{v}ę́ngjum \hld\ kveða \alst{v}ind koma &
\ind \alst{a}lla męnn \alst{y}fir.\eva

\bvb Webthrithner [quoth]: “\inx[P]{Rawswallower} he is called, who sits at the end of the heavens; an ettin in an eagle’s \inx[C]{hame}. From his wings, they say that the wind comes over all men.”\evb
\evg


\bvg {\small [Óðinn kvað:]}
\bva\mssnote{\Regius~8v/1, \AM~3r/24}Sęg þat \alst{t}íunda, \hld\ alls þú \alst{t}íva rǫk &
\ind ǫll \alst{V}afþrúðnir \alst{v}itir, &
hvaðan Njǫrðr of kom \hld\ með ása sonum; &
\alst{h}ofum ok \alst{h}ǫrgum \hld\ rę́ðr \alst{h}undmǫrgum &
\ind ok varð-at \alst{ǫ́}sum \alst{a}linn.\eva

\bvb {[Weden quoth:]} “Say the tenth, since thou of the \inx[P]{Rakes of the Tews} all, Webthrithner, knowest: Whence \inx[P]{Nearth} did come among sons of the \inx[G]{Ease}? Of \inx[C]{hove}[hoves] and \inx[C]{harrow}[harrows] he rules a hound-many,\footnoteB{This is probably a reference to the large presence of theophoric place-names relating to Nearth in Norway. Cf. also \Grimnismal\ 16 for Nearth’s connection with harrows.} and he was not begotten to the Ease.”\evb
\evg


\bvg {\small [Webthrithner quoth:]}
\bva\mssnote{\Regius~8v/3, \AM~3r/26}Í \alst{V}anahęimi \hld\ skópu hann \alst{v}ís ręgin &
\ind ok sęldu at \alst{g}íslingu \alst{g}oðum, &
í \alst{a}ldar rǫk \hld\ hann mun \alst{a}ptr koma &
\ind hęim með \alst{v}ísum \alst{v}ǫnum.\eva

\bvb {[Webthrithner quoth:]} “In \inx[L]{Waneham} the wise \inx[G]{Reins}\footnoteB{While \emph{ręgin} ‘Reins’ is usually just a synonym of \emph{goð} ‘gods’, it seems here to refer specifically to the Wanes, in contrast with the \inx[G]{Ease}.} shaped him, and sold him as a hostage to the gods. In the rake of the \inx[C]{eld}\footnoteB{i.e. the \inx[P]{Rakes of the Reins}.} he will come back, home among the wise \inx[G]{Wanes}.”\evb
\evg


\bvg {\small [Gainred quoth:]}
\bva\mssnote{\Regius~8v/5, \AM~3r/28}Sęg þat \alst{ę}llipta, \hld\ hvar \alst{ý}tar túnum í &
\ind \alst{h}ǫggvask \alst{h}vęrjan dag; &
\edtext{\alst{v}al þęir kjósa}{\lemma{val þęir kjósa ‘the slain they choose’}\Bfootnote{The same root words are present in \emph{valkyrja} ‘\inx[G]{walkirries}[walkirrie]’, though those are women, not men.}} \hld\ ok ríða \alst{v}ígi frá, &
\ind \alst{s}itja męir of \alst{s}áttir \alst{s}aman.\footnoteB{This and the next stanza are damaged in both \Regius\ and \AM; \Regius\ has only this st., but splits it in two, while \AM\ has l. 1 (abbreviated in t ms.: \emph{S. þ. e. XI}) and then jumps to the answer. The present two stanzas are reconstructed. TODO: use edtext instead}\eva

\bvb “Say the eleventh: Where men in yards hew away at each other every day? The slain they choose and from the battle ride; [then] they sit more at peace together.”\evb
\evg


\bvg {\small [Webthrithner quoth:]}
\bva\mssnote{\AM~3r/28}\alst{A}llir \alst{ęi}nhęrjar \hld\ \alst{Ó}ðins túnum í &
\ind \alst{h}ǫggvask \alst{h}vęrjan dag, &
\alst{v}al þęir kjósa \hld\ ok ríða \alst{v}ígi frá, &
\ind \alst{s}itja męir of \alst{s}áttir \alst{s}aman.\eva

\bvb {[Webthrithner quoth:]} “All the \inx[G]{Ownharriers} in Weden’s yards hew away at each other every day. The slain they choose and from the battle ride; [then] they sit more at peace together.”\evb
\evg


\bvg {\small [Gainred quoth:]}
\bva\mssnote{\Regius~8v/6, \AM~3v/1}Sęg þat \alst{t}olpta, \hld\ hví þú \alst{t}íva rǫk &
\ind ǫll \alst{V}afþrúðnir \alst{v}itir, &
frá \alst{jǫ}tna rúnum \hld\ ok \alst{a}llra goða &
\ind þú hit \alst{s}annasta \alst{s}ęgir, &
\ind hinn \alst{a}lsvinni \alst{jǫ}tunn.\eva

\bvb “Say the twelfth: Why thou, the rakes of the Tews all, Webthrithner, knowest? From the \inx[C]{rune}[runes] of the ettins and of all the gods speakest thou the truest, all-wise ettin.”\evb
\evg


\bvg {\small [Webthrithner quoth:]}
\bva\mssnote{\Regius~8v/8, \AM~3v/2}Frá \alst{jǫ}tna rúnum \hld\ ok \alst{a}llra goða &
\ind ek kann \alst{s}ęgja \alst{s}att, &
\ind því’t \alst{h}vęrn hęf’k \alst{h}ęim of komit, &
\alst{n}íu kom’k hęima \hld\ fyr \alst{n}iflhęl neðan; &
\ind hinig dęyja ór \alst{h}ęlju \alst{h}alir.\eva

\bvb {[Webthrithner quoth:]} “From the runes of the ettins and of all the gods I can speak truly, for I have come into each \inx[C]{Home}. Into nine Homes I came beneath \inx[L]{Nivelhell}; that way die men out of \inx[L]{Hell}.\footnoteB{Presumably lower underworlds, more severe than the ‘normal’ one. \textcite{FinnurEdda}\ considers \emph{ór hęlju} ‘out of Hell’ a later interpolation, presumably for metric reasons, but there is no textual support for it.}”\evb
\evg

\sectionline

\bvg {\small [Gainred quoth:]}
\bva\mssnote{\Regius~8v/11, \AM~3v/4}\alst{F}jǫlð ek \alst{f}ór, \hld\ \alst{f}jǫlð \alst{f}ręistaða’k, &
\ind fjǫlð ek \alst{r}ęynda \alst{r}ęgin; &
hvat lifir \alst{m}anna, \hld\ þá’s hinn \alst{m}ę́ra líðr &
\ind \alst{f}imbulvetr með \alst{f}irum?\eva

\bvb “Much I journeyed, much I tried, much I tested the Reins.\footnoteB{Cf. v. 3.} What remains of men, when the renowned \inx[L]{Fimble-winter} passes among people?”\evb
\evg


\bvg {\small [Webthrithner quoth:]}
\bva\mssnote{\Regius~8v/13, \AM~3v/6}\alst{L}íf ok \alst{L}ífþrasir, \hld\ en þau \alst{l}ęynask munu &
\ind í \alst{h}olti \alst{H}oddmímis; &
\alst{m}orgindǫggvar \hld\ þau sér at \alst{m}at hafa; &
\ind þaðan af \alst{a}ldir \alst{a}lask.\eva

\bvb {[Webthrithner quoth:]} “\inx[P]{Life} and \inx[P]{Lifethrasher}, but they will hide themselves in \inx[P]{Hoardmimer}’s wood.\footnoteB{Perhaps in the hollowed-out Uggdrassle.} Morning-dew [will] they have as their food; thence generations [will] be bred.”\evb
\evg


\bvg {\small [Gainred quoth:]}
\bva\mssnote{\Regius~8v/15, \AM~3v/8}\alst{F}jǫlð ek \alst{f}ór, \hld\ \alst{f}jǫlð \alst{f}ręistaða’k, &
\ind fjǫlð ek \alst{r}ęynda \alst{r}ęgin; &
hvaðan kømr \alst{s}ól \hld\ á hinn \alst{s}létta himin, &
\ind \edtext{es þessa hęfr \alst{F}ęnrir \alst{f}arit?}{\lemma{þessa hęfr Fęnrir farit ‘when Fenrer has this one slain.’}\Bfootnote{Cf. \Voluspa\ TODO. Here it is Fenrer himself who will swallow the sun unless it there be taken as a poetic synonym for ‘wolf’ (which undoubtedly is its original meaning). TODO}}\eva

\bvb “Much I journeyed, much I tried, much I tested the Reins. Whence comes Sun onto the smooth heaven, when \inx[P]{Fenrer} has this one\footnoteB{i.e. the current incarnation of the sun, as explained in the next v.} slain?”\evb
\evg


\bvg {\small [Webthrithner quoth:]}
\bva\mssnote{\Regius~8v/16, \AM~3v/9}\alst{Ęi}na dóttur \hld\ berr \alst{a}lfrǫðull, &
\ind áðr hana \alst{F}ęnrir \alst{f}ari; &
sú skal \alst{r}íða, \hld\ þá’s ręgin \alst{d}ęyja, &
\ind \alst{m}óður brautir \alst{m}ę́r.\eva

\bvb {[Webthrithner quoth:]} “One daughter the elf-wheel \ken*{= Sun} bears before Fenrer might slay her. She shall ride—when the Reins die—the maiden, her mother’s paths.”\evb
\evg


\bvg {\small [Gainred quoth:]}
\bva\mssnote{\Regius~8v/18, \AM~3v/10}\alst{F}jǫlð ek \alst{f}ór, \hld\ \alst{f}jǫlð \alst{f}ręistaða’k, &
\ind fjǫlð ek \alst{r}ęynda \alst{r}ęgin; &
hvęrjar ’ru \alst{m}ęyjar, \hld\ es líða \alst{m}ar yfir, &
\ind \alst{f}róðgęðjaðar \alst{f}ara.\eva

\bvb “Much I journeyed, much I tried, much I tested the Reins. Which are the maidens that pass over the ocean; learned-minded they go?”\evb
\evg


\bvg {\small [Webthrithner quoth:]}
\bva\mssnote{\Regius~8v/19, \AM~3v/11}\alst{Þ}ríar \alst{þ}jóðáar \hld\ falla \alst{þ}orp yfir &
\ind \alst{m}ęyja \alst{M}ǫgþrasis; &
\alst{h}amingjur ęinar \hld\ þę́r’s í \alst{h}ęimi eru, &
\ind þó þę́r með \alst{jǫ}tnum \alst{a}lask.\eva

\bvb {[Webthrithner quoth:]} “Three great rivers fall over the settlement of the maidens of Maythrasher; the only Hamings are they in the Home,\footnoteB{In Ettinham, or in the entire world?} though they are among the ettins begotten.”\evb
\evg


\bvg {\small [Gainred quoth:]}
\bva\mssnote{\Regius~8v/21, \AM~3v/13}\alst{F}jǫlð ek \alst{f}ór, \hld\ \alst{f}jǫlð \alst{f}ręistaða’k, &
\ind fjǫlð ek \alst{r}ęynda \alst{r}ęgin; &
hvęrir ráða \alst{ę́}sir \hld\ \alst{ęi}gnum goða, &
\ind þá’s \alst{s}loknar \alst{S}urtalogi?\eva

\bvb “Much I journeyed, much I tried, much I tested the Reins. Which Ease rule the estates of the gods, when the flame of \inx[P]{Surt} goes out?”\evb
\evg


\bvg {\small [Webthrithner quoth:]}
\bva\mssnote{\Regius~8v/22, \AM~3v/14}\alst{V}íðarr ok \alst{V}áli \hld\ byggva \alst{v}é goða, &
\ind þá’s \alst{s}loknar \alst{S}urtalogi; &
\alst{M}óði ok \alst{M}agni \hld\ skulu \alst{M}jǫllni hafa &
\ind \alst{V}ingnis at \alst{v}ígþroti.\eva

\bvb {[Webthrithner quoth:]} “\inx[P]{Wider} and \inx[P]{Wonnel} inhabit the \inx[C]{wigh}[wighs] of the gods, when the flame of Surt goes out. \inx[P]{Mood} and \inx[P]{Main} shall own \inx[P]{Millner}, when \inx[P]{Wingner} is too tired to fight.\footnoteB{lit. ‘at Wingner’s fight-exhaustion,’ referring to his death.}”\evb
\evg


\bvg {\small [Gainred quoth:]}
\bva\mssnote{\Regius~8v/24, \AM~3v/16}\alst{F}jǫlð ek \alst{f}ór, \hld\ \alst{f}jǫlð \alst{f}ręistaða’k, &
\ind fjǫlð ek \alst{r}ęynda \alst{r}ęgin; &
hvat verðr \alst{Ó}ðni \hld\ at \alst{a}ldrlagi, &
\ind þá’s \alst{r}júfask \alst{r}ęgin?\eva

\bvb “Much I journeyed, much I tried, much I tested the Reins. What brings Weden’s life to an end, when the Reins are rent?\footnoteB{Formulaic; see note to \Baldrsdraumar.}”\evb
\evg


\bvg {\small [Webthrithner quoth:]}
\bva\mssnote{\Regius~8v/25, \AM~3v/17}\alst{U}lfr glęypa \hld\ mun \alst{A}ldafǫðr, &
\ind þess mun \alst{V}íðarr \alst{v}reka; &
\alst{k}alda \alst{k}japta \hld\ hann \alst{k}lyfja mun &
\ind \alst{v}itnis \alst{v}ígi at.\eva

\bvb {[Webthrithner quoth:]} “The wolf will devour \inx[P]{Eldfather} \name{= Weden}; that will Wider avenge. The cold jaws he will cleave, of the Wolf at the battle.”\evb
\evg


\bvg {\small [Gainred quoth:]}
\bva\mssnote{\Regius~8v/27, \AM~3v/19}\alst{F}jǫlð ek \alst{f}ór, \hld\ \alst{f}jǫlð \alst{f}ręistaða’k, &
\ind fjǫlð ek \alst{r}ęynda \alst{r}ęgin; &
hvat mę́lti Óðinn, \hld\ áðr á bál stigi, &
\ind \alst{s}jalfr í ęyra \alst{s}yni?\eva

\bvb “Much I journeyed, much I tried, much I tested the Reins. What spoke Weden, before [he = Balder] would mount the pyre,\footnoteB{I agree with \textcite{FinnurEdda} that the subject is \emph{sonr} ‘son’ from the next line. The phrase \emph{stíga á} ‘step onto, mount’ is also used to refer to one stepping aboard a ship or mounting a horse (see \CV: \emph{stíga} for citations), and so its use for a person being borne onto the pyre seems formulaic. This has been compared with \Beowulf\ 1118b: \emph{gúðrinc ástáh} ‘the warrior mounted [his pyre]’, but the interpretation of that line is not controversial; \textcite{KlaeberBeowulf}[186] follow Grundtvig in emending \emph{gúðrinc} to \emph{gúðréc} ‘war-smoke’, relating it to \Beowulf\ 3144b (\emph{wuduréc ástáh} ‘wood-smoke rose up’, also in a description of a cremation). They state that \Grimnismal\ 54 ‘almost certainly refers not to Baldr but to Óðinn, probably imagined to mount the pyre in order to set fire to it.’} himself in the ear of the son \ken*{= Balder}?”\evb
\evg


\bvg {\small [Webthrithner quoth:]}
\bva\mssnote{\Regius~8v/28, \AM~3v/19}„\alst{Ęy} \edtext{manngi}{\Afootnote{\emph{manni} dat. sg. \Regius\AM\ is impossible; a subject is needed.}} vęit, \hld\ hvat þú í \alst{á}rdaga &
\ind \alst{s}agðir í ęyra \alst{s}yni; &
\alst{f}ęigum munni \hld\ mę́lta’k mína \alst{f}orna stafi &
\ind ok of \alst{r}agna \alst{r}ǫk.\eva

\bvb {[Webthrithner quoth:]} “Ever no man knows, what thou in days of yore saidst in the ear of the son. With \inx[C]{fey}\footnoteB{Webthrithner realizes that he was bound to die (\emph{fęigr} ‘fey’, a word with strong fatalistic connotations) from the moment he proposed the wager (v. 19), as no being can outwit Weden.} mouth I spoke my ancient \inx[C]{stave}[staves], and of the Rakes of the Reins.\evb
\evg


\bvg
\bva\mssnote{\Regius~8v/30, \AM~3v/21}Nú við \alst{Ó}ðin \hld\ dęilda’k mína \alst{o}rðspęki; &
\ind þú est ę́ \alst{v}ísastr \alst{v}era.“\eva

\bvb Now with Weden I shared my word-wisdom;\footnoteB{The same word-wisdom Weden in v. 5 set out to try.} thou art ever wisest of beings!\footnoteB{\emph{verr} literally means ‘husband, man,’ but here surely in the broader sense of ‘(male) being’. For other instances of gods being called men, see TODO.}”\evb
\evg
% Weden
	\bookStart{Dreams of Balder}[Baldrs draumar]
\def\thisBookCode{Baldrsdraumar}

% Introduction.
\begin{flushright}%
\textbf{Dating} \parencite{Sapp2022}: C10th (0.890)

\textbf{Meter:} \Fornyrdislag%
\end{flushright}

\section{Introduction}

The \textbf{Dreams of Balder} (\Baldrsdraumar) is not preserved in \Regius, but rather in the early C14th ms. \AM.  A younger redaction, characterized by a number of post-mediæval additions, is transmitted in several copies in later paper mss.

The poem begins \emph{in medias res}; \inx[P]{Balder} has been having nightmares, which the Gods meet at the Thing to discuss (1).  \inx[P]{Weden} rides to \inx[L]{Hell}, where he has an encounter with a bloody hound; he passes it and continues to “the high house of \inx[P]{Hell}” (2–3), from which he rides west, to the grave of a certain \inx[C]{wallow} whom he revives using magic (4). She asks which man has forced her out of the grave (5), and Weden introduces himself as Waytame, before asking for whom the benches of Hell are covered with gold (6). The wallow responds that barrels of mead stand brewed for Balder and that the gods are very anxious (7). Weden asks her who will slay Balder (8), and she responds that it is Hath, carrying a “high fame-beam” (9).  Weden asks who will avenge Balder’s death (10), the wallow responds that \inx[P]{Rind} will give birth to Weden’s son \inx[P]{Wonnel}, who will slay Hath when only one night old (11).  Weden then asks about some mysterious maidens (12), which apparently betrays his identity.  The wallow announces that she now knows that it is Weden, who in turn retorts that she is not a wallow, but rather the “mother of three thurses” (13). The wallow tells him to ride home and “be famous” and taunts him over his unavoidable death at the \inx[L]{Rakes of the Reins} (14).

\section{The Dreams of Balder}

\bvg\bva\mssnote{\AM~1v/18}%
\edtext{Sęnn vǫ́ru \alst{ę́}sir \hld\ \alst{a}llir á þingi &
ok \alst{ǫ́}synjur \hld\ \alst{a}llar á máli, &
ok umb þat \alst{r}éðu \hld\ \alst{r}íkir tívar:}{\lemma{Sęnn \dots\ tívar ‘Soon \dots\ Tews’}\Bfootnote{Formulaic, identically shared with \Thrymskvida\ 14/1–3.  See also \inx[L]{Thing of the Gods}.}} &
hví vę́ri \alst{B}aldri \hld\ \alst{b}allir draumar?\eva

\bvb%
{\huge S}oon were the \inx[G]{Eese} all at the \inx[C]{Thing}, \\
and the \inx[G]{Ossens} all at speech, \\
and of this counseled the mighty \inx[G]{Tews}: \\
Why did Balder have troubling dreams?\evb\evg


\bvg\bva\mssnote{\AM~1v/19}%
\alst{U}pp ręis \alst{Ó}ðinn, \hld\ \edtext{\alst{a}ldinn}{\Afootnote{emend.; \emph{alda} \AM}} gautr, &
ok hann á \alst{S}lęipni \hld\ \alst{s}ǫðul of lagði, &
ręið \alst{n}iðr þaðan \hld\ \alst{n}ifl-hęljar til; &
mǿtti \edtrans{\alst{h}velpi, \hld\ þęim’s ór \alst{h}ęlju kom}{the whelp that came out of Hell}{\Bfootnote{An otherwise unknown dog, sometimes identified with \inx[P]{Garm}.  The “hellhound” guarding the underworld is well known from world mythology, most famously the Greek \emph{Kérberos}.}}.\eva

\bvb Up rose Weden, the ancient Geat, \\
and he on \inx[P]{Slapner} the saddle did lay; \\
rode down thence to \inx[L]{Nivelhell}; \\
met the whelp that came out of Hell.\evb\evg


\bvg\bva\mssnote{\AM~1v/21}%
Sá vas \alst{b}lóðugr \hld\ of \alst{b}rjóst framan, &
ok \alst{g}aldrs fǫður \hld\ \edtext{\alst{g}ól of}{\Afootnote{\emph{golv} \AM}} lęngi, &
\alst{f}ramm ręið Óðinn, \hld\ \edtrans{\alst{f}old-vegr dunði}{the fold-way \ken{earth} resounded}{\Bfootnote{Cf. the description of \inx[P]{Thunder}’s riding in \Haustlong\ 14: \emph{dunði \dots\ mána vegr und hǫ́num} ‘the moon’s way \ken{sky/heaven} \dots\ resounded beneath him’); see further \Thrymskvida\ 21.}}, &
\alst{h}ann kom at \alst{h}ǫ́u \hld\ \alst{H}ęljar ranni.\eva

\bvb It was bloody on the front of its chest, \\
and at the father of \inx[C]{galder} \ken*{= Weden} for a long time bayed.— \\
Forth rode Weden—the fold-way \ken{earth} resounded— \\
he came to the high house of Hell.\evb\evg


\bvg\bva\mssnote{\AM~1v/22}%
Þá ręið \alst{Ó}ðinn \hld\ fyr \alst{au}stan dyrr, &
þar’s hann \alst{v}issi \hld\ \alst{v}ǫlu lęiði; &
nam hann \alst{v}ittugri \hld\ \edtrans{\alst{v}al-galdr}{slain-galder}{\Bfootnote{i.e. a galder to quicken the dead, in this case the wallow.  Cf. \Havamal\ 158 where Weden tells how He can bring hanged men back to life with runes.}} kveða, &
und’s \alst{n}auðug ręis, \hld\ \alst{n}ás orð of kvað:\eva

\bvb Then rode Weden east from the door, \\
there as he knew the wallow’s grave. \\
He began for the cunning woman to sing a slain-\inx[C]{galder}, \\
until forced she rose, a corpse’s words quoth:\evb\evg


\bvg\bva\mssnote{\AM~1v/24}%
„Hvat ’s \alst{m}anna þat \hld\ \alst{m}ér ó·kunnra, &
es mér hęfr \alst{au}kit \hld\ \edtrans{\alst{ę}rfitt sinni}{this toilsome journey}{\Bfootnote{i.e. the journey out of the grave.}}? &
\edtext{Vas’k \alst{s}nifin \alst{s}njóvi, \hld\ ok \alst{s}lęgin regni, &
ok \alst{d}rifin \alst{d}ǫggu, \hld\ \alst{d}auð vas’k lęngi.}{\lemma{Vas’k snifin \dots\ lęngi. ‘I was snowed \dots\ long.’}\Bfootnote{Cf. the similar description of a buried person in \HelgakvidaTwo\ 47–48 (TODO).}}“\eva

\bvb “What sort of man is this, to me unknown, \\
who has caused for me this toilsome journey? \\
I was snowed by snow and struck by rain, \\
and bespattered with dew—dead was I for long.”\evb\evg


\bvg\bva\mssnote{\AM~1v/25}\speakernote{[Óðinn kvað:]}%
„\alst{V}eg-tamr ek hęiti, \hld\ sonr em’k \alst{V}al-tams, &
sęg þú mér ór \alst{h}ęlju, \hld\ ek man ór \alst{h}ęimi; &
hvęim eru \alst{b}ękkir \hld\ baugum sánir, &
\alst{f}lęt \alst{f}agrliga \hld\ \alst{f}lóuð gulli?“\eva

\bvb\speakernoteb{[Weden quoth:]}%
“Waytame am I called, I am Waltame’s son; \\
tell me [the tidings] from Hell—I will [tell those] from the world. \\
For whom are the benches sown with \inx[C]{bigh}[bighs], \\
the floors fairly flooded with gold?”\evb\evg


\bvg\bva\mssnote{\AM~1v/27}\speakernote{[Vǫlva kvað:]}%
„Hér stęndr \alst{B}aldri \hld\ of \alst{b}rugginn mjǫðr, &
\alst{sk}írar vęigar, \hld\ \edtrans{liggr \alst{sk}jǫldr yfir}{a shield lies over [them]}{\Bfootnote{Shields covering casks of mead is a common trope. Cf. TODO.}}, &
en \alst{ǫ́}s-męgir \hld\ í \alst{o}f-vę́ni; &
\alst{n}auðug sagða’k, \hld\ \alst{n}ú mun’k þęgja.“\eva

\bvb\speakernoteb{[The wallow quoth:]}%
“Here for Balder mead stands brewed, \\
pure draughts—a shield lies over them; \\
but the os-lads \ken*{= Eese} [stand] in great suspense— \\
forced I spoke, now I will shut up!”\evb\evg


\bvg\bva\mssnote{\AM~1v/29}\speakernote{[Óðinn kvað:]}%
„\alst{Þ}ęgj-at-tu vǫlva, \hld\ \alst{þ}ik vil’k fregna, &
\alst{u}nds \alst{a}l-kunna, \hld\ vil’k \alst{ę}nn vita: &
hvęrr man \alst{B}aldri \hld\ at \alst{b}ana verða, &
ok \alst{Ó}ðins son \hld\ \alst{a}ldri rę́na?“\eva

\bvb\speakernoteb{[Weden quoth:]}%
“Shut not up, wallow—thee I wish to ask! \\
Until all is known I wish yet to know: \\
Who will become Balder’s bane \\
and rob Weden’s son of life?”\evb\evg


\bvg\bva\mssnote{\AM~2r/1}\speakernote{[Vǫlva kvað:]}%
„\alst{H}ǫðr berr \alst{h}ǫ́van \hld\ \edtext{\alst{h}róðr-baðm}{\Afootnote{emend.; \emph{hróðr-barm} \AM}} þinig, &
hann man \alst{B}aldri \hld\ at \alst{b}ana verða, &
ok \alst{Ó}ðins son \hld\ \alst{a}ldri rę́na; &
\alst{n}auðug sagða’k, \hld\ \alst{n}ú mun’k þęgja.“\eva

\bvb\speakernoteb{[The wallow quoth:]}%
“\inx[P]{Hath} bears the high glory-beam \ken{mistletoe} thither; \\
he will become Balder’s bane \\
and rob Weden’s son of life— \\
forced I spoke, now I will shut up!”\evb\evg


\bvg\bva\mssnote{\AM~2r/3}\speakernote{[Óðinn kvað:]}%
„\alst{Þ}ęgj-at-tu vǫlva, \hld\ \alst{þ}ik vil’k fregna, &
\alst{u}nds \alst{a}l-kunna, \hld\ vil’k \alst{ę}nn vita, &
hvęrr man \alst{h}ęipt \alst{H}ęði \hld\ \alst{h}ęfnt of vinna, &
eða \alst{B}aldrs \alst{b}ana \hld\ ȧ \alst{b}ál vega?“\eva

\bvb\speakernoteb{[Weden quoth:]}%
“Shut not up, wallow—thee I wish to ask! \\
Until all is known I wish yet to know: \\
Who will avenge that evil on Hath, \\
or cast on the pyre Balder’s bane?”\evb\evg


\bvg\bva\mssnote{\AM~2r/4}\speakernote{[Vǫlva kvað:]}%
„Rindr berr \edtext{\emph{\alst{V}ála}}{\Afootnote{required by alliteration; om. \AM}} \hld\ í \alst{v}estr-sǫlum, &
\edtext{sá man \alst{Ó}ðins sonr \hld\ \alst{ęi}n-nę́ttr vega; &
\alst{h}ǫnd of þvę́r-\edtext{\emph{at}}{\Afootnote{om. \AM}} \hld\ né \alst{h}ǫfuð kęmbir, &
áðr ȧ \alst{b}ál of \alst{b}err \hld\ \alst{B}aldrs and-skota;}{\lemma{sá \dots\ and-skota ‘he will \dots\ shooter’}\Bfootnote{These lines are, apart from the verb tense, identical to \Voluspa\ 32/4–33/2.  It is possible that both are building on a now-lost third poem; or that one has got these lines from the other.  (For discussion on the myth itself see introduction to \Voluspa\ 31–34.)}} &
\alst{n}auðug sagða’k, \hld\ \alst{n}ú mun’k þęgja.“\eva

\bvb\speakernoteb{[The wallow quoth:]}%
“Rind bears \inx[P]{Wonnel} in the western halls: \\
he will, Weden’s son, one night old, fight. \\
He washes not his hand nor combs his head \\
before onto the pyre he bears Balder’s shooter— \\
forced I spoke, now I will shut up.”\evb\evg


\bvg\bva\mssnote{\AM~2r/6}\speakernote{[Óðinn kvað:]}%
„\alst{Þ}ęgj-at-tu vǫlva, \hld\ \alst{þ}ik vil’k fregna, &
\alst{u}nds \alst{a}l-kunna, \hld\ vil’k \alst{ę}nn vita, &
hvęrjar ’ru \alst{m}ęyjar, \hld\ es at \alst{m}uni gráta &
ok á \alst{h}imin verpa \hld\ \alst{h}alsa-skautum?“\eva

\bvb\speakernoteb{[Weden quoth:]}%
“Shut not up, wallow—thee I wish to ask! \\
Until all is known I wish yet to know: \\
Which are the maidens that heartily weep, \\
and onto heaven throw the front-sheets?\footnoteB{According to \Gylfaginning\ 49 Hell promised to give Balder back to the Eese if “all things in the world, living and dead, cry for him”. The Eese relayed this message, and “the men and the animals and the earth and the stones and trees and all metals” cried for Balder. It may be that these maidens were included among the grievers (perhaps they were the Walkirries, and this is what reveals Weden’s identity?), but their identity is otherwise unknown.  They may perhaps be identified with the maidens in \Vafthrudnismal\ 49.}”\evb\evg


\bvg\bva\mssnote{\AM~2r/8}\speakernote{[Vǫlva kvað:]}%
„\alst{E}rt-at Veg-tamr, \hld\ sem \alst{e}k hugða, &
hęldr ert \alst{Ó}ðinn, \hld\ \alst{a}ldinn gautr!“ &
\speakernote{[Óðinn kvað:]}„Ert-at \alst{v}ǫlva \hld\ né \alst{v}ís kona, &
hęldr ert \alst{þ}riggja \hld\ \alst{þ}ursa móðir!“\eva

\bvb\speakernoteb{[The wallow quoth:]}%
“Thou art not Waytame as I thought, \\
rather art thou Weden, the ancient Geat!”— \\
\speakernoteb{[Weden quoth:]}%
“Thou art no \inx[C]{wallow} nor wise woman, \\
rather art thou three \inx[G]{Thurses}’ mother!”\evb\evg


\bvg\bva\mssnote{\AM~2r/9}\speakernote{[Vǫlva kvað:]}%
„\alst{H}ęim ríð Óðinn \hld\ \edtrans{ok ves \alst{h}róðigr}{and be renowned}{\Bfootnote{A sarcastic taunt, the sense being: “Your fame, Weden, will not save you!”}}, &
svá komi-t \alst{m}anna \hld\ \alst{m}ęirr aptr ȧ vit, &
es \alst{l}auss \alst{L}oki \hld\ \alst{l}íðr ór bǫndum &
ok \alst{r}agna \alst{r}ǫk \hld\ \edtrans{\alst{r}júfęndr}{rippers}{\Bfootnote{Presumably Surt and Lock with his children, as described in \Voluspa\ 40 ff.  The verb \emph{rjúfa} ‘\CV: to break, rip up, break a hole in’ is used in the same context in the formulaic \emph{þá’s rjúfask ręgin} ‘when the \inx[G]{Reins} are ripped’ (\Vafthrudnismal\ 52), \emph{und’s (of) rjúfask ręgin} ‘until the Reins are ripped’ (\Grimnismal\ 4, \Lokasenna\ 41 and \Sigrdrifumal\ 17).
Cf. also the similar sounding (but not or only very distantly related) verb \emph{rifna} ‘be riven, rent apart’ in Runic inscription Sö 154 (Skarpåker, Sweden).}} koma.“\eva
%NOTE: If printing mythological Eddic poems separately One may also compare the similar sounding (but not or only very distantly related) verb \emph{rifna} ‘be riven, rent apart’ used in reference to the destruction of the world in Runic inscription Sö 154 (Skarpåker), and Arn \emph{Hryn} (in \Skp\ II pp. 185–6, ll. 3/7–8, see also note there): \emph{meiri verði þinn an þeira \hld\ þrifnuðr allr, und’s himinn rifnar.} ‘greater than theirs may thy whole wealth be, until heaven is riven.’}} koma.“

\bvb\speakernoteb{[The wallow quoth:]}%
“Ride home, Weden, and be renowned! \\
So may no man come again to visit, \\
when loose Lock slips out of his bonds,\\
and [at] the \inx[L]{Rakes of the Reins} the rippers come!”\evb\evg


%TODO Late stanzas in paper manuscripts.

\sectionline
% Weden
	\bookStart{The Speeches of the High One}[Hávamǫ́l]

%Introduction.

The \textbf{Speeches of the High One} is the second poem of \Regius, which is also the only ancient manuscript in which it is attested. Several sts. are however cited or alluded to in other places, such as Eyv \emph{Hák} (TODO: formatting) 21 and \FostrbroedhraSaga\ TODO.

The poem as it currently comes down to us hardly seems like a single composition, but instead more like a “grab bag” of traditional poetry associated with the god Weden.  It contains at least two poems of practical live advice, two mythological narratives, scattered gnomic poetry about runes, and a list of galders.  Little unites these various strands other than their speaker.

Following previous authors, I identify several such strands, excepting various lone sts. that are probably later inserts.  In the present edition each of them is given a separate, short introduction:

\begin{longtabu} to \textwidth {|c c c c c c|}
	\hline
  1–79 & The Guest-strand; practical life advice, beginning with a guest arriving at a homestead. \\
  81–89 & Various scattered sts. of advice. \\
  90–101 & Weden’s failed seduction of Billing’s daughter. \\
  102–109 & Weden’s obtaining of the Mead of Poetry \\
  110–136 & The Speeches of Loddfathomer; Weden’s advice to Loddfathomer. \\
  137–145 & The Rune-tally; various sts. relating to runes and their magical use. \\
  146–164 & The Leed-tally; Weden’s listing of 18 galders. \\
  165 & Final st., composed by the redactor or collector of the above poetry. \\ [1ex]
  \hline
\end{longtabu}

TODO: Discuss the Heathen identity of the redactor and the purpose of such a redaction.

\sectionline

\section{The Guest-strand (sts. 1–79)}

The Guest-Strand (Old Norse: \emph{Gęstaþáttr}) is possibly the finest work in Norse poetry. Sadly, its structure has been obscured by various inserted and possibly displaced sts. My hope is to shed some light on the original vision behind the poem, while as usual not changing the order of sts. as they appear in the only surviving witness manuscript.

The poem moves through many elements of life, but in a poetically almost seamless way. To move from one topic to another, the poet often employs transitions where a st. recalls the structure of the previous one, but with a new subject. This is particularly evident in sts. 4–5 and 10–11.

The strand begins with a st. encouraging travellers to be wary of entering strange houses without first spying out who is inside (1), after which a voice inside of a farmstead (possibly Weden?) announces that a guest is waiting to be let in (2). The same speaker then lists several things which the newly arrived guest needs from the host, namely: fire, food and clothes (3), water, a towel, a great welcome, a good reception, an opportunity to speak and silence in return (4).

After this focus shifts to the conduct of the wanderer, with an introductory st. explaining that he needs wit (specifically \inx[C]{manwit} (\emph{manvit}); see Encyclopedia), lest he become a laughing-stock (5). He should be silent but attentive, and choose his words carefully (6–7). He should be confident in himself and his own decisions, and not rely too much on the opinions of others (8–9), since there is nothing better one may bring along on the journey than much manwit (10).

Here the advice moves to the subject alcohol. Where the best thing one may bring along on the journey is manwit, the worst is too much ale (11). It is not as good as men call it (12) since it “robs [them] of their senses”; it is even personified as a “heron of forgetfulness” (13). A drinking round is best when the participants do not drink too much, but rather regain their senses afterwards (14).

St. 15 contains some general advice; a royal child should be silent, thoughtful and bold in battle, and all men should stay happy, until they die.

TODO.

\sectionline

\bvg
\bva\alst{G}áttir allar \hld\ áðr \alst{g}angi framm &
\ind \edtext{of \alst{sk}oðask \alst{sk}yli,}{\Bfootnote{om. \GylfMS}} &
\ind of \alst{sk}yggnask \alst{sk}yli; &
því-at ó·\alst{v}íst ’s at \alst{v}ita, \hld\ hvar ó·\alst{v}inir &
\ind sitja á \alst{f}lęti \alst{f}yrir.\eva

\bvb All doorways—before one might go forth— \\
should be watched, \\
should be spied at; \\
for uncertain ’tis to know, where enemies \\
sit on the benches within.\evb
\evg


\bvg
\bva \edtrans{\alst{G}efęndr}{the givers}{\Bfootnote{The hosts.}} hęilir, \hld\ \alst{g}ęstr ’s inn kominn, &
\ind hvar skal \alst{s}itja \alst{s}já? &
mjǫk es \alst{b}ráðr \hld\ sá’s \edtrans{á \alst{b}rǫndum}{on the fires}{\Bfootnote{Possibly referring a Norwegian folk custom, wherein a guest would sit down on the wood-pile outside of the door, waiting until being let in. See further TODO SOME ARTICLE on this custom. The speaker thus announces to the hosts that a frozen, wet and tired guest has arrived and currently sits impatiently on the wood-pile, and ought to be taken in.}} skal &
\ind \edtrans{síns of \alst{f}ręista \alst{f}rama}{try his distinction}{\Bfootnote{Formulaic, also occurring in TODO other places.}}.\eva

\bvb Hail the givers, a guest is come in! \\
Where shall this one sit? \\
Very impatient is he, who on the fires shall \\
try his distinction.\evb
\evg


\bvg
\bva\alst{Ę}lds es þǫrf \hld\ þęim’s \alst{i}nn es kominn &
\ind ok á \alst{k}néi \alst{k}alinn, &
\alst{m}atar ok váða \hld\ es \alst{m}anni þǫrf, &
\ind þęim’s hęfr of \alst{f}jall \alst{f}arit.\eva

\bvb Of fire is there need for the one who is come in, \\
and cold about the knees; \\
of food and of clothing is there need for that man \\
who over the fell has fared.\evb
\evg


\bvg
\bva\alst{V}ats es þǫrf \hld\ þęim’s til \alst{v}erðar kømr, &
\ind \alst{þ}ęrru ok \alst{þ}jóð-laðar, &
\alst{g}óðs of ǿðis, \hld\ —ef sér \alst{g}eta mę́tti— &
\ind \alst{o}rðs ok \alst{ę}ndr-þǫgu.\eva

\bvb Of water is there need for the one who comes for a meal; \\
of a towel and of a great welcome; \\
of a good reception—if he might get one— \\
of speech, and of silence in return.\footnoteB{There is a well thought-out linear progression throughout this st.: The guest must first wash, then dry himself with a towel, then be welcomed to sit and eat at the table and speak with the host. The host has done his part, and now it is the guest’s turn. This nicely leads the transition to the following sts., where the proper conduct of the guest (first in speech, and then in various other areas) is discussed.}\evb
\evg


\bvg
\bva\alst{V}its es þǫrf \hld\ þęim’s \alst{v}íða ratar; &
\ind dę́lt es \alst{h}ęima \alst{h}vat; &
at \alst{au}ga-bragði \hld\ verðr sá’s \alst{ę}kki kann &
\ind ok með \alst{s}notrum \alst{s}itr.\eva

\bvb Of wit is there need for the one who widely roams; \\
everything is easy at home. \\
A laughing-stock\footnoteB{An idiom, \emph{auga-bragð} lit. ‘twinkling of an eye, moment’.} becomes he who nothing knows, \\
and among the clever sits.\evb
\evg


\bvg
\bva At \alst{h}yggjandi sinni \hld\ skyli-t maðr \alst{h}rǿsinn vesa, &
\ind hęldr \alst{g}ę́tinn at \alst{g}ęði, &
þá’s \alst{h}orskr ok þǫgull \hld\ kømr \alst{h}ęimis-garða til, &
\ind sjaldan verðr \alst{v}íti \alst{v}ǫrum. &
því-at \alst{ó}·brigðra vin \hld\ fę́r maðr \alst{a}ldri-gi, &
\ind an \alst{m}an-vit \alst{m}ikit.\eva

\bvb Of his thinking should man not be boastful; \\
rather guarding of his senses, \\
when sharp and silent he comes to a homestead; \\
sudden injury seldom strikes the wary, \\
for an unfickler friend man never gets \\
than much \inx[C]{manwit}.\evb
\evg


\bvg
\bva Hinn \alst{v}ari gęstr, \hld\ es til \alst{v}erðar kømr, &
\ind \alst{þ}unnu hljóði \alst{þ}ęgir; &
\alst{ęy}rum hlýðir, \hld\ en \alst{au}gum skoðar, &
\ind svá \edtext{nýsisk \alst{f}róðra hvęrr \alst{f}yrir}{\lemma{nýsisk \dots\ fyrir ‘looks \dots\ ahead’}\Bfootnote{Verb underlying the noun \emph{for-njósn} as found in \Sigrdrifumal\ 24.}}.\eva

\bvb The wary guest—when for a meal he comes— \\
with thin listening shuts up.\footnoteB{i.e. is in attentive silence.} \\
With ears he listens, but with eyes he observes; \\
so looks each learned man ahead.\evb
\evg


\bvg
\bva Hinn es \alst{s}ę́ll, \hld\ es \alst{s}ér of getr &
\ind \edtrans{\alst{l}of ok \alst{l}íkn-stafi}{praise and staves of liking}{\Bfootnote{\emph{líkn} ‘liking’ is a very interesting word. It is defined by \ONP\ as: ‘mercy, compassion, relief, comfort, help’. In the present poem its precise meaning seems to be something like ‘the state of being liked by your surroundings to the point where people are willing to help you out’. Cf. its two other occurrences in the present poem: sts. 120 and especially 123 (where it is likewise paired with \emph{lof} ‘praise’).}}; &
\alst{ó}·dę́lla ’s við þat, \hld\ es \alst{ęi}ga skal &
\ind \alst{a}nnars brjóstum \alst{í}.\eva

\bvb The one is blessed, who for himself gets \\
praise and staves of liking. \\
’Tis uneasy regarding that which one shall own \\
in another man’s breast.\evb
\evg


\bvg
\bva\alst{S}á es \alst{s}ę́ll, \hld\ es \alst{s}jalfr of á &
\ind \alst{l}of ok vit meðan \alst{l}ifir; &
því-at \alst{i}ll rǫ́ð \hld\ hęfr maðr \alst{o}pt þęgit &
\ind \alst{a}nnars brjóstum \alst{ó}r.\eva

\bvb He is blessed, who himself does own \\
praise and wits while he lives, \\
for ill counsels has man oft taken \\
out of another man’s breast.\evb
\evg


\bvg
\bva\alst{B}yrði \alst{b}ętri \hld\ berr-at maðr \alst{b}rautu at, &
\ind an sé \alst{m}an-vit \alst{m}ikit; &
\alst{au}ði bętra \hld\ þykkir þat í \alst{ó}·kunnum stað; &
\ind slíkt es \alst{v}á-laðs \alst{v}era.\eva

\bvb A better burden bears man not on the road \\
than much manwit. \\
In an unknown place it seems better than wealth; \\
such is the destitute man’s shelter.\evb
\evg

% TODO: NEW SECTION (Alcohol)

\bvg
\bva\alst{B}yrði \alst{b}ętri \hld\ berr-at maðr \alst{b}rautu at, &
\ind an sé \alst{m}an-vit \alst{m}ikit; &
\alst{v}eg-nest \alst{v}erra \hld\ \alst{v}egr-a \edtrans{\alst{v}ęlli at}{on the plain}{\Bfootnote{Formulaic, the word \emph{vǫllr} ‘plain, (uncultivated) field’ is also used in sts. 38 and 49. It is easily understood that the wild heaths and plains of Iron Age Norway were particularly unsafe places where a traveller needed to keep his wits about him, lest he fall victim to robbers or murderers (so st. 38).}}, &
\ind an sé \alst{o}f-drykkja \alst{ǫ}ls.\eva

\bvb A better burden bears man not on the road \\
than much manwit. \\
Worse way-provision he drags not along on the plain \\
than a too great drink of ale.\evb
\evg


\bvg
\bva Es-a svá \alst{g}ótt, \hld\ sęm \alst{g}ótt kveða, &
\ind \alst{ǫ}l \alst{a}lda sonum; &
því-at \alst{f}ę́ra vęit, \hld\ es \alst{f}lęira drekkr, &
\ind síns til \alst{g}ęðs \alst{g}umi.\eva

\bvb ’Tis not so good, as good they say, \\
ale for the sons of men; \\
for the less he knows, as the more he drinks, \\
man of his own senses.\evb
\evg


\bvg
\bva \edtrans{\alst{Ó}·minnis-hegri}{Forgetfulness-heron}{\Bfootnote{Lit. “unmemory-heron”; a rather interesting personification of drunkenness as a hovering bird.}} hęitir, \hld\ sá’s yfir \alst{ǫ}lðrum þrumir, &
\ind hann stelr \alst{g}ęði \alst{g}uma; &
þess \alst{f}ogls \alst{f}jǫðrum \hld\ ek \alst{f}jǫtraðr vas’k &
\ind í \alst{g}arði \alst{G}unnlaðar.\eva

\bvb Forgetfulness-heron is called he who over ale-feasts hovers: \\
he robs man of his senses. \\
With that bird’s feathers was I fettered \\
in the yards of \inx[P]{Guthlathe}.\evb
\evg


\bvg
\bva\alst{Ǫ}lr ek varð, \hld\ varð \alst{o}fr-ǫlvi, &
\ind at hins \alst{f}róða \alst{F}jalars; &
því es \alst{ǫ}lðr batst, \hld\ at \alst{a}ptr of hęimtir &
\ind hvęrr sitt \alst{g}ęð \alst{g}umi.\eva

\bvb Drunk I became—I became the drunkest by far— \\
at the learned Fealer’s [home].— \\
That ale-feast is best, where every man \\
fetches back his senses.\evb
\evg

% TODO: NEW SECTION (War)

\bvg
\bva\alst{Þ}agalt ok hugalt \hld\ skyli \alst{þ}jóðans barn &
\ind ok \alst{v}íg-djarft \alst{v}esa; &
\alst{g}laðr ok ręifr \hld\ skyli \alst{g}umna hvęrr, &
\ind unds sinn \alst{b}íðr \alst{b}ana.\eva

\bvb Silent and thoughtful should the ruler’s child \\
—and battle-bold—be. \\
Glad and cheerful should each man [be], \\
until he suffer his bane.\evb
\evg


\bvg
\bva\alst{Ó}·snjallr maðr \hld\ hyggsk munu \alst{ę}y lifa, &
\ind ef við \alst{v}íg \alst{v}arask; &
en \alst{ę}lli gefr hǫ́num \hld\ \alst{ę}ngi frið, &
\ind þótt hǫ́num \alst{g}ęirar \alst{g}efi.\eva

\bvb The unvalorous man thinks he will always live \\
if he of war be wary; \\
but old age gives him no peace, \\
although spears would give him.\footnoteB{The unvalorous man might have been spared by the spears, but death will still find him through miserable old age. Since death is unavoidable it is better to live bravely, even if one risks dying in battle, than to live cowardly and die of sickness. This connects well to the ancient view of the ‘straw-death’ (TODO).}\evb
\evg


\bvg
\bva\alst{K}ópir af-glapi, \hld\ es til \alst{k}ynnis kømr, &
\ind \alst{þ}ylsk hann umb eða \alst{þ}rumir; &
alt es \alst{s}ęnn, \hld\ ef \alst{s}ylg of getr, &
\ind uppi ’s þá \alst{g}ęð \alst{g}uma.\eva

\bvb Gapes the oaf when to visit he comes; \\
he mumbles about or loiters. \\
All at once—if a sip he gets— \\
are the senses of the man exposed.\evb
\evg


\bvg
\bva Sá ęinn \alst{v}ęit, \hld\ es \alst{v}íða ratar &
\ind ok hęfr \edtrans{\alst{f}jǫlð of \alst{f}arit}{journeyed much}{\Bfootnote{Formulaic, also occuring in \Vafthrudnismal\ 3, 44, and so on in the fixed lines spoken by Weden: \emph{Fjǫlð ek fór, \hld\ fjǫlð fręistaða’k, // fjǫlð ek ręynda ręgin} ‘Much I journeyed, much I tried, much I tested the \inx[G]{Reins}.’.}}, &
hvęrju \alst{g}ęði \hld\ stýrir \alst{g}umna hvęrr, &
\ind sá es \alst{v}itandi ’s \alst{v}its.\eva

\bvb He alone knows, who widely roams, \\
and has journeyed much: \\
his own senses does each man control, \\
who is knowing of his wits.\evb
\evg


\bvg
\bva\alst{H}aldi-t maðr á kęri, \hld\ drekki þó at \alst{h}ófi mjǫð, &
\ind \edtrans{mę́li \alst{þ}arft eða \alst{þ}ęgi}{he ought to speak the needful or shut up}{\Bfootnote{Formulaic, line occurs identically in \Vafthrudnismal\ 10/2.}}; &
\alst{ó}·kynnis þess \hld\ váar þik \alst{ę}ngi maðr, &
\ind at gangir \alst{s}nimma at \alst{s}ofa.\eva

\bvb Man ought not to hold onto the cask, yet drink mead in moderation;\footnoteB{Drinking horns at this time could not be set down, and so to “hold onto” may have been an expression for not drinking. The st. may also be referring to the toasting ritual wherein a single vessel would be passed around and drunk from by each person (indeed this is the origin of the Scandinavian toasting-word, \emph{skål} ‘prosit, cheers!’, lit. ‘bowl!’). At such celebrations “holding onto” the vessel and refusing to drink was very rude; as late as 1519 a man in Jämtland was killed in an argument resulting from his refusal to pass on to the bowl (see \textcite{Sjöberg1907}).} \\
he ought to speak the needful or shut up. \\
For that uncouthness will no man blame thee, \\
that thou go early to sleep.\evb
\evg


\bvg
\bva\alst{G}rǫ́ðugr halr, \hld\ nema \alst{g}ęðs viti, &
\ind \alst{e}tr sér \alst{a}ldr-trega; &
opt fę́r \alst{h}lǿgis, \hld\ es með \alst{h}orskum kømr, &
\ind \alst{m}anni hęimskum \alst{m}agi.\eva

\bvb The gluttonous man—unless he know his sense— \\
eats himself a life-sorrow. \\
Oft the belly, when among the sharp he comes, \\
brings a foolish man ridicule.\evb
\evg


\bvg
\bva\alst{H}jarðir þat vitu, \hld\ nę́r \alst{h}ęim skulu, &
\ind ok \alst{g}anga þá af \alst{g}rasi; &
en \alst{ó}·sviðr maðr \hld\ kann \alst{ę́}va-gi &
\ind síns of \alst{m}ál \alst{m}aga.\eva

\bvb Herds know when homewards they shall [turn], \\
and then part from the grass; \\
but an unwise man never knows \\
his own belly’s measure.\evb
\evg


\bvg
\bva\alst{V}e-sall maðr \hld\ ok \alst{i}lla skapi &
\ind \alst{h}lę́r at \alst{h}ví-vetna; &
hit-ki hann \alst{v}ęit, \hld\ es \alst{v}ita þyrpti, &
\ind at \edtrans{hann es-a \alst{v}amma \alst{v}anr}{he is not free of blemishes}{\Bfootnote{Formulaic, cf. \Lokasenna\ 30: \emph{es-a þér vamma vant} ‘thou art not free of blemishes’.}}.\eva

\bvb The wretched man and badly tempered \\
laughs at anything. \\
This he knows not, which he might need to know: \\
that he is not free of blemishes.\evb
\evg


\bvg
\bva\alst{Ó}·sviðr maðr \hld\ vakir umb \alst{a}llar nę́tr &
\ind ok \alst{h}yggr at \alst{h}ví-vetna; &
þá es \alst{m}óðr, \hld\ es at \alst{m}orni kømr; &
\ind alt es \alst{v}íl sęm \alst{v}as.\eva

\bvb The unwise man is awake during all nights, \\
and thinks of anything. \\
Then he is weary when the morning comes: \\
all the trouble is as it was.\evb
\evg


\bvg
\bva\alst{Ó}·snotr maðr \hld\ hyggr sér \alst{a}lla vesa &
\ind \alst{v}ið-hlę́jęndr \alst{v}ini; &
hit-ki hann \alst{f}iðr, \hld\ þótt of hann \alst{f}ár lesi, &
\ind ef með \alst{s}notrum \alst{s}itr.\eva

\bvb The unclever man thinks all to be \\
who laugh with him his friends. \\
This he finds not, that they still see flaws in him, \\
if among the clever he sits.\evb
\evg


\bvg
\bva\alst{Ó}·snotr maðr \hld\ hyggr sér \alst{a}lla vesa &
\ind \alst{v}ið-hlę́jęndr \alst{v}ini; &
\alst{þ}á þat fiðr \hld\ es at \alst{þ}ingi kømr, &
\ind at \edtrans{á \alst{f}or-mę́lęndr \alst{f}áa}{has spokesmen few}{\Bfootnote{Repeated in st. 62. He has few who are ready to take his side and speak up for him; the sense is that true friends are proven in conflict, not in easy things like laughing. The Thing was the old Germanic legal assembly, and so the specific reference here is to legal disputes, which, however, could easily turn into deadly feuds.}}.\eva

\bvb The unclever man thinks all to be \\
who laugh with him his friends. \\
Then he finds it, when to the \inx[C]{Thing} he comes, \\
that he has spokesmen few.\evb
\evg


\bvg
\bva\alst{Ó}·snotr maðr \hld\ þykkisk \alst{a}lt vita, &
\ind ef á sér i \alst{v}ǫ́ \alst{v}eru; &
hit-ki hann \alst{v}ęit, \hld\ hvat skal \alst{v}ið kveða, &
\ind ef hans \alst{f}ręista \alst{f}irar.\eva

\bvb The unclever man seems to know everything \\
if he finds shelter in a nook. \\
This he knows not, what he shall say in return \\
if men test him.\evb
\evg


\bvg
\bva\alst{Ó}·snotr maðr, \hld\ es með \alst{a}ldir kømr, &
\ind \alst{þ}at ’s batst at hann \alst{þ}ęgi; &
\alst{ę}ngi þat vęit, \hld\ at hann \alst{ę}kki kann, &
\ind nema hann \alst{m}ę́li til \alst{m}art. &
\alst{v}ęit-a maðr, \hld\ hinn’s \alst{v}ę́t-ki vęit, &
\ind þótt hann \alst{m}ę́li til \alst{m}art.\eva

\bvb The unclever man, when among people he comes, \\
’tis best that he shut up. \\
Noone knows that he nothing knows, \\
unless he speak too much. \\
The man knows not, who nothing knows, \\
that he speak too much.\evb
\evg


\bvg
\bva\alst{F}róðr sá þykkisk, \hld\ es \alst{f}regna kann, &
\ind ok \alst{s}ęgja hit \alst{s}ama, &
\alst{ęy}-vitu lęyna \hld\ męgu \alst{ý}ta synir &
\ind því es \alst{g}ęngr of \alst{g}uma.\eva

\bvb Learned seems he who can ask \\
and answer likewise. \\
Naught may the sons of men conceal \\
of that [gossip] which goes about a man.\evb
\evg


\bvg
\bva\alst{Ǿ}rna mę́lir, \hld\ sá’s \alst{ę́}va þęgir, &
\ind \alst{st}að-lausu \alst{st}afi; &
\edtext{\alst{h}rað-mę́lt tunga, \hld\ \edtrans{nema \alst{h}aldęndr ęigi}{unless it be held in place}{\Bfootnote{lit. ‘unless holders own it’ or ‘unless it own holders’. The ‘holders’ are perhaps the teeth which hold the tongue in place.}}, &
\ind opt sér ó·\alst{g}ótt of \alst{g}ęlr}{\lemma{hrað-mę́lt \dots\ of gęlr ‘A quick-spoken \dots\ for itself’}\Bfootnote{Formulaic. Cf. \Lokasenna\ 31.}}.\eva

\bvb Plenty enough speaks he who never shuts up \\
utterings of absurdity. \\
A quick-spoken tongue—unless it be held in place— \\
oft sings evil [into being] for itself.\evb
\evg


\bvg
\bva At \alst{au}ga-bragði \hld\ skal-a maðr \alst{a}nnan hafa, &
\ind þótt til \alst{k}ynnis \alst{k}omi; &
margr \alst{f}róðr þykkisk, \hld\ ef \alst{f}reginn es-at &
\ind ok nái \edtrans{\alst{þ}urr-fjallr}{dry-skinned}{\Bfootnote{i.e. ‘untested’, equivalent to the English idiom \emph{get one’s feet wet}. The word \emph{fell} \char`~ \emph{fjall} ‘skin, pelt’ is rare in Old Norse literature and only occurs in cpds, e.g. \Volundarkvida\ 11: \emph{ber-fjall} ‘bear-pelt’. Cf. however Swedish \emph{fjäll} ‘scale (on fish and reptiles)’}} \alst{þ}ruma.\eva

\bvb As a laughing-stock shall man not have another \\
when he comes to visit. \\
Many a one seems learned if he is not asked, \\
and manages to loiter about dry-skinned.\evb
\evg


\bvg
\bva\alst{F}róðr þykkisk \hld\ sá’s \alst{f}lótta tękr &
\ind \edtrans{\alst{g}ęstr}{guest}{\Bfootnote{Here probably ‘stranger’; when being mocked by a stranger it is best not to engage, since the conversation can quickly turn violent. Cf. sts. 122–123 and 125.}} at \alst{g}ęst hę́ðinn; &
\alst{v}ęit-a gǫrla \hld\ sá’s of \alst{v}erði glissir, &
\ind þótt með \alst{g}rǫmum \alst{g}lami.\eva

\bvb Learned seems he who takes to flight, \\
the guest, from a scoffing guest. \\
Clearly knows not he who grins over the food, \\
that he with fiends be prattling.\evb
\evg


\bvg
\bva\alst{G}umnar margir \hld\ erusk \alst{g}agn-hollir, &
\ind en at \alst{v}irði \alst{v}rekask; &
\alst{a}ldar róg \hld\ þat mun \alst{ę́} vesa; &
\ind órir \alst{g}ęstr við \alst{g}ęst.\eva

\bvb Many men are \inx[C]{hold} to each other, \\
but over a meal drive each other away. \\
The strife of mankind will that ever be; \\
guest raves against guest.\evb
\evg


\bvg
\bva\alst{Á}r-liga verðar \hld\ skyli maðr \alst{o}pt fáa, &
\ind nema til \alst{k}ynnis \alst{k}omi; &
\alst{s}itr ok \alst{s}nópir, \hld\ lę́tr sęm \alst{s}olginn sé, &
\ind ok kann \alst{f}regna at \alst{f}ǫ́u.\eva

\bvb An early meal should man oft get, \\
unless he come to visit: \\
he sits and idles haplessly, makes as if starved, \\
and can ask about little.\evb
\evg


\bvg
\bva\alst{A}f-hvarf mikit \hld\ es til \alst{i}lls vinar, &
\ind þótt á \alst{b}rautu \alst{b}úi, &
en til \alst{g}óðs vinar \hld\ liggja \alst{g}agn-vegir, &
\ind þótt hann sé \alst{f}irr \alst{f}arinn.\eva

\bvb A great detour ’tis to a bad friend, \\
although he on the highway live; \\
but to a good friend lie the finest ways, \\
although he far gone be.\evb
\evg


\bvg
\bva\alst{G}anga \edtext{skal}{\Afootnote{emend.; om. \Regius}}, \hld\ skal-a \alst{g}ęstr vesa &
\ind \alst{ęy} í \alst{ęi}num stað; &
\alst{l}júfr verðr \alst{l}ęiðr, \hld\ ef \alst{l}ęngi sitr &
\ind \alst{a}nnars flętjum \alst{á}.\eva

\bvb Go shall one; shall not be a guest \\
forever in one place. \\
The loved becomes loathed if for long he sits \\
on another man’s benches.\evb
\evg


\bvg
\bva\alst{B}ú es \alst{b}ętra, \hld\ þótt lítit sé, &
\ind \alst{h}alr es \alst{h}ęima \alst{h}vęrr; &
þótt \alst{t}vę́r gęitr ęigi \hld\ ok \alst{t}aug-ręptan sal, &
\ind þat ’s þó \alst{b}ętra an \alst{b}ǿn.\eva

\bvb A dwelling is better, though small it be: \\
each is a warrior at home. \\
Though two goats he own, and a cord-roofed hall, \\
that is yet better than begging.\evb
\evg


\bvg
\bva\alst{B}ú es \alst{b}ętra, \hld\ þótt lítit sé, &
\ind \alst{h}alr es \alst{h}ęima \alst{h}vęrr; &
\alst{b}lóðugt es hjarta \hld\ þęim’s \alst{b}iðja skal &
\ind sér í \alst{m}ál hvęrt \alst{m}atar.\eva

\bvb A dwelling is better, though small it be: \\
each is a warrior at home. \\
Bloody is the heart of the one who shall beg \\
for himself each meal of food.\evb
\evg


\bvg
\bva\alst{V}ǫ́pnum sínum \hld\ skal-a maðr \edtrans{\alst{v}ęlli á}{on the plain}{\Bfootnote{Formulaic, see note to st. 12.}} &
\ind \edtrans{\alst{f}eti ganga \alst{f}ramarr}{take one step further}{\Bfootnote{Formulaic. Cf. \Lokasenna\ 1: \emph{svát ęinu-gi feti gangir framarr} ‘so that thou not take one step further’.}}; &
því-at ó·\alst{v}íst ’s at \alst{v}ita, \hld\ nę́r verðr á \alst{v}egum úti &
\ind \alst{g}ęirs of þǫrf \alst{g}uma.\eva

\bvb From his weapons shall man not on the plain \\
take one step further; \\
for uncertain ’tis to know, when on the ways outside, \\
man comes in need of a spear.\evb
\evg


\bvg
\bva Fann’k-a \alst{m}ildan \alst{m}ann \hld\ eða svá \edtrans{\alst{m}atar góðan}{good of meat}{\Bfootnote{A Viking Age expression; see Encyclopedia.}}, &
\ind at vę́ri-t \alst{þ}iggja \alst{þ}egit; &
eða \alst{s}íns féar \hld\ \alst{s}vá-gi \edtext{[...]}{\Bfootnote{It is doubtless that a word has been lost here; the meter and sense require it. \textcite{FinnurEdda}\ suggests \emph{gløggvan} ‘miserly, stingy’, giving a litotes ‘so not stingy’, i.e., ‘so generous’.}}, &
\ind at \alst{l}ęið sé \alst{l}aun, ef þegi.\eva

\bvb I found not a generous man, or one so \inx[C]{good of meat}, \\
that a gift were not accepted; \\
or one of his \inx[C]{fee} so not [...], \\
that the rewards were loathed, if he accepted [them].\footnoteB{No man is so generous that he would refuse a gift presented to him, nor loathe receiving a favour as thanks for his generosity.}\evb
\evg


\bvg
\bva\alst{F}éar síns, \hld\ es \alst{f}ęngit hęfr, &
\ind skyli-t maðr \alst{þ}ǫrf \alst{þ}ola; &
opt sparir \alst{l}ęiðum \hld\ þat’s hęfr \alst{l}júfum hugat; &
\ind mart gęngr \alst{v}err an \alst{v}arir.\eva

\bvb Of his own \inx[C]{fee}, which he has earned, \\
should man not suffer need. \\
Oft one saves for the loathed what was meant for the loved;\\
many a thing goes worse than one expects.\evb
\evg


\bvg
\bva\alst{V}ǫ́pnum ok \alst{v}ǫ́ðum \hld\ skulu \alst{v}inir glęðjask; &
\ind þat ’s á \alst{s}jǫlfum \alst{s}ýnst; &
\alst{v}iðr-gefęndr ok ęndr-gefęndr \hld\ erusk \alst{v}inir lęngst, &
\ind ef þat bíðr at \alst{v}erða \alst{v}ęl.\eva

\bvb With weapons and garments shall friends gladden each other; \\
that is most seen on oneself.\footnoteB{i.e. in one’s own lived experience.} \\
Mutual givers and return-givers are friends for the longest, \\
if it\footnoteB{The friendship.} is to last long.\evb
\evg


\bvg
\bva\alst{V}in sínum \hld\ skal maðr \alst{v}inr \alst{v}esa, &
\ind ok \alst{g}jalda \alst{g}jǫf við \alst{g}jǫf; &
\alst{h}látr við \alst{h}látri \hld\ skyli \alst{h}ǫlðar taka, &
\ind en \alst{l}ausung við \alst{l}ygi.\eva

\bvb With his friend shall man be a friend, \\
and pay gift against gift; \\
laughter against laughter should men employ, \\
but duplicity against lie.\evb
\evg


\bvg
\bva\alst{V}in sínum \hld\ skal maðr \alst{v}inr vesa, &
\ind \alst{þ}ęim ok \alst{þ}ess vin; &
en \alst{ó}·vinar síns \hld\ skyli \alst{ę}ngi maðr &
\ind \alst{v}inar \alst{v}inr \alst{v}esa.\eva

\bvb With his friend shall man be a friend, \\
with him and his friend; \\
but with his enemy’s, should no man, \\
friend’s friend be.\evb
\evg


\bvg
\bva\alst{V}ęitst, ef \alst{v}in átt, \hld\ þann’s \alst{v}ęl trúir &
\ind ok vilt af hǫ́num \alst{g}ótt \alst{g}eta, &
\alst{g}ęði skalt við þann \hld\ ok \alst{g}jǫfum skipta, &
\ind \alst{f}ara at \alst{f}inna opt.\eva

\bvb Know, if thou have a friend, one on which thou well trust, \\
and wilt receive good from him: \\
thoughts and gifts shalt thou trade with that one, \\
{[and]} journey to find him oft.\footnoteB{Several lines of the present st. are shared with st. 119.}\evb
\evg


\bvg
\bva Ef þú \alst{á}tt \alst{a}nnan, \hld\ þann’s \alst{i}lla trúir, &
\ind vilt af hǫ́num þó \alst{g}ótt \alst{g}eta, &
\edtext{\alst{f}agrt skalt mę́la við þann, \hld\ en \alst{f}látt hyggja}{\lemma{fagrt \dots\ mę́la \dots\ flátt hyggja ‘fairly \dots\ speak \dots\ falsely think’}\Bfootnote{Formulaic, cf. sts. 90, 91.}} &
\ind ok gjalda \alst{l}ausung við \alst{l}ygi.\eva

\bvb If thou have another, one on which thou badly trust, \\
and wilt yet receive good from him: \\
fairly shalt thou speak with that one, but falsely think, \\
and pay duplicity against lie.\evb
\evg


\bvg
\bva Þat ’s \alst{ę}nn umb þann, \hld\ es þú \alst{i}lla trúir &
\ind ok þér es \alst{g}runr at \alst{g}ęði, &
\alst{h}lę́ja skalt við þęim \hld\ ok of \alst{h}ug mę́la; &
\ind \alst{g}lík skulu \alst{g}jǫld \alst{g}jǫfum.\eva

\bvb ’Tis yet regarding that one, on which thou badly trustest, \\
and who causes thy senses doubt:\footnoteB{lit. “and for thee is doubt in senses”.} \\
laugh shalt thou with him, and speak thoughtfully; \\
payments shall be equal to gifts.\footnoteB{Equivalent to the last line of the previous st. (“pay duplicity against lie”).}\evb
\evg


\bvg
\bva Ungr vas’k \alst{f}orðum, \hld\ \alst{f}ór’k ęinn saman, &
\ind þá varð’k \alst{v}illr \alst{v}ega; &
\alst{au}ðigr þóttumk, \hld\ es \alst{a}nnan fann’k, &
\ind \alst{m}aðr es \alst{m}anns gaman.\eva

\bvb Young was I once, I travelled alone; \\
then I became lost about the ways. \\
Wealthy I thought myself when another one I found; \\
man is man’s pleasure.\evb
\evg


\bvg
\bva\alst{M}ildir frǿknir \hld\ \alst{m}ęnn batst lifa, &
\ind \alst{s}jaldan \alst{s}út ala; &
\alst{ó}·snjallr maðr \hld\ \alst{u}ggir hvat-vetna, &
\ind sýtir ę́ \alst{g}løggr við \alst{g}jǫfum.\eva

\bvb Generous, bold men live the best; \\
seldom they nourish grief. \\
The unvalorous man is frightened by anything; \\
ever the stingy man grieves over gifts.\footnoteB{Refer back to st. 39; after receiving a gift, one was culturally obliged to give something back.}\evb
\evg


\bvg
\bva\alst{V}áðir mínar \hld\ gaf’k \alst{v}ęlli at &
\ind \alst{t}vęim \alst{t}ré-mǫnnum; &
\alst{r}ekkar þat þóttusk, \hld\ es \alst{r}ipt hǫfðu; &
\ind \alst{n}ęiss es \alst{n}ǫkkviðr halr.\eva

\bvb My garments I gave, on the plain, \\
to two tree-men. \\
Champions they seemed when cloaks they had; \\
shameful is the naked warrior.\footnoteB{One of the hardest sts. in the poem. After much thought I consider the probable sense to be that “the clothes make the man”. Under expensive gear a thin tree-man might be hiding, and likewise even a strong man (I see the choice of the word \emph{halr} ‘warrior’ rather than the more neutral \emph{maðr} ‘man, person’ as intentional) when naked and facing a heavily armoured opponent becomes as vulnerable as the ‘tree-man’ on a plain.}\evb
\evg


\bvg
\bva Hrørnar \alst{þ}ǫll, \hld\ sú’s stęndr \alst{þ}orpi á, &
\ind hlýr-at hęnni \alst{b}ǫrkr né \alst{b}arr; &
svá es \alst{m}aðr, \hld\ sá’s \alst{m}ann-gi ann; &
\ind hvat skal hann \alst{l}ęngi \alst{l}ifa?\eva

\bvb Wilters the pine that stands on the yard; \\
shields her not bark nor needle. \\
So is the man who loves no man; \\
for what shall he live for long?\evb
\evg


\bvg
\bva\alst{Ę}ldi hęitari \hld\ brinnr með \alst{i}llum vinum &
\ind \alst{f}riðr \alst{f}imm daga, &
en þá \alst{sl}oknar, \hld\ es hinn \alst{s}étti kømr, &
\ind ok \alst{v}ersnar allr \alst{v}in-skapr.\eva

\bvb Hotter than fire burns love among bad friends, \\
for \inx[C]{five days};\footnoteB{A reference to the five-day week (see also st. 74); the number is symbolic. See further Encyclopedia.} \\
but then goes out when the sixth one comes, \\
and all the friendship worsens.\evb
\evg


\bvg
\bva\alst{M}ikit ęitt \hld\ skal-a \alst{m}anni gefa; &
\ind opt kaupir sér í \alst{l}ítlu \alst{l}of, &
með \alst{h}ǫlfum \alst{h}lęif \hld\ ok með \alst{h}ǫllu kęri &
\ind \alst{f}ekk ek mér \alst{f}é-laga.\eva

\bvb Much at once shall one not give a man; \\
oft one buys oneself praise for little. \\
With half a loaf and an awry cask, \\
I got myself a partner.\evb
\evg


\bvg
\bva\alst{L}ítilla sanda, \hld\ \alst{l}ítilla sę́va, &
\ind lítil eru \alst{g}ęð \alst{g}uma; &
því-at \alst{a}llir męnn \hld\ \alst{u}rðu-t jafn-spakir; &
\ind \alst{h}ǫlf es ǫld \alst{h}var.\eva

\bvb Of small sands, of small seas; \\
small are the senses of man. \\
For all have not become evenly knowing; \\
half is every man.\footnoteB{The genitive “of small sands, of small seas” is probably a partitive, the sense being that man’s horizons are small; the universe is far greater than he and always will be. On the meaning of the second half of the st. I find that of \textcite{Athugasemdir1929} most convincing, namely that every man has both strengths and weaknesses. As nobody can excel at everything, nobody is complete; every person is “half” (and it should be added that ON \emph{halfr} has a more general sense of incompleteness than its English cognate). This interpretation fits particularly closely with sts. 71 and 132.}\evb
\evg


\bvg
\bva\alst{M}eðal-snotr \hld\ skyli \alst{m}anna hvęrr, &
\ind ę́va til \alst{s}notr \alst{s}éi; &
þęim es \alst{f}yrða \hld\ \alst{f}ęgrst at lifa, &
\ind es \alst{v}ęl mart \alst{v}itu.\eva

\bvb Middle-clever should each man be; \\
never too clever. \\
For those men ’tis fairest to live \\
who know well enough.\evb
\evg


\bvg
\bva\alst{M}eðal-snotr \hld\ skyli \alst{m}anna hvęrr, &
\ind ę́va til \alst{s}notr \alst{s}éi; &
\alst{s}notrs manns hjarta \hld\ verðr \alst{s}jaldan glatt, &
\ind ef sá ’s \alst{a}l-snotr es \alst{á}.\eva

\bvb Middle-clever should each man be; \\
never too clever. \\
The clever man’s heart is seldom gladdened, \\
if he is all-clever that owns [it].\evb
\evg


\bvg
\bva\alst{M}eðal-snotr \hld\ skyli \alst{m}anna hvęrr, &
\ind ę́va til \alst{s}notr \alst{s}éi; &
\alst{ø}r-lǫg sín \hld\ viti \alst{ę}ngi maðr fyrir; &
\ind þęim es \alst{s}orga-lausastr \alst{s}efi.\eva

\bvb Middle-clever should each man be; \\
never too clever. \\
His own \inx[C]{orlay} ought no man to know ahead; \\
his is the most sorrowless mind.\footnoteB{Who knows not his fate. It is fitting that Weden should say this, having knowledge of the inevitable destruction of the world and hisself.}\evb
\evg


\bvg
\bva\alst{B}randr af \alst{b}randi \hld\ \alst{b}rinnr unds \alst{b}runninn es, &
\ind \alst{f}uni kvęykisk af \alst{f}una; &
\alst{m}aðr af \alst{m}anni \hld\ verðr at \alst{m}áli kuðr; &
\ind en til \alst{d}ǿlskr af \alst{d}ul.\eva

\bvb Fire by fire burns until it is burnt [out]; \\
flame is quickened by flame. \\
Man by man becomes known through speech, \\
but the too hickish from delusion.\evb
\evg


\bvg
\bva\alst{Á}r skal rísa, \hld\ sá’s \alst{a}nnars vill &
\ind \alst{f}é eða \alst{f}jǫr hafa; &
sjaldan \alst{l}iggjandi ulfr \hld\ \alst{l}ę́r of getr, &
\ind né \alst{s}ofandi maðr \alst{s}igr.\eva

\bvb Early shall he rise who another man’s \\
\inx[C]{fee} or life will have. \\
Seldom gets the lying wolf the thigh, \\
nor the sleeping man victory.\evb
\evg


\bvg
\bva\alst{Á}r skal rísa, \hld\ sá’s á \alst{y}rkjęndr fáa, &
\ind ok ganga síns \alst{v}erka á \alst{v}it; &
\alst{m}art of dvęlr \hld\ þann’s umb \alst{m}orgin sefr, &
\ind \alst{h}alfr es auðr und \alst{h}vǫtum.\eva

\bvb Early he shall rise who owns workers few, \\
and go his work to meet. \\
Much is kept back from him who in the morning sleeps; \\
a half wealth is under the brisk.\footnoteB{The brisk man has already obtained a “half wealth” just by putting his work before his comfort (and sleeping in).}\evb
\evg


\bvg
\bva\alst{Þ}urra skíða \hld\ ok \alst{þ}akinna nę́fra, &
\ind þess kann \alst{m}aðr \alst{m}jǫt, &
ok þess \alst{v}iðar, \hld\ es \alst{v}innask męgi &
\ind \alst{m}ál ok \alst{m}issęri.\eva

\bvb Dry planks and of thatching birch bark: \\
of this man knows the measure— \\
and of that firewood which he may use \\
for a season and half-year.\footnoteB{i.e. over the winter.}\evb
\evg


\bvg
\bva\alst{Þ}vęginn ok męttr \hld\ ríði maðr \alst{þ}ingi at, &
\ind þótt sé-t \alst{v}ę́ddr til \alst{v}ęl; &
\alst{sk}úa ok bróka \hld\ \alst{sk}ammisk ęngi maðr &
\ind né \alst{h}ęsts in \alst{h}ęldr. \hld\ (\edtext{þótt hann \alst{h}afi-t góðan}{\lemma{þótt \dots\ góðan ‘although \dots\ good one’}\Bfootnote{As \textcite{FinnurEdda} points out this line is surely a late insert. The inserter was not aware of the rules of the \Ljodahattr\ meter and interpreted the c-verse as an a-verse in \Fornyrdislag.}}).\eva

\bvb Washed and full\footnoteB{A collocation. Cf. \Reginsmal\ TODO: \emph{kęmbðr} ‘combed’ — \emph{þvęginn} ‘washed’ — \emph{męttr} ‘full’; \Voluspa\ 33: \emph{þó} ‘washed’ — \emph{kęmbði} ‘combed’. These examples attest to the importance of personal hygiene in the culture, something further seen by the ubiquity of combs in pre-Christian graves. Cf. also Tacitī Germania 22: \emph{Statim e somno, quem plerumque in diem extrahunt, lavantur, saepius calida, ut apud quos plurimum hiems occupat. Lauti cibum capiunt: separatae singulis sedes et sua cuique mensa. Tum ad negotia nec minus saepe ad convivia procedunt armati.} ‘On waking from sleep, which they generally prolong to a late hour of the day, they take a bath, oftenest of warm water, which suits a country where winter is the longest of the seasons. After their bath they take their meal, each having a separate seat and table of his own. Then they go armed to business, or no less often to their festal meetings.’} ought a man to ride to the Thing, \\
although he be not clothed too well; \\
of his shoes and his breeches ought no man to be ashamed, \\
nor the more of his horse. (although he has not a good one.)\evb
\evg

\sectionline

The two following sts. are written in opposite order in \Regius, but a symbol at the start of each indicates that they should switch places.

\sectionline

\bvg
\bva\alst{S}napir ok gnapir, \hld\ es til \alst{s}ę́var kømr, &
\ind \alst{ǫ}rn á \alst{a}ldinn mar; &
svá es \alst{m}aðr, \hld\ es með \alst{m}ǫrgum kømr &
\ind ok \edtrans{á \alst{f}or-mę́lęndr \alst{f}áa}{has spokesmen few}{\Bfootnote{Shared with st. 25.}}.\eva

\bvb Snaps and stoops—when to the sea it comes— \\
the eagle on the aged ocean. \\
So is the man who among the many comes, \\
and has spokesmen few.\evb
\evg


\bvg
\bva\alst{F}regna ok sęgja \hld\ skal \alst{f}róðra hvęrr, &
\ind sá’s vill \alst{h}ęitinn \alst{h}orskr; &
\alst{ęi}nn vita \hld\ né \alst{a}nnarr skal, &
\ind \alst{þ}jóð vęit ef \alst{þ}rír ’ru.\eva

\bvb Ask and speak shall each learned man \\
who wishes to be called sharp. \\
\emph{One} shall know, but not another; \\
thirty\footnoteB{\emph{þjóð} lit. ‘people, nation’; cf. \Skaldskaparmal\ (TODO): \emph{þjóð eru þrír tigir} ‘thirty are a \emph{people}’.} know if there are three.\evb
\evg


\bvg
\bva\alst{R}íki sitt \hld\ skyli \alst{r}áð-snotra &
\ind hvęrr í \alst{h}ófi \alst{h}afa; &
\edtext{þá þat \alst{f}innr, \hld\ es með \alst{f}rǿknum kømr, &
\ind at \alst{ę}ngi es \alst{ęi}nna hvatastr.}{\lemma{þá \dots\ ęinna hvatastr ‘then \dots briskest of all’}\Bfootnote{Almost identical to \Reginsmal\ TODO/3–4, which however has \emph{flęirum} ‘more men’ for \emph{frǿknum} ‘the bold’.}}\eva

\bvb His own power should each counsel-clever \\
man use in moderation; \\
then he finds it—when among the bold he comes— \\
that none is the briskest of all.\footnoteB{i.e., every man has his match.}\evb
\evg


\bvg
\bva\alst{O}rða þęira, \hld\ es maðr \alst{ǫ}ðrum sęgir, &
\ind opt hann \alst{g}jǫld of \alst{g}etr.\eva

\bvb For those words which man to another says, \\
he oft gets recompense.\evb
\evg


\bvg
\bva \edtrans{\alst{M}ikils til}{Much too}{\Afootnote{written as one word \emph{mikilsti} \Regius}} snimma \hld\ kom’k í \alst{m}arga staði, &
\ind en til \alst{s}íð í \alst{s}uma; &
\alst{ǫ}l vas drukkit, \hld\ sumt vas \alst{ó}·lagat; &
\ind sjaldan hittir \alst{l}ęiðr í \alst{l}ið.\eva

\bvb Much too early I came to many places, \\
but too late to some. \\
Ale was drunk, some was unbrewed; \\
seldom finds the loathed one his place.\evb
\evg


\bvg
\bva\alst{H}ér ok \alst{h}var \hld\ myndi mér \alst{h}ęim of boðit, &
\ind ef þyrpta’k at \alst{m}ǫ́lun-gi \alst{m}at, &
eða \alst{t}vau lę́r hęngi \hld\ at hins \alst{t}ryggva vinar, &
\ind þar’s ek hafða \alst{ęi}tt \alst{e}tit.\eva

\bvb Here and there would I to a home be invited, \\
if at no meal-time I needed food; \\
or [if] two hams should hang at the trusty friend’s [home], \\
where I had eaten one.\footnoteB{Not everyone is hospitable, especially with regards to food, which was scarce and closely watched among subsistence farmers. The speaker notes that even a “trusty friend” (possibly sarcastic) would invite him more often if he could increase the amount of food rather than decrease it.}\evb
\evg


\bvg
\bva\alst{Ę}ldr es batstr \hld\ með \alst{ý}ta sonum &
\ind ok \alst{s}ólar \alst{s}ýn, &
\alst{h}ęilyndi sitt, \hld\ ef maðr \alst{h}afa náir, &
\ind án við \alst{l}ǫst at \alst{l}ifa.\eva

\bvb Fire is best among the sons of men, \\
and the sight of the sun; \\
one’s good health, if he manage to keep it— \\
{[and]} not living by vice.\evb
\evg


\bvg
\bva\alst{E}s-at maðr \alst{a}lls \edtrans{ve-sall}{unblessed}{\Bfootnote{Or ‘woe-blessed’. I have elsewhere translated this word as ‘wretched’, but I have presently rendered it this way to show the etymological relationship. The second element in this word is \emph{sę́ll}, but lacks i-umlaut due to Proto-Norse shortening of the vowel before the umlaut occurred or became phonemic. The ancestral Proto-Norse forms would be \emph{*sāliʀ} and \emph{*wajē-sāliʀ}. Cf. here ᚹᚨᛃᛖ-ᛗᚨᚱᛁᛉ \emph{wajē-mariʀ} ‘infamous’ on the Tjurkö bracteate, where the second element is the ancestor of ON \emph{mę́rr} ‘renowned, famous’. The expected descendant \emph{*ve-marr} is not attested.}\Bfootnote{I have chosen to translate \emph{sę́ll} as ‘blessed’, but it is not a past participle and could also be rendered as ‘lucky’. It carries with it a certain sense of innateness, in a way that modern Westerners may find foreign. So a king whose reign is one of peace (\emph{friðr}) is said to be \emph{frið-sę́ll} ‘blessed with peace’, while one who reigns during good harvests (\emph{ár}) is said to be \emph{ár-sę́ll} ‘blessed with harvests’. The harvests and peace are not due to environmental or political factors outside of his control, but rather spring from the king himself (TODO: Reference PCRN chapter).}}, \hld\ þótt sé \alst{i}lla hęill, &
\ind \alst{s}umr es af \edtext{\alst{s}onum}{\lemma{sonum \dots\ frę́ndum ‘sons \dots\ kinsmen’}\Bfootnote{Cf. st. 72 below, which stresses the importance of sons and kinsmen.}} \alst{s}ę́ll, &
sumr af \alst{f}rę́ndum, \hld\ sumr af \alst{f}é ǿrnu, &
\ind sumr af \alst{v}erkum \alst{v}ęl.\eva

\bvb Man is not all unblessed, though he of poor health be: \\
someone is blessed with sons; \\
someone with kinsmen, someone with ample \inx[C]{fee}, \\
someone with works done well.\evb
\evg


\bvg
\bva Bętra ’s \alst{l}ifðum, \hld\ \edtrans{an séi ó·\alst{l}ifðum}{than with the unliving}{\Bfootnote{emend.; \emph{⁊ ſęl lıfðo}m \Regius. The normalized reading \emph{ok sę́l-lifðum} ‘and for the blessed living’ is metrically defect, since \emph{sę́l-} is strongly stressed and thus should carry alliteration. For the original form of the line we may instead compare \Fafnismal\ 30: \emph{Hvǫtum ’s bętra \hld\ an sé óhvǫtum} ‘For the brisk ’tis better than it may be for the unbrisk’. The corruption is understandable; \emph{*en} (younger form of \emph{an}) ‘than’ was interpreted as \emph{en} ‘and, but’ and copied as \emph{⁊} (the tironian \emph{et}), while \emph{*séı ólıfðo}m (probably with the words cramped together) became \emph{sęl lıfðo}m.}}, &
\ind ęy getr \alst{k}vikr \alst{k}ú; &
\alst{ę}ld sá’k \alst{u}pp brinna \hld\ \alst{au}ðgum manni fyr, &
\ind en úti vas \alst{d}auðr fyr \alst{d}urum.\eva

\bvb ’Tis better for the living than it may be for the unliving: \\
ever gets the quick a cow.\footnoteB{A reference to the cattle-based economy (see also st. 76), the cow being used as a metonym: “new opportunities always present themselves for the living” (cf. churchly English ‘the \emph{quick} and the dead’, i.e. ‘the \emph{living} and the dead’).} \\
A fire I saw burning high for a wealthy man, \\
but outside he was dead before the doors.\footnoteB{The fire is probably the man’s funeral pyre. It is notable that his wealth is mentioned; according to Ibn Fadlan (TODO) two thirds of a great chieftain’s wealth was spent on his funeral. One notes the contrastive \emph{en} ‘but’, and may paraphrase it as something like “I saw a lavish funeral, \emph{but} the burning man was dead \emph{anyway}.” This interpretation is supported by the following st. (\Havamal\ 70, especially the second half), which expresses the same sentiment.”}\evb
\evg


\bvg
\bva\alst{H}altr ríðr \alst{h}rossi, \hld\ \alst{h}jǫrð rekr \alst{h}andar vanr, &
\ind \alst{d}aufr vegr ok \alst{d}ugir; &
\alst{b}lindr es \alst{b}ętri, \hld\ an \alst{b}ręndr séi; &
\ind \alst{n}ýtr mann-gi \alst{n}ás.\eva

\bvb A halt man rides a horse; a handless drives a herd; \\
a deaf fights and avails. \\
Blind is better than be burnt; \\
no man has use for a corpse.\evb
\evg


\bvg
\bva \edtrans{\alst{S}onr es bętri}{A son is better}{\Bfootnote{i.e. it is better for a man to have a son and heir than not, even if the father should die some time before he is born. The son can further his father’s lineage and memory (as exemplified by the raising of a “beat-stone”), and as the poet says, it is rare for a non-relative to do so.}}, \hld\ þótt sé \alst{s}íð of alinn &
\ind ęptir \alst{g}inginn \alst{g}uma; &
sjaldan \edtrans{\alst{b}autar-stęinar}{beat-stones}{\Bfootnote{Large standing stones raised in memory of someone.  Numerous such stones with runic inscriptions are known from migration period Norway, often near grave fields.  Some hold only single personal names or short phrases, like the stone from Sunde in Sunnfjord, western Norway (signum \emph{KJ 90}): ᚹᛁᛞᚢᚷᚨᛊᛏᛁᛉ \textbf{widugastiʀ} ‘Woodguest’, or the one from Bø in Rogaland, southwestern Norway (signum \emph{KJ 78}): ᚺᚾᚨᛒᛞᚨᛊ ᚺᛚᚨᛁᚹᚨ \textbf{hnabdas hlaiwa} ‘Naved’s grave’.  Others hold longer inscriptions, like the one from Kjølevik in Rogaland (signum \emph{KJ 75}): ᚺᚨᛞᚢᛚᚨᛁᚲᚨᛉ ᛖᚲᚺᚨᚷᚢᛊᛏᚨᛞᚨᛉ ᚺᛚᚨᚨᛁᚹᛁᛞᛟᛗᚨᚷᚢᛗᛁᚾᛁᚾᛟ \textbf{hadulaikaz ekhagustadaz hlaaiwidomaguminino} ‘Hathlac [lies here].  I, Haystald, buried my lad.’}} \hld\ standa \alst{b}rautu nę́r, &
\ind nema ręisi \alst{n}iðr at \alst{n}ið.\eva

\bvb A son is better, though he late be born \\
after a passed-on man; \\
seldom beat-stones near the highway stand, \\
save by kinsman for kinsman raised.\evb
\evg


\bvg
\bva \edtext{\alst{T}vęir ’ru ęins hęrjar, \hld\ \edtrans{\alst{t}unga ’s hǫfuðs bani}{the tongue is the head’s bane}{\Bfootnote{Formulaic or proverbial. Cf. the Old Swedish Heathen Law (my norm. following \textcite{Läffler1879}): \emph{Fallr þann orð havr givit—glǿpr orða vęrstr, tunga hovuð-bani—liggi i ú·gildum akri} ‘If he falls who has given the word (of insult)—wickedness is the worst of words, the tongue the head-bane-man—may he lie in an invalid (i.e. not properly enclosed) field.’}}; &
mér ’s í \alst{h}eðin \alst{h}vęrn \hld\ \alst{h}andar vę́ni.}{\lemma{Tvęir \dots\ vę́ni}\Bfootnote{The whole st. is undoubtedly a later insert as seen from the divergent meter and style.}}\eva

\bvb Two are of one host;\footnoteB{\emph{hęrjar} gen. sg. of \emph{hęrr} ‘host, army’ may alternatively be read as the nom. pl. meaning ‘harriers, raiders,’ present in \emph{ęinhęrjar} (\inx[G]{Ownharriers}). Thus ‘two are the destroyers of one (i.e. the person)’.} the tongue is the head’s bane;\footnoteB{The tongue and the head are part of the same body and need each other, yet the former often leads to the demise of the latter.} \\
in every cloak I expect a hand.\evb
\evg


\bvg
\bva\alst{N}ǫ́tt verðr fęginn, \hld\ sá’s \alst{n}esti trúir, &
\ind \alst{sk}ammar ’ru \alst{sk}ips ráar, &
\ind \alst{h}verf es \alst{h}aust-gríma; &
\alst{f}jǫlð of viðrir \hld\ á \edtrans{\alst{f}imm dǫgum}{five days}{\Bfootnote{i.e. ‘in a week’, see note to st. 51 and Encyclopedia: five days.}}, &
\ind en \alst{m}ęir á \alst{m}ánaði.\eva

\bvb At night rejoices he who trusts in his provisions; \\
short are the ship’s sailyards;\footnoteB{TODO: Write about the varying interpretations (Finnur, Cleasby, Skp) of this line.} \\
ever-shifting is the autumn night. \\
The weather shifts much in \inx[C]{five days}, \\
but more in a month.\evb
\evg


\bvg
\bva\alst{V}ęit-a hinn, \hld\ es \alst{v}ę́tki \alst{v}ęit, &
\ind margr verðr \edtrans{af \alst{au}rum}{by treasures}{\Afootnote{emend. from \emph{†aflꜹðrom†} \Regius}} \alst{a}pi; &
maðr es \alst{au}ðigr, \hld\ annarr \alst{ó}·auðigr, &
\ind skyli-t þann \alst{v}ítka \alst{v}áar.\eva

\bvb The one knows not, who nothing knows: \\
many a man becomes by treasures an \inx[C]{ape}. \\
A man is wealthy, another not wealthy; \\
one oughtn’t to curse him for his woe.\evb
\evg


\bvg
\bva\alst{D}ęyr \edtext{fé}{\lemma{fé \dots\ frę́ndr ‘Fee \dots\ kinsmen’}\Bfootnote{The import of this merism may be less clear to the modern reader. In the Germanic Iron Age farming society a man’s wealth was reckoned by how many heads of cattle (and the Norman loan-word \emph{cattle} is itself the same word as \emph{capital}) he owned (cf. st. 70 above, where “a cow” is used to express “an opportunity”), and his social power by the number of able male relatives ready to side with him in conflict (cf. st. 72 above and TODO: reference?). The meaning is thus: all your power will pass away, and so too must you, but if you leave a good reputation behind it can live on. For Indo-European poetic analogues, see \textcite[99\psqq]{West2007}.}}, \hld\ \alst{d}ęyja frę́ndr, &
\ind dęyr \alst{s}jalfr hit \alst{s}ama; &
en \alst{o}rðs-tírr \hld\ dęyr \alst{a}ldri-gi &
\ind hvęim’s sér \alst{g}óðan \alst{g}etr.\eva

\bvb \inx[C]{fee}[Fee] dies, kinsmen die, \\
oneself dies likewise; \\
but a word-glory never dies, \\
for whomever gets himself a good one.\evb
\evg


\bvg
\bva\alst{D}ęyr fé, \hld\ \alst{d}ęyja frę́ndr, &
\ind dęyr \alst{s}jalfr hit \alst{s}ama; &
\alst{e}k vęit \alst{ęi}nn \hld\ at \alst{a}ldri-gi dęyr: &
\ind \alst{d}ómr of \alst{d}auðan hvęrn.\eva

\bvb Fee dies, kinsmen die, \\
oneself dies likewise. \\
I know one that never dies: \\
the \inx[C]{Doom} o’er each man dead.\evb
\evg

\sectionline

It is likely that the original \emph{Gęsta-þáttr} ended here. The three following stanzas, especially the third, are poorly placed and seem like later inserts.

\sectionline

\bvg
\bva\alst{F}ullar grindr \hld\ sá’k fyr \alst{F}itjungs sonum, &
\ind nú bera þęir \alst{v}ánar \alst{v}ǫl; &
svá es \alst{au}ðr \hld\ sęm \alst{au}ga-bragð, &
\ind hann es \alst{v}altastr \alst{v}ina.\eva

\bvb Full pens I saw for the sons of Fitting; \\
now they carry the staff of hope.\footnoteB{A beggar’s staff.} \\
So is wealth like the twinkling of an eye: \\
it is the ficklest of friends.\evb
\evg


\bvg
\bva\alst{Ó}·snotr maðr \hld\ es \alst{ęi}gnask getr &
\ind \alst{f}é eða \alst{f}ljóðs mun-úð; &
\alst{m}etnaðr hǫ́num þróask, \hld\ en \alst{m}an-vit aldri-gi; &
\ind framm gęngr hann \alst{d}rjúgt í \alst{d}ul.\eva

\bvb The unclever man who comes to own \\
fee or a girl’s grace: \\
the pride in him flourishes, but never his manwit; \\
he goes forth far into delusion.\evb
\evg


\bvg
\bva Þat ’s þá \alst{r}ęynt, es þú at \edtext{\alst{r}únum spyrr, \hld\ hinum \alst{r}ęgin-kunnum}{\lemma{rúnum \dots\ ręgin-kunnum ‘runes \dots\ born of the Reins’}\Bfootnote{This expression also appears on the C4th–6th Noleby stone (in the acc. sg. \emph{rúnó ragina-kundó} ‘a rune born of the Reins’), which proves that the Eddic rune-magic is (at least in part) founded in oral tradition going back to the Heathen age. See also Encyclopedia \inx[C]{rune}.}}, &
\ind \edtext{þęim’s \alst{g}ørðu \alst{g}inn-ręgin &
\ind ok \alst{f}áði \alst{F}imbul-þulr;}{\lemma{þęim’s \dots\ Fimbul-þulr ‘those which \dots\ Fimble-Thyle’}\Bfootnote{Formulaic. Cf. st. 142 where these two lines occur almost identically, but in reverse order.}} &
\ind (\alst{þ}á hęfr hann batst, ef hann \alst{þ}ęgir.)\eva

\bvb That is then proven, which from the runes thou learnest, [from] the ones born of the Reins, \\
{[from]} those which the \inx[G]{yin-Reins} made, \\
and the Fimble-Thyle \name{= Weden} painted. \\
(Then he has it best, if he shuts up.)\footnoteB{This stanza, which deals with runic magic, and shares expressions with sts. in the Rune-Tally section (beginning with st. 138 below), hardly fits in its current placing. The last line with its shift in person is likely to be a later insert.}\evb
\evg

\sectionline

\section{Scattered stanzas of practical advice.}

These sts. are rather different, both in terms of meter and style.

\bvg
\bva At \alst{k}veldi skal dag lęyfa, \hld\ \alst{k}onu es bręnnd es, &
\alst{m}ę́ki es ręyndr es, \hld\ \alst{m}ęy es gefin es, &
\alst{í}s es \alst{y}fir kømr, \hld\ \alst{ǫ}l es drukkit es.\eva

\bvb At evening shall one praise day, a woman when she is burned, \\
a sword when it is tried, a maiden when she is given,\footnoteB{i.e. in marriage.} \\
ice when one crosses over, ale when it is drunk.\evb
\evg


\bvg
\bva Í \alst{v}indi skal \alst{v}ið hǫggva, \hld\ \alst{v}eðri á sę́ róa, &
\alst{m}yrkri við \alst{m}an spjalla— \hld\ \alst{m}ǫrg eru dags augu— &
á \alst{sk}ip skal \alst{sk}riðar orka, \hld\ en á \alst{sk}jǫld til hlífar, &
\alst{m}ę́ki til hǫggs, \hld\ en \alst{m}ęy til kossa.\eva

\bvb In wind shall one cut wood, in weather row at sea, \\
in darkness speak with a maiden—many are the eyes of day. \\
A ship shall one have for speed, and a shield for protection; \\
a sword for striking, and a maiden for kisses.\evb
\evg


\bvg
\bva Við \alst{ę}ld skal \alst{ǫ}l drekka, \hld\ en á \alst{í}si skríða, &
\alst{m}agran \alst{m}ar kaupa, \hld\ en \alst{m}ę́ki saurgan, &
\alst{h}ęima \alst{h}ęst fęita, \hld\ en \alst{h}und á búi.\eva

\bvb By fire shall one drink ale, and skate on ice; \\
buy a meager stallion, and a rusty sword; \\
at home fatten the horse, and the hound in its dwelling.\evb
\evg


\bvg
\bva\alst{M}ęyjar orðum \hld\ skyli \alst{m}ann-gi trúa, &
\ind né því’s \alst{k}veðr \alst{k}ona; &
\edtext{\edtext{því-at}{\Afootnote{om. \FostrbroedhraSaga}} á \alst{h}verfanda \alst{h}véli \hld\ \edtext{vǫ́ru}{\Afootnote{\emph{er} \FostrbroedhraSaga}} þęim \edtrans{\alst{h}jǫrtu skǫpuð}{hearts shaped}{\Afootnote{\emph{hjarta skapat} ‘heart shaped’ \FostrbroedhraSaga}}, &
\ind \edtext{\alst{b}rigð}{\Afootnote{ok brigð \FostrbroedhraSaga}} í \alst{b}rjóst of \edtext{lagit}{\Afootnote{\emph{laginn} \FostrbroedhraSaga}}.}{\lemma{þvít \dots\ lagið}\Bfootnote{Quoted in slightly divergent form in \FostrbroedhraSaga\ (Thott 1768 4°\textsuperscript{x}, fol. 210r) introduced with the words: \emph{Kom honum þá í hug kviðlingr sá, er kveðinn hafði verit um lausungar-konur:} ‘And then he remembered the ditty which had been composed about loose women:’}}\eva

\bvb A maiden’s words should no man trust, \\
nor that which a woman speaks. \\
For on a spinning wheel were their hearts shaped; \\
fickleness in their breasts was laid.\evb
\evg


\bvg
\bva\alst{B}restanda \alst{b}oga, \hld\ \alst{b}rinnanda loga, &
\alst{g}ínanda ulfi, \hld\ \alst{g}alandi krǫ́ku, &
\alst{r}ýtanda svíni, \hld\ \alst{r}ót-lausum viði, &
\alst{v}axanda \alst{v}ági, \hld\ \alst{v}ellanda katli,\eva

\bvb In the bursting bow, in the burning flame, \\
in the yawning wolf, in the crowing crow, \\
in the roaring swine, in the rootless tree, \\
in the waxing wave, in the swelling kettle,\evb
\evg


\bvg
\bva\alst{f}ljúganda \alst{f}lęini, \hld\ \alst{f}allandi bǫ́ru, &
\alst{í}si \alst{ęi}n-nę́ttum, \hld\ \alst{o}rmi hring-lęgnum, &
\alst{b}rúðar \alst{b}ęð-mǫ́lum \hld\ eða \alst{b}rotnu sverði, &
\alst{b}jarnar lęiki \hld\ eða \alst{b}arni konungs, &
\alst{s}júkum kalfi, \hld\ \alst{s}jalf-ráða þrę́li, &
\alst{v}ǫlu \alst{v}il-mę́li, \hld\ \alst{v}al ný-fęldum.\eva

\bvb in the flying spear, in the falling billow, \\
in one-night old ice, in the coiled-up serpent, \\
in the bed-speeches of a bride or in the broken sword, \\
in the play of a bear or in the child of a king, \\
in the sick calf, in the self-ruling thrall, \\
in the pleasing speech of a wallow, in newly felled corpses,\evb
\evg

\sectionline

In \Regius\ the following two sts. come in the opposite order, but it is clear from its \Malahattr\ meter and the dative case of the words that 88 should follow 86.  On the other hand st. 87, with its \Ljodahattr\ meter and self-enclosed form seems a separate composition, and was probably inserted after 86 due to its first line (\emph{akri ár-sǫ́num}), which is also in the dative.

\sectionline

\bvg
\bva[88]\alst{b}róður-\alst{b}ana sínum \hld\ þótt á \alst{b}rautu mǿti, &
\alst{h}úsi \alst{h}alf-brunnu, \hld\ \alst{h}ęsti al-skjótum, &
þá ’s \alst{jó}r \alst{ó}·nýtr, \hld\ ef \alst{ęi}nn fótr brotnar; &
verðr-it maðr svá \alst{t}ryggr \hld\ at þessu \alst{t}rúi ǫllu!\eva

\bvb in his brother’s bane-man—though on the highway they meet— \\
in the half-burned house, in the all-fleet horse: \\
then is the steed useless, if one foot breaks.— \\
There will be no man so trusting, that he trust in all this!\evb
\evg\stepcounter{stanza}


\bvg
\bva[87]\alst{A}kri \alst{á}r-sǫ́num \hld\ trúi \alst{ę}ngi maðr, &
\ind né til \alst{s}nimma \alst{s}yni; &
\alst{v}eðr rę́ðr akri, \hld\ en \alst{v}it syni; &
\ind \alst{h}ę́tt es þęira \alst{h}várt.\eva

\bvb In an early sown field ought no man to trust, \\
nor too soon in a son. \\
The weather rules the field, and the wits the son; \\
there is risk to them both.\evb
\evg\stepcounter{stanza}


\bvg
\bva Svá ’s \alst{f}riðr kvinna \hld\ þęira’s \alst{f}látt hyggja, &
sęm \alst{a}ki \alst{jó} ó·bryddum \hld\ á \alst{í}si hǫ́lum &
\alst{t}ęitum, \alst{t}vé-vetrum \hld\ ok sé \alst{t}amr illa, &
eða í \alst{b}yr óðum \hld\ \alst{b}ęiti stjórn-lausu, &
eða skyli \alst{h}altr \alst{h}ęnda \hld\ \alst{h}ręin í þá-fjalli.\eva

\bvb So is the love of women—those who falsely think— \\
like one rode an unshod horse on slippery ice: \\
a merry one, two winters old, and badly tamed— \\
or in mad wind tacked a rudderless [ship], \\
or [as if] a halt man should catch a reindeer on a thawing mountain.\evb
\evg

\sectionline

\section{Weden’s failed seduction of Billing’s daughter.}

The following sts. are united by their meter, \Ljodahattr\ (unlike most of the preceding sts., see introduction to them above) and by their logical progression, beginning with general maxims about love and relations between the sexes, before moving into the narrative about Billing’s daughter. The narrator is securely identified as Weden in st. 97.

\sectionline

\bvg
\bva\alst{B}ęrt nú mę́li’k, \hld\ því-at \edtrans{\alst{b}ę́ði}{both}{\Bfootnote{i.e. “both sides, both sexes”. The poet, a man, declares that he is not setting out to unfairly attack women; he is also aware of the faults of his own sex.}} vęit’k, &
\ind brigðr es \alst{k}arla hugr \alst{k}onum, &
\edtext{þá \alst{f}ęgrst mę́lum, \hld\ es \alst{f}lást hyggjum}{\lemma{fęgrst mę́lum \dots\ flást hyggjum ‘most fairly speak \dots\ most falsely we think’}\Bfootnote{Formulaic. Cf. st. 45.}}; &
\ind þat tę́lir \alst{h}orska \alst{h}ugi.\eva

\bvb Plainly I now speak, for I know both: \\
fickle is men’s thought towards women. \\
We then most fairly speak, when most falsely we think; \\
that entices sharp minds.\evb
\evg


\bvg
\bva \edtrans{\alst{F}agrt skal mę́la}{Fairly shall speak}{\Bfootnote{Formulaic. Cf. st. 45.}} \hld\ ok \alst{f}é bjóða, &
\ind sá’s vill \alst{f}ljóðs ǫ́st \alst{f}áa, &
\alst{l}íki \alst{l}ęyfa \hld\ hins \alst{l}jósa mans, &
\ind \edtrans{sá \alst{f}ę́r, es \alst{f}ríar}{he gets, who woos}{i.e., ‘he who woos her gets her’}.\eva

\bvb Fairly shall speak, and offer \inx[C]{fee}, \\
he who will earn a girl’s love; \\
{[he shall]} praise the body of the bright girl; \\
he gets, who woos.\evb
\evg


\bvg
\bva\alst{Á}star firna \hld\ skyli \alst{ę}ngi maðr &
\ind \alst{a}nnan \alst{a}ldri-gi; &
opt fáa á \alst{h}orskan, \hld\ es á \alst{h}ęimskan né fáa, &
\ind \alst{l}ost-fagrir \alst{l}itir.\eva

\bvb For [his] love should no man \\
ever blame another; \\
oft they seize the sharp, when they seize not the foolish, \\
lust-fair looks.\footnoteB{Looks so fair that they cause great lust.}\evb
\evg


\bvg
\bva\alst{Ęy}-vitar firna, \hld\ es maðr \alst{a}nnan skal, &
\ind þess es of margan \alst{g}ęngr \alst{g}uma; &
\alst{h}ęimska ór \alst{h}orskum \hld\ gęrir \alst{h}ǫlða sonu &
\ind sá hinn \alst{m}átki \alst{m}unr.\eva

\bvb For nothing shall man ever blame another, \\
which happens to many a man; \\
from sharp to foolish are the sons of men made \\
by the mighty love.\evb
\evg


\bvg
\bva\alst{H}ugr ęinn þat vęit, \hld\ es býr \alst{h}jarta nę́r, &
\ind ęinn es hann \alst{s}ér of \alst{s}efa; &
øng es \alst{s}ótt verri \hld\ hvęim \alst{s}notrum manni &
\ind an sér \alst{ø}ngu at \alst{u}na.\eva

\bvb The mind alone knows what lives close to the heart, \\
it is alone with its thoughts. \\
No sickness is worse for any clever man \\
than to with nothing be content.\evb
\evg


\bvg
\bva Þat þá \alst{r}ęynda’k, \hld\ es í \alst{r}ęyri sat’k, &
\ind ok vę́tta’k \alst{m}íns \alst{m}unar, &
\alst{h}old ok \alst{h}jarta \hld\ vas mér hin \alst{h}orska mę́r, &
\ind þęygi hana at \alst{h}ęldr \alst{h}ęf’k.\eva

\bvb That I then discovered, as I sat in the reed, \\
and awaited my pleasure. \\
My flesh and heart was that sharp maiden; \\
I hold her none the more.\evb
\evg


\bvg
\bva\alst{B}illings \edtrans{męy}{maiden}{\Bfootnote{i.e. ‘unmarried (virgin) daughter’.}} \hld\ ek fann \alst{b}ęðjum á &
\ind \alst{s}ól-hvíta \alst{s}ofa; &
\alst{ja}rls \alst{y}nði \hld\ þótti mér \alst{ę}kki vesa &
\ind nema við þat \alst{l}ík at \alst{l}ifa.\eva

\bvb Billing’s maiden I found on the beds, \\
sun-white, sleeping. \\
An earl’s pleasure seemed me naught to be, \\
except living alongside that body.\evb
\evg


\bvg
\bva „\alst{Au}k nę́r \alst{a}ptni \hld\ skalt \alst{Ó}ðinn koma, &
\ind ef vilt þér \alst{m}ę́la \alst{m}an, &
\alst{a}lt eru \alst{ó}·skǫp, \hld\ nema \alst{ęi}n vitim &
\ind \alst{s}likan lǫst \alst{s}aman.“\eva

\bvb {[Billing’s daughter:]} \\
“And by evening shalt thou, Weden, come, \\
if thou wilt get for thee the girl [me]; \\
all is misshapen, if we may not know, \\
alone, such a vice together.”\evb
\evg


\bvg
\bva\alst{A}ptr ek hvarf \hld\ ok \alst{u}nna þóttumk &
\ind \edtrans{\alst{v}ísum \alst{v}ilja frá}{away from my wise will}{\Bfootnote{i.e., “against my better judgment”; the wise choice would have been to walk away.}}; &
\alst{h}itt ek \alst{h}ugða, \hld\ at \alst{h}afa mynda’k &
\ind \alst{g}ęð hęnnar allt ok \alst{g}aman.\eva

\bvb Back I turned—and thought myself in love— \\
away from my wise will; \\
this I thought, that I would have \\
her senses all, and pleasure.\evb
\evg


\bvg
\bva Svá kom’k \alst{n}ę́st, \hld\ at hin \edtrans{\alst{n}ýta}{useful}{\Bfootnote{Sarcastic. Billing’s daughter had apparently summoned a lynch mob.}} vas &
\ind \alst{v}íg-drótt ǫll of \alst{v}akin; &
með \alst{b}rinnǫndum ljósum \hld\ ok \edtrans{\alst{b}ornum viði}{carried sticks}{\Bfootnote{lit. ‘carried wood’; the mob was armed with clubs.}}, &
\ind svá vas mér \edtrans{\alst{v}íl-stígr}{sad path}{\Bfootnote{Ambiguous, referring either to the beating he would have received at the hands of the mob, or to his walk of shame away from the hall.  The latter is perhaps more likely.}} of \alst{v}itaðr.\eva

\bvb So I came next, as was the useful \\
war-troop all awake; \\
with burning lights and with carried sticks; \\
so was for me a sad path marked out.\evb
\evg


\bvg
\bva \edtrans{\alst{Au}k nę́r morni}{And by morning}{\Bfootnote{Mirroring the beginning of st. 97 above.}}, \hld\ es vas’k \alst{ę}nn of kominn, &
\ind þá vas \alst{s}al-drótt of \alst{s}ofin; &
\edtrans{\alst{g}ręy ęitt}{a lone bitch}{\Bfootnote{The insult is easily understood: Weden is being asked to make love to the dog, “this is all you get!”}} þá fann’k \hld\ hinnar \edtrans{\alst{g}óðu}{good}{\Bfootnote{Possibly not sarcastic, but rather referring to her chastity.}} konu &
\ind \alst{b}undit \alst{b}ęðjum á.\eva

\bvb And by morning when I had come again, \\
then was the hall-troop asleep. \\
A lone bitch I then found, by the good woman \\
bound on the bed.\evb
\evg


\bvg
\bva Mǫrg es \alst{g}óð mę́r, \hld\ ef \alst{g}ǫrva kannar, &
\ind \alst{h}ug-brigð við \alst{h}ali; &
þá þat \alst{r}ęynda’k, \hld\ es hit \alst{r}áð-spaka &
\ind tęygða’k á \alst{f}lę́rðir \alst{f}ljóð; &
\alst{h}ǫ́ðungar \alst{h}vęrrar \hld\ lęitaði mér hit \alst{h}orska man &
\ind ok hafða’k þess \alst{v}ę́t-ki \alst{v}ífs.\eva

\bvb Many a good maiden—if one comes to know her well— \\
is heart-fickle towards men; \\
then I found that out, as into sins I lured \\
the counsel-clever maid: \\
all sorts of disgraces that sharp girl sought out for me, \\
and I had naught of that woman.\evb
\evg

\sectionline

\section{Weden’s obtaining of the Mead of Poetry}

The intricate myth of how Weden came to own the Mead of Poetry is told more fully in \Skaldskaparmal\ 5–6. That narrative goes as follows, with minor details left out:
After the war between the Eese and Wanes, the two tribes of gods reconcile through spitting into a vat. Not wanting to discard this token of their truce, they instead create a man out of the spit, calling him \inx[P]{Quasher}; he is so wise that he can answer any question posed to him, and so travels around the world in order to share his wisdom with humans.
Quasher eventually comes to the dwelling of two dwarfs, Fealer and Galer. They kill him and drain his blood into three vessels: two vats named Soon and Bothem, and a kettle named \inx[P]{Woderearer}. Through mixing the blood with honey they make a mead, with the power to turn anyone who drinks from it “a scold or man of learning (\emph{skald eða frǿða-maðr})”. The dwarfs then lie to the Eese about the murder, telling them that Quasher drowned in his own wisdom.
Some time later, the dwarfs murder an ettin named \inx[P]{Gilling} and his wife. Gilling’s son, \inx[P]{Sutting}, learns of this and prepares to drown the dwarfs. In exchange for their lives and as recompense for his father’s slaying, the dwarfs offer Sutting the “dear mead” (\emph{mjǫðinn dýra}; cf. here sts. 104 and 138). Sutting accepts the ransom and takes the mead home with him. He makes his daughter \inx[P]{Guthlathe} guard it.
Some time later, Weden is out journeying, and finds nine thralls mowing hay. He sharpens their scythes with a special whetstone, and the mowing improves greatly. He then throws it in the air and the thralls shortly kill each other over it. By evening Weden comes to the owner of the thralls, Bigh, Sutting’s brother. Bigh laments the death of his workmen, and so Weden, who calls himself \inx[P]{Baleworker}, offers to do the work of the thralls over the summer, in exchange for one drink of Sutting’s mead. Bigh tells him that Sutting alone owns the mead, but that he will accompany Baleworker to Sutting to ask for the drink.
The two arrive at Sutting, who as expected refuses to give any part of the mead away. Baleworker then tells Bigh that he will get to it anyway; he takes out the drill \inx[P]{Rate}, and tells Bigh to drill through the mountain, into the room where the mead is stored. Bigh first attempts to trick him by only drilling halfway, but eventually creates a narrow passage. Baleworker turns himself into a snake and crawls through it; as he does, Bigh tries to strike him the drill, but misses.
After coming through, Baleworker sees Guthlathe watching over the mead. He goes on to sleep with her for three nights, after which she promises him three sips of the mead. With each sip he swallows the contents of one of the three vessels, so that all of the mead ends up in his belly.
Having taken the mead, he dons his eagle-hame and flies away from the mountain. Sutting sees him, takes his own eagle-hame, and gives chase. The Eese see Weden in flight, and set out several large vat on the ground, into which Weden, still flying, spits out the mead. At this point Sutting has almost caught up with him, and so Weden “sends back” (\emph{sęnda aptr}, usually interpreted being sent out from the anus) some of the mead, presumably into his face. This portion becomes the lot of foolish poets (\emph{skald-fífla hlutr}), while the rest of the mead is given to the Eese and to skilled poets (\emph{þęim mǫnnum, er yrkja kunnu} ‘those men who can compose [poetry]’).

The core of this many-twisted myth is old. A close parallel is found in \Rigveda\ hymns 4.26–27. In these two hymns the \emph{soma} plant (who in the Vedic mythology is not just the plant and its resulting drink, but also a god, perhaps somewhat like Quasher) is first held within “a hundred iron forts” (4.27.1c: \emph{śatám púraḥ ā́yasīḥ}) by the archer \emph{Kr̥şānu}, before being stolen by a sweeping falcon. The falcon brings \emph{Soma} to \emph{Manu}, the ancestor of the Aryans and first sacrificer.

The resemblance to the last part of the \Skaldskaparmal\ account should be obvious, but, notably, the detail of the falcon is not found in any of the sts. below. This shows that the narrative of \Skaldskaparmal\ cannot be exclusively based on the sts. here below, but instead also relies on other, now-lost sources. This is also supported by the present sts. leaving out the narratives about Quasher, the two dwarfs, and Baye, along with some subtler narrative differences.

The order of the present sts. follows that of \Regius, their main witness manuscript. The strand begins with some social advice (102), after which the narrative follows (103–109). It is narrated in the first person by Weden himself. The sts. do not tell the myth in chronological order and leave much up to the listener; they are surely composed for an audience that already knows the story. The following narrative details are given:

\begin{enumerate}
	\setcounter{enumi}{103}
	\item Weden visits Sutting’s home, but does not receive a good reception.
	\item Guthlate falls in love with Weden, and gives him a drink of the Mead.
	\item Weden has to bore through the mountains with the drill Rate.
	\item Weden has “bought [the Mead] well”; possibly a euphemistic reference to sleeping with Guthlathe for it.
	\item Guthlathe indeed does sleep with Weden, though not expressely in exchange for the Mead.
	\item The following day (\emph{hins hindra dags}, see note to this word in the edited text below), a group of Rime-Thurses come to Weden’s hall, to ask him whether a Baleworker is among the Gods, or if he has been slain by Sutting.
	\item Switching to the third person (which may indicate that this is his answer to the Rime-Thurses), Weden says that he “thinks” that Weden has sworn an oath, but that his words cannot be trusted. After the “simble” (i.e. drinking feast, banquet; probably referring to the drink of the Mead), Weden betrayed Sutting and made Guthlathe weep.
\end{enumerate}

The underlying narrative seems to generally agree with that of \Skaldskaparmal, but unlike its more transactional affair, we here find a stronger emphasis on Weden’s cruel betrayal of Guthlathe. A notable detail not found in \Skaldskaparmal\ is Weden’s oath in st. 109. The content of the oath was most likely that Weden would marry Guthlathe, something supported by the language used (see note to st. 108: \emph{hins hindra dags}). The recipient of the oath, which Weden clearly broke, was either Sutting or Guthlathe. That Weden swore it to Sutting, and thus asked him for Guthlathe’s hand in marriage, may be suggested by the description of Sutting as \emph{svikvinn} ‘betrayed’ in st. 109. This view, however, has an internal narrative problem: in st. 103 Weden describes his interaction with Sutting as poor, and in st. 105 Weden is said to have had to bore through the mountains, but this may just have been to reach Sutting, rather than Guthlathe as in \Skaldskaparmal.
The recipient of the oath being Guthlathe would agree better with the \Skaldskaparmal\ narrative, and Sutting’s betrayer would instead be her.

\sectionline

\bvg
\bva Hęima \alst{g}laðr \alst{g}umi \hld\ ok við \alst{g}ęsti ręifr, &
\ind \alst{s}viðr skal of \alst{s}ik vesa; &
\alst{m}innigr ok \alst{m}ǫ́lugr, \hld\ ef vill \alst{m}arg-fróðr vesa; &
\ind opt skal \alst{g}óðs \alst{g}eta; &
\alst{f}imbul-\alst{f}ambi hęitir, \hld\ sá’s \alst{f}átt kann sęgja; &
\ind þat es \alst{ó}·snotrs \alst{a}ðal.\eva

\bvb At home shall man be glad and giving with the guest, \\
wise about himself; \\
{[he shall be]} of good memory and speech, if he wishes to be many-learned; \\
oft shall he speak of good. \\
A fimble-fool is he called who little can say; \\
that is an unclever man’s nature.\evb
\evg


\bvg
\bva Hinn \alst{a}ldna \alst{jǫ}tun sótta’k, \hld\ nú em’k \alst{a}ptr of kominn; &
\ind fátt gat’k \alst{þ}ęgjandi \alst{þ}ar; &
\alst{m}ǫrgum orðum \hld\ \alst{m}ę́lta’k í minn frama &
\ind í \alst{S}uttungs \alst{s}ǫlum.\eva

\bvb The old ettin \name{= Sutting} I sought, now am I come back; \\
I got little audience there. \\
Many words I spoke to my furtherance, \\
in the halls of Sutting.\evb
\evg


\bvg
\bva\alst{G}unn-lǫð mér of \alst{g}af \hld\ \alst{g}ullnum stóli á &
\ind \alst{d}rykk hins \alst{d}ýra mjaðar; &
\alst{i}ll \alst{i}ð-gjǫld \hld\ lét’k hana \alst{ę}ptir hafa &
\ind síns hins \alst{h}ęila \alst{h}ugar, &
\ind síns hins \alst{s}vára \alst{s}efa.\eva

\bvb \inx[P]{Guthlathe} did give me, on the golden throne, \\
a drink of the dear mead; \\
evil recompense I let her have afterwards, \\
for her whole heart, \\
for her severe affection.\evb
\evg


\bvg
\bva\alst{R}ata munn \hld\ létumk \alst{r}úms of fáa &
\ind ok of \alst{g}rjót \alst{g}naga; &
\alst{y}fir ok \alst{u}ndir \hld\ stóðumk \alst{jǫ}tna vegir, &
\ind svá \alst{h}ę́tta’k \alst{h}ǫfði til.\eva

\bvb Rate’s mouth I made to bring me room, \\
and gnaw away at the rocks. \\
Over and under me stood the roads of the ettins \ken{mountains}; \\
so I risked my head.\evb
\evg


\bvg
\bva \edtext{\alst{V}ęl kęypts hlutar \hld\ hęf’k \alst{v}ęl notit; &
\ind \alst{f}ás es \alst{f}róðum vant; &
því-at \edtrans{\alst{Ó}ð-rǿrir}{Woderearer}{\Bfootnote{One of the vessels in with the Mead of Poetry was held (see introduction to the present section above), here standing in for all the Mead.}} \hld\ es nú \alst{u}pp kominn &
\ind á \alst{a}lda vés \edtrans{\alst{ja}ðar}{rim}{\Bfootnote{metr. emend.; \emph{jarðar} \Regius\ has a long root-syllable, and does not fit grammatically.}}.}{\lemma{Vęl \dots\ jaðar}\Bfootnote{Taken on its own this st. would be somewhat difficult, but in context the import is clear: Weden says that He has made good use of the Mead of Poetry by bringing it to earth, making poetry (and surely likewise other intellectual disciplines) available to men.}}\eva

\bvb The well bought thing \ken*{Mead of Poetry} have I used well— \\
little is lacking for the learned, \\
for Woderearer is now come up \\
over the rim of the \inx[C]{wigh} of men \ken*{= Middenyard}.\evb
\evg


\bvg
\bva\alst{I}fi ’s mér \alst{á}, \hld\ at vę́ra’k \alst{ę}nn kominn &
\ind \alst{jǫ}tna gǫrðum \alst{ó}r, &
ef \alst{G}unn-laðar né nyta’k, \hld\ hinnar \alst{g}óðu konu, &
\ind es lǫgðumk \alst{a}rm \alst{y}fir.\eva

\bvb There is doubt in me, that I would yet be come \\
out of the yards of the Ettins, \\
if I had not used Guthlathe, that good woman \\
whom I laid my arm over.\evb
\evg


\bvg
\bva \edtrans{\alst{H}ins \alst{h}indra dags}{The following day}{\Bfootnote{This is the only occurrence of the comparative \emph{hindra} ‘following, next’ in the Norse (i.e. ‘belonging to Norway and its colonies’) literature. The superlative \emph{hindstr} ‘last, final’ does occur more often (e.g. \emph{indsta sinni} ‘the last time’, with loss of the \emph{h-}; see \CV: \emph{hindri}), and the possible derivative \emph{hindar-dags} ‘day after tomorrow, two days after’ is found twice, both times in the \Gulatingslog, chh. 37 and 266.  If we, however, search in the broader Scandinavian sphere, we find in the Swedish provicial laws an exact equivalent of the present phrase, namely OSwe. \emph{hindra-dagher}, a law-word referring specifically to the ‘day after the wedding’, used both on its own and in the expression \emph{hindra-dags gięf} ‘morning gift’.  If this is indeed the sense in the present stanza, two interpretations are possible: it either (i) refers sarcastically to Weden’s sleeping with Guthlathe (as would be done on the wedding night), or (ii) means that Weden actually married, or promised to marry, Guthlathe.  The latter interpretation may find support in st. 109, see notes there.}} \hld\ gingu \alst{h}rím-þursar &
\alst{H}áva ráðs at fregna, \hld\ \alst{H}áva \alst{h}ǫllu í, &
at \alst{B}ǫl-verki spurðu, \hld\ ef vę́ri með \alst{b}ǫndum kominn &
\ind eða hęfði hǫ́num \alst{S}uttungr of \alst{s}óit.\eva

\bvb The following day went the Rime-Thurses \\
to ask for the High One’s counsel, in the High One’s hall. \\
About Baleworker \name{= Weden} they asked, whether he were come among the bonds \ken{gods}, \\
or if Sutting had slain him.\evb
\evg


\bvg
\bva \edtext{Baug-ęið \alst{Ó}ðinn \hld\ hygg at \alst{u}nnit hafi, &
\ind hvat skal hans \alst{t}ryggðum \alst{t}rúa? &
\alst{S}uttung \alst{s}vikvinn \hld\ hann lét \alst{s}umbli frá &
\ind ok \alst{g}rǿtta \alst{G}unn-lǫðu}{\lemma{Baug-ęið \dots\ Gunn-lǫðu ‘A bigh-oath \dots\ brought to tears™}\Bfootnote{The exact narrative referred to in the stanza is hard to pin down, but I find the following most likely: Weden swore an oath on a bigh, its contents being that he would marry Guthlathe. Sutting then hosted a simble (banquet, drinking feast) for the new couple (cf. \emph{hins hindra dags} in st. 108), and Weden slept with her, but after. \emph{svikvinn} ‘betrayed’ and \emph{grǿtta} ‘brought to tears’ are (respectively masc. and fem.) acc. sg. past participles of the transitive verbs \emph{svíkva} ‘to betray’ and \emph{grǿta} ‘to make weep, bring to tears’. I read \emph{lét} as meaning ‘left, abandoned, forsook’.}}.\eva

\bvb A \inx[C]{bigh-oath} I ween that Weden has sworn— \\
how shall one trust his truces? \\
Away from the \inx[C]{simble} he left Sutting, betrayed, \\
and Guthlathe, brought to tears.\evb
\evg

\sectionline

\section{The Speeches of Loddfathomer}

\emph{Loddfáfnismǫ́l}. Advice given to Loddfathomer. In \Regius\ stanza 110 begins with a large initial \emph{M} in the margin, smaller than those of individual named poems, but larger than the typical initials for sts.

\sectionline

\bvg
\bva Mál ’s at \alst{þ}ylja \hld\ \alst{þ}ular stóli á; &
\ind \alst{U}rðar brunni \alst{a}t &
\alst{s}á’k ok þagða’k, \hld\ \alst{s}á’k ok hugða’k, &
\ind hlýdda’k á \alst{m}anna \alst{m}ál; &
of \alst{r}únar hęyrða’k dǿma, \hld\ né umb \alst{r}ǫ́ðum þǫgðu &
\ind \alst{H}áva \alst{h}ǫllu at, &
\ind \alst{H}áva \alst{h}ǫllu í &
\ind hęyrða’k \alst{s}ęgja \alst{s}vá:\eva

\bvb ’Tis time to \inx[C]{thill}, upon the \inx[C]{thyle}’s chair. \\
At the well of Weird \\
I saw and I shut up: I saw and I thought: \\
I heeded the matters of men. \\
Of runes I heard them speak, nor did they shut up about counsels, \\
at the High One’s \name{= Weden’s} hall \ken*{= Walhall}, \\
in the High One’s hall, \\
I heard [them] say thus:\footnoteB{The speaker, describing himself as a thyle (\emph{þulr} ‘sage, chanter of memorized poetry’), says that he will relate what he has heard said in Walhall. Considering the location, it seems almost certain that the giver of this advice was its owner, \inx[P]{Weden}. The receiver of the advice, \inx[P]{Loddfathomer} (see Encyclopedia for etymologies), is otherwise unknown.}\evb
\evg


\bvg
\bva\alst{R}ǫ́ðumk þér Loddfáfnir, \hld\ at \alst{r}ǫ́ð nemir, &
\ind \alst{n}jóta munt ef \alst{n}emr, &
\ind þér munu \alst{g}óð ef \alst{g}etr: &
\alst{n}ǫ́tt þú rís-at, \hld\ nema á \alst{n}jósn séir, &
\ind eða \edtrans{lęitir þér \alst{i}nnan \alst{ú}t staðar}{thou must look for thy place, [going] out from within}{\Bfootnote{A difficult line to translate faithfully, owing to \emph{innan út} ‘[going] out from within’ and the euphemistic expression \emph{lęita sér staðar} ‘look for one’s place’ for ‘shit’, something which at the time was done outside. The meaning of the line is thus ‘or if you are leaving your house to relieve yourself’.}}.\eva

\bvb I counsel thee, O Loddfathomer, that thou learn the counsels; \\
thou wilt have use if thou learn [them], \\
they will be good for thee if thou get [them]: \\
At night thou rise not, unless thou be scouting, \\
or [if] thou must look for thy place, [going] out from within.\evb
\evg


\bvg
\bva\alst{R}ǫ́ðumk þér Loddfáfnir, \hld\ at \alst{r}ǫ́ð nemir, &
\ind \alst{n}jóta munt ef \alst{n}emr, &
\ind þér munu \alst{g}óð ef \alst{g}etr: &
\alst{f}jǫl-kunnigri konu \hld\ skal-at-tu í \alst{f}aðmi sofa, &
\ind svá’t hon \alst{l}yki þik \alst{l}iðum.\eva

\bvb I counsel thee, O Loddfathomer, that thou learn the counsels; \\
thou wilt have use if thou learn [them], \\
they will be good for thee if thou get [them]: \\
In the bosom of a \inx[C]{many-cunning} woman shalt thou never sleep, \\
lest she might lock you in [her?] limbs.\evb
\evg


\bvg
\bva Hón svá \alst{g}ęrir \hld\ at \alst{g}áir ęigi &
\ind \alst{þ}ings né \alst{þ}jóðans máls; &
\alst{m}at þú vill-at \hld\ né \alst{m}anns-kis gaman &
\ind fęrr þú \alst{s}orga-fullr at \alst{s}ofa.\eva

\bvb She makes it so that thou heed not \\
the \inx[C]{Thing}, nor the ruler’s speech: \\
thou wilt [then] not have food, nor any man’s pleasure; \\
thou goest full of sorrows to sleep.\evb
\evg


\bvg
\bva\alst{R}ǫ́ðumk þér Loddfáfnir, \hld\ at \alst{r}ǫ́ð nemir, &
\ind \alst{n}jóta munt ef \alst{n}emr, &
\ind þér munu \alst{g}óð ef \alst{g}etr: &
\alst{a}nnars konu \hld\ tęyg þér \alst{a}ldri-gi &
\ind \edtrans{\alst{ęy}ra-rúnu}{ear-whisperer \ken{lover}}{\Bfootnote{This word is also used in \Voluspa\ TODO, in which male seducers of married women are among those being forced to wade through “heavy streams” in the afterlife, prob. a reference to bog burials (see there).}} \alst{a}t.\eva

\bvb I counsel thee, O Loddfathomer, that thou learn the counsels; \\
thou wilt have use if thou learn [them], \\
they will be good for thee if thou get [them]: \\
Never lure another man’s woman \\
into [becoming] thy ear-whisperer \ken{lover}.\evb
\evg


\bvg
\bva\alst{R}ǫ́ðumk þér Loddfáfnir, \hld\ en \alst{r}ǫ́ð nemir, &
\ind \alst{n}jóta munt ef \alst{n}emr, &
\ind þér munu \alst{g}óð ef \alst{g}etr: &
á \edtrans{\alst{f}jalli eða \alst{f}irði}{fell or firth}{\Bfootnote{i.e. ‘hiking through the mountains or travelling at sea’; a very Norse expression. This word pair is a formulaic merism, which occurs a few times in the Norwegian laws, but not elsewhere in poetry.}}, \hld\ ef þik \alst{f}ara tíðir, &
\ind fásk-tu at \alst{v}irði \alst{v}ęl.\eva

\bvb I counsel thee, O Loddfathomer—and thou oughtst to learn the counsels; \\
thou wilt have use if thou learn [them], \\
they will be good for thee if thou get [them]: \\
on the fell or firth—if thou desire to journey— \\
furnish thyself well with food.\evb
\evg


\bvg
\bva\alst{R}ǫ́ðumk þér Loddfáfnir, \hld\ en \alst{r}ǫ́ð nemir, &
\ind \alst{n}jóta munt ef \alst{n}emr, &
\ind þér munu \alst{g}óð ef \alst{g}etr: &
\alst{i}llan mann \hld\ lát \alst{a}ldri-gi &
\ind \edtext{\alst{ó}·hǫpp at þér \alst{v}ita}{\Bfootnote{An unambiguous instance of \emph{v} alliterating with a vowel.}}, &
því-at af \alst{i}llum manni \hld\ fę́r \alst{a}ldri-gi &
\ind \alst{g}jǫld hins \alst{g}óða hugar.\eva

\bvb I counsel thee, O Loddfathomer—and thou oughtst to learn the counsels; \\
thou wilt have use if thou learn [them], \\
they will be good for thee if thou get [them]: \\
An evil man let thou never \\
know of thy misfortunes, \\
for from an evil man gettest thou never \\
recompense for thy good heart.\evb
\evg


\bvg
\bva\edtrans{\alst{O}far-la}{Sorely}{\Bfootnote{Contraction of \emph{ofar-liga} ‘\CV: high up, in the upper part’, presumably meaning that the words were particularly grievous or insulting, i.e., they “got to him”.  Whether he was murdered or committed suicide is not clear.}} bíta \hld\ sá’k \alst{ęi}num hal &
\ind \alst{o}rð \alst{i}llrar konu, &
\alst{f}lá-rǫ́ð tunga \hld\ varð hǫ́num at \alst{f}jǫr-lagi &
\ind ok þęygi of \alst{s}anna \alst{s}ǫk.\eva

\bvb Sorely I saw biting, on one man, \\
an evil woman’s words; \\
a false-counseling tongue brought his life to its end, \\
and in no way over a truthful charge.\footnoteB{Cf. \Lokasenna\ 31/1: \emph{flǫ́ ’s þér tunga} ‘false is thy tongue’.}\evb
\evg


\bvg
\bva\alst{R}ǫ́ðumk þér Loddfáfnir, \hld\ en \alst{r}ǫ́ð nemir, &
\ind \alst{n}jóta munt ef \alst{n}emr, &
\ind þér munu \alst{g}óð ef \alst{g}etr: &
\alst{v}ęitst, ef \alst{v}in átt, \hld\ þann’s \alst{v}ęl trúir, &
\ind \alst{f}ar þú at \alst{f}inna opt; &
því-at \edtrans{\alst{h}rísi vęx \hld\ ok \alst{h}ǫ́u grasi}{with brushwood and with tall grass grows’}{\Bfootnote{Identical with \Grimnismal\ 17/1.}} &
\ind \alst{v}egr, es \alst{v}ę́t-ki trøðr.\eva

\bvb I counsel thee, O Loddfathomer—and thou oughtst to learn the counsels; \\
thou wilt have use if thou learn [them], \\
they will be good for thee if thou get [them]: \\
Know, if thou have a friend, one on which thou well trust, \\
journey to find him oft; \\
for with brushwood and tall grass grows \\
the way which no man treads.\evb
\evg


\bvg
\bva\alst{R}ǫ́ðumk þér Loddfáfnir, \hld\ en \alst{r}ǫ́ð nemir, &
\ind \alst{n}jóta munt ef \alst{n}emr, &
\ind þér munu \alst{g}óð ef \alst{g}etr: &
\alst{g}óðan mann \hld\ tęyg þér at \edtrans{\alst{g}aman-rúnum}{pleasure-runes}{\Bfootnote{Here “rune” apparently carries its root meaning of ‘whisper, counsel, speech’, thus ‘pleasing speech’.  Cf. st. 130 where this word reoccurs.}} &
\ind ok nem \edtrans{\alst{l}íknar-galdr}{liking-galder}{\Bfootnote{i.e. ways of speaking which will make one liked or popular. For \emph{líkn} see sts. 8 (with note) and 123.}} meðan \alst{l}ifir.\eva

\bvb I counsel thee, O Loddfathomer—and thou oughtst to learn the counsels; \\
thou wilt have use if thou learn [them], \\
they will be good for thee if thou get [them]: \\
Lure a good man to thee through pleasure-runes, \\
and learn liking-galder while thou livest.\evb
\evg


\bvg
\bva\alst{R}ǫ́ðumk þér Loddfáfnir, \hld\ en \alst{r}ǫ́ð nemir, &
\ind \alst{n}jóta munt ef \alst{n}emr, &
\ind þér munu \alst{g}óð ef \alst{g}etr: &
\alst{v}in þínum \hld\ \alst{v}es aldri-gi &
\ind \alst{f}yrri at \alst{f}laum-slitum. &
\alst{s}org etr hjarta, \hld\ ef þú \edtext{\alst{s}ęgja né náir &
\ind \alst{ęi}n-hvęrjum \alst{a}llan hug}{\lemma{sęgja \dots\ ęin-hvęrjum allan hug ‘tell anyone thy whole mind’}\Bfootnote{Cf. st. 124 which uses almost the same expression.}}.\eva

\bvb I counsel thee, O Loddfathomer—and thou oughtst to learn the counsels; \\
thou wilt have use if thou learn [them], \\
they will be good for thee if thou get [them]: \\gettest: \\
With thy friend be thou never the first \\
to tear apart the company. \\
Sorrow eats thy heart if thou cannot tell \\
anyone thy whole mind.\evb
\evg


\bvg
\bva\alst{R}ǫ́ðumk þér Loddfáfnir, \hld\ en \alst{r}ǫ́ð nemir, &
\ind \alst{n}jóta munt ef \alst{n}emr, &
\ind þér munu \alst{g}óð ef \alst{g}etr: &
\edtext{\alst{o}rðum skipta \hld\ skalt \alst{a}ldri-gi &
\ind við \edtrans{\alst{ó}·svinna \alst{a}pa}{unwise apes}{\Bfootnote{Formulaic. Cf. TODO.}}}{\lemma{orðum \dots\ apa ‘Words \dots\ apes’}\Bfootnote{Cf. st. 125 which gives similar advice.}},\eva

\bvb I counsel thee, O Loddfathomer—and thou oughtst to learn the counsels; \\
thou wilt have use if thou learn [them], \\
they will be good for thee if thou get [them]: \\
Words shalt thou never exchange \\
with unwise apes,\evb
\evg


\bvg
\bva \edtext{því-at af \alst{i}llum manni \hld\ munt \alst{a}ldri-gi &
\ind \alst{g}óðs laun of \alst{g}eta}{\lemma{því-at \dots\ geta ‘For \dots\ praise’}\Bfootnote{Cf. st. 117/6–7.}}, &
en \alst{g}óðr maðr \hld\ mun þik \alst{g}ørva męga &
\ind \edtrans{\alst{l}íkn-fastan}{steadfast in liking}{\Bfootnote{The first element \emph{líkn} ‘liking’ is somewhat difficult; see sts. 8 (with note) and 120.  For the present cpd \textcite{LaFargeGlossary} give a tentative ‘assured of favour’, while \CV\ gives ‘fast in goodwill, beloved’.}} at \alst{l}ofi.\eva

\bvb for from an evil man wilt thou never \\
get a reward for thy goodness, \\
but a good man will know to make thee \\
steadfast in liking by [his] praise.\evb
\evg


\bvg
\bva\alst{S}ifjum ’s þá blandit \hld\ hvęrr es \edtext{\alst{s}ęgja rę́ðr &
\ind \alst{ęi}num \alst{a}llan hug}{\lemma{sęgja \dots\ \alst{ęi}num \alst{a}llan hug ‘tell one man his whole mind’}\Bfootnote{Cf. st. 121 which uses almost the same expression.}}; &
alt es \alst{b}ętra \hld\ an sé \alst{b}rigðum at vesa: &
es-a sá \alst{v}inr ǫðrum \hld\ es \alst{v}ilt ęitt sęgir.\eva

\bvb Kinship is then blended, when any man decides to tell \\
one man his whole mind. \\
Everything is better than to be with the fickle; \\
he is no friend to another who says only that which is wanted.\evb
\evg


\bvg
\bva\alst{R}ǫ́ðumk þér Loddfáfnir, \hld\ en \alst{r}ǫ́ð nemir, &
\ind \alst{n}jóta munt ef \alst{n}emr, &
\ind þér munu \alst{g}óð ef \alst{g}etr: &
\edtrans{þrimr \alst{o}rðum}{With three words}{\Bfootnote{i.e. ‘not even with three words’. If one understands \emph{orð} to mean ‘speech’, it may be interpreted as that if one says something (the first speech) to which another man responds insultingly (the second speech), one should not respond a third time and turn it into a fight.}} sęnna \hld\ skal-at-tu þér við \alst{v}erra mann; &
\ind opt hinn \alst{b}ętri \alst{b}ilar, &
\ind þá’s hinn \alst{v}erri \alst{v}egr.\eva

\bvb I counsel thee, O Loddfathomer—and thou oughtst to learn the counsels; \\
thou wilt have use if thou learn [them], \\
they will be good for thee if thou get [them]: \\
With three words shalt thou not flyte with a worse man; \\
oft the better man breaks \\
when the worse man strikes.\footnoteB{Cf. st. 122.}\evb
\evg


\bvg
\bva\alst{R}ǫ́ðumk þér Loddfáfnir, \hld\ en \alst{r}ǫ́ð nemir, &
\ind \alst{n}jóta munt ef \alst{n}emr, &
\ind þér munu \alst{g}óð ef \alst{g}etr: &
\alst{sk}ó-smiðr þú vesir \hld\ né \alst{sk}ępti-smiðr, &
\ind nema \alst{s}jǫlfum þér \alst{s}éir. &
\alst{Sk}ór ’s \alst{sk}apaðr illa \hld\ eða \alst{sk}apt sé rangt, &
\ind þá ’s þér \alst{b}ǫls \alst{b}eðit.\eva

\bvb I counsel thee, O Loddfathomer—and thou oughtst to learn the counsels; \\
thou wilt have use if thou learn [them], \\
they will be good for thee if thou get [them]: \\
Be not a shoe-maker nor shaft-maker, \\
unless thou be one for thyself. \\
{[If]} the shoe is shaped badly or the shaft be crooked, \\
then for thee a \inx[C]{bale} is bidden.\footnoteB{i.e. ‘the customer will place a curse on you if he dislikes the wares’.}\evb
\evg


\bvg
\bva\alst{R}ǫ́ðumk þér Loddfáfnir, \hld\ en \alst{r}ǫ́ð nemir, &
\ind \alst{n}jóta munt ef \alst{n}emr, &
\ind þér munu \alst{g}óð ef \alst{g}etr: &
hvar’s \alst{b}ǫl kant, \hld\ kveð þér \alst{b}ǫlvi at &
\ind ok gef-at þínum \alst{f}jǫ́ndum \alst{f}rið.\eva

\bvb I counsel thee, O Loddfathomer—and thou oughtst to learn the counsels; \\
thou wilt have use if thou learn [them], \\
they will be good for thee if thou get [them]: \\
Whereever thou dost know a bale, call it a bale against thee, \\
and give not thy enemies peace.\footnoteB{i.e. “if somebody puts a curse on you, do not ignore it, but respond decisively”.  This st. has often been interpreted as a command to call out evil, even when committed towards somebody else, and while there is nothing in it that speaks clearly against that interpretation, it does not agree with the general spirit of the \Havamal, which is one of caution and shrewdness.}\evb
\evg


\bvg
\bva\alst{R}ǫ́ðumk þér Loddfáfnir, \hld\ en \alst{r}ǫ́ð nemir, &
\ind \alst{n}jóta munt ef \alst{n}emr, &
\ind þér munu \alst{g}óð ef \alst{g}etr: &
\alst{i}llu fęginn \hld\ ves \alst{a}ldri-gi, &
\ind \edtrans{en lát þér at \alst{g}óðu \alst{g}etit}{but [rather] let thyself be pleased by good}{\Bfootnote{This construction is equivalent to \CV: \emph{geta}, A. IV. with acc.}}.\eva

\bvb I counsel thee, O Loddfathomer—and thou oughtst to learn the counsels; \\
thou wilt have use if thou learn [them], \\
they will be good for thee if thou get [them]: \\
Rejoicing in evil be thou never, \\
but [rather] let thyself be pleased by good.\evb
\evg


\bvg
\bva\alst{R}ǫ́ðumk þér Loddfáfnir, \hld\ en \alst{r}ǫ́ð nemir, &
\ind \alst{n}jóta munt ef \alst{n}emr, &
\ind þér munu \alst{g}óð ef \alst{g}etr: &
\alst{u}pp líta \hld\ skal-at-tu í \alst{o}rrostu; &
—\alst{g}jalti \alst{g}líkir \hld\ verða \alst{g}umna synir— &
\ind síðr þitt of \alst{h}ęilli \alst{h}alir.\eva

\bvb I counsel thee, O Loddfathomer—and thou oughtst to learn the counsels; \\
thou wilt have use if thou learn [them], \\
they will be good for thee if thou get [them]: \\
Up shalt thou not look in battle \\
—alike to a madman become the sons of men— \\
lest men bewitch thy [sense/life/face].\footnoteB{A very difficult st. \CV\ explains \emph{gjalti} as an old dative of \emph{gǫltr} ‘boar, hog’, and thus sees the closely related phrase \emph{verða at gjalti} as “‘to be turned into a hog’, i.e. ‘to turn mad with terror’, esp. in a fight”. The vowel breaking is however unexpected here, since \emph{gǫltr} (< Proto-Norse \emph{*galtuʀ}) is an u-stem, which makes the stem-vowel in the dat. sg. \emph{gęlti} (< \emph{*galtiu}, cf. \textbf{kunimudiu}, dat. sg. of \emph{*Kunimunduʀ}, on the Tjurkö 1 bracteate) the result of i-umlaut rather than an original short \emph{*e}.

\textcite{LaFargeGlossary} instead explain the word as a borrowing from Old Irish \emph{geilt} ‘insane, mad’. \textcite{PettitEdda} follows this, and argues that the whole theme of the st. probably be of Celtic origin, giving several examples from Celtic literature of warriors going mad upon looking up into the sky during battle. In this case the men (\emph{halir}, which word seems to have an association with warriors; cf. 36–37, 49) would be to quote Pettit some sort of “supernatural sky warriors”, in my opinion most likely the \inx[G]{Ownharriers}.}\evb
\evg


\bvg
\bva\alst{R}ǫ́ðumk þér Loddfáfnir, \hld\ en \alst{r}ǫ́ð nemir, &
\ind \alst{n}jóta munt ef \alst{n}emr, &
\ind þér munu \alst{g}óð ef \alst{g}etr: &
Ef vilt þér \alst{g}óða konu \hld\ kvęðja at \edtrans{\alst{g}aman-rúnum}{pleasure-runes}{\Bfootnote{While easily interpreted as ‘sexual intercourse’, the word is used in st. 120 with a decidedly non-sexual meaning. Its base meaning is probably ‘good, light-hearted conversation’.}} &
\ind ok \alst{f}áa \alst{f}ǫgnuð af, &
\alst{f}ǫgru skalt hęita \hld\ ok láta \alst{f}ast vesa; &
\ind lęiðisk mann-gi \alst{g}ótt ef \alst{g}etr.\eva

\bvb I counsel thee, O Loddfathomer—and thou oughtst to learn the counsels; \\
thou wilt have use if thou learn [them], \\
they will be good for thee if thou get [them]: \\
If thou wilt for thyself greet a good woman to pleasure-runes, \\
and get good cheer from her; \\
fair things shalt thou promise, and let it be fast; \\
no man loathes a good thing if he gets it.\evb
\evg


\bvg
\bva\alst{R}ǫ́ðumk þér Loddfáfnir, \hld\ en \alst{r}ǫ́ð nemir, &
\ind \alst{n}jóta munt ef \alst{n}emr, &
\ind þér munu \alst{g}óð ef \alst{g}etr: &
\alst{v}aran bið’k þik \alst{v}esa \hld\ ok ęigi of·\alst{v}aran, &
ves við \alst{ǫ}l varastr, \hld\ ok við \alst{a}nnars konu &
ok við \alst{þ}at hit \alst{þ}riðja, \hld\ at \alst{þ}jófar né lęiki.\eva

\bvb I counsel thee, O Loddfathomer—and thou oughtst to learn the counsels; \\
thou wilt have use if thou learn [them], \\
they will be good for thee if thou get [them]: \\
Wary I ask thee to be, and not over-wary; \\
be thou wariest with ale, and with another man’s woman, \\
and with the third, that thieves do not outplay [thee].\evb
\evg


\bvg
\bva\alst{R}ǫ́ðumk þér Loddfáfnir, \hld\ en \alst{r}ǫ́ð nemir, &
\ind \alst{n}jóta munt ef \alst{n}emr, &
\ind þér munu \alst{g}óð ef \alst{g}etr: &
at \alst{h}áði né \alst{h}látri \hld\ \alst{h}af aldri-gi &
\ind \alst{g}ęst né \alst{g}anganda.\eva

\bvb I counsel thee, O Loddfathomer—and thou oughtst to learn the counsels; \\
thou wilt have use if thou learn [them], \\
they will be good for thee if thou get [them]: \\
In mockery or laughter have thou never \\
a guest nor wanderer.\evb
\evg


\bvg
\bva\alst{O}pt vitu \alst{ó}·gǫrla, \hld\ þęir’s sitja \alst{i}nni fyrir, &
\ind hvęrs þęir ’ru \alst{k}yns es \alst{k}oma; &
es-at maðr svá \alst{g}óðr \hld\ at \alst{g}alli né fylgi, &
\ind né svá \alst{i}llr at \alst{ęi}nu-gi dugi.\eva

\bvb Oft they know unclearly, those who sit further within, \\
of what kind are those who come; \\
there is no man so good that him follows no flaw, \\
nor so bad that he to nothing avails.\evb
\evg


\bvg
\bva\alst{R}ǫ́ðumk þér Loddfáfnir, \hld\ en \alst{r}ǫ́ð nemir, &
\ind \alst{n}jóta munt ef \alst{n}emr, &
\ind þér munu \alst{g}óð ef \alst{g}etr: &
at \alst{h}ǫ́rum þul \hld\ \alst{h}lę́ aldri-gi, &
\ind opt ’s \alst{g}ótt þat’s \alst{g}amlir kveða, &
opt ór \alst{sk}ǫrpum bęlg \hld\ \alst{sk}ilin orð koma &
\ind þęim’s \alst{h}angir með \alst{h}ǫ́um &
\ind ok \alst{sk}ollir með \alst{sk}rǫ́um, &
\ind ok \alst{v}áfir með \alst{v}íl-mǫgum.\eva

\bvb I counsel thee, O Loddfathomer—and thou oughtst to learn the counsels; \\
thou wilt have use if thou learn [them], \\
they will be good for thee if thou get [them]: \\
At a hoary thyle laugh thou never; \\
oft is good that which old men sing. \\
Oft out of a scorched leather discerning words come; \\
out of that one that hangs with hides, \\
and dangles with dry skins, \\
and sways among lads of toil \ken{thralls}.\footnoteB{TODO: Some note. \emph{vil-mǫgum} meaning ‘veal-stomachs’? Cf. Crawford’s video and Finnur on this.}\evb
\evg


\bvg
\bva\alst{R}ǫ́ðumk þér Loddfáfnir, \hld\ en \alst{r}ǫ́ð nemir, &
\ind \alst{n}jóta munt ef \alst{n}emr, &
\ind þér munu \alst{g}óð ef \alst{g}etr: &
\alst{g}ęst þú né \alst{g}ęyj-a \hld\ né á \alst{g}rind hrę́kir; &
\ind get þú \alst{v}ǫ́-luðum \alst{v}ęl.\eva

\bvb I counsel thee, O Loddfathomer—and thou oughtst to learn the counsels; \\
thou wilt have use if thou learn [them], \\
they will be good for thee if thou get [them]: \\
Bark not at a guest, nor spit at the gate;\footnoteB{Behind which the guest stands, waiting for the farmer to open.} \\
furnish the destitute well.\evb
\evg


\bvg
\bva\alst{R}amt es þat tré, \hld\ es \alst{r}íða skal &
\ind \alst{ǫ}llum at \alst{u}pp-loki; &
\alst{b}aug þú gef \hld\ eða þat \alst{b}iðja mun &
\ind þér \alst{l}ę́s hvęrs á \alst{l}iðu.\eva

\bvb Strong is that wood which shall swing \\
to open for all.\footnoteB{i.e. the beam of the gate in front of the farm.} \\
Give a bigh, or it will bid \\
every kind of guile onto thy limbs.\evb
\evg


\bvg
\bva\alst{R}ǫ́ðumk þér Loddfáfnir, \hld\ en \alst{r}ǫ́ð nemir, &
\ind \alst{n}jóta munt ef \alst{n}emr, &
\ind þér munu \alst{g}óð ef \alst{g}etr: &
hvar’s \alst{ǫ}l drekkir \hld\ kjós þér \alst{ja}rðar męgin, &
því-at \alst{jǫ}rð tękr við \alst{ǫ}lðri, \hld\ en \alst{ę}ldr við sóttum, &
\alst{ęi}k við \alst{a}bbindi, \hld\ \alst{a}x við fjǫl-kyngi, &
\alst{h}ǫll við \alst{h}ýrógi; \hld\ \alst{h}ęiptum skal mána kvęðja, &
\alst{b}ęiti við \alst{b}it-sóttum, \hld\ en við \alst{b}ǫlvi rúnar; &
\ind \alst{f}old skal við \alst{f}lóði taka.\eva

\bvb I counsel thee, O Loddfathomer, that thou learn the counsels; \\
thou wilt have use if thou learn [them], \\
they will be good for thee if thou get [them]: \\
Wherever thou drinkest ale, choose for thee Earth’s might, \\
for earth takes against drunkenness, but fire against sicknesses; \\
oak against dysentery, the ear [of corn] against sorcery, \\
bearded rye against hernia—in conflicts shall one invoke Moon\footnoteB{According to \Voluspa\ 5, the moon has some sort of power, and based on \Lokasenna\ P3 \emph{kvęðja} ‘greet, call’ seems to be the word used for invoking in prayer.}— \\
heather against bite-sicknesses; but \inx[C]{rune}[runes] against a \inx[C]{bale};\footnoteB{cf. sts. 124, 149.} \\
fold \ken{earth} shall one employ against flood.\evb
\evg

\sectionline

\section{The Rune-Tally}

These scattered sts. are introduced by a larger initial in \Regius, marking the beginning of a new section. They have the header \emph{Rúna-tals þáttr} ‘Strand of the Rune-Tally’ in younger paper mss. and generally give an archaic, mystic impression; it is as if they were drawn from the lips of an Odinic priest.

Apart from these stanzas, there are a few other instances of Runic magic. Closest at hand is st. 80 above, which would fit seamlessly into the present section. Outside of \Havamal\ there is \Sigrdrifumal\ 4–16, also preserved in \Regius.

\sectionline

\bvg
\bva\alst{V}ęit’k at ek hekk \hld\ \alst{v}indga męiði á &
\ind \alst{n}ę́tr allar \alst{n}íu, &
\alst{g}ęiri undaðr \hld\ ok \alst{g}efinn Óðni, &
\ind \alst{s}jalfr \alst{s}jǫlfum mér, &
á þęim \alst{m}ęiði, \hld\ es \alst{m}ann-gi vęit, &
\ind hvęrs af \alst{r}ótum \alst{r}innr.\eva

\bvb I know that I hung on the windy beam, \\
for nine nights all; \\
wounded by spear and given to Weden— \\
myself to myself— \\
on that beam, which no man knows, \\
of whose roots it runs.\evb
\evg


\bvg
\bva Við \edtext{\alst{h}lęifi mik sǿldu-t \hld\ né við \alst{h}orni-gi}{\lemma{hlęifi \dots\ horni-gi ‘loaf \dots\ horn’}\Bfootnote{i.e. “I was given neither food nor drink”.}}; &
\alst{n}ýsta ek \alst{n}iðr, \hld\ \alst{n}am’k upp rúnar, &
\alst{ǿ}pandi nam, \hld\ fell’k \alst{a}ptr þaðan.\eva

\bvb With loaf they relieved me not, nor with any horn. \\
I peered down, I took up the runes, \\
screaming I took; I fell back thence.\evb
\evg


\bvg
\bva\alst{F}imbul-ljóð níu \hld\ nam’k af hinum \alst{f}rę́gja syni &
\ind \alst{B}ǫlþorns, \alst{B}ęstlu fǫður, &
ok ek \alst{d}rykk of gat \hld\ hins \alst{d}ýra mjaðar &
\ind \alst{au}sinn \alst{Ó}ð-rǿri.\eva

\bvb Nine \inx[C]{fimble-leeds} I learned from the famous son \\
of \inx[P]{Balethorn}, \inx[P]{Bestle}’s father— \\
and a drink I got, of that dear mead \\
poured [from] \inx[P]{Woderearer}.\footnoteB{This st. fits poorly here and seems like an insert. It mentions \emph{ljóð} ‘leeds; (magical) songs, incantations’ rather than runes, and has nothing to do with Weden’s hanging on the tree. Bestle was Weden’s mother and Balethorn his maternal grandfather. The famous son of Balethorn would then be his maternal uncle. The custom of sending sons away to be fostered by their maternal uncles or grandfathers (which seems to be what is going on here) was quite common in Germanic society, cf. TODO.}\evb
\evg


\bvg
\bva Þá \edtrans{nam’k \alst{f}rę́vask}{I took to thrive}{\Bfootnote{A notorious mistranslation (TODO: source) has rendered these words as ‘I took semen’, seeing in them a reference to Weden taking the seed from hanged men in order to replenish his own powers, something never attested elsewhere. This notion, surely based on the word \emph{frę́} ‘seed’, has no philological grounding. \emph{frę́vask} is wo. doubt a reflexive verb, and regardless \emph{frę́} is used of plant seeds, not ejaculate.}} \hld\ ok \alst{f}róðr vesa &
\ind ok \alst{v}axa ok \alst{v}ęl hafask; &
\alst{o}rð mér af \alst{o}rði \hld\ \alst{o}rðs lęitaði &
\ind \alst{v}erk mér af \alst{v}erki \alst{v}erks.\eva

\bvb Then I began to flourish, and be learned, \\
and grow and have it well. \\
My word from a word a word sought out; \\
my work from a work a work.\footnoteB{Each good speech and deed quickly led to another.}\evb
\evg


\bvg
\bva \edtext{\alst{R}únar munt finna \hld\ ok \alst{r}áðna stafi}{\lemma{Rúnar \dots\ ok ráðna stafi}\Bfootnote{Formulaic. Cf. the long-line on the medieval runestone N 13 (excerpt): \emph{rúnar ek ríst \hld\ ok ráðna stafi} ‘runes I carve, and interpreted staves’.}}, &
\ind mjǫk \alst{st}óra \alst{st}afi, &
\ind mjǫk \alst{st}inna \alst{st}afi, &
\ind es \alst{f}áði \alst{F}imbul-þulr &
\ind ok \alst{g}ørðu \alst{g}inn-ręgin &
\ind ok \alst{r}ęist Hroptr \edtrans{\alst{r}agna}{of the Reins}{\Afootnote{\emph{‘rǫgna’} \Regius}}.\eva

\bvb \inx[C]{rune}[Runes] wilt thou find, and interpreted staves: \\
very large staves, \\
very stiff staves, \\
which \inx[P]{Fimble-Thyle} \name{= Weden} painted, \\
and the \inx[G]{yin-Reins} made, \\
and Roft \name{= Weden} of the Reins carved.\evb
\evg


\bvg
\bva\alst{Ó}ðinn með \alst{ǫ́}sum, \hld\ en fyr \alst{ǫ}lfum Dáinn, &
\ind \alst{D}valinn \alst{d}vergum fyrir, &
\ind \alst{Á}sviðr \alst{jǫ}tnum fyrir, &
\ind ek ręist \alst{s}jalfr \alst{s}umar.\eva

\bvb \inx[P]{Weden} among the \inx[G]{Eese}, but for the \inx[G]{Elves} \inx[P]{Dowen}; \\
\inx[P]{Dwollen} for the \inx[G]{Dwarfs}; \\
\inx[P]{Oswood} for the Ettins; \\
I myself carved some.\footnoteB{The identity of the speaker is not clear. One would expect him to be Weden.}\evb
\evg


\bvg
\bva Vęitst, hvé \alst{r}ísta skal? \hld\ Vęitst, hvé \alst{r}áða skal? &
Vęitst, hvé \alst{f}áa skal? \hld\ Vęitst, hvé \alst{f}ręista skal? &
Vęitst, hvé \alst{b}iðja skal? \hld\ Vęitst, hvé \alst{b}lóta skal? &
Vęitst, hvé \alst{s}ęnda skal? \hld\ Vęitst, hvé \alst{s}óa skal?\eva

\bvb Knowest thou how one shall carve? Knowest thou how one shall read? \\
Knowest thou how one shall paint? Knowest thou how one shall try? \\
Knowest thou how one shall bid? Knowest thou how one shall \inx[C]{bloot}? \\
Knowest thou one shall send? Knowest thou how one shall \inx[C]{soo}?\footnoteB{A neat semantic structure would be found if the former four verbs referred to \inx[C]{rune}[runes]: carving, interpreting, painting (with blood?), and divining; and the latter four referred to sacrifice: asking for boons, worshipping, sending (the sacrifice or the prayer; making sure the gods receive it), and slaying the victim. This may be supported by the following stanza, which repeats the last four verbs here in what looks like a sacrificial context. See further relevant Encyclopedia entries.}\footnoteB{The meter of this st. is unusual, but bears some resemblance to Vg 216 (the Högstena galder). TODO: Elaborate.}\evb
\evg


\bvg
\bva\alst{B}ętra ’s ó·\alst{b}eðit \hld\ an sé of·\alst{b}lótit, &
\ind ęy sér til \alst{g}ildis \alst{g}jǫf; &
bętra ’s ó·\alst{s}ęnt \hld\ an sé of·\alst{s}óit; &
\edtext{[...]}{\Bfootnote{It is almost certain that a line be missing here, which is very unfortunate.}}\eva

\bvb ’Tis better unbid than over\inx[C]{bloot}[blooted]; \\
a gift always sees repayment. \\
’Tis better unsent than over\inx[C]{soo}[sooed]; \\
{[...]}.\footnoteB{An identical progression of four verbs suggests a close relation with the previous st. — The sense seems to be that it is better not to sacrifice at all than to sacrifice in excess, since even a small gift (to the gods) will be rewarded. A ritual cycle of gifts and rewards between men and the gods is also seen in other Indo-European pagan literatures. Compare the Sanskrit \emph{Dehí me, dádāmi te} ‘Give to me, I give to thee’ and Latin \emph{dō ut dēs} ‘I give that thou might give’.}\evb
\evg


\bvg
\bva Svá \alst{Þ}undr of ręist \hld\ fyr \alst{þ}jóða rǫk, &
þar’s \alst{u}pp of ręis, \hld\ es \alst{a}ptr of kom.\eva

\bvb Thus \inx[P]{Thound} \name{= Weden} did carve for the rakes of nations, \\
where up he rose as back he came.\footnoteB{TODO: A very cryptic st.}\evb
\evg

\sectionline

\section{The Leed-Tally}

This section of \Havamal, the so-called the Leed-Tally (\emph{Ljóðatal}), is not separated from the preceding section (which is marked out with a large initial), but is usually taken as separate since it is a unified whole not much concerned with runes. The speaker (certainly Weden) recounts eighteen spells, aristocratic and Odinic in character; they deal with such things as healing (spell 2, 12), battle (3, 4, 5, 8, 11, 13), countering sorcery (6, 10), stilling the elements (7, 9), and seduction (16, 17).

In particular the fourth spell bears a strong likeness to the first Merseburg charm.


\bvg
\bva Ljóð \alst{þ}au kann’k, \hld\ es kann-at \alst{þ}jóðans kona &
\ind ok \alst{m}anns-kis \alst{m}ǫgr. &
\alst{H}jǫlp hęitir ęitt, \hld\ þat þér \alst{h}jalpa mun &
\ind við \alst{s}orgum ok \alst{s}ǫkum, \hld\ ok \alst{s}útum gǫrv-ǫllum.\eva

\bvb Those \inx[C]{leed}[leeds] I know, as knows not the ruler’s woman, \\
and no man’s lad: \\
Help is called one, it will help thee \\
against sorrows and sakes,\footnoteB{Legal proceedings.} and all kinds of griefs.\footnoteB{TODO: elaborate on translatioon}\evb
\evg


\bvg
\bva Þat kann’k \alst{a}nnat, \hld\ es þurfu \alst{ý}ta synir, &
\ind þęir’s vilja \alst{l}ę́knar \alst{l}ifa.\eva

\bvb I know another, which the sons of men need;\footnoteB{Identical wording to 164/2.} \\
those who wish to live as leechers.\evb
\evg


\bvg
\bva Þat kann’k \alst{þ}riðja, \hld\ ef mér verðr \alst{þ}ǫrf mikil &
\ind \alst{h}apts við mína \alst{h}ęipt-mǫgu, &
\alst{ę}ggjar dęyfi’k \hld\ minna \alst{a}nd-skota, &
\ind bíta-t þęim \alst{v}ǫ́pn né \edtrans{\alst{v}ęlir}{staffs}{\Bfootnote{This word cannot be \emph{vélir} ‘wiles’ due to the meter. It may refer to magical staffs. (TODO.)}}.\eva

\bvb I know the third, if I come in great need \\
of hindrance against my conflict-lads \ken{enemies}; \\
I dull the edges of my opponents; \\
for them bite not weapons nor staffs.\evb
\evg


\bvg
\bva Þat kann’k \alst{f}jórða, \hld\ ef mér \alst{f}yrðar bera &
\ind \alst{b}ǫnd at \alst{b}óg-limum, &
svá ek \alst{g}ęl, \hld\ at \alst{g}anga má’k, &
\ind sprettr mér af \alst{f}ótum \alst{f}jǫturr, &
\ind en af \alst{h}ǫndum \alst{h}apt.\eva

\bvb I know the fourth, if men should bear \\
bonds onto my shoulder-limbs \ken{arms}: \\
so I gale that I may walk; \\
springs off my feet the fetter, \\
and off my hands the bond.\footnoteB{Cf. \MerseburgOne\ (edited below under Charms and Spells), a galder that seems to have actually been used for the purpose of removing fetters.}\evb
\evg


\bvg
\bva Þat kann’k \alst{f}imta, \hld\ ef sé’k af \alst{f}ári skotinn &
\ind \alst{f}lęin í \alst{f}olki vaða, &
flýgr-a svá \alst{st}int, \hld\ at \alst{st}ǫðvi’g-a’k, &
\ind ef hann \alst{s}jónum of \alst{s}é’k.\eva

\bvb I know the fifth, if I see a dangerously shot \\
arrow wading in the troop; \\
it flies not so stiffly that I may not hinder it, \\
if I see it with my sights.\evb
\evg


\bvg
\bva Þat kann’k \alst{s}étta, \hld\ ef mik \alst{s}ę́rir þegn &
\ind á \alst{r}ótum \alst{r}ás viðar, &
þann \alst{h}al, \hld\ es mik \alst{h}ęipta kveðr, &
\ind þann eta \alst{m}ęin hęldr an \alst{m}ik.\eva

\bvb I know the sixth, if a thane should injure me \\
on the roots of a raw/sappy tree;\footnoteB{i.e., if he carves harmful magic runes into the roots. See note to \Skirnismal\ 32, where \emph{hrár viðr} ‘raw/sappy tree’ also occurs in a context of curse-magic.} \\
that man who sings hatred against me, \\
him eat the harms rather than me.\evb
\evg


\bvg
\bva Þat kann’k \alst{s}jaunda, \hld\ ef \alst{s}é’k hǫ́van loga &
\ind \alst{s}al of \alst{s}ess-mǫgum, &
\alst{b}rinnr-at svá \alst{b}ręitt, \hld\ at hǫ́num \alst{b}jargi’g-a’k; &
\ind þann kann’k \alst{g}aldr at \alst{g}ala.\eva

\bvb I know the seventh, if I see a high hall \\
burning over seat-lads \ken{warriors}: \\
it burns not so broadly that I do not save it\footnoteB{i.e. ‘if I see a hall burning with men trapped inside, no matter how large the flame is I can save both the hall and the men’.}— \\
that galder I can gale.\evb
\evg


\bvg
\bva Þat kann’k \alst{á}tta, \hld\ es \alst{ǫ}llum es &
\ind \alst{n}yt-sam-ligt at \alst{n}ema, &
\alst{h}var’s \edtrans{\alst{h}atr}{hatred}{\Bfootnote{i.e. with regard to the father’s inheritance.}} vęx \hld\ með \alst{h}ildings sonum, &
\ind þat má’k \alst{b}ǿta \alst{b}rátt.\eva

\bvb I know the eighth, which for all men is \\
useful to learn: \\
wherever hatred grows among a prince’s sons, \\
it I may shortly mend.\evb
\evg


\bvg
\bva Þat kann’k \alst{n}íunda, \hld\ ef mik \alst{n}auðr of stęndr &
\ind at bjarga \alst{f}ari mínu á \alst{f}loti, &
\alst{v}ind ek kyrri \hld\ \alst{v}ági á &
\ind ok \alst{s}vę́fi’k allan \alst{s}ę́.\eva

\bvb I know the ninth, if I am in need \\
to save my friend on a floater \ken{ship}: \\
the wind I calm on the wave, \\
and put all the sea asleep.\evb
\evg


\bvg
\bva Þat kann’k \alst{t}íunda, \hld\ ef sé’k \alst{t}ún-riður &
\ind \alst{l}ęika \alst{l}opti á, &
ek svá \alst{v}inn’k, \hld\ at \edtrans{þę́r \alst{v}illar fara}{they (\emph{fem.}) go astray}{\Bfootnote{emend.; \emph{þęir villir fara} ‘they (\emph{masc.}) go astray’ \Regius}} &
\ind sinna \alst{h}ęim-\alst{h}ama &
\ind sinna \alst{h}ęim-\alst{h}uga.\eva

\bvb I know the tenth, if I see \inx[G]{town-riders} \\
playing aloft: \\
I accomplish it so that they go astray \\
from their home-\inx[C]{hame}[hames]; \\
from their home-minds.\footnoteB{The \emph{riður} ‘(female) riders’ were witches who were thought to leave their hames (\emph{hamir} ‘skins, shapes’) in a form of astral projection in order to fly around in the air, tormenting villagers. Their original bodies would of course be lying in a comatose state, and with the bodies their original minds; their humanness. Weden was through his second sight able to see these riders, and could use his superior magical abilities in order to confuse them so that they were not able to return to their original hames or minds (but were instead forced to wander astray); a cruel fate. — Weden likewise brags about tricking riders in \Harbardsljod\ 20.}\evb
\evg


\bvg
\bva Þat kann’k \alst{ę}llipta, \hld\ ef skal’k til \alst{o}rrostu &
\ind \alst{l}ęiða \alst{l}ang-vini, &
und \alst{r}andir gęl’k, \hld\ en þęir með \alst{r}íki fara, &
\ind \alst{h}ęilir \alst{h}ildar til, &
\ind \alst{h}ęilir \alst{h}ildi frá, &
\ind koma þęir \alst{h}ęilir \alst{h}vaðan.\eva

\bvb I know the eleventh, if I shall into war \\
lead old friends: \\
beneath the shields I gale, and they go with power \\
healthy to the battle, \\
healthy from the battle; \\
they return healthy anywhence.\evb
\evg


\bvg
\bva Þat kann’k \alst{t}olpta, \hld\ ef sé’k á \alst{t}ré uppi &
\ind \alst{v}áfa \alst{v}irgil-ná, &
svá ek \alst{r}íst \hld\ ok í \alst{r}únum fá’k, &
\ind at sá \alst{g}ęngr \alst{g}umi. &
\ind ok \alst{m}ę́lir við \alst{m}ik.\eva

\bvb I know the twelfth, if I see high up on a tree \\
a gallow-corpse dangling: \\
so I carve and paint in the runes, \\
that that man walks \\
and speaks with me.\evb
\evg


\bvg
\bva Þat kann’k \alst{þ}rettánda \hld\ ef skal’k \alst{þ}egn ungan &
\ind \alst{v}erpa \alst{v}atni á, &
mun-at hann \alst{f}alla \hld\ þótt í \alst{f}olk komi, &
\ind \alst{h}nígr-a sá \alst{h}alr fyr \alst{h}jǫrum.\eva

\bvb I know the thirteenth, if I shall upon a young thane \\
throw water:\footnoteB{Describing the Heathen ritual of pouring water on a newborn child. Cf. \Rigsthula\ 7, 21, 34.} he will not fall though he should come into battle; \\
that warrior sinks not down before swords.\evb
\evg


\bvg
\bva Þat kann’k \alst{f}jórtánda, \hld\ ef skal’k \alst{f}yrða liði &
\ind \alst{t}ęlja \alst{t}íva fyr, &
\alst{á}sa ok \alst{a}lfa \hld\ ek kann \alst{a}llra skil, &
\ind fár kann ó·\alst{s}notr \alst{s}vá.\eva

\bvb I know the fourteenth, if before a retinue of men \\
I shall count forth the Tews: \\
of all the Eese and Elves I know the discernments;\footnoteB{Cf. \Hymiskvida\ 38, where the corresponding verb \emph{skilja} is used in the context of god-knowledge.} \\
few unwise men can do so.\evb
\evg


\bvg
\bva\alst{Þ}at kann’k fimtánda, \hld\ es gól \alst{Þ}jóð-rǿrir &
\ind \alst{d}vergr fyr \alst{D}ęllings \alst{d}urum, &
\alst{a}fl gól \alst{ǫ́}sum, \hld\ en \alst{ǫ}lfum frama, &
\ind \alst{h}yggju \alst{H}ropta-týi.\eva

\bvb I know the fifteenth, which Thedrearer galed, \\
the dwarf, before Delling’s doors. \\
Power he galed for the Eese, but for the Elves distinction; \\
thought for Roft-Tew \name{= Weden}.\evb
\evg


\bvg
\bva Þat kann’k \alst{s}extánda, \hld\ ef vil’k hins \alst{s}vinna mans &
\ind hafa \alst{g}ęð allt ok \alst{g}aman, &
\alst{h}ugi \alst{h}vęrfi’k \hld\ \alst{h}vit-armri konu &
\ind ok \alst{s}ný’k hęnnar ǫllum \alst{s}efa.\eva

\bvb I know the sixteenth, if I will from the wise girl \\
have her senses all, and pleasure; \\
the heart I change of the white-armed woman, \\
and I twist all her mind.\evb
\evg


\bvg
\bva Þat kann’k \alst{s}jautjánda \hld\ at mik \alst{s}ęint mun firrask &
\ind hit \alst{m}an-unga \alst{m}an.\eva

\bvb I know the seventeenth, that the girl-young girl \\
will lately shun me.\evb
\evg


\bvg
\bva\alst{L}jóða þessa \hld\ munt \alst{L}oddfáfnir &
\ind lengi \alst{v}anr \alst{v}esa; &
\ind þó sé þér \alst{g}óð ef \alst{g}etr, &
\ind \alst{n}ýt ef \alst{n}emr, &
\ind \alst{þ}ǫrf ef \alst{þ}iggr.\eva

\bvb These leeds wilt thou, Loddfathomer, \\
long be lacking! \\
Though they would be good for thee if thou get [them], \\
useful if thou learn [them], \\
needful if thou receive [them].\evb
\evg


\bvg
\bva Þat kann’k \alst{á}tjánda, \hld\ es \alst{ę́}va kęnni’k &
\ind \alst{m}ęy né \alst{m}anns konu, &
—\alst{a}lt es bętra \hld\ es \alst{ęi}nn of kann, &
\ind þat fylgir \alst{l}jóða \alst{l}okum— &
nema þęiri \alst{ęi}nni, \hld\ es mik \alst{a}rmi vęrr, &
\ind eða mín \alst{s}ystir \alst{s}éi.\eva

\bvb I know the eighteenth, which I never teach \\
a maiden nor man’s woman— \\
everything is better when one alone can do it; \\
that follows the end of the leeds— \\
save for her alone who holds me in her arm,\footnoteB{This expression is also used \Volundarkvida\ 2. — The one who wraps Weden in her arm may be his wife, Frie.  He has no known sister.} \\
or who is my sister.\evb
\evg


\bvg
\bva Nú eru \alst{H}áva mǫ́l kveðin \hld\ \alst{H}áva \alst{h}ǫllu í; &
\ind \alst{a}ll-þǫrf \alst{ý}ta sonum, &
\ind \alst{ó}·þǫrf \edtrans{\alst{jǫ}tna}{ettins}{\Afootnote{corrected in margin from earlier \emph{ýta} ‘men’ \Regius}} sonum; &
hęill sá’s \alst{k}vað, \hld\ hęill sá’s \alst{k}ann, &
\ind \alst{n}jóti sá’s \alst{n}am, &
\ind \alst{h}ęilir þęir’s \alst{h}lýddu.\eva

\bvb Now are the High One’s speeches sung, in the High One’s hall; \\
of great need for the sons of men, \\
of harm for the sons of ettins. \\
Hail he who sang; hail he who knows; \\
may he use who learned; \\
hail those who heeded!\evb
\evg
% Weden
	\bookStart{Speeches of Grimner}[Grímnismǫ́l]

\begin{flushright}%
\textbf{Dating} \parencite{Sapp2022}: C10th (0.976)

\textbf{Meter:} \Ljodahattr, \Fornyrdislag\ (2/3–4, 28/3–5, 45/3–5, 48/4, 49/1–2, 53), \Galdralag\ (46)%
\end{flushright}

\section{Introduction}

The \textbf{Speeches of Grimner} (\Grimnismal) are preserved whole in both \Regius\ and \AM.

The poem itself is enclosed by prose passages.  It is hard to say for how long these have accompanied the poem, but since they are found in both \Regius\ and \AM\ they must go back to a now-lost archetypal manuscript.  Together with sts. 1–3 and 53–55 of the poem they form a narrative frame for the gnomic stanzas.  The gnomic sts. themselves, the bulk of the poem, are mythological and sometimes obscure.  They align closely with other Eddic gnomic poems like \Havamal, \Vafthrudnismal, \Sigrdrifumal, and \Allvismal.

Weden begins by listing the individual dwellings of the gods (4–17).  The locations are numbered, but a few facts speak to these numbers being a later insert:

\begin{enumerate}
  \item The alliteration is never reliant on the numbers; if one compares the numbered questions in \Vafthrudnismal\ 20–42 the difference is striking.
  \item The numbering is inconsistent; Thunder’s realm (st. 4) is not counted, and Wider’s land (st. 17) has no numeral (perhaps since the form of the stanza would not allow it.)
  \item In sts. 11–15 cited in \Gylfaginning, the numbers are missing.
\end{enumerate}

%This section has been discussed in detail by \textcite{deVries1952} TODO! who considers it corrupt. Specifically, he sees the second half of v. 4 as a later insert, since it does not elaborate on the “holy land” mentioned in the first half. \textcite{Jackson1995} argues convincingly against this, showing how the first half serves as a generalized introduction to the list; the holy land is the dwelling-places of the gods.)

After this list come several sts relating to Weden and his hall, Walhall (18–23). Mentioned are the preparation of food in Walhall (18), Weden’s wolves (19) and ravens (20), the river through which the dead have to wade (21) and the gate through which they have to pass (22), the count of doors in Walhall (23), the count of doors in Thunder’s hall Bilshirner (24), and two animals which stand on the hall and gnaw on the branches of the tree Leered (25–26). From the latter animal’s—the stag Oakthirner’s—horns droplets fall into Wharyelmer, which is the origin of all rivers (26).

This introduces a list of mythic rivers (27–28), ending with the waters through which Thunder must wade on his way to Ugdrassle (29). This leads to a list of the horses ridden by the other gods on their way to Ugdrassle (31) which is followed by a description of the roots of Ugdrassle (31), then its animals (32–36) the Walkirries (37), and beings associated with the sun and moon (38–40), the things created from Yimer’s body (41–42) with a digression on the significance of the \inx[P]{bloot} for men in the present (43, see note there!), the creation of the ship Shidebladner (44) and finally a list of the noblest of several categories of things and groups (45).

After these lists Weden utters an unclear st. invoking the gods (46), before listing many of his names and the circumstances in which they were used (47–50). He then turns to Garfrith, disappointed by the inhospitality and poor conduct of his former protégé, and predicts his imminent death (51–53). He finally reveals himself by his true name, daring Garfrith to face him (53). After this he repeats several of his names (54), and the poem ends.

In the final prose section we are told that Garfrith, after learning that he was torturing Weden, hurried up to take the god away from the fires, but tripped and fell on his sword and died. After this his son Ayner ruled for a long time.

\sectionline

\section{From the sons of king Reading (\emph{Frá sonum Hrauðungs konungs})}

\bpg\bpa\mssnote{\Regius~8v/31, \AM~3v/23}%
Hrauðungr konungr átti tvá sonu. Hét annarr Agnarr, enn annarr Geirrøðr.
Agnarr var tíu vetra enn Geirrøðr átta vetra. Þeir reru tveir á báti með dorgar sínar at smá-fiski.
Vindr rak þá í haf út. Í nátt-myrkri brutu þeir við land ok gingu upp; fundu kot-bónda einn.
Þar vǫ́ru þeir um vetrinn. Kerling fostraði Agnar, enn karl Geirrøð.
At vári fekk karl þeim skip. Enn er þau kerling leiddu þá til strandar, þá mę́lti karl ein-mę́li við Geirrøð.
Þeir fengu byr ok kvǫ́mu til stǫðva fǫður síns. Geirrøðr var fram í skipi.
Hann hljóp upp á land enn hratt út skipinu, ok mę́lti: „Far þú þar er smyl hafi þik.“
Skipit rak út. Enn Geirrøðr gekk út til bǿjar; hánum var vel fagnat; þá var faðir hans andaðr.
Var þá Geirrøðr til konungs tekinn, ok varð maðr ágę́tr.\epa

\bpb King Reading had two sons. One was called Ayner, and the other Garfrith.
Ayner was ten winters old, but Garfrith eight winters. The two were rowing in a boat with their trolling-lines for small fishing.
The wind drove them out into the sea. In the dark of night they crashed onto land and walked ashore; they found a lone cottage farmer.
There they stayed over the winter. The wife fostered Ayner, but the husband Garfrith.\footnoteB{The wife was Frie, and the husband Weden; this is clarified by the following prose. The motif of Weden preferring the youngest brother is also found in \Rigsthula.}
In the spring the husband gave them ships, but when he and his wife led them to the shore, the husband spoke privately with Garfrith.\footnoteB{Surely instructing him to push his brother out to sea.}
They caught good wind, and came to their father’s harbour. Garfrith was in the front of the ship.
He leapt onto land and pushed out the ship, and spoke: ”Go thou whither the fiends may have thee!”
The ship drove out. But Garfrith walked towards the farm; he was welcomed well; by then was his father ended.
Garfrith was then taken as king, and became an excellent man.\epb\epg


\bpg\bpa\mssnote{\Regius~9r/10, \AM~4r/3}%
Óðinn ok Frigg sátu í Hliðskjǫlfu ok sá um heima alla.
Óðinn mę́lti: „Sér þú Agnar fóstra þinn, hvar hann elr bǫrn við gýgi í hellinum?
En Geirrøðr, fóstri minn, er konungr ok sitr nú at landi.“
Frigg segir: „Hann er mat-níðingr sá at hann kvelr gesti sína ef hánum þykkja of-margir koma.“
Óðinn segir at þat er in mesta lygi. Þau veðja um þetta mál.
Frigg sendi eskis-mey sína, Fullu, til Geirrøðar. Hon bað konung varask at eigi fyr-gerði hánum fjǫl-kunnigr maðr sá er þar var kominn í land, ok sagði þat mark á at engi hundr var svá ólmr at á hann myndi hlaupa.
En þat var inn mesti hé-gómi at Geirrøðr vę́ri eigi mat-góðr ok þó lę́tr hann hand-taka þann mann er eigi vildu hundar á ráða.
Sá var í feldi blám ok nefndisk Grímnir ok sagði ekki fleira frá sér þótt hann vę́ri at spurðr.
Konungr lét hann pína til sagna ok setja milli elda tveggja ok sat hann þar átta nę́tr.
Geirrøðr konungr átti son tíu vetra gamlan ok hét Agnarr eptir bróður hans.
Agnarr gekk at Grímni ok gaf hánum horn fullt at drekka, sagði at konungr gerði illa er hann lét pína hann sak-lausan.
Grímnir drakk af. Þá var eldrinn svá kominn at feldrinn brann af Grímni. Hann kvað:\epa

\bpb Weden and Frie sat in the \inx[L]{Lithshelf} and looked over all the Homes.\footnoteB{Very similar to the Longbeard Origin Myth (TODO: reference and elaborate).}
Weden spoke: “Dost thou see Ayner, thy foster-son, where he begets children with a troll-woman in her cave?\footnoteB{This may relate to Frie’s role as love-goddess. Ayner is in any case to be understood as a weak, effeminate man.}
But Garfrith, \emph{my} foster-son, is king and now rules his land.”
Frie says: “He is such a meat-nithing that he torments his guests if he thinks too many are coming!”
Weden says that this is the greatest lie; they make a wager over this matter.
Frie sent her handmaid, Full, to Garfrith’s hall. She bade the king be wary, lest he be destroyed by the \inx[C]{many-cunning} man who had come to his land; and said that his mark was that no hound was so fierce that it would rush at him.
But it was the greatest falsehood that Garfrith was not \inx[C]{good of meat}; and yet he has that man bound whom the hounds would not touch.
He was in a blue cloak and called himself Grimner, and did not tell anything more about himself, even though he was asked.
The king had him tortured that he would speak, and set him between two fires; and he sat there for eight nights.
King Garfrith had a son ten winters old, and he was called Ayner after his brother.
Ayner went up to Grimner and gave him a full horn to drink, saying that the king did badly as he had him tortured without cause.
Grimner drank it up. Then the fire had grown so much that the cloak burned on Grimner. He quoth:\epb\epg\stepcounter{prosea}

\sectionline

\section{The Speeches of Grimner}

\bvg\bva\mssnote{\Regius~9r/27, \AM~4r/17}„\alst{H}ęitr est \alst{h}ripuðr \hld\ ok \alst{h}ęldr til mikill, &
\ind gǫngumk \alst{f}irr \alst{f}uni! &
\alst{L}oði sviðnar, \hld\ þótt á \alst{l}opt bera’k; &
\ind brinnumk \alst{f}eldr \alst{f}yrir.\eva

\bvb “Hot art thou, flame, and rather too great; \\
\ind go far from me, fire! \\
The wool-cape is singed though I hold it aloft; \\
\ind the cloak burns before me!\evb\evg


\bvg\bva\mssnote{\Regius~9r/29, \AM~4r/18}%
\alst{Á}tta nę́tr \hld\ sat’k milli \alst{ę}lda hér, &
\ind svá’t mér \alst{m}ann-gi \alst{m}at né bauð &
nema \alst{ęi}nn Agnarr, \hld\ es \alst{ęi}nn skal ráða, &
\alst{G}ęirrøðar sonr, \hld\ \alst{G}otna landi.\eva

\bvb For eight nights I sat between the fires here, \\
\ind while no man offered me food, \\
save for Ayner alone, who alone shall rule— \\
Garfrith’s son—the land of the Gots!\evb\evg


\bvg\bva\mssnote{\Regius~9r/31, \AM~4r/20}%
\alst{H}ęill skalt, Agnarr, \hld\ alls \alst{h}ęilan biðr &
\ind þik \alst{V}era-týr \alst{v}esa; &
\alst{ęi}ns drykkjar \hld\ skalt \alst{a}ldri-gi &
\ind \edtrans{bętri \alst{g}jǫld}{better recompense}{\Bfootnote{Namely the mythic lore which takes up sts. 4–53.}} \alst{g}eta:\eva

\bvb Hale shalt thou be, Ayner, for hale \\
\ind does Were-Tew \name{= Weden} bid thee be! \\
For a single drink shalt thou never get \\
\ind better recompense.\evb\evg

\sectionline

\bvg\bva\mssnote{\Regius~9r/33, \AM~4r/22}%
\alst{L}and es hęilagt, \hld\ es \alst{l}iggja sé’k &
\ind \alst{ǫ́}sum ok \alst{ǫ}lfum nę́r; &
en í \alst{Þ}rúð-hęimi \hld\ skal \alst{Þ}órr vesa &
\ind \edtrans{unds of \alst{r}júfask \alst{r}ęgin}{until the Reins are ripped}{\Bfootnote{i.e. until the \inx[L]{Rakes of the Reins}.  A formulaic expression; see note to \Baldrsdraumar\ 14 for further occurrences.}}.\eva

\bvb The land is holy which lying I see \\
\ind near the \inx[F]{Eese and Elves}, \\
but in Thrithham shall Thunder dwell \\
\ind until the Reins are ripped.\evb\evg


\bvg\bva\mssnote{\Regius~9v/2, \AM~4r/23}%
\alst{Ý}-dalir hęita, \hld\ þar’s \alst{U}llr hęfir &
\ind \alst{s}ér of gǫrva \alst{s}ali; &
\alst{A}lf-hęim Fręy \hld\ gǫ́fu í \alst{á}r-daga &
\ind \alst{t}ívar at \edtrans{\alst{t}ann-féi}{tooth-gift}{\Bfootnote{The gift the child receives when he sheds his first tooth.}}.\eva

\bvb Yewdales they are called where Woulder has \\
\ind made for himself a hall. \\
Elfham to Free in days of yore \\
\ind the Tews as a tooth-gift gave.\evb\evg


\bvg\bva\mssnote{\Regius~9v/3, \AM~4r/25}%
\alst{B}ǿr es sá (hinn þriði), \hld\ es \alst{b}líð ręgin &
\ind \alst{s}ilfri þǫkðu \alst{s}ali; &
\alst{V}ala-skjǫlf hęitir, \hld\ \edtrans{es \alst{v}élti sér}{won through wiles}{\Bfootnote{Several previous editors and translators (e.g. \textcite{FinnurEdda}, \textcite{PettitEdda}, \textcite{LarringtonEdda}) have rendered this phrase with variants of “craftily made for himself”, where the verb \emph{véla} would mean ‘craftily make’.  To my knowledge this sense is never otherwise attested, and its common meaning is ‘defraud, trick, betray’.  A simpler reading would be to see this as a reference to the myth of the Ettin-smith who built the wall of Osyard.  The Gods had promised him Sun, Moon, and Frow, if he could build it in a year, but employed various tricks to hinder him.  When it at last looked like he would make it in time, Thunder slew him.  This myth is told in \Gylfaginning\ 42 and alluded to in \Voluspa\ 24–25.}} &
\ind \alst{ǫ́}ss í \alst{á}r-daga.\eva

\bvb Bower is (the third) one, where the blithe Reins \\
\ind with silver thatched a hall. \\
Waleshelf is it called which he won through wiles, \\
\ind the Os in days of yore.\evb\evg


\bvg\bva\mssnote{\Regius~9v/5, \AM~4r/26}%
\alst{S}økkva-bękkr hęitir (hinn fjórði), \hld\ en þar \alst{s}valar knegu &
\ind \alst{u}nnir glymja \alst{y}fir; &
þar þau \alst{Ó}ðinn ok Sága \hld\ drekka umb \alst{a}lla daga &
\ind \alst{g}lǫð ór \alst{g}ullnum kęrum.\eva

\bvb Sinkbench is (the fourth) one called, and there do cool \\
\ind waves clash over above; \\
there Weden and Sey drink all days, \\
\ind glad, out of golden casks.\evb\evg


\bvg\bva\mssnote{\Regius~9v/7, \AM~4r/28}%
\alst{G}laðs-hęimr hęitir (hinn fimti) \hld\ þar’s hin \alst{g}ull-bjarta &
\ind \alst{V}al-hǫll \alst{v}íð of þrumir; &
en þar \alst{H}roptr \hld\ kýss \alst{h}vęrjan dag &
\ind \alst{v}ápn-dauða \alst{v}era.\eva

\bvb Gladsham is (the fifth) one called, where the gold-bright \\
\ind Walhall wide stands fast; \\
and there Roft \name{= Weden} chooses every day \\
\ind weapon-dead warriors.\footnoteB{Cf. st. 14.}\evb\evg

\sectionline

In \AM\ the order of the following two sts. is reversed.

\sectionline

\bvg\bva\mssnote{\Regius~9v/9, \AM~4r/31}%
Mjǫk ’s \alst{au}ð-kęnnt \hld\ þęim’s til \alst{Ó}ðins koma &
\ind \edtext{\alst{s}al-kynni at \alst{s}éa}{\Afootnote{\emph{‘sia at sia’} \AM}}, &
\alst{v}argr hangir \hld\ fyr \alst{v}estan dyrr &
\ind ok drúpir \alst{ǫ}rn \alst{y}fir.\eva

\bvb Very easily recognized, for those who come to Weden, \\
\ind is the hall to see: \\
A wolf hangs before the western door, \\
\ind and an eagle droops over.\footnoteB{Something very similar is found in Widukind’s History of the Saxons, book 1:12.  The Saxons have just conquered a fortress, and \emph{mane [...] facto ad orientalem portam ponunt aquilam, aramque victoriae construentes secundum errorem paternum sacra sua propria veneratione venerati sunt} ‘at the coming of morning they set an eagle at the eastern gate, and, building an altar of victory, they worshipped it with their own holy worship in accordance with their ancestral error.’  The altar was pledged to \inx[P]{Ermin}, whom the author identifies with Mars or Hermes, but who is surely Weden.

According to \textcite{HyltenCavallius1863}[156] it was custom in Wärend, southern Sweden to hang the bodies of killed wolves high up in old oaks, and killed birds of prey above the stable-door.}\evb\evg%TODO: bibliography for both works


\bvg\bva\mssnote{\Regius~9v/10, \AM~4r/30}%
Mjǫk ’s \alst{au}ð-kęnnt \hld\ þęim’s til \alst{Ó}ðins koma &
\ind \alst{s}al-kynni at \alst{s}éa, &
\edtrans{\alst{sk}ǫptum}{shafts}{\Bfootnote{Spear-shafts.}} ’s rann rępt, \hld\ \alst{sk}jǫldum ’s salr þakiðr, &
\ind \alst{b}rynjum of \alst{b}ękki stráat.\eva

\bvb Very easily recognized, for those who come to Weden, \\
\ind is the hall to see: \\
With shafts is the house roofed, with shields is the hall thatched; \\
\ind with byrnies the benches strewn.\evb\evg


\bvg%TODO: All variants are not yet noted.
\bva\mssnote{\Regius~9v/12, \AM~4v/2, \GylfMS}%
\alst{Þ}rym-hęimr hęitir \edtrans{(hinn sétti)}{the sixth}{\Afootnote{om. \GylfMS}}, \hld\ \edtrans{es}{where}{\Afootnote{\emph{þar nú} ‘where now’}} \alst{Þ}jatsi \edtrans{bjó}{dwelled}{\Afootnote{om. \Wormianus; \emph{býr} ‘dwells’ \Upsaliensis}}, &
\ind sá hinn \edtrans{\edtext{\alst{á}m-átki}{\Afootnote{\emph{mátki} \Upsaliensis}} \alst{jǫ}tunn}{uncanny ettin}{\Bfootnote{Formulaic. See note to \Voluspa\ 8.}}; &
en nú \alst{Sk}aði byggvir, \hld\ \alst{sk}ír brúðr \edtrans{goða}{of the Gods}{\Afootnote{\emph{guma} ‘of men’ \Upsaliensis}}, &
\ind \alst{f}ornar toptir \alst{f}ǫður.\eva

\bvb Thrimham is (the sixth) one called, where Thedse dwelled, \\
\ind that uncanny ettin; \\
but now Shede bedwells—the pure bride of the Gods— \\
\ind the ancient plots of her father.\evb\evg


\bvg\bva\mssnote{\Regius~9v/14, \AM~4v/3, \GylfMS}%
\alst{B}ręiða-\alst{b}lik \edtrans{eru (hin sjaundu)}{are (the seventh)}{\Bfootnote{\emph{hęita} ‘[they] are called’ \GylfMS.}}, \hld\ en þar \alst{B}aldr hęfir &
\ind \alst{s}ér of gǫrva \alst{s}ali, &
á því \alst{l}andi \hld\ es \alst{l}iggja vęit’k &
\ind \alst{f}ę́sta \edtrans{\alst{f}ęikn-stafi}{wicked deeds}{\Bfootnote{Lit. ‘staves of wickedness’, where ‘stave’ originally means something like ‘word, speech’.  Cf. \Beowulf\ 1018b: \emph{fâcen-stafas}, referring to treacherous intrigues among the \inx[G]{Shieldings}.}}.\eva

\bvb Broadblicks are (the seventh), and there Balder has \\
\ind made for himself a hall, \\
on that land where I know lying \\
\ind the fewest wicked deeds.\evb\evg


\bvg\bva\mssnote{\Regius~9v/16, \AM~4v/5, \GylfMS}%
\alst{H}imin-bjǫrg \edtrans{eru (hin ǫ́ttu)}{are (the eighth)}{\Bfootnote{\emph{hęita} ‘[they] are called’ \GylfMS.}}, \hld\ en þar \alst{H}ęim-dall &
\ind kveða \alst{v}alda \alst{v}éum; &
þar \edtrans{\alst{v}ǫrðr goða}{Watchman of the Gods}{\Bfootnote{Formulaic epithet of Homedal, also occurring in \Lokasenna\ 49 and possibly in \Skirnismal\ 28: \emph{vǫrðr með goðum} ‘the Watchman among the Gods’.  \Gylfaginning\ 27, where the present stanza is cited, gives some further details: \emph{Hann býr þar er heitir Himinbjǫrg við Bifrǫst. Hann er vǫrðr goða ok sitr þar við himins enda at gę́ta brúarinnar fyrir berg-risum. Hann þarf minna svefn en fugl. Hann sér jafnt nótt sem dag hundrað rasta frá sér; hann heyrir ok þat, er gras vex á jǫrðu eða ull á sauðum, ok allt þat er hę́ra lę́tr.} ‘He lives at the place called the Heavenbarrows near Bivrest. He \ken*{= Homedal} is the Watchman of the Gods and sits there at Heaven’s end to guard the bridge against barrow-risers.  He needs less sleep than a bird.  Both night and day he sees a hundred rests away from him; he also hear when grass grows on the ground or wool on sheep, and everything which sounds louder.’}} \hld\ drekkr í \alst{v}ę́ru ranni &
\ind \alst{g}laðr \edtext{hinn}{\Afootnote{so \AM\GylfMS; om. \Regius}} \alst{g}óða mjǫð.\eva

\bvb Heavenbarrows are (the eighth), and there Homedal, \\
\ind they say, wields over wighs. \\
There the Watchman of the Gods \ken*{= Homedal} drinks in the tranquil house, \\
\ind glad, the good mead.\evb\evg


\bvg\bva\mssnote{\Regius~9v/17, \AM~4v/6, \GylfMS}%
\alst{F}olk-vangr \edtrans{es (hinn níundi)}{is (the ninth)}{\Bfootnote{\emph{hęitir} ‘[one] is called’ \GylfMS}}, \hld\ en þar \alst{F}ręyja rę́ðr &
\ind \alst{s}essa kostum í \alst{s}al; &
\alst{h}alfan val \hld\ hon kýss \alst{h}vęrjan dag, &
\ind en halfan \alst{Ó}ðinn \alst{á}.\eva

\bvb Folkwong is (the ninth), and there Frow decides \\
\ind the choice of seats in the hall; \\
half the slain she chooses each day, \\
\ind but half does Weden own.\footnoteB{This st. is cited and closely paraphrased in \Gylfaginning\ 24. — The roots of \emph{kjósa val} ‘choose the slain’ are the same as those in \inx[C]{walkirrie} (\emph{val-kyrja} ‘chooser of the slain’), and as Frow is a prominent goddess this would surely make her the chief walkirrie.
This is paralleled by \Sorlathattr, where Frow assumes the name \inx[C]{Gandle} (\emph{Gǫndul}, a name attested in several lists of walkirries; see \Voluspa\ 30 and Notes) and incites the legendary never-ending Conflict of the Headnings (\emph{Hjaðningavíg}).
In spite of this parallel, there are good reasons to believe that the chief walkirrie was \inx[C]{Frie}, Weden’s wife.
First, one of the functions of the walkirries is to bear ale to the Oneharriers (\Grimnismal\ 37). This mirrors royal Germanic banquets attested in heroic poetry, where the host’s wife or daughter would pour ale to his retainers and guests (the so-called ‘lady with a mead cup’ ritual; see \textcite{Enright1996} and \textcite{Riseley2014}). As Weden’s wife, we would expect Frie to have this role.
Second, at Balder’s funeral as attested in \Gylfaginning\ (TODO. chapter number), Weden rides with Frie and the Walkirries, while Frow rides alone with her cats. If she were chief walkirrie, it is rather strange that she should not ride with them.
Third, there are two separate myths where Frie and Weden contend over the fates of armies and men. These are the prose introduction to the present poem and the Longbeard origin myth (for which see Introduction to the present poem).}\evb\evg


\bvg\bva\mssnote{\Regius~9v/19, \AM~4v/8, \GylfMS}%
\alst{G}litnir \edtrans{es (hinn tíundi)}{is (the tenth)}{\Bfootnote{\emph{hęitir salr} ‘a hall is called’ \GylfMS}}, \hld\ hann ’s \alst{g}ulli studdr &
\ind ok \alst{s}ilfri þakðr it \alst{s}ama; &
en þar \alst{F}or-seti \hld\ byggir \alst{f}lęstan dag &
\ind ok \alst{s}vę́fir allar \alst{s}akir.\eva

\bvb Glitner is (the tenth): it is supported by gold, \\
\ind and thatched with silver likewise. \\
And there Foresitter dwells for most of the day, \\
\ind and puts all disputes to sleep.\evb\evg


\bvg\bva\mssnote{\Regius~9v/21, \AM~4v/9}%
\alst{N}óa-tún eru (hin ęlliptu), \hld\ en þar \alst{N}jǫrðr hęfir &
\ind \alst{s}ér of gǫrva \alst{s}ali; &
\edtrans{\alst{m}anna þęngill \hld\ hinn \alst{m}ęins-vani}{The lord of men, the guileless one}{\Bfootnote{Interesting epithets probably relating to Nearth’s roles in upholding the bounty of the land and the law.  Cf. my article on pre-Christian oaths (TODO).}} &
\ind \edtrans{\alst{h}ǫ́-timbruðum \alst{h}ǫrgi rę́ðr}{rules the harrow timbered on high}{\Bfootnote{The rare verb \emph{hǫ́-timbra} ‘timber on high’ otherwise only occurs in \Voluspa\ 7, likewise in connection with the \emph{hǫrgr} ‘harrow’.  The harrow is an outdoors holy place; see Index.  Cf. also \Vafthrudnismal\ 38 where Nearth is said to rule a great many hoves and harrows.}}.\eva

\bvb Nowetowns are (the eleventh), and there Nearth has \\
\ind made for himself a hall. \\
The lord of men, the guileless one, \\
\ind rules the \inx[C]{harrow} timbered on high.\evb\evg


\bvg\bva\mssnote{\Regius~9v/23, \AM~4v/11}%
\edtrans{\alst{H}rísi vęx \hld\ ok \alst{h}ǫ́u grasi}{with brushwood grows, and with tall grass,}{\Bfootnote{Identical to \Havamal\ 119/6.}} &
\ind \alst{V}íðars land, \alst{v}iði, &
en þar \alst{m}ǫgr of lę́tsk \hld\ af \alst{m}ars baki &
\ind \alst{f}rǿkn at hęfna \alst{f}ǫður.\eva

\bvb With brushwood grows, and with tall grass, \\
\ind \inx[P]{Wider}’s land, with wood, \\
and there the lad vows from the back of his steed, \\
\ind brave, to avenge his father.\footnoteB{At the Rakes of the Reins Wider avenges His father, Weden.  See \Voluspa\ 54–55, \Vafthrudnismal\ 53.}\evb\evg


\bvg\bva\mssnote{\Regius~9v/24, \AM~4v/12, \GylfMS}%
\alst{A}nd-hrímnir \hld\ lę́tr í \alst{Ę}ld-hrímni &
\ind \alst{S}ę́-hrímni \alst{s}oðinn, &
\alst{f}lęska bętst, \hld\ en þat \alst{f}áir vitu, &
\ind við hvat \alst{ę}in-hęrjar \alst{a}lask.\eva

\bvb Andrimner lets Sowrimner \\
\ind in Eldrimner be boiled. \\
The best of meats, but few know this: \\
\ind by what the \inx[G]{Oneharriers} are nourished.\footnoteB{The cook Andrimner ‘face-sooty’ cooks the boar Sowrimner ‘sow-sooty’ in the cauldron Eldrimner ‘fire-sooty’; by this meat are the Oneharriers nouished.}\evb\evg


\bvg\bva\mssnote{\Regius~9v/26, \AM~4v/14, \GylfMS}%
\edtext{\alst{G}era ok Freka \hld\ sęðr \alst{g}unn-tamiðr, &
\ind \alst{h}róðigr \alst{H}ęrjafǫðr, &
en við \alst{v}ín ęitt \hld\ \alst{v}ápn-gǫfugr &
\ind \alst{Ó}ðinn \alst{ę́} lifir.}{\lemma{Gera \dots\ lifir ‘Gar \dots\ live’}\Bfootnote{With what Weden feeds his two hounds it is not said, but it is most likely with the corpses of dead warriors.  The wine on which he subsists may perhaps be identified with drink offerings.  Cf. the 7th century \emph{vita} of Saint Columban (TODO: cite source), describing a rite of the Swabians: \emph{Quo cum moraretur, et inter habitatores loci illius progrederetur, reperit eos sacrificium profanum litare velle, vasque magnum, quod vulgo cupam vocant, quod viginti et sex modios amplius minusve capiebat, cervisia plenum in medio habebant positum. Ad quod vir Dei accessit, et sciscitatur quid de illo fieri vellent. Illi aiunt Deo suo Vodano, quem Mercurium vocant alii, se velle litare.} ‘While he was satying there and going about the dwellers of that place, he found out that they were going to offer a profane sacrifice, and a large cask called a \emph{cupa}, which held about twenty-six measures, was filled with beer and set in their midst.  When the man of God asked what they wanted to do with it, they answered that they were wanted to offer to their God Wodan, whom others call Mercury.’}}\eva

\bvb \inx[P]{Gar and Freak} does the battle-accustomed \\
\ind glorious Father of Hosts \name{= Weden} feed; \\
but on wine alone, esteemed of weapons, \\
\ind Weden ever lives.\evb\evg


\bvg\bva\mssnote{\Regius~9v/28, \AM~4v/15, \GylfMS}%
\alst{H}uginn ok Muninn \hld\ fljúga \alst{h}vęrjan dag &
\ind \edtrans{\alst{jǫ}rmun-grund}{ermin-ground}{\Bfootnote{i.e. ‘the immense ground’ (for the rare prefix \inx[C]{ermin-} see Index), denoting the earth as a vast flat expanse of land. This compound also occurs in a kenning in the st. on the late C10th Karlevi stone (Öl 1) referring to the unbounded sea as \emph{Ęndils jǫrmungrund} ‘Andle’s ermin-ground’ (Andle being a known “sea-king”), and in \Beowulf\ 859 as \emph{eormen-grund} carrying the same sense.}} \alst{y}fir; &
\alst{ó}umk of Hugin, \hld\ at \alst{a}ptr né komi-t; &
\ind þó séumk \alst{m}ęir of \alst{M}unin.\eva

\bvb Highen and Minden fly every day \\
\ind over the ermin-ground \ken{earth}. \\
I worry for Highen, that he might not come back, \\
\ind yet I fear more for Minden.\evb\evg


\bvg\bva\mssnote{\Regius~9v/30, \AM~4v/17}%
\alst{Þ}ýtr \alst{Þ}und, \hld\ unir \edtext{\alst{Þ}jóð-vitnis &
\ind fiskr}{\lemma{Þjóðvitnis fiskr ‘Thedwitner’s fish’}\Bfootnote{\emph{Þjóðvitnir} is easily analyzed as \emph{þjóð-} ‘great, main’ + \emph{vitnir} ‘wolf’.  The great wolf is naturally the \inx[P]{Fenrerswolf}, the brother of the Middenyardswyrm.  That the Wyrm can be called a fish is shown by \Hymiskvida\ 24.}} flóði í; &
\alst{á}ar-straumr \hld\ þykkir \alst{o}f-mikill &
\ind \alst{v}al-glaumi at \alst{v}aða.\eva

\bvb \inx[P]{Thound} roars; Thedwitner’s fish \\
\ind thrives in the flood. \\
The river-stream seems far too great \\
\ind for the noisy slain host to wade.\footnoteB{A difficult stanza.  Thound may be the river surrounding Walhall, which the dead have to pass over to reach it.  The stanza may also be referring to the punishment of criminals in waters; see note to \Voluspa\ 38 for discussion on that.}\evb\evg


\bvg\bva\mssnote{\Regius~9v/32, \AM~4v/18}\edtrans{\alst{V}al-grind}{Walgrind}{\Bfootnote{‘Slain-gate;’ the gate standing before Walhall.}} hęitir \hld\ es stęndr \alst{v}ęlli á &
\ind \alst{h}ęilǫg fyr \alst{h}ęlgum durum; &
\alst{f}orn ’s sú grind, \hld\ en þat \alst{f}áir vitu, &
\ind hvé hǫ́n ’s í \alst{l}ás of \alst{l}okin.\eva

\bvb \inx[L]{Walgrind} ’tis called, which stands on the plain, \\
\ind holy, before the holy doors. \\
Old is that gate, but few know this: \\
\ind how its lock is locked.\evb\evg


\bvg\bva\mssnote{\Regius~9v/34, \AM~4v/22}%
\alst{F}imm hundruð golfa \hld\ ok umb \alst{f}jórum tøgum &
\ind svá hygg’k \alst{B}il-skirni með \alst{b}ugum; &
\alst{r}anna þęira, \hld\ es \alst{r}ępt vita’k, &
\ind \alst{m}íns vęit’k męst \alst{m}agar.\eva

\bvb With five hundred floors, and around fourty, \\
\ind so I judge \inx[L]{Bilshirner} altogether. \\
Of those houses which I might know rafted \\
\ind I know my lad’s \ken*{= Thunder} to be the greatest.\evb\evg


\bvg\bva\mssnote{\Regius~10r/2, \AM~4v/20}%
\alst{F}imm hundruð dura \hld\ ok umb \alst{f}jórum tøgum, &
\ind svá hygg at \alst{V}alhǫllu \alst{v}esa; &
\edtrans{\alst{á}tta hundruð}{eight hundred}{\Bfootnote{The hundred is probably here the long hundred (120, rather than 100), which gives a sum of \(640 * 960 = 614~400\) Oneharriers.}} \alst{Ę}in-hęrja \hld\ ganga ór \alst{ęi}num durum, &
\ind þá’s fara við \alst{v}itni at \alst{v}ega.\eva

\bvb Five hundred doors, and around fourty, \\
\ind so I judge there to be on Walhall. \\
Eight hundred \inx[G]{Oneharriers} go out of one door, \\
\ind when to fight with the wolf they go.\evb\evg


\bvg\bva\mssnote{\Regius~10r/4, \AM~4v/24}%
\alst{H}ęið-rún hęitir gęit, \hld\ es stęndr \edtrans{\alst{h}ǫllu á Hęrja-fǫðrs}{on the hall of the Father of Hosts}{\Bfootnote{The hall of Weden, i.e. Walhall.  \emph{Hęrja-fǫðrs} looks like an unmetrical addition.}} &
\ind ok bítr af \alst{L}ę́-raðs \alst{l}imum; &
\edtrans{\alst{sk}ap-kęr}{shape-vats}{\Bfootnote{According to \CV\ the central beer-vat, from which drinks were poured into smaller vessels.}} fylla \hld\ skal \edtrans{hins \alst{sk}íra mjaðar}{the pure mead}{\Bfootnote{The mead is the goat’s milk.}}, &
\ind kná-at sú \alst{v}ęig \alst{v}anask.\eva

\bvb Heathrune is the goat called which stands on the hall of the Father of Hosts, \\
\ind and bites off Leered’s branches. \\
The shape-vats shall she fill with the pure mead; \\
\ind those draughts cannot wane.\evb\evg


\bvg\bva\mssnote{\Regius~10r/6, \AM~4v/26}Ęik-þyrnir hęitir \alst{h}jǫrtr \hld\ es stęndr \alst{h}ǫllu á Hęrja-fǫðrs &
\ind ok bítr af \alst{L}ę́-raðs \alst{l}imum; &
en af hans \alst{h}ornum \hld\ drýpr í \alst{H}ver-gęlmi &
\ind þaðan ęiga \alst{v}ǫtn ǫll \alst{v}ega:\eva

\bvb Oakthirner is called the stag who stands on the hall of the Father of Hosts, \\
\ind and bites off Leered’s branches. \\
And from his horns [drops] drip into Wharyelmer; \\
\ind thence have all waters their ways:\evb\evg


\bvg\bva\mssnote{\Regius~10r/9, \AM~4v/28}%
\alst{S}íð ok Víð, \alst{S}ę́kin ok Ęikin, \hld\ \alst{S}vǫl ok Gunn-þró, &
\ind \alst{F}jǫrm ok \alst{F}imbul-þul, &
\ind \alst{R}ín ok \alst{R}innandi, &
\alst{G}ipul ok \alst{G}ǫpul, \hld\ \alst{G}ǫmul ok \alst{G}ęir-vimul, &
\ind þę́r \alst{h}verfa umb \alst{h}odd goða, &
\alst{Þ}yn ok Vin, \hld\ \alst{Þ}ǫll ok Hǫll, &
\ind \alst{G}rǫ́ð ok \alst{G}unn-þorin.\eva

\bvb Side and Wide, Seeken and Oaken, Swale and Guththrew, \\
\ind Ferm and Fimblethule, \\
\ind Rine and Rinnend, \\
Gipple, Gapple, Gamble and Garwimble— \\
\ind they run around the hoard of the Gods \ken*{= Osyard}— \\
Thin and Win, Thall and Hall, \\
\ind Gread and Guththorn.\evb\evg


\bvg\bva\mssnote{\Regius~10r/12, \AM~5r/1}%
\alst{V}ína hęitir enn, \hld\ ǫnnur \alst{V}eg-svinn, &
\ind \alst{þ}riðja \alst{Þ}jóð-numa; &
\alst{N}yt ok \alst{N}ǫt, \hld\ \alst{N}ǫnn ok \alst{H}rǫnn, &
\alst{S}líð ok \alst{H}ríð, \hld\ \alst{S}ylgr ok Ylgr, &
\alst{V}íð ok \alst{V}ǫ́n, \hld\ \alst{V}ǫnd ok Strǫnd, &
\alst{G}jǫll ok Lęiptr; \hld\ þę́r falla \alst{g}umnum nę́r &
\ind es falla til \alst{h}ęljar \alst{h}eðan. \eva

\bvb Wine is one further called, another Wayswith, \\
\ind a third Thedenumb; \\
Nit and Nat, Nan and Ran, \\
Slithe and Rithe, Sellow and Wellow, \\
Wide and Ween, Wand and Strand, \\
Yell and Laft—they fall near to men \\
\ind as they fall hence to Hell.\evb\evg


\bvg\bva\mssnote{\Regius~10r/15, \AM~5r/4, \GylfMS}%
\alst{K}ǫrmt ok Ǫrmt \hld\ ok \alst{k}ęr-laugar tvę́r &
\ind \edtrans{\alst{þ}ę́r skal \alst{Þ}órr vaða}{these shall Thunder wade}{\Bfootnote{For Thunder’s association with wading see TODO.}} &
\alst{d}ag hvęrn \hld\ es \alst{d}ǿma fęrr &
\ind at \alst{a}ski \alst{Y}gg-drasils; &
því-at \alst{ǫ́}s-brú \hld\ bręnn \alst{ǫ}ll loga &
\ind \alst{h}ęilǫg vǫtn \edtrans{\alst{h}lóa}{bellow}{\Bfootnote{A hapax. TODO.}}.\eva

\bvb Carmt and Armt, and the two Carlays, \\
\ind these shall Thunder wade \\
every day, when to judge he goes, \\
\ind at \inx[L]{Ugdrassle’s Ash}; \\
for the \inx[G]{eese}[os]-bridge \ken{rainbow} burns all with flame; \\
\ind the holy waters bellow.\evb\evg


\bvg\bva\mssnote{\Regius~10r/17, \AM~5r/6}%
\alst{G}laðr ok \alst{G}yllir, \hld\ \alst{G}lęr ok Skęið-brimir, &
\ind \alst{S}ilfrin-toppr ok \alst{S}inir, &
\alst{G}ísl ok Fal-hófnir, \hld\ \alst{G}ull-toppr ok Létt-feti, &
\ind þęim ríða \alst{ę́}sir \alst{jó}um &
\alst{d}ag hvęrn \hld\ es \alst{d}ǿma fara &
\ind at \alst{a}ski \alst{Y}gg-drasils.\eva

\bvb Glad and Gilder, Glare and Sheathbrimmer, \\
\ind Silvrentop and Sinewer; \\
Yissel and Fallowhofner, Goldtop and Lightfeet; \\
\ind on these horses ride the Eese, \\
every day, when to judge they go, \\
\ind at \inx[L]{Ugdrassle’s Ash}.\evb\evg


\bvg\bva\mssnote{\Regius~10r/20, \AM~5r/8}%
\alst{Þ}ríar rǿtr \hld\ standa á \alst{þ}ría vega &
\ind undan \alst{a}ski \alst{Y}gg-drasils; &
\alst{H}ęl býr und \alst{ęi}nni, \hld\ \alst{a}nnarri \alst{h}rím-þursar, &
\ind þriðju \alst{m}ęnnskir \alst{m}ęnn.\eva

\bvb Three roots grow on three ways, \\
\ind from beneath Ugdrassle’s Ash. \\
Hell lives enclosed by one, [by] the other the \inx[G]{Rime-Thurses}, \\
\ind {[by]} the third manly men.\evb\evg


\bvg\bva\mssnote{\Regius~10r/22, \AM~5r/9}%
\alst{R}ata-toskr hęitir íkorni \hld\ es \alst{r}inna skal &
\ind at \alst{a}ski \alst{Y}gg-drasils; &
\alst{a}rnar \alst{o}rð \hld\ hann skal \alst{o}fan bera &
\ind ok sęgja \alst{N}íð-hǫggvi \alst{n}iðr.\eva

\bvb Wratetusk is the squirrel called who shall run \\
\ind at Ugdrassle’s Ash. \\
The eagle’s words he shall carry from above, \\
\ind and say to Nithehewer below.\footnoteB{This st. and the following is paraphrased in \Gylfaginning\ 16 (excerpt):
\begin{quote}
  \emph{Þá mę́lti Gangleri: „Hvat er fleira at segja stór-merkja frá askinum?“ Hár segir: „Mart er þar af at segia. Ǫrn einn sitr í limum asksins, ok er hann margs vitandi, en í milli augna honum sitr haukr sá, er heitir Veðrfǫlnir. Íkorni sá, er heitir Rata-toskr, rennr upp ok niðr eptir askinum ok berr ǫfundar orð millum arnarins ok Níðhǫggs.} ‘Gangler spoke: “What more great marks are there to be said about the ash?” High says: “There is much to say about it. An eagle sits in the limbs of the ash, and he is much knowing, but between his eyes sits the hawk called Weatherfalner. The squirrel, which is called Wratetush, runs up and down along the ash and carries words of spite between the eagle and Nithehewer.”’
\end{quote}}\evb\evg


\bvg\bva\mssnote{\Regius~10r/23, \AM~5r/11}%
\alst{H}irtir ’ru ok fjórir \hld\ þęir’s af \alst{h}ę́fingar &
\ind á \alst{g}ag-halsir \alst{g}naga: &
\alst{D}áinn ok \alst{D}valinn, \hld\ \alst{D}ún-ęyrr ok \alst{D}ura-þrór.\eva

\bvb Harts are there also, four, those who TODO \\
\ind TODO gnaw: \\
Dowen and Dwollen, Downeer and Doorthrew.\footnoteB{Paraphrased in \Gylfaginning\ 16 immediately following a paraphrase of the last st.: \emph{En fjórir hirtir renna í limum asksins ok bíta barr; þeir heita svá: Dáinn, Dvalinn, Dún-eyrr, Dura-þrór.} ‘But four harts run in the limbs of the ash and bite its leaves; they are called thus: Dowen, Dwollen, Downeer, Doorthrew.’}\evb\evg


\bvg\bva\mssnote{\Regius~10r/25, \AM~5r/12, \GylfMS}%
\alst{O}rmar flęiri \hld\ liggja und \alst{a}ski \alst{Y}gg-drasils &
\ind an þat of \alst{h}yggi \alst{h}vęrr &
\ind \alst{ó}-sviðra \alst{a}pa:\eva

\bvb More worms lie under Ugdrassle’s Ash \\
\ind than any one would think \\
\ind among unwise \inx[C]{ape}[apes]:\footnoteB{Paraphrased in \Gylfaginning\ 16: \emph{En svá margir ormar eru í Hvergelmi með Níðhǫgg, at engi tunga má telja; svá segir hér:} ‘But so many worms are in Wharyelmer with Nithehewer that no tongue may count them. So it says here:’ after which st. 36 is quoted.}\evb\evg


\bvg\bva\mssnote{\Regius~10r/26, \AM~5r/13, \GylfMS}%
\alst{G}óinn ok Móinn, \hld\ þęir ’ru \alst{G}raf-vitnis synir, &
\ind \alst{G}rá-bakr ok \alst{G}raf-vǫlluðr, &
\alst{O}fnir ok Sváfnir, \hld\ hygg’k at \alst{ę́} skyli &
\ind \alst{m}ęiðs kvistu \alst{m}áa.\eva

\bvb Gowen and Mowen—they are Gravewitner’s sons— \\
\ind Greyback and Gravewalled; \\
Ovner and Sweefner, I ween, shall always \\
\ind injure the beam’s branches.\evb\evg


\bvg\bva\mssnote{\Regius~10r/28, \AM~5r/14}%
\alst{A}skr \alst{Y}gg-drasils \hld\ drýgir \alst{ę}rfiði &
\ind \alst{m}ęira an \alst{m}ęnn viti: &
\alst{h}jǫrtr bítr ofan \hld\ en á \alst{h}liðu fúnar, &
\ind skęrðir \alst{N}íð-hǫggr \alst{n}eðan.\eva

\bvb Ugdrassle’s Ash suffers hardship \\
\ind greater than men might know: \\
a hart bites it above and it rots on the side; \\
\ind Nithehewer harms it below.\evb\evg


\bvg\bva\mssnote{\Regius~10r/30, \AM~5r/16}%
\alst{H}rist ok Mist \hld\ vil’k at mér \alst{h}orn beri, &
\ind \alst{Sk}eggj-ǫld ok \alst{Sk}ǫgul, &
\edtrans{\alst{H}ildr ok Þrúðr}{Hild and Thrith}{\Afootnote{so \AM; \emph{Hildi ok Þrúði} \Regius\ stems from \emph{ꝺꝛ, ðꝛ} with r rotunda being interpreted and copied as \emph{ꝺı, ðr}, this becomes clear upon viewing the facsimile images.}}, \hld\ \alst{H}lǫkk ok \alst{H}ęr-fjǫtur, &
\ind \alst{G}ǫll ok \alst{G}ęir-ǫlul, &
\alst{R}and-gríð ok \alst{R}áð-gríð, \hld\ \alst{R}ęgin-lęif; &
\ind þę́r bera \alst{ęi}n-hęrjum \alst{ǫ}l.\eva

\bvb Rist and Mist I would have bearing to me a horn— \\
\ind Shageld and Shagle; \\
Hild and Thrith, Lank and Harfetter, \\
\ind Gall and Garannel, \\
Randgrith and Redegrith, Rainlaf— \\
\ind they bear the Oneharriers ale.\footnoteB{The women listed in this st. are Walkirries. Their names are known from other lists of Walkirries, but differ somewhat in form. TODO: Note these differences}\evb\evg


\bvg\bva\mssnote{\Regius~10r/32, \AM~5r/18}%
\edtrans{\alst{Á}r-vakr ok \alst{A}l-sviðr}{Yorewaker and Allswith}{\Bfootnote{These horses also appear in \Sigrdrifumal\ 15a/2; see note to the next st.}}, \hld\ skulu \alst{u}pp heðan &
\ind \alst{s}vangir \alst{s}ól draga; &
en und þęira \alst{b}ógum \hld\ fǫ́lu \alst{b}líð ręgin, &
\ind \alst{ę́}sir, \alst{í}sarn-kol.\eva

\bvb Yorewaker and Allswith shall from hence— \\
\ind slender [steeds]—pull up the sun, \\
and under their shoulders the blithe Reins hid \\
\ind —the Eese—iron-cooling.\footnoteB{According to \Gylfaginning\ 11 the gods took two horses to pull the sun’s chariot—Yorewaker and Allswith—and “under the shoulders of the horses the gods placed two wind-bellows to cool them, but in some sources (\emph{í sumum frǿðum}, presumably this st.) they are called iron-cooling (\emph{ísarn-kol}).”}\evb\evg


\bvg\bva\mssnote{\Regius~10v/2, \AM~5r/20}%
\alst{S}valinn hęitir, \hld\ hann stęndr \alst{s}ólu fyrir, &
\ind \alst{sk}jǫldr \alst{sk}ínanda goði; &
\alst{b}jǫrg ok \alst{b}rim \hld\ vęit’k at \alst{b}rinna skulu, &
\ind ef hann \alst{f}ęllr í \alst{f}rá.\eva

\bvb Swalen one is called, it stands before the sun: \\
\ind a shield [before] the shining god \ken{sun}. \\
Crags and surf I know shall burn, \\
\ind if it falls away.\footnoteB{The sun-disc was apparently thought to be a translucent shield, which protected the earth from the full power of the Sun behind it. Without it the whole world (“crags and surf”, \textsc{land} and \textsc{sea}; the totality of the earth) would burn up.  Cf. \Sigrdrifumal\ 15a/1, which mentions the “shield that stands before the shining god \ken{sun}”.}\evb\evg


\bvg\bva\mssnote{\Regius~10v/4, \AM~5r/21}%
\alst{Sk}oll hęitir ulfr, \hld\ es fylgir hinu \alst{sk}ír-lęita &
\ind goði til \alst{v}arna \alst{v}iðar, &
en annarr \alst{H}ati, \hld\ hann ’s \alst{H}róð-vitnis sonr, &
\ind sá skal fyr \alst{h}ęiða brúði \alst{h}imins.\eva

\bvb \inx[P]{Scoll} is called the wolf who follows the pure-faced \\
\ind god \ken*{= Sun} to the shelter of the woods. \\
But another is \inx[P]{Hate}, he is \inx[P]{Rothwitner}’s son— \\
\ind who shall [run] in front of the bright bride of heaven \ken*{= Sun}.\footnoteB{According to \Gylfaginning\ 12 Scoll chases the Sun and Hate chases the Moon (which is why he runs in front of the sun).  See note to \Voluspa\ 40 for discussion on these wolves.}\evb\evg


\bvg\bva\mssnote{\Regius~10v/6, \AM~5r/23, \\ \AMb~9v/14, \EddaBms~3v/11}%TODO: Critical notes for these next two stanzas based on the mss. Sigla for ms. B = AM 757 a 4°.
\edtext{Ór \alst{Y}mis holdi \hld\ vas \alst{jǫ}rð of skǫpuð, &
\ind en ór \edtrans{\alst{s}vęita}{blood}{\Afootnote{\emph{hans sára svęita} ‘blood of his wounds’ \AMb\EddaBms}\Bfootnote{For the sense, see note to this word in \Vafthrudnismal\ 21.}} \edtext{\alst{s}jór}{\Afootnote{so \AM\AMb\EddaBms; \emph{sę́r} \Regius}}, &
\alst{b}jǫrg ór \alst{b}ęinum, \hld\ \alst{b}aðmr ór hári, &
\ind en \edtrans{ór \alst{h}ausi \alst{h}iminn}{from his skull the heaven}{\Afootnote{\emph{himinn ór hausi hans} ‘the heaven from his skull’ \AMb\EddaBms}\Bfootnote{This suggests that the heavens were understood as a dome, something common among many ancients. This also fits well with the floating clouds being Yimer’s brains, as said in the following st.}}.}{\lemma{Ór \dots\ himinn ‘Out of \dots\ heaven’}\Bfootnote{This stanza is clearly related to \Vafthrudnismal\ 21, see note there.}}\eva

\bvb From \inx[P]{Yimer}’s flesh was the earth shaped, \\
\ind and from his blood the sea; \\
mountains from his bones, woods from his hair, \\
\ind and from his skull the heaven.\evb\evg


\bvg\bva\mssnote{\Regius~10v/8, \AM~5r/25, \\ \AMb~9v/16, \EddaBms~3v/12}%
\edtext{En ór hans \alst{b}rǫ́um \hld\ gørðu \alst{b}líð ręgin &
\ind \alst{M}ið-garð \alst{m}anna sonum,}{\lemma{En ór hans brǫ́um \dots\ manna sonum ‘But from his eyebrows \dots\ sons of men’}\Bfootnote{The gods fenced in Middenyard (‘the middle enclosure’) by using the hair of Yimer’s eyebrows as poles.}} &
en ór hans \alst{h}ęila \hld\ vǫ́ru þau hin \edtrans{\alst{h}arð-móðgu}{hard-minded}{\Afootnote{\emph{hríð-fęldu} ‘stormy’ \AMb\EddaBms}} &
\ind \alst{sk}ý ǫll of \alst{sk}ǫpuð.\eva

\bvb And from his eyebrows the blithe \inx[G]{Reins} made \\
\ind \inx[L]{Middenyard} for the sons of men, \\
and from his brains were the hard-minded \\
\ind clouds all shaped.\evb\evg


\bvg\bva\mssnote{\Regius~10v/9, \AM~5r/26}%
\edtext{\edtrans{\alst{U}llar}{Woulder’s}{\Bfootnote{It is uncertain why the rather obscure god Woulder is invoked here; it cannot be simply for the sake of alliteration, since \emph{Óðins} ‘Weden’s’ would work just as well.  It may be that Woulder had a particular role in the setting of the ritual fire, something supported by the large number of firesteel-shaped amulets at the archeological site of \emph{Lilla Ullevi} (‘Woulder’s little \inx[C]{wigh}’) in Sweden.  For this site see Index: \inx[P]{Woulder} and \textcite{afEdholm2009}.}} \edtrans{hylli}{holdness}{\Bfootnote{‘Favour, loyalty, grace’.  This root (from which also the adjective \emph{hollr} ‘hold; favourable, loyal, gracious’ and verb \emph{hylla} ‘to make hold’) is often to refer to godly grace in both a Heathen and Christian context.  See Index: \inx[C]{hold} and \inx[C]{holdness}.}} \hld\ hęfr ok \edtrans{\alst{a}llra goða}{All Gods}{\Bfootnote{Cf. \Sigrdrifumal\ 3–4, \Lokasenna\ 11, which both hail the Gods as a collective (the former as part of a genuine prayer, the latter subversively).  For the oneness of the Gods, see Index: \inx[C]{All Gods}.}} &
\ind hvęrr’s tękr \alst{f}yrstr á \alst{f}una, &
því-at \alst{o}pnir hęimar \hld\ verða umb \alst{á}sa sonum, &
\ind þá’s \alst{h}ęfja af \edtrans{\alst{h}vera}{kettles}{\Bfootnote{acc. pl. of \emph{hverr}, from PGmc. \emph{*hweraz}, from PIE \emph{*kʷer-} ‘pot, vessel’.  Interestingly the Sanskrit cognate \emph{carú} is occasionally used in reference to the vat wherein the ritual drink \emph{soma} is prepared (e.g. \Rigveda\ 10.167.4).}}.}{\lemma{ALL}\Bfootnote{This st. is one of the most difficult in the poem and many interpretations have been made.

The traditional explanation (e.g. \textcite{FinnurEdda}, Bellows, Sijmons and Gering (p. 208)) relates it to the poem’s frame narrative.  In this view, Weden, bound between the two fires, cryptically asks for a cauldron hanging above him to be moved so that the Gods will be able to see him through the smoke-vent and rescue him.  This explanation is strange given the stanza’s placement in the gnomic wisdom section of the poem, unless the whole section is taken to be a later insert (so Finnur), something for which there is no textual support.  The invocation of the obscure god Woulder is also left unexplained, and there is no mention of a cauldron elsewhere in the poem.

A better explanation is given by \textcite{Nordberg2005}, who argues that the stanza is another piece of gnomic wisdom, referring to the cooking of the sacrificial meal in large cauldrons during the \inx[C]{bloot}.  The st. describes the divine grace (\emph{hylli} ‘holdness’, see Note to l. 1) won by the ritualist who sets the fire onto which the cauldron is placed, since this act enables the Gods to become guests at the ritual meal.  Cf. \HakonarSaga\ 14, describing the traditional bloot in the Throndlaw (\emph{Þrǿnda-lǫg}), Norway: \emph{At veizlu þeiri skyldu allir menn ǫl eiga; þar var ok drepinn alls konar smali ok svá hross, [...] en slátr skyldi sjóða til mann-fagnaðar; eldar skyldu vera á miðju gólfi í hofinu ok þar katlar yfir.} ‘At that gathering all men should have ale; thereat was also slain every kind of small cattle and likewise horses, [...] and the fresh meat should be cooked for men to enjoy.  There should be fires in the middle of the floor in the hove and kettles above them.’

This interpretation is especially interesting when one considers the immediately preceding stanzas 41–42, which deal with the ordering of the world through the dismembering of Yimer, the primordial victim sacrificed by the Gods.  It is known from other Indo-European branches that the ritual sacrifice in the present was seen as a reenactment of the primeval sacrifice in the mythic past, which was necessary for the continued existence of the world and the social order (cf. e.g. \Rigveda\ 10.90); for discussion see \textcite{Lincoln1986}, especially the first two chapters.  If this is correct, \Grimnismal\ 41–43 would then attest this conception also in the Germanic tradition.}}\eva

\bvb \inx[P]{Woulder}’s \inx[C]{holdness} and that of \inx[C]{All Gods} \\
\ind has whoever first touches the fire, \\
for the \inx[C]{Home}[Homes] open up for the Sons of the Eese, \\
\ind when men lift off the kettles.\evb\evg


\bvg\bva\mssnote{\Regius~10v/11, \AM~5r/28}%
\alst{Í}valda synir \hld\ gingu í \alst{á}r-daga &
\ind \alst{Sk}íð-blaðni at \alst{sk}apa, &
\alst{sk}ipa batst \hld\ \alst{sk}írum Fręy, &
\ind \alst{n}ýtum \alst{N}jarðar bur.\eva

\bvb Iwald’s sons went in days of yore \\
\ind Shidebladner for to shape: \\
the best of ships for the pure Free, \\
\ind for the useful Son of Nearth.\evb\evg


\bvg\bva\mssnote{\Regius~10v/13, \AM~5r/29}%
\alst{A}skr \alst{Y}gg-drasils, \hld\ hann ’s \alst{ǿ}ðstr viða &
\ind en \alst{Sk}íð-blaðnir \alst{sk}ipa, &
\alst{Ó}ðinn \alst{á}sa \hld\ en \alst{jó}a Slęipnir, &
\alst{B}il-rǫst \alst{b}rúa \hld\ en \alst{B}ragi skalda, &
\alst{H}á-brók \alst{h}auka \hld\ en \alst{h}unda Garmr.\eva

\bvb Ugdrassle’s Ash—it is the noblest of trees, \\
\ind and Shidebladner of ships; \\
Weden of the Eese and Slapner of steeds; \\
Bilrest of bridges and Bray of scolds; \\
Highbrook of hawks and Garm of hounds.\evb\evg

\sectionline

\bvg\bva\mssnote{\Regius~10v/15, \AM~5v/2}%
\alst{S}vipum hęf’k nú ypt \hld\ fyr \alst{s}ig-tíva sonum, &
\ind við þat skal \alst{v}il-bjǫrg \alst{v}aka, &
\alst{ǫ}llum \alst{ǫ́}sum \hld\ þat skal \alst{i}nn koma &
\ind \alst{Ę́}gis bękki \alst{á} &
\ind \alst{Ę́}gis drekku \alst{a}t.\eva

\bvb My gaze have I now lifted up before the sons of the victory-Tews \ken*{= Eese}— \\
\ind by that shall the willed rescue awake! \\
All the Eese shall it bring into here, \\
\ind upon Eagre’s bench, \\
\ind at Eagre’s drinking!\footnoteB{Weden suddenly announces that he has made the other gods aware of his situation; they will leave their feasting at Eagre’s hall (see \Hymiskvida\ and \Lokasenna) and instead come to his rescue.  He then begins to recount his names.}\evb\evg


\bvg\bva\mssnote{\Regius~10v/17, \AM~5v/4}Hétumk \alst{G}rímr, \hld\ hétumk \alst{G}anglęri, &
\ind \alst{H}ęrjann ok \alst{H}jalm-beri, &
\alst{Þ}ękkr ok \alst{Þ}riði, \hld\ \alst{Þ}undr ok Uðr, &
\ind \alst{H}ęl-blindi ok \alst{H}ár.\eva

\bvb I called myself Grim, I called myself Gangler, \\
\ind Harn and Helmbearer. \\
Theck and Third, Thound and Ith, \\
\ind Hellblinder and High.\evb\evg


\bvg\bva\mssnote{\Regius~10v/19, \AM~5v/5}%
\alst{S}aðr ok \alst{S}vipall \hld\ ok \alst{S}ann-getall, &
\ind \alst{H}ęr-tęitr ok \alst{H}nikarr, &
\alst{B}il-ęygr, \alst{B}ál-ęygr, \hld\ \alst{B}ǫl-verkr, Fjǫlnir, &
\alst{G}rímr ok \alst{G}rímnir, \hld\ \alst{G}lap-sviðr ok Fjǫl-sviðr.\eva

\bvb Sooth and Swiple and Soothgettle, \\
\ind Hartote and Nicker, \\
Bileye, Baleeye, Baleworker, Fillner, \\
Grim and Grimner, Glapswith and Fellswith.\evb\evg


\bvg\bva\mssnote{\Regius~10v/21, \AM~5v/7}%
\alst{S}íð-hǫttr, \alst{S}íð-skęggr, \hld\ \alst{S}ig-fǫðr, Hnikuðr, &
\alst{A}l-fǫðr, \alst{V}al-fǫðr, \hld\ \alst{A}t-ríðr ok Farma-týr; &
\alst{ęi}nu nafni \hld\ hétumk \alst{a}ldri-gi &
\ind síðst ek með \alst{f}olkum \alst{f}ór.\eva

\bvb Sidehat, Sideshag, Syefather, Nicked, \\
Allfather, Walfather, Atrider, and Farm-Tew— \\
by just one name have I never called myself, \\
\ind since among manfolk I fared.\evb\evg


\bvg\bva\mssnote{\Regius~10v/23, \AM~5v/9}%
\alst{G}rímni mik hétu \hld\ at \alst{G}ęir-raðar, &
\ind en \alst{Ja}lk at \alst{Ǫ́}s-mundar; &
en þá \alst{K}jalar \hld\ es ek \alst{k}jalka dró, &
\ind \alst{Þ}rór \alst{þ}ingum at.\eva

\bvb Grimner they called me at Garfrith’s [home], \\
\ind but Yelk at Osmund’s, \\
but Keller whenas I drew the sled; \\
\ind Throo at \inx[C]{Thing}[Things].\footnoteB{Presumably referencing other now-lost myths involving Weden travelling in disguise. The last is possibly a reference to the name under which Weden would be invoked at the start of Things (legal assemblies, see Index).}\evb\evg


\bvg\bva\mssnote{\Regius~10v/24, \AM~5v/10}%
\alst{Ó}ski ok \alst{Ó}mi, \hld\ \alst{Ja}fn-hár ok Biflindi, &
\ind \alst{G}ǫndlir ok Hár-barðr með \alst{g}oðum.\eva

\bvb Wish and Ome, Evenhigh and Bivlend; \\
\ind Gandler and Hoarbeard among Gods.\evb\evg


\bvg\bva\mssnote{\Regius~10v/25, \AM~5v/11}%
\alst{S}viðurr ok \alst{S}viðrir \hld\ es ek hét at \alst{S}økk-mímis &
\ind ok dulða’k þann hinn \alst{a}ldna \alst{jǫ}tun &
þá’s \alst{M}ið-vitnis vas’k \hld\ ins \alst{m}ę́ra burar &
\ind \alst{o}rðinn \alst{ęi}n-bani.\eva

\bvb Swither and Swithrer, as I was called at Sink-Mimer’s, \\
\ind and I deceived that aged ettin, \\
when of Midwitner’s famous son \\
\ind I had become the lone slayer.\evb\evg


\bvg\bva\mssnote{\Regius~10v/28, \AM~5v/13}%
\alst{Ǫ}lr est Gęir-røðr, \hld\ hęfr þú \alst{o}f-drukkit; &
\alst{m}iklu est hnugginn, \hld\ es þú est \alst{m}ínu gęngi, &
\edtrans{\alst{ǫ}llum \alst{ęi}n-hęrjum}{of all the Oneharriers}{\Bfootnote{Linguistically, Garfrith is not bereft of the support of the Oneharriers but rather of the Oneharriers themselves, but the sense is the same.  By breaking the Odinic code of conduct he has lost Weden’s favour, and thus been excluded from the community of oath-bound warriors, the Oneharriers.

On the other hand a king who behaved well could expect to have the truce of the Oneharriers; this was the case for Hathkin the Good according to the poem composed about him (Eyv \emph{Hák} in \Skp\ 1).  In that poem (st. 16/1–2) \inx[P]{Bray} greets him in the hall of the Gods, saying: \emph{Ęin-hęrja grið · skalt allra hafa; / þigg þú at Ǫ́sum ǫl.} ‘All the Oneharriers’ truce shalt thou have; take ale from the \inx[G]{Eese}!’}} \hld\ ok \alst{Ó}ðins hylli.\eva

\bvb Worse for ale art thou, Garfrith; thou hast over-drunk. \\
Of much art thou bereft when thou art [bereft] of my support, \\
of all the \inx[G]{Oneharriers}, and of Weden’s \inx[C]{holdness}.\evb\evg


\bvg\bva\mssnote{\Regius~10v/30, \AM~5v/15}%
\alst{F}jǫlð þér sagða’k, \hld\ en þú \alst{f}átt of mant, &
\ind of þik \alst{v}éla \alst{v}inir; &
\edtext{\alst{m}ę́ki liggja \hld\ sé’k \edtext{\alst{m}íns vinar}{\linenum{|2--3}\lemma{vinir, míns vinar ‘friends, my friend’}\Bfootnote{Weden stresses his friendship with Garfrith by using the word \emph{vinr} ‘friend’ twice.  The followers of a god were his friends; see note to \Havamal\ 157.}} &
\ind allan í \alst{d}ręyra \alst{d}rifinn.}{\lemma{mę́ki \dots\ drifinn. ‘The sword \dots\ gore.’}\Bfootnote{Weden foretells Garfrith’s coming death.}}\eva

\bvb Much I told thee, but thou recallest little; \\
\ind ’tis friends that deal with thee! \\
The sword of my friend I see lying \\
\ind all drenched in gore.\evb\evg


\bvg\bva\mssnote{\Regius~10v/31, \AM~5v/16}%
\alst{Ę}gg-móðan val \hld\ nú mun \alst{Y}ggr hafa, &
\ind þitt vęit’k \alst{l}íf of \alst{l}iðit; &
\alst{v}arar ’ru \edtrans{dísir}{Dises}{\Bfootnote{i.e. the Norns, fates, who have determined his hour of death.  Cf. \Fafnismal\ TODO, \Hamdismal\ TODO. }}, \hld\ nú knátt \alst{Ó}ðin séa; &
\ind nálgask \alst{m}ik ef þú \alst{m}ęgir!\eva

\bvb An edge-tired corpse will Ug now have: \\
\ind I know thy life to be past. \\
Wary are the \inx[G]{Dises}, now dost thou see Weden— \\
\ind come near me, if thou mayst!\evb\evg


\bvg\bva\mssnote{\Regius~11r/2, \AM~5v/18}%
\alst{Ó}ðinn nú hęiti’k, \hld\ \alst{Y}ggr áðan hét’k, &
\ind hétumk \alst{Þ}undr fyr \alst{þ}at, &
\alst{V}akr ok Skilfingr, \hld\ \alst{V}ǫ́fuðr ok Hropta-týr &
\ind \alst{G}autr ok Jalkr með \alst{g}oðum.\eva

\bvb Weden am I called now, Ug was I called earlier, \\
\ind I called myself Thound before that; \\
Wacker and Shilving, Waved and Roft-Tew, \\
\ind Geat and Gelding among the Gods.\evb\evg


\bvg\bva\mssnote{\Regius~11r/4, \AM~5v/20}%
\alst{O}fnir ok Sváfnir \hld\ hygg’k at \alst{o}rðnir sé &
\ind \alst{a}llir at \alst{ęi}num mér.\eva

\bvb Ovner and Swebner, I ween, have arisen \\
\ind all from me alone.\evb\evg


\bpg\bpa\mssnote{\Regius~11r/5, \AM~5v/21}Geir-røðr konungr sat, ok hafði sverð um kné sér ok brugðit til miðs. En er hann heyrði, at Óðinn var þar kominn, stóð hann upp, ok vildi taka Óðin frá eldinum. Sverðit slapp ór hendi hánum; vissu hjǫltin niðr. Konungr drap fę́ti, ok steyptist á-fram, en sverðit stóð í gǫgnum hann, ok fekk \edtext{hann}{\Afootnote{þar af \AM}} bana. \edtext{Óðinn hvarf þá.}{\Afootnote{om. \AM}} En Agnarr \edtext{var þar}{\Afootnote{varð \AM}} konungr \edtext{lengi síðan.}{\Afootnote{om. \AM}}\epa

\bpb King Garfrith sat and had the sword about his knee, and it was brandished half-way up. But when he heard that Weden were come there, he stood up and would take Weden from the fire. The sword slipped out of his hand; the hilt pointed downwards. The king tripped and stooped forth, but the sword went through him, and he received his bane. Weden then disappeared, but Ayner was there king for a long while afterwards.\epb\epg
% Weden
	\bookStart{Leeds of Hoarbeard}[Hárbarðsljóð]

\begin{flushright}%
\textbf{Dating} \parencite{Sapp2022}: early C11th (0.578)–late C11th (0.377)

\textbf{Meter:} Unclear (TODO)%
\end{flushright}

\section{Introduction}

The poem can be seen as an allegory on class relations, namely between the self-owning yeomen farmers and the warlike earls, represented through their patron gods.

Of all Eddic poems \Harbardsljod\ is probably the strangest in terms of form. Verse length varies greatly, and many of the lines (see especially the final verse) are of an obscene length reminiscent of late continental Germanic poems like the Heliand; some simply have no metrical qualities at all. The young clitic definite is (uniquely) employed frequently throughout the poem. These criteria would seem to point towards a late origin for the poem (though not later than the late C13th, when \Regius\ was written).

Against this late origin speaks the presence of rare words (e.g. \emph{ǫgurr} v. 13) and a thorough understanding of the personalities of the two gods which would seem unlikely to stem from several centuries after the conversion of Iceland. The model devised by Sapp gives the poem a 57.8\% likelihood of being from the early C11th, and a 37.7\% likelihood of being from the late 11th. These scores are most similar to those obtained by \Gripisspa, a poem that on the surface seems much more archaic.

What could we then be dealing with? It may of course be that the poem is heavily corrupt, but there is no good evidence for this (apart from the above-mentioned irregularities). Most lines are readily understandable and fit well both within their respective context and the poem as a whole. I think a better solution to this problem is to assume that the poem has been acted out as a sort of carnivalesque theatre, with two masked actors, each playing one of the gods. This would explain the variations in meter and line length, and the prose; some lines were simply shouted out, and the lack of alliteration in them would then have a powerful, discordant effect.

This is shown also by uses of the word ‘here’ in sts. 9 and 14. TODO: mention concept of "double scene" by Lars Lönnroth?

\sectionline

\section{The Leed of Hoarbeard}

\bpg\bpa\mssnote{\Regius~12r/30}%
Þórr fór ór austr-vegi ok kom at sundi einu. Ǫðrum megum sundsins var ferju-karlinn með skipit. Þórr kallaði:\epa

\bpb Thunder journeyed from the Eastern Way and came to a sound. At the other side of the sound was the ferryman with the ship. Thunder called out:\epb\epg


\bvg\bva\mssnote{\Regius~12r/32}%
„Hvęrr ’s sá \alst{s}vęinn \alst{s}vęina \hld\ es stęndr fyr \alst{s}undit handan?“\eva

\bvb “Who is that swain of swains, standing here across the sound?”\evb\evg


\bvg\bva\speakernote{Hann svaraði:}\mssnote{\Regius~12v/1}%
„Hvęrr ’s sá \alst{k}arl \alst{k}arla \hld\ es \alst{k}allar of váginn?“\eva

\bvb He answered: \\
“Who is that churl of churls, calling out over the wave?”\evb\evg


\bvg\bva\mssnote{\Regius~12v/2}%
„\alst{F}ęr þú mik of sundit, \hld\ \alst{f}ǿði’k þik á morgun; &
\alst{m}ęis hęfi’k á baki, \hld\ verðr-a \alst{m}atrinn bętri. &
Át’k í \alst{h}víld \hld\ áðr ek \alst{h}ęiman fór, &
\alst{s}íldr ok \edtrans{hafra}{oatmeal/he-goats}{\Bfootnote{The easiest reading here is the acc. pl. of \emph{hafr} ‘he-goat’. Thunder also eats his goats in \Gylfaginning\ 44, where he butchers and cooks them in the evening and brings them back to life by blessing them with his hammer at dawn. \textcite{FinnurEdda} and \textcite{PettitEdda} prefer this reading; see also note to next stanza.—Many other scholars have here read an accusative plural of \emph{hafri} ‘oat’, i.e. ‘porridge, oatmeal’. Stiles (forthcoming TODO) connects this with Indrá’s (who is the Vedic equivalent of Thunder) “partner and yokemate” (\Rigveda\ 6.56.2) Pūṣán’s eating porridge (e.g. 6.56.1, 57.2). Another similarity Stiles notes between Thunder and Pūṣan is that both have chariots driven by goats (e.g. 6.57.3: “Goats are the draft-animals for the one”, 58.2: “Having goats as his horses”). Whether the Vedic tradition has split an original god into two or whether Thunder has absorbed elements of another god is hard to say.}}; \hld\ \alst{s}aðr em’k ęnn þęss.“\eva

\bvb\speakernoteb{[Thunder quoth:]} \\
“Ferry me over the sound, I feed thee in the morning! \\
A basket have I on my back; the food does not get better.\footnoteB{i.e. ‘you will not get better food than that.’} \\
I ate for a while before I journeyed from home, \\
herring and oatmeal/he-goats; I am still full from that.”\evb\evg


\bvg\bva\mssnote{\Regius~12v/5}%
„Ár-ligum \alst{v}erkum hrósar þú, \alst{v}ęrðinum; \hld\ \alst{v}ęitst-at-tu fyr gǫrla, &
\alst{d}ǫpr ’ru þín hęim-kynni, \hld\ \alst{d}auð hygg’k at þín móðir sé.“\eva

\bvb “Of early works boastest thou; of eating!\footnoteB{TODO. This is pretty difficult. From the previous stanza \emph{vęrðinum} seems to be referring to eating.} Thou knowest not clearly [what lies] before [thee]: \\
dismal is the state of thy home—I think that thy mother is dead!”\evb\evg


\bvg\bva\mssnote{\Regius~12v/6}%
„\alst{Þ}at sęgir þú nú \hld\ es hvęrjum \alst{þ}ikkir &
\alst{m}ęst at vita— \hld\ at mín \alst{m}óðir dauð sé.“\eva

\bvb “Thou now sayest that which to every man seems \\
most important to know—that my mother is dead!”\evb\evg


\bvg\bva\mssnote{\Regius~12v/8}%
„\alst{Þ}ęygi ’s sem \alst{þ}ú \hld\ \alst{þ}rjú bú ęigir góð; &
\alst{b}ęr-\alst{b}ęinn þú stęndr \hld\ ok hęfir \alst{b}rautinga gørvi, \hld\ þat-ki at þú hafir \alst{b}rę́kr þínar.“\eva

\bvb “But it is hardly as if thou own three good homesteads; \\
bare-legged thou standest, and hast the gear of a tramp; it is not even as if thou own thy breeches!”\evb\evg


\bvg\bva\mssnote{\Regius~12v/9}%
„\alst{St}ýr-ðu hingat ęikjunni, \hld\ ek mun þér \alst{st}ǫðna kęnna &
eða \alst{h}vęrr á skipit \hld\ es þú \alst{h}ęldr við landit?“\eva

\bvb “Steer hither the boat! I will show thee to the harbour— \\
or who owns the ship which thou holdest by the shore?”\evb\evg


\bvg\bva\mssnote{\Regius~12v/11}%
„\alst{H}ildólfr sá \alst{h}ęitir \hld\ es mik \alst{h}alda bað, &
\alst{r}ekkr inn \alst{r}áð-svinni \hld\ es býr í \alst{R}áðs-ęyjar-sundi; &
bað-at hann \alst{h}lęnni-męnn flytja \hld\ eða \alst{h}rossa-þjófa, &
\alst{g}óða ęina \hld\ ok þá’s ek \alst{g}ørva kunna; &
\alst{s}ęg-ðu til nafns þíns \hld\ ef þú vill of \alst{s}undit fara.“\eva

\bvb “Hildolf he is called, who asked me to hold it, \\
the counsel-wise man who lives in Redeseysound. \\
He bade me not take highwaymen nor horse-thieves; \\
good men only, and those whom I know well— \\
say thy name if thou wilt go over the sound!”\evb\evg


\bvg\bva\mssnote{\Regius~12v/15}%
„\alst{S}ęgja mun’k til nafns míns \hld\ þótt ek \alst{s}ękr sjá’k &
ok til \alst{a}lls \alst{ø}ðlis: \hld\ Ek em \alst{Ó}ðins sonr, &
\alst{M}ęila bróðir \hld\ ęn \alst{M}agna faðir, &
\alst{þ}rúð-valdr goða \hld\ við \alst{Þ}ór knátt-u hér dǿma! &
\alst{H}ins vil’k nú spyrja, \hld\ hvat þú \alst{h}ęitir?“\eva

\bvb “I will say my name—although I should be charged— \\
and all my origin: I am Weden’s son, \\
Male’s brother and Main’s father, \\
the strength-wielder of the Gods; with Thunder dost thou here speak! \\
Now I will ask something else: What art thou called?”\evb\evg


\bvg\bva\mssnote{\Regius~12v/18}„\alst{H}ár-barðr ek \alst{h}ęiti, \hld\ \alst{h}yl’k of nafn sjaldan.“\eva

\bvb “Hoarbeard I am called, seldom I conceal my name.”\evb\evg


\bvg\bva\mssnote{\Regius~12v/18}„Hvat skalt-u of \alst{n}afn hylja \hld\ \alst{n}ema þú sakar ęigir?“\eva

\bvb “Why shalt thou conceal thy name, unless thou have charges?”\evb\evg


\bvg\bva\mssnote{\Regius~12v/19}„En þótt ek \alst{s}akar ęiga, \hld\ fyr \alst{s}líkum sem þú est &
þá mun’k \alst{f}orða \alst{f}jǫrvi mínu \hld\ nema ek \alst{f}ęigr sé.“\eva

\bvb “But though I had charges—for such a one as thou art \\
then I will protect my life, unless I be \inx[C]{fey}.”\evb\evg


\bvg\bva\mssnote{\Regius~12v/21}„Harm ljótan mér þikkir í því &
at \alst{v}aða of \alst{v}áginn til þín \hld\ ok \alst{v}ę́ta \edtrans{ǫgur}{burden}{\Bfootnote{The sense of this word is not clear, though it is probably the same as the first element of the compound \emph{ǫgur-stund} ‘burdensome hour’, found in \Volundarkvida\ 42. Some authors have read it as a crude euphemism for ‘penis’, which would not be out of character for this poem. I however consider the best interpretation to be that of an author whose name I've forgotten (TODO!), namely that Thunder is referring to the food he carries on his back (cf. v. 3).}} minn; &
skylda’k launa \alst{k}ǫgur-svęini \hld\ þínum \alst{k}angin-yrði \hld\ ef ek \alst{k}omumk yfir sundit.“\eva

\bvb “An ugly harm it seems to me \\
to wade o’er the wave to thee, and wet my burden. \\
I would repay thee, swaddle-swain, for thy mocking words, if I could bring myself over the sound.”\evb\evg


\bvg\bva\mssnote{\Regius~12v/23}„\alst{H}ér mun’k standa \hld\ ok þín \alst{h}eðan bíða; &
fannt-a-tu mann inn \alst{h}arðara \hld\ at \alst{H}rungni dauðan.“\eva

\bvb “\emph{Here} will I stand, and \emph{from here} await thee; \\
thou hast not found a harder man since \inx[P]{Rungner} died!\footnoteB{Rungner was an ettin famously slain by Thunder, TODO.  Hoarbeard’s mention of that battle sets off a long argument over the deeds of the two.}”\evb\evg


\bvg\bva\mssnote{\Regius~12v/25}„\alst{H}ins vilt-u nú geta \hld\ es vit \alst{H}rungnir dęildum, &
sá inn \alst{st}ór-úðgi jǫtunn, \hld\ es ór \alst{st}ęini vas hǫfuðit á, &
þó lét’k hann \alst{f}alla \hld\ ok \alst{f}yrir hníga; &
\ind hvat vannt-u þá meðan, Hárbarðr?“\eva

\bvb “This wilt thou now mention, when I and Rungner dealt with each other, \\
that great-minded ettin on whom the head was of stone.  \\
Yet I made him fall, and kneel down before [me]— \\
what didst thou then meanwhile, Hoarbeard?”\evb\evg


\bvg\bva\mssnote{\Regius~12v/27}„Vas’k með \alst{F}jǫl-vari \hld\ \alst{f}imm vetr alla &
í \alst{ęy} þęiri \hld\ es \alst{A}l-grǿn hęitir; &
\alst{v}ega vér þar knǫ́ttum \hld\ ok \alst{v}al fęlla, &
\alst{m}args at fręista, \hld\ \alst{m}ans at kosta.“\eva

\bvb “I was with Felwar for all of five winters \\
in that island which Allgreen is called. \\
There we did fight and fell corpses; \\
many a girl to tempt and win.\footnoteB{I read \emph{margs} ‘many a’ as modifying \emph{mans} ‘girl’, i.e. \emph{margs mans at fręista, at kosta} ‘to tempt and to win many a girl’.}”\evb\evg


\bvg\bva\mssnote{\Regius~12v/30}„Hversu snúnuðu yðr konur yðrar?“\eva

\bvb “How did your women pleasure (TODO!!!) you?.\footnoteB{Seemingly a prose line; see Introduction.}”\evb\evg


\bvg\bva\mssnote{\Regius~12v/30}„\alst{Sp}arkar ǫ́ttum vér konur \hld\ ef oss at \alst{sp}ǫkum yrði; &
\alst{h}orskar ǫ́ttum vér konur \hld\ ef oss \alst{h}ollar vę́ri, &
þę́r ór \alst{s}andi \hld\ \alst{s}íma undu &
\ind ok ór \alst{d}ali \alst{d}júpum &
\ind \alst{g}rund of \alst{g}rófu; &
varð’k þęim ęinn \alst{ǫ}llum \hld\ \alst{ø}fri at rǫ́ðum; &
\ind hvílda’k hjá \alst{s}ystrum \alst{s}jau &
\ind ok hafða’k \alst{g}ęð þęira allt ok \alst{g}aman; &
\ind hvat vannt-u þá meðan, Þórr?“\eva

\bvb “We \ken*{I} owned frisky women, if they became pleasing toward us \ken*{me}; \\
we \ken*{I} owned clever women, if they were \inx[C]{hold} toward us \ken*{me}; \\
they wound a rope out of the sand, \\
and out of a deep dale \\
dug up the ground. \\
I alone became superior to them all in counsels, \\
I rested next to those seven sisters, \\
and had their senses all, and pleasure— \\
what didst thou then meanwhile, Thunder?”\evb\evg


\bvg\bva\mssnote{\Regius~13r/2, \AM~1r/1 (l. 4b ff.)}„Ek drap \alst{Þ}jatsa, \hld\ hinn \alst{þ}rúð-móðga jǫtun, &
\alst{u}pp ek varp \alst{au}gum \hld\ \alst{A}ll-valda sonar &
\ind á þann hinn \alst{h}ęiða \alst{h}imin; &
þau ’ru \alst{m}ęrki \alst{m}ęst \hld\ \alst{m}inna verka, &
\ind þau’s allir męnn \alst{s}íðan of \alst{s}éa; &
\ind hvat vannt-u þá meðan, Hárbarðr?“\eva

\bvb “I slew \inx[C]{Thedse}, the strength-minded ettin; \\
Up I threw the eyes of Allwald’s son \ken*{= Thedse} \\
onto the clear heaven! \\
Those are the greatest marks of my works, \\
those which all men since may see\footnoteB{Here we seem to have a rare example of native Germanic star-lore. Is the exact constellation identifiable? TODO.}— \\
what didst thou then meanwhile, Hoarbeard?”\evb\evg


\bvg\bva\mssnote{\Regius~13r/5, \AM~1r/1}„\alst{M}iklar \alst{m}an-vélar \hld\ hafða’k við \alst{m}yrk-riður &
\ind þá’s ek \alst{v}élta þę́r frá \alst{v}erum. &
\alst{H}arðan jǫtun \hld\ hugða’k \alst{H}lébarð vesa; &
\ind \alst{g}af hann mér \alst{g}amban-tęin &
\ind en ek \alst{v}élta hann ór \alst{v}iti.“\eva

\bvb “Great girl-tricks did I have against \inx[C]{mirk-rideresses}, \\
when I lured them away from men.\footnoteB{Alternatiely ‘away from [their] husbands’.  The \emph{riður} ‘(female) riders’ were witches thought to torment people and cause disease and suffering. See \Havamal\ 156 for discussion.} \\
A hard ettin I judged Leebeard to be; \\
he gave me a \inx[C]{gombentoe}, \\
but I tricked him out of his wits.”\evb\evg


\bvg\bva\mssnote{\Regius~13r/7, \AM~1r/3}„Illum huga launaðir þú þá \alst{g}óðar \alst{g}jafar.“\eva

\bvb “With an evil mind didst thou repay the good gift.”\evb\evg


\bvg\bva\mssnote{\Regius~13r/8, \AM~1r/4}„Þat hęfir \alst{ęi}k \hld\ es af \alst{a}nnarri skęfr; &
\ind umb \alst{s}ik es hvęrr í \alst{s}líku— &
\ind hvat vannt-u þá meðan, Þórr?“\eva

\bvb “An oak has that which it chafes from another; \\
each man is for himself in such— \\
what didst thou then meanwhile, Thunder?”\evb\evg


\bvg\bva\mssnote{\Regius~13r/9, \AM~1r/4}„Ek vas \alst{au}str \hld\ ok \alst{jǫ}tna barða’k &
\alst{b}rúðir \alst{b}ǫl-vísar \hld\ es til \alst{b}jargs gingu; &
mikil myndi \alst{ę́}tt \alst{jǫ}tna \hld\ ef \alst{a}llir lifði, &
vę́tr myndi \alst{m}anna \hld\ undir \alst{M}ið-garði— &
\ind hvat vannt-u þá meðan, Hárbarðr?\eva

\bvb “I was in the East, and bashed ettins: \\
bale-wise brides who walked to the mountain. \\
Great would the lineage of ettins be if all lived, \\
naught would remain of men within Middenyard\footnoteB{A remarkable clear statement, the underlying worldview of which is far from unique to this stanza; in \Hymiskvida\ 11, for instance, Thunder is described as “the opponent of Rooder”, “the friend of manly retinues” and “Wigh-ward”, referring to His role in slaying ettins and guarding men and their shrines (\inx[C]{wigh}[wighs]).  For Thunder’s killing of women cf. sts. 37–39 below and Lindow 1988.}— \\
what didst thou then meanwhile, Hoarbeard?”\evb\evg


\bvg\bva\mssnote{\Regius~13r/11, \AM~1r/6}„\alst{V}as’k á \alst{V}allandi \hld\ ok \alst{v}ígum fylgða’k, &
\alst{a}tta ek \alst{jǫ}frum \hld\ en \alst{a}ldri-gi sę́tta’k; &
\alst{Ó}ðinn á \alst{ja}rla \hld\ þá’s í \alst{v}al falla &
\ind en \alst{Þ}órr á \alst{þ}rę́la kyn.“\eva

\bvb “I was in \inx[L]{Walland} and followed battles; \\
I incited princes and never reconciled them. \\
Weden owns the earls which fall among the slain, \\
but Thunder owns the kin of thralls.\footnoteB{We see here a sort of aristocratic, Odinic disregard for lower life and life as a good in itself; where Thunder boasts of saving men, Weden sarcastically responds that he caused the deaths of men so that he could have them for himself.}”\evb\evg


\bvg\bva\mssnote{\Regius~13r/13, \AM~1r/8}„\alst{Ó}·\alst{ja}fnt skipta \hld\ es þú myndir með \alst{ǫ́}sum liði &
\ind ef þú ę́ttir \alst{v}il-gi mikils \alst{v}ald.“\eva

\bvb “Translation.”\evb%TODO: There’s something very weird going on here.
\evg


\bvg\bva\mssnote{\Regius~13r/14, \AM~1r/9}„Þórr á \alst{a}fl \alst{ǿ}rit \hld\ en \alst{ę}kki hjarta; &
af \alst{h}rę́ðslu ok \alst{h}ug-blęyði \hld\ þér vas í \alst{h}andska troðit &
\ind ok \alst{þ}óttisk-a þú \alst{þ}á \alst{Þ}órr vesa; &
\alst{h}vár-ki þá þorðir \hld\ fyr \alst{h}rę́ðslu þinni &
hnjósa né \alst{f}ísa \hld\ svá’t \alst{F}jalarr hęyrði.“\eva

\bvb “Thunder has ample strength but little heart; \\
for fear and heart-softness didst thou tread into a glove, \\
and then seemedest thou not to be Thunder. \\
Thou daredest not for thy fear— \\
sneeze nor fart lest Feller should hear.\footnoteB{This story is also referenced in \Lokasenna\ 60, and is told in full in \Gylfaginning\ 45: Lock, Thunder, and his servants Thelve and Wrash had journeyed east for a long time when they came upon a large hall, with an opening on one end as wide as the building.  They rested inside, but in the middle of the night they were awakened by a great earthquake.  Thunder rose and led the party to a side-room to the right in the middle of the hall. He stayed closest to the opening with his hammer ready, while the terrified others were further inside.  At daybreak they left the hall and found the huge ettin \emph{Skrymir} (\inx[P]{Shrimer}) asleep outside.  His snoring had caused the earth-quakes, and the hall was his mitten; the side-room was its thumb.}”\evb\evg


\bvg\bva\mssnote{\Regius~13r/17, \AM~1r/11}„\alst{H}ár-barðr hinn ragi, \hld\ munda’k þik í \alst{H}ęl drepa &
\ind ef mę́tta’k \alst{s}ęilask of \alst{s}und.“\eva

\bvb “Hoarbeard the \inx[C]{queer}, I would strike thee into \inx[L]{Hell}, \\
if I might sail o’er the sound!”\evb\evg


\bvg\bva\mssnote{\Regius~13r/18, \AM~1r/12}„Hvat skyldir of \alst{s}und \alst{s}ęilask \hld\ es \alst{s}akir ’ru alls øngar? &
\ind hvat vannt-u þá meðan, Þórr?“\eva

\bvb “Why should thou sail o’er the sound when there are no offenses?— \\
what didst thou then meanwhile, Thunder?”\evb\evg


\bvg\bva\mssnote{\Regius~13r/19, \AM~1r/13}„Ek vas \alst{au}str \hld\ ok \alst{á}na varða’k &
þá’s mik \alst{s}óttu \hld\ þęir \alst{S}várangs synir; &
\alst{g}rjóti mik bǫrðu, \hld\ \alst{g}agni urðu þó lítt fęgnir, &
þó urðu mik \alst{f}yrri \hld\ \alst{f}riðar at biðja. &
\ind hvat vannt-u þá meðan, Hárbarðr?“\eva

\bvb “I was in the east and guarded the river \\
when I was attacked by Sweering’s sons. \\
With rocks they bashed me—still they rejoiced little in victory, \\
still they had to beg me first for peace— \\
what didst thou then meanwhile, Hoarbeard?”\evb\evg


\bvg\bva\mssnote{\Regius~13r/22, \AM~1r/15}„Ek vas \alst{au}str \hld\ ok við \alst{ęi}n-hvęrja dǿmða’k, &
\alst{l}ék’k við ina \alst{l}ind-hvítu \hld\ ok \alst{l}ǫng þing háða’k, &
\alst{g}ladda’k ina \alst{g}ull-bjǫrtu, \hld\ \alst{g}amni mę́r unði.“\eva

\bvb “I was in the east, and spoke with a certain woman; \\
I played with the linen-white, and held long-lasting trysts:\footnoteB{\emph{þing} (see \inx[C]{Thing}) usually means ‘legal assembly’, but clearly not here.} \\
I gladdened the gold-bright—the maiden enjoyed pleasure.”\evb\evg


\bvg\bva\mssnote{\Regius~13r/24, \AM~1r/17}„Góð ǫ́ttu þęir man-kynni þar þá.“\eva

\bvb “Then they had good girl-visits there.”\evb\evg


\bvg\bva\mssnote{\Regius~13r/24, \AM~1r/17}„\alst{L}iðs þíns vę́ra’k þá þurfi, Þórr, \hld\ at hęlda’k þęiri inni \alst{l}ín-hvítu męy.“\eva

\bvb “Of thy help I might have been in need then, Thunder, that I might hold that linen-white maiden.”\evb\evg


\bvg\bva\mssnote{\Regius~13r/25, \AM~1r/18}„Ek mynda þér þat þá \alst{v}ęita \hld\ ef ek \alst{v}iðr of kǿmisk.“\eva

\bvb “I would then have granted thee that, if I were able.”\evb\evg


\bvg\bva\mssnote{\Regius~13r/26, \AM~1r/18}„Ek mynda þér þá \alst{t}rúa, \hld\ nema mik í \alst{t}ryggð véltir.“\eva

\bvb “I would then have trusted thee, unless thou shouldst betray my trust.”\evb\evg


\bvg\bva\mssnote{\Regius~13r/27, \AM~1r/19}„Em’k-at ek sá \alst{h}ę́l-bítr \hld\ sem \alst{h}úð-skór forn á vár.“\eva

\bvb “I am not such a heel-biter as an old hide-shoe in spring.\footnoteB{Proverbial (a heel-biter being someone who betrays his companions); the leather of a shoe would become very stiff and chafing over the winter.}”\evb\evg


\bvg\bva\mssnote{\Regius~13r/28, \AM~1r/20}\ind „Hvat vannt-u þá meðan, Þórr?“\eva

\bvb “What didst thou then meanwhile, Thunder?”\evb\evg


\bvg\bva\mssnote{\Regius~13r/28, \AM~1r/20}„\alst{B}rúðir \alst{b}er-sęrkja \hld\ \alst{b}arða’k í Hlés-ęyju; &
þę́r hǫfðu \alst{v}ęrst unnit, \hld \alst{v}élta þjóð alla.“\eva

\bvb “The brides of bearserks I bashed in Leesie; \\
they had done the worst thing: deceived a whole people.”\evb\evg


\bvg\bva\mssnote{\Regius~13r/29, \AM~1r/21}„\alst{K}lę́ki vannt-u þá, Þórr, \hld\ es þú á \alst{k}onum barðir.“\eva

\bvb “A great disgrace didst thou then, Thunder, when thou didst bash women.”\evb\evg


\bvg\bva\mssnote{\Regius~13r/30, \AM~1r/22}„\alst{V}argynjur vǫ́ru þę́r \hld\ en \alst{v}ar-la konur, &
\alst{sk}ęlldu \alst{sk}ip mitt \hld\ es ek \alst{sk}orðat hafða’k, &
ǿgðu mér járn-lurki \hld\ en ęltu Þjálfa. &
\ind hvat vannt-u þá meðan, Hárbarðr?“\eva

\bvb “She-wolves were they, and hardly women; \\
they overturned my ship which I had propped; \\
terrorised me with an iron-cudgel, and chased Thelve around— \\
what didst thou then meanwhile, Hoarbeard?”\evb\evg


\bvg\bva\mssnote{\Regius~13r/32, \AM~1r/23}„Ek vas’k í hęrnum \hld\ es hingat gørðisk &
gnę́fa gunn-fana, \hld\ gęir at rjóða.“\eva

\bvb “I was in the warband, when it readied itself here \\
to raise the war-standard, to redden the spear.”\evb\evg


\bvg\bva\mssnote{\Regius~13v/1, \AM~1r/24}„Þess vilt-u nú geta, es þú fórt oss \edtext{ó·ljúfan}{\Bfootnote{oliyfan \AM; †olubann† \Regius}} at bjóða!“\eva

\bvb “This wilt thou now mention, that thou didst journey to attack us!”\evb\evg


\bvg\bva\mssnote{\Regius~13v/2, \AM~1r/25}„\alst{B}ǿta skal þér þat þá \hld\ munda \alst{b}augi &
sem \alst{ja}fnęndr \alst{u}nnu \hld\ þęir’s \alst{o}kkr vilja sę́tta.“\eva

\bvb “Then, I shall repay thee for that, with a hand-bigh, \\
bestowed by the mediators who wish to reconcile us two.”\evb\evg


\bvg\bva\mssnote{\Regius~13v/3, \AM~1r/26}„Hvar namt þęssi \hld\ in hnǿfi-ligu orð &
es hęyrða’k aldrigi \hld\ hnǿfi-ligri?“\eva

\bvb “Where didst thou learn these sarcastic words, \\
which I never heard more sarcastic?”\evb\evg


\bvg\bva\mssnote{\Regius~13v/5, \AM~1r/27}„Nam’k at mǫnnum þęim inum aldrǿnum es búa í hęimis-skógum.“\eva

\bvb “I learned them from the old men who dwell in the home-forests.”\evb\evg


\bvg\bva\mssnote{\Regius~13v/5, \AM~1v/1}„Þó gefr þú gótt nafn dysjum, es þú kallar þat hęimis-skóga.“\eva

\bvb “Yet thou givest a good name to poor cairns,\footnoteB{cf. Weden’s waking the dead in various poems.} as thou callest them home-forests.”\evb\evg


\bvg\bva\mssnote{\Regius~13v/6, \AM~1v/2}„Svá dǿmi’k of slíkt far.“\eva

\bvb “So I speak about such matters.”\evb\evg


\bvg\bva\mssnote{\Regius~13v/7, \AM~1v/2}„\alst{O}rð-kringi þín \hld\ mun þér \alst{i}lla koma &
\ind ef ek rę́ð á \alst{v}ág at \alst{v}aða; &
\alst{u}lfi hę́ra \hld\ hygg’k at \alst{ǿ}pa mynir &
\ind ef \alst{h}lýtr af \alst{h}amri \alst{h}ǫgg.“\eva

\bvb “Thy glibness of word will bring thee harm, \\
if I decide to wade over the wave; \\
higher than a wolf I judge that thou wilt scream, \\
if thou suffer a strike from the hammer.”\evb\evg


\bvg\bva\mssnote{\Regius~13v/9, \AM~1v/4}„Sif á \edtrans{\alst{h}ó}{lover}{\Bfootnote{Most translators take this acc. sg. word as an alternative form of \emph{hórr} m. ‘adulterer’ (gen. \emph{hórs}), containing the same root as \emph{hóra} f. ‘whore, prostitute’, \emph{hór} n. ‘adultery, fornication’, ModEngl. whore. The \emph{-r} has presumably been interpreted as the masc. nom. sg. ending, giving nom. \emph{*hór}, gen. \emph{*hós}. Further, this accusation is also found in \Lokasenna\ TODO, where Lock says that he has been Sib’s lover (\emph{hórr}). Notably, \CV\ interprets this word as the unrelated \emph{hór} m. ‘pot-hook’, “insinuating that Thor busied himself with cooking and dairy-work.” This seems very unlikely when considering Thunder’s response in the next verse: “I think that thou liest!” and the parallel in \Lokasenna.}} \alst{h}ęima, \hld\ \alst{h}ans munt fund vilja, &
\alst{þ}ann munt \alst{þ}ręk drýgja, \hld\ \alst{þ}at ’s þér skyldara.“\eva

\bvb “Sib has a lover at home; \emph{him} wilt thou wish to meet! \\
Against that one shalt thou use thy strength—that is for thee more urgent!”\evb\evg


\bvg\bva\mssnote{\Regius~13v/10, \AM~1v/5}„\alst{M}ę́lir þú at \alst{m}unns ráði \hld\ svá’t \alst{m}ér skyldi vęrst þikkja, &
\alst{h}alr inn \alst{h}ug-blauði, \hld\ \alst{h}ygg’k at þú ljúgir.“\eva

\bvb “Thou speakest according to thy mouth’s counsel that which should seem to me the worst; \\
O heart-soft man, I think that thou liest!”\evb\evg


\bvg\bva\mssnote{\Regius~13v/12, \AM~1v/6}„\alst{S}att hygg’k mik \alst{s}ęgja, \hld\ \alst{s}ęinn est at fǫr þinni, &
\alst{l}angt myndir nú kominn, Þórr, \hld\ ef þú \edtrans{\alst{l}itum fǿrir}{brought thy colours}{\Bfootnote{Very unclear expression. \emph{fǿra litum} TODO.}}.“\eva

\bvb “I think myself to speak truly: thou art late on thy journey; \\
far wouldst thou now have come, Thunder, if thou had brought thy colours.”\evb\evg


\bvg\bva\mssnote{\Regius~13v/14, \AM~1v/8}„\alst{H}árbarðr inn ragi, \hld\ \alst{h}ęldr hęfir nú mik dvalðan!“\eva

\bvb “Hoarbeard the queer; thou hast now much delayed me!”\evb\evg


\bvg\bva\mssnote{\Regius~13v/14, \AM~1v/8}„\alst{Á}sa-Þórs \hld\ hugða’k \alst{a}ldri-gi myndu &
\ind glępja \alst{f}é-hirði \alst{f}arar.“\eva

\bvb “The journey of Thunder of the Eese I never thought \\
that a shepherd would divert.”\evb\evg


\bvg\bva\mssnote{\Regius~13v/15, \AM~1v/9}„\alst{R}áð mun’k þér nú \alst{r}áða: \hld\ \alst{R}ó þú hingat bátinum, &
\alst{h}ę́ttum \alst{h}ǿtingi, \hld\ \alst{h}itt fǫður Magna!“\eva

\bvb “I will now give thee a counsel: Row the boat hither, \\
stop the taunting, come to the father of Main \ken*{= Thunder = me}!”\evb\evg


\bvg\bva\mssnote{\Regius~13v/17, \AM~1v/10}„\alst{F}ar þú \alst{f}irr sundi, \hld\ þér skal \alst{f}ars synja!“\eva

\bvb “Go far from the sound; the ferry shall be denied thee!”\evb\evg


\bvg\bva\mssnote{\Regius~13v/17, \AM~1v/11}„\alst{V}ísa þú mér nú lęiðina \hld\ alls þú vill mik ęigi of \alst{v}áginn fęrja!“\eva

\bvb “Now show me the way, since thou wilt not ferry me o’er the wave!”\evb\evg


\bvg\bva\mssnote{\Regius~13v/18, \AM~1v/11}„\alst{L}ítit ’s at synja, \hld\ \alst{l}angt ’s at fara; &
\alst{st}und ’s til \alst{st}okksins, \hld\ ǫnnur til \alst{st}ęinsins, &
halt svá til \alst{v}instra \alst{v}egsins \hld\ unds þú hittir \alst{V}er-land; &
\alst{þ}ar mun Fjǫrgyn \hld\ hitta \alst{Þ}ór, son sinn, &
ok mun hǫ́n kęnna hǫ́num \alst{ǫ́}ttunga brautir \hld\ til \alst{Ó}ðins landa.“\eva

\bvb “It is little to deny; it is long to journey: \\
an hour to the log, another to the stone;  \\
keep thus to the left road, until thou dost find Wereland;  \\
there will Firgyn find Thunder, her son, \\
and she will teach him the ancestral roads, to Weden’s lands \ken*{= Osyard}.”\evb\evg


\bvg\bva\mssnote{\Regius~13v/22, \AM~1v/14}„Mun’k taka þangat í dag?“\eva

\bvb “Will I arrive thither today?”\evb\evg


\bvg\bva\mssnote{\Regius~13v/22, \AM~1v/14}„Taka við víl ok \alst{ę}rfiði \hld\ at \alst{u}pp-vesandi sólu &
es ek get þána.“\eva

\bvb “[Thou wilt] arrive, with toil and hardship, at the rising of the sun as I guess it is thawing.”\evb\evg


\bvg\bva\mssnote{\Regius~13v/23, \AM~1v/15}„\alst{Sk}ammt mun nú mál okkat vesa, \hld\ alls þú mér \alst{sk}ǿtingu ęinni svarar; &
launa mun ek þér \alst{f}ar-synjun \hld\ ef vit \alst{f}innumk í sinn annat. &
Far þú nú þar’s þik hafi allan gramir!“\eva

\bvb “Now our speech will be short as thou dost answer me only with scoffing; \\
I will reward thee for this ferry-denial if we meet another time. \\
Go now whither the fiends may have thee all!”\evb\evg

\sectionline
% Weden, Thunder
	\bookStart{The Lay of Thrim}[Þrymskviða]

\begin{flushright}%
Dating \parencite{Sapp2022}: C9th (0.741)–C10th (0.259)

Meter: \Fornyrdislag%
\end{flushright}

Compare \Haustlong, \Hymiskvida, other poems and refer to the SkP intro to one of the big Thunder poems. TODO.

\sectionline

\bvg
\bva \edtext{\alst{V}ręiðr}{\lemma{Vręiðr}\Afootnote{TODO: Note about ambiguity of alliteration.}} vas þá Ving-Þórr \hld\ es hann vaknaði &
ok síns hamars \hld\ of saknaði, &
skegg nam at hrista, \hld\ skǫr nam at dýja, &
réð Jarðar burr \hld\ umb at þręifask.\eva

\bvb Wroth was then Wing-Thunder when he woke, and of his hammer was bereaved. His beard he took to shake, his locks he took to pull; resolved the son of Earth to look about.\evb
\evg


\bvg
\bva \edtext{Ok hann þat orða \hld\ allz fyrst of kvað:}{\lemma{Ok \dots\ of kvað ‘And ... did say’}\Bfootnote{The whole line is formulaic, occuring in five other places: sts. 3, 9 and 12 of the present poem; st 3 of \Oddrunargratr; st. 5 of \Brot.}} &
„Hęyr-ðu nú, Loki, \hld\ hvat ek nú mę́li &
es ęigi vęit \hld\ jarðar hvęrgi &
né upphimins: \hld\ áss es stolinn hamri!“\eva

\bvb And he that word first of all did say: “Hear thou now, Lock, what I now speak, which nowhere is known, not on earth nor Up-heaven:\footnoteB{Formulaic, see Encyclopedia: \inx[F]{Earth and Up-heaven}.} the \inx[G]{Ease}[os] \ken*{= Thunder = I} has been robbed of his hammer!”\evb
\evg


\bvg
\bva Gingu þęir fagra \hld\ Fręyju túna &
ok hann þat orða \hld\ allz fyrst of kvað: &
„Muntu mér, Fręyja, \hld\ fjaðrhams léa &
ef ek mínn hamar \hld\ mę́tta’k hitta?“\eva

\bvb Went they to the fair yards of \inx[P]{Frow}, and he that word first of all did say: “Wilt thou me, Frow, the \inx[P]{feather-hame} lend, if I my hammer might find?”\evb
\evg


\bvg
\bva „Þó mynda’k gefa þér \hld\ þótt ór gulli vę́ri &
ok þó sęlja \hld\ at vę́ri ór silfri.“\eva

\bvb {[Frow quoth:]} “I would yet give it to thee, though it were golden; and yet offer\footnoteB{\emph{sęlja} ‘sell’ here has its earlier meaning, cf. Gotish \emph{saljan} \textcite[116]{Streitberg}: ‘\emph{opfern}; \textgreek{θύειν}’.} it to thee, as it were silvern.”\footnoteB{Regaining the hammer is of such importance to the gods (cf. v. 17; without it the Ease stand powerless against the \inx[G]{Ettins}), that Frow would lend the feather-hame to the greedy and untrusty Lock, even if it were made out of gold or silver.}\evb
\evg

\bvg
\bva Fló þá Loki, \hld\ fjaðrhamr dunði, &
unz fyr útan kom \hld\ ása garða &
ok fyr innan kom \hld\ jǫtna hęima.\eva

\bvb Flew then Lock\footnoteB{Though Thunder is the one asking for the hame (“if I \emph{my} hammer might find”), Lock is the one that takes off flying.}—the feather-hame rustled—until outside he came of the \inx[L]{Osyard}[yards of the Ease], and inside he came of the \inx[L]{Ettinham}[homes of the Ettins].\evb
\evg


\bvg
\bva Þrymr sat á haugi, \hld\ þursa dróttinn, &
gręyjum sínum \hld\ gullbǫnd snøri &
ok mǫrum sínum \hld\ mǫn jafnaði.\eva

\bvb Thrim sat on the mound,\footnoteB{Apparently a typical seating position for ettins. See \Voluspa\ 42 for other attestations.} the lord of \inx[G]{Thurses}: on his greyhounds the golden leashes he twirled, and on his mares the manes he cut even.\footnoteB{The image suggested here reminds one of the ancient “master of animals” motif, especially as attested on panel A of the Gundestrup cauldron.}\evb
\evg


\bvg
\bva „Hvat ’s með ǫ́sum? \hld\ Hvat ’s með ǫlfum? &
Hví est ęinn kominn \hld\ í jǫtunhęima?“ &
„Illt es með ǫ́sum, \hld\ \edtext{illt es með ǫlfum}{\Bfootnote{Inserted in analogy with the first pair, regardless it is needed for metrical reasons.}}! &
Hęfir þú Hlórriða \hld\ hamar of folginn?“\eva

\bvb {[Thrim quoth:]} “What is with the Ease? What is with the elves? Why art thou alone come into the \inx[L]{Ettinham}[Ettin-homes]?”—{[Lock quoth:]} “’Tis ill with the Ease, ’tis ill with the elves! Hast thou the hammer of Loride \name{= Thunder} hidden?”\evb
\evg


\bvg
\bva „Ek hęfi Hlórriða \hld\ hamar of folginn &
átta rǫstum \hld\ fyr jǫrð neðan; &
hann ęngi maðr \hld\ aptr of hęimtir &
nęma fǿri mér \hld\ Fręyju at kvę́n.“\eva

\bvb {[Thrim quoth:]} “I have the hammer of Loride hidden, eight \inx[C]{rest}[rests] beneath the earth; it no man will fetch again, unless he bring me Frow as wife.”\evb
\evg


\bvg
\bva Fló þá Loki, \hld\ fjaðrhamr dunði, &
unz fyr útan kom \hld\ jǫtna hęima &
ok fyr innan kom \hld\ ása garða; &
mǿtti hann Þór \hld\ miðra garða &
ok \edtext{hann þat}{\Afootnote{emend.; \emph{þat hann} \Regius, with elsewhere unprecedented word order. Cf. note to st. 2.}} orða \hld\ allz fyrst of kvað:\eva

\bvb Flew then Lock—the feather-hame rustled—until outside he came of the homes of the Ettins, and inside he came of the yards of the Ease. He met Thunder in the middle of the yards, and he \ken*{= Thunder} that word first of all did say:\evb
\evg


\bvg
\bva „Hęfir þú ørendi \hld\ sem ęrfiði? &
Seg-ðu á lopti \hld\ lǫng tíðendi! &
Opt sitjanda \hld\ sǫgur of fallask, &
ok liggjandi \hld\ lygi of bęllir.“\eva

\bvb {[Thunder quoth:]} “Hast thou an errand, as hardship?\footnoteB{Thunder asks Lock if he has bad news. The collocation \emph{ørendi} ‘errand’ \dots\ \emph{ęrfiði} ‘hardship’ is formulaic and occurs in X (TODO!!) places, including in st. 5 of \HelgakvidaHjorvardssonar.} Say thou aloft, the long tidings! Often sitting, tales fail each other, and lying down, lies are dealt.”\footnoteB{Proverbial. If one sits down and thinks too much over bad news, details will be left out, excuses thought up. Thus it is best that Lock immediately tell Thunder what he has learned.}\evb
\evg


\bvg
\bva „Hefi’k ørendi \hld\ erfiði ok: &
Þrymr hęfir þinn hamar, \hld\ þursa dróttinn; &
hann ęngi maðr \hld\ aptr of hęimtir &
nęma hǫ́num fǿri \hld\ Fręyju at kvę́n.“\eva

\bvb {[Lock quoth:]} “I have an errand, hardship also: Thrim has thy hammer, the lord of Thurses; it no man will fetch again, unless he bring him Frow as wife.”\evb
\evg


\bvg
\bva Ganga þęir fagra \hld\ Fręyju at hitta &
ok hann þat orða \hld\ allz fyrst of kvað: &
„Bitt-u þik, Fręyja, \hld\ brúðar líni! &
Vit skulum aka tvau \hld\ í jǫtunhęima.“\eva

\bvb Go they the fair Frow to find, and he\footnoteB{Unclear. Possibly Lock, since he was the speaker of the last verse.} that word first of all did say: “Bind thee, Frow, with a bride’s linen\footnoteB{A linen band tied around the bride’s head. TODO: Reference this note.}! We two shall drive into the Ettin-homes.”\evb
\evg


\bvg
\bva Vręið varð þá Fręyja \hld\ ok fnasaði, &
allr ása salr \hld\ undir bifðisk, &
stǫkk þat it mikla \hld\ męn Brísinga: &
„Mik vęizt verða \hld\ vergjarnasta &
ef ek ęk með þér \hld\ í jǫtunhęima.“\eva

\bvb Wroth became then Frow, and snorted—the whole hall of the Ease trembled below—threw she off the great necklace of the Brisings: “Thou knowest that I will become the most man-eager,\footnoteB{Either Frow is speaking out of self-awareness of her own lustful inclinations, or the sense is that she will be accused of being lustful by the other gods, but there is no verb here corresponding to ‘accuse’. For Frow’s promiscuity see \Lokasenna\ 30 and Note.} if I drive with thee into the Ettin-homes.”\evb
\evg


\bvg
\bva Sęnn vǫ́ru ę́sir \hld\ allir á þingi &
ok ǫ́synjur \hld\ allar á máli, &
ok of þat réðu \hld\ ríkir tívar: &
hvé þęir Hlórriða \hld\ hamar of sǿtti.\eva

\bvb Soon were the \inx[G]{Ease} all at the \inx[C]{Thing}, and the \inx[C]{Ossens} all at speech, and of this counseled the mighty \inx[G]{Tews}:\footnoteB{Identical to \Baldrsdraumar\ 1.} how they the hammer of Loride would seek out.\evb
\evg


\bvg
\bva Þá kvað þat Hęimdallr, \hld\ hvítastr ása, &
vissi hann vęl framm \hld\ sęm vanir aðrir: &
„Bindu vér Þór þá \hld\ brúðar líni; &
hafi hann it mikla \hld\ męn Brísinga!\eva

\bvb Then quoth that \inx[P]{Homedall}, the whitest of the Ease; he knew well forth,\footnoteB{\emph{vita framm} ‘to know forth’, i.e. to know the future. Compare \emph{framvíss} ’forth-wise; prescient.’} like the other \inx[G]{Wanes}: “Let us bind Thunder with the bride’s linen; may he have the great \inx[P]{necklace of the Brisings}.\evb
\evg


\bvg
\bva Lǫ́tum und hǫ́num \hld\ hrynja lukla &
ok kvenváðir \hld\ umb kné falla &
en á brjósti \hld\ bręiða stęina &
ok hagliga \hld\ umb hǫfuð typpum!“\eva

\bvb Let us place by his side keys to jingle, and women’s garments to fall down about his knees, and on the breast broad stones, and skillfully let us tip his head!\footnoteB{This verse contains an interesting description of Viking age bridal dress: As the everyday manager of the household, keys were the mark of a respectable married woman. The “broad stones” on the breast are probably tortoise brooches, while the tipping of the head refers to some sort of bridal hat (TODO: Literature). Breast-brooches are also mentioned in \Volundarkvida\ 25, 36.}”\evb
\evg


\bvg
\bva Þá kvað þat Þórr, \hld\ þrúðugr áss: &
„Mik munu ę́sir \hld\ argan kalla &
ef ek bindask lę́t \hld\ brúðar líni!“\eva

\bvb Then quoth that Thunder, the mighty os: “Me would the Ease call \inx[C]{degenerate}, if I let myself be bound with bride’s linen!”\evb
\evg


\bvg
\bva Þá kvað þat Loki \hld\ Laufęyjar sonr: &
„Þęgi þú, Þórr, \hld\ þęira orða! &
Þegar munu jǫtnar \hld\ Ásgarð búa &
nęma þú þinn hamar \hld\ þér of hęimtir.“\eva

\bvb Then quoth that Lock, the son of Leafie: “Shut thou, Thunder, those words up! Shortly the Ettins will settle Osyard, unless thou thy hammer for thyself dost fetch!”\evb
\evg


\bvg
\bva Bundu þęir Þór þá \hld\ brúðar líni &
ok inu mikla \hld\ męni Brísinga, &
létu und hǫ́num \hld\ hrynja lukla &
ok kvenváðir \hld\ umb kné falla &
en á brjósti \hld\ bręiða stęina &
ok hagliga \hld\ of hǫfuð typpðu.\eva

\bvb Bound they Thunder then, with bride’s linen, and with the great necklace of the Brisings. They placed by his side keys to jingle, and women’s garments to fall down about his knees, and on the breast broad stones, and skillfully they tipped his head.\evb
\evg


\bvg
\bva Þá kvað þat Loki \hld\ Laufęyjar sonr: &
„Mun ek ok með þér \hld\ ambǫ́tt vesa, &
vit skulum aka tvau \hld\ í jǫtunhęima.“\eva

\bvb Then quoth that Lock, the son of Leafie: “I will also with thee be a handmaid; we two\footnoteB{The form used, \emph{tvau}, is the neuter plural, i.e. one of the pair is female and the other male. This is either an error due to mindless copying of v. 11, or a backhanded insult against Thunder.} shall drive into the Ettin-homes.”\evb
\evg


\bvg
\bva Sęnn vǫ́ru hafrar \hld\ hęim of vreknir, &
skyndir at skǫklum, \hld\ skyldu vęl renna; &
bjǫrg brotnuðu, \hld\ brann jǫrð loga; &
ók Óðins sonr \hld\ í jǫtunhęima.\eva

\bvb Soon \inx[C]{he-goats}\footnoteB{Thunder’s cart was driven by he-goats, and he is likewise called “the lord of he-goats” in \Hymiskvida\ 20, 31. See Encyclopedia.} were driven home, hasted onto the cart-poles; they were to run well. Crags burst, the earth burned with flame; the son of Weden \ken*{= Thunder} drove into the Ettin-homes.\footnoteB{A very similar but more detailed description of Thunder driving is found in Thedwolf’s \Haustlong\ 14–16. In both poems his wagon is drawn by he-goats, causing great cosmic disturbance: crags (\emph{bjǫrg} in both) are rent asunder and fires rage before him. See also \Baldrsdraumar\ 3 for a related description of Weden riding.}\evb
\evg


\bvg
\bva Þá kvað þat Þrymr, \hld\ þursa dróttinn: &
„Standið upp, jǫtnar, \hld\ ok stráið bękki! &
Nú fǿrið mér \hld\ Fręyju at kván, &
Njarðar dóttur \hld\ ór Nóatúnum.\eva

\bvb Then quoth that Thrim, the lord of Thurses: “Stand ye up, ettins, and strew the benches! Now bring me Frow as wife; the daughter of \inx[P]{Nearth} of the \inx[L]{Nowetowns}.\evb
\evg


\bvg
\bva Ganga hér at garði \hld\ gullhyrnðar kýr, &
\edtrans{øxn alsvartir}{all-black oxen}{\Bfootnote{Formulaic, also occurring in \Hymiskvida\ 18. That all-black (that is, spotlessly black) oxen were most valued is seen by the pairing with “golden-horned”. One may also compare Saxo (I.8.12), where the hero Hadding has to atone for his slaying of a heavenly being by the bloot of dark-coloured victims (\emph{furvae hostiae}): \emph{Siquidem propiciandorum numinum gratia Frø deo rem diuinam furuis hostiis fecit. Quem litationis morem annuo feriarum circuitu repetitum posteris imitandum reliquit. Frøblod Sueones uocant.} ‘In order to mollify the divinities he [= Hadding] did indeed make a holy sacrifice of dark-coloured victims to the god Frø. He repeated this mode of propitiation at an annual festival and left it to be imitated by his descendants. The Swedes call it Frøblot.’ This ancient ritual taboo is further paralleled e.g. by the Tanakh, where animals dedicated to Yhwh were to be without blemish (\textgreek{תָּמִ֖ים}, Leviticus 1:3)}}, \hld\ jǫtni at gamni, & %TODO: Hebrew.
fjǫlð á’k męiðma, \hld\ fjǫlð á’k męnja; &
ęinnar mér Fręyju \hld\ ávant þykkir.“\eva

\bvb Here march to the estate golden-horned cows, all-black oxen, to the enjoyment of the ettin \ken*{= me}. A great deal I own of treasures, a great deal I own of necklaces; Frow alone I think myself missing.”\evb
\evg


\bvg
\bva Vas þar at kveldi \hld\ of komit snimma &
ok fyr jǫtna \hld\ ǫl framm borit. &
Ęinn át oxa, \hld\ átta laxa, &
krásir allar, \hld\ þę́r’s konur skyldu, &
drakk Sifjar verr \hld\ sáld þrjú mjaðar.\eva

\bvb There was the evening come quickly, and before the ettins ale brought forth. Ate he \ken*{= Thunder} one ox, eight salmons, and all the dainties which were meant for the women; drank the husband of Sib \ken*{= Thunder} three sieves of mead.\footnoteB{Cf. \Hymiskvida\ 15. It is rather interesting that the same kenning is used in both verses when both concern Thunder’s great eating; possibly one poet was playing on the other’s expression, or they were both referencing some now-lost work.}\evb
\evg


\bvg
\bva Þá kvað þat Þrymr, \hld\ þursa dróttinn: &
„Hvar sáttu brúðir \hld\ bíta hvassara? &
Sá’k-a brúðir \hld\ bíta ęnn bręiðara &
né ęnn męira mjǫð \hld\ męy of drekka!“\eva

\bvb Then quoth that Thrim, the lord of Thurses: “Where sawest thou brides bite sharper? Saw I never brides bite yet broader, nor yet more mead a maiden drink.”\evb
\evg


\bvg
\bva Sat in alsnotra \hld\ ambǫ́tt fyr &
es orð of fann \hld\ við jǫtuns máli: &
„Át vę́tr Fręyja \hld\ átta nǫ́ttum, &
svá vas hón óðfús \hld\ í jǫtunhęima.“\eva

\bvb Sat the allclever maid-servant \ken*{= Lock} in front, when she a word did find against the speech of the ettin: “Ate Frow naught, for eight nights; so madly was she longing for the Ettin-homes.”\evb
\evg


\bvg
\bva Laut und línu, \hld\ lysti at kyssa, &
en hann útan stǫkk \hld\ ęnd-langan sal: &
„Hví eru ǫndótt \hld\ augu Fręyju? &
Þykki mér ór \hld\ augum brenna!“\eva

\bvb He looked ’neath the linen, he lusted for a kiss, but he from the outside leapt back, across the length of the hall: “Why are the eyes of Frow fiery? Methinks there is flame coming out of the eyes!\footnoteB{Lit. “Methinks out of the eyes burn.”}”\evb
\evg


\bvg
\bva Sat in alsnotra \hld\ ambǫ́tt \edtext{fyrir}{\Afootnote{add. \emph{†ſ.†} \Regius\ is possibly a lost word.}} &
es orð of fann \hld\ við jǫtuns máli: &
„Svaf vę́tr Fręyja \hld\ átta nǫ́ttum, &
svá vas hón óðfús \hld\ í jǫtunhęima.“\eva

\bvb Sat the allclever maid-servant \ken*{= Lock} in front, when she a word did find against the speech of the ettin: “Slept Frow naught, for eight nights; so madly was she longing for the Ettin-homes.”\evb
\evg


\bvg
\bva Inn kom in arma \hld\ jǫtna systir, &
hin es brúðfjár \hld\ biðja þorði: &
„Láttu þér af hǫndum \hld\ hringa rauða &
ef þú ǫðlask vill \hld\ ástir mínar, &
ástir mínar, \hld\ alla hylli!“\eva

\bvb In came the wretched sister of the ettins, the one who for the bride-price had dared ask: “Take off from thy hands the red rings, if thou wilt win my loves; my loves, [and] all [my] \inx[C]{holdness}.”\footnoteB{The sister, who already asked for the hammer, now has the audacity to ask Thunder (still disguised as Frow) to give her the very rings on his hands.}\evb
\evg


\bvg
\bva Þá kvað þat Þrymr, \hld\ þursa dróttinn: &
„Berið inn hamar \hld\ brúði at vígja, &
leggið Mjǫllni \hld\ í męyjar kné, &
vígið okkr saman \hld\ Várar hęndi!“\eva

\bvb Then quoth that Thrim, the lord of Thurses: “Bear ye in the hammer, the bride to bless; lay Millner in the maiden’s knee, bless us two together by the hand of \inx[P]{Ware}!\footnoteB{A minor goddess presumably presiding over marriage. See Encyclopedia.}”\evb
\evg


\bvg
\bva Hló Hlórriða \hld\ hugr í brjósti &
es harðhugaðr \hld\ hamar of þękkði; &
Þrym drap hann fyrstan, \hld\ þursa dróttin, &
ok ę́tt jǫtuns \hld\ alla lamði.\eva

\bvb Laughed the heart in Loride’s chest, when, hard-hearted, he recognized the hammer. Thrim he slew first, the lord of Thurses, and all the lineage of the ettin he thrashed.\evb
\evg


\bvg
\bva Drap hann ina ǫldnu \hld\ jǫtna systur, &
hin es brúðfjár \hld\ of beðit hafði; &
hón skell of hlaut \hld\ fyr skillinga &
en hǫgg hamars \hld\ fyr hringa fjǫlð.\eva

\bvb He slew the old sister of the ettins, the one who for the bride-price had asked; she received a smiting before shillings, and a strike of the hammer before a multitude of rings.\evb
\evg


\bvg
\bva Svá kom Óðins sonr \hld\ ęndr at hamri.\eva

\bvb Thus Weden’s son regained his hammer.\evb
\evg
% Thunder
	\bookStart{Lay of Hymer}[Hymiskviða]

\begin{flushright}%
\textbf{Dating} \parencite{Sapp2022}: C10th (0.694)

\textbf{Meter:} \Fornyrdislag%
\end{flushright}%

\section{Introduction}

The \textbf{Lay of Hymer} (\Hymiskvida) is attested in both \Regius\ and \AM.  The two mss. agree very well with each other; they share the same stanzas in the same order.  The most substantial difference is the title; \AM\ has \emph{Hymis kviða} ‘the lay of Hymer’ while \Regius\ instead has \emph{Þórr dró Miðgarðs-orm} ‘Thunder pulled the Middenyardswyrm’.

\subsection{Content}

At its core \Hymiskvida\ is a comedy about Thunder’s adventures in Ettinland.  This seems to have been a popular genre, which in the Poetic Edda is also represented by \Thrymskvida\ and to some degree \Harbardsljod.  Other related stories are Thunder’s journey to Outyards-Lock in \Gylfaginning\ 44–47, his fight with Rungner in \Skaldskaparmal\ 24, and his journey to Garfrith in \Skaldskaparmal\ 26 (edited in the present edition under Eddic fragments).  These tales involve fantastical events and a fair bit of humour, and usually end with Thunder having slaughtered yet more Ettins.

\subsubsection{The otherness of the Ettins}

The Ettins are very much an \emph{other} to the Gods, and this is something which \Hymiskvida\ strongly emphasizes:

\begin{itemize}
  \item The Ettins live in the far east (st. 5) in an inhospitable, frozen climate (st. 10) of mountains (sts. 2, 17) and lavafields (sts. 36, 38);
  \item they are physically deviant: misshapen (st. 10), grey-haired (st. 16), many-headed (sts. 8, 35), having bodies harder than stone (sts. 30–31);
  \item they are likened to apes (st. 20), whales (st. 36) and Danes (st. 17, see note!);
  \item they are stingy and inhospitable (sts. 9, 16);
  \item they are snide and cowardly (sts. 19–20, 25–26, 28–32).
\end{itemize}

In general the Ettins stand in direct opposition to the Old Germanic social norms represented by the Gods; the Gods instead live in a lush green world and are young, beautiful, generous, and brave.  The one exception in the poem is Tew’s mother in st. 8, who is blonde, beautiful, and hospitable; the mother of a god must also be godlike.

As natural inferiors and a threat to the social order the Ettins must be subjugated by the Gods, and the agent of this is Thunder.  Throughout the poem he constantly humiliaties the ettins Eagre and Hymer, recurringly through completing their challenges, which follow a similar scheme: Thunder is given a dangerous or near-impossible test of strength, but quickly accomplishes it through a combination of brawn and brain, humiliating the challenger.  The challenges consist of finding an enormously large kettle (st. 3, explicitly called Eagre’s “revenge”), wrestling one of Hymer’s oxen for bait (sts. 17–18), carrying home Hymer’s whales and boat (st. 26), breaking Hymer’s finest chalice (st. 28), and perhaps also taking away the cauldron (st. 33)—though that may just be Hymer wishing to finally be rid of the pestering gods.

In the end Thunder delivers justice by slaughtering Hymer and his troop of many-headed Ettins, probably his clansmen.

\subsubsection{The fishing expedition}

At the center of the poem stands Thunder’s fishing expedition, where he gets the Middenyardswyrm on his hook but ultimately fails to catch it.  One here finds a more reverent tone than elsewhere in the poem, especially in sts. 22–24.

This myth was very popular in the Wiking Age and is dealt with in five fragmentary Scaldic poems from the 9th or 10th centuries.  These are all found in quotations in \Skaldskaparmal; they are (by their \Skp\ 3 sigla) Bragi \emph{Þórr}, ÚlfrU \emph{Húsdr} 3–6, Ǫlv \emph{Þórr}, \emph{EVald} Þórr, and Ggnæv \emph{Þórr}.  In their present state the fragments are not complete narratives, but specifically focus on Thunder in the boat facing off against the hooked Wyrm pressed against the gunwale.  They also disagree on the course of events; in some of them the staring contest ends when the cowardly Hymer cuts the fishing line and the Wyrm sinks back unscathed into the sea (the version preferred by \Gylfaginning\ 48)—in others Thunder strikes the head off the Wyrm, slaying it.

In addition to literary sources there are also numerous pictorial depictions of the myth from the Wiking Age.  These are the Swedish runestones from Altuna (U 1611) and Linga (Sö 352), several Jutlandic picture stones from Hørdum, a Cumbrian picture stone from Gosforth, and the Gotlandic picture stone GP 21 from Ardre church.  The images depict the same scene as the Scaldic fragments: Thunder stands in the boat above the hooked Wyrm, often depicted as a fish; next to him is one companion.  Some of them have additional details like the use of the ox-head for bait (U 1611, Sö 352), or Thunder’s foot going through the boat (U 1611, Hørdum).

Other than \Hymiskvida\ the only complete retelling of the myth is found in \Gylfaginning\ 48, which may be summarized as follows:

{\small Thunder goes out alone into Middenyard in the shape of a young man (\emph{ungr dręngr}) without his goats and chariot.  In the evening he comes to the ettin Hymer and asks to stay the night.  At dawn Hymer plans to go fishing and Thunder asks to join him.  Hymer insults Thunder's small size and youth, and warns him that he usually takes long and arduous trips.  Thunder, angered, says that he will row very far, and then asks Hymer what bait they will use.  Hymer tells him to find it himself and so he turns to the flock of oxen, where he tears off the head from the greatest ox, one called Heavenrid (\emph{Himin-hrjóðr}).

The two go out to sea, and Thunder rows far past Hymer’s usual fishing waters.  Hymer, unhappy, warns him that if they row any further out they will be in danger of the Middenyardswyrm, but Thunder keeps on.  After some time he puts down the oars, readies his fishing line, hooks the ox-head and lowers it.  The Wyrm soon bites, and struggles so hard that Thunder is pressed against the gunwale.  In rage he brings himself into his Os-might (\emph{ás-męgin}) and pulls back with such force that his feet go through the bottom of the ship and press into the seabed.  The Wyrm's head goes up against the gunwale.  The two enemies ferociously stare at each other, Thunder “sharpening his eyes” and the Wyrm spitting venom.  Hymer is frightened, reaches for his bait-cutting knife, and cuts the line—the Wyrm then sinks back into the sea.  Thunder throws his hammer after it, “and men say that he struck off the monster’s head, but I think it true to tell thee that the Middenyardswyrm still lives and is lying in the outer sea.”  Thunder gives Hymer a punch to the ear so that he flies headfirst overboard; the god then wades back to land.}

This account is clearly based on multiple sources, certainly including the Scaldic fragments cited in \Skaldskaparmal.  It is hard to say whether Snorre had access to \Hymiskvida; the closest agreement is when it is said that \emph{Miðgarðs-ormr gein yfir uxa-hǫfuð’it, en ǫngull’inn vá í góm’inn orm’inum} ‘The Middenyardswyrm snapped at the ox-head and the hook went into the roof of the wyrm’s mouth’, which has some resemblance to st. 22, but it is not conclusive.  Some details must derive from now-lost texts available to Snorre: the detail of Thunder’s feet going through the boat is also found on the Swedish Altuna stone and the Danish Hørdum stone (but see note to st. 34/2 below), and the name Heavenrid is attested in \inx[C]{thule}[thules] listing names of oxen.

\sectionline

More broadly, the fishing is the Norse-Germanic version of the fight between Storm-god and Dragon which is found in a great many mythologies.  Some important examples include Indra and Vr̥tra in the Vedic (\Rigveda\ 1.32 et. c.), Marduk and Tiamat in the Babylonian (\emph{Enūma Eliš}), Zeus and Typhon in the Greek, and Yahweh and Leviathan in the Jewish (TODO: references).  With these analogies in mind it seems that the versions where Thunder slays the Wyrm probably reflect an older layer of Germanic mythology; at some point the fight between Thunder and the Wyrm was moved to the End Times (see \Voluspa\ 54), and then the Wyrm had to still be alive.

\subsubsection{\Hymiskvida\ as a composite}

In \Hymiskvida\ one can roughly identify the following strands:

\begin{enumerate}
  \item 1–6 Thunder tells the ettin Eagre to host a banquet for the Gods; Eagre in turn asks for a cauldron big enough to brew enough ale for them all.
  \item 7–16 Thunder and Tew go to visit Tew’s father, the stingy ettin Hymer, who owns such a cauldron; horrified at Thunder’s great appetite during the evening he tells them that they must go fishing for food.
  \item 17–19 Thunder says that he will do it, if he is given bait; Hymer challenges him to kill one of his oxen; Thunder tears off the head from one of them.
  \item 20–25 The three go fishing; Hymer pulls up some whales; with the ox-head as bait Thunder manages to hook the Middenyardswyrm itself; he loses it.
  \item 26–27 Hymer challenges Thunder to carry the boat and whales back to his farm; he does.
  \item 28–32 Hymer challenges Thunder to break a supposedly indestructible chalice; he succeeds by smashing it against the ettin’s forehead.
  \item 33–36 Thunder and Tew depart with the cauldron; they find themselves followed by Hymer and his ettins; Thunder kills them all.
  \item 37–38 One of Thunder’s goats goes halt.
  \item 39 Thunder returns to the Gods with Hymer’s cauldron; they host a banquet.
\end{enumerate}

The fishing expedition as found in the Scaldic fragments and \Gylfaginning\ 48 is represented by 3–4.  \Hymiskvida\ is the only source that places it within the context of Thunder and Tew obtaining a huge cauldron from Hymer for the sake of brewing ale, and also scatters several other strange incidents throughout.  It seems most likely, both from comparative mythology and the other sources just mentioned, that these additional narratives originally had nothing to do with Thunder’s encounter with the Wyrm.  They have instead been woven together into a single narrative late, probably by the poet himself for the sake of a more entertaining and complete story, but he has not been entirely successful in this, and there are a few loose strands.  The halt goat of sts. 37–38 finds a parallel in \Gylfaginning\ 44, where it serves as the origin story of Thunder’s two servants who are to play an important part in the narrative, but it is here an entirely superfluous detail, something the poet even anticipates in his direct address of the audience.  It is also strange that Lock should appear at that point, since he is never mentioned before or since in the poem.

A further proof of this is that the god Tew plays no role at all in the fishing expedition: he is last alluded to in st. 16 where Hymer speaks of “[us] three”, and then reappears in st. 33 where he fails to lift the cauldron.  The simplest explanation for this is that he originally had nothing to do with it; his role is to bridge the cauldron-narrative and the fishing expedition.  In the other variants of the latter, Thunder only has one companion, Hymer; this includes the pictoral depictions, which only show two figures on the boat.  It is also strange that he does not react at all to the murder of his father in front of him, although that fact is also in doubt; in \Skaldskaparmal\ 16 Tew is called the son of \inx[P]{Weden}.

\subsection{Style}

When speaking of a composite poem, one must distinguish between a text where several originally different works have been placed together mostly unchanged, and a text composed by a single author drawing from multiple sources.  A likely example of the former category is \Havamal, but \Hymiskvida\ undoubtedly belongs to the latter.  It has a distinct style and meter throughout which is unlike anything else in the Poetic Edda; indeed, the sharpest contrast is with the poem most similar content-wise, \Thrymskvida.  Where \Thrymskvida\ is written in a rustic style with fairly loose \Fornyrdislag\ meter and few kennings, \Hymiskvida\ uses an unusually strict, almost syllable-counting \Fornyrdislag, and fills the stanzas with ornate kennings, difficult grammatical constructions, and highly unnatural word order (see especially sts. 16, 20, and 39).

These are all traits one associates more closely with Scaldic poetry in intricate measures like \Drottkvett\ than Eddic poetry in \Fornyrdislag, and it seems clear that the anonymous poet of \Hymiskvida\ was highly trained in the Scaldic art and familiar with compositions in that genre.  Two kennings (17/4: \emph{brjótr berg-Dana}, 22/4: \emph{umb-gjǫrð allra landa}) are shared identically with Scaldic poems in \Drottkvett, and the direct address to the audience in st. 38 is otherwise only ever found in Scaldic poetry.

Metrically the poet has a particular fondness for four-syllable half-lines where the first two are heavy and the third is light, e.g. \emph{ør-kost hvera}.  This also explains his love of having the two-syllable preposition \emph{fyrir} ‘before, in front of’ at the end of a half-line, something done 6 times here—much more frequently than in any other \Fornyrdislag\ poem of the Poetic Edda.

\sectionline

\section{The Lay of Hymer}

\bvg\bva\mssnote{\Regius~13v/26, \AM~5v/25}%
Ár \alst{v}al-tívar \hld\ \alst{v}ęiðar nǫ́mu &
ok \alst{s}umbl-\alst{s}amir \hld\ \edtrans{áðr \alst{s}aðir yrði}{before they might eat}{\Bfootnote{Lit. “might become sated”.}}, &
\alst{h}ristu tęina \hld\ ok á \alst{h}laut sǫ́u, &
fundu at \alst{Ę́}gis \hld\ \alst{ø}r-kost hvera.\eva

\bvb Of yore the slain-Tews \name{Gods} had caught game, \\
and gathered at the \inx[C]{simble} before they might eat \\
they shook the twigs and looked at the \inx[C]{leat}; \\
they found at Eagre’s a great choice of cauldrons.\footnoteB{The Gods sprinkled the leat (\emph{hlaut} ‘sacrificial blood’) of the beasts and interpreted the pattern; they found it most auspicious to feast at Eagre’s. TODO: reference to leat-twigs.}\evb\evg


\bvg\bva\mssnote{\Regius~13v/28, \AM~5v/27}%
Sat \alst{b}erg-\alst{b}úi \hld\ \alst{b}arn-tęitr fyrir, &
\alst{m}jǫk glíkr \edtrans{\alst{m}ęgi \hld\ \alst{M}iskur-blinda}{lad of Misherblind}{\Bfootnote{An unexplained reference.  Misherblind might be another name for Firneet, Eagre’s father, in which case the line would be a tautology: “he looked much like himself”.}}, &
lęit í \alst{au}gu \hld\ \alst{Y}ggs barn í þrá: &
„þú skalt \alst{ǫ́}sum \hld\ \alst{o}pt sumbl \edtext{gęra}{\lemma{gęra ‘make’}\Afootnote{\emph{gefa} ‘give’ \AM}}!“\eva

\bvb The crag-dweller \ken*{\textsc{ettin} = Eagre} sat merry like a child before [him] \\
much alike to the lad of Misherblind. \\
Into his eyes looked the Ug’s \name{Weden’s} child \ken*{= Thunder} stubbornly: \\
“Thou shalt for the Eese oft make simbles!”\footnoteB{Having seen that Eagre has a great store of cauldrons, Thunder orders him to brew ale for the feasts of the Eese.}\evb\evg


\bvg\bva\mssnote{\Regius~13v/31, \AM~5v/29}%
\alst{Ǫ}nn fekk \alst{jǫ}tni \hld\ \alst{o}rð-bę́ginn halr, &
\alst{h}ugði at \alst{h}efndum \hld\ \alst{h}ann nę́st við goð, &
bað \alst{S}ifjar ver \hld\ \alst{s}ér fǿra hver, &
„þann’s ek \alst{ǫ}llum \alst{ǫ}l \hld\ \alst{y}ðr of hęita.“\eva

\bvb Great toil for the ettin the word-peevish man \ken*{= Thunder} caused; \\
he \ken*{= Eagre} thought of revenge, soon, against the god. \\
He bade Sib’s husband \ken*{= Thunder} bring him a cauldron, \\
“that one with which I for you all ale might warm.\footnoteB{Eagre gets back at Thunder by telling him that he needs a single cauldron which can hold enough ale to supply all the Eese.}”\evb\evg


\bvg\bva\mssnote{\Regius~14r/1, \AM~5v/30}%
Né þat \alst{m}ǫ́ttu \hld\ \alst{m}ę́rir tívar &
ok \alst{g}inn-ręgin \hld\ of \alst{g}eta hvęr-gi, &
unds af \alst{t}ryggðum \hld\ \alst{T}ýr Hlórriða &
\alst{ǫ́}st-ráð mikit \hld\ \alst{ęi}num sagði:\eva

\bvb That one could not the renowned \inx[G]{Tews} \\
and the \inx[G]{yin-Reins} anywhere get hold of— \\
until, out of loyalty, Tew to Loride \name{= Thunder} \\
a great loving counsel told alone:\evb\evg


\bvg\bva\mssnote{\Regius~14r/3, \AM~6r/2}„Býr fyr \alst{au}stan \hld\ \alst{É}li-vága &
\edtrans{\alst{h}und-víss}{hundred-wise}{\Bfootnote{Alternatively “hound-wise”; the prefix simply means “very”.}} \alst{H}ymir \hld\ at \alst{h}imins ęnda, &
á \alst{m}inn faðir \hld\ \alst{m}óðugr kętil, &
\edtext{\alst{r}úm-brugðinn}{\Afootnote{\emph{†rumbrygðan†} \AM}} hver \hld\ \alst{r}astar djúpan.“\eva

\bvb “Dwells to the east of the \inx[L]{Ilewaves} \\
the hundred-wise Hymer, at heaven’s end.\footnoteB{According to \Vafthrudnismal\ 31 the Ilewaves were the poisonous wild rushes from which the ettins emerged, and so it makes sense that they would be found in the east, where the ettins dwell.  That Hymer should dwell even to the east of them then illustrates his unusual ettin-ness.} \\
Owns my father \ken*{= Hymer}, fierce, a kettle: \\
a size-famed cauldron one \inx[C]{rest} deep.”\evb\evg


\bvg\bva\speakernote{[Þórr kvað:]}\mssnote{\Regius~14r/4, \AM~6r/4}%
„Vęitst, ef \alst{þ}iggjum \hld\ \alst{þ}ann lǫg-velli?“ &
\speakernote{[Týr kvað:]}„Ef, \alst{v}inr, \alst{v}élar \hld\ \alst{v}it gørvum til!“\eva

\bvb\speakernoteb{[Thunder quoth:]}%
“Knowest thou if we will receive that liquid-boiler \ken{cauldron}?” — \\
\speakernoteb{[Tew quoth:]}%
“If, friend, we two make use of wiles!”\footnoteB{Like elsewhere in this poem the speakers are not indicated, but it is most sensible that Thunder asks and Tew answers.}\evb\evg


\bvg\bva\mssnote{\Regius~14r/5, \AM~6r/4}%
Fóru \alst{d}rjúgum \hld\ \edtrans{\alst{d}ag þann framan}{from the beginning of the day}{\Afootnote{emend. after \textcite{FinnurEdda}; \emph{dag þann fram} ‘on that day forth’ \Regius; \emph{dag fráliga} ‘swiftly at day’ \AM}} &
\alst{Á}sgarði frá \hld\ unds til \edtrans{\alst{Ę}gils}{Eyel}{\Afootnote{so \Regius; \emph{Ę́gis} ‘Eagre’ \AM\ is probably from confusion with Eagre (the ettin) described earlier in the poem, though the shepherd may have shared his name.}} kvǫ́mu; &
\edtrans{\alst{h}irði \alst{h}afra \hld\ \alst{h}orn-gǫfgasta}{he kept the he-goats noblest of horns}{\Bfootnote{Eyel is not otherwise known but he seems to have been familiar to the original audience.  In any case he takes possession of Thunder’s two goats until he returns.}}; &
\alst{h}urfu at \alst{h}ǫllu \hld\ es \alst{H}ymir átti.\eva

\bvb They journeyed far from the beginning of the day, \\
away from Osyard, until to Eyel they came— \\
he kept the he-goats noblest of horns— \\
they turned to the hall which Hymer owned.\evb\evg


\bvg\bva\mssnote{\Regius~14r/7, \AM~6r/6}%
\alst{M}ǫgr fann ǫmmu, \hld\ \alst{m}jǫk lęiða sér, &
\edtrans{\alst{h}afði \alst{h}ǫfða \hld\ \alst{h}undruð níu}{of heads she had nine hundred}{\Bfootnote{Malformed bodies, especially with a deviant number of body parts, are typical of ettins.  Other examples include a three-headed thurse in \Skirnismal\ 31, the nine-headed ettin Thriwold (Bragi Frag 3 in \Skp\ 3), and the eight-armed Starked Eeldreng.  Cf. Introduction and st. 35 below.}}. &
en \edtrans{\alst{ǫ}nnur}{another woman}{\Bfootnote{The use of the word “son” in the following line reveals this as Tew’s mother.  The poet stresses her beauty of dress and countenance, in contrast to the grandmother.}} gekk \hld\ \alst{a}l-gullin framm &
\alst{b}rún-hvít \alst{b}era \hld\ \alst{b}jór-vęig syni:\eva

\bvb The lad \ken*{= Tew} found his grandmother very loathsome; \\
of heads she had nine hundred. \\
But another woman, all-golden, walked forth, \\
white-browed, bringing a beer-draught for [her] son \ken*{= Tew}:\evb\evg


\bvg\bva\speakernote{[Týs móðir:]}\mssnote{\Regius~14r/9, \AM~6r/8}%
„\alst{Á}tt-niðr \alst{jǫ}tna \hld\ \alst{e}k vilja’k ykkr &
\alst{h}ug-fulla tvá \hld\ und \alst{h}vera sętja; &
es \alst{m}ínn \edtrans{fríi}{lover}{\Afootnote{so \Regius; \emph{faðir} ‘father’ \AM}} \hld\ \alst{m}ǫrgu sinni &
\edtext{\alst{g}løggr við \alst{g}ęsti \hld\ \alst{g}ǫrr ills hugar}{\lemma{gløggr \dots\ hugar ‘stingy \dots\ mood’}\Bfootnote{Ettins are characteristically inhospitable, in stark opposition to the Old Germanic social norms; see Introduction to the poem above.  This statement foreshadows the later hunting expedition starting at st. 16 below.}}.“\eva

\bvb\speakernoteb{[Tew’s mother:]}“O clansman of ettins \ken*{= Tew}! I would wish to put \\
you two, full of heart, beneath the cauldrons. \\
Many a time has my lover \ken*{= Hymer} been \\
stingy with guests, quick to ill mood.”\evb\evg


\bvg\bva\mssnote{\Regius~14r/11, \AM~6r/9}%
En \alst{v}á-skapaðr \hld\ \alst{v}arð \edtrans{síð-búinn}{come late}{\Afootnote{om. \AM}}, &
\alst{h}arð-ráðr \alst{H}ymir, \hld\ \alst{h}ęim af vęiðum; &
\alst{g}ekk inn í sal, \hld\ \alst{g}lumðu \edtrans{jǫklar}{icicles}{\Bfootnote{In Hymer’s frozen beard.  In modern Icelandic the word \emph{jökull} has come to mean ‘glacier’, but its original sense (as found here) is that of its English cognate “icicle”.}}, &
vas \alst{k}arls, es \alst{k}om, \hld\ \alst{k}inn-skógr frørinn.\eva

\bvb And the misshapen one was come late, \\
hard-minded Hymer, home from the hunt. \\
He entered the hall; icicles clattered; \\
on the churl who came was the cheek-shaw \ken{beard} frozen.\evb\evg


\bvg\bva\speakernote{[Týs móðir:]}\mssnote{\Regius~14r/13, \AM~6r/11}%
„\edtext{Ves þú \alst{h}ęill, \alst{H}ymir, \hld\ í \alst{h}ugum góðum!}{\lemma{Ves þú hęill, \dots\ í hugum góðum! ‘Be thou hale \dots\ in good spirits!’}\Bfootnote{A formulaic greeting; cf. the almost identical greeting in \emph{N B380} (edited below under Galders).  Further afield cf. the type exemplified by \Beowulf\ 407a: \emph{Wæs þú, Hróðgâr, hâl} ‘Be thou, Rothgar, hale!’}} &
Nú ’s \alst{s}onr kominn \hld\ til \alst{s}ala þinna, &
sá’s \alst{v}it \alst{v}ę́ttum \hld\ af \alst{v}egi lǫngum; &
fylgir \alst{h}ǫ́num \hld\ \alst{H}róðrs and-skoti, &
\alst{v}inr \alst{v}er-liða; \hld\ \edtrans{\alst{V}éurr}{Wighward}{\Bfootnote{The guardian of \inx[C]{wigh}[wighs] (sanctuaries), a name of Thunder.}} hęitir sá.\eva

\bvb\speakernoteb{[Tew’s mother:]}“Be thou hale, Hymer, in good spirits! \\
Now the son has come to thy halls, \\
he whom we awaited, from a long way off. \\
Him follows the Rooder’s opponent \ken*{= Thunder}, \\
the friend of manly retinues—\inx[P]{Wighward} is he called.\evb\evg


\bvg\bva\mssnote{\Regius~14r/15, \AM~6r/13}\alst{S}é þú hvar \alst{s}itja \hld\ und \alst{s}alar gafli, &
\alst{s}vá \edtext{forða \alst{s}ér}{\Afootnote{\emph{forðask} \AM}}, \hld\ stęndr \edtrans{\alst{s}úl}{column}{\Afootnote{\emph{†sol†} \AM}} fyrir.“ &
\alst{S}undr stǫkk \alst{s}úla \hld\ fyr \alst{s}jón jǫtuns, &
en \edtext{\alst{a}llr}{\Afootnote{emend.; \emph{áðr} ‘earlier, before that’ \Regius\AM. TODO: elaborate, mention Finnur}} í tvau \hld\ \alst{á}ss brotnaði.\eva

\bvb See where they sit beneath the hall’s gable: \\
so they save themselves—a column stands before [them]!” \\
The column crashed down before the ettin’s gaze, \\
and all in two the roof-beam broke.\evb\evg


\bvg\bva\mssnote{\Regius~14r/17, \AM~6r/15}Stukku \alst{á}tta, \hld\ en \alst{ęi}nn af þęim &
\alst{h}verr \alst{h}arð-slęginn \hld\ \alst{h}ęill af þolli; &
\alst{f}ramm gingu þęir, \hld\ en \alst{f}orn jǫtunn &
\alst{s}jónum lęiddi \hld\ \alst{s}inn and-skota.\eva

\bvb Eight [cauldrons] crashed down, but one of them, \\
a hard-forged cauldron, [came] whole off its peg.\footnoteB{Nine cauldrons were hanging from the roof-beam supported by the column.  Eight of them broke and one remained whole, presumably the one they were looking for.} \\
Forth they went, but the ancient ettin \\
with his gaze tracked his opponent.\evb\evg


\bvg\bva\mssnote{\Regius~14r/19, \AM~6r/16}\edtrans{Sagði-t \alst{h}ǫ́num \hld\ \alst{h}ugr vęl}{His heart did not please him}{\Bfootnote{Lit. ‘his heart did not speak well to him’.}} þá’s sá &
\alst{g}ýgjar \edtrans{\alst{g}rǿti}{distresser}{\Afootnote{\emph{gę́ti} ‘keeper, warder’ \AM}} \hld\ á \alst{g}olf kominn, &
\alst{þ}ar vǫ́ru \alst{þ}jórar \hld\ \alst{þ}rír of tęknir, &
bað \edtrans{\alst{s}ęnn}{at once}{\Afootnote{\emph{sun} ‘[his] son \ken*{= Tew}?’ \AM}} jǫtunn \hld\ \alst{s}jóða ganga.\eva

\bvb His heart did not please him when he saw \\
the \inx[C]{gow}’s distresser \ken*{= Thunder} come on the floor. \\
There were three bulls a-taken: \\
the ettin bade them at once go cooking.\evb\evg


\bvg\bva\mssnote{\Regius~14r/21, \AM~6r/18}\alst{H}vęrn létu þęir \hld\ \alst{h}ǫfði skęmra &
auk á \alst{s}ęyði \hld\ \alst{s}íðan bǫ́ru, &
át \alst{S}ifjar verr \hld\ áðr \alst{s}ofa gingi, &
\alst{ęi}nn með \alst{ǫ}llu \hld\ \alst{ø}xn tvá Hymis.\eva

\bvb Each one they let shorten by a head, \\
and onto the cooking-pit then did bear: \\
Sib’s husband \ken*{= Thunder} ate—before he might go sleep— \\
alone by himself two of Hymer’s oxen.\footnoteB{Cf. \Thrymskvida\ 24 for another instance of Thunder’s great eating, which curiously also uses the kenning \emph{Sifjar verr} ‘Sib’s husband \ken*{= Thunder}’.}\evb\evg


\bvg\bva\mssnote{\Regius~14r/23, \AM~6r/19}Þótti \alst{h}ǫ́rum \hld\ \alst{H}rungnis spjalla &
\alst{v}erðr Hlórriða \hld\ \alst{v}ęl full-mikill, &
\edtext{„munum at \alst{a}ptni \hld\ \alst{ǫ}ðrum verða &
\alst{v}ið \alst{v}ęiði-mat \hld\ \alst{v}ér þrír lifa.“}{\lemma{munum \dots\ lifa. ‘the next \dots\ live.’}\Bfootnote{The poet is pushing at the limits of Old Norse syntax.  In prose word order it should be construed as: \emph{at ǫðrum aptni munum vér þrír verða lifa við vęiði-mat}, where \emph{verða} ‘have to, must’ is used like its modern German cognate \emph{werden}.

Hymer’s stinginess—he refuses to share more of his own food but instead forces his guests to go hunt—breaks all Indo-European rules of hospitality and illustrates the otherness of the Ettins.  See the Introduction above.}}\eva

\bvb To Rungner’s hoary friend \ken*{= Hymer} did seem \\
Loride’s \name{Thunder’s} eating far too great; \\
“the next evening we three will \\
on game-meat have to live.”\evb\evg


\bvg\bva\mssnote{\Regius~14r/24, \AM~6r/21}\alst{V}éurr kvaðsk \alst{v}ilja \hld\ á \alst{v}ág róa, &
ef \alst{b}allr jǫtunn \hld\ \alst{b}ęitur gę́fi. &
„\alst{H}verf þú til \edtext{\alst{h}jarðar}{\Afootnote{\emph{hallar} corr. \AM}}, \hld\ ef \alst{h}ug trúir, &
\edtrans{\alst{b}rjótr \alst{b}erg-Dana}{breaker of boulder-Danes \ken*{\textsc{ettins} > = Thunder}}{\Bfootnote{This kenning for Thunder also occurs in \Haustlong\ 18.  The ettin-kenning emphasises their otherness (see Introduction to the poem above) by equating them with ethnic foreigners.  Cf. \Thorsdrapa, where ettins are called Scots, Swedes, Danes, Ruges and Hareds; all peoples hostile to the Norwegian Earl Hathkin, at whose court that poem may have been composed.}}, \hld\ \alst{b}ęitur sǿkja.\eva

\bvb Wighward called himself willing to row on the wave, \\
if the stubborn ettin might give pieces of bait. \\
“Turn to the herd—if thou trust in thy heart, \\
O breaker of boulder-Danes \ken*{\textsc{ettins} > = Thunder}—to seek pieces of bait.\evb\evg


\bvg\bva\mssnote{\Regius~14r/26, \AM~6r/23}\alst{Þ}ess \edtext{vę́ntir mik}{\Afootnote{so \AM; \emph{vę́nti ek} \Regius}}, \hld\ at \alst{þ}ér \edtrans{myni-t}{will not}{\Afootnote{so \AM; \emph{myni} ‘will’ \Regius.  The \AM\ reading is preferable since it makes this the first of Hymer’s several challenges of strength to Thunder, which the god, to the ettin’s humiliation, easily accomplishes.}} &
\alst{ǫ}gn at \alst{o}xa \hld\ \alst{au}ð-feng vesa.“ &
\edtrans{\alst{S}vęinn}{The swain}{\Bfootnote{Thunder was apparently in the shape of a young boy.  This detail is also found in \Gylfaginning\ 48: \emph{Gekk hann út of Miðgarð svá sem ungr drengr \dots} ‘He went out about Middenyard in the shape of a young man’.}} \alst{s}ýsliga \hld\ \alst{s}vęif til skógar, &
þar’s \edtext{\alst{o}xi stóð \hld\ \alst{a}l-svartr}{\lemma{oxi \dots\ al-svartr ‘ox \dots\ all-black’}\Bfootnote{Formulaic, also occuring in \Thrymskvida\ 23; see note there for further parallels to the custom of sacrificing animals of certain colours.  It seems that all-black oxen were thought the noblest, and so Thunder’s slaying one instead of an inferior beast is probably intended to humiliate the stingy Hymer.

In \Gylfaginning\ 48 we read that: \emph{Hann tók inn mesta uxa’nn, er Himin-hrjóðr hét, ok sleit af hǫfuð’it ok fór með til sjávar.} ‘He took the greatest ox, which was called Heavenrid, and tore off its head and went with it to the sea’.}} fyrir.\eva

\bvb I ween that the baits from the ox \\
will not be an easy catch for thee!”— \\
The swain \ken*{= Thunder} swiftly turned to the wood, \\
where an ox stood, all-black, before [him].\evb\evg


\bvg\bva\mssnote{\Regius~14r/28, \AM~6r/24}Braut af \alst{þ}jóri \hld\ \alst{þ}urs ráð-bani &
\alst{h}ǫ́-tún ofan \hld\ \alst{h}orna tveggja. &
„\alst{V}erk þikkja þín \hld\ \alst{v}erri myklu &
\alst{k}jóla valdi \hld\ an \alst{k}yrr sitir.“\eva

\bvb From the bull broke the thurse’s death-planner \ken*{= Thunder} \\
the high meadow of the two horns \ken{head} from above.— \\
“Worse by far thy works do seem \\
to the wielder of ships \ken*{= Hymer = me} than if thou didst sit calm!”\evb\evg

\sectionline

{\small (The scene now shifts, and the party is out at sea.  It is possible that a stanza has here been lost, or that it would be indicated in some other way in the original performance.)}

\sectionline

\bvg\bva\mssnote{\Regius~14r/30, \AM~6r/26}%
Bað \alst{h}lunn-gota \hld\ \alst{h}afra dróttinn &
\edtext{\alst{á}tt-runn}{\Afootnote{\emph{†atrænn†} \AM}} \edtrans{\alst{a}pa}{ape}{\Bfootnote{The specific sense of \emph{api} ‘ape’ is uncertain.  It seems to generally refer to a fool, but see Encyclopedia.}} \hld\ \alst{ú}tar fǿra, &
\edtext{en \alst{s}á jǫtunn \hld\ \alst{s}ína \edtext{talði}{\Afootnote{\emph{milldi} corr. \AM}}, &
\alst{l}ítla fýsi \hld\ \edtext{\alst{l}ęngra at róa}{\Afootnote{metr. emend.; \emph{at róa lęngra} \Regius\AM}}.}{\lemma{en \dots\ róa. ‘but \dots\ longer.’}\Bfootnote{Thunder’s humorous humiliation of Hymer continues with the snide ettin now forced to row against his will.}}\eva

\bvb The Lord of He-goats \ken*{= Thunder} bade the kinsman of the \inx[C]{ape}\ \ken*{\textsc{ettin} = Hymer} \\
push the launcher-steed \ken{boat} further out, \\
but that ettin told of his \\
scarce wish to row longer.\evb\evg


\bvg\bva\mssnote{\Regius~14r/31, \AM~6r/27}Dró \edtrans{\alst{m}ę́rr}{famous}{\Afootnote{so \Regius; \emph{męir} ‘more, further’ \AM}} Hymir \hld\ \alst{m}óðugr hvala &
\alst{ęi}nn á \alst{ǫ}ngli \hld\ \alst{u}pp sęnn tváa; &
en \alst{a}ptr í skut \hld\ \alst{Ó}ðni sifjaðr &
\alst{V}éurr við \alst{v}élar \hld\ \alst{v}að gęrði sér.\eva

\bvb Famous, fierce Hymer pulled whales: \\
one on the hook, soon up two. \\
But back in the stern the Weden-related \\
Wighward craftily fixed his line.\evb\evg


\bvg\bva\mssnote{\Regius~14v/1, \AM~6r/29}\alst{Ę}gnði á \alst{ǫ}ngul \hld\ sá’s \alst{ǫ}ldum bergr, &
\alst{o}rms \alst{ęi}n-bani \hld\ \alst{o}xa hǫfði; &
\alst{g}ęin við \edtrans{agni}{bait}{\Afootnote{so \AM; \emph{ǫngli} ‘hook’ \Regius}} \hld\ sú’s \alst{g}oð fía &
\edtext{\alst{u}mb-gjǫrð neðan \hld\ \alst{a}llra landa.}{\lemma{umb-gjǫrð \dots\ allra landa ‘engirdler of all lands’}\Bfootnote{Also found in a fragment by Alewigh Snub (\Skp: Ǫlv \emph{Þórr}) quoted in \Skaldskaparmal\ 11: \emph{\alst{Ǿ}stisk \alst{a}llra landa \hld\ \alst{u}mb-gjǫrð ok sonr Jarðar.} ‘The engirdler of all lands and the son of Earth surged.’  Cf. also the Wyrm-kenning in Braye’s fragment quoted in the same chapter (\Skp: Bragi \emph{Þórr} 3): \emph{ęndi-sęiðr allra landa} ‘boundary-saithe of all lands’.

The poetic juxtaposition between the Storm-god and the Wyrm may be very old; cf. \Rigveda\ 1.32.13c: \emph{Índraś ca yád yuyudhā́tay Áhiś ca} ‘When Indra and the Wyrm (\emph{áhi}) fought each other.’}}\eva

\bvb Baited on the hook he who rescues men \ken*{= Thunder}—  \\
the Wyrm’s lone slayer—the ox’s head. \\
Snapped at the bait the one whom the Gods hate \ken*{= Middenyardswyrm}— \\
the engirdler of all lands—from below.\evb\evg


\bvg\bva\mssnote{\Regius~14v/3, \AM~6v/1}\alst{D}ró \alst{d}jarf-liga \hld\ \alst{d}áð-rakkr Þórr &
\alst{o}rm \alst{ęi}tr-fáan \hld\ \alst{u}pp at borði; &
\alst{h}amri kníði \hld\ \edtrans{\alst{h}ǫ́-fjall skarar}{high mountain of hair \ken{head}}{\Bfootnote{A rather unfitting kenning, since serpents do not have hair.}} &
\alst{o}f-ljótt \alst{o}fan \hld\ \alst{u}lfs hnit-bróður.\eva

\bvb Bravely pulled deed-ready Thunder \\
the venom-gleaming Wyrm up on the gunwale. \\
With the hammer he struck the high mountain of hair \ken{head}— \\
very hideous, from above—on the Wolf’s clash-brother \ken*{= Middenyardswyrm}.\evb\evg


\bvg\bva\mssnote{\Regius~14v/5, \AM~6v/2}\edtrans{\alst{H}raun-gǫlkn}{The desert-monsters}{\Bfootnote{Both mss. have \emph{hręin-}, which may mean either ‘clean’ or ‘reindeer’, neither of which fit. On the other hand \emph{hraun} \ONP: ‘stone/barren area, wasteland; lavafield’ is well attested in Scaldic kennings for ettins. The precise meaning of \emph{galkn} ‘monster’ (plural \emph{gǫlkn}) is unclear; but it is attested in three Scaldic verses, always in kennings of the type “troll-woman of the shield \ken{axe}”.  While the mss. spelling ‘\emph{galkn}’ (norm. \emph{gálkn}) could reflect either singular and plural, the form of the verb is plural.  This means that the word cannot be referring to the Middenyardswyrm, refuting the interpretation of \textcite{LarringtonEdda}: “the sea-wolf shrieked”.}} \edtext{\alst{h}rutu}{\Afootnote{so \AM; \emph{hlumðu} ‘dashed’ \Regius. End-rhyme is also used by the poet in st. 3/3.}}, \hld\ ęn \alst{h}ǫlkn þutu, &
\alst{f}ór hin \alst{f}orna \hld\ \alst{f}old ǫll saman; &
\edtext{[...]}{\Bfootnote{It is very likely that a line is missing here, since the stanzas in the poem otherwise consistently have four lines.  In other tellings of the myth it is at this point that Hymer cuts Thunder’s fishing line, so that is probably what has been lost.

For the reader’s enjoyment, based on other poets and \Gylfaginning\ 48, the translator has composed the following variant lines: \emph{unds vinr Hrungnis \hld\ vað Þórs of skar} ‘until the friend of Rungner \ken*{= Hymer} Thunder’s fishing-line did cut’; \emph{unds fǫlr Hymir \hld\ fekk á saxi} ‘until pale Hymer grasped the knife’.}} &
\alst{s}økkðisk \alst{s}íðan \hld\ \alst{s}á \edtrans{fiskr}{fish}{\Bfootnote{The Middenyardswyrm may also be called a fish in \Grimnismal\ 21; see note there.  In Scaldic sources it is often called a saithe (\emph{sęiðr}).}} í mar.\eva

\bvb The desert-monsters \ken{ettins} bounded and the bedrock resounded; \\
the ancient earth moved all at once. \\
{[...]}; \\
sank thereafter that fish \ken*{= Middenyardswyrm} into the sea.\evb\evg


\bvg\bva\mssnote{\Regius~14v/6, \AM~6v/3}%
\alst{Ó}-tęitr \alst{jǫ}tunn, \hld\ es \alst{a}ptr røru, &
\edtext{[...]}{\Bfootnote{Another likely missing line.  As said in the previous stanza the meter usually requires four lines; more importantly the first half of the sentence is incomplete without a verb.}} &
svá’t \edtrans{\alst{á}r}{in early morn}{\Bfootnote{\textcite{FinnurEdda}\ suggests \emph{svá’t at ǫ́r} ‘so that by the oar’, but this burdens the strict meter.  Assuming the present interpretation is correct, the three would have been out fishing throughout the night.}} Hymir \hld\ \alst{ę}kki mę́lti, &
\alst{v}ęifði rǿði \hld\ \alst{v}eðrs annars til.\eva

\bvb The unmerry ettin \ken*{= Hymer}, as they rowed back, \\
{[...]}, \\
so that in early morn Hymer said nothing; \\
he pulled the oar against the wind:\evb\evg


\bvg\bva\speakernote{[Hymir:]}\mssnote{\Regius~14v/8, \AM~6v/4}%
„Munt of \alst{v}inna \hld\ \alst{v}erk halft við mik, &
at \alst{h}ęim \alst{h}vali \hld\ \alst{h}af til bǿjar &
eða \alst{f}lot-brúsa \hld\ \alst{f}ęstir okkarn.“\eva

\bvb\speakernoteb{[Hymer quoth:]}%
“Thou wilt accomplish a half work by me, \\
if thou bring home the whales to the farm, \\
or our float-jar \ken{boat} do fasten.\footnoteB{Hymer tells Thunder who, having let go of the Wyrm, has nothing to show for the trip, that he can accomplish something half as great as the pulling of the whales if he carries them home and ties the boat by the shore.}”\evb\evg


\bvg\bva\mssnote{\Regius~14v/9, \AM~6v/6}%
\alst{G}ekk Hlórriði \hld\ \alst{g}ręip \edtext{á}{\Afootnote{\emph{til á} \Regius}} stafni &
vatt \edtrans{með \alst{au}stri}{with the bilge-water}{\Bfootnote{That is, the bilge-water was still inside the boat; another comic work of strength.}} \hld\ \alst{u}pp lǫg-fáki; &
\alst{ęi}nn með \alst{ǫ́}rum \hld\ ok með \alst{au}st-skotu &
\alst{b}ar til \alst{b}ǿjar \hld\ \alst{b}rim-svín jǫtuns &
ok \edtext{\edtext{\alst{h}olt-riða}{\Afootnote{\emph{†holtriba†} \Regius}} \hld\ \alst{h}ver}{\lemma{holt-riða hver}\Bfootnote{An uncertain and possibly corrupt kenning.  TODO: What do other editors and translators say?}} í gegnum. \eva

\bvb Loride \name{= Thunder} went, grasped the stern, \\
hurled up the lake-nag \ken{boat} with the bilge-water. \\
Alone with the oars and the bilge-bucket \\
he bore to the farm the ettin’s brim-swines \ken{whales}, \\
even through the cauldron of woodland ridges \ken{valley?}.\evb\evg


\bvg\bva\mssnote{\Regius~14v/12, \AM~6v/7}\edtext{\edtext{Ok}{\Afootnote{\emph{Enn} \AM}} \alst{ę}nn \alst{jǫ}tunn \hld\ umb \alst{a}fr-endi, &
\alst{þ}rá-girni vanr, \hld\ við \alst{Þ}ór sęnti, &
kvað-at mann \alst{r}amman, \hld\ þótt \alst{r}óa kynni, &
\alst{k}rǫptur-ligan, \hld\ nema \alst{k}alk bryti.}{\lemma{ALL}\Bfootnote{Even after witnessing numerous great feats of strength Hymer still refuses to admit Thunder’s superiority.  He now insists on challenging him to break his indestructible chalice.}}\eva

\bvb And still the ettin, used to stubbornness, \\
over strength of hand with Thunder flyted. \\
He called no man strong—although he could row, \\
mightily—unless he broke the chalice.\evb\evg


\bvg\bva\mssnote{\Regius~14v/14, \AM~6v/9}En \alst{H}lórriði, \hld\ es at \alst{h}ǫndum kom, &
\alst{b}rátt lét \alst{b}resta \hld\ \edtrans{\alst{b}ratt-stęin glęri}{steep stone with the glass}{\Bfootnote{He probably broke the stone columns in Hymer’s house with the chalice.}}, &
\alst{s}ló \edtrans{\alst{s}itjandi}{standing}{\Bfootnote{This word is ambiguous and can modify either Thunder (in which case it would mean “sitting”) or the columns (\emph{súlur}).  I have chosen the latter and read it as signifying their stability.}} \hld\ \alst{s}úlur í gǫgnum; &
bǫ́ru þó \alst{h}ęilan \hld\ fyr \alst{H}ymi síðan,\eva

\bvb But Loride \name{= Thunder} when it came to his hands \\
impatiently crushed steep stone with the glass. \\
He struck right through the standing columns, \\
still was it brought whole before Hymer thereafter,\evb\evg


\bvg\bva\mssnote{\Regius~14v/16, \AM~6v/10}unds þat hin \alst{f}ríða \hld\ \alst{f}riðla kęndi &
\alst{ǫ́}st-ráð mikit, \hld\ \alst{ęi}tt es vissi, &
„drep við \alst{h}aus \alst{H}ymis, \hld\ hann ’s \alst{h}arðari, &
\edtrans{\alst{k}ost-móðs}{choice-weary}{\Bfootnote{The gods have destroyed eight of his nine cauldrons, eaten his choicest food, and slain his finest bull.}} jǫtuns, \hld\ \alst{k}alki hvęrjum.“\eva

\bvb until the handsome mistress \ken*{Tew’s mother} gave \\
a great loving counsel, the one she knew: \\
“Strike against Hymer’s skull! It’s harder— \\
the choice-weary ettin’s—than any chalice.”\evb\evg


\bvg\bva\mssnote{\Regius~14v/18, \AM~6v/12}%
\alst{H}arðr \edtext{ręis}{\Afootnote{om. \AM}} á kné \hld\ \alst{h}afra dróttinn, &
fǿrðisk \alst{a}llra \hld\ í \alst{á}s-męgin; &
\alst{h}ęill vas karli \hld\ \alst{h}jalm-stofn ofan, &
en \alst{v}ín-fęrill \hld\ \alst{v}alr rifnaði.\eva

\bvb Hard on the knee rose the Lord of He-goats \ken*{= Thunder}, \\
drew himself into his highest Os-might.\footnoteB{What this actually means is not entirely clear, but a likely interpretation is that Thunder gains his true form—note that he was earlier, st. 18, in the shape of a young boy.  Compare \Gylfaginning\ in its description of Thunder attempting to pull up the Wyrm: \emph{Þá varð Þórr reiðr ok fǿrðist í ás-megin} “Then Thunder turned wroth and drew himself into his Os-might.”}— \\
Whole on the churl \ken*{= Hymer} was the helm-stump \ken{head} above, \\
but the round wine-track \ken{chalice} did rend apart.\evb\evg


\bvg\bva\speakernote{[Hymir kvað:]}\mssnote{\Regius~14v/20, \AM~6v/13}%
„\alst{M}ǫrg vęit’k \alst{m}ę́ti \hld\ \alst{m}ér gingin frá, &
\edtext{es}{\Afootnote{om. \Regius}} \alst{k}alki sé’k \hld\ \edtext{fyr}{\Afootnote{\emph{†yr†} \Regius}} \alst{k}néum hrundit,“ &
\alst{k}arl orð of \alst{k}vað: \hld\ „\edtext{\alst{k}ná’k-at sęgja &
\alst{a}ptr \alst{ę́}va-gi: \hld\ ‚þú ’st \alst{ǫ}lðr of hęitt.}{\lemma{kná’k-at \dots\ of hęitt. ‘I cannot \dots\ warmed!’}\Bfootnote{Hymer laments that with the loss of his finest vessel he will never be able to enjoy his drink again.  This is ironic since it was he who challenged Thunder to break it in the first place.}}‘\eva

\bvb\speakernoteb{[Hymer quoth:]}%
“I know many treasures are gone from me, \\
when I see the chalice thrown before [my] knees!”— \\
The churl \ken*{= Hymer} spoke words: “I cannot say \\
ever again: ‘Thou art, ale, well warmed!’\evb\evg


\bvg\bva\mssnote{\Regius~14v/22, \AM~6v/15}%
Þat ’s til \alst{k}ostar \hld\ ef \alst{k}oma mę́ttið &
\alst{ú}t ór \alst{ó}ru \hld\ \edtrans{\alst{ǫ}l-kjól}{ale-vessel \ken{cauldron}}{\Bfootnote{\emph{ǫl-kjól} is the accusative of \emph{ǫl-kjóll}, but in this construction (\CV: \emph{koma}, B) we would expect the dative \emph{ǫl-kjóli}.  Since the meter does not allow for this the poet has probably taken a grammatical liberty.}} \edtrans{hofi}{hall}{\Bfootnote{This is the only Old Norse occurrence of the word \emph{hof} in the sense “hall, house”—it otherwise only means “temple” (\inx[C]{hove}).  The West Germanic cognates consistently mean “hall”, but that is probably the original sense, so it is unclear if this is an instance of foreign (if so, most likely Anglo-Saxon) influence or just a poetic archaism.}}.“ &
\alst{T}ýr lęitaði \hld\ \alst{t}ysvar hrǿra; &
stóð at \alst{h}vǫ́ru \hld\ \alst{h}verr kyrr fyrir.\eva

\bvb It would be choicest if ye might take \\
out from our hall the ale-vessel \ken{cauldron}.” \\
Tew attempted, twice, to move it— \\
each time stood the cauldron still before [him].\evb\evg


\bvg\bva\mssnote{\Regius~14v/24, \AM~6v/16}%
\alst{F}aðir Móða \hld\ \alst{f}ekk á þręmi &
ok í \alst{g}ǫgnum stęig \hld\ \alst{g}olf niðr í sal; &
\alst{h}óf sér á \alst{h}ǫfuð upp \hld\ \alst{h}ver Sifjar verr, &
en á \alst{h}ę́lum \hld\ \edtrans{\alst{h}ringar skullu}{the rings clattered}{\Bfootnote{i.e. the chain-links.  This detail is mentioned in an example sentence contrasting long and short phonemes in \FGT: \emph{heyrði til hǫddu, þá er Þórr bar hverinn} ‘the sound of the pot-links (\emph{hadda}) was heard when Thunder bore the cauldron’.  According to \textcite{FinnurEdda}\ the chain (or \emph{hadda}) on a Wiking-age cauldron would have reached across, in which case this would be a reference to the cauldron’s enormous size, with its diameter—mentioned in st. 5 as one \inx[C]{rest}—being roughly the same as Thunder’s height.}}.\eva

\bvb The father of Moody \ken*{= Thunder} grasped the brim, \\
and stepped down through the floor in the hall.\footnoteB{In the account of \Gylfaginning\ Thunder is said to have stepped through the boat when trying to pull up the Middenyardswyrm.  This detail is also seen on the carving of the Altuna stone from Uppland, Sweden; it may have been transposed to this place in the narrative. TODO.} \\
Sib’s husband \ken*{= Thunder} heaved the cauldron up on his head, \\
but by his heels the rings clattered.\evb\evg


\bvg\bva\mssnote{\Regius~14v/26, \AM~6v/18}Fóru-t \alst{l}ęngi, \hld\ áðr \alst{l}íta nam &
\alst{a}ptr \alst{Ó}ðins sonr \hld\ \alst{ęi}nu sinni; &
sá ór \alst{h}ręysum \hld\ með \alst{H}ymi austan &
\edtext{\alst{f}olk-drótt}{\lemma{folk-drótt \dots\ fjǫl-hǫfðaða ‘war-troop \dots\ many-headed’}\Bfootnote{The adjective \emph{fjǫl-hǫfðaðr} means ‘many-headed, polycephalic’ and is not referring to the size of the host.  For many-headed ettins see st. 8 and for their malformed bodies in general see Introduction.}} \alst{f}ara \hld\ \alst{f}jǫl-hǫfðaða.\eva

\bvb They journeyed not for long before Weden’s son \ken*{= Thunder} \\
took to look back a single time. \\
He saw out of stone-heaps with Hymer from the east \\
a war-troop coming, many-headed.\evb\evg


\bvg\bva\mssnote{\Regius~14v/28, \AM~6v/19}\alst{H}óf sér af \alst{h}ęrðum \hld\ \alst{h}ver standandi, &
vęifði \alst{M}jǫllni \hld\ \alst{m}orð-gjǫrnum framm, &
ok \alst{h}raun-\alst{h}vala \hld\ \alst{h}ann alla drap.\eva

\bvb He heaved from his shoulders the cauldron, standing; \\
swung the murder-eager Millner forth, \\
and the desert-whales \ken{ettins} all he slew.\evb\evg


\bvg\bva\mssnote{\Regius~14v/30, \AM~6v/21}%
\edtext{Fóru-t \alst{l}ęngi, \hld\ áðr \alst{l}iggja nam &
\alst{h}afr \alst{H}lórriða \hld\ \alst{h}alf-dauðr fyrir, &
vas \edtext{\alst{sk}ę́r}{\Afootnote{emend. from meaningless \emph{†skirr†} \Regius\AM}} \alst{sk}ǫkuls \hld\ \alst{sk}akkr á bęini, &
en því hinn \alst{l}ę́-vísi \hld\ \alst{L}oki of olli.}{\lemma{ALL}\Bfootnote{The detail of Thunder’s halt goat is also found in \Gylfaginning\ 44:

{\small Thunder and Lock were on the way to visit Outyards-Lock and stayed the night at a certain farmer’s.  For supper Thunder cut his two goats and asked the farmer and his family to eat with him.  After they had eaten he spread the goatskins before the fire and asked the housefolk to throw the bones of the goats onto them.  Thelve, the farmer’s son, secretly pried open the thigh of one of the goats and ate the marrow.  At dawn Thunder blessed the goatskins with his hammer and the goats came back to life, but one of them had a halt leg.  The farmer begged for his life and offered to give up his two children: Thelve, his son, and Wrash, his daughter.  Thunder accepted this, and the two became his servants.}

The present stanza probably references a version of the myth where Lock had a part to play in the halting of the goat, perhaps by encouraging Thelve to pry the bone open.}}\eva

\bvb They journeyed not for long before Loride’s \name{= Thunder’s} he-goat \\
took to lie half-dead before [them]. \\
The colt of the cart-pole \ken{goat} was halt in the leg, \\
and that the guile-wise Lock had caused.\evb\evg


\bvg\bva\mssnote{\Regius~14v/32, \AM~6v/22}%
En \edtrans{ér}{ye}{\Bfootnote{The listeners.  A direct address to the audience of this type is otherwise unparalleled in Eddic mythological poetry.  Such are, however, typical for the Scaldic poetry with which this poem shares several traits; see Introduction above.}} \alst{h}ęyrt \alst{h}afið, \hld\ \edtext{\alst{h}vęrr kann umb þat &
\alst{g}oð-mǫ́lugra}{\lemma{hvęrr \dots\ goð-mǫ́lugra ‘each god-speaking man’}\Bfootnote{Literally “each of the god-speaking ones”.  \emph{goð-mǫ́lugr} ‘god-speaking’ is an hapax, but easily understood as “learned in the (lore of) the gods”.}} \hld\ \alst{g}ørr at skilja, &
\alst{h}vęr af \alst{h}raun-búa \hld\ \alst{h}ann laun of fekk, &
es \alst{b}ę́ði galt \hld\ \alst{b}ǫrn sín fyrir.\eva

\bvb But ye have heard—about that can \\
each god-speaking man more clearly discern— \\
which repayments \emph{he} [Thunder] from the desert-dweller [\textsc{ettin} = the farmer] got \\
when he paid up both his children for it.\evb\evg


\bvg\bva\mssnote{\Regius~15r/1, \AM~6v/24}\alst{Þ}rótt-ǫflugr kom \hld\ á \alst{þ}ing goða &
ok \alst{h}afði \alst{h}ver, \hld\ þann’s \alst{H}ymir átti; &
en \alst{v}éar hvęrjan \hld\ \alst{v}ęl skulu drekka &
\alst{ǫ}lðr at \alst{Ę́}gis \hld\ \edtrans{\alst{ęi}tt hǫr-męitið}{an \dots\ flax-cutting}{\Bfootnote{The latter word is an \emph{hapax} and very obscure.  \textcite{LaFargeGlossary} give several suggestions based on \textsc{winter}-kennings of the type “harm of the snake”, viz. \emph{ęitr-hǫr-męitir} ‘poison-rope-cutter \ken{snake > winter}’, \emph{ęitr-orm-męiðir} ‘poison-worm-injurer’ \ken{winter}.
A solution without emendation is to read \emph{ęitt} ‘one’ n. acc. sg. as modifying \emph{ǫlðr} n. acc. ‘ale-feast’, and \emph{hvęrjan} masc. acc. sg. ‘every’ as modifying \emph{hǫr-męitiðr} masc. acc. ‘flax-cutting’, a compound made up of \emph{hǫrr} ‘flax, cord’ and \emph{męita} ‘to cut’.  The whole thing might refer to an obscure harvest festival and give the poem something of an etiological purpose.  If this interpretation is correct it is not unlikely that \Hymiskvida\ was originally composed for performance at such a festival.}}.\eva

\bvb The valour-strong man \ken*{= Thunder} came to the \inx[C]{Thing} of the Gods, \\
and had the cauldron which Hymer had owned, \\
and the \inx[G]{Wighers} \name{Gods} well shall drink \\
an ale-feast at Eagre’s, each flax-cutting \ken{fall?}.\evb\evg

\sectionline
% Thunder, Tue
	\bookStart{The Flyting of Lock}[Lokasęnna]

\begin{flushright}%
Dating \parencite{Sapp2022}: C10th (0.965)

Meter: \Ljodahattr%
\end{flushright}

Preserved in \Regius, directly following \Hymiskvida, though the poems without doubt were originally separate; the stylistic differences are drastical.

The poem has been interpreted as blasphemous (TODO: elaborate), but shows no linguistic signs of being particularly late.

\sectionline

\section{From Eagre and the gods (\emph{Frá Ę́gi ok goðum})}

\bpg\bpa Ę́gir, er ǫðru nafni hét Gymir, hann hafði búit ásum ǫl þá er hann hafði fengit ketil inn mikla sem nú er sagt. Til þeirar veizlu kom Óðinn ok Frigg kona hans. Þórr kom eigi þvíat hann var í austrvegi. Sif var þar, kona Þórs; Bragi, ok Iðunn kona hans. Týr var þar, hann var einhendr; Fenrisulfr sleit hǫnd af hánum, þá er hann var bundinn. Þar var Njǫrðr ok kona hans Skaði; Freyr ok Freyja; Víðarr son Óðins. Loki var þar, ok þjónustumenn Freys, Byggvir ok Beyla. Mart var þar ása ok alfa. Ę́gir átti tvá þjónustumenn; Fimafengr ok Eldir. Þar var lýsigull haft fyr eldsljós; sjalft barsk þar ǫl. Þar var griðastadr mikill. Menn lofuðu mjǫk hversu góðir þjónustumenn Ę́gis vóru. Loki mátti eigi heyra þat, ok drap hann Fimafeng. Þá skóku ę́sir skjǫldu sína ok ǿptu at Loka, ok eltu hann braut til skógar, en þeir fóru at drekka. Loki hvarf aptr ok hitti úti Eldi; Loki kvaddi hann:\epa

\bpb \inx[P]{Eagre}, who by another name is called \inx[P]{Gymer}, had prepared an ale-feast for the Ease when he had got the great kettle as now is told.\footnoteB{See the immediately preceding \Hymiskvida.}

To that gathering came \inx[P]{Weden} and \inx[P]{Frie}, his woman. \inx[P]{Thunder} came not, for he was on the \inx[L]{Eastern Way}. Sib was there, Thunder’s woman; \inx[P]{Bray} and \inx[P]{Idun}, his woman. \inx[P]{Tue} was there, he was one-handed. The \inx[P]{Fenrerswolf} tore his hand off when it was bound.\footnoteB{This detail is probably brought up to chronologically date the events of the poem as happening after the binding of Fenrer in the mythology.} There was \inx[P]{Nearth}, and his woman \inx[P]{Shede}; \inx[P]{Free} and \inx[L]{Frow}; \inx[P]{Wider}, the son of \inx[P]{Weden}. \inx[P]{Lock} was there, and the servants of Free: \inx[P]{Bew} and \inx[P]{Beal}. There was a great many of the \inx[G]{Ease} and \inx[G]{Elves}\footnoteB{A formulaic expression, see \inx[F]{Ease and Elves}.}.

Eagre had two servants: \inx[P]{Femfinger} and \inx[P]{Elder}. There was glowing gold used instead of fire; the ale there poured itself. There was a great \inx[C]{grith-stead}.\footnoteB{A place wherein all violence was forbidden, see Encyclopedia.} Men greatly praised how good the servants of Eagre were. Lock could not stand that, and he slew Femfinger.

Then the Ease shook their shields and screamed at Lock,\footnoteB{Some sort of ancient war dance. Cf. the Old Swedish Heathen Law: “He screams three nithing-screams TODO”.} and chased him away to the forest, but then they went to drink. Lock came back and found Elder outside; Lock greeted him:\epb\epg

\sectionline

\bvg
\bva „Seg þú þat, Eldir, \hld\ \edtext{svá’t ęinugi &
\ind feti gangir framarr}{\lemma{svá’t \dots\ framarr ‘so that \dots\ further’}\Bfootnote{Cf. \Havamal\ 38: \emph{feti ganga framarr} ‘take one step further’.}}, &
hvat hér inni \hld\ hafa at ǫlmǫ́lum &
\ind sigtíva synir.“\eva

\bvb “Say thou it, Elder, so that thou take not one step further: what here within they bring up over the ale,\footnoteB{lit. ‘have for their ale-speeches’} the sons of the victory-Tues \ken{gods}.”\evb
\evg


\bvg {\small Elder quoth:}
\bva „Of vǫ́pn sín dǿma \hld\ ok of vígrisni sína &
\ind sigtíva synir; &
ása ok alfa, \hld\ es hér inni eru, &
\ind \edtext{manngi ’s þér í orði vinr.}{\lemma{manngi \dots\ vinr “none \dots\ words.”}\Bfootnote{i.e. “none of them say anything good about you.” — The (lack of) alliteration here is very notable, and also occurs in a c-line of v. 10 (see note there). Both of the two lines are otherwise perfect, and so it seems that \emph{v} (\textipa{/w/}) is participating in vowel-alliteration. Such is never encountered in scoldic poetry, it could have been delegated to the simpler Eddic styles. Alternatively the poem is of such age that it was composed before the North Germanic loss of \emph{v} before rounded vowels. This is supported by the fact that in both this stanza and st. 10 the words starting with vowels have cognates in other Germanic languages that begin with \emph{w}; in the case of \emph{ulfr} in v. 10 this consonant is well attested in old runic inscriptions.

If the alliteration indeed is on \emph{v}, this does not require dating the whole poem to the Proto-Norse period; perhaps the poet was aware of the change which had taken place a few generations before him, and employed the older form as an archaism. For metrical reasons the poem must certainly post-date the syncope period (in the C6th), but we know from the transitional C7th Blekinge runestones from Stentoften (DR 357), Gummarp (DR 358) and Istaby (DR 359) that the loss of \textipa{/w-/} occured after syncope anyway.

A C7th Proto-Norse form of the c-line might be: \emph{mannagí ’s þéʀ in worðé winʀ}.}}“\eva

\bvb “Of their weapons they converse, and of their fight-valiance, the sons of the victory-Tues \ken{gods}; of the Ease and Elves which are here within, none is thee a friend in words.”\evb
\evg


\bvg {\small Lock quoth:}
\bva „Inn skal ganga \hld\ Ę́gis hallir í &
\ind á þat sumbl at séa, &
\edtext{jǫll ok ǫ́fu}{\lemma{jǫll ok ǫ́fu “scorn and spite”}\Bfootnote{ioll oc áfo \Regius\. These two interesting words have been interpreted in a variety of ways: \CV\ sees the first word as \emph{jóll} ‘wild angelica’, whereas the second is taken to be an error for \emph{áfr} ‘a beverage [...] translated by Magnaeus by \emph{sorbitio avenacea}, a sort of common ale brewed of oats’.}} \hld\ fǿri’k ása sonum &
\ind ok blęnd’k þęim svá męini mjǫð.“\eva

\bvb “In shall I go into Eagre’s halls, for to see that \inx[C]{simble}; scorn and strife I bring to the sons of the Ease, and I mix for them so the mead with harm.”\evb
\evg


\bvg {\small Elder quoth:}
\bva „Vęizt, ef inn gęngr \hld\ Ę́gis hallir í &
\ind á þat sumbl at séa, &
hrópi ok rógi \hld\ ef ęyss á holl ręgin, &
\ind á þér munu þau þęrra þat.“\eva

\bvb “Know, if in thou goest into Eagre’s halls, for to see that simble: if slander and strife thou pourest onto the \inx[C]{hold} \inx[G]{Reins}, they will dry it off on thee.”\evb
\evg


\bvg {\small Lock quoth:}
\bva „Vęizt þat Ęldir, \hld\ ef ęinir skulum &
\ind sáryrðum sakask, &
auðigr verða \hld\ mun’k í andsvǫrum, &
\ind ef þú mę́lir til mart.“\eva

\bvb “Know it, Elder, if alone we two shall banter with wound-words: I will become wealthy in my answers, if thou speak too much.\footnoteB{Cf. \Havamal\ TODO mę́la til mart.}”\evb
\evg


\bpg
\bpa Síðan gekk Loki inn í hǫllina; en er þeir sá, er fyrir váru, hverr inn var kominn, þǫgnuðu þeir allir.\epa

\bpb Thereafter Lock walked into the hall, but when those who were there before him saw who was come inside, they all turned silent.\epb
\epg


\bvg {\small Lock quoth:}
\bva „Þyrstr ek kom \hld\ þessar hallar til &
\ind Loptr of langan veg, &
ǫ́su at biðja, \hld\ at mér ęinn gefi &
\ind mę́ran drykk mjaðar.\eva

\bvb “Thirsty I, Loft \name{= Lock}, came to these halls over a long way, to ask the Ease that they to me give a single renowned drink of mead.\evb
\evg


\bvg
\bva Hví þęgið ér svá \hld\ þrungin goð, &
\ind at mę́la né męguð; &
sessa ok staði \hld\ vęlið mér sumbli at, &
\ind eða hęitið mik heðan.“\eva

\bvb Why shut ye up so, pressed gods, that ye may not speak? Seats and places choose for me at the simble, or call me [away] hence.\footnoteB{i.e. “Cease your ambiguity; give me a seat or tell me to leave!”}”\evb
\evg


\bvg {\small Bray quoth:}
\bva „Sessa ok staði \hld\ vęlja þér sumbli at &
\ind ę́sir aldrigi; &
því’t ę́sir vitu \hld\ hvęim þęir alda skulu &
\ind gambansumbl of geta.“\eva

\bvb “Seats and places choose the Ease never for thee at the simble; for the Ease know which men they shall bid to the gomben-simble.”\evb
\evg


\bvg {\small [Lock quoth:]}
\bva „Mant þat Óðinn, \hld\ es vit í árdaga &
\ind blendum blóði saman? &
ǫlvi bęrgja \hld\ lézk ęigi mundu, &
\ind nema okkr vę́ri bǫ́ðum borit.“\eva

\bvb “Recallest thou, Weden, as we two in days of yore blended our blood together? Thou saidst thou wouldst not taste ale, unless it were for us both brought forth.”\evb
\evg


\bvg {\small [Weden quoth:]}
\bva \edtext{„Rís þú Víðarr \hld\ ok lát ulfs fǫður}{\lemma{Rís \dots\ fǫður ‘Rise \dots\ wolf’}\Bfootnote{For the alliteration see note to v. 2. A C7th Proto-Norse form of the c-line might be: \emph{Rís þú Wíðarʀ · auk lát wulfs faður}.}}
\ind sitja sumbli at,
síðr oss Loki \hld\ kveði lastastǫfum
\ind Ę́gis hǫllu í.“\eva

\bvb “Rise thou, Wider, and let the father of the wolf \ken*{= Lock} sit at the simble, lest Lock accuse us of fault in the hall of Eagre.”\evb
\evg


\bpg
\bpa Þá stóð Víðarr upp ok skenkti Loka, en áðr hann drykki, kvaddi hann ásuna:\epa

\bpb Then Wider stood up and poured to Lock, but before he [= Lock] drunk, he greeted the Ease:\epb
\epg


\bvg
\bva „Hęilir ę́sir, \hld\ hęilar ǫ́synjur &
\ind ok ǫll ginnhęilǫg goð, &
nema sá ęinn ǫ́ss \hld\ es innar sitr &
\ind Bragi bękkjum á.“\eva

\bvb “Hail the \inx[G]{Ease}! Hail the \inx[G]{Ossens}, and all the \inx[C]{gin-holy} gods!\footnoteB{The first two half-lines prayer formula are identical to \Sigrdrifumal\ 2–3, for which reason it is possibly of authentic Heathen origin. To the original audience Lock’s parody of it would then have been seen as highly offensive and blasphemous.} Save for that one \inx[G]{Ease}[os], who sits further within: Bray, on the benches.”\evb
\evg


\bvg {\small [Bray] quoth:}
\bva „Mar ok mę́ki \hld\ gef’k þér míns féar &
\ind ok bǿtir þér svá baugi Bragi, &
síðr þú ǫ́sum \hld\ ǫfund of gjaldir, &
\ind gręmjat goð at þér.“\eva

\bvb “Steed and sword I give thee of my own wealth, and so recompenses thee Bray with a \inx[C]{bigh}, since thou repayest the Ease with envy; do not anger the gods towards thee.”\evb
\evg


\bvg {\small [Lock] quoth:}
\bva „Jós ok armbauga \hld\ munt ę́ vesa &
\ind bęggja vanr Bragi, &
ása ok alfa, \hld\ es hér inni eru, &
\ind þú ert við víg varastr,
\ind ok skjarrastr við skot.“\eva

\bvb “Of both steed and arm-bighs wilt thou ever be, Bray, lacking; of the Ease and Elves which are here within, art thou the wariest of war, and the shyest of shot.”\evb
\evg


\bvg {\small [Bray] quoth:}
\bva „Vęit’k, ef fyr útan vę́ra’k, \hld\ sem fyr innan em’k, &
\ind Ę́gis hǫll of kominn, &
hǫfuð þitt \hld\ bę́ra’k í hęndi mér; &
\ind\edtext{lít’k þér þat fyr lygi}{\Bfootnote{‘litt ec þer þat fyr lygi’ \Regius. A variety of emendations have been proposed for this line. Simplest would be \emph{lítt es þér þat fyr lygi} ‘that is little [punishment] for thee for lying’. Based on the similarity of \emph{c} and \emph{ꞇ̇} (= \emph{tt}) \textcite{FinnurEdda} gives \emph{lykak þér þat fyr lygi} ‘so I would bring to thee for thy lie’.}}.“\eva

\bvb “I know if outside I were, as inside I am come into the hall of Eagre: thy head I would bear in my hands; this I see for thy lie.”\evb
\evg


\bvg {\small [Lock] quoth:}
\bva „Snjallr ert í sessi, \hld\ skalattu svá gęra, &
Bragi bękkskrautuðr; &
vega þú gakk \hld\ ef vręiðr séir; &
hyggsk vę́tr hvatr fyrir.“\eva

\bvb “Valiant art thou in the seat; thou shalt not do thus, Bray the bench-ornamenter! Go to strike if thou art wroth; the bold does not think in advance.\footnoteB{Cf. \Havamal\ nýsisk fróðra TODO, really the opposite sentiment.}”\evb
\evg


\bvg {\small [Idun] quoth:}
\bva „Bið’k, Bragi, \hld\ barna sifjar duga &
\ind ok allra óskmaga, &
at þú Loka \hld\ kveðir-a lastastǫfum &
\ind Ę́gis hǫllu í.“\eva

\bvb “I bid thee, O Bray, to respect the TODO, that thou not accuse Lock of fault in the hall of Eagre.”\evb
\evg


\bvg {\small [Lock] quoth:}
\bva „Þęgi þú, Iðunn, \hld\ þik kveð’k allra kvinna &
\ind vergjarnasta vesa &
síz þú arma þína \hld\ lagðir ítrþvęgna &
\ind umb þinn bróðurbana.“\eva

\bvb “Shut up thou, Idun: thee I say of all women to be the most man-eager, since thou laid thy beautifully washed arms around thy brother’s bane.”\evb
\evg


\bvg {\small [Idun] quoth:}
\bva „Loka ek kveð’k-a \hld\ lastastǫfum &
\ind Ę́gis hǫllu í; &
Braga ek kyrri \hld\ bjórręifan, &
\ind vil’k-at ek at it vręiðir vegisk.“\eva

\bvb “I do not accuse Lock of fault in the hall of Eagre. Bray I calm, cheerful from beer—I do not wish that ye two wroth ones may fight.”\evb
\evg


\bvg {\small [Giben] quoth:}
\bva „Hví it ę́sir tvęir \hld\ skuluð inni hér &
\ind sáryrðum sakask? &
Lofts-ki þat vęit \hld\ at hann lęikinn es &
\ind ok hann fjǫrgvall frjá.”\eva

\bvb “TODO”\evb
\evg


\bvg {\small [Lock] quoth:}
\bva „Þęgi þú, Gefjun, \hld\ þęss mun’k nú geta &
\ind es þik glapði at gęði: &
svęinn inn hvíti \hld\ es þér sigli gaf &
\ind ok þú lagðir lę́r yfir.“\eva

\bvb “Shut up thou, o Giben! Of him I will now speak, who confounded thy senses: the white swain, who gave thee a necklace, and thou laidest thy leg over [him].”\evb
\evg


\bvg {\small [Weden] quoth that:}
\bva „Ǿrr ert, Loki, \hld\ ok ørviti &
es þú fę́r þér Gęfjun at gręmi &
því’t aldar ørlǫg \hld\ hygg at hón ǫll of viti &
jafngǫrla sem ek.“\eva

\bvb “Mad art thou, o Lock, and out of wits, as thou incurrest the wrath of Giben; for, all orlays of people I judge that she might know, just as clearly as I.”\evb
\evg


\bvg {\small [Lock] quoth:}
\bva „Þęgi þú, Óðinn, \hld\ þú kunnir aldrigi &
\ind dęila víg með verum; &
opt þú gaft \hld\ þęim’s þú gefa skyldir-a, &
\ind inum slę́vurum, sigr.“\eva

\bvb “Shut up thou, o Weden: thou couldst never deal out war amongst men—often thou gavest to the ones thou shouldst not have given, to the slower men victory.”\evb
\evg


\bvg {\small [Weden] quoth:}
\bva „Vęizt ef ek gaf \hld\ þęim’s ek gefa né skylda, &
\ind inum slę́vurum, sigr, &
átta vetr \hld\ vast fyr jǫrð neðan &
\ind kýr mólkandi ok kona &
\ind ok hęfir þú þar bǫrn of borit &
\ind ok hugða’k þat args aðal.“\eva

\bvb “Know that if I gave to the ones I should not have given, to the slower men victory: for eight nights wast thou beneath the earth, milking cows and a woman, and there hast thou borne children, and I’ve judged that a degenerate’s nature.”\evb
\evg


\bvg {\small [Lock] quoth:}
\bva „En þik síga kóðu \hld\ Sámsęyju í &
\ind ok drapt á vett sem vǫlur, &
vitka líki \hld\ fórt verþjóð yfir, &
\ind ok hugða’k þat args aðal.“\eva

\bvb “But thou, they said, didst sink down upon Samsy, and thou beatst the drum like wallows [do]. In the likeness of a sorcerer thou journeyedst among the nations of men, and I’ve judged that a degenerate’s nature.”\evb
\evg


\bvg {\small [Frie] quoth:}
\bva „Ørlǫgum ykkrum \hld\ skylið aldrigi &
\ind sęgja sęggjum frá, &
hvat it ę́sir tvęir drýgðuð í árdaga; &
\ind firrisk ę́ forn rǫk firar.“\eva

\bvb “Regarding your two’s orlays should ye never speak to youths; that which ye two Ease did in days of yore—always may ancient rakes be shunned by men.”\evb
\evg


\bvg {\small [Lock] quoth:}
\bva „Þęgi þú, Frigg, \hld\ þú ert Fjǫrgyns mę́r &
\ind ok hęfir ę́ vergjǫrn verit, &
es þá Véa ok Vilja \hld\ lézt þér, Viðris kvę́n, &
\ind báða í baðm of tękit.“\eva

\bvb “Shut up thou, o Frie: thou art Firgyn’s maiden, and has always been man-eager—when Wigh and Will, thou letst, o Withrer’s wife, both in thy bosom take.”\evb
\evg


\bvg {\small [Frie] quoth:}
\bva „Vęizt ef inni ę́tta’k \hld\ Ę́gis hǫllum í &
\ind Baldri líkan bur &
út þú né kvę́mir \hld\ frá ása sonum &
\ind ok vę́ri þá at þér vręiðum vegit.“\eva

\bvb “Know, that if here inside I owned, in Eagre’s halls, a son alike to Balder: out came thou not, away from the sons of the Ease, and thou would be fought with wrath.”\evb
\evg


\bvg {\small [Lock] quoth:}
\bva „En vill þú, Frigg, \hld\ at ek flęiri tęlja &
\ind mína męinstafi: &
ek því réð \hld\ es þú ríða sér-at &
\ind síðan Baldr at sǫlum.“\eva

\bvb “Yet wilt thou, o Frie, that I count more of my harmful deeds: I caused it, that thou dost not hence see Balder riding toward the halls.”\evb
\evg


\bvg {\small [Frow] quoth:}
\bva „Ǿrr ert, Loki, \hld\ es þú yðra tęlr &
\ind ljóta lęiðstafi; &
ørlǫg Frigg \hld\ hygg at ǫll viti &
\ind þótt hón sjǫlf-gi sęgi.“\eva

\bvb “Mad art thou, o Lock, as thou countest your ugly loathsome deeds: all orlays I judge that Frie might know, although she says them not herself.”\evb
\evg


\bvg {\small [Lock] quoth:}
\bva „VERSE“\eva

\bvb “TRANSLATION”\evb
\evg


\bvg {\small [Frow] quoth:}
\bva „VERSE“\eva

\bvb “TRANSLATION”\evb
\evg


\bvg {\small [Lock] quoth:}
\bva „VERSE“\eva

\bvb “TRANSLATION”\evb
\evg


\bvg {\small [Nearth] quoth:}
\bva „VERSE“\eva

\bvb “TRANSLATION”\evb
\evg


\bvg {\small [Lock] quoth:}
\bva „VERSE“\eva

\bvb “TRANSLATION”\evb
\evg


\bvg {\small [Nearth] quoth:}
\bva „VERSE“\eva

\bvb “TRANSLATION”\evb
\evg


\bvg {\small [Lock] quoth:}
\bva „VERSE“\eva

\bvb “TRANSLATION”\evb
\evg


\bvg {\small [Tue] quoth:}
\bva „VERSE“\eva

\bvb “TRANSLATION”\evb
\evg


\bvg {\small [Lock] quoth:}
\bva „VERSE“\eva

\bvb “TRANSLATION”\evb
\evg


\bvg {\small [Tue] quoth:}
\bva „VERSE“\eva

\bvb “TRANSLATION”\evb
\evg


\bvg {\small [Lock] quoth:}
\bva „VERSE“\eva

\bvb “TRANSLATION”\evb
\evg


\bvg {\small [Free] quoth:}
\bva „VERSE“\eva

\bvb “TRANSLATION”\evb
\evg


\bvg {\small [Lock] quoth:}
\bva „VERSE“\eva

\bvb “TRANSLATION”\evb
\evg


\bvg {\small [Bew] quoth:}
\bva „VERSE“\eva

\bvb “TRANSLATION”\evb
\evg


\bvg {\small [Lock] quoth:}
\bva „VERSE“\eva

\bvb “TRANSLATION”\evb
\evg


\bvg {\small [Bew] quoth:}
\bva „VERSE“\eva

\bvb “TRANSLATION”\evb
\evg


\bvg {\small [Lock] quoth:}
\bva „VERSE“\eva

\bvb “TRANSLATION”\evb
\evg


\bvg {\small [Homedall] quoth:}
\bva „VERSE“\eva

\bvb “TRANSLATION”\evb
\evg


\bvg {\small [Lock] quoth:}
\bva „VERSE“\eva

\bvb “TRANSLATION”\evb
\evg
% Lock, all gods
%	\include{books/Speeches of Allwise.tex}% Thunder, Wisdom poem
	\bookStart{The Speeches of Shirner}[Skírnismǫ́l]

\begin{flushright}%
Dating \parencite{Sapp2022}: C10th (0.897)

Meter: \Ljodahattr, \Galdralag\ (TODO)%
\end{flushright}

% Introduction

The whole poem is attested in both \Regius\ and \AM. The name \emph{Skírnismǫ́l} ‘\textbf{Speeches of Shirner}’ comes from \AM; \Regius\ has the header \emph{Fǫr Skírnis} ‘Shirner’s journey’.

The same myth is told in \Gylfaginning\ 37. A single verse of the present poem is quoted there, namely the last one (42), with some minor differences in wording that would seem to stem from oral tradition (see Note there). One could speculate that the author of \Gylfaginning\ did not have a copy of this poem in front of him, but rather knew of the story through an oral tradition which included only the last verse. This seems unlikely for the chief reason that this paraphrase does not add a single detail not already in the present poem, but on the other hand condenses and abbreviates that which is already written here. Thus Shirner’s journey and curse (roughly vv. 10–38 here) is simply summarized in the following manner: “Then Shirner journeyed and requested the woman [i.e. Gird] for him [i.e. Free], and received her promise, that nine nights later she would come to the place which is called Barrey, and have a wedding with Free.”

On the other hand, the paragraph in \Gylfaginning\ 37 that corresponds to what is here P1 is much more detailed. It goes: “Gymer was a man called, and his woman Earbode; she was of the lineage of mountain-risers. Their daughter is Gird, who is fairest of all women. It was one day as Free had gone to Lithshelf and looked about all the Homes. And when he looked to the north he saw on a farm a large and fair house, and into that house walked a woman. And when she brought out her hands and closed the doors before her, then light shone off her hands—both into the air and onto the waters—and all the homes were brightened by her. That beauty, when he had set himself in that holy seat, harmed him so that he walked away filled with pain. And when he came home he spoke nothing. Nothing slept he, nothing drank he. Nobody dared to ask him to speak. Then Nearth had Shirner, Free’s shoe-swain, called unto him, and asked him to go to Free and ask him to speak, [...]”

It seems to me that this circumstance, where the part corresponding to the poem is a short paraphrase, but the part corresponding to the prose passage is much more detailed, can only have arisen if the former already had a fixed form, whereas the latter was freer and could vary with each retelling. For this, see further TODO.

\sectionline

\bpg
\bpa\mssnote{\Regius~11r/10, \AM~2r/11}Freyr, sonr Njarðar, hafði einn dag setsk í Hlið-skjálf ok sá um heima alla; hann sá í Jǫtun-heima ok sá þar mey fagra, þá er hon gekk frá skála fǫður síns til skemmu; þar af fekk hann hug-sóttir miklar. Skírnir hét skó-sveinn Freys. Njǫrðr bað hann kveðja Frey máls. Þá mę́lti Skaði:\epa

\bpb \inx[P]{Free}, son of \inx[P]{Nearth}, had one day set himself down in \inx[L]{Lithshelf} and looked about all the \inx[C]{Homes}. He looked into the \inx[L]{Ettinhomes} and saw there a fair maiden as she walked from her father’s hall to her bower; thereof he got great heart-aches. \inx[P]{Shirner} was called the shoe-swain of Free. Nearth asked him to speak with Free. Then \inx[P]{Shede} spoke:\epb
\epg


\bvg
\bva\mssnote{\Regius~11r/14, \AM~2r/15}„\edtext{Rís-tu nú Skírnir \hld\ ok gakk at bęiða}{\lemma{rís \dots\ bęiða ‘rise \dots\ speak’}\Bfootnote{Alliteration is missing here. A simple solution would be to replace \emph{gakk} ‘go’ with a synonym like \emph{rinn} ‘run’ or \emph{ráð} ‘resolve’, but this breaks the mirroring in 2/2.}} &
\ind okkarn \alst{m}ála \alst{m}ǫg, &
ok þess at \alst{f}regna \hld\ hvęim hinn \alst{f}róði séi &
\ind \alst{o}f-ręiði \edtrans{\alst{a}fi}{man}{\Bfootnote{While this word usually means ‘father’ or ‘grandfather’, it must here certainly mean ‘man’ without a connotation of old age. See further \CV.}}.“\eva

\bvb “Rise thou now, O Shirner, and go to ask \\
our lad \ken*{= Free} for speech; \\
and to learn at whom the wise \\
man \ken*{= Free} might be cross.”\evb
\evg


\bvg {\small Skírnir kvað:}
\bva\mssnote{\Regius~11r/15, \AM~2r/17}„\alst{I}llra \alst{o}rða \hld\ es mér \alst{ó}n at ykkrum syni, &
\ind ef ek gęng at \alst{m}ę́la við \alst{m}ǫg, &
ok þess at \alst{f}regna, \hld\ hvęim hinn \alst{f}róði séi &
\ind \alst{o}f-ręiði \alst{a}fi.“\eva

\bvb Shirner quoth: “Bad words I expect from your son \ken*{= Free},  \\
if I go with the lad to speak; \\
and to learn at whom the wise \\
man might be cross.”\evb
\evg

\sectionline

\bvg {\small Skírnir:}
\bva\mssnote{\Regius~11r/17, \AM~2r/18}„Sęg þat \alst{F}ręyr, \hld\ \alst{f}olk-valdi goða, &
\ind ok ek \alst{v}ilja \alst{v}ita, &
hví þú \alst{ęi}nn sitr \hld\ \alst{ę}nd-langa sali, &
\ind minn \alst{d}róttinn, of \alst{d}aga?“\eva

\bvb Shirner [quoth]: “Tell it, O Free, troop-wielder of the gods; \\
I too would wish to know: \\
why thou sittest alone in the endlong halls, \\
my lord, during the days?”\evb
\evg


\bvg {\small Fręyr:}
\bva\mssnote{\Regius~11r/19, \AM~2r/20}„Hví of \alst{s}ęgja’k þér, \hld\ \alst{s}ęggr hinn ungi, &
\ind \alst{m}ikinn \alst{m}óð-trega? &
því-at \alst{a}lf-rǫðull \hld\ lýsir of \alst{a}lla daga &
\ind ok þęygi at \alst{m}ínum \alst{m}unum.“\eva

\bvb Free [quoth]: “Why should I tell thee, O young youth, \\
{[of my]} great mood-grief? \\
For the elf-wheel \ken{sun} shines during all days, \\
and naught to my liking.”\evb
\evg


\bvg {\small Skírnir:}
\bva\mssnote{\Regius~11r/20, \AM~2r/21}„\alst{M}uni þína \hld\ hykk-a svá \alst{m}ikla vesa, &
\ind at þú mér \edtrans{\alst{s}ęggr}{youth}{\Bfootnote{This word usually means simply ‘man’, but it seems to have a specific connotation with youth. Its original meaning is ‘messenger’, and the semantic shift is thus: ‘messenger’ > ‘young man’ > ‘warrior/man’. The sense of ‘young man’ is also seen in \Volundarkvida\ 23, where it is used in reference to king Nithad’s two young sons. In the present stanza it answers Free’s addressing Shirner as \emph{sęggr hinn ungi} ‘the young youth’; Shirner points out that the two are of equal age, and so Free is as much of a young man as he.}} né \alst{s}ęgir; &
\alst{u}ngir saman \hld\ vǫ́rum í \alst{á}r-daga, &
\ind vęl mę́ttim \alst{t}vęir \alst{t}rúask.“\eva

\bvb Shirner [quoth]: “Thy liking I do not think so great, \\
that thou, O youth, should not tell me [of it]. \\
Young together were we in days of yore; \\
we two might well trust each other.”\evb
\evg


\bvg {\small Fręyr:}
\bva\mssnote{\Regius~11r/22, \AM~2r/23}„Í \alst{G}ymis gǫrðum \hld\ ek \alst{g}anga sá &
\ind \alst{m}ér tíða \alst{m}ęy; &
\alst{a}rmar lýstu, \hld\ en \alst{a}f þaðan &
\ind allt \edtrans{\alst{l}opt ok \alst{l}ǫgr}{air and sea}{\Bfootnote{Formulaic and very old, also paralleled in the Anglo-Saxon. TODO.}}.\eva

\bvb Free [quoth]: “In Gymer’s yards I saw walking \\
a maiden, dear to me. \\
The arms shone, but thereof \\
all the air and sea.\evb
\evg


\bvg
\bva\mssnote{\Regius~11r/24, \AM~2r/24}\alst{M}ę́r ’s mér tíðari \hld\ an \alst{m}anna hvęim &
\ind \alst{u}ngum í \alst{á}r-daga; &
\alst{á}sa ok \alst{a}lfa \hld\ þat vill \alst{ę}ngi maðr, &
\ind at vit \alst{s}átt \alst{s}éim.“\eva

\bvb The maiden is dearer to me than to any man \\
young in days of yore. \\
Of the \inx[F]{Ease and Elves} does no man\footnoteB{i.e. ‘person’. For other examples of gods being called men see note to final st. of \Vafthrudnismal\ (TODO).} wish \\
that we two should be brought together.”\evb
\evg


\bvg {\small Skírnir:}
\bva\mssnote{\Regius~11r/25, \AM~2r/25}„\alst{M}ar gef mér þá, \hld\ es mik of \alst{m}yrkvan beri &
\ind \alst{v}ísan \alst{v}afr-loga, &
ok þat \alst{s}verð, \hld\ es \alst{s}jalft vegisk &
\ind við \alst{jǫ}tna \alst{ę́}tt.“\eva

\bvb Shirner [quoth]: “Then give me the steed, which might bear me over the dark, \\
wise wavering-flame; \\
and that sword, which by itself might strike \\
against the line of the \inx[G]{Ettins}.”\evb
\evg


\bvg {\small Fręyr:}
\bva\mssnote{\Regius~11r/27, \AM~2r/27}„\alst{M}ar þér þann gef’k, \hld\ es þik of \alst{m}yrkvan \edtext{berr &
\ind \alst{v}ísan \alst{v}afr-loga, &
auk þat \alst{s}verð, \hld\ es \alst{s}jalft mun vegask, &
\ind ef sá ’s \alst{h}orskr es \alst{h}ęfr.“}{\lemma{berr ‘bears’; mun vegask, ef sá ’s horskr es hęfr ‘will strike, if he is wise who owns it’}\Bfootnote{In his response Free replaces the subjunctive verb forms (\emph{beri} ‘might bear’, \emph{vegisk} ‘might strike’) with indicative and future forms, giving a sense of certainity and authority. The steed and sword are faultless, and if Shirner fails on the mission, it would be only due to his own fault (“if he is sharp who owns it.”).}}\eva
%TODO? Change the line numbering from 1–4 to 1, 3–4.

\bvb Free [quoth]: “That steed I give thee which bears thee over the dark, \\
wise wavering-flame; \\
and that sword which by itself will strike, \\
if he is sharp who owns it.”\evb
\evg

\bpg\bpa Skírnir mę́lti við hestinn:\epa
\bpb Shirner spoke with the horse:\epb\epg

\bvg
\bva\mssnote{\Regius~11r/29, \AM~2r/28}„\alst{M}yrkt es úti, \hld\ \alst{m}ál kveð’k okkr fara &
\ind \alst{ú}rig fjǫll \alst{y}fir &
\ind \edtrans{\alst{þ}ursa}{of the Thurses}{\Afootnote{so \AM; \emph{þyria} \Regius}} \alst{þ}jóð yfir; &
\alst{b}áðir vit komumk \hld\ eða okkr \alst{b}áða tękr &
\ind sá hinn \edtrans{\alst{á}m-átki \alst{jǫ}tunn}{unnatural ettin}{\Bfootnote{Formulaic. See note to \Voluspa\ 8.}}.“\eva

\bvb “’Tis dark outside; I declare it time for us to journey \\
over the drizzling mountains, \\
over the tribe of the \inx[G]{Thurses}. \\
Both two [shall] we come [over], or us both does take \\
that unnatural ettin.\footnoteB{Shirner declares his intention not to abandon the horse given to him by his lord; they will either both make it, or both perish.}”\evb
\evg


\bpg
\bpa\mssnote{\Regius~11r/31, \AM~2v/1}Skírnir reið i Jǫtun-heima til Gymis garða; þar váru hundar ólmir ok bundnir fyrir skíð-garðs hliði þess, er um sal Gerðar var. Hann reið at þar, er fé-hirðir sat á haugi, ok kvaddi hann: \epa

\bpb Shirner rode into the Ettinhomes, to Gymer’s yards. There were fierce hounds bound in front of the slope of the wooden fence which surrounded Gird’s\footnoteB{It is first now that we are informed of the maiden’s name.} hall. He rode to where a shepherd sat on a mound, and greeted him:\epb
\epg


\bvg
\bva\mssnote{\Regius~11v/2, \AM~2v/4}„Sęg þat \alst{h}irðir, \hld\ es á \alst{h}augi sitr &
\ind ok \alst{v}arðar alla \alst{v}ega: &
hvé ek at \alst{a}nd-spilli \hld\ komumk hins \alst{u}nga mans &
\ind fyr \alst{g}ręyjum \alst{G}ymis.“\eva

\bvb “Say it, O herdsman, who sittest on the mound, \\
and wardest all the ways: \\
How I to discourse might come with the young girl \ken*{= Gird}, \\
past Gymer’s greyhounds?”\evb
\evg


\bvg {\small [Hirðir] kvað:}
\bva\mssnote{\Regius~11v/4, \AM~2v/5}„Hvárt est \alst{f}ęigr, \hld\ eða est \alst{f}ramm ginginn &
\ind [...]; &
\alst{a}nd-spillis vanr \hld\ þú skalt \alst{ę́} vesa &
\ind \edtrans{\alst{g}óðrar męyjar}{good maiden}{\Bfootnote{Formulaic, carrying with it a sense of chasttity. See note to \Havamal\ TODO for further occurences.}} \alst{G}ymis.“\eva

\bvb {[The herdsman]} quoth: \\
“Either art thou fey, or gone forth \ken{dead}; \\
{[...]}. \\
Lacking discourse shalt thou ever be, \\
with Gymer’s good maiden \ken*{= Gird}.”\evb
\evg


\bvg {\small [Skírnir] kvað:}
\bva\mssnote{\Regius~11v/6, \AM~2v/7}„\edtrans{\alst{K}ostir}{Choices}{\Bfootnote{i.e. ‘alternatives, other ways’.}} ’ru bętri \hld\ \edtrans{an}{than}{\Afootnote{so \AM; \emph{hęldr an at} ‘rather than to [be]’ \Regius}} \alst{k}løkkva séi &
\ind hvęim’s \alst{f}úss es \alst{f}ara, &
\alst{ęi}nu dǿgri \hld\ mér vas \alst{a}ldr of skapaðr &
\ind ok alt \alst{l}íf of \alst{l}agit.“\eva

\bvb {[Shirner]} quoth: \\
“Choices are better than sobbing \\
for whomever is eager to journey. \\
On a single day was my age shaped, \\
and all my life laid [in place].\footnoteB{The Germanic fatalistic worldview, wherein one’s course of life was predetermined at birth, is here clearly seen. Presumably after uttering these words Shirner rides through the fire surrounding the fortress. — The causative \emph{lęgja} ‘to lay (down, in place)’ is closely connected to fate; the expression is formulaic. Cf. \Lokasenna\ 48: \emph{í árdaga vas þér hit ljóta líf of lagit} ‘in days of yore was thy ugly life laid [in place]’ and \Voluspa\ 19: \emph{þę́r lǫg lǫgðu} ‘they [= the Norns] laid laws [in place]’.}”\evb
\evg


\bvg {\small [Gęrðr] kvað:}
\bva\mssnote{\Regius~11v/7, \AM~2v/8}„Hvat ’s þat \alst{h}lym \alst{h}lymja \hld\ es \alst{h}lymja hęyri’k nú til &
\ind \alst{o}ssum rǫnnum \alst{í}? &
\alst{jǫ}rð bifask, \hld\ en \alst{a}llir fyr &
\ind skjalfa \alst{g}arðar \alst{G}ymis.“\eva

\bvb {[Gird]} quoth: \\
“What is that din of dins, which I of dins now hear \\
in our halls? \\
The earth quakes, but before [me] tremble \\
all Gymer’s yards.”\evb
\evg


\bvg {\small Ambǫ́tt kvað:}
\bva\mssnote{\Regius~11v/9, \AM~2v/10}„\alst{M}aðr ’s hér úti, \hld\ stiginn af \alst{m}ars baki, &
\ind \alst{jó} lę́tr til \alst{ja}rðar taka.“\eva

\bvb A servant-woman quoth: \\
“A man is here outside, stepped down off horseback; \\
he lets take his steed to the ground.\footnoteB{According to \textcite{FinnurEdda} a still known (in his time) Icelandic expression; Shirner lets his horse graze.}”\evb
\evg


\bvg {\small [Gęrðr] kvað:}
\bva\mssnote{\Regius~11v/10, \AM~2v/11}„\alst{I}nn bið þú hann ganga \hld\ í \alst{o}kkarn sal &
\ind ok drekka hinn \alst{m}ę́ra \alst{m}jǫð, &
þó ek hitt \alst{ó}umk, \hld\ at hér \alst{ú}ti séi &
\ind minn \alst{b}róður-\alst{b}ani.“\eva

\bvb {[Gird]} quoth: \\
“Bid thou him to go in into our hall, \\
and to drink the renowned mead; \\
though I fear that here outside should be  \\
my brother’s bane.”\evb
\evg

\sectionline

\bvg {\small [Gęrðr kvað:]}
\bva\mssnote{\Regius~11v/12, \AM~2v/13}„Hvat ’s þat \alst{a}lfa \hld\ né \alst{á}sa sona, &
\ind né \alst{v}íssa \alst{v}ana; &
hví \alst{ęi}nn of komt \hld\ \alst{ęi}kinn fúr yfir &
\ind ór \alst{s}al-kynni at \alst{s}éa?“\eva

\bvb {[Gird quoth:]} \\
“What sort is that, not of Elves, nor of sons of the Ease, \\
nor of wise Wanes? \\
Why camest thou alone over the raging fire, \\
to see the state of our hall?”\evb
\evg


\bvg {\small [Skírnir kvað:]}
\bva\mssnote{\Regius~11v/14}„\alst{E}m’k-at \alst{a}lfa \hld\ né \alst{á}sa sona &
\ind né \alst{v}íssa \alst{v}ana, &
þó \alst{ęi}nn of kom’k \hld\ \alst{ęi}kinn fúr yfir &
\ind yður \alst{s}al-kynni at \alst{s}éa.\eva

\bvb {[Shirner quoth:]} \\
“I am not of Elves, nor of sons of the Ease, \\
nor of wise Wanes— \\
yet I came alone over the raging fire, \\
to see the state of your hall.\evb
\evg


\bvg
\bva\mssnote{\Regius~11v/15, \AM~2v/14}\alst{Ę}pli \alst{ę}llifu \hld\ hér hef’k \alst{a}l-gollin, &
\ind þau mun’k þér \alst{G}ęrðr \alst{g}efa, &
\alst{f}rið at kaupa, \hld\ at þú þér \alst{F}ręy kveðir &
\ind ó-\alst{l}ęiðastan at \alst{l}ifa.“\eva

\bvb Elven apples have I here, all-golden; \\
those I will to thee, O Gird, give \\
to purchase [thy] love, that thou callest Free for thee \\
most unloathsome \ken{most lovely} in life.\footnoteB{\emph{at lifa} here means seems to mean ‘in life/living’ rather than the typical infinitive sense ‘to live’; cf. st. 22 \emph{at dęila} ‘in sharing’ below. This is possibly an archaism.}”\evb
\evg


\bvg {\small [Gęrðr kvað:]}
\bva\mssnote{\Regius~11v/17, \AM~2v/15}„\alst{Ę}pli \alst{ę}llifu \hld\ ek þigg \alst{a}ldri-gi &
\ind at \alst{m}anns-kis \alst{m}unum, &
né vit \alst{F}ręyr, \hld\ meðan okkart \alst{f}jǫr lifir, &
\ind \alst{b}yggum \alst{b}ę́ði saman.“\eva

\bvb {[Gird quoth:]} \\
“Eleven apples [will] I never accept, \\
to any man’s liking; \\
nor [will] I and Free—while our lives remain\footnoteB{lit. ‘while our life-force lives’}— \\
dwell both together.”\evb
\evg


\bvg {\small [Skírnir kvað:]}
\bva\mssnote{\Regius~11v/19, \AM~2v/17 (ll. 1–2)}„\alst{B}aug þér þá gef’k, \hld\ þann’s \alst{b}ręndr of vas &
\ind með \alst{u}ngum \alst{Ó}ðins syni; &
\edtext{\alst{á}tta ’ru \alst{ja}fn-hǫfgir, \hld\ es \alst{a}f drjúpa &
\ind hina \alst{n}íundu hvęrja \alst{n}ǫ́tt.“}{\lemma{átta ... nǫ́tt ‘Eight ... night.’}\Bfootnote{In \AM\ these lines and 22:1–2 are missing. Instead 1–2 here and 22:3–4 are combined into one.}}\eva

\bvb {[Shirner quoth:]} \\
“The \inx[C]{bigh} I then give thee, that one which was burned \\
with Weden’s young son \ken*{= Balder}. \\
Eight are even-heavy, which from it drip, \\
every ninth night.\footnoteB{The bigh, while not named, is clearly Dreepner as known from \Gylfaginning\ 49, describing Balder’s funeral: “Weden laid on the pyre that gold ring which is called Dreepner. Its nature was such that every ninth night, eight even-heavy golden rings dripped from it.” When \inx[P]{Harmod} later comes to \inx[L]{Hell} to try to bring Balder back, Balder tells him to bring the ring back to Weden, as a token of memory.}”\evb
\evg


\bvg {\small [Gęrðr kvað:]}
\bva\mssnote{\Regius~11v/21, \AM~2v/18 (ll. 3–4)}„\alst{B}aug þikk-a’k, \hld\ þótt \alst{b}ręndr séi, &
\ind með \alst{u}ngum \alst{Ó}ðins syni; &
es-a mér \alst{g}olls vant \hld\ í \alst{g}ǫrðum \alst{G}ymis &
\ind at dęila \alst{f}é \alst{f}ǫður.“\eva

\bvb {[Gird quoth:]} \\
“The bigh I accept not, though it may have been burned \\
with Weden’s young son \ken*{= Balder}; \\
I have no want of gold in Gymer’s yards, \\
in sharing the \inx[C]{fee} of my father.”\evb
\evg


\bvg {\small [Skírnir kvað:]}
\bva\mssnote{\Regius~11v/23, \AM~2v/19}„Sér þú \alst{m}ę́ki, \alst{m}ę́r, \hld\ \alst{m}jóvan, \alst{m}ál-fáan, &
\ind es \alst{h}ęf’k í \alst{h}ęndi \alst{h}ér? &
\alst{h}ǫfuð \alst{h}ǫggva \hld\ mun’k þér \alst{h}alsi af, &
\ind nema mér \alst{s}ę́tt \alst{s}ęgir.“\eva

\bvb {[Shirner quoth:]} \\
“Seest thou, O maiden, this sword—slender, pictured-painted\footnoteB{The sword is inlaid with metal forming a pattern. The expression is formulaic, cf. TODO.}, \\
which I have here in my hand? \\
Hew the head will I, from thy neck, \\
unless thou come to terms with me.”\evb
\evg


\bvg {\small [Gęrðr kvað:]}
\bva\mssnote{\Regius~11v/25, \AM~2v/20}„\alst{Á}-nauð þola \hld\ vil’k \alst{a}ldri-gi &
\ind at \edtrans{\alst{m}anns-kis}{any man’s (lit. ‘no man’s)}{\Afootnote{\emph{mannz ænskis} \AM}} \alst{m}unum, &
þó hins \alst{g}et’k, \hld\ ef it \alst{G}ymir finnizk &
\alst{v}ígs ó-trauðir \hld\ at ykkr \alst{v}ega tíði.“\eva

\bvb {[Gird quoth:]} \\
“Stand coercion will I never, \\
to any man’s liking; \\
though I get this, if thou and Gymer meet— \\
men unreluctant of conflict—that ye two will wish to fight.\footnoteB{Gird says that she will never let herself be forced to marry Free, even if that means that her father and Shirner should fight over her.}”\evb
\evg


\bvg {\small [Skírnir kvað:]}
\bva\mssnote{\Regius~11v/27, \AM~2v/22}„Sér þú \alst{m}ę́ki, \alst{m}ę́r, \hld\ \alst{m}jóvan, \alst{m}ál-fáan, &
\ind es \alst{h}ęf’k í \alst{h}ęndi \alst{h}ér? &
fyr þessum \alst{ę}ggjum \hld\ hnígr sá hinn \alst{a}ldni jǫtunn, &
\ind verðr þinn \alst{f}ęigr \alst{f}aðir.\eva

\bvb {[Shirner quoth:]} \\
“Seest thou, O maiden, this sword—slender, pictured-painted— \\
which I have here in my hand? \\
By these edges sinks the aged ettin \ken*{= Gymer} down; \\
\inx[C]{fey} becomes thy father.\evb
\evg


\bvg
\bva\mssnote{\Regius~11v/28, \AM~2v/24}\edtrans{\alst{T}ams-vęndi}{taming-wand}{\Bfootnote{Has been interpreted as a sword, TODO.}} þik drep’k, \hld\ ęn þik \alst{t}ęmja mun’k, &
\ind \alst{m}ę́r, at mínum \alst{m}unum, &
þar skalt \alst{g}anga \hld\ es þik \alst{g}umna synir &
\ind \alst{s}íðan ę́va \alst{s}éi.\eva

\bvb With the taming-wand I strike thee, but thee will I tame, \\
O maiden, to my liking. \\
There shalt thou go, where thee the sons of men \\
never since may see.\evb
\evg


\bvg
\bva\mssnote{\Regius~11v/30, \AM~2v/26}\edtrans{\alst{A}ra þúfu \alst{á} \hld\ skalt \alst{á}r sitja}{On an eagle’s hill shalt thou sit in early morn}{\Afootnote{\emph{ár skalt sitja \hld\ ara þúfu á} ‘in early morn shalt thou sit on an eagle’s hill’ \AM}}, &
\ind \edtext{\alst{h}orfa \alst{h}ęimi ór; &
\ind snugga \alst{h}ęljar til}{\lemma{horfa hęimi ór; snugga hęljar til ‘turn out of the world; hanker after Hell’}\Afootnote{horfa ok snugga hęljar til ‘turn and hanker to hell’ \AM}}; &
\alst{m}atr sé þér męir lęiðr \hld\ an \alst{m}anna hvęim &
\ind hinn \alst{f}ráni ormr með \edtext{\alst{f}irum}{\Bfootnote{This is the last word of fol. 2v of \AM, after which the text cuts off.}}.\eva

\bvb On an eagle’s hill shalt thou sit in early morning; \\
turn out of the world; \\
hanker after \inx[L]{Hell}.\footnoteB{Gird will long for death.} \\
May food be for thee more loathsome, than to anyone \\
the gleaming serpent \ken*{the Middenyardsworm} among men.\footnoteB{Her food will be as disgusting as the Middenyardsworm (for its disgusting nature see Note to \Hymiskvida\ 22).}\evb
\evg


\bvg
\bva\mssnote{\Regius~11v/32}At \alst{u}ndr-sjónum verðir \hld\ es \alst{ú}t of kømr, &
\ind á þik \alst{H}rímnir \alst{h}ari &
\ind á þik \alst{h}ot-vetna stari, &
\alst{v}íð-kunnari \alst{v}erðir \hld\ an \alst{v}ǫrðr með goðum, &
\ind \alst{g}api þú \alst{g}rindum frá.\eva

\bvb A wondrous sight mayst thou become when thou comest out; \\
at thee may Rimner ogle; \\
at thee may anyone stare. \\
More widely known mayst thou become than the ward among the Gods \ken*{= Homedall}; \\
mayst thou gape from the gates.\evb
\evg


\bvg
\bva\mssnote{\Regius~12r/2}\edtrans{\alst{T}ópi ok ópi, \hld\ \alst{t}jǫsull ok ó-þoli}{Toop and oop, tessle and impatience}{\Bfootnote{The first three of these four words are magic curse words; I have left them untranslated. TODO: Potential meanings.}}, &
\ind vaxi þér \alst{t}ǫ́r með \alst{t}rega; &
\alst{s}ęzk þú niðr \hld\ en mun’k \alst{s}ęgja þér &
\ind \alst{s}váran \alst{s}ús-breka, &
\ind ok \alst{t}vinnan \alst{t}rega.\eva

\bvb Toop and oop, tessle and impatience; \\
may thy tear grow with grief! \\
Sit thyself down, and I will tell thee \\
a heavy roaring-breaker, \\
and a twined grief.\evb
\evg


\bvg
\bva\mssnote{\Regius~12r/3}Tramar \alst{g}nęypa \hld\ þik skulu \alst{g}ęrstan dag &
\ind \alst{jǫ}tna gǫrðum \alst{í}, &
til \alst{h}rím-þursa \alst{h}allar \hld\ þú skalt \alst{h}vęrjan dag &
\ind \alst{k}ranga \alst{k}osta-laus; &
\ind \alst{k}ranga \alst{k}osta-vǫn; &
\alst{g}rát at \alst{g}amni \hld\ skalt í \alst{g}ǫgn hafa &
\ind ok lęiða með \alst{t}ǫ́rum \alst{t}rega.\eva

\bvb Thee shall fiends torment at the dismal day, \\
in the yards of the Ettins. \\
To the halls of the Rime-thurses shalt thou every day \\
creep choiceless; \\
creep choice-lacking. \\
Weeping for joy shalt thou have in exchange, \\
and nurse grief with tears.\evb
\evg


\bvg
\bva\mssnote{\Regius~12r/7}Með \alst{þ}ursi \alst{þ}rí-hǫfðuðum \hld\ \alst{þ}ú skalt ę́ nara &
\ind eða \alst{v}er-laus \alst{v}esa, &
\ind þitt \alst{g}eð \alst{g}rípi; &
\ind þik \alst{m}orn \alst{m}orni &
ves þú sem \alst{þ}istill, \hld\ sá’s \alst{þ}runginn vas &
\ind í \alst{o}fan-verða \alst{ǫ́}nn.\eva

\bvb With a three-headed thurse shalt thou ever live, \\
or be husband-less. \\
May thy senses grasp; \\
may murrain mourn thee; \\
be thou like the thistle that was pressed \\
in the uppermost harvest season!\evb
\evg


\bvg
\bva\mssnote{\Regius~12r/9}\edtext{Til \alst{h}olts ek gekk \hld\ ok til \alst{h}rás viðar &
\ind \alst{g}amban-tęin at \alst{g}eta &
\ind \alst{g}amban-tęin ek \alst{g}at.}{\lemma{Til holts \dots\ gat. ‘To the wood \dots\ got.’}\Bfootnote{The \emph{gamban-tęin} ‘gombentoe’ seems to be the stick on which the runic curse is to be carved (possibly to be identified with the \emph{tams-vǫndr} ‘taming-wand’ of st. 26.) This interpretation is supported by \Havamal\ 152, which also uses the expression \emph{(h)rás viðr} ‘raw/sappy tree’ and seems to refer to a runic curse.}}\eva

\bvb To the wood I went, and to the raw/sappy tree, \\
the \inx[C]{gombentoe} for to get; \\
the gombentoe I got.\evb
\evg


\bvg
\bva\mssnote{\Regius~12r/10}\alst{R}ęiðr ’s þér Óðinn, \hld\ \alst{r}ęiðr ’s þér Ása-bragr, &
\ind þik skal \alst{F}ręyr \alst{f}íask, &
hin \alst{f}irin-illa mę́r, \hld\ en \alst{f}ingit hęfr &
\ind \alst{g}amban-ręiði \alst{g}oða.\eva

\bvb Wroth with thee is Weden; wroth with thee is Bray of the Ease \name*{= Thunder?}; \\
thee shall Free come to hate, \\
O horrible maiden, if thou hast earned \\
the gomben-wrath of the gods.\evb
\evg


\bvg
\bva\mssnote{\Regius~12r/12}\alst{H}ęyri jǫtnar, \hld\ \alst{h}ęyri \alst{h}rím-þursar, &
\alst{s}ynir \alst{S}uttunga, \hld\ \alst{s}jalfir ás-liðar, &
hvé \alst{f}yrir býð’k, \hld\ hvé \alst{f}yrir banna’k &
\ind \alst{m}anna glaum \alst{m}ani, &
\ind \alst{m}anna nyt \alst{m}ani.\eva

\bvb May Ettins hear, may Rime-thurses hear, \\
sons of Sutting, the Os-retinues \ken*{= Ease} themselves: \\
how I forbid, how I forban \\
the company of men from the maiden; \\
the use of men from the maiden.\evb
\evg


\bvg
\bva\mssnote{\Regius~12r/14}\alst{H}rím-grímnir hęitir þurs, \hld\ es þik \alst{h}afa skal &
\ind fyr \alst{n}á-grindr \alst{n}eðan, &
þar þér \alst{v}íl-męgir \hld\ á \alst{v}iðar-rótum &
\ind \alst{g}ęita-hland \alst{g}efi; &
\alst{ǿ}ðri drykkju \hld\ fá þú \alst{a}ldri-gi, &
\ind \alst{m}ę́r, af þínum \alst{m}unum, &
\ind \alst{m}ę́r, at \alst{m}ínum \alst{m}unum.\eva

\bvb Rimegrimner is called the thurse, who shall have thee, \\
down beneath Nawgrind— \\
where the lads of toil \ken{thralls} on the roots of the tree, \\
goat-piss may give thee. \\
A better drink [wilt] thou never get, \\
O maiden, against thy liking; \\
O maiden, to my liking!\evb
\evg


\bvg
\bva\mssnote{\Regius~12r/16}\edtrans{\alst{Þ}urs}{thurse}{\Bfootnote{Thurse is the name of the \textbf{þ}-rune (ᚦ); it is carved as part of the curse.}} ríst’k \alst{þ}ér \hld\ ok \edtrans{\alst{þ}ría stafi}{three staves}{\Bfootnote{Three runic letters, possibly representing each of the three following words (\emph{ęrgi} ‘degeneracy’ etc.). This expression also appears on the C7th Gummarp stone: \textbf{h\textsc{a}þuwol\textsc{a}fʀ s\textsc{a}te st\textsc{a}b\textsc{a} þri\textsc{a} fff} ‘Hathwolf placed three staves: fff’, where the \textbf{f}-rune (ᚠ) is standing for its name, \inx[C]{fee} (i.e. wealth, cattle).}}, &
\ind \edtrans{\alst{ę}rgi ok \alst{ǿ}ði ok \alst{ó}-þola}{degeneracy and madness and impatience}{\Bfootnote{Both \emph{ęrgi} ‘degeneracy’ and \emph{óþoli} ‘impatience’ (here probably with a sexual connotation), are found in the love magic charm on the rune stick B257 from Bryggen, here edited under Charms and Spells. \emph{ęrgi} is also found in the curse-formula on the C7th Proto-Norse runestones from Stentoften and Björketorp. See further introduction to B257.}}, &
svá ek þat \alst{a}f ríst \hld\ sem ek þat \alst{á} ręist, &
\ind ef gørask \alst{þ}arfar \alst{þ}ęss.“\eva

\bvb \inx[G]{Thurses}[Thurse] I carve for thee, and three staves: \\
\inx[C]{degeneracy} and madness and impatience. \\
So I carve it off as I carved it on, \\
if need arises of that.\footnoteB{Shirner has carved the curse (which will bring true all the threats from 26–35), but tells Gird that he will scrape it off if she will accept his demands. She then responds:}”\evb
\evg


\bvg {\small [Gęrðr kvað:]}
\bva\mssnote{\Regius~12r/19}„\alst{H}ęill ves þú \alst{h}ęldr, svęinn, \hld\ ok tak við \alst{h}rím-kalki &
\ind \alst{f}ullum \alst{f}orns mjaðar, &
þó hafða’k \alst{ę́}tlat, \hld\ at mynda’k \alst{a}ldri-gi &
\ind unna \edtrans{\alst{v}aningja}{the Waning \ken*{= Free}}{\Bfootnote{lit. ‘descendant of the \inx[G]{Wanes}’; a rare word. It only occurs at one other place in the Norse corpus, namely in the \inx[C]{thule} of boar-names. Boars were sacred to Free, TODO.}} \alst{v}ęl.“\eva

\bvb {[Gird quoth:]} \\
“Be thou rather hale, O swain, and receive the rime-chalice, \\
full of ancient mead\footnoteB{Occurs identically in \Lokasenna\ 52.}— \\
although I had intended that I never would \\
love the Waning \ken*{= Free} well.”\evb
\evg


\bvg {\small [Skírnir kvað:]}
\bva\mssnote{\Regius~12r/21}„\alst{Ø}rendi mín \hld\ vil’k \alst{ǫ}ll vita, &
\ind áðr ríða’k \alst{h}ęim \alst{h}eðan, &
nę́r á \alst{þ}ingi \hld\ munt hinum \alst{þ}roska &
\ind \alst{n}ęnna \alst{N}jarðar syni.“\eva

\bvb {[Shirner quoth:]} \\
“My errands all I wish to know, \\
before I ride home hence: \\
when on the \inx[C]{Thing} wilt thou with the vigorous \\
son of Nearth \ken*{= Free} be joined?”\evb
\evg


\bvg {\small [Gęrðr kvað:]}
\bva\mssnote{\Regius~12r/23}„\alst{B}arri hęitir, \hld\ es vit \alst{b}ę́ði vitum, &
\ind \alst{l}undr \alst{l}ognfara, &
en ępt \alst{n}ę́tr \alst{n}íu, \hld\ þar mun \alst{N}jarðar syni &
\ind \alst{G}ęrðr unna \alst{g}amans.“\eva

\bvb {[Gird quoth:]} \\
“Barrey is called—as we both know— \\
a grove of calm rushes, \\
and after nine nights there will to the son of Nearth \ken*{= Free} \\
Gird her pleasure grant.”\evb
\evg


\bpg
\bpa\mssnote{\Regius~12r/24}Þá reið Skírnir heim. Freyr stóð úti ok kvaddi hann ok spurði tíðenda:\epa

\bpb Then Shirner rode home. Free stood outside and greeted him and asked for the tidings:\epb
\epg


\bvg
\bva\mssnote{\Regius~12r/25}„\alst{S}ęg mér, Skírnir, \hld\ áðr verpir \alst{s}ǫðli af mar &
\ind ok stígir \alst{f}eti \alst{f}ramarr, &
hvat \alst{á}rnaðir \hld\ í \alst{Jǫ}tun-hęima &
\ind þíns eða \alst{m}íns \alst{m}unar?“\eva

\bvb “Tell me, O Shirner, before thou throwest the saddle off the steed, \\
and takest a step further: \\
what didst thou accomplish in the \inx[L]{Ettinhomes}, \\
to thy or my liking?”\evb
\evg


\bvg {\small [Skírnir kvað:]}
\bva\mssnote{\Regius~12r/27}„\alst{B}arri hęitir, \hld\ es vit \alst{b}áðir vitum, &
\ind \alst{l}undr \alst{l}ogn-fara, &
en ępt \alst{n}ę́tr \alst{n}íu, \hld\ þar mun \alst{N}jarðar syni &
\ind \alst{G}ęrðr unna \alst{g}amans.“\eva

\bvb {[Shirner quoth:]} \\
“Barrey is called—as we both know— \\
a grove of calm rushes, \\
and after nine nights there will to the son of Nearth \ken*{= Free} \\
Gird her pleasure grant.”\evb
\evg


\bvg {\small [Fręyr kvað:]}
\bva\mssnote{\Regius~12r/28, \GylfMS}\alst{L}ǫng es nǫ́tt, \hld\ \edtrans{\alst{l}angar ’u tvę́r}{long are two}{\Bfootnote{so \Regius; \emph{lǫng es ǫnnur} ‘long is another’ \GylfMS}}, &
\ind hvé of \alst{þ}ręyja’k \alst{þ}ríar? &
opt \alst{m}ér \alst{m}ánaðr \hld\ \alst{m}inni þótti &
\ind an sjá \alst{h}ǫlf \alst{h}ý-nǫ́tt.\eva

\bvb {[Free quoth:]} \\
Long is a night; long are two; \\
how should I yearn for three? \\
Oft a month to me seemed less, \\
than this half wedding-night.\footnoteB{The wedding-night (TODO: it's a hapax so explain the etymology?) is presumably half as it is not consumated.}\evb
\evg
% Free, Gird
	\bookStart{Thule of Righ}[Rígsþula]

\begin{flushright}%
\textbf{Dating} \parencite{Sapp2022}: early C11th (0.240), late C11th (0.204), late C12th (0.195), C13th (0.280)

\textbf{Meter:} \Fornyrdislag%
\end{flushright}

Dumezil hypothesis. Irish influence? Many interesting things to write here!

The language of \Rigsthula\ is highly formulaic, but also often unique to it. Of particular note are the alliteration between the adverb \emph{męirr} ‘further’ and \emph{miðra}, e.g. in st. 2/1: \emph{gekk męirr at þat}

\sectionline

\bpg
\bpa\mssnote{\Wormianus~78r/1}Svá sęgja męnn í fornum sǫgum, at ęinn-hvęrr af ǫ́sum, sá er Hęimdallr hét, fór fęrðar sinnar ok framm með sjóvar-strǫndu nǫkkurri, kom at ęinum húsa-bǿ ok nęfndisk Rigr; ęptir þęiri sǫgu er kvę́ði þetta.\epa

\bpb So say men in ancient \inx[C]{saw}[saws] that one of the \inx[G]{Eese}, he who was called \inx[P]{Homedal}, went on his journey and came forth along a certain lake shore, came upon a lone homestead and called himself Righ—according to that saw is this poem.\epb
\epg


\bvg\bva\mssnote{\Wormianus~78r/TODO}\edtrans{Ár}{Of yore}{\Afootnote{sens. emend. (see note); \emph{at} \Wormianus}\Bfootnote{Formulaic. It is very common for poems to begin with \emph{ár} ‘of yore, in the beginning, in the dawn’. Cf. \Voluspa\ 3/1, \Hymiskvida\ 1/1, \HelgakvidaOne\ 1/1, \GudrunOne\ 1/1, \Sigurdskamma\ 1/1}} kvǫ́ðu \alst{g}anga \hld\ \alst{g}rǿnar brautir &
\alst{ǫ}flgan ok \alst{a}ldinn \hld\ \alst{ǫ́}s kunnigan, &
\alst{r}amman ok \alst{r}ǫskvan \hld\ \alst{R}íg stíganda.\eva

\bvb Of yore, they said, did walk on green highways \\
a mighty and ancient \inx[G]{Eese}[os], cunning: \\
the strong and brisk Righ, striding.\evb\evg


\bvg\bva\mssnote{\Wormianus~78r/TODO}Gekk \alst{m}ęirr at þat \hld\ \alst{m}iðrar brautar, &
kom hann at \alst{h}úsi, \hld\ \alst{h}urð vas á gę́tti; &
\alst{i}nn nam at ganga, \hld\ \alst{ę}ldr vas á golfi, &
\alst{h}jón sǫ́tu þar \hld\ \alst{h}ǫ́r \edtext{at}{\Afootnote{sens. emend.; \emph{af} \Wormianus}} arni, &
\alst{Á}i ok \alst{Ę}dda \hld\ \alst{a}ldin-falda.\eva

\bvb Went he further after that in the middle of the road; \\
came to a house—the door was wide open. \\
He took to go inside; fire was on the floor. \\
A couple sat there, hoary by the hearth: \\
Great-Grandfather and Great-Grandmother, old-fashioned.\evb\evg


\bvg\bva\mssnote{\Wormianus~78r/TODO}\alst{R}igr kunni þęim \hld\ \alst{r}ǫ́ð at sęgja; &
\alst{m}ęirr sęttisk hann \hld\ \alst{m}iðra flętja &
en á \alst{h}lið \alst{h}vára \hld\ \alst{h}jón sal-kynna.\eva

\bvb Righ knew to tell them counsels, \\
further he set himself down on the middle of the bench, \\
and on either side the couple of the hall.\evb\evg


\bvg\bva\mssnote{\Wormianus~78r/TODO}%
Þá tók \alst{Ę}dda \hld\ \alst{ø}kkvinn hlęif, &
\alst{þ}ungan ok \alst{þ}ykkvan, \hld\ \alst{þ}runginn sǫ́ðum, &
bar hǫ́n \alst{m}ęirr at þat \hld\ \alst{m}iðra skutla, &
\alst{s}oð vas í bolla \hld\ \alst{s}ętti á bjóð; &
vas \alst{k}alfr soðinn \hld\ \alst{k}rása bętstr; &
\alst{r}ęis hann upp þaðan, \hld\ \alst{r}éðsk at sofna;\eva

\bvb Then took Great-Grandmother a lumpy loaf, \\
heavy and thick, stuffed with chaff, \\
she carried it further after that on the middle of a trencher, \\
broth was in a bowl, she set it on a plate— \\
a cooked calf was the best dainty; \\
he \ken*{= Righ} rose up thence, resolved to sleep.\evb\evg


\bvg\bva\mssnote{\Wormianus~78r/TODO}\alst{R}igr kunni þęim \hld\ \alst{r}ǫ́ð at sęgja; &
\alst{m}ęirr lagðisk hann \hld\ \alst{m}iðrar rękkju, &
en á \alst{h}lið \alst{h}vára \hld\ \alst{h}jón sal-kynna.\eva

\bvb Righ knew to tell them counsels; \\
further he laid himself down in the middle of the bed, \\
and on either side the couple of the hall.\evb\evg


\bvg\bva\mssnote{\Wormianus~78r/TODO}\alst{Þ}ar vas hann at þat \hld\ \alst{þ}rjár nę́tr saman; &
gekk hann \alst{m}ęirr at þat \hld\ \alst{m}iðrar brautar; &
liðu \alst{m}ęirr at þat \hld\ \alst{m}ǫ́nuðr níu.\eva

\bvb There he was after that for three nights in all; \\
went he further after that in the middle of the road; \\
passed further after that nine months.\evb\evg


\bvg\bva\mssnote{\Wormianus~78r/TODO}\alst{Jó}ð \alst{ó}l \alst{Ę}dda, \hld\ \edtrans{\alst{jó}su vatni}{they sprinkled it with water}{\Bfootnote{A reference to the Heathen naming ceremony wherein water would be poured on a newborn, somewhat resembling the Christian baptism.  See \Havamal\ 156.}} &
\edtrans{\alst{h}ǫrund-svartan}{swarthy of skin}{\Afootnote{emend.; \emph{hǫrfi svartan} ‘swarthy with flax(?)’ \Wormianus}}, \hld\ \alst{h}étu Þrę́l.\eva

\bvb Great-Grandmother begot a child—they sprinkled it with water: \\
swarthy of skin, they called it Thrall.\evb\evg


\bvg\bva\mssnote{\Wormianus~78r/TODO}Hann nam at \alst{v}axa \hld\ ok \alst{v}ęl dafna; &
vas þar á \alst{h}ǫndum \hld\ \alst{h}rokkit skinn, &
\alst{k}ropnir \alst{k}núar, \hld\ [...] &
\alst{f}ingr digrir, \hld\ \alst{f}úlligt and-lit, &
\alst{l}otr hryggr, \hld\ \alst{l}angir hę́lar.\eva

\bvb He took to grow and have it well; \\
there on his hands was wrinkled skin, \\
crooked knuckles, [...], \\
stubby fingers, loathsome face, \\
stooping back, long heels.\evb\evg


\bvg\bva\mssnote{\Wormianus~78r/TODO}Nam \alst{m}ęirr at þat \hld\ \alst{m}agns of kosta, &
\alst{b}ast at \alst{b}inda, \hld\ \alst{b}yrðar gørva; &
bar \alst{h}ęim at þat \hld\ \alst{h}rís gęrstan dag.\eva

\bvb He took further after that to try his strength: \\
bast to bind, burdens to make; \\
he carried home after that brushwood on a gloomy day.\evb\evg


\bvg\bva\mssnote{\Wormianus~78r/TODO}Þar kom at \alst{g}arði \hld\ \edtrans{\alst{g}ęngil-bęina}{gangle-boned woman}{\Bfootnote{Derogatory, somebody who (due to poverty) only travels by foot.}}, &
\alst{au}rr vas á \alst{i}ljum, \hld\ \alst{a}rmr sól-brunninn, &
\alst{n}iðr-bjúgt es \alst{n}ęf, \hld\ \alst{n}ęfndisk \edtrans{Þír}{Thew}{\Bfootnote{The name probably means ‘maid-servant’ or ‘female slave’. Unlike Thrall, it is not attested in any prose texts, but probably corresponds to OS \emph{thiwi} ‘maid(-servant)’, being further root-related to \emph{þéa \char`~\ þjá} ‘to enthral’, Proto-Norse \textbf{þewaʀ} ‘servant’, OE \emph{þéow} ‘slave, servant’,.}}.\eva

\bvb There came to the farm a gangle-boned woman: \\
mud was on her footsoles, her arm sunburnt, \\
downturned her face—she called herself Thew.\evb\evg


\bvg\bva\mssnote{\Wormianus~78r/TODO}\edtext{Męirr sęttisk hǫ́n \hld\ miðra flętja,}{\lemma{Męirr \dots\ flętja}\Afootnote{emend. based on other sts.; \emph{miðra flętja \hld\ męirr sęttisk hǫ́n} \Wormianus}} &
\alst{s}at hjá hęnni \hld\ \alst{s}onr húss, &
\alst{r}ǿddu ok \alst{r}ýndu, \hld\ \alst{r}ękkju gørðu &
\alst{Þ}rę́ll ok \alst{Þ}ír \hld\ \alst{þ}rungin dǿgr.\eva

\bvb Further she set herself down on the middle of the bench; \\
by her sat the son of the house \ken*{= Thrall}. \\
They spoke and whispered, made a bed— \\
Thrall and Thew—in hard-pressed nights.\evb\evg


\bvg\bva\mssnote{\Wormianus~78r/TODO}\alst{B}ǫrn ólu þau, \hld\ \alst{b}juggu ok unðu; &
\alst{h}ygg’k at \alst{h}éti \hld\ \alst{H}ręimr ok Fjósnir, &
\alst{K}lúrr ok \alst{K}lęggi, \hld\ \alst{K}ęfsir, Fúlnir, &
\alst{D}rumbr, \alst{D}igraldi, \hld\ \alst{D}rǫttr ok Hǫsvir, &
\alst{L}útr ok \alst{L}ęggjaldi; \hld\ \alst{l}ǫgðu garða, &
\alst{a}kra tǫddu, \hld\ \alst{u}nnu at svínum, &
\alst{g}ęita \alst{g}ę́ttu, \hld\ \alst{g}rófu torf.\eva

\bvb Children they begot—they settled and were content— \\
I think that they were called Rame and Feesner, \\
Clour and Cledge, Chafser, Foulner, \\
Drumber, Digrald, Drant and Hazer, \\
Lout and Ledgald.—They laid yard-fences, \\
dunged fields, fed swine, \\
herded goats, dug turf.\evb\evg


\bvg\bva\mssnote{\Wormianus~78r/TODO}\alst{D}ǿtr vǫ́ru þę́r \hld\ \alst{D}rumba ok Kumba, &
\alst{Ø}kkvin-kalfa \hld\ ok \alst{A}rin-nęfja, &
\alst{Y}sja ok \alst{A}mbǫ́tt, \hld\ \alst{Ęi}kin-tjasna, &
\alst{T}ǫtrug-hypja \hld\ ok \alst{T}rǫnu-bęina; &
\alst{þ}aðan eru komnar \hld\ \alst{þ}rę́la ę́ttir.\eva

\bvb The daughters were Drumb and Cumb; \\
Inkencalf and Arn-neb, \\
Yeaze and Ambight, Oakentezen, \\
Tattryhip and Tranebone— \\
from thence are come the lines of thralls.\evb\evg


\sectionline


\bvg\bva\mssnote{\Wormianus~78r/TODO}Gekk \alst{R}ígr at þat \hld\ \alst{r}éttar brautir &
kom hann at \edtrans{\alst{h}ǫllu}{hall}{\Afootnote{sens. and metr. emend., cf. st. TODO; om. \Wormianus}} \hld\ \alst{h}urð vas á skiði &
\alst{i}nn nam at ganga, \hld\ \alst{ę}ldr vas á golfi &
\alst{h}jón sǫ́tu þar \hld\ \alst{h}eldu á syslu.\eva

\bvb Went Righ after that on straight highways; \\
he came to a hall—the TODO. \\
He took to go inside; fire was on the floor. \\
A couple sat there, busy with their chores:\evb\evg


\bvg\bva\mssnote{\Wormianus~78r/TODO}\alst{M}aðr tęlgði þar \hld\ \alst{m}ęið til rifjar, &
vas \alst{sk}ęgg \alst{sk}apat, \hld\ \alst{sk}ǫr vas fyr ęnni &
\alst{sk}yrtu þrǫngva \hld\ \alst{sk}okkr vas á golfi.\eva

\bvb A man there carved a stick into a loom-beam. \\
His beard was shapely, locks hung down his forehead, \\
his shirt tight; a toolbox was on the floor.\evb\evg


\bvg\bva\mssnote{\Wormianus~78r/TODO}\alst{S}at þar kona, \hld\ \alst{s}vęigði rokk, &
\alst{b}ręiddi faðm, \hld\ \alst{b}jó til váðar; &
\alst{s}vęigr vas á hǫfði, \hld\ \alst{s}mokkr vas á bríngu, &
\alst{d}úkr vas á halsi, \hld\ \alst{d}vergar á ǫxlum; &
\alst{A}fi ok \alst{A}mma \hld\ \alst{ǫ́}ttu hús.\eva

\bvb There sat a woman, twirled a distaff, \\
stretched out her arms, readied a cloth. \\
A scarf was on her head, a smock on her breast, \\
a kerchief on her throat, brooches on her shoulders— \\
Grandfather and Grandmother owned a house.\evb\evg


\bvg\bva\mssnote{\Wormianus~78r/TODO}\alst{R}ígr kunni þęim \hld\ \alst{r}ǫ́ð at sęgja, &
\alst{r}ęis frá borði \hld\ \alst{r}éð at sofna. &
\alst{M}ęirr lagðisk hann \hld\ \alst{m}iðrar rękkju &
en á \alst{h}lið hvára \hld\ \alst{h}jón sal-kynna. &
\alst{Þ}ar vas hann at þat \hld\ \alst{þ}rjár nę́tr saman &
liðu \alst{m}ęirr at þat \hld\ \alst{m}ǫ́nuðr níu.\eva

\bvb Righ knew to tell them counsels; \\
rose from the table, resolved to sleep. \\
Further he laid himself down in the middle of the bed, \\
and on either side the couple of the hall. \\
There he was after that for three nights in all; \\
passed further after that nine months.\evb\evg


\bvg\bva\mssnote{\Wormianus~78r/TODO}Jóð ól Amma, \hld\ jósu vatni, &
\alst{k}ǫlluðu \alst{K}arl \hld\ \alst{k}ona svęip ripti &
\alst{r}auðan ok \alst{r}jóðan \hld\ \alst{r}iðuðu augu.\eva

\bvb Grandmother begot a child, they sprinkled it with water, \\
called it Churl; the woman wrapped him in cloth, \\
red and ruddy; his eyes trembled.\evb\evg


\bvg\bva\mssnote{\Wormianus~78r/TODO}Hann nam at \alst{v}axa \hld\ ok \alst{v}ęl dafna, &
ǫxn nam at tęmja \hld\ arðr at gørva &
hús at timbra \hld\ ok hlǫður smíða &
karta at gørva \hld\ ok kęyra plóg.\eva

\bvb He took to grow and have it well; \\
oxen he took to tame, the ard to make, \\
houses to timber and barns to craft, \\
carts to make and drive the plough.\evb\evg


\bvg\bva\mssnote{\Wormianus~78r/TODO}Hęim óku þá \hld\ Hangin-luklu &
gęita kyrtlu \hld\ giptu Karli. &
Snǫr hęitir sú, \hld\ sęttisk und ripti. &
Bjuggu hjón, \hld\ bauga dęildu, &
bręiddu blę́jur, \hld\ ok bú gørðu.\eva

\bvb Homewards then drove Hangenkey, \\
TODO, married her to Churl. \\
Daughter-in-law she is called; she set herself under a cloth. \\
The couple settled, shared their money, \\
spread fine cloth and made a home.\evb\evg


\bvg\bva\mssnote{\Wormianus~78r/TODO}\alst{B}ǫrn ólu þau, \hld\ \alst{b}juggu ok unðu; &
hét Halr ok Drengr, \hld\ Hǫldr, Þegn ok Smiðr, &
Bręiðr, Bóndi, \hld\ Bundin-skęggi, &
Búi ok Boddi \hld\ Bratt-skęggr ok Sęggr.\eva

\bvb Children they begot—they settled and were content— \\
they were called Hale and Drang, Haled, Thane and Smith, \\
Broad, Bond, Boundenshag, \\
Bower and Bod, Brantshag and Sedge.\evb\evg


\bvg\bva\mssnote{\Wormianus~78v/1}Enn hétu svá \hld\ ǫðrum nǫfnum &
Snot, Brúðr, Svanni, \hld\ Svarri, Sprakki, &
Fljóð, Sprund, ok Víf, \hld\ Fęima, Ristill— &
þaðan eru \alst{k}omnar \hld\ \alst{k}arla ę́ttir.\eva

\bvb Yet some were called so with other names: \\
Snot, Bride, Swannie, Swarrie, Sprackie, \\
Fleed, Sprund and Wife, Fome, Ristle— \\
from thence are come the lines of churls.\evb\evg


\sectionline


\bvg\bva\mssnote{\Wormianus~78v/TODO}Gekk Rigr þaðan \hld\ réttar brautir &
kom hann at sal, \hld\ suðr horfðu dyrr, &
vas hurð hnigin, \hld\ hringr vas í gę́tti.\eva

\bvb TODO: Translation.\evb\evg


\bvg\bva\mssnote{\Wormianus~78v/TODO}Gekk hann inn at þat \hld\ golf vas stráat &
sǫ́tu hjón \hld\ sǫ́usk í augu &
faðir ok móðir \hld\ fingrum at lęika.\eva

\bvb TODO: Translation.\evb\evg


\bvg\bva\mssnote{\Wormianus~78v/TODO}Sat hús-gumi \hld\ ok snøri stręng &
alm of bęndi \hld\ ǫrvar skępti; &
en hús-kona \hld\ hugði at ǫrmum, &
strauk of ripti \hld\ sterti ęrmar.\eva%TODO: sterti or stęrti?

\bvb Sat the man of the houise and twisted the bow-string, \\
bent the elmwood, shafted arrows— \\
but the housewife minded her arms, \\
smoothened the fabric, tightened the sleeves.\evb\evg


\bvg\bva\mssnote{\Wormianus~78v/TODO}Kęisti fald, \hld\ kinga vas á bringu, &
síðar slǿður, \hld\ sęrk blá-fáan; &
brún bjartari, \hld\ brjóst ljósara, &
hals hvítari \hld\ hręinni mjǫllu.\eva

\bvb The linen hood jutted out, a brooch was on her chest, \\
a long-hanging gown, her serk dyed blue; \\
her brow was brighter, her chest lighter, \\
her throat whiter than purest snow.\evb\evg


\bvg\bva\mssnote{\Wormianus~78v/TODO}Rigr kunni þęim \hld\ rǫ́ð at sęgja; &
męirr sęttisk hann \hld\ miðra flętja &
en á hlið hvára \hld\ hjón sal-kynna.\eva

\bvb Righ knew to tell them counsels, \\
further he set himself down on the middle of the floor-bench, \\
and on either side: the couple of the hall.\evb\evg


\bvg\bva\mssnote{\Wormianus~78v/TODO}Þá tók móðir \hld\ męrktan dúk, &
hvítan af hǫrvi, \hld\ hulði bjóð; &
hón tók at þat \hld\ hlęifa þunna, &
hvíta af hvęiti, \hld\ ok hulði dúk.\eva

\bvb Then took Mother a patterned cloth, \\
white of flax—she covered a platter. \\
She took after that thin loaves, \\
white of wheat—and covered the cloth.\footnoteB{Note the strong parallelism. The household can afford an excess of expensive fabric and bread; Mother can cover the platter with a patterned (\emph{męrktr}) flaxen cloth, and then cover the cloth with wheat-bread.}\evb\evg


\bvg\bva\mssnote{\Wormianus~78v/TODO}Framm sętti hón \hld\ skutla fulla &
silfri varða á bjóð &
fán ok flęski \hld\ ok fugla stęikta &
vín vas i kǫnnu \hld\ varðir kalkar; &
drukku ok dǿmðu; \hld\ dagr vas á sinnum.\eva

\bvb TODO: Translation.\evb\evg


\bvg\bva\mssnote{\Wormianus~78v/TODO}Rigr kunni þęim \hld\ rǫ́ð at sęgja, &
ręis Rigr at þat, \hld\ rękkju gørði.\eva

\bvb Righ knew to tell them counsels, \\
rose Righ after that, made the bed.\evb\evg

\bvg\bva\mssnote{\Wormianus~78v/TODO}Þar vas hann at þat \hld\ þrjár nę́tr saman; &
gekk hann męirr at þat \hld\ miðrar brautar; &
liðu męirr at þat \hld\ mǫ́nuðr níu.\eva

\bvb There he was after that for three nights in all; \\
went he further after that on the middle of the road; \\
passed further after that nine months.\evb\evg


\bvg\bva\mssnote{\Wormianus~78v/TODO}Svęin ól móðir, \hld\ silki vafði, &
jósu vatni— \hld\ Jarl létu hęita; &
blęikt vas hár, \hld\ bjartir vangar, &
\edtext{ǫtul vǫ́ro augu \hld\ sem yrmlingi}{\lemma{ǫtul \dots\ yrmlingi ‘fierce \dots\ the young serpent’}\Bfootnote{A person of noble stock being recognised as such through their appearance is a motif in Norse literature. Cf. esp. the incident at the beginning of \HelgakvidaTwo, where Hallow, disguised as a thrall-woman, is almost caught due to his unslavelike eyes, which are, as in the present stanza, likewise said to be \emph{ǫtul} ‘fierce, terrible’.}}.\eva

\bvb Mother begot a swain, swaddled him in silk; \\
they sprinkled him with water—let him be called Earl. \\
Pale was his hair, bright his cheeks, \\
fierce were his eyes, like the young serpent.\evb\evg


\bvg\bva\mssnote{\Wormianus~78v/TODO}Upp óx þar \hld\ Jarl á flętjum; &
lind nam at skęlfa, \hld\ lęggja stręngi, &
alm at bęygja, \hld\ ǫrvar skępta, &
flęin at flęyja, \hld\ frǫkkur dýja, &
hęstum ríða, \hld\ hundum verpa, &
sverðum bregða, \hld\ sund at fręmja.\eva

\bvb Up grew Earl there on the floor-benches; \\
he took to shake shields, fasten bow-strings, \\
bend elmwood, shaft arrows, \\
throw javelins, hoist frankish spears, \\
ride horses, throw hounds (TODO) \\,
brandish swords, practice swimming.\evb\evg


\bvg\bva\mssnote{\Wormianus~78v/TODO}\edtext{Kom þar ór runni \hld\ Rigr gangandi, &
Rigr gangandi, \hld\ rúnar kęnndi; &
sitt gaf hęiti, \hld\ son kveðsk ęiga; &
þann bað hann ęignask \hld\ óðal-vǫllu, &
óðal-vǫllu, \hld\ aldnar bygðir.}{\lemma{Kom \dots\ bygðir.}\Bfootnote{Righ approaches his son, Earl. He reveals himself as his father and initiates him into the warrior aristocracy through teaching him the runes and giving him the noble title Righ (henceforth he will be known as Righ Earl). Finally he instructs him to set out and win land for himself, which Righ Earl soon does.}}\eva

\bvb There came out of a brush Righ, walking: \\
Righ, walking, taught runes; \\
he gave his own name; said that he had a son; \\
he bade \emph{him} take the ethel-plains: \\
the ethel-plains, the ancient villages.\evb\evg


\bvg\bva\mssnote{\Wormianus~78v/TODO}Ręið hann męirr þaðan \hld\ myrkan við &
hélug fjǫll \hld\ unds at hǫllu kom; &
skapt nam at dýja, \hld\ skęlfði lind, &
hęsti hlęypti, \hld\ ok hjǫrvi brá; &
víg nam at vękja, \hld\ vǫll nam at rjóða, &
val nam at fęlla, \hld\ vá til landa.\eva

\bvb He \ken{= Righ-Earl} rode further thence through the mirky wood, \\
through the frosty fells, until to a hall he came— \\
the shaft he took to hoist, shook the linden shield, \\
leapt with the horse, and brandished the sword; \\
war he took to rouse, the plain he took to redden, \\
men he took to fell—he won the land.\evb\evg


\bvg\bva\mssnote{\Wormianus~78v/TODO}Réð hann ęinn at þat \hld\ átján búum; &
auð nam skipta \hld\ ǫllum vęita &
męiðmar ok mǫsma, \hld\ mara svang-rifja; &
\edtrans{hringum hręytti}{rings he scattered}{\Bfootnote{Cf. StarkSt Frag 1/2a \emph{hring-hręytanda} ‘ring-scattererer \ken{generous man}’ which contains the same words.}}, \hld\ hjó sundr baug.\eva

\bvb He alone ruled, after that, eighteen homesteads. \\
Wealth he took to hand out; to give all men \\
gifts and treasures, [and] slender-ribbed steeds; \\
rings he scattered; he cut apart a bigh.\evb\evg


\bvg\bva\mssnote{\Wormianus~78v/TODO}\edtext{Óku}{\Afootnote{\emph{okū} \Wormianus}} ę́rir \hld\ úrgar brautir &
kvǫ́mu at hǫllu \hld\ þar’s hęrsir bjó: &
mǿtti [...] \hld\ \edtext{mjó-fingraðri}{\Afootnote{the grammar requires \emph{-ri}; mjó-fingraði \Wormianus}} &
hvítri ok horskri, \hld\ hétu Ęrna.\eva

\bvb Messengers drove through drizzling roads, \\
came to the hall where a ruler lived; \\
met a slender-fingered, \\
white and wise—they called her Erne.\evb\evg


\bvg\bva\mssnote{\Wormianus~78v/TODO}Bǫ́ðu hęnnar \hld\ ok hęim óku, &
giptu Jarli, \hld\ \edtrans{gekk hón und líni}{she went ’neath the linen}{\Bfootnote{i.e. she donned the bridal veil; cf. \Thrymskvida\ 27.}}; &
saman bjuggu þau \hld\ ok sér unðu, &
ę́ttir jóku \hld\ ok aldrs nutu.\eva

\bvb They asked for her hand and drove home, \\
married her off to Earl—she went under the linen. \\
They settled together and were content with themselves, \\
grew their lineage and enjoyed life.\evb\evg


\bvg\bva\mssnote{\Wormianus~78v/TODO}Burr vas hinn ęlsti, \hld\ en Barn annat; &
Jóð ok Aðal, \hld\ Arfi, Mǫgr, &
Niðr ok Niðjungr, \hld\ (nǫ́mu lęika) &
Sonr ok Svęinn, \hld\ (sund ok tafl) &
Kundr hét ęinn; \hld\ Konr vas hinn yngsti.\eva

\bvb Byre was the oldest, and Bairn another; \\
TODO: Translation. \\
TODO: Translation (they learned to play)
Son and Swain (swimming and Tavel)
Kund was one called; Kin was the youngest.\evb\evg


\bvg\bva\mssnote{\Wormianus~78v/TODO}Upp óxu þar \hld\ Jarli bornir: &
hęsta tǫmðu, \hld\ hlífar bęndu, &
skęyti skófu, \hld\ skęlfðu aska. &
En \edtrans{Konr ungr}{Kin the Young}{\Bfootnote{The name is clearly a folk etymological pun on ON \emph{konungr} ‘king’, who held the highest social rank, above even the earls.}} \hld\ kunni rúnar: &
ę́vin-rúnar \hld\ ok aldr-rúnar.\eva

\bvb There grew up the sons of Earl: \\
horses they tamed, shield-rims they bent, \\
smoothened shafts, shook ashen spears.— \\
But Kin the Young knew runes: \\
ever-runes and life-runes.\evb\evg


\bvg\bva\mssnote{\Wormianus~78v/TODO}%
Męirr kunni hann \hld\ mǫnnum bjarga, &
ęggjar dęyfa, \hld\ ę́gi lę́gja; &
klǫk nam fugla, \hld\ kyrra ęlda, &
sǿfa ok svęfja, \hld\ sorgir lę́gja, &
afl ok ęljun \hld\ átta manna.\eva

\bvb Further he knew men to save, \\
blades to dull, the sea to lower; \\
he learned the chirping of birds, to calm fires, \\
to soothe and lull to sleep, to lower sorrows; \\
the strength and zeal of eight men.\evb\evg


\bvg\bva\mssnote{\Wormianus~78v/TODO}Hann við Rig Jarl \hld\ rúnar dęildi; &
brǫgðum bęitti \hld\ ok bętr kunni; &
þá ǫðladisk \hld\ ok þá ęiga gat, &
Rigr at hęita, \hld\ rúnar kunna.\eva

\bvb With Righ-Earl he shared runes; \\
TODO. \\
then he earned for himself, and got to own, \\
Righ to be called, runes to know.\evb\evg


\bvg\bva\mssnote{\Wormianus~78v/TODO}Ręið Konr ungr \hld\ kjǫrr ok skóga; &
kolfi flęygði \hld\ kyrði fugla; &
þá kvað þat kráka \hld\ —sat kvisti ęin— &
„Hvat skalt, Konr ungr, \hld\ kyrra fugla? &
Hęldr mę́tti þér \hld\ hęstum ríða &
{[...]} \hld\ ok hęr fęlla.\eva

\bvb Kin the Young rode through brushes and woods, \\
flung bolts, hunted birds. \\
Then quoth a crow—sat on a branch alone— \\
“Why shalt thou, Kin the Young, hunt birds? \\
Better it fit thee horses to ride, \\
{[...]}, and armies to fell.”\evb\evg


\bvg\bva\mssnote{\Wormianus~78v/TODO}Á Danr ok Danpr \hld\ dýrar hallir; &
ǿðra \edtrans{óðal}{ethel}{\Bfootnote{Ancestral farmland, in this case the eighteen homesteads owned by Earl.}} \hld\ an \edtrans{ér}{ye}{\Afootnote{metr. emend.; \emph{þér} ‘id.’ \Wormianus, which is simply a younger form of \emph{ér}, and shows that the poem has been linguistically modernised.}} hafið; &
þęir kunnu vel \hld\ \edtrans{kjól at riða}{ship to ride}{\Bfootnote{i.e. to sail.}}, &
\edtrans{ęgg at kęnna}{the blade to teach}{\Bfootnote{i.e. to fight, wage war.  A euphemism; to “teach someone the blade” is to fight him.}}, \hld\ undir rjúfa.\eva

\bvb Dan and Danp own costly halls: \\
nobler ethel than ye do— \\
they know well the ship to ride, \\
the blade to teach, wounds to tear.\evb\evg

\sectionline

At this point leaf 78 ends. The rest of the poem is lost.

\sectionline
% Righ
	\bookStart{Leeds of Hindle}[Hyndluljóð]

\begin{flushright}%
\textbf{Dating} \parencite{Sapp2022}: late C11th (0.996)

\textbf{Meter:} \Fornyrdislag%
\end{flushright}%

\sectionline

\bvg\bva „Vaki mę́r męyja, \hld\ vaki mín vina, &
Hyndla systir, \hld\ es í hęlli býr; &
nú ’s røkr røkra, \hld\ ríða vit skulum &
til Valhallar \hld\ ok til vés hęilags.\eva

\bvb {[Frow quoth:]} “Wake, O maiden of maidens; wake, my friend, \\
sister Hindle, who lives in the rock-face! \\
Now is the twilight of twilights; we two shall ride \\
to Walhall, and to the holy \inx[C]{wigh}!\evb\evg


\bvg\bva Biðjum Hęrjafǫðr \hld\ í hugum sitja, &
hann geldr ok gefr \hld\ gull \edtrans{verðugum}{to the worthy}{\Afootnote{emended to \emph{verðungu} ‘to the retinue’ by \textcite{FinnurEdda}, \textcite{GudniEdda}}}, &
gaf hann Hęrmóði \hld\ hjalm ok brynju, &
en Sigmundi \hld\ sverð at þiggja.\eva

\bvb Let us bid the Father of Hosts \name{= Weden} to remain in good spirits;  \\
he rewards and gives gold to the worthy.  \\
He gave \inx[P]{Harmod} helmet and byrnie, \\
and \inx[P]{Syemund} a sword to receive.\evb\evg


\bvg\bva Gefr hann \alst{s}igr \alst{s}onum, \hld\ en \alst{s}vinnum aura, &
\alst{m}ę́lsku \alst{m}ǫrgum \hld\ ok \alst{m}an-vit firum, &
\alst{b}yri gefr \alst{b}rǫgnum, \hld\ en \alst{b}rag skǫldum, &
gefr hann \alst{m}ann-sęmi \hld\ \alst{m}ǫrgum rekki.\eva

\bvb He gives victory to sons and silver to the wise, \\
speech to many and \inx[C]{manwit} to men. \\
Fair wind he gives to nobles and praise-song to \inx[C]{scald}[scalds]; \\
he gives manly valour to many a champion.\evb\evg


\bvg\bva \alst{Þ}ór mun’k blóta, \hld\ \alst{þ}ess mun’k biðja, &
at hann \alst{ę́} við þik \hld\ \alst{ęi}n-art láti; &
þó ’s hǫ́num \alst{ȯ}-títt \hld\ við \alst{jǫ}tuns brúðir.\eva

\bvb To Thunder I will \inx[C]{bloot}; of this I will bid, \\
that he always be upright with thee \\
even though he hates the ettin’s brides.\evb\evg


\bvg\bva Nú tak-tu \alst{u}lf þinn \hld\ \alst{ęi}nn af stalli, &
lát hann \alst{r}inna \hld\ með \alst{r}una mínum.“— &
„Sęinn es \alst{g}ǫltr þinn \hld\ \alst{g}oð-veg troða, &
vil’k-at \alst{m}ar \alst{m}ínn \hld\ \alst{m}ę́tan hlǿða.\eva

\bvb Now take thy one wolf from the stable; \\
let him run alongside my boar.”— \\
{[Hindle quoth:]} “Slow is thy boar to tread the Godways; \\
I will not load my noble steed.\evb\evg


\bvg\bva Flǫ́ est Fręyja, \hld\ es fręistar mín, &
\edtext{vísar þú augum \hld\ á oss þannig, &
es hafir ver þinn \hld\ í val-sinni}{\lemma{vísar \dots\ val-sinni ‘thou showest \dots\ slain-ways’}\Bfootnote{i.e., “You only show favour to me because you want me to help your lover”.  For the expression cf. \Sigrdrifumal\ 3/3 and note.}} &
Óttar unga \hld\ Innstęins bur.“\eva

\bvb False art thou, Frow, who temptest me; \\
thou showest thy eyes on us this way \\
since thou hast thy lover on the slain-ways: \\
the young Oughter, Instone's offspring.”\evb\evg


\bvg\bva „Dulið est Hyndla, \hld\ draums ę́tla’k þér, &
es kveðr ver minn \hld\ í valsinni.\eva

\bvb {[Frow quoth:]}%
Deluded art thou, Hindle; I think thee dreamy \\
as thou sayest that my man is on the slain-ways.\evb\evg


\bvg\bva Þar’s gǫltr glóar \hld\ Gullinbursti, &
\edtext{Hildisvíni}{\lemma{Hildisvíni ‘Hildswine’}\Bfootnote{Presumably an alternative name of Goldenbristle.}}, \hld\ es mér hagir gęrðu, &
dvergar tvęir \hld\ Dáinn ok Nabbi.\eva

\bvb There where the boar Goldenbristle glows, \\
the Hildswine, which the two skillful dwarfs \\
Dowen and Nab did make for me.\evb\evg


\bvg\bva Sęnn í sǫðlum \hld\ sitja vit skulum &
ok of jǫfra \hld\ ę́ttir dǿma, &
gumna þęira, \hld\ es frá goðum kómu.\eva

\bvb Soon in the saddles we two shall sit, \\
and speak about the lineages of princes, \\
of those men who are come from the gods.\evb\evg


\bvg\bva Þęir hafa vęðjat \hld\ vala malmi &
Óttarr ungi \hld\ ok Angantýr; &
skylt ’s at vęita, \hld\ svá’t skati hinn ungi & &
fǫður-lęifð hafi \hld\ ępt frę́ndr sína.\eva

\bvb They have wagered the Welsh ore \ken{gold}, \\
young Oughter and Ongenthew— \\
it \emph{must} be divulged, so that the young prince \\
may have the patrimony left by his kinsmen.\evb\evg


\bvg\bva Hǫrg hann mér gęrði \hld\ hlaðinn stęinum; &
nú ’s grjót þat \hld\ at glęri orðit; &
rauð hann í nýju \hld\ nauta blóði; &
ę́ trúði Óttarr \hld\ á ǫ́synjur.\eva

\bvb A \inx[C]{harrow} he made for me, loaded with stones; \\
now that stone-pile has turned into glass. \\
He reddened it in the fresh blood of oxen; \\
always did Oughter trust on the \inx[G]{Ossens}.\evb\evg


\bvg\bva Nú lát forna \hld\ niðja talða &
ok upp-bornar \hld\ ę́ttir manna &
hvat ’s Skjǫldunga, \hld\ hvat ’s Skilfinga, &
hvat ’s Ǫðlinga \hld\ hvat ’s Ylfinga & &
hvat ’s hǫld-borit, \hld\ hvat ’s hęrs-borit &
męst manna val \hld\ und Mið-garði?“\eva

\bvb Now let ancient kinsmen be counted, \\
and the high born lineages of men: \\
What is of the Shieldings? What is of the Shilvings? \\
What is of the Athlings? What is of the Wolvings? \\
What is born of hero? What is born of chief, \\
the mightiest choice of men in Middenyard?”\evb\evg


\bvg\bva „Þú est Óttarr \hld\ borinn Innstęini, &
en Innstęinn vas \hld\ Alfi inum gamla, &
Alfr vas Ulfi, \hld\ Ulfr Sę́fara, &
en Sę́fari \hld\ Svan inum rauða.\eva

\bvb {[Hindle quoth:]} “Thou\footnoteB{Hindle, maybe in a trance-like state, speaks straight to Oughter.} art, Oughter, born to Instone, \\
and Instone was born to Elf the old, \\
Elf to Wolf, Wolf to Seafare, \\
and Seafare to Swan the red.\evb\evg


\bvg\bva Móður átti faðir þinn \hld\ męnjum gǫfga, &
hygg at héti \hld\ Hlédís gyðja, &
Fróði vas faðir þęirar, \hld\ en \edtext{Fríund}{\Afootnote{emend. from meaningless \emph{†friaut†} \FlatMS}} móðir; &
ǫll þótti ę́tt sú \hld\ með yfir-mǫnnum.\eva

\bvb Thy father had thy mother, esteemed with neck-rings, \\
I think that she was called Leedise the \inx[C]{gidden}. \\
Frood was her father and Friend her mother; \\
all her lineage seemed to be among \inx[C]{overmen}.\evb\evg


\bvg\bva Auði vas áðr \hld\ ǫflgastr manna, &
Halfdanr fyrri \hld\ hę́str Skjǫldunga, &
frę́g vǫ́ru folk-víg, \hld\ þau’s framir gęrðu, &
hvarfla þóttu verk \hld\ með himins skautum.\eva

\bvb Ed was once the mightiest of men, \\
Halfdane earlier the highest of Shieldings. \\
Renowned were the troop-conflicts \ken{wars} which the famous ones made; \\
his \name{= Halfdane’s} works seemed to circle along the corners of heaven.\evb\evg


\bvg\bva Ęflðisk við Ęymund \hld\ ǿðstan manna &
en vá Sigtrygg \hld\ með svǫlum ęggjum, &
ęiga gekk Almvęig, \hld\ ǿðsta kvinna, &
ólu þau ok ǫ́ttu \hld\ átján sonu.\eva

\bvb He \name{= Halfdane} became the in-law of Iemund\footnoteB{lit. “[he] was strengthened by”.  Elmwey was Iemund’s daughter or sister.}, the noblest of men, \\
and he slew Syetrue with cool edges. \\
He went on to have Elmwey, the noblest of women; \\
they begot and had eighteen sons.\evb\evg


\bvg\bva Þaðan eru Skjǫldungar, \hld\ þaðan eru Skilfingar, &
þaðan eru Ǫðlingar, \hld\ þaðan eru Ynglingar, &
þaðan es hǫld-borit, \hld\ þaðan es hęrs-borit, &
mest manna val \hld\ und Mið-garði; &
alt ’s þat ę́tt þín, \hld\ Óttarr hęimski.\eva

\bvb Thence come the Shieldings! Thence come the Shilvings! \\
Thence come the Athlings! Thence come the Inglings!\footnote{Note the contradiction with v. 12. Since the Inglings have already been mentioned (under the name Shilvings, for the difference between the two see Encyclopedia), it seems likely that Wolvings is the original reading.} \\
Thence is born of hero! Thence is born of chief \\
the mightiest choice of men in Middenyard! \\
All of this is thy lineage, O foolish Oughter!”\evb\evg


\bvg\bva Vas Hildigunnr \hld\ hęnnar móðir, &
Svǫ́fu barn \hld\ ok Sę́-konungs; &
alt ’s þat ę́tt þín, \hld\ Óttarr hęimski. &
varði at viti svá, \hld\ viltu ęnn lęngra?\eva

\bvb Hildguth was her mother, \\
the child of Sweve and Sea-king. \\
All of this is thy lineage, O foolish Oughter!— \\
It is meaningful that one might know thus; wilt thou [hear] yet further?\evb\evg


\bvg\bva Dagr átti Þóru \hld\ dręngja móður, &
ólusk í ę́tt þar \hld\ ǿðstir kappar, &
Fraðmarr ok Gyrðr \hld\ ok Frekar báðir, &
Ámr ok Jǫsurmarr, \hld\ Alfr hinn gamli. &
varðar at viti svá, \hld\ viltu ęnn lęngra?\eva

\bvb Day had Thure, the mother of valiant men; \\
in that lineage were begotten the noblest champions: \\
Fradmer and Yird, and both Frekes; \\
Ame and Essirmer; Elf the old.— \\
It is meaningful that one might know thus; wilt thou [hear] yet further?\evb\evg


\bvg\bva Kętill hét vinr þęira \hld\ Klypps arf-þęgi, &
vas hann móður-faðir \hld\ móður þinnar; &
þar vas Fróði \hld\ fyrr ęnn Kári, &
en Hildi vas \hld\ Hóalfr of getinn.\eva

\bvb Kettle was their friend, the heir of Clip; \\
he was the father of thy mother's mother. \\
There was Frood, yet earlier Keer, \\
but by Hild was Highelf begotten.\evb\evg

... %TODO More dialogue

\sectionline
% Frow
	\bookStart{The Lay of Wayland}[Vǫlundarkviða]

\begin{flushright}%
Dating \parencite{Sapp2022}: C10th (0.428)–early C11th (0.475)

Meter: \Fornyrdislag%
\end{flushright}%

% Introduction

The \textbf{Lay of Wayland} (\Volundarkvida) is a story of immense psychological complexity, one of the masterpieces of Norse poetry.

The poem begins with a prose introduction, which survives in both \Regius\ and \AM.

Wayland gets his revenge on the whole royal household. He murders Nithad’s two young sons (affectionately, his “bear-cubs”) and thus ends his male lineage. Likewise he defangs Nithad’s “cunning wife” (she is never called anything else) by reducing her once powerful counsels to cold words; and finally he rapes Beadhild, depriving her of her maidenhood and value in marriage. They are thus reduced to the same state of complete powerlessness as he himself experienced, something clearly seen in the repetition of the adjective \emph{viljalauss} ‘powerless’; in v. 12 it describes Wayland after he wakes in shackles, but in v. 31 Nithad uses it to refer to his own mental state after the deaths of his sons. This sense of hopelessness is also seen in Beadhild’s haunting concluding speech. “I knew by naught struggle against him; I could by naught struggle against him.”

From the other versions of the story it is known that Beadhild gave birth to a son, Woody (OE \emph{Wudga}, \ThidreksSaga\ \emph{Viðga}, in Danish ballads \emph{Vidrik Verlandsøn}). He went on to become a great hero, and in the later heroic ballads by far eclipses his father. His birth seems heavily foreshadowed by Wayland forcing Nithad to swear an oath in v. 33, but he is nowhere directly mentioned in the poem, probably for artistic reasons.

Apart from this lay there is one other telling of the full story, namely the Strand of Wayland the Smith in \ThidreksSaga. While written in Old Norse, it is clear from the proper names and content that it is based on German sources (probably heroic ballads). Thus the native form \emph{Vǫlundr} is replaced with \emph{Velent} [\emph{sic}], \emph{Níðuðr} with \emph{Niðungr}. Interestingly there is a note within it showing that the native form was still known, namely about “Velent, the excellent smith, whom Warrings (\emph{Væringjar}) call Wayland (\emph{Vǫlundr})”. Apparently Wayland was so famous that “all men seem to praise his workmanship so, that the maker of any smith’s work which is made better than other works, is called a Wayland (\emph{Vǫlundr}) with regards to workmanship.”

Far more stark than minor differences of language is that of tone. The psychological complexity and tension of the older redaction is almost entirely gone: Wayland is no longer a mysterious wild man, but a chivalrous knight who can escape from any peril through his ingenuity and craftmanship. He is not kidnapped out of Nithad’s greed, nor hamstrung out of the suspicion of his cruel wife, but rather a loyal servant of Nithad’s, banished from the kingdom after defending himself against the king’s corrupt steward, and hamstrung after being caught attempting to poison the king’s food in revenge.

Most frustratingly the personality of Beadhild is entirely expulged. She is the anonymous “king’s daughter”, an unnamed maiden (\emph{jungfrú}, a borrowing from Low German) who is peacefully seduced by Wayland and quickly falls in love with him. Likewise the person of Nithad’s cunning wife is completely gone, and the murder of his sons no longer ends his lineage, since he has another, older son who survives him and takes over the kingdom. Wayland still flies away laughing after telling Nithad what he has done, but only four years (his son with Beadhild is three years old) later reconciliates with Nithad’s son, retrieves Beadhild and their son and lives a long life as a famous craftsman.

With this it is clearly seen that the story by the time of the \ThidreksSaga\ had been heavily distorted, a tragic victim of medieval romantic sensibilities. It does not have any high literary value, but is of interest since it shows the wide reception and variation of the narrative.

Finally there are also traces of the story in the Anglo-Saxon tradition, where it is alluded to in both \Waldere\ and \Deor, the latter of which particularly emphasising the powerlessness felt by Wayland and Beadhild (thus being much closer in spirit to the present poem than to \ThidreksSaga). Parts of the narrative are depicted on the early C8th Frank’s casket, where it is as prominent as the depiction of the Adoration of the Magi—a true testament to the weight with which it was regarded within that culture.

To illustrate the narrative correspondences and differences of the various redactions, I present the following table:
\begin{longtabu} to \textwidth {|c c c c c c|}
	\hline
	Person & \Volundarkvida & \ThidreksSaga & \Deor & \Waldere & Frank’s casket \\ [0.5ex]
	\hline\hline
  Wayland & \emph{Vǫlundr} & \emph{Velent} & \emph{Welund} & \emph{Weland} & + \\
  Wayland’s brothers & Agle and Slayfinn & Agle & − & − & Agle? \\
  Father of the brothers & A Finnish king & The riser Wade (\emph{Vaði}) of Zealand & − & − & − \\
  Nithad & \emph{Níðuðr}, lord of the Nears in Sweden & \emph{Niðungr}, king of \inx[G]{Thede} in Jutland & \emph{Niðhad} & \emph{Niðhad} & − \\
  Nithad’s daughter (Beadhild) & \emph{Bǫðvildr} & Unnamed & \emph{Beadohild} & − & + \\
  Nithad’s sons & Two & Three & More than one & − & At least one \\
  Wayland and Beadhild’s son (Woody) & − & Viðga & − & Widia & − \\
  Wives of the brothers & Wayland and Allwit, Agle and Alerune, Slayfinn and Swanwhite & Agle and Alerune & − & − & − \\ [0.5ex]
  \hline
  — & Wayland and his brothers ski and hunt animals. They settle in the Wolfdales, and one day find their wives. When they suddenly leave, Slayfinn and Agle go out to search for them, while Wayland remains alone, smithing rings and longing for his wife. & Wade sends Wayland away to learn smithing from dwarfs, and he becomes exceptionally skilled. He enters into the service of Nithad and becomes his trusted friend. One day Nithad asks Wayland to retrieve his victory-stone (the owner of which will always have victory in battle), in exchange for which he will be given half of his kingdom and his daughter in marriage. After retrieving it, Wayland is ambushed by Nithad’s steward, who asks him for the stone. When he refuses, the steward attacks him, but Wayland easily slays the steward, whose men flee. Wayland is banished for this, but returns and attempts to poison the king’s food. & − & − & − \\
  — & Nithad learns that Wayland is alone, and arrives with a large number of warriors. He abducts him, and on the counsel of his wife has him hamstrung. & After the trick is discovered, Wayland is caught and hamstrung by the king. & − & − & − \\
  — & Wayland is placed on the island Seastead, and forced to make jewellery for him. & Wayland pretends to reconcile with the king, acknowledging his error, and saying that he will never flee, even if he is able to. In return for this he is given a smithy, and as much gold and silver as he asks for. & − & − & − \\ [1ex]
	\hline
\end{longtabu}

\sectionline

\section{Regarding Wayland (\emph{Frá Vǫlundi})}

\bpg\bpa Níðuðr hét konungr í Svíþjóð.
Hann átti tvá sonu ok eina dóttur. Hon hét Bǫðvildr.
Brę́ðr váru þrír, synir Finnakonungs.
Hét einn Slagfiðr, annarr Egill, þriði Vǫlundr.
Þeir skriðu ok veiddu dýr. Þeir kómu í Úlfdali ok gerðu sér þar hús.
Þar er vatn, er heitir Úlfsjár.
Snemma of morgin fundu þeir á vatnsstrǫndu konur þrjár, ok spunnu lín.
Þar váru hjá þeim álftarhamir þeira. Þat váru valkyrjur.
Þar váru tvę́r dę́tr Hlǫðvés konungs, Hlaðguðr svanhvít ok Hervǫr alvitr, in þriðja var Ǫlrún Kjársdóttir af Vallandi.
Þeir hǫfðu þę́r heim til skála með sér. Fekk Egill Ǫlrúnar, en Slagfiðr Svanhvítrar, en Vǫlundr Alvitrar.
Þau bjuggu sjau vetr. Þá flugu þę́r at vitja víga ok kómu eigi aftr.
Þá skreið Egill at leita Ǫlrúnar, en Slagfiðr leitaði Svanhvítrar, en Vǫlundr sat í Úlfdǫlum.
Hann var hagastr maðr, svá’t menn viti í fornum sǫgum.
Níðuðr konungr lét hann hǫndum taka, svá sem hér er um kveðit:\epa

\bpb Nithad was named a king in Sweden.
He owned two sons and one daughter; she was called Beadhild.
There were three brothers, the sons of a king of the Finns.
One was called Slayfinn, another Agle, the third Wayland.
They travelled on skis and hunted wild animals. They came into the Wolfdales and made for themselves houses there.
There is a water there, called Wolfsea.
Early in the morning they found on the lake-shore three women, and they were spinning linen.
By them were their swan-\inx[C]{hame}[hames]; they were Walkirries.
Two of them were the daughters of king Ladwigh: Ladguth Swanwhite and Harware Allwit, the third was Alerune, daughter of \inx[P]{Kear} of \inx[G]{Walland}\footnoteB{The Roman emperor; see Encyclopedia.}.
The brothers brought the maidens with them to their halls. Agle got Alerune, but Slayfinn Swanwhite, but Wayland Allwit.
They lived there for seven winters, then they left to attend battles, and did not return.
Then Agle left on skis to look for Alerune, but Slayfinn sought out Swanwhite; but Wayland stayed in the Wolfdales.
He was the most skilled craftsman, as men know, in the ancient saws.
King Nithad had him captured, about which this has been sung:\epb\epg

\sectionline

\bvg
\bva \alst{M}ęyjar flugu sunnan \hld\ \alst{M}yrk-við í gǫgnum &
\alst{a}l-vitr \alst{u}ngar, \hld\ \edtrans{\alst{ø}r-lǫg drýgja;}{fulfill [their] orlay}{\Bfootnote{That is, to fulfill their already laid-down destinies, as described in P1 and st. 3. I disagree with \textcite{MCR2005}[103], who translates this phrase as ‘engage in war’, seeing the latter word as a borrowing from OE (cf. Dutch \emph{oorlog} ‘war’). In fact, the expression \emph{drýgja ørlǫg} is also attested in OE, namely in l. 29 of a poem on the Christian Doomsday (TODO?), about a man going to Hell for his sins: \emph{ond þonne á tó ealdre \hld\ orleg dreógeð} ‘And then (the sinner) suffers his orlay for ever and ever’}} &
þę́r á \alst{s}ę́var-strǫnd \hld\ \alst{s}ęttusk at hvílask &
\alst{d}rósir suð-rǿnar, \hld\ \alst{d}ýrt lín spunnu.\eva

\bvb Maidens flew from the south through Mirkwood\footnoteB{Mirkwood is surely referenced for its association with the war-ravaged lands of the Gots and Huns; a natural environment for Walkirries.}—young allwits\footnoteB{Maybe look at what this means. TODO.}—to fulfill [their] \inx[C]{orlay}. They on the lake-shore set down to rest; the southern ladies span expensive linen.\evb
\evg


\bvg
\bva \alst{Ęi}n nam þęira \hld\ \alst{Ę}gil at vęrja &
\alst{f}ǫgr mę́r \alst{f}ira \hld\ \alst{f}aðmi ljósum; &
ǫnnur vas \alst{S}vanhvít, \hld\ \alst{s}van-fjaðrar dró, &
\edtext{[...]}{\Bfootnote{A line mentioning the name of Slayfinn has certainly gone missing here.}} &
en hin \alst{þ}riðja \hld\ \alst{þ}ęira systir &
varði \alst{h}vítan \hld\ \alst{h}als Vǫlundar.\eva

\bvb One of them began—the fair maiden of men—to embrace Agle in her light bosom. Another was Swanwhite—her swan-feathers she rustled; but the third of the sisters warded the white throat of Wayland.\evb
\evg


\bvg
\bva \alst{S}ǫ́tu \alst{s}íðan \hld\ \alst{s}jau vetr at þat, &
en hinn \alst{á}tta \hld\ \alst{a}llan þrǫ́ðu, &
en hinn \alst{n}íunda \hld\ \alst{n}auðr of skilði, &
\alst{m}ęyjar fýstusk \hld\ á \alst{m}yrkvan við, &
\alst{a}l-vitr \alst{u}ngar \hld\ \alst{ø}r-lǫg drýgja.\eva

\bvb Then they stayed for seven winters at that, but all the eighth they yearned, but the ninth did need divorce them: the maidens longed for the mirky wood: the young allwits, to fulfill orlay.\footnoteB{As Walkirries the \inx[C]{orlay} (already laid-down destiny) of the sisters is to preside over battles for Weden. Remembering this duty they become increasingly restless, until they one day decide to leave when their husbands are out hunting. For the significance of Mirkwood, see note to st. 1.}\evb
\evg


\bvg
\bva Kom þar af \alst{v}ęiði \hld\ \alst{v}eðr-ęygr skyti &
Vǫlundr \alst{l}íðandi \hld\ of \alst{l}angan veg, &
\alst{S}lagfiðr ok Ęgill, \hld\ \alst{s}ali fundu auða, &
gingu \alst{ú}t ok \alst{i}nn \hld\ ok \alst{u}mb sǫ́usk.\eva

\bvb Came there from the hunt the weather-eyed shooter: Wayland passing over a long way. Slayfinn and Agle found the halls deserted; they walked out and in, and looked about.\evb
\evg


\bvg
\bva \alst{Au}str skręið \alst{Ę}gill \hld\ at \alst{Ǫ}lrúnu, &
en \alst{s}uðr \alst{S}lagfiðr \hld\ at \alst{S}vanhvítu, &
en \alst{ęi}nn Vǫlundr \hld\ sat í \alst{U}lf-dǫlum.\eva

\bvb East skied Agle for Alerune, but south Slayfinn for Swanwhite; but alone Wayland stayed in the Wolfdales.\evb
\evg


\bvg
\bva Hann sló \alst{g}oll rautt \hld\ við \alst{g}im fastan, &
\alst{l}ukði hann alla \hld\ \alst{l}inn-baugum vęl; &
\alst{s}vá bęið hann \hld\ \alst{s}innar ljóssar &
\alst{k}vánar, ef hǫ́num \hld\ of \alst{k}oma gęrði.\eva

\bvb He struck red gold by gemstone fastened, enclosed he all the serpent-\inx[C]{bigh}[bighs]\footnoteB{Armlets, torcs resembling serpents, perhaps even literally shaped like them; cf. the Viking age armlet found in a hoard in Undrom, Ångermanland, northern Sweden. Museum ID 108822 HST. TODO: Maybe include photo?} well; thus awaited he his bright wife, if to him she might come.\evb
\evg


\bvg
\bva Þat spyrr \alst{N}íðuðr, \hld\ \alst{N}íara dróttinn, &
at \alst{ęi}nn Vǫlundr \hld\ sat í \alst{U}lf-dǫlum; &
\alst{n}ǫ́ttum fóru sęggir, \hld\ \alst{n}ęglðar vǫ́ru brynjur, &
\alst{sk}ildir bliku þęira \hld\ við hinn \alst{sk}arða mána.\eva

\bvb This learns Nithad, lord of the \inx[G]{Nears}, that alone Wayland stayed in the Wolfdales. By night travelled warriors—nailed were their byrnies\footnoteB{The soldiers had plated armour.}—their shields gleamed by the waning moon.\evb
\evg


\bvg
\bva Stigu ór \alst{s}ǫðlum \hld\ at \alst{s}alar gafli, &
\edtext{gingu \alst{i}nn þaðan \hld\ \alst{ę}nd-langan sal}{\lemma{gingu \dots\ sal ‘went \dots\ hall’}\Bfootnote{Formulaic. The fixed variant line \emph{hón/hann inn of gekk \hld\ ęnd-langan sal} ‘he/she inside did go the endlong hall’ occurs in three other places: sts. 16 and 30 of the present poem, and st. 3 of \Oddrunargratr. \emph{ęnd-langr salr} ‘endlong hall’ occurs in two additional places: st. 27 of \Thrymskvida\ and st. 3 of \Skirnismal. — \emph{ęnd-langr} ‘endlong’ may be rendered as ‘throughout, the entire (length of)’.}}, &
sǫ́u þęir á \alst{b}ast \hld\ \alst{b}auga dręgna, &
\alst{s}jau hundruð allra, \hld\ es sá \alst{s}ęggr átti.\eva

\bvb They stepped down from the saddles by the hall’s gables; went thence inside the endlong hall; saw they on a bast-rope bighs drawn up: seven hundred in all which that man \ken*{= Wayland} owned.\evb
\evg


\bvg
\bva Ok þęir \alst{a}f tóku \hld\ ok þęir \alst{á} létu &
fyr \alst{ęi}nn \alst{ú}tan, \hld\ es \alst{a}f létu; &
kom þar af \alst{v}ęiði \hld\ \alst{v}eðr-ęygr skyti &
Vǫlundr \alst{l}íðandi \hld\ of \alst{l}angan veg.\eva

\bvb And they slid [them] off, and they slid [them] on; but for one, which off they slid.\footnoteB{Nithad’s men take off all the seven hundred rings (presumably to count them) and then put them back on, but they keep just one. This bigh is probably the one mentioned in sts. 17 and 26, since Beadhild has it already when Wayland is brought back after being captured. \textcite{FinnurEdda}\ writes (\emph{My translation from the Danish.}): “The ring which Nithad kept must have had special properties, and distinguished itself before others. There is no doubt that the ring is a flight ring; whether this was clear to the poet is however questionable. This much is certain, that Wayland seems to be able to fly away only after he has got back the ring; that is, the one which Beadhild brings him.” —The reader may for himself judge the plausibility of this, but it seems that Wayland, being an exceptionally handy craftsman, may just as well have crafted wings for himself without need for magical rings. This agrees with the Low German verison and the Daedalus myth, for both of which see the introduction to the present poem.}—Came there from the hunt the weather-eyed shooter: Wayland passing over a long way.\evb
\evg


\bvg
\bva Gekk \alst{b}rúnni \hld\ \alst{b}eru hold stęikja, &
\alst{á}r brann hrísi \hld\ \alst{a}ll-þurru fura, &
\alst{v}iðr hinn \alst{v}ind-þurri, \hld\ fyr \alst{V}ǫlundi.\eva

\bvb Went he the brown she-bear’s hull to roast; in early morning burned the twigs of all-dry pine—the wind-dry wood—before Wayland.\evb
\evg


\bvg
\bva Sat á \alst{b}er-fjalli, \hld\ \alst{b}auga talði, &
\alst{a}lfa ljóði \hld\ \alst{ęi}ns saknaði; &
\alst{h}ugði at \alst{h}ęfði \hld\ \alst{H}lǫðvés dóttir, &
\alst{A}l-vitr \alst{u}nga, \hld\ vę́ri \alst{a}ptr komin.\eva

\bvb Sat he on the bear-pelt, bighs he counted—the prince of elves was missing one! Thought he that Ladwigh’s daughter \ken*{= Harware} might have it; that the young allwit might be come back.\evb
\evg


\bvg
\bva \alst{S}at \alst{s}vá lęngi, \hld\ at \alst{s}ofnaði, &
ok \alst{v}aknaði \hld\ \alst{v}ilja-lauss; &
vissi sér á \alst{h}ǫndum \hld\ \alst{h}ǫfgar nauðir, &
en á \alst{f}ótum \hld\ \alst{f}jǫtur of spęntan.\eva

\bvb Sat he so long that asleep he fell, and he awoke, powerless. He knew on his hands tortuous restraints, and on his feet were fetters tightened.\evb
\evg


\bvg
\bva „Hvęrir ’ru \alst{jǫ}frar \hld\ þęir’s \alst{á} lǫgðu &
\alst{b}ęsti-síma \hld\ ok \alst{b}undu mik?“\eva

\bvb {[Wayland quoth:]} “Which are the princes, those that laid on thick bast-ropes, and bound me?”\evb
\evg


\bvg
\bva Kallaði \alst{n}ú \alst{N}íðuðr, \hld\ \alst{N}íara dróttinn: &
„Hvar gazt \alst{V}ǫlundr, \hld\ \alst{v}ísi alfa, &
\alst{ó}ra \alst{au}ra, \hld\ í \alst{U}lf-dǫlum? &
\alst{G}oll vas þar ęigi \hld\ á \alst{G}rana lęiðu, &
\alst{f}jarri hugða’k várt land \hld\ \alst{f}jǫllum Rínar.“\eva

\bvb Now called Nithad, lord of the Nears: “Where gottest thou, Wayland, leader of elves, \emph{our} ounces in the Wolfdales? Gold was there not on \inx[P]{Grane}’s path; far I thought our land from the fells of the Rhine.\footnoteB{Grane was the horse of the legendary hero \inx[P]{Siward}, slayer of the dragon \inx[P]{Fathomer}. These events were thought to have taken place in Germany. The sense of the is thus sarcastic: “Where did you get that gold? A dragon’s hoard?”.}”\evb
\evg


\bvg
\bva „\alst{M}an’k at \alst{m}ęiri \hld\ \alst{m}ę́ti ǫ́ttum, &
es vér \alst{h}ęil \alst{h}jú \hld\ \alst{h}ęima vǫ́rum: &
\alst{H}laðguðr ok \alst{H}ęrvǫr \hld\ borin vas \alst{H}lǫðvé, &
\alst{k}unn vas Ǫlrún \hld\ \alst{K}íars dóttir.“\eva

\bvb {[Wayland quoth:]} “I remember that we owned greater wealth, when we a whole household were at home: Ladguth and Harware were born to Ladwigh; known was Alerune, Kear’s daughter.”\footnoteB{Wayland responds rather cryptically. It seems that by asserting the noble lineages of the three swan-wives he gives a legitimate reason for his wealth, although he seems to be aware, judging by the tone, that the greedy Nithad neither cares nor believes him.}\evb
\evg

\sectionline

\bvg
\bva Úti stóð \alst{k}unnig \hld\ \alst{k}vǫ́n Níðaðar, &
\edtext{hón \alst{i}nn of gekk \hld\ \alst{ę}nd-langan sal}{\lemma{hon inn \dots\ sal ‘she inside \dots\ hall’}\Bfootnote{Formulaic, also occuring in st. 30 of the present poem and in \Oddrunargratr\ 3.}}, &
\alst{st}óð á golfi, \hld\ \alst{st}ilti rǫddu: &
„es-a sá nú \alst{h}ýrr, \hld\ es ór \alst{h}olti fęrr.\eva

\bvb Outside stood the cunning wife of Nithad; she inside did go the endlong hall; stood on the floor, steered her voice: “That one \ken*{= Wayland} is not mild now, who comes out of the wood.\evb
\evg


\bvg
\bva \alst{T}ęnn hǫ́num \alst{t}ęygjask \hld\ es hǫ́num’s \alst{t}ét sverð &
ok hann \alst{B}ǫðvildar \hld\ \alst{b}aug of þękkir, &
\alst{ǫ́}mun eru \alst{au}gu \hld\ \alst{o}rmi hinum frána; &
\alst{s}níðið ér hann \hld\ \alst{s}ina magni, &
ok \alst{s}ętið hann \alst{s}íðan \hld\ í \alst{S}ę́varstǫð.“\eva

\bvb His teeth are bared when he is shown the sword, and he recognizes Beadhild’s bigh; reminiscent are the eyes to the gleaming serpent’s.—Snithe ye from him the might of his sinews, and set him thereafter on Seastead!”\evb
\evg


\bpg
\bpa Svá var gǫrt, at skornar váru sinar í knés-fótum ok settr í holm einn, er þar var fyrir landi, er hét Sę́varstaðr. Þar smíðaði hann konungi alls-kyns gǫr-simar; engi maðr þorði at fara til hans, nema konungr einn. Vǫlundr kvað:\epa

\bpb Thus was done, that the sinews in his houghs were cut, and he was placed on a lonely islet lying there before the land, which was called Seastead. There he smithed for the king all manner of jewels. No man dared journey to him, save for the king alone. Wayland quoth:\epb
\epg


\bvg
\bva „\alst{S}é’k Níðaði \hld\ \alst{s}verð á linda, &
þat’s ek \alst{h}vęsta \hld\ sęm \alst{h}agast kunna’k &
ok ek \alst{h}ęrða’k \hld\ sęm \alst{h}ǿgst þótti; &
sá ’s mér \alst{f}ránn mę́kir \hld\ ę́ \alst{f}jarri borinn; &
\alst{s}é’kk-a þann Vǫlundi \hld\ til \alst{s}miðju borinn.\eva

\bvb “I see a sword on Nithad’s belt, that one I sharpened as most handily I knew, and hardened as most pleasingly seemed. Now that gleaming blade is ever far from me carried; I see it not for Wayland to the smithy carried.\evb
\evg


\bvg
\bva Nú \alst{b}err \alst{B}ǫðvildr \hld\ \alst{b}rúðar minnar &
—\alst{b}íð’k-a þęss \alst{b}ót— \hld\ \alst{b}auga rauða.“\eva

\bvb Now Beadhild bears my bride’s—I await no bettering for that—red bighs.”\evb
\evg


\bvg
\bva \edtrans{\alst{S}at—né \alst{s}vaf á-valt—}{He sat—he slept not—}{\Bfootnote{Compare \Gudrunarhvot\ TODO: \emph{hófu mik—né drękkðu—} ‘lifted me—drowned [me] not—’.}} \hld\ ok \alst{s}ló hamri; &
vél gęrði \alst{h}ęldr \hld\ \alst{h}vatt Níðaðí; &
\alst{d}rifu ungir tvęir \hld\ á \alst{d}ýr séa &
\alst{s}ynir Níðaðar \hld\ í \alst{S}ę́varstǫð.\eva

\bvb He sat—he slept not—and struck the hammer; he very boldly planned wiles for Nithad.—Two young ones drifted to look at precious things: Nithad’s sons, onto Seastead.\evb
\evg


\bvg
\bva \alst{K}vǫ́mu til \alst{k}istu, \hld\ \alst{k}rǫfðu lukla, &
\alst{o}pin vas \alst{i}llúð, \hld\ es \alst{í} sǫ́u, &
fjǫlð vas þar \alst{m}ęina, \hld\ es \alst{m}ǫgum sýndisk &
at vę́ri \alst{g}oll rautt \hld\ ok \alst{g}ǫr-simar.\eva

\bvb Came they to the chest, demanded the keys; open was the evil when inside they looked. A great deal was there of harms, which to the lads seemed like were it red gold and jewels.\evb
\evg


\bvg
\bva „Komið \alst{ęi}nir tvęir, \hld\ komið \alst{a}nnars dags; &
ykkr lę́t’k þat \alst{g}oll \hld\ of \alst{g}efit verða; &
\alst{s}ęgið-a męyjum \hld\ né \alst{s}al-þjóðum, &
\alst{m}anni ęngum, \hld\ at \alst{m}ik fyndið.“\eva

\bvb {[Wayland quoth:]} “Come alone ye two, come another day; to you I will let that gold be given. Say not to maidens nor to the people of the hall—to no man—that ye met me!”\evb
\evg


\bvg
\bva \alst{S}nimma kallaði \hld\ \alst{s}ęggr á annan, &
\alst{b}róðir á \alst{b}róður: \hld\ „gǫngum \alst{b}aug séa!“ &
\alst{K}vǫ́mu til \alst{k}istu, \hld\ \alst{k}rǫfðu lukla, &
\alst{o}pin vas \alst{i}llúð \hld\ es \alst{í} litu.\eva

\bvb Early called one youth to another, brother to brother: “Let us go see the bighs!” Came they to the chest, demanded the keys; open was the evil when inside they looked.\evb
\evg


\bvg
\bva Snęið af \alst{h}ǫfuð \hld\ \alst{h}úna þęira &
ok und \alst{f}ęn \alst{f}jǫturs \hld\ \alst{f}ǿtr of lagði, &
ęn þę́r \alst{sk}álar, \hld\ es und \alst{sk}ǫrum vǫ́ru, &
\alst{s}vęip útan \alst{s}ilfri, \hld\ \alst{s}ęldi Níðaði.\eva

\bvb He sliced off the heads of those bear-cubs\footnoteB{An affectionate term for the young boys. TODO: Relate to Bearserks?} \ken{boys}, and under the fetter’s fen\footnoteB{Very unclear. TODO.} their feet did lay; but the bowls which were under their curls \ken{skulls}, he coated with silver and gave to Nithad.\evb
\evg


\bvg
\bva \alst{E}n ór \alst{au}gum \hld\ \alst{ja}rkna-stęina &
sęndi \alst{k}unnigri \hld\ \alst{k}vǫ́n Níðaðar; &
en ór \alst{t}ǫnnum \hld\ \alst{t}vęggja þęira &
\alst{s}ló brjóst-kringlur, \hld\ \alst{s}ęndi Bǫðvildi.\eva

\bvb But out of the eyes earkenstones he sent to the cunning wife of Nithad; but out of the teeth of the two he struck breast-brooches, sent to Beadhild.\evb
\evg

\sectionline

\bvg
\bva Þá nam \alst{B}ǫðvildr \hld\ \alst{b}augi at hrósa &
\edtext{[...]}{\Bfootnote{The meter requires a half-line here, likely containing a more specific description of the bigh.}}\ \hld\ es brotit hafði, &
„\alst{þ}ori’k-a’k sęgja, \hld\ nema \alst{þ}ér ęinum.“\eva

\bvb Then Beadhild began to praise the ring,\footnoteB{The verse is without doubt incomplete, but the story can be gleaned: Beadhild breaks the bigh she has been given by her parents (previously mentioned in vv. 10 (see note there) and 17), and is afraid that her parents may become upset. She thus goes to Wayland in secret, asking him to repair it.} [...] which she had broken, “I dare not tell it, save to thee alone.”\evb
\evg


\bvg
\bva „Ek \alst{b}ǿti svá \hld\ \alst{b}rest á golli, &
at \alst{f}ęðr þínum \hld\ \alst{f}ęgri þykkir, &
ok \alst{m}ǿðr þinni \hld\ \alst{m}iklu bętri, &
ok \alst{s}jalfri þér \hld\ at \alst{s}ama hófi.“\eva

\bvb {[Wayland quoth:]} “I mend such the crack on the gold, that to thy father it fairer seems, and to thy mother far better, and to thyself of the same rank.”\evb
\evg


\bvg
\bva \alst{B}ar hann hána \alst{b}jóri, \hld\ því’t hann \alst{b}ętr kunni, &
\alst{s}vá’t hón í \alst{s}essi \hld\ of \alst{s}ofnaði. &
„Nú \alst{h}ęf’k \alst{h}ęfnt \hld\ \alst{h}arma minna &
\alst{a}llra nema \alst{ę}inna \hld\ \alst{í}-við-gjǫrnum.“\eva

\bvb He overcame her with beer—for he knew better\footnoteB{i.e. was more cunning, experienced than her.}—so that she in the seat asleep did fall. “Now have I avenged my harms—all but one\footnoteB{Presumably the deprivation of his mobility due to the hamstringing, which he resolves in the following stanza.}—on the insidious ones.\footnoteB{King Nithad and his family.}”\evb
\evg


\bvg
\bva „\alst{V}ęl ek,“ kvað \alst{V}ǫlundr, \hld\ „\alst{v}erða’k á fitjum, &
þęim’s mik \alst{N}íðaðar \hld\ \alst{n}ǫ́mu rekkar.“ &
\alst{H}lę́jandi Vǫlundr \hld\ \alst{h}ófsk at lopti, &
\alst{g}rátandi Bǫðvildr \hld\ \alst{g}ekk ór ęyju. &
tregði \alst{f}ǫr \alst{f}riðils \hld\ ok \alst{f}ǫður vręiði.\eva

\bvb “Well I”, quoth Wayland, “fall on my paddles; those which Nithad’s men bereaved me of!\footnoteB{\emph{C-V}: \emph{fit} ‘the webbed foot of water-birds’, the reader may picture for himself. Wayland has crafted a mechanism to take flight, regaining his mobility which he lost when he was hamstrung.}” Laughing Wayland threw himself in the air; weeping Beadhild went from the island: she grieved the lover’s flight, and the father’s fury.\evb
\evg

\sectionline

\bvg
\bva Úti stóð \alst{k}unnig \hld\ \alst{k}vǫ́n Níðaðar, &
ok hón \alst{i}nn of gekk \hld\ \alst{ę}nd-langan sal, &
en hann á \alst{s}al-garð \hld\ \alst{s}ęttisk at hvílask, &
„Vakir þú \alst{N}íðuðr, \hld\ \alst{N}íara dróttinn?“\eva

\bvb Outside stood the cunning wife of Nithad, and she inside did go the endlong hall—but he, on the courtyard, set down to rest. “Art thou awake, Nithad, lord of the Nears?”\evb
\evg


\bvg
\bva „\alst{V}aki’k á-\alst{v}alt \hld\ \alst{v}ilja-lauss, &
\alst{s}ofna’k minst, \hld\ síz \alst{s}onu dauða, &
\alst{k}ęll mik í hǫfuð, \hld\ \alst{k}ǫld erumk rǫ́ð þín, &
\alst{v}ilnumk þęss nú, \hld\ at við \alst{V}ǫlund dǿma’k.“\eva

\bvb {[Nithad quoth:]} “I am always awake, powerless; I fall asleep the least, since the death of my sons. My head freezes; cold are thy counsels—I wish now but that: to speak with Wayland.”\evb
\evg

\sectionline

\bvg
\bva „Sęg mér þat \alst{V}ǫlundr, \hld\ \alst{v}ísi alfa, &
af \alst{h}ęilum \alst{h}vat varð \hld\ \alst{h}únum mínum?“\eva

\bvb {[Nithad quoth:]} “Say it to me, Wayland, leader of elves: what became of my healthy bear-cubs \ken{boys}?”\evb
\evg


\bvg
\bva „\alst{Ęi}ða skalt mér \alst{á}ðr \hld\ \alst{a}lla vinna, &
at \alst{sk}ips borði \hld\ ok at \alst{sk}jaldar rǫnd, &
at \alst{m}ars bǿgi \hld\ ok at \alst{m}ę́kis ęgg &
at þú \alst{k}vęlj-at \hld\ \alst{k}vǫ́n Vǫlundar, &
né \alst{b}rúði minni \hld\ at \alst{b}ana verðir, &
þótt kvǫ́n \alst{ęi}gim, \hld\ þá’s \alst{é}r kunnið, &
eða \alst{jó}ð \alst{ęi}gim \hld\ \alst{i}nnan hallar.\eva

\bvb {[Wayland quoth:]} “Before that shalt thou swear to me all oaths:—by the deck of the ship and the rim of the shield, by the bough of the steed and the edge of the sword—that thou wilt not torment the wife of Wayland, nor of my bride become the bane, though a wife we might own, which ye might know; or a babe might own within the hall.\footnoteB{Wayland has Nithad swear an oath that he will not harm Beadhild, nor their (yet unborn) child. For the form of the oaths cf. TODO.}\evb
\evg


\bvg
\bva \alst{G}akk til smiðju, \hld\ es \alst{g}ęrðir þú, &
þar fiðr þú \alst{b}ęlgi \hld\ \alst{b}lóði stokna, &
snęið’k af \alst{h}ǫfuð \hld\ \alst{h}úna þinna &
ok und \alst{f}ęn \alst{f}jǫturs \hld\ \alst{f}ǿtr of lagða’k.\eva

\bvb Go to the smithy, which thou madest; there wilt thou find bellows, sprinkled with blood. I sliced off the heads of thy bear-cubs \ken{boys}, and under the fetter’s fen their feet did I lay.\evb
\evg


\bvg
\bva En þę́r \alst{sk}álar, \hld\ es und \alst{sk}ǫrum vǫ́ru, &
\alst{s}vęip’k útan \alst{s}ilfri, \hld\ \alst{s}ęlda’k Níðaði, &
\alst{e}n ór \alst{au}gum \hld\ \alst{ja}rkna-stęina, &
sęnda’k \alst{k}unnigri \hld\ \alst{k}vǫ́n Níðaðar.\eva

\bvb But the bowls, which were under their curls, I coated with silver and gave to Nithad. But out of the eyes earkenstones I sent to the cunning wife of Nithad.\evb
\evg


\bvg
\bva En ór \alst{t}ǫnnum \hld\ \alst{t}vęggja þęira &
\alst{s}ló’k brjóst-kringlur, \hld\ \alst{s}ęnda’k Bǫðvildi; &
nú gęngr \alst{B}ǫðvildr \hld\ \alst{b}arni aukin, &
\edtrans{\alst{ęi}nga dóttir \hld\ \alst{y}kkur bęggja.}{the only daughter of you both}{\Bfootnote{Formulaic, near-identical to \HervararSaga\ st. 25/1–2: (\emph{Vaki, Angantýr, \hld\ vękr þik Hęrvǫr, // ęinga dóttir \hld\ ykkr Svǫ́fu.} ‘Wake, Ongentew: Harware awakes thee, the only daughter of thee and Sweve.’ Cf. also \Beowulf\ 375a, 2997b: \emph{ángan dohtor} ‘only daughter (accusative)’.)}}“\eva

\bvb But out of the teeth of the two, I struck breast-brooches, sent to Beadhild. Now walks Beadhild, swollen with child; the only daughter of you both.”\evb
\evg


\bvg
\bva „\alst{M}ę́ltir-a þú þat \alst{m}ál, \hld\ es mik \alst{m}ęir tregi, &
né þik \alst{v}ilja’k \alst{V}ǫlundr \hld\ \alst{v}err of níta; &
es-at svá maðr \alst{h}ǫ́r, \hld\ at þik af \alst{h}ęsti taki, &
\alst{n}é svá ǫflugr, \hld\ at þik \alst{n}eðan skjóti, &
þar’s þú \alst{sk}ollir \hld\ við \alst{sk}ý uppi.“\eva

\bvb {[Nithad quoth:]} “Thou couldst not have spoken that speech which might grieve me more; nor could I worse wish, Wayland, to deny thee. There is no man so high that he from horse might take thee, nor so mighty that he might shoot thee from below, there as thou jeerest against the cloud-cover on high!”\evb
\evg


\bvg
\bva \alst{H}lę́jandi Vǫlundr \hld\ \alst{h}ófsk at lopti, &
en \alst{ó}-kátr Níðuðr \hld\ þá \alst{ę}ptir sat.\eva

\bvb Laughing Wayland threw himself in the air, but gloomy Nithad thereafter stayed.\evb
\evg

\sectionline

\bvg
\bva „Upp rís \alst{Þ}akkráðr, \hld\ \alst{þ}rę́ll minn bazti, &
\alst{b}ið \alst{B}ǫðvildi, \hld\ \edtext{męy hina \alst{b}rá-hvítu, &
gangi \alst{f}agr-varið}{\lemma{męy hina brá-hvítu \dots\ fagr-varið ‘the brow-white maiden \dots\ fair-clothed’}\Bfootnote{With these expressions Nithad strongly stresses the purity of his daughter (\emph{mę́r} ‘maiden’ here simply meaning ‘virgin’). Perhaps he thinks that her innocence can be restored if she dresses in fair clothes, but it will not be so.}} \hld\ við \alst{f}ǫður rǿða.“\eva

\bvb {[Nithad quoth:]} “Rise up, Thankred, my best thrall! Ask Beadhild—the brow-white maiden—to go fair-clothed with her father to counsel.”\evb
\evg

\sectionline

\bvg
\bva „Es þat \alst{s}att Bǫðvildr, \hld\ es \alst{s}ǫgðu mér, &
\alst{s}ǫ́tuð it Vǫlundr \hld\ \alst{s}aman í holmi?“\eva

\bvb {[Nithad quoth:]} “Is it true, Beadhild, as they said to me: stayed thou and Wayland together on the islet?”\evb
\evg


\bvg
\bva „\alst{S}att ’s þat Níðuðr \hld\ es \alst{s}agði þér: &
\alst{s}ǫ́tum vit Vǫlundr \hld\ \alst{s}aman í holmi &
\alst{ęi}na \alst{ǫ}gur-stund, \hld\ \alst{ę́}va skyldi; &
ek \alst{v}ę́tr hǫ́num \hld\ \alst{v}inna kunna’k, &
ek \alst{v}ę́tr hǫ́num \hld\ \alst{v}inna mátta’k.“\eva

\bvb {[Beadhild quoth:]} “’Tis true, Nithad, as \emph{he} said\footnoteB{Beadhild, knowing that the only one who is aware of what happened is Wayland, makes the subtle change in the conjugation, from her father’s general plural (“what \emph{they} said”), to the specific singular (“what \emph{he} said”).} to thee: stayed I and Wayland together on the islet, for one heavy hour—it should never [have been]! I knew by naught struggle against him; I could by naught struggle against him.\footnoteB{She was both mentally (\emph{kunna} ‘to know, understand’) and physically (\emph{mega} ‘to have strength to do, avail’) incapable of struggling against him. — As \textcite{FinnurEdda} comments, an unsurpassed final verse.}”\evb
\evg
%
	\bookStart{Eddic fragments}

\bvg {\small The Golder of Homedall}
\bva Níu em’k \alst{m}ǿðra \alst{m}ǫgr &
Níu em’k \alst{s}ystra \alst{s}onr\eva

\bvb Of nine mothers am I the lad, of nine sisters am I the son.\evb
\evg
% Eddic fragments

\part{Heroic poetry of the Codex Regius}% How to handle... Ideally we'd want a parallel edition with the Saw of the Walsings, since it and the heroic section of the CR clearly derive from the same source.
	\bookStart{First Lay of Hallow Hundingsbane}[Helgakviða Hundingsbana fyrsta]
\def\thisBookCode{HelgakvidaOne}

\begin{flushright}%
\textbf{Dating} \parencite{Sapp2022}: late C12th (0.805)

\textbf{Meter:} \Fornyrdislag%
\end{flushright}%

\section{Introduction}

This rather late poem is very well written.  Particularly beautiful are the introductory stanzas, which tell of Norns arriving in the night to predetermine Hallow’s life.

\sectionline

\section{First Lay of Hallow Hundingsbane}

\bpg\bpa Hér hefr upp kvę́ði frá Helga Hundings bana, þeira ok Hǫðbrodds. Vǫlsunga kviða.\epa

\bpb Here begins a lay regarding Hallow, bane of Hunding and his men, and of Hathbrod. A lay of the Walsings.\epb\epg

\sectionline

\bvg\bva\mssnote{\Regius~20r/21}%
\edtrans{\alst{Á}r vas \alst{a}lda}{It was early of ages}{\Bfootnote{This formulaic introduction immediately situates the events of the poem in the distant mytho-heroic past, indeed, if one compares \Voluspa\ 3/1 where the same line occurs, at the beginning of history.}} \hld\ þat’s \alst{a}rar gullu &
\alst{h}nigu \alst{h}ęilǫg vǫtn \hld\ af \alst{H}imin-fjǫllum; &
þá \alst{h}afði \alst{H}ęlga \hld\ inn \alst{h}ugum stóra &
\alst{B}orghildr \alst{b}orit \hld\ í \alst{B}rálundi.\eva

\bvb It was early of ages when eagles shrieked; \\
holy waters poured down from the Heavenfells; \\
then to Hallow the great of heart \\
had Burhild in Browlund given birth.\evb\evg


\bvg\bva\mssnote{\Regius~20r/23}%
\alst{N}ǫ́tt varð ï bǿ, \hld\ \alst{n}ornir kvǫ́mu, &
þę́r’s \alst{ǫ}ðlingi \hld\ \alst{a}ldr of skópu; &
þann bǫ́ðu \alst{f}ylki \hld\ \alst{f}rę́gstan verða &
ok \alst{b}uðlunga \hld\ \alst{b}ętstan þykkja.\eva

\bvb It turned night in the settlement; norns oncame, \\
they who shaped the athling’s age. \\
They bade that battle-arrayer become the noblest \\
and among princes seem the best.\evb\evg


\bvg\bva\mssnote{\Regius~20r/25}%
Snøru þę́r af \alst{a}fli \hld\ \alst{ø}r·lǫg-þǫ́ttu &
þȧ’s \alst{b}orgir \alst{b}raut \hld\ ï \alst{B}rálundi; &
þę́r of \alst{g}ręiddu \hld\ \alst{g}ollin-símu &
ok und \alst{m}ȧna sal \hld\ \alst{m}iðjan fęstu.\eva

\bvb They turned with strength orlay strands \\
when castles were broken in Browlund. \\
They arranged a golden cord \\
and beneath the moon’s hall \ken{sky/heaven} fastened it in the middle.\evb\evg


\bvg\bva\mssnote{\Regius~20r/27}%
Þę́r \alst{au}str ok vestr \hld\ \alst{ę}nda fǫ́lu, &
þar átti \alst{l}ofðungr \hld\ \alst{l}and ȧ milli, &
brá \alst{n}ipt \alst{N}era \hld\ ȧ \alst{n}orðr-vega &
\alst{ę}inni fęsti, \hld\ \alst{ęy} bað hȯn halda.\eva

\bvb In the east and west they hid its ends; \\
there the praised man owned land in between. \\
The kinswoman of Nare \ken{norn} pulled onto the northern ways \\
a single strand—she bade it hold forever.\evb\evg

TODO: more stanzas.

\sectionline
%
	\bookStart{Lay of Hallow Harwardson}[Hęlgakviða Hjǫrvarðssonar]
\def\thisBookCode{HelgakvidaHjorvardssonar}

\begin{flushright}%
\textbf{Dating} \parencite{Sapp2022}: early C11th (0.385)–late C11th (0.550)

\textbf{Meter:} \Fornyrdislag%
\end{flushright}%

Heroic poem.

\sectionline

\section{From Harward and Syelind (\emph{Frá Hjǫr·varði ok Sigr·linn})}

\bpg\bpa Hjǫr·varðr hét konungr; hann átti fjórar konur.  Ein hét Alf·hildr; sonr þeira hét Heðinn.  Ǫnnur hét Sę́·reiðr; þeira sonr hét Humlungr.  In þriðja hét Sinrjóð; þeira sonr hét Hymlingr.  Hjǫr·varðr konungr hafði þess heit strengt at eiga þá konu er hann vissi vę́nsta.  Hann spurði at Sváfnir konungr átti dóttur allra\footnote{‘vęnallra’ \emph{corr.} \Regius} fegrsta; sú hét Sigr·linn.  Ið·mundr hét jarl hans; Atli var hans sonr er fór at biðja Sigr·linnar til handa konungi.  Hann dvalðisk vetr-langt með Sváfni konungi.  Frán·marr hét þar jarl, fóstri Sigr·linnar; dóttir hans hét Álǫf.  Jarl’inn réð, at meyjar var synjat, ok fór jarlinn heim.  Atli jarls sonr stóð einn dag við lund nǫkkurn, en fugl sat í limunum uppi yfir hánum ok hafði heyrt til, at hans menn kǫlluðu vę́nstar konur þę́r, er Hjǫr·varðr konungr átti.  Fugl’inn kvakaði, en Atli hlýddi, hvat hann sagði. Hann kvað:\epa

\bpb Hearward was the name of a king; he had four women.  One was called Elfhild; their son was called Headen.  Another was called Searad; their son was called Humbling.  The third was called Sindred; their son was called Himbling.  King Hearward had made a vow to have those women whom he knew the most handsome.  He learned that king Swebner had a daughter fairest of all; she was called Syelind.  Ithmund was the name of his earl; Attle was his son, who journeyed to ask for Syelind’s hand on behalf of the king.  He stayed over the winter with king Swebner.  Frenmar was the name of an earl there, the foster-father of Syelin; his daughter was called Anlab.  The bird twittered, and Attle listened to what it said.  It quoth:\epb\epg


\bvg\bva%
„\alst{S}átt-u \alst{S}igr·linn, \hld\ \alst{S}váfnis dóttur, &
męyna fęgrstu \hld\ ï \alst{m}unar-hęimi? &
Þó \alst{h}ag-ligar \hld\ \alst{H}jǫr·varðs konur &
\alst{g}umnum þykkja \hld\ at \alst{G}lasis-lundi.“\eva

\bvb “Hast thou seen Syelind Swebner’s daughter, \\
the fairest of maidens in the realm of love \ken{world}? \\
Although to mankind Hearward’s wives \\
seem handsome in Glazerslund.”\evb\evg


\bvg\bva%
„Munt við \alst{A}tla \hld\ \alst{I}ð·mundar son &
\alst{f}ugl \alst{f}róð-hugaðr \hld\ \alst{f}lęira mę́la?“ &
„Mun’k ef mik \alst{b}uðlungr \hld\ \alst{b}lóta vildi &
ok \alst{k}ýs’k þat’s ek vil \hld\ ór \alst{k}onungs garði.“\eva

\bvb “Wilt thou with Attle Idmund’s son, \\
O wise-minded fowl, speak yet further?” \\
“I will, if the prince will make me a bloot, \\
and I may choose what I wish from the house of the king.”\evb\evg


\bvg\bva%
Kjós-at-tu Hjǫr·varð \hld\ né hans sonu &
\eva

\bvb 3\evb\evg


\bvg\bva%
Hof mun’k kjósa, \hld\ hǫrga marga, &
gull-hyrndar kýr \hld\ frȧ grams búi, &
ef hǫ́num Sigr·linn \hld\ søfr ȧ armi &
ok ȯ·nauðig \hld\ jǫfri fylgir.\eva

\bvb 4\evb\evg


\bpg\bpa%
Þetta var áðr Atli fǿri. En er hann kom heim ok konungr spurði hann tíðinda, hann kvað:\epa

\bpb TODO.\epb\epg

\bvg\bva%
Hǫfum erfiði \hld\ ok ękki ørendi;\eva

\bvb 5\evb\evg


\bpg\bpa%
TODO.\epa

\bpb TODO.\epb\epg


\bpg\bpa%
TODO.\epa

\bpb TODO.\epb\epg


\bvg\bva%
6\eva

\bvb 6\evb\evg


\bvg\bva%
7\eva

\bvb 7\evb\evg


\bvg\bva%
\alst{S}verð vęit’k liggja \hld\ ï \alst{S}igars-holmi, &
\alst{f}jórum \alst{f}ę́ra \hld\ enn \alst{f}imm tǫgu; &
\alst{ęi}tt es þęira \hld\ \alst{ǫ}llum bętra &
\alst{v}íg-nesta bǫl \hld\ ok \alst{v}arið gulli.\eva

\bvb Swords I know lying in Sigarsholm: \\
four less than fifty. \\
One of them is better than all—\\
a \inx[C]{bale} of war-covers(?) \ken{shields}—and covered with gold.\evb\evg


\bvg\bva%
\edtrans{\alst{H}ringr ’s ï \alst{h}jalti}{A ring is on its hilt}{\Bfootnote{The sword is a ring-sword.  It was popular among Germanic warriors of the Migration Period to have oath-ring on their sword-hilts as a symbol of fidelity to their lords.  This custom was largely or entirely extinct by the Wiking Age, and the detail thus serves to emphasize the high age of the sword.  A well preserved Norwegian ring-sword survives from Snartemo in Vest-Agder,  dating to around 500 CE (object ID C26001); see Fig. \ref{fig:snartemo}.}}, \hld\ \alst{h}ugr ’s ï miðju, &
\alst{ó}gn ’s ï \alst{o}ddi, \hld\ þęim’s \alst{ęi}ga getr; &
liggr með \alst{ę}ggju \hld\ \alst{o}rmr dręyr-fáiðr &
en ȧ \edtrans{\alst{v}al-bǫstu}{walbast}{\Bfootnote{An unclear part of the sword-hilt; see \Sigrdrifumal\ 6.}} \hld\ \alst{v}erpr naðr hala.\eva

\bvb A ring is on its hilt; heart is in the middle; \\
terror is in the point for him who gets to own it. \\
Along the edge lies a serpent painted in blood \\
and on the walbast an adder eats its tail.\evb\evg

\begin{figure}
\centering
\includegraphics[width=\textwidth]{Snartemo-hilt}
\caption{Hilt of the Snartemo sword, front and reverse.  Migration period, ca. 500 CE.  © Eirik Irgens Johnsen, \href{https://creativecommons.org/licenses/by-sa/4.0/deed.en}{CC BY-SA 4.0}.  \url{https://www.unimus.no/portal/\#/photos/d8932af5-1082-4938-9b4b-ca6b86f2bdfb}}
\label{fig:snartemo}
\end{figure}


\bpg\bpa%
TODO.\epa

\bpb TODO.\epb\epg


TODO: many stanzas


\bpg\bpa%
Helgi ok Sváfa er sagt at vę́ri endr-borin.\epa

\bpb Hallow and Sweve, it is said, were reborn.\epb\epg

\sectionline

	\bookStart{Second Lay of Hallow Hundingsbane}[Helgakviða Hundingsbana aðra]

\begin{flushright}%
\textbf{Dating} \parencite{Sapp2022}: late C11th (0.587)

\textbf{Meter:} \Fornyrdislag\ (TODO)%
\end{flushright}

\section{Introduction}

TODO: Introduction.

The latter part of the poem features a touching description of Syreun’s visit to Hallow’s grave.  It reflects a folkloric motif found in many traditional British ballads, e.g. Roud 50 (Sweet William’s Ghost), Roud 179 (the Lover’s Ghost or the Grey Cock), and Roud 22568 (the Night Visiting Song), where two lovers must part at cock-crow, although in some variants of 179 and 22568 the supernatural element is not explicit.  Compare the version recorded by \emph{The Dubliners} in 1972:

\begin{quote}\itshape I must away now; I can no longer tarry \\
This morning’s tempest I have to cross \\
I must be guided without a stumble \\
Into the arms I love the most. \\

And when he came to his true love’s dwelling \\
He knelt down gently upon a stone \\
And through her window he’s whispered lowly: \\
“Is my true lover within at home?” \\

“Wake up, wake up, love, it is thine own true lover \\
Wake up, wake up, love, and let me in \\
For I am tired, love, and oh so weary \\
And more than near drenched to the skin.” \\

She’s raised her off her down soft pillow \\
She’s raised her up and she’s let him in \\
And they were locked in each other’s arms \\
Until that long night was past and gone. \\

And when that long night was past and over \\
And when the small clouds began to grow \\
He’s taken her hand and they’ve kissed and parted \\
Then he saddled and mounted and away did go. \\

I must away now \emph{et. c.}\end{quote}

\sectionline

\section{The Second Lay of Hallow Hundingsbane}

... TODO ...

\bpg\bpa Hęlgi fekk Sigrúnar ok ǫ́ttu þau sonu; vas Hęlgi ęigi gamall.  Dagr Hǫgna sonr blótaði Óðin til fǫður-hefnda. Óðinn léði Dag gęirs síns.  Dagr fann Helga, mág sinn, þar sem hęitir at Fjǫturlundi.  Hann lagði í gǫgnum Hęlga með gęir’num.  Þar fell Hęlgi, en Dagr ręið til fjalla ok sagði Sigrúnu tíðindi:\epa

\bpb Hallow got Syerun and they had sons; Hallow was not old.  Day, son of Hain, made a \inx[C]{bloot} to Weden for the sake of avenging his father.  Weden lent Day his spear. Day found Hallow, his brother-in-law, where it is called Fetterlund; he ran through Hallow with the spear.  There Hallow fell, but Day rode to the fells and told Syerun the tidings:\epb\epg


\bvg\bva „\alst{T}rauðr em ek, systir, \hld\ \alst{t}rega þér at sęgja &
því-at ek hęfi \alst{n}auðigr \hld\ \alst{n}ipti grǿtta: &
\alst{F}ell í morgun \hld\ und \alst{F}jǫturlundi &
\alst{b}uðlungr sá’s vas \hld\ \alst{b}ętstr í hęimi &
ok \alst{h}ildingum \hld\ á \alst{h}alsi stóð.“\eva

\bvb “Regretful am I, O sister, to grieve thee by saying it— \\
for, forced, must I make my kinswoman weep: \\
this morning fell in Fetterlund \\
that noble who was the best in the world, \\
and on the throats of princes stood.”\evb\evg


\bvg\bva\speakernote{[Sigrún kvað:]}%
„Þik skyli \alst{a}llir \hld\ \alst{ęi}ðar bíta, &
þęir es \alst{H}ęlga \hld\ \alst{h}afðir unna, &
at inu \alst{l}jósa \hld\ \alst{L}ęiptrar vatni &
ok at \alst{ú}r-svǫlum \hld\ \alst{U}nnar steini!\eva

\bvb “\emph{Thee} should all oaths bite, \\
which thou to Hallow hast sworn, \\
by the shining water of Lafter, \\
and by the spray-cold stone of Ithe.\evb\evg


\bvg\bva \alst{Sk}ríði-at þat \alst{sk}ip, \hld\ es und þér \alst{sk}ríði, &
þótt \alst{ó}ska-byrr \hld\ \alst{e}ptir lęggisk! &
\alst{R}enni-a sá marr, \hld\ es und þér \alst{r}enni, &
þótt \alst{f}íęndr þína \hld\ \alst{f}orðask ęigir!\eva

\bvb May the ship not glide, which glides beneath thee, \\
though it has a wished-for gust behind it! \\
May the sea not run, which runs beneath thee, \\
though from thy foes thou must escape!\evb\evg


\bvg\bva \alst{B}íti-a þér þat sverð, \hld\ es þú \alst{b}ręgðir, &
nema \alst{s}jǫlfum þér \hld\ \alst{s}yngvi of hǫfði! &
Þá vę́ri þér \alst{h}ęfnt \hld\ \alst{H}ęlga dauða, &
ef þú \alst{v}ę́rir \alst{v}argr \hld\ á \alst{v}iðum úti, &
\alst{a}uðs \alst{a}nd-vani \hld\ ok \alst{a}lls gamans, &
\alst{h}ęfðir ęigi mat, \hld\ nema á \alst{h}rę́um spryngir!“\eva

\bvb May the sword not bite for thee, which thou brandishest, \\
save it sing over thy very own head! \\
\emph{Then} were on thee Hallow’s death avenged, \\
if thou wert a wolf in the woods outside, \\
deprived of wealth and all pleasure; \\
hadst no food, save thou plundered carrion!“\evb\evg


\bvg\bva\speakernote{Dagr kvað:}%
„\edtext{\alst{Ǿ}r ert, systir, \hld\ ok \alst{ø}r-vita}{\lemma{Ǿr \dots\ ok ør-viti ‘Mad \dots\ and out of wits’}\Bfootnote{Formulaic, also occurring in \Lokasenna\ and others TODO.}}, &
es \alst{b}rǿðr þínum \hld\ \alst{b}iðr for-skapa! &
\alst{Ęi}nn vęldr \alst{Ó}ðinn \hld\ \alst{ǫ}llu bǫlvi, &
því-at með \alst{s}ifjungum \hld\ \alst{s}ak-rúnar bar!\eva

\bvb\speakernoteb{Day quoth:}“Mad art thou, sister, and out of wits, \\
when onto thy brother thou dost bid a cruel \inx[C]{shape}. \\
Weden alone causes all the bale, \\
for he bore strife-runes among relatives!\evb\evg


\bvg\bva Þér \alst{b}ýðr \alst{b}róðir \hld\ \alst{b}auga rauða, &
ǫll \alst{V}andils-\alst{v}é \hld\ ok \alst{V}íg-dali; &
\alst{h}af \alst{h}alfan \alst{h}ęim \hld\ \alst{h}arms at gjǫldum &
\alst{b}rúðr \alst{b}aug-varið \hld\ ok \alst{b}úrir þínir.\eva

\bvb \emph{Thee} thy brother offers red bighs, \\
all Wendelswigh and the Wighdales. \\
Have half the realm as recompense for the injury, \\
O bigh-adorned bride—and thy sons, too.\evb\evg


\bvg\bva „\alst{S}it’k-a svá \alst{s}ę́l \hld\ at \alst{S}efa-fjǫllum, &
\alst{á}r né of nę́tr, \hld\ at ek \alst{u}na lífi, &
nema at \alst{l}iði \alst{l}ofðungs \hld\ \alst{l}jóma bręgði, &
renni und \alst{v}ísa \hld\ \alst{V}íg-blę́r þinig, &
\alst{g}ull-bitli vanr, \hld\ knega’k \alst{g}rami fagna!\eva

\bvb “I will not sit so happy in the Sevefells, \\
at dawn nor night, that I should be content with life, \\
unless the retinue of the man of praise were struck with light: \\
{[and]} beneath the ruler ran Wighblaw hither, \\
wont to the golden bit—{[and]} I might greet the prince!\evb\evg


\bvg\bva Svá \alst{h}afði \alst{H}ęlgi \hld\ \alst{h}rę́dda gǫrva &
\alst{f}jándr sína alla \hld\ ok \alst{f}rę́ndr þęira, &
sem fyr \alst{u}lfi \hld\ \alst{ó}ðar rynni &
\alst{g}ęitr af fjalli, \hld\ \alst{g}ęiska fullar!\eva

\bvb So would Hallow have terrified \\
his enemies all and their kinsmen, \\
like from a wolf did madly run \\
goats down a fell, full of fright.\evb\evg


\bvg\bva \edtext{Svá bar \alst{H}ęlgi \hld\ af \alst{h}ildingum &
sem \alst{í}tr-skapaðr \hld\ \alst{a}skr af þyrni &
eða sá \alst{d}ýr-kalfr \hld\ \alst{d}ǫggu slunginn &
es \alst{ø}fri fęrr \hld\ \alst{ǫ}llum dýrum, &
ok \alst{h}orn glóa \hld\ við \alst{h}imin sjalfan.“}{\lemma{ALL}\Bfootnote{Cf. the very similar description of Siward in \GudrunTwo\ 2.}}\eva

\bvb So did Hallow surpass the princes \\
like the nobly shaped ash the thorn, \\
or the deer-calf, dew-besprinkled, \\
who fares higher than all beasts, \\
and its horns gleam against heaven itself.”\evb\evg


\bpg\bpa Haugr var gǫrr eptir Helga.  En er hann kom til Valhallar, þá bauð Óðinn hánum ǫllu at ráða með sér.  Helgi kvað:\epa

\bpb A barrow was made for Hallow.  But when he came to Walhall Weden offered him to rule everything together with him.  Hallow quoth:\epb\epg


\bvg\bva „Þú skalt, \alst{H}undingr, \hld\ \alst{h}vęrjum manni &
\alst{f}ót-laug geta \hld\ ok \alst{f}una kynda; &
\alst{h}unda binda, \hld\ \alst{h}esta gę́ta, &
gefa \alst{s}vínum \alst{s}oð, \hld\ áðr \alst{s}ofa gangir!“\eva

\bvb “Thou shalt, Hunding, for every man \\
make a foot-bath and kindle the fire, \\
bind the hounds, feed the horses, \\
give broth to the swine—before thou mightst go to sleep!”\evb\evg


\bpg\bpa Ambótt Sigrúnar gekk um aptan hjá haugi Helga ok sá at Helgi reið til haugs’ins með marga menn. Ambótt kvað:\epa

\bpb Syerun’s maid-servant walked by Hallow’s barrow at evening, and saw that Hallow rode to the barrow with many men.  The maid-servant quoth:\epb\epg


\bvg\bva „Hvárt ’ru þat \alst{s}vik ęin \hld\ es \alst{s}éa þikkjumk &
eða \alst{r}agna \alst{r}ǫk \hld\ \alst{r}íða męnn dauðir, &
es \alst{jó}a \alst{y}ðra \hld\ \alst{o}ddum kęyrið, &
eða es \alst{h}ildingum \hld\ \alst{h}ęim-fǫr gefin?“\eva

\bvb “Either these are only tricks, as I seem to see \\
—or the \inx[L]{Rakes of the Reins}?—dead men riding; \\
as ye drive your steeds on by spear-points— \\
or are the princes granted leave to go home?”\evb\evg


\bvg\bva\speakernote{[Ęinn þęira kvað:]}%
„Es-a þat \alst{s}vik ęin \hld\ es \alst{s}éa þikkisk &
né \edtrans{\alst{a}ldar rof}{Ripping of the Age}{\Bfootnote{Formulaic.  Cf. TODO \emph{rjúfask ręgin}. This is the same root, only zero-grade.}} \hld\ þótt-u \alst{o}ss lítir, &
þótt vér \alst{jó}a \alst{ó}ra \hld\ \alst{o}ddum keyrim, &
né es \alst{h}ildingum \hld\ \alst{h}ęim-fǫr gefin.“\eva

\bvb\speakernoteb{[One of them quoth:]}%
“It is not only tricks, as thou seemest to see— \\
nor the Ripping of the Age, although thou behold us; \\
although we drive our steeds on by spear-points \\
the princes are not granted leave to go home.”\evb\evg


\bpg\bpa Heim gekk ambótt ok sagði Sigrúnu:\epa

\bpb The maid-servant walked home and said to Syerun:\epb\epg


\bvg\bva „Út gakk \alst{S}igrún, \hld\ frá \alst{S}ęfa-fjǫllum &
ef þik \alst{f}olks jaðarr \hld\ \alst{f}inna lystir; &
upp ’s \alst{h}augr lokinn, \hld\ kominn es \alst{H}ęlgi! &
\alst{D}ólg-spor \alst{d}ręyra \hld\ \alst{d}ǫglingr bað þik &
at þú \alst{s}ár-dropa \hld\ \alst{s}vęfja skyldir.“\eva

\bvb “Go out, O Syerun from the Sevefells, \\
if thou hast lust to find the leader of the troop! \\
The barrow is unlocked; Hallow is come! \\
The ruler of bloody wounds bade thee \\
that thou his wound-drops shouldst soothe.”\evb\evg


\bpg\bpa Sigrún gekk í haug’inn til Helga ok kvað:\epa

\bpb Syerun walked into Hallow’s barrow, and quoth:\epb\epg


\bvg\bva „Nú em’k svá \alst{f}ęgin \hld\ \alst{f}undi okkrum &
sem \alst{á}t-frękir \hld\ \alst{Ó}ðins haukar &
es \alst{v}al \alst{v}itu, \hld\ \alst{v}armar bráðir, &
eða \alst{d}ǫgg-litir \hld\ \alst{d}ags-brún séa.“\eva

\bvb “Now do I so rejoice at our meeting, \\
like do the ravenous hawks of Weden \ken{ravens} \\
when they know corpses, warm venison, \\
or, gleaming with dew, they see the day’s brow \ken{dawn}.\evb\evg


\bvg\bva Fyrr vil’k \alst{k}yssa \hld\ \alst{k}onung ó·lifðan &
an þú \alst{b}lóðugri \hld\ \alst{b}rynju kastir; &
\alst{h}ár ’s þitt, \alst{H}elgi, \hld\ \alst{h}élu þrungit, &
\edtrans{allr es \alst{v}ísi \hld\ \alst{v}al-dǫgg slęginn}{the prince is all with corpse-dew whipped}{\Bfootnote{Cf. \Baldrsdraumar\ 5, where the dead wallow says something similar.}}, &
\alst{h}ęndr úr-svalar \hld\ \alst{H}ǫgna mági; &
hvé skal’k þér, \alst{b}uðlungr, \hld\ þess \alst{b}ót of vinna?“\eva

\bvb Sooner would I kiss the unliving king, \\
than thou the bloody byrnie mightst cast away! \\
Thy hair is, O Hallow, with hoarfrost swollen; \\
the prince is all with corpse-dew \ken{blood} whipped; \\
the hands spray-cold on Hain’s in-law \ken*{= Hallow}.— \\
How shall I for thee, O noble, remedy that?”\evb\evg


\bvg\bva\speakernote{[Hęlgi kvað:]}„Ęin vęldr þú, \alst{S}igrún \hld\ frá \alst{S}efafjǫllum, &
es \alst{H}ęlgi es \hld\ \alst{h}arm-dǫgg slęginn: &
\alst{G}rę́tr þú, \alst{g}ull-varið, \hld\ \alst{g}rimmum tǫ́rum, &
\alst{s}ól-bjǫrt \alst{s}uð-rǿn, \hld\ áðr þú \alst{s}ofa gangir, &
hvęrt fęllr \alst{b}lóðugt \hld\ á \alst{b}rjóst grami, &
\alst{ú}r-svalt, \alst{i}nn-fjalgt \hld\ \alst{ę}kka þrungit.\eva

\bvb “Thou alone causest, O Syerun from the Sevefells, \\
that Hallow be with harm-dew whipped. \\
Thou weepest—O gold-covered—bitter tears— \\
O sun-bright southern lady—before thou go to sleep. \\
Each one falls bloody on the prince’s chest, \\
spray-cold, stifled, pressed forth by grief.\evb\evg


\bvg\bva Vęl skulum \alst{d}rekka \hld\ \alst{d}ýrar vęigar &
þótt \alst{m}isst hafim \hld\ \alst{m}unar ok landa! &
Skal \alst{ę}ngi maðr \hld\ \alst{a}ngr-ljóð kveða &
þótt mér á \alst{b}rjósti \hld\ \alst{b}ęnjar líti. &
Nú eru \edtext{\alst{b}rúðir \hld\ \alst{b}yrgðar í haugi, &
\alst{l}ofða dísir, \hld\ hjá oss}{\lemma{brúðir, dísir, oss ‘brides, dises, us’}\Bfootnote{Hallow speaks in the plural.  “Now has my bride, my goddess, come into the barrow, next to me, who am dead.”}} \alst{l}iðnum!“\eva

\bvb Well shall we drink dear draughts, \\
although we have lost both love and land! \\
Let no one sing songs of sorrow, \\
although he behold the wounds on my chest. \\
Now are the brides shut within the barrow, \\
the praised one’s \inx[C]{dise}[dises], next to us, passed-on.”\evb\evg


\bpg\bpa Sigrún bjó sę́ing í haug’inum.\epa

\bpb Syerun made a bed in the barrow:\epb\epg


\bvg\bva „\alst{H}ér hęfi’k þér, \alst{H}ęlgi, \hld\ \alst{h}vílu gørva, &
\alst{a}ngr-lausa mjǫk, \hld\ \alst{Y}lfinga niðr; &
vil’k þér í \alst{f}aðmi, \hld\ \alst{f}ylkir, sofna &
\edtrans{sem’k \alst{l}ofðungi \hld\ \alst{l}ifnum mynda’k!}{like I would with the living man of praise}{\Bfootnote{i.e. “just as I would if you were still alive.”}}“\eva

\bvb “Here I’ve for thee, Hallow, made a place of rest, \\
all without sorrow, O kinsman of the Wolvings! \\
I will in thy arms, O marshal, fall asleep, \\
like I would with the living man of praise.”\evb\evg


\bvg\bva\speakernote{[Hęlgi kvað:]}„Nú kveð’k \alst{ę}nskis \hld\ \alst{ø}r-vę́nt vesa, &
\alst{s}íð né \alst{s}nimma, \hld\ at \alst{S}efa-fjǫllum &
es þú á \alst{a}rmi \hld\ \alst{ó}·lifðum søfr, &
\alst{h}vít, í \alst{h}augi, \hld\ \alst{H}ǫgna dóttir, &
ok est-u \alst{k}vik, \hld\ in \alst{k}onung-borna!“\eva

\bvb\speakernoteb{[Hallow quoth:]}%
“Now, I say, there is naught more missing \\
neither late nor soon from the Sevefells, \\
when thou dost sleep on the unliving arm, \\
O white daughter of Hain—in the barrow, \\
and thou art alive!—of kingly birth.”\evb\evg

\sectionline

{\small (The night has passed; dawn is breaking, and Hallow speaks.  The manuscript does not indicate the change of scene.)}

\sectionline

\bvg\bva\speakernote{[Hęlgi kvað:]}„Mál ’s mér at \alst{r}íða \hld\ \edtrans{\alst{r}oðnar}{reddening}{\Bfootnote{From the rising dawn.}} brautir, &
láta \alst{f}ǫlvan jó \hld\ \alst{f}lug-stíg troða; &
skal’k fyr \alst{v}estan \hld\ \alst{v}ind-hjalms brúar &
áðr \alst{S}al-gofnir \hld\ \alst{s}igr-þjóð vęki.“\eva

\bvb “’Tis time for me to ride the reddening roads, \\
to let my pale steed tread the path of flight \ken{sky/heaven}. \\
I shall go west of the wind-helm’s bridges \ken{sky/heaven > clouds?}, \\
before Salgovner may awaken the victorious folk.”\evb\evg


\bpg\bpa Þęir Hęlgi riðu lęið sína, en þę́r fóru hęim til bǿjar. Annan aptan lét Sigrún ambótt halda vǫrð á haugi’num.  En at dag-setri, es Sigrún kom til haugs’ins, hón kvað:\epa

\bpb Hallow and his men rode on their way, but the women journeyed home to the farm. The next evening Syerun made her maid-servant keep watch on the barrow.  And at sunset as Syerun came to the barrow, she \ken*{= the maid-servant} quoth:\epb\epg


\bvg\bva „\alst{K}ominn vę́ri nú, \hld\ ef \alst{k}oma hygði, &
\alst{S}igmundar burr \hld\ frá \alst{s}ǫlum Óðins; &
kveð’k \alst{g}rams þinig \hld\ \alst{g}rę́nask vánir &
\edtrans{es á \alst{a}sk-limum \hld\ \alst{ę}rnir sitja}{when on ashen branches eagles sit}{\Bfootnote{i.e. “when the eagles roost on yonder trees”.  This is a sign of Hallow and his men not coming; if they were, the eagles would be following them and picking at their bodies.}} &
ok \edtext{\alst{d}rífr \alst{d}rótt ǫll \hld\ \alst{d}raum-þinga til}{\lemma{drífr \dots\ draum-þinga til ‘drifts off to dream-Things’}\Bfootnote{i.e. “falls asleep”.  A fine metaphor.}}.“\eva

\bvb “Come were now, if to come he had thought, \\
Syemund’s son \ken*{= Hallow} from Weden’s halls; \\
hopes fade, I say, of the prince’s coming, \\
when on ashen branches eagles sit, \\
and all mankind drifts off to dream-\inx[C]{Thing}[Things].\evb\evg


\bvg\bva Ves \alst{ęi}gi svá \alst{ǿ}r \hld\ at \alst{ęi}n farir, &
\alst{d}ís skjǫldunga, \hld\ \alst{d}raug-húsa til! &
Verða \alst{ǫ}flgari \hld\ \alst{a}llir á nǫ́ttum &
\alst{d}auðir \alst{d}ólgar, mę́r, \hld\ an of \alst{d}aga ljósa.“\eva

\bvb Be not so mad that thou journey alone, \\
O dise of the Shieldings, to the ghost-houses! \\
Mightier at night do all become \\
dead fiends, O maiden, than during the bright days!”\evb\evg


\bpg\bpa Sigrún varð skamm-líf af harmi ok trega. Þat var trúa í forneskju, at menn vę́ri endr-bornir, en þat er nú kǫlluð kerlinga-villa.  Helgi ok Sigrún er kallat at vę́ri endr-borin.  Hét hann þá Helgi Haddingjaskati en hon Kára Hálfdanar dóttir, svá sem kveðit er í \edtrans{Káruljóðum}{Leeds of Cheer}{\Bfootnote{A now-lost heroic poem.}}, ok var hon val-kyrja.\epa

\bpb Syerun became short-lived for pain and grief.  It was the belief in olden times that men were born again, but that is now called an old wives’ tale.  Of Hallow and Syerun it is said that they were born again.  He was then called Hallow Hardingskate and she Cheer Halfdanesdaughter, as is told in the Leeds of Cheer, and she was a walkirrie.\epb\epg

\sectionline
%
%	\bookStart{Spae of Griper}[Grípisspǫ́]

\begin{flushright}%
\textbf{Dating} \parencite{Sapp2022}: early C11th (0.616)–late C11th (0.313).

\textbf{Meter:} \Fornyrdislag%
\end{flushright}

\section{Introduction}

TODO: Introduction.  This poem is uniquely regular and well preserved; every single one of its 53 \Fornyrdislag\ stanzas all is four lines long.

The title is “From Sinfittle’s death”.

\section{From the Death of Sinfittle (\emph{Frá dauða Sinfjǫtla})}

\sectionline

\bpg\bpa Sigmundr Vǫlsungs sonr var konungr á Frakklandi. Sinfjǫtli var elztr hans sona, annarr Helgi, þriði Hámundr. Borghildr, kona Sigmundar, átti bróður er hét... en Sinfjǫtli, stjúp-sonr hennar, ok... báðu einnar konu báðir ok fyr þá sǫk drap Sinfjǫtli hann. En er hann kom heim þá bað Borghildr hann fara á brot en Sigmundr bauð henni fé-bǿtr ok þat varð hón at þiggja. En at erfi’nu bar Borghildr ǫl. Hon tók eitr mikit, horn fullt, ok bar Sinfjǫtla.  En er hann sá í horn’it skilði hann at eitr var í ok mę́lti til Sigmundar: „Gjǫr-óttr er drykkr’inn, ái!“  Sigmundr tók horn’it ok drakk af.  Svá er sagt at Sigmundr var harð-gǫrr at hvárki mátti hánum eitr granda útan né innan.  En allir synir hans stóðusk eitr á hǫrund útan.  Borghildr bar annat horn Sinfjǫtla ok bað drekka ok fór allt sem fyrr.  Ok enn it þriðja sinn bar hon hánum horn’it ok þó á-mę́lis-orð með ef hann drykki eigi af.  Hann mę́lti enn sem fyrr við Sigmund; hann sagði: „Láttu grǫn sía þá, sonr!“  Sinfjǫtli drakk ok varð þegar dauðr.  Sigmundr bar hann langar leiðir í fangi sér ok kom at firði einum mjóvum ok lǫngum ok var þar skip eitt lítit ok maðr einn á.  Hann bauð Sigmundi far of fjǫrð’inn.  En er Sigmundr bar lík’it út á skip’it þá var bátr’inn hlaðinn.  Karl mę́lti at Sigmundr skyldi fara fyr inn á fjǫrð’inn.  Karl hratt út skip’inu ok hvarf þegar.  Sigmundr konungr dvalðisk lengi í Danmǫrk í ríki Borghildar síðan er hann fekk hennar.  Fór Sigmundr þá suðr í Frakkland til þess ríkis er hann átti þar.  Þá fekk hann Hjǫrdísar, dóttur Eylima konungs.  Þeira sonr var Sigurðr.  Sigmundr konungr fell í orrustu fyr Hundings sonum.  En Hjǫrdís giptisk þá Álfi, syni Hjálpreks konungs.  Óx Sigurðr þar upp í barn-ǿsku.  Sigmundr ok allir synir hans vóru langt um fram alla menn aðra um afl ok vǫxt ok hug ok alla at-gørvi.  Sigurðr var þá allra framarstr ok hann kalla allir menn í forn-frǿðum um alla menn fram ok gǫfgastan her-konunga.\epa

\bpb TODO.\epb\epg


\bpg\bpa Grípir hét sonr Ęylima, bróðir Hjǫrdísar.  Hann réð lǫndum ok vas allra manna vitrastr ok fram-víss.  Sigurðr ręið ęinn saman ok kom til hallar Grípis.  Sigurðr vas auð-kęnndr.  Hann hitti mann at máli úti fyr hǫll’inni; sá nęfndisk Gęitir.  Þá kvaddi Sigurðr hann máls, ok spyrr:\epa

\bpb Griper was called the son of Ilime, Hardise’s brother.  He ruled lands and was wisest of all men, and forthwise.  Siward rode alone and came to Griper’s hall.  Siward was easily recognized.  He approached a man for speech outside of the hall; he was named Goater.  Then Siward greeted him with a speech, and asks:\epb\epg

\section{The Spae of Griper}

\bvg\bva „Hvęrr \alst{b}yggir hér \hld\ \alst{b}orgir þessar? &
Hvat þann \alst{þ}jóð-konung \hld\ \alst{þ}egnar nefna?“ &
„\alst{G}rípir hęitir \hld\ \alst{g}umna stjóri, &
sá’s \alst{f}astri rę́ðr \hld\ \alst{f}oldu ok þegnum.“\eva

\bvb “Who bedwells here these forts? \\
What is this great king called by thanes?” \\
“Griper is called the steerer of men \\
who rules the steadfast land and thanes.”\evb\evg


\bvg\bva \alst{M}ę́la nǫ́mu \hld\ ok \alst{m}argt hjala &
þá’s \alst{r}áð-spakir \hld\ \alst{r}ekkar fundusk. &
„Sęg-ðu \alst{m}ér ef þú vęizt, \hld\ \alst{m}óður-bróðir, &
hvé mun \alst{S}igurði \hld\ \alst{s}núna ę́vi?“\eva

\bvb They took to speak and chatter much, \\
when the council-wise champions found each other. \\
“Tell me, if thou knowest, O mother’s brother: \\
how will Siward’s age turn out?”\evb\evg


\bvg\bva „Þú \alst{m}unt \alst{m}aðr vesa \hld\ \alst{m}ę́ztr und sólu &
ok \alst{h}ę́str borinn \hld\ \alst{h}vęrjum jǫfri; &
\alst{g}jǫfull af \alst{g}ulli \hld\ en \alst{g}løggr flugar, &
\alst{í}tr á-liti \hld\ ok í \alst{o}rðum spakr.“\eva

\bvb „Thou wilt be a man noblest neath the sun, \\
and borne higher than every ruler, \\
giving with gold but stingy of flight, \\
radiant of hue and wise in words.“\evb\evg

TODO.

\bvg\bva Es-a með \alst{l}ǫstum \hld\ \alst{l}ǫgð ę́vi þér; &
lát-tu, inn \alst{í}tri, \hld\ þat, \alst{ǫ}ðlingr, nemask &
því at \alst{u}ppi mun \hld\ meðan \alst{ǫ}ld lifir, &
\alst{n}add-éls boði, \hld\ \alst{n}afn þitt vera.\eva

\bvb TODO. \\
For remembered will while mankind lives, \\
O beseecher of the sword-storm \ken{battle > warrior}, thy name be.\evb\evg

TODO.

\bvg\bva Þú munt \alst{h}víla, \hld\ \alst{h}ęrs odd-viti, &
\alst{m}ę́rr hjá \alst{m}ęyju \hld\ sem þín \alst{m}óðir sé; &
því mun \alst{u}ppi \hld\ meðan \alst{ǫ}ld lifir, &
\alst{þ}jóðar \alst{þ}ęngill, \hld\ \alst{þ}itt nafn vera.\eva

\bvb Thou wilt rest, O point-knower of the host \ken{warrior}, \\
renowned beside a maiden like she were thy mother. \\
For that will remembered while mankind lives, \\
O prince of the nation, thy name be.\evb\evg

TODO.

\bvg\bva Því skal \alst{h}ugga þik, \hld\ \alst{h}ęrs odd-viti, &
sú mun \alst{g}ipt lagit \hld\ á \alst{g}rams ę́vi; &
mun-at \alst{m}ę́tri \alst{m}aðr \hld\ á \alst{m}old koma &
und \alst{s}ólar \alst{s}jǫt \hld\ an, \alst{S}igurðr, þikkir.\eva

\bvb For that [she] shall soothe thee, O point-knower of the host; \\%TODO: "soothe"??
she will have laid venom in the ruler’s age. \\
No nobler man will come onto the earth \\
neath the sun’s seat \ken{sky/heaven}, than thou, Siward, seemest!\evb\evg


\bvg\bva \alst{Sk}iljumk hęilir; \hld\ mun-at \alst{sk}ǫpum vinna! &
Nú hęfir þú, Grípir, vęl \hld\ gørt sem bęiddak; &
fljótt myndir þú \hld\ fríðri sęgja &
mína ę́vi \hld\ ef þú mę́ttir þat!\eva

\bvb Let us part healthy; one will not withstand the \inx[C]{shape}[shapes]! \\
Now hast thou, Griper, well done as I asked; \\
shortly wouldst thou fairer speak \\
of my age, if thou couldst do that!\evb\evg

\sectionline
%
	\bookStart{The Speeches of Rein}[Ręginsmǫ́l]

\begin{flushright}%
Dating \parencite{Sapp2022}: C10th (0.666)–early C11th (0.259)

Meter: \Ljodahattr, \Fornyrdislag%
\end{flushright}

Like other poems from this section, it is better defined as a prosimetrum. The differing meter of the verses might suggest that they are taken from different poems.

\sectionline

\bpg\bpa Sigurðr gekk til stóðs Hjálpreks ok kaus sér af hest einn er Grani var kallaðr síðan. Þá var kominn Reginn til Hjálpreks, sonr Hreiðmars. Hann var hverjum manni hagari ok dvergr of vöxt. Hann var vitr, grimmr ok fjölkunnigr. Reginn veitti Sigurði fóstr ok kennzlu ok elskaði hann mjök. Hann sagði Sigurði frá forellri sínu ok þeim atburðum at Óðinn ok Hænir ok Loki höfðu komið til Andvarafors; í þeim forsi var fjölði fiska. Einn dvergr hét Andvari; hann var löngum í forsinum í geddu líki ok fekk sér þar matar. „Otr hét bróðir várr,“ kvað Reginn, „er oft fór í forsinn í otrs líki. Hann hafði tekið einn lax ok sat á árbakkanum ok át blundandi. Loki laust hann með steini til bana. Þóttuz æsir mjök heppnir verið hafa ok flógu belg af otrinum. Þat sama kveld sóttu þeir gisting til Hreiðmars ok sýndu veiði sína. Þá tóku vér þá höndum ok lögðum þeim fjörlausn at fylla otrbelginn með gulli ok hylja útan ok með rauðu gulli. Þá sendu þeir Loka at afla gullzins. Hann kom til Ránar ok fekk net hennar ok fór þá til Andvarafors ok kastaði netinu fyr gedduna en hon hljóp í netið. Þá mælti Loki:\epa

\bpb Siward went to Helpric’s stable and chose one horse, which was thereafter called Grane. Then Rein, son of Rethmar, was come to Helpric. He was more skilled than any man and a dwarf in stature. He was wise, cruel and feel-cunning. Rein fostered and taught Siward and love him very much. He told Siward about his own parents, and about the events that Weden, Heener and Lock had come to Andwareforce; in that force was a multitude of fish. A dwarf was named Andware; he was for a long time in the force in the likeness of a pike and got his food there. “Otter was our brother called,” said Rein, “who often journeyed in the force in the likeness of an otter. He had caught a salmon and sat on the riverbank and ate it with closed eyes Lock struck him with a stone unto his death. The Ease thought themselves to have been very lucky, and flayed the skin off the otter. The same evening they sought to pass the night at Rethmare’s house, and showed their catch. Then we bound them and proposed to them as a life-ransom that they would fill the otter-skin with gold, and also coat the outside with red gold. Then they sent Lock to get ahold of the gold. He came to Ran and got her net and then journeyed to Andwareforce and threw the net before the pike, and it jumped into the net. Then Lock spoke:”\epb\epg


TODO


\bvg
\bva Kęmbðr ok þvęginn \hld\ skal kǿnna hvęrr &
\ind ok at morni męttr. &
Því’t ósýnt es \hld\  hvar at aptni kømr; &
\ind illt ’s fyr hęill at hrapa.\eva

\bvb Combed and washed shall each keen man be, and well fed in morning,—for unknown it is where he will come in the evening; ’tis bad to run before one’s luck.\footnoteB{The language of the first half of this stanza is very close to \Havamal\ 61 and \Voluspa\ 33; for discussion on personal hygiene and bathing see note to the former.}\evb
\evg
%
	\bookStart{The Speeches of Fathomer}[Fáfnismǫ́l]

\begin{flushright}%
Dating \parencite{Sapp2022}: C10th (0.442), early C11th (0.402), late C11th (0.155)

Meter: \Ljodahattr\ (TODO)%
\end{flushright}

\sectionline

\bvg
\bva „Svęinn ok svęinn! \hld\ Hvęrjum estu svęini of borinn? &
\ind Hvęrra estu manna mǫgr? &
es þú á Fáfni rautt \hld\ þínn hinn frána mę́ki; &
\ind stǫndumk til hjarta hjǫrr!“\eva

\bvb {[Fathomer quoth:]} \\
“O swain and swain! To which swain art thou born; \\
of which men art thou son? \\
As thou on Fathomer hast reddened thy gleaming blade, \\
the sword stands unto my heart!”\evb
\evg


\bpg\bpa Sigurðr dulði nafns síns fyr því at þat var trúa þeira í forneskju at orð feigs manns mę́tti mikit ef hann bǫlvaði óvin sínum með nafni. Hann kvað:\epa

\bpb Siward concealed his name, because it was their belief in ancient times that the word of a \inx[C]{fey} man could do much if he baled his enemy by his name. He \ken*{= Siward} quoth:\epb\epg


\bvg
\bva „Gǫfugt dýr ek hęiti \hld\ en ek gęngit hef’k &
\ind hinn móður-lausi mǫgr, &
fǫður ek á’kk-a \hld\ sem fira synir, &
\ind gęng ek ęinn saman.“\eva

\bvb “Noble beast I am called, but gone have I, \\
the motherless lad. \\
A father I have not, like do the sons of men; \\
I go all alone.”\evb
\evg


\bvg
\bva „Vęitst, ef fǫður né átt-at \hld\ sem fira synir, &
\ind af hvęrju vastu undri alinn?
[...]“\eva

\bvb {[Fathomer quoth:]} \\
“Knowest thou, if thou hast not a father, like do the sons of men, \\
by which wonder thou wast begotten?”\evb
\evg


\bvg
\bva „Ę́tterni mitt \hld\ kveð’k þér ó·kunnigt vesa &
\ind ok mik sjalfan hit sama: &
Sigurðr ek hęiti \hld\ Sigmundr hét minn faðir &
\ind es hęf’k þik vápnum vegit.“\eva

\bvb {[Siward quoth:]} \\
“My lineage I declare is unknown to thee, \\
and my self the same.\footnoteB{The meaning is that Fathomer would not recognize Siward’s lineage (i.e. his father) or name, since he is an orphan who up until this point has not won any glory. He is not saying that he is lineage is unknown even to himself, since \emph{sjalfan mik} ‘my self’ is accusative, not dative.} \\
Siward am I called—Syemund was called my father— \\
who with weapons have struck thee.”\evb
\evg


\bvg
\bva „Hvęrr þik hvatti, \hld\ hví hvętjask lést, &
\ind mínu fjǫrvi at fara? &
Hinn frán-ęygi svęinn, \hld\ þú áttir fǫður bitran, &
\ind á-bornu skjór á skęið.“\eva

\bvb {[Fathomer quoth:]} \\
“Who goaded thee—why didst thou let thyself be goaded— \\
my life for to destroy? \\
O gleaming-eyed swain, thou haddest a sharp father; \\
inborn traits show quickly.\footnoteB{The original is unclear. \emph{á skęið} means roughly ‘rapidly, quickly’, whence the expression \emph{ríða á skęið} ‘\CV: to ride at full speed’, but the other words are uncertain. \textcite{LaFargeGlossary} read ‘your innate qualities show quickly’, suggesting two unattested words: an adjective \emph{*áborinn} ‘innate, inborn’ and a verb \emph{*skjóa} ‘to show’. Yet the lack of i-umlaut in the supposed 3rd sg. pres. ind. \emph{skjór} is difficult. We would expect \emph{**skýr}, as in \emph{skjóta} ‘to shoot,’ with 2nd/3rd sg. pres. ind \emph{skýtr}. A solution here would be reading a 2nd sg. pres. subj. \emph{skjóir}, with a vowel TODO}”\evb
\evg


\bvg
\bva „Hugr mik hvatti, \hld\ hendr mér full-týðu &
\ind ok minn inn hvassi hjǫrr; &
fár es hvatr \hld\ es hrøðask tękr &
\ind ef í barnǿsku ’s blauðr.“\eva

\bvb {[Siward quoth:]} \\
“My heart goaded me, my hands assisted me, \\
and this my sharp sword. \\
Few a man is brave when he takes to grow, \\
if in his youth he be soft.”\evb
\evg


\bvg
\bva „Vęit’k, ef þú vaxa nę́ðir \hld\ fyr þinna vina brjósti, &
\ind séi-t maðr þik vręiðan vega; &
nú ert haptr \hld\ ok hęr-numinn, &
\ind ę́ kveða bandingja bifask.“\eva

\bvb {[Fathomer quoth:]} \\
“TRANSLATION”\evb
\evg


\bvg
\bva „Því bregðr þú nú mér, Fáfnir, \hld\ at til fjarri sjá’k &
\ind mínum fęðr-munum, &
ęigi em’k haptr \hld\ þótt vę́ra hęr-numi; &
\ind þú fannt, at ek lauss lifi!“\eva

\bvb {[Siward quoth:]} \\
“TRANSLATION”\evb
\evg


\bvg
\bva „Hęipt-yrði ęin \hld\ tęlr þú þér í hví-vętna &
\ind en ek þér satt ęitt sęgi’k: &
It gjalla gull \hld\ ok it glóð-rauða fé, &
\ind þér verða þęir baugar at bana!“\eva

\bvb {[Fathomer quoth:]} \\
“With hateful words alone answerest thou anything, \\
but I tell thee truth alone: \\
The resounding gold and the glowing red wealth, \\
those bighs will become thy bane!”\evb
\evg


\bvg
\bva „Féi ráða \hld\ skal fyrða hvęrr &
\ind ę́ til ins ęina dags &
því-at ęinu sinni \hld\ skal alda hvęrr &
\ind fara til hęljar heðan.“\eva

\bvb {[Siward quoth:]} \\
“Rule [his] fee shall every man, \\
always, until the one day; \\
for at one time must every man \\
journey hence to Hell.\footnoteB{Siward dismisses the idea of the curse. He must die regardless of whether he takes the gold or not, and he would rather die wealthy and famous than poor and unknown.}”\evb
\evg


\bvg
\bva „Norna dóm \hld\ munt \edtrans{fyr nęsjum}{before the headlands}{\Bfootnote{Formulaic, the sense is that the doom of the norns is close at hand (TODO: How do other scholars explain this?). Cf. the last st. of Sonatorrek (TODO).}} hafa &
\ind ok ó·svinns apa; &
í vatni þú drukknar \hld\ ef í vindi rę́r; &
\ind allt es fęigs forað.“\eva

\bvb {[Fathomer quoth:]} \\
“The doom of the Norns shalt thou have before the headlands, \\
and that of an unwise ape. \\
In water [wilt] thou drown if thou row in wind; \\
everything is the pit of the \inx[C]{fey}.\footnoteB{That is, the cursed, death-doomed (fey) man will find sudden death no matter where he turns.}”\evb
\evg


\bvg
\bva „Sęg mér, Fáfnir, \hld\ alls þik fróðan kveða &
\ind ok vęl mart vita: &
Hvęrjar ’ru þę́r nornir \hld\ es nauð-gǫnglar ’ru &
\ind ok kjósa mǿðr frá mǫgum?“\eva

\bvb {[Siward quoth:]} \\
“Say to me, Fathomer, as they call thee wise, \\
and knowing well enough: \\
Which are those Norns who are need-going, \\
and choose mothers from their lads?”\evb
\evg


\bvg
\bva „Sundr-bornar mjǫk \hld\ hygg at nornir sé, &
\ind ęigu-t þę́r ę́tt saman; &
sumar ’ru ás-kunngar, \hld\ sumar alf-kunngar, &
\ind sumar dǿtr Dvalins.“\eva

\bvb {[Fathomer quoth:]} \\
“Of much sunry birth I judge the norns to be; \\
they come not from a common lineage: \\
Some are begotten of the Eese, some begotten of the Elves, \\
some are the daughters of Dwollen \ken{dwarfs}.”\evb
\evg


\bvg
\bva „Sęg mér þat, Fáfnir, \hld\ alls þik fróðan kveða &
\ind ok vęl margt vita, &
hvé sá holmr hęitir \hld\ es blanda hjǫr-lęgi &
\ind Surtr ok ę́sir saman.“\eva

\bvb {[Siward quoth:]} \\
“Say to me, Fathomer, as they call thee wise, \\
and knowing well enough: \\
What is the islet called, where Surt and the Eese \\
blend sword-water \ken{blood} together?”\evb
\evg


\bvg
\bva „Ó·skópnir hęitir \hld\ en þar ǫll skulu &
\ind gęirum lęika goð; &
Bil-rǫst brotnar \hld\ es á brott fara &
\ind ok svima í móðu marir.\eva

\bvb {[Fathomer quoth:]} \\
“Unshopner it is called, and there shall all \\
the Gods play with spears; \\
Bilrest shatters when they fare away, \\
and the horses swim in the sea.\evb
\evg

\sectionline

Fathomer continues speaking, but there is probably something missing here, since the transition is abrupt. Between its paraphrases of st. 15 and of st. 16, \VolsungaMS\ has \emph{Ok enn mę́lti Fáfnir: „Reginn bróðir minn veldr mínum dauða, ok þat hlę́gir mik, er hann veldr ok þínum dauða, ok ferr þá, sem hann vildi.“} ‘And further spoke Fathomer: “My brother Rein causes my death, and it gladdens me that he also causes thy death, and then it will go like he has willed.”’, which may either be a paraphrase of a lost st., or an addition by the redactor.

\sectionline

\bvg
\bva Ǿgis hjalm \hld\ bar’k of alda sonum &
\ind meðan of męnjum lá’k; &
ęinn rammari \hld\ hugðumk ǫllum vesa, &
\ind fann’k-a’k marga mǫgu.“\eva

\bvb A helmet of terror I carried over the sons of men \\
while on the rings I lay; \\
stronger than all I thought myself alone to be; \\
I did not find many men.”\evb
\evg


\bvg %NOTE: Heavily formulaic.
\bva „Ǿgis hjalmr \hld\ bergr ęinu-gi &
\ind hvar’s skulu vręiðir vega; &
þá þat finnr \hld\ es með flęirum kømr &
\ind at ęngi es ęinna hvatastr.“\eva

\bvb {[Siward quoth:]} \\
“A helmet of terror saves no man, \\
whereever wroth men should fight; \\
then he finds, when among the many he comes, \\
that none is the boldest of all.”\evb
\evg


\bvg
\bva „Ęitri ek fnę́sta \hld\ es á arfi lá’k &
\ind miklum míns fǫður.“\eva

\bvb {[Fathomer quoth:]} \\
“Venom I blew, while I lay on the great \\
inheritance of my father.”\evb
\evg


\bvg
\bva „Inn rammi ormr, \hld\ þú gørðir frę́s mikla &
\ind ok gatst harðan hug; &
\ind hęipt at męiri \hld\ verðr hǫlða sonum &
\ind at þann hjalm hafi.“\eva

\bvb {[Siward quoth:]} \\
“O mighty wyrm, thou madest a great snort, \\
and wonnest a hard heart; \\
TODO.”\evb
\evg


\bvg
\bva „Rę́ð’k þér nú, Sigurðr, \hld\ en þú ráð nemir &
\ind ok ríð hęim heðan; &
it gjalla gull \hld\ ok it glóð-rauða fé, &
\ind þér verða þęir baugar at bana!“\eva

\bvb {[Fathomer quoth:]} \\
“I counsel thee now, O Siward—and thou oughtst to take the counsel, \\
and ride home, hence! \\
The resounding gold and the glowing red wealth, \\
those bighs will become thy bane!”\evb
\evg


\bvg
\bva „Ráð ’s þér ráðit \hld\ en ek ríða mun &
\ind til þęss gulls es í lyngvi liggr, &
en þú, Fáfnir, ligg \hld\ í fjǫr-brotum &
\ind \edtrans{þar’s þik Hęl hafi}{where Hell may have thee}{\Bfootnote{Formulaic. TODO.}}!“\eva

\bvb {[Siward quoth:]} \\
“Thy counsel has been counseled—but I will ride, \\
to the gold which in the heather lies; \\
but \emph{thou}, Fathomer, lie in the blood-tracks, \\
where Hell may have thee!”\evb
\evg


\bvg% NOTE: Pun.
\bva „Ręginn mik réð, \hld\ hann þik ráða mun, &
\ind hann mun okkr verða bǫ́ðum at bana; &
fjǫr sitt láta \hld\ hygg at Fáfnir myni; &
\ind þitt varð nú męira męgin.“\eva

\bvb {[Fathomer quoth:]} \\
“Rein betrayed \emph{me}, he will betray \emph{thee}; \\
he will become the bane of us both; \\
give his life, I judge that Fathomer will; \\
thy strength was now the greater.”\evb
\evg


\bpg
\bpa Reginn var á brott horfinn meðan Sigurðr vá Fáfni ok kom þá aptr er Sigurðr strauk blóð af sverðinu. Reginn kvað:\epa

\bpb Rein had gone away while Siward smote Fathomer, and then came back as Siward wiped the blood off the sword. Rein quoth:\epb
\epg


\bvg
\bva „Hęill þú nú, Sigurðr, \hld\ nú hęfir sigr vegit &
\ind ok Fáfni of farit; &
manna þęira \hld\ es mold troða &
\ind þik kveð’k ó·blauðastan alinn.“\eva

\bvb {[SPEAKER quoth:]} \\
“Hail thee now, O Siward—now thou hast won victory \\
and Fathomer destroyed! \\
Of those men who tread on the earth \\
I declare \emph{thee} with least softness begotten.”\evb
\evg


\bvg
\bva „VERSE“\eva

\bvb {[SPEAKER quoth:]} \\
“TRANSLATION”\evb
\evg
%
	\bookStart{The Speeches of Syedrive}[Sigrdrífumǫ́l]

\begin{flushright}%
Dating \parencite{Sapp2022}: C10th (0.961)

\textbf{Meter: }\Ljodahattr%
\end{flushright}

% Introduction

% Preservation

\Sigrdrifumal\ is attested in two medieval mss., namely \Regius\ (which is the main mss. for the pres. ed) and \VolsungaMS\ (\VolsungaSaga\ ch. 21), which begins with a paraphrase of the present poem up to P2:

\begin{quote}
  \emph{Brynhildr segir, at tveir konungar bǫrðust. Hét annarr Hjalmgunnarr; hann var gamall ok hinn mesti hermaðr, ok hafði Óðinn honum sigr heitit, en annarr Agnarr eða Auða bróðir. „Ek fellda Hjalmgunnarr í orrostu, en Óðinn stakk mik svefn-þorni í hefnd þess ok kvað mik aldri síðan skyldu sigr hafa ok kvað mik giptast skulu. En ek strengða þess heit þar í mót at giptast engum þeim, er hrę́ðast kynni.“ Sigurðr mę́lti: „Kenn oss ráð til stórra hluta.“ Hun svarar: „Þér munuð betr kunna, en með þǫkkum vil ek kenna yðr, ef þat er nǫkkut, er vér kunnum, þat er yðr mę́tti líka, í rúnum eða ǫðrum hlutum, er liggja til hvers hlutar, ok drekkum bę́ði saman, ok gefi goðin okkr góðan dag, at þér verði nýt ok fregð at mínum vitrleik, ok þú munir eptir þat, er vit réðum.“ Brynhildr fylldi eitt ker ok fę́rði Sigurði ok mę́lti:}

  ‘Byrnhild says that two kings fought. One was called Helmguther; he was old and the greatest warrior, and Weden had promised him victory,
  but the other was called Eyner or Eade’s brother. “I felled Helmguther in battle, but Weden stung me with a sleeping-thorn as revenge for that, and declared that I should never thenceforth have victory, and said that I must marry, but I made a vow in response, to marry no man who could be frightened.” Siward spoke: “Teach us counsels regarding great things.” She answers: “Ye will know better, but with thanks I will teach you, if there is anything which we know that may please you, of runes or other things of importance; and let us both drink together, and may the gods give us two a good day, that thou have use and joy from my wisdom and that thou afterwards recall that which we two speak of.” Byrnhild filled a vessel and brought it to Siward and spoke:’
\end{quote}

After this it cites sts. 4–12 and 14–18 in uninterrupted sequence, and paraphrases sts. 19 ff. (TODO: edit these!).  The order of stanzas in \VolsungaMS\ is not identical to \Regius.  Both mss. have sts. 4–5 and 12, 14–18 in the same place, but the order of sts. 6–11 in between is divergent, as seen by the following table:

\begin{longtabu} to \textwidth {|c c c c|}
	\hline
	\multicolumn{2}{|c}{\emph{pres. ed.}} & \Regius & \VolsungaMS \\ [0.5ex]
	\hline\hline\endhead
	\hline\endfoot
	4 & Bjór fǿri’k þér & 4 & 6 \\
	5 & Sig-rúnar skalt rísta & 5 & 7 \\
  6 & Ǫl-rúnar skalt kunna & 6 & 10 \\
  7 & Full skal signa & 6* & 11 \\
  8 & Bjarg-rúnar skalt kunna & 7 & 12 \\
  9 & Brim-rúnar skalt rísta & 8 & 8 \\
  10 & Lim-rúnar skalt kunna & 9 & 13 \\
  11 & Mál-rúnar skalt kunna & 10 & 9 \\
  12 & Hug-rúnar skalt kunna & 11a & 14 \\
  13 & Á bjargi stóð & 11b–12 & − \\
  14 & Á skildi kvað ristnar & 13–14a & 15–17 \\
  15 & Allar vǫ́ru af skafnar & 14b–15 & 18 \\
  16 & Þat eru bókrúnar & 16 & 19 \\
  17 & Nú skalt kjósa & 17 & 20 \\
  18 & Mun’k-a ek flǿja & 18 & 21 \\ [1ex]
	\hline
\end{longtabu}

% Contents

The contents of the poem

\sectionline

\bvg\bva „Lęngi ek \alst{s}vaf, \hld\ lęngi ek \alst{s}ofnuð vas, &
\ind \alst{l}ǫng eru \alst{l}ýða \alst{l}ę́; &
\alst{Ó}ðinn því vęldr \hld\ es \alst{ęi}gi mátta’k &
\ind \alst{b}regða \alst{b}lund-stǫfum.“\eva

\bvb {[Syedrive quoth:]} “Long I slept, long was I asleep, \\
long are the guiles of men. \\
Weden doth cause that I could not \\
break the sleeping-staves.”\evb\evg


\bpg\bpa Sigurðr sęttisk niðr ok spyrr hana nafns. Hón tók þá horn fullt mjaðar ok gaf hǫ́num minnis-vęig.\epa

\bpb Siward set himself down, asking for her name. Then she took a horn full of mead, and gave him a draught of memory:\epb\epg


\bvg\bva Hęill \alst{D}agr, \hld\ hęilir \alst{D}ags synir, &
\ind hęil \alst{N}ǫ́tt ok \edtrans{\alst{n}ipt}{[her] kinswoman \ken*{= Earth}}{\Bfootnote{According to \Gylfaginning\ 10 Earth is the daughter of Night and \inx[P]{Aner}.}}! &
\edtrans{\alst{Ó}-ręiðum \alst{au}gum \hld\ lítið \alst{o}kkr þinig}{With unwrathful \ken{friendly} eyes look ye toward us two}{\Bfootnote{An archaic conception; the Gods turning Their friendly gaze toward the worshipper symbolises Their bestowing their favour, and the specific use of \emph{ó-ręiðr} ‘un-wroth’ shows that the wrath of the Gods was feared.  Compare \Hyndluljod\ 6.  Similar language is found in other ancient literatures, e.g. in the Hebrew Bible, most famously in the “Priestly Blessing” of Numbers 6:24–26 where Yahweh’s favour is expressed by “making His face shine” and “lifting His face” toward the receiver of the blessing, and also in Psalms 4:6 and the chorus of Psalms 80, contrasting with 80:17 where the Israelites are depicted as perishing before the rebuke of Yahweh’s face.}} &
\ind ok gefið \alst{s}itjǫndum \alst{s}igr!\eva

\bvb “Hail \inx[P]{Day}! Hail the sons of Day!\footnoteB{TODO. Who?} \\
Hail Night and [her] kinswoman \ken*{= Earth}! \\
With un-wroth \ken{friendly} eyes look ye toward us two, \\
and give the sitters \ken*{= us} victory.\evb\evg


\bvg\bva \edtrans{Hęilir \alst{ę́}sir, \hld\ hęilar \alst{ǫ́}synjur}{Hail the Eese! Hail the Ossens!}{\Bfootnote{Probably formulaic, subverted by Lock in \Lokasenna\ 11 (see note there for possible ritual use).}}, &
\ind hęil sjá in \alst{f}jǫl-nýta \alst{f}old! &
\alst{M}ál ok \alst{m}an-vit \hld\ gefið okkr \alst{m}ę́rum tvęim &
\ind ok \edtrans{\alst{l}ę́knis-hęndr}{healing-hands}{\Bfootnote{Hands with the power to heal (perhaps supernaturally). The singular form \emph{lę́knis-hǫnd} occurs in the semi-Christianized prayer on a c. 1300 stick from Ribe, Denmark (signum DR EM85;493).}} meðan \alst{l}ifum!\eva

\bvb Hail the \inx[G]{Eese}! Hail the \inx[G]{Ossens}! \\
Hail this bountiful fold \ken{earth}! \\
Speech and \inx[C]{manwit} give ye to us renowned two, \\
and \inx[C]{healing-hands}, while we live.”\evb\evg


\bpg\bpa Hon nefndisk Sigrdrífa ok var valkyrja. Hon sagði, at tveir konvngar bǫrðusk. Hét annarr Hjalmgunnarr; hann var þá gamall ok inn mesti hermaðr, ok hafði Óðinn hánum sigri heitit.
En \alst{a}nnarr hét \alst{A}gnarr, \hld\ \alst{Au}ðu bróðir // er \alst{v}ę́tr engi \hld\ \alst{v}ildi þiggja.
Sigrdrífa felldi Hjalm-gunnar í orrostunni. En Óðinn stakk hana svefn-þorni í hefnd þess ok kvað hana aldri skyldu síðan sigr vega í orrostu, ok kvað hana giftask skyldu, „en sagða’k hánum at strengða’k heit þar í mót, at giptask øngom þeim manni er hrę́ðask kynni.“ Hann segir ok biðr hana kenna sér speki ef hon vissi tíðendi ór ǫllum heimum. Sigrdrífa kvað:\epa

\bpb She called herself Syedrive and was a walkirrie. She said, that two kings fought. One was called Helmguther; he was then old and the greatest warrior, and Weden had promised him victory.
But the other was called Eyner, Eade’s brother, who in no way wished to surrender.
Syedrive felled Helmguther in the battle, but Weden stung her with a sleeping-thorn as revenge for that, and declared that she should never thenceforth cause victory in battle, and said that she must marry, “but I said to him that I made a vow in response, to marry no man who could be frightened.” He \ken*{= Siward} speaks and asks her to teach him wisdom, if she knew any tidings out of all the \inx[C]{Home}[Homes]. Syedrive quoth:\epb\epg


\bvg\bva\mssnote{\Regius~32r/18–20, \VolsungaMS~24v/12–14}„\alst{B}jór fǿri’k þér, \hld\ \edtrans{\alst{b}ryn-þings apaldr}{apple-tree of the byrnie-Thing \ken{battle > warrior}}{\Afootnote{\emph{bryn-þinga valdr} ‘wielder of byrnie-Things \ken{battles > warrior}’ \VolsungaMS}}, &
\alst{m}agni blandinn \hld\ ok \alst{m}ęgin-tíri, &
fullr es \alst{l}jóða \hld\ ok \alst{l}íkn-stafa, &
\alst{g}óðra \alst{g}aldra \hld\ ok \edtrans{\alst{g}aman-rúna}{pleasure-runes}{\Afootnote{\emph{gaman-†rędna†} \VolsungaMS}}.\eva

\bvb Beer I bring thee—apple-tree of the byrnie-\inx[C]{Thing} \ken{battle > warrior}!—mixed with might, and might-glory; it is full of \inx[C]{leed}[leeds] and grace-staves, of good \inx[C]{galder}[galders] and pleasure-\inx[C]{rune}[runes].\evb\evg


\bvg\bva\mssnote{\Regius~32r/20–22, \VolsungaMS~24v/14–16}\alst{S}ig-rúnar skalt rísta, \hld\ ef vilt \edtrans{\alst{s}igr hafa}{have victory}{\Afootnote{\emph{snotr vera} ‘be clever’ \VolsungaMS}}, &
\ind ok \edtext{rísta}{\Afootnote{\emph{†rist†} \VolsungaMS}} á \alst{h}jalti \alst{h}jǫrs, &
\edtrans{sumar}{some}{\Afootnote{om. \VolsungaMS}} á \edtext{\alst{v}étt-rimum}{\Afootnote{\emph{vétt-†rvnum†} \VolsungaMS}}, \hld\ \edtrans{sumar}{some}{\Afootnote{\emph{ok} ‘and’ \VolsungaMS}} á \edtext{\alst{v}al-bǫstum}{\Afootnote{\emph{val-†bystum†} \VolsungaMS}}, &
\ind ok nęfna \alst{t}ysvar \alst{T}ý.\eva

\bvb Victory-runes shalt thou know, if thou wilt have victory, and carve on the hilt of the sword; some on the weight-rims;\footnoteB{Unclear. TODO.} some on the wal-basts\footnoteB{Possibly the sword-pommel, the word also occurs in \HelgakvidaHjorvardssonar\ 9. TODO.}, and twice name \inx[P]{Tew}.\evb\evg


\bvg\bva\mssnote{\Regius~32r/22–24, \VolsungaMS~25r/1–3}\alst{Ǫ}l-rúnar skalt kunna \hld\ ef vilt \edtrans{at}{that}{\Afootnote{emend. from \emph{†a†} \VolsungaMS; om. \Regius}} \alst{a}nnars kvę́n &
\ind \edtext{véli-t þik í \alst{t}ryggð}{\Afootnote{\emph{véli þik eigi tryggð} \VolsungaMS}} ef \alst{t}rúir; &
á \alst{h}orni skal \edtrans{þę́r}{them}{\Afootnote{\emph{þat} ‘it’ \VolsungaMS}} rísta \hld\ ok á \alst{h}andar baki &
\ind ok męrkja á \alst{n}agli \edtrans{\alst{N}auð}{Need}{\Bfootnote{i.e. the n-rune, ᚾ.}}.\eva

\bvb Ale-runes shalt thou know, if thou wilt that another man’s wife not betray thee in troth if thou trustest [in her]. On the horn shall [one] carve them, and on the back of the hand, and mark Need on the nail.\evb\evg


\bvg\bva\mssnote{\Regius~32r/24–25, \VolsungaMS~25r/3–4}\edtrans{\alst{F}ull}{The cup}{\Afootnote{\emph{ǫl} ‘The ale’ \VolsungaMS\ breaks alliteration.}} skal signa \hld\ ok við \alst{f}ári séa &
\ind ok verpa \alst{l}auki í \alst{l}ǫg; &
\edtext{\alst{þ}á þat vęit’k, \hld\ at \alst{þ}ér verðr aldri-gi &
\ind \edtext{męini blandinn}{\Afootnote{emend.; \emph{męin-blandinn} \VolsungaMS}} mjǫðr.}{\lemma{þá \dots\ mjǫðr}\Bfootnote{only in \VolsungaMS; om. \Regius}}\eva

\bvb The cup shalt thou sign\footnoteB{Dedicate to the gods with a certain formula. TODO.}, and gaze against the danger, and throw in the liquid a leek. Then I know that it never will be mixed with harm, thy mead.\evb\evg


\bvg\bva\mssnote{\Regius~32r/25–26, \VolsungaMS~25r/5–7}\alst{B}jarg-rúnar skalt \edtrans{kunna}{know}{\Afootnote{\emph{nema} ‘learn’ \VolsungaMS}} \hld\ \edtrans{ef \alst{b}jarga vilt}{if thou wilt rescue}{\Afootnote{\emph{ef þú vilt borgit fá} ‘if thou wilt get rescued’ \VolsungaMS}} &
\ind ok lęysa \alst{k}ind frá \alst{k}onum; &
á \alst{l}ófa þę́r skal rísta \hld\ ok of \alst{l}iðu spęnna &
\ind ok biðja \edtrans{þá}{then}{\Afootnote{om. \VolsungaMS}} \edtrans{\alst{d}ísir}{dises}{\Bfootnote{Minor female deities; one of their roles was helping ailing women during childbirth.  Probably a synonym for the norns; cf. \Fafnismal\ 12.}} \alst{d}uga.\eva

\bvb Rescue-runes shalt thou know, if thou wilt rescue and loosen children from women;\footnoteB{i.e. during difficult childbirth. Cf. \Oddrunargratr, esp. st. TODO, for an example of galders used to avail childbirth.} on the palm shall [one] carve them, and wrap them around the joints, and then bid the dises to avail.\evb\evg


\bvg\bva\mssnote{\Regius~32r/27–29, \VolsungaMS~24v/16–19}\alst{B}rim-rúnar skalt \edtrans{rísta}{carve}{\Afootnote{\emph{gjǫra} ‘make’ \VolsungaMS}} \hld\ ef vilt \alst{b}orgit hafa &
\ind á \alst{s}undi \alst{s}egl-mǫrum; &
á \alst{st}afni \edtrans{skal rísta}{shall [one] carve}{\Afootnote{\emph{skal þę́r rísta} ‘shall [one] carve them’ \VolsungaMS}} \hld\ ok á \alst{st}jórnar blaði &
\ind ok \edtrans{lęggja \alst{ę}ld í \alst{á}r}{lay fire into the oar}{\Bfootnote{i.e. mark it with fire in some way.}};
\edtrans{es-a}{There is not}{\Afootnote{\emph{falla-t} ‘There fall not’ \VolsungaMS}} svá \alst{b}rattr \alst{b}reki \hld\ né svá \alst{b}láar unnir, &
\ind \edtext{þó kømsk-tu \alst{h}ęill af \alst{h}afi}{\lemma{þó ... hafi ‘that ... sea’}\Bfootnote{lit. ‘yet comest thou whole off the sea.’}}.\eva

\bvb Surf-runes shalt thou carve, if thou wilt rescue sail-steeds \ken{ships} on the sound; on the stem shall [one] carve, and on the rudder’s blade, and lay fire into the oar. There is not so steep a breaker nor so blue-black waves, that thou not come whole off the sea.\evb\evg


\bvg\bva\mssnote{\Regius~32r/29–31, \VolsungaMS~25r/7–9}\alst{L}im-rúnar skalt kunna \hld\ ef vilt \alst{l}ę́knir vesa &
\ind ok kunna \alst{s}ár at \alst{s}éa; &
á \alst{b}ęrki skal þę́r rísta \hld\ ok á \edtrans{\alst{b}aðmi}{beam}{\Afootnote{\emph{barri} ‘leaf’}} viðar, &
\ind \edtext{þęim’s}{\Afootnote{\emph{þęss es} \VolsungaMS}} \alst{l}úta austr \alst{l}imar.\eva

\bvb Limb-runes shalt thou know, if thou wilt be a leecher, and know how to look at wounds; on a birch shall [one] carve them, and on the beam of the wood: [on] the one whose limbs bow to the east.\footnoteB{Probably referring to a characteristically bent mountain birch bowing to the east.}\evb\evg


\bvg\bva\mssnote{\Regius~32r/31—34, \VolsungaMS~24v/19–21}\alst{M}ál-rúnar skalt kunna \hld\ ef \edtext{vilt}{\Afootnote{om. \VolsungaMS}} at \alst{m}ann-gi þér &
\ind \alst{h}ęiptum \edtext{gjaldi}{\Afootnote{\emph{†giallda†} \VolsungaMS}} \alst{h}arm; &
þę́r of \alst{v}indr, \hld\ þę́r of \alst{v}ęfr, &
\ind þę́r of \alst{s}ętr allar \alst{s}aman, &
á \alst{þ}ví \alst{þ}ingi \hld\ es \edtrans{\alst{þ}jóðir}{nations}{\Afootnote{\emph{męnn} \VolsungaMS\ breaks alliteration.}} skulu &
\ind í \alst{f}ulla dóma \alst{f}ara.\eva

\bvb Speech-runes shalt thou know, if thou wilt that no man should repay thy offences with harm; them thou windest, them thou weavest, them thou settest all together, on that Thing as nations shall go to full judgements.\evb\evg


\bvg\bva\mssnote{\Regius~32r/34–32v/3, \VolsungaMS~25r/9–10}\alst{H}ug-rúnar skalt \edtrans{kunna}{know}{\Afootnote{\emph{nema} ‘learn’ \VolsungaMS}} \hld\ ef vilt \alst{h}vęrjum vesa &
\ind \edtrans{\alst{g}ęð-svinnari}{sense-swifter}{\Afootnote{\emph{gęð-horskari} ‘sense-sharper’ \VolsungaMS}} \alst{g}uma; &
þę́r of \alst{r}éð, \hld\ þę́r of \alst{r}ęist, &
\ind þę́r of \alst{h}ugði \alst{H}roptr, &
\edtext{af þęim \alst{l}ęgi \hld\ es \alst{l}ekit hafði &
\ind ór \alst{h}ausi \alst{H}ęiðdraupnis &
\ind ok ór \alst{h}orni \alst{H}oddrofnis.}{\lemma{af \dots\ Hoddrofnis ‘from \dots\ Hoardrovner’s [horn].}\Bfootnote{om. \VolsungaMS}}\eva

\bvb Mind-runes shalt thou know, if thou wilt be sense-swifter than every man; them did counsel, them did carve, them did Roft think out, from that liquid which had leaked out of Heathdreepner’s skull and out of Hoardrovner’s horn.\evb\evg


\bvg\bva\mssnote{\Regius~32v/3–4}Á \alst{b}jargi stóð \hld\ með \alst{B}rimis ęggjar, &
\ind \alst{h}afði sér á \alst{h}ǫfði \alst{h}jalm; &
\ind þá \alst{m}ę́lti \alst{M}íms hǫfuð &
\ind \alst{f}róðligt it \alst{f}yrsta orð, &
\ind ok \alst{s}agði \alst{s}anna stafi.\eva

\bvb On the barrow [he] stood along Brimer’s edges; had on his head a helmet. Then spoke the Mime’s head, learnedly, the first word, and said true staves:\evb\evg


\bvg\bva[14a]\mssnote{\Regius~32v/5–7, \VolsungaMS~25r/11–13}Á \alst{sk}ildi kvað ristnar \hld\ þęim’s stęndr fyr \alst{sk}ínanda goði, &
\edtrans{á \alst{ęy}ra \alst{Á}rvakrs, \hld\ ok á}{on Yorewaker’s ear and on}{\Afootnote{om. \VolsungaMS}} \alst{A}lsvinns hófi, &
\edtext{á}{\Afootnote{\emph{ok á} \VolsungaMS}} því \alst{hv}éli \hld\ es \edtrans{snýsk}{turns}{\Afootnote{\emph{stęndr} ‘stands’ \VolsungaMS}} und ręið \edtrans{\alst{H}rungnis}{Rungner’s}{\Afootnote{emend. based on sense and meter; \emph{Ra\d{v}gnis} \Regius; \emph{Raugnis} \VolsungaMS}}, &
á \alst{S}lęipnis \edtrans{tǫnnum}{teeth}{\Afootnote{\emph{taumum} ‘reins’ \VolsungaMS}} \hld\ ok á \alst{s}lęða fjǫtrum,\eva

\bvb On a shield, [he] declared [there to be] carved [runes]—[on] the one that stands before the shining god\footnoteB{Cf. \Grimnismal\ 39, according to which the sun is covered by a shield, protecting the earth from its heat. Without it, the whole world will burn up.} \ken{sun}; on Yorewaker’s ear and on Allswith’s hoof,\footnoteB{The two horses that pull the sun across the heavens; cf. \Grimnismal\ 38.} on that wheel which turns beneath Rungner’s chariot, on Slopner’s teeth and on the fetters of sleds,\evb\evg


\bvg\bva[14b]\mssnote{\Regius~32v/7–9, \VolsungaMS~25r/13–15}á \alst{b}jarnar hrammi \hld\ ok á \alst{B}raga tungu, &
á \alst{u}lfs klóum \hld\ ok á \alst{a}rnar \edtext{nęfi}{\Afootnote{†nefiu† \VolsungaMS}}, &
á \alst{b}lóðgum vę́ngjum \hld\ ok á \alst{b}rúar sporði, &
á \alst{l}ausnar \alst{l}ófa \hld\ ok \edtext{á}{\Afootnote{om. \VolsungaMS}} \alst{l}íknar spori,\eva

\bvb on the bear’s paw and on Bray’s tongue, on the wolf’s claws and on the eagle’s beak, on bloody wings and on the bridge’s supports, on the palm of release and the track of grace,\evb\evg


\bvg\bva[14c]\mssnote{\Regius~32v/9–11, \VolsungaMS~25r/15–18}á \alst{g}lęri ok á \alst{g}ulli \hld\ ok á \edtrans{\alst{g}umna hęillum}{men’s luck-charms}{\Afootnote{\emph{góðu silfri} \VolsungaMS}}, &
í \alst{v}íni ok \alst{v}irtri \hld\ ok \edtrans{\alst{v}ili-sessi}{the comfortable seat}{\Afootnote{\emph{vǫlu sessi} ‘a \inx[C]{wallow}’s seat’ \VolsungaMS}\Bfootnote{\emph{í guma holdi} ‘in a man’s flesh’ add. \VolsungaMS\ is clearly an inserted line.}}, &
á \edtrans{\alst{G}ungnis oddi}{Gungner’s point}{\Afootnote{\emph{Gaupnis oddi} ‘Yeapner’s point’ (an elsewhere unknown spear) \VolsungaMS}} \hld\ ok á \edtrans{\alst{G}rana brjósti}{Grane’s chest}{\Afootnote{\emph{gýgjar brjósti} ‘a \inx[C]{gow}’s chest’}}, &
á \alst{n}ornar \alst{n}agli \hld\ ok á \alst{n}ęfi uglu;\eva

\bvb on glass and on gold and on men’s luck-charms, in wine and beerwort and the comfortable seat, on Gungner’s point and on Grane’s chest, on a norn’s nail and on an owl’s beak.\evb\evg\stepcounter{stanza}


\bvg\bva\mssnote{\Regius~32v/11–14, \VolsungaMS~25r/18–21}\alst{A}llar vǫ́ru \alst{a}f skafnar, \hld\ þę́r’s vǫ́ru \alst{á} ristnar, &
\ind ok \edtrans{\alst{h}vęrfðar}{mixed}{\Afootnote{\emph{†hrędar†} (for \emph{hrǿrðar} ‘stirred’?) \VolsungaMS}} við inn \alst{h}ęlga mjǫð &
\ind ok sęndar á \alst{v}íða \alst{v}ega: &
þę́r ’ru með \edtext{\alst{ǫ́}sum, \hld\ \edtrans{þę́r ’ru}{they are}{\Afootnote{\emph{sumar} ‘some’ \VolsungaMS}} með \alst{ǫ}lfum}{\lemma{ǫ́sum \dots\ ǫlfum ‘Eese \dots\ Elves’}\Afootnote{\emph{ǫlfum \dots\ ǫ́sum} ‘Elves \dots\ Eese’ \VolsungaMS}}, &
\ind \edtrans{sumar}{some}{\Afootnote{\emph{ok} ‘and’ \VolsungaMS}} með \alst{v}ísum \alst{v}ǫnum, &
\ind sumar hafa \alst{m}ęnskir \alst{m}ęnn.\eva

\bvb All were shaven off—those that were carved on— \\
and mixed into the holy mead, \\
and sent on wide ways: \\
They are among the Eese, they are among the Elves; \\
some among the wise Wanes; \\
some have manly men.\evb\evg


\bvg\bva\mssnote{\Regius~32v/14–16, \VolsungaMS~25r/21–25v/3}Þat eru \edtrans{\alst{b}ók-rúnar}{book-runes}{\Bfootnote{Or ‘beech-runes’.  The word may also be emended to \emph{bót-rúnar} ‘cure-runes’, since the letters \emph{c} and \emph{t} were, in the TODO miniscule used on Iceland, very similar.  This emendation is favourable for two reasons: (i) it makes more sense, since the semantic pair \emph{bót} ‘cure’ : \emph{bjarg} ‘rescue’ is surely stronger than \emph{bók} ‘book, beech’ : \emph{bjarg} ‘rescue’, and since the present stanza is specifically referring to the practical use of the runes; (ii) the pair \emph{bót-runar} : \emph{bjarg-rúnar} is already found in a runic charm (B 257, edited under Galders from Bryggen).}}, \hld\ \edtrans{þat eru}{those are}{\Afootnote{\emph{ok} ‘and’ \VolsungaMS}} \alst{b}jarg-rúnar &
\ind ok \alst{a}llar \alst{ǫ}l-rúnar &
\ind \edtrans{ok \alst{m}ę́tar}{and noble}{\Afootnote{\emph{ok mę́rar ok} ‘and renowned and’ \VolsungaMS}} \alst{m}ęgin-rúnar &
hvęim’s þę́r kná \alst{ó}·villtar \hld\ ok \edtext{\alst{ó}·spilltar}{\Afootnote{\emph{†of villtar†} \VolsungaMS}} &
\ind sér at \alst{h}ęillum \alst{h}afa; &
\ind \alst{n}jót-tu ef \alst{n}amt &
\ind unds \edtext{\alst{r}júfask}{\Afootnote{\emph{rjúfa} \VolsungaMS}} \alst{r}ęgin!\eva

\bvb They are book-runes, those are rescue-runes, \\
and all ale-runes, \\
and noble might-runes— \\
for whomever knows them unfalsified and uninjured \\
to use for himself as charms. \\
Use [them] if thou learn [them], \\
until the Reins are ripped!\evb\evg

\sectionline

\bvg\bva\mssnote{\Regius~32v/16–18, \VolsungaMS~25v/3–5}„Nú skalt \alst{k}jósa \hld\ alls þér ’s \alst{k}ostr of boðinn, &
\ind \alst{h}vassa vápna \alst{h}lynr, &
\alst{s}ǫgn eða þǫgn \hld\ haf þér \alst{s}jalfr í hug; &
\ind ǫll eru \alst{m}ęin of \alst{m}etin.“\eva

\bvb {[Syedrive quoth:]} \\
“Now shalt thou choose, as the choice is offered thee, \\
O maple-tree of sharp weapons \ken{warrior}! \\
Speech or silence have for thyself in thy heart; \\
all the harms are measured\footnoteB{i.e. in advance.}!”\evb\evg


\bvg\bva\mssnote{\Regius~32v/18–20, \VolsungaMS~25v/5–8}„Mun’k-a ek \alst{f}lǿja \hld\ þótt mik \alst{f}ęigan vitir, &
\ind em’k-a ek \edtrans{með}{with}{\Afootnote{om. \VolsungaMS}} \alst{b}lęyði \alst{b}orinn; &
\alst{á}st-rǫ́ð þín \hld\ ek vil \alst{ǫ}ll hafa &
\ind svá \alst{l}ęngi sem ek \alst{l}ifi.“\eva

\bvb {[Siward quoth:]} “I shall not flee, although thou know me to be fey; \\
I was not born with softness.\footnoteB{TODO: Note about this common heroic expression.} \\
Thy loving counsels, all, will I have \\
for as long as I may live.”\evb\evg


\bvg\bva\mssnote{\Regius~32v/20–22}„Þat rę́ð’k þér it \alst{f}yrsta \hld\ at við \alst{f}rę́ndr þína &
\ind \alst{v}amma-laust \alst{v}erir; &
\alst{s}íðr þú hęfnir \hld\ þótt þęir \alst{s}akar gøri; &
\ind þat kveða \alst{d}auðum \alst{d}uga.“\eva

\bvb {[Syedrive quoth:]} “This I counsel thee first: that thou against thy kinsmen \\
defend thyself faultlessly. \\
Late oughtst thou to take revenge, although they incur charges; \\
that, they say, befits the dead.\evb\evg


\bvg\bva\mssnote{\Regius~32v/22–24}Þat rę́ð’k þér \alst{a}nnat, \hld\ at \alst{ęi}ð né svęrir, &
\ind nema þann ’s \alst{s}aðr \alst{s}éi, &
\alst{g}rimmar \edtrans{simar}{strands}{\Bfootnote{i.e. ‘strands of fate’; cf. \HelgakvidaOne\ 3, where the norns are said to twist such strands. Often emended to \emph{limar} ‘ramifications’ in accordance with \Reginsmal\ 4, where that word is used in basically the same context. Such a scribal confusion is easily understood, since \emph{s} in this position was always spelled with long \emph{ſ} in the old mss. The paraphrase (see other note) is not conclusive, since it replaces this word with \emph{hefnd} ‘revenge’.}} \hld\ \alst{g}anga at tryggð-rofi; &
\ind armr es \alst{v}ára \alst{v}argr.\eva

\bvb This I counsel thee second: that thou not swear an oath, \\
save for the one which is true. \\
Grim strands follow the troth-breach; \\
wretched is the outlaw of vows.\footnoteB{The punishment is one of torment in the afterlife; see note to \Voluspa\ 39. — The whole stanza is paraphrased in \VolsungaSaga\ ch. 21: \emph{Ok sver eigi rangan eið, því at grimm hefnd fylgir griðrofi.} ‘And swear no wrong oath, for grim revenge follows the grith-breach.’}\evb\evg


\bvg\bva\mssnote{\Regius~32v/24–25}Þat rę́ð’k þér \alst{þ}riðja \hld\ at þú \alst{þ}ingi á &
\ind dęili-t við \alst{h}ęimska \alst{h}ali &
því-at \alst{ó}·sviðr maðr \hld\ lę́tr \alst{o}ft kveðin &
\ind \alst{v}erri orð an \alst{v}iti.\eva

\bvb This I counsel thee third: that thou on the Thing \\
not bandy with foolish men; \\
for an unwise man often lets be spoken \\
worse words than he ought to know.\evb\evg


\bvg\bva\mssnote{\Regius~32v/25–28}Allt es \alst{v}ant \hld\ ef \alst{v}ið þęgir; &
\ind þá þikkir þú með \alst{b}lęyði \alst{b}orinn &
\ind eða \alst{s}ǫnnu \alst{s}agðr; &
\ind \alst{h}ę́ttr es \alst{h}ęimis-kviðr &
\ind nema sér \alst{g}óðan \alst{g}eti. &
\alst{A}nnars dags \hld\ lát hans \alst{ǫ}ndu farit &
\ind ok \alst{l}auna svá \alst{l}ýðum \alst{l}ygi.\eva

\bvb Everything is wrong if thou shut up in reply; \\
then thou seemest born with softness, \\
or truthfully accused. \\
Risky is the hometown-verdict, \\
unless one get himself a good one. \\
On another day do destroy his life, \\
and thus repay the people for the lie.\evb\evg


\bvg\bva\mssnote{\Regius~32v/28–30}Þat rę́ð’k þér it \alst{f}jórða \hld\ ef býr \alst{f}or-dę́ða &
\ind \alst{v}amma-full á \alst{v}egi: &
\alst{g}anga ’s betra \hld\ an \alst{g}ista séi &
\ind þótt þik \alst{n}ǫ́tt of \alst{n}emi.\eva

\bvb This I counsel thee fourth: if there lives an evil-working woman, \\
full of faults, by the road, \\
to walk is better than to take lodgings, \\
although night overtake thee.\evb\evg


\bvg\bva\mssnote{\Regius~32v/30–32}\edtrans{\alst{F}or-njósnar}{looking-ahead}{\Bfootnote{Verbal noun to \emph{nýsask fyrir} ‘to look ahead’, as found in \Havamal\ 7.}} augu \hld\ þurfu \alst{f}ira synir &
\ind hvar’s skulu \alst{v}ręiðir \alst{v}ega; &
oft \alst{b}ǫl-vísar konur \hld\ sitja \alst{b}rautu nér; &
\ind þę́r’s dęyfa \alst{s}verð ok \alst{s}efa.\eva

\bvb Eyes of looking-ahead the sons of men need, \\
wherever wroth men should fight; \\
oft bale-wise women sit near the highway, \\
they who dull sword and sense.\evb\evg


\bvg\bva\mssnote{\Regius~32v/32–34}Þat rę́ð’k þér it \alst{f}immta, \hld\ þótt \alst{f}agrar séir &
\ind \alst{b}rúðir bękkjum á, &
\alst{s}ifja \alst{s}ilfr \hld\ lát-a þínum \alst{s}vefni ráða, &
\ind tęygj-at þér at \alst{k}ossi \alst{k}onur.\eva

\bvb This I counsel thee fifth: although thou seest \\
fair brides on the benches, \\
let not kinsmen’s silver rule thy sleep; \\
lure not women to thee for kisses.\evb\evg


\bvg\bva\mssnote{\Regius~32v/34}\edtext{Þat rę́ð’k þér it \alst{s}étta, \hld\ þótt með \alst{s}ęggjum fari}{\lemma{Þat \dots\ fari ‘That \dots\ may grow’}\Bfootnote{With these words fol. 32v of \Regius\ ends, and we have the “great lacuna”.  The rest of the stanzas are supplied from younger paper mss.}} &
\ind \alst{ǫ}lðr-mál til \alst{ǫ}fug: &
drukkinn \alst{d}ęila \hld\ skal-at við \alst{d}olg-viðu &
\ind margan stelr \alst{v}ín \alst{v}iti.\eva

\bvb This I counsel thee sixth: although among warriors may grow \\
the ale-speech too awry, \\
drunkenly deal shalt thou not with war-trees \ken{warriors}; \\
wine steals wit from many.\evb\evg

TODO: More stanzas from paper manuscripts.

\sectionline
%

% TODO: Summarize contents of "Great Lacuna" with excerpts from relevant Saws

%	\include{books/Fragmented Lay of Siward.tex}%
%	\include{books/From the Death of Siward.tex}%
%	\include{books/First Lay of Guthrun.tex}%
%	\include{books/Short Lay of Siward.tex}%
%	\include{books/Hell-ride of Byrnhild.tex}%
%	\include{books/Slaying of the Nivlings.tex}%
%	\include{books/Second Lay of Guthrun.tex}%
	\bookStart{The Third Lay of Guthrun}[Guðrúnarkviða þriðja]

\begin{flushright}%
Dating \parencite{Sapp2022}: C10th (0.731), early C11th (0.178)

Meter: \Fornyrdislag%
\end{flushright}

A very short narrative poem, depicting a single minor legendary event. It is especially notable for its depiction of a trial by ordeal and the mention of a woman being drowned in a bog.

Herch, one of Attle’s concubines tells Attle that she has seen his wife Guthrun sleeping with Thedric. Attle becomes distressed upon hearing this (P1). Guthrun asks him what is wrong (1), and he responds that Herch has accused her of sleeping with Thedric (2). Guthrun promises to to prove her innocence through a trial by ordeal involving picking up a white stone from boiling water (3). She further says that while she and Thedric did sit down together, they did so in mutual grief over the deaths of her brothers (4–5). She tells Attle to summon a German lord named Saxe, who knows how to carry out the trial. Seven hundred men arrive to witness the event (6). Before picking up the stone, Guthrun laments over her brothers’ deaths, saying that they would have disputed the accusation through violence, but that she must now prove her innocence by herself (7). She then puts her hand in the boiling water, and unscathed takes out the stones. She holds it up and shows it to the witnesses (8). Attle laughs, knowing that his wife has been faithful, and orders Herch to pick up the stone (9). She does so, but her hands are horribly scorched, and men lead her to a “foul bog”, presumably to be drowned (see above). The poet ends by laconically stating that Guthrun in such a way was “reconstituted for her affronts”.

\sectionline

\bpg
\bpa Herkja hét ambǫ́tt Atla; hón hafði verit frilla hans. Hón sagði Atla at hón hefði sét Þjóðrek ok Guðrúnu bę́ði saman. Atli var þá allókátr. Þá kvað Guðrún:\epa

\bpb Herch was named the female thrall of Attle; she had been his concubine. She told Attle that she had seen Thedric and Guthrun both together. Attle was then wholly displeased. Then Guthrun quoth:\epb
\epg


\bvg
\bva „Hvat ’s þér, Atli? \hld\ ę́, Buðla sonr, &
es þér hryggt í hug; \hld\ hví hlę́r þú ę́va? &
Hitt myndi ǿðra \hld\ jǫrlum þykkja &
at við męnn mę́ltir \hld\ ok mik sę́ir.“\eva

\bvb “What is with thee, Attle? Always, son of Bodle, art thou sad at heart; why laughest thou never? TODO.”\evb
\evg


\bvg
\bva „Tregr mik þat, Guðrún, \hld\ Gjúka dóttir, &
mér í hǫllu \hld\ Hęrkja sagði &
at þit Þjóðrekr \hld\ undir þaki svę́fið &
ok léttliga \hld\ líni vęrðið.“\eva

\bvb “It troubles me, Guthrun, Yivick’s daughter, as in the hall Herch has said me: that thou and Thedric beneath thatched roof slept, and ye lightly warded the linen.\footnoteB{i.e., they threw off their clothes and slept together.}”\evb
\evg


\bvg
\bva „Þér mun’k alls þęss \hld\ ęiða vinna &
at inum hvíta \hld\ hęlga stęini, &
at ek við Þjóðmar \hld\ þat-ki átta’k, &
es vǫrðr né verr \hld\ vinna knátti,—\eva

\bvb “To thee I will swear oaths regarding all of that—by the white, holy stone—that I did not do such a thing with Thedmar,\footnoteB{Historically, Thedmar was the father of Thedric, who took over the kingdom after his father’s death (see Encyclopedia). Thedmar may here be a scribal error for Thedric, a scribal error for “Thedmar’s son”, or a nickname due to conflation of the father and son.} which neither watchman nor warrior has been able to swear upon,—\footnoteB{Guthrun says that she will prove her innocence through a trial by ordeal (that is, by lifting “the white holy stone” out of boiling water; see st. 8). She further strengthens her position by pointing out that no reliable man has sworn an oath attesting to her guilt.}”\evb
\evg


\bvg
\bva Nema ek halsaða \hld\ hęrja stilli, &
jǫfur ónęisinn, \hld\ ęinu sinni; &
aðrar vǫ́ru \hld\ okkrar spękjur &
es vit hǫrmug tvau \hld\ hnigum at rúnum.\eva

\bvb Unless I embraced the stiller of hosts \ken*{\textsc{ruler} = Thedmar}—the unshamed prince—a single time. Different were our dealings, when we two distressed ones [Guthrun and Thedric] reclined in private conversation.\evb
\evg


\bvg
\bva Hér kom Þjóðrekr \hld\ með þrjá tøgu, &
lifa þęir né ęinir, \hld\ þriggja tega manna; &
hrinktu mik at brǿðrum \hld\ ok at brynjuðum, &
hrinktu mik at ǫllum \hld\ á hǫfuðniðjum.\eva

\bvb Here came Thedric with thirty; not one of those thirty men still live. Surround\footnoteB{\emph{hrinktu} consisting of \emph{hring}, 2nd sg. imper. of \emph{hringja} ‘surround, encircle’ + \emph{þú} ‘thou’. The clitic form \emph{-tu} has caused devoicing.} me with my brothers, and with byrnied men; surround me with all my close kinsmen.\evb
\evg


\bvg
\bva Sęnd at Saxa, \hld\ sunnmanna gram; &
hann kann hęlga \hld\ hver vellanda;“ &
sjau hundruð manna \hld\ í sal gingu &
áðr kvę́n konungs \hld\ í kętil tǿki.\eva

\bvb Send for Saxe, lord of the southmen; he knows how to hallow a swelling cauldron!” Seven hundred men went into the hall, before the wife of the king might touch the kettle.\evb
\evg


\bvg
\bva „Kemr-a nú Gunnarr, \hld\ kalli’k-a Hǫgna, &
sé’k-a síðan \hld\ svása brǿðr; &
sverði myndi Hǫgni \hld\ slíks harms reka, &
nú verð’k sjǫlf fyr mik \hld\ synja lýta.“\eva

\bvb “Now Guther comes not, I can not call on Hain; I see not thereafter [my] beloved brothers. With a sword would Hain avenge such an affront; now I will for myself disprove the slanders.”\evb
\evg


\bvg
\bva Brá hón til botns \hld\ bjǫrtum lófa &
ok hón upp of tók \hld\ jarknastęina: &
„Sé nú sęggir \hld\ —sykn em ek orðin &
hęilagliga— \hld\ hvé sjá hverr velli.“\eva

\bvb Brought she the bright palms to the bottom, and she up did take the earkenstones: “See now, men—I am proven innocent, through holy means—how this cauldron boils!”\evb
\evg


\bvg
\bva Hló þá Atla \hld\ hugr í brjósti &
es hann hęilar sá \hld\ hęndr Guðrúnar: &
„Nú skal Hęrkja \hld\ til hvers ganga, &
sú’s Guðrúnu \hld\ grandi vę́nti.“\eva

\bvb Then laughed the heart in Attle’s chest, when he saw unscathed the hands of Guthrun: “Now shall Herch go to the cauldron, she who to Guthrun hoped to cause harm.”\evb
\evg


\bvg
\bva Sá-at maðr armligt, \hld\ hvęrr es þat sá at, &
hvé þar á Hęrkju \hld\ hęndr sviðnuðu; &
lęiddu þá męy \hld\ í mýri fúla, &
svá þá Guðrún \hld\ sinna harma.\eva

\bvb Each man saw not something so pitiful, who saw that: how there on Herch the hands were scorched. Led they the maiden into the foul bog; thus was Guthrun reconstituted for her affronts.\evb
\evg
%
%	\include{books/Weeping of Ordrun.tex}%
	\bookStart{Lay of Attle}[Atlakviða]

\begin{flushright}%
\textbf{Dating} \parencite{Sapp2022}: C10th (0.719)–early C11th (0.212)

\textbf{Meter:} \Malahattr, \Fornyrdislag
\end{flushright}%

A famously archaic poem.

Attle sends his messenger Kneefrith to Guther (1). He arrives at Guther’s hall, where the mood is one of unease, and addresses Guther (2). Kneefrith invites him and his brother Hain to Attle’s court (3), offering them treasures, weapons and land (4–5). Guther asks his brother Hain for advice, since he has not heard of Attle having gold to give away (6).

\sectionline

\section{The Death of Attle (\emph{Dauði Atla})}

\bpg\bpa Guðrún Gjúkadóttir hefndi brǿðra sinna, svá sem frę́gt er orðit. Hon drap fyrst sonu Atla, en eptir drap hon Atla ok brendi hǫllina ok hirðina alla; um þetta er sjá kviða ort.\epa

\bpb Guthrun Yivicksdaughter avenged her brothers, as has become famous. She first killed the sons of Attle, and after that she killed Attle, and burned the hall and the whole hird. Regarding that this lay is wrought.\epb\epg

\sectionline

\bvg\bva \alst{A}tli sęndi \hld\ \alst{á}r til Gunnars &
\alst{k}unnan sęgg at ríða, \hld\ \alst{K}néfrøðr vas sá hęitinn; &
at \alst{g}ǫrðum kom hann \alst{G}júka \hld\ ok at \alst{G}unnars hǫllu, &
\alst{b}ękkjum arin-gręypum \hld\ ok at \alst{b}jóri svǫ́sum.\eva

\bvb Attle sent—of yore–to Guther \\
a well-known messenger to ride; \inx[P]{Kneefrith} he was called. \\
To the yards of Yivick he came, and to the hall of Guther; \\
to the hearth-surrounding benches, and to the lovely beer.\evb\evg


\bvg\bva \alst{D}rukku þar \alst{d}rótt-męgir \hld\ —ęn \edtrans{\alst{d}yljęndr}{concealed ones}{\Bfootnote{\textcite{FinnurEdda} reasonably interprets this as referring to Attle’s spies at Guther’s court.}} þǫgðu— &
\alst{v}ín í \edtrans{\alst{v}al-hǫllu}{the walhall}{\Bfootnote{The interpretation of this compound is difficult in the current context. The first element \emph{val-} could be (1) \emph{valr} ‘falcon’, referring to the aristocratic hunting practice; (2) \emph{valr} ‘\inx[G]{Wales}[Wale]’, cognate with ‘Welsh’ but in ON referring to the French or Romans, stressing the southern location or appearance of the hall; or (3) \emph{valr} ‘(collective) the battle-slain’, foreshadowing the inevitable death (\inx[C]{feyness}) of the \inx[G]{Yivickings}. If (3) is correct the word is linguistically identical to \inx[L]{Walhall}, Weden’s hall, whither the battle-slain go.}}, \hld\ \alst{v}ręiði sǫ́usk þęir Húna; &
\alst{k}allaði þá \alst{K}néfrøðr \hld\ \alst{k}aldri rǫddu, &
\alst{s}ęggr inn \alst{s}uð-rǿni \hld\ \alst{s}at hann á bękk hǫ́m:\eva

\bvb There the dright-lads \ken{warriors} drank—but the concealed ones shut up— \\
wine in the walhall; they feared the wrath of the Huns. \\
Then called Kneefrith with cold voice, \\
the southern messenger, he sat on a high bench:\evb\evg


\bvg\bva „\alst{A}tli mik hingat sęndi \hld\ ríða \alst{ø}ręndi, &
\alst{m}ar inum \alst{m}él-gręypa, \hld\ \alst{M}yrk-við inn ó·kunna &
at \alst{b}iðja yðr, Gunnarr, \hld\ at it á \alst{b}ękk kǿmið &
með \alst{h}jǫlmum arin-gręypum \hld\ at sǿkja \alst{h}ęim Atla.\eva

\bvb “Attle sent me hither to ride with an errand, \\
on the bit-champing steed through Mirkwood uncharted— \\
to ask you, O Guther, that ye two \ken*{= Guther and Hain} on the bench come, \\
with hearth-surrounding helmets, to seek the home of Attle.\evb\evg


\bvg\bva \alst{Sk}jǫldu kneguð þar vęlja \hld\ ok \alst{sk}afna aska, &
\alst{h}jalma gull-roðna \hld\ ok \alst{H}úna męngi, &
\alst{s}ilfr-gyllt \alst{s}ǫðul-klę́ði, \hld\ \alst{s}ęrki val-rauða, &
\alst{d}afar, \alst{d}arraða, \hld\ \alst{d}rǫsla mél-gręypa.\eva

\bvb There ye might choose shields, and shaven ash-spears, \\
helmets gold-reddened, and the multitude of the Huns, \\
silver-gilt saddle-cloths, blood-red serks, \\
daves, spears, bit-champing steeds.\evb\evg


\bvg\bva \alst{V}ǫll létsk ykkr ok myndu gefa \hld\ \alst{v}íðrar Gnita-hęiðar &
af \alst{g}ęiri \alst{g}jallanda \hld\ ok af \alst{g}ylltum stǫfnum, &
\alst{st}órar męiðmar \hld\ ok \alst{st}aði Danpar, &
hrís þat it \alst{m}ę́ra \hld\ es meðr \alst{M}yrk-við kalla.“\eva

\bvb He also declared himself willing to give you two the field of wide Gnit-heath, \\
{[and]} of yelling spears and of gilded prows, \\
great treasures and the place of Danp; \\
the renowned brush which men call Mirkwood.\evb\evg


\bvg\bva \alst{H}ǫfði vatt þá Gunnarr \hld\ ok \alst{H}ǫgna til sagði: &
„Hvat rę́ðr þú okkr, \alst{s}ęggr hinn ǿri, \hld\ alls vit \alst{s}líkt hęyrum? &
\alst{G}ull vissa’k ękki \hld\ á \alst{G}nita-hęiði, &
þat’s vit \alst{ę́}ttim-a \hld\ \alst{a}nnat slíkt.\eva

\bvb His head turned Guther then, and said to Hain: \\
“What dost thou counsel us two, O younger man, as such a thing we hear? \\
I knew of no gold on the Gnit-heath \\
which we two should not own as much of.\evb\evg


\bvg\bva \alst{S}jau ęigu vit \alst{s}al-hús \hld\ \alst{s}verða full, &
\alst{h}vęrju ’ru þęira \hld\ \alst{h}jǫlt ór gulli; &
\alst{m}ínn vęit’k \alst{m}ar bętstan \hld\ en \alst{m}ę́ki hvassastan, &
\alst{b}oga \alst{b}ękk-sǿma \hld\ en \alst{b}rynjur ór gulli;\eva

\bvb We own seven hall-houses filled with swords— \\
on each of them is a golden hilt; \\
I know my horse to be the best and {[my]} sword the sharpest, \\
{[my]} bow bench-fit and {[my]} byrnies golden,\evb\evg


\bvg\bva \alst{h}jalm ok skjǫld \alst{h}vítastan, \hld\ kominn ór \alst{h}ǫll Kíars; &
\alst{ęi}nn ’s mínn bętri \hld\ en sé \alst{a}llra Húna.“\eva

\bvb {[my]} helmet and shield the whitest, come from Choser’s hall; \\
mine alone is better, than [those] of all of the Huns might be!”\evb\evg


\bvg\bva „Hvat hyggr \alst{b}rúði \alst{b}ęndu \hld\ þá’s hón okkr \alst{b}aug sęndi, &
\alst{v}arinn \alst{v}ǫ́ðum hęiðingja? \hld\ Hykk at hón \alst{v}ǫrnuð byði! &
\alst{H}ár fann’k \alst{h}ęiðingja \hld\ riðit í \alst{h}ring rauðum; &
\alst{y}lfskr es vegr \alst{o}kkarr \hld\ at ríða \alst{ø}ręndi.“\eva

\bvb {[Hain quoth:]} \\
“What thinkest thou the bride meant when she sent us a bigh \\
wrapped with a heath-dweller’s cloth \ken{wolf > wolf’s hair}? I think she meant it as a warning! \\
A heath-dweller’s \ken{wolf’s} hair I found wrapped round the red ring: \\
wolven is our road, if we ride that errand!\footnoteB{That it is the more cautious Hain who speaks here is clear from Guther’s response in the following stanzas.  Whereas Hain judges the wolf-hair to be a warning of Hunnish treachery, Guther thinks that it is a warning that wolves will steal his treasure if he does not show up.}”\evb\evg


\bvg\bva \alst{N}iðjar-gi hvǫttu Gunnar \hld\ né \alst{n}áungr annarr, &
\alst{r}ýnęndr né \alst{r}áðęndr, \hld\ né þęir’s \alst{r}íkir vǫ́ru; &
\alst{k}vaddi þá Gunnarr \hld\ sęm \alst{k}onungr skyldi, &
\alst{m}ę́rr í \alst{m}jǫð-ranni \hld\ af \alst{m}óði stórum:\eva

\bvb No kinsmen Guther, nor any other relation, \\
not counselors nor advisors, nor those who were powerful. \\
Then Guther announced—as a king should, \\
renowned in the mead-hall—with great spirit:\evb\evg


\bvg\bva „Rís-tu nú, \edtrans{\alst{F}jǫrnir}{Ferner}{\Bfootnote{An otherwise unknown servant.}}, \hld\ lát-tu á \alst{f}lęt vaða &
\alst{g}reppa \alst{g}ull-skálir \hld\ með \alst{g}umna hǫndum!\eva

\bvb “Rise now, Ferner! Let on the benches wade forth \\
the golden bowls of warriors along the hands of men!\evb\evg


\bvg\bva \alst{U}lfr mun ráða \hld\ \alst{a}rfi Niflunga, &
\alst{g}amlir \alst{g}ran-varðir, \hld\ ef \alst{G}unnars missir; &
\alst{b}irnir \alst{b}lakk-fjallir \hld\ \alst{b}íta þref-tǫnnum, &
\alst{g}amna \alst{g}ręy-stóði, \hld\ ef \alst{G}unnarr né kømr-at.“\eva

\bvb The wolf will rule the inheritance of the Nivlings— \\
the old grey guardians \ken{wolves}—if Guther is absent. \\
Black-furred bears will bite with wrangling teeth— \\
amusing the bitch-pack—if Guther comes not.”\evb\evg


\bvg\bva \alst{L}ęiddu land-rǫgni \hld\ \edtrans{\alst{l}ýðar ó·nęisir}{unshamed men}{\Bfootnote{Compare the long-line on the Thorsberg chape (\char`~\ 160–240 AD): \emph{wlþuþewaʀ \hld\ ni wajē-māriʀ} ‘Wolthew, the not ill-famed \ken{famous}’.}}, &
\alst{g}rátęndr, \alst{g}unn-hvatan, \hld\ ór \alst{g}arði Húna; &
þá kvað þat inn \alst{ø̇}ri \hld\ \alst{ę}rfi-vǫrðr Hǫgna: &
„\alst{H}ęilir farið nú ok \alst{h}orskir \hld\ hvar’s ykkr \alst{h}ugr tęygir!“\eva

\bvb Unshamed men led the lord of the land, \\
weeping, the battle-bold man out of the yards of the Huns. \\
Then quoth this the young inheritance-ward \ken{son} of Hain: \\
“Fare ye two now whole and wise wherever your heart may draw you!”\evb\evg


\bvg\bva \alst{F}etum létu \alst{f}rǿknir \hld\ of \alst{f}jǫll at þyrja &
\alst{m}ar ina \alst{m}él-gręypu, \hld\ \alst{M}yrk-við inn ókunna; &
\alst{h}ristisk ǫll \alst{H}ún-mǫrk \hld\ þar’s \alst{h}arð-móðgir fóru, &
\alst{v}rǫ́ku þęir \alst{v}and-styggva \hld\ \alst{v}ǫllu al-grǿna.\eva

\bvb With strides the braves made the bit-champing steed \\
rush o’er the fells through Mirkwood uncharted. \\
All Hunmark shook where the hard-minded went forth; \\
they drove the whip-shy horse along the allgreen fields.\evb\evg


\bvg\bva \alst{L}and sǫ́u þęir Atla \hld\ ok \alst{l}ið-skjalfar djúpar; &
\alst{B}ikka greppar standa \hld\ á \alst{b}org inni hǫ́u, &
\alst{s}al of \alst{s}uðr-þjóðum, \hld\ \alst{s}lęginn sess-męiðum, &
\alst{b}undnum rǫndum, \hld\ \alst{b}lęikum skjǫldum,\eva

\bvb The land of Attle they saw, and ravines deep, \\
\inx[P]{Bicke}’s soldiers standing on the high stronghold, \\
the hall of the southfolk built with seat-beams, \\
with bound rims, with pale shields,\evb\evg


\bvg\bva \alst{d}afar, \alst{d}arraða; \hld\ en þar \alst{d}rakk Atli &
\alst{v}ín í \alst{v}al-hǫllu; \hld\ \alst{v}ęrðir sǫ́tu úti &
at \alst{v}arða þęim Gunnari \hld\ ef þęir hér \alst{v}itja kǿmi &
með \alst{g}ęiri \alst{g}jallanda \hld\ at vękja \alst{g}ram hildi.\eva

\bvb daves, spears. And there drank Attle \\
wine in the wal-hall—watchmen sat outside \\
to watch for Guther’s men, if they came here to visit, \\
with yelling spears to wake the ruler with war.\evb\evg


\bvg\bva \alst{S}ystir fann þęira \alst{s}nemmst \hld\ at þęir í \alst{s}al kvǫ́mu, &
\alst{b}rǿðr hęnnar \alst{b}áðir, \hld\ \alst{b}jóri vas hón lítt drukkin: &
„\alst{R}áðinn est nú, Gunnarr, \hld\ hvat munt, \alst{r}íkr, vinna &
við \alst{H}úna \alst{h}arm-brǫgðum? \hld\ \alst{H}ǫll gakk þú ór snemma!\eva

\bvb Their sister found soonest they they had come into the hall— \\
her brothers both—on beer was she lightly drunk: \\
“Betrayed art thou now, Guther; what wilt thou, powerful man, work \\
against the Hunnish harm-tricks? Go soon out of the hall!\footnoteB{Before anything evil might happen.}”\evb\evg


\bvg\bva \alst{B}ętr hęfðir þú, \alst{b}róðir, \hld\ at þú í \alst{b}rynju fǿrir, &
sęm \alst{h}jǫlmum arin-gręypum \hld\ at séa \alst{h}ęim Atla; &
\alst{s}ę́tir þú í \alst{s}ǫðlum \hld\ \alst{s}ól-hęiða daga, &
\alst{n}ái \alst{n}auð-fǫlva \hld\ létir \alst{n}ornir gráta,\eva

\bvb Better hadst thou, brother, if thou hadst gone in byrnie \\
with hearth-surrounding helmets, to see the home of Attle; \\
if thou hadst set in the saddle during sun-bright days \\
need-pale corpses; if thou madest the norns cry,\evb\evg


\bvg\bva \alst{H}úna skjald-męyjar \hld\ \alst{h}ęrfi kanna &
en \alst{A}tla sjalfan \hld\ létir í \alst{o}rm-garð koma; &
nú ’s sá \alst{o}rm-garðr \hld\ \alst{y}kkr of folginn.“\eva

\bvb {[and]} the Hunnish shield-maidens to know the harrow;\footnoteB{i.e. if he turned the Hunnish shield-maidens into enslaved farmhands.} \\
and Attle himself hadst thou brought in the snake-pit— \\
now that snake-pit has swallowed you two!”\evb\evg

Guther answers:

\bvg\bva „\alst{S}ęinað ’s nú, systir, \hld\ at \alst{s}amna Niflungum, &
\alst{l}angt ’s at \alst{l}ęita \hld\ \alst{l}ýða sinnis til, &
of \alst{r}osmu-fjǫll \alst{R}ínar, \hld\ \alst{r}ekka ó·nęissa.“\eva

\bvb “’Tis late now, sister, to gather the Nivlings; \\
’tis far to look for the support of men: \\
over the great fells of the Rhine for unshamed warriors.”\evb\evg


\bvg\bva \alst{F}engu þęir Gunnar \hld\ ok í \alst{f}jǫtur sęttu, &
\edtrans{vin \alst{B}orgunda}{the friend of the Burgends}{\Bfootnote{The historic Guther was king of the Burgundians.  The manuscript has a small stroke above the \emph{n} that abbreviates the syllable \emph{ir}, indicating the plural \emph{vinir} ‘friends’, who would then be the people binding Guther.  This is probably due to a scribal misunderstanding of a not uncommon type, since the significance of the kenning had been forgotten.  It is clearly old, for in \Waldere\ 46 Walder addresses Guther, whom he is just about to fight, by the identical phrase \emph{wine Burgenda}.}}, \hld\ ok \alst{b}undu fastla; &
\alst{s}jau hjó Hǫgni \hld\ \alst{s}verði hvǫssu &
en inum \alst{á}tta hratt hann \hld\ í \alst{ę}ld hęitan.\eva

\bvb They caught Guther and in fetters placed him \\
—the friend of the Burgends—and bound him firmly. \\
Hain smote seven with a sharp sword, \\
and the eighth one he threw into hot fire.\evb\evg


\bvg\bva \edtext{Svá skal \alst{f}rǿkn \hld\ \alst{f}jǫ́ndum vęrjask;}{\lemma{Svá \dots\ vęrjask}\Bfootnote{Line moved from the last st. to this one since it seems to connect semantically with the immediately following line, and results in two typical four-line stanzas.}} &
\alst{H}ǫgni varði \hld\ \alst{h}ęndr Gunnars. &
\alst{f}rǫ́gu \alst{f}rǿknan \hld\ ef \alst{f}jǫr vildi &
\alst{G}otna þjóðann \hld\ \alst{g}ulli kaupa.\eva

\bvb So shall a brave guard himself against foes; \\
Hain guarded the hands of Guther. \\
They asked the brave \ken*{Guther} if his \ken*{Hain’s} life he wished— \\
the ruler of the Gots—to buy with gold.\footnoteB{The Huns try to make Guther (the “ruler of the Gots”, cf. sts. 1, 3, 10) pay for Hain’s life.  Guther instead responds with the following.}\evb\evg


\bvg\bva „\alst{H}jarta skal mér \alst{H}ǫgna \hld\ í \alst{h}ęndi liggja &
\alst{b}lóðugt, ór \alst{b}rjósti \hld\ skorit \alst{b}ald-riða, &
\edtrans{\alst{s}axi \alst{s}líðr-bęitu}{slide-biting sax}{\Bfootnote{A short-sword with a blade so sharp that it draws blood when one slides the finger across it.}}, \hld\ \alst{s}yni þjóðans.“\eva

\bvb “The heart of Hain shall lie in my hands: \\
bloody from the breast, cut from the bold rider \ken*{= Hain}, \\
with a slide-biting sax, from the son of the sovereign \ken*{= Hain}.”\evb\evg


\bvg\bva Skǫ́ru þęir \alst{h}jarta \hld\ \alst{H}jalla ór brjósti, &
\alst{b}lóðugt, ok á \alst{b}jóð lǫgðu \hld\ ok \alst{b}ǫ́ru þat fyr Gunnar.\eva

\bvb They cut the heart of Helle from the breast, \\
bloody, and on a platter laid it, and bore it before Guther.\evb\evg


\bvg\bva Þá kvað þat \alst{G}unnarr, \hld\ \alst{g}umna dróttinn: &
„\alst{H}ér hęfi’k \alst{h}jarta \hld\ \alst{H}jalla ins blauða, &
ó·líkt \alst{h}jarta \hld\ \alst{H}ǫgna ins frǿkna, &
es mjǫk \alst{b}ifask \hld\ es á \alst{b}jóði liggr; &
\alst{b}ifðisk hǫlfu męirr \hld\ es í \alst{b}rjósti lá!“\eva

\bvb Then quoth this Guther, the lord of men: \\
“Here have I the heart of Helle the soft—unlike the heart of Hain the bold!— \\
which quivers greatly when on the platter it lies; \\
it quivered twice as much when in the breast it lay.”\evb\evg


\bvg\bva \alst{H}ló þá \alst{H}ǫgni \hld\ es til \alst{h}jarta skǫ́ru &
\alst{k}vikvan \alst{k}umbla-smið \hld\ —\alst{k}løkkva síðst hugði. &
\alst{B}lóðugt þat á \alst{b}jóð lǫgðu \hld\ ok \alst{b}ǫ́ru fyr Gunnar.\eva

\bvb Hain then laughed as to the heart they cut \\
the living wound-smith \ken*{\textsc{warrior} = Hain}; he thought least of sobbing. \\
Bloody on a platter they laid it, and bore it before Guther.\evb\evg


\bvg\bva Mę́rr kvað þat \alst{G}unnarr, \hld\ \alst{G}ęir-Niflungr: &
„\alst{H}ér hęfi’k \alst{h}jarta \hld\ \alst{H}ǫgna ins frǿkna, &
ó·líkt \alst{h}jarta \hld\ \alst{H}jalla ins blauða, &
es lítt \alst{b}ifask \hld\ es á \alst{b}jóði liggr; &
\alst{b}ifðisk svá-gi mjǫk \hld\ þá’s í \alst{b}rjósti lá!\eva

\bvb Renowned Guther quoth this, the Spear-Nivling: \\
“Here have I the heart of Hain the bold \\
—unlike the heart of Helle the soft!— \\
which quivers lightly when on the platter it lies; \\
it quivered not so much when in the breast it lay.\evb\evg


\bvg\bva Svá skalt, \alst{A}tli, \hld\ \alst{au}gum fjarri &
sęm \alst{m}unt \hld\ \alst{m}ęnjum verða; &
es und \alst{ęi}num mér \hld\ \alst{ǫ}ll of folgin &
\alst{h}odd Niflunga: \hld\ lifir-a nú \alst{H}ǫgni!\eva

\bvb Thus shalt thou, Attle, be as far from the eyes \\
as thou wilt from the neck-rings. \\
With me alone is hidden all \\
the hoard of the Nivlings—now Hain lives not!\evb\evg


\bvg\bva Ęy vas mér \alst{t}ýja \hld\ meðan vit \alst{t}vęir lifðum, &
nú ’s mér \alst{ę}ngi \hld\ es \alst{ęi}nn lifi’k; &
\alst{R}ín skal \alst{r}áða \hld\ \alst{r}óg-malmi skatna, &
svinn, \alst{ǫ́}s-kunna \hld\ \alst{a}rfi Niflunga.\eva

\bvb I always had doubt when we two lived; \\
now I have none when I alone live. \\
The Rhine shall rule the strife-ore of princes \ken{gold}: \\
the swift [river] the os-born patrimony of the Nivlings!\evb\evg


\bvg\bva Í \alst{v}eltanda \alst{v}atni \hld\ lýsask \alst{v}al-baugar &
\alst{h}ęldr an á \alst{h}ǫndum gull \hld\ skíni \alst{H}úna bǫrnum.“\eva

\bvb In tumbling water will the Welsh bighs gleam, \\
rather than gold on the hands shine for the children of Huns!”\evb\evg

\sectionline

\bvg\bva “Ýkvið ér hvél-vǫgnum, \hld\ haptr ’s nú í bǫndum!”\eva

\bvb “Turn ye the wheel-wagons, the captive is now in bonds!”\evb\evg


\bvg\bva Atli inn ríki\eva

\bvb TODO\evb\evg


\bvg\bva Svá gangi þér\eva

\bvb TODO\evb\evg


\bvg\bva ok meirr þaðan\eva

\bvb TODO\evb\evg


\bvg\bva \alst{L}ifanda gram \hld\ \alst{l}agði í garð, &
þann’s \alst{sk}riðinn vas, \hld\ \alst{sk}atna męngi, &
\alst{i}nnan \alst{o}rmum. \hld\ En \alst{ęi}nn Gunnarr &
\alst{h}ęipt-móðr hǫrpu \hld\ \alst{h}ęndi kníði; &
\alst{g}lumðu stręngir. \hld\ Svá skal \alst{g}olli &
\alst{f}rǿkn hring-drifi \hld\ við \alst{f}ira halda!\eva

\bvb The living prince was laid in the enclosure \\
(which was crawling) by a multitude of warriors \\
(with snakes inside). And Guther alone \\
spitefully struck the harp with his hand; \\
the strings rang out. \emph{So} shall hold \\
a brave ring-strewer his gold from men.\evb\evg


\bvg\bva Dynr vas í garði,\eva

\bvb TODO\evb\evg


\bvg\bva Út gekk þá Guðrún,\eva

\bvb TODO\evb\evg


\bvg\bva Umðu ǫlskálir\eva

\bvb TODO\evb\evg


\bvg\bva Út gekk þá Guðrún,\eva

\bvb TODO\evb\evg


\bvg\bva Skævaði þá in skírleita\eva

\bvb TODO\evb\evg


\bvg\bva Sona hefir þinna, \eva

\bvb TODO\evb\evg


\bvg\bva Kallar-a þú síðan \eva

\bvb TODO\evb\evg


\bvg\bva Ymr varð á bekkjum, \eva

\bvb TODO\evb\evg


\bvg\bva Gulli seri \eva

\bvb TODO\evb\evg


\bvg\bva Ȯ-varr Atli \hld\ móðan hafði sik drukkit; &
vápn hafði hann ękki, \hld\ varnaði-t við Guðrúnu; &
opt vas sá lęikr bętri \hld\ þá’s þau lint skyldu &
optarr of faðmask \hld\ fyr ǫðlingum.\eva

\bvb Unwary Attle had drunk himself tired; \\
he had no weapons; did not beware Guthrun. \\
Oft their play was better when they gently would \\
more often embrace each other before the athlings.\evb\evg


\bvg\bva Hǫ́n \alst{b}ęð \alst{b}roddi \hld\ gaf \alst{b}lóð at drekka, &
\alst{h}ęndi \alst{h}ęl-fu̇ssi, \hld\ ok \alst{h}velpa lęysti; &
\alst{h}ratt fyr \alst{h}allar dyrr \hld\ ok \alst{h}ús-karla vakði, &
\alst{b}randi, \alst{b}rúðr, hęitum; \hld\ þau lét hǫ́n gjǫld \alst{b}rǿðra.\eva

\bvb With a blade she gave the bed blood to drink, \\
—with a hell-eager hand—and loosened the whelps, \\
blocked the doors of the hall and awoke the housecarls, \\
the bride, with hot flame—so she repaid her brothers!\evb\evg


\bvg\bva \alst{Ę}ldi gaf hón \alst{a}lla \hld\ es \alst{i}nni vǫ́ru &
ok frá \alst{m}orði þęira Gunnars \hld\ komnir vǫ́ru ór \alst{M}yrk-hęimi; &
\alst{f}orn timbr \alst{f}ellu, \hld\ \alst{f}jarg-hús ruku, &
\alst{b}ǿr \alst{B}uðlunga, \hld\ \alst{b}runnu ok skjald-męyjar, &
\alst{i}nni; \alst{a}ldr-stamar \hld\ hnigu í \alst{ę}ld hęitan.\eva

\bvb To the fire she gave all who were within \\
and from the murder of Guther’s men had come from Mirkham. \\
Ancient timbers fell, great houses smoked— \\
the settlement of the Budlungs—also the shield–maidens burned \\
inside; short of life, they sunk into hot fire.\evb\evg


\bvg\bva Full-rǿtt’s umb þetta; \hld\ fęrr ęngi svá síðan &
brúðr í brynju \hld\ brǿðra at hęfna; &
hón hęfir þriggja \hld\ þjóð-konunga &
\edtrans{ban-orð borit}{borne the bane-word}{\Bfootnote{\footnoteB{i.e. “she has caused the deaths of three great kings.” This expression and its Germanic and Indo-European relatives is discussed in detail in \textcite{Watkins1995}[417--422].}}}, \hld\ bjǫrt, áðr sylti.\eva

\bvb ’Tis told fully about this: henceforth none will go so, \\
a bride in byrnie her brothers to avenge. \\
She has of three great kings \\
borne the bane-word—bright woman—before she must die.\evb\evg


\bvg\bva Enn segir gløggra í Atlamálum inum grǿn-lenskum.\eva

\bvb Yet says it more clearly in the Greenlendish Speeches of Attle.\evb\evg

\sectionline
%
%	\include{books/Greenlendish Speeches of Attle.tex}%
	\bookStart{The Instigation of Guthrun}[Guðrúnarhvǫt]

\begin{flushright}%
\textbf{Dating} \parencite{Sapp2022}: early C11th (0.781)–late C11th (0.177)

\textbf{Meter:} \Fornyrdislag
\end{flushright}%

TODO: INTRODUCTION.

\sectionline

\bvg\bva Þá frá’k \alst{s}ęnnu \hld\ \alst{s}líðr-fęng-ligasta, &
\alst{t}rauð mǫ́l \alst{t}alit \hld\ af \alst{t}rega stórum, &
es \alst{h}arð-\alst{h}uguð \hld\ \alst{h}vatti at vígi &
\alst{g}rimmum orðum \hld\ \alst{G}uðrún sonu:\eva

\bvb This gibing I’ve found most sharpily caught— \\
reluctant speeches told from great grief— \\
when the hard-minded woman incited to war, \\
with fierce words, Guthrun, her sons:\evb\evg


\bvg\bva „Hví \alst{s}itið? \hld\ Hví \alst{s}ofið lífi? &
Hví \alst{t}regr-at ykkr \hld\ \alst{t}ęiti at mę́la? &
es \alst{Jǫ}rmunrekr \hld\ \alst{y}ðra systur, &
\alst{u}nga at \alst{a}ldri, \hld\ \alst{j}óm of traddi,\eva

\bvb “Why do ye sit? Why do ye sleep your life away? \\
Why troubles it not you two to speak merrily? \\
when Erminric has had your sister, \\
young of age, trampled by steeds,”\evb\evg


\bvg\bva \alst{h}vítum ok svǫrtum \hld\ á \alst{h}ęr-vegi &
\alst{g}rám, \alst{g}ang-tǫmum \hld\ \alst{G}otna hrossum.\eva

\bvb “by ones white and black on the war path; \\
by grey, pacing, Gotish horses.”\evb\evg


TODO: Missing verses.


\bvg\bva Hlę́jandi Guðrún \hld\ hvarf til skęmmu, &
kumbl konunga \hld\ ór kęrum valði, &
síðar brynjur \hld\ ok sonum fǿrði; &
hlóðusk móðgir \hld\ á mara bógu.\eva

\bvb ...\evb\evg


\bvg\bva Þá kvað þat Hamðir \hld\ inn hugum-stóri: &
Svá koma’k męirr aftr \hld\ móður at vitja &
Geir-Njǫrðr hniginn \hld\ á Goð-þjóðu &
at þú ęrfi \hld\ at ǫll oss drykkir, &
at Svanhildi \hld\ ok sonu þína.\eva

\bvb ...\evb\evg


\bvg\bva Guðrún grátandi, \hld\ Gjúka dóttir, &
gekk treg-liga \hld\ á tái sitja &
ok at tęlja, \hld\ tǫ́rug-hlýra,
móðug spjǫll \hld\ á margan veg:\eva

\bvb ...\evb\evg


\bvg\bva „Þrjá vissa’k \alst{ę}lda, \hld\ þrjá vissa’k \alst{a}rna, &
\alst{v}as’k þrimr \alst{v}erum \hld\ \alst{v}egin at húsi; &
\alst{ęi}nn vas mér Sigurðr \hld\ \alst{ǫ}llum bętri &
es \alst{b}rǿðr mínir \hld\ at \alst{b}ana urðu.\eva

\bvb “I’ve known three fires; I’ve known three hearths; \\
for three husbands I’ve been carried to the house. \\
With me was Siward alone better than all, \\
he whose bane my brothers became.\evb\evg




TODO: Bunch of verses.




\bvg\bva \alst{M}inns-tu, Sigurðr, \hld\ hvat vit \alst{m}ę́ltum &
þá’s vit á \alst{b}ęð \hld\ \alst{b}ę́ði sǫ́tum? &
at þú \alst{m}yndir \alst{m}ín \hld\ \alst{m}óðugr vitja, &
\alst{h}alr, ór \alst{h}ęlju, \hld\ en ek þín ór \alst{h}ęimi.\eva

\bvb Recallest thou, Siward, what we two spoke, \\
when on the bed we both did sit? \\
That thou wouldst me, O mighty man, \\
visit from Hell, and I thee from the world.\evb\evg


\bvg\bva Hlaðið ér, \alst{ja}rlar, \hld\ \alst{ęi}ki-kǫstinn, &
látið þann und \edtrans{\alst{h}i\emph{mn}i}{heaven}{\Afootnote{emend.; \emph{hilmi} ‘prince’ \Regius}} \hld\ \alst{h}ę́stan verða! &
Męgi \alst{b}ręnna \alst{b}rjóst \hld\ \alst{b}ǫlva-fullt ęldr &
umb hjarta [\dots] \hld\ þiðni sorgir!\eva

\bvb Load, ye earls, the oaken pile \ken{pyre}! \\
Let it beneath heaven become the highest! \\
May fire burn my curse-filled chest, \\
unto the heart \dots\ may the sorrows melt away!\evb\evg


\bvg\bva \alst{Jǫ}rlum \alst{ǫ}llum \hld\ \alst{ó}ðal batni, &
\alst{s}nótum ǫllum \hld\ \alst{s}org at minni &
at þetta \alst{t}reg-róf \hld\ of \alst{t}alit vę́ri.\eva

\bvb For all earls may patrimony improve; \\
for all ladies sorrow decrease, \\
as this grief-chain was recounted!\evb\evg

\sectionline
%
%	\include{books/Speeches of Hamthew.tex}%

\part{Other heroic poetry}
%	\include{books/Song of Grotte.tex}% Song of Grotte
	\bookStart{The Lay of Hildbrand}

\begin{flushright}%
Dating: C8th

Meter: Germanic alliterative meter%
\end{flushright}%

% Introduction

For the text of original poem I present the manuscript text with as few textual emendations as possible. As for the orthography, I have found it impossible to produce a normalizated without too heavily distorting the received text, being as it is, a blend of several dialects (one need only observe the treatment of the name Thedric, which appears thrice, and each time in a markedly different form). Apart from my typical practice of capitalizing proper names, marking prefixes with ⟨·⟩ and compounds with ⟨-⟩, and using acute accents to signify long vowels, grave accents to signify now-monophthongized original diphthongs, and overdots to mark nasal vowels, I have done the following changes in order to clarify etymological relationships and make the text somewhat more wieldy. Of these, TODO–TODO have also been noted in the apparatus where they occur:
\begin{enumerate}
  \item Consistently replaced \emph{ƿ} (wynn) and \emph{uu} with \emph{w}.
  \item Consistently replaced \emph{c} with \emph{k}.
  \item Replaced \emph{th} with \emph{þ}.
  \item Replaced \emph{e} with \emph{ę} when reflecting an original a-vowel affected by \emph{i}-mutation.
  \item Replaced \emph{ó} with \emph{ǫ́} where originally an \emph{a}.
  \item Restored inconsistent second consonant shift of \emph{b} to \emph{p}.
  \item Removed unetymological double \emph{nn}.
  \item Restored initial \emph{h-} where etymological and/or metrically required.
  \item Removed initial \emph{h-} unetymological and/or metrically deficient.
\end{enumerate}

The punctuation of the original, entirely consisting of interpuncts, at times representing line breaks and cæsuræ and at others sporadically placed, has not been retained.

Where they appear in cæsuræ, the words \emph{quad Hiltibrant} ‘Hildbrand quoth’ (found in ll. 30, 49, and 58) replace the usual interpunct. Due to their hypermetrical nature, I had originally planned to remove these, and instead indicate the speaker in the margins—but after comparison with various Norse stanzas (e.g. \Reginsmal\ 3, wherein the words \emph{kvað Loki} ‘Lock quoth’ appear in the stanza’s first cæsura), I have come to believe that these represent an ancient oral interjection, seemingly going back as far as the Migration Period (as it seems incredulous to think that the scribe of \HildMS\ should have influenced the four centuries younger scribe of \Regius\ in such a minor point.)

% Summary

The poet gives a very short formulaic introduction, from which we can tell that the beginning of the poem is preserved (1–2). Hildbrand and Hathbrand, father and son, arm and dress themselves before riding into battle, each the head of an opposing host (3–6). Hildbrand asks Hathbrand about his name and lineage, saying that he knows all noble genealogies (7–13). Hathbrand gives his name, and says that the old men of his tribe have told him that his father was Hildbrand, a brave warrior. He abandoned the newborn Hathbrand in order to serve Thedric in his fight against Edwaker, but this was a long time ago, and Hathbrand doubts that he is still alive (14–29). Realizing that he is facing his son, Hildbrand invokes God as witness, and as a token of loyalty offers Hathbrand a golden bigh which the Hunnish king had given him (30–35). Hathbrand exclaims that treasures must be won by struggle alone and harshly insults his father’s manhood: he calls him an old Hun, and accuses him of having survived to old age through treachery (36–41). Hathbrand then reveals that he has learned from sailors on the Mediterranean that Hildbrand is dead (42–44).

After this follows three short speeches by Hildbrand. The second one is certainly spoken by him, but the other two may be misplaced or misattributed. Hildbrand first reflects on his son’s prosperity, saying that he can tell from his clothes that he has a good lord, and has (unlike himself) not suffered an exile’s fate (45–48). He then calls on God, and laments that after thirty years of war he is now forced to fight against his own son; still, he tells Hathbrand that he should easily be able to kill such an old man as himself, if he has the strength to it (49–57). Lastly, he (or Hathbrand, if we choose to emend) says that only the most degenerate easterner would refuse the fight; when the fight is over, one of the two will win and take away the armour of the other (58–62).

The two men then throw their javelins, each of which gets stuck in the opposing shield, before rushing into each other, hacking away at their shields until they become worthless (63–68). The rest of the poem was continued on the now-lost, following page(s).

\sectionline

\bvg
\bva[0]Ik gi·hòrta dat sęggen &
dat sih \alst{u}r·hèttun \hld\ \alst{ae}non muotín: &
\alst{H}ilti-brant ęnti \alst{H}adu-brant \hld\ untar \alst{h}ęrjun twèm &
\alst{s}unu-fatar·ungo \hld\ iro \alst{s}aro rihtun &
\alst{g}arutun sé iro \alst{g}u̇d-hamun \hld\ \alst{g}urtun sih iro swert ana &
\alst{h}ęlidos ubar \edtext{\alst{h}ringa}{\Afootnote{\emph{ringa} \HildMS}} \hld\ dó sie to dero \alst{h}iltju ritun.\eva

\bvb[0]I heard it said, \\
that two contenders alone did meet: \\
Hildbrand and Hathbrand, under two hosts.\footnoteB{i.e. each man was a champion of his respective army.} \\
Son and father ordered their armour, \\
readied their war-cloths, girded their swords on, \\
the heroes over the mail-coats—when to that battle they rode.\evb
\evg


\bvg
\bva[0][6]\alst{H}ilti-brant \edtext{gi·mahalta}{\Afootnote{\emph{heribrantes sunu} ‘Harbrand’s son’ add. \HildMS}} \hld\ her was \alst{h}èróro man &
\alst{f}erahes \alst{f}rótóro \hld\ her \alst{f}rágén gi·stuont &
\alst{f}òhém wortum \hld\ \edtext{hwer}{\Afootnote{\emph{wer} \HildMS}} sín \alst{f}ater wári &
\alst{f}irjo in \alst{f}olkhe \hld\ {[...]} &
{[...]} \hld\ „eddo \edtext{hwe-líhhes}{\Afootnote{\emph{welihhes} \HildMS}} \alst{k}nuosles dú sís &
ibu dú mí \alst{è}nan sagés \hld\ ik mí de \alst{ǫ́}dre wèt &
\alst{kh}ind in \edtext{\alst{kh}unink-ríkhe}{\Afootnote{\emph{chunnincriche} \HildMS}} \hld\ \alst{kh}u̇d ist mín al irmin-deot“\eva

\bvb[0]Hildbrand spoke—he was the hoarier man, \\
wiser of life—he began to ask, \\
with few words, who his father might be, \\
of men in the troop, [...] \\
“or of which lineage thou be; \\
if thou tell me one, I the others will know, \\
O child, in the kingdom I know all great men.”\evb
\evg


\bvg
\bva[0][13]\alst{H}adu-brant gi·mahalta \hld\ \alst{H}ilti-brantes sunu &
\edtext{„dat sagetun mí \hld\ u̇sere liuti}{\lemma{dat \dots\ liuti}\Bfootnote{this l. breaks no rhythmic rules (cf. l. 42), but the needed alliteration is missing.}} &
\alst{a}lte anti fróte \hld\ dea \alst{è}rhina wárun &
dat \alst{H}iltibrant haetti mín fater \hld\ ih hęittu \alst{H}adubrant &
forn her \alst{ò}star \edtext{gi·węit}{\Afootnote{\emph{gihueit} \HildMS}} \hld\ flòh her \alst{Ò}t-akhres níd &
hina miti \alst{Þ}eot-ríhhe \hld\ ęnti sínero \alst{d}egano filu &
her fur-\alst{l}aet in \alst{l}ante \hld\ \alst{l}úttila sitten &
\edtext{\alst{b}rút}{\Afootnote{\emph{prut} \HildMS}} in \alst{b}úre \hld\ \alst{b}arn un·wahsan &
\alst{a}rbjo-laosa \hld\ \edtext{her raet}{\Afootnote{\emph{heraet} \HildMS}} \alst{ò}star hina &
det síd \alst{D}et-ríhhe \hld\ \alst{d}arba gi·stuontum &
\edtext{\alst{f}ateres}{\Afootnote{\emph{fatereres} \HildMS}} mínes \hld\ dat was só \alst{f}riunt-laos man &
her was \alst{Ò}t-akhre \hld\ \alst{u}m·met tirri &
\alst{d}egano \alst{d}ękhisto \hld\ unti \edtext{\alst{D}eot-ríkhhe}{\Afootnote{add. \emph{darba gistontun} \HildMS}} &
her was eo \alst{f}olkhes at ęnte \hld\ imo was eo \edtext{\alst{f}eheta}{\Afootnote{\emph{peheta} \HildMS}} ti leop &
\alst{kh}úd was her \hld\ \edtext{\alst{kh}óném}{\Afootnote{\emph{chonnem} \HildMS}} mannum &
ni wániu ih iu líb habbe.“\eva

\bvb[0]Hathbrand spoke, Hildbrand’s son: \\
“This told me \emph{our} people— \\
the old and wise, those who earlier lived— \\
that Hildbrand was called my father—I am called Hathbrand. \\
Long ago he departed to the east—he fled Edwaker’s hate— \\
away with Thedrich and his multitude of thanes. \\
He forsook in the land a little one to stay: \\
a bride in the bower, a bairn ungrown, \\
inheritance-less—he rode east thither, \\
at the time when Thedrich was in great need \\
of my father—that was so friendless a man! \\
He was to Edwaker immeasurably hostile; \\
the dearest of thanes under Thedrich. \\
He was always at the front of the troop; him did always the fight gladden; \\
known was he among keen men; \\
I ween not that he have life.”\evb
\evg


\bvg
\bva[0][29]„wèttu \alst{i}rmin-got {\small (quad Hiltibrant)} \alst{o}bana ab \edtext{hewane}{\Afootnote{\emph{heuane} \HildMS}} &
dat dú neo dana halt mit sus sippan man &
dink ni gi·lęitós“ &
\alst{w}ant her dó ar arme \hld\ \alst{w}untane bauga &
\alst{kh}ęisur·ingu gi·tán \hld\ so imo sie der \alst{kh}uning gap &
\alst{h}unjo truhtin \hld\ „dat ih dír it nú bí \alst{h}uldí gibu“\eva

\bvb[0]“I call on Ermin-god as witness, from above in heaven, \\
that thou never again with such a closely related man lead dispute.” \\
Unwound he then from his arm some twisted \inx[C]{bigh}[bighs], \\
made by a Cæsar’s man, which the king had given him, \\
the lord of the Huns—“This I now give thee for [thy] \inx[C]{holdness}.\footnoteB{The giving of \emph{bighs} (armlets, torcs) in exchange for loyalty among warriors is well attested; see Encyclopedia. This encounter is particularly reminiscent of \Harbardsljod\ 42.}”\evb
\evg


\bvg
\bva[0][35]\alst{H}adu-brant gi·mahalta \hld\ \alst{H}ilti-brantes sunu &
„mit \alst{g}èru skal man \hld\ \alst{g}eba in·fȧhan &
\alst{o}rt widar \alst{o}rte \hld\ {[...]} &
dú bist dir \alst{a}ltér hun \hld\ \alst{u}m-met spáhér &
\alst{sp}ęnis mih mit díném wortun \hld\ wili mih dínu \alst{sp}eru werpan &
\edtext{bist}{\Afootnote{\emph{pist} \HildMS}} \alst{a}l-só gi·\alst{a}ltét man \hld\ só dú èwín \alst{i}n-wit fórtós &
dat \alst{s}agetun mí \hld\ \alst{s}èo-lídante &
\alst{w}estar ubar \alst{W}ęntil-sèo \hld\ dat man \alst{w}ík fur·nam &
tòt ist \alst{H}ilti-brant \hld\ \alst{H}ęri-brantes suno“\eva

\bvb[0]Hathbrand spoke, Hildbrand’s son: \\
“With spear shall one earn gifts, \\
point against point!\footnoteB{This ancient mindset was codified by the Indians as part of the \emph{kṣatra-dharma}, the code of the Warrior (\emph{Kṣatriya}) caste, which explicitly forbade the Warriors from taking gifts. So in a part of the Mahabharata (12.192.73), a (Kṣatriya) King refuses a gift from a priest, saying that “it is the duty prescribed for a Kṣatriya that he must fight and protect (people). Kṣatriya are said to be the givers, then, how can I take (this) from you?” (translation by \textcite{Hara1974})} \\
Thou art, old Hun, immeasurably clever; \\
thou dost lure me with thy words, at me wilt thou hurl thy spear! \\
Thou art thus an aged man, since thou always deceit didst work.— \\
\emph{This} told me seafarers, \\
in the west o’er the Wendle-sea\footnoteB{The Mediterranean, the name refers to the Vandals in North Africa.}, that war took that man: \\
dead is Hildbrand, Harbrand’s son!”\evb
\evg


\bvg
\bva[0][44]\alst{H}ilti-brant gi·mahalta \hld\ \alst{H}ęri-brantes suno &
„wela gi·sihu ih in díném hrustim &
dat dú \alst{h}abés \alst{h}ème \hld\ \alst{h}èrron góten &
dat dú noh bí desemo \alst{r}íkhe \hld\ \alst{r}ekkhjo ni wurti“\eva

\bvb[0]Hildbrand spoke, Harbrand’s son: \\
“I see well on thy equipment, \\
that thou hast a good lord at home, \\
that thou yet from this realm art not become an exile.”\evb
\evg


\bvg
\bva[0][48]„\alst{w}elaga nú \alst{w}altant got {\small (quad Hiltibrant)} \alst{w}è-wurt skihit &
ih wallóta \alst{s}umaro ęnti wintro \hld\ \alst{s}ehs-tik ur lante &
dar man mih eo \alst{sk}ęrita \hld\ in folk \edtrans{\alst{sk}eotantero}{shooters}{\Bfootnote{Cf. \Beowulf\ 702, where the OE cognate \emph{sceótend} stands for warriors in general.}} &
só man mir at \alst{b}urk ènigeru \hld\ \alst{b}anun ni gi·fasta &
nú skal mih \alst{s}wásat khind \hld\ \alst{s}wertu hauwan &
\alst{b}retón mit sínu \alst{b}illiu \hld\ eddo ih imo ti \alst{b}anin werdan. &
Doh maht dú nú \alst{ao}d-líhho \hld\ ibu dir dín \alst{ę}llen taok &
in sus \alst{h}èremo man \hld\ \alst{h}rusti gi·winnan &
\alst{r}auba \edtext{bi·\alst{r}ahanen}{\Afootnote{\emph{bihrahanen} \HildMS}} \hld\ ibu dú dar èníg \alst{r}eht habés!“\eva

\bvb[0]“Well now, wielding God! the woeful weird\footnoteB{i.e. ‘(inexorable) course of events’, not the norn; cf. \emph{grimmar urðir} TODO.} comes to pass. \\
I roamed for sixty summers and winters\footnoteB{i.e. thirty years. Hathbrand is thus exactly thirty years old, while Hildbrand is in his fifties or sixties.} out of the land, \\
where I was always placed in the troop of shooters, \\
as at no fortress my bane was fastened.— \\
Now shall my own child hew at me with sword; \\
beat down with his blade, or I become his bane. \\
Yet mayst thou now easily—if thy zeal avail thee— \\
from such a hoary man win the equipment; \\
bear away the booty—if thou thereto have any right!”\evb
\evg


\bvg
\bva[0][57]„der sí doh nú \alst{a}rgósto {\small (quad Hiltibrant)} \alst{ò}star-liuto &
der dir nú \alst{w}íges \alst{w}arne \hld\ nú dih es só \alst{w}el lustit &
gu̇dea gi·\alst{m}ęinun \hld\ niuse de \alst{m}ótti &
\edtext{hwędar}{\Afootnote{\emph{werdar} \HildMS}} sih \edtext{\alst{h}iutu dèro}{\Afootnote{metr. emend.; \emph{dero hiutu} \HildMS}} \alst{h}regilo \hld\ \edtext{\alst{h}ruomen}{\Afootnote{\emph{hrumen} \HildMS}} muotti &
\edtext{eddo}{\Afootnote{\emph{erdo} \HildMS}} desero \alst{b}runnóno \hld\ \alst{b}èdero waltan!“\eva

\bvb[0]“He be now the weakest of the eastern peoples, \\
who should refuse thee the fight, when thou so greatly cravest \\
to struggle together—try he who might, \\
which one of us today of these garments may boast, \\
or both of these byrnies may wield!”\evb
\evg


\bvg
\bva[0][62]Dó léttun sé \alst{ae}rist \hld\ \alst{a}skkim skrítan &
\alst{sk}arpén \alst{sk}úrim \hld\ dat in dem \alst{sk}iltim stónt &
dó \alst{st}óptun to·samane \hld\ \alst{st}aim-bort \edtext{hludun}{\Afootnote{\emph{chludun} \HildMS}} &
\alst{h}ewun harm-líkko \hld\ \alst{h}wítte skilti &
unti imo iro \alst{l}intún \hld\ \alst{l}úttilo wurtun &
gi·\alst{w}igan miti \alst{w}ábnum \hld\ [...]\eva

\bvb[0]Then let they first their ash-spears glide, \\
in sharp showers, that in the shields they stuck. \\
Then charged they into each other—the war-boards \ken{shields} resounded— \\
struck they bitterly the white shields, \\
until for them their lindens \ken{shields} became little, \\
worn down by the weapons, [...].\evb
\evg
%
%	\include{books/Fight at Finnsbury.tex}%
%	\include{books/Beewolf.tex}% Probably not happening.

\part{Galders: Poetic Charms, Spells, and Curses}

Under this section are gathered sundry \inx[C]{galder}[galders] (metrical magic charms) attested in Old Germanic languages.  I have only included those with clear Heathen or otherwise traditional elements (especially certain poetic formulae known from older texts).  Thoroughly Christian prayers are found below under “Poetry on Christian Subjects”.


\chapter{Continental Germanic galders}

\section{The Two Merseburg galders}\chapterStart{}

\begin{flushright}%
\textbf{Dating:} C9th–10th

\textbf{Meter:} \Fornyrdislag, \Galdralag%
\end{flushright}

These two galders, preserved in a manuscript (TODO) are some of the only surviving examples of genuine Heathen galders from the continent.  Both share a common two-part structure, each beginning with an \emph{historiola}—a “historical” account describing the successful effects of the galder in the mythic past—followed by an \emph{imperative} commanding that the willed magic effect take place in the present.

The first galder begins with the historiola describing a group of supernatural women in the midst of a battle, affecting its outcome by fastening or loosening fetters.  The imperative then commands that some fetters in the present be destroyed, so that captive(s) may escape.

The second galder begins with the historiola describing a group of Gods riding through the woods.  Among them is \inx[P]{Balder}, whose young foal sprains its foot.  Three Gods—the otherwise unknown goddess \inx[P]{Sithguth}, the goddess \inx[P]{Sun}, the god \inx[P]{Weden}—in turn chant a healing galder over it.  The imperative—apparently the galder sung by Weden—then commands that a present sprain be healed.

\sectionline

\bvg\bva%
Ęiris \alst{s}ázun idisi \hld\ \alst{s}ázun hera duo der; &
suma \alst{h}apt \alst{h}ęptidun \hld\ suma \alst{h}ęri lęzidun &
suma \alst{k}lubodun \hld\ umbi \edtrans{\alst{k}uonjo-widi}{chains}{\Bfootnote{A rare word apparently cognate with Gothic \emph{kuna-wida} ‘\emph{Fessel}; \textgreek{ἅλῠσῐς}’ \parencite[76]{Streitberg}, although the first element is not formally identical.}} &
\alst{i}n-sprink hapt-bandun \hld\ \alst{i}n-var vígandun &
\edtext{.H.}{\Bfootnote{The meaning of this letter, which is very clear and written in the same hand as the galders, is uncertain.  To me the most convincing suggestion is that it be read as \emph{.N.}, short for Latin \emph{nomen} ‘name’, presumably the name of the person whom the singer wishes to free from the fetters.}}\eva

\bvb Of yore sat dises, sat here, then there: \\
some fastened fetters, some hindered armies, \\
some cut chains asunder.— \\
Destroy the fetter-bonds, lead the way from the foes! \\
.H.\evb\evg


\bvg\bva%
\alst{Ph}ol ęnde Wuodan \hld\ \alst{v}uorun zi holza &
dú wart demo Balderes \alst{v}olon \hld\ sín \alst{v}uoz bi·ręnkit &
þú \edtrans{bi·guol en}{begaled him}{\Bfootnote{Sang a \inx[C]{galder} over the horse, the third past singular of \emph{bi·galan} ‘begale’, the transitive of \emph{galan} ‘gale, sing a galder’.  Cf. \Oddrunargratr\ TODO, where a midwife “gales” “bitter galders” over a birthing mother.}} \alst{S}inhtgunt \hld\ \alst{S}unna era swister &
þú bi·guol en \alst{F}rija \hld\ \alst{V}olla era swister &
þú bi·guol en \alst{W}uodan \hld\ só hé \alst{w}ola konda: &
„Só-se \alst{b}ên-ręnkí \hld\ só-se \alst{b}luot-ręnkí \hld\ só-se lidi-ręnkí &
\ind \alst{b}ên zi \alst{b}êna &
\ind \alst{b}luot zi \alst{b}luoda &
\alst{l}id zi ge·\alst{l}iden \hld\ só-se ge·\alst{l}ímida sín!“\eva

\bvb Phol and Weden journeyed in the woods; \\
then was the foot of Balder’s foal sprained. \\
Then \inx[P]{Sithguth} \inx[C]{begale}[begaled] him—\inx[P]{Sun} her sister; \\
then \inx[P]{Frie} begaled him—\inx[P]{Full} her sister; \\
then Weden begaled him, as well he knew: \\
“Like bone-sprain, like blood-sprain, like joint-sprain! \\
\ind Bone to bone, \\
\ind blood to blood, \\
joint to joints, like they were glued together!”\evb\evg

\sectionline

\bookStart{Against wyrms}[Contra vermes]

\begin{flushright}%
\textbf{Dating:} ?

\textbf{Meter:} \Fornyrdislag%para
\end{flushright}%

A manuscript charm against wyrms located in the bone-marrow, probably thought to cause aching.  The galder calls upon a chief worm, Nesse, and its nine offspring, to depart from the patient.  It lays out a path for the worms, who are to leave the sufferer’s body and instead go into an arrow or sharp point (\emph{strála}), probably a ritual implement used to pierce the affect area.

The structure “Go from X to Y, from Y to Z” may be very old, as it is also found in Romani charms collected by \textcite[27,28,95]{Leland1891}  The charm on p. 95 is also against wyrms.  Like in our galder the wyrms (\emph{kirmora}, from Sanskrit \emph{kŕ̥mi}, which is probably related to Germanic \emph{*wurmiz}, although the difference in the initial consonant is unusual—perhaps a taboo formation?) are to leave the body and instead go into the ritual implement, in the Gypsy charm an ointment.  I take me the freedom to reproduce this charm in full, with Leland’s introduction and translation:

“Before sunrise wolf’s milk (Wolfsmilch, rukeskro tçud) is collected, mixed with salt, garlic, and water, put into a pot, and boiled down to a brew. With a part of this the afflicted spot is rubbed, the rest is thrown into a brook, with the words:—

\begin{verse}
\emph{Kirmora jánen ándre tçud
Andrál tçud, andré sir
Andrál sir, andré páñi,
Panensá kiyá dádeske,
Kiyá Niváseske
Pçándel tumen shelehá
Eñávárdesh teñá!}
\end{verse}

\begin{verse}
‘Worms go in the milk,
From the milk into the garlic,
From the garlic into the water,
With the water to (your) father,
To the Nivasi,
He shall bind you with a rope,
Ninety-nine (yards long).’”
\end{verse}

\sectionline

\bvg\bva[]Gang út, \edtrans{\alst{N}esso}{Nesse}{\Bfootnote{The \emph{naming} of the daemon or being which is to be excised is common in ancient magic, including several other galders edited here.  The idea is that knowledge of the name of the entity gives the healer power over it.}}, \hld\ mid \alst{n}igun \alst{n}essi-klínon, &
ut fana þemo marge an þat \alst{b}ên, \hld\ fan þemo \alst{b}êne an þat flęsg, &
ut fan þemo flęsgke an þia \alst{h}úd, \hld\ ut fan þera \alst{h}úd an þesa strála. &
Drohtin, werþe só.\eva

\bvb Go out, O Nesse, with the nine small Nesses! \\
Out from the marrow into the bone, from the bone into the flesh, \\
out from the flesh into the skin, out from the skin into this arrow. \\
Lord, may it be so.\evb\evg

\sectionline



\chapter{Old English galders}

\section{Against Swarm (\emph{Wið ymbe})}

\begin{flushright}%
\textbf{Dating:} ?

\textbf{Meter:} \Fornyrdislag%para
\end{flushright}%

TODO. That bees are called “victory-wives” is interesting.

\sectionline

\bpg\bpa Wið ymbe nim eorþan, ofer·weorp mid þínre swíþran handa under þínum swíþran fét, and cwet:\epa

\bpb Against a swarm take earth, throw it with thy right hand under thy right foot, and say:\epb\epg


\bvg\bva \alst{F}ó ic under \alst{f}ót, \hld\ \alst{f}unde ic hit. &
Hwæt \alst{eo}rðe mæg \hld\ wið \alst{ea}lra wihta ge·hwilce &
and wið \alst{a}ndan \hld\ and wið \alst{æ}minde &
and wið \edtrans{þá \alst{m}icelan \hld\ \alst{m}annes tungan}{that mighty tongue of man}{\Bfootnote{The tongue is surely here standing in for “speech”, specifically galder; i.e., if the swarming of the bees were caused by an enemy’s cursing, the earth will disarm it.}}.\eva

\bvb I catch under foot, I may have found \emph{it}. \\
How, earth works against everywhich wight \\
and against mischief and against neglect \\
and against that mighty tongue of man.\evb\evg


\bpg\bpa And wiððon \edtrans{for·weorp ofer greót}{throw the grit over}{\Bfootnote{i.e. “throw the earth over the swarm”.}}, þonne hí swirman, and cweð:\epa

\bpb And with that throw the grit over, when they swarm, and say:\epb\epg


\bvg\bva \alst{S}itte gé, \alst{s}ige-wíf, \hld\ \alst{s}ígað to eorþan! &
Næfre gé \alst{w}ilde \hld\ to \alst{w}uda fleogan. &
Beo gé swá ge·\alst{m}indige \hld\ \alst{m}ínes gódes, &
swá bið \alst{m}anna ge·hwilc \hld\ \alst{m}etes and éþeles.\eva

\bvb Sit ye, victory-wives; sink to the earth! \\
Never ye would fly to the woods. \\
Be ye so mindful of \emph{my} good, \\
like is every man of his measure and homestead.\evb\evg

\sectionline

\section{Against a Dwarf (\emph{Wið dweorh})}\chapterStart{}
\setBookCode{WidDweorh}

\begin{flushright}%
\textbf{Dating:} TODO

\textbf{Meter:} \Fornyrdislag%
\end{flushright}

\subsection{Introduction}

TODO: Introduction.

\sectionline

\subsection{Text}

\bpg\bpa Mann sceal niman \emph{seofon} lytle of-lætan swylce mann mid ofrað, ond wrítan þás naman on ælcre oflætan: Maximianus, Malchus, Johannes, Martinianus, Dionisius, Constantinus, Serafion.  Þænne eft þæt galdor þæt hér æfter cweð[eð] mann sceal singan, ærest on þæt wynstre éare, þænne on þæt swíðre éare, þænne búfan þæs mannes moldan; ond gá þænne ân mæden-mann tó, ond hó hit ǫn his sweoran, ond dó mann swá þrý dagas.  Him bið sóna sél.\epa

\bpb One shall take seven small wafers, such as one offers [during the Mass], and write these names on each wafer: Maximianus, Malchus, Johannes, Martinianus, Dionysius, Constantinus, Seraphion.  After that shall one sing this galder which is henceforth said; first into the left ear, then into the right ear, then over the man’s head; and thereafter a maiden go forth, and hang it on his neck; and one do so for three days.  He will soon be well.\epb\epg


\bvg\bva%
Hér cóm \alst{i}n·gangan \hld\ \alst{i}n·spiden wiht, &
hæfde him his \alst{h}aman ǫn \alst{h}anda; \hld\ cwæð þæt þú his \alst{h}æncgest wǽre, &
\alst{l}ęgeþe þé his téage \emph{ǫ}n sweoran; \hld\ ǫn·gunnan him ǫf þæm \alst{l}ande líðan. &
Sóna swá hý ǫf þæm \alst{l}ande cóman \hld\ þá ǫn·gunnan him þá \emph{\alst{l}eomu} cólian.— &
Þá cóm in·gangan \hld\ déores sweostar; &
þá ge·\alst{æ}ndode héo \hld\ ond \alst{â}ðas swór, &
þæt næfre þis þæm \alst{a}dlegan \hld\ \emph{\alst{e}gl}ian ne móste &
né þæm þe þis \alst{g}aldor \hld\ be·\alst{g}ýtan mihte &
oððe þe þis \alst{g}aldor \hld\ on·\alst{g}alan cu̇ðe. &
Amen fiað.\eva

\bvb Here an inspiden wight came walking in, \\
had his harness in his hands, said that thou wert his horse, \\
laid his reins on thy neck; they began to ride away from the land. \\
As soon as they came away from the land then they began to cool limbs. \\
Then the beast’s sister came walking in; \\
then she made an end to it and swore oaths \\
that this never should torment the ailing man, \\
nor him who this galder might get, \\
nor whomever this galder could gale. \\
Amen, let it be.\evb\evg

\sectionline

\section{Against a Sudden Stitch (\emph{Wið fǽr-stice})}\chapterStart{}

\begin{flushright}%
\textbf{Dating:} ?

\textbf{Meter:} \Fornyrdislag%para
\end{flushright}%

Attested in \Lacnunga.

\sectionline

\bvg\bva \alst{H}lúde wǽran hý, lá, \alst{h}lúde, \hld\ þá hý ofer þone \alst{h}lǽw ridan, &
wǽran \alst{â}n-móde, \hld\ þá hý \alst{o}fer land ridan. &
Scyld þú þé nú, þú þysne \alst{n}íð \hld\ ge·\alst{n}esan móte. &
\alst{Ú}t, lýtel spere, \hld\ gif hér \alst{i}nne síe!\eva

\bvb Loud were they, lo, loud, when they rode over that mound; \\
they were steadfast, when they rode over land. \\
Shield thyself now; thou mayst escape this evil! \\
Out little spear, if here within it be!\evb\evg


\bvg\bva Stód under \alst{l}inde, \hld\ under \alst{l}eohtum scylde, &
þær þá \alst{m}ihtigan wíf \hld\ hýra \alst{m}ægen be·rǽddon &
and hý \alst{g}yllende \hld\ \alst{g}âras sændan; &
ic him \alst{ó}ðerne \hld\ \alst{e}ft wille sændan, &
\alst{f}léogende \alst{f}lâne \hld\ \alst{f}orane tó·géanes. &
\alst{Ú}t, lytel spere, \hld\ gif hit her \alst{i}nne sý!\eva

\bvb Stood under the linden \ken{shield}—under the light shield— \\
where those mighty wives their might arrayed, \\
and they yelling spears did send. \\
To them another [projectile] will I send back: \\
a flying arrow, aimed against [them]. \\
Out little spear, if here within it be!\evb\evg


\bvg\bva \alst{S}æt \alst{s}mið, \hld\ \alst{s}loh seax, &
lytel \alst{í}serna, \hld\ \alst{w}und swíðe. &
\alst{Ú}t, lytel spere, \hld\ gif her \alst{i}nne sý!\eva

\bvb Sat the smith, struck the sax: \\
a little iron-thing—a great wound. \\
Out little spear, if here within it be!\evb\evg


\bvg\bva \alst{S}yx \alst{s}miðas \alst{s}ætan, &
\alst{w}æl-spera \alst{w}orhtan. &
\alst{Ú}t, spere, \hld\ næs \alst{i}n, spere! &
Gif her \alst{i}nne sý \hld\ \alst{í}senes dǽl, &
\alst{h}æg-tessan ge·weorc, \hld\ \alst{h}it sceal ge·myltan.\eva

\bvb Six smiths sat, \\
wrought slaughter-spears. \\
Out, spear! Be not in, spear! \\
If here within be a part of iron, \\
the work of a \inx[C]{hag-tess}—\emph{it} shall melt!\evb\evg


\bvg\bva Gif þú wǽre on \alst{f}ell scoten \hld\ oððe wǽre on \alst{f}læsc scoten &
oððe wǽre on blód scoten \hld\ [...] &
oððe wǽre on \alst{l}ið scoten, \hld\ næfre ne sý þín \alst{l}íf atæsed;\eva

\bvb If thou wert shot in the skin, or wert shot in the flesh, \\
or wert shot in the blood, [...], \\
or wert shot in the limb—never be thy life injured.\evb\evg


\bvg\bva gif hit wǽre \alst{ė}sa ge·scot \hld\ oððe hit wǽre \alst{y}lfa ge·scot &
oððe hit wǽre \alst{h}æg-tessan ge·scot, \hld\ nú ic wille þín \alst{h}elpan: &
þis þé tó bóte \alst{ė}sa ge·scotes, \hld\ þis þé tó bóte \alst{y}lfa ge·scotes, &
þis þé tó bóte \alst{h}æg-tessan ge·scotes; \hld\ ic þín wille \alst{h}elpan.\eva

\bvb If it were Eese-shot, or it were Elf-shot,\footnoteB{Formulaic; see \inx[F]{Eese and Elves}. That they are held in the same category as the hag-tess—a witch—indicates Christian influence. Among the Germanic peoples the elves and Eese were originally beneficial, as seen by numerous names like Alfred (OE \emph{Ęlf-réd} ‘Elf-counsel’), Oswald (OE \emph{Ós-weald} ‘Os-power’), Elfwin (Lomb. \emph{Alb-oin} ‘Elf-friend’), Oshelm (Lomb. \emph{Anselm} ‘Os-helmet’).}  \\
or it were Hag-tess-shot—now I will help thee! \\
This for thee as cure against Eese-shot; this for thee as cure against Elf-shot;  \\
this for thee as cure against Hag-tess-shot—I will help thee!\evb\evg


\bvg\bva \alst{F}leo þær on \hld\ \alst{f}yrgen-hæfde! &
\alst{H}âl wes-tu, \hld\ \alst{h}elpe þín drihten! &
Nim þonne þæt seax, \hld\ ado on wætan.\eva

\bvb TODO. \\
Be thou hale, may the Lord help thee.\evb\evg

\sectionline

\section{The Nine Herbs galder}\chapterStart{}
\setBookCode{NineHerbs}

\begin{flushright}%
\textbf{Dating:} ?

\textbf{Meter:} \Fornyrdislag%para
\end{flushright}%

\subsection{Introduction}

TODO: introduction

\sectionline

\subsection{Text}

\bvg\bva%
Ge·\alst{m}yne ðú \alst{m}ug-wyrt \hld\ hwæt þú á·\alst{m}eldodest &
hwæt þu \alst{r}enadest \hld\ æt \alst{R}egen-melde?\eva

\bvb Rememberest thou, Mugwort, what thou didst declare, \\
what thou didst arrange at Reinmeld?\evb\evg


\bvg
\bva Una þú hâttest \hld\ yldost wyrta &
þú miht \edtext{wið III \hld\ and wið XXX}{\lemma{wið III and wið XXX ‘against three and against thirty’}\Bfootnote{I.e. ‘against a great number of foemen’.  ‘Three and thirty’ is formulaic and appears in many later English and Scottish folk-songs, viz. Child 4EFG, 18B, 20C, 30, 53BCDEIKM, 63EFH, 73I, 97AC, 100AG, 110BGH, 156G, 185A, 187A, 187C, 190A, 192A, 193B, 203C, 211A, 217GHLN, 244A, 268A, 269C, 281ABC.  Things described include horses, heads of cattle, warriors, days, years, winters.}} &
þú miht wiþ attre \hld\ and wið on·flyge &
þú miht wiþ þâm lâþan \hld\ ðe geond lond færð\eva

\bvb Un thou art called, the oldest of worts; \\
thou availest against three and against thirty; \\
thou availest against the venom and against the onflier; \\
thou availest against the loathsome one that journeys through the lands.\evb\evg


\bvg
\bva + Ond þú weg·bráde \hld\ wyrta módor &
éastan opene \hld\ innan mihtigu &
ofer ðy cræte curran \hld\ ofer ðy cwéne reodan &
\ind ofer ðy brýde brýodedon &
\ind ofer ðy fearras fnærdon.\eva

\bvb And thou, Waybroad, mother of worts, \\
open from the east, mighty from within. \\
Over thee TODO.\evb\evg


\bvg
\bva Eallum þu þon wið·stóde \hld\ and wið·stunedest &
swá ðú wið·stonde attre \hld\ and on·flyge &
and þǽm lâðan \hld\ þe geond lond fereð.\eva

\bvb Them all didst thou then withstand, and didst stop; \\
so mayst thou withstand the venom and the onflier, \\
and the loathsome one that journeys through the lands.\evb\evg


\bvg
\bva Stune hætte þéos wyrt, \hld\ héo on stâne ge·weox &
stond héo wið attre, \hld\ stunað héo wærce &
Stiðe héo hatte, \hld\ wið·stunað héo attre &
wreceð héo wrâðan, \hld\ weorpeð út attor.\eva

\bvb Stun is this wort called, she grew on stone; \\
she withstands venom, she stops aches. \\
Stithe is she called, she stops the venom; \\
she drives away the wroth one, casts out the venom.\evb\evg


\bvg
\bva + Þis is séo wyrt \hld\ séo wiþ wyrm ge·feaht &
þéos mæg wið attre, \hld\ héo mæg wið on·flyge; &
héo mæg wið ðâm lâþan \hld\ ðe geond lond fereþ.\eva

\bvb This is the wort that fought against the Wyrm; \\
this one avails against the venom, she avails against the onflier; \\
she avails against the loathsome one that journeys through the lands.\evb\evg


\bvg
\bva Fleoh þú nú attor-lâðe, \hld\ séo lǽsse ðá mâran &
séo mâre þá lǽssan, \hld\ oððæt him beigra bót sý!\eva

\bvb TODO\evb\evg


\bvg
\bva Ge·myne þú, mægðe,\hld\ hwæt þú á·meldodest &
hwæt ðú ge·ændadest \hld\ æt Alor-forda &
þæt nǽfre for ge·floge \hld\ feorh ne ge·sealde &
syþðan him mǫn mægðan \hld\ tú mete ge·gyrede\eva

\bvb TODO\evb\evg


\bvg
\bva Þis is séo wyrt \hld\ ðe wer-gulu hatte &
ðás on·sænde seolh \hld\ ofer sǽs hrygc &
ondan attres \hld\ óþres tó bóte\eva

\bvb TODO\evb\evg


\bvg
\bva Ðás VIIII magon \hld\ wið nygon attrum.\eva

\bvb These nine avail against nine venoms.\evb\evg


\bvg
\bva + Wyrm cóm snícan, \hld\ to·slât hé man &
ðá ge·nam Wóden \hld\ VIIII wuldor-tânas &
slóh ðá þá nǽddran \hld\ þæt héo on VIIII tó·fléah &
Þǽr ge·ændade æppel \hld\ and attor &
þæt héo nǽfre ne wolde \hld\ on hús búgan.\eva

\bvb A \inx[C]{Wyrm} came crawling; he tore apart a man. \\
Then took Weden nine glory-twigs, \\
slew then that adder, that it sprung into nine [parts]. \\
There ended apple and venom, \\
that she would never wish to enter a house.\evb\evg


\bvg
\bva + Fille and finule, \hld\ fela-mihtigu twá &
þá wyrte ge·sceop \hld\ wítig drihten &
hâlig on heofonum, \hld\ þá hé hongode; &
sętte and sęnde \hld\ on VII worulde &
earmum and éadigum \hld\ eallum tó bóte\eva

\bvb Fill and Fennel, the many-mighty two; \\
those worts the wise lord shaped, \\
holy in heaven when he hung. \\
He set and sent them into seven worlds, \\
for wretched men and for wealthy, for all men as a cure.\evb\evg


\bvg
\bva \alst{St}ǫnd héo wið wærce, \hld\ \alst{st}unað héo wið attre &
séo mæg wið III \hld\ \emph{and} wið XXX &
wið \emph{\alst{f}éondes} hǫnd \hld\ and wið \alst{f}ǽr-bregde &
wið \alst{m}alscrunge \hld\ \alst{m}ânra wihta\eva

\bvb She stands against ache, she stands against venom;
she avails against three and against thirty;
against \evb\evg


\bvg
\bva + Nu magon þás \emph{nygon} \alst{w}yrta \hld\ wið nygon \alst{w}uldor-ge·flogenum &
wið \emph{nygon} \alst{a}ttrum \hld\ and wið nygon \alst{o}n·flygnum &
wið ðý \alst{r}éadan attre, \hld\ wið ðý \alst{r}unlan attre &
wið ðý \alst{h}witan attre, \hld\ wið ðý \emph{\alst{h}æwe}nan attre &
wið ðý \alst{g}eolwan attre, \hld\ wið ðý \alst{g}rénan attre &
wið ðý \alst{w}onnan attre, \hld\ wið ðý \alst{w}edenan attre &
wið ðý \alst{b}rúnan attre, \hld\ wið ðý \alst{b}asewan attre &
wið \alst{w}yrm-ge·blæd, \hld\ wið \alst{w}æter-ge·blæd &
wið \alst{þ}orn-ge·blæd, \hld\ wið \alst{þ}ystel-ge·blæd &
wið \alst{ý}s-ge·blæd, \hld\ wið \alst{a}ttor-ge·blæd\eva

\bvb Now these nine worts avail against glory-onfliers: \\
against nine venoms and against nine onfliers; \\
against the red venom; against the TODO venom; \\
against the white venom; against the TODO venom; \\
against the yellow venom; against the green venom; \\
against the TODO venom; against the TODO venom; \\
against the brown venom; against the TODO venom; \\
against worm-TODO; against water-TODO; \\
against thorn-TODO; against thistle-TODO; \\
against ice-TODO; against venom-TODO.\evb\evg


\bvg
\bva Gif ænig \alst{a}ttor cume \hld\ \alst{éa}stan fleógan &
oððe ǽnig norðan cume &
oððe ǽnig \alst{w}estan \hld\ ofer \alst{w}er-þeóde\eva

\bvb If any venom should come flying from the east; \\
or any come from the north; \\
or any from the west, over mankind.\evb\evg


\bvg
\bva + Críst stód ofer \alst{á}dle \hld\ \alst{ǽ}ngan cundes &
Ic \alst{â}na wât \hld\ éa rinnende &
þǽr þá \alst{n}ygon \alst{n}ǽdran \hld\ \alst{n}éan be·healdað\eva

\bvb Christ stood over TODO; \\
I know one river running; \\
there the nine adders TODO.\evb\evg


\bvg
\bva Motan ealle \alst{w}éoda \hld\ nu \alst{w}yrtum á·springan &
\alst{s}ǽs tó·\alst{s}lúpan, \hld\ eal \alst{s}ealt wæter &
ðonne ic \alst{þ}is attor \hld\ of \alst{þ}é ge·bláwe.\eva

\bvb TODO\evb\evg


\bpg\bpa Mucgwyrt, weg-brade þe eastan open sy, lombes-cyrse, attor-laðan, mageðan, netelan, wudu-sur-æppel, fille and finul, ealde sapan. Ge·wyrc ða wyrta to duste, mængc wiþ þa sapan and wiþ þæs æpples gor. Wyrc slypan of wætere and of axsan, ge·nim finol, wyl on þære slyppan and beþe mid æggemongc, þonne he þa sealfe on do, ge ær ge æfter. Sing þæt galdor on æcre þara wyrta, :III: ær he hy wyrce and on þone æppel eal-swa; ond singe þon męn in þǫne mu̇ð and in þá éaran búta and on ðá wunde þæt ilce gealdor, ær he þá sealfe on dó.\epa

\bpb TODO.\epb\epg

\sectionline



\chapter{Old Norse galders}

\section{Ribe galder stick (DR EM85;493)}\chapterStart{}
\setBookCode{RibeStick}

\begin{flushright}%
\textbf{Dating:} Mediæval.%TODO

\textbf{Meter:} \Fornyrdislag, \Galdralag%para
\end{flushright}%

A wooden stick from the Danish city of Ribe.  The galder is syncretic and contains numerous pre-Christian elements in a Christian(ised) context.

The inscription may be conveniently divided into four parts.  Part one (ll. 1–4) contains an introductory prayer where the healer asks for the aid of natural forces (Earth, Up-heaven and the Sun) and Christian divinitities (God and Saint Mary) so that the healing may be successful.  Part two (ll. 5–8) ritually exorcises any sickness which may have entered any part of the body.  Part three (ll. 9–14) apparently warns the addressee that they will be haunted by “nine needs” (an old Heathen formula; see note) until they say the charm.  Part four (ll. 15, which is probably prose) gives the personal name “Bonde”, perhaps the addressee, and concludes with an “Amen”.

\sectionline

\bvg\bva%
\alst{Jo}rð bið ak varðę \hld\ ok \alst{u}p-himęn &
\alst{s}ól ok \alst{s}antę María \hld\ ok \alst{s}alfęn Guð dróttęn &
þęt hann \alst{l}ę́ mik \alst{l}ę́knęs-hand \hld\ ok \alst{l}yf-tungę &
at lyfę \alst{b}ifjandę \hld\ þęr \alst{b}ótę þarf.\eva

\bvb I pray Earth to protect and Up-heaven, \\
the Sun and Saint Mary, and the very Lord God, \\
that he lend me a leecher’s hand and medicine-tongue, \\
as medicine for the trembler who needs the cure.\evb\evg


\bvg\bva%
\ind Ór \alst{b}ak ok ór \alst{b}ryst &
\ind ór \alst{l}íkę ok ór \alst{l}im &
\ind ór \alst{ø̂}vęn ok ór \alst{ø̂}ręn &
\ind ór \alst{a}llę þé þęr \alst{i}llt kann í \alst{a}t kumę.\eva

\bvb Out of back and out of breast! \\
Out of body and out of limb! \\
Out of eyes and out of ears! \\
Out of everything, where evil which might come in!\evb\evg


\bvg\bva%
Svart hêtęr \alst{st}ênn \hld\ hann \alst{st}ę́r í hafę útę, &
\ind þęr liggęr á þé \alst{n}í\emph{u} \alst{n}auðęr; &
\ind þę́r skulę hvęrki \alst{s}ǿtęn \alst{s}ofę; &
\ind ęð \alst{v}armęn \alst{v}akę; &
\ind førr ęn þú þęssa \alst{b}ót \alst{b}iðęr,
\ind þęr \alst{a}k \alst{o}rð at kvę́ðę.\eva

\bvb Swart is a stone called; it stands out in the ocean. \\
\ind There lie on it nine needs; \\
\ind they will neither sleep sweetly \\
\ind nor wake warmly, \\
\ind until thou prayest this cure \\
\ind to which I have given the words.\evb\evg

\sectionline

\bookStart{The Canterbury Galder}

\begin{flushright}%
Dating: c. 1075

Meter: \Fornyrdislag%
\end{flushright}

This Old Norse galder is found in the Anglo-Saxon manuscript Cotton Caligula A XV.  It runs across the bottom margin of the two facing pages 123v and 124r and is written in very clear runes of the Wiking age long-stave type.  One rune, viz. \textbf{g} in \textbf{vigi} \emph{vegi} ‘smite’ is stung.  The inscription has no word separators.

The galder is of the same type as the two from Sigtuna (U Fv1933;134, U NOR1998;25) and clearly intended for healing; it ends with \emph{viðr áðra-vari} ‘against pus of veins’, and this is probably a declaration of purpose.

\sectionline

\bvg\bva[] Gyrils sár-þvara! &
\alst{F}ar-ðu nú, \hld\ \alst{f}undinn es-tu! &
\alst{Þ}órr vegi \alst{þ}ik \hld\ \alst{þ}ursa dróttinn! &
Jórils sár-þvara. &
Viðr áðra-vari.\eva

\bvb O Gyrel’s wound-TODO! \\
Go thou now, thou art found! \\
May Thunder smite thee, O lord of Thurses! \\
O Erel’s wound-causer. \\
Against pus of veins.\evb\evg

\bookStart{Sigtuna Rib}[U NOR1998;25]

\begin{flushright}%
\textbf{Dating:} c. 1100

\textbf{Meter:} \Fornyrdislag%
\end{flushright}

TODO: Introduction.

\sectionline

\bvg\bva[] Jórils \alst{v}rið, \dots\ \alst{v}aksna úr Króki! &
\alst{B}att han riðu \hld\ \alst{b}arði hann riðu, &
auk \alst{s}íða \alst{s}arð \hld\ \alst{s}ára rann. &
Vara hafir \alst{f}ullt \alst{f}engit; \hld\ \alst{f}lý braut, riða!\eva

\bvb O Erel’s trembling, grow out of Crook! \\
He bound the fever; he beat the fever, \\
and thereafter sodomised(?) the house of wounds. \\
The pus has he fully caught—fly away, fever!\evb\evg

\sectionline

\bookStart{Sigtuna Plate I}[U Fv1933;134]

\begin{flushright}%
\textbf{Dating:} C11th

\textbf{Meter:} \Fornyrdislag%
\end{flushright}

TODO: Introduction

\sectionline

\bvg\bva[] \alst{Þ}urs sár-riðu, \hld\ \alst{þ}ursa dróttinn; &
\alst{f}liu þú nú \hld\ \alst{f}undinn es! &
\ind Af þéʀ \alst{þ}ríaʀ \alst{þ}ráaʀ, ulfr; &
\ind af þéʀ \alst{n}íu \alst{n}ø̂þiʀ, ulfr! &
Efiʀ þessi séʀ, auk es uniʀ ulfr. &
Niut lyfja!\eva

\bvb O thurse of the wound-fever, O lord of Thurses; \\
fly thou now; found art thou! \\
Have for thee three yearnings, O wolf! \\
Have for thee nine needs, O wolf! \\
He has this for himself, and the wolf is content. \\
Benefit from the medicine!\evb\evg

\sectionline

\bookStart{Galders from Bryggen}

Several galders or magical inscriptions are part of the cache of mediæval rune-inscribed objects found at Bryggen in the city of Bergen, Norway.  For simplicity’s sake, they are here listed in ascending order of their runological numbers.

\sectionline

\section{B 257}

\begin{flushright}%
\textbf{Dating:} c. 1335

\textbf{Meter:} \Galdralag
\end{flushright}%

A stick inscribed on four planed sides.  Part of the stick is broken off, which renders the text incomplete.  The inscription is clearly a “love-charm” (that is, a piece of sexually coercive magic), addressed—as shown by the feminine dative \emph{sjalfri þér} ‘thy self’ on side D—to a woman.  The language closely resembles that of \Skirnismal\ 36, in which Shirner, Free’s servant, threatens to carve a runic inscription which will curse the ettin-woman Gird with \emph{ęrgi} ‘queerness, degeneracy’, \emph{ǿði} ‘madness’, and \emph{ó·þoli} ‘restlessness, impatience’ unless she sleep with his master.  It seems that we are here dealing with just such a surviving runic curse, and that \Skirnismal\ 36 is reflecting an  authentic form of Norse “love magic” (for it is unlikely that the present inscription should derive directly from that poem) by which a woman is cursed with sexual restlessness until she succumb to the will of the male curser.

A more distant parallel may be seen in the curse-formula found on the two C7th runic inscriptions from Stentoften and Björketorp (see TODO), wherein the destroyer of the respective monuments is cursed to become \emph{herma-lausaʀ argjú} ‘restless (a different root from \emph{ó·þoli} above!) with queerness’, i.e. ‘incessantly randy’.

Side D ends with a string of fake-Latin gibberish, a clear sign of Christian syncretic influence on the Old Norse-Germanic magical tradition.

\sectionline

\bvg\bva[A]Ríst ek \alst{b}ót-rúnar \hld\ ríst ek \alst{b}jarg-rúnar &
\ind \alst{ei}n-falt við \alst{ǫ}lfum &
\ind \alst{t}ví-falt við \alst{t}rollum &
\ind \alst{þ}rí-falt við \alst{þ}u\emph{rsum}\eva

\bvb I carve cure-runes, I carve rescue-runes: \\
onefold against elves, \\
twofold against trolls, \\
threefold against thurses.\evb\evg


\bvg\bva[B]Við inni \alst{sk}ǿðu \hld\ \alst{sk}ag-val-kyrju &
svá’t \alst{ei} megi \hld\ þó-at \alst{ę́} vili &
\alst{l}ę́-vís kona \hld\ \alst{l}ífi þínu g\emph{randa}.\eva

\bvb Against the scatheful shag-walkirrie, \\
so that she may not—though she always wants to— \\
that guile-wise woman—harm thy life.\evb\evg


\bvg\bva[C]Ek \alst{s}endir þér \hld\ ek \alst{s}é á þér &
\alst{y}lgjar \alst{e}rgi \hld\ ok \alst{ó}·þola; &
á þér hríni \alst{ó}·þoli \hld\ ok \alst{jǫ}tuns móð\emph{r}; &
\alst{s}it-tu aldri, \hld\ \alst{s}op-tu aldri.\eva

\bvb I send to thee, I see on thee \\
a she-wolf’s queerness and restlessness; \\
may restlessness stick on thee, and an ettin’s wrath! \\
Never sit, never sleep!\evb\evg


\bvg\bva[D]Ant mér sem sjalfri þér. &
\edtrans{\textbf{†Beirist rubus rabus et arantabus laus abus rosa gava†}}{...}{\Bfootnote{Latin-like gibberish.}}\eva

\bvb Love me like thy self. \\
...\evb\evg

\sectionline

\section{B 380}

\begin{flushright}%
\textbf{Dating:} ?

\textbf{Meter:} \Galdralag
\end{flushright}%

A short little charm explicitly invoking the two most important Heathen Gods, \inx[P]{Thunder} and \inx[P]{Weden}.  The inscription postdates the official conversion of Norway by over a hundred years, and it is an open question whether the two mentioned gods were still seen in a good light or whether they had already been assimilated into the Catholic system of demons and devils.  This question is important since it determines the context of the letter: was it well-wishing, assuming that the receiver was of like mind to the sender, or did he have more sinister intent than the first line lets on?  Judging from the first line, and from the half-Heathen contents of many other inscriptions found at Bryggen (some from as late as the C14th), I see it as crypto-Heathen.

\sectionline

\bvg\bva[]\edtrans{\alst{H}ęill sé þú \hld\ ok í \alst{h}ugum góðum}{Mayst thou be hale and in good spirits}{\Bfootnote{A formulaic greeting.  The very same line is found in \Hymiskvida\ 41; see note there for parallels.}}; &
\ind \alst{Þ}órr þik \alst{þ}iggi, &
\ind \edtrans{\alst{Ó}ðinn þik \alst{ęi}gi}{may Weden own thee}{\Bfootnote{See note to \Voluspa\ 23.}}.\eva

\bvb Mayst thou be hale and in good spirits; \\
may Thunder receive thee, \\
may Weden own thee.\evb\evg

\sectionline



%\section{Runic plates}
% Assorted spells

%	Giga-index at the end
	\bookStart{Index}

\section{Cultural and religious terms (C)}
\begin{itemize}

\inxitem{ape}[C] (ON. \emph{api}, OE. \emph{apa}, OS. \emph{apo}, OHG. \emph{affo}, PNWGmc. \emph{*apó})
  In the Old Norse the word seems to mean ‘fool, buffoon’, in the other old languages apparently ‘monkey’, though this sense should be a later development of the former; why would the early Germanic tribes have a word for an animal that they had never encountered?

\inxitem{aught}[C] (ON. \emph{ę́tt}, OE. \emph{ǽht} ‘possession, property’)
  The Nordic (paternal) clan or family line.

\inxitem{begale}[C] (OHG. \emph{bigalan})
  To affect something using \inx{galder}[C][galders]. See also \inx{gale}[C].

\inxitem{bigh}[C] (ON. \emph{baugr}, OE. \emph{béag}, OHG. \emph{boug})
  A torc or armlet, in the migration period used as currency or tokens of loyalty (see particularly \Hildebrandslied). often referenced in ruler-kennings.

\inxitem{bloot}[C] (ON. \emph{blót}, OE. \emph{blót}, OHG. \emph{bluoz})
  Sacrifice or a sacrificial feast.

\inxitem{bloot-kettle}[C]
  The large pots used for cooking the bloot-stew.

\inxitem{Doom}[C] (ON. \emph{dómr}, OE. \emph{dóm})
  Commonly ‘judgement’ (whence Doomsday, ‘day of judgement’), but also specifically referring to one’s fame or good reputation (that is, how other men judge one’s character and deeds). Thus \Havamal\ 77: “I know one that never dies: the \textbf{Doom} over each man dead.”; this is further illuminated by passages in \Beowulf\ such as 884b–887a: \\ \emph{... · Sigemunde gesprong \\ æfter déaðdæge · dóm unlýtel \\ syþðan wíges heard · wyrm ácwealde \\ hordes hyrde · ...} \\ “For Sighmund sprang up after his day of death unlittle \textbf{Doom}, since hard in conflict he defeated the \inx{Worm}, the herder of the hoard.”; \\ 953b–955a: \\ \emph{... · þú þé self hafast \\ dę́dum gefremed · þæt þín dóm lyfað \\ áwa tó aldre · ...} \\ “Thou hast for thyself by deeds accomplished that thy \textbf{Doom} lives for ever and ever.”

\inxitem{fee}[C] (ON. \emph{fé}, OE. \emph{féoh})
  Originally ‘cattle’, however also used in a broader sense to refer to one’s mobile wealth. For this cf. particularly \Havamal TODO.

\inxitem{feelcunning}[C] (ON. \emph{fjǫlkunnigr})
  Skilled with sorcery.

\inxitem{fimble-}[C] (ON. \emph{fimbul-})
  The ultimate, final, greatest. See \inx{Fimble-thyle}, \inx{Fimble-winter}.

\inxitem{five days}[C] (ON. \emph{fimm dagar})
  That the old Scandinavian week was \textbf{five days} long is well attested. According to the \Gulatingslog\ there were six weeks in a month, and the expression \textbf{five days} is used as the equivalent of \emph{week} in \Havamal 51 and 74, in the second of which it is contrasted with \emph{month}. Related to this is the legal term \emph{fifth} (ON. \emph{fimmt}, OSw. \emph{fæmt}), a meeting or gathering set to be held at a five-day notice. See \emph{fimt} in \CV, \LMNL\ for further discussion.

\inxitem{galder}[C] (ON. \emph{galdr}, OE. \emph{gealdor}, OHG. \emph{galdar})
  A magical spell or song. See the Merseburg charms (TODO?) for examples. See also \inx{gale}[C].

\inxitem{gand}[C] (ON. \emph{gandr}, Latin \emph{gandus})
  A witch’s familiar, a spirit sent out to do her bidding. See PCRN HS I:17, p. 361 and II:26, p. 656. TODO

\inxitem{gin-}[C] (ON. \emph{ginn-})
  A rare prefix, maybe referring to sacrosanctity. TODO.

\inxitem{hame}[C] (ON. \emph{hamr})
  A skin, shape. Individuals can through magic “shift hames” (ON. \emph{skipta hǫmum}), and leave their human \emph{hames} behind, instead entering into the shapes of wolves, bears, birds. During this process the original hame would be sleeping in a vulnerable state, as described in the Saw of the Walsings, chap. TODO: . See also \inx{feather-hame}, \inx{town-riders}, \inx{evening-riders}.

\inxitem{harrow}[C] (ON. \emph{hǫrgr}, OE. \emph{hearg}, PNWGmc. \emph{*harugaʀ})
  A cairn constructed for ritual purposes. \emph{Hind} 10 describes one: “A \inx{harrow} he made for me, loaded with stones; now that stone-pile is become into glass. He reddened [it] in fresh blood of oxen; Oughthere ever trusted on the osennies.” See also \inx{wigh}.

\inxitem{leat}[C] (ON. \emph{hlaut})
  Sacrificial blood (that is, taken from the animal), especially when used for auguries.

\inxitem{leat-twig}[C] (ON. \emph{hlauttęinn})
  A twig used to sprinkle the \inx{leat}[C] in auguries (presumably the pattern of the blood would then be inspected).

\inxitem{leed}[C] (ON. \emph{ljóð}, OE. \emph{léod})
  A magical chant or incantation. See also \inx{galder}[C], \inx{gale}[C], \inx{begale}[C].

\inxitem{manwit}[C] (ON. \emph{manvit})
  Practical sense and wisdom, situational awareness, ‘common sense’.

\inxitem{orlay}[C] (ON. \emph{ørlǫg}, OE. \emph{orlæg})
  One’s predetermined fate, destiny, purpose as decreed by the \inx{Norns}.

\inxitem{rest}[C] (ON. \emph{rǫst})
  The distance between two rest-stops, a geographical mile (about 1850 metres). See especially \CV.

\inxitem{rune}[C] (ON. \emph{rún}, OE. \emph{rún}, OS. \emph{rúna}, OHG. \emph{rúna}, Got. \emph{rúna}, PNWGmc. \emph{rūnu})
  An (esoteric) secret message or formula. That this—rather than ‘letter (of a Runic alphabet)’—is the original and proper sense is apparent from among others the Finnish borrowing \emph{runo} ‘poem; poetry; a division of a poem (specifically of the \emph{Kalevala})’, and its use in the singular in the earliest Runic inscriptions such as Noleby Vg 63 (which contains the linguistically indecipherable string of letters {ᚢᚾᚨᚦᛟᚢᛊᚢᚺᚢᚱᚨᚺᛊᚢᛊᛁᚺ[--]ᚨᛁ\rotatebox[origin=c]{180}{ᛏ}ᛁᚾ}, a \emph{rune} in the proper sense) or the recently discovered Svingerud fragment. Thus, Weden’s taking of the \emph{runes} should not be interpreted as merely a myth for the invention of profane writing, but rather the origin of esoteric incantations, not at all unlike Indian \emph{mantras}.
  The word for letter was instead \inx{stave}[C], see also there.

\inxitem{soo}[C] (ON. \emph{sóa})
  To ritually waste, the slaying in the animal sacrifice.

\inxitem{thill}[C] (ON. \emph{þylja})
  To chant poetry or lists (so called \inx{thules}[C][thule]) acquired by rote memorization. See also {thyle}[C].

\inxitem{Thing}[C] (ON., OE. \emph{þing}, OS. \emph{thing}, OHG. \emph{ding})
  The legal assembly and gathering place where matters would be settled and the law recited.

\inxitem{thyle}[C] (ON. \emph{þulr}, OE. \emph{þyle}, PNWGmc. \emph{*þuliʀ})
  A sage who through rote learning has acquired a large amount of mythological lore (cf. \emph{þula} 'a list in poetic form; a meaningless poem' and \emph{þylja} 'to recite, to chant'). Thus \inx{Weden} is the \inx{Fimble-thyle}, being the unbeaten master of lore, as can be seen in his wisdom contests (see \Allvismal, \Vafthrudnismal). Runic inscription DR 248 (Snoldelev) suggests the thyle may have tied to a specific place, and in Beowulf it seems to have been a court position, with \inx{Unferth} being described as the "thyle of Rothgar".

\inxitem{wale}[C] (ON. \emph{vǫlr})
  The staff or sceptre, especially of a wallow. TODO: archeological finds, mention Sutton Hoo.

\inxitem{wallow}[C] (ON. \emph{vǫlva}, OE. \emph{*wealwe} (cf. ON. \emph{svǫlva}, OE. \emph{swealwe} ‘swallow’))
  A sibyl, seeress, oracle. The word derives from the \inx{wale}[C], a staff or sceptre probably used for ritual purposes.

\inxitem{wigh}[C] (ON. \emph{vé}, OE. \emph{wéoh}, \emph{wíh}, PNWGmc. \emph{*wīhą})
  A holy shrine or sanctuary. It seems that where the \inx{harrow} was a pile of stones or cairn used for carrying out rituals, the \emph{wigh} was an enclosed space. The earliest Norse attestation is the runic inscription Ög N288 (Oklunda), which reads: “Guthhere <= Gunnarr> painted these runes, and he fled, guilty. Sought this wigh, and he fled into this clearing. And he bound. [...]” The implication seems to be that the wigh was considered so sacred that Guthhere could not be apprehended or punished for his crime while in it. — In Old English the word means ‘pagan idol’. It is not immediately clear which meaning is the original one, but in this edition the Norse sense has been adopted, since the Anglo-Saxon sources are all of a Christian nature. The \emph{Beewolf} name \emph{Wighstone} (\emph{Wīh-} or \emph{Wēohstān}) in any case suggests it is the Norse meaning, since ‘idol-stone’ makes little sense.

\inxitem{wode}[C] (ON. \emph{óðr}, OE. \emph{wód}, PNWGmc. \emph{*wōþuʀ})
  \inx{Hean}'s gift to men, though the name would suggest it be from \inx{Weden}. The word has several related meanings: ‘poetic inspiration’, ‘madness’, ‘rage’.

\end{itemize}


\section{Personal names, objects and events (P)}

\begin{itemize}

\inxitem{Attle}[P] (\emph{Attila}, ON. \emph{Atli}, OE. \emph{Ætla}, MHG. \emph{Etzel}, PNWGmc. \emph{*Attilō})
  The ruler of the \inx{Huns} (historically from 434–453). Husband of \inx{Guthrun}, and with her father of \inx{Earp and Oatle}. and murderer of
  I HHb 54, SiL 11, I Gr 23, ShS 28, 29, 33, 37, 54, 56, 57, II Gr 26, 38, 45, III Gr 1, 9, BnOr 0, OdW A, 2, 22, 23, 25, 26, 30, 31, AtD 0, AtL 1, 3, 15, 17, 18, 27, 31, 32, 34, 36, 37, 38, 41, 43, B, AtS 2, 4, 21, 22, 44, 52, 60, 64, 71, 73, 77, 80, 86, 87, 97, 98, 108, 113, 117, FGr 0, GrB 12, Ham 6.

\inxitem{Earp and Oatle}[P] (ON. \emph{Erpr ok Eitill})
  The sons of \inx{Attle} and \inx{Guthrun}.

\inxitem{Feather-hame}[P] (ON. \emph{fjaðrhamr})
  A \inx{hame} owned by the Ease that lets the wearer fly like a bird, more specifically a falcon.

\inxitem{Guthrun}[P] (ON. \emph{Guðrún})
  Daughter of king \inx{Yivick}, sister of \inx{Guthhere} and \inx{Hain}. The wife of \inx{Attle}.

\inxitem{Hain}[P][Hain 1] (ON. \emph{Hǫgni}, OE. \emph{Haguna}, \emph{Hagena}, OHG. \emph{Hagano}, Ger. \emph{Hagen}, PNWGmc. \emph{*Hagunō})
  A \inx{Nifling} and \inx{Yifking}, son of king \inx{Yivick}, brother of \inx{Guthhere} and \inx{Guthrun}. In \emph{AtL} he defeats seven warriors before being captured by \inx{Attle}, who has his heart cut out at the request of Guthhere.

\inxitem{2}[P] A petty king of \inx{East Geatland}, contemporary with \inx{Granmer}, the king of \inx{Southmanland} and Ingeld Illrede, the \inx{Ingling} king of \inx{Upland}.

\inxitem{Hindle}[P] (ON. \emph{Hyndla}) A witch awoken by Frow in \emph{Hind}.

\inxitem{Millner}[P] (ON. \emph{Mjǫllnir}, OE. \emph{*Meldne}, PNWGmc. \emph{*Meldunjaʀ})
  Powerful hammer owned by Thunder.

\inxitem{Oughter}[P] (ON. \emph{Óttarr}, OE. \emph{Óhthere}, PNWGmc. \emph{*Ōhtaharjaʀ})
  TODO

\inxitem{Rakes of the Reins}[P] (ON. \emph{ragna rǫk})
  The ‘sequence of events of the \inx{Reins}[G]’, euphemistic for the destruction of the Home.

\inxitem{Rakes of the Tues}[P] (ON. \emph{tíva rǫk})
  See the \inx{Rakes of the Reins}[P].

\inxitem{Rotholf}[P] (ON. \emph{Hrólfr kraki}, OE. \emph{Hróþulf}, PNWGmc. \emph{*Hrōþiwulfaʀ})
  A king of the \inx{Shieldings} (see family tree). As foreshadowed in \emph{Beewolf} (1017–9, 1180–90), he betrays the sons of \inx{Rothgar}, his cousins \inx{Rethrich and Rothmund}, in order to take the throne for himself. In the later Icelandic tradition this has been forgotten, and he is consistently portrayed as a heroic king.

\inxitem{Rothgar}[P] (ON. \emph{Hróarr}, OE. \emph{Hróþgár}, PNWGmc. \emph{*Hrōþigaiʀaʀ})
  A king of the \inx{Shieldings} (see family tree), one of the main characters in \emph{Beewolf}.

\inxitem{Weden}[P] (rhymes with \emph{leaden}; ON. \emph{Óðinn}, OE. \emph{Wóden}, \emph{Wéden}, OHG. \emph{Wuotan}, PNWGmc. \emph{*Wōdanaʀ})
  Chief of the \inx{Ease}, his name is clearly related to \inx{wode}, referring to his role as the patron of \inx{scolds} and \inx{bearserks}. For the meaning of his other names see \inx{Fimblethyle}, \inx{Harn} TODO. Husband of \inx{Frie}, and by her father of \inx{Bolder}. Also father of \inx{Thunder} by \inx{Earth}. Brother of \inx{Hean} and \inx{Lother}.

\inxitem{Yivick}[P] (ON. \emph{Gjúki}, OE. \emph{Gifica}, OHG. \emph{Gibicho}, MHG. \emph{Gibeche})
  King of the \inx{Burgends} (historically from late 300s–407) of the Nifling dynasty, founder of the \inx{Yifking} aught†. Father of \inx{Guthrun}, \inx{Guthhere} and \inx{Hain}.

\end{itemize}


\section{Groups and place names (G)}

TODO: Map of rough tribal areas. Geneaologies.

\begin{itemize}

\inxitem{Danes}[G] (ON. \emph{danir}, OE. \emph{dene}, PNWGmc. \emph{*daníʀ})
  A tribe in eastern modern-day Denmark and southern Sweden. They probably originated in Scania in southern Sweden, before moving westwards into the Danish isles and eventually Jutland, driving out the \inx{Earls} and \inx{Jutes}.
  Noted members: TODO
  Attestations: TODO

\inxitem{Dwarfs}[G] (ON. \emph{dvergar}, OE. \emph{dweorgas}, OHG. \emph{twerca}, PNWGmc. \emph{*dwergóʀ})
  Earthly (chthonic) supernatural beings, often referred to as living in rocks and mountains.
  Noted members: TODO
  Attestations: TODO

\inxitem{Ease}[G] (rhyming with \emph{geese}; ON. \emph{ę́sir}, OE. \emph{ése}, PNWGmc. \emph{*ansiwiʀ}; sg. \emph{os}, ON. \emph{áss}, OE. \emph{ós}, PNWGmc. \emph{*ansuʀ})
  A group of Gods, though the word can also refer to all the Gods. See \inx{Gods}, \inx{Tues}, \inx{Wanes}, \inx{Powers}.
  Noted members: \inx{Weden}, \inx{Thunder}, \inx{Frie}, \inx{Hath} and \inx{Bolder}
  Attestations: TODO

\inxitem{Ease and Elves}[G] (ON. \emph{ę́sir ok alfar}, OE. \emph{ése ende ielfe}, PNWGmc. \emph{*alβíʀ jah ansiwiʀ})
  A merism; both heavenly and earthly spiritual beings. Notably the words always occur in this order.

\inxitem{Elves}[G] (ON. \emph{alfar}, OE. \emph{ielfe}, PNWGmc. \emph{*alβíʀ})
  Earthly (chthonic) supernatural beings. Possibly ancestral spirits?
  Noted members: TODO
  Attestations: TODO

\inxitem{Ettins}[G] (ON. \emph{jǫtnar}, OE. \emph{eotenas}, PNWGmc. \emph{*etunóʀ})
  The fundamental enemies of the Gods, the agents of chaos and disorder. See \inx{Rises}, \inx{Thurses}.
  Noted members: \inx{Thrym}
  Attestations: TODO

\inxitem{Geats}[G] (ON. \emph{gautar}, OE. \emph{géatas}, PNWGmc. \emph{*gautóʀ} from \emph{*geut-} ‘to pour’, perhaps ‘the libators’)
  A tribe in what is today southern-central Sweden. See also \inx{Geatland}.
  Noted members: TODO
  Attestations: TODO

\inxitem{gin-Reins}[G] (ON. \emph{ginnręgin})
  \inx{gin-}[C] + \inx{Reins}[G]. The sacrosanct, highest divine powers.

\inxitem{Gods}[G] (ON. \emph{goð}, OE. \emph{godu}, OHG. \emph{gota}, PNWGmc. \emph{*godu})
  TODO.
  Noted members: TODO
  Attestations: TODO

\inxitem{Huns}[G] (ON. \emph{húnir}, OE. \emph{Húne}, OHG. \emph{Húni}, \emph{Hunni}, PNWGmc. \emph{*húníʀ})
  An invading Asiatic tribe in the Migration Period. In the legendary material their cultural and ethnic foreignness is not seen.
  Noted members: TODO
  Attestations: TODO

\inxitem{Nears}[G] (ON. \emph{níarar} ~ \emph{njárar})
  A Swedish tribe, only mentioned in \Volundarkvida, where it is ruled by king \inx{Nithad}[P]. The name and location may allow us to connect them with the Swedish province of Närke, cf. Old Swedish: \emph{Nærikiar} ‘inhabitants of Närke’, \emph{Nærisker} ‘belonging to Närke; Nearish’. The Old Swedish stem \emph{nær-} would then be a reduced form of \emph{níar-}, \emph{njár-}.

\inxitem{Reins}[G] (ON. \emph{rǫgn}, \emph{ręgin})
  The divine powers. Based on \Vafthrudnismal\ (TODO) the term may be more closely associated with the \inx{Wanes}[G] than the \inx{Ease}[G].

\inxitem{Saxons}[G] (ON. \emph{saxar}, OE. \emph{Seaxan}, \emph{Seaxe})
  TODO.
  Noted members: TODO
  Attestations: TODO

\inxitem{Shieldings}[G] (ON. \emph{skjǫldungar}, OE. \emph{Scyldingas}, PNWGmc. \emph{*skeldungóʀ})
  The descendants of \inx{Shield}[P], the legendary ruling dynasty of the \inx{Danes}. With \inx{Harward}'s death after his slaying of \inx{Rotholf} their rule ended. TODO
  Noted members: TODO
  Attestations: TODO

\inxitem{Shelfings}[G] (ON. \emph{skilfingar}, OE. \emph{Scilfingas}, PNWGmc. \emph{*skilβingóʀ})
  The descendants of \inx{Shelf}[P]. The exact difference between Shelfings and \inx{Inglings} is unclear. According to the Saw of Geatrich TODO
  Noted members: TODO
  Attestations: \Hyndluljod 15, 20

\inxitem{Swedes}[G] (ON. \emph{svíar}, OE. \emph{Swéon}, PNWGmc. \emph{*swihaníʀ})
  TODO.
  Noted members: TODO
  Attestations: TODO

\inxitem{Thurses}[G] (sg. Thurse; ON. \emph{þurs}, OE. \emph{þyrs}, OS. \emph{thuris}, OHG. \emph{duris}, PNWGmc. \emph{*þurisaʀ})
  Possibly a poetic synonym for \inx{Ettins}. See also \inx{Rime-Thurse}
  Noted members: TODO
  Attestations: Wal 8, Shr 31, 35, 36, Hyme 17, Thr 5, 10, 21, 24, 29, 30, Alw 2, I HHb 40, HHw 27.

\inxitem{Tues}[G] (ON. \emph{tívar}, PNWGmc. \emph{*tíwóʀ})
  A poetic synonym for \inx{Gods}.
  Noted members: —
  Attestations: TODO

\inxitem{Wanes}[G] (ON. \emph{vanir}, OE. \emph{wan-?})
  A tribe of the gods, associated with fertility, harvests and fishing.
  Noted members: TODO
  Attestations: TODO

\inxitem{Yifkings}[G] (ON. \emph{gjúkungar})
  The descendants of \inx{Yivick}, including \inx{Guthhere}, \inx{Guthrun} and \inx{Hain}.
  Noted members: TODO
  Attestations: TODO

\end{itemize}
%
%

\end{document}
