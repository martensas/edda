% This file should be compiled with XeLaTeX.

\documentclass{memoir}

% Font and typesetting
\usepackage[final]{microtype}

% Multiple languages
\usepackage{fontspec}
\usepackage{polyglossia}

\setdefaultlanguage{english}
\setotherlanguages{greek}

\setmainfont{Junicode}[
	Extension=.ttf,
	BoldFont=*-Bold,
	ItalicFont=*-Italic,
	BoldItalicFont=*-BoldItalic]
\newfontfamily\greekfont{Times New Roman}

\usepackage[usenames,dvipsnames]{xcolor}

% TODO: Underline that does not skip descender?
% Should be called \nsunderline
% ^ The point of this would have been to make the alliteration marking nicer. But it'd be bad since certain consonants (p, g, j, q) would simply not be marked.

% Headers.
\pagestyle{myheadings}
\aliaspagestyle{plain}{empty}% Remove footers from title pages
\makeoddhead{myheadings}{}{\thechapter}{\thepage}
\makeevenhead{myheadings}{\thepage}{The Ancient Germanic Poetry}{}


% Bibliography and citations
\usepackage[style=apa]{biblatex}
\addbibresource{bibliography.bib}
\DefineBibliographyStrings{english}{%
  sequens = {f\adddot},
  sequentes = {ff\adddot},
}

\newbibmacro*{textciteshorttitle}{% Cite short titles. From Stack Exchange: https://tex.stackexchange.com/questions/489928/biblatex-apa-textcite-with-shorttitle
  \ifbool{cbx:parens}
    {\bibcloseparen\global\boolfalse{cbx:parens}}%
    {}%
  \setunit{\compcitedelim}%
  \printfield[bibhyperref]{shorttitle}%
  \iffieldundef{postnote}{}{%
		\ifnumequal{\value{citecount}}{\value{citetotal}}{%
			\printunit{\global\booltrue{cbx:parens}\addspace\bibopenparen}
		}{}%
	}%
}%

\DeclareCiteCommand{\textciteshorttitle}
  {\usebibmacro{cite:init}%
   \usebibmacro{prenote}}
  {\usebibmacro{citeindex}%
   \usebibmacro{textciteshorttitle}}
  {}
  {\usebibmacro{textcite:postnote}%
   \usebibmacro{cite:post}}


% Critical edition
\usepackage[parapparatus, series={A,B}]{reledmac}

% Define verse and prose counters
\newcounter{versea}
\newcounter{verseb}
\newcounter{prosea}


% Various packages
\usepackage{xparse}% For better document commands
\usepackage{graphicx}% For rotating characters (used when citing runic inscriptions)
\usepackage{hyperref}% For index links
\usepackage{longtable}% Long tables.
\usepackage{tabu}% Table management
\usepackage{tipa}% IPA


\begin{document}

% Book and chapter commands
  \setsecnumdepth{none}% Disable numbering of sections
  \maxsecnumdepth{none}% Disable numbering of sections

	\NewDocumentCommand{\chapterStart}{o O{Chap}}{% Command at the start of chapter
		\setcounter{stanza}{0}%
		\setcounter{verseb}{0}%
		\setcounter{prosea}{0}%
		\stepcounter{section}%
		\IfNoValueF{#1}{%
			\begin{center}%
			\textbf{#2. \arabic{section}} \\
			{#1}\end{center}%
		}%
	}

	\NewDocumentCommand{\bookStart}{m o}{% Command at the start of book
		% arg 1 (mandatory): English title
		% arg 2 (optional):  Original title
	  \IfValueTF{#2}{%
      \chapter*{#1 \emph{(#2)}}%
      \def\booktitle{#1 \emph{(#2)}}%
    }{%
      \chapter*{#1}%
      \def\booktitle{#1}%
    }%
    \addcontentsline{toc}{chapter}{\booktitle}
    \setcounter{section}{0}% Set chapter count to zero.
    \chapterStart{}%
	}


% Verse format commands
	\numberstanzatrue% Use reledmac stanza numbering

	\NewDocumentCommand{\bvg}{o}{% Begin verse group
		\begin{ledgroup}%
		\beginnumbering%
		\setcounter{footnoteB}{0}%
	}

	\NewDocumentCommand{\bva}{o o}{% Begin verse a
		\IfValueT{#2}{% Optional line number
			\setlinenum{#2}%
		}%
		\IfValueTF{#1}{%
			\numberstanzafalse% If optional verse number exists or is 0 we disable numbering
		}{%
			\numberstanzatrue%
		}%
		\begin{large}\begin{stanza}%
		\IfValueT{#1}{%
			\IfEq{#1}{0}{}{% If optional verse number is NOT 0 we show it with flag.
				\flagstanza{\textbf{#1}}%
			}%
		}%
	}
	\NewDocumentCommand{\eva}{o}{% End verse a
		\& \end{stanza}\end{large}% End reledmac stanza
		\vspace{1.5mm}% Vertical space
	}

	\NewDocumentCommand{\bvb}{o}{% Begin verse b (see above)
		\IfNoValueTF{#1}{%
			\stepcounter{verseb}%
%			\textbf{\arabic{verseb} }%
		}{\IfEq{#1}{0}{}{%
%			\textbf{\textbf{#1}}
		}}%
		\noindent%
	}
	\NewDocumentCommand{\evb}{o}{% End verse b
		% Nothing (for now?)
	}

	\NewDocumentCommand{\evg}{o}{% End verse group
		\endnumbering\end{ledgroup}% End numbering and ledgroup
		\vspace{1cm}%
	}


% Prose format commands
	\NewDocumentCommand{\bpg}{o}{% Begin prose group
  	\begin{ledgroup}\beginnumbering%
  	\setcounter{footnoteB}{0}%
  }

  \NewDocumentCommand{\bpa}{o}{% Begin prose a
		\setlength{\leftskip}{1cm}% Indent hack
		\pstart%
		\noindent% No indent (is this bad)
		\begin{large}% Begin large text
		\IfNoValueTF{#1}{% Add prose number according to counter
			\stepcounter{prosea}% Step prose counter
			\flagstanza{\textbf{P\arabic{prosea}}} %
		}{\IfEq{#1}{0}{}{% If optional prose number is NOT 0 we show it.
			\flagstanza{\textbf{P#1}}%
		}}%%
	}
	\NewDocumentCommand{\epa}{o}{% End prose a
		\end{large}% End large
		\pend
		\setlength{\leftskip}{0cm}%
		\vspace{1.5mm}% Vertical space
	}

	\NewDocumentCommand{\bpb}{o}{% Begin prose b (see above)
		\noindent%
	}
	\NewDocumentCommand{\epb}{o}{% End prose b
		% nothing (for now?)
	}

	\NewDocumentCommand{\epg}{o}{% End verse group
		\endnumbering\end{ledgroup}% End numbering and ledgroup
		\vspace{1cm}%
	}


% Formatting of footnotes and critical apparatus
	% Edtext with automatic translation
	% EXAMPLE: \edtrans{orðinn}{become}{\Afootnote{borinn ‘borinn’ \Hauksbok}}
	\NewDocumentCommand{\edtrans}{m m m}{%
		\edtext{#1}{\lemma{\textnormal{#1} ‘#2’}#3}%
	}

	% Manuscript note
	\leftnoteupfalse\rightnoteupfalse% Make the notes flow downwards
	\NewDocumentCommand{\mssnote}{m}{%
		\ledouternote{[{#1}]}%
	}

	% Sidenote margin
	\setlength{\ledlsnotesep}{2 \ledlsnotesep}

	% Make A footnotes paragraphs
	\Xarrangement[A]{paragraph}
	% Make A footnotes italicized
%	\Xwrapcontent[A]{\emph}

	% Make numbering of B footnotes roman
	\renewcommand*{\thefootnoteB}{\alph{footnoteB}}

	\fnpos{% Footnote order
		{B}{familiar},%
		{A}{familiar},% No such notes should exist but the entry is needed.
		{A}{critical},%
		{B}{critical},%
	}


% Poem formatting
	% First line number at 2
	\firstlinenum{2}
	\linenumincrement{2}

	% Stanza indentation (required by reledmac)
	\setstanzaindents{5, 2, 2}
	\setcounter{stanzaindentsrepetition}{2}

	% Line numbers directly under verse number (kind of a hack)
	\setlength{\linenumsep}{-1.62pc}

	% Mark cæsura.
	\newcommand{\hld}{ · }%

	% Indent lines (in Ljóðaháttr or Galdralag).
	\newcommand{\ind}{%
		\hspace{1.5em}%
	}

	% Mark alliteration. This might not be present in the final version.
	% Update: It's very useful and definitely will.
	\NewDocumentCommand{\alst}{m}{%
%		\underline{#1}% Underline.
%		\textbf{#1}% Bold font.
  	\textcolor{Red}{#1}% Red font. Uses xcolor package.
	}

% Translation formatting

	%Line
	\newcommand{\sectionline}{%
		\begin{center}\line(1,0){5em}\end{center}%
	}

	% Mark kennings.
	\NewDocumentCommand{\ken}{sm}{%
		% No small caps; text inside the brackets.
		\IfBooleanTF{#1}{% No small caps
			{[{#2}]}% EXAMPLE: [= Wooden]
		}{%
			\textsc{[{#2}]}% EXAMPLE: [ETTIN]
		}%
	}%

	% Mark names with angular brackets.
	\NewDocumentCommand{\name}{m}{%
	% Text inside the brackets.
		⟨{#1}⟩% EXAMPLE: ⟨ettin⟩
	}%

% Encyclopedia commands
	\NewDocumentCommand{\inx}{o m O{#2}}{% Encyclopedia link (type, link, optional alt display)
		\IfNoValueTF{#1}{% TODO: phase this out
			ERRORERRORERROR%
		}{% Proper noun
			\hyperref[#1:#2]{#3\textsuperscript{#1}}%
		}%
	}

	\NewDocumentCommand{\inxitem}{o m}{% Encyclopedia label (type, word)
		\item[\textbf{#2}]%
		\phantomsection\label{#1:#2}%
	}

% Sigla
  \subsection{Languages}
\begin{itemize}%
	\item Eng. = Modern English
	\item Ger. = Modern German
	\item Got. = Gotnish (or Gothic)
	\item Lomb. = Lombardic
	\item MHG = Middle High German
	\item OE = Old English
	\item OF = Old Frisian
	\item OHG = Old High German
	\item ON = Old Norse
	\item OS = Old Saxon
	\item OSwe. = Old Swedish
	\item PGmc. = Proto-Germanic
	\item PN = Proto-Norse
	\item PNWGmc. = Proto-North-West Germanic
\end{itemize}

\subsection{Grammar}
\begin{itemize}%
	\item 1st = first-person
	\item 2nd = second-person
	\item 3rd = third-person
	\item acc. = accusative case
	\item cpd = compound
	\item dat. = dative case
	\item gen. = genitive case
	\item imper. = imperative mood
	\item ind. = indicative mood
	\item instr. = instrumental case
	\item nom. = nominative case
	\item pl. = plural number
	\item sg. = singular number
	\item subj. = subjunctive mood
\end{itemize}

\subsection{Other abbreviations}
\begin{itemize}%
	\item cert. = certainly
	\item c. = circa
	\item cf. = \emph{confere}; compare
	\item corr. = corrected in the ms.
	\item e. = excerpt (not the whole stanza)
	\item ed. = edition, edited (by)
	\item e.g. = \emph{exemplio gratia}; for instance
	\item emend. = emendation, emended (by)
	\item fol., foll. = folio, folios
	\item i.e. = \emph{id est}; that is
	\item l., ll. = line, lines
	\item lit. = literally
	\item metr. emend. = emended based on (secure) metrical criteria
	\item ms., mss. = manuscript, manuscripts
	\item norm. = normalised from the ms. spelling
	\item om. = omitted by
	\item p., pp. = page, pages
	\item tr. = translation, translated (by)
	\item sens. emend. = emended based on sense
	\item st., sts. = stanza, stanzas
	\item viz. = \emph{vidēlicet}; namely, to wit
	\item wo. = without
	\item wrt. = with regard to
\end{itemize}

% Old texts, primary sources
% The command codes must be as close to the original language titles as possible.
\subsection{Primary sources}

\newcommand{\Allvismal}{% Speeches of Allwise
	\emph{Alv}%
}
\newcommand{\Atlakvida}{% Lay of Attle
	\emph{Akv}%
}
\newcommand{\Atlamal}{% Speeches of Attle
	\emph{Am}%
}
\newcommand{\Baldrsdraumar}{% The Dreams of Balder
	\emph{Bdr}%
}
\newcommand{\Beowulf}{% Beewolf
	\emph{Beow}%
}
\newcommand{\Brot}{% Fragment of a Lay of Siward
	\emph{Brot}%
}
\newcommand{\Deor}{% Deer
	\emph{Deer}%
}
\newcommand{\EyrbyggjaSaga}{% Saw of Harware and Heathric
	\emph{Eb}%
}
\newcommand{\EgilsSaga}{% Saw of Harware and Heathric
	\emph{Eg}%
}
\newcommand{\Fafnismal}{% Speeches of Fathomer
	\emph{Fáfn}%
}
\newcommand{\FostrbroedhraSaga}{% Saw of the Foster-brothers
	\emph{FbrS}%
}

\newcommand{\FraLoka}{% From Lock
	\emph{From Lock}%
}%TODO: remove this

\newcommand{\Grettissaga}{% Saw of Gretter
	\emph{GrettS}%
}
\newcommand{\Grimnismal}{% Speeches of Grimner
	\emph{Grm}%
}
\newcommand{\Gripisspa}{% Spae of Griper
	\emph{Gríp}%
}
\newcommand{\Grottasongr}{% Song of Grotte
	\emph{Grotta}%
}
\newcommand{\Grougaldr}{% Galder of Growe
	\emph{Grg}%
}
\newcommand{\Gudrunarhvot}{% Guthrun’s Instigation
	\emph{Ghv}%
}
\newcommand{\GudrunOne}{% Guthrun’s First Lay
	\emph{Guðr I}%
}
\newcommand{\GudrunTwo}{% Guthrun’s Second Lay
	\emph{Guðr II}%
}
\newcommand{\GudrunThree}{% Guthrun’s Third Lay
	\emph{Guðr III}%
}
\newcommand{\Gulatingslog}{% Law of the Gole-thing
	\emph{Gula}%
}
\newcommand{\Gylfaginning}{% The Guiling of Yilver; for referring to Gylfaginning as a text
	\emph{Gylf}%
}
\newcommand{\Hakonarmal}{% Speeches of Hathkin
	\emph{Hákm}%
}

\newcommand{\HakonarSaga}{% Saw of Hathkin the good
	\emph{HákGóð}%
}
\newcommand{\Haleygjatal}{% Tally of the Hallowlendings
	\emph{HalT}%
}%TODO: Add

\newcommand{\Hamdismal}{% Speeches of Hamthew
	\emph{Hamð}%
}
\newcommand{\Harbardsljod}{% Leed of Hoarbeard
	\emph{Hárb}%
}
\newcommand{\Haustlong}{% Harvest-long
	\emph{Haustl}
}
\newcommand{\Havamal}{% Speeches of the High One
	\emph{Háv}%
}
\newcommand{\HelgakvidaHjorvardssonar}{% Lay of Hallow Harwardson
	\emph{HHj}%
}
\newcommand{\HelgakvidaOne}{% First Lay of Hallow Hundingsbane
	\emph{HHund I}%
}
\newcommand{\HelgakvidaTwo}{% Second Lay of Hallow Hundingsbane
	\emph{HHund II}%
}
\newcommand{\Heliand}{%
  \emph{Heli}%
}
\newcommand{\Helreid}{% Byrnhild’s Hell-ride
	\emph{Helr}%
}
\newcommand{\HervararSaga}{% Saw of Harware and Heathric
	\emph{HarS}%
}
\newcommand{\Hildebrandslied}{% Speeches of Hildbrand
	\emph{Hildebrand}%
}
\newcommand{\Hymiskvida}{% Lay of Hymer
	\emph{Hym}%
}
\newcommand{\Hyndluljod}{% Leed of Hindle
	\emph{Hdl}%
}

\newcommand{\Lacnunga}{% Leekning
	\emph{Lacning}%
}%TODO: add this

\newcommand{\Lokasenna}{% Flyting of Lock
	\emph{Lok}%
}

\newcommand{\Malshattakvadi}{
	\emph{Mhkv}
}%TODO: add this

\newcommand{\Mahabharata}{
	\emph{Mahā́bʰārata}
}
\newcommand{\MerseburgOne}{% First Merseburg charm
	\emph{Mers I}%
}
\newcommand{\MerseburgTwo}{% Second Merseburg charm
	\emph{Mers II}%
}
\newcommand{\Muspilli}{% Muspell
	\emph{Muspilli}%
}
\newcommand{\Oddrunargratr}{% Weeping of Ordrun
	\emph{Oddrgr}%
}
\newcommand{\Reginsmal}{% The Speeches of Rein
	\emph{Reg}%
}
\newcommand{\Rigsthula}{% Thule of Righ
	\emph{Rþ}%
}
\newcommand{\Rigveda}{%
	\emph{R̥V}%
}
\newcommand{\SaxonGenesis}{% Old Saxon Genesis
	\emph{OSGen}%
}
\newcommand{\Sigurdskamma}{% Short Lay of Siward
	\emph{Sigsk}%
}
\newcommand{\Sigrdrifumal}{% Speeches of Syedrive
	\emph{Sigrdr}%
}
\newcommand{\Skaldskaparmal}{% The Matter of Scoldship
	\emph{Skm}%
}
\newcommand{\Skirnismal}{% Speeches of Shirner
	\emph{Skm}%
}

\newcommand{\Sogubrot}{
	\emph{AncKings}
}%TODO: add
\newcommand{\Solarljod}{
	\emph{Sun}
}%TODO: add
\newcommand{\Sonatorrek}{
	\emph{Sont}
}%TODO: add
\newcommand{\Sorlathattr}{% Strand of Sarle
	\emph{Sarle}%
}%TODO: add
\newcommand{\ThidreksSaga}{% Saw of Thedrich
	\emph{ThidS}%
}%TODO: add

\newcommand{\Thorsdrapa}{% Drape of Thunder
	\emph{Þdr}%
}
\newcommand{\Thrymskvida}{% Lay of Thrim
	\emph{Þrk}%
}
\newcommand{\Vafthrudnismal}{% Speeches of Webthrithner
	\emph{Vafþ}%
}
\newcommand{\Volsathattr}{% Strand of Walse
	\emph{Vǫlsþ}%
}
\newcommand{\VolsungaSaga}{% Saw of the Walsings
	\emph{VǫlsS}%
}
\newcommand{\Volundarkvida}{% Lay of Wayland
	\emph{Vkv}%
}
\newcommand{\Voluspa}{% Spae of the Wallow
	\emph{Vsp}%
}

\newcommand{\Waldere}{% Walder
	\emph{Walder}%
}%TODO: add
\newcommand{\YnglingaSaga}{% Saw of the Inglings
	\emph{IngS}%
}%TODO: add
\newcommand{\Ynglingatal}{% Tally of the Inglings
	\emph{IngT}%
}%TODO: add

\begin{itemize}%
	\item \Allvismal\ = \emph{Allvíssmǫ́l} (Speeches of Allwise)
	\item \Atlakvida\ = \emph{Atlakviða} (Lay of Attle)
	\item \Atlamal\ = \emph{Atlamǫ́l} (Speeches of Attle)
	\item \Baldrsdraumar\ = \emph{Baldrs draumar} (Dreams of Balder)
	\item \Beowulf\ = \emph{Beowulf}
	\item \Brot\ = \emph{Brot af Sigurðarkviða} (Fragment of a Lay of Siward)
	\item \Deor\ = \emph{Déor} (Deer)
	\item \EyrbyggjaSaga\ = \emph{Eyrbyggja saga} (Saw of the Ere-dwellers)
	\item \Fafnismal\ = \emph{Fáfnismǫ́l} (Speeches of Fathomer)
	\item \FostrbroedhraSaga\ = \emph{Fóstrbrǿðra saga} (Saw of the Fosterbrothers)
	\item \Grettissaga\ = \emph{Grettis saga} (Saw of Gretter)
	\item \Grimnismal\ = \emph{Grímnis mǫ́l} (Speeches of Grimner)
	\item \Gripisspa\ = \emph{Grípisspǫ́} (Spae of Griper)
	\item \Grottasongr\ = \emph{Grottasǫngr} (Song of Grotte)
	\item \Grougaldr\ = \emph{Gróugaldr} (Galder of Growe)
	\item \Gudrunarhvot\ = \emph{Guðrúnarhvǫt} (Goading of Guthrun)
	\item \GudrunOne\ = \emph{Guðrúnarkviða I} (First Lay of Guthrun)
	\item \GudrunTwo\ = \emph{Guðrúnarkviða II} (Second Lay of Guthrun)
	\item \GudrunThree\ = \emph{Guðrúnarkviða III} (Third Lay of Guthrun)
	\item \Gulatingslog\ = \emph{Gulaþingslǫg} (Law of the Gole‑Thing)
	\item \Gylfaginning\ = \emph{Gylfaginning} (Beguiling of Yilver)
	\item \Hakonarmal\ = \emph{Hǫ́konarmǫ́l} (Speeches of Hathkin)
	\item \HakonarSaga\ = \emph{Hǫ́konar saga góða} (Saw of Hathkin the good)
	\item \Hamdismal\ = \emph{Hamðismǫ́l} (Speeches of Hamthew)
	\item \Harbardsljod\ = \emph{Hárbarðljóð} (Leeds of Hoarbeard)
	\item \Haustlong\ = \emph{Haustlǫng} (Harvest‑long)
	\item \Havamal\ = \emph{Hávamǫ́l} (Speeches of the High One)
	\item \HelgakvidaHjorvardssonar\ = \emph{Helgakviða Hjǫrvarðssonar} (Lay of Hallow Harwardson)
	\item \HelgakvidaOne\ = \emph{Helgakviða Hundingsbana I} (First Lay of Hallow Hundingsbane)
	\item \HelgakvidaTwo\ = \emph{Helgakviða Hundingsbana II} (Second Lay of Hallow Hundingsbane)
	\item \Heliand\ = \emph{Heliand}
	\item \Helreid\ = \emph{Helreið Brynhildar} (Hell‑ride of Byrnhild)
	\item \HervararSaga\ = \emph{Hervarar saga} (Saw of Harware and Heathric)
	\item \Hildebrandslied\ = \emph{Hildebrandslied}
	\item \Hymiskvida\ = \emph{Hymiskviða} (Lay of Hymer)
	\item \Hyndluljod\ = \emph{Hyndluljóð} (Leeds of Hindle)
	\item \Lokasenna\ = \emph{Lokasenna} (Flyting of Lock)
%	\item \Mahabharata\ = \emph{Mahā́bʰārata}
	\item \MerseburgOne\ = Merseburg galder I
	\item \MerseburgTwo\ = Merseburg galder II
	\item \Oddrunargratr\ = \emph{Oddrúnargrátr} (Weeping of Ordrun)
	\item \Reginsmal\ = \emph{Ręginsmǫ́l} (Speeches of Rein)
	\item \Rigsthula\ = \emph{Rigsþula} (Thule of Righ)
	\item \Rigveda\ = \emph{R̥g-vedá}, with translations from Jamison‑Brereton unless otherwise specified.
	\item \SaxonGenesis\ = \emph{Old Saxon Genesis}
	\item \Sigurdskamma\ = \emph{Sigurðarkviða skamma} (Short Lay of Siward)
	\item \Sigrdrifumal\ = \emph{Sigrdrífumǫ́l} (Speeches of Syedrive)
	\item \Skaldskaparmal\ = \emph{Skaldskaparmǫ́l} (Matter of Scoldship)
	\item \Skirnismal\ = \emph{Skírnismǫ́l} (Speeches of Shirner)
	\item \Thorsdrapa\ = \emph{Þórsdrápa} (Drape of Thunder)
	\item \Thrymskvida\ = \emph{Þrymskviða} (Lay of Thrim)
	\item \Vafthrudnismal\ = \emph{Vafþrúðnismǫ́l} (Speeches of Webthrithner)
	\item \Volsathattr\ = \emph{Vǫlsaþáttr} (Strand of Walse)
	\item \VolsungaSaga\ = \emph{Vǫlsunga saga} (Saw of the Walsings)
	\item \Volundarkvida\ = \emph{Vǫlundarkviða} (Lay of Wayland)
	\item \Voluspa\ = \emph{Vǫluspǫ́} (Spae of the Wallow)
\end{itemize}%


% Manuscripts
\newcommand{\AM}{% AM 748 I a 4to (https://handrit.is/manuscript/view/da/AM04-0748-I-a, https://books.google.se/books?id=L-MOAAAAQAAJ)
	\textbf{A}%
}
\newcommand{\AMb}{% AM 748 I b 4to (https://handrit.is/manuscript/view/is/AM04-0748-Ib)
	\textbf{A\textsubscript{b}}%
}
\newcommand{\EddaBms}{% AM 757 a 4° (https://handrit.is/manuscript/view/is/AM04-0757a)
	\textbf{B}%
}
\newcommand{\FlatMS}{% Flateyjarbok
	\textbf{F}%
}
\newcommand{\GylfMS}{% For referring to Gylfaginning manuscripts when stanzas are attested there.
	\textbf{G}%
}
\newcommand{\Hauksbok}{% Hauksbok
	\textbf{H}%
}
\newcommand{\VolsungaMS}{% NKS 1824 b 4° (https://skaldic.ku.dk/q?p=skp/mss/ms/512 and https://onp.ku.dk/onp/onp.php?b2195-53)
	\textbf{N}%
}
\newcommand{\Regius}{% Codex Regius (of the poetic edda)
	\textbf{R}%
}
\newcommand{\RegiusProse}{% Codex Regius of the Prose Edda
	\textbf{S}%
}
\newcommand{\Trajectinus}{% Codex Trajectinus
	\textbf{T}%
}
\newcommand{\Wormianus}{% Codex Wormianus (https://clarino.uib.no/menota/text/menota/AM-242-fol)
	\textbf{W}%
}
\newcommand{\Upsaliensis}{% Codex Upsaliensis
	\textbf{U}%
}
\newcommand{\HildMS}{% For referring to the Hildebrandslied manuscript.
	ms.%
}

\subsection{Manuscripts}
\begin{itemize}%
	\item \AM\ = AM 748 I a 4° (https://handrit.is/manuscript/view/da/AM04-0748-I-a)
	\item \AMb\ = AM 748 I b 4° (https://handrit.is/manuscript/view/is/AM04-0748-Ib)
	\item \EddaBms\ = AM 757 a 4° (https://handrit.is/manuscript/view/is/AM04-0757a)
	\item \FlatMS\ = Flatsęyjarbók, GKS 1005 fol. (https://handrit.is/manuscript/view/is/GKS02-1005)
	\item \GylfMS\ = all manuscripts of \Gylfaginning; equivalent to \RegiusProse\Trajectinus\Upsaliensis\Wormianus
	\item \Hauksbok\ = Hauksbók, AM 544 4° (https://handrit.is/manuscript/view/en/AM04-0544)
	\item \VolsungaMS\ = NKS 1824 b 4° (https://onp.ku.dk/onp/onp.php?m9641)
	\item \Regius\ = Codex Regius of the Poetic Edda, GKS 2365 4° (https://eae.ku.dk/q?p=eae/vols/text/1)
	\item \RegiusProse\ = Codex Regius of the Prose Edda, GKS 2367 4° (https://handrit.is/manuscript/view/is/GKS04-2367)
	\item \Trajectinus\ = Codex Trajectinus, Traj 1374ˣ
	\item \Upsaliensis\ = Codex Upsaliensis, DG 11
	\item \Wormianus\ = Codex Wormianus, AM 242 fol. (https://clarino.uib.no/menota/text/menota/AM-242-fol)
\end{itemize}

% Meters
\newcommand{\Drottkvett}{% Court-recited
	\emph{Court-recited meter}%
}
\newcommand{\Fornyrdislag}{% Law of Ancient Speeches
	\emph{Ancient-words-law}%
}
\newcommand{\Galdralag}{% Meter of Speeches
	\emph{Galders-law}%
}
\newcommand{\Ljodahattr}{% Meter of Leeds
	\emph{Leeds-meter}%
}
\newcommand{\Kviduhattr}{%
	\emph{Lay-meter}%
}
\newcommand{\Malahattr}{% Meter of Speeches
	\emph{Speeches-meter}%
}

%Modern books and editions (TODO: move these to bibliography)
\newcommand{\CV}{% Cleasby-Vigfússon dictionary of Old Norse
	\textciteshorttitle{CleasbyVigfusson}% \emph{C-V}%
}
\newcommand{\FGT}{% First Grammatical Treatise
	\textciteshorttitle{FGTHaugen}%
}
\newcommand{\ONP}{% Dictionary of Old Norse Prose
	\emph{ONP}%
}
\newcommand{\Skp}{% Skaldic Poetry of the Scandinavian Middle Ages
	\textciteshorttitle{SkP}%
}
%

% Render books
  % This file only contains the various books included. This is done so that it can be edited separately from main.tex, which contains the formatting code.

% Introduction, bibliography and abbreviations
\frontmatter%

\title{%
  \Huge The \textsc{Old Germanic Scoldship}, \\
  \huge\emph{or, \\
  \textsc{Scandinavian, English} and \textsc{German Mythic} and \textsc{Heroic Alliterative Poetry, Newly Translated, Edited} and \textsc{Commented upon}} \\
  \emph{by} \\
  \Huge \textsc{Konrad Olof Lennart Rosenberg}; \\ \emph{also \textsc{Including} a \textsc{List} of \textsc{Poetic Formulæ}, and \textsc{Several Essays} on the \textsc{Ancient \\ Common Germanic Culture} \\ and \textsc{Worldview}.}}

\maketitle

\newpage\thispagestyle{empty}

\begin{center} The following people have been especially helpful in giving corrections and general feedback: Ęinarr, Nikhilasurya Dwibhashyam, Joseph S. Hopkins, John Newman, Trevor L. Payne, Thibault.\end{center}

\begin{center} \emph{\alst{V}ęl kęypts hlutar \hld\ hęf’k \alst{v}ęl notit; \\
\alst{f}ás es \alst{f}róðum vant; \\
því-at \alst{Ó}ð-rǿrir \hld\ es nú \alst{u}pp kominn \\
á \alst{a}lda vés \alst{ja}ðar} \\
(\emph{Háva mǫ́l} 106)\end{center}

\newpage\thispagestyle{empty}

\tableofcontents

\newpage

\thispagestyle{empty}\section{Abbreviations}
  \begin{itemize}% Manuscript sigla
    \item \AM\ = AM 748 I a 4° (https://handrit.is/manuscript/view/da/AM04-0748-I-a)
    \item \AMb\ = AM 748 I b 4° (https://handrit.is/manuscript/view/is/AM04-0748-Ib)
    \item \EddaBms\ = AM 757 a 4° (https://handrit.is/manuscript/view/is/AM04-0757a)
    \item \FlatMS\ = Flatsęyjarbók, GKS 1005 fol. (https://handrit.is/manuscript/view/is/GKS02-1005)
    \item \Hauksbok\ = Hauksbók, AM 544 4° (https://handrit.is/manuscript/view/en/AM04-0544)
    \item \VolsungaMS\ = NKS 1824 b 4° (https://onp.ku.dk/onp/onp.php?m9641)
    \item \Regius\ = Codex Regius of the Poetic Edda, GKS 2365 4° (https://eae.ku.dk/q.php?p=cr/poems)
    \item \RegiusProse\ = Codex Regius of the Prose Edda, GKS 2367 4° (https://handrit.is/manuscript/view/is/GKS04-2367)
    \item \Trajectinus\ = Codex Trajectinus, Traj 1374ˣ
    \item \Upsaliensis\ = Codex Upsaliensis, DG 11
    \item \Wormianus\ = Codex Wormianus, AM 242 fol. (https://clarino.uib.no/menota/text/menota/AM-242-fol)
  \end{itemize}

  \begin{itemize}% Languages
    \item Eng. = Modern English
    \item Ger. = Modern German
    \item Got. = Gotnish (or Gothic)
    \item Lomb. = Lombardic
    \item MHG = Middle High German
    \item OE = Old English
    \item OF = Old Frisian
    \item OHG = Old High German
    \item ON = Old Norse
    \item OS = Old Saxon
    \item OSwe. = Old Swedish
    \item PGmc. = Proto-Germanic
    \item PN = Proto-Norse
    \item PNWGmc. = Proto-North-West Germanic
  \end{itemize}

  \begin{itemize}% Grammar
    \item 1st = first-person
    \item 2nd = second-person
    \item 3rd = third-person
    \item acc. = accusative case
    \item cpd = compound
    \item dat. = dative case
    \item gen. = genitive case
    \item imper. = imperative mood
    \item ind. = indicative mood
    \item instr. = instrumental case
    \item nom. = nominative case
    \item pl. = plural number
    \item sg. = singular number
    \item subj. = subjunctive mood
  \end{itemize}

  \begin{itemize}% Other abbreviations
    \item cert. = certainly
    \item c. = circa
    \item cf. = \emph{confere}; compare
    \item corr. = corrected in the ms.
    \item e. = excerpt (not the whole stanza)
    \item ed. = edition, edited (by)
    \item e.g. = \emph{exemplio gratia}; for instance
    \item emend. = emendation, emended (by)
    \item fol., foll. = folio, folios
    \item i.e. = \emph{id est}; that is
    \item l., ll. = line, lines
    \item lit. = literally
    \item metr. emend. = emended based on (secure) metrical criteria
    \item ms., mss. = manuscript, manuscripts
    \item norm. = normalised from the ms. spelling
    \item om. = omitted by
    \item p., pp. = page, pages
    \item tr. = translation, translated (by)
    \item sens. emend. = emended based on sense
    \item st., sts. = stanza, stanzas
    \item viz. = \emph{vidēlicet}; namely, to wit
    \item wo. = without
    \item wrt. = with regard to
  \end{itemize}

\newpage

\bookStart{Introduction (INCOMPLETE!)}

\section{Introduction to Eddic poetry}
  Don't go too indepth on individual poems! Each one will have its own introduction.
  \subsection{Metrics and conventions}
    Alliteration
    Kennings
  \subsection{How can we know the age of the Eddic poems?}
    Linguistic criteria
    Archeological evidence
    Comparison with known Christian texts (Sólarljóð, Hugsvinnsmál)
    Snorri thought they were old
    Saxo had access to them
    Many of them clearly describe non-Icelandic surroundings
      Especially Hávamál is clearly Norwegian

\section{Ancient Germanic cult(ure)}
  \subsection{Economy (fee)}
  \subsection{Morals}
    Honour, personal integrity
    Notes on the terms \emph{argr} and \emph{ergi}
  \subsection{Religious conceptions}
    Cosmic cycles
    Reincarnation
    Analogies with other Indo-European traditions

\section{Notes to English translation}
  Point about literal translation for use by scholars of comparative mythology
    The “guiding star” of this translation effort has been literality and consistency. All previous translations (to my knowledge) have such issues as: rendering identically repeated phrases differently at various places; covering up or obscuring technical and cultural terminology; simplifying kennings and other expressions—and this often without notes, to a point where the original meaning is, at times, unrecognizable.
    While I wholly encourage all readers of sufficient interest to study Old Norse (and other ancient Germanic languages!), perhaps even using the present edition as a tool, I also realize that this is a demanding ask which not all interested students and scholars of comparative mythology, anthropology, literature, religion and other fields will be able to fulfill. I therefore want these groups to be able to have a text that is as close to the original as possible, at the very least when it regards sense and expression.
  \subsection{Anglish proper nouns}
    One of the most idiosyncratic parts of the present edition will be its handling of proper nouns. I have opted to render all cultural and religious terms, names of places, heroes, gods, and other entities by their English cognates (thus \emph{Thunder} for Old Norse \emph{Þórr}) and where such do not exist, their philologically expected English (\emph{Anglish}) forms (e.g. \emph{wallow} for Old Norse \emph{vǫlva}).
    One reason for this is ideological. I believe that these myths and poems are a common Germanic or Northern European heritage, and should be treated as such. The English once knew gods such as Weden and Thunder, and called them by names naturally evolved in their language. So too did the Germans and Scandinavians, of course, and I would hope that any translators into those languages would follow this spirit and render the names in their natural forms there as well.\footnote{For instance in German perhaps Wuten, Donner, Froh, in Swedish Oden, Tor, Frö.}
    Another is philological. Forms like Odin and Thor are, while now commonly accepted, debased. They do not even represent the Old Norse pronunciation as accurate as would be possible (for instance, Odin would be better anglicized as Othin; the dental fricative still survives in English!), and many are difficult for English speakers to pronounce. I shudder when hearing a word like \emph{ę́sir} pronounced /aɪˈsɪ:ɹ/

\section{Notes to critical edition}
  My goal with the critical editing of the texts has been to produce something as close to the original mss. as possible, without excessive emendation to the preserved recension(s). There are texts in three languages in the present edition, namely Old Norse, Old English and Old High German. Old Norse texts have been normalized according to roughly the same orthography as \textcite{FinnurEdda}. On the other hand the Old High German and Old English texts have only been lightly normalized, correcting obvious errors and marking vowel length with acute accents.

  \subsection{Normalization}
    The general principle in normalizing texts has been to strive for a uniform orthography across languages, where the same sound is written with the same character. This of course means disregarding local manuscript traditions and philological tradition, but I see this as justified. My goal is to render the texts themselves in a manner that gives as much information to the reader as possible—not to present a facsimile edition for students of paleography. Anyway, such obvious aspects of the original manuscripts as the long \emph{ſ}, arbitrary punctuation, arbitrary spelling, and lack of line breaks are almost never reproduced in modern editions of Old Germanic poetry.

    \subsubsection{Normalization of poetry}
    \begin{enumerate}
      \item Lines are broken at each long-line, not each half-line. This follows traditional practice for the publication of West Germanic poetry, while departing from that of Old Norse poetry.
      \item Cæsuræ are represented with the interpunct (·).
      \item Alliterations are marked with red colour.
    \end{enumerate}

    \subsubsection{Normalization of Old West Norse}
    The orthography is inspired by \textcite{FinnurEdda} in that it strives for a more archaic form than that of the surviving mss., one that instead represents the poetry as it may (in many cases, must) originally have looked. For this reason, it often has more in common with the proposed orthography of the First Grammatical Treatise than with the standard Old Icelandic orthography seen in most editions. The following list describes the differences from the standard orthography.

    \begin{enumerate}
    \item I distinguish short \emph{e} (from etymological short \emph{e}) and short \emph{ę} (from etymological short \emph{a} + \emph{i}-umlaut).
    \item I distinguish long \emph{á} and \emph{ǫ́}, as done by the First Grammatical Treatise.
    \item I use \emph{ǿ} and \emph{ę́} rather than the traditional \emph{œ} and \emph{æ}, to represent the vowels descended from Proto-Norse \emph{ō} and \emph{ā} after \emph{i}-umlaut (cf. the short \emph{ø, ę} < \emph{o, a} + \emph{i}-umlaut).
    \item I distinguish long nasal \emph{ȧ, ė, ï, ȯ, u̇} from long oral \emph{á, é, í, ó, ú}, as done by the First Grammatical Treatise.
    \item I restore the old \emph{s}—which in modern Scandinavian and even in most Old Norse manuscripts has become \emph{r}, but which is found consistently in old manuscripts such as AM 237 a fol (c. 1150), and fossilized in forms like \emph{þaz} (i.e. \emph{þat’s}) in \Regius—in the words \emph{es} ‘which, that, where, when’, and in inflections of \emph{vesa} (later \emph{vera}) such as \emph{es} ‘is’ (3rd sg. pres. ind.) and \emph{vas} (3rd sg. pret. ind.). The following forms retain the \emph{r}, as it is there the result of Verner’s law, and not of this (much younger) sound change: the pl. pres. ind. (\emph{erum} \&c.), the pl. pret. ind. (\emph{vǫ́rum} \&c.), and the pl. pret. subj. (\emph{vę́rim} \&c.)
    \item When metrically benefactory, I contract \emph{ek} ‘I’, \emph{eru} ‘are’, and \emph{es} ‘which; is’ to \emph{’k}, \emph{’ru} and \emph{’s}, respectively.
    \item I use \textcite{FinnurEdda}’s way of distinguishing between the relative particle \emph{es} and the verb \emph{es}: the first is appended to the previous word with only an apostrophe (e.g. \emph{hann’s} ‘he who’), while the second is separated by a space (e.g. \emph{hann ’s} ‘he is’).
    \end{enumerate}

    \subsubsection{Normalization of Old English}

    \subsubsection{Normalization of Old High German}

  \subsection{Manuscripts}

    \subsubsection{Eddic poetry}
    There are two surviving ancient mss. which contain full Eddic poems.

    The first and most important is GKS 2365 4to, here \Regius. It dates to the 1270s and has 45 surviving leaves, containing TODO poems. Of these 10 are mythological, and the rest heroic, dealing with legends mostly of the Migration Period. Notably, following fol. 32, there is a large gap of missing pages. This occurs in the heroic section, specifically cutting off \Sigrdrifumal. It is unclear how many leaves and poems went missing.
    \Regius\ is not just a compilation of poems, it shows editorial input as well. Several of the mythological poems are separated by short prose sections, which tie them together into a loose frame narrative, though it is clear from their style and composition that they are originally separate works. When it comes to the heroic poems long prose sections occur both within and between them, creating a \inx[C]{saw}-like narrative where the prose in many cases holds up the poetry, rather than the reverse. For further literature see TODO.

    The second ms. is AM 748 I a 4to, here \AM. It dates to the 1300s and is but a fragment, consisting of just 6 leaves. It contains only mythological poems, and in a different order from \Regius; unlike it there is no trace of a frame narrative. On the first two leaves are contained the final stanzas of \Harbardsljod\ (1r–v), the complete \Baldrsdraumar\ (1v–2r), and the first verses of \Skirnismal, after which a single leaf has been lost. The next four leaves follow eachother and contain the second half of \Vafthrudnismal, the complete \Grimnismal\ and \Hymiskvida, and the beginning of the prose introduction to \Volundarkvida. \AM\ is the only medieval manuscript attesting \Baldrsdraumar, and its variants of the poems attested in \Regius\ are clearly not copied from it, but rather derive from a common ancestor. This makes it very valuable for textual criticism. For further literature see TODO.

    Several Eddic poems are quoted in \Gylfaginning, namely (TODO): \Voluspa, \Vafthrudnismal, \Grimnismal. The text also quotes a few fragmentary verses of Eddic character (possibly from lost Eddic poems), which have here been edited together with their surrounding prose passages. For \Gylfaginning\ I have relied on the following four main mss.:\begin{enumerate}
	   \item The Codex Regius of the Prose Edda \RegiusProse\ (GKS 2367 4to; 1300-1350)
     \item The Codex Trajectinus \Trajectinus\ (Traj 1374; a c. 1595 paper copy of a ms. closely related to \RegiusProse.)
     \item The Codex Wormianus \Wormianus\ (AM 242 fol.; 1340–70)
     \item The Codex Upsaliensis \Upsaliensis\ (DG 11; 1300–25)\end{enumerate}

    For discussion on their internal stemmatics and origins I refer to \textcite{Haukur2017}. When all employed witness mss. of \Gylfaginning\ agree on a reading the siglum \GylfMS\ is used in the critical apparatus, which is thus equivalent to \RegiusProse\Trajectinus\Wormianus\Upsaliensis.

    A few other Eddic poems have also been edited. One of them, \Rigsthula, only survives in \Wormianus, though it is sadly incomplete (see its Introduction). Other Eddic poems survive only in younger paper mss., namely: TODO. While I have not consulted these paper mss. for poems attested in medieval mss., I have had to rely on them for these poems. Their exclusive survival there does not necessarily prove them to be late antiquarian works, as is clearly shown by \Baldrsdraumar, which among medieval mss. is only attested in the fragmentary \AM. It thus cannot be excluded that some of these poems would have existed in other lost medieval mss., perhaps even in the lost pages of \Regius\ or \AM.

    \subsubsection{West Germanic poetry}

    As none of the West Germanic poems edited here (TODO: Will we be editing other poems than Hildebrandslied?) survive in more than one copy, the specific details of their transmission is discussed in their individual Introductions.

  \printbibliography% Does it work?
%

\mainmatter%

\part{Heathen Mythic Poetry}% Theology, mythology, order generally independent of mss.
	\bookStart{Spae of the Wallow}[Vǫluspǫ́]

\begin{flushright}%
\textbf{Dating} \parencite{Sapp2022}: C10th (0.865)–early C11th (0.121)

\textbf{Meter:} \Fornyrdislag%
\end{flushright}

\section{Introduction}

The \textbf{Spae of the Wallow} (\Voluspa) is the most comprehensive mythological text surviving from Heathen times.  The poem is a \inx[C]{spae} (\emph{spǫ́} ‘prophecy’) in the form of a monologue spoken by a \inx[C]{wallow} (\emph{vǫlva} ‘seeress, sibyl, prophetess’) summoned by the god Weden in order to relate mythological knowledge.  Weden’s frequent journeys to question various beings about mythological lore should be seen in the light of his incessant lust for knowledge and wisdom.  The most similar instance is \Baldrsdraumar, wherein Weden summons another wallow out of her grave in \inx[L]{Hell} in order to find out why the god \inx[P]{Balder} is having ominous nightmares.  There is also \Vafthrudnismal, wherein Weden challenges the wise ettin \inx[P]{Webthrithner} to a wisdom contest and defeats him.  These journeys are further alluded to in \Harbardsljod\ TODO.

In its being a mythic catalogue \Voluspa\ also resembles (parts of) poems like \Havamal, \Grimnismal, \Sigrdrifumal, and \Allvismal, but it differs from them all in a key way: instead of being a motley collection of scattered mythological lore, \Voluspa\ offers a chronological overview of the whole Norse mythic timeline, from the creation of the world to its demise and rebirth.

That is not to say that the events in it are described in a straight-forward manner; they are related in a highly allusive fashion that presupposes that the audience is already familiar with them.  There may also be some later omissions and inserts that make the poem more difficult to read.

\Voluspa\ is attested in full in two independent recensions.  The first and most important is \Regius, where it is the first poem and found on foll. 1r–3r; the other is \Hauksbok, where it is found in the middle of a large collection of saws and Catholics works at 20r–21r.

Many stanzas from the poem are also cited or paraphrased in \Gylfaginning, for which \Voluspa\ was clearly one of the main sources.  These paraphrases are still of critical value, e.g. in st. 19, where \emph{sal} ‘hall’ in the paraphrase agrees with \Hauksbok\ against \Regius\ \emph{sę́} ‘lake’.  For the four mss. of \Gylfaginning—\RegiusProse, \Trajectinus, \Wormianus, and \Upsaliensis—see the General Introduction.

For the differences between the mss. the reader may consult the following table prepared by the editor.  The several stanzas in \Gylfaginning, which are quoted independently and with little relation to the order of the original poem, are marked with plus signs.  The sequences containg uninterrupted quotations of several stanzas are marked with an incrementing alphabetic symbol, so that \emph{B1} is the first stanza in the second sequence, and so on.  When a stanza found in a ms. is strongly divergent (e.g. st. 10, where \Gylfaginning\ omits the first two half-lines), its number is followed by a star.  The stanzas beginning with \emph{Þȧ gingu ręgin ǫll} ‘Then went the Reins all’ are represented by the half-line immediately following.

\begin{longtabu} to \textwidth {|c c c c c c|}
	\hline
	\multicolumn{2}{|c}{\emph{pres. ed.}} & \Regius & \Hauksbok & \RegiusProse\Trajectinus\Wormianus & \Upsaliensis \\ [0.5ex]
	\hline\hline\endhead
	\hline\endfoot
	1 & Hljóðs bið’k allar & 1 & 1 & − & − \\
	2 & Ek man jǫtna & 2 & 2 & − & − \\
	3 & Ár vas alda & 3 & 3 & + & + \\
	4 & áðr Burs synir & 4 & 4 & − & − \\
	5 & Sól varp sunnan & 5 & 5 & +* & +* \\
	6 & \dots\ nǫ́tt ok niðjum & 6 & 6 & − & − \\
	7 & Hittusk ę̇sir & 7 & 7 & − & − \\
	8 & Tęflðu ï tu̇ni & 8 & 8 & − & − \\
	9 & \dots\ hvęrr skyldi dverga & 9 & 9 & B1 & B1 \\
	10 & Þar vas Móðsognir & 10 & 10 & B2* & B2* \\
	11–15 & \emph{Dwarf-tallies} & 11–15 & 11–16 & + & + \\
	16 & Unds þrír kvǫ̇mu & 16 & 17 & − & − \\
	17 & Ǫnd þau né ǫ́ttu & 17 & 18 & − & − \\
	18 & Ask vęit’k standa & 18 & 19 & + & + \\
	19 & Þaðan koma męyjar & 19–20 & 20–21 & − & − \\
	20 & Þat man hǫ̇n folk-víg & 21–22 & 27 & − & − \\
	21 & Hęiði hétu & 23 & 28 & − & − \\
	22 & \dots\ hvárt skyldu ę̇sir & 24 & 29 & − & − \\
	23 & Flęygði Óðinn & 25 & 30 & − & − \\
	24 & \dots\ hvęrr hęfði lopt alt & 26 & 22 & C1 & C1 \\
	25 & Þȯrr ęinn þar vá & 27 & 23 & C2* & C2* \\
	26 & Vęit hǫ̇n Hęimdalar & 28 & 24 & − & − \\
	27 & Ęin sat hǫ̇n úti & 29 & − & − & − \\
	28 & Alt vęit’k, Óðinn & 29 & − & + & + \\
	29 & Valði hęnni Hęr-fǫðr & 30 & − & − & − \\
	30 & Sá hǫ̇n val-kyrjur & 31 & − & − & − \\
	31 & Ek sá Baldri & 32 & − & − & − \\
	32 & Varð af męiði & 33 & − & − & − \\
	33 & Þó hann ę́va hęndr & 34 & − & − & − \\
	H1 & Þȧ kná Váli & − & 31 & − & − \\
	34a & Hapt sá hǫ̇n liggja & 35a & − & − & − \\
	34b & þar sitr Sigyn & 35b & 32 & − & − \\
	35 & Ǫ́ fęllr austan & 36 & − & − & − \\
	36 & Stóð fyr norðan & 36 & − & − & − \\
	37 & Sal sá hǫ̇n standa & 37 & 36 & E1 & E1 \\
	38 & Sér hǫ̇n þar vaða & 38 & 37 & E2* & E2* \\
	39 & Austr býr hin aldna & 39 & 25 & A1 & A1 \\
	40 & Fyllisk fjǫrvi & 40 & 26 & A2 & A2 \\
	41 & Sat þar ȧ haugi & 41 & 34 & − & − \\
	42 & Gól of ǫ̇sum & 42 & 35 & − & − \\
	43, 48, 56 & Gęyr (nú) Garmr mjǫk & 43, 46, 55 & 33, 38, 43, 48, 51 & − & − \\
	44 & Brǿðr munu bęrjask & 44 & 39 & − & − \\
	45 & Lęika Mïms synir & 45 & 40 & D1* & D1* \\
	H2 & Hrę́ðask allir & − & 41 & − & − \\
	46 & Hvat ’s með ǫ̇sum? & 49 & 42 & D2 & D2* \\
	48 & Hrymr ękr austan & 47 & 44 & D3 & − \\
	49 & Kjóll fęrr austan & 48 & 45 & D4 & − \\
	50 & Surtr fęrr sunnan & 50 & 46 & +, D5 (cited twice) & + \\
	51 & Þȧ kømr Hlïnar & 51 & 47 & D6 & − \\
	52 & Þȧ kømr hinn mikli & 52 & − & D7 & − \\
	H3 & Gïnn lopt yfir & − & 48 & — & − \\
	53 & Þȧ kømr hinn mę́ri & 53* & 49* & D8 & − \\
	54 & Sól tér sortna & 54 & 50 & D9 & − \\
	56 & Sér hǫ̇n upp koma & 56 & 52 & − & − \\
	57 & Finnask ę̇sir & 57* & 53 & − & − \\
	58 & Þar munu ęptir & 58 & 54 & − & − \\
	59 & Munu ȯ·sánir & 59 & 55 & − & − \\
	60 & Þȧ kná Hø̇nir & 60 & 56 & − & − \\
	61 & Sal sér hǫ̇n standa & 61 & 57 & + & + \\
	H4 & Þȧ kømr hinn ríki & − & 58 & − & − \\
	62 & Þar kømr hinn dimmi & 62 & 59 & − & − \\ [1ex]
	\hline
\end{longtabu}


\sectionline

The poem begins with a bid for silence (1), and the wallow recalling her earliest memories (2). She then recounts the ordering of the world by the gods (3–6) and the golden age of peace and plenty (7–8), which is, however, interrupted by the intrusion of three unidentified ettin-maidens (8, and see note there). After this follow two verses about the shaping of the dwarfs (9–10), and then several originally separate \emph{dwarf-tallies} (11–15), which are without doubt later inserts. Returning to the main narrative thread is described the creation and endowment of the first man and woman (16–17), Ugdrassle’s Ash (18), and the three \inx[G]{norns} living under it (19).

At this point the two full redactions of the poem (\Regius\ and \Hauksbok) diverge. Because of its older age and greater count of stanzas I have here followed the order of \Regius: the wallow recalls how a woman named Goldwey was sacrificed and reborn three times (20), and how she, under the name Heath, practiced sorcery and witchcraft (21). She then recalls the first war in the world, between the Eese and Wanes (22–23), and alludes to the slaying of the smith, who according to \Gylfaginning\ 42 was promised \inx[P]{Frow} and the sun and moon in exchange for building the wall of Osyard (24-25). This is followed by a cryptic verse describing Homedal’s hidden silence or hearing (26).

In \Hauksbok\ the structure is quite different. After the description of the norns (19), the Eese immediately go to decide what action to take regarding the promising of Frow to the ettin (24-25), and Homedal’s hearing is described (26). Then follow the two sts about the wolves that will swallow the sun and moon (40-41), and after this come sts 20–23 in the same order as \Regius\ (see above).

TODO.

\sectionline

\section{The Spae of the Wallow}

\bvg\bva\mssnote{\Regius~1r/2, \Hauksbok~20r/1}%
\edtext{„\edtrans{\alst{H}ljóðs bið’k}{For hearing I ask}{\Bfootnote{The same introductory expression is found in st. 2 of Eyel’s Head-ransom (Egill \emph{Hfl} in \Skp\ 5): \emph{hljóðs biðjum hann} ‘for hearing we [I] ask him’.}} allar \hld\ \edtrans{\alst{h}ęlgar}{holy}{\Afootnote{so \Hauksbok; om. \Regius}\Bfootnote{That the omission of this word in \Regius\ is nothing more than a scribal error is clearly shown by the meter; the a-verse in \emph{Hljóðs bið \alst{e}k \hld\ \alst{a}llar kindir} is only three syllables long, and has highly unnatural alliteration on the unstressed \emph{ek} rather than the expected first nominal \emph{hljóðs}.}} kindir, &
\edtrans{\alst{m}ęiri ok \alst{m}inni}{greater and lesser}{\Bfootnote{It is ambiguous to which phrase these adjectives belong.  It may either be (a) ‘holy kindreds greater and lesser’, which could be equivalent to the phrase \inx[F]{Eese and Elves} (both earthly and heavenly supernatural beings; see Index for occurrences); or (b) ‘greater and lesser lads of Homedal’.  (b) is probably to be preferred as the more natural reading, in which case ‘greater or lesser’ may refer literally to physical size (the younger and older members of the audience) or more figuratively to the various social classes.}} \hld\ \edtrans{\alst{m}ǫgu Hęimdalar}{lads of Homedal \ken{men}}{\Bfootnote{Homedal sired the three castes of men, as told in \Rigsthula.}}; &
\alst{v}ilt at, \edtrans{\alst{V}al-fǫðr}{Walfather}{\Bfootnote{That is, “Father of the Slain”.  This name is probably used of Weden since he awoke her from her grave; cf. st. 62/4.}}, \hld\ \alst{v}ęl fram tęlja’k &
\alst{f}orn spjǫll \alst{f}ira, \hld\ \edtrans{þau’s \alst{f}ręmst of man}{which I foremost recall}{\Bfootnote{Cf. \Vafthrudnismal\ 34–35 with similar phrasing.}}?}{\lemma{ALL}\Bfootnote{The wallow begins by asking for the silence of both gods and men, a meristic expression \parencite[99--100]{West2007}.  The whole introductory formula has Indo-European parallels; see \textcite[63,92--93,312]{West2007}.}}\eva

\bvb “For hearing I ask all holy races \ken{gods}, \\
greater and lesser lads of \inx[P]{Homedal} \ken{men}! \\
Wilt thou, \inx[P]{Walfather} \name{= Weden}, that I well tell forth \\
the ancient sayings of men which I foremost recall?\evb\evg


\bvg\bva\mssnote{\Regius~1r/4, \Hauksbok~20r/2}%
\alst{E}k man \alst{jǫ}tna \hld\ \alst{á}r of borna, &
þȧ’s \alst{f}orðum mik \hld\ \alst{f}ǿdda hǫfðu; &
\alst{n}íu man’k hęima, \hld\ \alst{n}íu \edtext{ïviðjur}{\Afootnote{so all.  \Regius\ has previously been as read \emph{‘iviði’}, but this was made obsolete by an x-ray scan undertaken by \textcite{StefanKarlsson1979} revealing a tiny abbreviation mark for \emph{-ur}.}\Bfootnote{Evil-working women or ogresses; this word also appears in a list of names for trollwomen (Þul \emph{Trollkvenna} 3 in \Skp\ 3).  The word is a fem. \emph{jōn}-stem.  A commonly suggested etymology is \emph{í} ‘in’ + \emph{viðr} ‘wood’ (i.e. forest-dwellers), but this would be an unusual formation, and leaves the \emph{-j-} unexplained.  A more plausible etymology is an agent-noun based on \emph{*ïvið} ‘guile, malice’, attested in the cpd. \emph{ïvið-gjarn} (\Volundarkvida\ 28).  This etymology can also explain the \emph{-j-}, since its WGmc. cognates OE \emph{inwid}, OS \emph{inwid}, and OHG \emph{inwit} show it to be a neutr. \emph{ja}-stem.}}, &
\edtrans{\alst{m}jǫt-við \alst{m}ę́ran \hld\ fyr \alst{m}old neðan.}{the renowned measure-tree beneath the soil.}{\Bfootnote{Probably \inx[L]{Ugdrassle’s Ash}, being still a seed.}}\eva

\bvb I recall \inx[G]{Ettins} born of yore, \\
those who formerly had nourished me. \\
Nine \inx[C]{Home}[Homes] I recall, nine \inx[G]{Inwithies}; \\
the renowned measure-tree beneath the soil.\evb\evg


\bvg\bva\mssnote{\Regius~1r/6, \Hauksbok~20r/4, \GylfMS}%
\alst{Á}r vas \alst{a}lda \hld\ \edtrans{þar’s \alst{Y}mir byggði}{where Yimer dwelled}{\Afootnote{\emph{þat’s ękki vas} ‘when nothing was’ \GylfMS}}, &
vas-a \alst{s}andr né \alst{s}ę́r, \hld\ né \alst{s}valar unnir; &
\edtext{\alst{jǫ}rð fannsk \alst{ę́}va \hld\ né \alst{u}pp-himinn}{\lemma{jǫrð \dots\ né upp-himinn ‘Earth \dots\ nor Up-heaven’}\Bfootnote{A well-attested formulaic cosmological word-pair found in all four Old Germanic languages with alliterative poetic traditions (viz. ON, OE, OS, OHG), especially in the context of the creation and destruction of the world.  See Index: \inx[F]{Earth and Upheaven}.}}; &
\edtrans{\alst{g}ap vas \alst{g}innunga}{there was the Gap of Ginnings \ken{air/midspace}}{\Bfootnote{In \Gylfaginning\ Snorre presents \emph{ginnunga-gap} as a physical place existing between Earth and Upheaven during the beginning of the universe, but that may simply be an idiosyncrasy of that author, and finds no support in older sources.  Indeed the present stanza is the only occurrence of the combination of the words \emph{gap} and \emph{ginnunga}, outside of Snorre’s Edda.

I reject as unfounded the traditional translation “yawning chaos”, and instead agree with Meissner in reading \emph{gap ginnunga} as a kenning “gap of hawks \ken{air}”, where \emph{ginnunga} is gen. pl. of \emph{ginnungr} ‘hawk’.  The kenning-type “land, path of the bird \ken{air}” is conventional \parencite[108]{Meissner1921}, and the determinant \emph{ginnungr} is also found in a kenning in \Haustlong\ 15: \emph{ǫll endi-lǫ́g ginnunga vé} ‘all the end-low mansions of hawks \ken{skies}’.  This interpretation is confirmed by \Skaldskaparmal\ 74, which lists it among synonyms (\emph{hęiti}) for the air: \emph{Lopt heitir ginnunga-gap ok meðal-heimr, fogl-heimr, veðr-heimr.} ‘Air is called gap of ginnings and middle-home, bird-home, weather-home.’%
%TODO: bibliography, Meissner is 20. Luft, b

In the old Germanic cosmology the air was the midspace (whence \emph{meðal-heimr} ‘middle-home’) between Earth and Upheaven; not synonymous with the latter.  This is also why \Haustlong\ 15 speaks of the “low \textsc{skies}”, contrasted with “Upheaven” or High Heaven in st. 16.}}, \hld\ en \alst{g}ras \edtrans{hvęrgi}{nowhere}{\Afootnote{\emph{ękki} ‘not’ \Hauksbok}};\eva

\bvb It was early of ages where \inx[P]{Yimer} dwelled; \\
there was not sand nor sea nor cool waves. \\
\inx[L]{Earth} was never found, nor \inx[L]{Up-heaven}; \\
there was the \inx[L]{Gap of Ginnings} \ken{air/midspace}, but grass nowhere,\footnoteB{A more extensive creation narrative is found in \Gylfaginning\ 4–5, according to which the world first consisted of two extremities: the frozen Nivelham in the north and scorching Muspellsham in the south. From Nivelham the freezing venom-rivers called the \inx[L]{Ilewaves} ran until they froze to ice, while burning lava flowed from Muspellsham. The ice and lava met in the Gap of Ginnings, “which was as calm as windless air”, and there combined to form the first being, \inx[P]{Yimer}, who was the ancestor of the ettins.}\evb\evg


\bvg\bva\mssnote{\Regius~1r/8, \Hauksbok~20r/5}%
áðr \edtrans{\alst{B}urs synir}{the Sons of Byre}{\Bfootnote{In \Gylfaginning\ 6 identified as Weden, Will, and Wigh.  They sacrificed Yimer and shaped the world out of his body, for which cf. \Grimnismal\ 41–42, \Vafthrudnismal\ 21.}} \hld\ \alst{b}jǫðum of ypðu, &
þęir es \alst{M}ið-garð \hld\ \alst{m}ę́ran skópu; &
\alst{s}ól skęin \alst{s}unnan \hld\ ȧ \alst{s}alar stęina; &
þȧ vas \alst{g}rund \alst{g}róin \hld\ \edtrans{\alst{g}rø̇num lauki}{green leek}{\Bfootnote{A sign of the golden age, for the leek was in ancient times held to be the noblest plant. See Index.}}.\eva

\bvb before the \inx[P]{Sons of Byre} uplifted the flatlands, \\
they who shaped renowned \inx[L]{Middenyard}. \\
The sun shone from the south on the stones of the hall; \\
then was the ground grown with green leek.\evb\evg


\bvg\bva\mssnote{\Regius~1r/11, \Hauksbok~20r/7, \GylfMS}%
\edtext{\alst{S}ól varp \alst{s}unnan, \hld\ \edtrans{\alst{s}inni Mȧna}{Moon’s companion}{\Bfootnote{At times translated as ‘her moon’, understanding \emph{sinni} as dat. sg. f. of \emph{sïnn} ‘its (reflexive)’.  This cannot be correct since ON possessives are inflected based on the gender of the noun they modify, not the gender of the possessor.  \emph{mȧni} ‘moon’ is masculine, and so ‘her moon’ would be \emph{sïnum Mȧna}.}}, &
\alst{h}ęndi hinni \alst{h}ǿgri \hld\ of \edtrans{\alst{h}imin-jǫður}{heaven’s rim}{\Afootnote{composite; \emph{himin †iodyr†} \Regius; \emph{ioður} \Hauksbok.}\Bfootnote{Some recent editors have taken it upon themselves to normalize the reading of \Regius\ as \emph{himin-jó-dýr} ‘heaven-horse-beast’, which is not just nonsensical but also unmetrical due the stress pattern.  On the other hand the reading of \Hauksbok, normalized to \emph{jǫður} ‘rim, edge’, is clearly deficient since it lacks the neccessary alliteration on \emph{h}.  If we see \emph{iodyr} \Regius\ as corrupted from \emph{*iodur} we can restore \emph{himin-jǫður}, as done here.}}}{\lemma{Sól \dots\ himin-jǫður ‘Sun \dots\ heaven’s rim’}\Afootnote{om. \GylfMS.}\Bfootnote{Probably a poetic description of the dawn; the Sun lifted herself up over the horizon and rose for the first time.}}; &
\alst{S}ól þat né vissi, \hld\ hvar hǫ̇n \alst{s}ali átti; &
\edtext{\alst{st}jǫrnur þat né vissu, \hld\ hvar þę́r \alst{st}aði ǫ́ttu}{\lemma{stjǫrnur \dots\ ǫ́ttu}\Afootnote{In \GylfMS\ this line comes last, so that the order is sun, moon, stars.}}; &
\edtext{\alst{M}ȧni þat né vissi, \hld\ hvat hann \alst{m}ęgins átti.}{\lemma{Mȧni \dots\ átti ‘Moon \dots\ had’}\Bfootnote{The moon was believed to have supernatural powers and could be invoked in conflict (cf. \Havamal\ 137/7.)}}
\eva

\bvb The Sun cast from the south—the Moon’s companion— \\
her right hand over heaven’s rim. \\
The Sun knew not where halls she had; \\
the stars knew not where seats they had; \\
the Moon knew not what sort of might he had.\evb\evg


\bvg\bva\mssnote{\Regius~1r/13, \Hauksbok~20r/9}%
\edtext{Þȧ gingu \alst{r}ęgin ǫll \hld\ ȧ \edtrans{\alst{r}ǫk-stóla}{rake-seats}{\Bfootnote{Their seats of judgment at the \inx[C]{Thing}.}}, &
\alst{g}inn-hęilǫg \alst{g}oð, \hld\ ok umb þat \alst{g}ę̇ttusk}{\lemma{Þȧ \dots\ gę̇ttusk ‘Then \dots\ of this.’}\Bfootnote{A formulaic expression for the convening of the \inx[L]{Thing of the Gods}, identically repeated below in sts. 9/1–2, 22/1–2, and 24/1–2.  Cf. also the formula shared between \Baldrsdraumar\ 1/1–3 and \Thrymskvida\ 14/1–3, which follows the structure of the present formula very closely: \emph{Sęnn vǫ́ru ę̇sir \hld\ allir ȧ þingi // ok ǫ̇synjur \hld\ allar ȧ máli, // ok umb þat réðu \hld\ ríkir tívar.} ‘Soon were the \inx[G]{Eese} all at the \inx[C]{Thing}, // and the \inx[G]{Ossens} all at speech, // and of this counseled the mighty \inx[G]{Tews}.’

In the five occurrences of these two formulae outside of the present stanza, the demonstrative pronoun \emph{þat} ‘this’ clearly refers to an immediately following question introduced by a \emph{hv}-word (e.g. \Thrymskvida\ 14/4: \emph{hvé þęir Hlórriða \hld\ hamar of sǿtti?} ‘how they Loride’s \name{= Thunder’s} hammer would find?’)  Following this pattern we would expect to find such a question following \emph{umb þat gę̇ttusk} ‘took counsel of that’ in the present stanza, and it seems reasonable plausible (but not certain) that one has been lost in transmission.}}. &
\edtext{\alst{N}ǫ́tt ok \alst{n}iðjum \hld\ \alst{n}ǫfn of gǫ́fu, &
\alst{m}orgin hétu \hld\ ok \alst{m}iðjan dag, &
\alst{u}ndurn ok \alst{a}ptan, \hld\ \alst{ǫ́}rum at tęlja.}{\lemma{Nǫ́tt \dots\ tęlja ‘To night \dots\ tally’}\Bfootnote{Cf. \Vafthrudnismal\ 23, where it is said that the sun and moon turn round in heaven \emph{ǫldum at ár-tali} ‘for mankind’s tally of years’, and 25, where it is said that the Reins created the moon-phases for the same purpose.}}\eva

\bvb Then went the Reins all onto the rake-seats: \\
the Yin-holy Gods, and from each other took counsel of that. \\
To night and the moon-phases names they gave; \\
morning they named, and middle day, \\
afternoon and evening, the years for to tally.\evb\evg


\bvg\bva\mssnote{\Regius~1r/16, \Hauksbok~20r/10}%
Hittusk \alst{ę̇}sir \hld\ ȧ \alst{I}ða-vęlli, &
\edtext{þęir’s \alst{h}ǫrg ok \alst{h}of \hld\ \alst{h}ǫ́-timbruðu}{\lemma{þęir’s \dots\ hǫ́-timbruðu ‘they who \dots\ timbered on high’}\Afootnote{\emph{afls kostuðu \hld\ alls freistuðu} ‘[their] strength they tried; everything they tempted’ \Hauksbok}\Bfootnote{Two formulæ. — \emph{hǫrgr ok hof} ‘harrow and hove’ is a merism, i.e. ritual structures made of stone and wood; cf. \Vafthrudnismal\ 38 and \HelgakvidaHjorvardssonar\ TODO, as well as the Norwegian Christian laws that impose ‘the burning of hoves and the breaking of harrows’ (\emph{brenna hof ok brjóta hǫrga}). — \emph{hǫ́-timbra} ‘timber on high’ is a rare compound. Its only other occurrence in the ON corpus is in \Grimnismal\ 16, where it describes a harrow ruled by Nearth. —

This line has often been wondered at; why would the Gods themselves make cultic buildings?  Yet they partake in ritual slaughter of beasts, divination, and feasting (e.g. \Voluspa\ 61, \Hymiskvida\ 1, 39, \Lokasenna, \Haustlong\ 2), and their deeds form the precedent for upright human behaviour.}}; &
\alst{a}fla lǫgðu, \hld\ \alst{au}ð smíðuðu, &
\alst{t}angir skópu \hld\ ok \alst{t}ól gęrðu.\eva

\bvb The Eese found each other on the \inx[L]{Idewolds}, \\
they who \inx[C]{harrow} and \inx[C]{hove} timbered on high. \\
Hearths they laid, wealth they smithed, \\
tongs they shaped and tools they made.\evb\evg


\bvg\bva\mssnote{\Regius~1r/18, \Hauksbok~20r/12}%
\edtext{\edtrans{\alst{T}ęflðu}{played Tables}{\Bfootnote{A verb derived from \emph{tafl} ‘board game’, an old borrowing from Latin \emph{tabula}.  “Tables” is used as a cognate translation; the exact type of board game referred to is unimportant.}} ï \alst{t}u̇ni, \hld\ \alst{t}ęitir vǫ́ru, &
\edtrans{\alst{v}as þęim \edtrans{\alst{v}éttir-gis}{nothing}{\Bfootnote{An archaic gen. of \emph{vę́tt-ki} ‘nothing’; the \emph{-ir} representing a fossilized i-stem genitive, for \emph{véttr} ‘thing’ comes from PGmc. \emph{*wihtiʀ}.  The only other occurrence of this form is in the highly linguistically archaic Icelandic Homily Book (ms. Holm perg 15 4°, fol. 36v/30).}} \hld\ \alst{v}ant ór gulli}{for them was nothing golden wanting}{\Bfootnote{Indeed even the bricks they played with were of gold. See st. 58.}}, &
unds \edtext{\alst{þ}ríar kvǫ̇mu \hld\ \alst{þ}ursa męyjar}{\lemma{þríar \dots\ þursa męyjar ‘three maidens of Thurses’}\Bfootnote{These three maidens are never mentioned again (unless they are taken to be the three norns in st. 19, but they would then be introduced twice). It is possible that an additional stanza giving further information about them has been lost. If it originally existed, it was already absent in the version used for \Gylfaginning, since no additional information is found there.}}, &
\edtrans{\alst{ȧ}m-átkar}{uncanny}{\Bfootnote{The word \emph{ám-áttigr} has a clear association with supernatural beings; trolls and ettins. It occurs in four other places in \Regius. In \Grimnismal\ 11, \Skirnismal\ 10 and \HelgakvidaHjorvardssonar\ 17 it modifies \emph{jǫtunn} ‘ettin’ in a \Ljodahattr\ c-line. In \HelgakvidaHjorvardssonar\ 14 it is used by the daughter of an ettin to refer to a human hero.}} mjǫk, \hld\ ór \alst{Jǫ}tun-hęimum.}{\lemma{ALL}\Bfootnote{The whole stanza is paraphrased in \Gylfaginning\ ch. 14: \emph{Ok því nę́st smíðuðu þeir málm ok stein ok tré ok svá gnóg-liga þann málm, er gull heitir, at ǫll bús-gǫgn ok ǫll reiði-gǫgn hǫfðu þeir af gulli, ok er sú ǫld kǫlluð gull-aldr, áðr en spilltist af til-kvámu kvinnanna; þę́r kómu ór Jǫtun-heimum.} ‘And after this they smithed ore and stone and wood, and so abundantly [did they smith] that ore which is called gold, that all their house tools and riding tools were golden. And that age is called the golden age, before it was spoiled by the arrival of the women; they came from Ettinham.’}}\eva

\bvb They played \inx[C]{Tables} in the yard; merry were they; \\
for them was nothing golden wanting— \\
until three maidens of \inx[G]{Thurses} came, \\
most uncanny, out of \inx[L]{Ettinham}.\evb\evg

\sectionline

\bvg\bva\mssnote{\Regius~1r/20, \Hauksbok~20r/14, \GylfMS}%
\edtext{Þȧ gingu \alst{r}ęgin ǫll \hld\ ȧ \alst{r}ǫk-stóla, &
\alst{g}inn-hęilǫg \alst{g}oð, \hld\ ok umb þat \alst{g}ę̇ttusk: &
\edtrans{Hvęrr skyldi \alst{d}verga}{Who would \dots\ of dwarfs’}{\Afootnote{so \Regius\Wormianus\Upsaliensis; \emph{at skyldi dverga} ‘That they would \dots\ of dwarfs’ \RegiusProse\Trajectinus; \emph{hverir skyldu dvergar} ‘Which dwarfs would [shape the retinues]’ \Hauksbok}} \hld\ \edtrans{\alst{d}rótt}{the retinue}{\Afootnote{so \GylfMS; \emph{drotin} ‘the lord’ \Regius; \emph{dróttir} ‘the retinues’ \Hauksbok}} \edtrans{of skępja}{shape}{\Afootnote{\emph{spekia} ‘soothe’ \Upsaliensis}} &
\edtrans{ór \edtrans{\alst{b}rimi \alst{b}lóðgu}{bloody surf}{\Afootnote{so \Hauksbok\RegiusProse\Wormianus\Upsaliensis; \emph{Brimis blóði} ‘the blood of Brimmer’ \Regius\Trajectinus}} \hld\ ok ór \edtrans{\alst{b}lǫ́um}{blue-black}{\Afootnote{metr. emend. from \emph{blám} \Regius; \emph{Bláins} ‘Blown’s’ \Hauksbok\Wormianus; \emph{Bláms} \RegiusProse\Trajectinus\Upsaliensis\ is prob. a corrupt form of \emph{Bláins}}} lęggjum}{from the bloody surf and from the blue-black legs}{\Bfootnote{I think that the poem simply telling of “the bloody surf” and “the blue-black legs” fits better with its general allusive style, but this requires a composite reading.  If we read \emph{Bláinn} ‘Blown’ (named in the \inx[C]{thule}[thules] as a dwarf) instead of \emph{blǫ́um} ‘blue-black’, then following Gurevich (\emph{Skp} 2017, p. 693) we may see a kenning “the legs of Blown \name{dwarf} \ken{stones}”.  Blown has otherwise usually been read as a poetic name for Yimer, but it is not attested anywhere else. — The “blood” and “legs” are in any case those of Yimer; from his bones were made the rocks, and from his blood the sea (see \Grimnismal\ 41, \Vafthrudnismal\ 21).  Dwarfs of course dwell in rocks and earth; cf. for instance \Ynglingatal\ 2, where the Swedish king Swayther (\emph{Svęigðir}) runs into a rock in pursuit of a dwarf.  More difficult to explain is the creation of dwarfs from the sea.  Einheri suggests that it may be referring to the formation of salt-stones by means of evaporating salty seawater.}}?}{\lemma{ALL}\Bfootnote{After the Golden Age is spoiled, the Gods must get their metal in some other way.  For this they need the dwarfs, who are connected with finding minerals, perhaps through techniques similar to dousing.  Ancient ideas about the spontaneous generation of maggots in flesh (likened to minerals in the earth) are also clearly at play. — \Gylfaginning\ 14 continues with its paraphrase: \emph{Þar nę́st settust goðin upp í sę́ti sín ok réttu dóma sína ok minntust, hvaðan dvergar hǫfðu kviknat í moldinni ok niðri í jǫrðunni, svá sem maðkar í holdi. Dvergarnir hǫfðu skipazt fyrst ok tekit kviknun í holdi Ymis ok váru þá maðkar, en af atkvę́ðum goðanna urðu þeir vitandi mann-vits ok hǫfðu manns líki ok búa þó í jǫrðu ok í steinum. Móðsognir var ǿðstr ok annarr Durinn. Svá segir í Vǫluspá:} ‘Thereafter the gods set themselves up in their seats and made their judgments and remembered whence the dwarfs had come to life in the ground and down in the earth like maggots in flesh. The dwarfs had first taken shape and come to life in Yimer’s flesh and were then maggots, but by the decrees of the gods they became knowing of manwit and had a man’s likeness, and even so they live in the earth and in stones. Moodsowner was the highest in rank, and second Dorn. So it says in the Spae of the Wallow:’ after which the text quotes the present st. and 10/3–4.}}\eva

\bvb Then went the Reins all onto the rake-seats: \\
the Yin-holy Gods, and from each other took counsel of this: \\
Who would shape the retinue of \inx[G]{Dwarfs}, \\
from the bloody surf and from the blue-black legs?\evb\evg


\bvg\bva\mssnote{\Regius~1r/21, \Hauksbok~20r/15, \GylfMS}%
\edtext{\edtext{Þar vas \alst{M}óðsognir}{\Afootnote{so \Hauksbok; \emph{Þar †mótſognir vitnir†} ‘there Mootsowner wolf(?)’ \Regius. The prose of \Gylfaginning\ 14 agrees with \Hauksbok\ that the correct form of the name is \emph{Móðsognir}, not \emph{Mótsognir}.}} \hld\ \alst{m}ę́tstr of orðinn &
\alst{d}verga allra, \hld\ en \alst{D}urinn annarr;}{\lemma{Þar \dots\ annarr ‘There \dots\ second’}\Bfootnote{om. \GylfMS, but the author must have had the full stanza, since he paraphrases these lines (see Note to ALL for st. 9 above).}} &
\edtext{\edtext{þęir \alst{m}an-líkun \hld\ \alst{m}ǫrg of gęrðu}{\lemma{þęir \dots\ gęrðu ‘They \dots\ did make’}\Afootnote{so \Regius\Hauksbok\Upsaliensis; \emph{þar man-líkun \hld\ mǫrg of gęrðusk} ‘There man-likenesses many were made’ \RegiusProse\Trajectinus\Wormianus}}, &
\alst{d}vergar \edtrans{ï}{in}{\Afootnote{so \GylfMS\Hauksbok; \emph{ór} ‘out of’ \Regius}} jǫrðu, \hld\ \edtrans{sęm \alst{D}urinn sagði}{as Dorn said}{\Afootnote{so \Regius\Hauksbok\RegiusProse\Wormianus; \emph{sem †dur menn† sagði} ‘as door-men(?) said’ \Trajectinus; \emph{sem †þeim dyrinn kendi†} ‘as the beasts(?) taught them’ \Upsaliensis}}.}{\lemma{þęir \dots\ sagði ‘They \dots\ said.’}\Bfootnote{The mss. readings offer two conflicting narratives of the creation of the dwarfs.  Either they arose on their own; this is supported by the prose of \Gylfaginning\ (see note to previous st.) and by the form of the stanza quoted there (but it may have been changed to correspond to the author’s vision). On the other hand, both \Regius\ and \Hauksbok\ have the dwarfs Moodsowner and Dorn shaping “man-likenesses” out of soil. The present edition follows the second version.}}\eva

\bvb There was Moodsowner made the worthiest \\
of all dwarfs, but Dorn [was] second. \\
They man-likenesses many did make: \\
dwarfs in the earth, as Dorn said.\evb\evg

\sectionline

{\small The following sts. (11–15) contain two originally distinct lists of dwarf-names; part of them are almost certainly later inserts.  It is proof enough that there is a repetition of names (Oakenshield, Great-grandfather) and more than one formulaic conclusion.

Sts. 11–13, having no repeated names, seem to belong together. If they do, st. 12, which contains the formulaic conclusion to the list, should probably switch places with 13.

Sts. 14–15 form the second group, having an introduction and a conclusion which both mention the dwarf Loffer.}

\sectionline

%TODO: move these stanzas to appendix?
\bvg\bva\mssnote{\Regius~1r/23, \Hauksbok~20r/17, \GylfMS}%
\alst{N}ýi ok \alst{N}iði, \hld\ \alst{N}orðri, Suðri, &
\alst{Au}stri, Vestri, \hld\ \alst{A}l-þjófr, Dvalinn, &
\alst{B}ívurr, \alst{B}ávurr, \hld\ \alst{B}ǫmburr, Nóri, &
\alst{Ȧ}nn ok \alst{Ȧ}narr, \hld\ \alst{Á}i, Mjǫð-vitnir.\eva

\bvb New and Nithe, Norther and Souther, \\
Easter and Wester, Allthief, Dwollen, \\
Bewer, Bower, Bamber, Noor, \\
Own and Owner, Great-grandfather, Meadwitner.\evb\evg


\bvg\bva\mssnote{\Regius~1r/25, \Hauksbok~20r/18, \GylfMS}%
\alst{V}ęigr ok Gand-alfr, \hld\ \alst{V}ind-alfr, Þráinn, &
\alst{Þ}ękkr ok \alst{Þ}orinn, \hld\ \alst{Þ}rór, Vitr ok Litr, &
\alst{N}ár ok \alst{N}ý-ráðr— \hld\ \alst{n}ú hęf’k dverga &
—\alst{R}ęginn ok \alst{R}áð-sviðr— \hld\ \alst{r}étt of talða.\eva

\bvb Wey and Gandelf, Windelf, Thrown, \\
Thetch and Thorn, Threw, Wit and Lit, \\
Nee and Newred—now have I the dwarfs— \\
Rain and Redswith—rightly tallied.\evb\evg


\bvg\bva\mssnote{\Regius~1r/28, \Hauksbok~20r/20, \GylfMS}%
\alst{F}íli, Kíli, \hld\ \alst{F}undinn, Náli, &
\alst{H}ępti, Víli, \hld\ \alst{H}annarr, Svíurr, &
\alst{F}rár, Horn-bori, \hld\ \alst{F}rę́gr ok Lȯni, &
\alst{Au}r-vangr, \alst{Ja}ri, \hld\ \alst{Ęi}kin-skjaldi.\eva

\bvb Filer, Chiler, Found and Needler, \\
Hefter, Wiler, Hanner, Swigher, \\
Fraw, Hornborer, Fray and Looner, \\
Earwong, Earer, Oakenshield.\evb\evg


\bvg\bva\mssnote{\Regius~1r/30, \Hauksbok~20r/22, \GylfMS}%
Mál es \alst{d}verga \hld\ ï \alst{D}valins liði &
\alst{l}jȯna kindum \hld\ til \alst{L}ofars tęlja, &
\edtext{þęir}{\Afootnote{\emph{þeim} \Hauksbok}} es \alst{s}óttu \hld\ frȧ \alst{s}alar stęini &
\alst{Au}r-vanga sjǫt \hld\ til \alst{Jǫ}ru-valla.\eva

\bvb ’Tis time to tally the dwarfs in Dwollen’s troop \\
{[back]} to Loffer for the races of men;\footnoteB{A standard genealogical introduction (cf. \Haleygjatal\ 1: \emph{meðan hans ę́tt \dots\ til goða tęljum} ‘while we tally his line \dots\ [back] to the gods’).  The (patrilineal) line of dwarfs is to be counted back to their progenitor, Loffer.  This possibly disagrees with st. 10, where Moodsowner is said to be the foremost (and presumably the oldest) of the dwarfs, and Loffer is not mentioned, but such details were probably not very important.} \\
they who sought, from the stone of the hall, \\
the seat of the \inx[L]{Earwongs} unto the \inx[L]{Erwolds}.\footnoteB{Cf. \Gylfaginning\ 14: “But these came from Swornshigh (\emph{Svarinshaugr}) to the Earwongs on the Erwolds, and thereof i Loffer come—these are their names: Sherper (\emph{Skirpir}), Werper (\emph{Virpir}), Showfind, Great-grandfather, Elf and Ing (\emph{Ingi}), Oakenshield, Fale (\emph{Falr}), Frost, Finn, Ginner.”}\evb\evg


\bvg\bva\mssnote{\Regius~1r/32, \Hauksbok~20r/24, \GylfMS}%
Þar vas \alst{D}raupnir \hld\ ok \alst{D}olg-þrasir, &
\alst{H}ár, \alst{H}aug-spori, \hld\ \alst{H}lé-vangr, Glói, &
\alst{Sk}irfir, Virfir, \hld\ \alst{Sk}áfiðr, Ái, &
\alst{A}lfr ok \alst{Y}ngvi, \hld\ \alst{Ęi}kin-skjaldi, &
\alst{F}jalarr ok \alst{F}rosti, \hld\ \alst{F}innr ok Ginnarr; &
Þat mun \edtext{\alst{ę́}}{\Afootnote{om. \Regius}} \alst{u}ppi, \hld\ meðan \alst{ǫ}ld lifir, &
\alst{l}ang-niðja-tal \hld\ \edtext{til}{\Afootnote{om. \Hauksbok}} \alst{L}ofars hafat.\eva

\bvb There was Dreepner and Dollowthrasher, \\
High, Highspurer, Leewong, Glower, \\
Sherver, Werver, Showfind, Great-grandfather, \\
Elf and Ing, Oakenshield, \\
Feller and Frost, Finn and Ginner.— \\
It will ever be remembered while the age lives,\footnoteB{Two archaic formulæ. The first literally ‘that will ever [be] up above’, cf. \HervararSaga\ TODO: “We two are cursed, brother, thy bane am I become! That will ever be remembered (\emph{þat mun ę́ uppi}, but both mss. \emph{þat mun enn uppi}), evil is the doom of the norns!” The second is found in a runic inscription, U 323 (980–1015): “Ever will lie—while the age lives (\textbf{meþ + altr + lifiʀ} \emph{með aldr lifir})—the hard-hammered bridge, broad, after a good man.” An especially close parallel is found in Þstf \emph{Stuttdr} (st. 5, Kari Ellen Gade ed. in \Skp\ II): \emph{Ęy mun uppi \hld\ Ęndils, meðan stęndr // sól-borgar salr, \hld\ svǫr-gǿðis fǫr.} ‘Always will be remembered—while the hall of the sun’s stronghold \ken{sky/heaven > earth} stands—the journey of the fattener of Andle’s bird \ken{raven/eagle > warrior}.’} \\
the tally of kinsmen lifted to Lofer.\evb\evg

\sectionline

\bvg\bva\mssnote{\Regius~1v/1, \Hauksbok~20r/26}%
\edtext{\edtrans{Unds}{Until}{\Bfootnote{We seem to be missing a preceding clause here, probably as part of a now-lost stanza.  It is of course impossible to say what this st. would have contained, but it may have given a reason for the creation of men.}} \edtrans{\edtext{\alst{þ}r\emph{í}r}{\Afootnote{emend.; \emph{þrjár} \Regius\Hauksbok}} kvǫ̇mu \hld\ \edtext{ór \alst{þ}ví liði}{\Afootnote{\emph{þussa brúðir} \Hauksbok.}}}{Until three came out of that host}{\Bfootnote{Both mss. show influence from st. 8 in using the fem. \emph{þrjár} for masc. \emph{þrír}.  \Hauksbok\ goes further in replacing \emph{ór því liði} ‘out of that host’ with \emph{þussa brúðir} ‘brides of thurses’.  That these are errors is clearly shown by the masculine \emph{ǫflgir ok ȧstkir ę̇sir} in l. 2.}} &
\edtrans{\alst{ǫ}flgir ok \alst{ȧ}stkir}{strong and lovely}{\Afootnote{\emph{ȧstkir ok ǫflgir} (norm.) ‘lovely and strong’ \Hauksbok}} \hld\ \alst{ę̇}sir \edtrans{at húsi}{along the houses}{\Bfootnote{An adverbial; the gods were walking on the outskirts of their settlement.}}; &
fundu ȧ \alst{l}andi \hld\ \alst{l}ítt męgandi &
\edtrans{\alst{A}sk ok \alst{Ę}mblu}{Ash and Emble}{\Bfootnote{Ash (nom. \emph{Askr}) is easily identified with the same-named wood species (\emph{Fraxinus excelsior}), but the etymology of Emble (nom. \emph{Ęmbla}) is much more difficult to explain.  Her name is often translated as “Elm” (so Neil Price), but the ON word for that tree is the masc. \emph{almr} ‘elm’.  Metathesis from earlier \emph{*Ęlma}, a derivative of the same type as \emph{þęlla} ‘young fir tree’ < \emph{þǫll} ‘fir tree’, is possible but uncertain.}} \hld\ \alst{ø}r-lǫg-lausa.}{\lemma{ALL}\Bfootnote{This stanza and the next are paraphrased in \Gylfaginning\ 9: \emph{Þá er þeir gengu með sę́var-strǫndu Bors synir, fundu þeir tré tvau ok tóku upp trén ok skǫpuðu af menn. Gaf inn fyrsti ǫnd ok líf, annarr vit ok hrę́ring, þriði á-sjónu, mál ok heyrn ok sjón. Gáfu þeim klę́ði ok nǫfn; hét karl-maðr’inn Askr, en kona’n Embla, ok ólst þaðan af mann-kind’in, sú er byggð’in var gefinn undir Mið-garði.} ‘When the sons of Byre (cf. st. 4) walked along the sea-shore they found two trees (\emph{tré}, alt. ‘pieces of wood’) and they took up the trees and shaped men out of them.  The first one gave breath and life; the second wit and movement; the third outward appearance, speech and hearing and sight.  They gave them clothes and names: the male was called Ash and the woman Emble.  And from them was begotten mankind, to which the dwelling within Middenyard was given.’ — Based on \Gylfaginning, the myth is traditionally seen as referring to pieces of driftwood, but that may be a later Icelandic or Snorroeanean interpretation.  As pointed out by \textcite{Hultgård2006}, the comparative evidence suggests that the first humans were in fact originally seen as living, growing trees, and there is really nothing in the \Voluspa\ that speaks against such an interpretation.  The story is probably the reason why words for trees are used extensively by Norse poets in kennings for men and women (see \textciteshorttitle{SkP} I, p. lxxv ff., \cite[245,266--272, 410]{Meissner1921}), more commonly in Scaldic poetry, but at times also in Eddic poetry, e.g. in \Sigrdrifumal\ 5: \emph{bryn-þings apaldr} ‘apple-tree of the byrnie-\inx[C]{Thing} \ken{battle > warrior}’.}}\eva

\bvb Until three came out of that host: \\
strong and lovely Eese along the houses; \\
they found on land the little availing \\
Ash and Emble, \inx[C]{orlay}-less.\evb\evg


\bvg\bva\mssnote{\Regius~1v/3, \Hauksbok~20r/27}%
\edtrans{\alst{Ǫ}nd}{Breath}{\Bfootnote{The breath (animating spirit) of life, which sets living things apart from the unliving.  Cf. \Gylfaginning\ 3: \emph{Hitt er þó mest, er hann gerði manninn ok gaf honum ǫnd þá, er lifa skal ok aldri týnast, þótt líkaminn fúni at moldu eða brenni at ǫsku} ‘Yet the greatest thing is when he [= Weden the Allfather] made man and gave him that “breath” which shall live and never perish even though the body molders to dust or burns to ashes.’  On Christian Scandinavian memorial runestones from the C11th onwards this word is used interchangably with the Anglo-Saxon borrowing \emph{sál} ‘soul’; compare e.g. Sö 10 \emph{Guð hjalpi ǫnd hans} ‘God help his “breath”’, Sö 8 \emph{Guð hjalpi sǫ́lu hans} ‘God help his soul’, and the frequent (at least 14 separate inscriptions) pairing of the two, like e.g. U 358 \emph{Guð hjalpi hans ǫnd ok sálu} ‘God help his “breath” and soul.’  It seems likely that this idea of an immortal “breath”, instead of being pagan, stems from the Latin \emph{spiritus} which means both ‘breath’ and ‘spirit’.  In old poems a person gives up his “breath” when he dies and stops breathing, cf. \HelgakvidaHjorvardssonar, \Sigrdrifumal, \Sigurdskamma\ TODO.}} þau né \alst{ǫ́}ttu, \hld\ \alst{ó}ð þau né hǫfðu, &
\alst{l}ǫ́ né \alst{l}ę́ti \hld\ né \alst{l}itu góða; &
\alst{ǫ}nd gaf \alst{Ó}ðinn, \hld\ \alst{ó}ð gaf Hø̇nir, &
\alst{l}ǫ́ gaf \alst{L}óðurr \hld\ ok \alst{l}itu góða.\eva

\bvb Breath they owned not, \inx[C]{wode} they had not, \\
not craft nor sound nor good colour. \\
Breath gave Weden, wode gave Heener, \\
craft gave Lother, and good colour.\evb\evg


\bvg\bva\mssnote{\Regius~1v/5, \Hauksbok~20r/29, \GylfMS}%
\alst{A}sk vęit’k \edtext{standa}{\lemma{standa ‘standing’}\Afootnote{so \Regius\Hauksbok\Upsaliensis; \emph{ausinn} ‘sprinkled’ \RegiusProse\Trajectinus\Wormianus}}, \hld\ hęitir \edtext{\alst{Y}gg-drasill}{\Afootnote{\emph{Ygg-drasils} \RegiusProse}}, &
\alst{h}ǫ́r \edtrans{baðmr}{beam}{\Afootnote{\emph{borinn} ‘born’ \Upsaliensis\ wo. doubt corrupt.}}, \edtrans{\edtrans{ausinn}{sprinkled}{\Afootnote{\emph{hęilagr} ‘holy’ \GylfMS}} \hld\ \alst{h}víta auri}{sprinkled with white mud}{\Bfootnote{Possibly relevant is the Indian ritual pouring of beverages like milk onto the phallic \emph{líṅga}, although Nikhil Surya Dwibhashyam considers this an indigenous Indian practice foreign to the old Vedic religion.  Cf. st. 26 below.}}; &
þaðan koma \alst{d}ǫggvar \hld\ \edtext{þę́r’s}{\Afootnote{\emph{es} \RegiusProse\Trajectinus}} ï \alst{d}ala falla; &
stęndr \edtext{\alst{ę́}}{\Afootnote{\emph{om.} \Upsaliensis}} \alst{y}fir \edtext{grø̇nn}{\Afootnote{\emph{†grvnn†} \RegiusProse; \emph{†grein†} \Upsaliensis}} \hld\ \alst{U}rðar brunni.\eva

\bvb An ash I know standing, ’tis called \inx[L]{Ugdrassle}: \\
a high beam \ken{tree} sprinkled with white mud. \\
Thence come the dew-drops which fall in the dales; \\
it stands ever green over \inx[L]{Weird’s Well}.\evb\evg


\bvg\bva\mssnote{\Regius~1v/8, \Hauksbok~20r/31}%
\edtext{Þaðan koma \alst{m}ęyjar \hld\ \alst{m}args vitandi &
\alst{þ}ríar ór þęim \edtrans{sal}{hall}{\Afootnote{so \Hauksbok, \GylfMS\ (in the paraphrase); \emph{sę́} ‘lake’ \Regius}} \hld\ es \edtrans{und}{under}{\Afootnote{\emph{ȧ} ‘on’ \Hauksbok}} \edtrans{\alst{þ}olli}{tree}{\Bfootnote{Literally ‘fir’, but the word is only used for the alliteration.  The same may perhaps apply to \emph{askr} ‘ash’ above, the species being indeterminate.}} stęndr; &
\alst{U}rð hétu \alst{ęi}na, \hld\ \alst{a}ðra Verðandi, &
—\edtrans{\alst{sk}ǫ́ru ȧ \alst{sk}íði}{they scored billets}{\Bfootnote{Unclear; perhaps they carve tallies for the number of years allotted to each human being.}}— \hld\ \alst{Sk}uld hina þriðju &
þę́r \alst{l}ǫg \alst{l}ǫgðu, \hld\ þę́r \alst{l}íf køru, &
\alst{a}lda bǫrnum, \hld\ \alst{ø}r-lǫg \edtrans{sęggja}{of youths}{\Afootnote{\emph{at sęgja} ‘to say’ \Hauksbok}}.}{\lemma{ALL}\Bfootnote{The st. is paraphrased in \Gylfaginning\ 15: \emph{Þar stendr salr einn fagr undir askinum við brunninn, ok ór þeim sal koma þrjár meyjar, þę́r er svá heita: Urðr, Verðandi, Skuld. Þessar meyjar skapa mǫnnum aldr; þę́r kǫllum vér nornir.} ‘There stands a single fair hall beneath the ash-tree by the well, and out of that hall come three maidens, who are called so: Weird, Werthing, Shild. These maidens shape the ages of men; we call them norns.’}}\eva

\bvb Thence come maidens, much knowing: \\
three out of the hall which stands beneath the tree. \\
Weird they called one, the other Werthing \\
—they scored billets—Shild the third. \\
They laid law, they chose lives \\
for the children of mankind, the \inx[C]{orlay} of youths.\evb\evg


\bvg\bva\mssnote{\Regius~1v/11, \Hauksbok~20v/5}%
Þat man hǫ̇n \edtrans{\alst{f}olk-víg}{troop-conflict}{\Bfootnote{\emph{folk} here carries its older meaning ‘troop, band’, as seen in the Slavic borrowing exemplified by Russian \textrussian{полк} ‘regiment, host, army’.}} \hld\ \alst{f}yrst ï hęimi, &
es \alst{G}ull-vęigu \hld\ \alst{g}ęirum studdu &
ok ï \alst{h}ǫll \alst{H}áars \hld\ \alst{h}ȧna bręnndu, &
\edtext{\alst{þ}rysvar bręnndu}{\Afootnote{\emph{†þrysvar brendv þrysvar brendv†} \Hauksbok}} \hld\ \alst{þ}rysvar borna, &
\alst{o}pt, \alst{ȯ}-sjaldan, \hld\ þó hǫ̇n \alst{ę}nn lifir.\eva

\bvb That troop-conflict she recalls first in the \inx[C]{Home}, \\
when Goldwey with spears they goaded, \\
and in the hall of \inx[P]{Higher} \name{= Weden} \ken*{= Walhall} they burned her; \\
thrice they burned the thrice born, \\
often, unseldom, though she still lives.\footnoteB{Very cryptic. TODO: check Snorri. Goldwey was apparently slain, burned and reborn three times (in short succession?) by the Eese.}\evb\evg


\bvg\bva\mssnote{\Regius~1v/13, \Hauksbok~20v/7}%
\alst{H}ęiði \alst{h}étu, \hld\ hvar’s til \alst{h}úsa kom, &
\edtext{\alst{v}ǫlu}{\Afootnote{\emph{ok vǫlu} \Hauksbok}} \alst{v}ęl-spáa, \hld\ \alst{v}itti ganda; &
\alst{s}ęið hǫ́n \edtrans{hvar’s hǫ́n kunni}{where she could}{\Afootnote{so \Hauksbok; \emph{hǫ́n kunni} ‘she knew’ \Regius}}, \hld\ \alst{s}ęið hǫ́n \edtrans{hug lęikinn}{deluded minds}{\Afootnote{so \Hauksbok; \emph{leikinn} \Regius}}; &
\alst{ę́} vas hǫ̇n \alst{a}ngan \hld\ \alst{i}llrar brúðar.\eva

\bvb Heath they called—where to houses she came— \\
the well-spaeing \inx[C]{wallow}; she bewitched \inx[C]{gand}[gands]. \\
She sorcered where she could; she sorcered deluded minds; \\
she was always the love of any evil bride.\evb\evg


\bvg\bva\mssnote{\Regius~1v/16, \Hauksbok~20v/9}%
Þȧ gingu \alst{r}ęgin ǫll \hld\ ȧ \alst{r}ǫk-stóla, &
\alst{g}inn-hęilǫg goð, \hld\ ok umb þat \alst{g}ę̇ttusk: &
Hvárt skyldu \alst{ę̇}sir \hld\ \alst{a}f-ráð gjalda, &
eða skyldu \edtrans{\alst{g}oð’in ǫll}{all the Gods}{\Bfootnote{The clitic definite \emph{-in} is very rare in older Norse poetry; this is its only occurence in \Voluspa. — Here “all the Gods” (viz., the Eese \emph{and} the Wanes) seem to be contrasted with the \inx[G]{Eese}, a subset.}} \hld\ \alst{g}ildi ęiga?\eva

\bvb Then went the Reins all onto the rake-seats: \\
the Yin-holy Gods, and from each other took counsel of this: \\
Whether the Eese should yield tribute, \\
or should all the Gods hold a banquet?\evb\evg


\bvg\bva\mssnote{\Regius~1v/17, \Hauksbok~20v/11}%
\edtrans{\alst{F}lęygði Óðinn \hld\ ok ï \alst{f}olk of skaut}{Weden hurled and shot into the troop}{\Bfootnote{The object, a spear, is understood.  This first spear-throw was reenacted in a ritual well attested in Icelandic literature, wherein the king leading his troops would hurl the first spear into the opposing host, typically with the phrase \emph{Óðinn á yðr alla} ‘Weden owns you all!’  The battle-slain were thusly devoted to Weden, and they would join him as \inx[G]{Oneharriers} in \inx[L]{Walhall}.  The sacrifice of an entire army or nation was not uncommon in ancient warfare, and examples are also found among the Hebrews (the \textgreek{חֵרֶם} \emph{ḥērem}) and the Romans (the \emph{devotio}, Livy 8:9).  Weden is also described as “owning” dead warriors in \Harbardsljod\ TODO, and in runic inscription \emph{N B380} (edited below under Galders), a sort of greeting wherein the receiver is wished to be owned by Weden (and “received” by Thunder).  For further literature see \textciteshorttitle{PCRN-HS} II:24, p. 560, II:25, p. 617, and especially III:42, p. 1166ff.}}; &
þat vas ęnn \alst{f}olk-víg \hld\ \edtrans{\alst{f}yrr}{earlier}{\Afootnote{so \Hauksbok; \emph{fyrst} ‘first’ \Regius. The \Regius\ reading cannot be correct as this st. is describing a different war, and thus not the first. It has probably arisen due to the similarity with st. 20/1.}} ï hęimi; &
\alst{b}rotinn vas \alst{b}orð-vęggr \hld\ \alst{b}orgar ȧsa, &
knǫ́ttu \alst{v}anir \edtrans{\alst{v}íg-spǫ́}{war-spae}{\Bfootnote{The Wanes used a magic prophecy (\emph{spǫ́} ‘spae’) to win the battle and sack \inx[L]{Osyard}, the stronghold of the Eese.}} \hld\ \alst{v}ǫllu sporna.\eva

\bvb Weden hurled and shot into the troop; \\
that was yet a troop-conflict earlier in the \inx[L]{Home}. \\
Broken was the plank-wall of the stronghold of the Eese; \\
the Wanes by a war-\inx[C]{spae} did tread the fields.\evb\evg


\bvg\bva\mssnote{\Regius~1v/19, \Hauksbok~20r/34, \GylfMS}%
\edtext{Þȧ gingu \alst{r}ęgin ǫll \hld\ ȧ \alst{r}ǫk-stóla, &
\alst{g}inn-hęilǫg \alst{g}oð, \hld\ ok umb þat \alst{g}ę̇ttusk: &
Hvęrr hęfði \alst{l}opt alt \hld\ \alst{l}ę́vi blandit &
eða \alst{ę́}tt \alst{jǫ}tuns \hld\ \alst{Ó}ðs męy gefna?}{\lemma{ALL}\Bfootnote{After their stronghold, protected only by a plank-wall (\emph{borð-vęggr}), is sacked by the Wanes, the Eese decide to build a stronger wall.  The story of the wall-builder is told in \Gylfaginning\ 42, which ends by quoting sts. 24–25.  An \inx[G]{Ettins}[ettin] craftsman approached the Eese and asked to build them a great wall.  His price was \inx[P]{Frow}’s hand, and the \inx[P]{Sun} and \inx[P]{Moon}, but only if he could complete the entire wall alone in a single winter.  He also asked for permission to use his workhorse, \inx[P]{Swaddlefare}, which Lock granted him.  The agreement was sealed with strong oaths.  The horse was, however, unexpectedly strong, and when three days were left before summer the wall was almost finished.  The panicked Eese then turned to Lock and forced him to deal with the horse.  His solution was to turn into a mare to distract the ettin’s workhorse, which worked; the two were out all night, and Lock was made pregnant, later giving birth to \inx[P]{Slapner}.  When the ettin realised that he would not finish the wall on time he came into his greatest ettin-wrath, at which point the Eese called on Thunder; he showed up and quickly slew the builder.}}\eva

\bvb Then went the Reins all onto the rake-seats: \\
the Yin-holy Gods, and from each other took counsel of this: \\
Who might have blended all the air with deceit, \\
or to the ettin’s lineage given \inx[P]{Wode}’s maiden \ken*{= Frow}?\evb\evg


\bvg\bva\mssnote{\Regius~1v/20, \Hauksbok~20r/36, \GylfMS}%
\edtext{\alst{Þ}ȯrr ęinn \edtrans{\alst{þ}ar vá}{fought there}{\Afootnote{so \Hauksbok\Trajectinus\Upsaliensis; \emph{þar var} ‘was there’ \Regius; \emph{þat vann} ‘accomplished it’ \RegiusProse; \emph{þat vá} ‘fought it’ \Wormianus}} \hld\ \alst{þ}runginn móði, &
\edtrans{hann \alst{s}jaldan \alst{s}itr \hld\ es \alst{s}líkt of fregn;}{he seldom sits when of such he learns}{\Bfootnote{When he learns of an ettin encroaching on the gods (see Note to 24/ALL).  Thunder is the defender of the gods (\Thrymskvida\ 18, Þdís \emph{Þórr} in \Skp\ III) and is willing to break even oaths sworn to an ettin for this purpose (cf. \Lokasenna\ 57–64).}} &
\edtext{\alst{ȧ} gingusk \alst{ęi}ðar, \hld\ \alst{o}rð ok sǿri, &
\alst{m}ǫ́l ǫll \alst{m}ęgin-lig, \hld\ es ȧ \alst{m}eðal \edtext{fóru}{\lemma{fóru ‘had gone’}\Afootnote{\emph{vǫ́ru} ‘had been’ \Hauksbok\Trajectinus}}.}{\lemma{ȧ \dots\ fóru.}\Afootnote{om. \Wormianus}}}{\lemma{ALL}\Afootnote{The order of the lines is that of \Regius\Hauksbok; in \GylfMS\ the two helmings (\emph{Þȯrr \dots\ fregn;} and \emph{ȧ \dots\ fóru.}) are reversed.}}\eva

\bvb Thunder alone fought there, pressed by wrath; \\
he seldom sits when of such he learns. \\
Trampled were oaths, speeches and vows, \\
the mighty treaties all which had gone between them.\evb\evg

\sectionline

\bvg\bva\mssnote{\Regius~1v/23, \Hauksbok~20v/1}%
Vęit hǫ̇n \alst{H}ęimdalar \hld\ \alst{h}ljóð of folgit &
und \edtrans{\alst{h}ęið-vǫnum}{shady}{\Bfootnote{Literally ‘light-less’, \emph{hęiðr} referring especially to the light of a clear sky.}} \hld\ \alst{h}ęlgum baðmi; &
\alst{ǫ́} sér hǫ̇n \alst{au}sask \hld\ \edtrans{\alst{au}rgum}{muddy}{\Bfootnote{Which should be the same mud (\emph{aurr}) as in st. 19, there said of Weird’s Well.}} forsi &
af \edtrans{\alst{v}ęði \alst{V}al-fǫðrs}{Walfather’s pledge}{\Bfootnote{Weden placed his eye in Mimer’s well, which gives wisdom to any man who drinks from it.  So \Gylfaginning\ 15: \emph{Þar kom Alfǫðr ok beiddisk eins drykkjar af brunninum, en hann fekk eigi, fyrr en hann lagði auga sitt at veði.} ‘There came Allfather and asked for a single drink from the well, but he did not get it before he laid down his eye as a pledge.’}}. \hld\ \edtrans{\alst{V}ituð ér ęnn eða hvat?}{Know ye yet, or what?}{\Bfootnote{“Do you, Weden, know enough now, or what?”, repeated in 28, 33, 34, 38, 40, 47, 60, and 61.  Similar refrains are found in \Baldrsdraumar\ and \Hyndluljod.}}\eva

\bvb She knows Homedal’s sound \ken*{= Horn of Yell?} hidden \\
beneath the shady, hallowed beam \ken*{= Ugdrassle’s Ash?}. \\
A river she sees being fed by a muddy torrent \\
from Walfather’s pledge \ken*{= Mimer’s well}.—Know ye yet, or what?”\evb\evg

\sectionline

\bvg\bva\mssnote{\Regius~1v/25}%
\edtrans{\alst{Ęi}n sat hǫ̇n \alst{ú}ti}{Alone sat she outside}{\Bfootnote{To \emph{sitja úti} ‘sit outside’ has a cultural connotation of meditation in order to connect or communicate with the otherworld; cf. the noun \emph{úti-seta}.  This line is directly repeated in \Sigurdskamma\ 6/1a.}}, \hld\ þȧ’s hinn \alst{a}ldni kom &
\alst{y}ggjungr \alst{ȧ}sa \hld\ ok ï \alst{au}gu lęit: &
‚hvęrs \alst{f}regnið mik? \hld\ hví \edtrans{\alst{f}ręistið}{tempt}{\Bfootnote{\emph{fręista} ‘tempt’ has a sense of testing someone, especially intellectually.  Cf. \Havamal\ 2, 26, \Vafthrudnismal\ 3, 5.}} mïn?\eva

\bvb Alone sat she outside when the old one came, \\
the Terrifier of the Eese \ken*{= Weden}, and looked into her eyes. \\
\speakernoteb{[The Wallow:]}%
‘Of what ask ye me? Why tempt ye me?\evb\evg


\bvg\bva\mssnote{\Regius~1v/26, \GylfMS}%
\alst{A}lt vęit’k, \alst{Ó}ðinn, \hld\ hvar \alst{au}ga falt &
\edtrans{ï hinum \alst{m}ę́ra}{in the renowned}{\Afootnote{so \Wormianus; \emph{þitt} (corr.) \emph{i enom męra} ‘id.’ \Regius; \emph{j þeim enom meira} ‘in the greater’ \Trajectinus; \emph{i þeim envm mæra} ‘in the renowned’ \Upsaliensis; \emph{vr þeim envm mę́ra} ‘out of the renowned’ \RegiusProse}} \hld\ \alst{M}ímis brunni; &
drekkr \alst{m}jǫð \alst{M}ímir \hld\ \alst{m}orgin hvęrjan &
af \edtrans{\alst{v}ęði}{pledge}{\Afootnote{\emph{†veiði†} \RegiusProse}} \alst{V}al-fǫðrs.‘ \hld\ \alst{V}ituð ér ęnn eða hvat?\eva

\bvb I know it all, Weden, where thine eye thou hidst: \\
in the renowned \inx[L]{Mimer’s Well} \\
drinks Mimer mead every morning \\
from Walfather’s pledge.’—Know ye yet, or what?\evb\evg


\bvg\bva\mssnote{\Regius~1v/29}%
Valði hęnni \alst{H}ęr-fǫðr \hld\ \alst{h}ringa ok męn, &
\edtrans{fekk \alst{sp}jǫll \alst{sp}ak-lig}{got foresighted tidings}{\Afootnote{emend.; \emph{fe spioll spaclig} \Regius}\Bfootnote{The reading of \Regius\ may be interpreted either as (1): \emph{fé-spjǫll spak-lig} ‘foresighted wealth-spells’ or (2) \emph{fé, spjǫll spak-lig} ‘wealth, foresighted tidings’; both are metrically deficient.  In (1) a second element in a cpd. like \emph{fé-spjǫll} cannot carry alliteration, and (2) has three strongly stressed nominals; in both cases \emph{fé} which stands first would be expected to carry the alliteration.  The word \emph{fé} ‘wealth, cattle’ also makes little sense in context, since Weden is the one giving her expensive jewellery.

The emendation places the verb \emph{fekk} ‘got, received’ for \emph{fé}.  Verbs carry less stress than verbs, and the line is thus metrically equivalent to 28/3b \emph{drekkr mjǫð Mímir}.  The line parallels st. 1, where the wallow likewise says that she will relate \emph{spjǫll} ‘tidings, sayings’ (cf. English \emph{gospel}, lit. ‘good news’ which originally translates the Greek \textgreek{εὐαγγέλιον}).  For discussion on this reading see \textcite[51--53]{Haukur2020}, \textcite[16]{Males2023}.}} \hld\ ok \edtrans{\alst{sp}á-ganda}{spae-gands}{\Bfootnote{Spirits sent out in order to gather hidden wisdom and spaes.  See relevant Index entries.}}; &
sá \alst{v}ítt ok umb \alst{v}ítt \hld\ of \alst{v}er-ǫld hvęrja.\eva

\bvb Host-father \name{= Weden} chose for her rings and a necklace, \\
he got foresighted tidings and \inx[C]{spae}-\inx[C]{gands}— \\
she saw widely and more widely, o’er every world.\evb\evg


\bvg\bva\mssnote{\Regius~1v/30}%
Sá hǫ̇n \alst{v}al-kyrjur \hld\ \alst{v}ítt of komnar, &
\alst{g}ǫrvar at ríða \hld\ til \edtrans{\alst{g}oð-þjóðar}{land of the Gots}{\Bfootnote{Ambiguous; ON \emph{goð-þjóð} may mean either (1) ‘land of the Gots’ or (2) ‘land of the Gods’, for the difficult cluster \emph{tþ} in \emph{Got-þjóð} ‘land of the Gots’ was at some point changed to \emph{ðþ}.  Alternative 1 is preferred since it is attested in three other places in \Regius, viz. \Helreid\ TODO and \Gudrunarhvot\ TODO and TODO, whereas 2 is entirely unattested. — It is interesting that ON \emph{Got-þjóð} reflects the attested Gotnish self-name, \emph{Gut-þiuda} (found in the October 29 entry of the Gotnish calender, TODO: reference).  The Walkirries have a particular association with the Gots, who fought the greatest battles of the Migration Period; cf. note to \Volundarkvida\ 1/1b.}}: &
\edtext{\alst{Sk}uld hélt \alst{sk}ildi, \hld\ en \alst{Sk}ǫgul ǫnnur, &
\alst{G}unnr, Hildr, \alst{G}ǫndul \hld\ ok \alst{G}ęir-skǫgul; &
\alst{n}ú eru talðar \hld\ \edtrans{\alst{N}ǫnnur Hęrjans}{Nans of Harn \name{= Weden}}{\Bfootnote{\emph{Nanna} ‘\inx[P]{Nan}’ (the name itself is a nursing word) was the wife of \inx[P]{Balder}, but the word is here certainly being used to refer generically to ‘maidens, women’.  Cf. Þul \emph{Ásynja} (\Skp\ 3), where the walkirries are kenned \emph{Óðins męyjar} ‘Weden’s maidens’.}}, &
\alst{g}ǫrvar at ríða \hld\ \alst{g}rund, val-kyrjur.}{\lemma{Skuld \dots\ val-kyrjur. ‘Shild \dots\ walkirries.’}\Bfootnote{Judging especially by the out-of-place phrase \emph{nú eru talðar} ‘now are tallied’, these four lines seem to be a later insert from a \inx[C]{thule} counting the walkirries.}}\eva

\bvb She saw \inx[G]{Walkirries} come from afar, \\
ready to ride to the land of the \inx[C]{Gots}. \\
Shild held a shield and Shagle another, \\
Guth, Hild, Gandle and Goreshagle— \\
now are tallied the Nans of Harn \name{= Weden}, \\
ready to ride the ground, the walkirries.\evb\evg

\sectionline

{\small Told allusively in \Voluspa\ 31–33 is the myth about Balder’s death.  Balder, the son of Weden and Frie, was slain with an arrow shot by his blind half-brother Hath, whose hand was guided by Lock.  Weden could not slay Hath, who was his son, and so he seduced the woman Rind, apparently through love-magic (Cormac Awmundson’s TODO: \emph{sęið Yggr til rindar} ‘Ug won Rind through sorcery’).  Rind gave birth to Wonnel, who grew very fast; after just one day he was big enough to kill Hath, which he also did, avenging Balder’s death.  The other important sources for this myth are \Baldrsdraumar\ 8–11, \Gylfaginning\ 49, and \textcite{Saxo} 3.4.1–8.

The language of \Baldrsdraumar\ is so similar to the present sts. that they must be of common origin; \Baldrsdraumar\ 11/2–4 is near-identical to \Voluspa\ 32/4–33/2.  The biggest narrative difference is that \Baldrsdraumar\ mentions Rind, who is not found in \Voluspa.

The most elaborate narrative is found in \Gylfaginning\ 49, which may be shortly summarised as follows: Balder has terrible nightmares about his own death, and so his mother Frie makes all sorts of things (fire, water, venom, metals, stones, trees, diseases, beasts, et c.) swear oaths not to harm him.  After this the Eese make sport of shooting and striking at him, since he cannot be harmed.  Lock is annoyed by this and approaches Frie while disguised as a woman.  He finds out from her that there is one thing that did not swear the oath—the mistletoe, which was thought too young.  Lock takes a mistletoe and a bow and gives it to the blind god Hath, showing him where to shoot.  Hath does so, and kills Balder.  After this \Gylfaginning\ describes Balder’s funeral (treated poetically in Wolf Ugson’s fragmentary \emph{House-drape}, ÚlfrU \emph{Húsdr} in \Skp\ III) and how the gods attempted to “weep Balder out of hell”, which failed (see Eddic Fragments in the present ed.)  \Gylfaginning\ 50 goes on to describe how the Eese punished Lock (see st. 34 below.)

It is notable that \Gylfaginning\ 49–50 fails to mention Wonnel.  This part of the myth may have been left out for moral reasons, but was certainly known to the author of the Prose Edda; cf. \Gylfaginning\ 30: \emph{Áli eða Váli heitir einn, sonr Óðins ok Rindar. Hann er djarfr í orrostum ok mjǫk happ-skęytr} ‘Onnel or Wonnel one is called, the son of Weden and Rind. He is brave in battles and a very lucky shot’ and \Skaldskaparmal\ 19: \emph{Hvernig skal kenna Vála? Svá, at kalla hann son Óðins ok Rindar, [\dots] hefni-ás Baldrs, dólg Haðar ok bana hans, [\dots]} ‘How shall one ken Wonnel? Namely by calling him the son of Weden and Rind, [\dots] avenging \inx[C]{os} of Balder, the foe of Hath and his bane, [\dots].’

The last source is \textcite{Saxo} 3.4.1–8, who retells the revenge narrative in typical euhemerized form; his versions of Hath and Balder are distinctly human generals and rulers. It may be summarized as follows: Weden takes counsel from a group of seers; one of them, Horsethief the Finn, foretells that Rind, daughter of the Russian king, will bear him another son to avenge Balder.  Weden soon enlists in the king’s army and leads it to great victories, but is continually spurned by the daughter.  He tries various other disguises but is still refused.  At last he disguises himself as an old woman and becomes her physician.  When she turns sick, he binds her, supposedly in order to give her a certain foul potion—he instead rapes her, apparently with her father’s consent.  Their son, Bo, grows up to become a fierce raider.  One day Weden summons him and reminds him of his duty to avenge his brother, Balder.  Bo slays Hath in a duel, but soon perishes from his wounds.}%TODO: add Saxo’s Latin names in parenthesis

\sectionline

\bvg\bva\mssnote{\Regius~2r/2}%
Ek sá \alst{B}aldri, \hld\ \alst{b}lóðgum \edtrans{tífur}{victim’s}{\Bfootnote{This word is rather difficult and possibly corrupt.  It may be connected with \emph{týr} ‘tew, god’, but the dat. sg. of \emph{týr} is \emph{tívi} and the intrusive \emph{r} is unexplained.  A better explanation is given by \CV, who connect it with OE \emph{tiber, tifer} ‘victim, hostage’, but this also has some problems.  \emph{blóðgum} ‘bloody’ is masc. dat. sg., but OE \emph{tiber} is neuter.  If we are dealing with a masc. noun \emph{*tífurr} with the same declension as \emph{jǫfurr}, we would expect dat. sg. \emph{*tífri}, not \emph{tífur} (which would however be the expected acc. sg.).}}, &
\alst{Ó}ðins barni, \hld\ \alst{ø}r-lǫg \edtrans{folgin}{sealed}{\Bfootnote{Or “hidden”.  The verb \emph{fela} ‘hide, conceal’ is used in poetry to describe burial in mounds, as in \Ynglingatal\ 24 (“[...] And afterwards the victory-havers hid (\emph{fǫ́lu}) the ruler on Borrey.”) or the C10th Karlevi stone (“Hidden (\textbf{fulkin} \emph{folginn}) in this mound lies he whom the greatest deeds followed; [...]”)}}; &
stóð of \alst{v}axinn \hld\ \alst{v}ǫllum hę́ri &
\alst{m}jór ok \alst{m}jǫk fagr \hld\ \alst{m}istil-tęinn.\eva

\bvb I saw Balder’s—the bloody victim’s, \\
Weden’s child’s—\inx[C]{orlay} sealed: \\
there stood grown—higher than the plains, \\
slender and most fair—the mistletoe.\evb\evg


\bvg\bva\mssnote{\Regius~2r/4}%
Varð af \alst{m}ęiði, \hld\ þęim’s \alst{m}ę́r sýndisk, &
\alst{h}arm-flaug \alst{h}ę́ttlig, \hld\ \alst{H}ǫðr nam skjóta. &
\alst{B}aldrs \alst{b}róðir vas \hld\ of \alst{b}orinn snimma, &
sá nam, \alst{Ó}ðins sonr, \hld\ \alst{ęi}n-nę́ttr vega.\eva

\bvb Of the tree which slender seemed \\
became a baneful harm-flier—Hath took to shoot. \\
Balder’s brother \ken*{= Wonnel} was born early; \\
he took, Weden’s son, one night old, to fight.\evb\evg


\bvg\bva\mssnote{\Regius~2r/6}%
\edtext{Þó}{\lemma{Þó \dots\ kęmbði ‘washed \dots\ combed’}\Bfootnote{A collocation, see note to \Havamal\ 61 for discussion and other examples. Wonnel, being oathbound and on the mission to avenge his brother, could not engage in such acts of personal vanity.}} ę́va \alst{h}ęndr \hld\ né \alst{h}ǫfuð kęmbði, &
áðr ȧ \alst{b}ál of \alst{b}ar \hld\ \alst{B}aldrs and-skota; &
en \alst{F}rigg of grét \hld\ ï \alst{F}ęn-sǫlum &
\edtrans{\alst{v}ǫ́ \alst{V}al-hallar}{the woe of Walhall}{\Bfootnote{The deaths of two sons; Balder and Hath.}}. \hld\ \alst{V}ituð ér ęnn eða hvat?\eva

\bvb He washed ne’er his hands nor combed his head, \\
before onto the pyre he bore Balder’s opponent \ken*{= Hath}, \\
and Frie lamented in the Fenhalls \\
the woe of \inx[L]{Walhall}.—Know ye yet, or what?\evb\evg

\sectionline

{\small After Balder was avenged the Eese went to catch Lock.  They bound him up with his son’s intestines.  A snake was then placed over his face to drip venom onto it.  His wife, Syein, sat over him and caught the venom in a small basin; when she had to empty it he writhed so greatly that the earth shook.  This myth is found in \FraLoka\ (the prose at the end of \Lokasenna) and \Gylfaginning\ 50.}

\sectionline

\bvg\bva\mssnote{\Regius~2r/8, \Hauksbok~20v/13}%
\edtext{\alst{H}apt sá hǫ̇n liggja \hld\ und \alst{H}vera-lundi &
\edtrans{\alst{l}ę́-gjarns}{guile-eager}{\Bfootnote{A formulaic epithet of Lock. See note to TODO for other examples and discussion.}} líki \hld\ \alst{L}oka ȧ-þękkjan;}{\lemma{Hapt \dots\ ȧ-þękkjan ‘A captive \dots\ to Lock,’}\Afootnote{Replaced with H1 \Hauksbok.}} &
\alst{þ}ar sitr Sigyn \hld\ \alst{þ}ęygi of sïnum &
\alst{v}eri \alst{v}ęl-glýjuð. \hld\ \alst{V}ituð ér ęnn eða hvat?\eva

\bvb A captive \ken*{= Lock} she saw lying beneath Wharlund: \\
a guile-eager man’s form, alike to Lock, \\
There sits Syein not at all cheerful, \\
o’er her husband.—Know ye yet, or what?\evb\evg

\sectionline

{\small The following sts. are paraphrased in \Gylfaginning\ 52:

\begin{quote}
	\emph{Þá mę́lti Gangleri: „Hvat verðr þá eptir, er brenndr er himinn ok jǫrð ok heimr allr, ok dauð goðin ǫll ok allir Einherjar ok alt mann-folk, ok hafið ér áðr sagt, at hverr maðr skal lifa í nǫkkvǫrum heimi um allar aldir?“}

	\emph{Þá svarar Þriði: „Margar eru þá vistir góðar ok margar illar; batst er þá at vera á Gimléi á himni, ok all-gótt er til góðs drykkjar þeim, er þat þykkir gaman, í þeim sal, er Brimir heitir; hann stendr ok á himni. Sá er ok góðr salr, er stendr á Niða-fjǫllum, gørr af rauðu gulli; sá heitir Sindri. Í þessum sǫlum skulu byggja góðir menn ok sið-látir.}

	\emph{Á Ná-strǫndum er mikill salr ok illr ok horfa norðr dyrr; hann er ok ofinn allr orma-hryggjum sem vanda-hús, en orma hǫfuð ǫll vitu inn í húsit ok blása eitri, svá at eptir salnum renna eitr-ár, ok vaða þę́r ár eið-rofar ok morð-vargar, svá sem hér segir:“}
\end{quote}

\begin{quote}
	‘Then spoke Gangler: “What will then remain, when heaven and earth and the whole world is burned, and gods are dead and all the Oneharriers and all man-kind—and [still] ye have said earlier, that each man will live in some world for all ages?”

	Then answers Third: “Many good dwellings are there then, and many ill: it is then best to be in Gimlee in the heaven, and it is very good of good drink for those who find joy in that, in the hall which is called Brimmer; it also stands in heaven. Another good hall is the one which stands on the Nithfells, made from red gold; it is called Sinder. In these halls good and well-mannered men will dwell.

	On Neestrand is a great and bad hall, and its doors face north. It is all woven with the spines of serpents like a wicker-house, but the heads of the serpents all look into the house and blow venom, so that through the hall rivers of venom run, and in those rivers wade oath-breakers and murder-wargs, as is said here:”’
\end{quote}

after which are quoted sts. 37 and 38/1–2, followed by the prose: \emph{En í Hver-gelmi er verst} ‘But in Wharyelmer is is worst’ and 38/4.}

\sectionline

\bvg\bva\mssnote{\Regius~2r/10}%
\alst{Ǫ́} fęllr \alst{au}stan \hld\ of \alst{ęi}tr-dala &
\alst{s}ǫxum ok \alst{s}verðum, \hld\ \edtrans{\alst{S}líðr}{Slide}{\Bfootnote{i.e. ‘very sharp’. Cf. \Atlakvida\ 23: \emph{sax slíðr-bęitt} ‘slide-biting sax’.}} hęitir sú.\eva

\bvb A river falls from the east, above the venom-dales; \\
{[a river]} of saxes and swords, Slide is that one called.\footnoteB{TODO. There are other examples of such a river.}\evb\evg


\bvg\bva\mssnote{\Regius~2r/11}%
Stóð fyr \alst{n}orðan \hld\ ȧ \edtrans{\alst{N}iða-vǫllum}{Nithwolds}{\Afootnote{\emph{Niða-fjǫllum} ‘Nithfells’ \Regius\Wormianus\ (paraphrase); \emph{fjǫllom nǫkkurum} ‘some certain fells’ \Trajectinus}} &
\alst{s}alr ór gulli \hld\ \alst{S}indra ę́ttar; &
en \alst{a}nnarr stóð \hld\ ȧ \alst{Ȯ}kólni, &
\alst{b}jór-salr jǫtuns, \hld\ \edtrans{en sá \alst{B}rimir hęitir}{and it is called Brimmer}{\Bfootnote{It is not clear if this is the name of the ettin or the hall itself. The author of \Gylfaginning\ considered it the name of the hall.}}.\eva

\bvb Stood to the north on the Nithwolds, \\
a hall of gold, of Sinder’s lineage \ken{dwarfs}. \\
But another one stood on Uncolner, \\
an ettin’s beer-hall, and it is called Brimmer.\evb\evg


\bvg\bva\mssnote{\Regius~2r/13, \Hauksbok~20v/19, \GylfMS}%
\alst{S}al \edtrans{sá hǫ̇n}{she saw}{\Afootnote{\emph{vęit’k} ‘I know’ \GylfMS; cf. st. 61.}} standa \hld\ \alst{s}ólu fjarri &
\alst{N}á-strǫndu ȧ, \hld\ \alst{n}orðr horfa dyrr; &
falla \alst{ęi}tr-dropar \hld\ \alst{i}nn umb ljóra, &
sá ’s \alst{u}ndinn salr \hld\ \alst{o}rma hryggjum.\eva

\bvb A hall she saw standing, far from the sun, \\
on Neestrand; north face its doors. \\
Venom-drops fall in through the smoke-vent; \\
that hall is wound with the spines of snakes.\evb\evg


\bvg\bva\mssnote{\Regius~2r/15, \Hauksbok~20v/21, \GylfMS}%
\edtrans{Sá hǫ̇n}{she saw}{\Afootnote{so \Regius; \emph{ser hon} ‘she sees’ \Hauksbok; \emph{skulu} ‘shall [be]’ \GylfMS}} \alst{þ}ar vaða \hld\ \alst{þ}unga strauma &
\alst{m}ęnn \alst{m}ęin-svara \hld\ ok \edtrans{\alst{m}orð-varga}{murder-wargs}{\Bfootnote{Murderous outlaws.}} &
ok þann’s \alst{a}nnars glępr \hld\ \alst{ęy}ra-ru̇nu. &
Þar \edtrans{saug}{sucked}{\Afootnote{so \Hauksbok; \emph{†súg†} \Regius; \emph{kvęlr} ‘torments’ \GylfMS}} \alst{N}íð-hǫggr \hld\ \alst{n}ái fram-gingna; &
slęit \alst{v}argr \alst{v}era. \hld\ \alst{V}ituð ér ęnn eða hvat?\eva

\bvb She saw there wading through heavy streams \\
false-swearing men and murder-wargs, \\
and the one who beguiles another’s ear-whisperer \ken{wife}. \\
There sucked \inx[P]{Nithehewer} from corpses passed-on; \\
the warg tore at men.—Know ye yet, or what?\footnoteB{In this st. is clearly described watery punishment in the Heathen afterlife, also seen in \Reginsmal\ 3–4 and possibly in \Grimnismal\ 21. The crimes are what one might expect from the Germanic worldview: perjury, shameful murder, and adultery with a married woman. In Anglo-Saxon and Nordic laws the committer of such crimes gained the title of \inx[C]{nithing}, that is, one afflicted with \inx[C]{nithe} (severe shame). It is not surprising then that such nithings would be tortured by a creature named Nithehewer ‘Nithe-striker’. The practice of burying in bogs and flood-marks (or generally outside of settlements) is well attested in sources about Germanic culture from Tacitī Germania onwards—I consider it likely that the heavy streams in this stanza and others represent such graves. This is further elaborated on in \textcite{GermanicGems2}.}\evb\evg


\bvg\bva\mssnote{\Regius~2r/17, \Hauksbok~20v/2, \GylfMS}%
\edtrans{\alst{Au}str}{In the east}{\Bfootnote{The cardinal direction associated with ettins and other monsters.}} \edtrans{býr}{dwells}{\Afootnote{so \Hauksbok\GylfMS; \emph{sat} ’sat/stayed’ \Regius}} hin \edtrans{\alst{a}ldna}{old}{\Afootnote{\emph{arma} ‘wretched’ \Upsaliensis}} \hld\ ï \edtrans{\alst{Éa}rn-viði}{Ironwood}{\Afootnote{metr. emend.; \emph{Járnviði} \Regius\Hauksbok\RegiusProse\Wormianus\Upsaliensis; \emph{Járn-viðjum} ‘Ironwoods’ \Trajectinus}} &
ok \edtrans{\alst{f}ǿðir}{nourishes}{\Afootnote{so \Hauksbok\GylfMS; \emph{fǿddi} ‘nourished’ \Regius}} þar \hld\ \alst{F}ęnris kindir; &
verðr \edtext{af}{\Afootnote{\emph{ór} \Trajectinus\RegiusProse}} þęim \alst{ǫ}llum \hld\ \alst{ęi}nna nøkkurr &
\alst{t}ungls \edtrans{\alst{t}júgari}{seizer}{\Afootnote{\emph{†tuigan†} \Trajectinus; \emph{tregari} ‘griever’ \Upsaliensis. As the young agentive suffix \emph{-ari} is found nowhere else in the poem it is possible that this word is corrupt. If it is, it must have occurred early in the transmission, as reflexes of \emph{tjúgari} are found in all surviving mss.}} \hld\ ï \alst{t}rolls hami.\eva

\bvb In the east dwells the old woman, in \inx[L]{Ironwood}, \\
and nourishes there the kindreds of \inx[P]{Fenrer} \ken{wolves}; \\
from them all comes one most certain: \\
a seizer of the Moon in a troll’s \inx[C]{hame}.\footnoteB{The old hag raises the cubs of the wolf Fenrer, of which a particularly fierce one will swallow the moon. According to \Grimnismal\ 40 the sun is chased by a wolf called Skoll, while another wolf, Hate Rothswitner’s son, runs in front of her. This is elaborated upon in \Gylfaginning\ 12, where it is said that Skoll swallows the moon, while Hate swallows the sun. High then explains that “A lone troll-woman (\emph{gýgr}) lives to the east of Middenyard in that forest called Ironwood”, and “feeds the sons of many ettins, all in the likenesses of wolves, and thereof these wolves (i.e. Skoll and Hate) come. And it is also said that from that lineage a single one becomes the mightiest, and he is called \inx[P]{Moongarm}. He fills himself with the life of all those men who die and he swallows the moon and stains heaven and all the air with blood. Thereof the sun loses its rays and the winds are violent and moan hither and thither, and thus it says in the Spae of the Wallow: [...]” after which this and the following st. are quoted. This seems very much like a composite from several sources—probably \Voluspa\ 40–41 and \Grimnismal\ 40—but becomes contradictory when it states that two wolves swallow the moon.
Assuming that this is only a confusion on the part of the author of \Gylfaginning, this st. and the next must be describing Skoll, but it is of course not impossible that there was confusion about the exact details of these events among the Heathen poets. In favour of that seems to speak \Vafthrudnismal\ 46–47, where the sun is said to be swallowed by Fenrer (but see note there).}\evb\evg


\bvg\bva\mssnote{\Regius~2r/19, \Hauksbok~20v/4, \GylfMS}%
\alst{F}yllisk \alst{f}jǫrvi \hld\ \alst{f}ęigra manna, &
\alst{r}ýðr \alst{r}agna sjǫt \hld\ \alst{r}auðum dręyra, &
\alst{s}vǫrt verða \alst{s}ól-skin \hld\ of \alst{s}umur ęptir, &
\alst{v}eðr ǫll \alst{v}á-lynd. \hld\ \alst{V}ituð ér ęnn eða hvat?\eva

\bvb He fills himself with the lifeblood of \inx[C]{fey} men; \\
he reddens the abode of the \inx[G]{Reins} with red gore. \\
Black turn the sun’s rays in summers thereafter; \\
the winds all woeful.—Know ye yet, or what?\evb\evg


\bvg\bva\mssnote{\Regius~2r/21, \Hauksbok~20v/16}%
\edtrans{\alst{S}at þar ȧ haugi}{There sat on the mound}{\Bfootnote{The motif of ettins sitting on burial mounds is also found in \Thrymskvida\ 6 and \Skirnismal\ P2.  The significance of this is uncertain,.}} \hld\ ok \alst{s}ló hǫrpu &
\alst{g}ýgjar hirðir, \hld\ \alst{g}laðr Ęggþér; &
\alst{g}ól of hǫ̇num \hld\ ï \edtrans{\alst{G}agl-viði}{Galewood}{\Bfootnote{An otherwise unknown location; the first element is \emph{gagl} ‘wild goose’.  Galewood is perhaps the same as Ironwood.}} &
\alst{f}agr-rauðr hani, \hld\ sá’s \alst{F}jalarr hęitir.\eva

\bvb There sat on the mound and struck the harp \\
the gow’s herdsman, glad \inx[P]{Edgethew}.\footnoteB{Edgethew “herds” the flock of monstrous wolves for the old woman in st. 39.} \\
Over him crowed in \inx[L]{Galewood} \\
a fair-red cock, he who is called Feller.\evb\evg


\bvg\bva\mssnote{\Regius~2r/23, \Hauksbok~20v/18}%
\alst{G}ól of ǫ̇sum \hld\ \alst{G}ullin-kambi, &
sá vękr \alst{h}ǫlða \hld\ at \alst{H}ęrja-fǫðrs, &
en \alst{a}nnarr gęlr \hld\ fyr \alst{jǫ}rð neðan &
\alst{s}ót-rauðr hani \hld\ at \alst{s}ǫlum Hęljar.\eva

\bvb Over the Eese crowed Goldencomb; \\
he wakes men at the Father of Hosts’s \name{= Weden’s} [hall]— \\
but another one crows beneath the earth: \\
a soot-red cock at the halls of Hell.\evb\evg

\sectionline

{\small With the crowing of these three cocks (the first in Ettinham, the second in Walhall, the third in Hell) the destruction of the world begins, and immediately afterwards we get the first occurrence of the refrain stanza (ON \emph{stęf}).}

\sectionline

\bvg\bva\mssnote{\Regius~2r/25}%
\alst{G}ęyr \alst{G}armr mjǫk \hld\ fyr \alst{G}nipa-hęlli, &
\alst{f}ęstr mun slitna, \hld\ en \alst{F}reki rinna; &
\alst{f}jǫlð vęit hǫ̇n \alst{f}rǿða, \hld\ \alst{f}ramm sé’k lęngra &
of \alst{r}agna \alst{r}ǫk, \hld\ \alst{r}ǫmm sig-tíva.\eva

\bvb Garm barks much before the Gnip-halls; \\
the rope will tear and the Wolf run. \\
She knows much wisdom; I foresee further \\
about the mighty \inx[L]{Rakes of the Reins}, of the victory-Tews \ken{gods}.\evb\evg


\bvg\bva\mssnote{\Regius~2r/28, \Hauksbok~20v/24, \GylfMS}%
\alst{B}rǿðr munu \alst{b}ęrjask \hld\ ok at \alst{b}ǫnum verðask, &
munu \edtrans{\alst{s}ystrungar}{the children of sisters}{\Afootnote{\emph{†stystrungar†} \Trajectinus}} \hld\ \edtrans{\alst{s}ifjum spilla}{defile the kinship}{\Bfootnote{I.e. “commit incest”, probably referring to marriages between first cousins.  Compare related words found in the laws, like \emph{frę́nd-semis-spell} ‘incest’ and especially \emph{sifja-spell} ‘id.’ — The idea of incest as a sign of later ages is also found in \Rigveda\ 10.10.10a–b (norm. and tr., Nikhil S. Dwibhashyam. (2025, Aug. 31). \emph{Véda quote 6}. https://nikhilsd.com/dvq/6/): \emph{Ā́ gʰā tā́ gaccʰān \hld\ úttarā yugā́ni, // yátra jāmáyaḥ \hld\ kr̥ṇávann ájāmi} ‘There shall come indeed those later ages when relatives shall do (acts) not (fit for) relatives.’}}; &
\alst{h}art ’s \edtrans{ï \alst{h}ęimi}{in the Home}{\Afootnote{so \Regius\Hauksbok\Upsaliensis; \emph{með hǫlðum} ‘among men’ \RegiusProse\Trajectinus\Wormianus}}, \hld\ \alst{h}ór-dȯmr mikill, &
\alst{sk}ęggj-ǫld, \alst{sk}alm-ǫld, \hld\ \edtrans{\alst{sk}ildir}{shields}{\Afootnote{\emph{’ru} ‘are’ add. \Regius}} \edtrans{klofnir}{split}{\Afootnote{\emph{klofna} ‘become split’ \Upsaliensis}}, &
\edtrans{\alst{v}ind-ǫld}{wind-age}{\Bfootnote{In \Hauksbok\ the \emph{v} is capitalized, marking the beginning of a new stanza.}}, \alst{v}arg-ǫld, \hld\ \edtrans{áðr}{before}{\Afootnote{\emph{unz} (norm.) ‘until’ \Upsaliensis}} \edtrans{\alst{v}er-ǫld}{man-age}{\Bfootnote{Translated as such since it stands next to various other compounds ending in \emph{ǫld} ‘age’.  ON \emph{ver-ǫld} is cognate with English “world”, but in ON that sense is usually expressed with \emph{hęimr} (e.g. l. 3 of the present stanza).}} \edtrans{stęypisk}{tumbles down}{\Bfootnote{\emph{grundir gjalla \hld\ gífr fljúgandi} (norm.) ‘foundations shrill, fiends flying’ add. after this l. \Hauksbok}} &
\edtext{mun \edtext{\alst{ę}ngi}{\Afootnote{\emph{†enn†} \Upsaliensis}} maðr \hld\ \alst{ǫ}ðrum þyrma.}{\lemma{mun \dots\ þyrma ‘before \dots\ spare’}\Bfootnote{om. \RegiusProse\Trajectinus\Wormianus}}\eva

\bvb Brothers will fight and become each other’s slayers; \\
the children of sisters will defile the kinship. \\
’Tis hard in the Home; whoredom is great: \\
axe-age, sword-age—shields are split— \\
wind-age, warg-age! Before the man-age tumbles down, \\
no man will another spare.\evb\evg

\sectionline

{\small Sts. 45–54 (with the omission of the refrain-stanza 47) are cited in sequence in \Gylfaginning\ 51.}

\sectionline

\bvg\bva\mssnote{\Regius~2r/32, \Hauksbok~20v/27, \GylfMS}%
\edtext{Lęika \alst{M}íms synir, \hld\ en \alst{m}jǫtuðr kyndisk &
at hinu \alst{g}alla \hld\ \alst{G}jallar-horni;}{\lemma{Lęika \dots\ Gjallar-horni; ‘Mime’s \dots Yell.’}\Bfootnote{om. \GylfMS}} &
\alst{h}ǫ́tt blę́ss \alst{H}ęimdallr, \hld\ \alst{h}orn ’s ȧ lopti; &
\edtrans{\alst{m}ę́lir}{speaks}{\Afootnote{\emph{†mey†} \RegiusProse; \emph{†nie†} \Trajectinus}} Óðinn \hld\ við \alst{M}íms hǫfuð; &
\edtext{skęlfr \alst{Y}ggdrasils \hld\ \alst{a}skr standandi, &
\alst{y}mr it \alst{a}ldna tré, \hld\ en \alst{jǫ}tunn losnar.}{\lemma{Skęlfr \dots\ losnar ‘Ugdrassle’s \dots\ loosens’}\Bfootnote{so \Hauksbok\GylfMS; in \Regius\ the two lines are reversed.}}
\eva

\bvb Mime’s sons play and the Metted is kindled \\
at [the sound of] the shrill Horn of Yell. \\
High blows Homedal; the horn is aloft; \\
Weden speaks with the head of Mime. \\
Ugdrassle’s Ash trembles, standing: \\
the old tree creaks and the ettin loosens.\evb\evg


\bvg\bva\mssnote{\Regius~2v/8, \Hauksbok~20v/30, \GylfMS}%
Hvat ’s með \alst{ǫ̇}sum? \hld\ hvat ’s með \edtrans{\alst{ǫ}lfum}{Elves}{\Afootnote{\emph{ǫ̇synjum} ‘Ossens’ \Upsaliensis}}? &
\edtext{gnýr \alst{a}llr \alst{Jǫ}tun-hęimr, \hld\ \alst{ę̇}sir ’ru ȧ \edtrans{þingi}{the Thing}{\Bfootnote{Viz. the \inx[L]{Thing of the Gods}; see note to st 6/1–2 and Index.}},}{\lemma{gnýr \dots\ þingi}\Afootnote{om. \Upsaliensis}} &
\alst{st}ynja dvergar \hld\ fyr \edtext{\alst{st}ęin-durum}{\Afootnote{\emph{stęins} \Upsaliensis; \emph{stęin-dyrum} \Hauksbok\Wormianus\Upsaliensis}} &
\edtext{\edtext{\alst{v}ęgg-bergs}{\Afootnote{\emph{veg-bergs} \Hauksbok\Trajectinus\Wormianus}} \alst{v}ísir}{\Afootnote{om. \Upsaliensis}}. \hld\ \alst{V}ituð ér ęnn eða hvat?\eva

\bvb What is with the Eese? What is with the Elves? \\
All Ettinham roars; the Eese are at the Thing. \\
Dwarfs groan before gates of stone, \\
the hillside’s princes.—Know ye yet, or what?\evb\evg


\bvg\bva\mssnote{\Regius~2v/4, \Hauksbok~20v/32}%
\alst{G}ęyr nú \alst{G}armr mjǫk \hld\ fyr \alst{G}nipa-hęlli, &
\alst{f}ęstr mun slitna, \hld\ en \alst{f}reki rinna; &
\alst{f}jǫlð vęit hǫ̇n \alst{f}rǿða, \hld\ \alst{f}ramm sé’k lęngra &
of \alst{r}agna \alst{r}ǫk \hld\ \alst{r}ǫmm sig-tíva.\eva

\bvb Now Garm barks much before the Gnip-halls; \\
the rope will tear and the Wolf run. \\
She knows much wisdom; I foresee further \\
about the mighty Rakes of the Reins, of the victory-Tews \ken{gods}.\evb\evg


\bvg\bva\mssnote{\Regius~2v/4, \Hauksbok~20v/32, \RegiusProse\Trajectinus\Wormianus}%
\alst{H}rymr ękr austan, \hld\ \alst{h}ęfsk lind fyrir, &
snýsk \alst{Jǫ}rmun-gandr \hld\ ï \alst{jǫ}tun-móði, &
\alst{o}rmr knýr \alst{u}nnir, \hld\ \edtrans{en \alst{a}ri hlakkar}{and the eagle screams}{\Afootnote{\emph{ǫrn mun hlakka} ‘the eagle will scream’ \RegiusProse\Trajectinus}}, &
slítr \alst{n}ái \alst{n}ęf-fǫlr; \hld\ \alst{N}agl-far losnar.\eva

\bvb Rim drives from the east, holding his shield before him; \\
Ermingand writhes about in ettin-wrath. \\
The Wyrm propels the waves and the eagle screams: \\
the pale-beak tears at corpses; Nailfare loosens.\evb\evg


\bvg\bva\mssnote{\Regius~2v/6, \Hauksbok~20v/34, \RegiusProse\Trajectinus\Wormianus}%
\alst{K}jóll fęrr austan \hld\ \alst{k}oma munu Múspells &
of \alst{l}ǫg \alst{l}ýðir, \hld\ en \alst{L}oki stýrir; &
\alst{f}ara \alst{f}ífl-męgir \hld\ með \alst{f}reka allir, &
þęim es \alst{b}róðir \hld\ \alst{B}ýlęists ï fǫr.\eva

\bvb A ship fares from the east—come will Muspell’s \\
subjects o’er the sea—and Lock steers it. \\
The devil-lads journey all with the Wolf; \\
with them comes the brother of Bylest \ken*{= Lock} along.\evb\evg


\bvg\bva\mssnote{\Regius~2v/10, \Hauksbok~20v/36, \GylfMS}%
\edtext{\alst{S}urtr}{\Afootnote{\emph{Svartr} \Upsaliensis}} fęrr \alst{s}unnan \hld\ með \alst{s}viga lę́vi, &
skínn af \alst{s}verði \hld\ \edtrans{\alst{s}ól val-tíva}{sun of the slain-Tew}{\Bfootnote{\emph{val-tíva} is here taken as gen. sg. of \emph{val-tívar} ‘slain-Tews’, for which cf. st. 59 below, but the sense of this is obscure.  Perhaps it means that Surt’s sword shines as bright as the heavenly Gods?  The word may also (so \CV) be read as gen. sg. of unattested \emph{*val-tívi} ‘tew of the slain’, referring to Surt, but this is tautological: “Surt comes from the south with fire; from his sword shines the sun of Surt”.}}; &
\alst{g}rjót-bjǫrg \alst{g}nata, \hld\ en \edtrans{\alst{g}ífr rata}{fiends reel}{\Afootnote{\emph{guðar hrata} ‘[but] the gods stagger’ \Upsaliensis}\Bfootnote{The reading of \Upsaliensis\ is wo. doubt corrupt; the anachronistic masc. pl. ending \emph{-ar} is proof enough, for the word \emph{goð} \char`~\ \emph{guð} ‘gods’ was always neuter in heathen times.}}, &
troða \alst{h}alir \edtrans{\alst{h}ęl-veg}{Hellway}{\Bfootnote{The road on which one has to travel after death to reach his final resting place.  Cf. \Helreid.}}, \hld\ en \alst{h}iminn klofnar.\eva

\bvb Surt comes from the south with the twig’s betrayer \ken{fire}; \\
from the sword shines the sun of the slain-Tews. \\
Boulders clash and the fiends reel; \\
men tread the \inx[L]{Hellway} and heaven is split.\evb\evg

\sectionline

{\small The following two sts. describe how Weden fights the Wolf and dies, and how he is then avenged by Wider.  This fight is also mentioned in \Vafthrudnismal\ 53.}

\sectionline

\bvg\bva\mssnote{\Regius~2v/13, \Hauksbok~20v/37, \RegiusProse\Trajectinus\Wormianus}%
Þȧ kømr \edtrans{\alst{H}línar \hld\ \alst{h}armr annarr}{Line’s second sorrow}{\Bfootnote{The first sorrow being the death of Balder.  Line is described in \Gylfaginning\ 35 as a minor goddess \emph{sett til gę́zlu yfir þeim mǫnnum, er Frigg vill forða við háska nǫkkurum} ‘placed to watch over those men which Frie wishes to protect against any particular danger’. In spite of this almost all translators and editors have understood Line as synonymous with Frie, or even asked whether her existence as a distinct goddess is not something invented by the author of \Gylfaginning.  \textcite{Hopkins2017} argues that this need not be the case; as a maidservant of Frie, Line’s two sorrows would consist in her failure to protect both the son and husband of her mistress.}} framm, &
es \alst{Ó}ðinn fęrr \hld\ við \alst{u}lf vega, &
—en \edtrans{\alst{b}ani \alst{B}ęlja}{the bane of Bellower \ken*{= Free}}{\Bfootnote{Bellower (ON \emph{Bęli}) was slain by Free in an obscure duel; see Index.}} \hld\ \alst{b}jartr at Surti— &
þȧ mun \alst{F}riggjar \hld\ \alst{f}alla \edtext{angan}{\Afootnote{so \Hauksbok\GylfMS; \emph{angantyr} \Regius}}.\eva

\bvb Then comes \inx[P]{Line}’s second sorrow to pass, \\
when Weden goes to fight the Wolf \\
—but the bane of Bellower \ken*{= Free}, bright, against Surt— \\
then will Frie’s beloved \ken*{= Weden} fall.\evb\evg


\bvg\bva\mssnote{\Regius~2v/15, \RegiusProse\Trajectinus\Wormianus}%
\edtrans{Þȧ kømr hinn \alst{m}ikli \hld\ \alst{m}ǫgr Sig-fǫður}{Then comes the great lad of Syefather}{\Afootnote{\emph{Gęngr Óðins sonr \hld\ við ulf vega} ‘Weden’s son goes the Wolf to fight’ \GylfMS.}}, &
\alst{V}íðarr \edtext{\alst{v}ega}{\Afootnote{\emph{of veg} \GylfMS}} \hld\ at \alst{v}al-dýri; &
lę́tr \alst{m}ęgi \edtrans{Hveðrungs}{Whethring}{\Bfootnote{An obscure name for \inx[P]{Lock}, whose son is the Wolf.}} \hld\ \alst{m}und of standa &
\alst{h}jǫr til \alst{h}jarta; \hld\ þȧ ’s \alst{h}efnt fǫður.\eva

\bvb Then comes the great lad of \inx[P]{Syefather}, \\
Wider, to fight that slaughter-beast. \\
He lets his hand through \inx[P]{Whethring}’s lad \ken*{= the Wolf} \\
drive the sword to the heart—then the father is avenged!\evb\evg


\bvg\bva\mssnote{\Regius~2v/17, \Hauksbok~20v/41, \RegiusProse\Trajectinus\Wormianus}%
\edtext{\edtrans{\edtrans{Þȧ kømr}{Then comes}{\Afootnote{\emph{Gęngr} ‘Goes’ \GylfMS}} hinn \alst{m}ę́ri \hld\ \alst{m}ǫgr \edtrans{Hlǫðynjar}{Lathyn}{\Afootnote{add. \emph{gęngr Óðins sonr \hld\ við orm vega.} ‘Weden’s son goes the Wyrm to fight.’ \Regius.}},}{Then comes the renowned lad of Lathyn}{\Afootnote{om. \Hauksbok.}} &
\edtrans{gęngr \alst{f}et níu \hld\ \alst{F}jǫrgynjar burr}{nine paces goes Firgyn’s son}{\Afootnote{om. \GylfMS.}} &
\alst{n}ęppr frȧ \alst{n}aðri, \hld\ \alst{n}íðs ȯ-kvíðnum; &
\edtrans{munu \alst{h}alir allir \hld\ \alst{h}ęim-stǫð ryðja}{All men will clear their homesteads}{\Bfootnote{After Thunder is slain the Earth is no longer habitable.  Cf. \Harbardsljod\ TODO, \Thrymskvida\ 18.}} &
\edtext{es af \alst{m}óði drepr}{\Afootnote{\emph{drepr hann af móði} \Regius}} \hld\ \edtrans{\alst{M}ið-garðs véurr}{Middenyard’s Wighward}{\Bfootnote{“The Guardian of the Sanctuaries of Middenyard”; a fitting kenning.}}.}{\lemma{ALL}\Bfootnote{The present version of the stanza is an amalgamation of all three mss. (\Regius, \Hauksbok\ and \GylfMS), based most closely on the latter two, which have the last 3 lines in the same order.  \Regius\ has the lines in the following order: 1, 5, 4, 2, 3.  It also inserts another line between 1 and 5.}}\eva

\bvb Then comes the renowned lad of Lathyn \ken*{= Thunder}; \\
nine paces walks Firgyn’s son \ken*{= Thunder} \\
pained, away from the loathsome adder \ken*{= Middenyardswyrm}. \\
All men will clear their homesteads \\
when Middenyard’s Wighward strikes out of wrath.\evb\evg


\bvg\bva\mssnote{\Regius~2v/20, \Hauksbok~21r/1, \GylfMS}%
\alst{S}ól tér \alst{s}ortna, \hld\ \edtrans{\edtrans{\alst{s}økkr}{sinks}{\Afootnote{so \RegiusProse\Trajectinus\Wormianus; \emph{sígr} ‘descends’ \Regius\Hauksbok\Upsaliensis}} fold ï mar}{the fold sinks into the sea}{\Bfootnote{The reading \emph{søkkr} ‘sinks’ is supported by Arn \emph{Þorfdr} 24 (\Skp\ II), which is probably based on the present line: \emph{Bjǫrt verðr sól at svartri; \hld\ søkkr fold ï mar døkkvan;} ‘Bright, the sun turns to black; the fold sinks into the dark sea’.}}, &
\alst{h}verfa af \alst{h}imni \hld\ \alst{h}ęiðar stjǫrnur; &
gęisar \alst{ęi}mi \hld\ við \alst{a}ldr-nara; &
lęikr \alst{h}ǫ́r \alst{h}iti \hld\ við \alst{h}imin sjalfan.\eva

\bvb The sun starts to blacken; the fold \name{earth} sinks into the sea; \\
from heaven fade the shining stars. \\
Smoke rages from the life-nourisher \ken{fire}; \\
the high heat licks heaven itself.\evb\evg


\bvg\bva\mssnote{\Regius~2v/22, \Hauksbok~21r/2}%
\alst{G}ęyr nú \alst{G}armr mjǫk \hld\ fyr \alst{G}nipa-hęlli, &
\alst{f}ęstr mun slitna, \hld\ en \alst{f}reki rinna; &
\alst{f}jǫlð vęit hǫ̇n \alst{f}rǿða, \hld\ \alst{f}ramm sé’k lęngra &
of \alst{r}agna \alst{r}ǫk, \hld\ \alst{r}ǫmm sig-tíva.\eva

\bvb Now Garm barks much before the Gnip-halls; \\
the rope will tear and the Wolf run. \\
She knows much wisdom; I foresee further \\
about the mighty Rakes of the Reins, of the Victory-Tews \ken{gods}.\evb\evg

%TODO: Insert image?

\sectionline

{\small With the last repetition of the refrain stanza the destruction has reached its apex.  Sts. 56–59 are paraphrased in \Gylfaginning\ ch. 53:

\begin{quote}
	\emph{Þá mę́lti Gangleri: „Hvárt lifa nǫkkur goðin þá, eða er þá nǫkkur jǫrð eða himinn?“ Hárr segir: „Upp skýtr jǫrðunni þá ór sę́num, ok er þá grǿn ok fǫgr. Vaxa þá akrar ó·sánir. Víðarr ok Váli lifa, svá at eigi hefir sę́rinn ok Surta-logi grandat þeim, ok byggja þeir á Iða-velli, þar sem fyrr var Ás-garðr, ok þar koma þá synir Þórs, Móði ok Magni, ok hafa þar Mjǫllni. Því nę́st koma þar Baldr ok Hǫðr frá Heljar, setjast þá allir samt, ok talast við, ok minnast á rúnar sínar, ok rǿða of tíðendi þau, er fyrrum hǫfðu verit, of Mið-garðs-orm ok um Fenris-úlf. Þá finna þeir í grasinu gull-tǫflur þę́r, er ę́sirnir hǫfðu átt. Svá er sagt:“}
\end{quote}

\begin{quote}
	‘Then spoke Gangler: “Do any of the gods then live, or is there then any earth or heaven?”  High says: “The earth then shoots up from the seas, and it is then green and fair.  Then grow acres unsown.  Wider and Wonnel live, for the sea and Surt’s flame have not harmed them, and they settle on the Idewolds where there earlier was Osyard; and then the sons of Thunder, Mood and Main, come there, and there they have Millner.  Next come Balder and Hath from Hell; then they all make peace with each other and discuss and think back on their runes, and speak about the tidings which had been in antiquity, about the Middenyardswyrm and about the Fenrerswolf.  Then they find in the grass those golden game-bricks which the Eese had owned. So it is said:”’
\end{quote}

after which is quoted \Vafthrudnismal\ 51.}

\sectionline

\bvg\bva\mssnote{\Regius~2v/23, \Hauksbok~21r/4}%
Sér hǫ̇n \alst{u}pp koma \hld\ \edtrans{\alst{ǫ}ðru sinni}{a second time}{\Bfootnote{The first time probably being the lifting of the Earth in st. 4.}} &
\alst{jǫ}rð ór \alst{ę́}gi \hld\ \alst{i}ðja-grø̇na; &
\alst{f}alla \alst{f}orsar, \hld\ \alst{f}lýgr ǫrn yfir, &
sá’s ȧ \alst{f}jalli \hld\ \alst{f}iska vęiðir.\eva

\bvb She sees coming up a second time \\
Earth from the ocean, ever green anew. \\
Torrents fall, flies the eagle above, \\
which on the fells catches fish.\evb\evg


\bvg\bva\mssnote{\Regius~2v/24, \Hauksbok~21r/5}%
\edtrans{Finnask}{find each other}{\Afootnote{\emph{hittask} \Hauksbok\ provides closer parallelism with st. 7, but for the same reason it may also have replaced earlier \emph{finnask}.}} \alst{ę̇}sir \hld\ ȧ \alst{I}ða-vęlli &
ok umb \edtrans{\alst{m}old-þinur}{Earth-cord}{\Bfootnote{Cf. the kenning for the Middenyardswyrm in ÚlfrU \emph{Húsdr} 4: \emph{stirð-þinull storðar} ‘the stiff cord of the land \ken*{= Middenyardswyrm}’}} \hld\ \alst{m}ǫ́tkan dø̇ma, &
\edtrans{ok \alst{m}innask þar \hld\ ȧ \alst{m}ęgin-dȯma}{and there think back on mighty verdicts}{\Afootnote{om. \Regius}} &
ok ȧ \alst{F}imbul-týs \hld\ \alst{f}ornar ru̇nar.\eva

\bvb The Eese find each other on the Idewolds, \\
and of the mighty Earth-cord \ken*{= Middenyardswyrm} judge, \\
and there think back on mighty verdicts, \\
and on Fimble-Tew’s \name{= Weden’s} ancient runes.\evb\evg


\bvg\bva\mssnote{\Regius~2v/26, \Hauksbok~21r/7}%
Þar munu \alst{ę}ptir \hld\ \edtext{\alst{u}ndr-samligar &
\alst{g}ullnar tǫflur}{\lemma{undr-samligar gullnar tǫflur ‘wondersome golden game-bricks’}\Bfootnote{A fine literary device.  In st. 8 the golden age of the Eese, exemplified by their playing board games, was spoiled by the three ettin-women.  The rediscovering of the golden board game then betokens a new golden age.}} \hld\ ï \alst{g}rasi finnask, &
þę́r’s ï \alst{á}r-daga \hld\ \alst{á}ttar hǫfðu.\eva

\bvb There will afterwards wondersome \\
golden game-bricks in the grass be found, \\
those which in days of yore they had owned.\evb\evg


\bvg\bva\mssnote{\Regius~2v/28, \Hauksbok~21r/9}%
Munu \alst{ȯ}-sánir \hld\ \alst{a}krar vaxa, &
\alst{b}ǫls mun alls \alst{b}atna, \hld\ mun \alst{B}aldr koma; &
búa \alst{H}ǫðr ok Baldr \hld\ \alst{H}ropts sig-toptir, &
\alst{v}ęl \alst{v}al-tívar. \hld\ \alst{V}ituð ér ęnn eða hvat?\eva

\bvb Unsown will acres grow; \\
the bale will all be bettered; Balder will come. \\
Hath and Balder bedwell Roft’s \name{= Weden’s} victory-plots \\
well, the slain-Tews.—Know ye yet, or what?\footnoteB{The evil of Hath’s slaying Balder will be forgotten as the two live together in peace.}\evb\evg


\bvg\bva\mssnote{\Regius~2v/30, \Hauksbok~21r/11}%
Þȧ kná \alst{H}ø̇nir \hld\ \edtrans{\alst{h}laut-við kjósa}{choose the leat-wood}{\Bfootnote{Foresee the future by means of twigs drenched in the blood of slaughtered beasts.  See \Hymiskvida\ 1 and Index: \inx[C]{leat}.}} &
ok \alst{b}urir \alst{b}yggva \hld\ \edtrans{\alst{b}rǿðra tvęggja}{the two brothers}{\Bfootnote{The present translation understands \emph{tvęggja} as the gen. pl. of \emph{tvęir} ‘two’; the two brothers are presumably Hath and Balder, mentioned in the previous stanza.
Since the original ms. does not capitalize proper nouns one could also read \emph{brǿðra Tvęggja} ‘the brothers of Tway \name{= Weden}’.  Weden’s brothers are attested in \Gylfaginning\ 6 as \inx[P]{Will} and \inx[P]{Wigh}; they are never said to have children.}} &
\alst{v}ind-hęim \alst{v}íðan. \hld\ \alst{V}ituð ér ęnn eða hvat?\eva

\bvb Then does Heener choose the \inx[C]{leat}-wood, \\
and the sons of the two brothers settle \\
the wide wind-home \ken{sky/heaven}.—Know ye yet, or what?\evb\evg


\bvg\bva\mssnote{\Regius~2v/31, \Hauksbok~21r/12, \GylfMS}%
\alst{S}al \edtrans{sér hǫ̇n}{she sees}{\Afootnote{\emph{vęit’k} ‘I know’ \GylfMS}} standa \hld\ \alst{s}ólu fęgra, &
\edtrans{\alst{g}ulli þakðan}{thatched with gold}{\Afootnote{\emph{gulli bętra} ‘better than gold’ \RegiusProse\Trajectinus}}, \hld\ ȧ \edtext{\alst{G}imléi}{\Afootnote{metr. emend.; \emph{Gimlé} \Regius\Hauksbok\GylfMS}}; &
\edtrans{þar}{there}{\Afootnote{\emph{þann} ‘[in] that [hall]’ \Trajectinus\Wormianus}} skulu \alst{d}yggvar \hld\ \alst{d}róttir byggva &
ok umb \alst{a}ldr-daga \hld\ \alst{y}nðis njóta.\eva

\bvb A hall she sees standing, fairer than the sun, \\
thatched with gold, on Gemlee; \\
there shall faithful folk settle, \\
and in their days of life enjoy delight.\evb\evg


\bvg\bva\mssnote{\Regius~3r/2, \Hauksbok~21r/15}%
Þar kømr hinn \alst{d}immi \hld\ \alst{d}ręki fljúgandi, &
\alst{n}aðr frȧnn \alst{n}eðan \hld\ frȧ \alst{N}iða-fjǫllum; &
berr sér ï \alst{f}jǫðrum \hld\ —\alst{f}lýgr vǫll yfir— &
\alst{N}íð-hǫggr \alst{n}ái; \hld\ \edtrans{\alst{n}ú mun hǫ̇n søkkvask}{Now she will sink!}{\Bfootnote{The wallow, referring to herself in third person, descends back down into her grave, whence Weden woke her.  Cf. the very last half-line of \Helreid: \emph{søkkst-u, gýgjar-kyn} ‘sink, thou gow’s kin!’}}.\eva

\bvb Then comes the gloomy dragon flying, \\
the gleaming adder up from the \inx[L]{Nithfells}. \\
He carries in his feathers—he flies over the field— \\
Nithehewer, corpses.—Now she will sink!”\evb\evg

\sectionline

\section{Stanzas from \emph{Hauksbók}}

{\small \Hauksbok\ has a few substantial inserts and differences from \Regius.  Their style strongly suggests that they are later compositions.}

\sectionline

{\small 34/1–2 are replaced by the following.}

\bvg\bva[H1]\mssnote{\Hauksbok~20v/12}%
Þȧ kná \edtrans{\alst{V}áli}{Wonnel}{\Afootnote{emend.; \emph{Vála} \Hauksbok}} \hld\ \alst{v}íg-bǫnd snúa &
\alst{h}ęldr vǫ́ru \alst{h}arð-gǫr \hld\ \alst{h}ǫpt ór þǫrmum.\eva

\bvb Then did \inx[C]{Wonnel} the war-bonds twist: \\
the most sturdy fetters were made from intestines.\evb\evg

\sectionline

{\small 45/5–6 are followed by the following lines, forming another four-line stanza.}

\bvg\bva[H2]\mssnote{\Hauksbok~20v/28}%
\alst{H}rę́ðask allir \hld\ ȧ \alst{h}ęl-vegum &
áðr \alst{S}urtar þann \hld\ \alst{s}efi of glęypir.\eva

\bvb All are frightened on the Hell-ways, \\
before Surt’s kinsman does devour it.\evb\evg

\sectionline

{\small The following stanza appears between 52 and 53.}

\bvg\bva[H3]\mssnote{\Hauksbok~20v/39}%
\edtext{Gïnn \alst{l}opt yfir \hld\ \alst{l}indi jarðar, &
gapa \alst{ý}gs kjaptar \hld\ \alst{o}rms ï hę́ðum; &
mun \alst{Ó}ðins son \hld\ \edtrans{\alst{ęi}tri}{venom}{\Afootnote{emend.; \emph{ormi} ‘Wyrm’ \Hauksbok.}\Bfootnote{Cf. \Gylfaginning\ 51: “Thunder bears the bane-word from the Middenyardswyrm and strides nine paces away from it. Then he falls dead to the earth for the venom (\emph{ęitri}) which the Wyrm blows on him.”}} mǿta &
\alst{v}args at \edtext{da\emph{uða}}{\Afootnote{\emph{‘da...’} \Hauksbok}} \hld\ \alst{V}íðars niðja.}{\lemma{Gïnn \dots\ niðja.}\Bfootnote{The last part of the stanza is almost completely illegible.  I have relied on the reading of \textcite[13,44\psqq]{JonHelgason1971}.}}\eva

\bvb Over the air yawns the Girdle of the Earth \ken*{= Middenyardswyrm}; \\
the jaws of the fierce Wyrm gape in the heights. \\
Weden’s son \ken*{= Thunder} will meet the venom \\
of the Warg, after the deaths of Wider’s kinsmen \ken*{= the Eese}.\evb\evg

\sectionline

{\small The following half-stanza appears between 61 and 62; it is generally held to be a late Christian insert.}

\bvg\bva[H4]\mssnote{\Hauksbok~21r/14}%
Þȧ kømr hinn \alst{r}íki \hld\ at \alst{r}ęgin-dȯmi &
\alst{ǫ}flugr \alst{o}fan \hld\ sá’s \alst{ǫ}llu rę́ðr.\eva

\bvb Then comes the mighty one to the great judgment, \\
strong from above, he who rules everything.\evb\evg

\sectionline
% Weden, All Gods
	\bookStart{The Dreams of Balder}[Baldrs draumar]

% Introduction.
\begin{flushright}%
\textbf{Dating} \parencite{Sapp2022}: C10th (0.890)

\textbf{Meter:} \Fornyrdislag%
\end{flushright}

In ancient manuscripts only preserved in \AM, but the poem also survives in later manuscripts with a few extra stanzas (see below). It follows the structure of a riddle contest.

The poem begins \emph{in medias res}; \inx[P]{Balder} has been having nightmares, and so the gods meet at the Thing to figure out why (1). \inx[P]{Weden} rides to \inx[L]{Hell}, where he has an encounter with a bloody dog (2). It barks for a long time at him, but he passes it and continues to “the high house of \inx[P]{Hell}” (3), from which he rides west, to the grave of a certain \inx[C]{wallow}, whom he revives using magic (4). She asks which man has forced her out of the grave (5), and Weden introduces himself as Waytame, before asking for whom the benches of Hell are covered with gold (6). The wallow responds that barrels of mead stand brewed for Balder and that the gods are very anxious (7). Weden asks her who will slay Balder (8), and she responds that it is Hath, carrying a “high fame-beam” (9). Weden then asks her who will avenge Balder’s death by slaying Hath (10). The wallow responds that \inx[P]{Rind} will give birth to Weden’s son \inx[P]{Wonnel}, who will slay Hath when only one night old (11). Weden then asks about some mysterious maidens (12; see Note), which betrays his identity. The wallow tells him that she now knows his true identity, to which Weden responds that he does as well: she is not a wallow, but rather the “mother of three thurses” (13). She tells him to ride home and “be famous”, before reminding him of his death at the \inx[L]{Rakes of the Reins} (14).

\sectionline

\bvg\bva\mssnote{\AM~1v/18}%
\edtext{Sęnn vǫ́ru \alst{ę́}sir \hld\ \alst{a}llir á þingi &
ok \alst{ǫ́}synjur \hld\ \alst{a}llar á máli, &
ok umb þat \alst{r}éðu \hld\ \alst{r}íkir tívar:}{\lemma{Sęnn \dots\ tívar ‘Soon \dots\ Tews’}\Bfootnote{Formulaic, identically shared with \Thrymskvida\ 14/1–3.  See also \inx[L]{Thing of the Gods}.}} &
hví vę́ri \alst{B}aldri \hld\ \alst{b}allir draumar?\eva

\bvb Soon were the \inx[G]{Eese} all at the \inx[C]{Thing}, \\
and the \inx[G]{Ossens} all at speech, \\
and of this counseled the mighty \inx[G]{Tews}: \\
Why did Balder have troubling dreams?\evb\evg


\bvg\bva\mssnote{\AM~1v/19}%
\alst{U}pp ręis \alst{Ó}ðinn, \hld\ \alst{a}ldinn gautr, &
ok hann á \alst{S}lęipni \hld\ \alst{s}ǫðul of lagði, &
ręið \alst{n}iðr þaðan \hld\ \alst{n}ifl-hęljar til; &
mǿtti \alst{h}velpi, \hld\ þęim’s ór \alst{h}ęlju kom.\eva

\bvb Up rose Weden, the ancient Geat, \\
and he on \inx[P]{Slapner} the saddle did lay; \\
rode down thence to \inx[L]{Nivelhell}; \\
met the whelp that came out of Hell.\evb\evg


\bvg\bva\mssnote{\AM~1v/21}%
Sá vas \alst{b}lóðugr \hld\ of \alst{b}rjóst framan, &
ok \alst{g}aldrs fǫður \hld\ \alst{g}ól oflęngi, &
\alst{f}ramm ręið Óðinn, \hld\ \alst{f}old-vegr dunði, &
kom at \alst{h}ǫ́u \hld\ \alst{H}ęljar ranni.\eva

\bvb That one was bloody on the front of the chest, \\
and at the father of \inx[C]{galder} \ken*{= Weden} for a long time bayed.— \\
Forth rode Weden, the fold-way \ken{earth} resounded;\footnoteB{A similarity may be noted with the description of \inx[P]{Thunder}’s riding in \Haustlong\ 14: \emph{dunði \dots\ mána vegr und hǫ́num} ‘the moon’s way \ken{sky/heaven} \dots\ resounded beneath him’) and \Thrymskvida\ 20 (see also note there).} \\
he came to the high house of Hell.\evb\evg


\bvg\bva\mssnote{\AM~1v/22}%
Þá ręið \alst{Ó}ðinn \hld\ fyr \alst{au}stan dyrr, &
þar’s hann \alst{v}issi \hld\ \alst{v}ǫlu lęiði; &
nam hann \alst{v}ittugri \hld\ \edtrans{\alst{v}al-galdr}{slain-galder}{\Bfootnote{i.e. a galder to quicken the dead, in this case the wallow.  Cf. \Havamal\ 158 where Weden tells how He can bring hanged men back to life with runes.}} kveða, &
unds \alst{n}auðug ręis, \hld\ \alst{n}ás orð of kvað:\eva

\bvb Then rode Weden east from the door, \\
there as He knew the wallow’s grave; \\
He began for the cunning woman to sing a slain-\inx[C]{galder}, \\
until forced she rose, a corpse’s words quoth:\evb\evg


\bvg\bva\mssnote{\AM~1v/24}%
„Hvat ’s \alst{m}anna þat \hld\ \alst{m}ér ó·kunnra, &
es mér hęfr \alst{au}kit \hld\ \edtrans{\alst{ę}rfitt sinni}{this toilsome journey}{\Bfootnote{i.e. the journey out of the grave.}}? &
\edtext{Vas’k \alst{s}nifin \alst{s}nę́vi, \hld\ ok \alst{s}lęgin regni, &
ok \alst{d}rifin \alst{d}ǫggu, \hld\ \alst{d}auð vas’k lęngi.}{\lemma{Vas’k snifin \dots\ lęngi. ‘I was snowed \dots\ long.’}\Bfootnote{Cf. the similar description of a buried person in \HelgakvidaTwo\ 47–48 (TODO).}}“\eva

\bvb “What sort of man is this, unknown to me, \\
who has caused for me this toilsome journey? \\
I was snowed by snow and struck by rain, \\
and bespattered with dew—dead was I for long.”\evb\evg


\bvg\bva\mssnote{\AM~1v/25}\speakernote{[Óðinn kvað:]}%
„\alst{V}eg-tamr hęiti’k, \hld\ sonr em’k \alst{V}al-tams, &
sęg mér ór \alst{h}ęlju, \hld\ ek ór \alst{h}ęimi mun; &
hvęim eru \alst{b}ękkir \hld\ baugum sánir? &
\alst{f}lęt \alst{f}agrliga \hld\ \alst{f}lóuð eru gulli.“\eva

\bvb\speakernoteb{[Weden quoth:]}
“Waytame am I called, I am Waltame’s son; \\
tell me [the tidings] from Hell—I will [tell those] from the world. \\
For whom are the benches sown with \inx[C]{bigh}[bighs]? \\
Fairly are the floors flooded with gold.”\evb\evg


\bvg\bva\mssnote{\AM~1v/27}\speakernote{[Vǫlva kvað:]}%
„Hér stęndr \alst{B}aldri \hld\ of \alst{b}rugginn mjǫðr, &
\alst{sk}írar vęigar, \hld\ \edtrans{liggr \alst{sk}jǫldr yfir}{a shield lies over [them]}{\Bfootnote{Shields covering casks of mead is a common trope. Cf. TODO.}}, &
en \alst{á}s-męgir \hld\ í \alst{o}f-vę́ni; &
\alst{n}auðug sagða’k, \hld\ \alst{n}ú mun’k þęgja.“\eva

\bvb\speakernoteb{[The wallow quoth:]}
“Here stands brewed for Balder mead: \\
pure draughts—a shield lies over [them]; \\
but the os-lads \ken*{= Eese} [stand] in great suspense— \\
forced I spoke, now I will shut up!”\evb\evg


\bvg\bva\mssnote{\AM~1v/29}\speakernote{[Óðinn kvað:]}%
„\alst{Þ}ęgj-at vǫlva, \hld\ \alst{þ}ik vil’k fregna, &
\alst{u}nds es \alst{a}l-kunna, \hld\ vil’k \alst{ę}nn vita, &
hvęrr mun \alst{B}aldri \hld\ at \alst{b}ana verða, &
ok \alst{Ó}ðins son \hld\ \alst{a}ldri rę́na?“\eva

\bvb\speakernoteb{[Weden quoth:]}
“Shut not up, O wallow; thee I wish to ask! \\
Until all is known I wish to know further: \\
Who will become Balder’s bane, \\
and rob Weden’s son \ken*{= Balder} of age?”\evb\evg


\bvg\bva\mssnote{\AM~2r/1}\speakernote{[Vǫlva kvað:]}%
„\alst{H}ǫðr berr \alst{h}ǫ́van \hld\ \alst{h}róðr-baðm þinig, &
hann mun \alst{B}aldri \hld\ at \alst{b}ana verða, &
ok \alst{Ó}ðins son \hld\ \alst{a}ldri rę́na; &
\alst{n}auðug sagða’k, \hld\ \alst{n}ú mun’k þęgja.“\eva

\bvb\speakernoteb{[The wallow quoth:]}
“\inx[P]{Hath} bears the high fame-beam \ken{mistletoe} thither; \\
he will become Balder’s bane, \\
and rob Weden’s son \ken*{= Balder} of age— \\
forced I spoke, now I will shut up!”\evb\evg


\bvg\bva\mssnote{\AM~2r/3}\speakernote{[Óðinn kvað:]}%
„\alst{Þ}ęgj-at vǫlva, \hld\ \alst{þ}ik vil’k fregna, &
\alst{u}nds es \alst{a}l-kunna, \hld\ vil’k \alst{ę}nn vita, &
hvęrr mun \alst{h}ęipt \alst{H}ęði \hld\ \alst{h}ęfnt of vinna, &
eða \alst{B}aldrs \alst{b}ana \hld\ á \alst{b}ál vega?“\eva

\bvb\speakernoteb{[Weden quoth:]}
“Shut not up, O wallow; thee I wish to ask! \\
Until all is known I wish to know further: \\
Who will avenge that evil doing on Hath, \\
or drag onto the pyre Balder’s bane \ken*{= Hath}?”\evb\evg


\bvg\bva\mssnote{\AM~2r/4}\speakernote{[Vǫlva kvað:]}%
„Rindr berr \alst{V}ála \hld\ í \alst{v}estr-sǫlum, &
sá mun \alst{Ó}ðins sonr \hld\ \alst{ęi}n-nę́ttr vega; &
\alst{h}ǫnd of þvę́r-at \hld\ né \alst{h}ǫfuð kęmbir, &
áðr á \alst{b}ál of \alst{b}err \hld\ \alst{B}aldrs and-skota; &
\alst{n}auðug sagða’k, \hld\ \alst{n}ú mun’k þęgja.“\eva

\bvb\speakernoteb{[The wallow quoth:]}
“Rind bears \inx[P]{Wonnel} in the western halls: \\
he will—Weden’s son, one night old—fight. \\
He washes not his hand nor combs his head, \\
before onto the pyre he bears Balder’s opponent \ken*{= Hath}— \\
forced I spoke, now I will shut up.\footnoteB{The similarity in wording to the treatment of this myth in \Voluspa\ is striking; apart from the tense, ll. 2–4 here are near-identical to 32/4–33/2 there (for discussion on the narrative see introduction to \Voluspa\ 31–34). The irregularity of the stanza length might suggest that a line has been inserted in analogy with the aforementioned poem.}”\evb\evg


\bvg\bva\mssnote{\AM~2r/6}\speakernote{[Óðinn kvað:]}%
„\alst{Þ}ęgj-at vǫlva, \hld\ \alst{þ}ik vil’k fregna, &
\alst{u}nds es \alst{a}l-kunna, \hld\ vil’k \alst{ę}nn vita, &
hvęrjar ’ru \alst{m}ęyjar, \hld\ es at \alst{m}uni gráta &
ok á \alst{h}imin verpa \hld\ \alst{h}alsa-skautum?“\eva

\bvb\speakernoteb{[Weden quoth:]}
“Shut not up, O wallow; thee I wish to ask! \\
Until all is known I wish to know further: \\
Which are the maidens that weep heartily, \\
and onto heaven cast the front sheets?\footnoteB{According to \Gylfaginning\ 49 Hell promised to give Balder back to the Eese if “all things in the world, living and dead, cry for him”. The Eese relayed this message, and “the men and the animals and the earth and the stones and trees and all metals” cried for Balder. It may be that these maidens were included among the grievers (perhaps they were the walkirries, and this is what reveals Weden’s identity?), but their identity is otherwise unknown.}”\evb\evg


\bvg\bva\mssnote{\AM~2r/8}\speakernote{[Vǫlva kvað:]}%
„\alst{E}rt-at Veg-tamr, \hld\ sem \alst{e}k hugða, &
hęldr est \alst{Ó}ðinn, \hld\ \alst{a}ldinn gautr.“ &
\speakernote{[Óðinn kvað:]}„est-at \alst{v}ǫlva \hld\ \alst{n}é vís kona, &
hęldr est \alst{þ}riggja \hld\ \alst{þ}ursa móðir.“\eva

\bvb\speakernoteb{[The wallow quoth:]}
“Thou art not Waytame as I thought, \\
rather art thou Weden, the ancient Geat!”— \\
\speakernoteb{[Weden quoth:]}
“Thou art no \inx[C]{wallow} nor wise woman, \\
rather art thou the mother of three \inx[G]{Thurses}!”\evb\evg


\bvg\bva\mssnote{\AM~2r/9}\speakernote{[Vǫlva kvað:]}%
„\alst{H}ęim ríð Óðinn \hld\ \edtrans{ok \alst{h}róðigr ves}{and be renowned}{\Bfootnote{A sarcastic, taunting statement, the sense being: “Your fame, Weden, will not save you!”}}, &
svá komi-t \alst{m}anna \hld\ \alst{m}ęirr aptr á vit, &
es \alst{l}auss \alst{L}oki \hld\ \alst{l}íðr ór bǫndum &
ok \alst{r}agna \alst{r}ǫk \hld\ \edtrans{\alst{r}júfęndr}{rippers}{\Bfootnote{Presumably Surt and Lock with his children, as described in \Voluspa\ 40 ff.  The verb \emph{rjúfa} ‘\CV: to break, rip up, break a hole in’ is used in the same context in the formulaic \emph{þá’s rjúfask ręgin} ‘when the \inx[G]{Reins} are ripped’ (\Vafthrudnismal\ 52), \emph{unds (of) rjúfask ręgin} ‘until the Reins are ripped’ (\Grimnismal\ 4, \Lokasenna\ TODO and \Sigrdrifumal\ TODO).
One might also compare the similar sounding (but not or only very distantly related) verb \emph{rifna} ‘be riven, rent apart’ used in reference to the destruction of the world in Runic inscription Sö 154: \emph{Jǫrð sal rifna \hld\ ok upp-himinn} ‘Earth shall be riven, and Up-heaven’, and Arn \emph{Hryn} (in \Skp\ II pp. 185–6, ll. 3/7–8, see also note there): \emph{meiri verði þinn an þeira \hld\ þrifnuðr allr, unds himinn rifnar.} ‘greater than theirs may thy whole wealth be, until heaven is riven.’}} koma.“\eva

\bvb\speakernoteb{[The wallow quoth:]}
“Ride home, O Weden, and be renowned! \\
So may no other man come again to visit [me], \\
when loose Lock slips out of his bonds,\\
and [at] the \inx[L]{Rakes of the Reins} the rippers come!”\evb\evg


%TODO Late stanzas in paper manuscripts.

\sectionline
% Weden, Balder
	\bookStart{Speeches of the High One}[Hávamǫ́l]

\begin{flushright}%
\textbf{Dating:} See individual sections.

\textbf{Meter:} \Ljodahattr, \Galdralag, \Malahattr
\end{flushright}%

\section{Introduction}

The \textbf{Speeches of the High One} is the second poem of \Regius, which is the only medieval witness manuscript.  Several sts. are however cited or alluded to in other places, such as Eyv \emph{Hák} (TODO: formatting) 21 and \FostrbroedhraSaga\ TODO.

The poem before us does not very much seem like a single composition by one poet, but instead much more like a collection of scattered traditional poetry associated with the god Weden.  It seems to contain at least two poems of practical life advice, two mythological narratives, scattered gnomic poetry about runes, and a list of galders.  These various strands are united by their presumed speaker, namely Weden in His function as God of Wisdom.

Following previous authors, I identify the following strands, excepting various lone sts. that are probably later inserts.  In the present edition each of the following is given a separate, short introduction:

\begin{enumerate}
	\item 1–79 The Guest-strand; practical life advice, beginning with a guest arriving at a homestead
  \item 81–90 Various scattered sts. of advice
  \item 91–102 Weden’s failed seduction of Billing’s daughter
  \item 103–110 Weden’s obtaining of the Mead of Poetry
  \item 111–137 The Speeches of Loddfathomer; Weden’s advice to Loddfathomer
  \item 138–146 The Rune-tally; various sts. relating to runes and their magical use
  \item 146–165 The Leed-tally; Weden’s listing of 18 galders
\end{enumerate}

Two questions shortly arise: who was the redactor (i.e., the person who set these strands together, and gave the new work the title \emph{Háva mǫ́l}), and what was his motive?  While a detailed and sufficient answer will probably never be found, a careful reading of the final stanza, 165, gives us some clues.  By its prayer-like blessing, which brings up the Heathen dichotomy between the Gods and Ettins (the friends and enemies of Mankind, respectively) and calls the contents of the poem (which include unambiguous Heathen ritual instructions) “very useful” (\emph{all-þǫrf}); and by its reference to the process of oral transmission, the whole poem in something resembling the current form must (it seems) have been put together no later than the early 11th century, in a pre-scribal, pre-monastic, Heathen context. (Iceland converted around year 1000, but people surely clung to the old traditions for some time longer.)

As seen by the emphasis on the usefulness of the poetry, the reason for this redaction was not strictly antiquarian, but foremost utilitarian; the redactor gathered an amount of traditional poetry he found useful (whether for its life-advice or mythology) into a single poem, which could then be learned by heart by anyone.  In this he certainly achieved his goal.  The \Havamal\ is by far the greatest surviving collection of pre-Christian Norse advice poetry, and has functioned like a Noah’s Ark—or Hoardmimer’s Wood—for that genre.  Thus, those scattered stanzas which were not included by the redactor—and many must have existed—are now forever lost.

\sectionline

\section{The Guest-strand (1–79)}

The Guest-Strand (Old Norse: \emph{Gęsta-þáttr}) is one of the most interesting surviving works of Norse poetry.  Sadly, its structure has been obscured by the insertion of unrelated sts. and by poor translations.  My hope is to shed some light on the original coherence of the strand, while respecting the text as it appears in the manuscript.  As I do not think it can do each stanza justice, and since there is not exactly a clear progression of themes, I will not here attempt a stanza-by-stanza summary of this strand. Rather, I will give some important observations and then let the reader read for himself.

The Strand is a piece of advice poetry, and takes its outset in a wanderer’s arriving as a guest at a Norse farmstead.  It first (roughly sts. 1–4) discusses the mutual responsibilites between guest and host, and then moves on to broader human interactions, with a particular focus on alcohol, war, friendship and human wisdom.  While there is some coherence and nice transitions are frequently employed in order to shift from one theme to another (e.g. between sts. 4 and 5, or 10 and 11), the poem is not clearly divided into sections, nor is there (after the very first stanzas) a linear progression from one theme to another.

At all turns the poem advices caution and shrewdness.  A man should always carry his “manwit” (ON \emph{man-vit}, a word somewhat analogous with the English “common sense”) with him; he should think before he speaks

The poem moves seamlessly between various parts of life.  To do so the poet often employs transitions where a st. repeats the structure of the previous one, but with a new subject.  This is particularly evident in sts. 4–5 and 10–11.

TODO.

\sectionline

\bvg\bva\alst{G}ȧttir allar \hld\ áðr \alst{g}angi framm &
\ind \edtext{of \alst{sk}oðask \alst{sk}yli,}{\Bfootnote{om. \GylfMS}} &
\ind of \alst{sk}yggnask \alst{sk}yli; &
því-at \alst{ȯ}-víst ’s at vita, \hld\ hvar \alst{ȯ}-vinir &
\ind sitja ȧ \alst{f}lęti \alst{f}yrir.\eva

\bvb All doorways—before one might go forth \\
\ind he should spy round; \\
\ind he should pry round; \\
for it is unsure to know where enemies \\
\ind sit on the benches within.\evb\evg


\bvg\bva \alst{G}efęndr hęilir, \hld\ \alst{g}ęstr ’s inn kominn, &
\ind hvar skal \alst{s}itja \alst{s}já? &
mjǫk es \alst{b}ráðr \hld\ sá’s \edtrans{ȧ \alst{b}rǫndum}{on the fires}{\Bfootnote{Possibly referring a Norwegian folk custom, wherein a guest would sit down on the wood-pile outside of the door, waiting until being let in; see further TODO SOME ARTICLE on this custom.  The speaker is announcing to the hosts (or “givers”) that a guest, frozen, wet and tired, is currently sitting on the wood-pile, and ought to be let in.}} skal &
\ind \edtrans{síns of \alst{f}ręista \alst{f}rama}{test his furtherance}{\Bfootnote{Try his luck, see how far he gets.  The same line is also found in \Vafthrudnismal\ 11, 13, 15, 17.}}.\eva

\bvb O givers, hail! A guest has come in; \\
\ind where shall this one sit? \\
Very hurried is he who on the fires shall \\
\ind test his furtherance.\evb\evg


\bvg\bva\alst{Ę}lds es þǫrf \hld\ þęim’s \alst{i}nn es kominn &
\ind ok ȧ \alst{k}néi \alst{k}alinn, &
\alst{m}atar ok váða \hld\ es \alst{m}anni þǫrf, &
\ind þęim’s hęfr of \alst{f}jall \alst{f}arit.\eva

\bvb Of fire there is need for the one who is come in, \\
\ind and cold about the knees; \\
of food and of clothing there is need for the man \\
\ind who over the fell has fared.\evb\evg


\bvg\bva\edtext{\alst{V}ats es þǫrf \hld\ þęim’s til \alst{v}erðar kømr, &
\ind \alst{þ}ęrru ok \alst{þ}jóð-laðar, &
\alst{g}óðs of ǿðis, \hld\ —ef sér \alst{g}eta mę́tti— &
\ind \alst{o}rðs ok \edtrans{\alst{ę}ndr-þǫgu}{silence in return}{\Bfootnote{One may note that the verb \emph{þęgja} ‘shut up, be silent’—of which \emph{*þaga}, which only appears in the present cpd., is a derivative formed in the same way as \emph{saga} ‘saw, history’ to \emph{sęgja} ‘say, speak’—and the related noun \emph{þǫgn} ‘silence’ are frequently used at the beginning of Scaldic poems (e.g. Arn \emph{Magndr} 1\textsuperscript{II}: \emph{þegi sęim-brotar} ‘may gold-breakers \ken{generous men} be silent’, Egill \emph{Berdr} 1\textsuperscript{V}: \emph{hyggi \dots\ til þagnar þinn lýðr} ‘may thy retinue focus on silence’, Glúmr \emph{Gráf} 1\textsuperscript{I}: \emph{biðjum vér þagnar} ‘we ask for silence’).}}.}{\lemma{ALL}\Bfootnote{There is a good train of thought throughout the st.: the guest must first wash and dry himself, and then be welcomed to sit and eat at the table.  After the host has furnished him with these amenities the need for proper conduct now shifts onto the guest, who must speak and speak wisely.}}\eva

\bvb Of water there is need for the one who comes for a meal, \\
\ind of a towel and a hearty welcome; \\
of a good reception—if he might get one— \\
\ind of a word, of and silence in return.\evb\evg


\bvg\bva\alst{V}its es þǫrf \hld\ þęim’s \alst{v}íða ratar; &
\ind dę́lt es \alst{h}ęima \alst{h}vat; &
\edtrans{at \alst{au}ga-bragði}{Into a laughing-stock}{\Bfootnote{Idomatic.  \emph{auga-bragð} literally means ‘twinkling of an eye, moment’; the sense here is thus something like ‘a quick glance of derision’.}} \hld\ verðr sá’s \alst{ę}kki kann &
\ind ok með \alst{s}notrum \alst{s}itr.\eva

\bvb Of wit there is need for the one who widely roams; \\
\ind everything is easy at home. \\
Into a laughing-stock turns he who nothing knows, \\
\ind and among the clever sits.\evb\evg


\bvg\bva At \alst{h}yggjandi sinni \hld\ skyli-t maðr \alst{h}rǿsinn vesa, &
\ind hęldr \alst{g}ę́tinn at \alst{g}ęði, &
þȧ’s \alst{h}orskr ok þǫgull \hld\ kømr \alst{h}ęimis-garða til, &
\ind sjaldan verðr \alst{v}íti \alst{v}ǫrum. &
því-at \alst{ȯ}-brigðra vin \hld\ fę̇r \edtrans{maðr}{man}{\Bfootnote{In \Regius\ abbreviated with the rune ᛘ \textbf{m} “man”, the first of 45 such instances in the present poem.  While Anglo-Saxon Latin-script mss. use several runes ideographically (e.g. ᛟ \textbf{o} for OE \emph{ǿðel} ‘homeland, patrimony’), there are (to my knowledge) no Scandinavian examples with runes other than ᛘ.  The tradition of ideographic runes standing for their names is ancient and goes back to the time before Latin writing, as proven by the inscriptions from Stentoften (DR 357) and Ingelstad (Ög 43), which use the runes ᛃ \textbf{j} for \emph{ár} ‘year, good harvest’ and ᛞ \textbf{d} for \emph{dagʀ} ‘day’, respectively.  For rune names see below: Anonymous Runerow Poems.}} \alst{a}ldri-gi, &
\ind an \alst{m}an-vit \alst{m}ikit.\eva

\bvb Of his thinking should man not be boastful, \\
\ind but rather guarding of his senses \\
when sharp and silent he comes to a homestead; \\
\ind sudden harm seldom strikes the wary, \\
for an unfickler friend man never gets \\
\ind than much \inx[C]{manwit}.\evb\evg


\bvg\bva Hinn \alst{v}ari gęstr, \hld\ es til \alst{v}erðar kømr, &
\ind \edtrans{\alst{þ}unnu hljóði \alst{þ}ęgir}{shupts up and listens closely}{\Bfootnote{lit. ‘shuts up with thin (i.e. attentive) listening’.}}; &
\alst{ęy}rum hlýðir, \hld\ en \alst{au}gum skoðar, &
\ind svá \edtext{nýsisk \alst{f}róðra hvęrr \alst{f}yrir}{\lemma{nýsisk fyrir ‘looks ahead’}\Bfootnote{This verb underlies the noun \emph{for-njósn} as found in \Sigrdrifumal\ 25.}}.\eva

\bvb The wary guest—when for a meal he comes— \\
\ind shuts up and listens closely. \\
With ears he listens and with eyes he watches; \\
\ind so looks each learned man ahead.\evb\evg


\bvg\bva Hinn es \alst{s}ę́ll, \hld\ es \alst{s}ér of getr &
\ind \edtrans{\alst{l}of ok \alst{l}íkn-stafi}{praise and staves of liking}{\Bfootnote{\emph{líkn} ‘liking’ is a very interesting word.  It is defined by \ONP\ as: ‘mercy, compassion, relief, comfort, help’.  In the present poem its precise meaning seems to be something like ‘the state of being liked by your surroundings to the point where people are willing to help you out’.  Cf. its two other occurrences in the present poem: sts. 120 and especially 123 (where it is likewise paired with \emph{lof} ‘praise’).}}; &
\alst{ȯ}-dę́lla ’s við þat, \hld\ es \alst{ęi}ga skal &
\ind \alst{a}nnars brjóstum \alst{í}.\eva

\bvb This one is blessed, who for himself does get \\
\ind praise and staves of liking. \\
It is uneasy regarding that which one shall own \\
\ind in another man’s chest.\evb\evg


\bvg\bva%
\edtrans{\alst{S}á}{That one}{\Bfootnote{Contrasting with \emph{hinn} ‘this one’ in the previous stanza.}} es \alst{s}ę́ll, \hld\ es \alst{s}jalfr of á &
\ind \alst{l}of ok vit meðan \alst{l}ifir; &
því-at \alst{i}ll rǫ́ð \hld\ hęfr maðr \alst{o}pt þęgit &
\ind \alst{a}nnars brjóstum \alst{ó}r.\eva

\bvb That one is blessed, who himself does have \\
\ind praise and wits while he lives; \\
for ill counsels has man oft taken \\
\ind out of another man’s chest.\evb\evg


\bvg\bva\alst{B}yrði \alst{b}ętri \hld\ berr-at maðr \alst{b}rautu at, &
\ind an sé \alst{m}an-vit \alst{m}ikit; &
\alst{au}ði bętra \hld\ þykkir þat í \alst{ȯ}-kunnum stað; &
\ind slíkt es \alst{v}á-laðs \alst{v}era.\eva

\bvb A better burden bears man not on the road \\
\ind than much manwit. \\
In an unknown place it seems better than wealth; \\
\ind such is the destitute man’s shelter.\evb\evg


\bvg\bva\alst{B}yrði \alst{b}ętri \hld\ berr-at maðr \alst{b}rautu at, &
\ind an sé \alst{m}an-vit \alst{m}ikit; &
\alst{v}eg-nest \alst{v}erra \hld\ \alst{v}egr-a \edtrans{\alst{v}ęlli at}{on the plain}{\Bfootnote{Formulaic, the word \emph{vǫllr} ‘plain, (uncultivated) field’ is also used in sts. 38 and 49. It is easily understood that the wild heaths and plains of Iron Age Norway were particularly unsafe places where a traveller needed to keep his wits about him, lest he fall victim to robbers or murderers (so st. 38).}}, &
\ind an sé \alst{o}f-drykkja \alst{ǫ}ls.\eva

\bvb A better burden bears man not on the road \\
\ind than much manwit. \\
Worse way-provision he drags not along on the plain \\
\ind than a too great drink of ale.\evb\evg


\bvg\bva Es-a svá \alst{g}ótt, \hld\ sęm \alst{g}ótt kveða, &
\ind \alst{ǫ}l \alst{a}lda sonum; &
því-at \alst{f}ę́ra vęit, \hld\ es \alst{f}lęira drekkr, &
\ind síns til \alst{g}ęðs \alst{g}umi.\eva

\bvb It is not so good, as good they say, \\
\ind ale for the sons of men; \\
for the less he knows, as the more he drinks, \\
\ind man of his own senses.\evb\evg


\bvg\bva \edtrans{\alst{Ó}·minnis-hegri}{Forgetfulness-heron}{\Bfootnote{Lit. “unmemory-heron”; a rather interesting personification of drunkenness as a hovering bird.}} hęitir, \hld\ sá’s yfir \alst{ǫ}lðrum þrumir, &
\ind hann stelr \alst{g}ęði \alst{g}uma; &
þess \alst{f}ogls \alst{f}jǫðrum \hld\ ek \alst{f}jǫtraðr vas’k &
\ind í \alst{g}arði \alst{G}unnlaðar.\eva

\bvb Forgetfulness-heron is he called, who hovers over ale-feasts; \\
\ind he robs man of his senses. \\
By that bird’s feathers I was fettered \\
\ind in the yards of \inx[P]{Guthlathe}.\evb\evg


\bvg\bva\alst{Ǫ}lr ek varð, \hld\ varð \alst{o}fr-ǫlvi, &
\ind at hins \alst{f}róða \alst{F}jalars; &
því es \alst{ǫ}lðr batst, \hld\ at \alst{a}ptr of hęimtir &
\ind hvęrr sitt \alst{g}ęð \alst{g}umi.\eva

\bvb Drunk I became—I became the drunkest by far— \\
\ind at the learned Fealer’s. \\
That ale-feast is best, where every man \\
\ind gets back to his senses.\evb\evg


\bvg\bva\alst{Þ}agalt ok hugalt \hld\ skyli \alst{þ}jóðans barn &
\ind ok \alst{v}íg-djarft \alst{v}esa; &
\alst{g}laðr ok ręifr \hld\ skyli \alst{g}umna hvęrr, &
\ind unds sinn \alst{b}íðr \alst{b}ana.\eva

\bvb Silent and thoughtful should the king’s child \\
\ind —and battle-bold—be. \\
Glad and cheerful should every man be, \\
\ind until he suffer his bane.\evb\evg


\bvg\bva\alst{Ó}·snjallr maðr \hld\ hyggsk munu \alst{ę}y lifa, &
\ind ef við \alst{v}íg \alst{v}arask; &
en \alst{ę}lli gefr hǫ́num \hld\ \alst{ę}ngi frið, &
\ind þótt hǫ́num \alst{g}ęirar \alst{g}efi.\eva

\bvb The unvalorous man thinks he will always live \\
\ind if he of war be wary; \\
but old age gives him no peace, \\
\ind which yet spears would give him.\footnoteB{The unvalorous man might have been spared by the spears, but death will still find him through miserable old age. Since death is unavoidable it is better to live bravely, even if one risks dying in battle, than to live cowardly and die of sickness. This connects well to the ancient view of the ‘straw-death’ (TODO).}\evb\evg


\bvg\bva\alst{K}ópir af-glapi, \hld\ es til \alst{k}ynnis kømr, &
\ind \alst{þ}ylsk hann umb eða \alst{þ}rumir; &
allt es \alst{s}ęnn, \hld\ ef \alst{s}ylg of getr, &
\ind uppi ’s þȧ \alst{g}ęð \alst{g}uma.\eva

\bvb Gapes the oaf when to visit he comes; \\
\ind he mumbles about or loiters. \\
All at once—if a sip he gets— \\
\ind exposed is the mind of the man.\evb\evg


\bvg\bva Sá ęinn \alst{v}ęit, \hld\ es \alst{v}íða ratar &
\ind ok \edtrans{hęfr \alst{f}jǫlð of \alst{f}arit}{has journeyed much}{\Bfootnote{Cf. \Vafthrudnismal\ 3, 44, et.c., where Weden repeats: \emph{Fjǫlð ek fór, \hld\ fjǫlð fręistaða’k, // fjǫlð ek ręynda ręgin} ‘Much I journeyed, much I tried, much I tested the \inx[G]{Reins}.’}}, &
hvęrju \alst{g}ęði \hld\ stýrir \alst{g}umna hvęrr, &
\ind sá es \alst{v}itandi ’s \alst{v}its.\eva

\bvb He alone knows, who widely roams, \\
\ind and has journeyed much, \\
which sort of mind every man wields, \\
\ind who is knowing of his wits.\evb\evg


\bvg\bva\edtrans{\alst{H}aldi-t maðr ȧ kęri}{Man ought not to hold onto the cask}{\Bfootnote{Perhaps referring to a toast wherein a drinking vessel would be passed around in a circle and each member would drink.  Such toasts were drunk for a long time in Northern Europe—indeed this is the origin of the Scandinavian toasting-word, \emph{skål} ‘prosit, cheers!’, lit. ‘bowl!’.  “Holding onto” the vessel (and not letting the next person drink) was surely seen as very rude; as late as 1519 a man in Jämtland was killed in an argument resulting from his refusal to pass on the bowl (see \textcite{Sjöberg1907}).  The sense is thus: “Do not refuse a toast when offered (but do not drink too much, either!)”}}, \hld\ drekki þó at \alst{h}ófi mjǫð, &
\ind \edtrans{mę́li \alst{þ}arft eða \alst{þ}ęgi}{he ought to speak the needful or shut up}{\Bfootnote{Formulaic, line occurs identically in \Vafthrudnismal\ 10/2.}}; &
\alst{ȯ}-kynnis þess \hld\ váar þik \alst{ę}ngi maðr, &
\ind at gangir \alst{s}nimma at \alst{s}ofa.\eva

\bvb Man ought not to hold onto the cask, but still drink mead in moderation; \\
\ind he ought to speak the needful or shut up. \\
For that uncouthness will no man blame thee, \\
\ind that thou go early to sleep.\evb\evg


\bvg\bva\alst{G}rǫ́ðugr halr, \hld\ nema \alst{g}ęðs viti, &
\ind \alst{e}tr sér \alst{a}ldr-trega; &
opt fę̇r \alst{h}lǿgis, \hld\ es með \alst{h}orskum kømr, &
\ind \alst{m}anni hęimskum \alst{m}agi.\eva

\bvb The gluttonous man—unless he know his sense— \\
\ind eats himself a life-sorrow. \\
Oft the belly, when among the sharp he comes, \\
\ind brings the foolish man ridicule.\evb\evg


\bvg\bva\alst{H}jarðir þat vitu, \hld\ nę́r \alst{h}ęim skulu, &
\ind ok \alst{g}anga þȧ af \alst{g}rasi; &
en \alst{ȯ}-sviðr maðr \hld\ kann \alst{ę́}va-gi &
\ind síns of \alst{m}ál \alst{m}aga.\eva

\bvb Herds know when home they shall [go], \\
\ind and then part from the grass; \\
but an unwise man never knows \\
\ind his own belly’s measure.\evb\evg


\bvg\bva\alst{V}e-sall maðr \hld\ ok \alst{i}lla skapi &
\ind \alst{h}lę́r at \alst{h}ví-vetna; &
hitt-ki hann \alst{v}ęit, \hld\ es \alst{v}ita þyrpti, &
\ind at \edtrans{hann es-a \alst{v}amma \alst{v}anr}{he is not free of blemishes}{\Bfootnote{Formulaic, cf. \Lokasenna\ 30: \emph{es-a þér vamma vant} ‘thou art not free of blemishes’.}}.\eva

\bvb The wretched man and badly turned out \\
\ind laughs at anything. \\
This he knows not, which he might need to know: \\
\ind that he is not free of blemishes.\evb\evg


\bvg\bva\alst{Ó}-sviðr maðr \hld\ vakir umb \alst{a}llar nę́tr &
\ind ok \alst{h}yggr at \alst{h}ví-vetna; &
þȧ es \alst{m}óðr, \hld\ es at \alst{m}orni kømr; &
\ind alt es \alst{v}íl sęm \alst{v}as.\eva

\bvb The unwise man is awake for all nights \\
\ind and thinks of anything. \\
Then he is weary when the morning comes: \\
\ind all the trouble is as it was.\evb\evg


\bvg\bva\alst{Ȯ}-snotr maðr \hld\ hyggr sér \alst{a}lla vesa &
\ind \alst{v}ið-hlę́jęndr \alst{v}ini; &
hitt-ki hann \alst{f}iðr, \hld\ þótt of hann \alst{f}ár lesi, &
\ind ef með \alst{s}notrum \alst{s}itr.\eva

\bvb The unclever man thinks all those \\
\ind who laugh with him his friends. \\
This he finds not, that they yet make sport in him, \\
\ind if among the clever he sits.\evb\evg


\bvg\bva\alst{Ȯ}-snotr maðr \hld\ hyggr sér \alst{a}lla vesa &
\ind \alst{v}ið-hlę́jęndr \alst{v}ini; &
\alst{þ}ȧ þat fiðr \hld\ es at \alst{þ}ingi kømr, &
\ind at \edtrans{á \alst{f}or-mę́lęndr \alst{f}áa}{has spokesmen few}{\Bfootnote{Repeated in st. 62.  He has few who are ready to take his side and speak up for him (in legal proceedings); true friends are proven in hard times, not in drunken chatter.  The Thing was the old Germanic legal assembly, where smaller disputes might easily turn into deadly feuds.}}.\eva

\bvb The unclever man thinks all those \\
\ind who laugh with him his friends. \\
Then he finds, when to the \inx[C]{Thing} he comes, \\
\ind that he has spokesmen few.\evb\evg


\bvg\bva\alst{Ȯ}-snotr maðr \hld\ þykkisk \alst{a}llt vita, &
\ind ef á sér í \edtrans{\alst{v}ǫ̇}{nook}{\Bfootnote{From earlier \emph{*vrǫ̇}; cf. Swedish \emph{vrå} ‘corner, nook’, rare English \emph{wroo} ‘id.’  The present stanza is to my knowledge the only Norse attestation of the form \emph{vǫ̇}, which features a rare Western sound change from \emph{vr-} to \emph{v-}.  The more common change \emph{vr-} to \emph{r-} yields \emph{rǫ̇}, which is the normal Norse form. — Tangentially this word is brought up in \FGT\ as an example of a word with nasal \emph{ǫ̇}, and contrasted with oral \emph{ǫ́} in \emph{rǫ́} ‘sailyard’.}} \alst{v}eru; &
hitt-ki hann \alst{v}ęit, \hld\ hvat skal \alst{v}ið kveða, &
\ind ef hans \alst{f}ręista \alst{f}irar.\eva

\bvb The unclever man seems to know everything \\
\ind if he takes shelter in a nook. \\
This he knows not, what he shall answer \\
\ind if men test him.\evb\evg


\bvg\bva\alst{Ȯ}-snotr maðr, \hld\ es með \alst{a}ldir kømr, &
\ind \alst{þ}at ’s batst at hann \alst{þ}ęgi; &
\alst{ę}ngi þat vęit, \hld\ at hann \alst{ę}kki kann, &
\ind nema hann \alst{m}ę́li til \alst{m}art. &
\alst{v}ęit-a maðr, \hld\ hinn’s \alst{v}ę́t-ki vęit, &
\ind þótt hann \alst{m}ę́li til \alst{m}art.\eva

\bvb The unclever man when among people he comes— \\
\ind it is best that he shut up. \\
No one knows that he nothing knows, \\
\ind unless he speak too much. \\
The man knows not, who nothing knows, \\
\ind that he speak too much.\evb\evg


\bvg\bva\alst{F}róðr sá þykkisk, \hld\ es \edtext{\alst{f}regna kann, &
\ind ok \alst{s}ęgja}{\lemma{fregna \dots\ sęgja ‘ask \dots\ answer’}\Bfootnote{Perhaps specifically in the context of a riddling contest of wisdom.}} hit \alst{s}ama, &
\alst{ęy}-vitu lęyna \hld\ męgu \alst{ý}ta synir &
\ind því es \alst{g}ęngr of \alst{g}uma.\eva

\bvb Learned seems he who can ask \\
\ind and answer the same [way]. \\
In no way may the sons of men hide \\
\ind that which eludes a man.\evb\evg


\bvg\bva\alst{Ǿ}rna mę́lir, \hld\ sá’s \alst{ę́}va þęgir, &
\ind \alst{st}að-lausu \alst{st}afi; &
\edtext{\alst{h}rað-mę́lt tunga, \hld\ \edtrans{nema \alst{h}aldęndr ęigi}{unless it be held in place}{\Bfootnote{lit. ‘unless holders own it’ or ‘unless it own holders’. The ‘holders’ are perhaps the teeth which hold the tongue in place.}}, &
\ind opt sér ȯ-\alst{g}ótt of \alst{g}ęlr}{\lemma{hrað-mę́lt \dots\ of gęlr ‘A quick-spoken \dots\ for itself’}\Bfootnote{Formulaic. Cf. \Lokasenna\ 31.}}.\eva

\bvb He who never shuts up speaks plenty many \\
\ind utterings of absurdity. \\
A quick-spoken tongue—unless it be held in place— \\
\ind oft sings evil [into being] for itself.\evb\evg


\bvg\bva At \alst{au}ga-bragði \hld\ skal-a maðr \alst{a}nnan hafa, &
\ind þótt til \alst{k}ynnis \alst{k}omi; &
margr \alst{f}róðr þykkisk, \hld\ ef \alst{f}reginn es-at &
\ind ok nái \edtrans{\alst{þ}urr-fjallr}{dry-skinned}{\Bfootnote{i.e. ‘untested’, equivalent to the English idiom \emph{get one’s feet wet}.  The word \emph{fell} \char`~\ \emph{fjall} ‘skin, pelt’ is rare in Old Norse literature and only occurs in cpds, e.g. \Volundarkvida\ 11: \emph{ber-fjall} ‘bear-pelt’.  It survives in modern Swedish \emph{fjäll} ‘scale (on fish and reptiles)’}} \alst{þ}ruma.\eva

\bvb For a laughing-stock shall man not have another \\
\ind when he comes to visit. \\
Many a one seems learned if he is not asked, \\
\ind and gets to loiter about dry-skinned.\evb\evg


\bvg\bva\alst{F}róðr þykkisk \hld\ sá’s \alst{f}lótta tękr &
\ind \edtrans{\alst{g}ęstr}{guest}{\Bfootnote{The situation hinted at in this and the following stanza is that two guests—unknown to eachother—have come to the same homestead.  The sense is that when mocked by a stranger it is best not to engage, since the dealing may quickly turn violent.  Cf. sts. 122, 123, and 125.}} at \alst{g}ęst hę́ðinn; &
\alst{v}ęit-a gǫrla \hld\ sá’s of \alst{v}erði glissir, &
\ind þótt með \alst{g}rǫmum \alst{g}lami.\eva

\bvb Learned seems he who takes to flight, \\
\ind the guest, from a scoffing guest. \\
He knows not clearly, who grins over the food, \\
\ind that he be flirting with fiends.\evb\evg


\bvg\bva\alst{G}umnar margir \hld\ erusk \alst{g}agn-hollir, &
\ind en at \alst{v}irði \alst{v}rekask; &
\alst{a}ldar róg \hld\ þat mun \alst{ę́} vesa; &
\ind órir \alst{g}ęstr við \alst{g}ęst.\eva

\bvb Many men are well true to each other, \\
\ind but over food drive each other away. \\
The strife of mankind will that ever be; \\
\ind guest raves against guest.\evb\evg


\bvg\bva\alst{Á}r-liga verðar \hld\ skyli maðr \alst{o}pt fȧa, &
\ind nema til \alst{k}ynnis \alst{k}omi; &
\alst{s}itr ok \alst{s}nópir, \hld\ lę́tr sęm \alst{s}olginn sé, &
\ind ok kann \alst{f}regna at \alst{f}ǫ́u.\eva

\bvb An early meal should man oft get, \\
\ind unless he come to visit: \\
he sits and sulks, sounds as if starved, \\
\ind and can ask about little.\evb\evg


\bvg\bva\alst{A}f·hvarf mikit \hld\ es til \alst{i}lls vinar, &
\ind þótt ȧ \alst{b}rautu \alst{b}úi, &
en til \alst{g}óðs vinar \hld\ liggja \alst{g}agn-vegir, &
\ind þótt hann sé \alst{f}irr \alst{f}arinn.\eva

\bvb A great detour it is to a bad friend, \\
\ind although he live on the road; \\
but to a good friend lie the finest ways, \\
\ind although he far gone be.\evb\evg


\bvg\bva\alst{G}anga \edtext{skal}{\Afootnote{emend.; om. \Regius}}, \hld\ skal-a \alst{g}ęstr vesa &
\ind \alst{ęy} í \alst{ęi}num stað; &
\alst{l}júfr verðr \alst{l}ęiðr, \hld\ ef \alst{l}ęngi sitr &
\ind \alst{a}nnars flętjum \alst{ȧ}.\eva

\bvb One shall go; he shall not be a guest \\
\ind forever in one place. \\
The loved becomes loathed if for long he sits \\
\ind on another man’s benches.\footnoteB{The customary length of stay in old times was three nights.  So Eyel’s saw, ch. 78: \emph{þat var engi siðr, at sitja lengr en þrjár nę́tr at kynni.} ‘it was not customary to stay longer than three nights when visiting.’  Compare a much Jutish saying: \emph{en tredje dags gjæst stinker} ‘a third day’s guest stinks’, which closely resembles a maxim attributed to Benjamin Franklin: “Guests, like fish, begin to smell after three days.”  It is probably with respect to such proverbs that Auden and Taylor translate the latter half of the present stanza “He starts to stink who outstays his welcome, / in a hall that is not his own.”}\evb\evg


\bvg\bva\edtrans{\alst{B}ú es \alst{b}ętra, \hld\ þótt lítit sé}{A dwelling is better though small it be}{\Bfootnote{The b-line is missing the necessary alliteration, but no good emendation suggests itself.}}, &
\ind \alst{h}alr es \alst{h}ęima \alst{h}vęrr; &
þótt \alst{t}vę́r gęitr ęigi \hld\ ok \alst{t}aug-ręptan sal, &
\ind þat ’s þó \alst{b}ętra an \alst{b}ǿn.\eva

\bvb A dwelling is better though small it be; \\
\ind each is a hero at home. \\
Though two goats he own and a cord-roofed hall, \\
\ind it is yet better than begging.\evb\evg


\bvg\bva\alst{B}ú es \alst{b}ętra, \hld\ þótt lítit sé, &
\ind \alst{h}alr es \alst{h}ęima \alst{h}vęrr; &
\alst{b}lóðugt es hjarta \hld\ þęim’s \alst{b}iðja skal &
\ind sér í \alst{m}ál hvęrt \alst{m}atar.\eva

\bvb A dwelling is better though small it be; \\
\ind each is a hero at home. \\
Bloody is the heart in him who shall beg \\
\ind for his every meal of food.\evb\evg


\bvg\bva%
\alst{V}ǫ́pnum sínum \hld\ skal-a maðr \edtrans{\alst{v}ęlli ȧ}{on the plain}{\Bfootnote{Formulaic, see note to st. 12.}} &
\ind \edtrans{\alst{f}eti ganga \alst{f}ramarr}{take one step further}{\Bfootnote{Formulaic. Cf. \Lokasenna\ 1: \emph{svá’t ęinu-gi feti gangir framarr} ‘so that thou not take one step further’.}}; &
því-at ȯ-\alst{v}íst ’s at \alst{v}ita, \hld\ nę́r verðr ȧ \alst{v}egum úti &
\ind \alst{g}ęirs of þǫrf \alst{g}uma.\eva

\bvb From his weapons shall man on the plain \\
\ind not take one step further; \\
for it is unsure to know, when on the ways outside, \\
\ind man comes in need of a spear.\evb\evg


\bvg\bva%
Fann’k-a \alst{m}ildan \alst{m}ann \hld\ eða svá \edtrans{\alst{m}atar góðan}{good of meat}{\Bfootnote{A Viking Age expression; see Encyclopedia.}}, &
\ind at vę́ri-t \alst{þ}iggja \alst{þ}egit; &
eða \alst{s}íns féar \hld\ \alst{s}vá-gi \edtext{[...]}{\Bfootnote{It is doubtless that a word has been lost here; the meter and sense require it. \textcite{FinnurEdda}\ suggests \emph{gløggvan} ‘miserly, stingy’, giving a litotes ‘so unstingy’, i.e., ‘so generous’.}}, &
\ind at \alst{l}ęið sé \alst{l}aun, ef þegi.\eva

\bvb I found not a generous man or one so \inx[C]{good of meat}, \\
\ind that a gift were not accepted; \\
or one with his \inx[C]{fee} so not [...], \\
\ind that the repayments were loathed, if he accepted [them].\footnoteB{No man is so generous that he would refuse a gift presented to him, nor loathe receiving a favour as thanks for his generosity.}\evb\evg


\bvg\bva%
\alst{F}éar síns, \hld\ es \alst{f}ęngit hęfr, &
\ind skyli-t maðr \alst{þ}ǫrf \alst{þ}ola; &
opt sparir \alst{l}ęiðum \hld\ þat’s hęfr \alst{l}júfum hugat; &
\ind mart gęngr \alst{v}err an \alst{v}arir.\eva

\bvb Of his own \inx[C]{fee} which he has earned \\
\ind should man not suffer need. \\
One oft saves for the loathed what one meant for the loved; \\
\ind much goes worse than expected.\evb\evg


\bvg\bva%
\edtrans{\alst{V}ǫ́pnum ok \alst{v}ǫ́ðum}{With weapons and garments}{\Bfootnote{i.e. weapons and armour (the “garments” are probably no silks); friends are supposed to help each other and strengthen their “violence capital”.  This alliterative word-pair is formulaic and in other occurences exclusively refers to implements of war; cf. e.g. \Beowulf\ 39, where \inx[P]{Shield}’s pyre-ship is loaded with \emph{hilde-wǽpnum \alst\ ǫnd heaðo-wǽdum} ‘war-weapons and battle-garments’.}} \hld\ skulu \alst{v}inir glęðjask; &
\ind \edtrans{þat ’s ȧ \alst{s}jǫlfum \alst{s}ýnst}{that is best seen on oneself}{\Bfootnote{i.e. in one’s own experience.}}; &
\alst{v}iðr-gefęndr ok ęndr-gefęndr \hld\ erusk \alst{v}inir lęngst, &
\ind ef \edtrans{þat}{it}{\Bfootnote{The friendship.}} bíðr at \alst{v}erða \alst{v}ęl.\eva

\bvb With weapons and garments shall friends gladden each other; \\
\ind that is best seen on oneself. \\
Givers-back and givers-again are friends for the longest \\
\ind if it comes to last long.\evb\evg


\bvg\bva\alst{V}in sínum \hld\ skal maðr \alst{v}inr \alst{v}esa, &
\ind ok \alst{g}jalda \alst{g}jǫf við \alst{g}jǫf; &
\alst{h}látr við \alst{h}látri \hld\ skyli \alst{h}ǫlðar taka, &
\ind en \alst{l}ausung við \alst{l}ygi.\eva

\bvb With his friend shall man be a friend, \\
\ind and pay gift against gift; \\
laughter against laughter should men employ, \\
\ind but duplicity against lie.\evb\evg


\bvg\bva\alst{V}in sínum \hld\ skal maðr \alst{v}inr vesa, &
\ind \alst{þ}ęim ok \alst{þ}ess vin; &
en \alst{ȯ}-vinar síns \hld\ skyli \alst{ę}ngi maðr &
\ind \alst{v}inar \alst{v}inr \alst{v}esa.\eva

\bvb With his friend shall man be a friend, \\
\ind with him and his friend; \\
but his enemy’s, should no man, \\
\ind friend’s friend be.\evb\evg


\bvg\bva\alst{V}ęitst, ef \alst{v}in átt, \hld\ þann’s \alst{v}ęl trúir &
\ind ok vilt af hǫ́num \alst{g}ótt \alst{g}eta, &
\alst{g}ęði skalt við þann \hld\ ok \alst{g}jǫfum skipta, &
\ind \alst{f}ara at \alst{f}inna opt.\eva

\bvb Thou knowest, if thou have a friend whom thou well trust, \\
\ind and wilt receive good from him: \\
thoughts and gifts shalt thou trade with him; \\
\ind journey to find him oft.\footnoteB{Several lines of the present st. are shared with st. 119.}\evb\evg


\bvg\bva Ef þú \alst{á}tt \alst{a}nnan, \hld\ þann’s \alst{i}lla trúir, &
\ind vilt af hǫ́num þó \alst{g}ótt \alst{g}eta, &
\edtext{\alst{f}agrt skalt mę́la við þann, \hld\ en \alst{f}látt hyggja}{\lemma{fagrt \dots\ mę́la \dots\ flátt hyggja ‘fairly \dots\ speak \dots\ falsely think’}\Bfootnote{Formulaic, cf. sts. 90, 91.}} &
\ind ok gjalda \alst{l}ausung við \alst{l}ygi.\eva

\bvb If thou have another whom thou badly trust, \\
\ind and wilt yet receive good from him: \\
fairly shalt thou speak with him, but falsely think, \\
\ind and pay duplicity against lie.\evb\evg


\bvg\bva Þat ’s \alst{ę}nn umb þann, \hld\ es þú \alst{i}lla trúir &
\ind ok þér es \alst{g}runr at \alst{g}ęði, &
\alst{h}lę́ja skalt við þęim \hld\ ok of \alst{h}ug mę́la; &
\ind \alst{g}lík skulu \alst{g}jǫld \alst{g}jǫfum.\eva

\bvb It is yet regarding the one whom thou trust badly, \\
\ind and whose intentions toward thee are suspect: \\
thou shalt laugh with him and speak with care; \\
\ind repayments shall be equal to gifts.\footnoteB{Equivalent to the last line of the previous st. (“pay duplicity against lie”).}\evb\evg


\bvg\bva Ungr vas’k \alst{f}orðum, \hld\ \alst{f}ór’k ęinn saman, &
\ind þȧ varð’k \alst{v}illr \alst{v}ega; &
\alst{au}ðigr þȯttumk, \hld\ es \alst{a}nnan fann’k, &
\ind \alst{m}aðr es \alst{m}anns gaman.\eva

\bvb Young was I once; I travelled alone; \\
\ind then I became lost of ways. \\
Wealthy I thought myself when another one I found; \\
\ind man is man’s pleasure.\evb\evg


\bvg\bva\alst{M}ildir frǿknir \hld\ \alst{m}ęnn batst lifa, &
\ind \alst{s}jaldan \alst{s}út ala; &
en \edtext{\alst{ȯ}-snjallr}{\linenum{|3||4}\lemma{ȯ-snjallr, gløggr ‘unvalorous, stingy’}\Bfootnote{Contrasting respectively with \emph{frǿkn, mildr} ‘brave, generous’ in the first half of the stanza; very fine parallelism.}} maðr \hld\ \alst{u}ggir hvat-vetna, &
\ind \edtext{sýtir ę́ \alst{g}løggr við \alst{g}jǫfum}{\lemma{sýtir \dots\ gjǫfum ‘the stingy man \dots\ gifts’}\Bfootnote{Cf. st. 39.  After receiving a gift, one was culturally obliged to give something back.}}.\eva

\bvb Generous, brave men live best: \\
\ind seldom they nourish sorrow— \\
but the unvalorous man is frightened by anything, \\
\ind the stingy always grieves over gifts.\evb\evg


\bvg\bva\alst{V}áðir mínar \hld\ gaf’k \alst{v}ęlli at &
\ind \alst{t}vęim \alst{t}ré-mǫnnum; &
\alst{r}ekkar þat þȯttusk, \hld\ es \alst{r}ipt hǫfðu; &
\ind \alst{n}ęiss es \alst{n}ǫkkviðr halr.\eva

\bvb My garments I gave, on the plain, \\
\ind to two tree-men. \\
Champions they seemed when cloaks they had; \\
\ind shameful is the naked hero.\footnoteB{One of the harder sts. in the poem.  The probable sense is that “the clothes make the man” (or warrior): under expensive gear a thin tree-man might be lurking, and likewise even a mighty man (the choice of the word \emph{halr} ‘hero, warrior’ (cf. sts. 36, 37) rather than the more neutral \emph{maðr} ‘man, person’ is surely intentional) can never defend himself against a heavily armoured opponent.  Without his arms, he becomes as vulnerable as the “tree-man” on the plain.}\evb\evg


\bvg\bva Hrørnar \alst{þ}ǫll, \hld\ sú’s stęndr \alst{þ}orpi ȧ, &
\ind hlýr-at hęnni \alst{b}ǫrkr né \alst{b}arr; &
svá es \alst{m}aðr, \hld\ sá’s \alst{m}ann-gi ann; &
\ind hvat skal hann \alst{l}ęngi \alst{l}ifa?\eva

\bvb Wilters the pine that stands on the yard; \\
\ind shields her not bark nor leaf. \\
So is the man who loves no man; \\
\ind why shall he live for long?\evb\evg


\bvg\bva\alst{Ę}ldi hęitari \hld\ brinnr með \alst{i}llum vinum &
\ind \alst{f}riðr \edtrans{\alst{f}imm daga}{for five days}{\Bfootnote{i.e. “for a week”, which was originally five days long.  See also st. 74 and the Encyclopedia: \inx[C]{five days}.}}, &
en þȧ \alst{sl}oknar, \hld\ es hinn \alst{s}étti kømr, &
\ind ok \alst{v}ersnar allr \alst{v}in-skapr.\eva

\bvb Hotter than fire burns love among bad friends, \\
\ind for \inx[C]{five days}; \\
but then goes out when the sixth one comes, \\
\ind and all the friendship worsens.\evb\evg


\bvg\bva\alst{M}ikit ęitt \hld\ skal-a \alst{m}anni gefa; &
\ind opt kaupir sér í \alst{l}ítlu \alst{l}of, &
með \alst{h}ǫlfum \alst{h}lęif \hld\ ok með \alst{h}ǫllu kęri &
\ind \alst{f}ekk ek mér \alst{f}é-laga.\eva

\bvb Much at once shall one not give a man; \\
\ind oft one buys oneself praise for little. \\
With half a loaf and an awry cask \\
\ind I got myself a partner.\evb\evg


\bvg\bva\edtrans{\alst{L}ítilla sanda, \hld\ \alst{l}ítilla sę́va}{Of small sands, of small seas}{\Bfootnote{Probably a partitive genitive, the sense being that man’s “horizons” are small; the universe will always be far greater than him.}}, &
\ind lítil eru \alst{g}ęð \alst{g}uma; &
\edtext{því-at \alst{a}llir męnn \hld\ \alst{u}rðu-t jafn-spakir; &
\ind \alst{h}ǫlf es ǫld \alst{h}var.}{\lemma{því-at \dots\ ǫld hvar. ‘For \dots\ every man.’}\Bfootnote{On the meaning of the second half of this stanza I find the view of \textcite{Athugasemdir1929} most convincing; namely that every man has both strengths and weaknesses in terms of wisdom.  As nobody can excel at everything, nobody is complete; every person is “half” (and it should be added that ON \emph{halfr} has a more general sense of incompleteness than its English cognate).  This interpretation fits particularly closely with sts. 71 and 132. — This stanza introduces several stanzas dealing with wisdom and foolishness.}}\eva

\bvb Of small sands, of small seas: \\
\ind small are the senses of man. \\
For all have not become evenly knowing; \\
\ind half is every man.\evb\evg


\bvg\bva\alst{M}eðal-snotr \hld\ skyli \alst{m}anna hvęrr, &
\ind ę́va til \alst{s}notr \alst{s}éi; &
þęim es \alst{f}yrða \hld\ \alst{f}ęgrst at lifa, &
\ind es \alst{v}ęl mart \alst{v}itu.\eva

\bvb Middle-clever should each man be; \\
\ind never too clever. \\
For those men it is fairest to live, \\
\ind who know well enough.\evb\evg


\bvg\bva\alst{M}eðal-snotr \hld\ skyli \alst{m}anna hvęrr, &
\ind ę́va til \alst{s}notr \alst{s}éi; &
\alst{s}notrs manns hjarta \hld\ verðr \alst{s}jaldan glatt, &
\ind ef sá ’s \alst{a}l-snotr es \alst{á}.\eva

\bvb Middle-clever should each man be; \\
\ind never too clever. \\
The clever man’s heart is seldom glad, \\
\ind if its owner is all-clever.\evb\evg


\bvg\bva\alst{M}eðal-snotr \hld\ skyli \alst{m}anna hvęrr, &
\ind ę́va til \alst{s}notr \alst{s}éi; &
\alst{ø}r·lǫg sín \hld\ viti \alst{ę}ngi maðr fyrir; &
\ind \edtrans{þęim es \alst{s}orga-lausastr \alst{s}efi.}{his is the most sorrowless mind.}{\Bfootnote{i.e. he who is ignorant of his fate.  It is surely fitting that Weden should say this, having knowledge of the inevitable destruction of the world and himself (see \inx[L]{Rakes of the Reins}).}}\eva

\bvb Middle-clever should each man be; \\
\ind never too clever. \\
His own \inx[C]{orlay} ought no man to know ahead; \\
\ind his is the most sorrowless mind.\evb\evg


\bvg\bva\alst{B}randr af \alst{b}randi \hld\ \alst{b}rinnr unds \alst{b}runninn es, &
\ind \alst{f}uni kvęykisk af \alst{f}una; &
\alst{m}aðr af \alst{m}anni \hld\ verðr at \alst{m}áli kuðr; &
\ind en til \edtrans{\alst{d}ǿlskr}{hickish}{\Bfootnote{Derived from an ablaut variant of \emph{dalr} ‘valley, dale’ + \emph{-iskr} ‘-ish’, the sense being ‘provincial, not having left his (home) valley’.  Cf. the Icelandic tribal names like \emph{vatns-dǿlir} and \emph{lang-dǿlir} ‘inhabitants of \emph{Vatns-dalr} (Waterdale), \emph{Lang-dalr} (Longdale)’.}} af \alst{d}ul.\eva

\bvb Fire by fire burns until it is burned [out]; \\
\ind flame is quickened by flame. \\
Man by man becomes known through speech, \\
\ind but the too hickish from his folly.\evb\evg


\bvg\bva\alst{Á}r skal rísa, \hld\ sá’s \alst{a}nnars vill &
\ind \alst{f}é eða \alst{f}jǫr hafa; &
sjaldan \alst{l}iggjandi ulfr \hld\ \alst{l}ę́r of getr, &
\ind né \alst{s}ofandi maðr \alst{s}igr.\eva

\bvb Early shall he rise who another man’s \\
\ind \inx[C]{fee} or life will have. \\
Seldom gets the lying wolf the thigh, \\
\ind nor the sleeping man victory.\evb\evg


\bvg\bva\alst{Á}r skal rísa, \hld\ sá’s á \alst{y}rkjęndr fáa, &
\ind ok ganga síns \alst{v}erka ȧ \alst{v}it; &
\alst{m}art of dvęlr \hld\ þann’s umb \alst{m}orgin sefr, &
\ind \edtrans{\alst{h}alfr es auðr und \alst{h}vǫtum}{the brisk has half the wealth}{\Bfootnote{i.e. the brisk man has already claimed half of a fortune by simply choosing to wake up early.}}.\eva

\bvb Early shall he rise who has workmen few, \\
\ind and go his work to meet. \\
Much is kept back from him who in the morning sleeps; \\
\ind the brisk has half the wealth.\evb\evg


\bvg\bva\alst{Þ}urra skíða \hld\ ok \alst{þ}akinna nę́fra, &
\ind þess kann \alst{m}aðr \alst{m}jǫt, &
ok þess \alst{v}iðar, \hld\ es \alst{v}innask męgi &
\ind \edtrans{\alst{m}ál ok \alst{m}issęri}{for a season and half-year}{\Bfootnote{Over nine months.}}.\eva

\bvb Of dry billets and thatching birch bark— \\
\ind of \emph{this} man knows the measure— \\
and of that firewood which he may use \\
\ind for a season and half-year.\evb\evg


\bvg\bva\edtrans{\alst{Þ}vęginn ok męttr}{Washed and full}{\Bfootnote{A formulaic collocation.  Cf. \Reginsmal\ 25 (\emph{kęmbðr} ‘combed’ — \emph{þvęginn} ‘washed’ — \emph{męttr} ‘full’) and \Voluspa\ 33: (\emph{þó} ‘washed’ — \emph{kęmbði} ‘combed’).  These examples attest to the importance of personal hygiene in the culture, something further seen by the ubiquity of combs in pre-Christian graves (TODO: archeological reference).
The whole thing reminds of the passage from \emph{Germania} ch. 22: \emph{Statim ē somnō, quem plērumque in diem extrahunt, lavantur, saepius calidā, ut apud quōs plūrimum hiems occupat.  Lautī cibum capiunt: sēparātae singulīs sēdēs et sua cuique mēnsa.  Tum ad negōtia nec minus saepe ad convīvia prōcēdunt armātī.} ‘On waking from sleep, which they generally prolong to a late hour of the day, they take a bath, oftenest of warm water, which suits a country where winter is the longest of the seasons.  After their bath they take their meal, each having a separate seat and table of his own.  Then they go armed to business, or no less often to their festal meetings (\emph{convivia}, i.e., their Things).’}} \hld\ ríði maðr \alst{þ}ingi at, &
\ind þótt sé-t \alst{v}ę́ddr til \alst{v}ęl; &
\alst{sk}úa ok bróka \hld\ \alst{sk}ammisk ęngi maðr &
\ind né \alst{h}ęsts in \alst{h}ęldr, &
\ind \edtrans{þótt hann \alst{h}afi-t góðan}{although he has not a good one}{\Bfootnote{\textcite{FinnurEdda} considers this a late insert, and I agree.  It seems that the inserter was not aware of the rules of the \Ljodahattr\ meter and interpreted the preceding c-verse (\emph{né hęsts in hęldr}) as an a-verse of \Fornyrdislag.}}.\eva

\bvb Washed and full ought a man to ride to the \inx[C]{Thing}, \\
\ind although he be not clothed too well; \\
of his shoes and breeches ought no man to be ashamed, \\
\ind nor the more of his horse, \\
\ind even though he haven’t a good one.\evb\evg

\sectionline

{\small The two following sts. are written in opposite order in \Regius, but a symbol at the start of each indicates that they should switch places.}

\sectionline

\bvg\bva\alst{S}napir ok gnapir, \hld\ es til \alst{s}ę́var kømr, &
\ind \alst{ǫ}rn ȧ \alst{a}ldinn mar; &
svá es \alst{m}aðr, \hld\ es með \alst{m}ǫrgum kømr &
\ind ok \edtrans{á \alst{f}or-mę́lęndr \alst{f}áa}{has spokesmen few}{\Bfootnote{Shared with st. 25.}}.\eva

\bvb He snaps and stoops when to the sea he comes, \\
\ind the eagle on the aged ocean. \\
So is the man who among the many comes, \\
\ind and has spokesmen few.\evb\evg


\bvg\bva\alst{F}regna ok sęgja \hld\ skal \alst{f}róðra hvęrr, &
\ind sá’s vill \alst{h}ęitinn \alst{h}orskr; &
\alst{ęi}nn vita \hld\ né \alst{a}nnarr skal, &
\ind \edtrans{\alst{þ}jóð}{thirty}{\Bfootnote{Or “people, nation”; the sense is in any case “many, everybody”.  For the translation “thirty” cf. \Skaldskaparmal\ 82, a list of poetic expressions for various numerals: \emph{\emph{þjóð} eru þrír tigir} ‘a \emph{nation} is thirty’ etc.}} vęit ef \alst{þ}rír ’ru.\eva

\bvb Ask and answer shall each learned man \\
\ind who wishes to be called sharp. \\
\emph{One} shall know, another shall not; \\
\ind thirty know if there are three.\evb\evg


\bvg\bva\alst{R}íki sitt \hld\ skyli \alst{r}áð-snotra &
\ind hvęrr í \alst{h}ófi \alst{h}afa; &
\edtext{þȧ þat \alst{f}innr, \hld\ es með \alst{f}rǿknum kømr, &
\ind at \alst{ę}ngi es \alst{ęi}nna hvatastr.}{\lemma{þȧ \dots\ ęinna hvatastr ‘then \dots briskest of all’}\Bfootnote{Almost identical to \Reginsmal\ TODO/3–4, which however has \emph{flęirum} ‘more men’ instead of \emph{frǿknum} ‘the bold’.}}\eva

\bvb His own power should each counsel-clever \\
\ind man use in moderation. \\
This he then finds when among the bold he comes— \\
\ind that none is the briskest of all.\footnoteB{i.e., every man has his match.}\evb\evg


\bvg\bva\alst{O}rða þęira, \hld\ es maðr \alst{ǫ}ðrum sęgir, &
\ind opt hann \alst{g}jǫld of \alst{g}etr.\eva

\bvb For those words which man says to another \\
\ind he oft gets recompense.\evb\evg


\bvg\bva \edtrans{\alst{M}ikils til}{Much too}{\Afootnote{written as one word \emph{mikilsti} \Regius}} snimma \hld\ kom’k í \alst{m}arga staði, &
\ind en til \alst{s}íð í \alst{s}uma; &
\alst{ǫ}l vas drukkit, \hld\ sumt vas \alst{ȯ}-lagat; &
\ind sjaldan hittir \alst{l}ęiðr í \alst{l}ið.\eva

\bvb Much too early I came to many places, \\
\ind and too late to some: \\
The ale was drunk up, some was unbrewed— \\
\ind seldom finds the loathed his place.\footnoteB{i.e., “there are no wrong times, only wrong people”.}\evb\evg


\bvg\bva\alst{H}ér ok \alst{h}var \hld\ myndi mér \alst{h}ęim of boðit, &
\ind ef þyrpta’k at \alst{m}ǫ́lun-gi \alst{m}at, &
eða \alst{t}vau lę́r hęngi \hld\ at hins \alst{t}ryggva vinar, &
\ind þar’s ek hafða \alst{ęi}tt \alst{e}tit.\eva

\bvb Here and there would I to a home be invited, \\
\ind if at meal-time I needed no food; \\
or if two hams should hang at the trusty friend’s [home], \\
\ind where I had eaten one.\footnoteB{Not everyone is hospitable, especially with regards to food, which was scarce and closely watched among the Norse subsistence farmers.  The poet notes that even a “trusty friend” (possibly sarcastic) would invite him over more often if he brought more food than he ate.}\evb\evg


\bvg\bva\alst{Ę}ldr es batstr \hld\ með \alst{ý}ta sonum &
\ind ok \alst{s}ólar \alst{s}ýn, &
\alst{h}ęilyndi sitt, \hld\ ef maðr \alst{h}afa náir, &
\ind án við \alst{l}ǫst at \alst{l}ifa.\eva

\bvb Fire is best among the sons of men, \\
\ind and the sight of the sun; \\
one’s good health, if he manage to keep it— \\
\ind {[and]} living free from vice.\evb\evg


\bvg\bva\alst{E}s-at maðr \alst{a}lls \edtrans{ve-sall}{unblessed}{\Bfootnote{Or ‘woe-blessed’.  I have elsewhere translated this word as ‘wretched’, but have presently rendered it this way to show the etymological relationship.  The second element in this compound is \emph{sę́ll}, which lacks i-umlaut due to a shortening of the vowel before the umlaut became phonemic.  The ancestral Proto-Norse forms would be \emph{*sāliʀ} and \emph{*wajē-sāliʀ}.  Cf. ᚹᚨᛃᛖ-ᛗᚨᚱᛁᛉ \emph{wajē-mariʀ} ‘infamous’ on the Tjurkö bracteate, where the second element is the ancestor of ON \emph{mę́rr} ‘renowned, famous’; the expected descendant \emph{*ve-marr} is not attested.
I have chosen to translate \emph{sę́ll} as ‘blessed’, but it is not a past participle and could also be rendered as ‘lucky’ or ‘blissful’.  It carries a certain sense of innateness that is foreign to modern Western culture.  Thus a king whose land experiences bountiful harvests (\emph{ár}) is said to be \emph{ár-sę́ll} ‘blessed with harvests’, while one whose kingdom is at peace (\emph{friðr}) is said to be \emph{frið-sę́ll} ‘blessed with peace’.  In this worldview the state of the realm is not due to uncontrollable environmental or political factors, but rather arises from the very person of the king (TODO: Reference PCRN chapter).}}, \hld\ þótt sé \alst{i}lla hęill, &
\ind \alst{s}umr es af \edtext{\alst{s}onum}{\lemma{sonum \dots\ frę́ndum ‘sons \dots\ kinsmen’}\Bfootnote{Cf. st. 72 below, which stresses the importance of sons and kinsmen.}} \alst{s}ę́ll, &
sumr af \alst{f}rę́ndum, \hld\ sumr af \alst{f}é ǿrnu, &
\ind sumr af \alst{v}erkum \alst{v}ęl.\eva

\bvb Man is not all unblessed, though he of poor health be: \\
\ind someone is blessed with sons; \\
someone with kinsmen, someone with ample \inx[C]{fee}, \\
\ind someone with works done well.\evb\evg


\bvg\bva Bętra ’s \alst{l}ifðum, \hld\ \edtrans{an séi ȯ-\alst{l}ifðum}{than with the unliving}{\Afootnote{emend.; \emph{⁊ ſęl lıfðo}m \Regius.}\Bfootnote{The reading of \Regius, which would be normalized as \emph{ok sę́l-lifðum} ‘and for the blessed living’, is metrically defect since \emph{sę́l-} is strongly stressed and should carry alliteration.  For the original form of the line we may instead cf. \Fafnismal\ 30: \emph{Hvǫtum ’s bętra \hld\ an sé ȯ-hvǫtum} ‘It is better for the brisk than it may be for the unbrisk’.  The corruption has probably happened in the following way: \emph{*en} (younger form of \emph{an} ‘than’) in the prototype was misinterpreted as \emph{en} ‘and, but’ and copied as \emph{⁊} (the tironian \emph{et}), while \emph{*séı ólıfðo}m (probably with the words cramped together) became \emph{sęl lıfðo}m.}}, &
\ind \edtrans{ęy getr \alst{k}vikr \alst{k}ú}{always gets the quick a cow}{\Bfootnote{i.e., “new opportunities always present themselves for the living”.  A reference to the cattle-based economy (see also st. 76), the cow being used as a metonym:  (cf. churchly English ‘the \emph{quick} and the dead’, i.e. ‘the \emph{living} and the dead’).}}; &
\alst{ę}ld sá’k \alst{u}pp brinna \hld\ \alst{au}ðgum manni fyr, &
\ind en úti vas \alst{d}auðr fyr \alst{d}urum.\eva

\bvb It is better for the living than it may be for the unliving: \\
\ind ever the quick gets the cow. \\
A fire I saw burning high for a wealthy man, \\
\ind but outside he was dead before the doors.\footnoteB{The fire is presumably the man’s funeral pyre, on which a considerable amount of his wealth has been spent; according to ibn Fadlan (TODO) two thirds of a dead chieftain’s estate was spent on his funeral.  One notes the contrastive \emph{en} ‘but’ and may understand it as follows: “I saw a lavish funeral held for a man, but he was still dead.”  This interpretation is supported by the \Havamal\ 71 below, which expresses the same sentiment.}\evb\evg


\bvg\bva\alst{H}altr ríðr \alst{h}rossi, \hld\ \alst{h}jǫrð rekr \alst{h}andar vanr, &
\ind \alst{d}aufr vegr ok \alst{d}ugir; &
\alst{b}lindr es \alst{b}ętri, \hld\ an \alst{b}ręnndr séi; &
\ind \alst{n}ýtr mann-gi \alst{n}ás.\eva

\bvb A halt man rides a horse; a handless drives a herd; \\
\ind a deaf fights and avails. \\
Blind is better than be burned; \\
\ind no man has use for a corpse.\evb\evg


\bvg\bva \edtrans{\alst{S}onr es bętri}{A son is better}{\Bfootnote{i.e. it is better for a man to have a son and heir than not, even if the father should die some time before he is born. The son can further his father’s lineage and memory (as exemplified by the raising of a “beat-stone”), and as the poet says, it is rare for a non-relative to do so.}}, \hld\ þótt sé \alst{s}íð of alinn &
\ind ęptir \alst{g}inginn \alst{g}uma; &
sjaldan \edtrans{\alst{b}autar-stęinar}{beat-stones}{\Bfootnote{Large standing stones raised in memory of someone.  Numerous such stones with runic inscriptions are known from migration period Norway, often near grave fields.  Some hold only single personal names or short phrases, like the stone from Sunde in Sunnfjord, western Norway (signum \emph{KJ 90}): ᚹᛁᛞᚢᚷᚨᛊᛏᛁᛉ \textbf{widugastiʀ} ‘Woodguest’, or the one from Bø in Rogaland, southwestern Norway (signum \emph{KJ 78}): ᚺᚾᚨᛒᛞᚨᛊ ᚺᛚᚨᛁᚹᚨ \textbf{hnabdas hlaiwa} ‘Naved’s grave’.  Others hold longer inscriptions, like the one from Kjølevik in Rogaland (signum \emph{KJ 75}): ᚺᚨᛞᚢᛚᚨᛁᚲᚨᛉ ᛖᚲᚺᚨᚷᚢᛊᛏᚨᛞᚨᛉ ᚺᛚᚨᚨᛁᚹᛁᛞᛟᛗᚨᚷᚢᛗᛁᚾᛁᚾᛟ \textbf{hadulaikaz ekhagustadaz hlaaiwidomaguminino} ‘Hathlac [lies here].  I, Haystald, buried my lad.’}} \hld\ standa \alst{b}rautu nę́r, &
\ind nema ręisi \alst{n}iðr at \alst{n}ið.\eva

\bvb A son is better, though he late be born \\
\ind after a passed-on man. \\
Seldom beat-stones stand near the road, \\
\ind save by kinsman for kinsman raised.\evb\evg


\bvg\bva \edtext{\edtrans{\alst{T}vęir ’ru ęins hęrjar}{Two are of one host}{\Bfootnote{i.e. “the tongue and head belong to the same body (but the former often leads to the latter’s demise).” — \emph{hęrjar} is an inflected form of \emph{hęrr} ‘host, army’, but its function is ambiguous; it can either be (1) the gen. sg., as adopted here, or (2) the nom. pl. ‘harriers, raiders’ (cf. \emph{ęin-hęrjar} ‘\inx[G]{Oneharriers}’) which would translate as “two are the destroyers of one”, i.e. “the tongue and head often lead to the demise of the body”.}}, \hld\ \edtrans{\alst{t}unga es hǫfuðs bani}{the tongue is the head’s bane}{\Bfootnote{Formulaic or proverbial.  Cf. the Old Swedish “Heathen Law”, which describes how a duel should be conducted following an insult to a man’s honour (my norm. and trans. following \textcite{Läffler1879}): \emph{Fallr þann orð havr givit—glǿpr orða vęrstr,} tunga hovuð-bani—\emph{liggi i ú·gildum akri} ‘If he falls who has given the [insulting] word—an insult is the worst of words, \emph{the tongue the head-bane}—may he lie in an unhallowed field.’}}; &
mér ’s í \alst{h}eðin \alst{h}vęrn \hld\ \edtrans{\alst{h}andar}{a hand}{\Bfootnote{i.e. a hand holding a dagger.}} vę́ni.}{\lemma{ALL}\Bfootnote{The whole st. fits poorly in context, and the metre and style are very out of place; it is probably a later insert.}}\eva

\bvb Two are of one host: the tongue is the head’s bane; \\
in every cloak I expect a hand.\evb\evg


\bvg\bva\alst{N}ǫ́tt verðr fęginn, \hld\ sá’s \alst{n}esti trúir, &
\ind \edtrans{\alst{sk}ammar ’ru \alst{sk}ips ráar}{short are a ship’s sailyards}{\Bfootnote{TODO: Write about the varying interpretations (Finnur, Cleasby, Skp) of this line.}}, &
\ind \alst{h}verf es \alst{h}aust-gríma; &
\alst{f}jǫlð of viðrir \hld\ ȧ \edtrans{\alst{f}imm dǫgum}{five days}{\Bfootnote{i.e. “in a week” (which was originally five days long), paralleling “month” in the next line.  See note to st. 51 and Encyclopedia.}}, &
\ind en \alst{m}ęir ȧ \alst{m}ánaði.\eva

\bvb At night he rejoices, who trusts in his provisions; \\
\ind short are a ship’s sailyards; \\
\ind shifty is a stormy fall night. \\
The weather changes much in \inx[C]{five days}; \\
\ind even more in a month.\evb\evg


\bvg\bva\alst{V}ęit-a hinn, \hld\ es \alst{v}ę́tki \alst{v}ęit, &
\ind \edtrans{margr verðr \edtrans{af \alst{au}rum}{from wealth}{\Afootnote{emend. from meaningless \emph{†aflꜹðrom†} \Regius}} \alst{a}pi}{many a man turns an ape from wealth}{\Bfootnote{Cf. \Solarljod\ 34/4: \emph{margan hefr auðr apat} ‘wealth has aped many a man’, which also lends support to the emendation.}}; &
maðr es \alst{au}ðigr, \hld\ annarr \alst{ȯ}-auðigr, &
\ind skyli-t þann \alst{v}ítka \alst{v}áar.\eva

\bvb The one knows not, who nothing knows: \\
\ind many a man turns an \inx[C]{ape} from wealth. \\
A man is wealthy, another not wealthy; \\
\ind one oughtn’t to curse him for his woe.\evb\evg


\bvg\bva\alst{D}ęyr \edtext{fé}{\lemma{fé \dots\ frę́ndr ‘Fee \dots\ kinsmen’}\Bfootnote{The import of this merism may be less clear to the modern reader. In the Germanic Iron Age farming society a man’s wealth was reckoned by how many heads of cattle (and the Norman loan-word \emph{cattle} is itself the same word as \emph{capital}) he owned (cf. st. 70 above, where “a cow” is used to express “an opportunity”), and his social power by the number of able male relatives ready to side with him in conflict (cf. st. 72 above and TODO: reference?). The meaning is thus: all your power will pass away, and so too must you, but if you leave a good reputation behind it can live on. For Indo-European poetic analogues, see \textcite[99\psqq]{West2007}.}}, \hld\ \alst{d}ęyja frę́ndr, &
\ind dęyr \alst{s}jalfr hit \alst{s}ama; &
en \alst{o}rðs-tírr \hld\ dęyr \alst{a}ldri-gi &
\ind hvęim’s sér \alst{g}óðan \alst{g}etr.\eva

\bvb \inx[C]{fee}[Fee] dies, kinsmen die, \\
\ind oneself dies the same [way]; \\
but a word-glory never dies, \\
\ind for whomever gets himself a good one.\evb\evg


\bvg\bva\alst{D}ęyr fé, \hld\ \alst{d}ęyja frę́ndr, &
\ind dęyr \alst{s}jalfr hit \alst{s}ama; &
\alst{e}k vęit \alst{ęi}nn \hld\ at \alst{a}ldri-gi dęyr: &
\ind \alst{d}ómr of \alst{d}auðan hvęrn.\eva

\bvb Fee dies, kinsmen die, \\
\ind oneself dies the same [way]. \\
I know one that never dies: \\
\ind the \inx[C]{Doom} o’er each man dead.\evb\evg

\sectionline

{\small It is likely that the original Guest-Strand ended here.  The three following stanzas, especially the third, are poorly placed and seem like later inserts.}

\sectionline

\bvg\bva\alst{F}ullar grindr \hld\ sá’k fyr \alst{F}itjungs sonum, &
\ind nú bera þęir \edtrans{\alst{v}ánar \alst{v}ǫl}{the staff of hope}{\Bfootnote{A beggar’s staff.}}; &
svá es \alst{au}ðr \hld\ sęm \alst{au}ga-bragð, &
\ind hann es \alst{v}altastr \alst{v}ina.\eva

\bvb Full pens I saw for the sons of Fitting; \\
\ind now they carry the staff of hope. \\
So is wealth like the twinkling of an eye: \\
\ind it is the ficklest of friends.\evb\evg


\bvg\bva\alst{Ȯ}-snotr maðr \hld\ es \alst{ęi}gnask getr &
\ind \alst{f}é eða \alst{f}ljóðs mun-úð; &
\alst{m}etnaðr hǫ́num þróask, \hld\ en \alst{m}an-vit aldri-gi; &
\ind framm gęngr hann \alst{d}rjúgt í \alst{d}ul.\eva

\bvb The unclever man who comes to own \\
\ind fee or a girl’s loving grace: \\
his pride flourishes, but never his manwit; \\
\ind he goes forth far in folly.\evb\evg


\bvg\bva Þat ’s þȧ \alst{r}ęynt, es þú at \edtext{\alst{r}únum spyrr, \hld\ hinum \alst{r}ęgin-kunnum}{\lemma{rúnum \dots\ ręgin-kunnum ‘runes \dots\ born of the Reins’}\Bfootnote{This expression also appears on the C4th–6th Noleby stone (in the acc. sg. \emph{rúnó ragina-kundó} ‘a rune born of the Reins’), which proves that the Eddic rune-magic is (at least in part) founded in oral tradition going back to the Heathen age. See also Encyclopedia \inx[C]{rune}.}}, &
\ind \edtext{þęim’s \alst{g}ørðu \alst{g}inn-ręgin &
\ind ok \alst{f}áði \alst{F}imbul-þulr;}{\lemma{þęim’s \dots\ Fimbul-þulr ‘those which \dots\ Fimble-Thyle’}\Bfootnote{Formulaic. Cf. st. 142 where these two lines occur almost identically, but in reverse order.}} &
\ind \alst{þ}ȧ hęfr hann batst, ef hann \alst{þ}ęgir.\eva

\bvb That is then proven, which thou learnest from the runes, those born of the Reins, \\
\ind those which the \inx[G]{yin-Reins} made, \\
\ind and the Fimble-Thyle \name{= Weden} painted.— \\
\ind Then he has it best, if he shuts up.\footnoteB{This stanza, which deals with runic magic and shares expressions with sts. in the Rune-Tally section (beginning with st. 138 below), hardly fits in its current place.  The last line with its shift in person is likely to be a later insert.}\evb\evg

\sectionline

\section{Scattered stanzas of practical advice (81–90)}

{\small The following stanzas are distinguished by a prevalence of \Malahattr\ and the common subject matter.}

\sectionline

\bvg\bva%
At \alst{k}veldi skal dag lęyfa, \hld\ \alst{k}onu es bręnnd es, &
\alst{m}ę́ki es ręyndr es, \hld\ \alst{m}ęy es gefin es, &
\alst{í}s es \alst{y}fir kømr, \hld\ \alst{ǫ}l es drukkit es.\eva

\bvb At evening shall one praise day, a woman when she is burned, \\
a sword when it is tried, a maiden when she is given,\footnoteB{i.e. in marriage.} \\
ice when one crosses over, ale when it is drunk.\evb\evg


\bvg\bva%
Í \alst{v}indi skal \alst{v}ið hǫggva, \hld\ \edtrans{\alst{v}eðri}{weather}{\Bfootnote{i.e. ‘in good weather’; elsewhere the word \emph{veðr} typically means ‘storm’, but that can hardly be the sense here.}} ȧ sę́ róa, &
\alst{m}yrkri við \alst{m}an spjalla— \hld\ \alst{m}ǫrg eru dags augu— &
ȧ \alst{sk}ip skal \alst{sk}riðar orka, \hld\ en ȧ \alst{sk}jǫld til hlífar, &
\alst{m}ę́ki til hǫggs, \hld\ en \alst{m}ęy til kossa.\eva

\bvb In wind shall one cut wood, in weather row at sea, \\
in darkness speak with a maiden—many are the eyes of day. \\
A ship shall one have for speed, and a shield for protection; \\
a sword for striking, and a maiden for kisses.\evb\evg


\bvg\bva%
Við \alst{ę}ld skal \alst{ǫ}l drekka, \hld\ en ȧ \alst{í}si skríða, &
\alst{m}agran \edtext{\alst{m}ar kaupa, \hld\ en \alst{m}ę́ki}{\lemma{mar \dots\ mę́ki ‘steed \dots\ sword’}\Bfootnote{Formulaic pair, also occurring in \Lokasenna\ 12/1, \Volundarkvida\ 33/3, \Atlakvida\ 7/3.}} saurgan, &
\alst{h}ęima \alst{h}ęst fęita, \hld\ en \alst{h}und ȧ búi.\eva

\bvb One shall drink ale by fire and skate on ice; \\
buy a starved steed and a rusty sword; \\
fatten the horse at home and the hound in its dwelling.\evb\evg


\bvg\bva%
\alst{M}ęyjar orðum \hld\ skyli \alst{m}ann-gi trúa, &
\ind né því’s \alst{k}veðr \alst{k}ona; &
\edtext{\edtext{því-at}{\Afootnote{om. \FostrbroedhraSaga}} ȧ \alst{h}verfanda \alst{h}véli \hld\ \edtext{vǫ́ru}{\Afootnote{\emph{er} \FostrbroedhraSaga}} þęim \edtrans{\alst{h}jǫrtu skǫpuð}{hearts shaped}{\Afootnote{\emph{hjarta skapat} ‘heart shaped’ \FostrbroedhraSaga}}, &
\ind \edtext{\alst{b}rigð}{\Afootnote{ok brigð \FostrbroedhraSaga}} í \alst{b}rjóst of \edtext{lagit}{\Afootnote{\emph{laginn} \FostrbroedhraSaga}}.}{\lemma{þvít \dots\ lagið}\Bfootnote{Quoted in slightly divergent form in \FostrbroedhraSaga\ (Thott 1768 4°\textsuperscript{x}, fol. 210r) introduced with the words: \emph{Kom honum þá í hug kviðlingr sá, er kveðinn hafði verit um lausungar-konur:} ‘And then he remembered the ditty which had been composed about loose women:’}}\eva

\bvb A maiden’s words should no man trust, \\
\ind nor that which a woman speaks. \\
For on a whirling wheel their hearts were shaped; \\
\ind fickleness laid in their breasts.\evb\evg


\bvg\bva%
\alst{B}restanda \alst{b}oga, \hld\ \alst{b}rinnanda loga, &
\alst{g}ínanda ulfi, \hld\ \alst{g}alandi krǫ́ku, &
\alst{r}ýtanda svíni, \hld\ \alst{r}ót-lausum viði, &
\alst{v}axanda \alst{v}ági, \hld\ \alst{v}ellanda katli,\eva

\bvb In bursting bow, in burning flame, \\
in yawning wolf, in crowing crow, \\
in roaring swine, in rootless tree, \\
in waxing wave, in boiling kettle,\evb\evg


\bvg\bva%
\alst{f}ljúganda \alst{f}lęini, \hld\ \alst{f}allandi bǫ́ru, &
\alst{í}si \alst{ęi}n-nę́ttum, \hld\ \alst{o}rmi hring-lęgnum, &
\alst{b}rúðar \alst{b}ęð-mǫ́lum \hld\ eða \alst{b}rotnu sverði, &
\alst{b}jarnar lęiki \hld\ eða \alst{b}arni konungs,\eva

\bvb in flying spear, in falling billow, \\
in one-night old ice, in coiled-up serpent, \\
in bride’s bed-speech, or in broken sword, \\
in bear’s play, or in king’s child,\evb\evg


\bvg\bva%
\alst{s}júkum kalfi, \hld\ \alst{s}jalf-ráða þrę́li, &
\edtrans{\alst{v}ǫlu \alst{v}il-mę́li}{in wallow’s pleasing speech}{\Bfootnote{i.e. in a favourable prophecy (\inx[C]{spae}).}}, \hld\ \alst{v}al ný-fęldum.\eva

\bvb in sick calf, in self-willing thrall, \\
in wallow’s pleasing speech, in newly felled corpses,\evb\evg

\sectionline

{\small In \Regius\ the following two sts. come in the opposite order, but it seems probable from its \Malahattr\ meter and the dative case of the words that 89 should follow 87.  On the other hand st. 88, with its \Ljodahattr\ meter and self-enclosed form seems a separate composition, and was probably inserted after 87 due to its first line (\emph{akri ár-sǫ́num}), which is also in the dative.}

\sectionline

\bvg\bva[89]%
\alst{b}róður-\alst{b}ana sínum \hld\ þótt ȧ \alst{b}rautu mǿti, &
\alst{h}úsi \alst{h}alf-brunnu, \hld\ \alst{h}ęsti al-skjótum, &
þȧ ’s \alst{jó}r \alst{ȯ}-nýtr, \hld\ ef \alst{ęi}nn fótr brotnar; &
verðr-it maðr svá \alst{t}ryggr \hld\ at þessu \alst{t}rúi ǫllu!\eva

\bvb in one’s brother’s bane—though on the road ye meet— \\
in half-burned house, in all-fleet horse— \\
the steed is useless if one foot breaks. \\
No man be so trusting that he trust in all this!\evb\evg\stepcounter{stanza}


\bvg\bva[88]%
\alst{A}kri \alst{á}r-sǫ́num \hld\ trúi \alst{ę}ngi maðr, &
\ind né til \alst{s}nimma \alst{s}yni; &
\alst{v}eðr rę́ðr akri, \hld\ en \alst{v}it syni; &
\ind \alst{h}ę́tt es þęira \alst{h}várt.\eva

\bvb In an early sown field ought no man to trust, \\
\ind nor too soon in a son. \\
The weather rules the field and the wits the son: \\
\ind there is risk to them both.\evb\evg\stepcounter{stanza}


\bvg\bva Svá ’s \alst{f}riðr kvinna \hld\ þęira’s \alst{f}látt hyggja, &
sęm \alst{a}ki \alst{jó} ȯ-bryddum \hld\ ȧ \alst{í}si hǫ́lum &
\alst{t}ęitum, \alst{t}vé-vetrum \hld\ ok sé \alst{t}amr illa, &
eða í \alst{b}yr óðum \hld\ \alst{b}ęiti stjórn-lausu, &
eða skyli \alst{h}altr \alst{h}ęnda \hld\ \alst{h}ręin \edtrans{í þá-fjalli}{on a thawing fell}{\Bfootnote{i.e. in springtime, when the melting ice on the ground is most slippery.}}.\eva

\bvb So is the love of those women who falsely think \\
like one rode an unshod horse on slippery ice— \\
a merry one, two winters old, and ill-tamed— \\
or in mad wind tacked a rudderless [ship], \\
or a halt man should catch a reindeer on a thawing fell.\evb\evg

\sectionline

\section{Weden’s failed seduction of Billing’s daughter (91–102)}

The following sts. are united by their meter, \Ljodahattr\ (unlike most of the preceding sts., see introduction to them above), style and content.  The strand begins with general maxims about love and relations between the sexes, before moving on to the narrative about Billing’s daughter.

\sectionline

\bvg\bva\alst{B}ęrt nú mę́li’k, \hld\ því-at \edtrans{\alst{b}ę́ði}{both}{\Bfootnote{i.e. both sides, both sexes.  The (male) poet declares that he will not attack the fair sex unfairly; he is also aware of men’s faults.}} vęit’k, &
\ind brigðr es \alst{k}arla hugr \alst{k}onum, &
\edtext{þȧ \alst{f}ęgrst mę́lum, \hld\ es \alst{f}lást hyggjum}{\lemma{fęgrst mę́lum \dots\ flást hyggjum ‘speak fairest \dots\ think falsest’}\Bfootnote{Formulaic.  Cf. st. 45.}}; &
\ind \edtrans{þat tę́lir \alst{h}orska \alst{h}ugi}{that entraps sharp minds}{\Bfootnote{i.e., love (or sexual infatuation—the poet does not distinguish between them) turns even wise men into liars or otherwise dishonest persons.  Cf. \Malshattakvadi\ 20/1–2, which is probably partly based on this stanza:
\emph{Ást-blindir ’ru seggir svá \hld\ sumir, at þykkja mjǫk fás gá;} \\
\emph{þannig verðr um man-sǫng mę́lt: \hld\ marga hefr þat hyggna tę́lt.}
‘Some men are so love-blind, that they seem to heed very little; // for that sake it is said about love-song: many thinking men has it entrapped.’}}.\eva

\bvb Plainly I now speak, for I know both: \\
\ind fickle is men’s thought towards women. \\
We then speak fairest when we think falsest; \\
\ind that entraps sharp minds.\evb\evg


\bvg\bva \edtrans{\alst{F}agrt skal mę́la}{Fairly shall speak}{\Bfootnote{Formulaic. Cf. st. 45.}} \hld\ ok \alst{f}é bjóða, &
\ind sá’s vill \alst{f}ljóðs ǫ́st \alst{f}ȧa, &
\alst{l}íki \alst{l}ęyfa \hld\ hins \alst{l}jósa mans, &
\ind \edtrans{sá \alst{f}ę̇r, es \alst{f}ríar}{he gets, who woos}{\Bfootnote{i.e., “he who courts her gets her”.}}.\eva

\bvb Fairly shall speak, and offer \inx[C]{fee}, \\
\ind he who will get a woman’s love; \\
praise the body of the bright girl; \\
\ind he gets, who woos.\evb\evg


\bvg\bva\alst{Á}star firna \hld\ skyli \alst{ę}ngi maðr &
\ind \alst{a}nnan \alst{a}ldri-gi; &
opt fȧa ȧ \alst{h}orskan, \hld\ es ȧ \alst{h}ęimskan né fȧa, &
\ind \edtrans{\alst{l}ost-fagrir \alst{l}itir}{lust-fair hues}{\Bfootnote{i.e. a (woman with a) countenance so beautiful that men cannot help but lust after her.}}.\eva

\bvb For [matters of] love should no man \\
\ind ever blame another; \\
oft they seize the sharp when they seize not the foolish, \\
\ind the lust-fair hues.\evb\evg


\bvg\bva\alst{Ęy}-vitar firna, \hld\ es maðr \alst{a}nnan skal, &
\ind þess es of margan \alst{g}ęngr \alst{g}uma; &
\alst{h}ęimska ór \alst{h}orskum \hld\ gęrir \alst{h}ǫlða sonu &
\ind sá hinn \alst{m}átki \alst{m}unr.\eva

\bvb In no way shall man blame another \\
\ind for that which happens to many a man; \\
from sharp to fools are the sons of men made \\
\ind by that mighty thing, love.\evb\evg


\bvg\bva\alst{H}ugr ęinn þat vęit, \hld\ es býr \alst{h}jarta nę́r, &
\ind ęinn es hann \alst{s}ér of \alst{s}efa; &
øng es \alst{s}ótt verri \hld\ hvęim \alst{s}notrum manni &
\ind an sér \alst{ø}ngu at \alst{u}na.\eva

\bvb The mind alone knows what dwells close to the heart; \\
\ind it is alone with its thoughts. \\
No sickness is worse for any clever man \\
\ind than with nothing to be content.\evb\evg


\bvg\bva Þat þȧ \alst{r}ęynda’k, \hld\ es í \alst{r}ęyri sat’k, &
\ind ok vę̇tta’k \alst{m}íns \alst{m}unar, &
\alst{h}old ok \alst{h}jarta \hld\ vas mér hin \alst{h}orska mę́r, &
\ind þęygi hana at \alst{h}ęldr \alst{h}ęf’k.\eva

\bvb I experienced it then, as I sat in the reed, \\
\ind and awaited my love. \\
My flesh and heart was that sharp maiden— \\
\ind I have her none the more.\evb\evg


\bvg\bva\alst{B}illings \edtrans{męy}{maiden}{\Bfootnote{i.e. unmarried (virgin) daughter.}} \hld\ ek fann \alst{b}ęðjum ȧ &
\ind \alst{s}ól-hvíta \alst{s}ofa; &
\alst{ja}rls \alst{y}nði \hld\ þȯtti mér \alst{ę}kki vesa &
\ind nema við þat \alst{l}ík at \alst{l}ifa.\eva

\bvb Billing’s maiden I found on the beds, \\
\ind sun-white, asleep. \\
An earl’s pleasure seemed me naught to be, \\
\ind save living alongside that body.\evb\evg


\bvg\bva\speakernote{[Billings mę́r:]}„\alst{Au}k nę́r \alst{a}ptni \hld\ skalt \alst{Ó}ðinn koma, &
\ind ef vilt þér \alst{m}ę́la \alst{m}an, &
\alst{a}llt eru \alst{ȯ}-skǫp, \hld\ nema \alst{ęi}n vitim &
\ind \alst{s}likan lǫst \alst{s}aman.“\eva

\bvb\speakernoteb{[Billing’s maiden:]}
“And by evening shalt thou, Weden, come, \\
\ind if thou wilt get for thee the girl [me]; \\
everything’s misshapen unless we alone should know, \\
\ind such a vice together.”\evb\evg


\bvg\bva\alst{A}ptr ek hvarf \hld\ ok \alst{u}nna þȯttumk &
\ind \edtrans{\alst{v}ísum \alst{v}ilja frȧ}{away from my wise will}{\Bfootnote{i.e., “against my better judgment”; the wise choice would have been to walk away.}}; &
\alst{h}itt ek \alst{h}ugða, \hld\ at \alst{h}afa mynda’k &
\ind \alst{g}ęð hęnnar allt ok \alst{g}aman.\eva

\bvb Back I turned—and thought myself in love— \\
\ind away from my wise will; \\
\emph{this} I thought: that I would have \\
\ind her senses all, and pleasure.\evb\evg


\bvg\bva Svá kom’k \alst{n}ę́st, \hld\ at hin \edtrans{\alst{n}ýta}{useful}{\Bfootnote{Sarcastic. Billing’s daughter had apparently summoned a lynch mob.}} vas &
\ind \alst{v}íg-drótt ǫll of \alst{v}akin, &
með \alst{b}rinnǫndum ljósum \hld\ ok \edtrans{\alst{b}ornum viði}{carried sticks}{\Bfootnote{lit. ‘carried wood’; the mob was armed with clubs.}}, &
\ind svá vas mér \edtrans{\alst{v}íl-stígr}{sad path}{\Bfootnote{Ambiguous, referring either to the beating he would have received at the hands of the mob, or to his walk of shame away from the hall.  The latter is perhaps more likely.}} of \alst{v}itaðr.\eva

\bvb So I came next, as the useful \\
\ind war-troop was all awake; \\
with burning lights and with carried sticks; \\
\ind so a sad path was marked out for me.\evb\evg


\bvg\bva \edtrans{\alst{Au}k nę́r morni}{And by morning}{\Bfootnote{Mirroring the beginning of st. 97 above.}}, \hld\ es vas’k \alst{ę}nn of kominn, &
\ind þȧ vas \alst{s}al-drótt of \alst{s}ofin; &
\edtrans{\alst{g}ręy ęitt}{A lone bitch}{\Bfootnote{The insult is clearly understood; Weden is compared to a horny dog, and mockingly asked to make love to one—“this is all you get, you dog!”}} þȧ fann’k \hld\ hinnar \edtrans{\alst{g}óðu}{good}{\Bfootnote{Possibly not sarcastic, but rather referring to her chastity.}} konu &
\ind \alst{b}undit \alst{b}ęðjum ȧ.\eva

\bvb And by morning when I had come again, \\
\ind then was the hall-troop asleep. \\
A lone bitch I then found, by the good woman \\
\ind bound on the beds.\evb\evg


\bvg\bva Mǫrg es \edtrans{\alst{g}óð mę́r}{good maiden}{\Bfootnote{A formulaic expression; the “goodness” here refers to faithfulness and chastity.  Cf. \Skirnismal\ 12, TODO.}}, \hld\ ef \alst{g}ǫrva kannar, &
\ind \alst{h}ug-brigð við \alst{h}ali; &
þȧ þat \alst{r}ęynda’k, \hld\ es hit \alst{r}áð-spaka &
\ind tęygða’k ȧ \alst{f}lę́rðir \alst{f}ljóð; &
\alst{h}ǫ́ðungar \alst{h}vęrrar \hld\ lęitaði mér hit \alst{h}orska man &
\ind ok hafða’k þess \alst{v}ę́t-ki \alst{v}ífs.\eva

\bvb Many a good maiden—if one comes to know her well— \\
\ind is heart-fickle towards men. \\
I found that out when the counsel-clever \\
\ind lady into sins I lured: \\
all kinds of disgraces that sharp girl sought out for me, \\
\ind and I had naught of the woman.\evb\evg

\sectionline

\section{Weden’s theft of the Mead of Poetry (103–110)}

The intricate myth of how Weden came to own the Mead of Poetry is told more fully in \Skaldskaparmal\ 5–6. That narrative goes as follows, with minor details left out:
After the war between the Eese and Wanes, the two tribes of gods reconcile through spitting into a vat. Not wanting to discard this token of their truce, they instead create a man out of the spit, calling him \inx[P]{Quasher}; he is so wise that he can answer any question posed to him, and so travels around the world in order to share his wisdom with humans.
Quasher eventually comes to the dwelling of two dwarfs, Fealer and Galer. They kill him and drain his blood into three vessels: two vats named Soon and Bothem, and a kettle named \inx[P]{Woderearer}. Through mixing the blood with honey they make a mead, with the power to turn anyone who drinks from it “a scold or man of learning (\emph{skald eða frǿða-maðr})”. The dwarfs then lie to the Eese about the murder, telling them that Quasher drowned in his own wisdom.
Some time later, the dwarfs murder an ettin named \inx[P]{Gilling} and his wife. Gilling’s son, \inx[P]{Sutting}, learns of this and prepares to drown the dwarfs. In exchange for their lives and as recompense for his father’s slaying, the dwarfs offer Sutting the “dear mead” (\emph{mjǫðinn dýra}; cf. here sts. 105 and 140). Sutting accepts the ransom and takes the mead home with him. He makes his daughter \inx[P]{Guthlathe} guard it.
Some time later, Weden is out journeying, and finds nine thralls mowing hay. He sharpens their scythes with a special whetstone, and the mowing improves greatly. He then throws it in the air and the thralls shortly kill each other over it. By evening Weden comes to the owner of the thralls, Bigh, Sutting’s brother. Bigh laments the death of his workmen, and so Weden, who calls himself \inx[P]{Baleworker}, offers to do the work of the thralls over the summer, in exchange for one drink of Sutting’s mead. Bigh tells him that Sutting alone owns the mead, but that he will accompany Baleworker to Sutting to ask for the drink.
The two arrive at Sutting, who as expected refuses to give any part of the mead away. Baleworker then tells Bigh that he will get to it anyway; he takes out the drill \inx[P]{Rate}, and tells Bigh to drill through the mountain, into the room where the mead is stored. Bigh first attempts to trick him by only drilling halfway, but eventually creates a narrow passage. Baleworker turns himself into a snake and crawls through it; as he does, Bigh tries to strike him the drill, but misses.
After coming through, Baleworker sees Guthlathe watching over the mead. He goes on to sleep with her for three nights, after which she promises him three sips of the mead. With each sip he swallows the contents of one of the three vessels, so that all of the mead ends up in his belly.
Having taken the mead, he dons his eagle-hame and flies away from the mountain. Sutting sees him, takes his own eagle-hame, and gives chase. The Eese see Weden in flight, and set out several large vat on the ground, into which Weden, still flying, spits out the mead. At this point Sutting has almost caught up with him, and so Weden “sends back” (\emph{sęnda aptr}, usually interpreted being sent out from the anus) some of the mead, presumably into his face. This portion becomes the lot of foolish poets (\emph{skald-fífla hlutr}), while the rest of the mead is given to the Eese and to skilled poets (\emph{þęim mǫnnum, er yrkja kunnu} ‘those men who can compose [poetry]’).

The core of this many-twisted myth is old. A close parallel is found in \Rigveda\ hymns 4.26–27. In these two hymns the \emph{soma} plant (who in the Vedic mythology is not just the plant and its resulting drink, but also a god, perhaps somewhat like Quasher) is first held within “a hundred iron forts” (4.27.1c: \emph{śatám púraḥ ā́yasīḥ}) by the archer \emph{Kr̥şānu}, before being stolen by a sweeping falcon. The falcon brings \emph{Soma} to \emph{Manu}, the ancestor of the Aryans and first sacrificer.

The resemblance to the last part of the \Skaldskaparmal\ account should be obvious, but, notably, the detail of the falcon is not found in any of the sts. below. This shows that the narrative of \Skaldskaparmal\ cannot be exclusively based on the sts. here below, but instead also relies on other, now-lost sources. This is also supported by the present sts. leaving out the narratives about Quasher, the two dwarfs, and Baye, along with some subtler narrative differences.

The order of the present sts. follows that of \Regius, their main witness manuscript. The strand begins with some social advice (103), after which the narrative follows (104–110). It is narrated in the first person by Weden himself. The sts. do not tell the myth in chronological order and leave much up to the listener; they are surely composed for an audience that already knows the story. The following narrative details are given:

\begin{enumerate}
	\setcounter{enumi}{103}
	\item Weden visits Sutting’s home, but does not receive a good reception.
	\item Guthlate falls in love with Weden, and gives him a drink of the Mead.
	\item Weden has to bore through the mountains with the drill Rate.
	\item Weden has “bought [the Mead] well”; possibly a euphemistic reference to sleeping with Guthlathe for it.
	\item Guthlathe indeed does sleep with Weden, though not expressely in exchange for the Mead.
	\item The following day (\emph{hins hindra dags}, see note to this word in the edited text below), a group of Rime-Thurses come to Weden’s hall, to ask him whether a Baleworker is among the Gods, or if he has been slain by Sutting.
	\item Switching to the third person (which may indicate that this is his answer to the Rime-Thurses), Weden says that he “thinks” that Weden has sworn an oath, but that his words cannot be trusted. After the “simble” (i.e. drinking feast, banquet; probably referring to the drink of the Mead), Weden betrayed Sutting and made Guthlathe weep.
\end{enumerate}

The underlying narrative seems to generally agree with that of \Skaldskaparmal, but unlike its more transactional affair, we here find a stronger emphasis on Weden’s cruel betrayal of Guthlathe. A notable detail not found in \Skaldskaparmal\ is Weden’s oath in st. 109. The content of the oath was most likely that Weden would marry Guthlathe, something supported by the language used (see note to st. 108: \emph{hins hindra dags}). The recipient of the oath, which Weden clearly broke, was either Sutting or Guthlathe. That Weden swore it to Sutting, and thus asked him for Guthlathe’s hand in marriage, may be suggested by the description of Sutting as \emph{svikvinn} ‘betrayed’ in st. 109. This view, however, has an internal narrative problem: in st. 103 Weden describes his interaction with Sutting as poor, and in st. 105 Weden is said to have had to bore through the mountains, but this may just have been to reach Sutting, rather than Guthlathe as in \Skaldskaparmal.
The recipient of the oath being Guthlathe would agree better with the \Skaldskaparmal\ narrative, and Sutting’s betrayer would instead be her.

\sectionline

\bvg\bva Hęima \alst{g}laðr \alst{g}umi \hld\ ok við \alst{g}ęsti ręifr, &
\ind \alst{s}viðr skal of \alst{s}ik vesa; &
\alst{m}innigr ok \alst{m}ǫ́lugr, \hld\ ef vill \alst{m}arg-fróðr vesa; &
\ind opt skal \alst{g}óðs \alst{g}eta; &
\alst{f}imbul-\alst{f}ambi hęitir, \hld\ sá’s \alst{f}átt kann sęgja; &
\ind þat es \alst{ȯ}-snotrs \alst{a}ðal.\eva

\bvb At home shall man be glad and giving with the guest, \\
\ind wise about himself. \\
Of good memory and speech, if he wishes to be many-learned; \\
\ind oft shall he speak of good. \\
A fimble-fool is he called who little can say; \\
\ind that is the unclever man’s nature.\evb\evg


\bvg\bva Hinn \alst{a}ldna \alst{jǫ}tun sótta’k, \hld\ nú em’k \alst{a}ptr of kominn; &
\ind fátt gat’k \alst{þ}ęgjandi \alst{þ}ar; &
\alst{m}ǫrgum orðum \hld\ \alst{m}ę́lta’k í minn frama &
\ind í \alst{S}uttungs \alst{s}ǫlum.\eva

\bvb The old ettin \ken*{= Sutting} I sought, now am I come back; \\
\ind I got little hearing there. \\
Many words I spoke to my furtherance, \\
\ind in the halls of Sutting.\evb\evg


\bvg\bva\alst{G}unn-lǫð mér of \alst{g}af \hld\ \alst{g}ullnum stóli ȧ &
\ind \alst{d}rykk hins \alst{d}ýra mjaðar; &
\alst{i}ll \alst{i}ð-gjǫld \hld\ lét’k hana \alst{ę}ptir hafa &
\ind síns hins \alst{h}ęila \alst{h}ugar, &
\ind síns hins \alst{s}vára \alst{s}efa.\eva

\bvb \inx[P]{Guthlathe} gave me on the golden throne \\
\ind a drink of the dear mead; \\
evil recompense I let her have afterwards, \\
\ind for her whole heart, \\
\ind for her severe affection.\evb\evg


\bvg\bva\alst{R}ata munn \hld\ létumk \alst{r}úms of fȧa &
\ind ok of \alst{g}rjót \alst{g}naga; &
\alst{y}fir ok \alst{u}ndir \hld\ stóðumk \alst{jǫ}tna vegir, &
\ind svá \alst{h}ę́tta’k \alst{h}ǫfði til.\eva

\bvb Rate’s mouth I made to bring me room, \\
\ind and gnaw away at the rocks. \\
Over and under me stood the roads of the ettins \ken{mountains}; \\
\ind so I risked my head.\evb\evg


\bvg\bva \edtext{\alst{V}ęl kęypts hlutar \hld\ hęf’k \alst{v}ęl notit; &
\ind \alst{f}ás es \alst{f}róðum vant; &
því-at \edtrans{\alst{Ó}ð-rǿrir}{Woderearer}{\Bfootnote{One of the vessels in with the Mead of Poetry was held (see introduction to the present section above), here standing in for all the Mead.}} \hld\ es nú \alst{u}pp kominn &
\ind ȧ \alst{a}lda vés \edtrans{\alst{ja}ðar}{rim}{\Bfootnote{metr. emend.; \emph{jarðar} \Regius\ has a long root-syllable, and does not fit grammatically.}}.}{\lemma{Vęl \dots\ jaðar}\Bfootnote{Taken on its own this st. would be somewhat difficult, but in context the import is clear: Weden says that He has made good use of the Mead of Poetry by bringing it to earth, making poetry (and surely likewise other intellectual disciplines) available to men.}}\eva

\bvb The well bought thing \ken*{Mead of Poetry} have I used well— \\
\ind little do the learned lack, \\
for Woderearer is now come up \\
\ind over the rim of the \inx[C]{wigh} of men \ken*{= Middenyard}.\evb\evg


\bvg\bva\alst{I}fi ’s mér \alst{ȧ}, \hld\ at vę́ra’k \alst{ę}nn kominn &
\ind \alst{jǫ}tna gǫrðum \alst{ó}r, &
ef \alst{G}unn-laðar né nyta’k, \hld\ hinnar \alst{g}óðu konu, &
\ind es lǫgðumk \alst{a}rm \alst{y}fir.\eva

\bvb There is doubt in me, if I would yet be come \\
\ind out of the yards of the Ettins, \\
if Guthlathe I had not used, that good woman \\
\ind whom I laid my arm over.\evb\evg


\bvg\bva \edtrans{\alst{H}ins \alst{h}indra dags}{The following day}{\Bfootnote{This is the only occurrence of the comparative \emph{hindra} ‘following, next’ in the Norse (i.e. ‘belonging to Norway and its colonies’) literature. The superlative \emph{hindstr} ‘last, final’ does occur more often (e.g. \emph{indsta sinni} ‘the last time’, with loss of the \emph{h-}; see \CV: \emph{hindri}), and the possible derivative \emph{hindar-dags} ‘day after tomorrow, two days after’ is found twice, both times in the \Gulatingslog, chh. 37 and 266.  If we, however, search in the broader Scandinavian sphere, we find in the Swedish provicial laws an exact equivalent of the present phrase, namely OSwe. \emph{hindra-dagher}, a law-word referring specifically to the ‘day after the wedding’, used both on its own and in the expression \emph{hindra-dags gięf} ‘morning gift’.  If this is indeed the sense in the present stanza, two interpretations are possible: it either (i) refers sarcastically to Weden’s sleeping with Guthlathe (as would be done on the wedding night), or (ii) means that Weden actually married, or promised to marry, Guthlathe.  The latter interpretation may find support in st. 109, see notes there.}} \hld\ gingu \alst{h}rím-þursar &
\alst{H}áva ráðs at fregna, \hld\ \alst{H}áva \alst{h}ǫllu í, &
at \alst{B}ǫl-verki spurðu, \hld\ ef vę́ri með \alst{b}ǫndum kominn &
\ind eða hęfði hǫ́num \alst{S}uttungr of \alst{s}óit.\eva

\bvb The following day went the Rime-Thurses \\
\ind to ask for the High One’s counsel, in the High One’s hall. \\
About Baleworker \name{= Weden} they asked, if he were come among the bonds \name{gods}, \\
\ind or if Sutting had slain him.\evb\evg


\bvg\bva \edtext{Baug-ęið \alst{Ó}ðinn \hld\ hygg at \alst{u}nnit hafi, &
\ind hvat skal hans \alst{t}ryggðum \alst{t}rúa? &
\alst{S}uttung \alst{s}vikvinn \hld\ hann lét \alst{s}umbli frȧ &
\ind ok \alst{g}rǿtta \alst{G}unn-lǫðu}{\lemma{Baug-ęið \dots\ Gunn-lǫðu ‘A bigh-oath \dots\ brought to tears™}\Bfootnote{The exact narrative referred to in the stanza is hard to pin down, but I find the following most likely: Weden swore an oath on a bigh, its contents being that he would marry Guthlathe. Sutting then hosted a simble (banquet, drinking feast) for the new couple (cf. \emph{hins hindra dags} in st. 108), and Weden slept with her, but after. \emph{svikvinn} ‘betrayed’ and \emph{grǿtta} ‘brought to tears’ are (respectively masc. and fem.) acc. sg. past participles of the transitive verbs \emph{svíkva} ‘to betray’ and \emph{grǿta} ‘to make weep, bring to tears’. I read \emph{lét} as meaning ‘left, abandoned, forsook’.}}.\eva

\bvb A \inx[C]{bigh-oath} I ween that Weden has sworn— \\
\ind how shall one trust his truces? \\
Away from the \inx[C]{simble} he left Sutting betrayed, \\
\ind and Guthlathe, made to weep.\evb\evg

\sectionline

\section{The Speeches of Loddfathomer (\emph{Loddfáfnis mǫ́l}, 111–137)}

A series of advice stanzas addressed to \inx[P]{Loddfathomer}, an otherwise unknown figure who is clearly mythological.  The name is a compound: the first element, \emph{lodd-}, is related to ON \emph{loddari} ‘juggler, tramp’, OE \emph{loddere} ‘pauper, beggar’; the second, \emph{Fáfnir} (\inx[P]{Fathomer}), is the name of a famous Wyrm and literally means ‘embracer’.  This name gives a picture of an archetypal greedy fool; he is taught by Weden, his opposite.

The section division is found in \Regius.  Stanza 111 has a large initial \emph{M}, albeit smaller than those which introduce new chapters and poems, and the beginning of the following section, the \emph{Rune-Tally}, is also clearly marked by an initial.

\sectionline

\bvg\bva Mál ’s at \alst{þ}ylja \hld\ \alst{þ}ular stóli ȧ; &
\ind \alst{U}rðar brunni \alst{a}t &
\alst{s}á’k ok þagða’k, \hld\ \alst{s}á’k ok hugða’k, &
\ind hlýdda’k ȧ \alst{m}anna \alst{m}ál; &
of \alst{r}únar hęyrða’k dǿma, \hld\ né of \alst{r}ǫ́ðum þǫgðu &
\ind \alst{H}áva \alst{h}ǫllu at, &
\ind \alst{H}áva \alst{h}ǫllu í &
\ind hęyrða’k \alst{s}ęgja \alst{s}vá:\eva

\bvb It is time to \inx[C]{thill}, upon the \inx[C]{thyle}’s chair. \\
\ind At the \inx[L]{Well of Weird} \\
I saw and shut up; I saw and I thought; \\
\ind I heeded the matters of men. \\
Of runes I heard them speak, nor were they silent of counsels \\
\ind at the High One’s hall, \\
\ind in the High One’s hall; \\
\ind I heard them say so:\evb\evg


\bvg\bva\alst{R}ǫ́ðumk þér Loddfáfnir, \hld\ at \alst{r}ǫ́ð nemir, &
\ind \alst{n}jóta munt ef \alst{n}emr, &
\ind þér munu \alst{g}óð ef \alst{g}etr: &
\alst{n}ǫ́tt þú rís-at, \hld\ nema ȧ \alst{n}jósn séir, &
\ind eða \edtrans{lęitir þér \alst{i}nnan \alst{ú}t staðar}{or thou look for thy place outside}{\Bfootnote{Lit. word-for-word “or thou look for thee from within out a place”, which becomes nonsensical.  \emph{lęita sér staðar} ‘look for one’s place’ is a euphemism, i.e. “to relieve oneself”, which was done outside.}}.\eva

\bvb I counsel thee, O Loddfathomer, that thou learn the counsels; \\
\ind thou wilt have use if thou learn, \\
\ind they will be good for thee if thou get: \\
At night do not rise, unless thou be scouting, \\
\ind or thou look for thy place outside.\evb\evg


\bvg\bva\alst{R}ǫ́ðumk þér Loddfáfnir, \hld\ at \alst{r}ǫ́ð nemir, &
\ind \alst{n}jóta munt ef \alst{n}emr, &
\ind þér munu \alst{g}óð ef \alst{g}etr: &
\alst{f}jǫl-kunnigri konu \hld\ skal-at-tu í \alst{f}aðmi sofa, &
\ind svá’t hon \alst{l}yki þik \alst{l}iðum.\eva

\bvb I counsel thee, O Loddfathomer, that thou learn the counsels; \\
\ind thou wilt have use if thou learn, \\
\ind they will be good for thee if thou get: \\
By a \inx[C]{many-cunning} woman’s bosom shalt thou never sleep, \\
\ind lest she lock thee in [her?] limbs.\evb\evg


\bvg\bva Hǫ́n svá \alst{g}ørir \hld\ at \edtrans{\alst{g}ȧir}{heed}{\Bfootnote{The nasal vowel here is based on Elfdalian \emph{gą̊}.}} ęigi &
\ind \alst{þ}ings né \alst{þ}jóðans máls; &
\alst{m}at þú vill-at \hld\ né \alst{m}anns-kis gaman &
\ind fęrr þú \alst{s}orga-fullr at \alst{s}ofa.\eva

\bvb She makes it so that thou heed not \\
\ind \inx[C]{Thing}’s or ruler’s speech; \\
thou hast no wish for food nor any man’s pleasure; \\
\ind thou goest sorrowful to sleep.\evb\evg


\bvg\bva\alst{R}ǫ́ðumk þér Loddfáfnir, \hld\ at \alst{r}ǫ́ð nemir, &
\ind \alst{n}jóta munt ef \alst{n}emr, &
\ind þér munu \alst{g}óð ef \alst{g}etr: &
\alst{a}nnars konu \hld\ tęyg þér \alst{a}ldri-gi &
\ind \edtrans{\alst{ęy}ra-rúnu}{ear-whisperer \ken{lover}}{\Bfootnote{This word is also used in \Voluspa\ 38, in which male seducers of married women are among those being forced to wade through “heavy streams” in the afterlife.}} \alst{a}t.\eva

\bvb I counsel thee, O Loddfathomer, that thou learn the counsels; \\
\ind thou wilt have use if thou learn, \\
\ind they will be good for thee if thou get: \\
Another man’s woman do never tug \\
\ind into becoming thy ear-whisperer \ken{lover}.\evb\evg


\bvg\bva\alst{R}ǫ́ðumk þér Loddfáfnir, \hld\ en \alst{r}ǫ́ð nemir, &
\ind \alst{n}jóta munt ef \alst{n}emr, &
\ind þér munu \alst{g}óð ef \alst{g}etr: &
\edtrans{\alst{f}jalli eða \alst{f}irði}{on fell or firth}{\Bfootnote{i.e. ‘hiking through mountains or travelling at sea’; a very Norwegian expression.  This word pair is a formulaic merism; this is its only poetic attestation, but it is found a few times in the Old Norwegian laws.}}, \hld\ ef þik \alst{f}ara tíðir, &
\ind fȧsk-tu at \alst{v}irði \alst{v}ęl.\eva

\bvb I counsel thee, O Loddfathomer—and thou oughtst to learn the counsels; \\
\ind thou wilt have use if thou learn, \\
\ind they will be good for thee if thou get: \\
on fell or firth—if thou desire to journey— \\
\ind furnish thyself well with food.\evb\evg


\bvg\bva\alst{R}ǫ́ðumk þér Loddfáfnir, \hld\ en \alst{r}ǫ́ð nemir, &
\ind \alst{n}jóta munt ef \alst{n}emr, &
\ind þér munu \alst{g}óð ef \alst{g}etr: &
\alst{i}llan mann \hld\ lát \alst{a}ldri-gi &
\ind \edtext{\alst{ȯ}-hǫpp at þér \alst{v}ita}{\Bfootnote{An unambiguous instance of \emph{v} alliterating with a vowel.}}, &
því-at af \alst{i}llum manni \hld\ fę̇r \alst{a}ldri-gi &
\ind \alst{g}jǫld hins \alst{g}óða hugar.\eva

\bvb I counsel thee, O Loddfathomer—and thou oughtst to learn the counsels; \\
\ind thou wilt have use if thou learn, \\
\ind they will be good for thee if thou get: \\
An evil man do never let \\
\ind know of thy misfortunes; \\
for from an evil man gettest thou never \\
\ind rewards for thy good will.\evb\evg


\bvg\bva\edtrans{\alst{O}far-la}{Sorely}{\Bfootnote{Contraction of \emph{ofar-liga} ‘\CV: high up, in the upper part’, presumably meaning that the words were particularly grievous or insulting, i.e., they “got to him”.  Whether he was murdered or committed suicide is not clear.}} bíta \hld\ sá’k \alst{ęi}num hal &
\ind \alst{o}rð \alst{i}llrar konu, &
\edtrans{\alst{f}lá-rǫ́ð tunga}{a false-counseling tongue}{\Bfootnote{Cf. \Lokasenna\ 31/1: \emph{flǫ́ ’s þér tunga} ‘false is thy tongue’.}} \hld\ varð hǫ́num at \alst{f}jǫr-lagi &
\ind ok þęygi of \alst{s}anna \alst{s}ǫk.\eva

\bvb Sorely biting I saw at a lonely man \\
\ind the words of an evil woman; \\
a false-counseling tongue brought his life to its end, \\
\ind and in no way over a truthful charge.\evb\evg


\bvg\bva\alst{R}ǫ́ðumk þér Loddfáfnir, \hld\ en \alst{r}ǫ́ð nemir, &
\ind \alst{n}jóta munt ef \alst{n}emr, &
\ind þér munu \alst{g}óð ef \alst{g}etr: &
\alst{v}ęitst, ef \alst{v}in átt, \hld\ þann’s \alst{v}ęl trúir, &
\ind \alst{f}ar þú at \alst{f}inna opt; &
því-at \edtrans{\alst{h}rísi vęx \hld\ ok \alst{h}ǫ́u grasi}{with brushwood and with tall grass grows}{\Bfootnote{Identical to \Grimnismal\ 17/1.}} &
\ind \alst{v}egr, es \alst{v}ę́t-ki trøðr.\eva

\bvb I counsel thee, O Loddfathomer—and thou oughtst to learn the counsels; \\
\ind thou wilt have use if thou learn, \\
\ind they will be good for thee if thou get: \\
Thou knowest, if thou have a friend whom thou well trust: \\
\ind journey to find him oft; \\
for with brushwood and tall grass grows \\
\ind the way which no one treads.\evb\evg


\bvg\bva\alst{R}ǫ́ðumk þér Loddfáfnir, \hld\ en \alst{r}ǫ́ð nemir, &
\ind \alst{n}jóta munt ef \alst{n}emr, &
\ind þér munu \alst{g}óð ef \alst{g}etr: &
\alst{g}óðan mann \hld\ tęyg þér at \edtrans{\alst{g}aman-rúnum}{pleasure-runes}{\Bfootnote{Here “rune” appears to carry its root meaning of ‘whisper, counsel, speech’, thus ‘pleasing speech’.  Cf. st. 129 where this word reoccurs.}} &
\ind ok nem \edtrans{\alst{l}íknar-galdr}{liking-galders}{\Bfootnote{i.e. ways of speaking which will make one liked or popular.  For \emph{líkn} ‘liking’ see sts. 8 (with note) and 123.}} meðan \alst{l}ifir.\eva

\bvb I counsel thee, O Loddfathomer—and thou oughtst to learn the counsels; \\
\ind thou wilt have use if thou learn, \\
\ind they will be good for thee if thou get: \\
A good man do tug toward thee with pleasure-runes, \\
\ind and learn liking-galders while thou livest.\evb\evg


\bvg\bva\alst{R}ǫ́ðumk þér Loddfáfnir, \hld\ en \alst{r}ǫ́ð nemir, &
\ind \alst{n}jóta munt ef \alst{n}emr, &
\ind þér munu \alst{g}óð ef \alst{g}etr: &
\alst{v}in þínum \hld\ \alst{v}es aldri-gi &
\ind \alst{f}yrri at \alst{f}laum-slitum. &
\alst{s}org etr hjarta, \hld\ ef þú \edtext{\alst{s}ęgja né náir &
\ind \alst{ęi}n-hvęrjum \alst{a}llan hug}{\lemma{sęgja \dots\ ęin-hvęrjum allan hug ‘tell anyone thy whole mind’}\Bfootnote{Cf. st. 123 which uses almost the same expression.}}.\eva

\bvb I counsel thee, O Loddfathomer—and thou oughtst to learn the counsels; \\
\ind thou wilt have use if thou learn, \\
\ind they will be good for thee if thou get: \\
With thy friend be thou never the first \\
\ind to tear the relation apart. \\
Sorrow will eat thy heart if thou canst not tell \\
\ind anyone thy whole mind.\evb\evg


\bvg\bva\alst{R}ǫ́ðumk þér Loddfáfnir, \hld\ en \alst{r}ǫ́ð nemir, &
\ind \alst{n}jóta munt ef \alst{n}emr, &
\ind þér munu \alst{g}óð ef \alst{g}etr: &
\edtext{\alst{o}rðum skipta \hld\ skalt \alst{a}ldri-gi &
\ind við \edtrans{\alst{ȯ}-svinna \alst{a}pa}{unwise apes}{\Bfootnote{Formulaic; cf. \Grimnismal\ 33, \Fafnismal\ 11.}}}{\lemma{orðum \dots\ apa ‘Words \dots\ apes’}\Bfootnote{Cf. st. 125 which gives similar advice.}},\eva

\bvb I counsel thee, O Loddfathomer—and thou oughtst to learn the counsels; \\
\ind thou wilt have use if thou learn, \\
\ind they will be good for thee if thou get: \\
Words shalt thou never exchange \\
\ind with unwise apes,\evb\evg


\bvg\bva \edtext{því-at af \alst{i}llum manni \hld\ munt \alst{a}ldri-gi &
\ind \alst{g}óðs laun of \alst{g}eta}{\lemma{því-at \dots\ geta ‘For \dots\ praise’}\Bfootnote{Cf. st. 117/6–7.}}, &
en \alst{g}óðr maðr \hld\ mun þik \alst{g}ørva męga &
\ind \edtrans{\alst{l}íkn-fastan}{steadfast in liking}{\Bfootnote{The first element \emph{líkn} ‘liking’ is somewhat difficult; see sts. 8 (with note) and 120.  For the present cpd \textcite{LaFargeGlossary} give a tentative ‘assured of favour’, while \CV\ gives ‘fast in goodwill, beloved’.}} at \alst{l}ofi.\eva

\bvb for from an evil man wilt thou never \\
\ind get a reward for thy goodness, \\
but a good man will know to make thee \\
\ind steadfast in liking by [his] praise.\evb\evg


\bvg\bva\alst{S}ifjum ’s þȧ blandit \hld\ hvęrr es \edtext{\alst{s}ęgja rę́ðr &
\ind \alst{ęi}num \alst{a}llan hug}{\lemma{sęgja \dots\ \alst{ęi}num \alst{a}llan hug ‘tell one man his whole mind’}\Bfootnote{Cf. st. 121 which uses almost the same expression.}}; &
alt es \alst{b}ętra \hld\ an sé \alst{b}rigðum at vesa: &
es-a sá \alst{v}inr ǫðrum \hld\ es \alst{v}ilt ęitt sęgir.\eva

\bvb Kinship is blended for whomever resolves to tell \\
\ind one man his whole mind. \\
Everything is better than to be with the fickle; \\
he is no friend to another who tells only what is pleasant.\evb\evg


\bvg\bva\alst{R}ǫ́ðumk þér Loddfáfnir, \hld\ en \alst{r}ǫ́ð nemir, &
\ind \alst{n}jóta munt ef \alst{n}emr, &
\ind þér munu \alst{g}óð ef \alst{g}etr: &
\edtrans{þrimr \alst{o}rðum}{With three words}{\Bfootnote{i.e. ‘not even with three words’. If one understands \emph{orð} to mean ‘speech’, it may be interpreted as that if one says something (the first speech) to which another man responds insultingly (the second speech), one should not respond a third time and turn it into a fight.}} sęnna \hld\ skal-at-tu þér við \alst{v}erra mann; &
\ind opt hinn \alst{b}ętri \alst{b}ilar, &
\ind þȧ’s hinn \alst{v}erri \alst{v}egr.\eva

\bvb I counsel thee, O Loddfathomer—and thou oughtst to learn the counsels; \\
\ind thou wilt have use if thou learn, \\
\ind they will be good for thee if thou get: \\
With three words shalt thou not flyte with a worse man; \\
\ind oft the better man breaks \\
\ind when the worse man strikes.\footnoteB{Cf. st. 121.}\evb\evg


\bvg\bva\alst{R}ǫ́ðumk þér Loddfáfnir, \hld\ en \alst{r}ǫ́ð nemir, &
\ind \alst{n}jóta munt ef \alst{n}emr, &
\ind þér munu \alst{g}óð ef \alst{g}etr: &
\alst{sk}ó-smiðr þú vesir \hld\ né \alst{sk}ępti-smiðr, &
\ind nema \alst{s}jǫlfum þér \alst{s}éir. &
\alst{Sk}ór ’s \alst{sk}apaðr illa \hld\ eða \alst{sk}apt sé rangt, &
\ind þȧ ’s þér \alst{b}ǫls \alst{b}eðit.\eva

\bvb I counsel thee, O Loddfathomer—and thou oughtst to learn the counsels; \\
\ind thou wilt have use if thou learn, \\
\ind they will be good for thee if thou get: \\
Be not a shoe-maker nor shaft-maker, \\
\ind unless thou be one for thyself. \\
The shoe is shaped badly or the shaft be crooked— \\
\ind then for thee a \inx[C]{bale} is bid.\footnoteB{i.e. the customer will place a curse on you if he dislikes the wares.}\evb\evg


\bvg\bva\alst{R}ǫ́ðumk þér Loddfáfnir, \hld\ en \alst{r}ǫ́ð nemir, &
\ind \alst{n}jóta munt ef \alst{n}emr, &
\ind þér munu \alst{g}óð ef \alst{g}etr: &
hvar’s \alst{b}ǫl kant, \hld\ kveð þér \alst{b}ǫlvi at &
\ind ok gef-at þínum \alst{f}jǫ́ndum \alst{f}rið.\eva

\bvb I counsel thee, O Loddfathomer—and thou oughtst to learn the counsels; \\
\ind thou wilt have use if thou learn, \\
\ind they will be good for thee if thou get: \\
Wherever thou knowest a bale, call it a bale against thee, \\
\ind and give not thy foes peace.\footnoteB{i.e. “if somebody puts a curse on you, do not ignore it, but respond decisively”.  This st. has often been interpreted as a command to call out evil, even when committed towards somebody else, and while there is nothing in it that speaks clearly against that interpretation, it does not agree with the general spirit of the \Havamal, which is one of caution and shrewdness.}\evb\evg


\bvg\bva\alst{R}ǫ́ðumk þér Loddfáfnir, \hld\ en \alst{r}ǫ́ð nemir, &
\ind \alst{n}jóta munt ef \alst{n}emr, &
\ind þér munu \alst{g}óð ef \alst{g}etr: &
\alst{i}llu fęginn \hld\ ves \alst{a}ldri-gi, &
\ind \edtrans{en lát þér at \alst{g}óðu \alst{g}etit}{but [rather] let thyself be pleased by good}{\Bfootnote{This construction is equivalent to \CV: \emph{geta}, A. IV. with acc.}}.\eva

\bvb I counsel thee, O Loddfathomer—and thou oughtst to learn the counsels; \\
\ind thou wilt have use if thou learn, \\
\ind they will be good for thee if thou get: \\
Rejoicing in evil be thou never, \\
\ind but let thyself be pleased by good.\evb\evg


\bvg\bva\alst{R}ǫ́ðumk þér Loddfáfnir, \hld\ en \alst{r}ǫ́ð nemir, &
\ind \alst{n}jóta munt ef \alst{n}emr, &
\ind þér munu \alst{g}óð ef \alst{g}etr: &
\alst{u}pp líta \hld\ skal-at-tu í \alst{o}rrostu; &
—\alst{g}jalti \alst{g}líkir \hld\ verða \alst{g}umna synir— &
\ind síðr þitt of \alst{h}ęilli \alst{h}alir.\eva

\bvb I counsel thee, O Loddfathomer—and thou oughtst to learn the counsels; \\
\ind thou wilt have use if thou learn, \\
\ind they will be good for thee if thou get: \\
Up shalt thou not look in battle \\
—alike to a madman become the sons of men— \\
\ind lest men bewitch thy [sense/life/face].\footnoteB{A very difficult st. \CV\ explains \emph{gjalti} as an old dative of \emph{gǫltr} ‘boar, hog’, and thus sees the closely related phrase \emph{verða at gjalti} as “‘to be turned into a hog’, i.e. ‘to turn mad with terror’, esp. in a fight”. The vowel breaking is however unexpected here, since \emph{gǫltr} (< Proto-Norse \emph{*galtuʀ}) is an u-stem, which makes the stem-vowel in the dat. sg. \emph{gęlti} (< \emph{*galtiu}, cf. \textbf{kunimudiu}, dat. sg. of \emph{*Kunimunduʀ}, on the Tjurkö 1 bracteate) the result of i-umlaut rather than an original short \emph{*e}.

\textcite{LaFargeGlossary} instead explain the word as a borrowing from Old Irish \emph{geilt} ‘insane, mad’. \textcite{PettitEdda} follows this, and argues that the whole theme of the st. probably be of Celtic origin, giving several examples from Celtic literature of warriors going mad upon looking up into the sky during battle. In this case the men (\emph{halir}, which word seems to have an association with warriors; cf. 36–37, 49) would be to quote Pettit some sort of “supernatural sky warriors”, in my opinion most likely the \inx[G]{Oneharriers}.}\evb\evg


\bvg\bva\alst{R}ǫ́ðumk þér Loddfáfnir, \hld\ en \alst{r}ǫ́ð nemir, &
\ind \alst{n}jóta munt ef \alst{n}emr, &
\ind þér munu \alst{g}óð ef \alst{g}etr: &
Ef vilt þér \alst{g}óða konu \hld\ kvęðja at \edtrans{\alst{g}aman-rúnum}{pleasure-runes}{\Bfootnote{While easily interpreted as ‘sexual intercourse’, the word is used in st. 120 with a decidedly non-sexual meaning.  Its base meaning is probably ‘good conversation’.}} &
\ind ok \alst{f}ȧa \alst{f}ǫgnuð af, &
\alst{f}ǫgru skalt hęita \hld\ ok láta \alst{f}ast vesa; &
\ind lęiðisk mann-gi \alst{g}ótt ef \alst{g}etr.\eva

\bvb I counsel thee, O Loddfathomer—and thou oughtst to learn the counsels; \\
\ind thou wilt have use if thou learn, \\
\ind they will be good for thee if thou get: \\
If thou wilt for thyself greet a good woman to pleasure-runes, \\
\ind and get good cheer from her; \\
fair things shalt thou promise, and let it be fast; \\
\ind no man loathes a good thing if he gets it.\evb\evg


\bvg\bva\alst{R}ǫ́ðumk þér Loddfáfnir, \hld\ en \alst{r}ǫ́ð nemir, &
\ind \alst{n}jóta munt ef \alst{n}emr, &
\ind þér munu \alst{g}óð ef \alst{g}etr: &
\alst{v}aran bið’k þik \alst{v}esa \hld\ ok ęigi of·\alst{v}aran, &
ves við \alst{ǫ}l varastr, \hld\ ok við \alst{a}nnars konu &
ok við \alst{þ}at hit \alst{þ}riðja, \hld\ at \alst{þ}jófar né lęiki.\eva

\bvb I counsel thee, O Loddfathomer—and thou oughtst to learn the counsels; \\
\ind thou wilt have use if thou learn, \\
\ind they will be good for thee if thou get: \\
Wary I ask thee to be, and not over-wary; \\
be thou wariest with ale, and with another man’s woman, \\
and with the third, that thieves do not outplay [thee].\evb\evg


\bvg\bva\alst{R}ǫ́ðumk þér Loddfáfnir, \hld\ en \alst{r}ǫ́ð nemir, &
\ind \alst{n}jóta munt ef \alst{n}emr, &
\ind þér munu \alst{g}óð ef \alst{g}etr: &
at \alst{h}áði né \alst{h}látri \hld\ \alst{h}af aldri-gi &
\ind \alst{g}ęst né \alst{g}anganda.\eva

\bvb I counsel thee, O Loddfathomer—and thou oughtst to learn the counsels; \\
\ind thou wilt have use if thou learn, \\
\ind they will be good for thee if thou get: \\
In scorn or laughter do never have \\
\ind a guest or wanderer.\evb\evg


\bvg\bva\alst{O}pt vitu \alst{ȯ}-gǫrla, \hld\ þęir’s sitja \alst{i}nni fyrir, &
\ind hvęrs þęir ’ru \alst{k}yns es \alst{k}oma; &
es-at maðr svá \alst{g}óðr \hld\ at \alst{g}alli né fylgi, &
\ind né svá \alst{i}llr at \alst{ęi}nu-gi dugi.\eva

\bvb Oft they know unclearly, who sit further within, \\
\ind of what kind are those who come; \\
there is no man so good that no flaw follows, \\
\ind nor so bad that he for nothing avails.\evb\evg


\bvg\bva\alst{R}ǫ́ðumk þér Loddfáfnir, \hld\ en \alst{r}ǫ́ð nemir, &
\ind \alst{n}jóta munt ef \alst{n}emr, &
\ind þér munu \alst{g}óð ef \alst{g}etr: &
at \alst{h}ǫ́rum þul \hld\ \alst{h}lę́ aldri-gi, &
\ind opt ’s \alst{g}ótt þat’s \alst{g}amlir kveða, &
opt ór \alst{sk}ǫrpum bęlg \hld\ \alst{sk}ilin orð koma &
\ind þęim’s \alst{h}angir með \alst{h}ǫ́um &
\ind ok \alst{sk}ollir með \alst{sk}rǫ́um, &
\ind ok \alst{v}áfir með \alst{v}íl-mǫgum.\eva

\bvb I counsel thee, O Loddfathomer—and thou oughtst to learn the counsels; \\
\ind thou wilt have use if thou learn, \\
\ind they will be good for thee if thou get: \\
At a hoary thyle do never laugh; \\
\ind oft is good that which old men sing. \\
Oft from scorched leather come discerning words; \\
\ind from him who hangs with hides, \\
\ind and dangles with dry skins, \\
\ind and sways among lads of toil \ken{thralls}.\footnoteB{TODO: Some note. \emph{vil-mǫgum} meaning ‘veal-stomachs’? Cf. Crawford’s video and Finnur on this.}\evb\evg


\bvg\bva\alst{R}ǫ́ðumk þér Loddfáfnir, \hld\ en \alst{r}ǫ́ð nemir, &
\ind \alst{n}jóta munt ef \alst{n}emr, &
\ind þér munu \alst{g}óð ef \alst{g}etr: &
\alst{g}ęst þú né \alst{g}ęyj-a \hld\ \edtrans{né ȧ \alst{g}rind hrę́kir}{nor spit at the gate}{\Bfootnote{The guest is presumably standing behind gate waiting for the farmer to open it and let him in.}}; &
\ind get þú \alst{v}ǫ́-luðum \alst{v}ęl.\eva

\bvb I counsel thee, O Loddfathomer—and thou oughtst to learn the counsels; \\
\ind thou wilt have use if thou learn, \\
\ind they will be good for thee if thou get: \\
At a guest bark not, nor spit at the gate; \\
\ind furnish the destitute well.\evb\evg


\bvg\bva\alst{R}ammt es þat tré, \hld\ es \alst{r}íða skal &
\ind \alst{ǫ}llum at \alst{u}pp-loki; &
\alst{b}aug þú gef \hld\ eða þat \alst{b}iðja mun &
\ind þér \alst{l}ę́s hvęrs ȧ \alst{l}iðu.\eva

\bvb Strong is that wood which shall swing \\
\ind to open up for all.\footnoteB{i.e. the beam of the gate in front of the farm.} \\
Do give a bigh, or it will bid \\
\ind every kind of guile onto thy limbs.\evb\evg


\bvg\bva\alst{R}ǫ́ðumk þér Loddfáfnir, \hld\ en \alst{r}ǫ́ð nemir, &
\ind \alst{n}jóta munt ef \alst{n}emr, &
\ind þér munu \alst{g}óð ef \alst{g}etr: &
hvar’s \alst{ǫ}l drekkir \hld\ kjós þér \alst{ja}rðar męgin, &
því-at \alst{jǫ}rð tękr við \alst{ǫ}lðri, \hld\ en \alst{ę}ldr við sóttum, &
\alst{ęi}k við \alst{a}bbindi, \hld\ \alst{a}x við fjǫl-kyngi, &
\alst{h}ǫll við \alst{h}ýrógi; \hld\ \edtrans{\alst{h}ęiptum skal Mána kvęðja}{in feuds shall one hail Moon}{\Bfootnote{Cf. \Voluspa\ 5 which mentions the “Moon’s might”; for which He is presumably here invoked.  For \emph{kvęðja} ‘hail, invoke’ cf. \Lokasenna\ P3.}}, &
\alst{b}ęiti við \alst{b}it-sóttum, \hld\ en við \alst{b}ǫlvi rúnar; &
\ind \alst{f}old skal við \alst{f}lóði taka.\eva

\bvb I counsel thee, O Loddfathomer, that thou learn the counsels; \\
\ind thou wilt have use if thou learn, \\
\ind they will be good for thee if thou get: \\
Wherever thou drinkest ale choose thee Earth’s might, \\
for earth takes against drunkenness, and fire against sicknesses; \\
oak against dysentery; the ear [of corn] against sorcery; \\
bearded rye against hernia—in feuds shall one hail Moon— \\
heather against bite-sicknesses, and \inx[C]{rune}[runes] against a \inx[C]{bale};\footnoteB{cf. sts. 126, 152.} \\
\ind earth shall one have against flood.\evb\evg

\sectionline

\section{The Rune-Tally (138–146)}

This group of stanzas is introduced by a large initial in \Regius, marking the beginning of a new section.  In younger paper manuscripts they have the header \emph{Rúna-tals þáttr} ‘Strand of the Rune-Tally’, and generally give an archaic, mystic impression; at times one gets a feeling that they were drawn from the lips of an Odinic priest.

Apart from these stanzas there are a few other manuscript attestations of similar Runic magic.  Closest at hand is st. 80 above, which would fit seamlessly into the present section.  Outside of \Havamal\ there is \Sigrdrifumal\ 5–17, also preserved in \Regius.

\sectionline

\bvg\bva\alst{V}ęit’k at ek hekk \hld\ \edtrans{\alst{v}indga-męiði}{the windy beam}{\Bfootnote{Generally understood to be a variant of \emph{vinga-męiðr} ‘gallows tree’ found in three Scaldic stanzas (\Skp\ signa: Egill Lv 14, Eyv \emph{Hál} 5, Anon (FoGT) 17).}} ȧ &
\ind \alst{n}ę́tr allar \alst{n}íu, &
\alst{g}ęiri undaðr \hld\ ok \alst{g}efinn Óðni, &
\ind \alst{s}jalfr \alst{s}jǫlfum mér, &
ȧ þęim \alst{m}ęiði, \hld\ es \alst{m}ann-gi vęit, &
\ind hvęrs af \alst{r}ótum \alst{r}innr.\eva

\bvb I know that I hung on the windy beam \\
\ind for nine nights all, \\
wounded by spear and given to Weden, \\
\ind myself to myself— \\
on that beam where no man knows \\
\ind of whose roots it runs.\evb\evg


\bvg\bva Við \edtext{\alst{h}lęifi mik sǿldu-t \hld\ né við \alst{h}orni-gi}{\lemma{hlęifi \dots\ horni-gi ‘loaf \dots\ horn’}\Bfootnote{i.e. “I got neither bread to eat nor ale to drink.”}}; &
\alst{n}ýsta ek \alst{n}iðr, \hld\ \alst{n}am’k upp rúnar, &
\alst{ǿ}pandi nam, \hld\ fell’k \alst{a}ptr þaðan.\eva

\bvb With loaf they relieved me not, nor with any horn. \\
I peered down; I took up the runes; \\
screaming I took—I fell back thence.\evb\evg


\bvg\bva\edtrans{\alst{F}imbul-ljóð níu}{Nine fimble-leeds}{\Bfootnote{Nine very great chants or spells (\inx[C]{galders}), compare the eighteen leeds below (st. 147 onward).  It is unclear what this has to do with Weden’s Hanging; this stanza may be an insert.}} \hld\ nam’k af \edtext{hinum \alst{f}rę́gja syni &
\ind \alst{B}ǫlþorns, \alst{B}ęstlu fǫður,}{\lemma{hinum frę́gja syni Bǫlþorns, Bęstlu fǫður ‘the famed son of Balethorn, Bestle’s father’}\Bfootnote{According to \Gylfaginning\ 6: \emph{[Borr] fekk þeirar konu, er Bestla hét, dóttir Bǫlþorns jötuns, ok fengu þau þrjá sonu; hét einn Óðinn, annarr Vili, þriði Vé [\dots]} ‘[Byre] got the wife called Bestle, the daughter of the ettin Balethorn, and they had three sons: one was called Weden, the other Will, the third Wigh.’  Balethorn’s son is then Weden’s uncle, an instance of the old Indo-European custom of sending sons away to be fostered by the mother’s male relations.  Cf. TODO: some reference.}} &
ok ek \alst{d}rykk of gat \hld\ hins \alst{d}ýra mjaðar &
\ind \alst{au}sinn \alst{Ó}ð-rǿri.\eva

\bvb Nine \inx[C]{fimble}-leeds I learned from the famed son \\
\ind of \inx[P]{Balethorn}, \inx[P]{Bestle}’s father— \\
and a drink I got of the dear mead \\
\ind poured from \inx[P]{Woderearer}.\evb\evg


\bvg\bva Þȧ \edtrans{nam’k \alst{f}rę́vask}{I began to flourish}{\Bfootnote{A notorious mistranslation popularized by \textcite{Greenberg1988} has rendered these words as “I took semen”.  They would supposedly reference Weden stealing the ejaculate from hanged men in order to replenish his own powers—something not otherwise attested.  This preposterous notion makes no sense in the context of the text and has no philological grounding.  While Old Norse \emph{frę́} does mean “seed”, it only refers to the seeds of plants, not the seed animals or men.  Regardless, \emph{frę́vask} is without doubt a reflexive verb literally meaning something like ‘cultivate oneself’.}} \hld\ ok \alst{f}róðr vesa &
\ind ok \alst{v}axa ok \alst{v}ęl hafask; &
\edtext{\alst{o}rð mér af \alst{o}rði \hld\ \alst{o}rðs lęitaði &
\alst{v}erk mér af \alst{v}erki \hld\ \alst{v}erks lęitaði.}{\lemma{orð \dots\ lęitaði. ‘My word \dots sought out.’}\Bfootnote{Every good speech led to another; every good deed likewise.}}\eva

\bvb Then I took to flourish and be wise, \\
\ind and grow and have it well. \\
My word from a word a word sought out; \\
my work from a work a work sought out.\evb\evg


\bvg\bva\edtrans{\alst{R}únar munt finna \hld\ ok \alst{r}áðna stafi}{Runes wilt thou find, and interpreted staves’}{\Bfootnote{A strong resemblance is found in the long-line on the medieval runestone N 13: \emph{rúnar ek ríst \hld\ ok ráðna stafi} ‘runes I carve, and interpreted staves.’}}, &
\ind mjǫk \alst{st}óra \alst{st}afi, &
\ind mjǫk \alst{st}inna \alst{st}afi, &
\ind es \alst{f}áði \alst{F}imbul-þulr &
\ind ok \alst{g}ørðu \alst{g}inn-ręgin &
\ind ok \alst{r}ęist Hroptr \edtrans{\alst{r}agna}{of the Reins}{\Afootnote{\emph{‘rǫgna’} \Regius}}.\eva

\bvb \inx[C]{rune}[Runes] wilt thou find, and interpreted staves: \\
\ind very large staves, \\
\ind very stiff staves, \\
\ind which \inx[P]{Fimble-Thyle} \name{= Weden} painted, \\
\ind and the \inx[G]{yin-Reins} made, \\
\ind and Roft \name{= Weden} of the Reins carved.\evb\evg


\bvg\bva\alst{Ó}ðinn með \alst{ǫ̇}sum, \hld\ en fyr \alst{ǫ}lfum Dáinn, &
\ind \alst{D}valinn \alst{d}vergum fyrir, &
\ind \alst{Á}sviðr \alst{jǫ}tnum fyrir, &
\ind \edtrans{ek}{I}{\Bfootnote{The identity of the speaker is unclear; one would expect it to be Weden, but He is already named in line 1.}} ręist \alst{s}jalfr \alst{s}umar.\eva

\bvb \inx[P]{Weden} among the \inx[G]{Eese} and \inx[P]{Dowen} for the \inx[G]{Elves}; \\
\ind \inx[P]{Dwollen} for the \inx[G]{Dwarfs}; \\
\ind \inx[P]{Oswith} for the Ettins; \\
\ind I myself carved some.\evb\evg


\bvg\bva Vęitst, hvé \alst{r}ísta skal? \hld\ Vęitst, hvé \alst{r}áða skal? &
Vęitst, hvé \alst{f}áa skal? \hld\ Vęitst, hvé \alst{f}ręista skal? &
Vęitst, hvé \alst{b}iðja skal? \hld\ Vęitst, hvé \alst{b}lóta skal? &
Vęitst, hvé \alst{s}ęnda skal? \hld\ Vęitst, hvé \alst{s}óa skal?\eva

\bvb Knowest thou how one shall carve? Knowest thou how one shall read? \\
Knowest thou how one shall paint? Knowest thou how one shall try? \\
Knowest thou how one shall bid? Knowest thou how one shall \inx[C]{bloot}? \\
Knowest thou one shall send? Knowest thou how one shall \inx[C]{soo}?\footnoteB{The first four verbs seem to refer to runes: carving, interpreting, colouring (with blood?), and divining from them.

The latter four refer to sacrifice: supplication, worship, sending (the sacrifice or the prayer to the gods), and wasting the victim.  The following stanza repeats these verbs in what looks like a sacrificial context.  See further relevant Encyclopedia entries.}\footnoteB{The meter of this st. is unusual, but bears some resemblance to Vg 216 (the Högstena galder). TODO: Elaborate.}\evb\evg


\bvg\bva\alst{B}ętra ’s ȯ-\alst{b}eðit \hld\ an sé of·\alst{b}lótit, &
\ind ęy sér til \alst{g}ildis \alst{g}jǫf; &
bętra ’s ȯ-\alst{s}ęnt \hld\ an sé of·\alst{s}óit; &
\edtext{[...]}{\Bfootnote{For metrical reasons it is very likely that a line has been lost here.}}\eva

\bvb It is better unbid than over-\inx[C]{bloot}[blooted]; \\
\ind a gift always sees to a reward. \\
It is better unsent than over-\inx[C]{soo}[sooed]; \\
{[...]}.\footnoteB{An identical progression of four verbs suggests a close relation with the previous st. — The sense seems to be that it is better not to sacrifice at all than to sacrifice in excess, since even a small gift (to the gods) will be rewarded. A ritual cycle of gifts and rewards between men and the gods is also seen in other Indo-European pagan literatures. Compare the Sanskrit \emph{Dehí me, dádāmi te} ‘Give to me, I give to thee’ and Latin \emph{dō ut dēs} ‘I give that thou might give’.}\evb\evg


\bvg\bva Svá \alst{Þ}undr of ręist \hld\ fyr \alst{þ}jóða rǫk, &
þar’s \alst{u}pp of ręis, \hld\ es \alst{a}ptr of kom.\eva

\bvb So \inx[P]{Thound} \name{= Weden} did carve for the rakes of nations, \\
where up he rose as back he came.\footnoteB{TODO: A very cryptic st.}\evb\evg

\sectionline

\section{The Leed-Tally (147–165)}

This section of \Havamal, the so-called the Leed-Tally (\emph{Ljóðatal}), is not separated from the preceding section (which is marked out with a large initial), but is usually taken as separate since it is a self-contained list not much concerned with runes.  The speaker, Weden, addressing Loddfathomer, lists eighteen galders or spells he knows.  The spells themselves are not given; only their purpose.  They are aristocratic and Odinic in character and deal with such things as battle (3, 4, 5, 8, 11, 13), healing (galder 2, 12), countering sorcery (6, 10), controlling the elements (7, 9), and seduction (16, 17).  The eighteenth and last spell is a mystery; not even its purpose is told, and it is known only by Weden and his closest women.

There is a clear relation to other known Germanic galders.  The fourth bears a strong likeness to \Grougaldr\ 10, and its effect (removing fetters) is shared with the High German \MerseburgOne, an actual galder of that type.  The mysterious eighteenth spell finds an interesting parallel in the unknowable eighteenth question posed by Weden in \Vafthrudnismal\ 54.

\sectionline

\bvg\bva Ljóð \alst{þ}au kann’k, \hld\ es kann-at \alst{þ}jóðans kona &
\ind ok \alst{m}anns-kis \alst{m}ǫgr. &
\alst{H}jǫlp hęitir ęitt, \hld\ þat þér \alst{h}jalpa mun &
við \alst{s}orgum ok \edtrans{\alst{s}ǫkum}{sakes}{\Bfootnote{Legal charges, the first element of English \emph{sakeless}.}}, \hld\ ok \alst{s}útum gǫrv-ǫllum.\eva

\bvb Those \inx[C]{leed}[leeds] I know, which no king’s wife knows, \\
\ind and no man’s lad. \\
Help is one called, it will help thee \\
against sorrows and sakes, and all kinds of griefs.\evb\evg


\bvg\bva Þat kann’k \alst{a}nnat, \hld\ es \edtrans{þurfu \alst{ý}ta synir}{the sons of men need}{\Bfootnote{Cf. the similar wording in 166/2.}}, &
\ind þęir’s vilja \alst{l}ę́knar \alst{l}ifa.\eva

\bvb This I know second, which the sons of men need, \\
\ind who wish to live as leechers.\evb\evg


\bvg\bva Þat kann’k \alst{þ}riðja, \hld\ ef mér verðr \alst{þ}ǫrf mikil &
\ind \alst{h}apts við mína \alst{h}ęipt-mǫgu, &
\alst{ę}ggjar dęyfi’k \hld\ minna \alst{a}nd-skota, &
\ind bíta-t þęim \alst{v}ǫ́pn né \edtrans{\alst{v}ęlir}{staffs}{\Bfootnote{Plural of \emph{vǫlr}, here referring to the magic staff or sceptre used by witches and warlocks; the word \emph{vǫlva} ‘\inx[C]{wallow}’ (seeress, prophetess) is probably derived from this word.  The reading \emph{vélir} ‘wiles, tricks, deceits’ must be excluded for metrical reasons, since a c-verse in \Ljodahattr\ cannot end in a trochée.}}.\eva

\bvb This I know third, if I come in great need \\
\ind of hindrance against my feud-lads \ken{foes}; \\
I dull the edges of my opponents; \\
\ind for them bite no weapons nor staffs.\evb\evg


\bvg\bva Þat kann’k \alst{f}jórða, \hld\ ef mér \alst{f}yrðar bera &
\ind \alst{b}ǫnd at \alst{b}óg-limum, &
svá ek \alst{g}ęl, \hld\ at \alst{g}anga má’k, &
\ind sprettr mér af \alst{f}ótum \alst{f}jǫturr, &
\ind en af \alst{h}ǫndum \alst{h}apt.\eva

\bvb This I know fourth, if men bear \\
\ind bonds onto my shoulder-limbs: \\
\emph{so} I gale that I may walk; \\
\ind springs from my feet the fetter, \\
\ind and from my hands the bond.\footnoteB{Cf. \Grougaldr\ 10, which is very similar to the present stanza, and \MerseburgOne\ (edited below under Galders), a galder that seems to have actually been used for the purpose of removing fetters.}\evb\evg


\bvg\bva Þat kann’k \alst{f}imta, \hld\ ef sé’k af \alst{f}ári skotinn &
\ind \alst{f}lęin í \alst{f}olki vaða, &
flýgr-a svá \alst{st}int, \hld\ at \alst{st}ǫðvi’g-a’k, &
\ind ef hann \alst{s}jónum of \alst{s}é’k.\eva

\bvb This I know fifth, if I see a dangerously shot \\
\ind arrow in the troop wading: \\
it flies not so stiff that I may not stop it, \\
\ind if I see it with my sights.\evb\evg


\bvg\bva Þat kann’k \alst{s}étta, \hld\ \edtext{ef mik \alst{s}ę́rir þegn &
\ind ȧ \alst{r}ótum \edtrans{\alst{r}ás}{raw/sappy}{\Bfootnote{The normal form of this word is \emph{hrár} (cf. \Skirnismal\ 32), but the required alliteration with \emph{rótum} makes it impossible here.}} viðar}{\lemma{ef mik sę́rir þegn ȧ rótum rás viðar ‘if a thane wounds me on the roots of a raw/sappy tree’}\Bfootnote{I.e., “if a man carves a runic curse against me”.  The sappy wood was apparently of importance for the curse; cf. the curious account of \Grettissaga\ 79, where a hag curses Gretter in the following way: after finding a small tree and planing a small smooth surface onto a scorched side of it, she carves runes in its roots and reddens them with her own blood.  She then chants \inx[C]{galder}[galders] while walking counter-clockwise around it.  Lastly she pushes it out to sea, praying for it to drift to Gretter’s homestead and curse him.  Cf. also \Skirnismal\ 32 where Shirner goes to a \emph{hrár viðr} ‘raw/sappy tree’ to get a certain curse-object.}}, &
þann \alst{h}al, \hld\ es mik \alst{h}ęipta kveðr, &
\ind þann eta \alst{m}ęin hęldr an \alst{m}ik.\eva

\bvb This I know sixth, if a thane wounds me \\
\ind on the roots of a raw/sappy tree: \\
that man who sings hatred against me, \\
\ind \emph{him} the harms eat instead of me.\evb\evg


\bvg\bva Þat kann’k \alst{s}jaunda, \hld\ ef \alst{s}é’k hǫ́van loga &
\ind \alst{s}al of \alst{s}ess-mǫgum, &
\alst{b}rinnr-at svá \alst{b}ręitt, \hld\ at hǫ́num \alst{b}jargi’g-a’k; &
\ind þann kann’k \alst{g}aldr at \alst{g}ala.\eva

\bvb This I know seventh, if I see a high hall \\
\ind blazing over seat-lads \ken{warriors}: \\
it burns not so broadly that I may not save it\footnoteB{i.e. “if I see a hall burning with men trapped inside, no matter how large the flame is I can save both the hall and the men.”}— \\
\ind that galder I can gale.\evb\evg


\bvg\bva Þat kann’k \alst{á}tta, \hld\ es \alst{ǫ}llum es &
\ind \alst{n}yt-sam-ligt at \alst{n}ema, &
\alst{h}var’s \edtrans{\alst{h}atr}{hatred}{\Bfootnote{i.e. with regard to the father’s inheritance.}} vęx \hld\ með \alst{h}ildings sonum, &
\ind þat má’k \alst{b}ǿta \alst{b}rátt.\eva

\bvb This I know eighth, which for all men is \\
\ind useful to learn: \\
wherever hatred grows among a prince’s sons, \\
\ind it I may shortly mend.\evb\evg


\bvg\bva Þat kann’k \alst{n}íunda, \hld\ ef mik \alst{n}auðr of stęndr &
\ind at bjarga \alst{f}ari mínu ȧ \alst{f}loti, &
\alst{v}ind ek kyrri \hld\ \alst{v}ági ȧ &
\ind ok \alst{s}vę́fi’k allan \alst{s}ę́.\eva

\bvb This I know ninth, if I come in need \\
\ind of saving my ride on a floater \ken{ship}: \\
the wind I calm upon the wave, \\
\ind and put all the sea asleep.\evb\evg


\bvg\bva Þat kann’k \alst{t}íunda, \hld\ ef sé’k \edtrans{\alst{t}ún-riður}{town-rideresses}{\Bfootnote{The \emph{riður} ‘rideresses’ were witches who would leave their original human shapes or skins (\emph{hamir}) in order to fly around in the air tormenting and poisoning villagers.  Their original bodies would then be lying in a coma-like state, something like “astral projection”.  It was not the case that their whole mental faculties would disconnect from their bodies, but rather they would leave behind something of their humanity, which was thought to be inextricably linked to their human bodies.  Through his second sight, Weden was could see these riders, and would then use his superior magical wisdom to confuse them so that they would not be able to return to their human “home-shapes” or minds, but would instead be forced to stray as tormented bodyless ghosts; a cruel fate.  Weden also brags about tricking riders in \Harbardsljod\ 20.}} &
\ind \alst{l}ęika \alst{l}opti ȧ, &
ek svá \alst{v}inn’k, \hld\ at \edtrans{þę́r \alst{v}illar fara}{they (\emph{fem.}) go astray}{\Afootnote{emend.; \emph{þęir villir fara} ‘they (\emph{masc.}) go astray’ \Regius}} &
\ind sinna \alst{h}ęim-\alst{h}ama &
\ind sinna \alst{h}ęim-\alst{h}uga.\eva

\bvb This I know tenth, if I see \inx[G]{town-rideresses} \\
\ind playing aloft: \\
I work it so that they go astray \\
\ind of their home-\inx[C]{hame}[hames], \\
\ind of their home-minds.\evb\evg


\bvg\bva Þat kann’k \alst{ę}llipta, \hld\ ef skal’k til \alst{o}rrostu &
\ind \alst{l}ęiða \edtrans{\alst{l}ang-vini}{old friends}{\Bfootnote{In Germanic paganism the followers and protégés of a god were his friends (\emph{vinir}); already in \Beowulf\ we see that the Shieldings are called the \emph{Ing-wine} ‘friends of \inx[P]{Ing}’.  Two other places where it is used of Weden are \Grimnismal\ 54 and \Sonatorrek\ 22, where Eyel speaks about his friendship (\emph{vinan}) with Weden.  In \Hymiskvida\ 11 Thunder is the \emph{vinr ver-liða} ‘friend of manly retinues’.}}, &
und \alst{r}andir gęl’k, \hld\ en þęir með \alst{r}íki fara, &
\ind \alst{h}ęilir \alst{h}ildar til, &
\ind \alst{h}ęilir \alst{h}ildi frȧ, &
\ind koma þęir \alst{h}ęilir \alst{h}vaðan.\eva

\bvb This I know eleventh, if I shall into the fray \\
\ind lead old friends: \\
beneath the shields I gale, and they go with power \\
\ind healthy to the battle, \\
\ind healthy from the battle; \\
\ind they return healthy anywhence.\evb\evg


\bvg\bva Þat kann’k \alst{t}olpta, \hld\ ef sé’k ȧ \alst{t}ré uppi &
\ind \alst{v}áfa \alst{v}irgil-ná, &
svá ek \alst{r}íst \hld\ ok í \alst{r}únum fá’k, &
\ind at sá \alst{g}ęngr \alst{g}umi. &
\ind ok \alst{m}ę́lir við \alst{m}ik.\eva

\bvb This I know twelfth, if I see in a tree up high \\
\ind a gallow-corpse dangling: \\
so I carve and paint in the runes, \\
\ind that that man walks \\
\ind and speaks with me.\evb\evg


\bvg\bva Þat kann’k \alst{þ}rettánda \hld\ \edtext{ef skal’k \alst{þ}egn ungan &
\ind \alst{v}erpa \alst{v}atni ȧ,}{\lemma{ef skal’k þegn ungan verpa vatni ȧ ‘if on a young thane I shall sprinkle water’}\Bfootnote{A reference to the Heathen name-giving ceremony in which the infant would be sprinkled with water; cf. the attestations in \Rigsthula\ 7, 21, 34.}} &
mun-at hann \alst{f}alla \hld\ þótt í \alst{f}olk komi, &
\ind \alst{h}nígr-a sá \alst{h}alr fyr \alst{h}jǫrum.\eva

\bvb This I know thirteenth, if on a young thane \\
\ind I shall sprinkle water: \\
he will not fall though he come into battle; \\
\ind that warrior sinks not down before swords.\evb\evg


\bvg\bva Þat kann’k \alst{f}jórtánda, \hld\ ef skal’k \alst{f}yrða liði &
\ind \alst{t}ęlja \alst{t}íva fyr, &
\alst{ȧ}sa ok \alst{a}lfa \hld\ ek kann \alst{a}llra \edtrans{skil}{discernments}{\Bfootnote{Their unique traits.  Cf. \Hymiskvida\ 38, where the corresponding verb \emph{skilja} ‘to discern, understand’ is used in the context of god-lore.}}, &
\ind fár kann ȯ-\alst{s}notr \alst{s}vá.\eva

\bvb This I know fourteenth, if before a retinue of men \\
\ind I shall count forth the Tews: \\
of the Eese and Elves all I know the discernments; \\
\ind few unwise men can do so.\evb\evg


\bvg\bva\alst{Þ}at kann’k fimtánda, \hld\ es gól \alst{Þ}jóð-rǿrir &
\ind \alst{d}vergr fyr \alst{D}ęllings \alst{d}urum, &
\alst{a}fl gól \alst{ǫ́}sum, \hld\ en \alst{ǫ}lfum frama, &
\ind \alst{h}yggju \alst{H}ropta-tý.\eva

\bvb This I know fifteenth, which Thedrearer galed, \\
\ind the dwarf, before Delling’s doors. \\
Strength he galed for the Eese, and fame for the Elves, \\
\ind thought for Roft-Tew \name{= Weden}.\evb\evg


\bvg\bva Þat kann’k \alst{s}extánda, \hld\ ef vil’k hins \alst{s}vinna mans &
\ind hafa \alst{g}ęð allt ok \alst{g}aman, &
\alst{h}ugi \alst{h}vęrfi’k \hld\ \alst{h}vit-armri konu &
\ind ok \alst{s}ný’k hęnnar ǫllum \alst{s}efa.\eva

\bvb This I know sixteenth, if I will from the smart girl \\
\ind have her senses all, and pleasure: \\
the heart I change of the white-armed woman, \\
\ind and I twist her whole mind.\evb\evg


\bvg\bva Þat kann’k \alst{s}jautjánda \hld\ at mik \alst{s}ęint mun firrask &
\ind hit \alst{m}an-unga \alst{m}an.\eva

\bvb This I know seventeenth, that the girl-young girl \\
\ind will lately shun me.\evb\evg


\bvg\bva\alst{L}jóða þessa \hld\ munt \alst{L}oddfáfnir &
\ind lęngi \alst{v}anr \alst{v}esa; &
\ind þó sé þér \alst{g}óð ef \alst{g}etr, &
\ind \alst{n}ýt ef \alst{n}emr, &
\ind \alst{þ}ǫrf ef \alst{þ}iggr.\eva

\bvb These leeds wilt thou, Loddfathomer, \\
\ind for long be lacking! \\
Though they would be good for thee if thou get, \\
\ind useful if thou learn, \\
\ind needful if thou receive.\evb\evg


\bvg\bva Þat kann’k \alst{á}tjánda, \hld\ es \alst{ę́}va kęnni’k &
\ind \alst{m}ęy né \alst{m}anns konu, &
—\alst{a}llt es bętra \hld\ es \alst{ęi}nn of kann, &
\ind þat fylgir \alst{l}jóða \alst{l}okum— &
nema þęiri \alst{ęi}nni, \hld\ es \edtrans{mik \alst{a}rmi vęrr}{holds me in her arm}{\Bfootnote{A similar expression is also used \Volundarkvida\ 2.  The one who wraps Weden in her arm may be his wife, \inx[P]{Frie}.}}, &
\ind eða mín \alst{s}ystir \alst{s}éi.\eva

\bvb This I know eighteenth, which I never teach \\
\ind a maiden nor man’s woman— \\
everything is better which one alone knows; \\
\ind that follows the last of the leeds— \\
save for her alone who holds me in her arm, \\
\ind or is my sister.\evb\evg

\sectionline

\bvg\bva Nú eru \alst{H}áva mǫ́l kveðin \hld\ \alst{H}áva \alst{h}ǫllu í; &
\ind \alst{a}ll-þǫrf \alst{ý}ta sonum, &
\ind \alst{ȯ}-þǫrf \edtrans{\alst{jǫ}tna}{ettins}{\Afootnote{corrected in margin from \emph{ýta} ‘men’ \Regius}} sonum; &
hęill sá’s \edtext{\alst{k}vað, \hld\ hęill sá’s \alst{k}ann, &
\ind \alst{n}jóti sá’s \alst{n}am, &
\ind \alst{h}ęilir þęir’s \alst{h}lýddu}{\lemma{kvað, kann, nam, hlýddu ‘sang, knows, learned, heeded’}\Bfootnote{The implied subject is the speeches, i.e. ‘hail he who sang them, hail he who knows them,’ et.c.}}.\eva

\bvb Now are the High One’s speeches sung in the High One’s hall, \\
\ind of great use for the sons of men, \\
\ind of harm for the sons of ettins. \\
Hail he who sang; hail he who knows; \\
\ind may he use who learned; \\
\ind hail they who heeded!\evb\evg

\sectionline
% Weden, cultural context
	\bookStart{Speeches of Webthrithner}[Vafþrúðnismǫ́l]

\begin{flushright}%
\textbf{Dating} \parencite{Sapp2022}: C10th (0.894)

\textbf{Meter:} \Ljodahattr%
\end{flushright}%

\section{Introduction}

The \textbf{Speeches of Webthrithner} (\Vafthrudnismal) are found in full in \Regius; the latter half (from st. 20 onwards) in \AM.  Several stanzas are also cited in \Gylfaginning.

\subsection{Structure}

The poem essentially consists of a riddle contest between the god Weden and the ettin Webthrithner.  Far from being a loose collection of mythic lore, it has a tight structure and logical plan throughout.  The whole may be divided into 4 sections, first the prologue, where Weden takes counsel from his wife Frie and sets out for Webthrithner’s hall (sts. 1--10).  The remaining 3 sections form the contest, and consist of alternating stanzas where one part asks and the other answers.  They are distinguished from each other by means of repeated refrains in the question stanzas, and consist of Webthrithner’s 4 unnumbered questions (11--19), Weden’s 12 numbered questions (20--43), and Weden’s 6 unnumbered questions about the end times (44--55).

The following table illustrates the refrains; for stanza 40 see note there:

\begin{small}
\begin{center}
\begin{tabular}{|c c|}
  \hline
  11–17 & \emph{Sęg mér/þat, Gagnráðr, \hld\ alls á golfi vill | þíns of fręista frama} \\
  20–42 & \emph{Sęg þat (hit) \emph{N}(:a) \hld\ ... | ... Vaf-þrúðnir vitir} \\
  44–54 & \emph{Fjǫlð ek fór, \hld\ fjǫlð fręistaða’k, | fjǫlð ek ręynda ręgin} \\
  \hline
\end{tabular}
\end{center}
\end{small}

Something must be said on the numerology of the questions—it is hardly a coincidence that Weden asks exactly 18 questions, this being a multiple of the sacred number 9.  It is notable that another Odinic list, Leed-tally (sts. 147–165) of \Havamal, also has 18 items, especially that the 18th spell there, like the 18th question here, is a mystery known only to Weden himself.

\subsection{Summary}

Weden asks his wife, Frie, for counsel, as he is curious about the ancient wisdom which the ettin Webthrithner might possess (1). Frie expresses worry, since she considers Webthrithner stronger than all other ettins (2), but Weden says that he has travelled far and wide, and wishes to know what Webthrithner’s hall is like (3).  Frie wishes him good luck against the ettin (4) and he departs, to challenge Webthrithner’s \emph{orð-spęki} ‘word-wisdom’ (5). He arrives at the ettin’s hall and introduces himself (6); Webthrithner promptly declares that Weden will not come out of the hall unless he be wiser than him (7).  Weden introduces himself as Gainred, saying that he has travelled far in need of Webthrithner’s hospitality (8).  Webthrithner invites Gainred to sit down (9), who in turn utters a gnomic stanza (10) not unlike those of the first section of \Havamal.

Webthrithner begins by asking four mythological questions, each answered by Gainred in turn. The questions concern the horse that pulls the Day (11–12) and the one that pulls the Night (13–14), the river which divides the gods and ettins (15–16), and the plain where \inx[P]{Surt} and the gods will fight (17–18).

Webthrithner calls the guest learned and invites him to sit.  He declares that the loser of the contest must give his head (19).  The roles are now reversed, and Gainred poses twelve numbered questions to the ettin.  He asks about the origins of earth and heaven (20–21), of sun and moon (22–23), of day, night, and the phases of the moon (24–25), and of winter and summer (26–27); then about the earliest being, namely the ettin \inx[P]{Earyelmer} (28–29), his origins (30–31) and how he reproduced asexually (32–33). Gainred continues by asking what Webthrithner himself first remembers (34–35), about the origin of the wind (36–37), the god \inx[P]{Nearth} (38–39), Walhall and the Oneharriers (40–41), and where Webthrithner has learned all this wisdom (42–43).

The tone of the questions now changes, and Gainred asks six questions concerning the end times, all beginning with the same refrain.  He asks which humans will survive after the \inx[L]{Fimblewinter} (44–45), how the sun can rise after Fenrer has destroyed it (46–47), about some obscure maidens (48–49; see there), which Eese will survive after the flame of Surt goes out (50–51), and how Weden will die (52–53).  Finally, he asks the unknowable question: what did Weden speak in the ear of Balder before he was burned on the pyre? (54)

Webthrithner at last understands the identity of his challenger, since only Weden himself could know the answer to that question.  He laconically accepts his imminent death and the futility of his own wisdom; the poem ends with his admission that Weden will always be the wisest (55).

\sectionline

\section{The Speeches of Webthrithner}

\bvg\bva\speakernote{[Óðinn kvað:]}\mssnote{\Regius~7v/9}%
„Ráð mér nú \alst{F}rigg \hld\ alls mik \alst{f}ara tíðir &
\ind at \alst{v}itja \alst{V}af-þrúðnis; &
\edtext{\alst{f}or-vitni mikla \hld\ kveð’k mér ȧ \alst{f}ornum stǫfum &
\ind við þann hinn \alst{a}l-svinna \alst{jǫ}tun.}{\lemma{for-vitni \dots\ jǫtun. ‘Great \dots\ ettin.’}\Bfootnote{I.e. “I am very curious to learn his ancient words of wisdom.”  Cf. st. 55.}}“\eva

\bvb\speakernoteb{[\inx[P]{Weden} quoth:]}
“Counsel me now, \inx[P]{Frie}, as I long to journey \\
\ind to visit \inx[P]{Webthrithner}. \\
Great curiosity I have of ancient staves \\
\ind from that all-wise \inx[G]{Ettins}[ettin].”\evb\evg


\bvg\bva\speakernote{[Frigg kvað:]}\mssnote{\Regius~7v/12}%
„\alst{H}ęima lętja \hld\ mynda’k \alst{H}ęrja-fǫðr &
\ind ï \alst{g}ǫrðum \alst{g}oða; &
því-at \alst{ę}ngi \alst{jǫ}tun \hld\ hugða’k \alst{ja}fn-ramman &
\ind sęm \alst{V}af-þrúðni \alst{v}esa.“\eva

\bvb\speakernoteb{[Frie quoth:]}
“At home would I keep the Father of Hosts \ken*{= Weden}, \\
\ind in the yards of the Gods, \\
for no ettin have I judged to be \\
\ind as strong as Webthrithner.”\evb\evg


\bvg\bva\speakernote{[Óðinn kvað:]}\mssnote{\Regius~7v/13}%
„\alst{F}jǫlð ek \alst{f}ór, \hld\ \alst{f}jǫlð \alst{f}ręistaða’k, &
\ind fjǫlð ek \alst{r}ęynda \alst{r}ęgin; &
hitt \alst{v}il’k \alst{v}ita, \hld\ hvé \alst{V}af-þrúðnis &
\ind \alst{s}ala-kynni \alst{s}éi.“\eva

\bvb\speakernoteb{[Weden quoth:]}
“Much I journeyed, much I tried, \\
\ind much I tested the \inx[G]{Reins}! \\
This I wish to know: how Webthrithner’s \\
\ind halls may be.”\evb\evg


\bvg\bva\speakernote{[Frigg kvað:]}\mssnote{\Regius~7v/15}%
„\alst{H}ęill þú farir, \hld\ \alst{h}ęill þú aptr komir, &
\ind hęill ȧ \alst{s}innum \alst{s}éir; &
\alst{ǿ}ði þér dugi \hld\ hvar’s skalt, \alst{A}lda-fǫðr, &
\ind \alst{o}rðum mę́la \alst{jǫ}tun.“\eva

\bvb\speakernoteb{[Frie quoth:]}
“Whole mayst thou journey; whole mayst thou come back; \\
\ind whole mayst thou be on thy paths! \\
May thy wisdom avail thee where thou, Father of Men, \\
\ind with words shalt address the ettin!”\evb\evg


\bvg\bva\mssnote{\Regius~7v/17}%
\alst{F}ór þȧ Óðinn \hld\ at \alst{f}ręista orð-spęki &
\ind þess hins \alst{a}l-svinna \alst{jǫ}tuns; &
at hǫllu hann kom, \hld\ \edtext{\emph{es}}{\Afootnote{emend.; \emph{ok} \Regius}} átti \edtrans{Íms}{Ime’s}\Bfootnote{An unknown ettin.  The name is probably corrupt, since alliteration on \emph{h-} is required by the strongly stressed \emph{hǫllu} in the a-verse.  \textcite{FinnurEdda} emends to \emph{Hymir} ‘Hymer’.} faðir; &
\ind \alst{i}nn gekk \alst{Y}ggr þegar.\eva

\bvb Then journeyed Weden to test the word-wisdom \\
\ind of that all-wise ettin. \\
He came to the hall which Ime’s father \ken*{= Webthrithner} owned; \\
\ind \inx[P]{Ug} \name{= Weden} went soon inside.\evb\evg


\bvg\bva\speakernote{[Óðinn kvað:]}\mssnote{\Regius~7v/18}%
„\alst{H}ęill þú nú, Vaf-þrúðnir, \hld\ nú em’k ï \alst{h}ǫll kominn &
\ind ȧ þik \alst{s}jalfan \alst{s}éa; &
hitt vil’k \alst{f}yrst vita, \hld\ ef \alst{f}róðr séir &
\ind eða \alst{a}l-sviðr, \alst{jǫ}tunn.“\eva

\bvb\speakernoteb{[Weden quoth:]}%
“Hail thee now, Webthrithner! Now I am come into the hall \\
\ind to see thy very self! \\
This I wish first to know, if thou be learned \\
\ind or all-wise, ettin!”\evb\evg


\bvg\bva\speakernote{[Vafþrúðnir kvað:]}\mssnote{\Regius~7v/20}%
„Hvat ’s þat \alst{m}anna, \hld\ es ï \alst{m}ïnum sal &
\ind verpumk \alst{o}rði \alst{ȧ}? &
\alst{ú}t þú né kømr \hld\ \alst{ȯ}rum hǫllum frȧ, &
\ind nema þú inn \alst{s}notrari \alst{s}éir.“\eva

\bvb\speakernoteb{[Webthrithner quoth:]}%
“What sort of man is this who in \emph{my} hall \\
\ind throws a word at me? \\
Out wilt thou not come from our halls \\
\ind unless thou be the wiser man.”\evb\evg


\bvg\bva\speakernote{[Óðinn kvað:]}\mssnote{\Regius~7v/22}%
„\edtext{\alst{G}agnráðr}{\Bfootnote{The prose of \GylfMS\ has \emph{Gangráðr} ‘Gangred; Journey-adviser’ instead.}} hęiti’k, \hld\ nú em’k af \alst{g}ǫngu kominn, &
\ind \alst{þ}yrstr til \alst{þ}ïnna sala; &
\alst{l}aðar þurfi \hld\ hęf’k \alst{l}ęngi farit &
\ind ok þïnna \alst{a}nd-fanga, \alst{jǫ}tunn.“\eva

\bvb\speakernoteb{[Weden quoth:]}%
“\inx[P]{Gainred} I am called; now I am come from walking, \\
\ind thirsty, to thy halls. \\
In need of a welcome I’ve journeyed for long, \\
\ind and of thy reception, ettin!”\evb\evg


\bvg\bva\speakernote{[Vafþrúðnir kvað:]}\mssnote{\Regius~7v/24}%
„Hví þú þȧ, \alst{G}agnráðr, \hld\ mę́lisk af \alst{g}olfi fyrir? &
\ind far þú í \alst{s}ess í \alst{s}al; &
þȧ skal \alst{f}ręista, \hld\ hvárr \alst{f}lęira viti, &
\ind \alst{g}ęstr eða \edtrans{hinn \alst{g}amli þulr}{the old thyle}{\Bfootnote{Webthrithner himself, the thyle being the lorekeeper whose purpose it was to recite the old wisdom poems.  See Encyclopedia: \inx[C]{thyle}.}}.“\eva

\bvb\speakernoteb{[Webthrithner quoth:]}%
“Why then, Gainred, dost thou speak from the floor ahead? \\
\ind Take a seat in the hall! \\
Then it shall be tried which of the two might know more: \\
\ind the guest, or the old \inx[C]{thyle}.”\evb\evg


\bvg\bva\speakernote{[Óðinn kvað:]}\mssnote{\Regius~7v/26}%
„\alst{Ȯ}-auðigr maðr, \hld\ es til \alst{au}ðigs kømr, &
\ind \edtrans{mę́li \alst{þ}arft eða \alst{þ}ęgi}{ought to speak the needful or shut up}{\Bfootnote{Formulaic, this line occurs identically in \Havamal\ 19.}}; &
\alst{o}fr-mę́lgi mikil \hld\ hygg’k at \alst{i}lla geti &
\ind hvęim’s við \edtrans{\alst{k}ald-rifjaðan}{cold-ribbed}{\Bfootnote{Cold-hearted, cunning.}} \alst{k}ømr.“\eva

\bvb\speakernoteb{[Weden quoth:]}%
“An unwealthy man who to a wealthy comes \\
\ind ought to speak the needful or shut up. \\
Great over-speaking I think will bring ill \\
\ind for whomever by a cold-ribbed comes.”\evb\evg


\bvg\bva\speakernote{[Vafþrúðnir kvað:]}\mssnote{\Regius~7v/28}%
„Sęg mér, \alst{G}agnráðr, \hld\ alls ȧ \alst{g}olfi vill &
\ind \edtrans{þíns of \alst{f}ręista \alst{f}rama}{test thy furtherance}{\Bfootnote{I.e. “try your luck, see how far you get”.  Formulaic; cf. \Havamal\ 2.}}, &
hvé \alst{h}ęstr \alst{h}ęitir, \hld\ sá’s \alst{h}vęrjan dręgr &
\ind \alst{d}ag of \alst{d}rótt-mǫgu.“\eva

\bvb\speakernoteb{[Webthrithner quoth:]}%
“Tell me, Gainred, since on the floor thou wilt \\
\ind test thy furtherance, \\
what the horse is called which pulls every \\
\ind day over the lads of the folk \ken{men}.”\evb\evg


\bvg\bva\speakernote{[Óðinn kvað:]}\mssnote{\Regius~7v/30}%
„\alst{Sk}in-faxi hęitir, \hld\ es hinn \alst{sk}íra dręgr &
\ind \alst{d}ag of \alst{d}rótt-mǫgu; &
\alst{h}ęsta batstr \hld\ þykkir hann með \edtext{\emph{\alst{H}}ręið-gotum}{\Afootnote{metr. emend.; \emph{‘reið-gotom’} \Regius}\lemma{\emph{H}ręið-gotum ’Reth-Gots’}\Bfootnote{An old tribe name referring to the Eastern Gots around the Black Sea, apparently mentioned due to their location in the East.  The first element is unclear.  There may be a pun of sorts here, since \emph{goti} can mean both ‘Got’ and ‘horse’.}}; &
\ind ęy lýsir \alst{m}ǫn af \alst{m}ari.“\eva

\bvb\speakernoteb{[Weden quoth:]}%
“\inx[P]{Shinefax} is he called who pulls the bright \\
\ind day over the lads of the folk. \\
The best of horses he seems among the \inx[G]{Reth-Gots}; \\
\ind ever shines that stallion’s mane.”\evb\evg


\bvg\bva\speakernote{[Vafþrúðnir kvað:]}\mssnote{\Regius~7v/32}%
„Sęg þat, \alst{G}agnráðr, \hld\ alls ȧ \alst{g}olfi vill &
\ind þïns of \alst{f}ręista \alst{f}rama, &
hvé \alst{jó}r hęitir, \hld\ sá’s \alst{au}stan dręgr &
\ind \alst{n}ǫ́tt of \alst{n}ýt ręgin.“\eva

\bvb\speakernoteb{[Webthrithner quoth:]}%
“Tell this, Gainred, since on the floor thou wilt \\
\ind test thy furtherance, \\
what the steed is called which from the east does pull \\
\ind night over the useful \inx[G]{Reins}.”\evb\evg


\bvg\bva\speakernote{[Óðinn kvað:]}\mssnote{\Regius~7v/33}%
„\alst{H}rím-faxi \alst{h}ęitir, \hld\ es \alst{h}vęrja dręgr &
\ind \alst{n}ǫ́tt \edtext{of}{\Afootnote{emend.; \emph{ok} \Regius}} \alst{n}ýt ręgin; &
\alst{m}ėl-dropa \hld\ fęllir hann \alst{m}orgin hvęrjan; &
\ind \edtrans{þaðan kømr \alst{d}ǫgg of \alst{d}ala}{from thence comes the dew about the dales}{\Bfootnote{For another explanation of the origin of dew, see \Voluspa\ 18.}}.“\eva

\bvb\speakernoteb{[Weden quoth:]}%
“\inx[P]{Rimefax}\ is he called who pulls every \\
\ind night over the useful Reins. \\
Drool from his bit he lets fall each morning; \\
\ind from thence comes the dew about the dales.”\evb\evg


\bvg\bva\speakernote{[Vafþrúðnir kvað:]}\mssnote{\Regius~8r/1}%
„Sęg þat, \alst{G}agnráðr, \hld\ alls ȧ \alst{g}olfi vill &
\ind þíns of \alst{f}ręista \alst{f}rama, &
hvé \alst{ǫ́} hęitir, \hld\ sú’s dęilir með \alst{jǫ}tna sonum &
\ind \alst{g}rund, ok með \alst{g}oðum.“\eva

\bvb\speakernoteb{[Webthrithner quoth:]}%
“Tell this, Gainred, since on the floor thou wilt \\
\ind test thy furtherance, \\
what the river is called which divides the land \\
\ind between the sons of ettins and the gods.”\evb\evg


\bvg\bva\speakernote{[Óðinn kvað:]}\mssnote{\Regius~8r/2}%
„\edtrans{\alst{Í}fing}{Iving}{\Bfootnote{The border river is not known by this name from any other source, not even \Gylfaginning, which otherwise tends to relay even the most obscure lore.}} hęitir \alst{ǫ́}, \hld\ es dęilir með \alst{jǫ}tna sonum &
\ind \alst{g}rund, ok með \alst{g}oðum; &
\alst{o}pin rinna \hld\ hȯn skal umb \alst{a}ldr-daga; &
\ind verðr-at \alst{í}ss á \alst{ǫ́}u.“\eva

\bvb\speakernoteb{[Weden quoth:]}%
“\inx[L]{Iving} is the river called which divides the land \\
\ind between the sons of ettins and the gods. \\
Open shall it flow through its days of life; \\
\ind there forms no ice on that river.”\evb\evg


\bvg\bva\speakernote{[Vafþrúðnir kvað:]}\mssnote{\Regius~8r/3}%
„Sęg þat, \alst{G}agnráðr, \hld\ alls ȧ \alst{g}olfi vill &
\ind þïns of \alst{f}ręista \alst{f}rama, &
hvé \alst{v}ǫllr hęitir, \hld\ es finnask \alst{v}ígi at &
\ind \alst{S}urtr ok hin \alst{s}vǫ́su goð.“\eva

\bvb\speakernoteb{[Webthrithner quoth:]}%
“Tell this, Gainred, since on the floor thou wilt \\
\ind test thy furtherance, \\
what the plain is called where they find each other at war, \\
\ind \inx[P]{Surt} and the excellent Gods.”\evb\evg


\bvg\bva\speakernote{Óðinn:}\mssnote{\Regius~8r/4, \GylfMS}%
„\edtrans{\alst{V}ígríðr}{Wighride}{\Bfootnote{The plain where the gods will fight Surt at the \inx[L]{Rakes of the Reins}.}} hęitir \alst{v}ǫllr, \hld\ es finnask \alst{v}ígi at &
\ind \alst{S}urtr ok hin \alst{s}vǫ́su goð; &
\alst{h}undrað rasta \hld\ hann ’s ȧ \alst{h}vęrjan veg; &
\ind sá ’s þęim \alst{v}ǫllr \alst{v}itaðr.“\eva

\bvb\speakernoteb{Weden:}%
“\inx[L]{Wighride} is the plain called where they find each other at war, \\
\ind Surt and the excellent Gods. \\
A hundred \inx[C]{rest}[rests] it stretches in every way; \\
\ind for them that plain is marked out.”\evb\evg


\bvg\bva\speakernote{Vafþrúðnir:}\mssnote{\Regius~8r/6}%
„\alst{F}róðr est nú gęstr, \hld\ \alst{f}ar á bękk jǫtuns, &
\ind ok mę́lumk ï \alst{s}essi \alst{s}aman; &
\alst{h}ǫfði vęðja \hld\ vit skulum \alst{h}ǫllu ï &
\ind \alst{g}ęstr, of \alst{g}oð-spęki.“\eva

\bvb\speakernoteb{Webthrithner:}%
“Learned art thou now, guest; go on the ettin’s bench \\
\ind and let us speak in the seat together! \\
Wager a head shall we two in the hall, \\
\ind O guest, over god-wisdom!”\evb\evg

\sectionline

{\small \Regius\ here has the header \emph{capitulum} ‘(new) chapter’, and introduces st. 20 with a large initial.}

\sectionline

\bvg\bva\speakernote{Óðinn:}\mssnote{\Regius~8r/9, \AM~3r/1}%
„Sęg þat hit \alst{ęi}na, \hld\ ef þïtt \edtext{\alst{ǿ}ði}{\lemma{ǿði}\Bfootnote{The first word on fol. 3r of \AM; from this point we have the poem in both manuscripts.}} dugir &
\ind ok þú \alst{V}af-þrúðnir \alst{v}itir, &
hvaðan \edtext{\alst{jǫ}rð of kom, \hld\ eða \alst{u}pp-himinn}{\lemma{jǫrð \dots\ eða upp-himinn ‘Earth \dots\ or Up-heaven’}\Bfootnote{An old Common Germanic formulaic merism; see Index of formulae: \inx[F]{Earth and Upheaven}.}} &
\ind \alst{f}yrst, hinn \alst{f}róði jǫtunn.“\eva

\bvb\speakernoteb{Weden:}%
“Tell this one, if thy wisdom avails \\
\ind and thou, Webthrithner, oughtst to know, \\
whence Earth did come, or \inx[L]{Up-heaven}, \\
\ind first, O learned ettin.”\evb\evg


\bvg\bva\speakernote{Vafþrúðnir:}\mssnote{\Regius~8r/10, \AM~3r/2}%
„\edtext{Ór \alst{Y}mis holdi \hld\ vas \alst{jǫ}rð of skǫpuð, &
\ind en ór \alst{b}ęinum \alst{b}jǫrg, &
\edtrans{\alst{h}iminn ór \alst{h}ausi}{the heaven from the skull}{\Bfootnote{The heavens are understood as a dome, a view common to many ancient peoples. This also fits well with the floating clouds being Yimer’s brains, as told in \Grimnismal\ 42.}} \hld\ hins \alst{h}rím-kalda jǫtuns, &
\ind en \edtrans{ór \edtrans{\alst{s}vęita}{blood}{\Bfootnote{In poetry \emph{svęiti} ‘sweat’ almost always means ‘blood’. This is shared with OE \emph{swât}, as seen e.g. in \Beowulf\ 1286a: \emph{sweord \emph{swâte} fâh} ‘sword stained with “sweat”’, 2689b–2690: \emph{hé ge-blódegod wearð / sâwul-dríore; \hld\ \emph{swât} ýðum wéoll.} ‘he was bloodied in soul-gore; the “sweat” gushed in waves’.}} \alst{s}ę́r}{from his blood the sea}{\Bfootnote{According to \Gylfaginning\ 7, the slaying of Yimer produced so much blood that it drowned the whole race of Rime-Thurses save one; for this see st. 35 below. —%
Cf. \Sonatorrek\ 3/3: \emph{jǫtuns hals \hld\ undir þjóta} ‘the neck-wounds of the ettin \ken{seas} roar’, which attests that Yimer was slain by decapitation, the typical way of wasting beasts of sacrifice (so e.g. \Hymiskvida\ 15).  That this is not a mere literary construct is proven by the excavation of the Wiking Age Hove-steads (\emph{Hofstaðir}) on Iceland, where bulls were seasonally slain in what was undoubtedly ritual sacrifice: “The most likely reconstruction from the forensics of the skulls requires at least a two-person team, one of whom struck the animal between the eyes (effectively killing it and certainly stunning it into momentary immobility) while the second swung a fairly broad-bladed axe at the neck or base of the skull for a beheading stroke.”  After the slaying (and presumed feasting on the meat), their skulls were displayed for a prolonged period of time (\citeauthor{Gavin2007}, \citeyear{Gavin2007}, p. 23).  \citeauthor{Gavin2007} note that this was not the usual manner of slaughtering animals on Iceland, and even has practical downsides compared to a slower cutting of the throat, like splintered bones and damage to the cutting blade.  On the other hand, the swift beheading and flow of blood would have great dramatic effect, and, what the authors neglect to mention, clearly reenact the slaying of Yimer: the separation of the skull (heaven) from the body (earth), and the great flow of blood (sea-water) from the neck-wound, lastly the burial of the body in the earth, and the display of the skull on high to symbolize the heaven.}}.%NOTE: this is a hack for parencite which for whatever reason doesn't work inside a Bfootnote with multiple authors. reledmac...
“}{\lemma{ALL}\Bfootnote{The gods sacrificed Yimer and created the world from his body, as told more fully in \Grimnismal\ 41–42; for the deeper religious significance of this myth see note to \Grimnismal\ 43. — The whole st. bears very close resemblance to \Grimnismal\ 41; ll. 1 and 4 here are identical to ll. 1–2 there, and ll. 2 and 3a here are clearly related to ll. 3a and 4 there.  Still, the sts. are distinct enough that the one cannot be a direct scribal copy of the other, and the relationship is more likely to be oral.  Both have probably been composed in the same West Norwegian milieu, deriving from an older Common Germanic tradition (cf. the Hymn from Wessobrunn under Poetry on Christian Subjects).}}%
\eva

\bvb\speakernoteb{Webthrithner:}%
“From \inx[P]{Yimer}’s flesh was the earth shaped, \\
\ind and from his bones the mountains; \\
the heaven from the skull of that rime-cold ettin, \\
\ind and from his blood the sea.”\evb\evg


\bvg\bva\speakernote{Óðinn:}\mssnote{\Regius~8r/12, \AM~3r/3}%
„Sęg þat \alst{a}nnat, \hld\ ef þïtt \alst{ǿ}ði dugir &
\ind ok þú \alst{V}af-þrúðnir \alst{v}itir, &
hvaðan \alst{M}ȧni of kom, \hld\ svá’t fęrr \alst{m}ęnn yfir, &
\ind eða \alst{S}ól hit \alst{s}ama.“\eva

\bvb\speakernoteb{Weden:}%
“Tell this other, if thy wisdom avails, \\
\ind and thou, Webthrithner, oughtst to know, \\
whence Moon did come who journeys over men, \\
\ind or Sun likewise.”\evb\evg


\bvg\bva\speakernote{Vafþrúðnir:}\mssnote{\Regius~8r/13, \AM~3r/4}%
„\edtrans{\alst{M}undil-fǿri}{Mundlefarer}{\Bfootnote{An otherwise unknown figure; see Index for etymology, which likens the cosmos to a Wiking Age flour-mill turned by a handle.}} hęitir, \hld\ hann ’s \alst{M}ȧna faðir &
\ind ok svá \alst{S}ólar hit \alst{s}ama; &
\alst{h}imin \alst{h}verfa \hld\ þau skulu \alst{h}vęrjan dag &
\ind \edtrans{\alst{ǫ}ldum at \alst{á}r-tali}{for mankind’s tally of years}{\Bfootnote{According to \Voluspa\ 6 the Gods gave names to night, the moon-phases, morning, midday, afternoon, and evening \emph{ǫ́rum at tęlja} ‘the years for to tally’. — Numerous examples of the chronological reckoning of the Heathen Icelanders are found in Are’s Book of Icelanders and in the Book of Landtakings.  Both of them relate the years to the Christian Common Era, but they must originally have been based on the reigns of kings, of which many examples are found in those two books.}}.“\eva

\bvb\speakernoteb{Webthrithner:}%
“\inx[P]{Mundlefare} he is called—he is the father of Moon, \\
\ind and so of Sun likewise. \\
Turn round heaven shall they every day, \\
\ind for mankind’s tally of years.”\evb\evg


\bvg\bva\speakernote{Óðinn:}\mssnote{\Regius~8r/15, \AM~3r/6}%
„\alst{S}ęg þat hit þriðja, \hld\ alls þik \alst{s}vinnan kveða &
\ind ok þú \alst{V}af-þrúðnir \alst{v}itir, &
hvaðan \alst{D}agr of kom, \hld\ sá’s fęrr \alst{d}rótt yfir, &
\ind eða \alst{N}ǫ́tt með \alst{n}iðum.“\eva

\bvb\speakernoteb{Weden:}%
“Tell this third, since they call thee wise, \\
\ind and thou, Webthrithner, oughtst to know, \\
whence Day did come who journeys over the folk, \\
\ind or Night with the moon-phases.”\evb\evg


\bvg\bva\speakernote{Vafþrúðnir:}\mssnote{\Regius~8r/17, \AM~3r/8}%
„\alst{D}ęllingr hęitir, \hld\ hann ’s \alst{D}ags faðir, &
\ind en \alst{N}ǫ́tt vas \alst{N}ǫrvi borin; &
\edtrans{\alst{n}ý ok \alst{n}ið}{The waxing and waning}{\Bfootnote{The phases of the moon, by which months were reckoned.  Cf. \Voluspa\ 6.}} \hld\ skópu \alst{n}ýt ręgin &
\ind \alst{ǫ}ldum at \alst{á}r-tali.“\eva

\bvb\speakernoteb{Webthrithner:}%
“\inx[P]{Delling} he is calledL; he is the father of \inx[P]{Day}, \\
\ind but \inx[P]{Night} was born to \inx[P]{Narrow}. \\
The waxing and waning did the useful Reins create \\
\ind for mankind’s tally of years.”\evb\evg


\bvg\bva\speakernote{Óðinn kvað:}\mssnote{\Regius~8r/18, \AM~3r/9}%
„Sęg þat hit \alst{f}jórða, \hld\ alls þik \alst{f}róðan kveða, &
\ind ok þú \alst{V}af-þrúðnir \alst{v}itir, &
hvaðan \alst{v}etr of kom \hld\ eða \alst{v}armt sumar &
\ind \alst{f}yrst með \alst{f}róð ręgin.“\eva

\bvb\speakernoteb{Weden quoth:}%
“Tell this fourth, since they call thee learned, \\
\ind and thou, Webthrithner, oughtst to know, \\
whence winter did come, or warm summer, \\
\ind first, amidst the learned Reins.”\evb\evg


\bvg\bva\speakernote{Vafþrúðnir:}\mssnote{\Regius~8r/20, \AM~3r/10}%
„\alst{V}ind-svalr hęitir, \hld\ hann’s \alst{V}etrar faðir, &
\ind en \alst{S}vǫ́suðr \alst{S}umars.“ &
\edtext{[...]}{\Bfootnote{A second half of the st. seems to be missing; its contents are entirely unknown.  No gap is indicated in the mss.}}\eva

\bvb\speakernoteb{Webthrithner:}%
“\inx[P]{Windswoll}\ is he called; he is \inx[P]{Winter}’s father; \\
\ind but \inx[P]{Sosuth}\ [is] \inx[P]{Summer}’s.”\evb\evg


\bvg\bva\speakernote{Óðinn kvað:}\mssnote{\Regius~8r/21, \AM~3r/11}%
„Sęg þat hit \alst{f}imta, \hld\ alls þik \alst{f}róðan kveða, &
\ind ok þú \alst{V}af-þrúðnir \alst{v}itir, &
\edtext{hvęrr \alst{ȧ}sa \alst{ę}ldstr \hld\ eða \alst{Y}mis niðja &
\ind \alst{y}rði í \alst{á}r-daga.}{\lemma{hvęrr \dots\ ár-daga ‘who \dots\ days of yore.’}\Bfootnote{I.e. “which was the very first being?”  Cf. the question on the cryptic C9th Malt Stone (DR NOR1988;5): \textbf{huaʀisi : alistiąsa}, perhaps \emph{Hvaʀ es inn ęlisti ȧsa?} ‘Who is the eldest of the Eese?’}}“\eva

\bvb\speakernoteb{Weden quoth:}%
“Tell this fifth, since they call thee learned, \\
\ind and thou, Webthrithner, oughtst to know, \\
who oldest of the \inx[G]{Eese} or of Yimer’s kinsmen \ken{ettins} \\
\ind arose in days of yore.”\evb\evg


\bvg\bva\speakernote{Vafþrúðnir:}\mssnote{\Regius~8r/22, \AM~3r/12}%
„\alst{Ø}r-ófi vetra \hld\ áðr vę́ri \alst{jǫ}rð of skǫpuð, &
\ind þȧ vas \alst{B}er-gęlmir \alst{b}orinn, &
\alst{Þ}rúð-gęlmir \hld\ vas \alst{þ}ess faðir, &
\ind en \alst{Au}r-gęlmir \alst{a}fi.“\eva

\bvb\speakernoteb{Webthrithner:}%
“Uncountable winters before the Earth was created, \\
\ind then was \inx[P]{Bareyelmer} born. \\
\inx[P]{Thrithyelmer} was that one’s father, \\
\ind and \inx[P]{Earyelmer} the grandfather.”\evb\evg


\bvg\bva\speakernote{Óðinn kvað:}\mssnote{\Regius~8r/23, \AM~3r/14, \GylfMS}%TODO: This stanza is not found in \U.
„\edtext{\alst{S}ęg þat hit \alst{s}étta, \hld\ alls þik \alst{s}vinnan kveða, &
\ind ok þú \alst{V}af-þrúðnir \alst{v}itir,}{\lemma{Sęg \dots\ vitir ‘Tell \dots\ know’}\Bfootnote{om. \GylfMS}} &
hvaðan \alst{Au}r-gęlmir kom \hld\ með \alst{jǫ}tna sonum &
\ind \alst{f}yrst, hinn \alst{f}róði jǫtunn.“\eva

\bvb\speakernoteb{Weden quoth:}%
“Tell this sixth, since they call thee wise, \\
\ind and thou, Webthrithner, oughtst to know, \\
whence Earyelmer came amidst the sons of ettins, \\
\ind first, O learned ettin.”\evb\evg


\bvg\bva\speakernote{Vafþrúðnir:}\mssnote{\Regius~8r/25, \AM~3r/15, \GylfMS}%
„\edtext{\alst{Ó}r \alst{É}li-vǫ́gum \hld\ stukku \alst{ęi}tr-dropar, &
\ind svá \alst{ó}x unds ór varð \alst{jǫ}tunn; &
\edtext{þar \alst{ȯ}rar \alst{ę́}ttir \hld\ kómu \alst{a}llar saman; &
\ind því’s \edtrans{þat}{it}{\Bfootnote{i.e. the ettin race.}} \alst{ę́} \alst{a}lt til \alst{a}talt.“}{\lemma{órar \dots\ atalt ‘Our \dots\ fierce’}\Bfootnote{so \GylfMS; om. \Regius\AM.}}}{\lemma{ALL}\Bfootnote{Over æons the splashing venom-drops combined until they formed a sentient being: this was Earyelmer, whom \Gylfaginning\ identifies with \inx[P]{Yimer}.  In \Gylfaginning\ 5 Snorre cites this stanza and the latter half of 30 in support of his lengthy and embellished creation narrative, but it is not certain that is what the older poet had in mind.

The Ilewaves are probably a reflex of the chaotic primeval Waters found in many West Eurasian mythologies, including Genesis 1:1–3 and \Rigveda\ 10.129.  Of these two foundational religious sources the latter is closer to the present stanza, and probably holds the more archaic conception.  Where we find in the Jewish narrative a proper \emph{creation}; at the very beginning of time God’s spirit is on the Waters and He makes the light shine over them, we find in these two Indo-European texts a \emph{spontaneous emergence} of a single primeval entity long before the Gods are born—here from the violent splashing of venom, in \Rigveda\ 10.129.3 from “the power of heat” (\emph{tápasaḥ mahinā́}).  This entity in turn asexually begets sexual beings—here through rubbing his limbs together, in \Rigveda\ 10.129.4 simply giving rise to “desire” (\emph{kā́ma}) which serves as the “primal seed of thought” (\emph{mánasaḥ rétaḥ prathamám})—and it is from these that the world is populated.}}\eva

\bvb\speakernoteb{Webthrithner:}%
“From the \inx[L]{Ilewaves} splashed venom-drops; \\
\ind so it grew until it formed an ettin. \\
Our lineages came there all together, \\
\ind thus it is ever all too fierce.”\evb\evg


\bvg\bva\speakernote{Óðinn kvað:}\mssnote{\Regius~8r/26, \AM~3r/16}%
„\alst{S}ęg þat hit \alst{s}jaunda, \hld\ alls þik \alst{s}vinnan kveða, &
\ind ok þú \alst{V}af-þrúðnir \alst{v}itir, &
hvé sá \alst{b}ǫrn gat \hld\ hinn \edtrans{\alst{b}aldni}{stubborn}{\Afootnote{so \AM; \emph{aldni} ‘the aged’ \Regius}} jǫtunn, &
\ind es hann hafði-t \alst{g}ýgjar \alst{g}aman.“\eva

\bvb\speakernoteb{Weden quoth:}%
“Tell this seventh, since they call thee wise, \\
\ind and thou, Webthrithner, oughtst to know, \\
how that one begot children, the stubborn ettin, \\
\ind when he knew not troll-woman’s pleasure.”\evb\evg


\bvg\bva\speakernote{Vafþrúðnir kvað:}\mssnote{\Regius~8r/27, \AM~3r/17}%
„\edtext{Und \alst{h}ęndi vaxa \hld\ kvǫ́ðu \alst{h}rím-þursi &
\ind \alst{m}ęy ok \alst{m}ǫg saman; &
\alst{f}ótr við \alst{f}ǿti}{\lemma{Und hęndi \dots\ fótr við fǿti ‘In the hand \dots\ Foot against foot’}\Bfootnote{The image is masturbatory and monstrous.  The stanza is paraphrased in \Gylfaginning\ 5: \emph{En svá er sagt, at þá er hann svaf, fekk hann sveita. Þá óx undir vinstri hendi honum maðr ok kona, ok annarr fótr hans gat son við ǫðrum, en þaðan af kómu ę́ttir.} ‘But so is said, that when he slept he began to sweat.  Then grew within his left hand a man and a woman, and one foot of his begat a son by the other, and thereof come the lineages [of Ettins].’}} \hld\ gat hins \alst{f}róða jǫtuns &
\ind \alst{s}ex-hǫfðaðan \alst{s}on.“\eva

\bvb\speakernoteb{Webthrithner quoth:}%
“In the hand of the \inx[G]{Rime-Thurses}[rime-thurse], they said, did grow \\
\ind a maiden and a lad together. \\
Foot against foot begat for the learned ettin \\
\ind a six-headed son.”\evb\evg


\bvg\bva\speakernote{Óðinn kvað:}\mssnote{\Regius~8r/29, \AM~3r/18}%
„\edtrans{Sęg þat hit ǫ́ttunda, \hld\ alls þik fróðan kveða,}{Tell this eighth, since they call thee learned}{\Bfootnote{This line lacks the required alliteration, but may easily be supplied by replacing \emph{alls þik fróðan kveða} with \emph{ef þïtt ǿði dugir} from sts. 20 and 22, or \emph{alls þik svinnan kveða} from 24.}} &
\ind ok þú \alst{V}af-þrúðnir \alst{v}itir, &
hvat \alst{f}yrst of mant \hld\ eða \alst{f}ręmst of vęitst, &
\ind þú est \alst{a}l-sviðr \alst{jǫ}tunn.“\eva

\bvb\speakernoteb{Weden quoth:}%
“Tell this eighth, since they call thee learned, \\
\ind and thou, Webthrithner, oughtst to know \\
what thou first recallest or foremost knowest— \\
\ind thou art all-wise, ettin!”\evb\evg


\bvg\bva\speakernote{Vafþrúðnir kvað:}\mssnote{\Regius~8r/30, \AM~3r/19, \GylfMS}%
„\alst{Ø}r-ófi vetra \hld\ áðr vę́ri \alst{jǫ}rð of skǫpuð, &
\ind þȧ vas \alst{B}er-gęlmir \alst{b}orinn; &
þat ek \alst{f}yrst of man, \hld\ \edtext{es hinn \alst{f}róði jǫtunn &
\ind ȧ vas \alst{l}úðr of \alst{l}agiðr.}{\lemma{es hinn fróði jǫtunn / ȧ vas lúðr of lagiðr ‘when the learned ettin on the tree-trunk was laid’}\Bfootnote{An obscure mythological reference.

\Gylfaginning\ explains it in the following way: the sons of \inx[P]{Byre} (that is, \inx[P]{Weden}, \inx[P]{Will} and \inx[P]{Wigh}) slew Yimer and when he died so much blood flowed from his wounds that the whole race of Ettins was drowned save for Bareyelmer and his household, who survived by getting up on his \emph{lúðr}.  This is clearly a variant of the Great Flood or Deluge myth.  It may have been found even among the Scandinavians, but it may also be Snorre’s invention based on the Bible, in which case the present stanza was about as obscure to him as it is to us.

In Old Norse prose \emph{lúðr} usually means ‘trumpet, blowing horn’, less commonly ‘flour-bin’; the underlying sense seems to be ‘hollowed-out wood’, which is why it is presently translated as “tree-trunk”.  Considering the transitive nature of Bareyelmer being laid (\emph{of lagiðr}) upon it, the stanza could be read as speaking of a ship burial, so that the first thing Webthrithner remembers is Bareyelmer’s funeral.}}“\eva

\bvb\speakernoteb{Webthrithner quoth:}%
“Uncountable winters before the Earth was created, \\
\ind then was Bareyelmer born. \\
It I first remember, when the learned ettin \\
\ind on the tree-trunk was laid.”\evb\evg


\bvg\bva\speakernote{Óðinn kvað:}\mssnote{\Regius~8r/32, \AM~3r/21}%
„\alst{S}ęg þat hit níunda, \hld\ alls þik \alst{s}vinnan kveða, &
\ind ok þú \alst{V}af-þrúðnir \alst{v}itir, &
hvaðan \alst{v}indr of kømr \hld\ svá’t fęrr \alst{v}ág yfir, &
\ind \edtrans{ę́ męnn hann \alst{s}jalfan of \alst{s}éa}{ever do men see hisself}{\Bfootnote{Perhaps a reference to sea which is never perfectly still, so that the wind is always seen on the waves.  It is also possible that a negative clitic \emph{-t} has been lost from the verb \emph{séa} ‘see’, in which case the line would read “\emph{never} do men see hisself”.}}.“\eva

\bvb\speakernoteb{Weden quoth:}%
“Tell this ninth, since they call thee wise, \\
\ind and thou, Webthrithner, oughtst to know: \\
whence the wind comes which fares over the wave— \\
\ind ever do men see hisself.”\evb\evg


\bvg\bva\speakernote{Vafþrúðnir:}\mssnote{\Regius~8r/34, \AM~3r/22, \GylfMS}%
„\alst{H}rę́-svęlgr \alst{h}ęitir, \hld\ es sitr ȧ \alst{h}imins ęnda, &
\ind \alst{jǫ}tunn ï \alst{a}rnar ham; &
af hans \alst{v}ę̇ngjum \hld\ kveða \alst{v}ind koma &
\ind \alst{a}lla męnn \alst{y}fir.“\eva

\bvb\speakernoteb{Webthrithner:}%
“\inx[P]{Rawswallower} is he called who sits at heaven’s end; \\
\ind an ettin in an eagle’s \inx[C]{hame}. \\
From his wings they say that the wind comes \\
\ind over all men.”\evb\evg


\bvg\bva\speakernote{[Óðinn kvað:]}\mssnote{\Regius~8v/1, \AM~3r/24}%
„Sęg þat hit \alst{t}íunda, \hld\ alls þú \alst{t}íva rǫk &
\ind ǫll \alst{V}afþrúðnir \alst{v}itir, &
hvaðan Njǫrðr of kom \hld\ með ȧsa sonum; &
\edtrans{\alst{h}ofum ok \alst{h}ǫrgum}{hoves and harrows}{\Bfootnote{A formulaic merism, see note to \Voluspa\ 7 for other occurrences.

This stanza seems to be referring to the large count of cultic places named after Nearth—\textcite{Brink2007} counts 13 attestations in Norway, 17 in Sweden, 3 in Denmark; in addition there are a few on Iceland (TODO).  For Nearth’s harrow cf. \Grimnismal\ 16, where it is said that Nearth \emph{rę́ðr hǫ́-timbruðum hǫrgi} ‘rules a high-timbered harrow’.  Also of interest is \Lokasenna\ 51, where a goddess speaks of her \emph{vé ok vangar} ‘wighs and wongs’, two terms common in cultic place names.  The underlying theological understanding seems to be that the god is physically present as a ruler of his shrine.}} \hld\ rę́ðr \alst{h}und-mǫrgum &
\ind ok varð-at \alst{ǫ́}sum \alst{a}linn.“\eva

\bvb\speakernoteb{[Weden quoth:]}%
“Tell this tenth, since thou the \inx[L]{Rakes of the Tews} \\
\ind all, Webthrithner, oughtst to know, \\
whence \inx[P]{Nearth} did come amidst the sons of the \inx[G]{Eese}; \\
\inx[C]{hove}[hoves] and \inx[C]{harrow}[harrows] he rules a hundred-many, \\
\ind and he was not by the Eese begotten.”\evb\evg


\bvg\bva\speakernote{[Vafþrúðnir kvað:]}\mssnote{\Regius~8v/3, \AM~3r/26}%
\edtext{„Ï \alst{V}ana-hęimi \hld\ skópu hann \alst{v}ís \edtrans{ręgin}{Reins}{\Bfootnote{\emph{ręgin} ‘the Reins, Powers’ is generally used simply to refer to the gods as a collective, but here seems to refer specifically to the \inx[G]{Wanes} in opposition to the \inx[G]{Eese}.}} &
\ind ok sęldu at \alst{g}íslingu \alst{g}oðum, &
ï \edtrans{\alst{a}ldar rǫk}{the Rakes of the Age}{\Bfootnote{The \inx[L]{Rakes of the Reins}, the End Times.}} \hld\ hann mun \alst{a}ptr koma &
\ind hęim með \alst{v}ísum \alst{v}ǫnum.“}{\lemma{ALL}\Bfootnote{Cf. \Gylfaginning, \YnglingaSaga\ TODO.}}\eva

\bvb\speakernoteb{[Webthrithner quoth:]}%
“In \inx[L]{Waneham} the wise \inx[G]{Reins} created him, \\
\ind and sold him as a hostage to/for the gods. \\
In the \inx[L]{Rakes of the Age} he will come back \\
\ind home amidst the wise \inx[G]{Wanes}.”\evb\evg


\bvg\bva\speakernote{[Óðinn kvað:]}\mssnote{\Regius~8v/5, \AM~3r/28}%
\edtext{„Sęg þat hit \alst{ę}llipta, \hld\ \emph{ef þïtt \alst{ǿ}ði dugir} &
\ind \emph{ok þú \alst{V}af-þrúðnir \alst{v}itir,} &
hvar \emph{\alst{a}llir} \hld\ \alst{ý}tar tu̇num ï &
\ind \alst{h}ǫggvask \alst{h}vęrjan dag.}{\lemma{ALL}\Bfootnote{This question-stanza is malformed in both \Regius\ and \AM\ and thus has to be partly reconstructed on the basis of st. 41.  The ms. preservation of 40–41 is as follows:

All four mss. of \Gylfaginning\ attest st. 41 with no textual variants.  \Regius\ has one complete stanza, which is clearly a mix between the question and the answer: \emph{Sęg-ðu þat hit ęllipta, hvar ýtar tu̇num ï hǫggvask hvęrjan dag? Val þęir kjósa ok ríða vígi frȧ sitja męirr of sáttir saman.} (normalised.)  \AM\ has only the very beginning of st. 40 (“Tell this eleventh”), followed by the full st. 41: \emph{Sęg þat hit ęllipta allir ęins hęrjar Óðins tu̇num ï hǫggvask hvęrjan dag. Val þęir kjósa ok ríða vígi frȧ sitja męirr of sáttir saman.} (norm.)  Although \Regius\ has a complete question-stanza it stands out by lacking a refrain in the first two lines, something found in all other questions in the poem (see Introduction); it also has no corresponding answer-stanza.

In order to restore stanza 40, the following conjectural reconstruction has been undertaken in the pres. ed.: in lines 1a–2 the refrain \emph{ef þïtt ǿði dugir ok þú Vaf-þrúðnir vitir} ‘if thy wisdom avails, and thou, Webthrithner, oughtst to know,’ has been inserted from sts. 20 and 22, which also have ordinal numbers alliterating with vowels; in line 3a the word \emph{allir} ‘all’ has been inserted from 41 to get vowel-alliteration with \emph{ýtar}.}}“\eva

\bvb\speakernoteb{[Weden quoth:]}%
“Tell this eleventh, if thy wisdom avails, \\
\ind and thou, Webthrithner, oughtst to know, \\
where all men in yards \\
\ind strike at each other every day.”\evb\evg


\bvg\bva\speakernote{[Vafþrúðnir kvað:]}\mssnote{\AM~3r/28, \GylfMS}%
„\alst{A}llir \edtext{\alst{ęi}n-hęrjar}{\Afootnote{so \GylfMS; \emph{ęins hęrjar} \AM}} \hld\ \alst{Ó}ðins tu̇num ï &
\ind \alst{h}ǫggvask \alst{h}vęrjan dag; &
\edtrans{\alst{v}al þęir kjósa}{The slain they choose}{\Bfootnote{It is from this verbal phrase that the female agent noun \emph{val-kyrja} ‘\inx[G]{walkirries}[walkirrie]’ is derived.}} \hld\ ok ríða \alst{v}ígi frȧ, &
\ind \alst{s}itja męirr of \alst{s}áttir \alst{s}aman.“\eva

\bvb\speakernoteb{[Webthrithner quoth:]}%
“All the \inx[G]{Oneharriers} in Weden’s yards \\
\ind strike at each other every day. \\
The slain they choose and they ride from the fray; \\
\ind then they sit at peace together.”\evb\evg


\bvg\bva\speakernote{[Óðinn kvað:]}\mssnote{\Regius~8v/6, \AM~3v/1}%
„Sęg þat hit \alst{t}olpta, \hld\ hví þú \alst{t}íva rǫk &
\ind ǫll \alst{V}af-þrúðnir \alst{v}itir? &
Frȧ \alst{jǫ}tna ru̇num \hld\ ok \alst{a}llra goða &
\ind þú hit \alst{s}annasta \alst{s}ęgir, &
\ind hinn \alst{a}l-svinni \alst{jǫ}tunn.“\eva

\bvb\speakernoteb{[Weden quoth:]}%
“Tell this twelfth, why thou the Rakes of the Tews \\
\ind all, Webthrithner, shouldst know? \\
From the \inx[C]{rune}[runes] of the ettins and of all the gods \\
\ind dost thou speak the most truly, \\
\ind O all-wise ettin!”\evb\evg


\bvg\bva\speakernote{[Vafþrúðnir kvað:]}\mssnote{\Regius~8v/8, \AM~3v/2}%
„Frȧ \alst{jǫ}tna ru̇num \hld\ ok \alst{a}llra goða &
\ind ek kann \alst{s}ęgja \alst{s}att, &
\ind því-at \alst{h}vęrn hęf’k \alst{h}ęim of komit, &
\edtext{\alst{n}íu kom’k hęima \hld\ fyr \alst{n}ifl-hęl neðan; &
\ind \alst{h}inig dęyja ór \alst{h}ęlju \alst{h}alir.}{\lemma{níu \dots\ halir. ‘Into nine \dots\ of Hell.’}\Bfootnote{Perhaps lower infernal underworlds.  \textcite{FinnurEdda}\ considers \emph{ór hęlju} ‘out of Hell’ a later interpolation, probably for metrical reasons.}}“\eva

\bvb\speakernoteb{[Webthrithner quoth:]}%
“From the runes of the ettins and of all the gods \\
\ind I can speak truly, \\
\ind for I have come into each \inx[C]{Home}. \\
Into nine Homes I came beneath \inx[L]{Nivelhell}; \\
\ind that way men die out of \inx[L]{Hell}.”\evb\evg

\sectionline

\bvg\bva\speakernote{[Óðinn kvað:]}\mssnote{\Regius~8v/11, \AM~3v/4}%
„\alst{F}jǫlð ek \alst{f}ór, \hld\ \alst{f}jǫlð \alst{f}ręistaða’k, &
\ind fjǫlð ek \alst{r}ęynda \alst{r}ęgin; &
hvat lifir \alst{m}anna, \hld\ þȧ’s hinn \alst{m}ę́ra líðr &
\ind \alst{f}imbul-vetr með \alst{f}irum?“\eva

\bvb\speakernoteb{[Weden quoth:]}%
“Much I journeyed, much I tried, \\
\ind much I tested the Reins. \\
What remains of men when the renowned \inx[L]{Fimble-winter} \\
\ind passes amidst the folk?”\evb\evg


\bvg\bva\speakernote{[Vafþrúðnir kvað:]}\mssnote{\Regius~8v/13, \AM~3v/6, \GylfMS}%
„\alst{L}íf ok \alst{L}ífþrasir, \hld\ en þau \alst{l}ęynask munu &
\ind ï \edtrans{\alst{h}olti \alst{H}odd-mímis}{in Hoardmimer’s wood}{\Bfootnote{Perhaps the hollowed-out \inx[L]{Uggdrassle’s Ash.}}}; &
\alst{m}orgin-dǫggvar \hld\ þau sér at \alst{m}at hafa; &
\ind þaðan af \alst{a}ldir \alst{a}lask.“\eva

\bvb\speakernoteb{[Webthrithner quoth:]}%
“\inx[P]{Life} and \inx[P]{Lifethrasher}—but they will hide themselves \\
\ind in \inx[P]{Hoardmimer}’s wood. \\
Morning dew will they have for food; \\
\ind from thence is mankind begotten.”\evb\evg


\bvg\bva\speakernote{[Óðinn kvað:]}\mssnote{\Regius~8v/15, \AM~3v/8}%
„\alst{F}jǫlð ek \alst{f}ór, \hld\ \alst{f}jǫlð \alst{f}ręistaða’k, &
\ind fjǫlð ek \alst{r}ęynda \alst{r}ęgin; &
hvaðan kømr \alst{s}ól \hld\ ȧ hinn \alst{s}létta himin, &
\ind es \edtrans{þessa}{this one}{\Bfootnote{The present sun, as explained in the following st.}} hęfr \edtrans{\alst{F}ęnrir}{Fenrer}{\Bfootnote{Perhaps not the same “Fenrerswolf” that fights against Weden (cf. st. 53 below).  The word, which originally prob. means “Fen-creature”, may here simply be a generic poetic synonym for “wolf”.  For the wolves who chase the sun and moon see \Voluspa\ 40 and \Grimnismal\ 40.}} \alst{f}arit?\eva

\bvb\speakernoteb{[Weden quoth:]}%
“Much I journeyed, much I tried, \\
\ind much I tested the Reins! \\
whence comes Sun onto the smooth heaven, \\
\ind when \inx[P]{Fenrer} has destroyed this one?”\evb\evg


\bvg\bva\speakernote{[Vafþrúðnir kvað:]}\mssnote{\Regius~8v/16, \AM~3v/9, \GylfMS}%
„\alst{Ęi}na dóttur \hld\ berr \edtext{\alst{a}lf-rǫðull}{\lemma{alf-rǫðull ‘Elf-wheel’}\Bfootnote{A rare poetic synonym (\emph{hęiti}) for the sun.  It occurs in two other places: \Skirnismal\ 4/3, and a Scaldic loose stanza by Eanwind Spoiler-of-scalds (Eyv Lv 9 in \Skp\ 1).  It also appears in two lists of names for the sun: \Skaldskaparmal\ 69, Þul \emph{Sólar} 1/7 in \Skp\ 3, but these do not count as independent attestations since they may be drawing from any of the three earlier poems.)

Depending on the age of the cpd. the first element may reflect the semantics of PIE \emph{albʰós} ‘white’ (cf. Latin \emph{albus} ‘id.’).  The second element \emph{rǫðull} is not the normal ON word for “wheel”; it is inherited from PGmc. \emph{*radulaz} \char`~\ \emph{*raduraz}, whence also OE \emph{rǫdor} ‘heaven, sky’, OS \emph{radur, radul} ‘id.’  It is composed of the root of German \emph{Rad} ‘wheel’ with the agentive suffix \emph{*-ulaz} \char`~\ \emph{*-uraz} ‘(habitually) doing’ and thus means something like ‘circler, turner, revolver’.  The PIE root is \emph{*Hreth₂-} which e.g. yields Latin \emph{rota} ‘wheel’, Sanskrit \emph{rata} ‘chariot’.  In conclusion a more etymological translation may ‘white circler’.}}, &
\ind áðr hana \alst{F}ęnrir \alst{f}ari; &
sú skal \alst{r}íða, \hld\ þȧ’s \alst{r}ęgin dęyja, &
\ind \alst{m}óður brautir \alst{m}ę́r.“\eva

\bvb\speakernoteb{[Webthrithner quoth:]}%
“One daughter the Elf-wheel \name{= Sun} bears \\
\ind before Fenrer might slay her. \\
She shall ride—when the Reins die— \\
\ind the maiden, her mother’s paths.”\evb\evg


\bvg\bva\speakernote{[Óðinn kvað:]}\mssnote{\Regius~8v/18, \AM~3v/10}%
\alst{F}jǫlð ek \alst{f}ór, \hld\ \alst{f}jǫlð \alst{f}ręistaða’k, &
\ind fjǫlð ek \alst{r}ęynda \alst{r}ęgin; &
\edtext{hvęrjar ’ru \alst{m}ęyjar, \hld\ es líða \alst{m}ar yfir, &
\ind \alst{f}róð-gęðjaðar \alst{f}ara?}{\lemma{hvęrjar \dots\ fara? ‘Who \dots\ go?’}\Bfootnote{The identity of these maidens is very mysterious, and Webthrithner’s answer in the next st. does not give much more information.  Considering all other questions introduced with the words \emph{fjǫlð ek fór} et.c. have something to do with the end times, this one should as well.  With this in mind they are probably to be identified with the maidens Weden asks about in \Baldrsdraumar\ 12.}}\eva

\bvb\speakernoteb{[Weden quoth:]}%
“Much I journeyed, much I tried, \\
\ind much I tested the Reins! \\
Who are the maidens that pass over the ocean; \\
\ind wise-minded they go?”\evb\evg


\bvg\bva\speakernote{[Vafþrúðnir kvað:]}\mssnote{\Regius~8v/19, \AM~3v/11}%
\alst{Þ}ríar \alst{þ}jóð-á\emph{a}r \hld\ falla \alst{þ}orp yfir &
\ind \alst{m}ęyja \alst{M}ǫg-þrasis; &
\alst{h}amingjur ęinar \hld\ þę́r’s ï \alst{h}ęimi eru, &
\ind þó þę́r með \alst{jǫ}tnum \alst{a}lask.\eva

\bvb\speakernoteb{[Webthrithner quoth:]}%
“Three great rivers fall over the house \\
\ind of the maidens of Maythrasher; \\
they are the only Hamings in the Home, \\
\ind although they are raised amidst ettins.”\evb\evg


\bvg\bva\speakernote{[Óðinn kvað:]}\mssnote{\Regius~8v/21, \AM~3v/13}%
„\alst{F}jǫlð ek \alst{f}ór, \hld\ \alst{f}jǫlð \alst{f}ręistaða’k, &
\ind fjǫlð ek \alst{r}ęynda \alst{r}ęgin; &
hvęrir ráða \alst{ę́}sir \hld\ \alst{ęi}gnum goða, &
\ind þȧ’s \alst{s}loknar \edtrans{\alst{S}urta-logi}{the flame of Surt}{\Bfootnote{The flame which reaches up to Heaven itself and burns the entire world; see \Voluspa\ 50, 54.}}?“\eva

\bvb\speakernoteb{[Weden quoth:]}%
“Much I journeyed, much I tried, \\
\ind much I tested the Reins! \\
Which Eese rule the ownings of the Gods \\
\ind when the flame of \inx[P]{Surt} goes out?”\evb\evg


\bvg\bva\speakernote{[Vafþrúðnir kvað:]}\mssnote{\Regius~8v/22, \AM~3v/14, \GylfMS}%
„\alst{V}íðarr ok \alst{V}ȧli \hld\ byggva \alst{v}é goða, &
\ind þȧ’s \alst{s}loknar \alst{S}urta-logi; &
\alst{M}óði ok \alst{M}agni \hld\ skulu \alst{M}jǫllni hafa &
\ind \edtrans{\alst{V}ingnis at \alst{v}íg-þroti}{after Wingner expires in war}{\Bfootnote{After Thunder dies in his fight against the \inx[P]{Middenyardswyrm}.}}.“\eva

\bvb\speakernoteb{[Webthrithner quoth:]}%
“\inx[P]{Wider} and \inx[P]{Wonnel} bedwell the \inx[C]{wigh}[wighs] of the gods \\
\ind when the flame of Surt goes out. \\
\inx[P]{Mood} and \inx[P]{Main} shall have \inx[P]{Millner} \\
\ind after \inx[P]{Wingner} expires in war.”\evb\evg


\bvg\bva\speakernote{[Óðinn kvað:]}\mssnote{\Regius~8v/24, \AM~3v/16}%
„\alst{F}jǫlð ek \alst{f}ór, \hld\ \alst{f}jǫlð \alst{f}ręistaða’k, &
\ind fjǫlð ek \alst{r}ęynda \alst{r}ęgin; &
hvat verðr \alst{Ó}ðni \hld\ at \alst{a}ldr-lagi, &
\ind \edtrans{þȧ’s \alst{r}júfask \alst{r}ęgin?}{when the Reins are ripped?}{\Bfootnote{Formulaic; see note to \Baldrsdraumar\ 14/1.}}“\eva

\bvb\speakernoteb{[Weden quoth:]}%
“Much I journeyed, much I tried, \\
\ind much I tested the Reins! \\
What brings Weden’s life to an end, \\
\ind when the Reins are ripped?”\evb\evg


\bvg\bva\speakernote{[Vafþrúðnir kvað:]}\mssnote{\Regius~8v/25, \AM~3v/17}%
„\alst{U}lfr glęypa \hld\ mun \alst{A}lda-fǫðr, &
\ind þęss mun \alst{V}íðarr \alst{v}reka; &
\alst{k}alda \alst{k}japta \hld\ hann \alst{k}lyfja mun &
\ind \alst{v}itnis \alst{v}ígi at.“\eva

\bvb\speakernoteb{[Webthrithner quoth:]}%
“The Wolf will devour the Father of Men: \\
\ind that will Wider avenge. \\
The cold jaws he will split apart \\
\ind of the beast at the battle.”\evb\evg


\bvg\bva\speakernote{[Óðinn kvað:]}\mssnote{\Regius~8v/27, \AM~3v/19}%
„\alst{F}jǫlð ek \alst{f}ór, \hld\ \alst{f}jǫlð \alst{f}ręistaða’k, &
\ind fjǫlð ek \alst{r}ęynda \alst{r}ęgin; &
hvat mę́lti Óðinn, \hld\ áðr \edtrans{ȧ bál stigi}{step onto the pyre}{\Bfootnote{The phrase \emph{stíga á} ‘step onto, mount’ is also used to refer to one stepping aboard a ship or mounting a horse (see \CV: \emph{stíga} for citations).  Its use for a person being borne onto the funeral pyre has been compared with \Beowulf\ 1118b: \emph{gu̇ð-rinc á·stâh} ‘the war-champion mounted [his pyre]’, but the interpretation of that line is controversial; \textcite{KlaeberBeowulf}[186] follow Grundtvig in emending \emph{gu̇ð-rinc} to \emph{gu̇ð-réc} ‘war-smoke’ and compare it with \Beowulf\ 3144b (\emph{wudu-réc á·stâh} ‘wood-smoke rose up’, which also describes a cremation; (according to them) the present stanza “almost certainly refers not to Baldr but to Óðinn, probably imagined to mount the pyre in order to set fire to it.”}}, &
\ind \alst{s}jalfr ï ęyra \alst{s}yni?“\eva

\bvb\speakernoteb{[Weden quoth:]}%
“Much I journeyed, much I tempted, \\
\ind much I tested the Reins! \\
What spoke Weden, before he would step onto the pyre, \\
\ind himself in the ear of his son \ken*{= Balder}?”\evb\evg


\bvg\bva\speakernote{[Vafþrúðnir kvað:]}\mssnote{\Regius~8v/28, \AM~3v/19}%
„\alst{Ęy} \edtext{mann-gi}{\Afootnote{\emph{manni} dat. sg. \Regius\AM\ is impossible; a subject is needed.}} vęit, \hld\ hvat þú í \alst{á}r-daga &
\ind \alst{s}agðir ï ęyra \alst{s}yni; &
\edtrans{\alst{f}ęigum}{fey}{\Bfootnote{A word with strong fatalistic connections. Webthrithner realises that he was bound to die from the moment he proposed the wager (st. 19), as no being can outwit Weden.}} munni \hld\ mę́lta’k \edtrans{mína \alst{f}orna stafi}{my ancient staves}{\Bfootnote{Referencing st. 1.}} &
\ind ok of \alst{r}agna \alst{r}ǫk; &
nú við \alst{Ó}ðin \hld\ dęilda’k mïna \edtrans{\alst{o}rð-spęki}{word-wisdom}{\Bfootnote{Referencing st. 5.}}; &
\ind þú est ę́ \alst{v}ísastr \edtrans{\alst{v}era}{of men}{\Bfootnote{\emph{verr} means ‘husband, man’ and is here used for reasons of alliteration; it does not imply that Weden is not a God.}}.“\eva

\bvb\speakernoteb{[Webthrithner quoth:]}%
“Never will man know what thou in days of yore \\
\ind saidst in the ear of thy son. \\
With a \inx[C]{fey} mouth I spoke my ancient \inx[C]{stave}[staves], \\
\ind and about the Rakes of the Reins. \\
Now with Weden have I shared my word-wisdom— \\
\ind thou art ever wisest of men!”\evb\evg

\sectionline
% Weden
	\bookStart{Speeches of Grimner}[Grímnismǫ́l]

\begin{flushright}%
\textbf{Dating} \parencite{Sapp2022}: C10th (0.976)

\textbf{Meter:} \Ljodahattr\ (1–2/2, 3–26, 27/4–27/7, 28/1–28/2, 28/6–28/7, 29–33/2, 35–45/2, 46/1–46/2, 47–48/2, 49/3–52, 54–57), \Fornyrdislag\ (2/3–2/4, 28/3–28/5, 33/3, 45/3–45/5, 48/3–48/4, 49/1–49/2, 53), \Galdralag\ (27/1–27/3, 34, 46/3–46/5)%
\end{flushright}

\section{Introduction}

The \textbf{Speeches of Grimner} (\Grimnismal) are preserved whole in both \Regius\ and \AM.

\subsection{Structure}

\Grimnismal\ essentially consists of several nested layers.  The outermost layer is the prose passages which bracket the actual poem (P1–P2).  It is hard to say for how long these have accompanied the verses, but since they are found in both \Regius\ and \AM\ they must go back to a now-lost manuscript archetype.  The second layer is sts. 1–3 and 53–55, which together with the prose form a narrative frame for the gnomic wisdom stanzas which make up the bulk of the poem and its core.  These gnomic stanzas are mythological and sometimes obscure, and align closely with other Eddic wisdom verse like \Havamal, \Vafthrudnismal, \Sigrdrifumal, and \Allvismal.

\subsection{Summary}
The text begins with the frame narrative, which tells the story of the two king’s sons Ayner and Garfrith.  Ayner is fostered by Frie, while the two winters younger Garfrith is fostered by her husband Weden himself.  After their father’s death it is Garfrith who becomes king, following his betrayal of his elder brother. (P1)  One day Weden and Frie are arguing over their respective foster-sons, and Frie accusses Garfrith of torturing wayfaring guests.  Weden sets out to test the hospitality of his protégé, but unbeknownst to him, his wife has already sent her handmaid in disguise to warn Garfrith about the coming of an evil wizard.  When Weden arrives he is thus promptly captured and placed between two fires so that he will reveal his name.  Garfrith’s young son, Ayner (clearly named after his uncle), kindly approaches the god and offers him a horn of drink.  Grimner drinks from it, and here the poem proper begins. (P2)

Weden begins by complaining about the fires which are now burning his cloak (1); he states that for eight nights not a soul has offered him any help save Ayner, Garfrith’s son, who will soon become king after his father (2).  As thanks for the drink he gives him good health, and will offer him holy knowledge (3).

Here the gnomic section begins as Weden lists the individual abodes of the gods (4–17).  The locations are numbered, but a few facts speak to these numbers being a later insert:

\begin{enumerate}
  \item The alliteration is never reliant on the numbers; if one compares the numbered questions in \Vafthrudnismal\ 20–42 the difference is striking.
  \item The numbering is inconsistent; Thunder’s realm (st. 4) is not counted, and Wider’s land (st. 17) has no numeral (perhaps since the form of the stanza would not allow it.)
  \item In sts. 11–15 cited in \Gylfaginning, the numbers are missing.
\end{enumerate}

%This section has been discussed in detail by \textcite{deVries1952} TODO! who considers it corrupt. Specifically, he sees the second half of v. 4 as a later insert, since it does not elaborate on the “holy land” mentioned in the first half. \textcite{Jackson1995} argues convincingly against this, showing how the first half serves as a generalized introduction to the list; the holy land is the dwelling-places of the gods.)

After this list come several sts. relating to Weden and his hall, Walhall (18–23). Mentioned are the preparation of food in Walhall (18), Weden’s wolves (19) and ravens (20), the river through which the dead have to wade (21) and the gate through which they have to pass (22), the count of doors in Walhall (23), the count of doors in Thunder’s hall Bilshirner (24), and two animals which stand on the hall and gnaw on the branches of the tree Leered (25–26). From the latter animal’s—the stag Oakthirner’s—horns droplets fall into Wharyelmer, which is the origin of all rivers (26).

This introduces a list of mythic rivers (27–28), ending with the waters through which Thunder must wade on his way to Ugdrassle (29). This leads to a list of the horses ridden by the other gods on their way to Ugdrassle (31) which is followed by a description of the roots of Ugdrassle (31), then its animals (32–36) the Walkirries (37), and beings associated with the sun and moon (38–40), the things created from Yimer’s body (41–42) with a digression on the significance of the \inx[P]{bloot} for men in the present (43, see note there!), the creation of the ship Shidebladner (44) and finally a list of the noblest of several categories of things and groups (45).

After these lists Weden utters an unclear st. invoking the gods (46), before listing many of his names and the circumstances in which they were used (47–50). He then turns to Garfrith, disappointed by the inhospitality and poor conduct of his former protégé, and predicts his imminent death (51–53). He finally reveals himself by his true name, daring Garfrith to face him (53). After this he repeats several of his names (54), and the poem ends.

In the final prose section we are told that Garfrith, after learning that he was torturing Weden, hurried up to take the god away from the fires, but tripped and fell on his sword and died. After this his son Ayner ruled for a long time.

\sectionline

\section{From the sons of king Reading (\emph{Frá sonum Hrauðungs konungs})}

\bpg\bpa\mssnote{\Regius~8v/31, \AM~3v/23}%
Hrauðungr konungr átti tvá sonu. Hét annarr Agnarr, enn annarr Geirrøðr.
Agnarr var tíu vetra enn Geirrøðr átta vetra. Þeir reru tveir á báti með dorgar sínar at smá-fiski.
Vindr rak þá í haf út. Í nátt-myrkri brutu þeir við land ok gingu upp; fundu kot-bónda einn.
Þar vǫ́ru þeir um vetrinn. Kerling fostraði Agnar, enn karl Geirrøð.
At vári fekk karl þeim skip. Enn er þau kerling leiddu þá til strandar, þá mę́lti karl ein-mę́li við Geirrøð.
Þeir fengu byr ok kvǫ́mu til stǫðva fǫður síns. Geirrøðr var fram í skipi.
Hann hljóp upp á land enn hratt út skipinu, ok mę́lti: „Far þú þar er smyl hafi þik.“
Skipit rak út. Enn Geirrøðr gekk út til bǿjar; hánum var vel fagnat; þá var faðir hans andaðr.
Var þá Geirrøðr til konungs tekinn, ok varð maðr ágę́tr.\epa

\bpb King Reading had two sons. One was called Ayner, and the other Garfrith.
Ayner was ten winters old, but Garfrith eight winters. The two were rowing in a boat with their trolling-lines for small fishing.
The wind drove them out into the sea. In the dark of night they crashed onto land and walked ashore; they found a lone cottage farmer.
There they stayed over the winter. The farmer’s wife fostered Ayner and the farmer Garfrith.\footnoteB{The husband and wife were Weden and Frie; this is clarified by the following prose. The motif of Weden preferring the youngest brother is also found in \Rigsthula.}
In the spring the husband gave them ships, but when he and his wife led them to the shore, the husband spoke privately with Garfrith.\footnoteB{Surely instructing him to push his brother out to sea.}
They caught good wind, and came to their father’s harbour. Garfrith was in the front of the ship.
He leapt onto land and pushed out the ship, and spoke: ”Go thou whither the fiends may have thee!”
The ship drove out. But Garfrith walked towards the farm; he was welcomed well; by then was his father ended.
Garfrith was then taken as king, and became an excellent man.\epb\epg


\bpg\bpa\mssnote{\Regius~9r/10, \AM~4r/3}%
Óðinn ok Frigg sátu í Hliðskjǫlfu ok sá um heima alla.
Óðinn mę́lti: „Sér þú Agnar fóstra þinn, hvar hann elr bǫrn við gýgi í hellinum?
En Geirrøðr, fóstri minn, er konungr ok sitr nú at landi.“
Frigg segir: „Hann er mat-níðingr sá at hann kvelr gesti sína ef hánum þykkja of-margir koma.“
Óðinn segir at þat er in mesta lygi. Þau veðja um þetta mál.
Frigg sendi eskis-mey sína, Fullu, til Geirrøðar. Hon bað konung varask at eigi fyr-gerði hánum fjǫl-kunnigr maðr sá er þar var kominn í land, ok sagði þat mark á at engi hundr var svá ólmr at á hann myndi hlaupa.
En þat var inn mesti hé-gómi at Geirrøðr vę́ri eigi mat-góðr ok þó lę́tr hann hand-taka þann mann er eigi vildu hundar á ráða.
Sá var í feldi blám ok nefndisk Grímnir ok sagði ekki fleira frá sér þótt hann vę́ri at spurðr.
Konungr lét hann pína til sagna ok setja milli elda tveggja ok sat hann þar átta nę́tr.
Geirrøðr konungr átti son tíu vetra gamlan ok hét Agnarr eptir bróður hans.
Agnarr gekk at Grímni ok gaf hánum horn fullt at drekka, sagði at konungr gerði illa er hann lét pína hann sak-lausan.
Grímnir drakk af. Þá var eldrinn svá kominn at feldrinn brann af Grímni. Hann kvað:\epa

\bpb Weden and Frie sat in the \inx[L]{Lithshelf} and looked about all the Homes.\footnoteB{Very similar to the Longbeard Origin Myth (TODO: reference and elaborate).}
Weden spoke: “Dost thou see Ayner, thy foster-son, where he begets children with a troll-woman in her cave?\footnoteB{This may relate to Frie’s role as love-goddess. Ayner is in any case to be understood as a weak, effeminate man.}
But Garfrith, \emph{my} foster-son, is king and now rules his land.”
Frie says: “He is such a meat-nithing that he torments his guests if he thinks too many are coming!”
Weden says that this is the greatest lie; they make a wager over this matter.
Frie sent her handmaid, Full, to Garfrith’s hall. She bade the king be wary, lest he be destroyed by the \inx[C]{many-cunning} man who had come to his land; and said that his mark was that no hound was so fierce that it would rush at him.
But it was the greatest falsehood that Garfrith was not \inx[C]{good of meat}; and yet he has that man bound whom the hounds would not touch.
He was in a blue cloak and called himself Grimner, and did not tell anything more about himself, even though he was asked.
The king had him tortured that he would speak, and set him between two fires; and he sat there for eight nights.
King Garfrith had a son ten winters old, and he was called Ayner after his brother.
Ayner went up to Grimner and gave him a full horn to drink, saying that the king did badly as he had him tortured without cause.
Grimner drank it up. Then the fire had grown so much that the cloak burned on Grimner. He quoth:\epb\epg\stepcounter{prosea}

\sectionline

\section{The Speeches of Grimner}

\bvg\bva\mssnote{\Regius~9r/27, \AM~4r/17}%
„\alst{H}ęitr est \alst{h}ripuðr \hld\ ok \alst{h}ęldr til mikill, &
\ind gǫngumk \alst{f}irr \alst{f}uni! &
\alst{L}oði sviðnar, \hld\ þótt á \alst{l}opt bera’k; &
\ind brinnumk \alst{f}eldr \alst{f}yrir.\eva

\bvb “Hot art thou, flame, and rather too great; \\
\ind go far from me, fire! \\
The wool-cape is singed though I hold it aloft; \\
\ind the cloak burns before me!\evb\evg


\bvg\bva\mssnote{\Regius~9r/29, \AM~4r/18}%
\alst{Á}tta nę́tr \hld\ sat’k milli \alst{ę}lda hér, &
\ind svá’t mér \alst{m}ann-gi \alst{m}at né bauð &
nema \alst{ęi}nn Agnarr, \hld\ es \alst{ęi}nn skal ráða, &
\alst{G}ęirrøðar sonr, \hld\ \alst{G}otna landi.\eva

\bvb For eight nights I sat between the fires here, \\
\ind while no man offered me food, \\
save for Ayner alone, who alone shall rule— \\
Garfrith’s son—the land of the Gots!\evb\evg


\bvg\bva\mssnote{\Regius~9r/31, \AM~4r/20}%
\alst{H}ęill skalt, Agnarr, \hld\ alls \alst{h}ęilan biðr &
\ind þik \alst{V}era-týr \alst{v}esa; &
\alst{ęi}ns drykkjar \hld\ skalt \alst{a}ldri-gi &
\ind \edtrans{bętri \alst{g}jǫld}{better recompense}{\Bfootnote{Namely the mythic lore which takes up sts. 4–53.}} \alst{g}eta:\eva

\bvb Hale shalt thou be, Ayner, for hale \\
\ind does Were-Tew \name{= Weden} bid thee be! \\
For a single drink shalt thou never get \\
\ind better recompense.\evb\evg

\sectionline

\bvg\bva\mssnote{\Regius~9r/33, \AM~4r/22}%
\alst{L}and es hęilagt, \hld\ es \alst{l}iggja sé’k &
\ind \alst{ǫ́}sum ok \alst{ǫ}lfum nę́r; &
en í \alst{Þ}rúð-hęimi \hld\ skal \alst{Þ}órr vesa &
\ind \edtrans{unds of \alst{r}júfask \alst{r}ęgin}{until the Reins are ripped}{\Bfootnote{i.e. until the \inx[L]{Rakes of the Reins}.  A formulaic expression; see note to \Baldrsdraumar\ 14 for further occurrences.}}.\eva

\bvb The land is holy which lying I see \\
\ind near the \inx[F]{Eese and Elves}, \\
but in Thrithham shall Thunder dwell \\
\ind until the Reins are ripped.\evb\evg


\bvg\bva\mssnote{\Regius~9v/2, \AM~4r/23}%
\alst{Ý}-dalir hęita, \hld\ þar’s \alst{U}llr hęfir &
\ind \alst{s}ér of gǫrva \alst{s}ali; &
\alst{A}lf-hęim Fręy \hld\ gǫ́fu í \alst{á}r-daga &
\ind \alst{t}ívar at \edtrans{\alst{t}ann-féi}{tooth-gift}{\Bfootnote{The gift the child receives when he sheds his first tooth.}}.\eva

\bvb Yewdales they are called where Woulder has \\
\ind made for himself a hall. \\
Elfham to Free in days of yore \\
\ind the Tews as a tooth-gift gave.\evb\evg


\bvg\bva\mssnote{\Regius~9v/3, \AM~4r/25}%
\alst{B}ǿr es sá (hinn þriði), \hld\ es \alst{b}líð ręgin &
\ind \alst{s}ilfri þǫkðu \alst{s}ali; &
\alst{V}ala-skjǫlf hęitir, \hld\ \edtrans{es \alst{v}élti sér}{won through wiles}{\Bfootnote{Several previous editors and translators (e.g. \textcite{FinnurEdda}, \textcite{PettitEdda}, \textcite{LarringtonEdda}) have rendered this phrase with variants of “craftily made for himself”, where the verb \emph{véla} would mean ‘craftily make’.  To my knowledge this sense is never otherwise attested, and its common meaning is ‘defraud, trick, betray’.  A simpler reading would be to see this as a reference to the myth of the Ettin-smith who built the wall of Osyard.  The Gods had promised him Sun, Moon, and Frow, if he could build it in a year, but employed various tricks to hinder him.  When it at last looked like he would make it in time, Thunder slew him.  This myth is told in \Gylfaginning\ 42 and alluded to in \Voluspa\ 24–25.}} &
\ind \alst{ǫ́}ss í \alst{á}r-daga.\eva

\bvb Bower is (the third) one, where the blithe Reins \\
\ind with silver thatched a hall. \\
Waleshelf is it called which he won through wiles, \\
\ind the Os in days of yore.\evb\evg


\bvg\bva\mssnote{\Regius~9v/5, \AM~4r/26}%
\alst{S}økkva-bękkr hęitir (hinn fjórði), \hld\ en þar \alst{s}valar knegu &
\ind \alst{u}nnir glymja \alst{y}fir; &
þar þau \alst{Ó}ðinn ok Sága \hld\ drekka umb \alst{a}lla daga &
\ind \alst{g}lǫð ór \alst{g}ullnum kęrum.\eva

\bvb Sinkbench is (the fourth) one called, and there do cool \\
\ind waves clash over above; \\
there Weden and Sey drink all days, \\
\ind glad, out of golden casks.\evb\evg


\bvg\bva\mssnote{\Regius~9v/7, \AM~4r/28}%
\alst{G}laðs-hęimr hęitir (hinn fimti) \hld\ þar’s hin \alst{g}ull-bjarta &
\ind \alst{V}al-hǫll \alst{v}íð of þrumir; &
en þar \alst{H}roptr \hld\ kýss \alst{h}vęrjan dag &
\ind \alst{v}ápn-dauða \alst{v}era.\eva

\bvb Gladsham is (the fifth) one called, where the gold-bright \\
\ind Walhall, wide, stands fast, \\
and there Roft \name{= Weden} chooses every day \\
\ind weapon-dead warriors.\footnoteB{Cf. st. 14.}\evb\evg

\sectionline

In \AM\ the order of the following two sts. is reversed.

\sectionline

\bvg\bva\mssnote{\Regius~9v/9, \AM~4r/31}%
Mjǫk ’s \alst{au}ð-kęnnt \hld\ þęim’s til \alst{Ó}ðins koma &
\ind \edtext{\alst{s}al-kynni at \alst{s}éa}{\Afootnote{\emph{‘sia at sia’} \AM}}, &
\alst{v}argr hangir \hld\ fyr \alst{v}estan dyrr &
\ind ok drúpir \alst{ǫ}rn \alst{y}fir.\eva

\bvb Very easily recognized, for those who come to Weden, \\
\ind is the hall to see: \\
A wolf hangs before the western door, \\
\ind and an eagle droops above.\footnoteB{Something very similar is found in Widukind’s \emph{History of the Saxons} 1:12.  The Saxons have just conquered a fortress, and \emph{mane [...] facto ad orientalem portam ponunt aquilam, aramque victoriae construentes secundum errorem paternum sacra sua propria veneratione venerati sunt} ‘at the coming of morning they set an eagle at the eastern gate, and, building an altar of victory, they worshipped it with their own holy worship in accordance with their ancestral error.’  The altar was pledged to \inx[P]{Ermin}, whom the author identifies with Mars or Hermes, but who is surely Weden.

According to \textcite[156]{HyltenCavallius1863} it was custom in Wärend, southern Sweden to hang the bodies of killed wolves high up in old oaks, and killed birds of prey above the stable-door.}\evb\evg%TODO: bibliography for Widukind


\bvg\bva\mssnote{\Regius~9v/10, \AM~4r/30}%
Mjǫk ’s \alst{au}ð-kęnnt \hld\ þęim’s til \alst{Ó}ðins koma &
\ind \alst{s}al-kynni at \alst{s}éa, &
\edtrans{\alst{sk}ǫptum}{shafts}{\Bfootnote{Spear-shafts.}} ’s rann rępt, \hld\ \alst{sk}jǫldum ’s salr þakiðr, &
\ind \alst{b}rynjum of \alst{b}ękki stráat.\eva

\bvb Very easily recognized, for those who come to Weden, \\
\ind is the hall to see: \\
With shafts is the house roofed, with shields is the hall thatched; \\
\ind with byrnies the benches strewn.\evb\evg


\bvg\bva\mssnote{\Regius~9v/12, \AM~4v/2, \GylfMS}%TODO: All variants are not yet noted.
\alst{Þ}rym-hęimr hęitir \edtrans{(hinn sétti)}{the sixth}{\Afootnote{om. \GylfMS}}, \hld\ \edtrans{es}{where}{\Afootnote{\emph{þar nú} ‘where now’}} \alst{Þ}jatsi \edtrans{bjó}{dwelled}{\Afootnote{om. \Wormianus; \emph{býr} ‘dwells’ \Upsaliensis}}, &
\ind sá hinn \edtrans{\edtext{\alst{á}m-átki}{\Afootnote{\emph{mátki} \Upsaliensis}} \alst{jǫ}tunn}{uncanny ettin}{\Bfootnote{Formulaic. See note to \Voluspa\ 8.}}; &
en nú \alst{Sk}aði byggvir, \hld\ \alst{sk}ír brúðr \edtrans{goða}{of the Gods}{\Afootnote{\emph{guma} ‘of men’ \Upsaliensis}}, &
\ind \alst{f}ornar toptir \alst{f}ǫður.\eva

\bvb Thrimham is (the sixth) one called, where Thedse dwelled, \\
\ind that uncanny ettin; \\
but now Shede bedwells—the pure bride of the Gods— \\
\ind the ancient plots of her father.\evb\evg


\bvg\bva\mssnote{\Regius~9v/14, \AM~4v/3, \GylfMS}%
\alst{B}ręiða-\alst{b}lik \edtrans{eru (hin sjaundu)}{are (the seventh)}{\Bfootnote{\emph{hęita} ‘[they] are called’ \GylfMS.}}, \hld\ en þar \alst{B}aldr hęfir &
\ind \alst{s}ér of gǫrva \alst{s}ali, &
á því \alst{l}andi \hld\ es \alst{l}iggja vęit’k &
\ind \alst{f}ę́sta \edtrans{\alst{f}ęikn-stafi}{wicked deeds}{\Bfootnote{Lit. ‘staves of wickedness’, where ‘stave’ originally means something like ‘word, speech’.  Cf. \Beowulf\ 1018b: \emph{fâcen-stafas}, referring to treacherous intrigues among the \inx[G]{Shieldings}.}}.\eva

\bvb Broadblicks are (the seventh), and there Balder has \\
\ind made for himself a hall, \\
on that land where I know lying \\
\ind the fewest wicked deeds.\evb\evg


\bvg\bva\mssnote{\Regius~9v/16, \AM~4v/5, \GylfMS}%
\alst{H}imin-bjǫrg \edtrans{eru (hin ǫ́ttu)}{are (the eighth)}{\Bfootnote{\emph{hęita} ‘[they] are called’ \GylfMS.}}, \hld\ en þar \alst{H}ęim-dall &
\ind kveða \alst{v}alda \alst{v}éum; &
þar \edtrans{\alst{v}ǫrðr goða}{Watchman of the Gods}{\Bfootnote{Formulaic epithet of Homedal, also occurring in \Lokasenna\ 49 and possibly in \Skirnismal\ 28: \emph{vǫrðr með goðum} ‘the Watchman among the Gods’.  \Gylfaginning\ 27, where the present stanza is cited, gives some further details: \emph{Hann býr þar er heitir Himinbjǫrg við Bifrǫst. Hann er vǫrðr goða ok sitr þar við himins enda at gę́ta brúarinnar fyrir berg-risum. Hann þarf minna svefn en fugl. Hann sér jafnt nótt sem dag hundrað rasta frá sér; hann heyrir ok þat, er gras vex á jǫrðu eða ull á sauðum, ok allt þat er hę́ra lę́tr.} ‘He \ken*{= Homedal} lives at the place called the Heavenbarrows near Bivrest. He is the Watchman of the Gods and sits there at Heaven’s end to guard the bridge against barrow-risers.  He needs less sleep than a bird.  In night as in day he always sees a hundred \inx[C]{rest}[rests] away; he also hears when grass grows on the earth or wool on sheep, and all which makes more sound.’}} \hld\ drekkr í \alst{v}ę́ru ranni &
\ind \alst{g}laðr \edtext{hinn}{\Afootnote{so \AM\GylfMS; om. \Regius}} \alst{g}óða mjǫð.\eva

\bvb Heavenbarrows are (the eighth), and there Homedal, \\
\ind they say, wields over wighs. \\
There the Watchman of the Gods \ken*{= Homedal} drinks in the tranquil house, \\
\ind glad, the good mead.\evb\evg


\bvg\bva\mssnote{\Regius~9v/17, \AM~4v/6, \GylfMS}%
\alst{F}olk-vangr \edtrans{es (hinn níundi)}{is (the ninth)}{\Bfootnote{\emph{hęitir} ‘[one] is called’ \GylfMS}}, \hld\ en þar \alst{F}ręyja rę́ðr &
\ind \alst{s}essa kostum í \alst{s}al; &
\alst{h}alfan val \hld\ hon kýss \alst{h}vęrjan dag, &
\ind en halfan \alst{Ó}ðinn \alst{á}.\eva

\bvb Folkwong is (the ninth), and there Frow decides \\
\ind the choice of seats in the hall; \\
half the slain she chooses each day, \\
\ind but half does Weden own.\footnoteB{This st. is cited and closely paraphrased in \Gylfaginning\ 24. — The roots of \emph{kjósa val} ‘choose the slain’ are the same as those in \inx[C]{walkirrie} (\emph{val-kyrja} ‘chooser of the slain’), and as Frow is a prominent goddess this would surely make her the chief walkirrie.
This is paralleled by \Sorlathattr, where Frow assumes the name \inx[C]{Gandle} (\emph{Gǫndul}, a name attested in several lists of walkirries; see \Voluspa\ 30 and Notes) and incites the legendary never-ending Conflict of the Headnings (\emph{Hjaðningavíg}).
In spite of this parallel, there are good reasons to believe that the chief walkirrie was \inx[C]{Frie}, Weden’s wife.
First, one of the functions of the walkirries is to bear ale to the Oneharriers (\Grimnismal\ 37). This mirrors royal Germanic banquets attested in heroic poetry, where the host’s wife or daughter would pour ale to his retainers and guests (the so-called ‘lady with a mead cup’ ritual; see \textcite{Enright1996} and \textcite{Riseley2014}). As Weden’s wife, we would expect Frie to have this role.
Second, at Balder’s funeral as attested in \Gylfaginning\ (TODO. chapter number), Weden rides with Frie and the Walkirries, while Frow rides alone with her cats. If she were chief walkirrie, it is rather strange that she should not ride with them.
Third, there are two separate myths where Frie and Weden contend over the fates of armies and men. These are the prose introduction to the present poem and the Longbeard origin myth (for which see Introduction to the present poem).}\evb\evg


\bvg\bva\mssnote{\Regius~9v/19, \AM~4v/8, \GylfMS}%
\alst{G}litnir \edtrans{es (hinn tíundi)}{is (the tenth)}{\Bfootnote{\emph{hęitir salr} ‘a hall is called’ \GylfMS}}, \hld\ hann ’s \alst{g}ulli studdr &
\ind ok \alst{s}ilfri þakðr it \alst{s}ama; &
en þar \alst{F}or-seti \hld\ byggir \alst{f}lęstan dag &
\ind ok \alst{s}vę́fir allar \alst{s}akir.\eva

\bvb Glitner is (the tenth): it is supported by gold, \\
\ind and thatched with silver likewise. \\
And there Foresitter dwells for most of the day, \\
\ind and puts all disputes to sleep.\evb\evg


\bvg\bva\mssnote{\Regius~9v/21, \AM~4v/9}%
\alst{N}óa-tún eru (hin ęlliptu), \hld\ en þar \alst{N}jǫrðr hęfir &
\ind \alst{s}ér of gǫrva \alst{s}ali; &
\edtrans{\alst{m}anna þęngill \hld\ hinn \alst{m}ęins-vani}{The lord of men, the guileless one}{\Bfootnote{Interesting epithets probably relating to Nearth’s roles in upholding the bounty of the land and the law.  Cf. my article on pre-Christian oaths (TODO).}} &
\ind \edtrans{\alst{h}ǫ́-timbruðum \alst{h}ǫrgi rę́ðr}{rules the harrow timbered on high}{\Bfootnote{The rare verb \emph{hǫ́-timbra} ‘timber on high’ otherwise only occurs in \Voluspa\ 7, likewise in connection with the \emph{hǫrgr} ‘harrow’.  The harrow is an outdoors holy place; see Index.  Cf. also \Vafthrudnismal\ 38 where Nearth is said to rule a great many hoves and harrows.}}.\eva

\bvb Nowetowns are (the eleventh), and there Nearth has \\
\ind made for himself a hall. \\
The lord of men, the guileless one, \\
\ind rules the \inx[C]{harrow} timbered on high.\evb\evg


\bvg\bva\mssnote{\Regius~9v/23, \AM~4v/11}%
\edtrans{\alst{H}rísi vęx \hld\ ok \alst{h}ǫ́u grasi}{with brushwood grows, and with tall grass,}{\Bfootnote{Identical to \Havamal\ 119/6.}} &
\ind \alst{V}íðars land, \alst{v}iði, &
en þar \alst{m}ǫgr of lę́tsk \hld\ af \alst{m}ars baki &
\ind \alst{f}rǿkn at hęfna \alst{f}ǫður.\eva

\bvb With brushwood grows, and with tall grass, \\
\ind \inx[P]{Wider}’s land, with wood, \\
and there the lad vows from the back of his steed, \\
\ind brave, to avenge his father.\footnoteB{At the Rakes of the Reins Wider avenges His father, Weden.  See \Voluspa\ 51–52, \Vafthrudnismal\ 53.}\evb\evg


\bvg\bva\mssnote{\Regius~9v/24, \AM~4v/12, \GylfMS}%
\alst{A}nd-hrímnir \hld\ lę́tr í \alst{Ę}ld-hrímni &
\ind \alst{S}ę́-hrímni \alst{s}oðinn, &
\alst{f}lęska bętst, \hld\ en þat \alst{f}áir vitu, &
\ind við hvat \alst{ę}in-hęrjar \alst{a}lask.\eva

\bvb Andrimner lets Sowrimner \\
\ind in Eldrimner be boiled. \\
The best of meats, but few know this: \\
\ind by what the \inx[G]{Oneharriers} are nourished.\footnoteB{The cook Andrimner ‘face-sooty’ cooks the boar Sowrimner ‘sow-sooty’ in the cauldron Eldrimner ‘fire-sooty’; by this meat are the Oneharriers nouished.}\evb\evg


\bvg\bva\mssnote{\Regius~9v/26, \AM~4v/14, \GylfMS}%
\edtext{\alst{G}era ok Freka \hld\ sęðr \alst{g}unn-tamiðr, &
\ind \alst{h}róðigr \alst{H}ęrjafǫðr, &
en við \alst{v}ín ęitt \hld\ \alst{v}ápn-gǫfugr &
\ind \alst{Ó}ðinn \alst{ę́} lifir.}{\lemma{Gera \dots\ lifir ‘Gar \dots\ live’}\Bfootnote{With what Weden feeds his two hounds it is not said, but it is most likely with the corpses of dead warriors.  The wine on which he subsists may perhaps be identified with drink offerings.  Cf. the 7th century \emph{vita} of Saint Columban (TODO: cite source), describing a rite of the Swabians: \emph{Quo cum moraretur, et inter habitatores loci illius progrederetur, reperit eos sacrificium profanum litare velle, vasque magnum, quod vulgo cupam vocant, quod viginti et sex modios amplius minusve capiebat, cervisia plenum in medio habebant positum. Ad quod vir Dei accessit, et sciscitatur quid de illo fieri vellent. Illi aiunt Deo suo Vodano, quem Mercurium vocant alii, se velle litare.} ‘While he was satying there and going about the dwellers of that place, he found out that they were going to offer a profane sacrifice, and a large cask called a \emph{cupa}, which held about twenty-six measures, was filled with beer and set in their midst.  When the man of God asked what they wanted to do with it, they answered that they were wanted to offer to their God Wodan, whom others call Mercury.’}}\eva

\bvb \inx[P]{Gar and Freak} does the battle-accustomed \\
\ind glorious Father of Hosts \name{= Weden} feed; \\
but on wine alone, esteemed of weapons, \\
\ind Weden ever lives.\evb\evg


\bvg\bva\mssnote{\Regius~9v/28, \AM~4v/15, \GylfMS}%
\alst{H}uginn ok Muninn \hld\ fljúga \alst{h}vęrjan dag &
\ind \edtrans{\alst{jǫ}rmun-grund}{ermin-ground}{\Bfootnote{i.e. ‘the immense ground’ (for the rare prefix \inx[C]{ermin-} see Index), denoting the earth as a vast flat expanse of land. This compound also occurs in a kenning in the st. on the late C10th Karlevi stone (Öl 1) referring to the unbounded sea as \emph{Ęndils jǫrmungrund} ‘Andle’s ermin-ground’ (Andle being a known “sea-king”), and in \Beowulf\ 859 as \emph{eormen-grund} carrying the same sense.}} \alst{y}fir; &
\alst{ó}umk of Hugin, \hld\ at \alst{a}ptr né komi-t; &
\ind þó séumk \alst{m}ęir of \alst{M}unin.\eva

\bvb Highen and Minden fly every day \\
\ind over the ermin-ground \ken{earth}. \\
I worry for Highen, that he might not come back, \\
\ind yet I fear more for Minden.\evb\evg


\bvg\bva\mssnote{\Regius~9v/30, \AM~4v/17}%
\alst{Þ}ýtr \alst{Þ}und, \hld\ unir \edtext{\alst{Þ}jóð-vitnis &
\ind fiskr}{\lemma{Þjóðvitnis fiskr ‘Thedwitner’s fish’}\Bfootnote{\emph{Þjóðvitnir} is easily analyzed as \emph{þjóð-} ‘great, main’ + \emph{vitnir} ‘wolf’.  The great wolf is naturally the \inx[P]{Fenrerswolf}, the brother of the Middenyardswyrm.  That the Wyrm can be called a fish is shown by \Hymiskvida\ 24.}} flóði í; &
\alst{á}ar-straumr \hld\ þykkir \alst{o}f-mikill &
\ind \alst{v}al-glaumi at \alst{v}aða.\eva

\bvb \inx[P]{Thound} roars; Thedwitner’s fish \\
\ind thrives in the flood. \\
The river-stream seems far too great \\
\ind for the noisy slain host to wade.\footnoteB{A difficult stanza.  Thound may be the river surrounding Walhall, which the dead have to pass over to reach it.  The stanza may also be referring to the punishment of criminals in waters; see note to \Voluspa\ 38 for discussion on that.}\evb\evg


\bvg\bva\mssnote{\Regius~9v/32, \AM~4v/18}%
\edtrans{\alst{V}al-grind}{Walgrind}{\Bfootnote{‘Slain-gate’, the gate standing before Walhall.}} hęitir \hld\ es stęndr \alst{v}ęlli á &
\ind \alst{h}ęilǫg fyr \alst{h}ęlgum durum; &
\alst{f}orn ’s sú grind, \hld\ en þat \alst{f}áir vitu, &
\ind hvé hǫ́n ’s í \alst{l}ás of \alst{l}okin.\eva

\bvb \inx[L]{Walgrind} ’tis called, which stands on the plain, \\
\ind holy, before the holy doors. \\
Old is that gate, but few know this: \\
\ind how its lock is locked.\evb\evg


\bvg\bva\mssnote{\Regius~9v/34, \AM~4v/22}%
\alst{F}imm hundruð golfa \hld\ ok umb \alst{f}jórum tøgum &
\ind svá hygg’k \alst{B}il-skirni með \alst{b}ugum; &
\alst{r}anna þęira, \hld\ es \alst{r}ępt vita’k, &
\ind \alst{m}íns vęit’k męst \alst{m}agar.\eva

\bvb With five hundred floors, and around fourty, \\
\ind so I judge \inx[L]{Bilshirner} altogether. \\
Of those houses which I might know rafted \\
\ind I know my lad’s \ken*{= Thunder} to be the greatest.\evb\evg


\bvg\bva\mssnote{\Regius~10r/2, \AM~4v/20}%
\alst{F}imm hundruð dura \hld\ ok umb \alst{f}jórum tøgum, &
\ind svá hygg at \alst{V}alhǫllu \alst{v}esa; &
\edtrans{\alst{á}tta hundruð}{eight hundred}{\Bfootnote{The hundred is probably here the long hundred (120, rather than 100), which gives a sum of \(640 * 960 = 614~400\) Oneharriers.}} \alst{Ę}in-hęrja \hld\ ganga ór \alst{ęi}num durum, &
\ind þá’s fara við \alst{v}itni at \alst{v}ega.\eva

\bvb Five hundred doors, and around fourty, \\
\ind so I judge there to be on Walhall. \\
Eight hundred \inx[G]{Oneharriers} go out of one door, \\
\ind when to fight with the wolf they go.\evb\evg


\bvg\bva\mssnote{\Regius~10r/4, \AM~4v/24}%
\alst{H}ęið-rún hęitir gęit, \hld\ es stęndr \edtrans{\alst{h}ǫllu á Hęrja-fǫðrs}{on the hall of the Father of Hosts}{\Bfootnote{The hall of Weden, i.e. Walhall.  \emph{Hęrja-fǫðrs} looks like an unmetrical addition.}} &
\ind ok bítr af \alst{L}ę́-raðs \alst{l}imum; &
\edtrans{\alst{sk}ap-kęr}{shape-vats}{\Bfootnote{According to \CV\ the central beer-vat, from which drinks were poured into smaller vessels.}} fylla \hld\ skal \edtrans{hins \alst{sk}íra mjaðar}{the pure mead}{\Bfootnote{The mead is the goat’s milk.}}, &
\ind kná-at sú \alst{v}ęig \alst{v}anask.\eva

\bvb Heathrune is the goat called which stands on the hall of the Father of Hosts, \\
\ind and bites off Leered’s branches. \\
The shape-vats shall she fill with the pure mead; \\
\ind those draughts cannot wane.\evb\evg


\bvg\bva\mssnote{\Regius~10r/6, \AM~4v/26}%
Ęik-þyrnir hęitir \alst{h}jǫrtr \hld\ es stęndr \alst{h}ǫllu á Hęrja-fǫðrs &
\ind ok bítr af \alst{L}ę́-raðs \alst{l}imum; &
en af hans \alst{h}ornum \hld\ drýpr í \alst{H}ver-gęlmi &
\ind þaðan ęiga \alst{v}ǫtn ǫll \alst{v}ega:\eva

\bvb Oakthirner is called the stag who stands on the hall of the Father of Hosts, \\
\ind and bites off Leered’s branches. \\
And from his horns [drops] drip into Wharyelmer; \\
\ind thence have all waters their ways:\evb\evg


\bvg\bva\mssnote{\Regius~10r/9, \AM~4v/28}%
\alst{S}íð ok Víð, \alst{S}ę́kin ok Ęikin, \hld\ \alst{S}vǫl ok Gunn-þró, &
\ind \alst{F}jǫrm ok \alst{F}imbul-þul, &
\ind \alst{R}ín ok \alst{R}innandi, &
\alst{G}ipul ok \alst{G}ǫpul, \hld\ \alst{G}ǫmul ok \alst{G}ęir-vimul, &
\ind þę́r \alst{h}verfa umb \alst{h}odd goða, &
\alst{Þ}yn ok Vin, \hld\ \alst{Þ}ǫll ok Hǫll, &
\ind \alst{G}rǫ́ð ok \alst{G}unn-þorin.\eva

\bvb Side and Wide, Seeken and Oaken, Swale and Guththrew, \\
\ind Ferm and Fimblethule, \\
\ind Rine and Rinnend, \\
Gipple, Gapple, Gamble and Garwimble— \\
\ind they run around the hoard of the Gods \ken*{= Osyard}— \\
Thin and Win, Thall and Hall, \\
\ind Gread and Guththorn.\evb\evg


\bvg\bva\mssnote{\Regius~10r/12, \AM~5r/1}%
\alst{V}ína hęitir enn, \hld\ ǫnnur \alst{V}eg-svinn, &
\ind \alst{þ}riðja \alst{Þ}jóð-numa; &
\alst{N}yt ok \alst{N}ǫt, \hld\ \alst{N}ǫnn ok \alst{H}rǫnn, &
\alst{S}líð ok \alst{H}ríð, \hld\ \alst{S}ylgr ok Ylgr, &
\alst{V}íð ok \alst{V}ǫ́n, \hld\ \alst{V}ǫnd ok Strǫnd, &
\alst{G}jǫll ok Lęiptr; \hld\ þę́r falla \alst{g}umnum nę́r &
\ind es falla til \alst{h}ęljar \alst{h}eðan. \eva

\bvb Wine is one further called, another Wayswith, \\
\ind a third Thedenumb; \\
Nit and Nat, Nan and Ran, \\
Slithe and Rithe, Sellow and Wellow, \\
Wide and Ween, Wand and Strand, \\
Yell and Laft—they fall near to men \\
\ind as they fall hence to Hell.\evb\evg


\bvg\bva\mssnote{\Regius~10r/15, \AM~5r/4, \GylfMS}%
\alst{K}ǫrmt ok Ǫrmt \hld\ ok \alst{k}ęr-laugar tvę́r &
\ind \edtrans{\alst{þ}ę́r skal \alst{Þ}órr vaða}{these shall Thunder wade}{\Bfootnote{Thunder is commonly associated with wading.  See TODO.}} &
\alst{d}ag hvęrn \hld\ es \alst{d}ǿma fęrr &
\ind at \alst{a}ski \alst{Y}gg-drasils; &
því-at \alst{ǫ́}s-brú \hld\ bręnn \alst{ǫ}ll loga &
\ind \alst{h}ęilǫg vǫtn \edtrans{\alst{h}lóa}{bellow}{\Bfootnote{A hapax. TODO.}}.\eva

\bvb Carmt and Armt, and the two Carlays, \\
\ind these shall Thunder wade \\
every day, when to judge he goes, \\
\ind at \inx[L]{Ugdrassle’s Ash}; \\
for the \inx[G]{eese}[os]-bridge \ken{rainbow} burns all with flame; \\
\ind the holy waters bellow.\evb\evg


\bvg\bva\mssnote{\Regius~10r/17, \AM~5r/6}%
\alst{G}laðr ok \alst{G}yllir, \hld\ \alst{G}lęr ok Skęið-brimir, &
\ind \alst{S}ilfrin-toppr ok \alst{S}inir, &
\alst{G}ísl ok Fal-hófnir, \hld\ \alst{G}ull-toppr ok Létt-feti, &
\ind þęim ríða \alst{ę́}sir \alst{jó}um &
\alst{d}ag hvęrn \hld\ es \alst{d}ǿma fara &
\ind at \alst{a}ski \alst{Y}gg-drasils.\eva

\bvb Glad and Gilder, Glare and Sheathbrimmer, \\
\ind Silvrentop and Sinewer; \\
Yissel and Fallowhofner, Goldtop and Lightfeet; \\
\ind on these horses ride the Eese, \\
every day, when to judge they go, \\
\ind at \inx[L]{Ugdrassle’s Ash}.\evb\evg


\bvg\bva\mssnote{\Regius~10r/20, \AM~5r/8}%
\alst{Þ}ríar rǿtr \hld\ standa á \alst{þ}ría vega &
\ind undan \alst{a}ski \alst{Y}gg-drasils; &
\alst{H}ęl býr und \alst{ęi}nni, \hld\ \alst{a}nnarri \alst{h}rím-þursar, &
\ind þriðju \alst{m}ęnnskir \alst{m}ęnn.\eva

\bvb Three roots grow on three ways, \\
\ind from beneath Ugdrassle’s Ash. \\
Hell lives enclosed by one, [by] the other the \inx[G]{Rime-Thurses}, \\
\ind {[by]} the third manly men.\evb\evg


\bvg\bva\mssnote{\Regius~10r/22, \AM~5r/9}%
\alst{R}ata-toskr hęitir íkorni \hld\ es \alst{r}inna skal &
\ind at \alst{a}ski \alst{Y}gg-drasils; &
\alst{a}rnar \alst{o}rð \hld\ hann skal \alst{o}fan bera &
\ind ok sęgja \alst{N}íð-hǫggvi \alst{n}iðr.\eva

\bvb Wratetusk is the squirrel called who shall run \\
\ind at Ugdrassle’s Ash. \\
The eagle’s words he shall carry from above, \\
\ind and say to Nithehewer below.\footnoteB{This st. and the following is paraphrased in \Gylfaginning\ 16 (excerpt):
\begin{quote}
  \emph{Þá mę́lti Gangleri: „Hvat er fleira at segja stór-merkja frá askinum?“ Hár segir: „Mart er þar af at segia. Ǫrn einn sitr í limum asksins, ok er hann margs vitandi, en í milli augna honum sitr haukr sá, er heitir Veðrfǫlnir. Íkorni sá, er heitir Rata-toskr, rennr upp ok niðr eptir askinum ok berr ǫfundar orð millum arnarins ok Níðhǫggs.} ‘Gangler spoke: “What more great marks are there to be said about the ash?” High says: “There is much to say about it. An eagle sits in the limbs of the ash, and he is much knowing, but between his eyes sits the hawk called Weatherfalner. The squirrel, which is called Wratetush, runs up and down along the ash and carries words of spite between the eagle and Nithehewer.”’
\end{quote}}\evb\evg


\bvg\bva\mssnote{\Regius~10r/23, \AM~5r/11}%
\alst{H}irtir ’ru ok fjórir \hld\ þęir’s af \alst{h}ę́fingar &
\ind á \alst{g}ag-halsir \alst{g}naga: &
\alst{D}áinn ok \alst{D}valinn, \hld\ \alst{D}ún-ęyrr ok \alst{D}ura-þrór.\eva

\bvb Harts are there also, four, those who TODO \\
\ind TODO gnaw: \\
Dowen and Dwollen, Downeer and Doorthrew.\footnoteB{Paraphrased in \Gylfaginning\ 16 immediately following a paraphrase of the last st.: \emph{En fjórir hirtir renna í limum asksins ok bíta barr; þeir heita svá: Dáinn, Dvalinn, Dún-eyrr, Dura-þrór.} ‘But four harts run in the limbs of the ash and bite its leaves; they are called thus: Dowen, Dwollen, Downeer, Doorthrew.’}\evb\evg


\bvg\bva\mssnote{\Regius~10r/25, \AM~5r/12, \GylfMS}%
\alst{O}rmar flęiri \hld\ liggja und \alst{a}ski \alst{Y}gg-drasils &
\ind an þat of \alst{h}yggi \alst{h}vęrr &
\ind \alst{ó}-sviðra \alst{a}pa:\eva

\bvb More worms lie under Ugdrassle’s Ash \\
\ind than any one would think \\
\ind among unwise \inx[C]{ape}[apes]:\footnoteB{Paraphrased in \Gylfaginning\ 16: \emph{En svá margir ormar eru í Hvergelmi með Níðhǫgg, at engi tunga má telja; svá segir hér:} ‘But so many worms are in Wharyelmer with Nithehewer that no tongue may count them. So it says here:’ after which st. 36 is quoted.}\evb\evg


\bvg\bva\mssnote{\Regius~10r/26, \AM~5r/13, \GylfMS}%
\alst{G}óinn ok Móinn, \hld\ þęir ’ru \alst{G}raf-vitnis synir, &
\ind \alst{G}rá-bakr ok \alst{G}raf-vǫlluðr, &
\alst{O}fnir ok Sváfnir, \hld\ hygg’k at \alst{ę́} skyli &
\ind \alst{m}ęiðs kvistu \alst{m}áa.\eva

\bvb Gowen and Mowen—they are Gravewitner’s sons— \\
\ind Greyback and Gravewalled; \\
Ovner and Swebner, I ween, shall always \\
\ind injure the beam’s branches.\evb\evg


\bvg\bva\mssnote{\Regius~10r/28, \AM~5r/14}%
\alst{A}skr \alst{Y}gg-drasils \hld\ drýgir \alst{ę}rfiði &
\ind \alst{m}ęira an \alst{m}ęnn viti: &
\alst{h}jǫrtr bítr ofan \hld\ en á \alst{h}liðu fúnar, &
\ind skęrðir \alst{N}íð-hǫggr \alst{n}eðan.\eva

\bvb Ugdrassle’s Ash suffers hardship \\
\ind greater than men might know: \\
a hart bites it above and it rots on the side; \\
\ind Nithehewer harms it below.\evb\evg


\bvg\bva\mssnote{\Regius~10r/30, \AM~5r/16}%
\alst{H}rist ok Mist \hld\ vil’k at mér \alst{h}orn beri, &
\ind \alst{Sk}eggj-ǫld ok \alst{Sk}ǫgul, &
\edtrans{\alst{H}ildr ok Þrúðr}{Hild and Thrith}{\Afootnote{so \AM; \emph{Hildi ok Þrúði} \Regius\ stems from \emph{ꝺꝛ, ðꝛ} with r rotunda being interpreted and copied as \emph{ꝺı, ðr}, this becomes clear upon viewing the facsimile images.}}, \hld\ \alst{H}lǫkk ok \alst{H}ęr-fjǫtur, &
\ind \alst{G}ǫll ok \alst{G}ęir-ǫlul, &
\alst{R}and-gríð ok \alst{R}áð-gríð, \hld\ \alst{R}ęgin-lęif; &
\ind \edtrans{þę́r bera \alst{ęi}n-hęrjum \alst{ǫ}l.}{they bring the Oneharriers ale.}{\Bfootnote{As cupbearers in Walhall.  Pouring drinks was traditionally done by the ruler’s kinswomen during a feast, in heroic legend most famously Rothgar’s wife and daughter in \Beowulf.  The Walkirries may be daughters of Weden; see note to \Voluspa\ 30/5.  For the reception of dead warriors see also note to st. 53/3 below.}}\eva

\bvb Rist and Mist I would have bring me a horn— \\
\ind Shageld and Shagle; \\
Hild and Thrith, Lank and Harfetter, \\
\ind Gall and Garannel, \\
Randgrith and Redegrith, Rainlaf— \\
\ind they bring the Oneharriers ale.\evb\evg


\bvg\bva\mssnote{\Regius~10r/32, \AM~5r/18}%
\edtrans{\alst{Á}r-vakr ok \alst{A}l-sviðr}{Yorewaker and Allswith}{\Bfootnote{These horses also appear in \Sigrdrifumal\ 15a/2; see note to the next st.}}, \hld\ skulu \alst{u}pp heðan &
\ind \alst{s}vangir \alst{s}ól draga; &
en und þęira \alst{b}ógum \hld\ fǫ́lu \alst{b}líð ręgin, &
\ind \alst{ę́}sir, \alst{í}sarn-kol.\eva

\bvb Yorewaker and Allswith shall from hence— \\
\ind slender [steeds]—pull up the sun, \\
and under their shoulders the blithe Reins hid \\
\ind —the Eese—iron-cooling.\footnoteB{According to \Gylfaginning\ 11 the gods took two horses to pull the sun’s chariot—Yorewaker and Allswith—and “under the shoulders of the horses the gods placed two wind-bellows to cool them, but in some sources (\emph{í sumum frǿðum}, presumably this st.) they are called iron-cooling (\emph{ísarn-kol}).”}\evb\evg


\bvg\bva\mssnote{\Regius~10v/2, \AM~5r/20}%
\alst{S}valinn hęitir, \hld\ hann stęndr \alst{s}ólu fyrir, &
\ind \alst{sk}jǫldr \alst{sk}ínanda goði; &
\alst{b}jǫrg ok \alst{b}rim \hld\ vęit’k at \alst{b}rinna skulu, &
\ind ef hann \alst{f}ęllr í \alst{f}rá.\eva

\bvb Swalen one is called, it stands before the sun: \\
\ind a shield [before] the shining god \ken{sun}. \\
Crags and surf I know shall burn, \\
\ind if it falls away.\footnoteB{The sun-disc was apparently thought to be a translucent shield, which protected the earth from the full power of the Sun behind it. Without it the whole world (“crags and surf”, \textsc{land} and \textsc{sea}; the totality of the earth) would burn up.  Cf. \Sigrdrifumal\ 15a/1, which mentions the “shield that stands before the shining god \ken{sun}”.}\evb\evg


\bvg\bva\mssnote{\Regius~10v/4, \AM~5r/21}%
\edtext{\alst{Sk}oll hęitir ulfr, \hld\ es fylgir hinu \alst{sk}ír-lęita &
\ind goði til \alst{v}arna \alst{v}iðar, &
en annarr \alst{H}ati, \hld\ hann ’s \alst{H}róð-vitnis sonr, &
\ind sá skal fyr \alst{h}ęiða brúði \alst{h}imins.}{\lemma{ALL}\Bfootnote{According to \Gylfaginning\ 12 Scoll chases the Sun and Hate chases the Moon (which is why he runs in front of the sun).  See note to \Voluspa\ 40 for discussion on these wolves.}}\eva

\bvb \inx[P]{Scoll} is the wolf called who follows the pure-faced \\
\ind god \ken*{= Sun} to the shelter of the woods, \\
but second \inx[P]{Hate}; he is \inx[P]{Rothwitner}’s son— \\
\ind who shall [run] in front of the bright bride of heaven \ken*{= Sun}.\evb\evg


\bvg\bva\mssnote{\Regius~10v/6, \AM~5r/23, \\ \AMb~9v/14, \EddaBms~3v/11}%TODO: Critical notes for these next two stanzas based on the mss. Sigla for ms. B = AM 757 a 4°.
\edtext{Ór \alst{Y}mis holdi \hld\ vas \alst{jǫ}rð of skǫpuð, &
\ind en ór \edtrans{\alst{s}vęita}{blood}{\Afootnote{\emph{hans sára svęita} ‘blood of his wounds’ \AMb\EddaBms}} \edtext{\alst{s}jór}{\Afootnote{so \AM\AMb\EddaBms; \emph{sę́r} \Regius}}, &
\alst{b}jǫrg ór \alst{b}ęinum, \hld\ \alst{b}aðmr ór hári, &
\ind en \edtrans{ór \alst{h}ausi \alst{h}iminn}{from his skull the heaven}{\Afootnote{\emph{himinn ór hausi hans} ‘the heaven from his skull’ \AMb\EddaBms}}.}{\lemma{ALL}\Bfootnote{This stanza is clearly closely related to \Vafthrudnismal\ 21; see there for notes.}}\eva

\bvb From \inx[P]{Yimer}’s flesh was the earth shaped, \\
\ind and from his blood the sea, \\
mountains from his bones, woods from his hair, \\
\ind and from his skull the heaven.\evb\evg


\bvg\bva\mssnote{\Regius~10v/8, \AM~5r/25, \\ \AMb~9v/16, \EddaBms~3v/12}%
\edtext{En ór hans \alst{b}rǫ́um \hld\ gørðu \alst{b}líð ręgin &
\ind \alst{M}ið-garð \alst{m}anna sonum,}{\lemma{En ór hans brǫ́um \hld\ gørðu blíð ręgin / Mið-garð manna sonum ‘And from his brows the blithe Reins made Middenyard for the sons of men’}\Bfootnote{The Gods fenced in Middenyard (‘the middle enclosure’) by using the strands of Yimer’s eyebrows as poles.}} &
en ór hans \alst{h}ęila \hld\ vǫ́ru þau hin \edtrans{\alst{h}arð-móðgu}{hard-minded}{\Afootnote{\emph{hríð-fęldu} ‘stormy’ \AMb\EddaBms}} &
\ind \alst{sk}ý ǫll of \alst{sk}ǫpuð.\eva

\bvb And from his brows the blithe \inx[G]{Reins} made \\
\ind \inx[L]{Middenyard} for the sons of men, \\
and from his brains were the hard-minded \\
\ind clouds all shaped.\evb\evg


\bvg\bva\mssnote{\Regius~10v/9, \AM~5r/26}%
\edtext{\edtrans{\alst{U}llar}{Woulder’s}{\Bfootnote{It is uncertain why the rather obscure god Woulder is invoked here.  It cannot be simply for the sake of alliteration, since \emph{Óðins} ‘Weden’s’ would work just as well.  It is possible that Woulder had a particular role in the setting of the ritual fire, which would find support in the large number of firesteel-shaped amulets at the archeological site of \emph{Lilla Ullevi} (‘Woulder’s little \inx[C]{wigh}’) in Sweden; see Index: \inx[P]{Woulder} and \textcite{afEdholm2009}.}} \edtrans{hylli}{holdness}{\Bfootnote{‘Favour, loyalty, grace’.  This root (from which also the adjective \emph{hollr} ‘hold; favourable, loyal, gracious’ and verb \emph{hylla} ‘to make hold’) is used to refer to the grace of god(s) in both Heathen and Christian texts.  See Index: \inx[C]{hold} and \inx[C]{holdness}.}} \hld\ hęfr ok \edtrans{\alst{a}llra goða}{All Gods}{\Bfootnote{Cf. \Sigrdrifumal\ 3–4, \Lokasenna\ 11, which both hail the Gods as a collective (the former as part of a genuine prayer, the latter subversively).  For the oneness of the Gods see Index: \inx[C]{All Gods}.}} &
\ind hvęrr’s \edtext{tękr \alst{f}yrstr ȧ \alst{f}una}{\lemma{tękr \dots\ ȧ funa ‘starts the fire’}\Bfootnote{An otherwise unattested phrase, for which cf. \emph{taka ęld} ‘light a fire’. With \emph{ȧ} ‘on’ the verb \emph{taka} ‘take’ has a variety of idiomatic senses like ‘touch, react to, get involved in, get on, et c’.}}, &
því-at \alst{o}pnir hęimar \hld\ verða umb \alst{ȧ}sa sonum, &
\ind þȧ’s \alst{h}ęfja af \edtrans{\alst{h}vera}{kettles}{\Bfootnote{Acc. pl. of \emph{hverr}, from PGmc. \emph{*hweraz}, from PIE \emph{*kʷer-} ‘pot, vessel’.  The Sanskrit cognate \emph{carú} is occasionally used in reference to the vat from which the ritual drink \emph{sóma} is drunk (\Rigveda\ 10.167.4), but any particular religious significance for the PIE root cannot be reconstructed.}}.}{\lemma{ALL}\Bfootnote{This st. is one of the most difficult in the poem and many interpretations have been made.

\indent The traditional view (e.g. \textcite{FinnurEdda}, Bellows, Sijmons and Gering (p. 208)) relates it to the poem’s frame narrative.  Weden, bound between the two fires, cryptically asks for a cauldron hanging above him from the roof to be moved aside so that the Gods will be able to see him through the smoke-vent and rescue him.  This explanation leaves very much unexplained, namely the stanza’s placement in the gnomic wisdom section of the poem (unless the whole section is taken to be a later insert—so Finnur—, for which there is no textual support), the invocation of the obscure god Woulder, the lack of mention of a cauldron elsewhere in the poem, and the big question of why the gods would bestow their grace unto the person who first set the fire which is presently torturing Weden.

I find the interpretation of \textcite{Nordberg2005} more convincing.  He argues that the st. is another piece of gnomic wisdom, referring to the cooking of the sacrificial meal in large cauldrons during the \inx[C]{bloot}.  This has textual support, e.g. \HakonarSaga\ 14, describing the traditional bloot in the Throndlaw (\emph{Þrǿnda-lǫg}), Norway: \emph{At veizlu þeiri skyldu allir menn ǫl eiga; þar var ok drepinn alls konar smali ok svá hross, [...] en slátr skyldi sjóða til mann-fagnaðar; eldar skyldu vera á miðju gólfi í hofinu ok þar katlar yfir.} ‘At that gathering all men were to have ale; thereat were also slain all kinds of small cattle and likewise horses, [...] and the fresh meat was to be be cooked for men to enjoy.  There were to be fires in the middle of the floor in the hove and kettles above them.’  According to this view, the stanza is speaking of the Heavenly favour (\emph{hylli}) earned by the ritualist who sets the cooking fire, since that act enables the Gods to become guests at the ritual meal.

Nordberg’s interpretation is especially interesting when one considers the immediately preceding stanzas 41–42 which describe the ordering of the world by the Gods through the sacrifice and dismembering of Yimer, the primordial victim.  (That the slaying of Yimer was in fact a sacrifice is supported by the manner in which it was done, viz. beheading, which was the typical manner of slaying sacrificial bulls in the Wiking Age; see note to \Vafthrudnismal\ 21/4.)  In other Indo-European religions—most famously the Vedic \emph{Púruṣa}, \Rigveda\ 10.90—the first sacrifice of a Great Being serves as the model for all future sacrifice, the performance of which reenacts the creation and enables the continued existence of the world and the social order \parencite{Lincoln1986}, and the sequence \Grimnismal\ 41–43 would then attest this also in the Germanic tradition.   For the role of fire in Germanic and Vedic sacrifice see \textcite{Kaliff2005}.}}\eva

\bvb \inx[P]{Woulder}’s \inx[C]{holdness} and that of \inx[C]{All Gods} \\
\ind has whoever first starts the fire, \\
for the \inx[C]{Home}[Homes] open up for the sons of the Eese \ken{gods}, \\
\ind when men lift off the kettles.\evb\evg


\bvg\bva\mssnote{\Regius~10v/11, \AM~5r/28}%
\alst{Í}valda synir \hld\ gingu í \alst{á}r-daga &
\ind \alst{Sk}íð-blaðni at \alst{sk}apa, &
\alst{sk}ipa batst \hld\ \alst{sk}írum Fręy, &
\ind \alst{n}ýtum \alst{N}jarðar bur.\eva

\bvb Iwald’s sons went in days of yore \\
\ind Shidebladner for to shape: \\
the best of ships for the pure Free, \\
\ind for the useful Son of Nearth.\evb\evg


\bvg\bva\mssnote{\Regius~10v/13, \AM~5r/29}%
\alst{A}skr \alst{Y}gg-drasils, \hld\ hann ’s \alst{ǿ}ðstr viða &
\ind en \alst{Sk}íð-blaðnir \alst{sk}ipa, &
\alst{Ó}ðinn \alst{á}sa \hld\ en \alst{jó}a Slęipnir, &
\alst{B}il-rǫst \alst{b}rúa \hld\ en \alst{B}ragi skalda, &
\alst{H}á-brók \alst{h}auka \hld\ en \alst{h}unda Garmr.\eva

\bvb Ugdrassle’s Ash—it is the noblest of trees, \\
\ind and Shidebladner of ships; \\
Weden of the Eese and Slapner of steeds; \\
Bilrest of bridges and Bray of scolds; \\
Highbrook of hawks and Garm of hounds.\evb\evg

\sectionline

\bvg\bva\mssnote{\Regius~10v/15, \AM~5v/2}%
\alst{S}vipum hęf’k nú ypt \hld\ fyr \alst{s}ig-tíva sonum, &
\ind við þat skal \alst{v}il-bjǫrg \alst{v}aka, &
\alst{ǫ}llum \alst{ǫ́}sum \hld\ þat skal \alst{i}nn koma &
\ind \alst{Ę́}gis bękki \alst{á} &
\ind \alst{Ę́}gis drekku \alst{a}t.\eva

\bvb My gaze I’ve now lifted up before the sons of the victory-Tews \ken*{= Eese}— \\
\ind by that shall the willed rescue awake! \\
All the Eese shall it bring in, \\
\ind upon Eagre’s bench, \\
\ind at Eagre’s drinking!\footnoteB{Weden suddenly announces that he has made the other gods aware of his situation; they will leave their feasting at Eagre’s hall (see \Hymiskvida\ and \Lokasenna) and instead come to his rescue.  He then begins to recount his names.}\evb\evg


\bvg\bva\mssnote{\Regius~10v/17, \AM~5v/4, \GylfMS}%
Hétumk \alst{G}rímr, \hld\ hétumk \alst{G}anglęri, &
\ind \alst{H}ęrjann ok \alst{H}jalm-beri, &
\alst{Þ}ękkr ok \alst{Þ}riði, \hld\ \alst{Þ}undr ok Uðr, &
\ind \alst{H}ęl-blindi ok \alst{H}ár.\eva

\bvb I called myself Grim, I called myself Gangler, \\
\ind Harn and Helmbearer. \\
Theck and Third, Thound and Ith, \\
\ind Hellblinder and High.\evb\evg


\bvg\bva\mssnote{\Regius~10v/19, \AM~5v/5, \GylfMS}%
\alst{S}aðr ok \alst{S}vipall \hld\ ok \alst{S}ann-getall, &
\ind \alst{H}ęr-tęitr ok \alst{H}nikarr, &
\alst{B}il-ęygr, \alst{B}ál-ęygr, \hld\ \alst{B}ǫl-verkr, Fjǫlnir, &
\alst{G}rímr ok \alst{G}rímnir, \hld\ \alst{G}lap-sviðr ok Fjǫl-sviðr.\eva

\bvb Sooth and Swiple and Soothgettle, \\
\ind Hartote and Nicker, \\
Bileye, Baleeye, Baleworker, Fillner, \\
Grim and Grimner, Glapswith and Fellswith.\evb\evg


\bvg\bva\mssnote{\Regius~10v/21, \AM~5v/7, \GylfMS}%
\alst{S}íð-hǫttr, \alst{S}íð-skęggr, \hld\ \alst{S}ig-fǫðr, Hnikuðr, &
\alst{A}l-fǫðr, \alst{V}al-fǫðr, \hld\ \alst{A}t-ríðr ok Farma-týr; &
\alst{ęi}nu nafni \hld\ hétumk \alst{a}ldri-gi &
\ind síðst ek með \alst{f}olkum \alst{f}ór.\eva

\bvb Sidehat, Sideshag, Syefather, Nicked, \\
Allfather, Walfather, Atrider, and Farm-Tew— \\
by just one name have I never called myself, \\
\ind since among manfolk I fared.\evb\evg


\bvg\bva\mssnote{\Regius~10v/23, \AM~5v/9}%
\alst{G}rímni mik hétu \hld\ at \alst{G}ęir-raðar, &
\ind en \alst{Ja}lk at \alst{Ǫ́}s-mundar; &
en þá \alst{K}jalar \hld\ es ek \alst{k}jalka dró, &
\ind \alst{Þ}rór \alst{þ}ingum at.\eva

\bvb Grimner they called me at Garfrith’s [home], \\
\ind but Yelk at Osmund’s, \\
but Keller whenas I drew the sled; \\
\ind Throo at \inx[C]{Thing}[Things].\footnoteB{Presumably referencing other now-lost myths involving Weden travelling in disguise. The last is possibly a reference to the name under which Weden would be invoked at the start of Things (legal assemblies, see Index).}\evb\evg


\bvg\bva\mssnote{\Regius~10v/24, \AM~5v/10, \GylfMS}%
\alst{Ó}ski ok \alst{Ó}mi, \hld\ \alst{Ja}fn-hár ok Biflindi, &
\ind \alst{G}ǫndlir ok Hár-barðr með \alst{g}oðum.\eva

\bvb Wish and Ome, Evenhigh and Bivlend; \\
\ind Gandler and Hoarbeard among Gods.\evb\evg


\bvg\bva\mssnote{\Regius~10v/25, \AM~5v/11}%
\alst{S}viðurr ok \alst{S}viðrir \hld\ es ek hét at \alst{S}økk-mímis &
\ind ok dulða’k þann hinn \alst{a}ldna \alst{jǫ}tun &
þá’s \alst{M}ið-vitnis vas’k \hld\ ins \alst{m}ę́ra burar &
\ind \alst{o}rðinn \alst{ęi}n-bani.\eva

\bvb Swither and Swithrer, as I was called at Sink-Mimer’s, \\
\ind and I deceived that aged ettin, \\
when of Midwitner’s famous son \\
\ind I had become the lone slayer.\evb\evg


\bvg\bva\mssnote{\Regius~10v/28, \AM~5v/13}%
\alst{Ǫ}lr est Gęir-røðr, \hld\ hęfr þú \alst{o}f-drukkit; &
\alst{m}iklu est hnugginn, \hld\ es þú est \alst{m}ínu gęngi, &
\edtrans{\alst{ǫ}llum \alst{ęi}n-hęrjum}{of all the Oneharriers}{\Bfootnote{Linguistically, Garfrith is not bereft of the support of the Oneharriers but rather of the Oneharriers themselves, but the sense is the same.  By breaking the Odinic code of conduct he has lost Weden’s favour, and thus been excluded from the community of oath-bound warriors, the Oneharriers.

On the other hand a righteous king could expect to have the truce of the Oneharriers; this was the case for Hathkin the Good according to the poem composed about him (Eyv \emph{Hák} in \Skp\ 1).  In that poem (st. 16/1–2) \inx[P]{Bray} greets him in the hall of the Gods, saying: \emph{Ęin-hęrja grið · skalt allra hafa; / þigg þú at ǫ́sum ǫl.} ‘All the Oneharriers’ truce shalt thou have; accept ale from the \inx[G]{Eese}!’}} \hld\ ok \alst{Ó}ðins hylli.\eva

\bvb Worse for ale art thou, Garfrith; thou hast over-drunk. \\
Of much art thou bereft when thou art [bereft] of my support, \\
of all the \inx[G]{Oneharriers}, and of Weden’s \inx[C]{holdness}.\evb\evg


\bvg\bva\mssnote{\Regius~10v/30, \AM~5v/15}%
\alst{F}jǫlð þér sagða’k, \hld\ en þú \alst{f}átt of mant, &
\ind of þik \alst{v}éla \edtext{\alst{v}inir}{\linenum{|2|||3}\lemma{vinir, míns vinar ‘friends, my friend’}\Bfootnote{Weden stresses his friendship with Garfrith by using the word \emph{vinr} ‘friend’ twice.  The followers of a god were his friends; see note to \Havamal\ 157.}}; &
\edtext{\alst{m}ę́ki liggja \hld\ sé’k \alst{m}íns vinar &
\ind allan í \alst{d}ręyra \alst{d}rifinn.}{\lemma{mę́ki liggja \hld\ sé’k míns vinar / allan í dręyra drifinn. ‘The sword of my friend I see lying all drenched in gore.’}\Bfootnote{Weden foresees Garfrith’s imminent death.}}\eva

\bvb Much I told thee, but thou recallest little; \\
\ind ’tis friends that deal with thee! \\
The sword of my friend I see lying \\
\ind all drenched in gore.\evb\evg


\bvg\bva\mssnote{\Regius~10v/31, \AM~5v/16}%
\alst{Ę}gg-móðan val \hld\ nú mun \alst{Y}ggr hafa, &
\ind þitt vęit’k \alst{l}íf of \alst{l}iðit; &
\alst{v}arar ’ru \edtrans{dísir}{Dises}{\Bfootnote{The Norns, fates, who have determined his hour of death.  Cf. \Fafnismal\ TODO, \Hamdismal\ TODO.}}, \hld\ nú knátt \alst{Ó}ðin séa; &
\ind nálgask \alst{m}ik ef þú \alst{m}ęgir!\eva

\bvb An edge-tired corpse will Ug now have: \\
\ind I know thy life to be past. \\
Wary are the \inx[G]{Dises}, now dost thou see Weden— \\
\ind come near me, if thou mayst!\evb\evg


\bvg\bva\mssnote{\Regius~11r/2, \AM~5v/18}%
\alst{Ó}ðinn nú hęiti’k, \hld\ \alst{Y}ggr áðan hét’k, &
\ind hétumk \alst{Þ}undr fyr \alst{þ}at, &
\alst{V}akr ok Skilfingr, \hld\ \alst{V}ǫ́fuðr ok Hropta-týr &
\ind \alst{G}autr ok Jalkr með \alst{g}oðum.\eva

\bvb Weden am I called now, Ug was I called earlier, \\
\ind I called myself Thound before that; \\
Wacker and Shilving, Waved and Roft-Tew, \\
\ind Geat and Gelding among the Gods.\evb\evg


\bvg\bva\mssnote{\Regius~11r/4, \AM~5v/20}%
\edtrans{\alst{O}fnir ok Sváfnir}{Ovner and Swebner}{\Bfootnote{The names of two serpents in 35/3a above.}} \hld\ hygg’k at \alst{o}rðnir sé &
\ind \alst{a}llir at \alst{ęi}num mér.\eva

\bvb Ovner and Swebner, I ween, have come \\
\ind all from me alone.\evb\evg


\bpg\bpa\mssnote{\Regius~11r/5, \AM~5v/21}%
Geir-røðr konungr sat, ok hafði sverð um kné sér ok brugðit til miðs. En er hann heyrði, at Óðinn var þar kominn, stóð hann upp, ok vildi taka Óðin frá eldinum. Sverðit slapp ór hendi hánum; vissu hjǫltin niðr. Konungr drap fę́ti, ok steyptist á-fram, en sverðit stóð í gǫgnum hann, ok fekk \edtext{hann}{\Afootnote{þar af \AM}} bana. \edtext{Óðinn hvarf þá.}{\Afootnote{om. \AM}} En Agnarr \edtext{var þar}{\Afootnote{varð \AM}} konungr \edtext{lengi síðan.}{\Afootnote{om. \AM}}\epa

\bpb King Garfrith sat and had a sword about his knee, and it was brandished half-way up. And when he heard that Weden were come there, he stood up and would take Weden from the fire. The sword slipped out of his hand; the hilt pointed downwards. The king tripped and stooped forth, but the sword went through him, and he received his bane. Weden then disappeared, but Ayner was there king for a long while afterwards.\epb\epg
% Weden
	\bookStart{Speeches of Shirner}[Skírnismǫ́l]

\begin{flushright}%
\textbf{Dating} \parencite{Sapp2022}: C10th (0.897)

\textbf{Meter:} \Ljodahattr, \Galdralag\ (TODO)%
\end{flushright}

\section{Introduction}

The \textbf{Speeches of Shirner} (\Skirnismal) are attested in full in both \Regius\ and \AM.  The name \emph{Skírnismǫ́l} ‘Speeches of Shirner’ comes from \AM; \Regius\ instead has \emph{Fǫr Skírnis} ‘Shirner’s journey’.

The same narrative is found in \Gylfaginning\ 37, which also quotes one stanza of the present poem.  That account begins with a long introduction, corresponding to P1–2:

\begin{quote}‘Gymer was a man called, and his woman Earbode; she was of the lineage of mountain-risers. Their daughter is Gird, who is fairest of all women.  It was one day when Free had gone to Lithshelf and looked about all the Homes.  And when he looked north he saw on a farm a great and fine house, and to that house walked a woman, and when she lifted her hands and closed the doors behind her it shone from her hands into both the air and onto the waters, and all the homes were brightened by her.  And that beauty which he had seen in that holy seat harmed him so greatly that he walked away filled with grief, and when he came home he spoke nothing; he neither slept nor drank.  Noone dared to get words out of him.’\end{quote}

After this it paraphrases sts. 3–9, describing Shirner’s interaction with Free:

\begin{quote}‘Then Nearth had Shirner, Free’s shoe-swain, called unto him, and asked him to go to Free and bid him to speak and ask at whom he was so wroth that he would not speak with men.  And Shirner said that he would go, although not eagerly, and said that he expected ill answers from him.

And when he came to Free he asked why Free were so downcast and spoke nothing with men.  Then Free answers, and said that he had seen a fair woman and for her sakes was he so full of grief that he would not live long if he should not reach her, “and now shalt thou journey to ask for her hand for me, and have her home hither whether her father wants to or not, and I shall reward thee well for that.”

Then Shirner answers; said so, that he will go on the errand-journey, but Free shall give him his sword; it was such a good sword that it struck by itself.  And Free did not refuse that and gave him the sword.’\end{quote}

The rest of the poem (sts. 10–38) is summarised very succinctly:

\begin{quote}‘Then Shirner journeyed and asked for the woman’s [Gird’s] hand for him [Free], and got her promise that nine nights later she would come to that place which is called Barrey and have a wedding with Free.  And when Shirner told Free his errand, then he quoth this:’\end{quote}

After which the author quotes a variant of stanza 42, with some minor differences in wording that seem to stem from oral tradition (see Note to that st.)  He last explains that \emph{Þessi sǫk er til þess, er Freyr var svá vápn-lauss, er hann barðist við Belja ok drap hann með hjartar-horni.} ‘This is the reason for why Free was so weaponless when he fought against Bellow, and he slew him with a hart’s horn.’

It seems near-certain that the author of \Gylfaginning\ had access to a version of \Skirnismal; not a single detail in his paraphrase is not found in the present version of the poem, although the introductory prose differs a fair bit, and Shirner’s curse is entirely omitted.  This is easily understood if his version was written down from a slightly different oral tradition; the poetry, being in bound form, would be much more stable than the more fluid introductory prose.

To sum up a narrative mythic poem in prose form and then quote one or two stanzas is something probably done elsewhere in \Gylfaginning; see the Eddic fragments from Snorre’s Edda below.

\sectionline

\section{The Speeches of Shirner}

\bpg\bpa\mssnote{\Regius~11r/10, \AM~2r/11}%
Freyr, sonr Njarðar, hafði einn dag setsk í Hlið-skjálf ok sá um heima alla; hann sá í Jǫtun-heima ok sá þar mey fagra, þá er hon gekk frá skála fǫður síns til skemmu; þar af fekk hann hug-sóttir miklar. Skírnir hét skó-sveinn Freys. Njǫrðr bað hann kveðja Frey máls. Þá mę́lti Skaði:\epa

\bpb \inx[P]{Free}, son of \inx[P]{Nearth}, had one day set himself in \inx[L]{Lithshelf} and looked about all the \inx[C]{Homes}.  He looked into the \inx[L]{Ettinhomes} and saw there a fair maiden as she walked from her father’s hall to her bower; thereof he got great heart-aches.  \inx[P]{Shirner} was called the shoe-swain of Free.  Nearth asked him to speak with Free.  Then \inx[P]{Shede} spoke:\epb\epg


\bvg\bva\mssnote{\Regius~11r/14, \AM~2r/15}%
„\edtext{Rís-tu nú Skírnir \hld\ ok gakk at bęiða}{\lemma{rís \dots\ bęiða ‘Rise \dots\ ask’}\Bfootnote{Alliteration is missing here. A simple solution would be to replace \emph{gakk} ‘go’ with a synonym like \emph{rinn} ‘run’ or \emph{ráð} ‘resolve’, but this lessens the semantic mirroring with l. 2/2 below (though, the insertion of the verb \emph{ganga} in the present stanza may in fact be due to influence from 2/2).}} &
\ind okkarn \alst{m}ála \alst{m}ǫg, &
ok þess at \alst{f}regna \hld\ hvęim hinn \alst{f}róði séi &
\ind \alst{o}f-ręiði \edtrans{\alst{a}fi}{man}{\Bfootnote{While this word usually means “father” or “grandfather”, it should here mean “man” without a connotation of old age. See further \CV.}}.“\eva

\bvb “Rise thou now, Shirner, and go to ask \\
\ind our lad for speech; \\
and to learn at whom the wise \\
\ind man might be cross.”\evb\evg


\bvg\bva\speakernote{Skírnir kvað:}\mssnote{\Regius~11r/15, \AM~2r/17}%
„\alst{I}llra \alst{o}rða \hld\ es mér \alst{ó}n at ykkrum syni, &
\ind ef ek gęng at \alst{m}ę́la við \alst{m}ǫg, &
ok þess at \alst{f}regna, \hld\ hvęim hinn \alst{f}róði séi &
\ind \alst{o}f-ręiði \alst{a}fi.“\eva

\bvb\speakernoteb{Shirner quoth:}%
“Bad words I expect from your son,  \\
\ind if I go to speak with the lad, \\
and to learn at whom the wise \\
\ind man might be cross.”\evb\evg

\sectionline

\bvg\bva\speakernote{Skírnir:}\mssnote{\Regius~11r/17, \AM~2r/18}%
„Sęg þat \alst{F}ręyr, \hld\ \alst{f}olk-valdi goða, &
\ind ok ek \alst{v}ilja \alst{v}ita, &
hví þú \alst{ęi}nn sitr \hld\ \alst{ę}nd-langa sali, &
\ind minn \alst{d}róttinn, of \alst{d}aga?“\eva

\bvb\speakernoteb{Shirner [quoth]:}%
“Tell it, O Free, troop-wielder of the gods— \\
\ind I too would wish to know, \\
why thou sittest alone in the endlong halls, \\
\ind my lord, during the days.”\evb\evg


\bvg\bva\speakernote{Fręyr:}\mssnote{\Regius~11r/19, \AM~2r/20}%
„Hví of \alst{s}ęgja’k þér, \hld\ \alst{s}ęggr hinn ungi, &
\ind \alst{m}ikinn \alst{m}óð-trega? &
því-at \edtrans{\alst{a}lf-rǫðull}{elf-wheel}{\Bfootnote{A rare poetic synonym (\emph{hęiti}) for the sun; see note to \Vafthrudnismal\ 47/1.}} \hld\ lýsir of \alst{a}lla daga &
\ind ok þęygi at \alst{m}ínum \alst{m}unum.“\eva

\bvb\speakernoteb{Free [quoth]:}%
“Why should I tell thee, O young youth, \\
\ind my great heartache? \\
For the elf-wheel \name{= Sun} shines during all days, \\
\ind and nowise to my liking.”\evb\evg


\bvg\bva\speakernote{Skírnir:}\mssnote{\Regius~11r/20, \AM~2r/21}%
„\alst{M}uni þína \hld\ hykk-a svá \alst{m}ikla vesa, &
\ind at þú mér \edtrans{\alst{s}ęggr}{youth}{\Bfootnote{This word usually means simply ‘man’, but it seems to have a specific connotation with youth. Its original meaning is ‘messenger’, and the semantic shift is thus: ‘messenger’ > ‘young man’ > ‘warrior/man’. The sense of ‘young man’ is also seen in \Volundarkvida\ 23, where it is used in reference to king Nithad’s two young sons. In the present stanza it answers Free’s addressing Shirner as \emph{sęggr hinn ungi} ‘the young youth’; Shirner points out that the two are of equal age, and so Free is as much of a young man as he.}} né \alst{s}ęgir; &
\alst{u}ngir saman \hld\ vǫ́rum í \alst{á}r-daga, &
\ind vęl mę́ttim \alst{t}vęir \alst{t}rúask.“\eva

\bvb\speakernoteb{Shirner [quoth]:}%
“Thy liking I do not think so great, \\
\ind that thou, O youth, should not tell me. \\
Young together were we in days of yore; \\
\ind we two might well trust each other.”\evb\evg


\bvg\bva\speakernote{Fręyr:}\mssnote{\Regius~11r/22, \AM~2r/23}%
„Í \alst{G}ymis gǫrðum \hld\ ek \alst{g}anga sá &
\ind \alst{m}ér tíða \alst{m}ęy; &
\alst{a}rmar lýstu, \hld\ en \alst{a}f þaðan &
\ind allt \edtrans{\alst{l}opt ok \alst{l}ǫgr}{air and sea}{\Bfootnote{Formulaic and very old, also paralleled in the Anglo-Saxon. TODO.}}.\eva

\bvb\speakernoteb{Free [quoth]:}%
“In Gymer’s yards I saw walking \\
\ind a maiden, dear to me. \\
Her arms shone and thereof \\
\ind all the air and sea.\evb\evg


\bvg\bva\mssnote{\Regius~11r/24, \AM~2r/24}%
\alst{M}ę́r ’s mér tíðari \hld\ an \alst{m}anna hvęim &
\ind \alst{u}ngum í \alst{á}r-daga; &
\alst{á}sa ok \alst{a}lfa \hld\ þat vill \alst{ę}ngi maðr, &
\ind at vit \alst{s}átt \alst{s}éim.“\eva

\bvb The maiden is dearer to me than to any man \\
\ind young in days of yore. \\
Of the \inx[F]{Eese and Elves} does no man\footnoteB{i.e. ‘person’. For other examples of gods being called men see note to final st. of \Vafthrudnismal\ 55.} wish \\
\ind that we two should be brought together.”\evb\evg


\bvg\bva\speakernote{Skírnir:}\mssnote{\Regius~11r/25, \AM~2r/25}%
„\alst{M}ar gef mér þá, \hld\ es mik of \alst{m}yrkvan beri &
\ind \alst{v}ísan \alst{v}afr-loga, &
ok þat \alst{s}verð, \hld\ es \alst{s}jalft vegisk &
\ind við \alst{jǫ}tna \alst{ę́}tt.“\eva

\bvb\speakernoteb{Shirner [quoth]:}%
“The steed then give me, which might bear me over the dark, \\
\ind wise wavering-flame; \\
and that sword, which by itself might strike \\
\ind against the line of the \inx[G]{Ettins}.”\evb\evg


\bvg\bva\speakernote{Fręyr:}\mssnote{\Regius~11r/27, \AM~2r/27}%
„\alst{M}ar þér þann gef’k, \hld\ es þik of \alst{m}yrkvan \edtext{berr &
\ind \alst{v}ísan \alst{v}afr-loga, &
auk þat \alst{s}verð, \hld\ es \alst{s}jalft mun vegask, &
\ind ef sá ’s \alst{h}orskr es \alst{h}ęfr.“}{\lemma{berr ‘bears’; mun vegask, ef sá ’s horskr es hęfr ‘will strike, if he is wise who owns it’}\Bfootnote{In his response Free replaces the subjunctive verb forms (\emph{beri} ‘might bear’, \emph{vegisk} ‘might strike’) with indicative and future forms, giving a sense of certainity and authority. The steed and sword are faultless, and if Shirner fails on the mission, it would be only due to his own fault (“if he is sharp who owns it.”).}}\eva
%TODO? Change the line numbering from 1–4 to 1, 3–4.

\bvb\speakernoteb{Free [quoth]:}%
“That steed I give thee, which bears thee over the dark, \\
\ind wise wavering-flame; \\
and that sword which by itself will strike, \\
\ind if he is wise who owns it.”\evb\evg


\bpg\bpa Skírnir mę́lti við hest’inn:\epa
\bpb Shirner spoke with the horse:\epb\epg


\bvg\bva\mssnote{\Regius~11r/29, \AM~2r/28}%
„\alst{M}yrkt es úti, \hld\ \alst{m}ál kveð’k okkr fara &
\ind \alst{ú}rig fjǫll \alst{y}fir &
\ind \edtrans{\alst{þ}ursa}{of the Thurses}{\Afootnote{so \AM; \emph{þyria} \Regius}} \alst{þ}jóð yfir; &
\alst{b}áðir vit komumk \hld\ eða okkr \alst{b}áða tękr &
\ind sá hinn \edtrans{\alst{á}m-átki \alst{jǫ}tunn}{uncanny ettin}{\Bfootnote{Formulaic. See note to \Voluspa\ 8.}}.“\eva

\bvb “’Tis dark outside; I declare it time for us to journey \\
\ind over the drizzling mountains, \\
\ind over the tribe of \inx[G]{Thurses}. \\
We will both come, or us both does take \\
\ind that uncanny ettin.\footnoteB{Shirner declares his intention not to abandon the horse given to him by his lord; they will either both make it, or both perish.}”\evb\evg


\bpg
\bpa\mssnote{\Regius~11r/31, \AM~2v/1}%
Skírnir reið i Jǫtun-heima til Gymis garða; þar váru hundar ólmir ok bundnir fyrir skíð-garðs hliði þess, er um sal Gerðar var. Hann reið at þar, er fé-hirðir sat á haugi, ok kvaddi hann: \epa

\bpb Shirner rode into the Ettinhomes, to Gymer’s yards. There were fierce hounds bound in front of the slope of the wooden fence which surrounded Gird’s\footnoteB{It is first now that we are informed of the maiden’s name.} hall. He rode to where a shepherd sat on a mound, and greeted him:\epb\epg


\bvg\bva\mssnote{\Regius~11v/2, \AM~2v/4}%
„Sęg þat \alst{h}irðir, \hld\ es á \alst{h}augi sitr &
\ind ok \alst{v}arðar alla \alst{v}ega: &
hvé ek at \alst{a}nd-spilli \hld\ komumk hins \alst{u}nga mans &
\ind fyr \alst{g}ręyjum \alst{G}ymis.“\eva

\bvb “Tell this, O herdsman, who on the mound sittest, \\
\ind and watchest all the ways, \\
how I to discourse might come with the young girl \ken*{= Gird}, \\
\ind past the greyhounds of Gymer.”\evb\evg


\bvg\bva\speakernote{[Hirðir] kvað:}\mssnote{\Regius~11v/4, \AM~2v/5}%
„Hvárt est \alst{f}ęigr, \hld\ eða est \alst{f}ramm ginginn &
\ind [...]; &
\alst{a}nd-spillis vanr \hld\ þú skalt \alst{ę́} vesa &
\ind \edtrans{\alst{g}óðrar męyjar}{good maiden}{\Bfootnote{Formulaic, carrying with it a sense of chastity.  See note to \Havamal\ 102/1 for further occurrences.}} \alst{G}ymis.“\eva

\bvb\speakernoteb{[The herdsman] quoth:}%
“Either art thou fey, or gone forth [dead]; \\
\ind {[...]}. \\
Discourse-less shalt thou always be, \\
\ind with the good maiden of Gymer \ken*{= Gird}.”\evb\evg


\bvg\bva\speakernote{[Skírnir] kvað:}\mssnote{\Regius~11v/6, \AM~2v/7}
„\edtrans{\alst{K}ostir}{Choices}{\Bfootnote{i.e. ‘alternatives, other ways’.}} ’ru bętri \hld\ \edtrans{an}{than}{\Afootnote{so \AM; \emph{hęldr an at} ‘rather than to [be]’ \Regius}} \alst{k}løkkva séi &
\ind hvęim es \alst{f}úss es \alst{f}ara, &
\alst{ęi}nu dǿgri \hld\ mér vas \alst{a}ldr of skapaðr &
\ind ok alt \alst{l}íf of \alst{l}agit.“\eva

\bvb\speakernoteb{[Shirner] quoth:}%
“Choices are better than sobbing might be \\
\ind for whomever is eager to journey. \\
In one half-day my age was shaped, \\
\ind and all my life laid down.\footnoteB{An excellent example of the fatalistic Germanic worldview, in which one’s course of life was determined (“laid down”) at birth (“in one half-day”).  Presumably after uttering these words Shirner rides through the fire surrounding the fortress. — The causative \emph{lęgja} ‘to lay (down, in place)’ is closely connected to fate; the expression is formulaic.  Cf. \Lokasenna\ 48: \emph{í ár-daga vas þér hit ljóta líf of lagit} ‘in days of yore was thy ugly life laid down’ and \Voluspa\ 19: \emph{þę́r lǫg lǫgðu} ‘they [= the Norns] laid down laws’.}”\evb\evg


\bvg\bva\speakernote{[Gęrðr] kvað:}\mssnote{\Regius~11v/7, \AM~2v/8}%
„Hvat ’s þat \alst{h}lym \alst{h}lymja \hld\ es \alst{h}lymja hęyri’k nú til &
\ind \alst{o}ssum rǫnnum \alst{í}? &
\alst{jǫ}rð bifask, \hld\ en \alst{a}llir fyr &
\ind skjalfa \alst{g}arðar \alst{G}ymis.“\eva

\bvb\speakernoteb{[Gird] quoth:}%
“What is that din of dins, which I of dins now hear \\
\ind in our halls? \\
The earth quakes, and before me tremble \\
\ind all Gymer’s yards.”\evb\evg


\bvg\bva\speakernote{Ambǫ́tt kvað:}\mssnote{\Regius~11v/9, \AM~2v/10}%
„\alst{M}aðr ’s hér úti, \hld\ stiginn af \alst{m}ars baki, &
\ind \alst{jó} lę́tr til \alst{ja}rðar taka.“\eva

\bvb\speakernoteb{A servant-woman quoth:}%
“A man is here outside, stepped down off horseback; \\
\ind he lets take his steed to the ground.\footnoteB{“He lets his horse graze.” According to \textcite{FinnurEdda} an Icelandic expression still known in his time.}”\evb\evg


\bvg\bva\speakernote{[Gęrðr] kvað:}\mssnote{\Regius~11v/10, \AM~2v/11}%
„\alst{I}nn bið þú hann ganga \hld\ í \alst{o}kkarn sal &
\ind ok drekka hinn \alst{m}ę́ra \alst{m}jǫð, &
þó ek hitt \alst{ó}umk, \hld\ at hér \alst{ú}ti séi &
\ind minn \alst{b}róður-\alst{b}ani.“\eva

\bvb\speakernoteb{[Gird] quoth:}%
“Bid thou him to go in into our hall, \\
\ind and to drink the renowned mead; \\
though I fear that here outside should be  \\
\ind my brother’s bane.”\evb\evg

\sectionline

\bvg\bva\speakernote{[Gęrðr] kvað:}\mssnote{\Regius~11v/12, \AM~2v/13}%
„Hvat ’s þat \alst{a}lfa \hld\ né \alst{á}sa sona, &
\ind né \alst{v}íssa \alst{v}ana; &
hví \alst{ęi}nn of komt \hld\ \alst{ęi}kinn fúr yfir &
\ind ór \alst{s}al-kynni at \alst{s}éa?“\eva

\bvb\speakernoteb{[Gird quoth:]}%
“What kind is that, not of Elves, nor of sons of the Eese, \\
\ind nor of wise Wanes? \\
Why camest thou alone over the raging fire, \\
\ind to see the state of our hall?”\evb\evg


\bvg\bva\speakernote{[Skírnir kvað:]}\mssnote{\Regius~11v/14}%
„\alst{E}m’k-at \alst{a}lfa \hld\ né \alst{á}sa sona &
\ind né \alst{v}íssa \alst{v}ana, &
þó \alst{ęi}nn of kom’k \hld\ \alst{ęi}kinn fúr yfir &
\ind yður \alst{s}al-kynni at \alst{s}éa.\eva

\bvb\speakernoteb{[Shirner quoth:]}%
“I am not of Elves, nor of sons of the Eese, \\
\ind nor of wise Wanes— \\
yet I came alone over the raging fire, \\
\ind to see the state of your hall.\evb\evg


\bvg\bva\mssnote{\Regius~11v/15, \AM~2v/14}%
\alst{Ę}pli \alst{ę}llifu \hld\ hér hef’k \alst{a}l-gullin, &
\ind þau mun’k þér \alst{G}ęrðr \alst{g}efa, &
\alst{f}rið at kaupa, \hld\ at þú þér \alst{F}ręy kveðir &
\ind ó·\alst{l}ęiðastan at \alst{l}ifa.“\eva

\bvb Eleven apples have I here, all-golden; \\
\ind those I will to thee, O Gird, give \\
to buy [thy] love, that thou callest Free for thee \\
\ind most unloathsome [lovely] in life.\footnoteB{\emph{at lifa} here means seems to mean ‘in life/living’ rather than the typical infinitive sense ‘to live’; cf. st. 22 \emph{at dęila} ‘in sharing’ below. This is possibly an archaism.}”\evb\evg


\bvg\bva\speakernote{[Gęrðr] kvað:}\mssnote{\Regius~11v/17, \AM~2v/15}%
„\alst{Ę}pli \alst{ę}llifu \hld\ ek þigg \alst{a}ldri-gi &
\ind at \alst{m}anns-kis \alst{m}unum, &
né vit \alst{F}ręyr, \hld\ meðan okkart \alst{f}jǫr lifir, &
\ind \alst{b}yggum \alst{b}ę́ði saman.“\eva

\bvb\speakernoteb{[Gird quoth:]}%
“Eleven apples will I never take, \\
\ind to any man’s liking; \\
nor will I and Free while our life remains \\
\ind dwell both together.”\evb\evg


\bvg\bva\speakernote{[Skírnir kvað:]}\mssnote{\Regius~11v/19, \AM~2v/17 (ll. 1–2)}%
„\alst{B}aug þér þá gef’k, \hld\ þann’s \alst{b}ręndr of vas &
\ind með \alst{u}ngum \alst{Ó}ðins syni; &
\edtext{\alst{á}tta ’ru \alst{ja}fn-hǫfgir, \hld\ es \alst{a}f drjúpa &
\ind hina \alst{n}íundu hvęrja \alst{n}ǫ́tt.“}{\lemma{átta ... nǫ́tt ‘Eight ... night.’}\Bfootnote{In \AM\ these lines and 22:1–2 are missing.  Instead 1–2 here and 22:3–4 are combined into one.}}\eva

\bvb\speakernoteb{[Shirner quoth:]}%
“The \inx[C]{bigh} I then give thee, which was burned \\
\ind with Weden’s young son \ken*{= Balder}. \\
Eight are even-heavy, which from it drip, \\
\ind every ninth night.\footnoteB{The bigh, while not named, is clearly Dreepner as known from \Gylfaginning\ 49, describing Balder’s funeral: “Weden laid on the pyre that gold ring which is called Dreepner. Its nature was such that every ninth night, eight even-heavy golden rings dripped from it.” When \inx[P]{Harmod} later comes to \inx[L]{Hell} to try to bring Balder back, Balder tells him to bring the ring back to Weden, as a token of memory.}”\evb\evg


\bvg\bva\speakernote{[Gęrðr] kvað:}\mssnote{\Regius~11v/21, \AM~2v/18 (ll. 3–4)}%
„\alst{B}aug þikk-a’k, \hld\ þótt \alst{b}ręndr séi, &
\ind með \alst{u}ngum \alst{Ó}ðins syni; &
es-a mér \alst{g}ulls vant \hld\ í \alst{g}ǫrðum \alst{G}ymis &
\ind at dęila \alst{f}é \alst{f}ǫður.“\eva

\bvb\speakernoteb{[Gird quoth:]}%
“The bigh I take not, though it may have been burned \\
\ind with Weden’s young son \ken*{= Balder}; \\
I’m not wanting gold in Gymer’s yards, \\
\ind in sharing the \inx[C]{fee} of my father.”\evb\evg


\bvg\bva\speakernote{[Skírnir kvað:]}\mssnote{\Regius~11v/23, \AM~2v/19}%
„Sér þú \alst{m}ę́ki, \alst{m}ę́r, \hld\ \alst{m}jóvan, \edtrans{\alst{m}ál-fáan}{picture-painted}{\Bfootnote{The sword is inlaid with metal (perhaps gold or silver) forming a pattern.  The expression is formulaic; cf. TODO.}}, &
\ind es \alst{h}ęf’k í \alst{h}ęndi \alst{h}ér? &
\alst{h}ǫfuð \alst{h}ǫggva \hld\ mun’k þér \alst{h}alsi af, &
\ind nema mér \alst{s}ę́tt \alst{s}ęgir.“\eva

\bvb\speakernoteb{[Shirner quoth:]}%
“Seest thou this sword, maiden—slender, pictured-painted—, \\
\ind which I have in my hand here? \\
Strike the head will I from thy neck, \\
\ind unless thou come to terms with me.”\evb\evg


\bvg\bva\speakernote{[Gęrðr kvað:]}\mssnote{\Regius~11v/25, \AM~2v/20}%
„\alst{Á}-nauð þola \hld\ vil’k \alst{a}ldri-gi &
\ind at \edtrans{\alst{m}anns-kis}{any man’s (lit. ‘no man’s)}{\Afootnote{\emph{manns ęnskis} \AM}} \alst{m}unum, &
þó hins \alst{g}et’k, \hld\ ef it \alst{G}ymir finniðsk &
\alst{v}ígs ó·trauðir \hld\ at ykkr \alst{v}ega tíði.“\eva

\bvb\speakernoteb{[Gird quoth:]}%
“Stand coercion will I never, \\
\ind to any man’s liking; \\
though I get this, if thou and Gymer meet— \\
men unreluctant of conflict—that ye two will come to fight.\footnoteB{Gird says that she will never let herself be forced to marry Free, even if that means that her father and Shirner should fight over her.}”\evb\evg


\bvg\bva\speakernote{[Skírnir kvað:]}\mssnote{\Regius~11v/27, \AM~2v/22}%
„Sér þú \alst{m}ę́ki, \alst{m}ę́r, \hld\ \alst{m}jóvan, \alst{m}ál-fáan, &
\ind es \alst{h}ęf’k í \alst{h}ęndi \alst{h}ér? &
fyr þessum \alst{ę}ggjum \hld\ hnígr sá hinn \alst{a}ldni jǫtunn, &
\ind verðr þinn \alst{f}ęigr \alst{f}aðir.\eva

\bvb\speakernoteb{[Shirner quoth:]}%
“Seest thou this sword, maiden—slender, pictured-painted—, \\
\ind which I have in my hand here? \\
By these edges sinks the aged ettin \ken*{= Gymer} down; \\
\ind \inx[C]{fey} becomes thy father.\evb\evg


\bvg\bva\mssnote{\Regius~11v/28, \AM~2v/24}%
\edtrans{\alst{T}ams-vęndi}{taming-wand}{\Bfootnote{Has been interpreted as a sword, TODO.}} þik drep’k, \hld\ ęn þik \alst{t}ęmja mun’k, &
\ind \alst{m}ę́r, at mínum \alst{m}unum, &
þar skalt \alst{g}anga \hld\ es þik \alst{g}umna synir &
\ind \alst{s}íðan ę́va \alst{s}éi.\eva

\bvb With the taming-wand I strike thee—and thee I will tame, \\
\ind O maiden, to my liking! \\
Thou shalt go where the sons of men \\
\ind never since may see thee!\evb\evg


\bvg\bva\mssnote{\Regius~11v/30, \AM~2v/26}%
\edtrans{\alst{A}ra þúfu \alst{á} \hld\ skalt \alst{á}r sitja}{On an eagle’s perch shalt thou sit at dawn}{\Afootnote{\emph{ár skalt sitja \hld\ ara þúfu á} ‘at dawn shalt thou sit on an eagle’s perch’ \AM}}, &
\ind \edtext{\alst{h}orfa \alst{h}ęimi ór; &
\ind snugga \alst{h}ęljar til}{\lemma{horfa hęimi ór; snugga hęljar til ‘turn out of the world; hanker after Hell’}\Afootnote{\emph{horfa ok snugga hęljar til} ‘turn and hanker after Hell’ \AM}\Bfootnote{i.e. “you will look toward and yearn for the underworld”.}}; &
\alst{m}atr sé þér męir lęiðr \hld\ an \alst{m}anna hvęim &
\ind hinn \alst{f}ráni ormr með \edtext{\alst{f}irum}{\Bfootnote{This is the last word of fol. 2v of \AM, after which the text cuts off.}}.\eva

\bvb On an eagle’s perch shalt thou sit at dawn; \\
\ind turn away from the world, \\
\ind hanker after \inx[L]{Hell}! \\
Let thy food be more loathsome than to any man \\
\ind the gleaming serpent \ken*{= the Middenyardswyrm} among the folk.\footnoteB{Her food will be more disgusting than the \inx[C]{Middenyardswyrm}, for which cf. \Hymiskvida\ 22.}\evb\evg


\bvg\bva\mssnote{\Regius~11v/32}%
At \alst{u}ndr-sjónum verðir \hld\ es \alst{ú}t of kømr, &
\ind á þik \alst{H}rímnir \alst{h}ari &
\ind á þik \alst{h}ot-vetna stari, &
\alst{v}íð-kunnari \alst{v}erðir \hld\ an \alst{v}ǫrðr með goðum, &
\ind \alst{g}api þú \alst{g}rindum frá.\eva

\bvb A wondrous sight be thou when thou comest out; \\
\ind at thee let Rimner ogle; \\
\ind at thee let anyone stare! \\
Be thou more widely known than the Watchman among the Gods \ken*{= Homedal}; \\
\ind may thou gape from the gates!\evb\evg


\bvg\bva\mssnote{\Regius~12r/2}%
\edtrans{\alst{T}ópi ok ópi, \hld\ \alst{t}jǫsull ok ó·þoli}{Toop and woop, tarsle and restlessness}{\Bfootnote{The first three words are magic curse words without clear meaning; I have left them untranslated.  \emph{tjǫsull} may perhaps be related to OE \emph{teors} ‘penis’ and mean ‘little phallus’.}}, &
\ind vaxi þér \alst{t}ǫ́r með \alst{t}rega; &
\alst{s}ętsk þú niðr \hld\ en mun’k \alst{s}ęgja þér &
\ind \alst{s}váran \alst{s}ús-breka, &
\ind ok \alst{t}vinnan \alst{t}rega.\eva

\bvb Toop and woop, tarsle and restlessness— \\
\ind may thy tears grow with grief! \\
Sit thyself down, and I will tell thee \\
\ind a heavy roaring-breaker, \\
\ind and a twined grief.\evb\evg


\bvg\bva\mssnote{\Regius~12r/3}%
Tramar \alst{g}nęypa \hld\ þik skulu \alst{g}ęrstan dag &
\ind \alst{jǫ}tna gǫrðum \alst{í}, &
til \alst{h}rím-þursa \alst{h}allar \hld\ þú skalt \alst{h}vęrjan dag &
\ind \alst{k}ranga \alst{k}osta-laus; &
\ind \alst{k}ranga \alst{k}osta-vǫn; &
\alst{g}rát at \alst{g}amni \hld\ skalt í \alst{g}ǫgn hafa &
\ind ok lęiða með \alst{t}ǫ́rum \alst{t}rega.\eva

\bvb Fiends shall pine thee on a gloomy day, \\
\ind in the yards of the Ettins. \\
To the hall of Rime-Thurses shalt thou every day \\
\ind crawl choice-less; \\
\ind crawl choice-lacking. \\
Weeping for joy shalt thou have in exchange, \\
\ind and nurse grief with tears.\evb\evg


\bvg\bva\mssnote{\Regius~12r/7}%
Með \edtrans{\alst{þ}ursi \alst{þ}rí-hǫfðuðum}{three-headed thurse}{\Bfootnote{Ettins often have an abnormal number of body parts.  For their “manyheadedness” see note to \Hymiskvida\ 8/2.}} \hld\ \alst{þ}ú skalt ę́ nara &
\ind eða \alst{v}er-laus \alst{v}esa; &
\ind þitt \alst{g}ęð \alst{g}rípi, &
\ind þik \alst{m}orn \alst{m}orni; &
\edtrans{ves þú sem \alst{þ}istill}{be thou like the thistle}{\Bfootnote{The thistle was apparently held to be a worthless plant; cf. the English galder against a cattle-thief (Charm IX in margins of CCCC 41. TODO: edit this!) cursing him to be \emph{swá bréðel swa séo þystel} ‘as wretched as the thistle’.}}, \hld\ sá’s \alst{þ}runginn vas &
\ind í \alst{o}fan-verða \alst{ǫ́}nn.\eva

\bvb With a three-headed thurse shalt thou always live, \\
\ind or be husband-less. \\
\ind May thy senses seize; \\
\ind may murrain mourn thee; \\
be thou like the thistle that was pressed \\
\ind during highest harvest!\evb\evg


\bvg\bva\mssnote{\Regius~12r/9}%
Til \alst{h}olts ek gekk \hld\ ok til \alst{h}rás viðar &
\ind \edtrans{\alst{g}amban-tęin}{gombentoe}{\Bfootnote{Perhaps “mighty twig”.  A compound consisting of the very rare word \emph{gamban} ‘magic/curse?’ and \emph{tęinn} ‘twig, branch’ (cf. \emph{mistil-tęinn} ‘mistle-toe’).  This may be the stick on which the runic curse in st. 36 below should be carved, or it is to be identified with the \emph{tams-vǫndr} ‘taming-wand’ of st. 26 above.  Cf. \Havamal\ 152, which speaks about a runic curse carved on \emph{rótum rás viðar} ‘the roots of a raw/sappy tree’.}} at \alst{g}eta &
\ind \alst{g}amban-tęin ek \alst{g}at.\eva

\bvb To the wood I went, and to the raw/sappy tree, \\
\ind the \inx[C]{gombentoe} for to get; \\
\ind the gombentoe I got.\evb\evg


\bvg\bva\mssnote{\Regius~12r/10}%
\alst{R}ęiðr ’s þér Óðinn, \hld\ \alst{r}ęiðr ’s þér Ása-bragr, &
\ind þik skal \alst{F}ręyr \alst{f}íask, &
hin \alst{f}irin-illa mę́r, \hld\ en \alst{f}ingit hęfr &
\ind \alst{g}amban-ręiði \alst{g}oða.\eva

\bvb Wroth with thee is Weden; wroth with thee is Eesebray \name{= Thunder}; \\
\ind thee shall Free come to hate, \\
O most wicked maiden, if thou hast earned \\
\ind the gomben-wrath of the gods.\evb\evg


\bvg\bva\mssnote{\Regius~12r/12}%
\alst{H}ęyri jǫtnar, \hld\ \alst{h}ęyri \alst{h}rím-þursar, &
\alst{s}ynir \alst{S}uttunga, \hld\ \alst{s}jalfir ás-liðar, &
hvé \alst{f}yrir býð’k, \hld\ hvé \alst{f}yrir banna’k &
\ind \alst{m}anna glaum \alst{m}ani, &
\ind \alst{m}anna nyt \alst{m}ani.\eva

\bvb Let hear Ettins, let hear Rime-thurses, \\
sons of Sutting, the very Os-Troops \ken*{= Eese} themselves! \\
how I forbid, how I forban \\
\ind men’s fellowship from the maid, \\
\ind men’s joy from the maid!\evb\evg


\bvg\bva\mssnote{\Regius~12r/14}%
\alst{H}rím-grímnir hęitir þurs, \hld\ es þik \alst{h}afa skal &
\ind fyr \alst{n}á-grindr \alst{n}eðan, &
þar þér \alst{v}íl-męgir \hld\ á \alst{v}iðar rótum &
\ind \alst{g}ęita-hland \alst{g}efi; &
\alst{ǿ}ðri drykkju \hld\ fá þú \alst{a}ldri-gi, &
\ind \alst{m}ę́r, af þínum \alst{m}unum, &
\ind \alst{m}ę́r, at \alst{m}ínum \alst{m}unum.\eva

\bvb Rimegrimner is called the thurse who thee shall have \\
\ind down beneath Nawgrind, \\
where the lads of toil \ken{thralls} on the roots of a tree, \\
\ind goat-piss will give thee. \\
A finer drink do thou never get, \\
\ind O maiden, against thy liking, \\
\ind O maiden, to my liking!\evb\evg


\bvg\bva\mssnote{\Regius~12r/16}%
\edtrans{\alst{Þ}urs}{thurse}{\Bfootnote{Thurse is the name of the \textbf{þ}-rune (ᚦ); it is carved as part of the curse.}} ríst’k \alst{þ}ér \hld\ ok \edtrans{\alst{þ}ría stafi}{three staves}{\Bfootnote{Three runic letters (or phrases) representing the three following words (\emph{ęrgi} ‘queerness, degeneracy’ etc.).  The ritual practice of carving “three staves” is first found on the C7th Gummarp stone: \textbf{h\textsc{a}þuwol\textsc{a}fʀ s\textsc{a}te st\textsc{a}b\textsc{a} þri\textsc{a} fff} ‘Hathwolf placed three staves: fff’, where the \textbf{f}-rune (ᚠ) stands for its name \inx[C]{fee} (i.e. ‘wealth, cattle’) and is thus meant to bring wealth.}}, &
\ind \edtrans{\alst{ę}rgi ok \alst{ǿ}ði ok \alst{ó}·þola}{queerness and madness and restlessness}{\Bfootnote{Both \emph{ęrgi} ‘queerness, degeneracy’ and \emph{ó·þoli} ‘restlessness’ (here probably from strong lust) are found in the love magic charm on the rune stick B257 from Bryggen (edited below under Galders).  \emph{ęrgi} is also found in the curse-formula on the C7th Proto-Norse runestones from Stentoften and Björketorp.  See further introduction to B257.}}, &
svá ek þat \alst{a}f ríst \hld\ sem ek þat \alst{á} ręist, &
\ind ef gørask \alst{þ}arfar \alst{þ}ęss.“\eva

\bvb \inx[G]{Thurses}[Thurse] I carve for thee, and three staves: \\
\inx[C]{queerness} and madness and restlessness.— \\
\ind So I carve it \emph{off}, like I carved it \emph{on}, \\
if there be need for that.\footnoteB{Shirner has carved the curse (which will make true the curse), but tells Gird that he will scrape it off if she \ind accepts his demands. She promptly does.}”\evb\evg


\bvg\bva\speakernote{[Gęrðr kvað:]}\mssnote{\Regius~12r/19}%
„\edtext{\alst{H}ęill ves þú \alst{h}ęldr, svęinn, \hld\ ok tak við \alst{h}rím-kalki &
\ind \alst{f}ullum \alst{f}orns mjaðar,}{\lemma{Hęill \dots\ mjaðar ‘Hale \dots\ mead’}\Bfootnote{Formulaic; the same lines occur in \Lokasenna\ 53.}} &
þó hafða’k \alst{ę́}tlat, \hld\ at mynda’k \alst{a}ldri-gi &
\ind unna \edtrans{\alst{v}aningja}{the Waning \ken*{= Free}}{\Bfootnote{lit. ‘descendant of the \inx[G]{Wanes}’.  A rare word.  Its only other occurence in the Norse corpus is in a \inx[C]{thule} of boar-names.  Boars were sacred to Free, TODO.}} \alst{v}ęl.“\eva

\bvb\speakernoteb{[Gird quoth:]}%
“Hale be thou rather, swain, and receive the rime-chalice, \\
\ind full of ancient mead, \\
even though I had intended that I never would \\
\ind love the Waning \ken*{= Free} well.”\evb\evg


\bvg\bva\speakernote{[Skírnir kvað:]}\mssnote{\Regius~12r/21}%
„\alst{Ø}rendi mín \hld\ vil’k \alst{ǫ}ll vita, &
\ind áðr ríða’k \alst{h}ęim \alst{h}eðan, &
nę́r á \alst{þ}ingi \hld\ munt hinum \alst{þ}roska &
\ind \alst{n}ęnna \alst{N}jarðar syni.“\eva

\bvb\speakernoteb{[Shirner quoth:]}%
“My errands all I wish to know, \\
\ind before I ride home hence: \\
when on the \inx[C]{Thing} wilt thou with the vigorous \\
\ind son of Nearth \ken*{= Free} be joined?”\evb\evg


\bvg\bva\speakernote{[Gęrðr kvað:]}\mssnote{\Regius~12r/23}%
„\alst{B}arri hęitir, \hld\ es vit \alst{b}ę́ði vitum, &
\ind \alst{l}undr \alst{l}ogn-fara, &
en ępt \alst{n}ę́tr \alst{n}íu, \hld\ þar mun \alst{N}jarðar syni &
\ind \alst{G}ęrðr unna \alst{g}amans.“\eva

\bvb\speakernoteb{[Gird quoth:]}%
“Barrey is called—as we both know— \\
\ind a grove of calm rushes, \\
and after nine nights there will to the son of Nearth \\
\ind Gird her pleasure grant.”\evb\evg


\bpg\bpa\mssnote{\Regius~12r/24}%
Þá reið Skírnir heim. Freyr stóð úti ok kvaddi hann ok spurði tíðenda:\epa

\bpb Then Shirner rode home. Free stood outside and greeted him and asked for the tidings:\epb\epg


\bvg\bva\mssnote{\Regius~12r/25}%
„\alst{S}ęg mér, Skírnir, \hld\ áðr verpir \alst{s}ǫðli af mar &
\ind ok stígir \alst{f}eti \alst{f}ramarr, &
hvat \alst{á}rnaðir \hld\ í \alst{Jǫ}tun-hęima &
\ind þíns eða \alst{m}íns \alst{m}unar?“\eva

\bvb “Tell me, O Shirner, before thou throw the saddle off the steed, \\
\ind and take a step further: \\
what hast thou accomplished in the \inx[L]{Ettinhomes}, \\
\ind to thy or my liking?”\evb\evg


\bvg\bva\speakernote{[Skírnir kvað:]}\mssnote{\Regius~12r/27}%
„\alst{B}arri hęitir, \hld\ es vit \alst{b}áðir vitum, &
\ind \alst{l}undr \alst{l}ogn-fara, &
en ępt \alst{n}ę́tr \alst{n}íu, \hld\ þar mun \alst{N}jarðar syni &
\ind \alst{G}ęrðr unna \alst{g}amans.“\eva

\bvb\speakernoteb{[Shirner quoth:]}%
“Barrey is called—as we both know— \\
\ind a grove of calm rushes, \\
and after nine nights there will to the son of Nearth \\
\ind Gird grant her pleasure.”\evb\evg


\bvg\bva\speakernote{[Fręyr kvað:]}\mssnote{\Regius~12r/28, \GylfMS}%
\alst{L}ǫng es nǫ́tt, \hld\ \edtrans{\alst{l}angar ’u tvę́r}{long are two}{\Afootnote{\emph{lǫng es ǫnnur} ‘long is another’ \GylfMS}}, &
\ind \edtext{hvé of \alst{þ}ręyja’k \alst{þ}ríar?}{\Afootnote{\emph{hvé męga’k þręyja þríar} \GylfMS}} &
opt \alst{m}ér \alst{m}ánaðr \hld\ \alst{m}inni þótti &
\ind an sjá \alst{h}ǫlf \alst{h}ý-nǫ́tt.\eva

\bvb\speakernoteb{[Free quoth:]}%
Long is a night, long are two— \\
\ind how can I yearn for three? \\
Oft a month to me seemed less \\
\ind than this half wedding-night.\footnoteB{The wedding-night (TODO: it's a hapax so explain the etymology?) is presumably half as it is not consumated.}\evb\evg

\sectionline
% Free
	\bookStart{The Leed of Hoarbeard}[Hárbarðsljóð]

\begin{flushright}%
Dating \parencite{Sapp2022}: early C11th (0.578)–late C11th (0.377)

Meter: Unclear (TODO)%
\end{flushright}

In my opinion the poem can be seen as an allegory on class relations, namely between the self-owning Norwegian and later Icelandic farmers, and the warlike Norwegian earls.

Of all Eddic poems this one is probably the strangest in terms of form. Verse length varies greatly, and many of the lines (see especially the final verse) are of an obscene length reminiscent of late continental Germanic poems like the Heliand; some simply have no metrical qualities at all. The young clitic definite is (uniquely) employed frequently throughout the poem. These criteria would seem to point towards a late origin for the poem (though not later than the late C13th, when \Regius\ was written).

Against this late origin speaks the presence of rare words (e.g. \emph{ǫgurr} v. 13) and a thorough understanding of the personalities of the two gods which would seem unlikely to stem from several centuries after the conversion of Iceland. The model devised by Sapp gives the poem a 57.8\% likelihood of being from the early C11th, and a 37.7\% likelihood of being from the late 11th. These scores are most similar to those obtained by \Gripisspa, a poem that on the surface seems much more archaic.

What could we then be dealing with? It may of course be that the poem is heavily corrupt, but there is no good evidence for this (apart from the above-mentioned irregularities). Most lines are readily understandable and fit well both within their respective context and the poem as a whole. I think a better solution to this problem is to assume that the poem has been acted out as a sort of carnivalesque theatre, with two masked actors, each playing one of the gods. This would explain the variations in meter and line length, and the prose; some lines were simply shouted out, and the lack of alliteration in them would then have a powerful, discordant effect.

This is shown also by uses of the word ‘here’ in vv. 9 and 14. TODO: mention concept of "double scene" by Lars Lönnroth?


\sectionline


\bpg
\bpa\mssnote{\Regius~12r/30}Þórr fór ór austr-vegi ok kom at sundi einu. Ǫðrum megum sundsins var ferju-karlinn með skipit. Þórr kallaði:\epa

\bpb Thunder journeyed from the Eastern Way and came to a sound. At the other side of the sound was the ferryman with the ship. Thunder called out:\epb
\epg


\bvg
\bva\mssnote{\Regius~12r/32}„Hvęrr ’s sá svęinn svęina \hld\ es stęndr fyr sundit handan?“\eva

\bvb “Who is that swain of swains, that stands across the sound?”\evb
\evg


\bvg
\bva\mssnote{\Regius~12v/1}„Hvęrr ’s sá karl karla \hld\ es kallar of váginn?“\eva

\bvb “Who is that churl of churls, that calls out over the wave?”\evb
\evg


\bvg
\bva\mssnote{\Regius~12v/2}„Fęr þú mik of sundit, \hld\ fǿði’k þik á morgun; &
męis hęfi’k á baki, \hld\ verðr-a matrinn bętri. &
Át’k í hvíld \hld\ áðr ek hęiman fór, &
síldr ok \edtrans{hafra}{oatmeal/he-goats}{\Bfootnote{The easiest reading here is the acc. pl. of \emph{hafr} ‘he-goat’. Thunder also eats his goats in \Gylfaginning\ 44, where he butchers and cooks them in the evening and brings them back to life by blessing them with his hammer at dawn. \textcite{FinnurEdda} and \textcite{PettitEdda} prefer this reading; see also note to next stanza.—Many other scholars have here read an accusative plural of \emph{hafri} ‘oat’, i.e. ‘porridge, oatmeal’. Stiles (forthcoming TODO) connects this with Indrá’s (who is the Vedic equivalent of Thunder) “partner and yokemate” (\Rigveda\ 6.56.2) Pūṣán’s eating porridge (e.g. 6.56.1, 57.2). Another similarity Stiles notes between Thunder and Pūṣan is that both have chariots driven by goats (e.g. 6.57.3: “Goats are the draft-animals for the one”, 58.2: “Having goats as his horses”). Whether the Vedic tradition has split an original god into two or whether Thunder has absorbed elements of another god is hard to say.}}; \hld\ saðr em’k ęnn þęss.“\eva

\bvb {[Thunder quoth:]} “Ferry me over the sound, I feed thee in the morning! \\
A basket have I on my back; the food does not get better.\footnoteB{i.e. ‘you will not get better food than that.’} \\
I ate for a while before I journeyed from home, \\
herring and oatmeal/he-goats; I am still full from that.”\evb
\evg


\bvg
\bva\mssnote{\Regius~12v/5}„Ár-ligum verkum \hld\ hrósar þú, vęrðinum; \hld\ vęitst-at-tu fyr gǫrla, &
dǫpr ’ru þín hęim-kynni, \hld\ dauð hygg’k at þín móðir sé.“\eva

\bvb “Of early works boastest thou; of eating!\footnoteB{TODO. This is pretty difficult. From the previous stanza \emph{vęrðinum} seems to be referring to eating.} Thou knowest not clearly [what lies] before [thee]: \\
dismal is the state of thy home—dead I ween thy mother be!”\evb
\evg


\bvg
\bva\mssnote{\Regius~12v/6}„Þat sęgir þú nú \hld\ es hvęrjum þikkir &
męst at vita— \hld\ at mín móðir dauð sé.“\eva

\bvb “Thou now sayest that which to each man seems \\
most [important] to know: that my mother be dead!”\evb
\evg


\bvg
\bva\mssnote{\Regius~12v/8}„Þęygi ’s sem þú \hld\ þrjú bú ęigir góð; &
bęr-bęinn þú stęndr \hld\ ok hęfir brautinga gørvi, \hld\ þat-ki at þú hafir brę́kr þínar.“\eva

\bvb “’Tis hardly as if thou might own three good homesteads; \\
bare-legged thou standest, and hast the gear of a tramp; ’tis not even as if thou have thy own breeches!”\evb
\evg


\bvg
\bva\mssnote{\Regius~12v/9}„Stýrðu hingat ęikjunni, \hld\ ek mun þér stǫðna kęnna &
eða hvęrr á skipit \hld\ es þú hęldr við landit?“\eva

\bvb “Steer hither the boat! I will show thee to the harbour— \\
or who owns the ship which thou holdest by the shore?”\evb
\evg


\bvg
\bva\mssnote{\Regius~12v/11}„Hildólfr sá hęitir \hld\ es mik halda bað, &
rekkr inn ráð-svinni \hld\ es býr í Ráðs-ęyjar-sundi; &
bað-at hann hlęnni-męnn flytja \hld\ eða hrossa-þjófa, &
góða ęina \hld\ ok þá’s ek gørva kunna; &
sęg-ðu til nafns þíns \hld\ ef þú vill of sundit fara.“\eva

\bvb “Hildolf is he called who asked me to hold it, \\
the counsel-wise man who lives in Redeseysound. \\
He bade me not take highwaymen nor horse-thiefs; \\
good men only, and those whom I know well— \\
state thy name if thou wilt fare o’er the sound!”\evb
\evg


\bvg
\bva\mssnote{\Regius~12v/15}„Sęgja mun’k til nafns míns \hld\ þótt ek sękr sjá’k &
ok til alls øðlis: \hld\ Ek em Óðins sonr, &
Męila bróðir \hld\ ęn Magna faðir, &
þrúð-valdr goða \hld\ við Þór knátt-u hér dǿma! &
Hins vil’k nú spyrja \hld\ hvat þú hęitir?“\eva

\bvb “I will state my name—[and would] even if I were charged— \\
and all my origin: I am Weden’s son, \\
Male’s brother and Main’s father, \\
the strength-wielder of the Gods; with Thunder dost thou here speak! \\
This will I now ask, what thou art called?”\evb
\evg


\bvg
\bva\mssnote{\Regius~12v/18}„Hárbarðr ek hęiti, \hld\ hyl’k of nafn sjaldan.“\eva

\bvb “Hoarbeard I am called, seldom I conceal my name.”\evb
\evg


\bvg
\bva\mssnote{\Regius~12v/18}„Hvat skalt-u of nafn hylja \hld\ nema þú sakar ęigir?“\eva

\bvb “Why shalt thou conceal thy name, unless thou have charges?”\evb
\evg


\bvg
\bva\mssnote{\Regius~12v/19}„En þótt ek sakar ęiga \hld\ fyr slíkum sem þú est &
þá mun’k forða fjǫrvi mínu \hld\ nema ek fęigr sé.“\eva

\bvb “Even if I should have charges, for such a one as thou art \\
would I still protect by life, lest I be \inx[C]{fey}.”\evb
\evg


\bvg
\bva\mssnote{\Regius~12v/21}„Harm ljótan mér þikkir í því &
at vaða of váginn til þín \hld\ ok vę́ta \edtrans{ǫgur}{burden}{\Bfootnote{The sense of this word is not clear, though it is probably the same as the first element of the compound \emph{ǫgur-stund} ‘burdensome hour’, found in \Volundarkvida\ 42. Some authors have read it as a crude euphemism for ‘penis’, which would not be out of character for this poem. I however consider the best interpretation to be that of an author whose name I've forgotten (TODO!), namely that Thunder is referring to the food he carries on his back (cf. v. 3).}} minn; &
skylda’k launa kǫgur-sveini \hld\ þínum kangin-yrði \hld\ ef ek komumk yfir sundit.“\eva

\bvb “An ugly harm it seems to me \\
to wade o’er the wave to thee, and wet my burden. \\
I would repay thee, swaddle-swain, for thy mocking words, if myself I could bring over the sound.”\evb
\evg


\bvg
\bva\mssnote{\Regius~12v/23}„Hér mun’k standa \hld\ ok þín heðan bíða; &
fannt-a-tu mann inn harðara \hld\ at Hrungni dauðan.“\eva

\bvb “\emph{Here} will I stand, and \emph{hence} await thee;  \\
thou foundest not a harder man since the death of \inx[P]{Rungner}!\footnoteB{Rungner was an ettin slain by Thunder, TODO. Hoarbeard’s mentioning of him sets off a long interchange, wherein the two boast of their deeds, and ask what the other one was doing meanwhile.}”\evb
\evg


\bvg
\bva\mssnote{\Regius~12v/25}„Hins vilt-u nú geta \hld\ es vit Hrungnir dęildum, &
sá inn stór-úðgi jǫtunn, \hld\ es ór stęini vas hǫfuðit á, &
þó lét’k hann falla \hld\ ok fyr hníga; &
\ind hvat vannt-u þá meðan, Hárbarðr?“\eva

\bvb “This wilt thou now mention, of when I and Rungner dealt with each other, \\
that great-minded ettin on whom the head was made of stone.  \\
Yet I let him fall, and sink down before [me]— \\
what didst thou then meanwhile, Hoarbeard?”\evb
\evg


\bvg
\bva\mssnote{\Regius~12v/27}„Vas’k með Fjǫl-vari \hld\ fimm vetr alla &
í ęy þeiri \hld\ er Algrǿn hęitir; &
vega vér þar knǫ́ttum \hld\ ok val fęlla, &
margs at fręista, \hld\ mans at kosta.“\eva

\bvb “I was with Felwar for all of five winters \\
in that island which Allgreen is called. \\
There we knew to fight, and fell corpses; \\
many to tempt, a girl to win.\footnoteB{I read \emph{margs} ‘many a’ as modifying \emph{mans} ‘girl’, thus giving ‘(we knew) to tempt and to win many a girl’.}”\evb
\evg


\bvg
\bva\mssnote{\Regius~12v/30}„Hversu snúnuðu yðr konur yðrar?“\eva

\bvb “How did your women pleasure (TODO!!!) you?.\footnoteB{Seemingly a prose line; see Introduction.}”\evb
\evg


\bvg
\bva\mssnote{\Regius~12v/30}„Sparkar ǫ́ttum vér konur \hld\ ef oss at spǫkum yrði; &
horskar ǫ́ttum vér konur \hld\ ef oss hollar vę́ri, &
þę́r ór sandi \hld\ síma undu &
\ind ok ór dali djúpum &
\ind grund of grófu; &
varð’k þęim ęinn ǫllum \hld\ øfri at rǫ́ðum; &
\ind hvílda’k hjá systrum sjau &
\ind ok hafða’k gęð þęira allt ok gaman;
\ind hvat vannt-u þá meðan, Þórr?“\eva

\bvb “We \ken*{I} owned frisky women, if they were pleasing towards us \ken*{me}; \\
we \ken*{I} owned wise women, if they were \inx[C]{hold} towards us \ken*{me}; \\
out of the sand a rope they wound, \\
and out of a deep dale \\
dug up the ground; \\
I alone became superior to all of them in counsels; \\
I rested by those sisters seven, \\
and had their senses all, and pleasure— \\
what didst thou then meanwhile, Thunder?”\evb
\evg


\bvg
\bva\mssnote{\Regius~13r/2, \AM~1r/1 (l. 4b ff.)}„Ek drap Þjatsa, \hld\ hinn þrúð-móðga jǫtun, &
upp ek varp augum \hld\ Allvalda sonar &
\ind á þann hinn hęiða himin; &
þau ’ru męrki męst \hld\ minna verka, &
\ind þau’s allir męnn síðan of sé; &
\ind hvat vannt-u þá meðan, Hárbarðr?“\eva

\bvb “I slew \inx[C]{Thedse}, the strength-minded ettin; \\
up I threw the eyes of Allwald’s son \ken*{= Thedse} \\
onto that bright heaven; \\
those are the greatest marks of my works, \\
those that all men since do see\footnoteB{Here we seem to have a rare example of native Germanic star-lore. Is the exact constellation identifiable? TODO.}— \\
what didst thou then meanwhile, Hoarbeard?”\evb
\evg


\bvg
\bva\mssnote{\Regius~13r/5, \AM~1r/1}„Miklar man-vélar \hld\ hafða’k við myrk-riður &
\ind þá’s ek vélta þę́r frá verum; &
harðan jǫtun \hld\ hugða’k Hlébarð vesa; &
\ind gaf hann mér gamban-tęin &
\ind en ek vélta hann ór viti.“\eva

\bvb “Great girl-tricks I used against \inx[C]{mirk-riders}, \\
when I tricked them away from their husbands.\footnoteB{Alternatiely ‘away from men’. The \emph{riður} ‘(female) riders’ were witches thought to torment people and cause disease and suffering. See \Havamal\ 156 for discussion.} \\
A hard ettin I judged Leebeard to be; he gave me a \inx[C]{gombentoe}, but I tricked him out of his wits.”\evb
\evg


\bvg
\bva\mssnote{\Regius~13r/7, \AM~1r/3}„Illum huga launaðir þú þá góðar gjafar.“\eva

\bvb “With an evil mind rewardedst thou that good gift.”\evb
\evg


\bvg
\bva\mssnote{\Regius~13r/8, \AM~1r/4}„Þat hęfir ęik \hld\ es af annarri skęfr; &
\ind umb sik es hvęrr í slíku; &
\ind hvat vannt-u þá meðan, Þórr?“\eva

\bvb “An oak has that which it shaves from another; \\
each [man] is for himself in such [a matter]— \\
what didst thou then meanwhile, Thunder?”\evb
\evg


\bvg
\bva\mssnote{\Regius~13r/9, \AM~1r/4}„Ek vas austr \hld\ ok jǫtna barða’k &
brúðir bǫl-vísar \hld\ es til bjargs gingu; &
mikil myndi ę́tt jǫtna \hld\ ef allir lifði, &
vę́tr myndi manna \hld\ undir Mið-garði; &
\ind hvat vannt-u þá meðan, Hárbarðr?\eva

\bvb “I was in the east, and ettins I fought; bale-wise brides who walked to the mountain. Great would the lineage of ettins be if all lived; naught would remain of men within Middenyard\footnoteB{A remarkable clear statement of purpose. This conception is far from unique to this verse; in \Hymiskvida\ 11, for instance, Thunder is described as “the opponent of Rooder”, “the friend of manly retinues” and “Wighward”, attesting his role in the slaying of ettins and the protection of men and their sanctuaries (\inx[C]{wigh}[wighs]). kenned as the wigh-ward (sanctuary-defender) of Middenyard. For Thunder’s killing of women cf. vv. 37–39 below and also}—what didst thou then meanwhile, Hoarbeard?”\evb
\evg


\bvg
\bva\mssnote{\Regius~13r/11, \AM~1r/6}„Vas’k á Vallandi \hld\ ok vígum fylgða’k, &
atta ek jǫfrum \hld\ en aldrigi sę́tta’k; &
Óðinn á jarla \hld\ þá’s í val falla &
\ind en Þórr á þrę́la kyn.“\eva

\bvb “I was in \inx[L]{Walland} and followed conflicts; I goaded princes on, but never reconciled them. Weden owns the earls which fall among the slain, but Thunder owns the kin of thralls.\footnoteB{We see here a sort of aristocratic, Odinic disregard for lower life and life as a good in itself; where Thunder boasts of saving men, Weden sarcastically responds that he caused the deaths of men so that he could have them for himself.}”\evb
\evg


\bvg
\bva\mssnote{\Regius~13r/13, \AM~1r/8}„Ójafnt skipta \hld\ es þú myndir með ǫ́sum liði &
\ind ef þú ę́ttir vilgi mikils vald.“\eva

\bvb “Translation.”\evb%TODO: There’s something very weird going on here.
\evg


\bvg
\bva\mssnote{\Regius~13r/14, \AM~1r/9}„Þórr á afl ǿrit \hld\ en ękki hjarta; &
af hrę́ðslu ok hug-blęyði \hld\ þér vas í handska troðit &
\ind ok þóttisk-a þú þá Þórr vesa; &
hvárki þá þorðir \hld\ fyr hrę́ðslu þinni &
hnjósa né físa \hld\ svá’t Fjalarr hęyrði.“\eva

\bvb “Thunder owns ample strength, but no heart; out of fear and mind-softness didst thou tread into a glove, and then seemedest thou not to be Thunder. Thou daredest neither—for thy fear—to sneeze nor to fart so that Feller might hear [it].\footnoteB{This story is also referenced in \Lokasenna\ TODO. It is elaborated heavily on in \Gylfaginning\ 45: Thunder, Lock, and the siblings Thelve and Wrash had travelled east for a long time when they discovered a large hall, with an opening on one end, as wide as the building. They took rest inside, but in the middle of the night there was a great earthquake and the ground beneath them trembled. Thunder rose and led the party to a side-room to the right in the middle of the hall. He sat closest to the opening with his hammer ready, while the others sat terrified further inside. At daybreak they left the hall and found a huge ettin named \emph{Skrymir} (\inx[P]{Shrimer}) sleeping next to them. His snoring had caused the earth-quakes, and the hall was his mitten; the side-room was the thumb-part.}”\evb
\evg


\bvg
\bva\mssnote{\Regius~13r/17, \AM~1r/11}„Hárbarðr hinn ragi, \hld\ munda’k þik í Hęl drepa &
\ind ef mę́tta’k sęilask of sund.“\eva

\bvb “Hoarbeard the \inx[C]{degenerate}, I would strike thee into \inx[L]{Hell}, if I might sail o’er the sound!”\evb
\evg


\bvg
\bva\mssnote{\Regius~13r/18, \AM~1r/12}„Hvat skyldir of sund sęilask \hld\ es sakir ’ru alls øngar? &
\ind hvat vannt-u þá meðan, Þórr?“\eva

\bvb “Why should thou sail o’er the sound when there are no offenses?—what didst thou then meanwhile, Thunder?”\evb
\evg


\bvg
\bva\mssnote{\Regius~13r/19, \AM~1r/13}„Ek vas austr \hld\ ok ána varða’k &
þá’s mik sóttu \hld\ þęir Svárangs synir; &
grjóti mik bǫrðu, \hld\ gagni urðu þó lítt fęgnir, &
þó urðu mik fyrri \hld\ friðar at biðja. &
\ind hvat vannt-u þá meðan, Hárbarðr?“\eva

\bvb “I was in the east, and warded the river, when the sons of Sweering attacked me. With rocks they fought me, yet they rejoiced little in victory; yet they earlier had to beg me for peace—what didst thou then meanwhile, Hoarbeard?”\evb
\evg


\bvg
\bva\mssnote{\Regius~13r/22, \AM~1r/15}„Ek vas austr \hld\ ok við ęin-hvęrja dǿmða’k, &
lék’k við ina lind-hvítu \hld\ ok lǫng þing háða’k, &
gladda’k ina gull-bjǫrtu, \hld\ gamni mę́r unði.“\eva

\bvb “I was in the east, and with a certain woman conversed; I played with the linen-white one, and held long-lasting trysts:\footnoteB{\emph{þing} (see \inx[C]{Thing}) usually means ‘legal assembly’, but clearly not here.} I gladdened the gold-bright one; the maiden enjoyed pleasure.”\evb
\evg


\bvg
\bva\mssnote{\Regius~13r/24, \AM~1r/17}„Góð ǫ́ttu þęir man-kynni þar þá.“\eva

\bvb “Then they had good girl-visits there.”\evb
\evg


\bvg
\bva\mssnote{\Regius~13r/24, \AM~1r/17}„Liðs þíns vę́ra’k þá þurfi, Þórr, \hld\ at hęlda’k þęiri inni lín-hvítu męy.“\eva

\bvb “Of thy help I might have been in need then, Thunder, that I might hold that linen-white maiden.”\evb
\evg


\bvg
\bva\mssnote{\Regius~13r/25, \AM~1r/18}„Ek mynda þér þat þá vęita \hld\ ef ek viðr of kę́misk.“\eva

\bvb “I would then have granted thee that, if I were able.”\evb
\evg


\bvg
\bva\mssnote{\Regius~13r/26, \AM~1r/18}„Ek mynda þér þá trúa, \hld\ nema mik í tryggð véltir.“\eva

\bvb “I would then have trusted thee, unless thou betrayed my trust.”\evb
\evg


\bvg
\bva\mssnote{\Regius~13r/27, \AM~1r/19}„Em’k-at ek sá hę́lbítr \hld\ sem húð-skór forn á vár.“\eva

\bvb “I am not such a heel-biter as an old hide-shoe in spring.\footnoteB{Proverbial (a heel-biter being someone who betrays his companions); the leather of a shoe would become very stiff and chafing over the winter.}”\evb
\evg


\bvg
\bva\mssnote{\Regius~13r/28, \AM~1r/20}\ind Hvat Shed þá meðan, Þórr?“\eva

\bvb “What didst thou then meanwhile, Thunder?”\evb
\evg


\bvg
\bva\mssnote{\Regius~13r/28, \AM~1r/20}„Brúðir ber-sęrkja \hld\ barða’k í Hlés-ęyju; &
þę́r hǫfðu vęrst unnit, \hld vélta þjóð alla.“\eva

\bvb “The brides of bearserks I fought in Leesie; they had done the worst thing: deceived a whole people.”\evb
\evg


\bvg
\bva\mssnote{\Regius~13r/29, \AM~1r/21}„Klę́ki  þá, Þórr, \hld\ es þú á konum barðir.“\eva

\bvb “A great disgrace didst thou then, Thunder, when thou foughtest women.”\evb
\evg


\bvg
\bva\mssnote{\Regius~13r/30, \AM~1r/22}„Vargynjur vǫ́ru þę́r \hld\ en varla konur, &
skęlldu skip mitt \hld\ es ek skorðat hafða’k, &
ǿgðu mér járn-lurki \hld\ en ęltu Þjálfa. &
\ind hvat vannt-u þá meðan, Hárbarðr?“\eva

\bvb “She-wolves were they, but hardly women; they knocked my ship which I had propped; frightened me with an iron-cudgel, but chased Thelve around—what didst thou then meanwhile, Hoarbeard?”\evb
\evg


\bvg
\bva\mssnote{\Regius~13r/32, \AM~1r/23}„Ek vas’k í hęrnum \hld\ es hingat gjǫrðisk &
gnę́fa gunn-fana, \hld\ gęir at rjóða.“\eva

\bvb “I was in the army, as hence it made ready to raise the war-standard, to redden the spear.”\evb
\evg


\bvg
\bva\mssnote{\Regius~13v/1, \AM~1r/24}„Þess vilt-u nú geta, es þú fórt oss \edtext{ó-ljúfan}{\Bfootnote{oliyfan \AM; †olubann† \Regius}} at bjóða!“\eva

\bvb “This wilt thou now mention, as thou wentest to bid us \ken*{= the Ease} hatred!”\evb
\evg


\bvg
\bva\mssnote{\Regius~13v/2, \AM~1r/25}„Bǿta skal þér þat þá \hld\ munda baugi &
sem jafnęndr unnu \hld\ þęir’s okkr vilja sę́tta.“\eva

\bvb “I will then restore thee for that with a hand-bigh, like the settlers [have] considered, those who wish to reconcile us two.”\evb
\evg


\bvg
\bva\mssnote{\Regius~13v/3, \AM~1r/26}„Hvar namt þęssi \hld\ in hnǿfi-ligu orð &
es hęyrða’k aldrigi \hld\ hnǿfi-ligri?“\eva

\bvb “Where learnedst thou these sarcastic words, which I never heard more sarcastic?”\evb
\evg


\bvg
\bva\mssnote{\Regius~13v/5, \AM~1r/27}„Nam’k at mǫnnum þęim inum aldrǿnum es búa í hęimis-skógum.“\eva

\bvb “I learned them from the old men who dwell in the home-forests.”\evb
\evg


\bvg
\bva\mssnote{\Regius~13v/5, \AM~1v/1}„Þó gefr þú gótt nafn dysjum, es þú kallar þat hęimis-skóga.“\eva

\bvb “Yet thou givest a good name to poor cairns,\footnoteB{cf. his waking the dead in various poems TODO.} as thou callest them home-forests.”\evb
\evg


\bvg
\bva\mssnote{\Regius~13v/6, \AM~1v/2}„Svá dǿmi’k of slíkt far.“\eva

\bvb “So I speak about such matters.”\evb
\evg


\bvg
\bva\mssnote{\Regius~13v/7, \AM~1v/2}„Orð-kringi þín \hld\ mun þér illa koma &
\ind ef ek rę́ð á vág at vaða; &
ulfi hę́ra \hld\ hygg’k at ǿpa mynir &
\ind ef hlýtr af hamri hǫgg.“\eva

\bvb “Thy word-glibness will bring thee evil, if I resolve to wade on the wave; higher than a wolf I think that thou wilt scream, if thou suffer a strike from the hammer.”\evb
\evg


\bvg
\bva\mssnote{\Regius~13v/9, \AM~1v/4}„Sif á \edtrans{hó}{lover}{\Bfootnote{Most translators take this acc. sg. word as an alternative form of \emph{hórr} m. ‘adulterer’ (gen. \emph{hórs}), containing the same root as \emph{hóra} f. ‘whore, prostitute’, \emph{hór} n. ‘adultery, fornication’, ModEngl. whore. The \emph{-r} has presumably been interpreted as the masc. nom. sg. ending, giving nom. \emph{*hór}, gen. \emph{*hós}. Further, this accusation is also found in \Lokasenna\ TODO, where Lock says that he has been Sib’s lover (\emph{hórr}). Notably, \CV\ interprets this word as the unrelated \emph{hór} m. ‘pot-hook’, “insinuating that Thor busied himself with cooking and dairy-work.” This seems very unlikely when considering Thunder’s response in the next verse: “I think that thou liest!” and the parallel in \Lokasenna.}} hęima, \hld\ hans munt fund vilja, &
þann munt þręk drýgja, \hld\ þat ’s þér skyldara.“\eva

\bvb “Sib has a lover at home; him wilt thou wish to meet! On that one shalt thou use thy strength—that befits thee more!”\evb
\evg


\bvg
\bva\mssnote{\Regius~13v/10, \AM~1v/5}„Mę́lir þú at munns ráði \hld\ svá’t mér skyldi vęrst þikkja, &
halr inn hug-blauði, \hld\ hygg’k at þú ljúgir.“\eva

\bvb “Thou speakest to the counsel of thy mouth that which would seem to me the worst; heart-soft man, I think that thou liest!”\evb
\evg


\bvg
\bva\mssnote{\Regius~13v/12, \AM~1v/6}„Satt hygg’k mik sęgja, \hld\ sęinn est at fǫr þinni, &
langt myndir nú kominn, Þórr, \hld\ ef þú \edtrans{litum fǿrir}{brought thy colours}{\Bfootnote{Very unclear expression. \emph{fǿra litum} TODO.}}.“\eva

\bvb “I think myself to speak truly: late art thou in thy journey; far would thou now be come, Thunder, if thou had brought thy colours.”\evb
\evg


\bvg
\bva\mssnote{\Regius~13v/14, \AM~1v/8}„Hárbarðr inn ragi, \hld\ hęldr hęfir nú mik dvalðan!“\eva

\bvb “Hoarbeard the degenerate; thou hast now delayed me greatly!”\evb
\evg


\bvg
\bva\mssnote{\Regius~13v/14, \AM~1v/8}„Ása-Þórs \hld\ hugða’k aldrigi myndu &
\ind glępja fé-hirði farar.“\eva

\bvb “The journey of Thunder of the Ease I never thought that a shepherd \ken*{= I} would divert.”\evb
\evg


\bvg
\bva\mssnote{\Regius~13v/15, \AM~1v/9}„Ráð mun’k þér nú ráða: \hld\ Ró þú hingat bátinum, &
hę́ttum hǿtingi, \hld\ hitt fǫður Magna!“\eva

\bvb “I will now give thee a counsel: Row hither the boat; seize with the taunting; come to the father of Main \ken*{= Thunder = me}!”\evb
\evg


\bvg
\bva\mssnote{\Regius~13v/17, \AM~1v/10}„Far þú firr sundi, \hld\ þér skal fars synja!“\eva

\bvb “Go far from the sound; the ferry shall be denied thee!”\evb
\evg


\bvg
\bva\mssnote{\Regius~13v/17, \AM~1v/11}„Vísa þú mér nú lęiðina \hld\ alls þú vill mik ęigi of váginn fęrja!“\eva

\bvb “Show me now the path, as thou wilt not ferry me o’er the wave!”\evb
\evg


\bvg
\bva\mssnote{\Regius~13v/18, \AM~1v/11}„Lítit ’s at synja, \hld\ langt ’s at fara; &
stund ’s til stokksins, \hld\ ǫnnur til stęinsins, &
halt svá til vinstra vegsins \hld\ unds þú hittir Ver-land; &
þar mun Fjǫrgyn \hld\ hitta Þór, son sinn, &
ok mun hǫ́n kęnna hǫ́num ǫ́ttunga brautir \hld\ til Óðins landa.“\eva

\bvb “’Tis little to deny, ’tis long to journey: an hour to the log, another to the stone; hold thus to the left road, until thou findest Wereland; there will Firgyn find Thunder, her son, and she will show him to the highways of her ancestors, to Weden’s lands \ken*{= Osyard}.”\evb
\evg


\bvg
\bva\mssnote{\Regius~13v/22, \AM~1v/14}„Mun’k taka þangat í dag?“\eva

\bvb “Will I come thither today?”\evb
\evg


\bvg
\bva\mssnote{\Regius~13v/22, \AM~1v/14}„Taka við víl ok ęrfiði \hld\ at upp-vesandi sólu &
es ek get þána.“\eva

\bvb “[Thou wilt] come with toil and hardship at the rising of the sun, as I think it is thawing.”\evb
\evg


\bvg
\bva\mssnote{\Regius~13v/23, \AM~1v/15}„Skammt mun nú mál okkat vesa, \hld\ alls þú mér skǿtingu ęinni svarar; &
launa mun ek þér far-synjun \hld\ ef vit finnumk í sinn annat. &
Far þú nú þar’s þik hafi allan gramir!“\eva

\bvb “Short will now our speech be, as thou answerest me with scoffing alone; I will reward thee for this ferry-denial if we meet another time. Now go, whither the fiends may have all of thee!”\evb
\evg
% Weden, Thunder
	\bookStart{Lay of Hymer}[Hymiskviða]

\begin{flushright}%
\textbf{Dating} \parencite{Sapp2022}: C10th (0.694)–early C11th (0.268)

\textbf{Meter:} \Fornyrdislag%
\end{flushright}%

\section{Introduction}

Attested in two manuscripts, \Regius\ and \AM.  The two agree very well; they share the same stanzas and they come in the same order.  The most substantial difference is the header; \AM\ has \emph{Hymis-kviða} ‘the Lay of Hymer’, while \Regius\ instead has \emph{Þórr dró Mið-garðs-orm} ‘Thunder pulled the Middenyardswyrm’.

The poem is a comedy about Thunder’s adventures in Ettinland.  This was probably a popular genre, and is also represented by \Thrymskvida, but in spite of these similarities of contents the two poems are far apart stylistically.  Whereas \Thrymskvida\ is written in a simple and sparse style with free \Fornyrdislag-meter and few kennings, the form of \Fornyrdislag\ used in \Hymiskvida\ is unusually strict, almost syllable-counting, and the stanzas are filled with intricate kennings, difficult grammatical constructions and forced word order.  In this way \Hymiskvida\ is more akin to Scaldic poetry in intricate measures like \Drottkvett\ than to typical Eddic poetry in \Fornyrdislag.

For this reason it seems likely that the anonymous poet of \Hymiskvida\ was highly trained in the Scaldic arts, and familiar with composition in more advanced meters.  (See TODO: Difference between Scaldic and Eddic).  Apart from style and meter, the Scaldic composition context of \Hymiskvida\ is also supported by both its dating and subject.  There are five extant Scaldic poetic fragments (TODO: list them) that deal with Thunder’s fishing expedition, mostly from the 10th century.

These Scaldic fragments are fragmentary, and (in what survives of them) mostly focus on the scene where Thunder faces off against the hooked Wyrm pressed to the gunwale.  There are some interesting verbal correspondences between these fragments and \Hymiskvida—most strikingly the kenning for the Middenyardswyrm in st. 22/4 below—that may also support a common composition context.  The fragments do not all agree with each other; in some of them the encounter ends with the cowardly Hymer cutting off the fishing line and the Wyrm sinking back unharmed into the sea (the version preferred by Snorre)—in others Thunder strikes the head off the Wyrm, presumably slaying it.

Numerous pictoral depictions of the myth are found on Wiking Age objects.  These are the Swedish Altuna (U 1611) and Linga (Sö 352) runestones, the picture stones from Hørdum, Northern Jutland, a picture stone from Gosforth in Cumbria, and others (TODO).  They typically show Thunder standing in the boat with His hammer raised, and the hooked Wyrm below it.  Several smaller details also appear on these objects: the use of the ox-head for bait (U 1611, Sö 352), Thunder’s feet going through the ship (U 1611, Hørdum).

\Gylfaginning\ 48 gives a complete narrative, here paraphrased for the sake of shortness:

{\small Thunder goes out into Middenyard in the shape of a young man (\emph{ungr dręngr}), without his chariot, his goats, or his typical travelling gear.  In the evening he comes to the ettin Hymer and begs for lodgings.  At dawn Hymer plans to go fishing, and so Thunder asks to join in.  Hymer insults Thunder's small stature and youth, and questions his ability to go on such a long and arduous trip as he usually takes.  Thunder, angered, says that he will row very far, and then asks Hymer what bait they will use.  Hymer tells him to get his own bait, and so he turns to Hymer’s flock of oxen and tears off the head from his greatest ox, one named Heavenrid.  The two go out to sea, and Thunder rows far past Hymer's usual fishing spot.  Hymer, unhappy, warns him that if they row any further out they'll be in danger of the Middenyardswyrm, but Thunder goes on.  Eventually Thunder puts away the oars, readies a fishing line, hooks the ox-head and lowers it.  The Wyrm soon bites, and struggles so hard that Thunder is pressed against the gunwale.  This angers the god, and he brings himself into his Os-might.  Strengthened, he pulls back with such force that his feet go through the bottom of the ship and press into the sea-floor; the Wyrm's head goes up against the gunwale.  The two archenemies furiously stare at each other, Thunder “sharpening his eyes” and the Wyrm spitting venom.  Hymer is frightened, reaches for his bait-cutting knife, and cuts off the line—the Wyrm then sinks back into the sea.  Thunder throws the hammer after it, “and men say that he struck off the monster’s head, but I think it true to tell thee, that the Middenyardswyrm still lives and lies in the outer sea.”  Thunder then punches Hymer’s ear with his fist so that he is thrown overboard head-first; the god then wades back to land.}

This account is clearly based on several sources, possibly including the present poem.  The closest wording correspondence is when it is said that \emph{Miðgarðs-ormr gein yfir uxa-hǫfuð’it, en ǫngull’inn vá í góm’inn orm’inum} ‘The Middenyardswyrm yawned over the ox-head, and the hook went into the roof of the wyrm’s mouth’, which is decently close to st. 22 below.  The name Heavenrid (\emph{Himinhrjóðr}) is otherwise only found in thules listing names of oxen, and the interesting detail of Thunder’s feet going through the boat is only paralleled by the Swedish Altuna stone (though see note to st. 34/2 below).

While \Gylfaginning\ 48, the Scaldic fragments, and \Hymiskvida\ all share the central narrative of the fishing expedition, \Hymiskvida\ has several additional narratives woven into it.  That is not to say that \Hymiskvida\ consists of multiple originally separate poems.  Unlike, say, \Havamal, which has noticable differences of style and language between its constiuent strands, \Hymiskvida\ comes off as a strong stylistic and narrative whole, composed by a single poet and thereafter transmitted faithfully.  One may roughly identify the following narrative divisions in \Hymiskvida, of which only numbers 2–4 are found in the other sources for the myth of Thunder’s fishing:

\begin{enumerate}
  \item 1–6 Thunder attempts to force the ettin Eagre to host a banquet for the Gods; Eagre in turn asks for a cauldron big enough to brew enough ale for them all.
  \item 7–16 Thunder and Tew go to visit the stingy ettin Hymer, who owns such a cauldron; horrified at Thunder’s great appetite during the evening, Hymer tells them that they must eat fish the next.
  \item 17–19 Thunder says that he will go fishing if he is given bait; Hymer challenges him to kill one of his oxen for bait; Thunder tears off the head of one.
  \item 20–25 Hymer, Thunder and Tew go fishing; Hymer pulls up some whales; with the ox-head as bait Thunder manages to hook the Middenyardswyrm itself; he loses it.
  \item 26–27 Hymer challenges Thunder to carry the boat and whales back to his farm; he does.
  \item 28–32 Hymer challenges Thunder to break a supposedly indestructible chalice; he succeeds by smashing it against the ettin’s forehead.
  \item 33–36 Thunder and Tew depart with the cauldron; they find themselves followed by a troop led by Hymer; Thunder kills them all.
  \item 37–38 Lock makes the leg of one of Thunder’s goats halt.
  \item 39 Thunder returns to the Gods with Hymer’s cauldron; they host a banquet.
\end{enumerate}

The fishing expedition, found at the very center of the poem, is thus framed by the unique narrative of Thunder and Tew obtaining a huge cauldron from Hymer for the sake of brewing ale, and several other superfluous narratives scattered throughout.  The poet has not been entirely successful in his endeavour, and there are several loose strands.  Most notably the god Tew plays no role at all in the fishing expedition, probably because he was not originally in it; in other variants of the myth (including pictoral depictions, like that from Gosforth), Thunder is only accompanied by Hymer.  Tew also lacks a reaction to the murder of his father Hymer, and this familiar relationship is also unparalleled; in \Skaldskaparmal\ 16 Tew is called Weden’s Son.   Also unclear is the function of Lock’s halting one of Thunder’s goats (sts. 37–38); he does not appear anywhere else in the poem.

\sectionline

The poem has some interesting reoccurring themes.  The “otherness” of the Ettins, specifically Hymer, is constantly emphasized in several ways:
\begin{itemize}
  \item they live far to the East (st. 5) in an inhospitable, frozen climate (st. 10), associated with mountains (sts. 2, 17) and lava-fields (st. 36)
  \item they are physically deviant, being misshapen (st. 10), grey-haired (st. 16), many-headed (sts. 8, 35), and very hard-boned (sts. 30–31); they are even likened to apes (st. 20), whales (st. 36) and Danes (st. 17; see note!),
  \item they are stingy and inhospitable (sts. 9, 16),
  \item and sarcastic and cowardly (st. 19–20, 25–26, 28–32).
\end{itemize}

In these ways the Ettins oppose the Old Germanic social norms as represented by the Gods, who live in a lush green climate and are young, beautiful and generous.  The one exception is of course Tew’s mother in st. 8, who is light-haired (in contrast to the swarthy grandmother, presumably) and generous.  Perhaps the poet is implying that it is from her that Tew has inherited his good traits?

The last point, viz. sarcasm and cowardice, is seen throughout the poem in the way Thunder comically humiliates the Ettins, especially by completing challenges issued to him.  These follow a similar format: Thunder is given a near-impossible test of strength, which he shortly completes through a mix of physical strength and cleverness, humiliating the challenger.  These tests are finding a huge kettle (st. 3, explicitly called Eagre’s “revenge” (\emph{hęfnd}), taking one of Hymer’s oxen for bait (st. 17–18), carrying home Hymer’s whales and boat (st. 26), breaking Hymer’s finest chalice (st. 28), and perhaps also taking away the kettle (st. 33)—though that may just be Hymer’s wishing to finally be rid of the pestering gods.

Much like in \Thrymskvida\ the conflict is finally resolved with righteous hammer-slaughter.  After the Gods leave, Hymer tries to get his revenge by ambushing them, but Thunder takes his trusty hammer and kills them all.  The poem is clearly humorous and meant to be performed before an audience (see st. 38 where the poet directly addresses the listeners).  The original performance context may perhaps be gleaned from the difficult final stanza. TODO: It hints at a performance at a harvest bloot.

\sectionline

\section{The Lay of Hymer}

\bvg\bva\mssnote{\Regius~13v/26, \AM~5v/25}%
Ár \alst{v}al-tívar \hld\ \alst{v}ęiðar nǫ́mu &
ok \alst{s}umbl-\alst{s}amir \hld\ \edtrans{áðr \alst{s}aðir yrði}{before they might eat}{\Bfootnote{Lit. “might become sated”.}}, &
\alst{h}ristu tęina \hld\ ok á \alst{h}laut sǫ́u, &
fundu at \alst{Ę́}gis \hld\ \alst{ø}r-kost hvera.\eva

\bvb Of yore the slain-Tews \ken{gods} had caught game, \\
and together at the \inx[C]{simble} before they might eat \\
they shook the twigs and looked at the \inx[C]{leat}; \\
they found at Eagre’s a great choice of cauldrons.\footnoteB{The gods sprinkled the leat (\emph{hlaut} ‘sacrificial blood’) of the beasts and interpreted the pattern; they found it most auspicious to feast at Eagre’s. TODO: reference to leat-twigs.}\evb\evg


\bvg\bva\mssnote{\Regius~13v/28, \AM~5v/27}%
Sat \alst{b}erg-\alst{b}úi \hld\ \alst{b}arn-tęitr fyrir, &
\alst{m}jǫk glíkr \edtext{\alst{m}ęgi \hld\ \alst{M}iskur-blinda}{\lemma{męgi Miskur-blinda ‘lad of Misherblind’}\Bfootnote{An unexplained reference.  Misherblind might be another name for Firneet, Eagre’s father.}}, &
lęit í \alst{au}gu \hld\ \alst{Y}ggs barn í þrá: &
„þú skalt \alst{ǫ́}sum \hld\ \alst{o}pt sumbl \edtext{gęra}{\lemma{gęra ‘host’}\Afootnote{\emph{gefa} ‘give’ \AM}}!“\eva

\bvb Sat the mountain-dweller \ken*{\textsc{ettin} = Eagre} there, merry like a child, \\
much alike to the lad of Misherblind; \\
into his eyes looked the child of Ug \name{= Weden} \ken*{= Thunder} stubbornly: \\
“Thou shalt oft hold simbles for the Eese!”\footnoteB{Having seen that Eagre has a great store of cauldrons, Thunder orders him to brew ale for the feasts of the Eese.}\evb\evg


\bvg\bva\mssnote{\Regius~13v/31, \AM~5v/29}%
\alst{Ǫ}nn fekk \alst{jǫ}tni \hld\ \alst{o}rð-bę́ginn halr, &
\alst{h}ugði at \alst{h}efndum \hld\ \alst{h}ann nę́st við goð, &
bað \alst{S}ifjar ver \hld\ \alst{s}ér fǿra hver, &
„þann’s ek \alst{ǫ}llum \alst{ǫ}l \hld\ \alst{y}ðr of hęita.“\eva

\bvb Great toil for the ettin the word-peevish man \ken*{= Thunder} caused; \\
he \ken*{= Eagre} thought of revenge, soon, against the god; \\
he bade Sib’s husband \ken*{= Thunder} bring him a cauldron, \\
“that one with which I for you all ale might heat.\footnoteB{Eagre gets back at Thunder by telling him that he needs a single cauldron which can hold enough ale to supply all the Eese.}”\evb\evg


\bvg\bva\mssnote{\Regius~14r/1, \AM~5v/30}%
Né þat \alst{m}ǫ́ttu \hld\ \alst{m}ę́rir tívar &
ok \alst{g}inn-ręgin \hld\ of \alst{g}eta hvęr-gi, &
unds af \alst{t}ryggðum \hld\ \alst{T}ýr Hlórriða &
\alst{á}st-ráð mikit \hld\ \alst{ęi}num sagði:\eva

\bvb But that one might the renowned \inx[G]{Tews} \\
and the \inx[G]{yin-Reins} nowhere get ahold of— \\
until, out of loyalty, a great loving counsel \\
Tew to Loride \name{= Thunder} alone did say:\evb\evg


\bvg\bva\mssnote{\Regius~14r/3, \AM~6r/2}„Býr fyr \alst{au}stan \hld\ \alst{É}li-vága &
\alst{h}und-víss \alst{H}ymir \hld\ at \alst{h}imins ęnda, &
á \alst{m}inn faðir \hld\ \alst{m}óðugr kętil, &
\edtext{\alst{r}úm-brugðinn}{\Afootnote{\emph{†rumbrygðan†} \AM}} hver \hld\ \alst{r}astar djúpan.“\eva

\bvb “Dwells to the east of the \inx[L]{Ilewaves} \\
the hound-wise Hymer, at heaven’s end.\footnoteB{According to \Vafthrudnismal\ 31 the Ilewaves were the poisonous wild rushes out of which the ettins emerged, and so it only makes sense that they would be found in the east, where the ettins dwell. Hymer’s dwelling even further east than them illustrates his fierce nature.} \\
Owns my father \ken*{= Hymer}, fierce, a kettle: \\
a size-famed cauldron one \inx[C]{rest} deep.”\evb\evg


\bvg\bva\speakernote{[Þórr kvað:]}\mssnote{\Regius~14r/4, \AM~6r/4}%
„Vęitst, ef \alst{þ}iggjum \hld\ \alst{þ}ann lǫg-velli?“ &
\speakernote{[Týr kvað:]}„Ef, \alst{v}inr, \alst{v}élar \hld\ \alst{v}it gørvum til!“\eva

\bvb\speakernoteb{[Thunder quoth:]}%
“Knowest thou if we will receive that liquid-boiler \ken{cauldron}?” — \\
\speakernoteb{[Tew quoth:]}%
“If, friend, we two make use of wiles!”\footnoteB{Like elsewhere in this poem the speakers are not indicated, but it is most sensible that Thunder asks and Tew answers.}\evb\evg

\bvg\bva\mssnote{\Regius~14r/5, \AM~6r/4}%
Fóru \alst{d}rjúgum \hld\ \edtrans{\alst{d}ag þann framan}{from the beginning of the day}{\Afootnote{emend. after \textcite{FinnurEdda}; \emph{dag þann fram} ‘on that day forth’ \Regius; \emph{dag fráliga} ‘swiftly at day’ \AM}} &
\alst{Á}sgarði frá \hld\ unds til \edtrans{\alst{Ę}gils}{Eyel}{\Afootnote{so \Regius; \emph{Ę́gis} ‘Eagre’ \AM\ is probably from confusion with Eagre (the ettin) described earlier in the poem, though the shepherd may have shared his name.}} kvǫ́mu; &
\edtrans{\alst{h}irði \alst{h}afra \hld\ \alst{h}orn-gǫfgasta}{he kept the he-goats most esteemed of horns}{\Bfootnote{He took care of Thunder’s two goats.}}; &
\alst{h}urfu at \alst{h}ǫllu \hld\ es \alst{H}ymir átti.\eva

\bvb They journeyed long from the beginning of the day, \\
away from Osyard, until to Agle they came— \\
he herded the he-goats noblest of horns— \\
they turned to the hall which Hymer owned.\evb\evg


\bvg\bva\mssnote{\Regius~14r/7, \AM~6r/6}%
\alst{M}ǫgr fann ǫmmu, \hld\ \alst{m}jǫk lęiða sér, &
\alst{h}afði \alst{h}ǫfða \hld\ \alst{h}undruð níu. &
en \edtrans{\alst{ǫ}nnur}{another woman}{\Bfootnote{The use of the word “son” in the following line reveals this as Tew’s mother.  The poet stresses her beauty of dress and countenance, in contrast to the grandmother.}} gekk \hld\ \alst{a}l-gullin framm &
\alst{b}rún-hvít \alst{b}era \hld\ \alst{b}jór-vęig syni:\eva

\bvb The lad \ken*{= Tew} found his grandmother very loathsome; \\
of heads she had nine hundred. \\
But another woman, all-golden, walked forth, \\
white-browed, bringing a beer-draught for [her] son \ken*{= Tew}:\evb\evg


\bvg\bva\speakernote{[Týs móðir:]}\mssnote{\Regius~14r/9, \AM~6r/8}%
„\alst{Á}tt-niðr \alst{jǫ}tna \hld\ \alst{e}k vilja’k ykkr &
\alst{h}ug-fulla tvá \hld\ und \alst{h}vera sętja; &
es \alst{m}ínn \edtrans{fríi}{lover}{\Afootnote{so \Regius; \emph{faðir} ‘father’ \AM}} \hld\ \alst{m}ǫrgu sinni &
\edtext{\alst{g}løggr við \alst{g}ęsti \hld\ \alst{g}ǫrr ills hugar}{\lemma{gløggr \dots\ hugar ‘stingy \dots\ mood’}\Bfootnote{Ettins are characteristically inhospitable, in stark opposition to the Old Germanic social norms; see Introduction to the poem above.  This statement foreshadows the later hunting expedition starting at st. 16 below.}}.“\eva

\bvb\speakernoteb{[Tew’s mother:]}“O descendant of ettins \ken*{= Tew}, \emph{I} would wish to hide \\
you two, full of heart, under the cauldrons; \\
many a time has my lover \ken*{= Hymer} been \\
stingy with guests, quick to bad mood.”\evb\evg


\bvg\bva\mssnote{\Regius~14r/11, \AM~6r/9}En \alst{v}á-skapaðr \hld\ \alst{v}arð \edtrans{síð-búinn}{come late}{\Afootnote{om. \AM}}, &
\alst{h}arð-ráðr \alst{H}ymir, \hld\ \alst{h}ęim af vęiðum; &
\alst{g}ekk inn í sal, \hld\ \alst{g}lumðu \edtrans{jǫklar}{icicles}{\Bfootnote{viz. in Hymer’s frozen beard.  In modern Icelandic the word \emph{jökull} has come to mean ‘glacier’, but its original meaning (as found in the present stanza) is that of its English cognate ‘icicle’.}}, &
vas \alst{k}arls, es \alst{k}om, \hld\ \alst{k}inn-skógr frørinn.\eva

\bvb But the misshapen one was come late, \\
hard-minded Hymer, home from the hunt. \\
He entered the hall—the icicles clattered— \\
on the churl who came \ken*{= Hymer} was the cheek-shaw \ken{beard} frozen.\evb\evg


\bvg\bva\speakernote{[Týs móðir:]}\mssnote{\Regius~14r/13, \AM~6r/11}„\edtext{Ves þú \alst{h}ęill, \alst{H}ymir, \hld\ í \alst{h}ugum góðum!}{\lemma{Ves þú hęill, \dots\ í hugum góðum! ‘Be thou hale \dots\ in good spirits!’}\Bfootnote{A formulaic greeting; cf. the almost identical greeting in \emph{N B380} (edited below under Galders).  Further afield cf. the type exemplified by \Beowulf\ 407a: \emph{Wæs þú, Hróðgâr, hâl} ‘Be thou, Rothgar, hale!’}} &
Nú ’s \alst{s}onr kominn \hld\ til \alst{s}ala þinna, &
sá’s \alst{v}it \alst{v}ę́ttum \hld\ af \alst{v}egi lǫngum; &
fylgir \alst{h}ǫ́num \hld\ \alst{H}róðrs and-skoti, &
\alst{v}inr \alst{v}er-liða; \hld\ \alst{V}éurr hęitir sá.\eva

\bvb\speakernoteb{[Tew’s mother:]}“Be thou hale, Hymer, in good spirits! \\
Now the son \ken*{= Tew} is come to thy halls, \\
the one whom we have been awaiting from a long way off. \\
Follows him the opponent of Rooder \name{ettin}, \\
the friend of manly retinues; \inx[P]{Wighward} \name{= Thunder} is that one called.\evb\evg


\bvg\bva\mssnote{\Regius~14r/15, \AM~6r/13}\alst{S}é þú hvar \alst{s}itja \hld\ und \alst{s}alar gafli, &
\alst{s}vá \edtext{forða \alst{s}ér}{\Afootnote{\emph{forðask} \AM}}, \hld\ stęndr \edtrans{\alst{s}úl}{column}{\Afootnote{\emph{†sol†} \AM}} fyrir.“ &
\alst{S}undr stǫkk \alst{s}úla \hld\ fyr \alst{s}jón jǫtuns, &
en \edtext{\alst{a}llr}{\Afootnote{emend.; \emph{áðr} ‘earlier, before that’ \Regius\AM. TODO: elaborate, mention Finnur}} í tvau \hld\ \alst{á}ss brotnaði.\eva

\bvb See where they sit beneath the hall’s gable: \\
\emph{so} they save themselves—a column stands before them!\footnoteB{Tew’s mother reveals the hiding place of the gods.}” \\
The column crashed down before the ettin’s gaze \ken*{= Hymer}, \\
and all in two the roof-beam broke.\evb\evg


\bvg\bva\mssnote{\Regius~14r/17, \AM~6r/15}Stukku \alst{á}tta, \hld\ en \alst{ęi}nn af þęim &
\alst{h}verr \alst{h}arð-slęginn \hld\ \alst{h}ęill af þolli; &
\alst{f}ramm gingu þęir, \hld\ en \alst{f}orn jǫtunn &
\alst{s}jónum lęiddi \hld\ \alst{s}inn and-skota.\eva

\bvb Eight [cauldrons] crashed down, but one of them— \\
a hard-forged cauldron—[came] whole off its peg.\footnoteB{Nine cauldrons were hanging from the roof-beam supported by the column.  Eight of them broke, but a single one remained whole; this is presumably the cauldron the Gods will later get.} \\
Forth they went, and the ancient ettin \ken*{= Hymer} \\
with his gaze tracked his very opponent \ken*{= Thunder}.\evb\evg


\bvg\bva\mssnote{\Regius~14r/19, \AM~6r/16}\edtrans{Sagði-t \alst{h}ǫ́num \hld\ \alst{h}ugr vęl}{His heart did not please him}{\Bfootnote{Lit. ‘his heart did not speak well to him’.}} þá’s sá &
\alst{g}ýgjar \edtrans{\alst{g}rǿti}{distresser}{\Afootnote{\emph{gę́ti} ‘keeper, warder’ \AM}} \hld\ á \alst{g}olf kominn, &
\alst{þ}ar vǫ́ru \alst{þ}jórar \hld\ \alst{þ}rír of tęknir, &
bað \edtrans{\alst{s}ęnn}{at once}{\Afootnote{\emph{sun} ‘[his] son \ken*{= Tew}?’ \AM}} jǫtunn \hld\ \alst{s}jóða ganga.\eva

\bvb His heart did not please him when as he saw \\
the \inx[C]{gow}’s distresser \ken*{= Thunder} come onto the floor. \\
There three bulls were a-taken: \\
the ettin bade them at once be cooked.\evb\evg


\bvg\bva\mssnote{\Regius~14r/21, \AM~6r/18}\alst{H}vęrn létu þęir \hld\ \alst{h}ǫfði skęmra &
auk á \alst{s}ęyði \hld\ \alst{s}íðan bǫ́ru, &
át \alst{S}ifjar verr \hld\ áðr \alst{s}ofa gingi, &
\alst{ęi}nn með \alst{ǫ}llu \hld\ \alst{ø}xn tvá Hymis.\eva

\bvb Each one they let shorten by a head, \\
and onto the cooking-pit then did carry: \\
Sib’s husband \ken*{= Thunder} ate—before he might go sleep— \\
alone by himself two of Hymer’s oxen.\footnoteB{Cf. \Thrymskvida\ 24 for another instance of Thunder’s great eating, which curiously also uses the kenning \emph{Sifjar verr} ‘Sib’s husband \ken*{= Thunder}’.}\evb\evg


\bvg\bva\mssnote{\Regius~14r/23, \AM~6r/19}Þótti \alst{h}ǫ́rum \hld\ \alst{H}rungnis spjalla &
\alst{v}erðr Hlórriða \hld\ \alst{v}ęl full-mikill, &
„\edtext{munum at \alst{a}ptni \hld\ \alst{ǫ}ðrum verða &
\alst{v}ið \alst{v}ęiði-mat \hld\ \alst{v}ér þrír lifa.}{\lemma{munum \dots\ lifa ‘the next \dots\ live’}\Bfootnote{The poet is pushing at the limits of Old Norse syntax with this word order.  In prose word order it should be construed as: \emph{at ǫðrum aptni munum vér þrír verða lifa við vęiði-mat}, where \emph{verða} ‘have to, must’ is used like its modern German cognate \emph{werden}.

Hymer’s stinginess—he refuses to share more of his own food but instead forces his guests to go hunt—breaks all Indo-European rules of hospitality and illustrates the otherness of the Ettins.  See Introduction to the poem.}}“\eva

\bvb To Rungner’s hoary friend \ken*{= Hymer} did seem \\
Loride’s \name{Thunder’s} eating far too great; \\
“the following evening we three will \\
on game-meat have to live.”\evb\evg


\bvg\bva\mssnote{\Regius~14r/24, \AM~6r/21}\alst{V}éurr kvaðsk \alst{v}ilja \hld\ á \alst{v}ág róa, &
ef \alst{b}allr jǫtunn \hld\ \alst{b}ęitur gę́fi. &
„\alst{H}verf þú til \edtext{\alst{h}jarðar}{\Afootnote{\emph{hallar} corr. \AM}}, \hld\ ef \alst{h}ug trúir, &
\alst{b}rjótr \edtrans{\alst{b}erg-Dana}{boulder-Danes \ken{ettins}}{\Bfootnote{Kennings of this type emphasize the otherness of the Ettins (see Introduction to the poem above) by equating them with ethnic foreigners, and are well known from Anlif Gothrunson’s Drape for Thunder (\emph{Þórsdrápa}), where Ettins are called Scots, Swedes, Danes, Ruges and Hareds; all ethnic enemies of the Norwegian Earl Hathkin, at whose court that poem may have been composed.}}, \hld\ \alst{b}ęitur sǿkja.\eva

\bvb Wighward \name{= Thunder} called himself willing to row on the wave, \\
if the baleful ettin might give pieces of bait. \\
“Turn to the herd—if thou trust in thy heart, \\
O breaker of boulder-Danes \ken*{\textsc{ettins} > = Thunder}—to seek pieces of bait.\evb\evg


\bvg\bva\mssnote{\Regius~14r/26, \AM~6r/23}\alst{Þ}ess \edtext{vę́ntir mik}{\Afootnote{so \AM; \emph{vę́nti ek} \Regius}}, \hld\ at \alst{þ}ér \edtrans{myni-t}{will not}{\Afootnote{so \AM; \emph{myni} ‘will’ \Regius.  The \AM\ reading is preferable since it makes this the first of Hymer’s several challenges of strength to Thunder, which the god, to the ettin’s humiliation, easily accomplishes.}} &
\alst{ǫ}gn at \alst{o}xa \hld\ \alst{au}ð-feng vesa.“ &
\edtrans{\alst{S}vęinn}{The swain}{\Bfootnote{Thunder was apparently in the shape of a youth.  This detail is also found in \Gylfaginning\ 48, where Snorre writes: \emph{Gekk hann út of Miðgarð svá sem ungr drengr \dots} ‘He went out about Middenyard in the shape of a young warrior’.}} \alst{s}ýsliga \hld\ \alst{s}vęif til skógar, &
þar’s \edtext{\alst{o}xi stóð \hld\ \alst{a}l-svartr}{\lemma{oxi \dots\ alsvartr ‘all-black \dots\ ox’}\Bfootnote{Formulaic, also occuring in \Thrymskvida\ 23; see note there for further parallels to the custom of sacrificing animals of certain colours.  It seems that all-black oxen were thought the noblest, and so Thunder’s slaying one instead of an inferior beast is probably intended to humiliate the stingy Hymer.

In \Gylfaginning\ 48 we read that: \emph{Hann tók inn mesta uxann, er Himin-hrjóðr hét, ok sleit af hǫfuðit ok fór með til sjávar.} ‘He took the greatest ox, which was called Heavenrid, and tore of its head and went with it to the sea’.}} fyrir.\eva

\bvb I expect that the bait from the ox \\
will not be an easy catch for thee!”— \\
The swain \ken*{= Thunder} swiftly turned to the wood, \\
where an ox stood, all-black, before [him].\evb\evg


\bvg\bva\mssnote{\Regius~14r/28, \AM~6r/24}Braut af \alst{þ}jóri \hld\ \alst{þ}urs ráð-bani &
\alst{h}ǫ́-tún ofan \hld\ \alst{h}orna tveggja. &
„\alst{V}erk þikkja þín \hld\ \alst{v}erri myklu &
\alst{k}jóla valdi \hld\ an \alst{k}yrr sitir.“\eva

\bvb Off the bull broke the counsel-slayer of the thurse \ken*{= Thunder} \\
the high meadow of the two horns \ken{head} from above.— \\
“Worse by far thy works do seem \\
to the wielder of ships \ken*{= Hymer = me} than if thou mightst sat calm.\footnoteB{I had originally taken this as Hymer snidely belittling Thunder’s feat of pulling the head off the ox (presumably by the horns); he would have earned greater glory had he simply sat and done nothing. However, it may also be read as a factual statement; Thunder just killed one of his finest oxen, and Hymer would certainly have preferred that he had not.}”\evb\evg

\sectionline

{\small The scene now shifts, and the party is out at sea.  It is possible that a stanza has here been lost, or that it would be indicated in some other way in the original performance.}

\sectionline

\bvg\bva\mssnote{\Regius~14r/30, \AM~6r/26}%
Bað \alst{h}lunn-gota \hld\ \alst{h}afra dróttinn &
\edtext{\alst{á}tt-runn}{\Afootnote{\emph{†atrænn†} \AM}} \edtrans{\alst{a}pa}{ape}{\Bfootnote{The specific sense of \emph{api} ‘ape’ is uncertain.  It seems to generally refer to a fool, but see Encyclopedia.}} \hld\ \alst{ú}tar fǿra, &
\edtext{en \alst{s}á jǫtunn \hld\ \alst{s}ína \edtext{talði}{\Afootnote{\emph{milldi} corr. \AM}}, &
\alst{l}ítla fýsi \hld\ \edtext{\alst{l}ęngra at róa}{\Afootnote{metr. emend.; \emph{at róa lęngra} \Regius\AM}}.}{\lemma{en \dots\ róa. ‘but \dots\ longer.’}\Bfootnote{Thunder’s humorous humiliation of Hymer continues with the previously spiteful ettin now forced to row against his will.}}\eva

\bvb The Lord of he-goats \ken*{= Thunder} bade the kinsman of the \inx[C]{ape}\ \ken*{\textsc{ettin} = Hymer} \\
push the launching-steed \ken{boat} further out; \\
but that ettin told of his \\
scarce wish to row longer.\evb\evg


\bvg\bva\mssnote{\Regius~14r/31, \AM~6r/27}Dró \edtrans{\alst{m}ę́rr}{famous}{\Afootnote{so \Regius; \emph{męir} ‘more, further’ \AM}} Hymir \hld\ \alst{m}óðugr hvala &
\alst{ęi}nn á \alst{ǫ}ngli \hld\ \alst{u}pp sęnn tváa; &
en \alst{a}ptr í skut \hld\ \alst{Ó}ðni sifjaðr &
\alst{V}éurr við \alst{v}élar \hld\ \alst{v}að gęrði sér.\eva

\bvb Famous, fierce Hymer pulled whales: \\
one on the hook, soon up two. \\
But back in the stern the Weden-related \\
Wighward \name{= Thunder} craftily fixed His line.\evb\evg


\bvg\bva\mssnote{\Regius~14v/1, \AM~6r/29}\alst{Ę}gnði á \alst{ǫ}ngul \hld\ sá’s \alst{ǫ}ldum bergr, &
\alst{o}rms \alst{ęi}n-bani \hld\ \alst{o}xa hǫfði; &
\alst{g}ęin við \edtrans{agni}{bait}{\Afootnote{so \AM; \emph{ǫngli} ‘hook’ \Regius}}, \hld\ sú’s \alst{g}oð fía, &
\edtext{\alst{u}mb-gjǫrð neðan \hld\ \alst{a}llra landa}{\lemma{umb-gjǫrð \dots\ allra landa ‘encircler of all lands’}\Bfootnote{This kenning occurs identically in a fragment by C9th scold Alewigh Snub (Ǫlv \emph{Þórr} in \emph{SkP} III).}}.\eva

\bvb Baited on the hook He who rescues men \ken*{= Thunder}—  \\
the Wyrm’s Lone Slayer—the ox’s head. \\
Snapped at the bait the one whom the Gods hate \ken*{= Middenyardswyrm}— \\
the encircler of all lands—from below.\evb\evg


\bvg\bva\mssnote{\Regius~14v/3, \AM~6v/1}\alst{D}ró \alst{d}jarf-liga \hld\ \alst{d}áð-rakkr Þórr &
\alst{o}rm \alst{ęi}tr-fáan \hld\ \alst{u}pp at borði; &
\alst{h}amri kníði \hld\ \edtrans{\alst{h}ǫ́-fjall skarar}{high mountain of hair \ken{head}}{\Bfootnote{A rather unfitting kenning, since serpents do not have hair.}} &
\alst{o}f-ljótt \alst{o}fan \hld\ \alst{u}lfs hnit-bróður.\eva

\bvb Bravely deed-ready Thunder pulled \\
the venom-glistening Wyrm up on the gunwale; \\
with the hammer He struck the high mountain of hair \ken{head}— \\
very hideous, from above—on the Wolf’s clash-brother \ken*{= Middenyardswyrm}.\evb\evg


\bvg\bva\mssnote{\Regius~14v/5, \AM~6v/2}\edtrans{\alst{H}raun-gǫlkn}{The lavafield-monsters}{\Bfootnote{Both mss. have \emph{hręin-}, which may mean either ‘clean’ or ‘reindeer’, neither of which fit. On the other hand \emph{hraun} \ONP: ‘stone/barren area, wasteland; lavafield’ is well attested in Scaldic kennings for ettins. The precise meaning of \emph{galkn} ‘monster’ (plural \emph{gǫlkn}) is unclear; but it is attested in three Scaldic verses, always in kennings of the type “troll-woman of the shield \ken{axe}”.  While the mss. spelling ‘\emph{galkn}’ (norm. \emph{gálkn}) could reflect either singular and plural, the form of the verb is plural.  This means that the word cannot be referring to the Middenyardswyrm, refuting the interpretation of \textcite{LarringtonEdda}: “the sea-wolf shrieked”.}} \edtext{\alst{h}rutu}{\Afootnote{so \AM; \emph{hlumðu} ‘dashed’ \Regius. End-rhyme is also used by the poet in st. 3/3.}}, \hld\ ęn \alst{h}ǫlkn þutu, &
\alst{f}ór hin \alst{f}orna \hld\ \alst{f}old ǫll saman; &
\edtext{[...]}{\Bfootnote{It is very likely that a line is missing here, since the stanzas in the poem otherwise consistently have four lines.  In other tellings of the myth it is at this point that Hymer cuts Thunder’s fishing line, so that is probably what has been lost.

It is of course impossible to know what exact form it had, but for the reader’s enjoyment, based on other poets and the account in \Gylfaginning\ (see introduction to the present poem) I’ve composed the following variant lines: \emph{unds vinr Hrungnis \hld\ vað Þórs of skar} ‘until the friend of Rungner \ken*{= Hymer} Thunder’s fishing-line did cut’; \emph{unds fǫlr Hymir \hld\ fekk á saxi} ‘until pale Hymer grasped the knife’.}} &
\alst{s}økkðisk \alst{s}íðan \hld\ \alst{s}á \edtrans{fiskr}{fish}{\Bfootnote{The Middenyardswyrm may also be called a “fish” in \Grimnismal\ 21; see note there.}} í mar.\eva

\bvb The lavafield-monsters \ken{ettins} bounded and the bedrock resounded; \\
the ancient earth moved all at once; \\
{[...]}; \\
sank thereafter that fish \ken*{= Middenyardswyrm} into the sea.\evb\evg


\bvg\bva\mssnote{\Regius~14v/6, \AM~6v/3}\alst{Ó}-tęitr \alst{jǫ}tunn, \hld\ es \alst{a}ptr røru, &
\edtext{[...]}{\Bfootnote{Another missing line.  As said in the previous stanza the meter usually requires four lines, and also the first half of the sentence is incomplete without a verb.}} &
svá’t \edtrans{\alst{á}r}{in the early morning}{\Bfootnote{\textcite{FinnurEdda}\ suggests \emph{svá’t at ǫ́r} ‘so that by the oar’, but this burdens the meter.  Assuming my interpretation is correct, the three would have been out fishing throughout the night.}} Hymir \hld\ \alst{ę}kki mę́lti, &
\alst{v}ęifði rǿði \hld\ \alst{v}eðrs annars til.\eva

\bvb The unmerry ettin \ken*{= Hymer}, as they rowed back, \\
{[...]}, \\
so that in early morn Hymer said nothing; \\
he pulled the oar against the wind:\evb\evg


\bvg\bva\speakernote{[Hymir:]}\mssnote{\Regius~14v/8, \AM~6v/4}„Munt of \alst{v}inna \hld\ \alst{v}erk halft við mik, &
at \alst{h}ęim \alst{h}vali \hld\ \alst{h}af til bǿjar &
eða \alst{f}lot-brúsa \hld\ \alst{f}ęstir okkarn.“\eva

\bvb\speakernoteb{[Hymer quoth:]}“Thou wilt accomplish a half work against me, \\
if thou take home the whales to the farm, \\
or our float-jar \ken{boat} do fasten.\footnoteB{Hymer tells Thunder, who having let go of the Wyrm now has nothing to show for the trip, that he can accomplish something half as good as the pulling of the whales if he carries them home or ties the boat (by the shore).}”\evb\evg


\bvg\bva\mssnote{\Regius~14v/9, \AM~6v/6}\alst{G}ekk Hlórriði \hld\ \alst{g}ręip \edtext{á}{\Afootnote{\emph{til á} \Regius}} stafni &
vatt \edtrans{með \alst{au}stri}{with the bilge-water}{\Bfootnote{That is, the bilge-water was still inside the boat.  As anyone who has handled one knows, this water weighs very much, so this was another great work of strength.}} \hld\ \alst{u}pp lǫg-fáki; &
\alst{ęi}nn með \alst{ǫ́}rum \hld\ ok með \alst{au}st-skotu &
\alst{b}ar til \alst{b}ǿjar \hld\ \alst{b}rim-svín jǫtuns &
ok \edtext{\edtext{\alst{h}olt-riða}{\Afootnote{\emph{†holtriba†} \Regius}} \hld\ \alst{h}ver}{\lemma{holt-riða hver}\Bfootnote{An uncertain and possibly corrupt kenning.  TODO: What do other editors and translators say?}} í gegnum. \eva

\bvb Loride \name{= Thunder} went, grasped the stern, \\
hurled up the lake-nag \ken{boat} with the bilge-water; \\
alone with the oars and the bilge-bucket \\
he bore to the farm the ettin’s brim-swines \ken{whales}, \\
even through the cauldron of woodland ridges \ken{valley?}.\evb\evg


\bvg\bva\mssnote{\Regius~14v/12, \AM~6v/7}\edtext{\edtext{Ok}{\Afootnote{\emph{enn} \AM}} \alst{ę}nn \alst{jǫ}tunn \hld\ umb \alst{a}frendi, &
\alst{þ}rá-girni vanr, \hld\ við \alst{Þ}ór sęnti, &
kvað-at mann \alst{r}amman, \hld\ þótt \alst{r}óa kynni, &
\alst{k}rǫptur-ligan, \hld\ nema \alst{k}alk bryti.}{\lemma{ALL}\Bfootnote{Even after witnessing numerous great feats of strength Hymer still refuses to admit Thunder’s superiority.  He now insists on challenging him with breaking his indestructible chalice.}}\eva

\bvb And yet the ettin, used to stubbornness, \\
over strength of hand did flyte with Thunder; \\
he called no man strong—although he could row, \\
mightily—unless he broke the chalice.\evb\evg


\bvg\bva\mssnote{\Regius~14v/14, \AM~6v/9}En \alst{H}lórriði, \hld\ es at \alst{h}ǫndum kom, &
\alst{b}rátt lét \alst{b}resta \hld\ \edtrans{\alst{b}ratt-stęin glęri}{steep stone with glass}{\Bfootnote{That is, he broke the stone columns in Hymer’s house with the chalice.}}, &
\alst{s}ló \edtrans{\alst{s}itjandi}{fastened}{\Bfootnote{This word is ambiguous and can modify either Thunder (in which case it would mean “sitting”) or the columns (\emph{súlur}).  I have chosen the latter and read it as signifying their stability.}} \hld\ \alst{s}úlur í gǫgnum; &
bǫ́ru þó \alst{h}ęilan \hld\ fyr \alst{H}ymi síðan.\eva

\bvb But Loride \name{= Thunder}, when it came to his hands, \\
impatiently crushed steep stone with glass; \\
he struck right through the fastened columns; \\
it was still brought whole before Hymer afterward.\evb\evg


\bvg\bva\mssnote{\Regius~14v/16, \AM~6v/10}Unds þat hin \alst{f}ríða \hld\ \alst{f}riðla kęndi &
\alst{á}st-ráð mikit, \hld\ \alst{ęi}tt es vissi, &
„drep við \alst{h}aus \alst{H}ymis, \hld\ hann ’s \alst{h}arðari, &
\edtrans{\alst{k}ost-móðs jǫtuns}{the choice-weary ettin’s}{\Bfootnote{Presumably referring to the Gods’ having already eaten all his choicest food and slain his finest bull.}}, \hld\ \alst{k}alki hvęrjum.“\eva

\bvb Until the handsome mistress \ken*{= Tew’s mother} gave \\
a great loving counsel, the one she knew: \\
“Strike against Hymer’s skull; it is harder— \\
the choice-weary ettin’s—than every chalice.”\evb\evg


\bvg\bva\mssnote{\Regius~14v/18, \AM~6v/12}\alst{H}arðr \edtext{ręis}{\Afootnote{om. \AM}} á kné \hld\ \alst{h}afra dróttinn, &
fǿrðisk \alst{a}llra \hld\ í \alst{á}s-męgin; &
\alst{h}ęill vas karli \hld\ \alst{h}jalm-stofn ofan, &
en \alst{v}ín-fęrill \hld\ \alst{v}alr rifnaði.\eva

\bvb Hard on the knee rose the Lord of he-goats \ken*{= Thunder}; \\
He drew Himself into His highest Os-might.\footnoteB{Compare \Gylfaginning\ in its description of Thunder attempting to pull up the Wyrm: \emph{Þá varð Þórr reiðr ok fę́rðist í ás-megin} “Then Thunder became wroth, and drew himself into his os-might.”}— \\
Whole was on the churl \ken*{= Hymer} the helmet-stump \ken{head} above, \\
but the round wine-track \ken{chalice} rent apart.\evb\evg


\bvg\bva\speakernote{[Hymir kvað:]}\mssnote{\Regius~14v/20, \AM~6v/13}„\alst{M}ǫrg vęit’k \alst{m}ę́ti \hld\ \alst{m}ér gingin frá, &
\edtext{es}{\Afootnote{om. \Regius}} \alst{k}alki sé’k \hld\ \edtext{fyr}{\Afootnote{\emph{†yr†} \Regius}} \alst{k}néum hrundit,“ &
\alst{k}arl orð of \alst{k}vað: \hld\ „\edtext{\alst{k}ná’k-at sęgja &
\alst{a}ptr \alst{ę́}va-gi: \hld\ ‚þú ’st \alst{ǫ}lðr of hęitt.}{\lemma{kná’k-at \dots\ of hęitt. ‘I cannot \dots\ O ale!’}\Bfootnote{Hymer laments that with the loss of his finest vessel he will never be able to enjoy his drink again.  There is strong irony here since it was he himself who challenged Thunder to break it.}}‘\eva

\bvb\speakernoteb{[Hymer quoth:]}“I know many treasures have passed from me, \\
when I see the chalice thrown before [his] knees!”— \\
The churl spoke \ken*{= Hymer} words: “I cannot say \\
ever again: ‘Thou art brewed, O Ale!’\evb\evg


\bvg\bva\mssnote{\Regius~14v/22, \AM~6v/15}Þat ’s til \alst{k}ostar \hld\ ef \alst{k}oma mę́ttið &
\alst{ú}t ór \alst{ó}ru \hld\ \edtrans{\alst{ǫ}l-kjól}{ale-ship \ken{cauldron}}{\Bfootnote{\emph{ǫl-kjól} is the accusative of \emph{ǫl-kjóll}, but in this context (\CV: \emph{koma}, B) we would expect the dative \emph{ǫl-kjóli}.  The meter does not allow for this, however.}} \edtrans{hofi}{hall}{\Bfootnote{This is the only Old Norse occurrence of the word \emph{hof} in the sense ‘hall, house’; it otherwise only means ‘temple’ (\inx[C]{hove}).  The West Germanic cognates consistently mean ‘hall’, and that is probably the original sense, so it is unclear if this is an instance of foreign influence (if so, most likely Anglo-Saxon) or just a poetic archaism.}}.“ &
\alst{T}ýr lęitaði \hld\ \alst{t}ysvar hrǿra; &
stóð at \alst{h}vǫ́ru \hld\ \alst{h}verr kyrr fyrir.\eva

\bvb It would be best if ye might bring \\
the ale-ship \ken{cauldron} out of our hall.” \\
Tew attempted, twice, to move it— \\
each time stood the cauldron still before [him].\evb\evg


\bvg\bva\mssnote{\Regius~14v/24, \AM~6v/16}\alst{F}aðir Móða \hld\ \alst{f}ekk á þręmi &
ok í \alst{g}ǫgnum stęig \hld\ \alst{g}olf niðr í sal; &
\alst{h}óf sér á \alst{h}ǫfuð upp \hld\ \alst{h}ver Sifjar verr, &
en á \alst{h}ę́lum \hld\ \edtrans{\alst{h}ringar skullu}{the rings clattered}{\Bfootnote{i.e. the chain-links.  This detail is mentioned in an example sentence contrasting long and short phonemes in \FGT: \emph{heyrði til hǫddu, þá er Þórr bar hverinn} ‘the sound of the pot-links (\emph{hadda}) was heard when Thunder bore the cauldron’.  According to \textcite{FinnurEdda}\ the chain (or \emph{hadda}) on a Wiking-age cauldron would have reached across, in which case this would be a reference to the cauldron’s enormous size, with its diameter—mentioned in st. 5 as one \inx[C]{rest}—being roughly the same as Thunder’s height.}}.\eva

\bvb The father of Moody \ken*{= Thunder} grasped the brim, \\
and stepped down through the floor in the hall;\footnoteB{In the account of \Gylfaginning\ Thunder is said to have stepped through the boat when trying to pull up the Middenyardswyrm.  This detail is also seen on the carving of the Altuna stone from Uppland, Sweden; it may have been transposed to this place in the narrative. TODO.} \\
Sib’s husband \ken*{= Thunder} heaved the cauldron up onto his head, \\
and at his heels the rings clattered.\evb\evg


\bvg\bva\mssnote{\Regius~14v/26, \AM~6v/18}Fóru-t \alst{l}ęngi, \hld\ áðr \alst{l}íta nam &
\alst{a}ptr \alst{Ó}ðins sonr \hld\ \alst{ęi}nu sinni; &
sá ór \alst{h}ręysum \hld\ með \alst{H}ymi austan &
\edtext{\alst{f}olk-drótt}{\lemma{folk-drótt \dots\ fjǫl-hǫfðaða ‘war-troop \dots\ many-headed’}\Bfootnote{A deviant number of body parts, especially heads, is typical of ettins.  See Introduction and note to st. 8 above.}} \alst{f}ara \hld\ \alst{f}jǫl-hǫfðaða.\eva

\bvb They journeyed not for long before Weden’s son \ken*{= Thunder} \\
took to look back a single time— \\
he saw out of stone-heaps, with Hymer from the east, \\
a war-troop coming, many-headed.\evb\evg


\bvg\bva\mssnote{\Regius~14v/28, \AM~6v/19}\alst{H}óf sér af \alst{h}ęrðum \hld\ \alst{h}ver standandi, &
vęifði \alst{M}jǫllni \hld\ \alst{m}orð-gjǫrnum framm, &
ok \alst{h}raun-\alst{h}vala \hld\ \alst{h}ann alla drap.\eva

\bvb He heaved off his shoulders the cauldron, standing; \\
he swung the murder-eager Millner forth, \\
and the rock-whales \ken{ettins} all he slew.\evb\evg


\bvg\bva\mssnote{\Regius~14v/30, \AM~6v/21}\edtext{Fóru-t \alst{l}ęngi, \hld\ áðr \alst{l}iggja nam &
\alst{h}afr \alst{H}lórriða \hld\ \alst{h}alf-dauðr fyrir, &
vas \edtext{\alst{sk}ę́r}{\Afootnote{emend. from meaningless \emph{†skirr†} \Regius\AM}} \alst{sk}ǫkuls \hld\ \alst{sk}akkr á bęini, &
en því hinn \alst{l}ę́-vísi \hld\ \alst{L}oki of olli.}{\lemma{Fóru-t \dots\ olli. ‘They journeyed \dots\ caused.’}\Bfootnote{Lock, who is not mentioned earlier in the poem, was apparently placing curses on the returning party.  Snorre mentions this, TODO.}}\eva

\bvb They journeyed not for long before Loride’s \name{= Thunder’s} he-goat \\
took to lie half-dead before [them]; \\
the steed of the cart-pole \ken{goat} was halt in the leg, \\
but that the guile-wise Lock had caused.\evb\evg


\bvg\bva\mssnote{\Regius~14v/32, \AM~6v/22}En \edtrans{ér}{ye}{\Bfootnote{The audience.  As pointed out by \textcite{FinnurEdda} an address to the audience of this type is otherwise unparalleled in Eddic mythological poetry.  Such are however fairly common in Scaldic poetry, with which this poem shares several traits (see Introduction above).}} \alst{h}ęyrt \alst{h}afið, \hld\ \alst{h}vęrr kann umb þat &
\edtrans{\alst{g}oð-mǫ́lugra}{god-speaking}{\Bfootnote{This word is a hapax, but easily understood.  One who is \emph{goð-mǫ́lugr} is ‘able to speak about the god-lore’, i.e. ‘versed in the mythology’.}} \hld\ \alst{g}ørr at skilja, &
\alst{h}vęr af \alst{h}raun-búa \hld\ \alst{h}ann laun of fekk, &
es \alst{b}ę́ði galt \hld\ \alst{b}ǫrn sín fyrir.\eva

\bvb But ye have heard—about that can \\
any god-speaking man more clearly discern— \\
which recompense he \ken*{= Thunder} from the lavafield-dweller \ken{ettin} got, \\
as he yielded up both his own children for it.\evb\evg


\bvg\bva\mssnote{\Regius~15r/1, \AM~6v/24}\alst{Þ}rótt-ǫflugr kom \hld\ á \alst{þ}ing goða &
ok \alst{h}afði \alst{h}ver, \hld\ þann’s \alst{H}ymir átti; &
en \alst{v}éar hvęrjan \hld\ \alst{v}ęl skulu drekka &
\alst{ǫ}lðr at \alst{Ę́}gis \hld\ \edtrans{\alst{ęi}tt hǫr-męitið}{one \dots\ flax-cutting}{\Bfootnote{A very obscure kenning. \textcite{LaFargeGlossary} give several interpretations, viz. \emph{ęitr-hǫr-męitir} ‘poison-rope-cutter \ken{snake > winter}’, \emph{ęitr-orm-męiðir} ‘poison-worm-injurer’ \ken{winter}. The solution with the minimal amount of emendation is to read \emph{ęitt} ‘one’ as modifying \emph{ǫlðr} ‘ale-feast’, and \emph{hvęrjan} ‘every’ as modifying \emph{hǫr-męitiðr} ‘flax-cutting’, a compound made up of \emph{hǫrr} ‘flax, cord’ and \emph{męita} ‘to cut’, seemingly referring to an obscure harvest festival. This interpretation is by no means certain.}}.\eva

\bvb The valour-mighty one \ken*{= Thunder} came onto the \inx[C]{Thing} of the gods, \\
and had that cauldron which Hymer [had] owned; \\
but well the \inx[G]{Wighers} \name{= gods} shall drink one \\
ale-feast at Eagre’s, every flax-cutting \ken{fall?}.\evb\evg

\sectionline
% Thunder, Tew
	\bookStart{The Flyting of Lock}[Lokasęnna]

\begin{flushright}%
Dating \parencite{Sapp2022}: C10th (0.965)

Meter: \Ljodahattr%
\end{flushright}

Preserved in \Regius, directly following \Hymiskvida, though the poems without doubt were originally separate; the stylistic differences are drastical.

The poem has been interpreted as blasphemous (TODO: elaborate), but shows no linguistic signs of being particularly late.

\sectionline

\section{From Eagre and the Gods (\emph{Frá Ę́gi ok goðum})}

\bpg\bpa Ę́gir, er ǫðru nafni hét Gymir, hann hafði búit ásum ǫl þá er hann hafði fengit ketil inn mikla sem nú er sagt. Til þeirar veitslu kom Óðinn ok Frigg kona hans. Þórr kom eigi því at hann var í austr-vegi. Sif var þar, kona Þórs; Bragi, ok Iðunn kona hans. Týr var þar, hann var ein-hendr; Fenrisulfr sleit hǫnd af hánum, þá er hann var bundinn. Þar var Njǫrðr ok kona hans Skaði; Freyr ok Freyja; Víðarr son Óðins. Loki var þar, ok þjónustu-menn Freys, Byggvir ok Beyla. Mart var þar ása ok alfa. Ę́gir átti tvá þjónustu-menn; Fimafengr ok Eldir. Þar var lýsi-gull haft fyr elds-ljós; sjalft barsk þar ǫl. Þar var griða-stadr mikill. Menn lofuðu mjǫk hversu góðir þjónustu-menn Ę́gis vóru. Loki mátti eigi heyra þat, ok drap hann Fimafeng. Þá skóku ę́sir skjǫldu sína ok ǿptu at Loka, ok eltu hann braut til skógar, en þeir fóru at drekka. Loki hvarf aptr ok hitti úti Eldi; Loki kvaddi hann:\epa

\bpb \inx[P]{Eagre}, who by another name is called \inx[P]{Gymer}, had prepared an ale-feast for the Ease when he had got the great kettle as is now told.\footnoteB{See the immediately preceding \Hymiskvida.}

To that gathering came \inx[P]{Weden} and \inx[P]{Frie}, his woman. \inx[P]{Thunder} came not, for he was on the \inx[L]{Eastern Way}. Sib was there, Thunder’s woman; \inx[P]{Bray} and \inx[P]{Idun}, his woman. \inx[P]{Tew} was there, he was one-handed. The \inx[P]{Fenrerswolf} tore his hand off when it was bound.\footnoteB{This detail is probably brought up to chronologically date the events of the poem as happening after the binding of Fenrer in the mythology.} There was \inx[P]{Nearth}, and his woman \inx[P]{Shede}; \inx[P]{Free} and \inx[L]{Frow}; \inx[P]{Wider}, the son of \inx[P]{Weden}. \inx[P]{Lock} was there, and the servants of Free: \inx[P]{Bew} and \inx[P]{Beal}. There was a great many of the \inx[G]{Ease} and \inx[G]{Elves}\footnoteB{A formulaic expression, see \inx[F]{Ease and Elves}.}.

Eagre had two servants: \inx[P]{Femfinger} and \inx[P]{Elder}. There was glowing gold used instead of fire; the ale there poured itself. There was a great \inx[C]{grith-stead}.\footnoteB{A place wherein all violence was forbidden, see Encyclopedia.} Men greatly praised how good the servants of Eagre were. Lock could not stand that, and he slew Femfinger.

Then the Ease shook their shields and screamed at Lock,\footnoteB{Some sort of ancient war dance. Cf. the Old Swedish Heathen Law: “He screams three nithing-screams TODO”.} and chased him away to the forest—but then they went [back] to drinking. Lock came back and found Elder outside; Lock greeted him:\epb\epg

\sectionline

\bvg
\bva „Seg þú þat, \alst{E}ldir, \hld\ \edtext{svá’t \alst{ęi}nu-gi &
\ind \alst{f}eti gangir \alst{f}ramarr}{\lemma{svá’t \dots\ framarr ‘so that \dots\ further’}\Bfootnote{Cf. \Havamal\ 38: \emph{feti ganga framarr} ‘take one step further’.}}, &
hvat hér \alst{i}nni \hld\ hafa at \alst{ǫ}l-mǫ́lum &
\ind \alst{s}ig-tíva \alst{s}ynir.“\eva

\bvb “Say thou it, O Elder, so that thou not \\
take one step further: \\
what here within for their ale-speeches have \\
the sons of the victory-Tews \ken{gods}?\footnoteB{i.e. ‘what do they speak about over the ale?’}”\evb
\evg


\bvg {\small Ęldir:}
\bva „Of \alst{v}ǫ́pn sín dǿma \hld\ ok of \alst{v}íg-risni sína &
\ind \alst{s}ig-tíva \alst{s}ynir; &
\alst{á}sa ok \alst{a}lfa, \hld\ es hér \alst{i}nni eru, &
\ind \edtext{mann-gi ’s þér í orði vinr.}{\lemma{mann-gi \dots\ vinr ‘none \dots\ words.’}\Bfootnote{i.e. “none of them say anything good about you.” — The (lack of) alliteration here is very notable, and also occurs in st. 10 (between \emph{Víðarr} and \emph{ulfs}, see note there). It could simply be explained by the line being corrupt, but as there are no signs of that we ought to look for other explanations. I see two, namely that (a) the semi-vowel \emph{v} (\textipa{/w/}) is participating in vowel-alliteration with \emph{o}. Such an alliteration between \emph{v} and true vowels is never encountered in Scoldic poetry, but it might have been existed in the simpler Eddic styles; or that (2) the poem (or at least the relevant lines) is of such old age that it was composed before the North Germanic loss of \emph{v} before rounded vowels. This is supported by the fact that in both the present st. and st. 10 the words beginning with vowels (\emph{orð} ‘word’, \emph{ulfr} ‘wolf’) have cognates in other Germanic languages that begin with \emph{w}, and in the case of the word \emph{ulfr} this consonant is also attested in several old Scandinavian runic inscriptions. For metrical reasons the lines must postdate syncope, but on the basis of three clearly related C7th runestones from Blekinge (from Stentoften, Gummarp, and Istaby; DR 357–359) the loss of \emph{w} before rounded vowels is shown also to have occurred after some syncope (so DR 359 \textbf{h\textsc{a}þuwulafʀ} \emph{Haþuwulᵃfʀ}). Of course, even if the alliteration indeed is on \emph{v}, this does not require dating the whole poem to the late Proto-Norse period (indeed, according to the analysis done by \textcite{Sapp2022}, it is not even the linguistically oldest poem preserved); the older forms could simply be an archaism.

A C7th Proto-Norse form of the c-line might be: \emph{*mannagí ’s þéʀ in worðé winiʀ}.}}“\eva

\bvb \small Elder quoth:
“Of their weapons they speak, and of their fight-valiance, \\
the sons of the victory-Tews \ken{gods}; \\
of the Ease and Elves which are here within \\
none is thee a friend in words.”\evb
\evg


\bvg {\small Loki kvað:}
\bva „\alst{I}nn skal ganga \hld\ \alst{Ę́}gis hallir í &
\ind á þat \alst{s}umbl at \alst{s}éa, &
\edtrans{\alst{jǫ}ll ok \alst{ǫ́}fu}{scorn and hatred}{\Bfootnote{\emph{ioll oc áfo} \Regius. These two interesting words have been interpreted in a variety of ways: \CV\ sees the first word as \emph{jóll} ‘wild angelica’, whereas the second is taken to be an error for \emph{áfr} ‘a beverage [...] translated by Magnaeus by \emph{sorbitio avenacea}, a sort of common ale brewed of oats’. TODO: What do other editors say? Esp. Kommentar.}} \hld\ fǿri’k \alst{á}sa sonum &
\ind ok \edtext{blęnd’k þęim svá \alst{m}ęini \alst{m}jǫð}{\lemma{blęnd’k \dots\ męini mjǫð ‘I mix \dots\ the mead with harm’}\Bfootnote{Formulaic, cf. \Sigrdrifumal\ TODO (and others?).}}.“\eva

\bvb Lock quoth:
“In shall I go into Eagre’s halls, \\
for to see that \inx[C]{simble}; \\
scorn and hatred I bring to the sons of the Ease, \\
and I mix for them so the mead with harm.”\evb
\evg


\bvg {\small Ęldir kvað:}
\bva „Vęitst, ef \alst{i}nn gęngr \hld\ \alst{Ę́}gis hallir í &
\ind á þat \alst{s}umbl at \alst{s}éa, &
\alst{h}rópi ok rógi \hld\ ef ęyss á \alst{h}oll ręgin, &
\ind á \alst{þ}ér munu þau \alst{þ}ęrra \alst{þ}at.“\eva

\bvb Elder quoth:
“Know, if in thou goest into Eagre’s halls, \\
for to see that simble: \\
if slander and strife thou pourest onto the \inx[C]{hold} \inx[G]{Reins}, \\
on \emph{thee} will they dry it off.”\evb
\evg


\bvg {\small Loki kvað:}
\bva „Vęitst þat \alst{Ę}ldir, \hld\ ef \alst{ęi}nir skulum &
\ind \alst{s}ár-yrðum \alst{s}akask, &
\alst{au}ðigr verða \hld\ mun’k í \alst{a}nd-svǫrum, &
\ind ef þú \alst{m}ę́lir til \alst{m}art!“\eva

\bvb Lock quoth:
“Know that, O Elder, if alone we [two] shall \\
banter with wounding words: \\
wealthy will I in my answers become, \\
if thou speak too much!\footnoteB{Cf. \Havamal\ TODO mę́la til mart.}”\evb
\evg


\bpg
\bpa Síðan gekk Loki inn í hǫllina; en er þeir sá, er fyrir váru, hverr inn var kominn, þǫgnuðu þeir allir.\epa

\bpb Thereafter Lock went into the hall, but when those who were there before him saw who was come inside, they all turned silent.\epb
\epg


\bvg {\small Loki kvað:}
\bva „\alst{Þ}yrstr ek kom \hld\ \alst{þ}ęssar hallar til &
\ind \alst{L}optr of \alst{l}angan veg, &
\alst{ǫ́}su at biðja, \hld\ \edtext{at mér \alst{ęi}nn gefi &
\ind \alst{m}ę́ran drykk \alst{m}jaðar.}{\lemma{at mér \dots\ mjaðar ‘to me ... of mead’}\Bfootnote{The language describing the mead if formulaic; cf. \Havamal\ 104, 138, \Skirnismal\ 16 (TODO: more refs).}}\eva

\bvb Lock quoth:
“Thirsty to these halls came I, \\
Loft \name{= Lock}, over a long way, \\
to ask the Ease that they give to me \\
one renowned drink of mead.\evb
\evg


\bvg
\bva Hví \alst{þ}ęgið ér svá \hld\ \alst{þ}rungin goð, &
\ind at \alst{m}ę́la né \alst{m}ęguð; &
\alst{s}essa ok staði \hld\ vęlið mér \alst{s}umbli at, &
\ind eða \alst{h}ęitið mik \alst{h}eðan!“\eva

\bvb Why shut ye so up, O pressed Gods, \\
that ye cannot speak? \\
Seats and places choose for me at the simble, \\
or call me hence [away]!\footnoteB{i.e. “Cease your ambiguity; give me a seat or tell me to leave!”}”\evb
\evg


\bvg {\small Bragi:}
\bva „\alst{S}essa ok staði \hld\ vęlja þér \alst{s}umbli at &
\ind \alst{ę́}sir \alst{a}ldri-gi; &
því-at \alst{ę́}sir vitu \hld\ hvęim \alst{a}lda skulu &
\ind \alst{g}amban-sumbl of \alst{g}eta.“\eva

\bvb Bray [quoth]:
“Seats and places choose for thee at the simble \\
never the Ease, \\
for the Ease know for which man they shall \\
prepare the gomben-simble.”\evb
\evg


\bvg {\small [Loki:]}
\bva „Mant þat \alst{Ó}ðinn, \hld\ es vit í \alst{á}r-daga &
\ind \alst{b}lendum \alst{b}lóði saman? &
\alst{ǫ}lvi bęrgja \hld\ létsk \alst{ęi}gi mundu, &
\ind nema okkr vę́ri \alst{b}ǫ́ðum \alst{b}orit.“\eva

\bvb {[Lock quoth:]}
“Recallest thou, Weden, as we two in days of yore \\
blended our blood together? \\
Thou declaredst that thou wouldst not taste ale, \\
unless it were for us both borne forth!”\evb
\evg


\bvg {\small [Óðinn:]}
\bva \edtext{„Rís þú Víðarr \hld\ ok lát ulfs fǫður}{\lemma{Rís \dots\ fǫður ‘Rise \dots\ father’}\Bfootnote{For the (lack of) alliteration see note to st. 2. A C7th Proto-Norse form of the c-line might be: \emph{*Rís þú Wíðarʀ · auk lát wulfs faður}.}} &
\ind \alst{s}itja \alst{s}umbli at, &
síðr oss \alst{L}oki \hld\ kvęði \alst{l}asta-stǫfum &
\ind \alst{Ę́}gis hǫllu \alst{í}.“\eva

\bvb {[Weden quoth:]}
“Rise thou, Wider, and let the Wolf’s father \ken*{= Lock} \\
sit at the simble, \\
lest Lock should greet us with words of vice \\
in Eagre’s hall.”\evb
\evg


\bpg
\bpa Þá stóð Víðarr upp ok skenkti Loka, en áðr hann drykki, kvaddi hann ásuna:\epa

\bpb Then Wider stood up and poured to Lock, but before he \ken*{= Lock} drunk, he greeted the Ease:\epb
\epg


\bvg
\bva „Hęilir \alst{ę́}sir, \hld\ hęilar \alst{ǫ́}synjur &
\ind ok ǫll \alst{g}inn-hęilǫg \alst{g}oð, &
nema sá \alst{ęi}nn \alst{ǫ́}ss \hld\ es \alst{i}nnar sitr &
\ind \alst{B}ragi \alst{b}ękkjum á.“\eva

\bvb “Hail the \inx[G]{Ease}! Hail the \inx[G]{Ossens}, \\
and all \inx[C]{yin-holy} Gods!\footnoteB{The first two half-lines prayer formula are identical to \Sigrdrifumal\ 2–3; it may be of authentic Heathen origin, used in cup-offerings, with the second half of the stanza being used to ask for a boon. Lock subverts it by instead insulting one of the gods present, something that may have been highly offensive to the original audience.} \\
Save for that one \inx[G]{Ease}[os] who sits further within: \\
Bray, on the benches.”\evb
\evg


\bvg {\small [Bragi] kvað:}
\bva „\edtrans{\alst{M}ar ok \alst{m}ę́ki}{Steed and sword}{\Bfootnote{Formulaic, also occuring in \Skirnismal\ TODO.}} \hld\ gef’k þér \alst{m}íns féar &
\ind ok \alst{b}ǿtir þér svá \alst{b}augi \alst{B}ragi, &
síðr þú \alst{ǫ́}sum \hld\ \alst{ǫ}fund of gjaldir— &
\ind \alst{g}ręm þú ęigi \alst{g}oð at þér!“\eva

\bvb {[Bray]} quoth:
“Steed and sword I give thee of my own wealth, \\
and so restores thee Bray with a \inx[C]{bigh}, \\
lest thou shouldst yield envy to the Ease— \\
anger not the Gods against thee!”\evb
\evg


\bvg {\small [Loki] kvað:}
\bva „\alst{Jó}s ok \alst{a}rm-bauga \hld\ munt \alst{ę́} vesa &
\ind \alst{b}ęggja vanr \alst{B}ragi, &
\alst{á}sa ok \alst{a}lfa, \hld\ es hér \alst{i}nni eru, &
\ind þú est við \alst{v}íg \alst{v}arastr, &
\ind ok \alst{sk}jarrastr við \alst{sk}ot.“\eva

\bvb {[Lock]} quoth:
“Of both steed and arm-bighs wilt thou ever \\
O Bray, be lacking! \\
Of the Ease and Elves which are here within, \\
thou art with war wariest \\
and shiest with shot.”\evb
\evg


\bvg {\small {[Bragi]} kvað:}
\bva „Vęit’k, ef fyr \alst{ú}tan vę́ra’k, \hld\ svá sem fyr \alst{i}nnan em’k, &
\ind \alst{Ę́}gis hǫll \alst{o}f kominn, &
\alst{h}ǫfuð þitt \hld\ bę́ra’k í \alst{h}ęndi mér; &
\ind\edtext{\alst{l}ít’k þér þat fyr \alst{l}ygi}{\Bfootnote{\emph{‘litt ec þer þat fyr lygi’} \Regius. A variety of emendations have been proposed for this line. Simplest would be \emph{lítt es þér þat fyr lygi} ‘that is little [punishment] for thee for lying’. Based on the similarity of \emph{ꞇ̇} (= \emph{tt}) and \emph{c} \textcite{FinnurEdda} gives \emph{lykak þér þat fyr lygi} ‘so I would bring to thee for thy lie’.}}.“\eva

\bvb {[Bray]} quoth:
“I know if outside I were, as inside I am \\
come into Eagre’s hall:\footnoteB{As explicitly said in P1, the rule of \inx[C]{grith} (a truce of non-violence, even between enemies; see Encyclopedia) applied inside the hall. Being bound to it, Bray (or the other gods) cannot injure Lock.} \\
thy head I would bear in my hands; \\
this I see for thy lie.”\evb
\evg


\bvg {\small [Loki] kvað:}
\bva „\alst{S}njallr est í \alst{s}essi, \hld\ skal-at-tu \alst{s}vá gęra, &
\ind \alst{B}ragi \alst{b}ękk-skrautuðr; &
\alst{v}ega þú gakk \hld\ ef \alst{v}ręiðr séir; &
\ind \alst{h}yggsk vę́tr \alst{h}vatr fyrir.“\eva

\bvb {[Lock]} quoth:
“Valiant art thou in the seat; [but] thou shalt not do thus, \\
O Bray the bench-ornamenter! \\
Go thou to fight if thou art wroth; \\
the bold thinks not in advance.\footnoteB{Lock attacks Bray’s invoking of the rule of grith; a truly brave man would not care about such a thing.}”\evb
\evg


\bvg {\small [Iðunn] kvað:}
\bva „\alst{B}ið ek, \alst{B}ragi, \hld\ \alst{b}arna sifjar duga &
\ind ok allra \alst{ó}sk-maga, &
at þú \alst{L}oka \hld\ kveðir-a \alst{l}asta-stǫfum &
\ind \alst{Ę́}gis hǫllu \alst{í}.“\eva

\bvb {[Idun]} quoth:
“I bid thee, O Bray, to respect the TODO, \\
and all the TODO, \\
that thou not greet Lock with words of vice \\
in Eagre’s hall.”\evb
\evg


\bvg {\small [Loki] kvað:}
\bva „Þęgi þú, \alst{I}ðunn, \hld\ þik kveð’k \alst{a}llra kvinna &
\ind \alst{v}er-gjarnasta \alst{v}esa &
síðst þú \alst{a}rma þína \hld\ lagðir \alst{í}tr-þvęgna &
\ind umb þinn \alst{b}róður-\alst{b}ana.“\eva

\bvb {[Lock]} quoth:
“Shut up thou, Idun: Thee I declare, of all women, \\
most man-eager to be, \\
since thy nobly washed arms thou cast \\
about thy brother’s bane.”\evb
\evg


\bvg {\small [Iðunn] kvað:}
\bva „\alst{L}oka ek kveð’k-a \hld\ \alst{l}asta-stǫfum &
\ind \alst{Ę́}gis hǫllu \alst{í}; &
\alst{B}raga ek kyrri \hld\ \alst{b}jór-ręifan, &
\ind \alst{v}il’k-at at it \alst{v}ręiðir \alst{v}egisk.“\eva

\bvb {[Idun]} quoth:
“I greet not Lock with words of vice, \\
in Eagre’s hall. \\
Bray I calm, made rowdy from beer— \\
I wish not that ye two wroth ones should fight.”\evb
\evg


\bvg {\small [Gefjun] kvað:}
\bva „Hví it \alst{ę́}sir tvęir \hld\ skuluð \alst{i}nni hér &
\ind \alst{s}ár-yrðum \alst{s}akask? &
\alst{L}ofts-ki þat vęit \hld\ at hann \alst{l}ęikinn es &
\ind ok hann \alst{f}jǫrg-vall \alst{f}réa.”\eva

\bvb {[Giben]} quoth:
“Why shall ye two Ease here within, \\
with wound-words each other blame? \\
Loft \name{= Lock} knows not that he is being played, \\
and him TODO.”\evb
\evg


\bvg {\small [Loki] kvað:}
\bva „\alst{Þ}ęgi þú, Gęfjun, \hld\ \alst{þ}ęss mun’k nú geta &
\ind es þik \alst{g}lapði at \alst{g}ęði: &
\alst{s}vęinn inn hvíti \hld\ es þér \alst{s}igli gaf &
\ind ok þú \alst{l}agðir \alst{l}ę́r yfir.“\eva

\bvb {[Lock]} quoth:
“Shut up thou, Giben: \emph{Him} will I now mention, \\
who seduced thy senses: \\
the white swain who gave thee a necklace, \\
and thou cast o’er [him] thy leg!”\evb
\evg


\bvg {\small [Óðinn kvað] þat:}
\bva „\edtext{\alst{Ǿ}rr est, Loki, \hld\ ok \alst{ø}r-viti}{\lemma{Ǿrr \dots\ ok ør-viti ‘Mad \dots\ and out of wits’}\Bfootnote{Formulaic, occurs at two other places (TODO), and is probably alluded to in st. TODO of the present poem.}} &
\ind es þú fę́r þér \alst{G}ęfjun at \alst{g}ręmi &
því-at \alst{a}ldar \alst{ø}r-lǫg \hld\ hygg at \alst{ǫ}ll of viti &
\ind \alst{ja}fn-gǫrla sem \alst{e}k.“\eva

\bvb {[Weden quoth]} this: \\
“Mad art thou, Lock, and out of wits, \\
as thou earnest Giben’s anger against thee, \\
for all orlays of people I ween that she should know, \\
just as clearly as I.”\evb
\evg


\bvg {\small [Loki] kvað:}
\bva „Þęgi þú, \alst{Ó}ðinn, \hld\ þú kunnir \alst{a}ldri-gi &
\ind dęila \alst{v}íg með \alst{v}erum; &
opt þú \alst{g}aft \hld\ þęim’s \alst{g}efa skyldir-a, &
\ind inum \alst{s}lę́vurum, \alst{s}igr.“\eva

\bvb {[Lock]} quoth:
“Shut up thou, Weden: Thou couldst never \\
deal out war amongst men— \\
oft thou gavest to them thou shouldst not have given, \\
to the slower men victory.”\evb
\evg


\bvg {\small [Óðinn] kvað:}
\bva „Vęitst ef ek \alst{g}af \hld\ þęim’s \alst{g}efa né skylda, &
\ind inum \alst{s}lę́vurum, \alst{s}igr, &
\alst{á}tta vetr \hld\ vast fyr \alst{jǫ}rð neðan &
\ind \alst{k}ýr mólkandi ok \alst{k}ona &
\ind ok hęfir þar \alst{b}ǫrn of \alst{b}orit &
\ind ok hugða’k þat \alst{a}rgs \alst{a}ðal.“\eva

\bvb {[Weden]} quoth:
“Know that if I gave to them I should not have given, \\
to the slower men victory: \\
for eight nights wast thou beneath the earth, \\
milking cows and a woman, \\
and there hast thou borne children, \\
and I’ve judged that a degenerate’s nature.”\evb
\evg


\bvg {\small [Loki] kvað:}
\bva „En þik \alst{s}íga kóðu \hld\ \alst{S}ámsęyju í &
\ind ok drapt á \alst{v}ett sem \alst{v}ǫlur, &
\alst{v}itka líki \hld\ fórt \alst{v}er-þjóð yfir, &
\ind ok hugða’k þat \alst{a}rgs \alst{a}ðal.“\eva

\bvb {[Lock]} quoth:
“But thou, they said, didst sink down into Samsy, \\
and thou beatst the drum like [do] wallows. \\
In the likeness of a sorcerer thou journeyedst among the nations of men, \\
and I’ve judged that a degenerate’s nature.”\evb
\evg


\bvg {\small [Frigg kvað:]}
\bva „\alst{Ø}r-lǫgum \alst{y}kkrum \hld\ skylið \alst{a}ldri-gi &
\ind \alst{s}ęgja \alst{s}ęggjum frá, &
hvat it \alst{ę́}sir tvęir \hld\ drýgðuð í \alst{á}r-daga; &
\ind \alst{f}irrisk ę́ \alst{f}orn rǫk \alst{f}irar.“\eva

\bvb {[Frie quoth:]}
“Of your orlays should ye two never \\
speak to youths, \\
that which ye two Ease did in days of yore— \\
always be ancient rakes shunned by men.”\evb
\evg


\bvg {\small [Loki kvað:]}
\bva „Þęgi þú, \alst{F}rigg, \hld\ þú est \alst{F}jǫrgyns mę́r &
\ind ok hęfir ę́ \alst{v}er-gjǫrn \alst{v}esit, &
es þá \alst{V}éa ok Vilja \hld\ létst þér, \alst{V}iðris kvę́n, &
\ind \alst{b}áða í \alst{b}aðm of tękit.“\eva

\bvb {[Lock quoth:]}
“Shut up thou, Frie: Thou art Firgyn’s maiden, \\
and has always been man-eager: \\
as [when] Wigh and Will, thou hadst, O Withrer’s wife, \\
both in thy bosom taken.”\evb
\evg


\bvg {\small [Frigg kvað:]}
\bva „Vęitst ef \alst{i}nni \alst{ę́}tta’k \hld\ \alst{Ę́}gis hǫllum \alst{í} &
\ind \alst{B}aldri líkan \alst{b}ur &
\alst{ú}t né kvę́mir \hld\ frá \alst{á}sa sonum &
\ind ok vę́ri þá at þér \alst{v}ręiðum \alst{v}egit.“\eva

\bvb {[Frie quoth:]}
“Know, that if within I owned, in Eagre’s halls, \\
a son alike to Balder: \\
out came thou not from the sons of the Ease, \\
and thou wouldst be fought with wrath.”\evb
\evg


\bvg {\small [Loki kvað:]}
\bva „Enn vill þú, \alst{F}rigg, \hld\ at ek \alst{f}lęiri tęlja &
\ind \alst{m}ína \alst{m}ęin-stafi: &
ek því \alst{r}éð \hld\ es þú \alst{r}íða sér-at &
\ind \alst{s}íðan Baldr at \alst{s}ǫlum.“\eva

\bvb {[Lock quoth:]}
“Yet wilt thou, Frie, that I count more \\
of my harmful deeds: \\
I caused it, that thou seest not riding \\
henceforth Balder to the halls.”\evb
\evg


\bvg {\small [Fręyja kvað:]}
\bva „\alst{Ǿ}rr est, Loki, \hld\ es þú \alst{y}ðra tęlr &
\ind \alst{l}jóta \alst{l}ęið-stafi; &
\alst{ø}r-lǫg Frigg \hld\ hygg at \alst{ǫ}ll viti &
\ind þótt hón \alst{s}jǫlf-gi \alst{s}ęgi.“\eva

\bvb {[Frow quoth:]}
“Mad art thou, Lock, as thou countest \\
your ugly loathsome deeds: \\
all orlays I ween that Frie should know, \\
although she says them not herself.”\evb
\evg


\bvg {\small [Loki kvað:]}
\bva „Þęgi þú, \alst{F}ręyja, \hld\ þik kann’k \alst{f}ull-gørva; &
\ind es-a þér \edtrans{\alst{v}amma \alst{v}ant}{free of blemishes}{\Bfootnote{Formulaic, cf. \Havamal\ 22: \emph{hann es-a vamma vanr} ‘he is not free of blemishes’.}}: &
\alst{á}sa ok \alst{a}lfa, \hld\ es hér \alst{i}nni eru, &
\ind \alst{h}vęrr \alst{h}ęfir þinn \alst{h}ór vesit.“\eva

\bvb {[Lock quoth:]}
“Shut up thou, Frow: I know thee full well— \\
thou art not free of blemishes: \\
of the Ease and Elves which are here within \\
has each one been thy lover.”\evb
\evg


\bvg {\small [Fręyja kvað:]}
\bva \edtext{„\alst{F}lǫ́ ’s þér tunga, \hld\ hygg at þér \alst{f}ręmr myni &
\ind ó-\alst{g}ótt of \alst{g}ala;}{\lemma{Flǫ́ ... gala; ‘False ... thee’}\Bfootnote{The language is again strikingly similar to \Havamal, particularly 29/3–4: “A quick-spoken tongue—unless it be held in place—oft sings evil [into being] for itself (\emph{opt sér ó-gótt of gęlr}).” and 116/3–4: “a false-counseling tongue (\emph{flá-rǫ́ð tunga}) brought his life to its end, and in no way over a truthful charge.”}} &
vręiðir ’ru þér \alst{ę́}sir \hld\ ok \alst{ǫ́}synjur, &
\ind \alst{h}ryggr munt \alst{h}ęim fara.“\eva

\bvb {[Frow quoth:]}
“False is thy tongue, I ween that it henceforth will \\
sing evil [into being] for thee. \\
Wroth against thee are the Ease and Ossens: \\
grieved wilt thou journey home.\footnoteB{Frow predicts the future; Lock will regret his insults.}”\evb
\evg


\bvg {\small Loki:}
\bva „Þęgi þú, \alst{F}ręyja, \hld\ þú est \alst{f}or-dę́ða &
\ind ok \alst{m}ęini blandin \alst{m}jǫk, &
síðst-u at \alst{b}rǿðr þínum \hld\ siðu \alst{b}líð ręgin &
\ind ok myndir þá, \alst{F}ręyja, \alst{f}rata.“\eva

\bvb Lock [quoth]:
“Shut up thou, Frow: Thou art an evil-working woman, \\
and much mixed with harm, \\
since against thy brother the blithe Reins soth thee, \\
and wouldst thou then, O Frow, fart.”\evb
\evg


\bvg {\small Njǫrðr:}
\bva „Þat ’s \alst{v}á-lítit \hld\ þótt sér \alst{v}arðir \alst{v}ers fái, &
\ind \alst{h}ós eða \alst{h}várs; &
hitt es \alst{u}ndr \hld\ es \alst{á}ss ragr &
\ind es hér \alst{i}nn of kominn &
\ind ok hęfir sá \alst{b}ǫrn of \alst{b}orit.“\eva

\bvb Nearth [quoth]:
“It is little woe that women should get themselves a man, \\
an adulterer or whomever; \\
this is a wonder, as a degenerate os is come here within, \\
and that one has born children!”\evb
\evg


\bvg {\small Loki:}
\bva „\alst{Þ}ęgi þú, Njǫrðr, \hld\ \alst{þ}ú vast austr heðan &
\ind \alst{g}ísl of sęndr at \alst{g}oðum; &
\alst{H}ymis meyjar \hld\ hǫfðu þik at \alst{h}land-trogi &
\ind ok þér í \alst{m}unn \alst{m}igu.“\eva

\bvb Lock [quoth]:
“Shut up thou, Nearth: Thou wast east hence \\
sent [as] a hostage for the Gods. \\
Hymer’s maidens had thee for a urinal, \\
and pissed thee in the mouth!”\evb
\evg


\bvg {\small Njǫrðr:}
\bva „Sú esumk \alst{l}íkn \hld\ es vas’k \alst{l}angt heðan &
\ind \alst{g}ísl of sęndr at \alst{g}oðum: &
þá ek \alst{m}ǫg gat \hld\ þann’s \alst{m}ann-gi fíar, &
\ind ok þikkir sá \alst{á}sa \alst{ja}ðarr.“\eva

\bvb Nearth [quoth]:
“That is my relief, as I was far-away hence \\
sent [as] a hostage for the Gods:
[that] I then begot that lad whom no man hates \ken*{= Free},
and he seems the peak of the Ease.”\evb
\evg


\bvg {\small Loki:}
\bva „\alst{H}ę́tt-u nú, Njǫrðr, \hld\ haf á \alst{h}ófi þik; &
\ind mun’k-a því \alst{l}ęyna \alst{l}ęngr: &
við \alst{s}ystur þinni \hld\ gatst \alst{s}líkan mǫg, &
\ind ok es-a þó \alst{ó}nu \alst{v}err.“\eva

\bvb Lock [quoth]:
“Stop thou now, Nearth, restrain thyself; \\
I will no longer hide it: \\
by thy sister begotst thou such a lad,
and there can be expected nothing worse.”\evb
\evg


\bvg {\small Týr:}
\bva „Fręyr ’s \alst{b}ętstr \hld\ allra \alst{b}all-riða &
\ind \alst{á}sa gǫrðum \alst{í}; &
\alst{m}ęy né grǿtir \hld\ né \alst{m}anns konu, &
\ind ok lęysir ór \alst{h}ǫptum \alst{h}vęrn.“\eva

\bvb Tew [quoth]:
“Free is the best of all bold riders \\
in the yards of the Ease;  \\
he makes no maiden cry, nor [any] man’s woman,
and loosens each from his bonds!”\evb
\evg


\bvg {\small Loki:}
\bva „\alst{Þ}ęgi þú, Týr, \hld\ \alst{þ}ú kunnir aldri-gi &
\ind \edtrans{bera \alst{t}ilt með \alst{t}vęim}{settle strife among two}{\Bfootnote{Uncertain. See TODO.}}; &
\alst{h}andar ennar \alst{h}ǿgri \hld\ mun’k \alst{h}innar geta &
\ind es þér slęit \alst{F}ęnrir \alst{f}rá.“\eva

\bvb Lock [quoth]:
“Shut up thou, Tew: \emph{Thou} couldst never \\
settle strife among two; \\
the right hand I will next mention, \\
which from thee Fenrer tore.”\evb
\evg


\bvg {\small Týr:}
\bva „\alst{H}andar em’k vanr \hld\ en þú \alst{H}róðrs-vitnis; &
\ind \alst{b}ǫl es \alst{b}ęggja þráa; &
ulf-gi hęfir ok vel \hld\ es í bǫndum skal &
\ind bíða \alst{r}agna \alst{r}økrs.“\eva

\bvb Tew [quoth]:
“A hand am I lacking, but thou Rothwitner; \\
both yearnings are a bale! \\
Nor does the Wolf have it well, who in bonds shall \\
await the Twilight of the Reins.”\evb
\evg


\bvg {\small Loki:}
\bva „\alst{Þ}ęgi þú, Týr, \hld\ \alst{þ}at varð þinni konu &
\ind at hon átti \alst{m}ǫg við \alst{m}ér! &
\edtrans{\alst{Ǫ}ln}{mackerel}{\Bfootnote{Very uncertain. See TODO.}} né pęnning \hld\ hafðir þess \alst{a}ldri-gi &
\ind \alst{v}an-réttis, \alst{v}ę-sall.“\eva

\bvb Lock [quoth]:
“Shut up thou, Tew: \emph{It} happened to thy woman \\
that she had a lad by me! \\
A mackerel nor a penny hadst thou never for that \\
injustice, O wretch!”\evb
\evg


\bvg {\small Fręyr:}
\bva „\alst{U}lf sé’k liggja \hld\ \alst{á}ar-ósi fyr &
\ind unds \alst{r}júfask \alst{r}ęgin; &
því munt \alst{n}ę́st, \hld\ nema \alst{n}ú þęgir, &
\ind \alst{b}undinn, \alst{b}ǫlva smiðr!“\eva

\bvb Free [quoth]:
“The Wolf I see lying before the river-mouth, \\
until the Reins are ripped; \\
therefore wilt thou next—unless thou \emph{now} shut up— \\
be bound, O smith of bales!”\evb
\evg


\bvg {\small Loki:}
\bva „\alst{G}ulli kęypta \hld\ létst \alst{G}ymis dóttur &
\ind ok \alst{s}ęldir þitt \alst{s}vá \alst{s}verð, &
en es \alst{M}úspells synir \hld\ ríða \alst{M}yrk-við yfir &
\ind \alst{v}ęitst-a þá, \alst{v}ę-sall, hvé \alst{v}egr!“\eva

\bvb Lock [quoth]:
“Bought with gold hadst thou Gymer’s daughter \ken*{= Gird}, \\
and didst so sell thy sword—
but when Muspell’s sons ride over Mirkwood \\
knowest thou, not, O wretch, how to fight!”\evb
\evg


\bvg {\small Byggvir:}
\bva „Vęitst ef \alst{ø}ðli \alst{ę́}tta’k \hld\ sem \alst{I}ngunar-Fręyr, &
\ind ok \alst{s}vá \alst{s}ę́l-ligt \alst{s}etr: &
\alst{m}ęrgi smę́ra \hld\ \alst{m}ølða’k þá \alst{m}ęin-krǫ́ku &
\ind ok \alst{l}ęmða alla í \alst{l}iðu.“\eva

\bvb Bewe [quoth]:
“Know, if I owned a pedigree like Ingwin-Free, \\
and such blessed pasture—
smaller than marrow would I mill this harm-crow \ken*{= Lock},
and beat all its limbs lame!”\evb
\evg


\bvg {\small Loki:}
\bva „Hvat ’s þat it \alst{l}itla \hld\ es þat \alst{l}ǫggra sé’k &
\ind ok \alst{s}nap-víst \alst{s}napir? &
At \alst{ęy}rum Fręys \hld\ munt \alst{ę́} vesa &
\ind ok und \alst{k}vęrnum \alst{k}laka.“\eva

\bvb Lock [quoth]:
“What is this little thing which I see crawling, \\
and snap-wisely snapping? \\
At Free’s ears wilt thou always be, \\
and chirping under mills.”\evb
\evg


\bvg {\small [Byggvir kvað:]}
\bva „\alst{B}yggvir ek hęiti, \hld\ en mik \alst{b}ráðan kveða &
\ind \alst{g}oð ǫll ok \alst{g}umar; &
því em’k \alst{h}ér \alst{h}róðugr \hld\ at drekka \alst{H}ropts męgir &
\ind \alst{a}llir \alst{ǫ}l saman.“\eva

\bvb {[Bewe quoth:]}
“Bewe I am called, but hurried do call me \\
all Gods, and men; \\
therefore am I here glorious, as Roft’s lads \ken{ease} drink \\
ale all together.”\evb
\evg


\bvg {\small [Loki kvað:]}
\bva „\alst{Þ}ęgi þú, Byggvir, \hld\ \alst{þ}ú kunnir aldri-gi &
\ind dęila með \alst{m}ǫnnum \alst{m}at; &
ok þik í \alst{f}lęts strá \hld\ \alst{f}inna né mǫ́ttu &
\ind þá’s \alst{v}ǫ́gu \alst{v}erar.“\eva

\bvb {[Lock quoth:]}
“Shut up thou, Bewe: \emph{Thou} couldst never \\
divide food among men, \\
and in the bench-straw could they not find thee, \\
when warriors fought.”\evb
\evg


\bvg {\small [Hęimdallr kvað:]}
\bva „\alst{Ǫ}lr est, Loki \hld\ svá’t es \alst{ø}r-viti, &
\ind hví né \alst{l}ętsk-a þú, \alst{L}oki? &
því-at \alst{o}f-drykkja \hld\ vęldr \alst{a}lda hvęim &
\ind es sína \alst{m}ę́lgi né \alst{m}an-at.“\eva

\bvb {[Homedall quoth:]}
“Drunk art thou, Lock, so that thou art out of wits; \\
why dost thou not hold back, O Lock? \\
For over-drinking causes for every man \\
that he no longer recalls his speech.”\evb
\evg


\bvg {\small [Loki kvað:]}
\bva „\alst{Þ}ęgi þú, Hęimdallr, \hld\ \alst{þ}ér vas í ár-daga &
\ind it \alst{l}jóta \edtrans{\alst{l}íf of \alst{l}agit}{life laid [in place]}{\Bfootnote{Formulaic. See TODO.}}; &
\alst{ǫ}rgu baki \hld\ munt \alst{ę́} vesa &
\ind ok \alst{v}aka \edtrans{\alst{v}ǫrðr goða}{Ward of the Gods}{\Bfootnote{Formulaic epithet of Homedall. See note to \Grimnismal\ 13.}}.“\eva

\bvb {[Lock quoth:]}
“Shut up thou, Homedall: For \emph{thee} was in days of yore \\
the ugly life laid [in place]; \\
with a stiff back wilt thou ever be \\
and waking, [as] the Ward of the Gods.”\evb
\evg


\bvg {\small [X kvað:]}
\bva „\alst{L}étt ’s þér, Loki; \hld\ mun-at-tu \alst{l}ęngi svá &
\ind \alst{l}ęika \alst{l}ausum hala, &
því at þik á \alst{h}jǫrvi skulu \hld\ ins \alst{h}rím-kalda magar &
\ind \alst{g}ǫrnum binda \alst{g}oð.“\eva

\bvb “’Tis light for thee, Lock—thou wilt not so for long \\
play with loose tail: \\
for on a sword shall, with the rime-cold lad’s \\
guts, the Gods bind thee.”\evb
\evg


\bvg {\small [Loki kvað:]}
\bva „Vęitst ef mik á \alst{h}jǫrvi skulu \hld\ ins \alst{h}rím-kalda magar &
\ind \alst{g}ǫrnum binda \alst{g}oð, &
\alst{f}yrstr ok øfstr \hld\ vas’k at \alst{f}jǫr-lagi &
\ind \alst{þ}ar’s vér á \alst{Þ}jatsa \alst{þ}rifum.“\eva

\bvb {[Lock quoth:]}
“Know, if on a sword shall, with the rime-cold lad’s \\
guts, the Gods bind me: \\
first and highest was I in life-taking, \\
where we laid hands on Thedse.”\evb
\evg


\bvg {\small [X kvað:]}
\bva „Veitst ef \alst{f}yrstr ok øfstr \hld\ vast at \alst{f}jǫr-lagi &
\ind \alst{þ}á’s ér á \alst{Þ}jatsa \alst{þ}rifuð, &
frá mínum \alst{v}éum \hld\ ok \alst{v}ǫngum skulu &
\ind þér ę́ \alst{k}ǫld rǫ́ð \alst{k}oma.“\eva

\bvb “Know, if first and highest thou wast in life-taking, \\
when ye laid hands on Thedse: \\
from my wighs and wongs shall \\
for thee always cold counsels come.”\evb
\evg


\bvg {\small [Loki kvað:]}
\bva „\alst{L}éttari í mǫ́lum \hld\ vast við \alst{L}aufęyjar son &
\ind þá’s létsk mér á \alst{b}ęð þinn \alst{b}oðit; &
\alst{g}etit verðr oss slíks \hld\ ef vér \alst{g}ǫrva skulum &
\ind tęlja \alst{v}ǫmmin \alst{v}ǫ́r.“\eva

\bvb {[Lock quoth:]}
“Lighter of speech wast thou with Leafie’s son \ken*{= Lock = me} \\
when thou hadst me invited to thy bed; \\
such is told of us, if we shall clearly \\
tell our blemishes.\evb
\evg


\bpg\bpa Þá gekk Sif fram ok byrlaði Loka í hrím-kálki mjǫð ok mę́lti:\epa

\bpb Then Sib walked forth and poured for Lock mead into a rime-chalice, and spoke:\epb\epg


\bvg
\bva „\alst{H}ęill ves þú nú, Loki, \hld\ ok tak við \alst{h}rím-kálki &
\ind \alst{f}ullum \alst{f}orns mjaðar, &
hęldr þú hana \alst{ęi}na \hld\ látir með \alst{á}sa sonum &
\ind \alst{v}amma-lausa \alst{v}esa.“\eva

\bvb “Hale be thou now, O Lock, and receive the rime-chalice, \\
full of ancient mead, \\
that thou rather let her [me] alone, among the sons of the Ease, \\
remain blemish-less.\footnoteB{Sib attempts to bribe Lock with drink, so that she alone will remain unaccused among the gods.}”\evb
\evg


\bpg\bpa Hann tók við horni ok drakk af:\epa

\bpb He received the horn and drank from it:\epb\epg


\bvg
\bva „\alst{Ęi}n þú vę́rir \hld\ \alst{e}f þú svá vę́rir, &
\ind \alst{v}ǫr ok grǫm at \alst{v}eri; &
ęinn ek \alst{v}ęit, \hld\ svá’t ek \alst{v}ita þikkjumk, &
\ind \alst{h}ór ok af \alst{H}lórriða, &
\ind ok vas þat sá inn \edtrans{\alst{l}ę́-vísi \alst{L}oki}{guile-wise Lock}{\Bfootnote{Formulaic, also occuring in \Hymiskvida\ 37. Cf. also \Voluspa\ 35 where Lock is called \emph{lę́-gjarn} ‘guile-eager’ and note to \Voluspa\ 17 where Lother (possibly to be identified with Lock) gives men \emph{lǫ́}, which may be an accusative form of \emph{lę́}.}}.“\eva

\bvb “Alone were thou, if thou so were \\
wary and wroth against man. \\
I know one—which I think myself to know— \\
adulterer behind even Loride’s back, \\
and that was the guile-wise Lock!”\evb
\evg


\bvg {\small [Bęyla kvað:]}
\bva „\edtext{\alst{F}jǫll ǫll skjalfa, \hld\ hygg á \alst{f}ǫr vesa &
\ind \alst{h}ęiman \alst{H}lórriða;}{\lemma{Fjǫll \dots\ Hlórriða ‘The fells \dots\ to be’}\Bfootnote{Thunder’s movement is often signalled by such disturbance in poetry. See note to \Thrymskvida\ 21.}} &
hann \alst{r}ę́ðr \alst{r}ó \hld\ þeim’s \alst{r}ǿgir hér &
\ind \alst{g}oð ǫll ok \alst{g}uma!“\eva

\bvb {[Beal quoth:]}
“The fells all quake—I think on the journey \\
from home Loride to be; \\
he brings calm to the one who here maligns \\
all Gods and men!”\evb
\evg


\bvg {\small [Loki kvað:]}
\bva „Þęgi þú, \alst{B}ęyla, \hld\ þú est \alst{B}yggvis kvę́n &
\ind ok \alst{m}ęini blandin \alst{m}jǫk; &
\alst{ó}-kynjan męira \hld\ kom-a með \alst{á}sa sonum; &
\ind ǫll est, \alst{d}ęigja, \alst{d}ritin.“\eva

\bvb {[Lock quoth:]}
“Shut up thou, Beal: Thou art Bewe’s wife, \\
and much mixed with harm; \\
a greater disgrace came not among the sons of the Ease; \\
thou art all, O kneadess, shitty!”\evb
\evg


\bpg\bpa Þá kom Þórr at ok kvað:\epa

\bpb Then Thunder arrived and quoth:\epb\epg


\bvg
\bva „\alst{Þ}ęgi þú, rǫg vę́ttr, \hld\ \alst{þ}ér skal minn \alst{þ}rúð-hamarr, &
\ind \alst{M}jǫllnir, \alst{m}ál fyr-nema! &
\alst{H}ęrða klett \hld\ drep’k þér \alst{h}alsi af, &
\ind ok verðr þá þínu \alst{f}jǫrvi of \alst{f}arit.“\eva

\bvb “Shut up thou, degenerate wight: Thee shall my thrith-hammer \\
Millner, deprive of speech! \\
The shoulder-rock \ken{head} I strike off thy neck, \\
and then is thy life destroyed!”\evb
\evg


\bvg {\small [Loki kvað:]}
\bva „\alst{Ja}rðar burr \hld\ es hér nú \alst{i}nn kominn; &
\ind hví \alst{þ}rasir þú svá, \alst{Þ}órr? &
En þá þorir \alst{ę}kki \hld\ es skalt við \alst{u}lfinn vega &
\ind ok \alst{s}velgr hann allan \alst{S}ig-fǫður.“\eva

\bvb {[Lock quoth:]}
“The son of Earth is now here come inside, \\
why thrashest thou so, O Thunder? \\
But then darest thou not, as thou shalt fight against the wolf, \\
and he swallows Syefather \name{= Weden} whole.”\evb
\evg


\bvg {\small [Þórr kvað:]}
\bva „\alst{Þ}ęgi þú, rǫg vę́ttr, \hld\ \alst{þ}ér skal minn \alst{þ}rúð-hamarr, &
\ind \alst{M}jǫllnir, \alst{m}ál fyr-nema! &
\alst{U}pp ek þér verp \hld\ ok á \alst{au}str-vega &
\ind \alst{s}íðan þik mann-gi \alst{s}ér.“\eva

\bvb {[Thunder quoth:]}
“Shut up thou, degenerate wight: Thee shall my thrith-hammer \\
Millner, deprive of speech! \\
Up I throw thee, and onto the eastern ways \\
thereafter no man sees thee!”\evb
\evg


\bvg {\small [Loki kvað:]}
\bva „\alst{Au}str-fǫrum þínum \hld\ skalt \alst{a}ldri-gi &
\ind \alst{s}ęgja \alst{s}ęggjum frá &
síðst í \alst{h}anska þumlungi \hld\ \alst{h}núkðir þú, ęin-hęri, &
\ind \edtrans{ok \alst{þ}óttisk-a \alst{þ}á \alst{Þ}órr vesa}{didst not seem to be Thunder then}{\Bfootnote{Cf. \Harbardsljod\ TODO.}}!“\eva

\bvb {[Lock quoth:]}
“Of thy eastern journeys shalt thou never \\
speak to youths, \\
since in the thumb of a glove thou didst crawl, O Ownharrier, \\
and didst not seem to be Thunder then!”\evb
\evg


\bvg {\small [Þórr kvað:]}
\bva „\alst{Þ}ęgi þú, rǫg vę́ttr, \hld\ \alst{þ}ér skal minn \alst{þ}rúð-hamarr, &
\ind \alst{M}jǫllnir, \alst{m}ál fyr-nema! &
\alst{h}ęndi inni \alst{h}ǿgri \hld\ drep’k þik \alst{H}rungnis bana, &
\ind svá’t þér \alst{b}rotnar \alst{b}eina hvat.“\eva

\bvb {[Thunder quoth:]}
“Shut up thou, degenerate wight: Thee shall my thrith-hammer \\
Millner, deprive of speech! \\
With the right hand I strike thee with Rungner’s bane, \\
so that every bone in thee breaks.”\evb
\evg


\bvg {\small [Loki kvað:]}
\bva „\alst{L}ifa ę́tla’k mér \hld\ \alst{l}angan aldr &
\ind þótt \alst{h}ǿtir \alst{h}amri mér; &
\alst{sk}arpar álar \hld\ þóttu þér \alst{Sk}rymis vesa &
\ind ok máttir-a þá \alst{n}ęsti \alst{n}áa &
\ind ok svaltsk þá \alst{h}ungri \alst{h}ęill.“\eva

\bvb {[Lock quoth:]}
“For myself I intend to live a long life, \\
even though thou threatenest me with the hammer; \\
TODO.”\evb
\evg


\bvg {\small [Þórr kvað:]}
\bva „\alst{Þ}ęgi þú, rǫg vę́ttr, \hld\ \alst{þ}ér skal minn \alst{þ}rúð-hamarr, &
\ind \alst{M}jǫllnir, \alst{m}ál fyr-nema! &
\alst{H}rungnis bani \hld\ mun þér í \alst{h}ęl koma &
\ind fyr \alst{N}á-grindr \alst{n}eðan.“\eva

\bvb {[Thunder quoth:]}
“Shut up thou, degenerate wight: Thee shall my thrith-hammer \\
Millner, deprive of speech! \\
Rungner’s bane will take thee to hell, \\
down beneath Neegrind!”\evb
\evg


\bvg {\small [Loki kvað:]}
\bva „Kvað’k fyr \alst{ǫ́}sum, \hld\ kvað’k fyr \alst{á}sa sonum, &
\ind þat’s mik \alst{h}vatti \alst{h}ugr, &
en fyr þér \alst{ęi}num \hld\ mun’k \alst{ú}t ganga &
\ind því-at ek \alst{v}ęit at þú \alst{v}egr.\eva

\bvb {[Lock quoth:]}
“I spoke before the Ease, I spoke before the sons of the Ease \\
whatever my mind did goad me. \\
but for thee alone will I go out, \\
for I know that thou strikest.\evb
\evg


\bvg
\bva \alst{Ǫ}l gørðir þú, \alst{Ę́}gir, \hld\ en þú \alst{a}ldri munt &
\ind \alst{s}íðan \alst{s}umbl of gøra; &
\alst{ęi}ga þín \alst{ǫ}ll, \hld\ es hér \alst{i}nni es, &
\ind \alst{l}ęiki yfir \alst{l}ogi &
\ind ok \alst{b}renni þér á \alst{b}aki.“\eva

\bvb Ale madest thou, Eagre, but thou wilt never \\
since make a simble; \\
all thy ownings which are here within, \\
over [them] may flame play, \\
and burn thee on the back!”\evb
\evg

\sectionline

\section{From Lock (\emph{Frá Loka})}

The myth told here is known from two other places. Closest at hand is \Voluspa\

\Gylfaginning\ 50 has a longer but somewhat different account: the Ease captured Lock’s two sons, Wonnel and “Nare or Narve”. They turned Wonnel into a wolf (\emph{vargr}, which also means ‘outlaw’) and had him tear his brother Narve apart. Narve’s intestines were then taken and used to bind Lock on top of three pointed stones, with one digging into his shoulder-blades, the other digging into his loins, and the third digging into his houghs. The intestines then turned into iron.

Since the author of \Gylfaginning\ knew \Voluspa, it is possible that he combined a text similar to \FraLoka\ with this st., interpreting \emph{Vála víg-bǫnd} as ‘Wonnel’s war-bonds’ and \emph{vargr} as ‘wolf’ rather than the more probable ‘outlaw’. Wonnel is otherwise only known as the son of Weden, and there is no reason why he could not also bound Lock. For further differences between \Gylfaginning\ and \FraLoka\ see introduction to \FraLoka\ below

\sectionline

\bpg\bpa En eptir þetta falst Loki í Fránangrs-forsi í lax líki. Þar tóku ę́sir hann. Hann var bundinn með þǫrmum sonar Nara; en Narfi, sonr hans, varð at vargi. Skaði tók eitr-orm ok festi upp yfir and-lit Loka; draup þar ór eitr. Sigyn, kona Loka, sat þar ok helt munn-laug undir eitrit. En er munn-laugin var full bar hon út eitrit, en meðan draup eitrit á Loka. Þá kipptist hann svá hart við, at þaðan af skalf jǫrð ǫll; þat eru nú kallaðir land-skjálftar.\epa

\bpb But after this Lock hid himself in the Freenangersforce in the form of a salmon. There the Ease took him. He was bound with the intestines of his son Nare, but his son Narve became an outlaw. Shede took a venomous serpent and fastened it over Lock’s face; out of it dripped venom. Syein, Lock’s wife, sat there and held a basin [for hand-washing] under the venom. But when the basin was full she bore out the venom, and meanwhile the venom dripped on Lock. Then he revolted so hard that thence all the earth quaked; that is now called earth-quakes.\epb\epg
% Lock, All Gods
	\bookStart{Lay of Thrim}[Þrymskviða]

\begin{flushright}%
\textbf{Dating} \parencite{Sapp2022}: C9th (0.741)

\textbf{Meter:} \Fornyrdislag%
\end{flushright}

Compare \Haustlong, \Hymiskvida, other poems and refer to the SkP intro to one of the big Thunder poems. TODO.

\sectionline

\bvg\bva \edtrans{\alst{V}ręiðr}{Wroth}{\Bfootnote{The \emph{vr-} is restored for the sake of the alliteration, but is not strictly metrically neccessary; cf. st 13.  The manuscript has \emph{r-}.  In any case the poem (generally considered to be the oldest Eddic poem) most likely predates the change \emph{vr-} > \emph{r-}.}} vas þá \alst{V}ing-Þórr \hld\ es hann \alst{v}aknaði &
ok \alst{s}íns hamars \hld\ of \alst{s}aknaði, &
\edtext{\alst{sk}ęgg nam at hrista, \hld\ \alst{sk}ǫr nam at dýja}{\lemma{skęgg \dots\ dýja ‘beard \dots\ pull’}\Bfootnote{Apparently formulaic. Cf. a certain heroic poem (TODO).}}, &
réð \alst{Ja}rðar burr \hld\ \alst{u}mb at þręifask.\eva

\bvb Wroth was then Wing-Thunder when he woke, \\
and of his hammer was bereaved. \\
His beard he took to rustle, his locks he took to rip; \\
the son of Earth resolved to grope about.\evb\evg


\bvg\bva \edtext{\alst{O}k hann þat \alst{o}rða \hld\ \alst{a}lls fyrst of kvað:}{\lemma{Ok \dots\ of kvað ‘And ... did say’}\Bfootnote{The whole line is formulaic, occuring in five other places: sts. 3, 9 and 12 of the present poem; st 3 of \Oddrunargratr; st. 5 of \Brot.}} &
„\alst{H}ęyr-ðu nú, Loki, \hld\ \alst{h}vat ek nú mę́li &
es \alst{ęi}gi vęit \hld\ \edtext{\alst{ja}rðar hvęr-gi &
né \alst{u}pp-himins}{\lemma{jarðar \dots\ upp-himins ‘earth \dots\ up-heaven’}\Bfootnote{Formulaic, see Encyclopedia: \inx[F]{Earth and Up-heaven}.}}: \hld\ \alst{á}ss es stolinn hamri!“\eva

\bvb And he this word first of all did say: \\
“Hear thou now, Lock, what I now speak, \\
which no man knows anywhere on earth \\
nor in up-heaven: the \inx[G]{Eese}[os] \ken*{= Thunder = I} is robbed of His hammer!”\evb\evg


\bvg\bva Gingu þęir \alst{f}agra \hld\ \alst{F}ręyju túna &
\alst{o}k hann þat \alst{o}rða \hld\ \alst{a}lls fyrst of kvað: &
„Munt-u mér, \alst{F}ręyja, \hld\ \edtrans{\alst{f}jaðr-hams}{feather-hame}{\Bfootnote{A “feather-skin” by which the wearer can transform or fly like a bird.}} léa &
ef ek \alst{m}ínn hamar \hld\ \alst{m}ę́tta’k hitta?“\eva

\bvb Went they to the fair yards of \inx[P]{Frow}, \\
and he this word first of all did say: \\
“Wilt thou me, O Frow, the \inx[C]{feather-hame} lend, \\
if I my hammer might find?”\evb\evg


\bvg\bva\speakernote{Fręyja kvað:}%
„Þó mynda’k \alst{g}efa þér \hld\ þótt ór \alst{g}ulli vę́ri &
ok þó \edtrans{\alst{s}ęlja}{hand}{\Bfootnote{\emph{sęlja}, cognate of English \emph{sell}, here has its older sense of ‘hand over’, cf. Gotish \emph{saljan} \textcite[116]{Streitberg}: ‘\emph{opfern}; \textgreek{θύειν}’.}} \hld\ at vę́ri ór \alst{s}ilfri.“\eva

\bvb\speakernoteb{[Frow quoth:]}%
“Yet would I give it to thee though it were golden, \\
and yet hand it to thee if it were silvern.”\evb\evg


\bvg\bva \alst{F}ló þá \edtrans{Loki}{Lock}{\Bfootnote{Though Thunder is the one asking for the feather-hame (“if I \emph{my} hammer might find”), Lock is the one that takes off flying with it.}}, \hld\ \alst{f}jaðr-hamr dunði, &
unds fyr \alst{ú}tan kom \hld\ \alst{ȧ}sa garða &
ok fyr \alst{i}nnan kom \hld\ \alst{jǫ}tna hęima.\eva

\bvb Flew then Lock—the feather-hame rustled— \\
until he came outside the \inx[L]{Osyard}[Yards of the Eese], \\
and he came inside the \inx[L]{Ettinham}[Homes of the Ettins].\evb\evg


\bvg\bva \alst{Þ}rymr \edtrans{sat á haugi}{sat on the mound}{\Bfootnote{Apparently a typical seat for ettins.  See \Voluspa\ 42 for other attestations.}}, \hld\ \edtrans{\alst{þ}ursa dróttinn}{lord of Thurses}{\Bfootnote{This formulaic expression also occurs in several Runic charms against such thursen lords (see below under Galders); an example of the close connection between mythology and ritual.}}, &
\edtext{\alst{g}ręyjum sínum \hld\ \alst{g}ull-bǫnd snøri &
ok \alst{m}ǫrum sínum}{\lemma{gręyjum sínum \dots\ mǫrum sínum ‘his greyhounds \dots\ his steeds’}\Bfootnote{Thrim sits surrounded by dogs and horses.  The scene is reminiscent of the ancient “master of animals” motif, especially as attested on panel A of the Gundestrup cauldron.}} \hld\ \alst{m}ǫn jafnaði.\eva

\bvb Thrim sat on the mound, the lord of \inx[G]{Thurses}: \\
on his greyhounds the golden leashes he twirled, \\
and on his steeds the manes he evened.\evb\evg


\bvg\bva\speakernote{[Þrymr kvað:]}%
„\edtrans{Hvat ’s með \alst{ǫ̇}sum? \hld\ Hvat ’s með \alst{ǫ}lfum?}{What is with the Eese? What is with the Elves?}{\Bfootnote{Formulaic, the same line occurs in \Voluspa\ 47.}} &
Hví est \alst{ęi}nn kominn \hld\ í \alst{jǫ}tun-hęima?“ &
\speakernote{[Loki kvað:]}%
„\alst{I}llt ’s með \alst{ǫ̇}sum, \hld\ \edtext{\alst{i}llt ’s með \alst{ǫ}lfum}{\Afootnote{Required by the meter; om. \Regius}}! &
\alst{H}ęfir þú \alst{H}lórriða \hld\ \alst{h}amar of folginn?“\eva

\bvb\speakernoteb{[Thrim quoth:]}%
“What’s with the Eese? What’s with the Elves? \\
Why art thou alone come into the \inx[L]{Ettinham}[Ettin-homes]?”— \\
\speakernoteb{[Lock quoth:]}%
“’Tis ill with the Eese! ’Tis ill with the Elves! \\
Hast thou the hammer of Loride \name{= Thunder} hid?”\evb\evg


\bvg\bva\speakernote{[Þrymr kvað:]}„Ek \alst{h}ęfi \alst{H}lórriða \hld\ \alst{h}amar of folginn &
\alst{á}tta rǫstum \hld\ fyr \alst{jǫ}rð neðan; &
hann \alst{ę}ngi maðr \hld\ \alst{a}ptr of hęimtir &
nema \alst{f}ǿri mér \hld\ \alst{F}ręyju at kvę́n.“\eva

\bvb\speakernoteb{[Thrim quoth:]}
“I have the hammer of Loride hid \\
eight \inx[C]{rest}[rests] beneath the earth! \\
It no man will fetch back, \\
unless he bring me Frow for a wife.”\evb\evg


\bvg\bva \alst{F}ló þá Loki, \hld\ \alst{f}jaðr-hamr dunði, &
unds fyr \alst{ú}tan kom \hld\ \alst{jǫ}tna hęima &
ok fyr \alst{i}nnan kom \hld\ \alst{á}sa garða; &
\alst{m}ǿtti hann Þór \hld\ \alst{m}iðra garða &
\alst{o}k \edtext{hann þat}{\Afootnote{emend.; \emph{þat hann} \Regius, with elsewhere unprecedented word order. Cf. note to st. 2.}} \alst{o}rða \hld\ \alst{a}lls fyrst of kvað:\eva

\bvb Flew then Lock—the feather-hame rustled— \\
until he came outside the Yards of the Eese, \\
and he came inside the Homes of the Ettins. \\
Met he Thunder in the middle yards, \\
and he \ken*{= Thunder} that word first of all did say:\evb\evg


\bvg\bva „\edtrans{Hęfir þú \alst{ø}rendi \hld\ sem \alst{ę}rfiði?}{Hast thou an errand of hardship?}{\Bfootnote{Thunder asks Lock if he has bad news.  The collocation \emph{ørendi} ‘errand’ \dots\ \emph{ęrfiði} ‘trouble, hardship’ is formulaic and occurs in X other (TODO!!) places, including in st. 5 of \HelgakvidaHjorvardssonar.}} &
Seg-ðu á \alst{l}opti \hld\ \alst{l}ǫng tíðendi! &
Opt \alst{s}itjanda \hld\ \alst{s}ǫgur of fallask, &
ok \alst{l}iggjandi \hld\ \alst{l}ygi of bęllir.“\eva

\bvb “Hast thou an errand of hardship? \\
Tell thou aloft the long tidings! \\
Oft the sitter’s tales fail each other \\
and the lier blows up his lie.”\footnoteB{Proverbial. If one sits or lies (\emph{liggjandi} means to ‘lie down’; it is rather unfortunate that the two sound the same in English) down and thinks too much over bad news, details will be left out, excuses thought up. Thus it is best that Lock immediately tell Thunder what he has learned.}\evb\evg


\bvg\bva\speakernote{[Loki kvað:]}„Hefi’k \alst{ø}rendi, \hld\ \alst{ę}rfiði ok: &
\alst{Þ}rymr hęfir þinn hamar, \hld\ \alst{þ}ursa dróttinn; &
hann \alst{ę}ngi maðr \hld\ \alst{a}ptr of hęimtir &
nema hǫ́num \alst{f}ǿri \hld\ \alst{F}ręyju at kvę́n.“\eva

\bvb\speakernoteb{[Lock quoth:]}
“I have an errand, hardship also: \\
Thrim has thy hammer, the lord of Thurses. \\
It no man will fetch back, \\
unless he bring him Frow for a wife.”\evb\evg


\bvg\bva Ganga þęir \alst{f}agra \hld\ \alst{F}ręyju at hitta &
\alst{o}k \edtrans{hann}{he}{\Bfootnote{The speaker is either Thunder or Lock.}} þat \alst{o}rða \hld\ \alst{a}lls fyrst of kvað: &
„\alst{B}itt-u þik, Fręyja, \hld\ \edtrans{\alst{b}rúðar líni!}{bride’s linen}{\Bfootnote{i.e. bridal cloth.}} &
Vit skulum \alst{a}ka tvau \hld\ í \alst{jǫ}tun-hęima.“\eva

\bvb Go they the fair Frow to find, \\
and he this word first of all did say: \\
“Bind thyself, Frow, with a bride’s linen! \\
We two shall drive into the Ettin-homes.”\evb\evg


\bvg\bva Vręið varð þá \alst{F}ręyja \hld\ ok \alst{f}nasaði, &
\alst{a}llr \alst{ȧ}sa salr \hld\ \alst{u}ndir bifðisk, &
stǫkk þat it \alst{m}ikla \hld\ \edtrans{\alst{m}ęn Brísinga}{torc of the Brisings}{\Bfootnote{A legendary jewel owned by Frow.}}: &
„Mik \alst{v}ęitst \edtrans{\alst{v}erða \hld\ \alst{v}er-gjarnasta}{become the most man-eager}{\Bfootnote{Presumably Frow is speaking out of self-awareness of her own lustful inclinations, i.e., she will be gripped by uncontrollable lust.  It is also possible that she complains about being accused of promiscuity by the other gods, but that is not the literal sense.  For Frow’s promiscuity cf. \Lokasenna\ 30, and also st. 26 of that poem where Frie is likewise called \emph{ver-gjǫrn} ‘man-eager’.}} &
ef ek \alst{ę}k með þér \hld\ í \alst{jǫ}tun-hęima.“\eva

\bvb Wroth became then Frow, and snorted; \\
the whole hall of the Eese quivered below; \\
down crashed the great \inx[P]{torc of the Brisings}— \\
“Thou knowest that I will become the most man-eager, \\
if I drive with thee into the Ettin-homes.”\evb\evg


\bvg\bva \edtext{Sęnn vǫ́ru \alst{ę̇}sir \hld\ \alst{a}llir á þingi &
ok \alst{ǫ̇}synjur \hld\ \alst{a}llar á máli, &
ok umb þat \alst{r}éðu \hld\ \alst{r}íkir tívar:}{\lemma{Sęnn \dots\ tívar ‘Soon \dots\ Tews’}\Bfootnote{The exact same three lines also occur \Baldrsdraumar\ 1/1–3; see Note there.}} &
\alst{h}vé þęir \alst{H}lórriða \hld\ \alst{h}amar of sǿtti?\eva

\bvb Soon were the \inx[G]{Eese} all at the \inx[C]{Thing}, \\
and the \inx[G]{Ossens} all at speech, \\
and of this counseled the mighty \inx[G]{Tews}: \\
How they Loride’s \name{= Thunder’s} hammer would get?\evb\evg


\bvg\bva Þá kvað þat \alst{H}ęimdallr, \hld\ \alst{h}vítastr ása, &
\edtrans{\alst{v}issi \alst{v}ęl framm}{he foreknew well}{\Bfootnote{i.e. saw the future.  Compare the derived adjective \emph{fram-víss} ’forth-wise, prescient.’}} \hld\ sęm \alst{v}anir aðrir: &
„\alst{B}indu vér Þór þá \hld\ \alst{b}rúðar líni; &
hafi hann it \alst{m}ikla \hld\ \alst{m}ęn Brísinga!\eva

\bvb Then quoth this \inx[P]{Homedal}, whitest of the Eese; \\
he foreknew well like the other \inx[G]{Wanes}: \\
“Let us bind Thunder then, with a bride’s linen; \\
he may have the great torc of the Brisings.\evb\evg


\bvg\bva Lǫ́tum und \alst{h}ǫ́num \hld\ \alst{h}rynja lukla &
ok \alst{k}ven-váðir \hld\ umb \alst{k}né falla &
en á \alst{b}rjósti \hld\ \alst{b}ręiða stęina &
ok \alst{h}ag-liga \hld\ umb \alst{h}ǫfuð typpum!“\eva

\bvb Let us set by his side keys to jingle, \\
and women’s garments to fall about the knees, \\
but on the breast broad stones, \\
and skillfully let us tip his head.\footnoteB{An interesting description of Wiking age bridal dress.  As the everyday manager of the household, keys were the mark of a respectable married woman.  The “broad stones” on the breast may be tortoise brooches (also mentioned in \Volundarkvida\ 25, 36.) or beads.  The tipping of the head refers to some sort of bridal hat, perhaps a veil (TODO: Literature).}”\evb\evg


\bvg\bva Þá kvað þat \alst{Þ}órr, \hld\ \alst{þ}rúðugr áss: &
„Mik munu \alst{ę́}sir \hld\ \alst{a}rgan kalla &
ef ek \alst{b}indask lę́t \hld\ \alst{b}rúðar líni!“\eva

\bvb Then quoth this Thunder, the mighty Os: \\
“Me will the Eese call \inx[C]{queer}, \\
if I let myself be bound with a bride’s linen!”\evb\evg


\bvg\bva Þá kvað þat \alst{L}oki \hld\ \alst{L}aufęyjar sonr: &
„\alst{Þ}ęgi þú, \alst{Þ}órr, \hld\ \alst{þ}ęira orða! &
\edtext{Þegar munu \alst{jǫ}tnar \hld\ \alst{Ǫ̇}s-garð búa &
nema \alst{þ}ú \alst{þ}inn hamar \hld\ \alst{þ}ér of hęimtir.}{\lemma{Þegar \dots\ hęimtir. ‘Shortly \dots\ dost fetch!’}\Bfootnote{Guarding Osyard from transgressive and destructive forces was Thunder’s task, and the hammer his most important tool.  Cf. \Harbardsljod\ TODO, and a couplet by the obscure poet Thurbern Disescold, cited in \Skaldskaparmal\ 11: \emph{Þȯrr hęfr Yggs með ǫ́rum \hld\ Ǫ̇sgarð af þrek varðan.} ‘Thunder has with the messengers of Ug \ken{gods} mightily guarded Osyard.’}}“\eva

\bvb Then quoth this Lock, Leafie’s son: \\
“Shut up thou, Thunder, with those words! \\
Shortly the Ettins will settle Osyard, \\
unless thou thy hammer for thyself dost fetch!”\evb\evg


\bvg\bva \alst{B}undu þęir Þór þá \hld\ \alst{b}rúðar líni &
ok hinu \alst{m}ikla \hld\ \alst{m}ęni Brísinga, &
létu und \alst{h}ǫ́num \hld\ \alst{h}rynja lukla &
ok \alst{k}ven-váðir \hld\ umb \alst{k}né falla &
en á \alst{b}rjósti \hld\ \alst{b}ręiða stęina &
ok \alst{h}ag-liga \hld\ of \alst{h}ǫfuð typpðu.\eva

\bvb Bound they Thunder then with a bride’s linen, \\
and with the great torc of the Brisings. \\
They set by his side keys to jingle, \\
and women’s garments to fall about the knees, \\
but on the breast broad stones, \\
and skillfully they tipped his head.\evb\evg


\bvg\bva Þá kvað þat \alst{L}oki \hld\ \alst{L}aufęyjar sonr: &
„Mun’k \alst{au}k með þér \hld\ \alst{a}mbǫ́tt vesa, &
vit skulum \alst{a}ka tvau \hld\ í \alst{jǫ}tun-hęima.“\eva

\bvb Then quoth this Lock, Leafie’s son: \\
“I will also with thee be a handmaid; \\
we two\footnoteB{The form used, \emph{tvau}, is the neuter plural, i.e. one of the pair is female and the other male. This is either an error due to mindless copying of v. 11, or a backhanded insult against Thunder.} shall drive into the Ettin-homes.”\evb\evg


\bvg\bva Sęnn vǫ́ru \edtrans{\alst{h}afrar}{he-goats}{\Bfootnote{Thunder’s chariot was driven by two he-goats, whence he is called “the Lord of He-goats” (e.g. in \Hymiskvida\ 20, 31).}} \hld\ \alst{h}ęim of vreknir, &
\alst{sk}yndir at \alst{sk}ǫklum, \hld\ \alst{sk}yldu vęl renna; &
\alst{b}jǫrg \alst{b}rotnuðu, \hld\ \alst{b}rann jǫrð loga; &
\alst{ó}k \alst{Ó}ðins sonr \hld\ í \alst{jǫ}tun-hęima.\eva

\bvb Soon were the \inx[C]{he-goats} driven home, \\
hastened onto the cart-poles; they were to run well. \\
Crags burst, earth burned with flame; \\
drove Weden’s son \ken*{= Thunder} into the Ettin-homes.\footnoteB{Thunder’s driving is often connected with cosmic disturbance.  So, his arrival in \Lokasenna\ 55 is signalled by the mountains quaking.  The description most similar to the present stanza is found in Thedwolf’s \Haustlong\ 14–16, where crags (\emph{bjǫrg}) burst asunder and fires rage before him.  A possibly Indo-European parallel to this is the Vedic myth of Indra breaking the mountains and releasing the rivers (as described most famously in \Rigveda\ 1.32).  Cf. \Baldrsdraumar\ 3 where the ground rumbles beneath the riding Weden.}\evb\evg


\bvg\bva Þá kvað þat \alst{Þ}rymr, \hld\ \alst{þ}ursa dróttinn: &
„\alst{St}andið upp, jǫtnar, \hld\ ok \alst{st}ráið bękki! &
Nú \alst{f}ǿrið mér \hld\ \alst{F}ręyju at kván, &
\alst{N}jarðar dóttur \hld\ ór \alst{N}óa-túnum.\eva

\bvb Then quoth this Thrim, the lord of Thurses: \\
“Stand up, O ettins, and strew the benches! \\
Now bring me Frow for a wife, \\
\inx[P]{Nearth}’s daughter from the \inx[L]{Nowetowns}.\evb\evg


\bvg\bva \alst{G}anga hér at \alst{g}arði \hld\ \alst{g}ull-hyrnðar kýr, &
\edtrans{\alst{ø}xn \alst{a}l-svartir}{all-black oxen}{\Bfootnote{Formulaic, also occurring in \Hymiskvida\ 18.  That all-black (i.e. spotlessly black) oxen were most valued is seen by the pairing with “golden-horned”.  One may also compare \textcite{Saxo}[1.8.12], where the hero Hadding has to atone for his slaying of a heavenly being by a sacrifice of dark-coloured victims (\emph{furvae hostiae}): \emph{Siquidem propiciandorum numinum gratia Frø deo rem diuinam furuis hostiis fecit. Quem litationis morem annuo feriarum circuitu repetitum posteris imitandum reliquit. Frøblod Sueones uocant.} ‘In order to mollify the divinities he [= Hadding] did indeed make a holy sacrifice of dark-coloured victims to the god Frø.  He repeated this mode of propitiation at an annual festival and left it to be imitated by his descendants.  The Swedes call it Frøblot.’  This ancient ritual taboo finds parallel even in the Tanakh, where animals dedicated to YHWH were to be without blemish (\textgreek{תָּמִ֖ים}, Leviticus 1:3)}}, \hld\ \alst{jǫ}tni at gamni, & %TODO: Hebrew.
fjǫlð á’k \alst{m}ęiðma, \hld\ fjǫlð á’k \alst{m}ęnja; &
\alst{ęi}nnar mér Fręyju \hld\ \alst{á}-vant þykkir.“\eva

\bvb Here march to the farm golden-horned kine, \\
all-black oxen to the ettin’s [my] pleasure. \\
A multitude I own of treasures, a multitude I own of torcs— \\
only Frow I think myself missing.”\evb\evg


\bvg\bva Vas þar at \alst{k}veldi \hld\ of \alst{k}omit snimma &
\alst{o}k fyr \alst{jǫ}tna \hld\ \alst{ǫ}l framm borit. &
\alst{Ęi}nn át \alst{o}xa, \hld\ \alst{á}tta laxa, &
\alst{k}rásir allar, \hld\ þę́r’s \alst{k}onur skyldu, &
drakk \alst{S}ifjar verr \hld\ \alst{s}áld þrjú mjaðar.\eva

\bvb There was the evening come early, \\
and for the ettins ale brought forth. \\
Alone ate he \ken*{= Thunder} an ox, eight salmons, \\
all the dainties meant for the women; \\
drank Sib’s husband \ken*{= Thunder} three sieves of mead.\footnoteB{Cf. \Hymiskvida\ 15, where Thunder eats two of Hymer’s oxen.  It is rather interesting that the same kenning is used in both stanzas relating the god’s great eating; perhaps one poet was playing on the other’s expression, or they were both referencing another, now-lost work.}\evb\evg


\bvg\bva Þá kvað þat \alst{Þ}rymr, \hld\ \alst{þ}ursa dróttinn: &
„Hvar sátt-u \alst{b}rúðir \hld\ \alst{b}íta hvassara? &
Sá’k-a \alst{b}rúðir \hld\ \alst{b}íta ęnn \alst{b}ręiðara &
né ęnn \alst{m}ęira \alst{m}jǫð \hld\ \alst{m}ęy of drekka!“\eva

\bvb Then quoth this Thrim, the lord of Thurses: \\
“Where didst thou see brides bite sharper? \\
I never saw brides bite yet broader; \\
nor yet more mead a maiden drink!”\evb\evg


\bvg\bva Sat hin \alst{a}l-snotra \hld\ \alst{a}mbǫ́tt fyrir &
es \alst{o}rð of fann \hld\ við \alst{jǫ}tuns máli: &
„\alst{Á}t vę́tr Fręyja \hld\ \alst{á}tta nǫ́ttum, &
svá vas hón \alst{ó}ð-fús \hld\ í \alst{jǫ}tun-hęima.“\eva

\bvb Sat the all-clever handmaid \ken*{= Lock} in front, \\
who a word did find against the ettin’s speech: \\
“Frow ate naught for eight nights; \\
so madly she longed for the Ettin-homes.”\evb\evg


\bvg\bva \alst{L}aut und \alst{l}ínu, \hld\ \alst{l}ysti at kyssa, &
en hann \alst{ú}tan stǫkk \hld\ \alst{ę}nd-langan sal: &
„Hví eru \alst{ǫ}ndótt \hld\ \alst{au}gu Fręyju? &
\edtrans{Þykki mér \alst{ó}r \hld\ \alst{au}gum brenna!}{Methinks it burning from the eyes!}{\Bfootnote{The meter of this line is very poor: the first half-line is only three syllables long, and the alliteration falls on \emph{ór} ‘from’, which has no reason to be stressed.  It would be much improved by inserting \emph{ęldar} ‘fires’ between \emph{augum} ‘eyes’ and \emph{brenna} ‘burns’, and this expression is actually attested in \Gylfaginning\ 51: \emph{Eldar brenna ór augum hans ok nǫsum} ‘Fires burn from his eyes and nostrils’.}}“\eva

\bvb He \ken*{= Thrim} looked ’neath the linen, lusted to kiss— \\
but flung back out across the length of the hall— \\
“Why are the eyes of Frow blazing? \\
Methinks it burning from the eyes!”\evb\evg


\bvg\bva Sat hin \alst{a}l-snotra \hld\ \alst{a}mbǫ́tt \edtext{fyrir}{\Afootnote{add. \emph{†ſ.†} \Regius.}} &
es \alst{o}rð of fann \hld\ við \alst{jǫ}tuns máli: &
„Svaf vę́tr Fręyja \hld\ átta nǫ́ttum, &
svá vas hón \alst{ó}ð-fús \hld\ í \alst{jǫ}tun-hęima.“\eva

\bvb Sat the all-clever handmaid in front, \\
who a word did find against the ettin’s speech: \\
“Frow slept naught for eight nights; \\
so madly she longed for the Ettin-homes.”\evb\evg


\bvg\bva \alst{I}nn kom hin \alst{a}rma \hld\ \alst{jǫ}tna systir, &
hin’s \alst{b}rúð-féar \hld\ \alst{b}iðja þorði: &
„Lát þér af \alst{h}ǫndum \hld\ \alst{h}ringa rauða &
ef þú \alst{ǫ}ðlask vill \hld\ \alst{á}stir mínar, &
\edtrans{\alst{á}stir mínar, \hld\ \alst{a}lla hylli}{my love; all [my] holdness’}{\Bfootnote{Probably formulaic.  There are no preserved parallels in poetry, but there may be one in \Gylfaginning\ 49 (excerpt, following the death of Balder):
\emph{En er goðin vitkuðust, þá mę́lti Frigg ok spurði, hverr sá vę́ri með ásum, er \emph{eignast vildi „allar ástir mínar}} (so \Trajectinus\Wormianus; \emph{ástir hennar} ‘her loves’ \RegiusProse\Upsaliensis) \emph{\emph{ok hylli}, ok vili hann ríða á hel-veg ok freista, ef hann fái fundit Baldr, ok bjóða Helju út-lausn, ef hon vill láta fara Baldr heim í Ás-garð.“} ‘But when the gods came back to their wits, then Frie spoke and asked which one among the Eese \emph{would own “all my loves and holdness}, and will ride on the \inx[L]{Hellway} and see if he may find Balder and offer Hell a ransom if she will let Balder come home to Osyard.”’
We can tell from the citation of a \Ljodahattr\ stanza at the end of ch. 49 (see Eddic Fragments below) that Snorre knew one or more now-lost Eddic poems about Balder’s death, and it may be that one of these poems contained the same two long-lines as the present stanza.  For such a sharing of whole lines cf. e.g. st. 14/1–3 above, which are identical to \Baldrsdraumar\ 1/1–3.}}!“\eva

\bvb In came the wretched sister of the ettins, \\
she who for the bride-fee \ken*{= Millner} dared ask: \\
“Slide off from thy hands the red rings, \\
if thou wilt win my love; \\
my love, [and] all [my] \inx[C]{holdness}.”\footnoteB{The sister, who was apparently the one who asked for the Hammer, now has the audacity to ask Thunder (disguised as Frow) to give her the very rings on his hands.}\evb\evg


\bvg\bva Þá kvað þat \alst{Þ}rymr, \hld\ \alst{þ}ursa dróttinn: &
„\alst{B}erið inn hamar \hld\ \alst{b}rúði at vígja, &
lęggið \alst{M}jǫllni \hld\ í \alst{m}ęyjar kné, &
\alst{v}ígið okkr saman \hld\ \edtrans{\alst{V}árar}{Ware}{\Bfootnote{According to Snorre one of the goddesses, presiding over vows between men and women.  See Encyclopedia.}} hęndi!“\eva

\bvb Then quoth this Thrim, the lord of Thurses: \\
“Bear ye in the hammer the bride for to bless; \\
lay ye Millner in the maiden’s knee; \\
bless ye us together by \inx[P]{Ware}’s hand!”\evb\evg


\bvg\bva \alst{H}ló \alst{H}lórriða \hld\ \alst{h}ugr í brjósti &
es \alst{h}arð-\alst{h}ugaðr \hld\ \alst{h}amar of þękkði; &
\alst{Þ}rym drap hann fyrstan, \hld\ \alst{þ}ursa dróttin, &
ok \alst{ę́}tt \alst{jǫ}tuns \hld\ \alst{a}lla lamði.\eva

\bvb Laughed Loride’s \name{= Thunder’s} heart in his chest, \\
when, hard-hearted, he recognised the hammer. \\
Thrim he smote first, the lord of Thurses, \\
and all the ettin’s lineage he beat lame.\evb\evg


\bvg\bva Drap hann ina \alst{ǫ}ldnu \hld\ \alst{jǫ}tna systur, &
hin’s \alst{b}rúð-féar \hld\ of \alst{b}eðit hafði; &
hón \alst{sk}ell of hlaut \hld\ fyr \alst{sk}illinga, &
en \alst{h}ǫgg \alst{h}amars \hld\ fyr \alst{h}ringa fjǫlð. &
Svá kom \alst{Ó}ðins sonr \hld\ \alst{ę}ndr at hamri.\eva

\bvb He smote the aged sister of the ettins, \\
she who for the bride-fee had asked; \\
a smiting she got for shillings, \\
and a strike of the hammer for a multitude of rings.— \\
So came Weden’s son back to his hammer.\evb\evg

\sectionline
% Thunder
%	\bookStart{Speeches of Allwise}[Alvíssmǫ́l]
\def\thisBookCode{Alvissmal}

\begin{flushright}%
\textbf{Dating} \parencite{Sapp2022}: C10th (0.851)

\textbf{Meter:} \Ljodahattr%
\end{flushright}

\section{Introduction}

The \textbf{Speeches of Allwose} (\Alvissmal) is essentially a list of poetic synonyms set in a frame narrative of Thunder being visited by a dwarf insisting that he has been promised his daughter’s hand.  The synonyms are often archaic, representing older common Indo-European and Germanic words which have been displaced by younger words in the common register.  Some are not found elsewhere.

The translation is currently incomplete.

\section{The Speeches of Allwise}

\bvg\bva%
„\alst{B}ękki \alst{b}ręiða \hld\ nú skal \alst{b}rúðr með mér &
\ind hęim ï \alst{s}inni \alst{s}núask; &
\alst{h}ratat of mę́gi \hld\ mun \alst{h}vęrjum þikkja; &
\ind \alst{h}ęima skal-at \alst{h}víld nema.“\eva

\bvb “{\huge S}pread out on the benches shall now the bride with me; \\
\ind turn home by my side. \\
A hurried engagement it will seem to each; \\
\ind at home shall she not take rest!”\evb\evg


\bvg\bva%
„Hvat ’s þat \alst{f}ira; \hld\ hví ert svá \alst{f}ǫlr umb nasar; &
\ind vast-u ï \alst{n}ǫ́tt með \alst{n}á? &
\alst{Þ}ursa líki \hld\ þikki mér ȧ \alst{þ}ér vesa; &
\ind ert-at-tu til \alst{b}rúðar \alst{b}orinn.“\eva

\bvb “What sort of man is this; why art thou so pale about the nose; \\
\ind wast thou tonight with a corpse? \\
The likeness of a thurse methinks thou art; \\
\ind thou wast not born for a bride!”\evb\evg


\bvg\bva%
„\alst{A}l-víss ek hęiti \hld\ bý’k fyr \alst{jǫ}rð neðan &
\ind á’k undir \alst{st}ęini \alst{st}að. &
\edtrans{\alst{v}agna \alst{v}ers}{man of wagons}{\Bfootnote{The “wagons” may here be constellations in the heavens, namely the \emph{Charles’ Wain} (Great Bear, Big Dipper) and \emph{Women’s Wain} (Little Bear, Little Dipper).  Cf. \Skaldskaparmal\ 31, where heaven/the sky is kenned \emph{land sólar ok tungls ok himin-tungla, vagna ok veðra} ‘the land of sun and moon, and the heavenly bodies, wagons and winds.’}} \hld\ ek em á \alst{v}it kominn &
\ind bręgði ęngi \alst{f}ǫstu hęiti \alst{f}ira.“\eva

\bvb “Allwise I am called; I live beneath the earth; \\
\ind I own under a stone my home. \\
The man of wagons \ken*{= Thunder} I am come to visit; \\
\ind let no man break a firm promise!”\evb\evg


\bvg\bva%
„Ek mun \alst{b}ręgda \hld\ því’t ek \alst{b}rúðar ȧ &
\ind \alst{f}lęst umb rǫ́ð sem \alst{f}aðir. &
vas’k-a ek \alst{h}ęima \hld\ þá’s þér \alst{h}ęitit vas &
\ind at sá ęinn es \alst{g}jǫf es með \alst{g}oðum.“\eva

\bvb “\emph{I} will break it, for about the bride \\
\ind I have the greatest say, as her father. \\
I was not at home when it was promised thee, \\
\ind but he [I] alone is the giver among the gods!”\evb\evg


\bvg\bva%
„Hvat ’s þat \alst{r}ekka \hld\ es ï \alst{r}ǫ́ðum tęlsk &
\ind \alst{f}ljóðs ins \alst{f}agr-glóa; &
\alst{f}jarra-\alst{f}lęina \hld\ þik munu \alst{f}áir kunna; &
\ind hvęrr hęfir þik \alst{b}augum \alst{b}orit?“\eva

\bvb “What sort of champion is this who claims to have a say \\
\ind about the fair-glowing girl? \\
O foreign tramp, few men will know thee; \\
\ind who has borne bighs to thee?”\evb\evg


\bvg\bva%
\alst{V}ing-Þȯrr ek hęiti \hld\ ek hęfi \alst{v}íða ratat &
\ind \alst{s}onr em’k \alst{S}íð-grana; &
at \alst{ȯ}-sátt mïnni \hld\ skalt þat it \alst{u}nga man hafa &
\ind ok þat \alst{g}jaf-orð \alst{g}eta.\eva

\bvb “\inx[P]{Wing-Thunder} I am called; I have widely roamed; \\
\ind I am the son of Sidegrane. \\
Against my assent shalt thou have this young girl, \\
\ind and get that gift-word!”\evb\evg


\bvg\bva%
\alst{S}áttir þïnar \hld\ es ek vil \alst{s}nemma hafa &
\ind ok þat \alst{g}jaf-orð \alst{g}eta. &
\alst{ęi}ga vilja \hld\ heldr an \alst{á}n vera &
\ind þat it \alst{m}jall-hvíta \alst{m}an.\eva

\bvb “Thy assent I wish to have soon, \\
\ind and get that gift-word, \\
I would rather have than be without \\
\ind this snow-white girl.”\evb\evg


\bvg\bva%
„\alst{M}ęyjar ǫ̇stum \hld\ \alst{m}un-a þér verða &
\ind \alst{v}ísi gęstr of \alst{v}arið, &
ef þú ór \alst{h}ęimi kant \hld\ \alst{h}vęrjum at sęgja &
\ind alt þat’s ek \alst{v}il \alst{v}ita.\eva

\bvb “The maiden’s love will not be thee, \\
\ind O wise guest, denied, \\
if thou from every home canst tell \\
\ind all I wish to know:\evb\evg


\bvg\bva%
Sęg-ðu mér þat \alst{A}l-víss \hld\ \alst{ǫ}ll of rǫk fira &
\ind \alst{v}ǫrumk dvergr at \alst{v}itir, &
hvé sú \alst{jǫ}rð hęitir \hld\ es liggr fyr \alst{a}lda sonum &
\ind \alst{h}ęimi \alst{h}vęrjum ï.“\eva

\bvb Tell me this, Allwise—of all rakes of men, \\
\ind I think, dwarf, that thou mighst know: \\
what the earth is called which lies before the sons of men \\
\ind in every home.”\evb\evg


\bvg\bva%
„\alst{Jǫ}rð hęitir með mǫnnum \hld\ en með \alst{ǫ}lfum fold. &
\ind kalla \alst{v}ega \alst{v}anir. &
\alst{ï}-grǿn \alst{jǫ}tnar \hld\ \alst{a}lfar gróandi &
\ind kalla \alst{au}r \alst{u}pp-ręgin.“\eva

\bvb “‘Earth’ it is called among men, but among elves ‘fold’; \\
\ind call it ‘ways’ the Wanes; \\
‘evergreen’ ettins, elves ‘growing’; \\
\ind call it ‘mud’ the Up-reins.”\evb\evg


\bvg\bva%
Sęg-ðu mér þat \alst{A}l-víss \hld\ \alst{ǫ}ll of rǫk fira &
\ind \alst{v}ǫrumk dvergr at \alst{v}itir; &
hvé sá himinn hęitir \hld\ \edtrans{erakendi}{...}{\Bfootnote{A string too corrupt to restore without excessive conjecture; it at least appears to contain the relative pronoun \emph{er} ‘which’, younger form of \emph{es} and the adjective \emph{kęnndr} ‘known’.  Based on the first line, the alliteration must have fallen on \emph{h-}, and the root that first suggests itself is \emph{hę́ð} ‘height’.  A possible restoration is then \emph{es ȧ hę́ð es kęnndr} ‘which is known on high’.}} &
\ind \alst{h}ęimi \alst{h}vęrjum ï.\eva

\bvb “Tell me this, Allwise—of all rakes of men, \\
\ind I think, dwarf, that thou mighst know: \\
what the heaven is called ... \\
\ind in every home.”\evb\evg


\bvg\bva%
\alst{H}iminn hęitir með mǫnnum \hld\ en \alst{H}lýrnir með goðum &
\ind kalla \alst{V}ind-ófni \alst{v}anir; &
\alst{u}pp-hęim \alst{jǫ}tnar \hld\ \alst{a}lfar fagra-rę́fr &
\ind \alst{d}vergar \alst{d}rjúpan sal.\eva

\bvb “‘Heaven’ it is called among Men but ‘Leerner’ among Gods; \\
\ind ‘Wind-ovner’ call it the Wanes; \\
‘upham’ Ettins, Elves ‘fair roof’, \\
\ind Dwarfs ‘dripping hall’.”\evb\evg


\bvg\bva%
Sęg-ðu mér þat \alst{A}l-víss \hld\ \alst{ǫ}ll of rǫk fira &
\ind \alst{v}ǫrumk dvergr at \alst{v}itir; &
hvęrsu máni hęitir \hld\ sá’s męnn sjá &
\ind \alst{h}ęimi \alst{h}vęrjum ï.\eva

\bvb “Tell me this, Allwise—of all rakes of men, \\
\ind I think, dwarf, that thou mighst know: \\
how the moon is called which men do see \\
\ind in every home.”\evb\evg


\bvg\bva%
\alst{M}áni hęitir með \alst{m}ǫnnum \hld\ en \alst{M}ylinn með goðum, &
\ind kalla \alst{h}verfanda \alst{h}vél \alst{h}ęlju ï; &
\alst{sk}yndi jǫtnar \hld\ en \alst{sk}in dvergar &
\ind kalla \alst{a}lfar \edtrans{\alst{á}r-tala}{year-tallier}{\Bfootnote{The moon was important in the Germanic calendar (witness \emph{month}, a “moon-th”).  Cf. \Voluspa\ 6 and \Vafthrudnismal\ 23, 25.}}.\eva

\bvb “Moon it is called among Men, but ‘Milen’ with Gods, \\
\ind they call it ‘turning wheel’ in Hell, \\
‘hurrier’ Ettins and ‘shine’ Dwarfs; \\
\ind Elves call it ‘year-tallier’.”\evb\evg


\bvg\bva%
Sęg-ðu mér þat \alst{A}l-víss \hld\ \alst{ǫ}ll of rǫk fira &
\ind \alst{v}ǫrumk dvergr at \alst{v}itir; &
hvé sú sól hęitir \hld\ es sjá alda synir. &
\ind \alst{h}ęimi \alst{h}vęrjum ï.\eva

\bvb “Tell me this, Allwise—of all rakes of men, \\
\ind I think, dwarf, that thou mighst know: \\
what the sun is called, which the sons of men see, \\
\ind in every home.”\evb\evg


\bvg\bva%
\alst{S}ól hęitir með mǫnnum \hld\ en \alst{S}unna með goðum &
\ind kalla \alst{d}vergar \alst{D}valins lęika; &
\alst{Ęy}-glói \alst{jǫ}tnar \hld\ \alst{a}lfar fagra-hvél &
\ind \alst{a}l-skír \alst{á}sa synir.\eva

\bvb TODO.\evb\evg


\bvg\bva%
„Sęg-ðu mér þat \alst{A}l-víss \hld\ \alst{ǫ}ll of rǫk fira &
\ind \alst{v}ǫrumk dvergr at \alst{v}itir; &
hvé þau ský hęita \hld\ es skúrum blandask &
\ind \alst{h}ęimi \alst{h}vęrjum ï.“\eva

\bvb “Tell me this, Allwise—of all rakes of men, \\
\ind I think, dwarf, that thou mighst know: \\
what the clouds are called where showers are mixed \\
in every home.”\evb\evg


\bvg\bva%
\alst{Sk}ý hęita með mǫnnum, \hld\ en \alst{sk}úr-vǫ́n með goðum; &
\ind kalla \alst{v}ind-flot \alst{v}anir; &
\alst{ú}r-vǫ́n \alst{jǫ}tnar, \hld\ \alst{a}lfar veðr-męgin; &
\ind kalla ï hęlju \alst{h}jalm \alst{h}uliðs.\eva

\bvb “Clouds they are called among Men, but ‘shower-hope’ among Gods; \\
\ind ‘wind-fat’ the Wanes call them; \\
‘drizzle-hope’ the Ettins, Elves ‘weather-strength’; \\
\ind in Hell they call them ‘helmet of the hidden’.”\evb\evg


\bvg\bva%
„Sęg-ðu mér þat \alst{A}l-víss \hld\ \alst{ǫ}ll of rǫk fira &
\ind \alst{v}ǫrumk dvergr at \alst{v}itir; &
hvé sá \alst{v}indr hęitir \hld\ es \alst{v}íðast fęrr &
\ind \alst{h}ęimi \alst{h}vęrjum ï.“\eva

\bvb TODO.\evb\evg


\bvg\bva%
\alst{V}indr hęitir með mǫnnum, \hld\ en \alst{V}ǫ́fuðr með goðum; &
\ind kalla \alst{g}nęggjuð ginn-ręgin. &
\alst{ǿ}pi \alst{jǫ}tnar \hld\ \alst{a}lfar dyn-fara &
\ind kalla ï \alst{h}ęlju \alst{H}viðuð.\eva

\bvb “Wind it is called among Men but ‘Waver’ among Gods, \\
\ind ‘neigher’ call it the Yin-Reins; \\
‘weeper’ Ettins, Elves ‘din-farer’; \\
\ind in Hell they call it ‘stormer’.”\evb\evg


\bvg\bva%
„Sęg-ðu mér þat \alst{A}l-víss \hld\ \alst{ǫ}ll of rǫk fira &
\ind \alst{v}ǫrumk dvergr at \alst{v}itir; &
hvé þat \alst{l}ogn hęitir \hld\ es \alst{l}iggja skal &
\ind \alst{h}ęimi \alst{h}vęrjum ï.“\eva

\bvb “Tell me this, Allwise—of all rakes of men, \\
\ind I think, dwarf, that thou mighst know: \\
what the calm is called, which shall lie \\
\ind in every home.”\evb\evg


\bvg\bva%
„\alst{L}ogn hęitir með mǫnnum, \hld\ en \alst{l}ę́gi með goðum, &
\ind kalla \alst{v}inds flot \alst{v}anir; &
\alst{o}f-hlý \alst{jǫ}tnar \hld\ \alst{a}lfar dag-sefa, &
\ind kalla \alst{d}vergar \alst{d}ags veru.“\eva

\bvb “Calm it is called among men and ‘lowering’ among gods, \\
\ind ‘wind’s fat’ call the Wanes; \\
‘great lee’ Ettins, Elves ‘day-sleep’, \\
\ind call it Dwarfs ‘day’s rest’.”\evb\evg


\bvg\bva%
Sęg-ðu mér þat \alst{A}l-víss \hld\ \alst{ǫ}ll of rǫk fira &
\ind \alst{v}ǫrumk dvergr at \alst{v}itir; &
hvé sá \alst{m}arr hęitir \hld\ es \alst{m}ęnn róa &
\ind \alst{h}ęimi \alst{h}vęrjum ï.\eva

\bvb “Tell me this, Allwise—of all rakes of men, \\
\ind I think, dwarf, that thou mighst know: \\
what the ocean is called, where men do row, \\
\ind in every home.”\evb\evg


\bvg\bva%
\alst{S}ę́r hęitir með mǫnnum, \hld\ en \alst{s}ï-lę́gja með goðum, &
\ind kalla \alst{v}ág \alst{v}anir; &
\alst{á}l-hęim \alst{jǫ}tnar, \hld\ \alst{a}lfar laga-staf, &
\ind kalla \alst{d}vergar \alst{d}júpan mar.\eva

\bvb “Sea it is called among men but ‘ever-low’ among gods; \\
\ind ‘wave’ the Wanes call it; \\
‘eelhome’ Ettins, Elves ‘staff of waters’; \\
\ind Dwarfs call it ‘deep ocean’.”\evb\evg


\bvg\bva%
Sęg-ðu mér þat \alst{A}l-víss \hld\ \alst{ǫ}ll of rǫk fira &
\ind \alst{v}ǫrumk dvergr at \alst{v}itir; &
hvé sá \alst{ę}ldr hęitir \hld\ es brenn fyr \alst{a}lda sonum &
\ind \alst{h}ęimi \alst{h}vęrjum ï.\eva

\bvb “Tell me this, Allwise—of all rakes of men, \\
\ind I think, dwarf, that thou mighst know: \\
what the fire is called, which burns for the sons of men, \\
\ind in every home.”\evb\evg


\bvg\bva%
„\alst{Ę}ldr hęitir með mǫnnum \hld\ en með \alst{ǫ́}sum funi &
\ind kalla \alst{v}ág \alst{v}anir; &
\alst{f}rekan jǫtnar \hld\ en \alst{f}or-bręnni dvergar &
\ind kalla ï \alst{h}ęlju \alst{h}rǫðuð.“\eva

\bvb “Fire it is called among men but among the Eese ‘flame’, \\
\ind ‘wave’ the Wanes call it; \\
‘the greedy’ Ettins, but ‘burner’ Dwarfs; \\
\ind in Hell they call it ‘hurrier’.”\evb\evg


\bvg\bva%
Sęg-ðu mér þat \alst{A}l-víss \hld\ \alst{ǫ}ll of rǫk fira &
\ind \alst{v}ǫrumk dvergr at \alst{v}itir; &
hvé \alst{v}iðr hęitir \hld\ es \alst{v}ęx fyr alda sonum &
\ind \alst{h}ęimi \alst{h}vęrjum ï.\eva

\bvb “Tell me this, Allwise—of all rakes of men, \\
\ind I think, dwarf, that thou mighst know: \\
what the wood is called, which grows for the sons of men, \\
\ind in every home.”\evb\evg


\bvg\bva%
\alst{V}iðr hęitir með mǫnnum. \hld\ en \edtext{\alst{v}allar fax}{\Afootnote{emend.; \emph{vallar-far} \Regius.}} með goðum &
\ind kalla \alst{h}líð-þang \alst{h}alir; &
\alst{ę}ldi \alst{jǫ}tnar \hld\ \alst{a}lfar fagr-lima &
\ind kalla \alst{v}ǫnd \alst{v}anir.\eva

\bvb “Wood it is called among men but ‘mane of the plain’ among gods, \\
\ind ‘slope-kelp’ heroes call it; \\
‘firewood’ Ettins, Elves ‘fair-limb’; \\
\ind ‘wands’ the Wanes call it.”\evb\evg


\bvg\bva%
„Sęg-ðu mér þat \alst{A}l-víss \hld\ \alst{ǫ}ll of rǫk fira &
\ind \alst{v}ǫrumk dvergr at \alst{v}itir; &
hvé sú \alst{n}ǫ́tt hęitir \hld\ in \alst{N}ǫrvi kęnda &
\ind \alst{h}ęimi \alst{h}vęrjum ï.“\eva

\bvb “Tell me this, Allwise—of all rakes of men, \\
\ind I think, dwarf, that thou mighst know: \\
what the night is called, begotten to Narrow, \\
\ind in every home.”\evb\evg


\bvg\bva%
„\alst{N}ǫ́tt hęitir með mǫnnum \hld\ en \alst{n}jól með goðum, &
\ind kalla \alst{g}rímu \alst{g}inn-ręgin; &
\alst{ȯ}-ljós \alst{jǫ}tnar \hld\ \alst{a}lfar svefn-gaman &
\ind kalla \alst{d}vergar \alst{d}raum-njǫrun.“\eva

\bvb “Night it is called among men but ‘nivel’ among the gods; \\
\ind call it ‘mask’ the yin-Reins. \\
‘Un-light’ ettins, elves ‘sleep-joy’; \\
\ind call it dwarfs ‘dream-Narn’.”\evb\evg


\bvg\bva%
„Sęg-ðu mér þat \alst{A}l-víss \hld\ \alst{ǫ}ll of rǫk fira &
\ind \alst{v}ǫrumk dvergr at \alst{v}itir; &
hvé þat \alst{s}ǫ́ð hęitir \hld\ es \alst{s}áa alda synir &
\ind \alst{h}ęimi \alst{h}vęrjum ï.“\eva

\bvb “Tell me this, Allwise—of all rakes of men, \\
\ind I think, dwarf, that thou mighst know: \\
what the seed is called, which the sons of men sow, \\
\ind in every home.”\evb\evg


\bvg\bva%
\alst{B}ygg hęitir með mǫnnum \hld\ en \alst{b}arr með goðom &
\ind kalla \alst{v}ǫxt \alst{v}anir. &
\alst{ę́}ti \alst{jǫ}tnar \hld\ \alst{a}lfar laga-staf &
\ind kalla ï \alst{h}ęlju \alst{h}nipinn.\eva

\bvb “Barley it is called among Men but ‘leaf’ among Gods; \\
\ind ‘growth’ the Wanes call it; \\
‘eating’ Ettins, Elves ‘staff of waters’; \\
\ind in Hell they call it ‘drooping’.”\evb\evg


\bvg\bva%
„Sęg-ðu mér þat \alst{A}l-víss \hld\ \alst{ǫ}ll of rǫk fira &
\ind \alst{v}ǫrumk dvergr at \alst{v}itir; &
hvé þat \alst{ǫ}l hęitir \hld\ es drekka \alst{a}lda synir &
\ind \alst{h}ęimi \alst{h}vęrjum ï.“\eva

\bvb “Tell me this, Allwise—of all rakes of men, \\
\ind I think, dwarf, that thou mighst know: \\
what the ale is called, which the sons of men drink, \\
\ind in every home.”\evb\evg


\bvg\bva%
\alst{Ǫ}l hęitir með mǫnnum \hld\ en með \alst{ǫ́}sum bjórr; &
\ind kalla \alst{v}ęig \alst{v}anir; &
\alst{h}ręina-lǫg jǫtnar \hld\ en ï \alst{h}ęlju mjǫð; &
\ind kalla \alst{s}umbl \alst{S}uttungs \alst{s}ynir.\eva

\bvb “Ale it is called among Men but among the Eese ‘beer’; \\
call it ‘draughts’ the Wanes; \\
‘pure water’ the Ettins but in Hell ‘mead’; \\
call it ‘simble’ Sutting’s Sons.”\evb\evg


\bvg\bva%
İ \alst{ęi}nu brjósti \hld\ ek sá’k \alst{a}ldri-gi &
\ind \alst{f}lęiri \alst{f}orna stafi; &
miklum \alst{t}ǫ́lum \hld\ ek kveð \alst{t}ę́ldan þik: &
\ind uppi ert \alst{d}vergr of \alst{d}agaðr; &
\ind nú skïnn \alst{s}ól ï \alst{s}ali.\eva

\bvb “In a single breast I never saw \\
\ind more ancient staves— \\
with mighty tricks I call thee tricked: \\
\ind thou art, dwarf, dayed up; \\
\ind now shines the sun into the halls!”\evb\evg

\sectionline
% Thunder, Wisdom poem
	\bookStart{The Thule of Righ}[Rígsþula]

\begin{flushright}%
\textbf{Dating} \parencite{Sapp2022}: early 1000s (0.240), late 1000s (0.204), late 1100s (0.195), 1200s (0.280)

\textbf{Meter:} \Fornyrdislag%
\end{flushright}

Dumezil hypothesis. Irish influence? Many interesting things to write here!

The language of \Rigsthula\ is highly formulaic, but also often unique to it. Of particular note are the alliteration between the adverb \emph{męirr} ‘further’ and \emph{miðra}, e.g. in st. 2/1: \emph{gekk męirr at þat}

\sectionline

\bpg
\bpa\mssnote{\Wormianus~78r/1}Svá sęgja męnn í fornum sǫgum, at ęinn-hvęrr af ǫ́sum, sá er Hęimdallr hét, fór fęrðar sinnar ok framm með sjóvar-strǫndu nǫkkurri, kom at ęinum húsa-bǿ ok nęfndisk Rigr; ęptir þęiri sǫgu er kvę́ði þetta.\epa

\bpb So say men in ancient \inx[C]{saw}[saws] that one of the \inx[G]{Eese}, he who was called \inx[P]{Homedall}, went on his journey and came forth along a certain lake shore, came upon a lone homestead and called himself Righ—according to that saw is this poem.\epb
\epg


\bvg\bva\mssnote{\Wormianus~78r/TODO}\edtrans{Ár}{Of yore}{\Afootnote{sens. emend. (see note); \emph{at} \Wormianus}\Bfootnote{Formulaic. It is very common for poems to begin with \emph{ár}. Cf. \Voluspa\ 3/1, \Hymiskvida\ 1/1, \HelgakvidaOne\ 1/1, \GudrunOne\ 1/1, \Sigurdskamma\ 1/1}} kvǫ́ðu ganga \hld\ grǿnar brautir &
ǫflgan ok aldinn \hld\ ǫ́s kunnigan, &
ramman ok rǫskvan \hld\ Ríg stíganda.\eva

\bvb Of yore, they said, did walk on green roads \\
a mighty and aged \inx[G]{Eese}[os], cunning: \\
the strong and brisk Righ, striding.\evb\evg


\bvg\bva\mssnote{\Wormianus~78r/TODO}Gekk męirr at þat \hld\ miðrar brautar, &
kom hann at húsi, \hld\ hurð vas á gę́tti; &
inn nam at ganga, \hld\ ęldr vas á golfi, &
hjón sǫ́tu þar \hld\ hǫ́r \edtext{at}{\Afootnote{sens. emend.; \emph{af} \Wormianus}} arni, &
Ái ok Ędda \hld\ aldin-falda.\eva

\bvb Went he further after that on the middle of the road, \\
came he to a house—the door was wide open. \\
He took to go inside, fire was on the floor. \\
A couple sat there, hoary by the hearth: \\
Great-Grandfather and Great-Grandmother, old-fashioned.\evb\evg


\bvg\bva\mssnote{\Wormianus~78r/TODO}Rigr kunni þęim \hld\ rǫ́ð at sęgja; &
męirr sęttisk hann \hld\ miðra flętja &
en á hlið hvára \hld\ hjón sal-kynna.\eva

\bvb Righ knew to tell them counsels, \\
further he set himself down on the middle of the floor-bench, \\
and on either side: the couple of the hall.\evb\evg


\bvg\bva\mssnote{\Wormianus~78r/TODO}Þá tók Ędda \hld\ økkvinn hlęif, &
þungan ok þykkvan, \hld\ þrunginn sǫ́ðum, &
bar hǫ́n męirr at þat \hld\ miðra skutla, &
soð vas í bolla \hld\ sętti á bjóð; &
vas kalfr soðinn \hld\ krása bętstr; &
ręis hann upp þaðan, \hld\ réðsk at sofna;\eva

\bvb Then took Great-Grandmother a lumpy loaf, \\
heavy and thick, stuffed with chaff, \\
she carried it further after that on the middle of a trencher, \\
broth was in a bowl, she set it on a plate— \\
a cooked calf was the best dainty; \\
he \ken*{= Righ} rose up thence, resolved to sleep.\evb\evg


\bvg\bva\mssnote{\Wormianus~78r/TODO}Rigr kunni þęim \hld\ rǫ́ð at sęgja; &
męirr lagðisk hann \hld\ miðrar rękkju, &
en á hlið hvára \hld\ hjón salkynna.\eva

\bvb Righ knew to tell them counsels; \\
further he laid himself down in the middle of the bed, \\
and on either side: the couple of the hall.\evb\evg


\bvg\bva\mssnote{\Wormianus~78r/TODO}Þar vas hann at þat \hld\ þrjár nę́tr saman; &
gekk hann męirr at þat \hld\ miðrar brautar; &
liðu męirr at þat \hld\ mǫ́nuðr níu.\eva

\bvb There he was after that for three nights in all; \\
went he further after that on the middle of the road; \\
passed further after that nine months.\evb\evg


\bvg\bva\mssnote{\Wormianus~78r/TODO}Jóð ól Ędda, \hld\ jósu vatni &
\edtrans{hǫrund-svartan}{swarthy of skin}{\Afootnote{emend.; \emph{hǫrfi svartan} ‘swarthy with flax(?)’ \Wormianus}}, \hld\ hétu Þrę́l.\eva

\bvb Great-Grandmother begot a child, they sprinkled it with water\footnoteB{A reference to the Heathen naming ceremony, somewhat resembling the Christian baptism, wherein water would be poured on a newborn. Cf. \Havamal\ 156.}— \\
swarthy of skin—they called it Thrall.\evb\evg


\bvg\bva\mssnote{\Wormianus~78r/TODO}Hann nam at vaxa \hld\ ok vęl dafna; &
vas þar á hǫndum \hld\ hrokkit skinn, &
kropnir knúar, \hld\ [...] &
fingr digrir, \hld\ fúlligt and-lit, &
lotr hryggr, \hld\ langir hę́lar.\eva

\bvb He took to grow, and thrive well; \\
there on his hands was wrinkled skin, \\
crooked knuckles, [...], \\
thick fingers, a foul face, \\
a stooping back, long heels.\evb\evg


\bvg\bva\mssnote{\Wormianus~78r/TODO}Nam męirr at þat \hld\ magns of kosta, &
bast at binda, \hld\ byrðar gørva; &
bar hęim at þat \hld\ hrís gęrstan dag.\eva

\bvb He took further after that to try his power: \\
bast to bind, burdens to make, \\
he carried home after that brushwood on a gloomy day.\footnoteB{The thrall had to work in even the most hostile weather.}\evb\evg


\bvg\bva\mssnote{\Wormianus~78r/TODO}Þar kom at garði \hld\ \edtrans{gęngil-bęina}{gangle-boned woman}{\Bfootnote{Derogatory, somebody who (due to poverty) only travels by foot.}}, &
aurr vas á iljum, \hld\ armr sól-brunninn, &
niðr-bjúgt es nęf, \hld\ nęfndisk \edtrans{Þír}{Thew}{\Bfootnote{The name probably means ‘maid-servant’ or ‘female slave’. Unlike Thrall, it is not attested in any prose texts, but probably corresponds to OS \emph{thiwi} ‘maid(-servant)’, being further root-related to \emph{þéa \char`~ þjá} ‘to enthral’, Proto-Norse \textbf{þewaʀ} ‘servant’, OE \emph{þéow} ‘slave, servant’,.}}.\eva

\bvb There came to the farm a gangle-boned woman: \\
mud was on her footsoles, her arm sunburnt, \\
downturned her face—she called herself Thew.\evb\evg


\bvg\bva\mssnote{\Wormianus~78r/TODO}\edtext{Męirr sęttisk hǫ́n \hld\ miðra flętja,}{\lemma{Męirr \dots\ flętja}\Afootnote{emend. based on other sts.; \emph{miðra flętja \hld\ męirr sęttisk hǫ́n} \Wormianus}} &
sat hjá hęnni \hld\ sonr húss, &
rǿddu ok rýndu, \hld\ rękkju gørðu &
Þrę́ll ok Þír \hld\ þrungin dǿgr.\eva

\bvb Further she set herself down on the middle of the floor-bench; \\
by her sat the son of the house \ken*{= Thrall}. \\
They spoke and whispered, made a bed— \\
Thrall and Thew—in hard-pressed nights.\evb\evg


\bvg\bva\mssnote{\Wormianus~78r/TODO}Bǫrn ólu þau, \hld\ bjuggu ok unðu; &
hygg’k at héti \hld\ Hręimr ok Fjósnir, &
Klúrr ok Klęggi, \hld\ Kęfsir, Fúlnir, &
Drumbr, Digraldi, \hld\ Drǫttr ok Hǫsvir, &
Lútr ok Lęggjaldi; \hld\ lǫgðu garða, &
akra tǫddu, \hld\ unnu at svínum, &
gęita gę́ttu, \hld\ grófu torf.\eva

\bvb Children they begot—they settled and were content— \\
I think that they were called Rame and Feesner, \\
Clour and Cledge, Chafser, Foulner, \\
Drumber, Digrald, Drant and Hazer, \\
Lout and Ledgald.—They laid yard-fences, \\
dunged the fields, fed the swine, \\
kept the goats, dug the turf.\evb\evg


\bvg\bva\mssnote{\Wormianus~78r/TODO}Dǿtr vǫ́ru þę́r \hld\ Drumba ok Kumba, &
Økkvin-kalfa \hld\ ok Arin-nęfja, &
Ysja ok Ambǫ́tt, \hld\ Ęikin-tjasna, &
Tǫtrug-hypja \hld\ ok Trǫnu-bęina; &
þaðan eru komnar \hld\ þrę́la ę́ttir.\eva

\bvb The daughters were these: Drumb and Cumb; \\
Inkencalf and Arn-neb,
Yeaze and Ambight, Oakentezen,
Tattryhip and Tranebone— \\
thence are come the lineages of thralls.\evb\evg


\sectionline


\bvg\bva\mssnote{\Wormianus~78r/TODO}Gekk Rigr at þat \hld\ réttar brautir &
kom hann at \edtrans{hǫllu}{hall}{\Afootnote{sens. and metr. emend., cf. st. TODO; om. \Wormianus}} \hld\ hurð vas á skiði &
inn nam at ganga \hld\ ęldr vas á golfi &
hjón sǫ́tu þar \hld\ heldu á syslu.\eva

\bvb TODO: Translation.\evb\evg


\bvg\bva\mssnote{\Wormianus~78r/TODO}Maðr tęlgði þar \hld\ męið til rifjar, &
vas skęgg skapat, \hld\ skǫr vas fyr ęnni &
skyrtu þrǫngva \hld\ skokkr vas á golfi.\eva

\bvb TODO: Translation.\evb\evg


\bvg\bva\mssnote{\Wormianus~78r/TODO}Sat þar kona, \hld\ svęigði rokk, &
bręiddi faðm, \hld\ bjó til váðar; &
svęigr vas á hǫfði, \hld\ smokkr vas á bríngu, &
dúkr vas á halsi, \hld\ dvergar á ǫxlum; &
Afi ok Amma \hld\ ǫ́ttu hús.\eva

\bvb TODO: Translation.\evb\evg


\bvg\bva\mssnote{\Wormianus~78r/TODO}Rigr kunni þęim \hld\ rǫ́ð at sęgja, &
ręis frá borði \hld\ réð at sofna. &
Męirr lagðisk hann \hld\ miðrar rękkju &
en á hlið hvára \hld\ hjón sal-kynna. &
Þar vas hann at þat \hld\ þrjár nę́tr saman &
liðu męirr at þat \hld\ mǫ́nuðr níu.\eva

\bvb Righ knew to tell them counsels, \\
rose from the table, resolved to sleep. \\
Further he laid himself down in the middle of the bed, \\
and on either side: the couple of the hall. \\
There he was after that for three nights in all; \\
passed further after that nine months.\evb\evg


\bvg\bva\mssnote{\Wormianus~78r/TODO}Jóð ól Amma, \hld\ jósu vatni, &
kǫlluðu Karl \hld\ kona svęip ripti &
rauðan ok rjóðan \hld\ riðuðu augu.\eva

\bvb Grandmother begot a child, they sprinkled it with water, \\
called it Churl; the woman wrapped him in cloth, \\
red and ruddy; his eyes trembled.\evb\evg


\bvg\bva\mssnote{\Wormianus~78r/TODO}Hann nam at vaxa \hld\ ok vęl dafna, &
ǫxn nam at tęmja \hld\ arðr at gørva &
hús at timbra \hld\ ok hlǫður smíða &
karta at gørva \hld\ ok kęyra plóg.\eva

\bvb TODO: Translation.\evb\evg


\bvg\bva\mssnote{\Wormianus~78r/TODO}Hęim óku þá \hld\ Hangin-luklu &
gęita kyrtlu \hld\ giptu Karli. &
Snǫr hęitir sú, \hld\ sęttisk und ripti. &
Bjuggu hjón, \hld\ bauga dęildu, &
bręiddu blę́jur, \hld\ ok bú gørðu.\eva

\bvb TODO: Translation.\evb\evg


\bvg\bva\mssnote{\Wormianus~78r/TODO}Bǫrn ólu þau, \hld\ bjuggu ok unðu; &
hét Halr ok Drengr, \hld\ Hǫldr, Þegn ok Smiðr, &
Bręiðr, Bóndi, \hld\ Bundin-skęggi, &
Búi ok Boddi \hld\ Bratt-skęggr ok Sęggr.\eva

\bvb Children they begot—they settled and were content— \\
TODO: Translation.\evb\evg


\bvg\bva\mssnote{\Wormianus~78v/1}Enn hétu svá \hld\ ǫðrum nǫfnum &
Snot, Brúðr, Svanni, \hld\ Svarri, Sprakki, &
Fljóð, Sprund, ok Víf, \hld\ Fęima, Ristill— &
þaðan eru komnar \hld\ karla ę́ttir.\eva

\bvb TODO: Translation.\evb\evg


\sectionline


\bvg\bva\mssnote{\Wormianus~78v/TODO}Gekk Rigr þaðan \hld\ réttar brautir &
kom hann at sal, \hld\ suðr horfðu dyrr, &
vas hurð hnigin, \hld\ hringr vas í gę́tti.\eva

\bvb TODO: Translation.\evb\evg


\bvg\bva\mssnote{\Wormianus~78v/TODO}Gekk hann inn at þat \hld\ golf vas stráat &
sǫ́tu hjón \hld\ sǫ́usk í augu &
faðir ok móðir \hld\ fingrum at lęika.\eva

\bvb TODO: Translation.\evb\evg


\bvg\bva\mssnote{\Wormianus~78v/TODO}Sat hús-gumi \hld\ ok snøri stręng &
alm of bęndi \hld\ ǫrvar skępti; &
en hús-kona \hld\ hugði at ǫrmum, &
strauk of ripti \hld\ sterti ęrmar.\eva%TODO: sterti or stęrti?

\bvb Sat the husband and twisted the bow-string, \\
bent the elmwood, shafted arrows— \\
but the housewife minded her arms, \\
smoothened the fabric, tightened the sleeves.\evb\evg


\bvg\bva\mssnote{\Wormianus~78v/TODO}Kęisti fald, \hld\ kinga vas á bringu, &
síðar slǿður, \hld\ sęrk blá-fáan; &
brún bjartari, \hld\ brjóst ljósara, &
hals hvítari \hld\ hręinni mjǫllu.\eva

\bvb The linen hood jutted out, a brooch was on her chest, \\
a long-hanging gown, her serk dyed blue;
her brow was brighter, her chest lighter, \\
her throat whiter than purest snow.\evb\evg


\bvg\bva\mssnote{\Wormianus~78v/TODO}Rigr kunni þęim \hld\ rǫ́ð at sęgja; &
męirr sęttisk hann \hld\ miðra flętja &
en á hlið hvára \hld\ hjón sal-kynna.\eva

\bvb Righ knew to tell them counsels, \\
further he set himself down on the middle of the floor-bench, \\
and on either side: the couple of the hall.\evb\evg


\bvg\bva\mssnote{\Wormianus~78v/TODO}Þá tók móðir \hld\ męrktan dúk, &
hvítan af hǫrvi, \hld\ hulði bjóð; &
hón tók at þat \hld\ hlęifa þunna, &
hvíta af hvęiti, \hld\ ok hulði dúk.\eva

\bvb Then took Mother a patterned cloth, \\
white of flax—she covered a platter. \\
She took after that thin loaves, \\
white of wheat—and covered the cloth.\footnoteB{Note the strong parallelism. The household can afford an excess of expensive fabric and bread; Mother can cover the platter with a patterned (\emph{męrktr}) flaxen cloth, and then cover the cloth with wheat-bread.}\evb\evg


\bvg\bva\mssnote{\Wormianus~78v/TODO}Framm sętti hón \hld\ skutla fulla &
silfri varða á bjóð &
fán ok flęski \hld\ ok fugla stęikta &
vín vas i kǫnnu \hld\ varðir kalkar; &
drukku ok dǿmðu; \hld\ dagr vas á sinnum.\eva

\bvb TODO: Translation.\evb\evg


\bvg\bva\mssnote{\Wormianus~78v/TODO}Rigr kunni þęim \hld\ rǫ́ð at sęgja, &
ręis Rigr at þat, \hld\ rękkju gørði.\eva

\bvb Righ knew to tell them counsels, \\
rose Righ after that, made the bed.\evb\evg

\bvg\bva\mssnote{\Wormianus~78v/TODO}Þar vas hann at þat \hld\ þrjár nę́tr saman; &
gekk hann męirr at þat \hld\ miðrar brautar; &
liðu męirr at þat \hld\ mǫ́nuðr níu.\eva

\bvb There he was after that for three nights in all; \\
went he further after that on the middle of the road; \\
passed further after that nine months.\evb\evg


\bvg\bva\mssnote{\Wormianus~78v/TODO}Svęin ól móðir, \hld\ silki vafði, &
jósu vatni— \hld\ Jarl létu hęita; &
blęikt vas hár, \hld\ bjartir vangar, &
\edtext{ǫtul vǫ́ro augu \hld\ sem yrmlingi}{\lemma{ǫtul \dots\ yrmlingi ‘fierce \dots\ the young serpent’}\Bfootnote{A person of noble stock being recognised as such through their appearance is a motif in Norse literature. Cf. esp. the incident at the beginning of \HelgakvidaTwo, where Hallow, disguised as a thrall-woman, is almost caught due to his unslavelike eyes, which are, as in the present stanza, likewise said to be \emph{ǫtul} ‘fierce, terrible’.}}.\eva

\bvb Mother begot a swain, swaddled him in silk; \\
they sprinkled him with water—let him be called Earl. \\
Pale was his hair, bright his cheeks, \\
fierce were his eyes, like the young serpent.\evb\evg


\bvg\bva\mssnote{\Wormianus~78v/TODO}Upp óx þar \hld\ Jarl á flętjum; &
lind nam at skęlfa, \hld\ lęggja stręngi, &
alm at bęygja, \hld\ ǫrvar skępta, &
flęin at flęyja, \hld\ frǫkkur dýja, &
hęstum ríða, \hld\ hundum verpa, &
sverðum bregða, \hld\ sund at fręmja.\eva

\bvb Up grew Earl there on the floor-benches; \\
he took to shake shields, fasten bow-strings, \\
bend elmwood, shaft arrows, \\
throw javelins, hoist frankish spears, \\
ride horses, throw hounds (TODO) \\,
brandish swords, practice swimming.\evb\evg


\bvg\bva\mssnote{\Wormianus~78v/TODO}\edtext{Kom þar ór runni \hld\ Rigr gangandi, &
Rigr gangandi, \hld\ rúnar kęnndi; &
sitt gaf hęiti, \hld\ son kveðsk ęiga; &
þann bað hann ęignask \hld\ óðal-vǫllu, &
óðal-vǫllu, \hld\ aldnar bygðir.}{\lemma{Kom \dots\ bygðir.}\Bfootnote{Righ approaches his son, Earl. He reveals himself as his father and initiates him into the warrior aristocracy through teaching him the runes and giving him the noble title Righ (henceforth he will be known as Righ Earl). Finally he instructs him to set out and win land for himself, which Righ Earl soon does.}}\eva

\bvb There came out of a brush Righ, walking: \\
Righ, walking, taught runes; \\
he gave his own name; said that he had a son; \\
he bade \emph{him} take the ethel-plains: \\
the ethel-plains, the ancient villages.\evb\evg


\bvg\bva\mssnote{\Wormianus~78v/TODO}Ręið hann męirr þaðan \hld\ myrkan við &
hélug fjǫll \hld\ unds at hǫllu kom; &
skapt nam at dýja, \hld\ skęlfði lind, &
hęsti hlęypti, \hld\ ok hjǫrvi brá; &
víg nam at vękja, \hld\ vǫll nam at rjóða, &
val nam at fęlla, \hld\ vá til landa.\eva

\bvb He \ken{= Righ-Earl} rode further thence through the mirky wood, \\
through the frosty fells, until to a hall he came— \\
the shaft he took to hoist, shook the linden shield, \\
leapt with the horse, and brandished the sword; \\
war he took to rouse, the plain he took to redden, \\
men he took to fell—he won the land.\evb\evg


\bvg\bva\mssnote{\Wormianus~78v/TODO}Réð hann ęinn at þat \hld\ átján búum; &
auð nam skipta \hld\ ǫllum vęita &
męiðmar ok mǫsma, \hld\ mara svang-rifja; &
\edtrans{hringum hręytti}{rings he scattered}{\Bfootnote{Cf. StarkSt Frag 1/2a \emph{hring-hręytanda} ‘ring-scattererer \ken{generous man}’ which contains the same words.}}, \hld\ hjó sundr baug.\eva

\bvb He alone ruled, after that, eighteen homesteads. \\
Wealth he took to hand out; to give all men \\
gifts and treasures, [and] slender-ribbed steeds; \\
rings he scattered; he cut apart a bigh.\evb\evg


\bvg\bva\mssnote{\Wormianus~78v/TODO}\edtext{Óku}{\Afootnote{\emph{okū} \Wormianus}} ę́rir \hld\ úrgar brautir &
kvǫ́mu at hǫllu \hld\ þar’s hęrsir bjó: &
mǿtti [...] \hld\ \edtext{mjó-fingraðri}{\Afootnote{the grammar requires \emph{-ri}; mjó-fingraði \Wormianus}} &
hvítri ok horskri, \hld\ hétu Ęrna.\eva

\bvb Messengers drove through drizzling roads, \\
came to the hall where a ruler lived; \\
met a slender-fingered, \\
white and wise—they called her Erne.\evb\evg


\bvg\bva\mssnote{\Wormianus~78v/TODO}Bǫ́ðu hęnnar \hld\ ok hęim óku, &
giptu Jarli, \hld\ \edtrans{gekk hón und líni}{she went ’neath the linen}{\Bfootnote{i.e. she donned the bridal veil; cf. \Thrymskvida\ 27.}}; &
saman bjuggu þau \hld\ ok sér unðu, &
ę́ttir jóku \hld\ ok aldrs nutu.\eva

\bvb They asked for her hand and drove home, \\
married her off to Earl—she went under the linen. \\
They settled together and were content with themselves, \\
grew their lineage and enjoyed life.\evb\evg


\bvg\bva\mssnote{\Wormianus~78v/TODO}Burr vas hinn ęlsti, \hld\ en Barn annat; &
Jóð ok Aðal, \hld\ Arfi, Mǫgr, &
Niðr ok Niðjungr, \hld\ (nǫ́mu lęika) &
Sonr ok Svęinn, \hld\ (sund ok tafl) &
Kundr hét ęinn; \hld\ Konr vas hinn yngsti.\eva

\bvb Byre was the oldest, and Bairn another; \\
TODO: Translation. \\
TODO: Translation (they learned to play)
Son and Swain (swimming and Tavel)
Kund was one called; Kin was the youngest.\evb\evg


\bvg\bva\mssnote{\Wormianus~78v/TODO}Upp óxu þar \hld\ Jarli bornir: &
hęsta tǫmðu, \hld\ hlífar bęndu, &
skęyti skófu, \hld\ skęlfðu aska. &
En \edtrans{Konr ungr}{Kin the Young}{\Bfootnote{The name is clearly a folk etymological pun on ON \emph{konungr} ‘king’, who held the highest social rank, above even the earls.}} \hld\ kunni rúnar: &
ę́vin-rúnar \hld\ ok aldr-rúnar.\eva

\bvb There grew up the sons of Earl: \\
horses they tamed, shield-rims they bent, \\
smoothened shafts, shook ash-spears.— \\
But Kin the Young knew runes: \\
ever-runes and life-runes.\evb\evg


\bvg\bva\mssnote{\Wormianus~78v/TODO}Męirr kunni hann \hld\ mǫnnum bjarga, &
ęggjar dęyfa, \hld\ ę́gi lę́gja. &
Klǫk nam fugla, \hld\ kyrra ęlda, &
sǿfa ok svęfja, \hld\ sorgir lę́gja, &
afl ok ęljun \hld\ átta manna.\eva

\bvb Further he knew men to save, \\
blades to dull, the sea to lower. \\
He learned the chirps of birds, to calm fires, \\
to soothe and lull to sleep, to lower sorrows, \\
the strength and zeal of eight men.\evb\evg


\bvg\bva\mssnote{\Wormianus~78v/TODO}Hann við Rig Jarl \hld\ rúnar dęildi; &
brǫgðum bęitti \hld\ ok bętr kunni; &
þá ǫðladisk \hld\ ok þá ęiga gat, &
Rigr at hęita, \hld\ rúnar kunna.\eva

\bvb With Righ-Earl he shared runes; \\
TODO. \\
then he earned for himself, and got to own, \\
Righ to be called, runes to know.\evb\evg


\bvg\bva\mssnote{\Wormianus~78v/TODO}Ręið Konr ungr \hld\ kjǫrr ok skóga; &
kolfi flęygði \hld\ kyrði fugla; &
þá kvað þat kráka \hld\ —sat kvisti ęin— &
„Hvat skalt, Konr ungr, \hld\ kyrra fugla? &
Hęldr mę́tti þér \hld\ hęstum ríða &
{[...]} \hld\ ok hęr fęlla.\eva

\bvb Kin the Young rode through brushes and woods; \\
he flung bolts, he calmed birds. \\
Then quoth a crow—it sat lone on a twig—: \\
“For what shalt thou, Kin the Young, calm birds? \\
Better it fit thee horses to ride, \\
{[...]}, and armies to fell.”\evb\evg


\bvg\bva\mssnote{\Wormianus~78v/TODO}Á Danr ok Danpr \hld\ dýrar hallir; &
ǿðra \edtrans{óðal}{ethel}{\Bfootnote{Ancestral farmland, in this case the eighteen homesteads owned by Earl.}} \hld\ an \edtrans{ér}{ye}{\Afootnote{metr. emend.; \emph{þér} ‘id.’ \Wormianus, which is simply a younger form of \emph{ér}, and shows that the poem has been linguistically modernised.}} hafið; &
þęir kunnu vel \hld\ \edtrans{kjól at riða}{ship to ride}{\Bfootnote{i.e. to sail.}}, &
\edtrans{ęgg at kęnna}{the blade to teach}{\Bfootnote{i.e. to fight, wage war.  Apparently a euphemism; to “teach someone the blade” is to fight him.}}, \hld\ undir rjúfa.\eva

\bvb Dan and Danp own costly halls: \\
nobler ethel than ye do— \\
they know well the ship to ride, \\
the blade to teach, wounds to tear.\evb\evg

\sectionline

At this point leaf 78 ends. The rest of the poem is lost.

\sectionline
% Righ (Homedal/Weden)
	\bookStart{Eddic fragments from Snorre’s Edda}
%TODO: Further discussion on the fragments.

A number of Eddic lines, stanzas and groups of stanzas are quoted in Snorre’s Edda.  The majority of them are taken from longer Eddic poems preserved in full in other manuscripts (primarily \Regius\ and \AM), but a few are found nowhere else.  These fragments will be edited in the present section.

The fragments have some things in common: they are generally pieces of spoken dialogue quoted in the context of longer narrative prose sections, and are, with one exception (Homedal’s galder, see below), not introduced by reference to their source but rather with phrases like \emph{þá kvað hann} ‘then he quoth’.

\sectionline

\section{A lost riddle-poem}

This half-stanza is quoted in \Gylfaginning\ 2, being the second Eddic verse in the text, following \Havamal\ 1 in the same chapter, which is uttered by Yilfer himself when he enters the hall of the Eese. The whole section is clearly referencing other Eddic mythic wisdom contests and particularly reminiscent of \Vafthrudnismal.

\bpg\bpa Hann sá þrjú há-sę́ti ok hvert upp frá ǫðru, ok sátu þrír menn sinn í hverju. Þá spurði hann, hvert nafn hǫfðingja þeira vę́ri. Sá svarar, er hann leiddi inn, at sá, er í inu neðsta hásę́ti sat, var konungr, ok heitir Hárr, en þar nę́st sá, er heitir Jafnhárr, en sá ofast, er Þriði heitir. Þá spyrr Hárr komandann, hvárt fleira er erendi hans, en heimill er matr ok drykkr honum sem ǫllum þar í Háva hǫll. Hann segir, at fyrst vill hann spyrja, ef nǫkkurr er fróðr maðr inni. Hárr segir, at hann komi eigi heill út, nema hann sé fróðari,\epa

\bpb He [= Yilfer] saw three high-seats and each higher than the other, and three men sat there, each in his own seat. Then he asked what the names of those chieftains were. He who led him in answers that the one who sat in the lowest high-seat was a king called High, and next to him he who is called Evenhigh, and uppermost he who is called Third. Then High asks the guest whether he has any other errands, but food and drink will be freely offered him, like all men there in the High One’s hall. He [= Yilfer] asks whether anyone within is a learned man.  High says that he will not come out whole unless he be more learned [than he],\epb\epg

\bvg\bva „ok statt-u \alst{f}ramm \hld\ meðan þú \alst{f}regn &
\ind \alst{s}itja skal \alst{s}á es \alst{s}ęgir.“\eva

\bvb “and stand forth while thou askest; \\
\ind sit shall he who speaks!”\evb\evg

\sectionline

\section{Nearth and Shede}

The following passage is almost the whole of \Gylfaginning\ 23, excepting at the very end \emph{svá er sagt} ‘so it is said’, after which is quoted \Grimnismal\ 11.
Notably, the two stanzas cited here are also found translated in \textcite{Saxo}[1.8.18--19], where they are said to have been spoken by Hadding and Rainhild, respectively.  For discussion \textcite{Hopkins2021}.

\sectionline

\bpg\bpa Inn þriði áss er sá, er kallaðr er Njǫrðr. Hann býr á himni, þar sem heitir Nóatún. Hann rę́ðr fyrir gǫngu vinds ok stillir sjá ok eld. Á hann skal heita til sę́-fara ok til veiða. Hann er svá auðigr ok fé-sę́ll, at hann má gefa þeim auð, landa eða lausa-fjár. Á hann skal til þess heita. Eigi er Njǫrðr ása ę́ttar. Hann var upp fǿddr í Vana-heimi, en Vanir gísluðu hann goðunum ok tóku í mót at gíslingu þann, er Hǿnir heitir. Hann varð at sę́tt með goðum ok Vǫnum. Njǫrðr á þá konu, er Skaði heitir, dóttir Þjatsa jǫtuns. Skaði vill hafa bú-stað þann, er átt hafði faðir hennar, þat er á fjǫllum nǫkkurum, þar sem heitir Þrym-heimr, en Njǫrðr vill vera nę́r sę́. Þau sę́ttust á þat, at þau skyldu vera níu nę́tr í Þrym-heimi, en þá aðrar níu at Nóa-túnum. En er Njǫrðr kom aftr til Nóatúna af fjallinu, þá kvað hann þetta:\epa

\bpb The third Os is that one who is called Nearth. He lives in the heaven in the place called Nowetowns. He rules the course of the wind, and stills sea and fire. On him shall one call for sea-faring and for hunting. He is so wealthy and blessed with money that he may give them a wealth of lands or loose property; on him shall one call for that sake. Nearth is not of the lineage of the Eese. He was brought up in Wanehome, but the Wanes gave him as a hostage to the gods, and in return got as hostage that one who is called Heener. He was used to reconcile the gods and the Wanes. Nearth has that woman who is called Shede, the daughter of the ettin Thedse. Shede wishes to have the dwelling which her father had owned, which lies on some fells in the place called Thrimham—but Nearth wishes to live by the sea. They agreed with each other that they would live for nine nights in Thrimham, but the other nine at Nowetowns. But when Nearth came back to the Nowetowns from the fell, he quoth this:\epb\epg

\bvg\bva „\alst{L}ęið erumk fjǫll, \hld\ vas’k-a \alst{l}ęngi á, &
\ind \alst{n}ę́tr ęinar \alst{n}íu; &
\alst{u}lfa þytr \hld\ mér þótti \alst{i}llr vesa &
\ind hjá \alst{s}ǫngvi \alst{s}vana.“\eva

\bvb “The fells are loathsome to me; I was not long thereon— \\
\ind only for nine nights. \\
The howling of the wolves thought me evil, \\
\ind compared to the song of swans.”\evb\evg

\bpg\bpa Þá kvað Skaði þetta:\epa

\bpb Then Shede quoth this:\epb\epg

\bvg\bva „\alst{S}ofa né mát’k-a’k \hld\ \alst{s}ę́var bęðjum á &
\ind \alst{f}ugls jarmi \alst{f}yrir; &
sá mik \alst{v}ękr \hld\ es af \alst{v}íði kømr &
\ind \alst{m}orgun hvęrjan \alst{m}ár.“\eva

\bvb “I could not sleep on the beds of the sea \\
\ind for the bleating of the bird. \\
He awakes me, when from the wide sea he comes, \\
\ind every morning, the mew.”\evb\evg

\bpg\bpa Þá fór Skaði upp á fjall ok byggði í Þrym-heimi, ok ferr hon mjǫk á skíðum ok með boga ok skýtr dýr. Hon heitir ǫndur-goð eða ǫndur-dís.\epa

\bpb Then Shede went up to the fells and dwelled in Thrimham, and she often goes on skis with her bow and shoots beasts. She is called ski-god or ski-dise.\epb\epg

\sectionline

\section{Homedal’s Galder (\emph{Hęimdallargaldr})}

This mysterious fragment is quoted in \Gylfaginning\ 27, the chapter describing Homedal, which is here reproduced in full. The fragment consists of two c-lines and appears to be the end of a stanza in the fitting meter \Galdralag.

The same poem is mentioned again in \Skaldskaparmal\ 15: \emph{Heimdallar hǫfuð heitir sverð. Svá er sagt, at hann var lostinn manns hǫfði í gegnum. Um þat er kveðit í Heimdallar-galdri, ok er síðan kallat hǫfuð mjǫtuðr Heimdallar} ‘A sword is called Homedal’s head. So is said that he was run through with a man’s head.  About that it is sung in Homedal’s galder, and henceforth the head is called Homedal’s bane.’

\sectionline

\bpg\bpa Heimdallr heitir einn. Hann er kallaðr hvíti áss; hann er mikill ok heilagr. Hann báru at syni meyjar níu ok allar systr; hann heitir ok Hallinskíði ok Gullintanni; tennr hans váru af gulli. Hestr hans heitir Gulltoppr. Hann býr þar er heitir Himinbjǫrg við Bifrǫst; hann er vǫrðr goða ok sitr þar við himins enda at gę́ta brúarinnar fyrir berg-risum. Hann þarf minna svefn en fugl. Hann sér jafnt nótt sem dag hundrað rasta frá sér; hann heyrir ok þat, er gras vex á jǫrðu eða ull á sauðum, ok allt þat er hę́ra lę́tr. Hann hefir lúðr þann er Gjallar-horn heitir, ok heyrir blástr hans í alla heima. Heimdallar sverð er kallat hǫfuð manns. Hér er svá sagt:\epa

\bpb Homedal one is named.  He is called the White Os; he is great and holy.  He was born as the son of nine maidens, sisters all.  He is also named Haldenshid and Goldentooth; his tooth were of gold.  His horse is called Goldtop.  He lives at the place called the Heavenbarrows near Bivrest.  He is the Watchman of the Gods and sits there at Heaven’s end to guard the bridge against barrow-risers.  He needs less sleep than a bird.  Both night and day he sees a hundred rests away from him; he also hear when grass grows on the ground or wool on sheep, and everything which sounds louder. He has the basoon called the Horn of Yell, and his blowing can be heard in all realms.  Homedal’s sword is called a man’s head.  Here it says so:\epb\epg

\sectionline

(Here the text cites \Grimnismal\ 13; see there.)

\sectionline

\bpg\bpa Ok enn segir hann sjalfr í Heimdallar-galdri:\epa

\bpb And further he himself says in Homedal’s Galder:\epb\epg


\bvg\bva „Níu em’k \edtrans{\alst{m}ǿðra}{mothers}{\Afootnote{so \RegiusProse\Trajectinus\Wormianus; \emph{męyja} ‘maidens’ \Upsaliensis}} \alst{m}ǫgr, &
níu em’k \alst{s}ystra \edtrans{\alst{s}onr}{son}{\Afootnote{om. \Trajectinus}}.“\eva

\bvb “Of nine mothers I’m the lad, \\
of nine sisters I’m the son.”\evb\evg

\sectionline

\section{Gna and the Wanes}

The following passage is from \Gylfaginning\ 35, which lists the \inx[G]{Ossens}.

\sectionline

\bpg\bpa Fjórtánda Gná, hana sendir Frigg í ymsa heima at ørindum sínum. Hon á þann hest, er renn lopt ok lǫg, er heitir Hóf-varpnir. Þat var eitt sinn, er hon reið, at vanir nǫkkvǫrir sá reið hennar í loptinu. Þa mę́lti einn:\epa

\bpb The fourteenth is Gna; Frie sends her into every home to do her errands. She owns the horse who runs through air and sea, and is called Hoofwarpner. It was one time when she rode that some Wanes saw her riding in the air. Then one spoke:\epb\epg

\bvg\bva „Hvat þar \alst{f}lýgr, \hld\ hvat þar \alst{f}ęrr, &
\ind eða at \alst{l}opti \alst{l}íðr?“\eva

\bvb “What flies there, what fares there, \\
\ind or passes through the air?”\evb\evg


\bpg\bpa Hon svarar:\epa

\bpb She answers:\epb\epg


\bvg\bva „Né ek \alst{f}lýg, \hld\ þó ek \alst{f}ęr &
\ind ok at \alst{l}opti \alst{l}ið’k &
á \alst{H}óf-varpni, \hld\ þęim’s \alst{H}am-skęrpir &
\ind \alst{g}at við \alst{G}arð-rofu.“\eva

\bvb “I fly not, though I fare, \\
\ind and pass through the air, \\
on Hoofwarpner, whom Hamsherper \\
\ind begot with Yardrove.”\evb\evg


\bpg\bpa Af Gnár nafni er svá kallat, at þat gnę́far, er hátt ferr:\epa

\bpb From Gna’s name it is so called that something which fares high up \emph{protrudes}.\epb\epg

\sectionline

\section{Balder’s Death}

\Gylfaginning\ 49 contains the narrative of Balder’s death, beginning with his ominous dreams, and ending with the Eese failing to “weep him out of Hell” (for a summary and discussion of the myth and its attestations, see the introduction to \Voluspa\ 31–33). At the end of the chapter, a single \Ljodahattr\ speech-stanza is quoted.

\sectionline

\bpg\bpa Því nę́st sendu ę́sir um allan heim ørind-reka at biðja, at Baldr vę́ri grátinn ór Helju, en allir gerðu þat, menninir ok kykvendin ok jǫrðin ok steinarnir ok tré ok allr málmr, svá sem þú munt sét hafa, at þessir lutir gráta, þá er þeir koma ór frosti ok í hita. Þá er sendi-menn fóru heim ok hǫfðu vel rekit sín ørindi, finna þeir í helli nǫkkvǫrum, hvar gýgr sat; hon nefndist Þǫkk. Þeir biðja hana gráta Baldr ór helju, hon segir:\epa

\bpb Next after that the Eese sent an errand-runner through all the \inx[C]{Home}, to ask that Balder be wept out of hell. And all did that, the men and the beasts and the earth and the stones and trees and all bedrock, as thou must have seen, that these things weep when they come out of cold and into heat. When the messengers journeyed home, and had ran their errand well, they find in a certain cave that a \inx[C]{gow} sat there; she called herself Thanks. They ask her to weep Balder out of hell. She says:\epb\epg


\bvg\bva „\alst{Þ}ǫkk mun gráta \hld\ \alst{þ}urrum tǫ́rum &
\ind \alst{B}aldrs \alst{b}ál-farar; &
\alst{k}yks né dauðs \hld\ naut’k-a \alst{K}arls sonar &
\ind \alst{h}afi \alst{H}ęl því’s \alst{h}ęfir.“\eva

\bvb “Thanks will weep–with dry tears \\
\ind for Balder’s pyre-journey \ken{death}. \\
Neither alive nor dead did I benefit from Churl’s son \ken*{= Balder}; \\
\ind let Hell have what she has!”\evb\evg


\bpg\bpa En þess geta menn, at þar hafi verit Loki Laufeyjarson, er flest hefir illt gørt með ásum.\epa

\bpb But men guess that this must have been Lock, Leafy’s son, who has done the most evil among the Eese.\epb\epg

\sectionline

\section{Thunder’s Journey to Garfrith}

\Skaldskaparmal\ 26, here edited in part, is the only surviving retelling of Thunder’s journey to the ettin Garfrith, and his following fight with, and slaying of, him and his two daughters, Yelp and Grope. This was apparently a well-known story, and is also mentioned in Vetrl Lv 1/1b (quoted in \Skaldskaparmal\ 11, which lists kennings for Thunder): \emph{stétt of Gjǫlp dauða} ‘thou didst step over the dead Yelp’.
The prose of \Skaldskaparmal\ 26 seems to be based on an earlier, now-lost poem in \Ljodahattr, from which it quotes two stanzas. The first is found in all four main manuscripts, while the second is found only in \Upsaliensis. Both are spoken by Thunder and closely resemble each other stylistically, which is why they most likely come from the same poem.

\sectionline

\bpg\bpa Þá fór Þórr til ár þeirar, er Vimur heitir, allra á mest. Þá spennti hann sik megin-gjǫrðum ok studdi for-streymis Gríðar-vǫl, en Loki helt undir megin-gjarðar. Ok þá er Þórr kom á miðja ána, þá óx svá mjǫk áin, at uppi braut á ǫxl honum. Þá kvað Þórr þetta:\epa

\bpb Then Thunder journeyed to that river which is called Wimbre, greatest of all rivers. Then he wrapped his might-girdle around himself and leaned upon Grith’s stave against the stream, and Lock held up the might-girdle. And when Thunder came to the middle of the river, then it waxed so great that it broke over his shoulders. Then Thunder quoth this:\epb\epg


\bvg\bva „\alst{V}ax-at-tu nú, \alst{V}imur, \hld\ alls mik þik \alst{v}aða tíðir &
\ind \alst{jǫ}tna garða \alst{í}; &
\alst{v}ęitst, ef þú \alst{v}ęx \hld\ at þá \alst{v}ęx mér ǫ́s-męgin &
\ind jafn-\alst{h}átt upp sem \alst{h}iminn.“\eva

\bvb “Wax not now, O Wimbre, as I wish to wade through thee \\
\ind into the yards of the ettins. \\
Thou knowest, if thou waxest, then my os-might waxes \\
\ind up as high as the heaven.”\evb\evg


\bpg\bpa Þá sér Þórr uppi í gljúfrum nǫkkurum, at Gjálp, dóttir Geirrøðar \edtrans{stóð þar tveim megin árinnar, ok gerði hon ár-vǫxtinn.}{stood on both sides of the river, and she caused the river’s growth}{\Bfootnote{She stood with her legs spread and befouled the river.}} Þá tók Þórr upp ór ánni stein mikinn ok kastaði at henni ok mę́lti svá: „At ósi skal á stemma.“ Eigi missti hann, þar er hann kastaði til, ok í því bili bar hann at landi ok fekk tekit reyni-runn nǫkkurn ok steig svá ór ánni. Því er þat orð-tak haft, at reynir er bjǫrg Þórs.\epa

\bpb Then Thunder sees that up in some certain gorges Yelp, daughter of Garfrith, stood on both sides of the river, and she caused the river’s growth. Then Thunder took up from the river a great stone and threw it at her and spoke so: “At its source shall the river be dammed.” He did not miss his target, and in that moment he threw himself towards land and got hold of a certain rowan shrub, and thus stepped out of the river. From this comes the saying that the rowan is Thunder’s deliverance.\epb\epg


\bpg\bpa En er Þórr kom til Geirrøðar, þá var þeim fé-lǫgum vísat fyrst í geita-hús til her-bergis, ok var þar einn stóll til sę́tis, ok sat Þórr þar. Þá varð hann þess varr, at stóllinn fór undir honum upp at rę́fri. Hann stakk Gríðar-veli upp í raftana ok lét sígast fast á stólinn. Varð þá brestr mikill, ok fylgði skrę́kr. Þar hǫfðu verit undir stólinum dǿtr Geirrøðar, Gjálp ok Greip, ok hafði hann brotit hrygginn í báðum. Þa kvað Þórr:\epa

\bpb And when Thunder came to Garfrith’s home the fellows were first shown into a goathouse for lodgings, and therein one chair was for sitting, and Thunder sat down on it. Then he noticed that the chair beneath him was moving up toward the roof. He thrusted Grith’s stave up against the rafters and made it push firm onto the chair. Then there was a great crack, followed by a shriek; there beneath the chair had been the daughters of Garfrith, Yelp and Grope, and he had broken both their backs. Then Thunder quoth:\epb\epg

\bvg\bva „\alst{Ęi}nu \edtrans{\emph{sinni}}{time}{\Bfootnote{metr. and sens. emend.; om. \Upsaliensis}} \hld\ nęytta’k \alst{a}lls męgins &
\ind \alst{jǫ}tna gǫrðum \alst{í} &
þá’s \alst{G}jǫlp ok \alst{G}ręip, \hld\ dǿtr \alst{G}ęir-raðar, &
\ind vildu \alst{h}ęfja mik til \alst{h}imins.“\eva

\bvb “Only one time I used all my might \\
\ind in the yards of the ettins, \\
when Yelp and Grope, daughters of Garfrith, \\
\ind would lift me to the heaven.”\evb\evg

\sectionline

\section{On the Making of Glapner}

The following stanza about the making of Glapner, the fetter used to bind the Fenrerswolf, is found in the short work on kennings today called the \emph{Little Scalda} (\emph{Lítla skálda}), which text was probably used as a source by Snorre; see further \textcite[129--47]{Males2020}.  A variant of this stanza is transparently paraphrased in \Gylfaginning\ 28: \emph{Hann var gǫrr af sex hlutum: af dyn kattarins ok af skeggi konunnar ok af rótum bjargsins ok af sinum bjarnarins ok af anda fisksins ok af fogls hráka.} ‘It [Glapner] was made of six things: of the cat’s din and of the woman’s beard and of the mountain’s root and of the bear’s sinews and of the fish’s breath and of the fowl’s spittle.’  The two differences—\emph{hráka} ‘spittle’ for \emph{mjǫlk} ‘milk’, and the inverted order of lines 2 and 3—suggest that Snorre had access to a somewhat different version.  It is not attributed to any named poem.

\sectionline

\bvg\bva Ór \alst{k}attar dyn \hld\ ok ór \alst{k}onu skeggi, &
ór \alst{f}isks anda \hld\ ok ór \alst{f}ugla mjǫlk, &
ór \alst{b}ergs rótum \hld\ ok \alst{b}jarnar sinum, &
\ind ór því vas hann \alst{G}leipnir \alst{g}ǫrr.\eva

\bvb “From cat’s din and from woman’s beard; \\
from fish’s breath and from fowls’ milk; \\
from mountain’s roots and bear’s sinews; \\
\ind from this was Glapner made.”\evb\evg

\sectionline
% Eddic fragments

\part{Heroic Poetry of the Codex Regius}% How to handle... Ideally we'd want a parallel edition with the Saw of the Walsings, since it and the heroic section of the CR clearly derive from the same source.
	\bookStart{Lay of Wayland}[Vǫlundarkviða]

\begin{flushright}%
\textbf{Dating} \parencite{Sapp2022}: C10th (0.428)–early C11th (0.475)

\textbf{Meter:} \Fornyrdislag%
\end{flushright}%

\section{Introduction}

The \textbf{Lay of Wayland} (\Volundarkvida) is a psychologically complex, finely wrought poem.

Wayland gets his revenge on the whole royal household. He murders Nithad’s two young sons (affectionately, his “bear-cubs”) and thus ends his male lineage. Likewise he defangs Nithad’s “cunning wife” (she is never called anything else) by reducing her once powerful counsels to cold words; and finally he rapes Beadhild, depriving her of her maidenhood and value in marriage. They are thus reduced to the same state of complete powerlessness as he himself experienced, something clearly seen in the repetition of the adjective \emph{viljalauss} ‘powerless’; in st. 12 it describes Wayland after he wakes in shackles, but in st. 31 Nithad uses it to refer to his own mental state after the deaths of his sons. This sense of hopelessness concludes the poem in Beadhild’s haunting words: “I nowise knew withstand him; I nowise could withstand him.”

From the other versions of the story it is known that Beadhild gave birth to a son, Woody (OE \emph{Wudga}, \ThidreksSaga\ \emph{Viðga}, in Danish ballads \emph{Vidrik Verlandsøn}). He went on to become a great hero, and in the later heroic ballads by far eclipses his father. His birth seems heavily foreshadowed by Wayland forcing Nithad to swear an oath in st. 33, but he is nowhere directly mentioned in the poem, probably for artistic reasons.

Apart from this lay there is one other telling of the full story, namely the Strand of Wayland the Smith in \ThidreksSaga. While written in Old Norse, it is clear from the proper names and content that it is based on German sources (probably heroic ballads). Thus the native form \emph{Vǫlundr} is replaced with the Low German \emph{Velent} [\emph{sic}], \emph{Níðuðr} with \emph{Niðungr}. Interestingly there is a note within it showing that the native form was still known, namely about “Velent, the excellent smith, whom Warrings (\emph{væringjar}) call Wayland (\emph{Vǫlundr})”. Apparently Wayland was so famous that “all men seem to praise his workmanship so, that the maker of any smith’s work which is made better than other works, is called a Wayland (\emph{Vǫlundr}) with regards to workmanship.”

Far more stark than minor differences of language is that of tone. The psychological complexity and tension of the older redaction is almost entirely gone: Wayland is no longer a mysterious wild man, but a chivalrous knight who can escape from any peril through his ingenuity and craftmanship. He is not kidnapped out of Nithad’s greed, nor hamstrung out of the suspicion of his cruel wife, but rather a loyal servant of Nithad’s, banished from the kingdom after defending himself against the king’s corrupt steward, and hamstrung after being caught attempting to poison the king’s food in revenge.

Most frustratingly the personality of Beadhild is entirely expulged. She is the anonymous “king’s daughter”, an unnamed maiden (\emph{jungfrú}, a borrowing from Low German) who is peacefully seduced by Wayland and quickly falls in love with him. Likewise the person of Nithad’s cunning wife is completely gone, and the murder of his sons no longer ends his lineage, since he has another, older son who survives him and takes over the kingdom. Wayland still flies away laughing after telling Nithad what he has done, but only four years (his son with Beadhild is three years old) later reconciliates with Nithad’s son, retrieves Beadhild and their son and lives a long life as a famous craftsman.

Thus, by the time of the \ThidreksSaga\ the old story of Wayland had been heavily distorted, a tragic victim of chivalric sensibilities.  This younger version does not have any high literary value, but is of course still of interest since it shows the wide reception and variation of the narrative.

Finally there are also traces of the story in the Anglo-Saxon tradition, where it is alluded to in both \Waldere\ and \Deor, the latter of which particularly emphasising the powerlessness felt by Wayland and Beadhild (thus being much closer in spirit to the present poem than to \ThidreksSaga). Parts of the narrative are depicted on the early C8th Frank’s casket, where it is as prominent as the depiction of the Adoration of the Magi—a true testament to the weight with which it was regarded within that culture.

\sectionline

\section{From Wayland (\emph{Frá Vǫlundi})}

\bpg\bpa\mssnote{\Regius~18r/4, \AM~6v/26}Níðuðr hét konungr í Svíþjóð.
Hann átti tvá sonu ok eina dóttur; \edtrans{hon hét}{she was called}{\Afootnote{so \Regius; ok hét hon ‘and she was called’ \AM}} Bǫðvildr.
Brǿðr \edtrans{vǫ́ru}{were}{\Afootnote{so \AM; om. \Regius}} þrír, synir Finna konungs. Hét einn Slagfiðr, annarr Egill, þriði Vǫlundr.
Þeir skriðu ok veiddu dýr. Þeir kvǫ́mu í Úlfdali ok gerðu \edtext{sér þar hús.
Þar er vatn, er heitir Úlfsjár.
Snemma of morgin fundu þeir á vatsstrǫndu konur þrjár, ok spunnu lín. Þar váru hjá þeim álftarhamir þeira; þat váru valkyrjur.
Þar váru tvę́r dǿtr Hlǫðvés konungs: Hlaðguðr svanhvít ok Hervǫr alvitr. In þriðja var Ǫlrún \edtext{Kjárs dóttir af Vallandi}{\lemma{Kjárs [\dots] af Vallandi ‘Choser of Walland’}\Bfootnote{I.e. “Cæsar of Rome”; a legendary form of the Roman emperor. See Index.}}.
Þeir hǫfðu þę́r heim til skála með sér. Fekk Egill Ǫlrúnar, en Slagfiðr Svanhvítrar, en Vǫlundr Alvitrar.
Þau bjuggu sjau vetr. Þá flugu þę́r at vitja víga ok kvǫ́mu eigi aptr.
Þá skreið Egill at leita Ǫlrúnar, en Slagfiðr leitaði Svanhvítrar, en Vǫlundr sat í Úlfdǫlum.
Hann var hagastr maðr, svá at menn viti í fornum sǫgum.
Níðuðr konungr lét hann hǫndum taka, svá sem hér er um kveðit:}{\lemma{sér þar hús \dots\ um kveðit ‘for themselves houses \dots\ sung of’}\Afootnote{so \Regius; om. (due to loss of the following foll. in the ms.) \AM}}\epa

\bpb Nithad was a king called in Sweden.
He had two sons and one daughter; she was called Beadhild.
Three brothers were there; the sons of a king of the Finns. One was called Slayfinn, the other Eyel, the third Wayland.
They fared on skis and hunted wild beasts. They came into the Wolfdales and made for themselves houses there.
There is a lake there which is called the Wolfsea.
Early in the morning they found on the lake-shore three women, and they span linen. There were by them their swan-\inx[C]{hame}[hames]; those were Walkirries.
There were two daughters of king Ladwigh: Ladguth Swanwhite and Harware Elwight. The third was Alerune, daughter of \inx[P]{Choser} of \inx[G]{Walland}.
The men took the women to their halls with them.  Eyel got Alerune, and Slayfinn Swanwhite, and Wayland the Elwight.
The couples lived there for seven winters; then the women left to attend battles, and did not come back.
Then Eyel fared on skis to search for Alerune, but Slayfinn searched for Swanwhite—but Wayland stayed in the Wolfdales.
He was the most skilled craftsman whom men know of in the ancient saws. King Nithad had him taken, as it is here sung of:\epb\epg

\sectionline

\section{The Lay of Wayland}

\bvg\bva\mssnote{\Regius~18r/19}\alst{M}ęyjar flugu sunnan \hld\ \edtrans{\alst{M}yrk-við}{Mirkwood}{\Bfootnote{A great border forest, surely referenced for its association with the war-ravaged lands of the Gots and Huns; a natural environment for \inx[G]{Walkirries}.}} í gǫgnum &
\edtrans{\alst{a}l-vitr}{elwights}{\Bfootnote{“Strange beings, foreign wights”, reflecting a hypothetical \emph{*alja-wihtiz}.}} \alst{u}ngar, \hld\ \edtrans{\alst{ø}r-lǫg drýgja;}{fulfill orlay}{\Bfootnote{That is, to fulfill their preordained destinies, and act according to their innate nature as described in P1 and st. 3.  \textcite[103]{MCR2005} and some other editors see these words as a sign of English influence and translate \emph{drýgja ør-lǫg} as “engage in war”, considering \emph{ør-lǫg} a semantic borrowing from the OE \emph{or-leg} which is taken to mean the same as Dutch \emph{oorlog} ‘war’.  This is unneccessary; ON \emph{ør-lǫg} otherwise means ‘fate, destiny’, and so may its OE cognate as seen by the equivalent phrase found in l. 29 of a poem on the Christian Doomsday (TODO?), where a man going to Hell for his sins \emph{þǫnne â tó ealdre \hld\ or·leg dreógeð} ‘then for ever and ever [he] suffers his orlay’.}} &
þę́r á \alst{s}ę́var-strǫnd \hld\ \alst{s}ęttusk at hvílask, &
\alst{d}rósir suð-rǿnar \hld\ \alst{d}ýrt lín spunnu.\eva

\bvb Maidens flew from the south through \inx[L]{Mirkwood} \\
—young elwights—to fulfill \inx[C]{orlay}. \\
They on the lake-shore set down to rest; \\
the southern ladies span costly linen.\evb\evg


\bvg\bva\mssnote{\Regius~18r/21}\alst{Ęi}n nam þęira \hld\ \alst{Ę}gil at vęrja &
\alst{f}ǫgr mę́r \alst{f}ira \hld\ \alst{f}aðmi ljósum; &
ǫnnur vas \alst{S}vanhvít, \hld\ \alst{s}van-fjaðrar dró, &
\edtext{[...]}{\Bfootnote{A line mentioning Slayfinn has probably been lost here.}} &
en hin \alst{þ}riðja \hld\ \alst{þ}ęira systir &
varði \edtrans{\alst{h}vítan}{white}{\Bfootnote{Pale skin being a sign of noble ancestry; cf. 17/3.}} \hld\ \alst{h}als Vǫlundar.\eva

\bvb One of them took to embrace Eyel \\
—the fair maiden among men—in her pale bosom. \\
Second was Swanwhite; her swan-feathers she rustled, \\
{[...]} \\
And the third sister among them \\
embraced the white throat of Wayland.\evb\evg


\bvg\bva\mssnote{\Regius~18r/24}\alst{S}ǫ́tu \alst{s}íðan \hld\ \alst{s}jau vetr at þat, &
en hinn \alst{á}tta \hld\ \alst{a}llan þrǫ́ðu, &
en hinn \alst{n}íunda \hld\ \alst{n}auðr of skilði, &
\alst{m}ęyjar fýstusk \hld\ á \alst{m}yrkvan við, &
\alst{a}l-vitr \alst{u}ngar \hld\ \alst{ø}r-lǫg drýgja.\eva

\bvb They stayed then seven winters after that, \\
and all the eighth they yearned, \\
and the ninth did need divorce them. \\
The maidens longed for the Mirky Wood: \\
the young elwights, to fulfill orlay.\evb\evg


\bvg\bva\mssnote{\Regius~18r/26}Kom þar af \alst{v}ęiði \hld\ \alst{v}eðr-ęygr skyti &
\edtext{Vǫlundr \alst{l}íðandi \hld\ of \alst{l}angan veg,}{\lemma{Vǫlundr \dots\ veg ‘Wayland \dots\ way’}\Afootnote{emend. based on st. 9/3–4; om. \Regius}} &
\alst{S}lagfiðr ok Ęgill, \hld\ \alst{s}ali fundu auða, &
gingu \alst{ú}t ok \alst{i}nn \hld\ ok \alst{u}mb sǫ́usk.\eva

\bvb Came there from the hunt the stormy-eyed shooter: \\
Wayland passing over a long way. \\
Slayfinn and Eyel found the halls deserted; \\
they walked out and in, and looked about.\evb\evg


\bvg\bva\mssnote{\Regius~18r/27}\alst{Au}str skręið \alst{Ę}gill \hld\ at \alst{Ǫ}lrúnu, &
en \alst{s}uðr \alst{S}lagfiðr \hld\ at \alst{S}vanhvítu, &
en \alst{ęi}nn Vǫlundr \hld\ sat í \alst{U}lf-dǫlum.\eva

\bvb East skied Eyel after Alerune, \\
and south Slayfinn after Swanwhite, \\
and alone Wayland stayed in the Wolfdales.\evb\evg


\bvg\bva\mssnote{\Regius~18r/29}Hann sló \alst{g}ull rautt \hld\ við \alst{g}im fastan, &
\alst{l}ukði alla \hld\ \edtrans{\alst{l}inn-baugum}{serpent-bighs}{\Bfootnote{It is unclear whether this word refers to rings actually shaped like snakes or is merely a poetic description of twisted rings.  Archeological examples of the former include the so-called “snake-head rings” (German \emph{Schlangenkopfringe}, Swedish \emph{ormhuvudringar}) from the Migration Period, and the snake- or dragon-shaped armlet from the Wiking Age found in a hoard in Undrom, Ångermanland, northern Sweden (108822 HST). https://samlingar.shm.se/object/5C5658C4-0813-4DFF-947F-E5E4C4BAB965.}} vęl; &
\alst{s}vá bęið hann \hld\ \alst{s}innar ljóssar &
\alst{k}vánar, ef hǫ́num \hld\ \alst{k}oma gęrði.\eva

\bvb He struck red gold by fastened gem; \\
he enclosed all the serpent-\inx[C]{bigh}[bighs] well; \\
so he awaited his own bright wife, \\
if to him she might come.\evb\evg


\bvg\bva\mssnote{\Regius~18r/31}Þat spyrr \alst{N}íðuðr, \hld\ \edtrans{\alst{N}íara}{the Nears}{\Bfootnote{An obscure tribe, perhaps the residents of \emph{Närke}, an ancient province of Sweden. See Index.}} dróttinn, &
at \alst{ęi}nn Vǫlundr \hld\ sat í \alst{U}lf-dǫlum; &
\alst{n}ǫ́ttum fóru sęggir, \hld\ \edtrans{\alst{n}ęglðar vǫ́ru brynjur}{nailed were their byrnies}{\Bfootnote{The “byrnies” here are apparently some kind of costly plate armour.}}, &
\alst{sk}ildir bliku þęira \hld\ við hinn \alst{sk}arða mána.\eva

\bvb This learns Nithad, lord of the \inx[G]{Nears}, \\
that alone Wayland stayed in the Wolfdales. \\
Nightily journeyed warriors—nailed were their byrnies— \\
their shields gleamed by the waning moon.\evb\evg


\bvg\bva\mssnote{\Regius~18r/33}Stigu ór \alst{s}ǫðlum \hld\ at \alst{s}alar gafli, &
\edtext{gingu \alst{i}nn þaðan \hld\ \alst{ę}nd-langan sal}{\lemma{gingu \dots\ sal ‘went \dots\ hall’}\Bfootnote{Formulaic. The fixed variant line \emph{hón/hann inn of gekk \hld\ ęnd-langan sal} ‘he/she inside did go the endlong hall’ (i.e. ‘through the entire length of the hall’, cf. English “livelong”) occurs in three other places: sts. 16 and 30 of the present poem, and st. 3 of \Oddrunargratr. \emph{ęnd-langr salr} ‘endlong hall’ occurs in two additional places: st. 27 of \Thrymskvida\ and st. 3 of \Skirnismal.}}, &
sǫ́u á \alst{b}ast \hld\ \alst{b}auga dręgna, &
\alst{s}jau hundruð allra, \hld\ es sá \alst{s}ęggr átti.\eva

\bvb They stepped off their saddles by the hall’s gables; \\
went thence inside the endlong hall; \\
saw they on a bast-rope bighs drawn up, \\
seven hundred in all, which that man owned.\evb\evg


\bvg\bva\mssnote{\Regius~18v/2}Ok þęir \alst{a}f tóku \hld\ ok þęir \alst{á} létu &
\edtrans{fyr \alst{ęi}nn \alst{ú}tan, \hld\ es \alst{a}f létu}{save for one, which off they slid}{\Bfootnote{This bigh is probably the one mentioned in sts. 17 and 26, since Beadhild has it already when Wayland is brought back after being captured. It may have been kept for its particular beauty. \textcite{FinnurEdda}\ writes (\emph{my translation from the Danish}): “The ring which Nithad kept must have had special properties, and distinguished itself before others.  There is no doubt that the ring is a flight ring; whether this was clear to the poet is however questionable.  This much is certain, that Wayland seems to be able to fly away only after he has got back the ring; that is, the one which Beadhild brings him.”  This is by no means certain.  Wayland was a craftsman of legendary skill and could certainly have built wings for himself without a magical flight-ring.  That is what he does in the Low German version; it is also what happens in the related Daidalos myth.  For both of these see the introduction to the present poem.}}. &
Kom þar af \alst{v}ęiði \hld\ \alst{v}eðr-ęygr skyti &
Vǫlundr \alst{l}íðandi \hld\ of \alst{l}angan veg.\eva

\bvb And they took off and they slid on, \\
save for one which they slid off.— \\
Came there from the hunt the stormy-eyed shooter: \\
Wayland passing over a long way.\evb\evg


\bvg\bva\mssnote{\Regius~18v/4}Gekk hann \alst{b}rúnni \hld\ \alst{b}eru hold stęikja; &
\edtext{\alst{á}r}{\Afootnote{metr. and sens. emend.; \emph{hár} \Regius}} brann hrísi \hld\ \alst{a}ll-þurr fura, &
\alst{v}iðr hinn \alst{v}ind-þurri, \hld\ fyr \alst{V}ǫlundi.\eva

\bvb Went he the brown she-bear’s flesh to roast; \\
in early morning burned the twigs of all-dry pine— \\
the wood wind-dry—before Wayland.\evb\evg


\bvg\bva\mssnote{\Regius~18v/5}Sat á \alst{b}er-fjalli, \hld\ \edtrans{\alst{b}auga talði}{bighs he counted}{\Bfootnote{Wayland’s grief and loneliness are skilfully illustrated by his counting all seven hundred rings, something which had apparently become a habit for him.}}, &
\edtrans{\alst{a}lfa ljóði}{prince of elves}{\Bfootnote{Probably referring to Wayland’s nature as a Wild Man, something also seen by his hunting of bears, skiing, and fierce gaze, all associated with his Finnish or Saami ancestry.  Cf. 14/2b and 32/1b, where Nithad calls him \emph{vísi alfa} ‘chief of elves’.}} \hld\ \alst{ęi}ns saknaði; &
\alst{h}ugði at \alst{h}ęfði \hld\ \alst{H}lǫðvés dóttir, &
\alst{a}l-vitr \alst{u}nga \hld\ vę́ri \alst{a}ptr komin.\eva

\bvb Sat he on the bear-pelt, bighs he counted— \\
the prince of elves was missing one! \\
Thought he that Ladwigh’s daughter \ken*{= Harware} might have it, \\
that the young elwight might be come back.\evb\evg


\bvg\bva\mssnote{\Regius~18v/7}\alst{S}at \alst{s}vá lęngi, \hld\ at \alst{s}ofnaði, &
ok \alst{v}aknaði \hld\ \alst{v}ilja-lauss; &
vissi sér á \alst{h}ǫndum \hld\ \alst{h}ǫfgar nauðir, &
en á \alst{f}ótum \hld\ \alst{f}jǫtur of spęnntan.\eva

\bvb Sat he so long that asleep he fell, \\
and he awoke, powerless. \\
He knew on his hands heavy restraints, \\
and on his feet a fetter tight.\evb\evg


\bvg\bva\mssnote{\Regius~18v/9}\speakernote{[Vǫlundr kvað:]}%
„Hvęrir ’ru \alst{jǫ}frar \hld\ þęir’s \alst{á} lǫgðu &
\alst{b}ęsti-síma \hld\ ok \alst{b}undu mik?“\eva

\bvb\speakernoteb{[Wayland quoth:]}%
“Which are the princes that laid on \\
the bast-cordage, and bound me?”\evb\evg


\bvg\bva\mssnote{\Regius~18v/10}Kallaði \alst{n}ú \alst{N}íðuðr, \hld\ \alst{N}íara dróttinn: &
„Hvar gatst, \alst{V}ǫlundr, \hld\ \alst{v}ísi alfa, &
\alst{ó}ra \alst{au}ra, \hld\ í \alst{U}lf-dǫlum? &
\alst{G}ull vas þar ęigi \hld\ á \alst{G}rana lęiðu, &
\alst{f}jarri hugða’k várt land \hld\ \alst{f}jǫllum Rínar.“\eva

\bvb Now called Nithad, lord of the Nears: \\
“Where didst thou, Wayland, chief of elves, \\
get \emph{our} ounces in the Wolfdales? \\
Gold was there not on \inx[P]{Grane}’s path; \\
far I thought our land from the fells of the Rhine.\footnoteB{Grane was the horse of the legendary hero \inx[P]{Siward}, who slew the dragon \inx[P]{Fathomer} and took his gold.  Nithad’s speech is sarcastic: “Is there a dragon’s hoard in the Wolfdales?”}”\evb\evg


\bvg\bva\mssnote{\Regius~18v/13}\speakernote{[Vǫlundr kvað:]}%
„\alst{M}an’k at \alst{m}ęiri \hld\ \alst{m}ę́ti ǫ́ttum, &
es vér \alst{h}ęil \alst{h}jú \hld\ \alst{h}ęima vǫ́rum: &
\alst{H}laðguðr ok \alst{H}ęrvǫr \hld\ borin vas \alst{H}lǫðvé, &
\alst{k}unn vas Ǫlrún \hld\ \alst{K}íars dóttir.“\eva

\bvb\speakernoteb{[Wayland quoth:]}%
“I recall that we owned greater wealth \\
when we a whole household were at home. \\
Ladguth and Harware were born to Ladwigh; \\
known was Alerune, Choser’s daughter.”\footnoteB{Wayland responds rather cryptically and almost seems to be speaking to himself.  By asserting the noble lineages of the three swan-wives he gives a legitimate origin for his wealth, but he is aware that Nithad neither believes him nor cares.}\evb\evg

\sectionline

\bvg\bva\mssnote{\Regius~18v/15}%
\edtext{Úti stóð \alst{k}unnig \hld\ \alst{k}vǫ́n Níðaðar,}{\lemma{Úti \dots\ Níðaðar ‘Outside \dots\ of Nithad’}\Afootnote{emend. based on st. 30/1–2; om. \Regius}} &
\edtext{hón \alst{i}nn of gekk \hld\ \alst{ę}nd-langan sal}{\lemma{hón \dots\ sal ‘she went \dots\ hall’}\Bfootnote{Formulaic, also occuring in st. 30 of the present poem and in \Oddrunargratr\ 3.}}, &
\alst{st}óð á golfi, \hld\ \alst{st}ilti rǫddu: &
„es-a sá nú \alst{h}ýrr, \hld\ es ór \alst{h}olti fęrr.“\eva

\bvb Outside stood the cunning wife of Nithad; \\
she went inside the endlong hall, \\
stood on the floor, steered her voice: \\
“He is not mild now, who comes out of the wood.”\evb\evg


\bpg\bpa\mssnote{\Regius~18v/16}%
Níðuðr konungr gaf dóttur sinni Bǫðvildi gull-hring þann er hann tók af bastinu at Vǫlundar, en hann sjalfr bar sverðit er Vǫlundr átti. En dróttning kvað:\epa

\bpb King Nithad gave his daughter Beadhild the golden ring which he took from the bast rope in Wayland’s hall, but he himself carried the sword which Wayland had owned. And the queen quoth:\epb\epg


\bvg\bva\mssnote{\Regius~18v/19}\alst{T}ęnn hǫ́num \alst{t}ęygjask \hld\ es hǫ́num ’s \alst{t}ét sverð, &
ok hann \alst{B}ǫðvildar \hld\ \alst{b}aug of þękkir, &
\alst{ǫ́}mun eru \alst{au}gu \hld\ \alst{o}rmi hinum frána; &
\alst{s}níðið ér hann \hld\ \alst{s}ina magni, &
ok \alst{s}ętið hann \alst{s}íðan \hld\ í \alst{S}ę́varstǫð.“\eva

\bvb His teeth are bared when he is shown the sword, \\
and Beadhild’s bigh he recognizes; \\
reminiscent are his eyes to the gleaming serpent’s. \\
Snithe ye from him the might of his sinews, \\
and set him thereafter on Seastead!”\evb\evg


\bpg\bpa\mssnote{\Regius~18v/21}%
Svá var gǫrt, at skornar váru sinar í knés-fótum ok settr í holm einn, er þar var fyrir landi, er hét Sę́varstaðr. Þar smíðaði hann konungi alls-kyns gǫr-simar; engi maðr þorði at fara til hans, nema konungr einn. Vǫlundr kvað:\epa

\bpb So it was done that the sinews in his houghs were cut, and he was placed on the lonely islet which there lay before the land, which was called Seastead. There he forged for the king every kind of jewelry.  No man dared go to him save the king alone.  Wayland quoth:\epb\epg


\bvg\bva\mssnote{\Regius~18v/24}%
„\edtrans{Skínn}{shines}{\Bfootnote{Metrically deficient, since \emph{sk-} and \emph{s-} cannot alliterate.  A possible emendation is \emph{se’k} ‘I see’.}} Níðaði \hld\ sverð á linda, &
þat’s ek \alst{h}vęsta \hld\ sęm \alst{h}agast kunna’k &
ok ek \alst{h}ęrða’k \hld\ sęm \alst{h}ǿgst þótti; &
sá ’s mér \alst{f}ránn mę́kir \hld\ ę́ \alst{f}jarri borinn; &
\alst{s}é’k-a þann Vǫlundi \hld\ til \alst{s}miðju borinn.\eva

\bvb “The sword shines on Nithad’s belt, \\
which I sharpened as most handily I could, \\
and I hardened as most pleasingly seemed. \\
That gleaming blade is ever further from me carried; \\
I see it not for Wayland to the smithy carried!\evb\evg


\bvg\bva\mssnote{\Regius~18v/27}%
Nú \alst{b}err \alst{B}ǫðvildr \hld\ \alst{b}rúðar minnar &
—\alst{b}íð’k-a þęss \alst{b}ót— \hld\ \alst{b}auga rauða.“\eva

\bvb Now does Beadhild bear my bride’s \\
—I await no recompense for that—red bighs.”\evb\evg


\bvg\bva\mssnote{\Regius~18v/28}%
\edtrans{\alst{S}at—né \alst{s}vaf á-valt—}{He sat—never slept—}{\Bfootnote{Compare \Gudrunarhvot\ TODO: \emph{hófu mik—né drękkðu—} ‘they lifted me—they drowned [me] not—’.}} \hld\ ok \alst{s}ló hamri; &
vél gęrði \alst{h}ęldr \hld\ \alst{h}vatt Níðaði; &
\alst{d}rifu ungir tvęir \hld\ á \alst{d}ýr séa &
\alst{s}ynir Níðaðar \hld\ í \alst{S}ę́varstǫð.\eva

\bvb He sat—never slept—and struck the hammer; \\
wiles he most boldly planned for Nithad. \\
Two young ones were drifting to see costly things: \\
Nithad’s sons, to Seastead.\evb\evg


\bvg\bva\mssnote{\Regius~18v/30}%
\alst{K}vǫ́mu til \alst{k}istu, \hld\ \alst{k}rǫfðu lukla, &
\alst{o}pin vas \alst{i}ll-úð, \hld\ es þęir \alst{í} sǫ́u, &
fjǫlð vas þar \alst{m}ęina, \hld\ es \alst{m}ǫgum sýndisk &
at vę́ri \alst{g}ull rautt \hld\ ok \alst{g}ǫr-simar.\eva

\bvb Came they to the chest, demanded the keys; \\
open was the evil when inside they saw. \\
A host was there of harms, which to the lads seemed \\
like were they red gold and jewelry.\evb\evg


\bvg\bva\mssnote{\Regius~18v/33}\speakernote{[Vǫlundr kvað:]}%
„Komið \alst{ęi}nir tvęir, \hld\ komið \alst{a}nnars dags; &
ykkr lę́t’k þat \alst{g}ull \hld\ of \alst{g}efit verða; &
\alst{s}ęgið-a męyjum \hld\ né \alst{s}al-þjóðum, &
\alst{m}anni øngum, \hld\ at \alst{m}ik fyndið.“\eva

\bvb\speakernoteb{[Wayland quoth:]}%
“Come alone ye two, come another day; \\
to you, I say, this gold will be given. \\
Tell no maidens nor hall-folk \\
—not a man!—that \emph{me} ye met.”\evb\evg


\bvg\bva\mssnote{\Regius~19r/1}\alst{S}nimma kallaði \hld\ \alst{s}ęggr á annan, &
\alst{b}róðir á \alst{b}róður: \hld\ „gǫngum \alst{b}aug séa!“ &
\alst{K}vǫ́mu til \alst{k}istu, \hld\ \alst{k}rǫfðu lukla, &
\alst{o}pin vas \alst{i}ll-úð \hld\ es þęir \alst{í} litu.\eva

\bvb Early called one youth to another, \\
brother to brother: “Let us go see the bighs!” \\
Came they to the chest, demanded the keys; \\
open was the evil when inside they looked.\evb\evg


\bvg\bva\mssnote{\Regius~19r/3}Snęið af \alst{h}ǫfuð \hld\ \edtrans{\alst{h}úna}{bear-cubs}{\Bfootnote{An affectionate term for young boys, perhaps relating to warrior-initiations done in bear-skins.  This word is repeated by Nithad in st. 32 and mirrored by Wayland in st. 34.}} þęira &
ok und \edtrans{\alst{f}ęn \alst{f}jǫturs}{the fetter’s fen}{\Bfootnote{Unclear.  The smithy or islet may be Wayland’s “fetter”, in which case he buried them in a fen on the island.}} \hld\ \alst{f}ǿtr of lagði, &
ęn \edtrans{þę́r \alst{sk}álar, \hld\ es und \alst{sk}ǫrum vǫ́ru}{those bowls which were under their curls}{\Bfootnote{i.e. their skulls.}}, &
\alst{s}vęip útan \alst{s}ilfri, \hld\ \alst{s}ęldi Níðaði.\eva

\bvb He sliced off the heads of those bear-cubs, \\
and under the fetter’s fen their feet he laid. \\
And the bowls which were under their curls \\
he coated with silver, gave to Nithad.\evb\evg


\bvg\bva\mssnote{\Regius~19r/5}\alst{E}n ór \alst{au}gum \hld\ \edtrans{\alst{ja}rkna-stęina}{arkenstones}{\Bfootnote{Probably round crystals.}} &
sęndi \alst{k}unnigri \hld\ \alst{k}vǫ́n Níðaðar; &
en ór \alst{t}ǫnnum \hld\ \alst{t}vęggja þęira &
\alst{s}ló brjóst-kringlur, \hld\ \alst{s}ęndi Bǫðvildi.\eva

\bvb And from the eyes arkenstones \\
he sent to the cunning wife of Nithad. \\
And from the teeth of the two \\
he struck breast-brooches, sent to Beadhild.\evb\evg

\sectionline

{\small Something appears to be missing here, but the narrative can be gleaned.  Beadhild breaks the bigh given to her by Nithad (mentioned above in sts. 10—see note there—and 17), and fears her father’s anger.  She goes to Wayland in secret and asks him to mend it.  The sight of this ring reminds Wayland of his wife, and he decides to rape Beadhild.}

\sectionline

\bvg\bva\mssnote{\Regius~19r/7}%
Þá nam \alst{B}ǫðvildr \hld\ \alst{b}augi at hrósa &
\edtext{[...]}{\Bfootnote{The meter requires a half-line here, perhaps containing a repetition of 1a: \emph{baugi at hrósa} ‘the bigh to praise’.}}\ \hld\ es brotit hafði, &
„\alst{þ}ori’g-a’k sęgja, \hld\ nema \alst{þ}ér ęinum.“\eva

\bvb Then Beadhild began the bigh to praise, \\
{[...]} which she had broken, \\
“I dare not tell, save to thee alone.”\evb\evg


\bvg\bva\mssnote{\Regius~19r/8}\speakernote{Vǫlundr kvað:}%
„Ek \alst{b}ǿti svá \hld\ \alst{b}rest á gulli, &
at \alst{f}ęðr þínum \hld\ \alst{f}ęgri þykkir, &
ok \alst{m}ǿðr þinni \hld\ \alst{m}iklu bętri, &
ok \alst{s}jalfri þér \hld\ at \alst{s}ama hófi.“\eva

\bvb\speakernoteb{Wayland quoth:}
“I will so mend the crack on the gold, \\
that to thy father it fairer seems, \\
and to thy mother even better, \\
and to thyself of the same rank.”\evb\evg


\bvg\bva\mssnote{\Regius~19r/10}\alst{B}ar hána \alst{b}jóri, \hld\ \edtrans{því-at \alst{b}ętr kunni}{for he knew better}{\Bfootnote{i.e. he was more cunning than her.}}, &
\alst{s}vá’t hǫ́n í \alst{s}essi \hld\ of \alst{s}ofnaði. &
„Nú \alst{h}ęfi’k \alst{h}ęfnt \hld\ \alst{h}arma minna &
\alst{a}llra \edtrans{nema \alst{ę}inna}{save one}{\Bfootnote{Presumably the deprivation of his mobility due to the hamstringing, which he resolves by crafting his flight suit.}} \hld\ \edtrans{\alst{í}-við-gjarna}{insidious ones}{\Bfootnote{King Nithad and his house.}}.“\eva

\bvb He overcame her with beer—for he knew better— \\
so that she in the seat did fall asleep. \\
“Now have I avenged my harms, \\
all, save one, on the insidious ones.”\evb\evg

\sectionline

\bvg\bva\mssnote{\Regius~19r/12}„\alst{V}ęl ek,“ kvað \alst{V}ǫlundr, \hld\ „\alst{v}erða’k á \edtrans{fitjum}{paddles}{\Bfootnote{\CV: \emph{fit} ‘the webbed foot of water-birds’, here a reference to the flight-suit which allows Wayland to regain his freedom.}}, &
þęim’s mik \alst{N}íðaðar \hld\ \alst{n}ǫ́mu rekkar.“ &
\alst{H}lę́jandi Vǫlundr \hld\ \alst{h}ófsk at lopti, &
\alst{g}rátandi Bǫðvildr \hld\ \alst{g}ekk ór ęyju. &
tregði \alst{f}ǫr \alst{f}riðils \hld\ ok \alst{f}ǫður ręiði.\eva

\bvb “Well I”, quoth Wayland, “fall on my paddles; \\
those of which Nithad’s men bereaved me!” \\
Laughing, Wayland threw himself in the air; \\
weeping, Beadhild went from the island, \\
grieved the lover’s flight and the father’s wrath.\evb\evg

\sectionline

\bvg\bva\mssnote{\Regius~19r/14}%
Úti stęndr \alst{k}unnig \hld\ \alst{k}vǫ́n Níðaðar, &
ok hón \alst{i}nn of gekk \hld\ \alst{ę}nd-langan sal, &
en hann á \alst{s}al-garð \hld\ \alst{s}ęttisk at hvílask, &
„Vakir þú \alst{N}íðuðr, \hld\ \alst{N}íara dróttinn?“\eva

\bvb Outside stands the cunning wife of Nithad, \\
and she inside did go the endlong hall. \\
But he on the courtyard set down to rest. \\
“Art thou awake, O Nithad, lord of the Nears?”\evb\evg


\bvg\bva\mssnote{\Regius~19r/17}\speakernote{[Níðuðr kvað:]}%
„\edtrans{\alst{V}aki’k á-valt \hld\ \alst{v}ilja-lauss}{I am always awake, powerless}{\Bfootnote{This line references sts. 12 and 20, but there Wayland was the powerless man who never slept.  By his revenge the suffering has been transferred onto Nithad.}}, &
\alst{s}ofna’k minst, \hld\ síðst \alst{s}onu dauða, &
\alst{k}ęll mik í hǫfuð, \hld\ \edtrans{\alst{k}ǫld erumk rǫ́ð þín}{cold seem thy counsels}{\Bfootnote{A severe insult to a woman of power, for such counsels to her husband was how she would influence worldly affairs.  In this way Wayland’s revenge reaches also Nithad’s wife.}}, &
\alst{v}ilnumk þęss nú, \hld\ at við \alst{V}ǫlund dǿma’k.“\eva

\bvb\speakernoteb{[Nithad quoth:]}%
“I am always awake, powerless; \\
I sleep the least since my sons died. \\
My head turns cold; cold seem thy counsels— \\
I would now but that I with Wayland may speak.”\evb\evg

\sectionline

\bvg\bva\mssnote{\Regius~19r/19}\speakernote{[Níðuðr kvað:]}%
„Sęg mér þat \alst{V}ǫlundr, \hld\ \alst{v}ísi alfa, &
af \alst{h}ęilum \alst{h}vat varð \hld\ \alst{h}únum mínum?“\eva

\bvb\speakernoteb{[Nithad quoth:]}%
“Tell me this, O Wayland, chief of elves: \\
what became of my healthy bear-cubs?”\evb\evg


\bvg\bva\mssnote{\Regius~19r/20}\speakernote{[Vǫlundr kvað:]}%
„\alst{Ęi}ða skalt mér \alst{á}ðr \hld\ \alst{a}lla vinna, &
\edtext{at \alst{sk}ips borði \hld\ ok at \alst{sk}jaldar rǫnd, &
at \alst{m}ars bǿgi \hld\ ok at \alst{m}ę́kis ęgg}{\lemma{at skips \dots\ ęgg ‘by deck \dots\ of sword’}\Bfootnote{Nithad must swear the oaths by his tools of trade as a warrior; by extension on his martial honour.  Cf. \HelgakvidaTwo, where broken oaths are to come back “biting” the oath-breaker by cursing his ship, horse, and sword, in that order.}} &
at þú \edtrans{\alst{k}vęlj-at}{shalt not torment}{\Bfootnote{A negative imperative.  The normal 2nd. sg. imper. of \emph{kvęlja} is \emph{kvęl}, but the negative clitic \emph{-at} causes the \emph{-j-} to reappear in a rare \emph{liaison} effect.  See Rosenberg (2024): “A Norse sandhi?” (TODO: add to bibliography).}} \hld\ \edtext{\alst{k}vǫ́n Vǫlundar, &
né \alst{b}rúði minni}{\lemma{kvǫ́n Vǫlundar ‘wife of Wayland’, brúði minni ‘my bride’}\Bfootnote{Beadhild, who is now pregnant.}} \hld\ at \alst{b}ana verðir, &
þótt kvǫ́n \alst{ęi}gim, \hld\ þá’s \alst{é}r kunnið, &
eða \alst{jó}ð \alst{ęi}gim \hld\ \alst{i}nnan hallar.\eva

\bvb\speakernoteb{[Wayland quoth:]}%
“Oaths shalt thou first all swear to me— \\
by the ship’s wall and the shield’s rim, \\
by the steed’s bough and the sword’s edge— \\
that thou shalt not torment the wife of Wayland, \\
nor of my bride become the bane, \\
though a wife we might own whom ye might know; \\
or a babe might own within the hall.\evb\evg


\bvg\bva\mssnote{\Regius~19r/24}%
\alst{G}akk til smiðju, \hld\ þęirar’s \alst{g}ørðir, &
þar fiðr \alst{b}ęlgi \hld\ \alst{b}lóði stokna, &
snęið’k af \alst{h}ǫfuð \hld\ \alst{h}úna þinna &
ok und \alst{f}ęn \alst{f}jǫturs \hld\ \alst{f}ǿtr of lagða’k.\eva

\bvb Go to the smithy which thou madest; \\
there wilt thou find bellows blood-besprinkled. \\
I sliced off the heads of thy bear-cubs, \\
and under the fetter’s fen their feet I laid.\evb\evg


\bvg\bva\mssnote{\Regius~19r/26}%
En þę́r \alst{sk}álar, \hld\ es und \alst{sk}ǫrum vǫ́ru, &
\alst{s}vęip’k útan \alst{s}ilfri, \hld\ \alst{s}ęlda’k Níðaði, &
\alst{e}n ór \alst{au}gum \hld\ \alst{ja}rkna-stęina, &
sęnda’k \alst{k}unnigri \hld\ \alst{k}vǫ́n Níðaðar.\eva

\bvb And the bowls which were under their curls, \\
I coated with silver, gave to Nithad. \\
And from the eyes arkenstones \\
I sent to the cunning wife of Nithad.\evb\evg


\bvg\bva\mssnote{\Regius~19r/28}%
En ór \alst{t}ǫnnum \hld\ \alst{t}vęggja þęira &
\alst{s}ló’k brjóst-kringlur, \hld\ \alst{s}ęnda’k Bǫðvildi; &
nú gęngr \alst{B}ǫðvildr \hld\ \alst{b}arni aukin, &
\edtrans{\alst{ęi}nga dóttir \hld\ \alst{y}kkur bęggja.}{the only daughter of you both}{\Bfootnote{Formulaic, near-identical to \HervararSaga\ st. 25/1–2: (\emph{Vaki, Angantýr, \hld\ vękr þik Hęrvǫr, // ęinga dóttir \hld\ ykkur Svǫ́fu.} ‘Wake, Ongentew: Harware awakes thee, the only daughter of thee and Sweve.’ Cf. also \Beowulf\ 375a, 2997b: \emph{ángan dohtor} ‘only daughter (accusative)’.)}}“\eva

\bvb And from the teeth of the two \\
I struck breast-brooches, sent to Beadhild. \\
Now goes Beadhild swollen with child; \\
the only daughter of you both.”\evb\evg


\bvg\bva\mssnote{\Regius~19r/30}\speakernote{[Níðuðr kvað:]}%
„\alst{M}ę́ltir-a þat \alst{m}ál, \hld\ es mik \alst{m}ęirr tregi, &
né þik \alst{v}ilja’k \alst{V}ǫlundr \hld\ \alst{v}err of níta; &
es-at svá maðr \alst{h}ǫ́r, \hld\ at þik af \alst{h}ęsti taki, &
\alst{n}é svá ǫflugr, \hld\ at þik \alst{n}eðan skjóti, &
þar’s þú \alst{sk}ollir \hld\ við \alst{sk}ý uppi.“\eva

\bvb\speakernoteb{[Nithad quoth:]}%
“Thou couldst not have spoken a speech which would grieve me more; \\
nor could I worse wish, Wayland, to deny thee. \\
There is no man so high that he might take thee from a horse, \\
nor so strong that he might shoot thee from below, \\
where thou dost jeer by the clouds above!”\evb\evg


\bvg\bva\mssnote{\Regius~19v/1}%
\alst{H}lę́jandi Vǫlundr \hld\ \alst{h}ófsk at lopti, &
en \alst{ó}-kátr Níðuðr \hld\ sat þá \alst{ę}ptir.\eva

\bvb Laughing, Wayland threw himself in the air; \\
but, gloomy, Nithad stayed behind.\evb\evg

\sectionline

\bvg\bva\mssnote{\Regius~19v/2}\speakernote{[Níðuðr kvað:]}%
„Upp rís \edtrans{\alst{Þ}akkráðr}{Thankred}{\Bfootnote{A German name never found elsewhere in ON, but equivalent to MHG \emph{Dancrát}.}}, \hld\ \alst{þ}rę́ll minn batsti, &
\alst{b}ið \alst{B}ǫðvildi, \hld\ \edtext{męy hina \alst{b}rá-hvítu, &
gangi \alst{f}agr-varið}{\lemma{męy hina brá-hvítu \dots\ fagr-varið ‘the brow-white maiden \dots\ fair-clothed’}\Bfootnote{Nithad still has some doubt in his heart and by these words tries to convince himself of the innocence of his daughter (\emph{mę́r} ‘maiden, virgin’).}} \hld\ við \alst{f}ǫður rǿða.“\eva

\bvb\speakernoteb{[Nithad quoth:]}%
“Rise up, Thankred, my best thrall; \\
bid Beadhild, the brow-white maiden, \\
to go, fair-clothed, with her father to counsel.”\evb\evg

\sectionline

\bvg\bva\mssnote{\Regius~19v/3}\speakernote{[Níðuðr kvað:]}%
„Es þat \alst{s}att Bǫðvildr, \hld\ es \alst{s}ǫgðu mér, &
\alst{s}ǫ́tuð it Vǫlundr \hld\ \alst{s}aman í holmi?“\eva

\bvb\speakernoteb{[Nithad quoth:]}%
“Is it true, Beadhild, as they told me— \\
stayed thou and Wayland together on the islet?”\evb\evg


\bvg\bva\mssnote{\Regius~19v/4}\speakernote{[Bǫðvildr kvað:]}%
„\alst{S}att ’s þat Níðuðr \hld\ es \edtrans{\alst{s}agði}{\emph{he} told}{\Bfootnote{Beadhild knows that Wayland is the only one aware of the rape and thus deduces that \emph{he} told her father.  She makes a subtle change in the conjugation from her father’s general third person plural (“what they told”), to the specific singular form (“what \emph{he} told”).}} þér: &
\alst{s}ǫ́tum vit Vǫlundr \hld\ \alst{s}aman í holmi &
\alst{ęi}na \alst{ǫ}gur-stund, \hld\ \alst{ę́}va skyldi; &
ek \alst{v}ę́tr hǫ́num \hld\ \edtext{\alst{v}inna}{\Afootnote{metr. and sens. emend.; om. \Regius}} \edtext{kunna’k, &
ek \alst{v}ę́tr hǫ́num \hld\ \alst{v}inna mátta’k}{\lemma{kunna’k ‘knew’, mátta’k ‘could’}\Bfootnote{Beadhild could defend herself neither mentally (\emph{kunna} ‘to know, understand’) nor physically (\emph{mega} ‘to have strength to do, avail’).  A powerful final stanza.}}.“\eva

\bvb\speakernoteb{[Beadhild quoth:]}%
“True it is, Nithad, as \emph{he} told thee— \\
I and Wayland stayed together on the islet \\
for one heavy hour—it should never have been. \\
I nowise knew withstand him; \\
I nowise could withstand him.”\evb\evg

\sectionline
%
	\bookStart{First Lay of Hallow Hundingsbane}[Helgakviða Hundingsbana fyrsta]

\begin{flushright}%
\textbf{Dating} \parencite{Sapp2022}: late C12th (0.805)

\textbf{Meter:} \Fornyrdislag%
\end{flushright}%

This rather late poem is very well written.  Particularly beautiful are the introductory stanzas, which tell of Norns arriving in the night to predetermine Hallow’s life.

\sectionline

\bpg\bpa Hér hefr upp kvę́ði frá Helga Hundings bana, þeira ok Hǫðbrodds. Vǫlsunga kviða.\epa

\bpb Here begins a lay regarding Hallow, bane of Hunding and his men, and of Hathbrod. A lay of the Walsings.\epb\epg

\sectionline

\bvg\bva\mssnote{\Regius~20r/21}%
\edtrans{\alst{Á}r vas \alst{a}lda}{It was the dawn of elds}{\Bfootnote{This formulaic introduction immediately situates the events of the poem in the distant mytho-heroic past, indeed, if one compares \Voluspa\ 3, at the beginning of history.}} \hld\ þat’s \alst{a}rar gullu &
\alst{h}nigu \alst{h}ęilǫg vǫtn \hld\ af \alst{H}imin-fjǫllum; &
þá \alst{h}afði \alst{H}ęlga \hld\ inn \alst{h}ugum stóra &
\alst{B}orghildr \alst{b}orit \hld\ í \alst{B}rálundi.\eva

\bvb It was the dawn of \inx[C]{eld}[elds], when eagles shrieked; \\
holy waters poured down from the Heavenfells; \\
then to Hallow the great of heart \\
Burhild in Browlund had given birth.\evb\evg


\bvg\bva\mssnote{\Regius~20r/23}%
\alst{N}ǫ́tt varð í bǿ, \hld\ \alst{n}ornir kvǫ́mu, &
þę́r’s \alst{ǫ}ðlingi \hld\ \alst{a}ldr of skópu; &
þann bǫ́ðu \alst{f}ylki \hld\ \alst{f}rę́gstan verða &
ok \alst{b}uðlunga \hld\ \alst{b}ętstan þykkja.\eva

\bvb It turned night in the settlement; norns did come, \\
they who shaped the athling’s age. \\
They bade that battle-arrayer become the noblest, \\
and among princes seem the best.\evb\evg


\bvg\bva\mssnote{\Regius~20r/25}%
Sneru þę́r af \alst{a}fli \hld\ \alst{ø}r·lǫg-þǫ́ttu &
þá’s \alst{b}orgir \alst{b}raut \hld\ í \alst{B}rálundi; &
þę́r um \alst{g}ręiddu \hld\ \alst{g}ullin-símu &
ok und \alst{m}ána sal \hld\ \alst{m}iðjan fęstu.\eva

\bvb They turned mightily orlay-strands \\
when castles were broken in Browlund. \\
They wrapped a golden band, \\
and beneath the moon’s hall \ken{sky/heaven} fastened it in the middle.\evb\evg


\bvg\bva\mssnote{\Regius~20r/27}%
Þę́r \alst{au}str ok vestr \hld\ \alst{ę}nda fǫ́lu, &
þar átti \alst{l}ofðungr \hld\ \alst{l}and á milli, &
brá \alst{n}ipt \alst{N}era \hld\ á \alst{n}orðr-vega &
\alst{ę}inni fęsti, \hld\ \alst{ęy} bað hon halda.\eva

\bvb They in the east and west hid its ends; \\
there the praised one owned land in between. \\
The kinswoman of Nare tugged onto the northern ways \\
a single cord—she bade it hold forever.\evb\evg

TODO: more stanzas.

\sectionline
%
	\bookStart{The Lay of Hallow Harwardson}[Hęlgakviða Hjǫrvarðssonar]

\begin{flushright}%
Dating \parencite{Sapp2022}: early C11th (0.385)–late C11th (0.550)

Meter: \Fornyrdislag%
\end{flushright}%

Heroic poem.

\sectionline

\section{From Harward and Syelind (\emph{Frá Hjǫrvarði ok Sigrlinn})}

\bpg\bpa Hjǫrvarðr hét konungr. Hann átti fjórar konur. Ein hét Alfhildr; sonr þeira hét Heðinn. Ǫnnur hét Sę́reiþr; þeira sonr hét Humlungr. In þriðja hét Sinrjóð; þeira sonr hét Hymlingr. Hjǫrvarðr konungr hafði þess heit strengt at eiga þá konu er hann vissi vę́nsta. Hann spurði at Sváfnir konungr átti dóttur allra\footnote{‘vęnallra’ \emph{corr.} \Regius} fegrsta; sú hét Sigrlinn. Iðmundr hét jarl hans; Atli var hans sonr er fór at biðja Sigrlinnar til handa konungi. Hann dvalðisk vetrlangt með Sváfni konungi. Fránmarr hét þar jarl, fóstri Sigrlinnar; dóttir hans hét Álǫf. Jarlinn réð, at meyjar var synjat, ok fór jarlinn heim. Atli jarls sonr stóð einn dag við lund nǫkkurn, en fugl sat í limunum uppi yfir hánum ok hafði heyrt til, at hans menn kǫlluðu vę́nstar konur þę́r, er Hjǫrvarðr konungr átti. Fuglinn kvakaði, en Atli hlýddi, hvat hann sagði. Hann kvað:\epa

\bpb TODO. He quoth:\epb
\epg

\bvg\bva „Sátt-u Sigrlinn, \hld\ Sváfnis dóttur, &
męyna fęgrstu \hld\ ï munar-hęimi? &
Þó hagligar \hld\ Hjǫrvarðs konur &
gumnum þykkja \hld\ at Glasislundi.“\eva

\bvb 1\evb
\evg


\bvg\bva „Munt við Atla \hld\ Iðmundar son &
fugl fróð-hugaðr \hld\ flęira mę́la?“ &
„Mun’k ef mik buðlungr \hld\ blóta vildi &
ok kýs’k þat’s ek vil \hld\ ór konungs garði.“\eva

\bvb 2\evb
\evg


\bvg\bva Kjós-at-tu Hjǫrvarð TODO\eva

\bvb 3\evb
\evg


\bvg\bva Hof mun ek kjósa, TODO\eva

\bvb 4\evb
\evg


\bvg\bva Hǫfum erfiði \hld\ ok ękki ørendi;\eva

\bvb 5\evb
\evg


\bvg\bva 6\eva

\bvb 6\evb
\evg


\bvg\bva 7\eva

\bvb 7\evb
\evg


\bvg\bva Sverð vęit’k liggja \hld\ ï Sigarsholmi, &
fjórum fę́ra \hld\ enn fimm tǫgu; &
ęitt es þęira \hld\ ǫllum bętra &
vígnesta bǫl \hld\ ok varið gulli.\eva

\bvb Swords I know lying, in Syeharsholm, four less than fifty. One of them is better than all—the \inx[C]{bale} of war-needles\footnoteB{The kenning \emph{vígnest} also appears in} \ken{spears?}—and inlaid with gold.\evb
\evg


\bvg\bva Hringr ’s ï hjalti, \hld\ hugr ’s ï miðju, &
ógn ’s ï oddi, \hld\ þęim’s ęiga getr; &
liggr með ęggju \hld\ ormr dręyrfáiðr &
en ȧ valbǫstu \hld\ verpr naðr hala.\eva

\bvb A ring is in the hilt; courage is in the middle; fear is in the point, for the one who gets to own it; along the blade lies a serpent painted in blood, but on the walbast\footnoteB{An unclear part of the sword-hilt; see \Sigrdrifumal\ 7.} an adder chases its tail.\evb
\evg

	\bookStart{Second Lay of Hallow Hundingsbane}[Helgakviða Hundingsbana aðra]

\begin{flushright}%
\textbf{Dating} \parencite{Sapp2022}: early C11th (0.346)–late C11th (0.587)

\textbf{Meter:} \Fornyrdislag\ (TODO)%
\end{flushright}

TODO: Introduction.  Similarities to ballads like the Lover’s Ghost, the Grey Cock.

\sectionline

... TODO ...

\bpg\bpa Hęlgi fekk Sigrúnar ok ǫ́ttu þau sonu; vas Hęlgi ęigi gamall.  Dagr Hǫgna sonr blótaði Óðin til fǫður-hefnda. Óðinn léði Dag gęirs síns.  Dagr fann Helga, mág sinn, þar sem hęitir at Fjǫturlundi.  Hann lagði í gǫgnum Hęlga með gęir’num.  Þar fell Hęlgi, en Dagr ręið til fjalla ok sagði Sigrúnu tíðindi:\epa

\bpb Hallow got Syerun and they had sons; Hallow was not old.  Day, son of Hain, made a \inx[C]{bloot} to Weden for the sake of avenging his father.  Weden lent Day his spear. Day found Hallow, his brother-in-law, where it is called Fetterlund; he ran through Hallow with the spear.  There Hallow fell, but Day rode to the fells and told Syerun the tidings:\epb\epg


\bvg\bva „\alst{T}rauðr em ek, systir, \hld\ \alst{t}rega þér at sęgja &
því-at ek hęfi \alst{n}auðigr \hld\ \alst{n}ipti grǿtta: &
\alst{F}ell í morgun \hld\ und \alst{F}jǫturlundi &
\alst{b}uðlungr sá’s vas \hld\ \alst{b}ętstr í hęimi &
ok \alst{h}ildingum \hld\ á \alst{h}alsi stóð.“\eva

\bvb “Regretful am I, O sister, to grieve thee by saying it— \\
for, forced, must I make my kinswoman weep: \\
this morning fell in Fetterlund \\
that noble who was the best in the world, \\
and on the throats of princes stood.”\evb\evg


\bvg\bva\speakernote{[Sigrún kvað:]}%
„Þik skyli \alst{a}llir \hld\ \alst{ęi}ðar bíta, &
þęir es \alst{H}ęlga \hld\ \alst{h}afðir unna, &
at inu \alst{l}jósa \hld\ \alst{L}ęiptrar vatni &
ok at \alst{ú}r-svǫlum \hld\ \alst{U}nnar steini!\eva

\bvb “\emph{Thee} should all oaths bite, \\
which thou to Hallow hast sworn, \\
by the shining water of Lafter, \\
and by the spray-cold stone of Ithe.\evb\evg


\bvg\bva \alst{Sk}ríði-at þat \alst{sk}ip, \hld\ es und þér \alst{sk}ríði, &
þótt \alst{ó}ska-byrr \hld\ \alst{e}ptir lęggisk! &
\alst{R}enni-a sá marr, \hld\ es und þér \alst{r}enni, &
þótt \alst{f}íęndr þína \hld\ \alst{f}orðask ęigir!\eva

\bvb May the ship not glide, which glides beneath thee, \\
though it has a wished-for gust behind it! \\
May the sea not run, which runs beneath thee, \\
though from thy foes thou must escape!\evb\evg


\bvg\bva \alst{B}íti-a þér þat sverð, \hld\ es þú \alst{b}ręgðir, &
nema \alst{s}jǫlfum þér \hld\ \alst{s}yngvi of hǫfði! &
Þá vę́ri þér \alst{h}ęfnt \hld\ \alst{H}ęlga dauða, &
ef þú \alst{v}ę́rir \alst{v}argr \hld\ á \alst{v}iðum úti, &
\alst{a}uðs \alst{a}nd-vani \hld\ ok \alst{a}lls gamans, &
\alst{h}ęfðir ęigi mat, \hld\ nema á \alst{h}rę́um spryngir!“\eva

\bvb May the sword not bite for thee, which thou brandishest, \\
save it sing over thy very own head! \\
\emph{Then} were on thee Hallow’s death avenged, \\
if thou wert a wolf in the woods outside, \\
deprived of wealth and all pleasure; \\
hadst no food, save thou plundered carrion!“\evb\evg


\bvg\bva\speakernote{Dagr kvað:}%
„\edtext{\alst{Ǿ}r ert, systir, \hld\ ok \alst{ø}r-vita}{\lemma{Ǿr \dots\ ok ør-viti ‘Mad \dots\ and out of wits’}\Bfootnote{Formulaic, also occurring in \Lokasenna\ and others TODO.}}, &
es \alst{b}rǿðr þínum \hld\ \alst{b}iðr for-skapa! &
\alst{Ęi}nn vęldr \alst{Ó}ðinn \hld\ \alst{ǫ}llu bǫlvi, &
því-at með \alst{s}ifjungum \hld\ \alst{s}ak-rúnar bar!\eva

\bvb\speakernoteb{Day quoth:}“Mad art thou, sister, and out of wits, \\
when onto thy brother thou dost bid a cruel \inx[C]{shape}. \\
Weden alone causes all the bale, \\
for he bore strife-runes among relatives!\evb\evg


\bvg\bva Þér \alst{b}ýðr \alst{b}róðir \hld\ \alst{b}auga rauða, &
ǫll \alst{V}andils-\alst{v}é \hld\ ok \alst{V}íg-dali; &
\alst{h}af \alst{h}alfan \alst{h}ęim \hld\ \alst{h}arms at gjǫldum &
\alst{b}rúðr \alst{b}aug-varið \hld\ ok \alst{b}úrir þínir.\eva

\bvb \emph{Thee} thy brother offers red bighs, \\
all Wendelswigh and the Wighdales. \\
Have half the realm as recompense for the injury, \\
O bigh-adorned bride—and thy sons, too.\evb\evg


\bvg\bva „\alst{S}it’k-a svá \alst{s}ę́l \hld\ at \alst{S}efa-fjǫllum, &
\alst{á}r né of nę́tr, \hld\ at ek \alst{u}na lífi, &
nema at \alst{l}iði \alst{l}ofðungs \hld\ \alst{l}jóma bręgði, &
renni und \alst{v}ísa \hld\ \alst{V}íg-blę́r þinig, &
\alst{g}ull-bitli vanr, \hld\ knega’k \alst{g}rami fagna!\eva

\bvb “I will not sit so happy in the Sevefells, \\
at dawn nor night, that I should be content with life, \\
unless the retinue of the man of praise were struck with light: \\
{[and]} beneath the ruler ran Wighblaw hither, \\
wont to the golden bit—{[and]} I might greet the prince!\evb\evg


\bvg\bva Svá \alst{h}afði \alst{H}ęlgi \hld\ \alst{h}rę́dda gǫrva &
\alst{f}jándr sína alla \hld\ ok \alst{f}rę́ndr þęira, &
sem fyr \alst{u}lfi \hld\ \alst{ó}ðar rynni &
\alst{g}ęitr af fjalli, \hld\ \alst{g}ęiska fullar!\eva

\bvb So would Hallow have terrified \\
his enemies all and their kinsmen, \\
like from a wolf did madly run \\
goats down a fell, full of fright.\evb\evg


\bvg\bva \edtext{Svá bar \alst{H}ęlgi \hld\ af \alst{h}ildingum &
sem \alst{í}tr-skapaðr \hld\ \alst{a}skr af þyrni &
eða sá \alst{d}ýr-kalfr \hld\ \alst{d}ǫggu slunginn &
es \alst{ø}fri fęrr \hld\ \alst{ǫ}llum dýrum, &
ok \alst{h}orn glóa \hld\ við \alst{h}imin sjalfan.“}{\lemma{ALL}\Bfootnote{Cf. the very similar description of Siward in \GudrunTwo\ 2.}}\eva

\bvb So did Hallow surpass the princes \\
like the nobly shaped ash the thorn, \\
or the deer-calf, dew-besprinkled, \\
who fares higher than all beasts, \\
and its horns gleam against heaven itself.”\evb\evg


\bpg\bpa Haugr var gǫrr eptir Helga.  En er hann kom til Valhallar, þá bauð Óðinn hánum ǫllu at ráða með sér.  Helgi kvað:\epa

\bpb A barrow was made for Hallow.  But when he came to Walhall Weden offered him to rule everything together with him.  Hallow quoth:\epb\epg


\bvg\bva „Þú skalt, \alst{H}undingr, \hld\ \alst{h}vęrjum manni &
\alst{f}ót-laug geta \hld\ ok \alst{f}una kynda; &
\alst{h}unda binda, \hld\ \alst{h}esta gę́ta, &
gefa \alst{s}vínum \alst{s}oð, \hld\ áðr \alst{s}ofa gangir!“\eva

\bvb “Thou shalt, Hunding, for every man \\
make a foot-bath and kindle the fire, \\
bind the hounds, feed the horses, \\
give broth to the swine—before thou mightst go to sleep!”\evb\evg


\bpg\bpa Ambótt Sigrúnar gekk um aptan hjá haugi Helga ok sá at Helgi reið til haugs’ins með marga menn. Ambótt kvað:\epa

\bpb Syerun’s maid-servant walked by Hallow’s barrow at evening, and saw that Hallow rode to the barrow with many men.  The maid-servant quoth:\epb\epg


\bvg\bva „Hvárt ’ru þat \alst{s}vik ęin \hld\ es \alst{s}éa þikkjumk &
eða \alst{r}agna \alst{r}ǫk \hld\ \alst{r}íða męnn dauðir, &
es \alst{jó}a \alst{y}ðra \hld\ \alst{o}ddum kęyrið, &
eða es \alst{h}ildingum \hld\ \alst{h}ęim-fǫr gefin?“\eva

\bvb “Either these are only tricks, as I seem to see \\
—or the \inx[L]{Rakes of the Reins}?—dead men riding; \\
as ye drive your steeds on by spear-points— \\
or are the princes granted leave to go home?”\evb\evg


\bvg\bva\speakernote{[Ęinn þęira kvað:]}%
„Es-a þat \alst{s}vik ęin \hld\ es \alst{s}éa þikkisk &
né \edtrans{\alst{a}ldar rof}{Ripping of the Age}{\Bfootnote{Formulaic.  Cf. TODO \emph{rjúfask ręgin}. This is the same root, only zero-grade.}} \hld\ þótt-u \alst{o}ss lítir, &
þótt vér \alst{jó}a \alst{ó}ra \hld\ \alst{o}ddum keyrim, &
né es \alst{h}ildingum \hld\ \alst{h}ęim-fǫr gefin.“\eva

\bvb\speakernoteb{[One of them quoth:]}%
“It is not only tricks, as thou seemest to see— \\
nor the Ripping of the Age, although thou behold us; \\
although we drive our steeds on by spear-points \\
the princes are not granted leave to go home.”\evb\evg


\bpg\bpa Heim gekk ambótt ok sagði Sigrúnu:\epa

\bpb The maid-servant walked home and said to Syerun:\epb\epg


\bvg\bva „Út gakk \alst{S}igrún, \hld\ frá \alst{S}ęfa-fjǫllum &
ef þik \alst{f}olks jaðarr \hld\ \alst{f}inna lystir; &
upp ’s \alst{h}augr lokinn, \hld\ kominn es \alst{H}ęlgi! &
\alst{D}ólg-spor \alst{d}ręyra \hld\ \alst{d}ǫglingr bað þik &
at þú \alst{s}ár-dropa \hld\ \alst{s}vęfja skyldir.“\eva

\bvb “Go out, O Syerun from the Sevefells, \\
if thou hast lust to find the leader of the troop! \\
The barrow is unlocked; Hallow is come! \\
The ruler of bloody wounds bade thee \\
that thou his wound-drops shouldst soothe.”\evb\evg


\bpg\bpa Sigrún gekk í haug’inn til Helga ok kvað:\epa

\bpb Syerun walked into Hallow’s barrow, and quoth:\epb\epg


\bvg\bva „Nú em’k svá \alst{f}ęgin \hld\ \alst{f}undi okkrum &
sem \alst{á}t-frękir \hld\ \alst{Ó}ðins haukar &
es \alst{v}al \alst{v}itu, \hld\ \alst{v}armar bráðir, &
eða \alst{d}ǫgg-litir \hld\ \alst{d}ags-brún séa.“\eva

\bvb “Now do I so rejoice at our meeting, \\
like do the ravenous hawks of Weden \ken{ravens} \\
when they know corpses, warm venison, \\
or, gleaming with dew, they see the day’s brow \ken{dawn}.\evb\evg


\bvg\bva Fyrr vil’k \alst{k}yssa \hld\ \alst{k}onung ó·lifðan &
an þú \alst{b}lóðugri \hld\ \alst{b}rynju kastir; &
\alst{h}ár ’s þitt, \alst{H}elgi, \hld\ \alst{h}élu þrungit, &
\edtrans{allr es \alst{v}ísi \hld\ \alst{v}al-dǫgg slęginn}{the prince is all with corpse-dew whipped}{\Bfootnote{Cf. \Baldrsdraumar\ 5, where the dead wallow says something similar.}}, &
\alst{h}ęndr úr-svalar \hld\ \alst{H}ǫgna mági; &
hvé skal’k þér, \alst{b}uðlungr, \hld\ þess \alst{b}ót of vinna?“\eva

\bvb Sooner would I kiss the unliving king, \\
than thou the bloody byrnie mightst cast away! \\
Thy hair is, O Hallow, with hoarfrost swollen; \\
the prince is all with corpse-dew \ken{blood} whipped; \\
the hands spray-cold on Hain’s in-law \ken*{= Hallow}.— \\
How shall I for thee, O noble, remedy that?”\evb\evg


\bvg\bva\speakernote{[Hęlgi kvað:]}„Ęin vęldr þú, \alst{S}igrún \hld\ frá \alst{S}efafjǫllum, &
es \alst{H}ęlgi es \hld\ \alst{h}arm-dǫgg slęginn: &
\alst{G}rę́tr þú, \alst{g}ull-varið, \hld\ \alst{g}rimmum tǫ́rum, &
\alst{s}ól-bjǫrt \alst{s}uð-rǿn, \hld\ áðr þú \alst{s}ofa gangir, &
hvęrt fęllr \alst{b}lóðugt \hld\ á \alst{b}rjóst grami, &
\alst{ú}r-svalt, \alst{i}nn-fjalgt \hld\ \alst{ę}kka þrungit.\eva

\bvb “Thou alone causest, O Syerun from the Sevefells, \\
that Hallow be with harm-dew whipped. \\
Thou weepest—O gold-covered—bitter tears— \\
O sun-bright southern lady—before thou go to sleep. \\
Each one falls bloody on the prince’s chest, \\
spray-cold, stifled, pressed forth by grief.\evb\evg


\bvg\bva Vęl skulum \alst{d}rekka \hld\ \alst{d}ýrar vęigar &
þótt \alst{m}isst hafim \hld\ \alst{m}unar ok landa! &
Skal \alst{ę}ngi maðr \hld\ \alst{a}ngr-ljóð kveða &
þótt mér á \alst{b}rjósti \hld\ \alst{b}ęnjar líti. &
Nú eru \edtext{\alst{b}rúðir \hld\ \alst{b}yrgðar í haugi, &
\alst{l}ofða dísir, \hld\ hjá oss}{\lemma{brúðir, dísir, oss ‘brides, dises, us’}\Bfootnote{Hallow speaks in the plural.  “Now has my bride, my goddess, come into the barrow, next to me, who am dead.”}} \alst{l}iðnum!“\eva

\bvb Well shall we drink dear draughts, \\
although we have lost both love and land! \\
Let no one sing songs of sorrow, \\
although he behold the wounds on my chest. \\
Now are the brides shut within the barrow, \\
the praised one’s \inx[C]{dise}[dises], next to us, passed-on.”\evb\evg


\bpg\bpa Sigrún bjó sę́ing í haug’inum.\epa

\bpb Syerun made a bed in the barrow:\epb\epg


\bvg\bva „\alst{H}ér hęfi’k þér, \alst{H}ęlgi, \hld\ \alst{h}vílu gørva, &
\alst{a}ngr-lausa mjǫk, \hld\ \alst{Y}lfinga niðr; &
vil’k þér í \alst{f}aðmi, \hld\ \alst{f}ylkir, sofna &
\edtrans{sem’k \alst{l}ofðungi \hld\ \alst{l}ifnum mynda’k!}{like I would with the living man of praise}{\Bfootnote{i.e. “just as I would if you were still alive.”}}“\eva

\bvb “Here I’ve for thee, Hallow, made a place of rest, \\
all without sorrow, O kinsman of the Wolvings! \\
I will in thy arms, O marshal, fall asleep, \\
like I would with the living man of praise.”\evb\evg


\bvg\bva\speakernote{[Hęlgi kvað:]}„Nú kveð’k \alst{ę}nskis \hld\ \alst{ø}r-vę́nt vesa, &
\alst{s}íð né \alst{s}nimma, \hld\ at \alst{S}efa-fjǫllum &
es þú á \alst{a}rmi \hld\ \alst{ó}·lifðum søfr, &
\alst{h}vít, í \alst{h}augi, \hld\ \alst{H}ǫgna dóttir, &
ok est-u \alst{k}vik, \hld\ in \alst{k}onung-borna!“\eva

\bvb\speakernoteb{[Hallow quoth:]}%
“Now, I say, there is naught more missing \\
neither late nor soon from the Sevefells, \\
when thou dost sleep on the unliving arm, \\
O white daughter of Hain—in the barrow, \\
and thou art alive!—of kingly birth.”\evb\evg

\sectionline

{\small (The night has passed; dawn is breaking, and Hallow speaks.  The manuscript does not indicate the change of scene.)}

\sectionline

\bvg\bva\speakernote{[Hęlgi kvað:]}„Mál ’s mér at \alst{r}íða \hld\ \edtrans{\alst{r}oðnar}{reddening}{\Bfootnote{From the rising dawn.}} brautir, &
láta \alst{f}ǫlvan jó \hld\ \alst{f}lug-stíg troða; &
skal’k fyr \alst{v}estan \hld\ \alst{v}ind-hjalms brúar &
áðr \alst{S}al-gofnir \hld\ \alst{s}igr-þjóð vęki.“\eva

\bvb “’Tis time for me to ride the reddening roads, \\
to let my pale steed tread the path of flight \ken{sky/heaven}. \\
I shall go west of the wind-helm’s bridges \ken{sky/heaven > clouds?}, \\
before Salgovner may awaken the victorious folk.”\evb\evg


\bpg\bpa Þęir Hęlgi riðu lęið sína, en þę́r fóru hęim til bǿjar. Annan aptan lét Sigrún ambótt halda vǫrð á haugi’num.  En at dag-setri, es Sigrún kom til haugs’ins, hón kvað:\epa

\bpb Hallow and his men rode on their way, but the women journeyed home to the farm. The next evening Syerun made her maid-servant keep watch on the barrow.  And at sunset as Syerun came to the barrow, she \ken*{= the maid-servant} quoth:\epb\epg


\bvg\bva „\alst{K}ominn vę́ri nú, \hld\ ef \alst{k}oma hygði, &
\alst{S}igmundar burr \hld\ frá \alst{s}ǫlum Óðins; &
kveð’k \alst{g}rams þinig \hld\ \alst{g}rę́nask vánir &
\edtrans{es á \alst{a}sk-limum \hld\ \alst{ę}rnir sitja}{when on ashen branches eagles sit}{\Bfootnote{i.e. “when the eagles roost on yonder trees”.  This is a sign of Hallow and his men not coming; if they were, the eagles would be following them and picking at their bodies.}} &
ok \edtext{\alst{d}rífr \alst{d}rótt ǫll \hld\ \alst{d}raum-þinga til}{\lemma{drífr \dots\ draum-þinga til ‘drifts off to dream-Things’}\Bfootnote{i.e. “falls asleep”.  A fine metaphor.}}.“\eva

\bvb “Come were now, if to come he had thought, \\
Syemund’s son \ken*{= Hallow} from Weden’s halls; \\
hopes fade, I say, of the prince’s coming, \\
when on ashen branches eagles sit, \\
and all mankind drifts off to dream-\inx[C]{Thing}[Things].\evb\evg


\bvg\bva Ves \alst{ęi}gi svá \alst{ǿ}r \hld\ at \alst{ęi}n farir, &
\alst{d}ís skjǫldunga, \hld\ \alst{d}raug-húsa til! &
Verða \alst{ǫ}flgari \hld\ \alst{a}llir á nǫ́ttum &
\alst{d}auðir \alst{d}ólgar, mę́r, \hld\ an of \alst{d}aga ljósa.“\eva

\bvb Be not so mad that thou journey alone, \\
O dise of the Shieldings, to the ghost-houses! \\
Mightier at night do all become \\
dead fiends, O maiden, than during the bright days!”\evb\evg


\bpg\bpa Sigrún varð skamm-líf af harmi ok trega. Þat var trúa í forneskju, at menn vę́ri endr-bornir, en þat er nú kǫlluð kerlinga-villa.  Helgi ok Sigrún er kallat at vę́ri endr-borin.  Hét hann þá Helgi Haddingjaskati en hon Kára Hálfdanar dóttir, svá sem kveðit er í \edtrans{Káruljóðum}{Leeds of Cheer}{\Bfootnote{A now-lost heroic poem.}}, ok var hon val-kyrja.\epa

\bpb Syerun became short-lived for pain and grief.  It was the belief in olden times that men were born again, but that is now called an old wives’ tale.  Of Hallow and Syerun it is said that they were born again.  He was then called Hallow Hardingskate and she Cheer Halfdanesdaughter, as is told in the Leeds of Cheer, and she was a walkirrie.\epb\epg

\sectionline
%
	\bookStart{Spae of Griper}[Grípisspǫ́]

\begin{flushright}%
\textbf{Dating} \parencite{Sapp2022}: early C11th (0.616)–late C11th (0.313).

\textbf{Meter:} \Fornyrdislag%
\end{flushright}

\section{Introduction}

TODO: Introduction.

This poem is very regular and well preserved; every single one of its 53 \Fornyrdislag\ stanzas is four lines long.

\section{From the Death of Sinfittle (\emph{Frá dauða Sinfjǫtla})}

\sectionline

\bpg\bpa Sigmundr Vǫlsungs sonr var konungr á Frakklandi. Sinfjǫtli var elztr hans sona, annarr Helgi, þriði Hámundr. Borghildr, kona Sigmundar, átti bróður er hét... en Sinfjǫtli, stjúp-sonr hennar, ok... báðu einnar konu báðir ok fyr þá sǫk drap Sinfjǫtli hann. En er hann kom heim þá bað Borghildr hann fara á brot en Sigmundr bauð henni fé-bǿtr ok þat varð hón at þiggja. En at erfi’nu bar Borghildr ǫl. Hon tók eitr mikit, horn fullt, ok bar Sinfjǫtla.  En er hann sá í horn’it skilði hann at eitr var í ok mę́lti til Sigmundar: „Gjǫr-óttr er drykkr’inn, ái!“  Sigmundr tók horn’it ok drakk af.  Svá er sagt at Sigmundr var harð-gǫrr at hvárki mátti hánum eitr granda útan né innan.  En allir synir hans stóðusk eitr á hǫrund útan.  Borghildr bar annat horn Sinfjǫtla ok bað drekka ok fór allt sem fyrr.  Ok enn it þriðja sinn bar hon hánum horn’it ok þó á-mę́lis-orð með ef hann drykki eigi af.  Hann mę́lti enn sem fyrr við Sigmund; hann sagði: „Láttu grǫn sía þá, sonr!“  Sinfjǫtli drakk ok varð þegar dauðr.  Sigmundr bar hann langar leiðir í fangi sér ok kom at firði einum mjóvum ok lǫngum ok var þar skip eitt lítit ok maðr einn á.  Hann bauð Sigmundi far of fjǫrð’inn.  En er Sigmundr bar lík’it út á skip’it þá var bátr’inn hlaðinn.  Karl mę́lti at Sigmundr skyldi fara fyr inn á fjǫrð’inn.  Karl hratt út skip’inu ok hvarf þegar.  Sigmundr konungr dvalðisk lengi í Danmǫrk í ríki Borghildar síðan er hann fekk hennar.  Fór Sigmundr þá suðr í Frakkland til þess ríkis er hann átti þar.  Þá fekk hann Hjǫrdísar, dóttur Eylima konungs.  Þeira sonr var Sigurðr.  Sigmundr konungr fell í orrustu fyr Hundings sonum.  En Hjǫrdís giptisk þá Álfi, syni Hjálpreks konungs.  Óx Sigurðr þar upp í barn-ǿsku.  Sigmundr ok allir synir hans vóru langt um fram alla menn aðra um afl ok vǫxt ok hug ok alla at-gørvi.  Sigurðr var þá allra framarstr ok hann kalla allir menn í forn-frǿðum um alla menn fram ok gǫfgastan her-konunga.\epa

\bpb TODO.\epb\epg


\bpg\bpa Grípir hét sonr Ęylima, bróðir Hjǫrdísar.  Hann réð lǫndum ok vas allra manna vitrastr ok fram-víss.  Sigurðr ręið ęinn saman ok kom til hallar Grípis.  Sigurðr vas auð-kęnndr.  Hann hitti mann at máli úti fyr hǫll’inni; sá nęfndisk Gęitir.  Þá kvaddi Sigurðr hann máls, ok spyrr:\epa

\bpb Griper was called the son of Ilime, Hardise’s brother.  He ruled lands and was wisest of all men, and forthwise.  Siward rode alone and came to Griper’s hall.  Siward was easily recognized.  He approached a man for speech outside of the hall; he was named Goater.  Then Siward greeted him with a speech, and asks:\epb\epg

\section{The Spae of Griper}

\bvg\bva „Hvęrr \alst{b}yggir hér \hld\ \alst{b}orgir þessar? &
Hvat þann \alst{þ}jóð-konung \hld\ \alst{þ}egnar nefna?“ &
„\alst{G}rípir hęitir \hld\ \alst{g}umna stjóri, &
sá’s \alst{f}astri rę́ðr \hld\ \alst{f}oldu ok þegnum.“\eva

\bvb “Who bedwells here these forts? \\
What is this great king called by thanes?” \\
“Griper is called the steerer of men \\
who rules the steadfast land and thanes.”\evb\evg


\bvg\bva \alst{M}ę́la nǫ́mu \hld\ ok \alst{m}argt hjala &
þá’s \alst{r}áð-spakir \hld\ \alst{r}ekkar fundusk. &
„Sęg-ðu \alst{m}ér ef þú vęizt, \hld\ \alst{m}óður-bróðir, &
hvé mun \alst{S}igurði \hld\ \alst{s}núna ę́vi?“\eva

\bvb They took to speak and chatter much, \\
when the council-wise champions found each other. \\
“Tell me, if thou knowest, O mother’s brother: \\
how will Siward’s age turn out?”\evb\evg


\bvg\bva „Þú \alst{m}unt \alst{m}aðr vesa \hld\ \alst{m}ę́ztr und sólu &
ok \alst{h}ę́str borinn \hld\ \alst{h}vęrjum jǫfri; &
\alst{g}jǫfull af \alst{g}ulli \hld\ en \alst{g}løggr flugar, &
\alst{í}tr á-liti \hld\ ok í \alst{o}rðum spakr.“\eva

\bvb „Thou wilt be a man noblest neath the sun, \\
and borne higher than every ruler, \\
giving with gold but stingy of flight, \\
radiant of hue and wise in words.“\evb\evg

TODO.

\bvg\bva Es-a með \alst{l}ǫstum \hld\ \alst{l}ǫgð ę́vi þér; &
lát-tu, inn \alst{í}tri, \hld\ þat, \alst{ǫ}ðlingr, nemask &
því at \alst{u}ppi mun \hld\ meðan \alst{ǫ}ld lifir, &
\alst{n}add-éls boði, \hld\ \alst{n}afn þitt vera.\eva

\bvb TODO. \\
For remembered will while mankind lives, \\
O beseecher of the sword-storm \ken{battle > warrior}, thy name be.\evb\evg

TODO.

\bvg\bva Þú munt \alst{h}víla, \hld\ \alst{h}ęrs odd-viti, &
\alst{m}ę́rr hjá \alst{m}ęyju \hld\ sem þín \alst{m}óðir sé; &
því mun \alst{u}ppi \hld\ meðan \alst{ǫ}ld lifir, &
\alst{þ}jóðar \alst{þ}ęngill, \hld\ \alst{þ}itt nafn vera.\eva

\bvb Thou wilt rest, O point-knower of the host \ken{warrior}, \\
renowned beside a maiden like she were thy mother. \\
For that will remembered while mankind lives, \\
O prince of the nation, thy name be.\evb\evg

TODO.

\bvg\bva Því skal \alst{h}ugga þik, \hld\ \alst{h}ęrs odd-viti, &
sú mun \alst{g}ipt lagit \hld\ á \alst{g}rams ę́vi; &
mun-at \alst{m}ę́tri \alst{m}aðr \hld\ á \alst{m}old koma &
und \alst{s}ólar \alst{s}jǫt \hld\ an, \alst{S}igurðr, þikkir.\eva

\bvb For that [she] shall soothe thee, O point-knower of the host; \\%TODO: "soothe"??
she will have laid venom in the ruler’s age. \\
No nobler man will come onto the earth \\
neath the sun’s seat \ken{sky/heaven}, than thou, Siward, seemest!\evb\evg


\bvg\bva \alst{Sk}iljumk hęilir; \hld\ mun-at \alst{sk}ǫpum vinna! &
Nú hęfir þú, Grípir, vęl \hld\ gørt sem bęiddak; &
fljótt myndir þú \hld\ fríðri sęgja &
mína ę́vi \hld\ ef þú mę́ttir þat!\eva

\bvb Let us part healthy; one will not withstand the \inx[C]{shape}[shapes]! \\
Now hast thou, Griper, well done as I asked; \\
shortly wouldst thou fairer speak \\
of my age, if thou couldst do that!\evb\evg

\sectionline
%
	\bookStart{Speeches of Rein}[Ręginsmǫ́l]

\begin{flushright}%
\textbf{Dating} \parencite{Sapp2022}: C10th (0.666)–early C11th (0.259)

\textbf{Meter:} \Ljodahattr, \Fornyrdislag%
\end{flushright}

\Reginsmal\ is the first of a group of three similarly structured “poems” in an unbroken narrative sequence in \Regius; it is followed by \Fafnismal\ and \Sigrdrifumal.  The division into three poems (indeed their very names) is a product of later philology, and as Bellows says, is perhaps not logically sound.  The titles in the \Regius\ serve more like chapter headings than titles of new poems, and their placement does not exactly agree with the editorial boundaries of the three poems.  In the present edition the division into three poems has been kept for reasons of convention, since the vast majority of readers will be expecting to find the familiar \emph{Ręginsmǫ́l} or \emph{Fáfnismǫ́l}.

The whole group is probably best seen as a long \emph{prosimetrum} that should be read as a single text, rather than three distinct poems.  Indeed almost all of the narrative is carried by prose, while the poetry is almost exclusively dialogue.

The poetry comes in two meters, \Ljodahattr\ and \Fornyrdislag.  The \Ljodahattr\ stanzas of \Reginsmal–\Fafnismal–\Sigrdrifumal\ are greatly alike in style, and probably originally derive from the same composition; this may also be said for the \Fornyrdislag-stanzas.

\Reginsmal\ clearly serves as the basis for \VolsungaSaga\ 14–15 and 17–18 (for ch. 16 see \Gripisspa), where sts. 1–2, 6 and 18 below are quoted directly.

\sectionline

\bpg\bpa Sigurðr gekk til stóðs Hjálp-reks ok kaus sér af hest einn er Grani var kallaðr síðan. Þá var kominn Reginn til Hjálp-reks, sonr Hreið-mars. Hann var hverjum manni hagari ok dvergr of vǫxt. Hann var vitr, grimmr ok fjǫl-kunnigr. Reginn veitti Sigurði fóstr ok kennslu ok elskaði hann mjǫk. Hann sagði Sigurði frá for·ellri sínu ok þeim at·burðum at Óðinn ok Hǿnir ok Loki hǫfðu komit til And-vara-fors; í þeim forsi var fjǫlði fiska. Einn dvergr hét And-vari; hann var lǫngum í forsinum í geddu líki ok fekk sér þar matar. „Otr hét bróðir várr,“ kvað Reginn, „er oft fór í forsinn í otrs líki. Hann hafði tekit einn lax ok sat á ár-bakkanum ok át blundandi. Loki laust hann með steini til bana. Þóttust ę́sir mjǫk heppnir verit hafa ok flógu belg af otrinum. Þat sama kveld sóttu þeir gisting til Hreið-mars ok sýndu veiði sína. Þá tóku vér þá hǫndum ok lǫgðum þeim fjǫr-lausn at fylla otr-belginn með gulli ok hylja útan ok með rauðu gulli. Þá sendu þeir Loka at afla gullsins. Hann kom til Ránar ok fekk net hennar ok fór þá til And-vara-fors ok kastaði netinu fyr gedduna en hon hljóp í netit. Þá mę́lti Loki:\epa

\bpb Siward went to Helpric’s stable and thereof chose for himself one horse which was thenceforth called Grane. Then Rein, son of Rethmar, was come to Helpric. He was craftier than every man and a dwarf in stature; he was clever, cruel and \inx[C]{many-cunning}. Rein granted Siward fosterage and teaching, and loved him much. He told Siward about his parentage, and about the events that Weden, Heener and Lock had come to Andwaresforce; in that force was a multitude of fish. One dwarf was called Andware; he was for a long time in the force in the likeness of a pike and got his food there. “Otter was our brother called,” said Rein, “who often went forth in the force in the likeness of an otter. He had taken a salmon and sat on the riverbank and ate it with his eyes closed. Lock beat him with a stone to his death. The Eese thought themselves to have been very lucky and flayed the skin from the otter. The same evening they sought lodgings at Rethmar’s house, and showed their catch. Then we bound them and gave them as a life-ransom to fill the otter-skin with gold and cover even the outside with red gold. Then they sent Lock to procure the gold. He came to Ran and got her net, and then journeyed to Andwaresforce and threw the net in front of the pike, and it jumped into the net. Then spoke Lock:\epb\epg


\bvg\bva „Hvat ’s þat \alst{f}iska \hld\ es rinn \alst{f}lóði í; &
\ind kann-at sér við \alst{v}íti \alst{v}arask? &
\alst{H}ǫfuð þitt \hld\ lęys-tu \alst{h}ęlju ór; &
\ind finn mér \alst{l}indar \alst{l}oga!“\eva

\bvb “What kind of fish is this that runs in the flood? \\
\ind It cannot ward itself from harm. \\
Redeem thy head out of Hell; \\
\ind find me the linden’s flame \ken{gold}!”\evb\evg


\bvg\bva „\alst{A}nd-vari ek hęiti, \hld\ \alst{Ó}inn hét minn faðir, &
\ind margan hęfi’k \alst{f}ors of \alst{f}arit. &
\alst{Au}mlig norn \hld\ skóp oss í \alst{á}r-daga &
\ind at ek skylda í \alst{v}atni \alst{v}aða.“\eva

\bvb “Andware I am called; Owen was called my father; \\
\ind through many a force have I fared. \\
A wretched norn shaped for us in days of yore, \\
\ind that I should in the water wade.”\evb\evg


\bvg\bva „Sęg-ðu þat, \alst{A}nd-vari, \small{(kvað Loki)} ef þú \alst{ęi}ga vill &
\ind \alst{l}íf í \alst{l}ýða sǫlum: &
Hvęr \alst{g}jǫld \hld\ fȧa \alst{g}umna synir &
\ind ef hǫggvask \alst{o}rðum \alst{ȧ}?“\eva

\bvb “Tell this, Andware—quoth Lock—if thou wilt own \\
\ind life in the halls of men: \\
Which recompense do the sons of men get, \\
\ind if they hew at each other with words?”\evb\evg


\bvg\bva „Ofr-\alst{g}jǫld \hld\ fȧa \alst{g}umna synir &
\ind þęir’s \alst{V}að-gęlmi \alst{v}aða; &
\alst{ȯ}-saðra orða \hld\ hvęrr’s á \alst{a}nnan lýgr, &
\ind of \alst{l}ęngi \alst{l}ęiða \alst{l}imar.“\eva

\bvb “Great recompense do the sons of men get, \\
\ind those who in \inx[L]{Wadyelmer} wade. \\
By the branches of untrue words is each \\
\ind who lies to another long followed.\footnoteB{Watery torment in the afterlife for oath-breakers and liars is well attested in the Germanic sources. See note to \Voluspa\ 39 for discussion.}”\evb\evg


\bpg\bpa Loki sá allt gull þat er And-vari átti. En er hann hafði fram reitt gullit, þá hafði hann eftir einn hring ok tók Loki þann af hánum. Dvergrinn gekk inn í steininn ok mę́lti:\epa

\bpb Lock saw all the gold which Andware owned. But when he had readied all the gold, then he still had one ring, and Lock took it from him. The dwarf went into the stone and spoke:\epb\epg


\bvg\bva „Þat skal \alst{g}ull \hld\ es \alst{G}ustr átti &
\alst{b}rǿðrum tvęim \hld\ at \alst{b}ana verða &
ok \alst{ǫ}ðlingum \hld\ \alst{á}tta at rógi; &
\alst{m}un \alst{m}íns féar \hld\ \alst{m}ann-gi njóta.“\eva

\bvb “That gold which Gust owned shall \\
for two brothers become the bane, \\
and for eight nobles the [cause of] strife; \\
of my wealth will no man benefit.”\evb\evg


\bpg\bpa Ę́sir reiddu Hreið-mari féit ok tráðu upp otr-belginn ok reistu á fǿtr; þá skyldu ę́sirnir hlaða upp gullinu ok hylja. En er þat var gørt gekk Hreið-marr framm ok sá eitt grana-hár ok bað hylja. Þá dró Óðinn framm hringinn And-vara-naut ok hulði hárit.\epa

\bpb The Eese readied the wealth for Rethmar and stuffed the otter-skin and raised it on its feet. Then the Eese should fill it up with gold and cover it. But when that was done Rethmar stepped forth, and saw a single whisker-strand and bade it be covered. Then Weden drew forth the ring Andwaresgift and covered the strand.\epb\epg


\bvg\bva „\alst{G}ull ’s þér nú ręitt {\small (kvað Loki)} en þú \alst{g}jǫld hęfir &
\ind \alst{m}ikil \alst{m}íns hǫfuðs; &
\alst{s}yni þínum \hld\ verðr-a \alst{s}ę́la skǫpuð; &
\ind þat verðr ykkarr \alst{b}ęggja \alst{b}ani!“\eva

\bvb “The gold is now readied for thee—quoth Lock—and thou hast the great \\
\ind payment for my head. \\
For thy son no welfare will be made; \\
\ind it will be the bane of you both!”\evb\evg

Hreiðmarr sagði:

\bvg\bva „\alst{G}jafar þú \alst{g}aft— \hld\ \alst{g}aft-at ǫ́st-gjafar, &
\ind gaft-at af \alst{h}ęilum \alst{h}ug! &
\alst{F}jǫrvi yðru \hld\ skylduð ér \alst{f}irrðir vesa &
\ind ef vissa’k þat \alst{f}ár \alst{f}yrir.“\eva

\bvb “Thou gavest a gift—gavest not a gift of love; \\
\ind gavest not out of true heart! \\
From your lives would ye be far taken, \\
\ind if I had known that danger before!”\evb\evg


\bvg\bva „Enn es \alst{v}erra, \hld\ þat \alst{v}ita þikkjumk, &
\ind \alst{n}iðja stríð um \alst{n}ept; &
\alst{jǫ}fra \alst{ó}-borna \hld\ hygg þá \alst{e}nn vesa &
\ind es þat ’s til \alst{h}atrs \alst{h}ugat.“\eva

\bvb “TODO.”\evb\evg


\bvg\bva „\alst{R}auðu gulli {\small (kvað Hreiðmarr)} hygg ek mik \alst{r}áða munu &
\ind svá \alst{l}engi sem ek \alst{l}ifi; &
\alst{h}ót þín \hld\ \alst{h}rę́ðumk ękki lyf &
\ind ok \alst{h}aldið \alst{h}ęim \alst{h}eðan!“\eva

\bvb “The red gold—quoth Rethmar—I think that I will rule \\
\ind so long as I live. \\
Thy threats I fear not at all (TODO) \\
and hold home from hence!”\evb\evg


\bpg\bpa Fáfnir ok Reginn krǫfðu Hreið-mar nið-gjalda eptir Otr, bróður sinn. Hann kvað nei við. En Fáfnir lagði sverði Hreið-mar, fǫður sinn, sofanda. Hreið-marr kallaði á dǿtr sínar:\epa

\bpb Fathomer and Rein demanded from Rethmar the kin-payment after Otter, their brother. He said no to it. But Fathomer ran the sword through Rethmar, his father, sleeping. Rethmar called on his daughters:\epb\epg


\bvg\bva „\alst{L}yng-hęiðr ok \alst{L}ofn-hęiðr, \hld\ vitið mínu \alst{l}ífi farit! &
\ind \edtrans{Mart ’s þat’s \alst{þ}ǫrf \alst{þ}éar!}{Much does need compel!}{\Bfootnote{Or “Much is required by neccessity”.  Rethmar refers to the duty of his daughters to avenge him, even by killing their own brother.}}“ &
\speakernote{Lyngheiðr svaraði:}„\alst{F}ǫ́ mun systir, \hld\ þótt \alst{f}ǫður missi, &
\ind \alst{h}ęfna \alst{h}lýra \alst{h}arms!“\eva

\bvb “O Lingheath and Lovenheath, witness my life destroyed! \\
\ind Much does need compel!” \\
\speakernoteb{Lingheath answered:}“Few a sister, though she miss her father, \\
\ind will avenge her brother’s harm!\evb\evg


\bvg\bva „Al þú þó \alst{d}óttur, {\small (kvað Hreiðmarr)} \alst{d}ís úlf-huguð, &
ef þú getr-at \alst{s}on \hld\ við \alst{s}iklingi; &
fȧ þú \alst{m}ęy \edtext{\alst{m}anni \hld\ \alst{m}ęgin-þarfar}{\Afootnote{\emph{mann imeginþarfar} \Regius}}, &
þá mun \alst{þ}ęirar sonr \hld\ \alst{þ}íns harms vreka.“\eva

\bvb “Beget yet a daughter—quoth Rethmar—a wolf-minded lady, \\
if thou gettest no son by the prince. \\
Wed that maiden to a man of great need, \\
then \emph{her} son will avenge thy harm!\footnoteB{Rethmar’s last words foretell the life of Siward, whose mother, Hardise, would then be Lingheath’s daughter.}”\evb\evg


\bpg\bpa Þá dó Hreið-marr, en Fáfnir tók gullit allt. Þá beiddisk Reginn at hafa fǫður-arf sinn, en Fáfnir galt þar nei við. Þá leitaði Reginn ráða við Lyng-heiði, systur sína, hvernig hann skyldi heimta fǫður-arf sinn. Hon kvað:\epa

\bpb Then Rethmar died and Fathomer took all the gold. Then Rein begged to have his father’s inheritance, but Fathomer gave back a no. Then Rein sought counsel from Lingheath, his sister, over how he should take his father’s inheritance. She quoth:\epb\epg


\bvg\bva „\edtrans{\alst{B}rúðar}{From the bride}{\Bfootnote{“From me.”  It seems that Lingheath here offers Rein her part of the inheritance.}} kvęðja \hld\ skalt \alst{b}líð-liga &
\ind \alst{a}rfs ok \alst{ǿ}ðra hugar; &
es-a þat \alst{h}ǿft \hld\ at þú \alst{h}jǫrvi skylir &
\ind kvęðja \alst{F}áfni \alst{f}éar!“\eva

\bvb “From the bride shalt thou blithely call \\
\ind for heritance and nobler thoughts; \\
it is not fitting that thou shouldst by sword \\
\ind call for Fathomer’s wealth!”\evb\evg


\bpg\bpa Þessa hluti sagði Reginn Sigurði. Einn dag, er hann kom til húsa Regins, var hánum vel fagnat. Reginn kvað:\epa

\bpb These things Rein told Siward. One day when he came to Rein’s house he was greeted heartily. Rein quoth:\epb\epg


\bvg\bva „\alst{K}ominn ’s hingat \hld\ \alst{k}onr Sig-mundar, &
\alst{s}ęggr inn \alst{s}nar-ráði, \hld\ til \alst{s}ala várra; &
\alst{m}óð hęfir \alst{m}ęira \hld\ an \alst{m}aðr gamall, &
ok es mér \alst{f}angs vǫ́n \hld\ at \alst{f}rekum ulfi.\eva

\bvb “Hither is come the son of Syemund \ken*{= Siward}, \\
the youth of quick counsel to our halls! \\
He has greater heart than an old man, \\
and I expect a catch from the hungry wolf.\evb\evg


\bvg\bva Ek mun \alst{f}ǿða \hld\ \alst{f}olk-djarfan gram; &
nú ’s \alst{y}ngva konr \hld\ með \alst{o}ss kominn; &
sjá mun \alst{r}ę́sir \hld\ \alst{r}íkstr und sólu, &
\edtext{þrymr um \alst{ǫ}ll lǫnd \hld\ \alst{ø}r·lǫg-símu}{\lemma{þrymr \dots\ ør·lǫg-símu ‘he fastens \dots\ orlay-strands’}\Bfootnote{“His fate is being fixed through all lands.”  Cf. the first four sts. of \HelgakvidaOne.}}.“\eva

\bvb I will raise the troop-bold prince; \\
now the son of the king is come amidst us! \\
This ruler will become mightiest under the sun; \\
he fastens through all lands his orlay-strands!”\evb\evg


\bpg\bpa Sigurðr var þá jafnan með Regin ok sagði hann Sigurði at Fáfnir lá á Gnita-heiði ok var í orms líki. Hann átti ǿgis-hjalm er ǫll kvikvendi hrę́ddusk við. Reginn gerði Sigurði sverð er Gramr hét. Þat var svá hvasst at hann brá því ofan í Rín ok lét reka ullar-lagð fyr straumi ok tók í sundr lagðinn sem vatnit. Því sverði klauf Sigurðr í sundr steðja Regins. Eptir þat eggjaði Reginn Sigurð at vega Fáfni. Hann sagði:\epa

\bpb Thereafter Siward was always with Rein, and he told Siward that Fathomer lay on the Gnit-heath and was in a Wyrm’s likeness; he owned the helm of awe by which all living things were frightened. Rein made Siward the sword called Gram; it was so sharp that he plunged it down into the Rhine, and let a lock of wool float down the stream, and it split the lock like it did the water. With that sword Siward split asunder the anvil of Rein; after that Rein urged Siward to slay Fathomer. He said:\epb\epg


\bvg\bva „\alst{H}átt munu \alst{h}lę́ja \hld\ \alst{H}undings synir &
þęir’s \alst{Ęy}-lima \hld\ \alst{a}ldrs synjuðu, &
ef \alst{m}ęirr tiggja \hld\ \alst{m}unar at sǿkja &
\alst{h}ringa rauða \hld\ an \alst{h}ęfnd fǫður.“\eva

\bvb “Loudly laugh will Hunding’s sons \\
—they who denied Ielime’s old age— \\
if the chief is more eager to seek \\
red rings than to avenge his father.”\evb\evg


\bpg\bpa Hjálp-rekr konungr fekk Sigurði skipa-lið til fǫður-hefnda. Þeir fengu storm mikinn ok beittu fyr bergs-nǫs nakkvara. Maðr einn stóð á berginu ok kvað:\epa

\bpb Helpric got Siward a ship-retinue for the avenging of his father. They caught a great storm, and tacked the ships before a group of crags. A lone man stood on the crag and quoth:\epb\epg


\bvg\bva „Hvęrir \alst{r}íða þar \hld\ \alst{R}ę́fils hestum &
\alst{h}ávar unnir, \hld\ \alst{h}af glymjanda? &
\alst{S}egl-vigg eru \hld\ \alst{s}vęita stokkin, &
mun-at \alst{v}ág-marar \hld\ \alst{v}ind of standask.“\eva

\bvb “Which men ride there Revil’s horses \ken{ships} \\
on the high waves, the roaring sea? \\
The sail-steeds are spattered with blood; \\
the wave-chargers will not bear the wind!”\evb\evg


\bvg\bva „Hér eru vér \alst{S}ig-urðr \hld\ á \alst{s}ę́-tréum; &
es oss \alst{b}yrr gefinn \hld\ við \alst{b}ana sjalfan; &
fellr \alst{b}rattr \alst{b}reki \hld\ \alst{b}rǫndum hę́ri, &
\alst{h}lunn-vigg \alst{h}rapa— \hld\ \alst{h}vęrr spyrr at því?“\eva

\bvb “Here are we, Siward [and his men], on sea-trees \ken{ships}; \\
we are given a gust toward death itself! \\
The steep breaker falls higher than flames; \\
the launcher-steeds rush forth—who asks of this?”\evb\evg


\bvg\bva „\alst{H}nikar hétu mik \hld\ þá’s \edtrans{\alst{H}ugin gladdi}{gladdened Highen}{\Bfootnote{A variant of the extremely common motif “feed the raven”, i.e., by the corpses of slain foes on the battlefield.}} &
\edtrans{\alst{V}ǫlsungr ungi}{young Walsing}{\Bfootnote{Siward’s grandfather, the founder of the Walsing dynasty.}} \hld\ ok \alst{v}egit hafði; &
nú mátt \alst{k}alla \hld\ \alst{k}arl af bergi, &
\alst{F}ęng eða \alst{F}jǫlni; \hld\ \alst{f}ar vil’k þiggja.“\eva

\bvb “Nicker they called me when young Walsing \\
gladdened Highen and had conquered. \\
Now mayst thou call me churl-from-the-crag, \\
Feng or Fillner—I wish to beg passage.”\evb\evg


\bpg\bpa Þeir viku at landi, ok gekk karl á skip, ok lę́gði þá veðrit.\epa

\bpb They turned to land and the man went on the ship, and then the weather calmed down.\epb\epg


\bvg\bva „Sęg mér þat, \alst{H}nikarr, \hld\ alls \alst{h}vár-tvęggja vęitst, &
\ind \alst{g}oða hęill ok \alst{g}uma: &
hvęr \alst{b}ǫzt eru \hld\ ef \alst{b}ęrjask skal, &
\ind hęill at \alst{s}verða \alst{s}vipun?“\eva

\bvb “Tell me this, Nicker, as thou knowest both \\
\ind the charms of gods and men: \\
Which are the best—if one shall fight— \\
\ind charms in the swinging of swords?”\evb\evg


\bvg\bva „Mǫrg eru \alst{g}óð \hld\ ef \alst{g}umar vissi, &
\ind hęill at \alst{s}verða \alst{s}vipun; &
\alst{d}yggja fylgju \hld\ hygg ins \alst{d}økkva vesa &
\ind at \alst{h}rotta-męiði \alst{h}rafns.\eva

\bvb “There are many good—if men knew them— \\
\ind charms in the swinging of swords. \\
A good followeress I judge the dark one \\
TODO..”\evb\evg


\bvg\bva Þat es \alst{a}nnat \hld\ ef ert \alst{ú}t of kominn &
\ind ok est á \alst{b}raut \alst{b}úinn: &
\alst{t}vá þú lítr \hld\ á \alst{t}ái standa &
\ind \alst{h}róðr-fúsa \alst{h}ali.\eva

\bvb “This is the other, if thou art come out \\
\ind and art ready on the road: \\
thou beholdest two standing on their toes \\
\ind glory-eager heroes.”\evb\evg


\bvg\bva Þat ’s it \alst{þ}riðja \hld\ ef \alst{þ}jóta hęyrir &
\ind \alst{u}lf und \alst{a}sk-limum, &
\alst{h}ęilla auðit \hld\ verðr þér af \alst{h}jalm-stǫfum &
\ind ef sér þá \alst{f}yrri \alst{f}ara.\eva

\bvb “This is the third, if thou hear howling \\
\ind a wolf beneath ashen branches \\
TODO..”\evb\evg


\bvg\bva Ęngr skal \alst{g}umna \hld\ í \alst{g}ǫgn vega &
\alst{s}íð skínandi \hld\ \alst{s}ystur mána; &
þęir \alst{s}igr hafa \hld\ es \alst{s}éa kunnu, &
\alst{h}jǫr-lęiks \alst{h}vatir, \hld\ eða \edtrans{\alst{h}amalt fylkja}{draw up the flying wedge}{\Bfootnote{This formation, known as the swine-array (\emph{svín-fylking}), was favoured by the Germanic peoples.  It is mentioned already in Tacitus Germania ch. 6: \emph{acies per cuneos componitur} ‘their line of battle is drawn up in a wedge-like formation’.
In the legendary saws it has a particular association with Weden; according \Sogubrot\ it was taught by Weden to the Danish king Harold Hildtooth, who went on to win great victories with it.  At last his rival, the Swedish king Siward Ring, was also taught it, and went on to slay Harold at the battle of the Browolds (\emph{Brávęllir}).  Cf. \Sogubrot\ 8:
\emph{Brúni segir: „Svá lítst mér sem Hringr muni búinn at berjask ok hans lið. Hann hefir undarliga fylkt. Hann hefir svín-fylkt her sínum, ok mun eigi gott at berjask við hann.“
Þá segir Haraldr konungr: „Hverr mun Hringi hafa kennt hamalt at fylkja? Ek hugða engan kunna nema mik ok Óðin, eða mun Óðinn vilja skjǫplast í sigr-gjǫfinni við mik? [...]“}
‘Brown says: “It seems to me that Ring is ready to fight, and his troop too. He has drawn up them in a wondersome way; he has drawn up his host in the swine-shape, and it will not be good to fight against him.
Then says king Harold: “Who will have taught Ring to draw up the flying wedge? I thought noone knew it save for me and Weden; or will Weden wish to fail in his giving me victory? [...]”’}}.\eva

\bvb No man shall fight facing \\
in evening the shining sister of Moon \ken{sun}. \\
They have the victory who can see \\
—men brisk in sword-play \ken{battle}—or draw up the flying wedge.\evb\evg


\bvg\bva Þat ’s \alst{f}ár mikit \hld\ ef \alst{f}ǿti drepr &
\ind þar’s þú at \alst{v}ígi \alst{v}ęðr; &
\alst{t}álar dísir \hld\ standa þér á \alst{t}vę́r hliðar &
\ind ok vilja þik \alst{s}áran \alst{s}éa.\eva

\bvb It is a great peril if thou stumble thy foot \\
\ind where you wade forth in war. \\
Treacherous dises stand on both sides of thee \\
\ind and wish to see thee harmed.\evb\evg


\bvg\bva \alst{K}ęmbðr ok þvęginn \hld\ skal \alst{k}ǿnna hvęrr &
\ind ok at \alst{m}orni \alst{m}ęttr, &
því-at \alst{ó}-sýnt es \hld\ hvar at \alst{a}ptni kømr; &
\ind illt ’s fyr \alst{h}ęill at \alst{h}rapa.\eva

\bvb Combed and washed shall each keen man be, \\
\ind and by morning full, \\
for ’tis unseen where by evening he comes; \\
\ind ’tis bad to rush ahead of the charms!\footnoteB{The wording of the first half of this stanza is very close to \Havamal\ 61 and \Voluspa\ 33; for discussion on personal hygiene and bathing see note to the former.}\evb\evg

\sectionline

\bpg\bpa Sigurðr átti orrustu mikla við Lyngva Hundings son ok brǿðr hans. Þar fell Lyngvi ok þeir þrír brǿðr. Eptir orrustu kvað Reginn:\epa

\bpb Siward had a great battle with Ling Hunding’s son and his brothers. There fell Ling and three of his brothers. After the battle Rein quoth:\epb\epg


\bvg\bva Nú ’s \alst{b}lóðugr ǫrn \hld\ \alst{b}itrum hjǫrvi &
\alst{b}ana Sigmundar \hld\ á \alst{b}aki ristinn; &
øngr es \alst{f}ręmri, \hld\ sá’s \alst{f}old ryði, &
\alst{h}ilmis arfi \hld\ ok \edtrans{\alst{H}ugin gladdi}{has gladdened Highen}{\Bfootnote{i.e. “has fed the raven (with corpses).”}}!\eva

\bvb Now the bloody eagle with a bitter sword \\
is carved on the back of Syemund’s bane. \\
No chieftain’s heir is more successful, \\
who clears the earth and has gladdened Highen!\evb\evg


\bpg\bpa Heim fór Sigurðr til Hjálpreks. Þá eggjaði Reginn Sigurð til at vega Fáfni. Sigurðr ok Reginn fóru upp á Gnitaheiði ok hittu þar slóð Fáfnis þá er hann skreið til vats. Þar gørði Sigurðr grǫf mikla á veginum ok gekk Sigurðr þar í. En er Fáfnir skreið af gullinu blés hann eitri ok hraut þat fyr ofan hǫfuð Sigurði. En er Fáfnir skreið yfir grǫfina þá lagði Sigurðr hann með sverði til hjarta. Fáfnir hristi sik ok barði hǫfði ok sporði. Sigurðr hljóp ór grǫfinni ok sá þá hvárr annan. Fáfnir kvað:\epa

\bpb Siward journeyed home to Helpric. Then Rein incited Siward to smite Fathomer. Siward and Rein journeyed up on the Gnit-heath and found there Siward’s trail as he was slithering to water. There Siward made a great trench in the way, and Siward went down into it. And when Fathomer slithered off the gold he blew venom, and it flew over Siward’s head. But when Fathomer slithered over the trench, then Siward ran him through with the sword to the heart. Fathomer shook himself and struck his head and spurned. Siward leapt out of the trench, and then each of them saw the other. Fathomer quoth:\epb\epg
%
	\bookStart{The Speeches of Fathomer}[Fáfnismǫ́l]

\begin{flushright}%
\textbf{Dating} \parencite{Sapp2022}: C10th (0.442)–early C11th (0.402)

\textbf{Meter:} \Ljodahattr\ (TODO)%
\end{flushright}

Titled \emph{Frá dauða Fáfnis} ‘From Fathomer’s death’ in \Regius.

\sectionline

\bvg\bva „Svęinn ok svęinn! \hld\ Hvęrjum est svęini of borinn? &
\ind Hvęrra est manna mǫgr? &
es þú á Fáfni rautt \hld\ þínn hinn frána mę́ki; &
\ind stǫndumk til hjarta hjǫrr!“\eva

\bvb {[Fathomer quoth:]} \\
“O swain and swain! To which swain art thou born; \\
of which men art thou son? \\
As thou on Fathomer hast reddened thy gleaming blade, \\
the sword stands unto my heart!”\evb\evg


\bpg\bpa Sigurðr dulði nafns síns fyr því at þat var trúa þeira í forneskju at orð feigs manns mę́tti mikit ef hann bǫlvaði óvin sínum með nafni. Hann kvað:\epa

\bpb Siward concealed his name, because it was their belief in ancient times that the word of a \inx[C]{fey} man could do much if he baled his enemy by his name. He \ken*{= Siward} quoth:\epb\epg


\bvg\bva „Gǫfugt dýr ek hęiti \hld\ en ek gęngit hef’k &
\ind hinn móður-lausi mǫgr, &
fǫður ek á’kk-a \hld\ sem fira synir, &
\ind gęng ek ęinn saman.“\eva

\bvb “Noble Deer am I called, and I have gone \\
as the motherless lad. \\
A father I have not like the sons of men; \\
I go alone.”\evb\evg


\bvg\bva „Vęitst, ef fǫður né átt-at \hld\ sem fira synir, &
\ind af hvęrju vastu undri alinn?
[...]“\eva

\bvb {[Fathomer quoth:]} \\
“Dost thou know, if thou hast no father, like do the sons of men, \\
by which wonder thou wast begotten?”\evb\evg


\bvg\bva „Ę́tterni mitt \hld\ kveð’k þér ó·kunnigt vesa &
\ind ok mik sjalfan hit sama: &
Sigurðr ek hęiti \hld\ Sigmundr hét minn faðir &
\ind es hęf’k þik vǫ́pnum vegit.“\eva

\bvb {[Siward quoth:]} \\
“My lineage I declare is unknown to thee, \\
and my self the same.\footnoteB{The meaning is that Fathomer would not recognize Siward’s lineage (i.e. his father) or name, since he is an orphan who up until this point has not won any glory. He is not saying that he is lineage is unknown even to himself, since \emph{sjalfan mik} ‘my self’ is accusative, not dative.} \\
Siward am I called—Syemund was called my father— \\
who with weapons have struck thee.”\evb\evg


\bvg\bva „Hvęrr þik hvatti, \hld\ hví hvętjask lést, &
\ind mínu fjǫrvi at fara? &
Hinn frán-ęygi svęinn, \hld\ þú áttir fǫður bitran, &
\ind \edtrans{á-bornu skjór á skęið.}{inborn traits quickly show.}{\Bfootnote{The original is cryptic.  \emph{á skęið} means roughly ‘rapidly, quickly’, whence the expression \emph{ríða á skęið} ‘\CV: to ride at full speed’, but the other words are uncertain.  \textcite{LaFargeGlossary} read ‘your innate qualities show quickly’, suggesting two unattested words: an adjective \emph{*áborinn} ‘innate, inborn’ and a verb \emph{*skjóa} ‘to show’. Yet the lack of i-umlaut in the supposed 3rd sg. pres. ind. \emph{skjór} is difficult. We would expect \emph{**skýr}, as in \emph{skjóta} ‘to shoot,’ with 2nd/3rd sg. pres. ind \emph{skýtr}. A solution here would be reading a 2nd sg. pres. subj. \emph{skjóir}, with a vowel TODO}}“\eva

\bvb {[Fathomer quoth:]} \\
“Who goaded thee—why didst thou let thee be goaded— \\
my life for to destroy? \\
O gleaming-eyed swain, thou haddest a sharp father; \\
inborn traits quickly show!”\evb\evg


\bvg\bva „Hugr mik hvatti, \hld\ hendr mér full-týðu &
\ind ok minn inn hvassi hjǫrr; &
fár es hvatr \hld\ es hrøðask tękr &
\ind ef í barnǿsku ’s blauðr.“\eva

\bvb {[Siward quoth:]} \\
“My heart goaded me, my hands availed me, \\
and this my sharp sword. \\
Few a man is brave when he takes to grow, \\
if in youth he be soft.”\evb\evg


\bvg\bva „Vęit’k, ef þú vaxa nę́ðir \hld\ fyr þinna vina brjósti, &
\ind séi-t maðr þik vręiðan vega; &
nú est haptr \hld\ ok hęr-numinn, &
\ind ę́ kveða bandingja bifask.“\eva

\bvb {[Fathomer quoth:]} \\
“TRANSLATION”\evb\evg


\bvg\bva „Því bregðr þú nú mér, Fáfnir, \hld\ at til fjarri sjá’k &
\ind mínum fęðr-munum, &
ęigi em’k haptr \hld\ þótt vę́ra hęr-numi; &
\ind þú fannt, at ek lauss lifi!“\eva

\bvb {[Siward quoth:]} \\
“TRANSLATION”\evb\evg


\bvg\bva „Hęipt-yrði ęin \hld\ tęlr þú þér í hví-vętna &
\ind en ek þér satt ęitt sęgi’k: &
It gjalla gull \hld\ ok it glóð-rauða fé, &
\ind þér verða þęir baugar at bana!“\eva

\bvb {[Fathomer quoth:]} \\
“With only hateful words dost thou answer anything, \\
but I tell thee truth alone: \\
The resounding gold and the glowing red wealth, \\
those bighs will be thy bane!”\evb\evg


\bvg\bva „Féi ráða \hld\ skal fyrða hvęrr &
\ind ę́ til \edtrans{ins ęina dags}{the one day}{\Bfootnote{i.e. his predetermined time of death.  Siward dismisses the idea of the curse, since he knows that he will die regardless of whether he takes the gold or not; and he would rather die rich and famous than wretched and forgotten.}} &
því-at ęinu sinni \hld\ skal alda hvęrr &
\ind fara til hęljar heðan.“\eva

\bvb {[Siward quoth:]} \\
“Rule [his] wealth shall every man, \\
always, until the one day; \\
for at one time must every man \\
journey hence to Hell.”\evb\evg


\bvg\bva „Norna dóm \hld\ munt \edtrans{fyr nęsjum}{before the headlands}{\Bfootnote{i.e. ‘close at hand, imminent’.  A formulaic expression for imminent death, cf. the last st. of \Sonatorrek\ (TODO).}} hafa &
\ind ok ó·svinns apa; &
í vatni þú drukknar \hld\ ef í vindi rę́r; &
\ind allt es fęigs forað.“\eva

\bvb {[Fathomer quoth:]} \\
“The doom of the Norns shalt thou have before the headlands, \\
and that of an unwise ape. \\
In water wilt thou drown if thou row in wind; \\
everything is the pit of the \inx[C]{fey}.\footnoteB{That is, the cursed, death-doomed (fey) man will find sudden death no matter where he turns.}”\evb\evg


\bvg\bva „Sęg mér, Fáfnir, \hld\ alls þik fróðan kveða &
\ind ok vęl mart vita: &
Hvęrjar ’ru þę́r nornir \hld\ \edtrans{es nauð-gǫnglar ’ru}{that attend in need}{\Bfootnote{lit. ‘who are attendant in need’, i.e. who help ailing mothers during childbirth.  Cf. \Sigrdrifumal\ 8.}} &
\ind ok kjósa mǿðr frá mǫgum?“\eva

\bvb {[Siward quoth:]} \\
“Say to me, Fathomer, as they call thee wise, \\
and knowing well enough: \\
Who are the Norns that attend in need, \\
and choose mothers from their lads?”\evb\evg


\bvg\bva „Sundr-bornar mjǫk \hld\ hygg at nornir sé, &
\ind ęigu-t þę́r ę́tt saman; &
sumar ’ru ás-kunngar, \hld\ sumar alf-kunngar, &
\ind sumar dǿtr Dvalins.“\eva

\bvb {[Fathomer quoth:]} \\
“Of very sundry birth I judge the norns to be; \\
they come not from a common lineage: \\
Some are begotten of the Eese, some begotten of the Elves, \\
some are the daughters of Dwollen \ken{dwarfs}.”\evb\evg


\bvg\bva „Sęg mér þat, Fáfnir, \hld\ alls þik fróðan kveða &
\ind ok vęl margt vita, &
hvé sá holmr hęitir \hld\ es blanda hjǫr-lęgi &
\ind Surtr ok ę́sir saman.“\eva

\bvb {[Siward quoth:]} \\
“Say to me, Fathomer, as they call thee wise, \\
and knowing well enough: \\
What is the islet called, where Surt and the Eese \\
blend sword-water \ken{blood} together?”\evb\evg


\bvg\bva „Ó·skópnir hęitir \hld\ en þar ǫll skulu &
\ind gęirum lęika goð; &
Bil-rǫst brotnar \hld\ es á brott fara &
\ind ok svima í móðu marir.\eva

\bvb {[Fathomer quoth:]} \\
“Unshopner it is called, and there shall all \\
the Gods play with spears; \\
Bilrest shatters when they fare away, \\
and the horses swim in the sea.\evb\evg

\sectionline

Fathomer continues speaking, but there is probably something missing here, since the transition is abrupt. Between its paraphrases of st. 15 and of st. 16, \VolsungaMS\ has \emph{Ok enn mę́lti Fáfnir: „Reginn bróðir minn veldr mínum dauða, ok þat hlę́gir mik, er hann veldr ok þínum dauða, ok ferr þá, sem hann vildi.“} ‘And further spoke Fathomer: “My brother Rein causes my death, and it gladdens me that he also causes thy death, and then it will go like he has willed.”’, which may either be a paraphrase of a lost st., or an addition by the redactor.

\sectionline

\bvg\bva Ǿgis hjalm \hld\ bar’k of alda sonum &
\ind meðan of męnjum lá’k; &
ęinn rammari \hld\ hugðumk ǫllum vesa, &
\ind fann’k-a’k marga mǫgu.“\eva

\bvb A helmet of terror I carried over the sons of men \\
while on the rings I lay; \\
stronger than all I thought myself alone to be; \\
I did not find many men.”\evb\evg


\bvg %NOTE: Heavily formulaic.
\bva „Ǿgis hjalmr \hld\ bergr ęinu-gi &
\ind hvar’s skulu vręiðir vega; &
þá þat finnr \hld\ es með flęirum kømr &
\ind at ęngi es ęinna hvatastr.“\eva

\bvb {[Siward quoth:]} \\
“A helmet of terror saves no man, \\
wherever wroth men should fight; \\
then he finds, when among the many he comes, \\
that none is the boldest of all.”\evb\evg


\bvg\bva „Ęitri ek fnę́sta \hld\ es á arfi lá’k &
\ind miklum míns fǫður.“\eva

\bvb {[Fathomer quoth:]} \\
“Venom I snorted, while I lay on the great \\
inheritance of my father.”\evb\evg


\bvg\bva „Inn rammi ormr, \hld\ þú gørðir frę́s mikla &
\ind ok gatst harðan hug; &
\ind hęipt at męiri \hld\ verðr hǫlða sonum &
\ind at þann hjalm hafi.“\eva

\bvb {[Siward quoth:]} \\
“O mighty wyrm, thou madest a great snort, \\
and didst get a hard heart; \\
TODO.”\evb\evg


\bvg\bva „Rę́ð’k þér nú, Sigurðr, \hld\ en þú ráð nemir &
\ind ok ríð hęim heðan; &
it gjalla gull \hld\ ok it glóð-rauða fé, &
\ind þér verða þęir baugar at bana!“\eva

\bvb {[Fathomer quoth:]} \\
“I counsel thee now, O Siward—and thou oughtst to take the counsel, \\
and ride home, hence! \\
The resounding gold and the glowing red wealth, \\
those bighs will become thy bane!”\evb\evg


\bvg\bva „Ráð ’s þér ráðit \hld\ en ek ríða mun &
\ind til þęss gulls es í lyngvi liggr, &
en þú, Fáfnir, ligg \hld\ í fjǫr-brotum &
\ind \edtrans{þar’s þik Hęl hafi}{where Hell may have thee}{\Bfootnote{Formulaic. TODO.}}!“\eva

\bvb {[Siward quoth:]} \\
“Thy counsel has been counseled—but I will ride, \\
to the gold which in the heather lies; \\
but \emph{thou}, Fathomer, lie in the blood-tracks, \\
where Hell may have thee!”\evb\evg


\bvg% NOTE: Pun.
\bva „Ręginn mik réð, \hld\ hann þik ráða mun, &
\ind hann mun okkr verða bǫ́ðum at bana; &
fjǫr sitt láta \hld\ hygg at Fáfnir myni; &
\ind þitt varð nú męira męgin.“\eva

\bvb {[Fathomer quoth:]} \\
“Rein betrayed \emph{me}, he will betray \emph{thee}; \\
he will become the bane of us both; \\
give his life, I judge that Fathomer will; \\
thy strength was now the greater.”\evb\evg


\bpg
\bpa Reginn var á brott horfinn meðan Sigurðr vá Fáfni ok kom þá aptr er Sigurðr strauk blóð af sverðinu. Reginn kvað:\epa

\bpb Rein had gone away while Siward smote Fathomer, and then came back as Siward wiped the blood off the sword. Rein quoth:\epb
\epg


\bvg\bva „Hęill þú nú, Sigurðr, \hld\ nú hęfir sigr vegit &
\ind ok Fáfni of farit; &
manna þęira \hld\ es mold troða &
\ind þik kveð’k ó·blauðastan alinn.“\eva

\bvb {[SPEAKER quoth:]} \\
“Hail thee now, O Siward—now thou hast won victory \\
and Fathomer destroyed! \\
Of those men who tread on the earth \\
I declare \emph{thee} with least softness begotten.”\evb\evg


\bvg\bva „VERSE“\eva

\bvb {[SPEAKER quoth:]} \\
“TRANSLATION”\evb\evg

\sectionline
%
	\bookStart{The Speeches of Sighdrive}[Sigrdrífumǫ́l]

\begin{flushright}%
Dating \parencite{Sapp2022}: C10th (0.961)

Meter: \Ljodahattr%
\end{flushright}

% Introduction

Many of the verses are quoted in \VolsungaSaga, but notably the two prayer-verses are missing; possibly an instance of Christian censorship. TODO

\sectionline

\bvg {\small [Sighdrive quoth:]}
\bva „Lęngi ek svaf, \hld\ lęngi ek sofnuð vas, &
\ind lǫng eru lýða lę́; &
Óðinn því vęldr \hld\ es ęigi mátta’k &
\ind bregða blundstǫfum.“\eva

\bvb “Long I slept, long was I asleep, long are the deceits”\evb
\evg

\bpg
\bpa Sigurðr sęttisk niðr ok spyrr hana nafns. Hón tók þá horn fullt mjaðar ok gaf hǫ́num minnisvęig.\epa

\bpb Siward set himself down, asking for her name. Then she took a horn full of mead, and gave him a mind-draught:\epb
\epg


\bvg
\bva Hęill \alst{D}agr, \hld\ hęilir \alst{D}ags synir, &
\ind hęil \alst{N}ǫ́tt ok \alst{n}ipt! &
\alst{Ó}ręiðum \alst{au}gum \hld\ lítið \alst{o}kkr þinig &
\ind ok gefið \alst{s}itjǫndum \alst{s}igr!\eva

\bvb “Hail \inx[P]{Day}! Hail the sons of Day!\footnoteB{TODO. Who?} Hail Night and [her] kinswoman \ken*{= Earth}!\footnoteB{According to \Gylfaginning\ 10 Earth is the daughter of Night and \inx[P]{Aner}.} With unwrathful eyes look ye upon us two, and give the sitting ones \ken*{= us} victory.\evb
\evg


\bvg
\bva Hęilir \alst{ę́}sir, \hld\ hęilar \alst{ǫ́}synjur, &
\ind hęil sjá in \alst{f}jǫlnýta \alst{f}old! &
\alst{M}ál ok \alst{m}anvit \hld\ gefið okkr \alst{m}ę́rum tvęim &
\ind ok \alst{l}ę́knishęndr meðan \alst{l}ifum!\eva

\bvb Hail the \inx[G]{Ease}! Hail the \inx[G]{Ossens}! Hail this bountiful fold \ken{earth}! Speech and \inx[C]{manwit} give ye us renowned two, and \inx[C]{healing-hands}\footnoteB{Hands with the power to heal (perhaps supernaturally). The singular form \emph{lę́knishǫnd} occurs in the semi-Christianized prayer on a c. 1300 stick from Ribe, Denmark (signum DR EM85;493).} while we live.”\evb
\evg


BPG
BPA Hon nefndisk Sigrdrífa ok var valkyrja. Hon sagði, at tveir konvngar bǫrðusk. Hét annarr Hjalmgunnarr; hann var þá gamall ok inn mesti hermaðr, ok hafði Óðinn hánum sigri heitit.
En \alst{a}nnarr hét \alst{A}gnarr, \hld\ \alst{Au}ðu bróðir // er \alst{v}ę́tr engi \hld\ \alst{v}ildi þiggja.
Sigrdrífa felldi Hjalmgunnar í orrostunni. En Óðinn stakk hana svefnþorni í hefnd þess ok kvað hana aldri skyldu síðan sigr vega í orrostu, ok kvað hana giftask skyldu, „en sagða’k hánum at strengða’k heit þar í mót, at giptask øngom þeim manni er hrę́ðask kynni.“ Hann segir ok biðr hana kenna sér speki ef hon\footnoteA{\emph{hánom} ms.} vissi tíðendi ór ǫllum heimum. Sigrdrífa kvað:EPA

BPB She called herself Sighdrive and was a walkirrie. She said that two kings fought. One of them was called Helmguther; he was then old and the greatest harrier, and Weden had promised him victory.
But another one was called Eyner, Eade’s brother, who in no way wished to accept.\footnoteB{i.e. ‘wished to lose’ TODO}
Sighdrive felled Helmguther in the battle, but Weden pierced her with the sleeping-thorn as revenge for that, and said that she would never thenceforth win victory in battle, and said that she must marry, “but I told him that I made a vow against that, to marry no man who could be frightened.” He [= Siward] speaks and asks her to teach him wisdom, if she knew any tidings out of all the \inx[C]{Home}[Homes]. Sighdrive quoth: EPB
EPG


\bvg
\bva „Bjór fǿri’k þér, \hld\ brynþings apaldr, &
magni blandinn \hld\ ok męgintíri, &
fullr ’s hann ljóða \hld\ ok líknstafa, &
góðra galdra \hld\ ok gamanrúna.\eva

\bvb Beer I bring thee—apple-tree of the byrnie-\inx[C]{Thing} \ken{battle > warrior}!—mixed with might, and might-glory; it is full of \inx[C]{leed}[leeds], and grace-staves, of good \inx[C]{galder}[galders], and pleasure-\inx[C]{rune}[runes].\evb
\evg


\bvg
\bva Sigrúnar skalt kunna, \hld\ ef vilt sigr hafa, &
\ind ok rísta á hjalti hjǫrs, &
sumar á véttrimum, \hld\ sumar á valbǫstum, &
\ind ok nęfna tysvar Tý.\eva

\bvb Victory-runes shalt thou know, if thou wilt have victory, and carve on the hilt of the sword; some on weight-rims;\footnoteB{Unclear.} some on walbasts\footnoteB{Possibly the sword-pommel, the word also occurs in \HelgakvidaHjorvardssonar\ 9.}, and name \inx[P]{Tue} twice.\evb
\evg


\bvg
\bva Ǫlrúnar skalt kunna \hld\ ef þu vilt annars kvę́n &
\ind vęli t þik i trygd ef þú trúir. &
á horni skal þér rísta \hld\ ok á handar baki &
\ind ok merkia a nagli nꜹþ.\eva

\bvb Ale-runes shalt thou know, if TODO\evb
\evg


\bvg
\bva Full skal signa \hld\ ok við fári séa &
\ind ok verpa lauki í lǫg; &
\edtext{þá þat vęitk, \hld\ at þér verðr aldri &
męini blandinn mjǫðr.}{\lemma{þá \dots\ mjǫðr}\Bfootnote{\emph{thus} \VolsungaSaga, \emph{om.} \Regius}}\eva

\bvb TODO\evb
\evg

...


\bvg
\bva Þá mę́lti \hld\ Míms hǫfuð &
\ind fróðligt it fyrsta orð, &
\ind ok sagði sanna stafi.\eva

\bvb Then spoke the head of Mime learnedly the first word, and said true staves:\evb
\evg


\bvg
\bva Á skildi kvað ristnar \hld\ þęim’s stęndr fyr skínanda goði, &
á ęyra Árvakrs, \hld\ ok á Alsvinns hófi, &
á því hvéli es snýz \hld\ undir ręið Hrungnis, &
á Slęipnis tǫnnum \hld\ ok á slęða fjǫtrum, &
á bjarnar hrammi \hld\ ok á Braga tungu, &
á ulfs klóm \hld\ ok á arnar nęfi, &
á blóðgum vę́ngjum \hld\ ok á brúar sporði, &
á lausnar lófa \hld\ ok á líknar spori, &
á glęri ok á gulli \hld\ ok á gumna hęillum, &
í víni ok virtri \hld\ ok vilisessi. &
Á Gungnis oddi \hld\ ok á Grana brjósti, &
á nornar nagli \hld\ ok á nęfi uglu;\eva

\bvb On a shield it said were carved [runes]—[the shield] that stands before the shining god\footnoteB{According to \Grimnismal\ 39 the sun is covered by a shield, protecting the earth from its heat. Without it, the whole world would burn up.} \ken{sun}—[also] on the ear of Yorewaker, on the hoof of Allswith,\footnoteB{The two horses that pull the sun across the heavens; cf. \Grimnismal\ 38.} on that wheel which turns beneath the chariot of Rungner, on the teeth of Slopner, and on the fetters of sleds, on the paw of the bear, and on the tongue of Bray, on the claws of the wolf, and on the beak of the eagle, on bloody wings, and on the supports of the bridge, on the palm of release, and the track of grace, on glass and on gold, and on the good healths of men, in wine and beerwort, and on the comfortable seat, on the point of Gungner, and on the breast of Grane, on the nail of a norn, and on the beak of an owl.\evb
\evg


\bvg
\bva Allar vǫ́ru af skafnar, \hld\ þę́r es vǫ́ru á ristnar, &
\ind ok hvęrfðar við inn hęlga mjǫð &
\ind ok sęndar á víða vega.\eva

\bvb All were shaven off—those that were carved on—and thrown into the holy mead, and sent on wide ways:\evb
\evg


\bvg
\bva Þę́r ’ru með ǫ́sum, \hld\ þę́r ’ru með ǫlfum, &
\ind sumar með vísum vǫnum, &
\ind sumar hafa męnskir męnn.\eva

\bvb They are among the Ease, they are among the Elves; some among wise Wanes; some manly men have.\evb
\evg

...


\bvg {\small [Sighdrive quoth:]}
\bva ...\eva

\bvb “Now shalt thou choose, as the choice is offered to thee, maple-tree of sharp weapons \ken{warrior}! Speech or silence have thou in thy own heart; all the harms are measured [by the Norns].”\evb
\evg


\bvg {\small [Siwrd quoth:]}
\bva ...\eva

\bvb “I shall not flee, although thou know me to be fey; I am not born with softness.\footnoteB{Note about this common heroic expression.} Thy loving counsels all will I have, for as long as I live.”\evb
\evg


\bvg {\small [Sighdrive quoth:]}
\bva ...\eva

\bvb “That I counsel thee first: that thou against thy kinsmen defend thyself faultlessly. Late ought thou to take revenge, although they incur charges; that they say befits the dead.\evb
\evg


\bvg
\bva Þat rę́ð’k þér annat, \hld\ at ęið né svęrir, &
\ind nema þann ’s saðr séi, &
grimmar simar \hld\ ganga at tryggðrofi; &
\ind armr es vára vargr.\eva

\bvb That I counsel thee second: that thou not swear an oath, save for that one which is true. Grim strands befall the troth-breaker; wretched is the outlaw of vows.\evb
\evg


\bvg
\bva ...\eva

\bvb That I counsel thee third: that thou on the Thing bandy not with foolish men; for an unwise man often lets be spoken worse words than he ought to know.\evb
\evg


\bvg
\bva ...\eva

\bvb All is missing if thou shut up towards it; then thou seemest born with softness, or truthfully accused. Risky is the verdict of neighbours, unless one gets himself a good one.\evb
\evg


\bvg
\bva ...\eva

\bvb At another day make his breath go away, and thus repay the people for the lie.\evb
\evg
%

	\bookStart{Fragments from the Saw of the Walsings}

\section{Introduction}

In \Regius, \Sigrdrifumal\ ends abruptly at stanza 27, after which a number of pages have gone missing; the so-called “great lacuna”.  The poetry contained in them undoubtedly belonged to the Walsing cycle, specifically concerning the life of Siward.

The author of \VolsungaSaga\ drew heavily from a collection of Walsing-cycle poetry closely related to \Regius.  He quotes many stanzas known from \Regius, but also some which do not survive anywhere else—these are the stanzas edited here.  They correspond to the story which would have been found in the great lacuna, and it is probable that they derive from the now-lost poems found there.

\sectionline

\bvg\bva%
Ristu af \alst{m}agni \hld\ \alst{m}ikla hellu, &
\alst{S}igmundr hjǫrvi \hld\ ok \alst{S}infjǫtli.\eva

\bvb They carved with strength the great stone, \\
\inx[P]{Syemund} with sword, and \inx[P]{Sinfittle}.\evb\evg

\sectionline

\bvg\bva%
\alst{Ę}ldr nam at \alst{ǿ}sask \hld\ en \alst{jǫ}rð at skjalfa &
ok \alst{h}ár logi \hld\ við \alst{h}imni gnę́fa; &
fár \alst{t}ręystisk þar \hld\ \alst{f}ylkis rekka &
\alst{ę}ld at ríða \hld\ né \alst{y}fir stíga.\eva

\bvb Fire took to rage and earth to shake \\
and high flame to rise against heaven. \\
Few there dared of the marshall’s champions \\
the fire to ride or to step over.\evb\evg


\bvg\bva%
\alst{S}igurðr Grana \hld\ \alst{s}verði kęyrði; &
\alst{ę}ldr sloknaði \hld\ fyr \alst{ǫ}ðlingi; &
\alst{l}ogi allr \alst{l}ę́gðisk \hld\ fyr \alst{l}of-gjǫrnum; &
bliku \alst{r}ęiði, \hld\ es \alst{R}eginn átti.\eva

\bvb Siward drove Grane on by sword; \\
the fire went out before the athling; \\
the flame all lowered before the praise-eager man; \\
the harness flashed which Rein had owned.\evb\evg

\sectionline

\bvg\bva%
\alst{S}igurðr vá at ormi, \hld\ en þat \alst{s}íðan mun &
\alst{ø}ngum fyrnask, \hld\ meðan \alst{ǫ}ld lifir. &
En \alst{h}lýri þinn \hld\ \alst{h}várki þorði &
\alst{ę}ld at ríða \hld\ né \alst{y}fir stíga.\eva

\bvb Siward smote the Wyrm, and that will afterwards \\
by none be forgotten while mankind lives, \\
but thy brother dared not either \\
the fire to ride or to step over.\evb\evg

\sectionline

\bvg\bva%
\alst{Ú}t gekk Sigurðr \hld\ \alst{a}nn-spjalli frá, &
\alst{h}oll-vinr lofða, \hld\ ok \alst{h}nípaði, &
svá at \alst{g}anga nam \hld\ \alst{g}unnar-fúsum &
\alst{s}undr of \alst{s}íður \hld\ \alst{s}erkr járn-ofinn.\eva

\bvb TODO: translation.\evb\evg


TODO: More stanzas?


\sectionline
%

%	\include{books/Fragmented Lay of Siward.tex}%
%	\include{books/From the Death of Siward.tex}%
	\bookStart{The First Lay of Guthrun}[Guðrúnarkviða fyrsta]

\begin{flushright}%
\textbf{Dating} \parencite{Sapp2022}: 900s (0.988)

\textbf{Meter:} \Fornyrdislag%
\end{flushright}

After Siward’s death Guthrun is so upset that she cannot make herself weep.

\sectionline

\bvg\bva Ár vas þat’s \alst{G}uðrún \hld\ \alst{g}ørðisk at dęyja, &
es hǫ́n \alst{s}at \alst{s}org-full \hld\ yfir \alst{S}igurði, &
gørði-t hǫ́n \alst{h}júfra \hld\ né \alst{h}ǫndum sláa &
né \alst{k}vęina umb \hld\ sem \alst{k}onur aðrar.\eva

\bvb TODO.\evb\evg


\bvg\bva Gingu \alst{ja}rlar \hld\ \alst{a}l-snotrir framm, &
þęir’s \alst{h}arðs \alst{h}ugar \hld\ \alst{h}ana lǫttu; &
þęygi \alst{G}uðrún \hld\ \alst{g}ráta mátti, &
svá vas hǫ́n \alst{m}óðug; \hld\ \alst{m}undi hǫ́n springa.\eva

\bvb TODO... \\
Nowise could Guthrun weep; \\
so moody was she; she would burst apart.\evb\evg


\bvg\bva Sǫ́tu \alst{í}trar \hld\ \alst{ja}rla brúðir &
\alst{g}olli búnar \hld\ fyr \alst{G}uðrúnu; &
hvęr \alst{s}agði þęira \hld\ \alst{s}ínn of-trega &
þann’s \alst{b}itrastan \hld\ of \alst{b}eðit hafði.\eva

\bvb TODO.\evb\evg


\bvg\bva Þá kvað Gjaflaug, \hld\ Gjúka systir: &
„Mik vęit’k á moldu \hld\ munar-lausasta; &
hęfi’k fimm vera \hld\ for-spell beðit, &
tvęggja dǿtra, \hld\ þriggja systra, &
átta brǿðra, \hld\ þó ek ęin lifi.“\eva

\bvb TODO.\evb\evg


\bvg\bva Þęygi Guðrún \hld\ gráta mátti; &
svá vas hǫ́n móðug \hld\ at mǫg dauðan &
ok harð-huguð \hld\ um hrør fylkis.\eva

\bvb Nowise could Guthrun weep; \\
so moody was she after the lad’s death, \\
and hard-minded over the marshaller’s corpse.\evb\evg


\bvg\bva Þá kvað þat Hęrborg, \hld\ Húna lands dróttning: &
„Hęfi’k harðara \hld\ harm at sęgja: &
mínir sjau synir \hld\ sunnan lands, &
verr inn átti, \hld\ í val fellu:\eva

\bvb TODO.\evb\evg


\bvg\bva Faðir ok móðir, \hld\ fjórir brǿðr,
þau á vági \hld\ vindr of lék,
barði bára \hld\ við borð-þili.\eva

\bvb TODO.\evb\evg


\bvg\bva Sjǫlf skylda’k gǫfga, \hld\ sjǫlf skylda’k gǫtva, &
sjǫlf skylda’k hǫndla, \hld\ hęr-fǫr þęira; &
þat ek allt of bęið \hld\ ęin misseri &
svá’t mér maðr ęngi \hld\ munar lęitaði.\eva

\bvb TODO.\evb\evg


\bvg\bva Þá varð’k hapta \hld\ ok hęr-numa &
sams misseris \hld\ síðan verða; &
skylda’k skręyta \hld\ ok skúa binda &
hęrsis kván \hld\ hvęrjan morgin.\eva

\bvb TODO.\evb\evg


\bvg\bva Hon ǿgði mér \hld\ af af-brýði &
ok hǫrðum mik \hld\ hǫggum kęyrði; &
fann’k hús-guma \hld\ hvęrgi inn bętra &
en hús-fręyju \hld\ hvęrgi verri.“\eva

\bvb TODO, \\
and with hard blows drove me on; \\
a better husband I never found, \\
and a worse housewife never.”\evb\evg


\bvg\bva Þęygi Guðrún \hld\ gráta mátti; &
svá vas hǫ́n móðug \hld\ at mǫg dauðan &
ok harð-huguð \hld\ um hrør fylkis.\eva

\bvb Nowise could Guthrun weep; \\
so moody was she after the lad’s death, \\
and hard-minded over the marshaller’s corpse.\evb\evg


\bvg\bva Þá kvað þat Gullrǫnd, \hld\ Gjúka dóttir: &
„Fá kannt, fóstra, \hld\ þótt fróð séir, &
ungu vífi \hld\ and-spjǫll bera.“ &
Varaði hǫ́n at hylja \hld\ umb hrør fylkis.\eva

\bvb TODO.\evb\evg


\bvg\bva Svipti hǫ́n blę́ju \hld\ af Sigurði &
ok vatt vęngi \hld\ fyr vífs knjám: &
„Líttu á ljúfan, \hld\ lęgg þú munn við grǫn &
sem þú halsaðir \hld\ hęilan stilli.“\eva

\bvb TODO.\evb\evg


\bvg\bva Á lęit Guðrún \hld\ ęinu sinni; &
sá hǫ́n dǫglings skǫr \hld\ dręyra runna, &
fránar sjónir \hld\ fylkis liðnar, &
hug-borg jǫfurs \hld\ hjǫrvi skorna.\eva

\bvb TODO.\evb\evg


\bvg\bva Þá hné Guðrún \hld\ hǫll við bólstri; &
haddr losnaði, \hld\ hlýr roðnaði &
en regns dropi \hld\ rann niðr umb kné.\eva

\bvb her hair loosened, her cheek reddened, \\
and a drop of rain ran down to her knee.\evb\evg


\bvg\bva Þá grét Guðrún, \hld\ Gjúka dóttir, &
svá’t tǫ́r flugu \hld\ tresk í gǫgnum &
ok gullu við \hld\ gę̇ss í túni, &
mę́rir fuglar \hld\ es mę́r átti.\eva

\bvb Then wept Guthrun, Yivick’s daughter, \\
so that the tears flew through the ... \\
and in response shrieked the geese in the yard, \\
the famous fowls which the maiden owned.\evb\evg


\bvg\bva Þá kvað þat Gullrǫnd, \hld\ Gjúka dóttir: &
ykkar vissa’k \hld\ ȧstir męstar &
manna allra \hld\ fyr mold ofan; &
unðir þú hvárki \hld\ úti né inni, &
systir mín, \hld\ nema hjá Sigurði.\eva

\bvb TODO.\evb\evg


\bvg\bva%
„\edtext{\alst{S}vá vas mínn \alst{S}igurðr \hld\ hjá \alst{s}onum Gjúka &
sęm vę́ri \edtrans{\alst{g}ęir-laukr}{garlic}{\Bfootnote{or ‘speer-leek’. I have opted for this translation based on etymology (cf. OE \emph{gâr-léac} ‘spear-leek’), but the botanical identity is unclear. \GudrunTwo\ 2 has \emph{grǿnn laukr} ‘green leek’ instead. For the cultural importance of leeks and onions see note to \Voluspa\ 4.}} \hld\ ór \alst{g}rasi vaxinn,}{\lemma{Svá vas \dots\ vaxinn ‘So was \dots\ grown’}\Bfootnote{These two lines are almost identical to \GudrunTwo\ 2/1–2. Since the present poem is probably older \parencite{Sapp2022}, it is likely the source.}} &
\edtext{eða vę́ri \alst{b}jartr stęinn \hld\ ȧ \alst{b}and dręginn: &
\alst{j}arkna-stęinn \hld\ yfir \alst{ǫ}ðlingum.}{\lemma{eða vę́ri \dots\ ǫðlingum. ‘or were \dots\ athlings.’}\Bfootnote{Beaded necklaces were commonly worn by Scandinavian women of the time, and the beads were mostly of opaque coloured glass. Siward is thus likened to a bright crystal, the sons of Yivick to (dull) glass.}}\eva

\bvb\speakernoteb{[Guthrun quoth:]}So was my Siward by the sons of Yivick \\
like were a garlic out of grass grown, \\
or were a bright stone drawn on a band: \\
an \inx[C]{arkenstone} over the athlings.\evb\evg

\bvg\bva%
Ek \alst{þ}ȯtta auk \hld\ \alst{þ}jóðans rekkum &
\alst{h}vęrri \alst{h}ę́rri \hld\ \alst{H}ęrjans dísi; &
nú em’k svá \alst{l}ítil \hld\ sem \alst{l}auf séa &
\alst{op}t í \alst{jǫ}lstrum \hld\ at \alst{jǫ}fur dauðan.\eva

\bvb I, too, seemed to the prince’s champions \\
higher than each lady of the Lord of Hosts; \\
now I am as small as if a leaf I were,
up in the willows, after the ruler’s death.\evb\evg

TODO...

\sectionline
%
%	\include{books/Short Lay of Siward.tex}%
	\bookStart{Hell-ride of Byrnhild}[Hęlręið Brynhildar]
\def\thisBookCode{Helreid}

\begin{flushright}%
\textbf{Dating} \parencite{Sapp2022}: late C11th (0.650)

\textbf{Meter:} \Fornyrdislag
\end{flushright}%

\section{Introduction}

{\small Byrnhild is burned on her pyre in a beautiful chariot or wagon.  In the afterlife she rides on the \inx[L]{Hellway} to reach her resting place in \inx[L]{Hell}, and meets a \inx[C]{gow} or troll-woman on the way.  The poem consists of their conversation.}

\sectionline

\bpg\bpa Eptir dauða Brynhildar vóru gǫr bǫ́l tvau: annat Sigurði, ok brann þat fyrr, en Brynhildr var á ǫðru brennd ok var hon \edtrans{í reið þeiri er guð-vefjum var tjǫlduð}{in that chariot which was covered with godweb}{\Bfootnote{The tent-covering of the chariot was made of precious garments.  For the burial of women in wagons and chariots, cf. TODO (Oseberg ship?).}}.  Svá er sagt at \edtrans{Brynhildr ók með reið’inni á hel-veg}{Byrnhild drove with the chariot on the Hellway}{\Bfootnote{This gives us some interesting insight into old afterlife beliefs. After Byrnhild is burned she ends up between the worlds of the dead and the living, the so-called “Hell-way”, or road to Hell (the underworld); she is buried in a chariot so that she will be able to travel comfortably. We may presume that the animals driving the chariot were slaughtered and burnt with her on the pyre.}} ok fór um tún þar er gýgr nǫkkur bjó.  Gýgr’in kvað:\epa

\bpb After Byrnhild’s death two pyres were made: one for Siward, and it burned earlier; but Byrnhild was burned on the other, and she was in that chariot which was covered with \inx[C]{godweb}.  It is said that Byrnhild drove with the chariot onto the Hellway and passed through a plot where there lived a certain \inx[C]{gow}. The gow quoth:\epb\epg

\section{Byrnhild rode the Hellway (\emph{Brynhildr ręið hęl-veg})}

\bvg\bva%
„Skalt í \alst{g}ǫgnum \hld\ \alst{g}anga ęigi &
\alst{g}rjóti studda \hld\ \alst{g}arða mína; &
\alst{b}ętr sǿmði þér \hld\ \alst{b}orða at rękja &
hęldr an \alst{v}itja \hld\ \alst{v}ers annarar.\eva

\bvb “Thou shalt in no way go through \\
these rock-supported yards of mine; \\
it befits thee better to weave tapestries, \\
rather than visit another woman’s man.\evb\evg


\bvg\bva%
Hvat skalt \alst{v}itja \hld\ af \alst{V}al-landi, &
\alst{h}var-fu̇st \alst{h}ǫfuð, \hld\ \alst{h}úsa minna? &
Þú hęfir, \alst{V}ǫ́r gulls, \hld\ ef þik \alst{v}ita lystir, &
\alst{m}ild, af hǫndum \hld\ \alst{m}anns blóð þvegit.“\eva

\bvb Why shalt thou visit from Walland, \\
O straying head, these houses of mine? \\
Thou hast, mild \inx[P]{Ware} of gold, if thou hast lust to know, \\
washed a man’s blood off thy hands.”\evb\evg

Byrnhild answers:

\bvg\bva%
„\alst{B}regð ęigi mér, \hld\ \alst{b}rúðr ór stęini, &
þótt ek \alst{v}ę́ra’k \hld\ í \alst{v}íkingu; &
\alst{e}k mun \alst{o}kkur \hld\ \alst{ǿ}ðri þikkja &
hvar’s męnn \alst{ę}ðli \hld\ \alst{o}kkart kunna.“\eva

\bvb “Upbraid me not, O bride from the stone, \\
though I may have been in the sea-raid; \\
of us two will I seem the nobler, \\
wherever men know our lineages.”\evb\evg

The gow:

\bvg\bva%
„Þú vast, \alst{B}ryn-hildr, \hld\ \alst{B}uðla dóttir, &
\alst{h}ęilli verstu \hld\ í \alst{h}ęim borin; &
þú hęfir \alst{G}júka \hld\ of \alst{g}latat bǫrnum &
ok \alst{b}úi þęira \hld\ \alst{b}rugðit góðu.“\eva

\bvb “Thou wast, O Byrnhild, Budle’s daughter, \\
with the worst luck born into the world; \\
thou hast destroyed Yivick’s children, \\
and deprived their house of good.”\evb\evg

Byrnhild:

\bvg\bva%
„Ek mun \alst{s}ęgja þér, \hld\ \alst{s}vinn, ór ręiðu &
\alst{v}it-laussi mjǫk, \hld\ ef þik \alst{v}ita lystir: &
hvé \alst{g}ørðu mik \hld\ \alst{G}júka arfar &
\alst{ȧ}sta-lausa \hld\ ok \alst{ęi}ð-rofa.\eva

\bvb “I will tell thee, wise from my chariot, \\
O very witless one, if thou hast lust to know, \\
how Yivick’s heirs did make me \\
loveless, and an oath-breakeress.\evb\evg


\bvg\bva%
Lét hami vára \hld\ hug-fullr konungr, &
átta systra, \hld\ undir ęik borit; &
vas’k vetra tólf, \hld\ ef þik vita lystir, &
es ungum gram \hld\ ęiða sęlda’k.\eva

\bvb TODO. \\
I was twelve winters old, if thou hast lust to know, \\
when to the young prince I swore oaths.\evb\evg


\bvg\bva%
Hétu mik allir \hld\ í Hlym-dǫlum &
Hildi und hjalmi, \hld\ hvęrr es kunni.\eva

\bvb They all called me in the Limdales, \\
a Hild ’neath the helmet, whoever knew me.\evb\evg


\bvg\bva%
Þá lét’k gamlan \hld\ á Goð-þjóðu &
Hjalm-Gunnar nę́st \hld\ hęljar ganga; &
gaf’k ungum sigr \hld\ Auðu bróður; &
þar varð mér Óðinn \hld\ of-ręiðr um þat.\eva

\bvb Then I next among the Gots \\
made old Helm-Guther go the way of Hell; \\
I gave victory to Ead’s young brother; \\
there Weden was furious with me for that.\evb\evg


\bvg\bva%
Lauk hann mik skjǫldum \hld\ í Skata-lundi, &
rauðum ok hvítum, \hld\ randir snurtu; &
þann bað hann slíta \hld\ svefni mínum &
es hvęr-gi lands \hld\ hrę́ðask kynni.\eva

\bvb He locked me in with shields in Shatelund, \\
with red ones and white; their rims clasped. \\
He bade that one end my sleep, \\
who of no land could be frightened.\evb\evg


\bvg\bva%
Lét umb sal minn \hld\ sunnan-verðan &
hávan brenna \hld\ hęr alls viðar; &
þar bað hann ęinn þegn \hld\ yfir at ríða, &
þann’s mér fǿrði gull \hld\ þat’s und Fáfni lá.\eva

\bvb He made around my hall a south-facing, \\
high host of all wood \ken{fire} burn; \\
there he bade one thane ride over, \\
he who brought me the gold which ’neath Fathomer lay.\evb\evg


\bvg\bva%
Ręið góðr Grana \hld\ gull-miðlandi &
þar’s fóstri minn \hld\ flętjum stýrði; &
ęinn þótti hann þar \hld\ ǫllum bętri, &
víkingr Dana, \hld\ í verðungu.\eva

\bvb On Grane rode the good gold-dealer, \\
where my foster-son ruled the benches; \\
alone he seemed there better than all, \\
the Wiking of Danes, in the warband.\evb\evg


\bvg\bva%
\alst{S}vǫ́fu vit ok unðum \hld\ í \alst{s}ę́ing ęinni &
sem hann minn \alst{b}róðir \hld\ of \alst{b}orinn vę́ri; &
\alst{h}várt-ki knátti \hld\ \alst{h}ǫnd yfir annat &
\alst{á}tta nǫ́ttum \hld\ \alst{o}kkart lęggja.\eva

\bvb We slept and were content in one bed, \\
as if he were born my brother: \\
neither did lay a hand o’er the other \\
for eight nights, of us two.\evb\evg


\bvg\bva%
Því brá mér \alst{G}uðrún, \hld\ \alst{G}júka dóttir, &
at ek \alst{S}igurði \hld\ \alst{s}vę́fa’k á armi; &
þar varð’k þęss \alst{v}ís \hld\ es \alst{v}ildi’g-a’k &
at þau \alst{v}éltu mik \hld\ í \alst{v}er-fangi.\eva

\bvb Thus Guthrun upbraided me, Yivick’s daughter, \\
that I slept on Siward’s arm; \\
there I became wise of that which I wanted not, \\
that those two had tricked me in the catch of man.\evb\evg


\bvg\bva%
Munu við \alst{o}f-stríð \hld\ \alst{a}lls til lęngi &
\alst{k}onur ok \alst{k}arlar \hld\ \alst{k}vikkvir fǿðask; &
vit skulum \alst{o}kkrum \hld\ \alst{a}ldri slíta, &
\alst{S}igurðr, \alst{s}aman. \hld\ \alst{S}økks-tu, gýgjar-kyn!“\eva

\bvb In great strife for far too long \\
will men and women alive be born. \\
We two shall end our age, \\
I and Siward, together.—Sink, thou gow’s kin!”\evb\evg

\sectionline
% Byrnhild’s hellride
	\bookStart{The Second Lay of Guthrun}[Guðrúnarkviða aðra]

\begin{flushright}%
Dating \parencite{Sapp2022}: C10th (0.731), early C11th (0.178)

Meter: \Fornyrdislag%
\end{flushright}

TODO.

\sectionline

\section{The Slaying of the Nivlings (\emph{Dráp Niflunga})}

\bpg\bpa Gunnarr ok Hǫgni tóku þá gullit allt, Fáfnis arf. Ó-friðr var þá milli Gjúkunga ok Atla; kenndi hann Gjúkungum vǫld um and-lát Brynhildar. Þat var til sę́tta, at þeir skyldu gipta hánum Guðrúnu, ok gáfu henni ó·minnis-veig at drekka áðr hon játti at giptast Atla. Synir Atla vóru þeir Erpr ok Eitill, en Svanhildr var Sigurðar dóttir ok Guðrúnar. Atli konungr bauð heim Gunnari ok Hǫgna, ok sendi Vinga eða Knéfrøð. Guðrún vissi vélar ok sendi með rúnum orð at þeir skyldu eigi koma ok til jar-tegna sendi hon Hǫgna hringinn Andvaranaut ok knýtti í vargs-hár. Gunnarr hafði beðit Oddrúnar, systur Atla, ok gat eigi; þá fekk hann Glaumvarar, en Hǫgni átti Kostberu. Þeira synir vóru þeir Sólarr ok Snę́varr ok Gjúki. En er Gjúkungar kómu til Atla, þá bað Guðrún sonu sína at þeir bę́ði Gjúkungum lífs en þeir vildu eigi. Hjarta var skorit ór Hǫgna en Gunnarr settr í orm-garð. Hann sló hǫrpu ok svę́fði ormana en naðra stakk hann til lifrar. Þjóðrekr konungr var með Atla ok hafði þar látit flesta alla menn sína. Þjóðrekr ok Guðrún kę́rðu harma sín á milli. Hon sagði hánum ok kvað:\epa

\bpb Guther and Hain took all the gold, Fathomer’s inheritance. There was then enmity between the Yivickings and Attle; he blamed the Yivickings for Byrnhild’s passing. They came to terms that they would marry away Guthrun to him, and TODO. She spoke to him and quoth:\epb\epg


\bvg
\bva „\alst{M}ę́r vas’k \alst{m}eyja; \hld\ \alst{m}óðir mik fǿddi, &
\alst{b}jǫrt í \alst{b}úri; \hld\ unna’k vęl \alst{b}rǿðrum— &
unds mik \alst{G}júki \hld\ \alst{g}ulli ręifði, &
\alst{g}ulli ręifði, \hld\ \alst{g}af Sigurði.\eva

\bvb “A maiden was I of maidens; my mother raised me bright in the bowers; I loved well my brothers—until Yivick with gold endowed me, with gold endowed me, and gave [me] to Siward.\evb
\evg


\bvg
\bva „\alst{S}vá vas \alst{S}igurðr \hld\ uf \alst{s}onum Gjúka &
sem vę́ri \edtrans{\alst{g}rǿnn laukr}{green leek}{\Bfootnote{This st. shows that the leek was held to be the noblest of plants, something also seen by \Voluspa\ 4, where \emph{grǿnn laukr} it specifically mentioned as growing in the world’s very first days. See note there for its mythological significance.}} \hld\ ór \alst{g}rasi vaxinn, &
eða \alst{h}jǫrtr \alst{h}á-bęinn \hld\ um \alst{h}vǫssum dýrum, &
eða \alst{g}ull \alst{g}lóð-rautt \hld\ af \alst{g}rǫ́u silfri.“\eva

\bvb “So was Siward above the sons of Yivick, as were a green leek grown out of grass, or a high-boned hart in the midst of wild beasts, or glowing-red gold from grey silver.\evb
\evg
%
	\bookStart{The Third Lay of Guthrun}[Guðrúnarkviða þriðja]

\begin{flushright}%
Dating \parencite{Sapp2022}: C10th (0.731), early C11th (0.178)

Meter: \Fornyrdislag%
\end{flushright}

A very short narrative poem, depicting a single minor legendary event. It is especially notable for its depiction of a trial by ordeal and the mention of a woman being drowned in a bog.

Herch, one of Attle’s concubines tells Attle that she has seen his wife Guthrun sleeping with Thedric. Attle becomes distressed upon hearing this (P1). Guthrun asks him what is wrong (1), and he responds that Herch has accused her of sleeping with Thedric (2). Guthrun promises to to prove her innocence through a trial by ordeal involving picking up a white stone from boiling water (3). She further says that while she and Thedric did sit down together, they did so in mutual grief over the deaths of her brothers (4–5). She tells Attle to summon a German lord named Saxe, who knows how to carry out the trial. Seven hundred men arrive to witness the event (6). Before picking up the stone, Guthrun laments over her brothers’ deaths, saying that they would have disputed the accusation through violence, but that she must now prove her innocence by herself (7). She then puts her hand in the boiling water, and unscathed takes out the stones. She holds it up and shows it to the witnesses (8). Attle laughs, knowing that his wife has been faithful, and orders Herch to pick up the stone (9). She does so, but her hands are horribly scorched, and men lead her to a “foul bog”, presumably to be drowned (see above). The poet ends by laconically stating that Guthrun in such a way was “reconstituted for her affronts”.

\sectionline

\bpg\bpa Herkja hét ambǫ́tt Atla; hón hafði verit frilla hans. Hón sagði Atla at hón hefði sét Þjóðrek ok Guðrúnu bę́ði saman. Atli var þá allókátr. Þá kvað Guðrún:\epa

\bpb Herch was named the female thrall of Attle; she had been his concubine. She told Attle that she had seen Thedric and Guthrun both together. Attle was then wholly displeased. Then Guthrun quoth:\epb\epg


\bvg\bva „Hvat ’s þér, \alst{A}tli? \hld\ \alst{ę́}, Buðla sonr, &
es þér \alst{h}ryggt í \alst{h}ug; \hld\ hví \alst{h}lę́r þú ę́va? &
Hitt myndi \alst{ǿ}ðra \hld\ \alst{jǫ}rlum þykkja &
at við \alst{m}ęnn \alst{m}ę́ltir \hld\ ok \alst{m}ik sę́ir.“\eva

\bvb “What is with thee, Attle? Always, O son of Bodle, \\
art thou sad at heart—why laughest thou never? \\
TODO.”\evb\evg

%TODO: Add speaker notes

\bvg\bva „Tregr mik þat, \alst{G}uðrún, \hld\ \alst{G}júka dóttir, &
mér í \alst{h}ǫllu \hld\ \alst{H}ęrkja sagði &
at \alst{þ}it \alst{Þ}jóðrekr \hld\ undir \alst{þ}aki svę́fið &
ok \alst{l}éttliga \hld\ \alst{l}íni vęrðið.“\eva

\bvb “This troubles me, Guthrun, Yivick’s daughter: \\
in the hall has Herch told me \\
that thou and Thedric beneath thatched roof slept, \\
and ye lightly warded the linen.\footnoteB{i.e., they threw off their clothes and slept together.}”\evb\evg


\bvg\bva „Þér mun’k \alst{a}lls þęss \hld\ \alst{ęi}ða vinna &
at inum \alst{h}víta \hld\ \alst{h}ęlga stęini, &
at ek við \alst{Þ}jóðmar \hld\ \alst{þ}at-ki átta’k, &
es \alst{v}ǫrðr né \alst{v}err \hld\ v\alst{i}nna knátti,—\eva

\bvb “To thee I will swear oaths of all of that— \\
by the white, holy stone— \\
that I did not do such a thing with Thedmar,\footnoteB{Historically, Thedmar was the father of Thedric, who took over the kingdom after his father’s death (see Encyclopedia). Thedmar may here be a scribal error for Thedric, a scribal error for “Thedmar’s son”, or a nickname due to conflation of the father and son.} \\
which neither wife nor husband has been able to swear upon,—\footnoteB{Guthrun says that she will prove her innocence through a trial by ordeal (that is, by lifting “the white holy stone” out of boiling water; see st. 8). She further strengthens her position by pointing out that no reliable person has sworn an oath attesting to her guilt.}\evb\evg


\bvg\bva nema ek \alst{h}alsaða \hld\ \alst{h}ęrja stilli, &
\alst{jǫ}fur \alst{ó}·nęisinn, \hld\ \alst{ęi}nu sinni; &
\alst{a}ðrar vǫ́ru \hld\ \alst{o}kkrar spękjur &
es vit \alst{h}ǫrmug tvau \hld\ \alst{h}nigum at rúnum.\eva

\bvb unless I embraced the stiller of hosts \ken*{\textsc{ruler} = Thedmar}, \\
the unshamed prince a single time. \\
Different were our dealings, \\
when we two distressed ones [Guthrun and Thedric] reclined in private conversation.\evb\evg


\bvg\bva Hér kom \alst{Þ}jóðrekr \hld\ með \alst{þ}ría tøgu, &
lifa \alst{þ}ęir né ęinir, \hld\ \alst{þ}riggja tega manna; &
\edtrans{hrink-tu}{surround}{\Bfootnote{Consisting of \emph{hring}, 2nd sg. imper. of \emph{hringja} ‘surround, encircle’ + \emph{þú} ‘thou’.  The clitic form \emph{-tu} has caused devoicing.}} mik at \alst{b}rǿðrum \hld\ ok at \alst{b}rynjuðum, &
\alst{h}rink-tu mik at ǫllum \hld\ á \alst{h}ǫfuð-niðjum.\eva

\bvb Here came Thedric with thirty men; \\
of those thirty none still lives. \\
Surround me with brothers and with byrnied men; \\
surround me with all close kinsmen.\evb\evg


\bvg\bva \alst{S}ęnd at \alst{S}axa, \hld\ \alst{s}unn-manna gram; &
\alst{h}ann kann \alst{h}ęlga \hld\ \alst{h}ver vellanda;“ &
\alst{s}jau hundruð manna \hld\ í \alst{s}al gingu &
áðr \alst{k}vę́n \alst{k}onungs \hld\ í \alst{k}ętil tǿki.\eva

\bvb Send for Saxe, the lord of the Southmen, \\
he can hallow a boiling cauldron!” \\
Seven hundred men went into the hall, \\
before the king’s wife the kettle did touch.\evb\evg


\bvg\bva „\alst{K}ømr-a nú Gunnarr, \hld\ \alst{k}alli’k-a Hǫgna, &
\alst{s}é’k-a \alst{s}íðan \hld\ \alst{s}vása brǿðr; &
\alst{s}verði myndi Hǫgni \hld\ \alst{s}líks harms reka, &
nú verð’k \alst{s}jǫlf fyr mik \hld\ \alst{s}ynja lýta.“\eva

\bvb “Now Guther comes not, I cannot call on Hain; \\
I see not thereafter [my] beloved brothers. \\
y the sword would Hain avenge such an affront; \\
now I must for myself disprove the slanders!”\evb\evg


\bvg\bva \alst{B}rá hón til \alst{b}otns \hld\ \alst{b}jǫrtum lófa &
ok hón \alst{u}pp of tók \hld\ jarkna-stęina: &
„\alst{S}é nú \alst{s}ęggir \hld\ —\alst{s}ykn em ek orðin &
\alst{h}ęilag-liga— \hld\ hvé sjá \alst{h}verr velli.“\eva

\bvb She brought her bright palms to the bottom, \\
and she up did take the earkenstones: \\
“Let men now see—I am proven innocent, \\
through holy means!—how this cauldron boils!”\evb\evg


\bvg\bva \alst{H}ló þá Atla \hld\ \alst{h}ugr í brjósti &
es hann \alst{h}ęilar sá \hld\ \alst{h}ęndr Guðrúnar: &
„Nú skal \alst{H}ęrkja \hld\ til \alst{h}vers ganga, &
sú’s \alst{G}uðrúnu \hld\ \alst{g}randi vę́nti.“\eva

\bvb Then laughed the heart in Attle’s chest, \\
when he saw the unscathed hands of Guthrun: \\
“Now shall Herch to the cauldron go, \\
she who to Guthrun hoped to cause harm.”\evb\evg


\bvg\bva \alst{S}á-at maðr armligt, \hld\ hvęrr es þat \alst{s}á-at, &
\alst{h}vé þar á \alst{H}ęrkju \hld\ \alst{h}ęndr sviðnuðu; &
lęiddu þá \alst{m}ęy \hld\ í \alst{m}ýri fúla, &
\alst{s}vá þá Guðrún \hld\ \alst{s}inna harma.\eva

\bvb Man has not seen something pitiful, who has not seen that: \\
how there on Herch the hands were scorched. \\
Led they the maiden into the foul bog; \\
so was Guthrun reconstituted for her affronts.\evb\evg

\sectionline
%
	\bookStart{The Weeping of Ordrun}[Oddrúnargrátr]

\begin{flushright}%
Dating \parencite{Sapp2022}: C10th (0.954)

Meter: \Fornyrdislag%
\end{flushright}%

% Introduction

\section{From Burgny and Ordrun (\emph{Frá Borgnýju ok Oddrúnu})}

\bpg\bpa Heiðrekr hét konungr; dóttir hans hét Borgný. Vilmundr hét sá er var friðill hennar. Hon mátti eigi fǿða bǫrn áðr til kom Oddrún, Atla systir; hon hafði verit unnusta Gunnars, Gjúka sonar. Um þessa sǫgu er hér kveðit:\epa

\bpb Heathric was a king called, his daughter was called Burgny. Wilmund was he called who was her lover. She could not bear children before Ordrun, Attle’s sister, came to her. She had been the lover of Guther, Yivick’s son. Of this saw is here sung:\epb\epg


\bvg
\bva Hęyrða’k \alst{s}ęgja \hld\ í \alst{s}ǫgum fornum &
hvé \alst{m}ę́r of kom \hld\ til \alst{M}orna-lands; &
\alst{ę}ngi mátti \hld\ fyr \alst{jǫ}rð ofan &
\alst{H}ęiðreks dóttur \hld\ \alst{h}jalpir vinna.\eva

\bvb I heard [it] said in ancient saws,\footnoteB{Probably formulaic; cf. \Hildebrandslied\ 1: \emph{ik gi-hórta dat seggen} ‘I heard it said’ which likewise uses the 1sg pret. of ‘hear’ and the infinitive of ‘say’. Both would go back to a Proto-Northwest Germanic phrase \emph{*ek (ga-)hauʀidō (þat) sagjaną}.} \\
how a maiden came to Mornland; \\
noone could—above the earth— \\
find help for Heathric’s daughter \ken*{= Burgny}.\evb
\evg


\bvg
\bva Þat frá \alst{O}ddrún, \hld\ \alst{A}tla systir, &
at sú \alst{m}ę́r hafði \hld\ \alst{m}iklar sóttir; &
brá hon af \alst{st}alli \hld\ \alst{st}jórn-bitluðum &
ok á \alst{s}vartan \hld\ \alst{s}ǫðul of lagði.\eva

\bvb This learned Ordrun, Attle’s sister, \\
that the maiden \ken*{= Burgny} had great ailments; \\
she grabbed from the stable a rudder-bitted steed, \\
and a black saddle on [it] did lay.\evb
\evg


\bvg
\bva Lét hon \alst{m}ar fara \hld\ \alst{m}old-veg sléttan &
unds at \alst{h}ári kom \hld\ \alst{h}ǫll standandi; &
\edtext{ok hon \alst{i}nn of gekk \hld\ \alst{ę}nd-langan sal;}{\lemma{ok hon \dots\ sal ‘and she ... hall’}\Bfootnote{The whole line is formulaic, see note to \Volundarkvida\ 8.}} &
\alst{s}vipti hon \alst{s}ǫðli \hld\ af \alst{s}vǫngum jó &
\edtext{ok hon þat \alst{o}rða \hld\ \alst{a}lls fyrst of kvað:}{\lemma{ok \dots\ of kvað ‘and ... did say’}\Bfootnote{The whole line is formulaic, see note to \Thrymskvida\ 2.}}\eva

\bvb She let the steed journey on the smooth soil-way \ken{earth}, \\
until she came to the high standing hall, \\
and she inside did go the endlong hall. \\
She drew the saddle off the slender horse, \\
and she that word first of all did say:\evb
\evg

TODO: More verses.
% Weeping of Ordrun
	\bookStart{The Lay of Attle}[Atlakviða]

\begin{flushright}%
Dating \parencite{Sapp2022}: C10th (0.719)–early C11th (0.212)

Meter: \Malahattr, \Fornyrdislag
\end{flushright}%

A famously archaic poem.

Attle sends his messenger Kneefrith to Guther (1). He arrives at Guther’s hall, where the mood is one of unease, and addresses Guther (2). Kneefrith invites him and his brother Hain to Attle’s court (3), offering them treasures, weapons and land (4–5). Guther asks his brother Hain for advice, since he has not heard of Attle having gold to give away (6).

\sectionline

\section{The Death of Attle (\emph{Dauði Atla})}

\bpg\bpa Guðrún Gjúkadóttir hefndi brǿðra sinna, svá sem frę́gt er orðit. Hon drap fyrst sonu Atla, en eptir drap hon Atla ok brendi hǫllina ok hirðina alla; um þetta er sjá kviða ort.\epa

\bpb Guthrun Yivicksdaughter avenged her brothers, as has become famous. She first killed the sons of Attle, and after that she killed Attle, and burned the hall and the whole hird. Regarding that this lay is wrought.\epb
\epg

\sectionline

\bvg\bva \alst{A}tli sęndi \hld\ \alst{á}r til Gunnars &
\alst{k}unnan sęgg at ríða, \hld\ \alst{K}néfrøðr vas sá hęitinn; &
at \alst{g}ǫrðum kom hann \alst{G}júka \hld\ ok at \alst{G}unnars hǫllu, &
\alst{b}ękkjum arin-gręypum \hld\ ok at \alst{b}jóri svǫ́sum.\eva

\bvb Attle sent—of yore–to Guther \\
a well-known messenger to ride; Kneefrith he was called. \\
To the yards of Yivick he came, and to the hall of Guther; \\
to the hearth-surrounding benches, and to the lovely beer.\evb\evg


\bvg\bva \alst{D}rukku þar \alst{d}rótt-męgir \hld\ —ęn \edtrans{\alst{d}yljęndr}{concealed ones}{\Bfootnote{\textcite{FinnurEdda} reasonably interprets this as referring to Attle’s spies at Guther’s court.}} þǫgðu— &
\alst{v}ín í \edtrans{\alst{v}al-hǫllu}{the walhall}{\Bfootnote{The interpretation of this compound is difficult in the current context. The first element \emph{val-} could be (1) \emph{valr} ‘falcon’, referring to the aristocratic hunting practice; (2) \emph{valr} ‘\inx[G]{Wales}[Wale]’, cognate with ‘Welsh’ but in ON referring to the French or Romans, stressing the southern location or appearance of the hall; or (3) \emph{valr} ‘(collective) the battle-slain’, foreshadowing the inevitable death (\inx[C]{feyness}) of the \inx[G]{Yivickings}. If (3) is correct the word is linguistically identical to \inx[L]{Walhall}, Weden’s hall, whither the battle-slain go.}}, \hld\ \alst{v}ręiði sǫ́usk þęir Húna; &
\alst{k}allaði þá \alst{K}néfrøðr \hld\ \alst{k}aldri rǫddu, &
\alst{s}ęggr inn \alst{s}uð-rǿni \hld\ \alst{s}at hann á bękk hǫ́m:\eva

\bvb There the dright-lads \ken{warriors} drank—but the concealed ones shut up— \\
wine in the walhall; they feared the wrath of the Huns. \\
Then called Kneefrith with cold voice, \\
the southern messenger, he sat on a high bench:\evb\evg


\bvg\bva „\alst{A}tli mik hingat sęndi \hld\ ríða \alst{ø}ręndi, &
\alst{m}ar inum \alst{m}él-gręypa, \hld\ \alst{M}yrk-við inn ó·kunna &
at \alst{b}iðja yðr, Gunnarr, \hld\ at it á \alst{b}ękk kǿmið &
með \alst{h}jǫlmum arin-gręypum \hld\ at sǿkja \alst{h}ęim Atla.\eva

\bvb “Attle sent me hither to ride with an errand, \\
on the bit-champing steed through uncharted Mirkwood— \\
to ask you, O Guther, that ye two \ken*{= Guther and Hain} on the bench come, \\
with hearth-surrounding helmets, to seek the home of Attle.\evb\evg


\bvg\bva \alst{Sk}jǫldu kneguð þar vęlja \hld\ ok \alst{sk}afna aska, &
\alst{h}jalma gull-roðna \hld\ ok \alst{H}úna męngi, &
\alst{s}ilfr-gyllt \alst{s}ǫðul-klę́ði, \hld\ \alst{s}ęrki val-rauða, &
\alst{d}afar, \alst{d}arraða, \hld\ \alst{d}rǫsla mél-gręypa.\eva

\bvb There ye might choose shields, and shaven ash-spears, \\
helmets gold-reddened, and the multitude of the Huns, \\
silver-gilt saddle-cloths, blood-red serks, \\
daves, spears, bit-champing steeds.\evb\evg


\bvg\bva \alst{V}ǫll létsk ykkr ok myndu gefa \hld\ \alst{v}íðrar Gnita-hęiðar &
af \alst{g}ęiri \alst{g}jallanda \hld\ ok af \alst{g}ylltum stǫfnum, &
\alst{st}órar męiðmar \hld\ ok \alst{st}aði Danpar, &
hrís þat it \alst{m}ę́ra \hld\ es meðr \alst{M}yrk-við kalla.“\eva

\bvb He also declared himself willing to give you two the field of wide Gnit-heath, \\
{[and]} of yelling spears and of gilded prows, \\
great treasures and the place of Danp; \\
the renowned brush which men call Mirkwood.\evb\evg


\bvg\bva \alst{H}ǫfði vatt þá Gunnarr \hld\ ok \alst{H}ǫgna til sagði: &
„Hvat rę́ðr þú okkr, \alst{s}ęggr hinn ǿri, \hld\ alls vit \alst{s}líkt hęyrum? &
\alst{G}ull vissa’k ękki \hld\ á \alst{G}nita-hęiði, &
þat’s vit \alst{ę́}ttim-a \hld\ \alst{a}nnat slíkt.\eva

\bvb His head turned Guther then, and said to Hain: \\
“What dost thou counsel us two, O younger man, as such a thing we hear? \\
I knew of no gold on the Gnit-heath \\
which we two should not own as much of.\evb\evg


\bvg\bva \alst{S}jau ęigu vit \alst{s}al-hús \hld\ \alst{s}verða full, &
\alst{h}vęrju ’ru þęira \hld\ \alst{h}jǫlt ór gulli; &
\alst{m}ínn vęit’k \alst{m}ar bętstan \hld\ en \alst{m}ę́ki hvassastan, &
\alst{b}oga \alst{b}ękk-sǿma \hld\ en \alst{b}rynjur ór gulli;\eva

\bvb We own seven hall-houses filled with swords— \\
on each of them is a golden hilt; \\
I know my horse to be the best and {[my]} sword the sharpest, \\
{[my]} bow bench-fit and {[my]} byrnies golden,\evb\evg


\bvg\bva \alst{h}jalm ok skjǫld \alst{h}vítastan, \hld\ kominn ór \alst{h}ǫll Kjárs; &
\alst{ęi}nn ’s mínn bętri \hld\ en sé \alst{a}llra Húna.“\eva

\bvb {[my]} helmet and whitest shield, come from Caser’s hall; \\
mine alone is better, than [those] of all of the Huns might be!”\evb\evg


\bvg\bva „Hvat hyggr \alst{b}rúði \alst{b}ęndu \hld\ þá’s hón okkr \alst{b}aug sęndi, &
\alst{v}arinn \alst{v}ǫ́ðum hęiðingja? \hld\ Hykk at hón \alst{v}ǫrnuð byði! &
\alst{H}ár fann’k \alst{h}ęiðingja \hld\ riðit í \alst{h}ring rauðum; &
\alst{y}lfskr es vegr \alst{o}kkarr \hld\ at ríða \alst{ø}ręndi.“\eva

\bvb {[Hain quoth:]} \\
“What dost thou think the bride meant, when she sent us two an armlet \\
wrapped with a heath-dweller’s garment \ken{wolf > wolf’s hair}? I think that she gave us a warning! \\
I found the heath-dweller’s \ken{wolf’s} hair tied through the red ring: \\
wolven is our road, if we ride that errand!\footnoteB{That it is the more cautious Hain who speaks here is clear from Guther’s response in the following stanzas.  Whereas Hain judges the wolf-hair to be a warning of Hunnish treachery, Guther thinks that it is a warning that wolves will steal his treasure if he does not show up.}”\evb\evg


\bvg\bva \alst{N}iðjar-gi hvǫttu Gunnar \hld\ né \alst{n}áungr annarr, &
\alst{r}ýnęndr né \alst{r}áðęndr, \hld\ né þęir’s \alst{r}íkir vǫ́ru; &
\alst{k}vaddi þá Gunnarr \hld\ sęm \alst{k}onungr skyldi, &
\alst{m}ę́rr í \alst{m}jǫð-ranni \hld\ af \alst{m}óði stórum:\eva

\bvb Kinsmen urged not Guther, nor any other relation, \\
not counselors nor advisors, nor those who were mighty. \\
Guther then announced—as a king should, \\
renowned in the mead-hall—with great spirit:\evb\evg


\bvg\bva „Rís-tu nú, \edtrans{\alst{F}jǫrnir}{Ferner}{\Bfootnote{An otherwise unknown servant.}}, \hld\ lát-tu á \alst{f}lęt vaða &
\alst{g}ręppa \alst{g}ull-skálir \hld\ með \alst{g}umna hǫndum!\eva

\bvb “Rise now, Ferner; let on the floorboards wade forth \\
the golden bowls of warriors along the hands of men!\evb\evg


\bvg\bva \alst{U}lfr mun ráða \hld\ \alst{a}rfi Niflunga, &
\alst{g}amlir \alst{g}ran-varðir, \hld\ ef \alst{G}unnars missir, &
\alst{b}irnir \alst{b}lakk-fjallir \hld\ \alst{b}íta þref-tǫnnum, &
\alst{g}amna \alst{g}ręy-stóði, \hld\ ef \alst{G}unnarr né kømr-at.“\eva

\bvb The wolf will rule the inheritance of the Nivlings— \\
the old grey guardians \ken{wolves}—if Guther is missing. \\
Black-furred bears [will] bite with wrangling teeth— \\
amusing the bitch-pack—if Guther comes not.”\evb\evg


\bvg\bva \alst{L}ęiddu land-rǫgni \hld\ \edtrans{\alst{l}ýðar ó·nęisir}{unshamed \ken{famous} people}{\Bfootnote{Compare the long-line on the Thorsberg chape (\~ 160–240): \emph{wlþuþewaʀ \hld\ ni wajē-māriʀ} ‘Wolthew, the not ill-famed \ken{famous}’.}}, &
\alst{g}rátęndr, \alst{g}unn-hvatan, \hld\ ór \alst{g}arði Húna; &
þá kvað þat inn \alst{ǿ}ri \hld\ \alst{ę}rfi-vǫrðr Hǫgna: &
„\alst{H}ęilir farið nú ok \alst{h}orskir \hld\ hvar’s ykkr \alst{h}ugr tęygir!“\eva

\bvb TODO Then quoth that the young inheritance-ward \ken{son} of Hain: “Whole fare ye two now, and wise, wherever your hearts may draw!”\evb\evg


\bvg\bva \alst{F}etum létu \alst{f}rǿknir \hld\ of \alst{f}jǫll at þyrja &
\alst{m}ar ina \alst{m}él-gręypu, \hld\ \alst{M}yrk-við inn ókunna; &
\alst{h}ristisk ǫll \alst{H}ún-mǫrk \hld\ þar’s \alst{h}arð-móðgir fóru, &
\alst{v}rǫ́ku þęir \alst{v}ann-styggva \hld\ \alst{v}ǫllu al-grǿna.\eva

\bvb By their feet made the valiant ones over the fellss \\
the bit-champing steed rush along, through uncharted Mirkwood. \\
TODO.\evb\evg


\bvg\bva \alst{L}and sǫ́u þęir Atla \hld\ ok \alst{l}ið-skjalfar djúpar &
\alst{B}ikka greppar standa \hld\ á \alst{b}org inni hǫ́u, &
\alst{s}al of \alst{s}uðr-þjóðum, \hld\ \alst{s}lęginn sess-męiðum, &
\alst{b}undnum rǫndum, \hld\ \alst{b}lęikum skjǫldum,\eva

\bvb They saw the land of Attle, and deep valleys(?); \\
the warriors of Bicke standing on the high fortress \\
TODO\evb\evg


\bvg\bva \alst{d}afar, \alst{d}arraða; \hld\ en þar \alst{d}rakk Atli &
\alst{v}ín í \alst{v}al-hǫllu; \hld\ \alst{v}ęrðir sǫ́tu úti &
at \alst{v}arða þęim Gunnari \hld\ ef þęir hér \alst{v}itja kǿmi &
með \alst{g}ęiri \alst{g}jallanda \hld\ at vękja \alst{g}ram hildi.\eva

\bvb daves, spears; but there drank Attle \\
wine in the wale-hall; the watchmen sat outside \\
to watch for Guther’s men, if they came here to visit, \\
with yelling spear, to wake the ruler with war.\evb\evg


\bvg\bva \alst{S}ystir fann þęira \alst{s}nemmst \hld\ at þęir í \alst{s}al kvǫ́mu, &
\alst{b}rǿðr hęnnar \alst{b}áðir, \hld\ \alst{b}jóri vas hón lítt drukkin: &
„\alst{R}áðinn est nú, Gunnarr, \hld\ hvat munt, \alst{r}íkr, vinna &
við \alst{H}úna \alst{h}arm-brǫgðum? \hld\ \alst{H}ǫll gakk þú ór snemma!\eva

\bvb Their sister found earliest they they had come into the hall, \\
both of her brothers—on beer was she lightly drunk: \\
“Betrayed art thou now, Guther; how wilt thou, powerful man, work \\
against the harm-tricks of the Huns? Go early out of the hall!\footnoteB{Before anything evil might happen.}”\evb\evg


\bvg\bva \alst{B}ętr hęfðir þú, \alst{b}róðir, \hld\ at þú í \alst{b}rynju fǿrir, &
sęm \alst{h}jǫlmum arin-gręypum \hld\ at séa \alst{h}ęim Atla; &
\alst{s}ę́tir þú í \alst{s}ǫðlum \hld\ \alst{s}ól-hęiða daga, &
\alst{n}ái \alst{n}auð-fǫlva \hld\ létir \alst{n}ornir gráta.\eva

\bvb Better hadst thou, brother, if thou went in byrnie \\
with hearth-surrounding helmets, to see the home of Attle— \\
if thou placed in the saddle—during sun-bright days— \\
need-pale corpses, [if thou] made the norns cry;\evb\evg


\bvg\bva \alst{H}úna skjald-męyjar \hld\ \alst{h}ęrfi kanna &
en \alst{A}tla sjalfan \hld\ létir í \alst{o}rm-garð koma; &
nú ’s sá \alst{o}rm-garðr \hld\ \alst{y}kkr of folginn.“\eva

\bvb {[if thou made]} the shield-maidens of the Huns to know the harrow,\footnoteB{i.e. if he turned the Hunnish shield-maidens into enslaved farmhands.} \\
and Attle himself thou brought into the snake-pit— \\
now is that snake-pit enclosing you two!”\evb\evg


\bvg\bva „\alst{S}ęinað ’s nú, systir, \hld\ at \alst{s}amna Niflungum, &
\alst{l}angt ’s at \alst{l}ęita \hld\ \alst{l}ýða sinnis til, &
of \alst{r}osmu-fjǫll \alst{R}ínar, \hld\ \alst{r}ekka ó·nęissa.“\eva

\bvb “’Tis late now, O sister, to gather the Nivlings; \\
’tis far to look for the support of men— \\
over the fells of the Rhine—for unshamed \ken{famous} warriors.”\evb\evg


\bvg\bva \alst{F}engu þęir Gunnar \hld\ ok í \alst{f}jǫtur sęttu, &
vinir \alst{B}orgunda, \hld\ ok \alst{b}undu fastla; &
\alst{s}jau hjó Hǫgni \hld\ \alst{s}verði hvǫssu &
en inum \alst{á}tta hratt hann \hld\ í \alst{ę}ld hęitan.\eva

\bvb Caught they Guther, and in fetters set him— \\
the friends of the Burgends—and bound them tightly. \\
Hain hewed down seven with sharp sword, \\
but the eighth one he threw into hot fire.\evb\evg


\bvg\bva \edtext{Svá skal \alst{f}rǿkn \hld\ \alst{f}jándum vęrjask;}{\lemma{Svá \dots\ vęrjask}\Bfootnote{Line moved from the last st. to this one since it seems to connect semantically with the immediately following line, and also creates a regular line distribution of 4-4 instead of 5-3.}} &
\alst{H}ǫgni varði \hld\ \alst{h}ęndr Gunnars. &
\alst{f}rǫ́gu \alst{f}rǿknan \hld\ ef \alst{f}jǫr vildi &
\alst{G}otna þjóðann \hld\ \alst{g}ulli kaupa.\eva

\bvb Thus shall the bold against fiends ward himself; \\
Hain warded the hands of Guther. \\
They asked the bold man \ken*{= Guther} if his life he wished— \\
the ruler of the Gots—to buy with gold.\footnoteB{The Huns ask Guther (it is clear that “ruler of the Gots” refers to him, cf. sts. 1, 3, 10) if he wishes to ransom Hain. He instead responds with the following:}\evb\evg


\bvg\bva „\alst{H}jarta skal mér \alst{H}ǫgna \hld\ í \alst{h}ęndi liggja &
\alst{b}lóðugt, ór \alst{b}rjósti \hld\ skorit \alst{b}ald-riða, &
\alst{s}axi \alst{s}líðr-bęitu, \hld\ \alst{s}yni þjóðans.“\eva

\bvb {[Guther quoth:]} “The heart of Hain shall lie in my hands: \\
bloody from the breast, cut from the bold rider \ken*{= Hain}, \\
with a slide-biting sax,\footnoteB{i.e. a short-sword with a blade so sharp that it draws blood when one slides the finger across it.} from the son of the sovereign \ken*{= Hain}.”\evb\evg


\bvg\bva Skǫ́ru þęir \alst{h}jarta \hld\ \alst{H}jalla ór brjósti, &
\alst{b}lóðugt, ok á \alst{b}jóð lǫgðu \hld\ ok \alst{b}ǫ́ru þat fyr Gunnar.\eva

\bvb Cut they the heart of Helle from the breast, \\
bloody, and on a platter laid it, and carried it before Guther.\evb\evg


\bvg\bva Þá kvað þat \alst{G}unnarr, \hld\ \alst{g}umna dróttinn: &
„\alst{H}ér hęfi’k \alst{h}jarta \hld\ \alst{H}jalla ins blauða, &
ó·líkt \alst{h}jarta \hld\ \alst{H}ǫgna ins frǿkna, &
es mjǫk \alst{b}ifask \hld\ es á \alst{b}jóði liggr; &
\alst{b}ifðisk hǫlfu męirr \hld\ es í \alst{b}rjósti lá!“\eva

\bvb Then quoth that Guther, the lord of men: \\
“Here have I the heart of Helle the soft—unlike the heart of Hain the bold!— \\
which much trembles when on the platter it lies; \\
it trembled twice as much when in the breast it lay.”\evb\evg


\bvg\bva \alst{H}ló þá \alst{H}ǫgni \hld\ es til \alst{h}jarta skǫ́ru &
\alst{k}vikvan \alst{k}umbla-smið; \hld\ \alst{k}løkkva síðst hugði; &
\alst{b}lóðugt þat á \alst{b}jóð lǫgðu \hld\ ok \alst{b}ǫ́ru fyr Gunnar.\eva

\bvb Hain laughed then, when unto the heart they cut \\
the living wound-smith \ken*{\textsc{warrior} = Hain}; he thought least of sobbing. \\
Bloody on a platter they laid it, and carried it before Guther.\evb\evg


\bvg\bva Mę́rr kvað þat \alst{G}unnarr, \hld\ \alst{G}ęir-Niflungr: &
„\alst{H}ér hęfi’k \alst{h}jarta \hld\ \alst{H}ǫgna ins frǿkna, &
ó·líkt \alst{h}jarta \hld\ \alst{H}jalla ins blauða, &
es lítt \alst{b}ifask \hld\ es á \alst{b}jóði liggr; &
\alst{b}ifðisk svá-gi mjǫk \hld\ þá’s í \alst{b}rjósti lá!\eva

\bvb Renowned, quoth Guther, the Spear-Nivling: \\
“Here have I the heart of Hain the bold—unlike the heart of Helle the soft!— \\
which little trembles, when on the platter it lies; \\
it trembled not so much when in the breast it lay.\evb\evg


\bvg\bva Svá skalt, \alst{A}tli, \hld\ \alst{au}gum fjarri &
sęm \alst{m}unt \hld\ \alst{m}ęnjum verða; &
es und ęinum mér \hld\ ǫll of folgin &
hodd Niflunga: \hld\ lifir-a nú Hǫgni!\eva

\bvb Thus shalt thou, Attle, be as far from the eyes \\
as thou wilt from the neck-rings. \\
With me alone are all concealed \\
the hoards of the Nivlings—now Hain lives not!\evb\evg


\bvg\bva Ęy vas mér týja \hld\ meðan vit tvęir lifðum, &
nú ’s mér ęngi \hld\ es ęinn lifi’k; &
Rín skal ráða \hld\ róg-malmi skatna, &
svinn, ǫ́s-kunna \hld\ arfi Niflunga.\eva

\bvb I was ever in doubt when we \emph{two} lived; \\
now I am not when alone I live. \\
The Rhine shall rule the strife-ore of princes \ken{gold}— \\
swift [river]—the os-born inheritance of the Nivlings!\evb\evg


\bvg\bva Í veltanda vatni \hld\ lýsask val-baugar &
hęldr an á hǫndum gull \hld\ skíni Húna bǫrnum.“\eva

\bvb In tumbling water [shall] the Welsh bighs gleam, \\
rather than gold might shine on the hands of the children of Huns!”\evb\evg


\bvg\bva “Ýkvið ér hvél-vǫgnum, \hld\ haptr ’s nú í bǫndum!”\eva

\bvb “Turn ye the wheel-wagons—the captive is now in bonds!”\evb\evg


\bvg\bva Atli inn ríki\eva

\bvb TODO\evb\evg


\bvg\bva Svá gangi þér\eva

\bvb TODO\evb\evg


\bvg\bva ok meirr þaðan\eva

\bvb TODO\evb\evg


\bvg\bva Lifanda gram\eva

\bvb TODO\evb\evg


\bvg\bva Glumðu stręngir;\eva

\bvb TODO\evb\evg


\bvg\bva Dynr vas í garði,\eva

\bvb TODO\evb\evg


\bvg\bva Út gekk þá Guðrún,\eva

\bvb TODO\evb\evg


\bvg\bva Umðu ǫlskálir\eva

\bvb TODO\evb\evg


\bvg\bva Út gekk þá Guðrún,\eva

\bvb TODO\evb\evg


\bvg\bva Skævaði þá in skírleita\eva

\bvb TODO\evb\evg


\bvg\bva Sona hefir þinna, \eva

\bvb TODO\evb\evg


\bvg\bva Kallar-a þú síðan \eva

\bvb TODO\evb\evg


\bvg\bva Ymr varð á bekkjum, \eva

\bvb TODO\evb\evg


\bvg\bva Gulli seri \eva

\bvb TODO\evb\evg


\bvg\bva Óvarr Atli, \eva

\bvb TODO\evb\evg


\bvg\bva Hon beð broddi \eva

\bvb TODO\evb\evg


\bvg\bva Ęldi gaf hón alla \hld\ es inni vǫ́ru &
ok frá morði þęira Gunnars \hld\ komnir vǫ́ru ór Myrk-hęimi; &
forn timbr fellu, \hld\ fjarg-hús ruku, &
bǿr Buðlunga, \hld\ brunnu ok skjald-męyjar, &
inni aldr-stamar \hld\ hnigu í ęld hęitan.\eva

\bvb To the fire she gave all those who were inside \\
and from the murder of Guther’s men had come out of Mirkham. \\
Ancient timbers fell; great houses smoked— \\
the settlement of the Buthlungs—burned also the shield–maidens; \\
inside aged trunks sank into hot fire.\evb\evg


\bvg\bva Full-rǿtt’s umb þetta; \hld\ fęrr ęngi svá síðan &
brúðr í brynju \hld\ brǿðra at hęfna; &
hón hęfir þriggja \hld\ þjóð-konunga &
\edtrans{ban-orð borit}{borne the bane-words}{\Bfootnote{\footnoteB{i.e. “she has caused the deaths of three great kings.” This expression and its Germanic and Indo-European relatives is discussed in detail in \textcite{Watkins1995}[417--422].}}}, \hld\ bjǫrt, áðr sylti.\eva

\bvb ’Tis told fully about this: none fares afterwards so, \\
a bride in byrnie, her brothers to avenge. \\
She has of three great kings \\
borne the bane-words—bright woman—before she must die.\evb\evg


\bvg\bva Enn segir gleggra í Atlamálum inum grǿn-lenskum.\eva

\bvb Yet says it more clearly in the Greenlendish Speeches of Attle.\evb\evg

\sectionline
% Lay of Attle
%	\include{books/Greenlendish Speeches of Attle.tex}%
	\bookStart{Goading of Guthrun}[Guðrúnarhvǫt]

\begin{flushright}%
\textbf{Dating} \parencite{Sapp2022}: early C11th (0.781)–late C11th (0.177)

\textbf{Meter:} \Fornyrdislag
\end{flushright}%

\section{Introduction}

TODO: INTRODUCTION.

\sectionline

\section{From Guthrun (\emph{Frá Guðrúnu})}

\bpg\bpa Guðrún gekk þá til sę́var er hon hafði drepit Atla, gekk út á sę́inn ok vildi fara sér. Hon mátti eigi søkkva. Rak hana yfir fjǫrðinn á land Jónakrs konungs. Hann fekk hennar. Þeira synir vóru þeir Sǫrli ok Erpr ok Hamðir. Þar fǿddisk upp Svanhildr Sigurðar dóttir. Hon var gift Jǫrmunrekk inum ríkja. Með hánum var Bikki. Hann réð þat at Randvér konungs son skyldi taka hana; þat sagði Bikki konungi. Konungr lét hengja Randvé en troða Svanhildi undir hrossa fótum. En er þat spurði Guðrún þá kvaddi hon sonu sína.\epa

\bpb Guthrun then went to the sea after she had slain Attle; walked out into the sea and wanted to take her own life. She could not sink. She was driven across the firth to the land of king Enacker. He got her. Their sons were Sarrel and Earp and Hamthew. There Swanhild, Siward’s daughter was raised up. She was married to Erminric the powerful; with him was \inx[P]{Bicke}. He counseled that Randwigh, the king’s son, should rape her; this Bicke told the king. The king had Randwigh hanged and Swanhild trampled under horses’ feet. But when Guthrun learned of this she called on her sons.\epb\epg

\sectionline

\section{The Goading of Guthrun}

\bvg\bva Þá frá’k \alst{s}ęnnu \hld\ \alst{s}líðr-fęng-ligasta, &
\alst{t}rauð mǫ́l \alst{t}alit \hld\ af \alst{t}rega stórum, &
es \alst{h}arð-\alst{h}uguð \hld\ \alst{h}vatti at vígi &
\alst{g}rimmum orðum \hld\ \alst{G}uðrún sonu:\eva

\bvb That gibing I’ve found most direly caught— \\
loth speeches told from great grief— \\
when hard-hearted she goaded to war, \\
with fierce words, Guthrun, her sons:\evb\evg


\bvg\bva „Hví \alst{s}itið? \hld\ Hví \alst{s}ofið lífi? &
Hví \alst{t}regr-at ykkr \hld\ \alst{t}ęiti at mę́la? &
\edtext{es \alst{Jǫ}rmunrekr \hld\ \alst{y}ðra systur, &
\alst{u}nga at \alst{a}ldri, \hld\ \alst{j}óm of traddi, &
\alst{h}vítum ok svǫrtum \hld\ ȧ \alst{h}ęr-vegi &
\alst{g}rǫ́m, \alst{g}ang-tǫmum \hld\ \alst{G}otna hrossum.}{\lemma{es \dots\ hrossum. ‘when \dots\ horses!’}\Bfootnote{Repeated almost identically in \Hamdismal\ 3.}}\eva

\bvb “Why sit ye two? Why sleep ye your lives away? \\
Why troubles it you not to speak merrily? \\
when Erminric has had your sister, \\
young of age, trampled with steeds; \\
with whites and blacks on the war-path, \\
with grey, pacing, Gotnish horses!\evb\evg


\bvg\bva Hlę́jandi Guðrún \hld\ hvarf til skęmmu, &
kumbl konunga \hld\ ór kęrum valði, &
síðar brynjur \hld\ ok sonum fǿrði; &
hlóðusk móðgir \hld\ á mara bógu.\eva

\bvb Laughing, Guthrun turned to her chamber \\
the heirlooms of kings from the chests she picked: \\
the long byrnies, and to her sons brought them; \\
the gloomy men loaded themselves on the backs of steeds.\evb\evg


\bvg\bva Þá kvað þat Hamðir \hld\ inn hugum stóri: &
„Svá kom-a’k męirr aptr \hld\ móður at vitja &
gęir-Njǫrðr hniginn \hld\ á Goð-þjóðu &
at þú ęrfi \hld\ at ǫll oss drykkir, &
at Svanhildi \hld\ ok sonu þína.“\eva

\bvb Then Hamthew quoth this, the great of heart: \\
“TODO. \\
that thou drink a death-toast to us all; \\
to Swanhild and thy sons.”\evb\evg


\bvg\bva Guðrún grátandi, \hld\ Gjúka dóttir, &
gekk treg-liga \hld\ á tái sitja &
ok at tęlja, \hld\ tǫ́rug-hlýra,
móðug spjǫll \hld\ á margan veg:\eva

\bvb Guthrun weeping, Yivick’s daughter, \\
walked TODO. \\
and to tell with teary cheeks \\
gloomy words in many ways:\evb\evg


\bvg\bva „Þrjá vissa’k \alst{ę}lda, \hld\ þrjá vissa’k \alst{a}rna, &
\alst{v}as’k þrimr \alst{v}erum \hld\ \alst{v}egin at húsi; &
\alst{ęi}nn vas mér Sigurðr \hld\ \alst{ǫ}llum bętri &
es \alst{b}rǿðr mínir \hld\ at \alst{b}ana urðu.\eva

\bvb “Three fires I’ve known, three hearths I’ve known; \\
for three husbands I’ve been brought to the house. \\
Alone was Siward to me better than them all, \\
he whose bane my brothers became.\evb\evg




TODO: Bunch of verses.


\bvg\bva \alst{G}ekk ek til strandar, \hld\ \alst{g}rǫm vas’k nornum, &
vilda’k hrinda \hld\ stríð grið þęirra; &
\alst{h}ófu mik, né drękkðu, \hld\ \alst{h}ávar bǫ́rur, &
því \alst{l}and of sté’k \hld\ at \alst{l}ifa skylda’k.\eva

\bvb I walked to the shore, wroth against the norns; \\
I wished to break their stubborn peace. \\
The high waves lifted me—drowned me not; \\
I stepped aland since I was meant to live.\evb\evg


\bvg\bva Gekk ek á \alst{b}ęð \hld\ —hugða’k mér fyr \alst{b}ętra— &
\alst{þ}riðja sinni \hld\ \alst{þ}jóð-konungi; &
\alst{ó}l ek mér \alst{jó}ð, \hld\ \alst{ę}rfi-vǫrðu &
{[...]} \hld\ Jónakrs \edtext{sona}{\Bfootnote{emend.; \emph{sonum} \Regius}}.\eva

\bvb TODO.\evb\evg


TODO: stanzas


\bvg\bva Fjǫlð man’k bǫlva, \hld\ [...] &
\alst{b}ęit-tu, Sigurðr, \hld\ inn \alst{b}lakka mar, &
\alst{h}ęst inn \alst{h}rað-fǿra \hld\ lát-tu \alst{h}inig renna! &
\alst{S}itr ęigi hér \hld\ \alst{s}nǫr né dóttir &
sú’s \alst{G}uðrúnu \hld\ \alst{g}ę́fi hnossir.\eva

\bvb I recall a multitude of bales; [...]; \\
saddle, O Siward, thy black steed, \\
the quick-pacing horse; let him run hither! \\
Here sits nowise TODO.\evb\evg


\bvg\bva \alst{M}inns-tu, Sigurðr, \hld\ hvat vit \alst{m}ę́ltum &
þȧ’s vit ȧ \alst{b}ęð \hld\ \alst{b}ę́ði sǫ́tum? &
at þú \alst{m}yndir \alst{m}ín \hld\ \alst{m}óðugr vitja, &
\alst{h}alr, ór \alst{h}ęlju, \hld\ en ek þín ór \alst{h}ęimi.\eva

\bvb Recallest thou, Siward, what we said, \\
when on the bed we both did sit? \\
That thou wouldst me, O mighty man, \\
visit from Hell, and I thee from the world.\evb\evg


\bvg\bva Hlaðið ér, \alst{ja}rlar, \hld\ \alst{ęi}ki-kǫstinn, &
látið þann und \edtrans{\alst{h}i\emph{mn}i}{heaven}{\Afootnote{emend.; \emph{hilmi} ‘prince’ \Regius}} \hld\ \alst{h}ę́stan verða! &
Męgi \alst{b}ręnna \alst{b}rjóst \hld\ \alst{b}ǫlva-fullt ęldr &
umb hjarta [\dots] \hld\ þiðni sorgir!“\eva

\bvb Load, ye earls, the oaken pile \ken{pyre}! \\
Let it beneath heaven become the highest! \\
May fire burn my curse-filled chest, \\
unto the heart \dots\ may the sorrows melt away!”\evb\evg


\bvg\bva \alst{Jǫ}rlum \alst{ǫ}llum \hld\ \alst{ó}ðal batni, &
\alst{s}nótum ǫllum \hld\ \alst{s}org at minni &
at þetta \alst{t}reg-róf \hld\ of \alst{t}alit vę́ri.\eva

\bvb For all earls may patrimony improve; \\
for all ladies sorrow decrease, \\
as this grief-chain was recounted!\evb\evg

\sectionline
% Guthrun’s Instigation
	\bookStart{The Speeches of Hamthew}[Hamðismǫ́l]

\begin{flushright}%
\textbf{Dating} \parencite{Sapp2022}: C10th (0.885)

\textbf{Meter:} \Fornyrdislag, \Malahattr% TODO
\end{flushright}%

% Introduction

Two poems?


\sectionline


... TODO ...


\bvg\bva \alst{V}ęl hǫfum vit \alst{v}egit, \hld\ stǫndum á \alst{v}al Gotna &
\alst{o}fan \alst{ę}gg-móðum \hld\ sem \alst{ę}rnir á kvisti; &
\alst{g}óðs hǫfum tírar fengit \hld\ þótt skylim nú eða í \alst{g}ę́r dęyja, &
\alst{k}vęld lifir maðr ekki \hld\ ęftir \alst{k}við norna.\eva

\bvb “Well have we two fought, we stand on the corpses of the Gots: \\
above the edge-weary \ken{killed} like eagles on a branch. \\
We have earned great glory, even if we should die now or tomorrow— \\
man lives not one evening after the verdict of the norns!”\evb\evg


\bvg\bva Þar fell \alst{S}ǫrli \hld\ at \alst{s}alar gafli, &
en \alst{H}amðir \alst{h}né \hld\ at \alst{h}ús-baki.\eva

\bvb There fell Sarrel by the gables of the hall, \\
but Hamthew sank down by the back of the house.\evb\evg

\sectionline
% Speeches of Hamthew

\part{Other Norse Heroic Poetry}
%	\include{books/Song of Grotte.tex}% Song of Grotte
	\bookStart{Leeds of Hindle}[Hyndluljóð]
\def\thisBookCode{Hyndluljod}

\begin{flushright}%
\textbf{Dating} \parencite{Sapp2022}: late C11th (0.996)

\textbf{Meter:} \Fornyrdislag%
\end{flushright}%

\section{Introduction}

A poorly preserved poem only found in \FlatMS.

\sectionline

\section{The Leeds of Hindle}

\bvg\bva%
„Vaki \alst{m}ę́r \alst{m}ęyja, \hld\ vaki \alst{m}ín vina, &
\alst{H}yndla systir, \hld\ es í \alst{h}ęlli býr; &
nú ’s \alst{r}økr \alst{r}økra, \hld\ \alst{r}íða vit skulum &
til \alst{V}al-hallar \hld\ ok til \alst{v}és hęilags.\eva

\bvb\speakernoteb{[Frow quoth:]}%
“Wake, maiden of maidens! Wake, my friend! \\
O Hindle, sister, who livest in the cave! \\
Now’s the twilight of twilights; we two shall ride \\
to Walhall and to the holy \inx[C]{wigh}!\evb\evg


\bvg\bva%
Biðjum \alst{H}ęrja-fǫðr \hld\ í \alst{h}ugum sitja, &
\edtrans{hann \alst{g}eldr ok \alst{g}efr \hld\ \alst{g}ull verðugum}{he pays and gives gold to the worthy}{\Bfootnote{Closely related to \HelgakvidaOne\ 9/3, which is why \textcite{FinnurEdda}, \textcite{GudniEdda} emend \emph{verðugum} ‘the worthy’ to \emph{verðungu} ‘the retinue’.}}, &
gaf hann \alst{H}ęrmóði \hld\ \alst{h}jalm ok brynju, &
en \alst{S}igmundi \hld\ \alst{s}verð at þiggja.\eva

\bvb Let us bid the Father of Hosts \name{= Weden} to remain in good spirits;  \\
he pays and gives gold to the worthy.  \\
He gave \inx[P]{Harmod} helmet and byrnie, \\
and \inx[P]{Syemund} a sword to receive.\evb\evg


\bvg\bva%
Gefr hann \alst{s}igr \alst{s}onum, \hld\ en \alst{s}vinnum \edtrans{aura}{ounces}{\Bfootnote{Of silver.}}, &
\alst{m}ę́lsku \alst{m}ǫrgum \hld\ ok \alst{m}an-vit firum, &
\alst{b}yri gefr \alst{b}rǫgnum, \hld\ en \alst{b}rag skǫldum, &
gefr hann \alst{m}ann-sęmi \hld\ \alst{m}ǫrgum rekki.\eva

\bvb He gives victory to sons and ounces to the wise, \\
speech to many and \inx[C]{manwit} to men. \\
Fair wind he gives to nobles and praise-song to \inx[C]{scald}[scalds]; \\
he gives manly valour to many a champion.\evb\evg


\bvg\bva%
\alst{Þ}ór mun’k blóta, \hld\ \alst{þ}ess mun’k biðja, &
at hann \alst{ę́} við þik \hld\ \alst{ęi}n-art láti; &
þó ’s hǫ́num \alst{ȯ}-títt \hld\ við \alst{jǫ}tuns brúðir.\eva

\bvb To Thunder I will \inx[C]{bloot}; of this I will bid, \\
that he always be upright with thee \\
even though he hates the ettin’s brides.\evb\evg


\bvg\bva%
Nú tak-tu \alst{u}lf þinn \hld\ \alst{ęi}nn af stalli, &
lát hann \alst{r}inna \hld\ með \alst{r}una mínum.“— &
„Sęinn es \alst{g}ǫltr þinn \hld\ \alst{g}oð-veg troða, &
vil’k-at \alst{m}ar \alst{m}ínn \hld\ \alst{m}ę́tan hlǿða.\eva

\bvb Now take thy one wolf from the stable; \\
let him run alongside my boar.”— \\
\speakernoteb{[Hindle quoth:]}%
“Slow is thy boar to tread the Godways; \\
I wish not to load my noble steed.\evb\evg


\bvg\bva%
\alst{F}lǫ́ ert \alst{F}ręyja, \hld\ es \alst{f}ręistar mín, &
\edtext{vísar þú \alst{au}gum \hld\ á \alst{o}ss þannig, &
es hafir ver þinn \hld\ í val-sinni}{\lemma{vísar \dots\ val-sinni ‘thou showest \dots\ slain-ways’}\Bfootnote{i.e., “You only show favour to me because you want me to help your lover”.  For the expression cf. \Sigrdrifumal\ 3/3 and note.}} &
Óttar unga \hld\ Innstęins bur.“\eva

\bvb False art thou, Frow, who temptest me; \\
thou showest thy eyes on us this way \\
since thou hast thy lover on the slain-path: \\
the young Oughter, Instone's offspring.”\evb\evg


\bvg\bva%
„\alst{D}ulið est Hyndla, \hld\ \alst{d}raums ę́tla’k þér, &
es kveðr \alst{v}er minn \hld\ í \alst{v}al-sinni.\eva

\bvb\speakernoteb{[Frow quoth:]}%
Deluded art thou, Hindle; I think thee dreamy \\
as thou sayest that my man is on the slain-path.\evb\evg


\bvg\bva%
Þar’s \alst{g}ǫltr \alst{g}lóar \hld\ \alst{G}ullinbursti, &
\edtrans{\alst{H}ildisvíni}{Hildswine}{\Bfootnote{The ‘battle-swine’, presumably an alternative name of Goldenbristle.}}, \hld\ es mér \alst{h}agir gęrðu, &
\alst{d}vergar tvęir \hld\ \alst{D}áinn ok Nabbi.\eva

\bvb There where the boar Goldenbristle glows, \\
the Hildswine, which for me made \\
the two skilful dwarfs Dowen and Nab.\evb\evg


\bvg\bva%
\alst{S}ęnn í \alst{s}ǫðlum \hld\ \alst{s}itja vit skulum &
ok of \alst{jǫ}fra \hld\ \alst{ę́}ttir dǿma, &
\alst{g}umna þęira, \hld\ es frá \alst{g}oðum kómu.\eva

\bvb Soon in the saddles we two shall sit, \\
and of rulers’ lineages speak, \\
of those men who came from the gods.\evb\evg


\bvg\bva%
Þęir hafa \alst{v}ęðjat \hld\ \alst{v}ala malmi &
\alst{Ó}ttarr \alst{u}ngi \hld\ ok \alst{A}ngantýr; &
\alst{sk}ylt ’s at vęita, \hld\ svá’t \alst{sk}ati hinn ungi &
\alst{f}ǫður-lęifð hafi \hld\ ępt \alst{f}rę́ndr sína.\eva

\bvb They have wagered the Welsh ore \ken{gold}, \\
young Oughter and Ongenthew— \\
it must be granted so that the young prince \\
may have the patrimony of his kinsmen.\evb\evg


\bvg\bva%
\alst{H}ǫrg hann mér gęrði \hld\ \alst{h}laðinn stęinum; &
nú ’s \alst{g}rjót þat \hld\ at \alst{g}lęri orðit; &
rauð hann í \alst{n}ýju \hld\ \alst{n}auta blóði; &
\alst{ę́} trúði \alst{Ó}ttarr \hld\ á \alst{ǫ́}synjur.\eva

\bvb A \inx[C]{harrow} he made me, loaded with stones; \\
now that stone-pile has turned into glass. \\
He reddened it in the fresh blood of oxen; \\
always did Oughter trust on the \inx[G]{Ossens}.\evb\evg


\bvg\bva%
\alst{N}ú lát forna \hld\ \alst{n}iðja talða &
ok \alst{u}pp-bornar \hld\ \alst{ę́}ttir manna &
hvat ’s Skjǫldunga, \hld\ hvat ’s Skilfinga, &
hvat ’s \alst{Ǫ}ðlinga \hld\ hvat ’s \alst{Y}lfinga & &
hvat ’s \alst{h}ǫld-borit, \hld\ hvat ’s \alst{h}ęrs-borit &
\alst{m}ęst \alst{m}anna val \hld\ \alst{u}nd Mið-garði?“\eva

\bvb Now let ancient kinsmen be counted, \\
and the high born lineages of men: \\
What’s of Shieldings? What’s of Shilvings? \\
What’s of Athlings? What’s of Wolvings? \\
What’s born of hero? What’s born of chief, \\
the greatest choice of men within Middenyard?”\evb\evg


\bvg\bva%
„Þú ert \alst{Ó}ttarr \hld\ borinn \alst{I}nnstęini, &
en \alst{I}nnstęinn vas \hld\ \alst{A}lfi inum gamla, &
\alst{A}lfr vas \alst{U}lfi, \hld\ \alst{U}lfr Sę́fara, &
en \alst{S}ę́fari \hld\ \alst{S}van inum rauða.\eva

\bvb\speakernoteb{[Hindle quoth:]}%
“Thou\footnoteB{Hindle, maybe in a trance-like state, speaks straight to Oughter.} art, Oughter, born to Instone, \\
and Instone was born to Elf the old, \\
Elf was to Wolf, Wolf to Seafarer, \\
and Seafarer to Swan the red.\evb\evg


\bvg\bva%
\alst{M}óður átti faðir þinn \hld\ \alst{m}ęnjum gǫfga, &
\alst{h}ygg at \alst{h}éti \hld\ \alst{H}lédís gyðja, &
\alst{F}róði vas \alst{f}aðir þęirar, \hld\ en \edtext{\alst{F}rí\emph{und}}{\Afootnote{emend. from meaningless \emph{†friaut†} \FlatMS}} móðir; &
\alst{ǫ}ll þótti \alst{ę́}tt sú \hld\ með \alst{y}fir-mǫnnum.\eva

\bvb Thy father won thy esteemed mother with torcs, \\
I think that she was called Leedise the \inx[C]{gidden}. \\
Frood was her father and Friend her mother; \\
all that lineage seemed to be among \inx[C]{overmen}.\evb\evg


\bvg\bva%
\alst{Au}ði vas \alst{á}ðr \hld\ \alst{ǫ}flgastr manna, &
\alst{H}alfdanr fyrri \hld\ \alst{h}ę́str Skjǫldunga, &
\alst{f}rę́g vǫ́ru \alst{f}olk-víg, \hld\ þau’s \alst{f}ramir gęrðu, &
\alst{h}varfla þóttu verk \hld\ með \alst{h}imins skautum.\eva

\bvb Ead was once the strongest of men, \\
Halfdane earlier the highest of Shieldings. \\
Famous were the troop-wars which the brave ones made; \\
his \name{= Halfdane’s} works seemed to whirl along the corners of heaven.\evb\evg


\bvg\bva%
\edtrans{\alst{Ę}flðisk}{became the in-law}{\Bfootnote{Lit. “was strengthened by”.  Elmwey was Eanmund’s daughter or sister.}} við \alst{Ęy}mund \hld\ \alst{ǿ}ðstan manna &
en vá \alst{S}igtrygg \hld\ með \alst{s}vǫlum ęggjum, &
\alst{ęi}ga gekk \alst{A}lmvęig, \hld\ \alst{ǿ}ðsta kvinna, &
\alst{ó}lu þau ok \alst{ǫ́}ttu \hld\ \alst{á}tján sonu.\eva

\bvb He \name{= Halfdane} became the in-law of Eanmund, the noblest of men, \\
but he slew Syetrue with cool edges. \\
He went to have Elmwey, the noblest of women; \\
they begot and had eighteen sons.\evb\evg


\bvg\bva%
Þaðan eru \alst{S}kjǫldungar, \hld\ þaðan eru \alst{S}kilfingar, &
þaðan eru \alst{Ǫ}ðlingar, \hld\ þaðan eru \alst{Y}nglingar, &
þaðan es \alst{h}ǫld-borit, \hld\ þaðan es \alst{h}ęrs-borit, &
\alst{m}est \alst{m}anna val \hld\ und \alst{M}ið-garði; &
allt ’s þat ę́tt þín, \hld\ Óttarr hęimski.\eva

\bvb Thence come Shieldings! Thence come Shilvings! \\
Thence come Athlings! Thence come Inglings!\footnote{Note the contradiction with v. 12. Since the Inglings have already been mentioned (under the name Shilvings, for the difference between the two see Index), it seems likely that Wolvings is the original reading.} \\
Thence is born of hero! Thence is born of chief \\
the greatest choice of men within Middenyard! \\
This is all thy lineage, O foolish Oughter!”\evb\evg


\bvg\bva%
Vas Hildigunnr \hld\ hęnnar móðir, &
Svǫ́fu barn \hld\ ok Sę́-konungs; &
alt ’s þat ę́tt þín, \hld\ Óttarr hęimski. &
varði at viti svá, \hld\ viltu ęnn lęngra?\eva

\bvb Hildguth was her mother, \\
the child of Sweve and Sea-king. \\
This is all thy lineage, O foolish Oughter!— \\
It is meaningful that one might know thus; wilt thou yet further?\evb\evg


\bvg\bva%
Dagr átti Þóru \hld\ dręngja móður, &
ólusk í ę́tt þar \hld\ ǿðstir kappar, &
Fraðmarr ok Gyrðr \hld\ ok Frekar báðir, &
Ámr ok Jǫsurmarr, \hld\ Alfr hinn gamli. &
varðar at viti svá, \hld\ viltu ęnn lęngra?\eva

\bvb Day had Thure, the mother of valiant men; \\
in that lineage were begotten the noblest champions: \\
Fradmer and Yird, and both Frekes; \\
Ame and Essirmer; Elf the old.— \\
It is meaningful that one might know thus; wilt thou yet further?\evb\evg


\bvg\bva%
Kętill hét vinr þęira \hld\ Klypps arf-þęgi, &
vas hann móður-faðir \hld\ móður þinnar; &
þar vas Fróði \hld\ fyrr ęnn Kári, &
en Hildi vas \hld\ Hóalfr of getinn.\eva

\bvb Kettle was their friend, the heir of Clip; \\
he was the father of thy mother's mother. \\
There was Frood, yet earlier Keer, \\
but by Hild was Highelf begotten.\evb\evg

... %TODO More dialogue

\sectionline
% Frow

\part{West Germanic Heroic Poetry}
	\bookStart{The Lay of Hildbrand}

\begin{flushright}%
\textbf{Dating:} 700s

\textbf{Meter:} \Fornyrdislag%para
\end{flushright}%

% Introduction

For the text of original poem I present the manuscript text with as few textual emendations as possible. As for the orthography, I have found it impossible to produce a normalised without too heavily distorting the received text, being as it is, a blend of several dialects (one need only observe the treatment of the name Thedric, which appears thrice, and each time in a markedly different form). Apart from my typical practice of capitalising proper names, marking prefixes with ⟨·⟩ and compounds with ⟨-⟩, and using acute accents to signify long vowels, circumflex accents to signify now-monophthongised original diphthongs, and overdots to mark nasal vowels, I have done the following changes in order to clarify etymological relationships and make the text somewhat more wieldy. Of these, 8–10 have also been noted in the apparatus where they occur:
\begin{enumerate}
  \item Consistently replaced both \emph{ƿ} (wynn) and \emph{uu} with \emph{w}.
  \item Consistently replaced \emph{c} with \emph{k}.
  \item Consistently replaced \emph{qu} with \emph{kw}.
  \item Consistently replaced \emph{t} with \emph{t̨} in positions affected by the Second Sound Shift.
  \item Replaced \emph{th} with \emph{þ}.
  \item Replaced \emph{e} with \emph{ę} when reflecting an original a-vowel affected by \emph{i}-mutation.
  \item Replaced \emph{ó} with \emph{ǫ́} where originally an \emph{a}.
  \item Removed unetymological double \emph{nn}.
  \item Restored initial \emph{h-} where etymological and/or metrically required.
  \item Removed initial \emph{h-} unetymological and/or metrically deficient.
\end{enumerate}

The punctuation of the original, entirely consisting of interpuncts, at times representing metrical breaks, at others sporadically placed, has not been retained.

Where they appear in cæsuræ, the words \emph{kwad Hilti-brant} ‘Hildbrand quoth’ (found in ll. 30, 49, and 58) replace the usual interpunct. Due to their hypermetrical nature, I had originally planned to remove these, and instead indicate the speaker in the margins—but after comparison with various Norse stanzas (e.g. \Reginsmal\ 3, wherein the words \emph{kvað Loki} ‘Lock quoth’ appear in the stanza’s first cæsura), I have come to believe that these represent an ancient oral interjection, seemingly going back as far as the Migration Period (as it seems incredulous to think that the scribe of \HildMS\ should have influenced the four centuries younger scribe of \Regius\ in such a minor point.)

% Summary

\sectionline

The poet gives a very short formulaic introduction, from which we can tell that the beginning of the poem is preserved (1–2). Hildbrand and Hathbrand, father and son, arm and dress themselves before riding into battle, each the head of an opposing host (3–6). Hildbrand asks Hathbrand about his name and lineage, saying that he knows all noble genealogies (7–13). Hathbrand gives his name, and says that the old men of his tribe have told him that his father was Hildbrand, a brave warrior. He abandoned the newborn Hathbrand in order to serve Thedric in his fight against Edwaker, but this was a long time ago, and Hathbrand doubts that he is still alive (14–29). Realising that he is facing his son, Hildbrand invokes God as witness, and as a token of loyalty offers Hathbrand a golden bigh which the Hunnish king had given him (30–35). Hathbrand exclaims that treasures must be won by struggle alone and harshly insults his father’s manhood: he calls him an old Hun, and accuses him of having survived to old age through treachery (36–41). Hathbrand then reveals that he has learned from sailors on the Mediterranean that Hildbrand is dead (42–44).

After this follow three short speeches by Hildbrand. The second one is certainly spoken by him, but the other two may be misplaced or misattributed. Hildbrand reflects on his son’s prosperity, saying that he can tell from his clothes that he has a good lord, and that he, unlike himself, has not suffered an exile’s fate (first speech: 45–48). He then calls on God, and laments that after thirty years of war he is now forced to fight against his own son; still, he tells Hathbrand that he should easily be able to kill such an old man as himself, if he has the strength to it (second speech: 49–57). Lastly, he (or Hathbrand, if we choose to emend) says that only the most queer easterner would refuse the fight when his opponent so greatly desires it. He accepts his fate and declares that when the duel is over, one of the two must win and rob the corpse of the other (third speech: 58–62).

The two men then throw their javelins, each of which gets stuck in the opposing shield, before rushing into each other, hacking away at their shields until they become worthless (63–68). The rest of the poem was continued on the now-lost, following page(s).

\sectionline

\bvg\bva[]%
Ik gi·hôrta dat̨ sęggen &
dat̨ sih \alst{u}r·hêt̨t̨un \hld\ \alst{ae}non muot̨ín: &
\alst{H}ilti-brant ęnti \alst{H}adu-brant \hld\ untar \alst{h}ęrjun t̨wêm &
\alst{s}unu-fatar·ungo \hld\ iro \alst{s}aro rihtun &
\alst{g}arutun sé iro \alst{g}u̇d-hamun \hld\ \alst{g}urtun sih iro swert ana &
\alst{h}ęlidos ubar \edtext{\alst{h}ringa}{\Afootnote{\emph{ringa} \HildMS}} \hld\ dó sie t̨ó dero \alst{h}iltu ritun.\eva

\bvb[0]I heard it said, \\
that two contenders alone did meet: \\
Hildbrand and Hathbrand, under two hosts.\footnoteB{i.e. each man was a champion of his respective army.} \\
Son and father ordered their armour, \\
readied their war-cloths, girded their swords on, \\
the heroes over the mail-coats—when to that battle they rode.\evb\evg


\bvg\bva[][6]%
\alst{H}ilti-brant \edtext{gi·mahalta}{\Afootnote{\emph{heribrantes sunu} ‘Harbrand’s son’ add. \HildMS}} \hld\ her was \alst{h}êróro man &
\alst{f}erahes \alst{f}rótóro \hld\ her \alst{f}rágén gi·stuont &
\alst{f}ôhém wortum \hld\ \edtext{hwer}{\Afootnote{\emph{wer} \HildMS}} sín \alst{f}ater wári &
\alst{f}irjo in \alst{f}olkhe \hld\ {[...]} &
{[...]} \hld\ „eddo \edtext{hwe-líhhes}{\Afootnote{\emph{welihhes} \HildMS}} \alst{k}nuosles dú sís &
ibu dú mí \alst{ê}nan sagés \hld\ ik mí de \alst{ǫ́}dre wêt &
\alst{kh}ind in \edtext{\alst{kh}unink-ríkhe}{\Afootnote{\emph{chunnincriche} \HildMS}} \hld\ \alst{kh}u̇d ist mín al irmin-deot“\eva

\bvb[0]Hildbrand spoke—he was the hoarier man, \\
more learned in life—he began to ask \\
in few words, who his father might be, \\
of men in the troop, [...] \\
“or of which lineage thou be; \\
if thou tell me one I the others will know, \\
O child, in the kingdom all great men are known to me.”\evb\evg


\bvg\bva[][13]%
\alst{H}adu-brant gi·mahalta \hld\ \alst{H}ilti-brantes sunu &
\edtext{„dat̨ sagetun mí \hld\ u̇sere liuti}{\lemma{dat \dots\ liuti}\Bfootnote{this l. breaks no rhythmic rules (cf. l. 42), but the needed alliteration is missing.}} &
\alst{a}lte anti fróte \hld\ dea \alst{ê}rhina wárun &
dat̨ \alst{H}ilti-brant haet̨t̨i mín fater \hld\ ih hęit̨t̨u \alst{H}adu-brant &
forn her \alst{ô}star \edtext{gi·węit̨}{\Afootnote{\emph{gihueit} \HildMS}} \hld\ flôh her \alst{Ô}t-akhres níd &
hina miti \alst{Þ}eot-ríhhe \hld\ ęnti sínero \alst{d}egano filu &
her fur-\alst{l}aet̨ in \alst{l}ante \hld\ \alst{l}út̨t̨ila sit̨t̨en &
\edtext{\alst{b}rút}{\Afootnote{\emph{prut} \HildMS}} in \alst{b}úre \hld\ \alst{b}arn un·wahsan &
\alst{a}rbjo-laosa \hld\ \edtext{her raet}{\Afootnote{\emph{heraet} \HildMS}} \alst{ô}star hina &
des síd \alst{D}et-ríhhe \hld\ \alst{d}arba \edtext{gi·stuontun}{\Afootnote{\emph{gistuontum} \HildMS}} &
\edtext{\alst{f}ateres}{\Afootnote{\emph{fatereres} \HildMS}} mínes \hld\ dat̨ was só \alst{f}riunt-laos man &
her was \alst{Ô}t-akhre \hld\ \alst{u}m·met̨ t̨irri &
\alst{d}egano \alst{d}ękhisto \hld\ unti \edtext{\alst{D}eot-ríkhhe}{\Afootnote{\emph{darba gistontun} add. \HildMS}} &
her was eo \alst{f}olkhes at̨ ęnte \hld\ imo was eo \edtext{\alst{f}eheta}{\Afootnote{\emph{peheta} \HildMS}} t̨i leop &
\alst{kh}u̇d was her \hld\ \edtext{\alst{kh}óném}{\Afootnote{\emph{chonnem} \HildMS}} mannum &
ni wániu ih iu líb habbe.“\eva

\bvb[0]Hathbrand spoke, Hildbrand’s son: \\
“\emph{This} \emph{our} people told me— \\
the old and learned, those who lived earlier— \\
that Hildbrand was called my father—I am called Hathbrand. \\
Long ago he turned east, he fled Edwaker’s hate, \\
hence with Thedrich and his multitude of thanes. \\
He left in the land a little one to stay: \\
a bride in the bower, a bairn ungrown, \\
inheritance-less—he rode east hence, \\
at which time Thedrich was in great need \\
of my father—that was so friendless a man! \\
He was immeasurably hostile to Edwaker, \\
the dearest of thanes under Thedrich. \\
He was always at the front of the troop; him did always the fight gladden; \\
known was he among keen men; \\
I ween not that he still have life.”\evb\evg


\bvg\bva[][29]%
„wêt̨t̨u \alst{I}rmin-got {\small (kwad Hilti-brant)} \alst{o}bana ab \edtext{hewane}{\Afootnote{\emph{heuane} \HildMS}} &
dat̨ dú neo dana halt mit sus sippan man &
dink ni gi·lęitós“ &
\alst{w}ant her dó ar arme \hld\ \alst{w}untane bauga &
\alst{kh}ęisur·ingu gi·tán \hld\ so imo sie der \alst{kh}uning gap &
\alst{h}unjo truhtin \hld\ „dat̨ ih dír it̨ nú bí \alst{h}uldí gibu“\eva

\bvb[0]“I call Ermin-god as witness above in heaven, \\
that thou never again with such a close relation lead dispute.” \\
He then unwound from his arm some twisted \inx[C]{bigh}[bighs], \\
made by a Cæsar’s man, which the king had given him, \\
the Lord of the Huns—“This I now give thee as [a sign of] \inx[C]{holdness}.\footnoteB{The giving of \emph{bighs} (armlets, torcs) in exchange for loyalty among warriors is well attested; see Encyclopedia.  This encounter is particularly reminiscent of \Harbardsljod\ 42.}”\evb\evg


\bvg\bva[][35]%
\alst{H}adu-brant gi·mahalta \hld\ \alst{H}ilti-brantes sunu: &
„\edtrans{mit \alst{g}êru skal man \hld\ \alst{g}eba in·fȧhan}{With spear shall one win gifts}{\Bfootnote{This ancient mindset was codified by the Indians as part of the \emph{kṣatra-dharma}, the code of the Warrior (\emph{kṣatriya}) caste, which explicitly forbade them from taking gifts.  So in a part of the Mahabharata (12.192.73), a Warrior King refuses a gift from a priest since “it is the duty prescribed for a Kṣatriya that he must fight and protect (people).  Kṣatriya are said to be the givers, then, how can I take (this) from you?” (\textcite{Hara1974} transl.)}} &
\alst{o}rt widar \alst{o}rte \hld\ {[...]} &
dú bist dir \alst{a}ltér hun \hld\ \alst{u}m·met̨ spáhér &
\alst{sp}ęnis mih mit díném wortun \hld\ wili mih dínu \alst{sp}eru werpan &
\edtext{bist}{\Afootnote{\emph{pist} \HildMS}} \alst{a}l-só gi·\alst{a}ltét man \hld\ só dú êwín \alst{i}n·wit fórtós &
dat̨ \alst{s}agetun mí \hld\ \alst{s}êo-lídante &
\alst{w}estar ubar \edtrans{\alst{W}ęntil-sêo}{Wendle-sea}{\Bfootnote{The Mediterranean, the name referring to the Wandals who for a time ruled North Africa.}} \hld\ dat̨ man \alst{w}ík fur·nam: &
tôt ist \alst{H}ilti-brant \hld\ \alst{H}ęri-brantes suno!“\eva

\bvb[0]Hathbrand spoke, Hildbrand’s son: \\
“With spear shall one win gifts, \\
point against point! \\
Thou art, old Hun, immeasurably clever: \\
thou dost lure me with thy words; at me wilt thou hurl thy spear! \\
Thou art thus an aged man, since thou always deceit didst work.— \\
\emph{This} told me seafarers \\
in the west over the Wendle-sea, that war took that man; \\
dead is Hildbrand, Harbrand’s son!”\evb\evg


\bvg\bva[][44]%
\alst{H}ilti-brant gi·mahalta \hld\ \alst{H}ęri-brantes suno: &
„wela gi·sihu ih in díném hrustim &
dat̨ dú \alst{h}abés \alst{h}ême \hld\ \alst{h}êrron góten &
dat̨ dú noh bí desemo \alst{r}íkhe \hld\ \alst{r}ekkhjo ni wurti“\eva

\bvb[0]Hildbrand spoke, Harbrand’s son: \\
“Well do I see from thy gear, \\
that thou hast a good lord at home, \\
that thou yet from this realm art not become an exile.”\evb\evg


\bvg\bva[][48]%
„\alst{w}elaga nú \alst{w}altant got {\small (kwad Hilti-brant)} \edtrans{\alst{w}ê-wurt}{woeful weird}{\Bfootnote{\emph{wurt} here meaning ‘inexorable course of events’, not the Old Norse norn; cf. ON \emph{grimmar urðir} ‘grim courses of events’ TODO.}} skihit &
ih wallóta \edtrans{\alst{s}umaro ęnti wintro \hld\ \alst{s}ehs-tik}{sixty summers and winters}{\Bfootnote{i.e. thirty years.  Hathbrand is then around thirty years old, while Hildbrand is in his fifties or sixties.}} ur lante &
dar man mih eo \alst{sk}ęrita \hld\ in folk \edtrans{\alst{sk}eot̨antero}{shooters}{\Bfootnote{Cf. \Beowulf\ 702, where the OE cognate \emph{sceótend} stands for “warriors” in general.}} &
só man mir at̨ \alst{b}urk ênigeru \hld\ \alst{b}anun ni gi·fasta &
nú skal mih \alst{s}wásat̨ khind \hld\ \alst{s}wertu hauwan &
\alst{b}retón mit sínu \alst{b}illju \hld\ eddo ih imo t̨i \alst{b}anin werdan. &
Doh maht dú nú \alst{ao}d-líhho \hld\ ibu dir dín \alst{ę}llen taok &
in sus \alst{h}êremo man \hld\ \alst{h}rusti gi·winnan &
\alst{r}auba \edtext{bi·\alst{r}ahanen}{\Afootnote{\emph{bihrahanen} \HildMS}} \hld\ ibu dú dar êníg \alst{r}eht habés!“\eva

\bvb[0]“Well now, O wielding God! the woeful weird comes to pass. \\
I roamed for sixty summers and winters away from the land, \\
where I always was placed in the troop of shooters, \\
as at no fortress my bane was fastened.— \\
Now shall my own child strike me with the sword, \\
beat me down with his blade—or I become his bane. \\
Yet thou mayst now easily—if thy zeal avail thee— \\
from such a hoary man win the equipment; \\
bear away the booty—if thou have any right to it!”\evb\evg


\bvg\bva[][57]%
„der sí doh nú \alst{a}rgósto {\small (kwad Hilti-brant)} \alst{ô}star-liuto &
der dir nú \alst{w}íges \alst{w}arne \hld\ nú dih es só \alst{w}el lustit &
gu̇dja gi·\alst{m}ęinun \hld\ niuse de \alst{m}ót̨t̨i &
\edtext{hwędar}{\Afootnote{\emph{werdar} \HildMS}} sih \edtext{\alst{h}iutu dêro}{\Afootnote{metr. emend.; \emph{dero hiutu} \HildMS}} \alst{h}regilo \hld\ \edtext{\alst{h}ruomen}{\Afootnote{\emph{hrumen} \HildMS}} muot̨t̨i &
\edtext{eddo}{\Afootnote{\emph{erdo} \HildMS}} desero \alst{b}runnóno \hld\ \alst{b}êdero waltan!“\eva

\bvb[0]“He be now the weakest of Easterners, \\
who should refuse thee the fight when thou so greatly cravest \\
to struggle together—try he who might, \\
which one of us today of these garments may boast, \\
or of these byrnies wield both!”\evb\evg


\bvg\bva[][62]%
Dó lét̨t̨un sé \alst{ae}rist \hld\ \alst{a}skkim skrítan &
\edtrans{\alst{sk}arpén \alst{sk}úrim}{in sharp showers}{\Bfootnote{Formulaic, also occurring in \Heliand\ 5137a.}} \hld\ dat̨ in dem \alst{sk}iltim stónt &
dó \alst{st}óptun t̨ó·samane \hld\ \alst{st}aim-bort \edtext{hludun}{\Afootnote{\emph{chludun} \HildMS}} &
\alst{h}ewun harm-líkko \hld\ \alst{h}wít̨t̨e skilti &
unti imo iro \alst{l}intún \hld\ \alst{l}út̨t̨ilo wurtun &
gi·\alst{w}igan miti \alst{w}ábnum \hld\ \edtext{[...]}{\Bfootnote{At this point the lone folio ends.  The rest of the poem would have been found on the now-lost following pages.  See Introduction to the poem.}}\eva

\bvb[0]Then let they first their ash-spears glide, \\
in sharp showers, that in the shields they stuck. \\
Then charged they into each other—the war-boards \ken{shields} resounded— \\
struck they harmfully the white shields, \\
until for them their lindens \ken{shields} became little, \\
worn down by the weapons, [...].\evb\evg

\sectionline
%
	\bookStart{Widsith}[Wídsïþ]

\begin{flushright}%
\textbf{Dating:} 600–700s (Neidorf 2013)

\textbf{Meter:} \Fornyrdislag%para
\end{flushright}%

\section{Introduction}

An archaic heroic poem.

\sectionline

\section{Widsith}

\bvg\bva \alst{W}íd-sïð maðolade, \hld\ \alst{w}ord-hord ǫn·leac, &
sé þe \alst{m}æ̂st \hld\ \alst{m}ærþa ofer eorþan, &
\alst{f}olca geond·\alst{f}ǿrde; \hld\ oft hé \alst{f}lętte ge·þah &
\alst{m}yne-lícne \alst{m}âþþum. \hld\ Hine frǫm \alst{M}yrgingum &
\alst{æ}þele \alst{ǫ}n·wócon. \hld\ He mid \alst{E}alh-hilde, &
\alst{f}æ̂lre \edtrans{\alst{f}reoþu-wębban}{peace-weaveress}{\Bfootnote{A woman used in a political marriage to bring peace between two tribes or families, in this case between King Edwin of the Mirgings (see ll. 97–98) and Erminric of the Gots.}}, \hld\ \alst{f}orman sïþe &
\edtrans{\alst{H}reð-cyninges}{Reth-King}{\Bfootnote{The king of the \inx[G]{Reth-Gots}, which is apparently just a poetic name for the (Eastern) Gots; cf. ll. 18, 57, 88–89.}} \hld\ \alst{h}âm ge·sóhte &
\alst{éa}stan of \alst{Ǫ}ngle, \hld\ \alst{Eo}rman-ríces, &
\alst{w}râþes \alst{w}ær-logan. \hld\ Ǫn·gǫnn þá \alst{w}orn sprecan:\eva

\bvb Widsith spoke, unlocked his word-hoard, \\
he who mots through tribes on earth \\
and nations had journeyed. Oft on the bench had he received \\
delightful treasures. From the Mirgings \\
his ancestry stemmed. Along with Elhild, \\
the good peace-weaveress, for the first time \\
had he sought out the Reth-King’s realm, \\
east of the Angles, [the realm of] \inx[P]{Erminric}, \\
the fierce oath-breaker.  He then began a long speech:\evb\evg


\bvg\bva „Fela ic \alst{m}ǫnna ge·frægn \hld\ \alst{m}ægþum wealdan. &
Sceal \alst{þ}eóda ge·hwylc \hld\ \alst{þ}éawum lifgan, &
\alst{eo}rl æfter \alst{ȯ}þrum \hld\ \alst{ǿ}ðle rǽdan, &
sé þe his \alst{þ}eóden-stól \hld\ ge·\alst{þ}éon wile.\eva

\bvb “A great deal of men I’ve learned ruling tribes. \\
Every person shall live in virtue; \\
each earl after the other lead his homeland, \\
he who on his ruling-seat will prosper.\evb\evg


\bvg\bva Þâra wæs Wala \hld\ hwíle sélast, &
ǫnd \alst{A}lexandreas \hld\ \alst{ea}lra rícost &
\alst{m}ǫnna cynnes, \hld\ ǫnd he \alst{m}æ̂st ge·þȧh &
þâra þe ic ofer \alst{f}oldan \hld\ ge·\alst{f}rægen hæbbe.\eva

\bvb Of them was Wale for a while the most blessed, \\
and Alexander of all the strongest \\
of mankind, and he prospered most \\
of those men over the earth of whom I’ve learned.\evb\evg


\bvg\bva Ætla weold Húnum, \hld\ Eorman-ríc Gotum, &
Becca Baningum, \hld\ Burgendum Gifica. &
Câsere weold Créacum \hld\ ǫnd Cælic Finnum, &
Hagena Holm-rycum \hld\ ǫnd Henden Glommum.\eva

\bvb Attle ruled the Huns, Erminric the Gots, \\
Bicke the Banings, the Burgends Yivick. \\
Choser ruled the Greeks and Calic the Finns, \\
Hain the Holmrighs and Henden the Glams.”\evb\evg


\bvg\bva Witta weold Swǽfum, \hld\ Wada Hælsingum, &
Meaca Myrgingum, \hld\ Mearc-healf Hundingum. &
Þeód-ríc weold Frǫncum, \hld\ Þyle Rǫndingum, &
Breoca Brǫndingum, \hld\ Billing Wernum.\eva

\bvb TODO.\evb\evg


\bvg\bva Ȯswine weold Eowum \hld\ ǫnd Ytum Gef-wulf, &
Finn Folc-walding \hld\ Fresna cynne. &
Sige-hęre lęngest \hld\ Sæ̂-dęnum weold, &
Hnæf Hocingum, \hld\ Helm Wulfingum, &
Wald Wóingum, \hld\ Wód Þyringum, &
Sæ̂-ferð Sycgum, \hld\ Swéom Ongend-þeow, &
Sceaft-hęre Ymbrum, \hld\ Sceafa Lǫng-beardum, &
Hún Hæt-werum \hld\ ǫnd Holen Wrosnum; &
Hring-wald wæs hâten \hld\ Hęre-farena cyning.\eva

\bvb TODO.\evb\evg


\bvg\bva Offa weold Ǫngle, \hld\ Ale-wíh Dęnum; &
sé wæs þâra manna \hld\ módgast ealra, &
no hwæþre he ofer Offan \hld\ eorl-scype fręmede, &
ac Offa ge·slóg \hld\ æ̂rest mǫnna, &
cniht-wesende, \hld\ cyne-ríca mæ̂st.\eva

\bvb Offe ruled the Angles, Alewigh the Danes; \\
of those men he was the bravest of all, \\
but he never furthered greater earlship than Offe, \\
for Offe won—youngest of men, \\
still a boy—the greatest of kingdoms.\evb\evg


\bvg\bva Nænig efen-eald him \hld\ eorl-scipe mâran &
ǫn orette: \hld\ âne sweorde &%TODO: orette.
męrce ge·mæ̂rde \hld\ wið Myrgingum &
bi Fifel-dore; \hld\ heoldon forð siþþan &
Ęngle ǫnd Swǽfe, \hld\ swá hit Offa ge·slóg.\eva

\bvb No man of his age accomplished \\
greater earlship: with but one sword \\
he marked the border against the Mirgings, \\
by Fiveldoor. It was thenceforth held \\
by the Angles and Sweves as Offe had won it.\evb\evg


\bvg\bva Hróþ-wulf ǫnd Hróð-gâr \hld\ heoldon lęngest &
sibbe æt·somne \hld\ suhtor-fædran, &
siþþan hý for·wrǽcon \hld\ Wícinga cynn &
ǫnd Ingeldes \hld\ ord for·bigdan, &
for·heowan æt Heorote \hld\ Heaðo-beardna þrym.\eva

\bvb Rotholf and Rothgar held for the longest \\
the peace together, uncle and nephew, \\
since they drove away the race of Wikings, \\
and bent down Ingeld’s spear-point; \\
at Hart they cut down the host of the Hathbeards.\evb\evg

\sectionline

\bvg\bva Swá ic geond·fǿrde fela \hld\ fręmdra lǫnda &
geond ginne grund. \hld\ Gódes ǫnd yfles &
þær ic cunnade; \hld\ cnósle bi·dæ̂led, &
fréo-mǽgum feor \hld\ folgade wíde.\eva

\bvb So I journeyed through a great deal of strange lands \\
through the wide world. Of good and evil \\
I there became acquainted; of kin deprived, \\
far from dear kinsmen, I strayed widely.\evb\evg


\bvg\bva For·þǫn ic mæg singan \hld\ ǫnd sęcgan spell, &
mæ̂nan fore męngo \hld\ in meodu-healle &
hú mé cyne-góde \hld\ cystum dohten.\eva

\bvb Therefore I can sing and tell tales, \\
recall before the many in the mead-hall, \\
how men of good kin treated me with grace.\evb\evg


\bvg\bva Ic wæs mid Húnum \hld\ ǫnd mid Hreð-gotum, &
mid Swéom ǫnd mid Géatum \hld\ ǫnd mid Su̇þ-dęnum. &
Mid Wenlum ic wæs ǫnd mid Wærnum \hld\ ǫnd mid wícingum; &
mid Gefþum ic wæs ǫnd mid Winedum \hld\ ǫnd mid Gefflegum; &
mid Englum ic wæs ǫnd mid Swǽfum \hld\ ǫnd mid Ænenum; &
mid Seaxum ic wæs ǫnd Sycgum \hld\ ǫnd mid Sweord-werum; &
mid Hronum ic wæs ǫnd mid Deanum \hld\ ǫnd mid Heaþo-réamum.\eva

\bvb I was among Huns and among Reth-Gots, \\
among Swedes and among Geats, and among South-Danes. \\
Among Wendles I was and among Warns, and among Wikings; \\
among Yefths I was and among Wends, and among Yefflegs; \\
among Angles I was and among Sweves, and among Anens; \\
among Saxes I was and among Sidges, and among Sword-weres; \\
among Ranes I was and among Deans, and among Hath-Reams.\evb\evg


\bvg\bva Mid \alst{Þ}yringum ic wæs \hld\ ǫnd mid \alst{Þ}rowendum, &
ǫnd mid \alst{B}urgendum, \hld\ þær ic \alst{b}éag ge·þâh; &
mé þær \alst{G}u̇ð-hęre for·\alst{g}eaf \hld\ \alst{g}læd-lícne maþþum &
\alst{s}ǫnges to léane. \hld\ Næs þæt \alst{s}æne cyning!\eva%TODO: Check sæne.

\bvb Among Thirings I was and among Throwends, \\
and among the Burgends, where I received a bigh. \\
There Guther gladdened me with treasures, \\
as reward for my song. That was not a bad king!\evb\evg


\bvg\bva Mid \alst{F}rǫncum ic wæs ǫnd mid \alst{F}rysum \hld\ ǫnd mid \alst{F}rumtingum; &
mid \alst{R}ugum ic wæs ǫnd mid Glommum \hld\ ǫnd mid \alst{R}úm-walum.\eva

\bvb Among Franks I was and among Frises, and among Frumtings; \\
among Ruges I was and among Glams, and among Rome-Wales.\evb\evg

\sectionline

\bvg\bva Swylce ic wæs ǫn \alst{Ea}tule \hld\ mid \alst{Æ}lf-wine, &
sé hæfde \alst{m}ǫn-cynnes, \hld\ \alst{m}íne ge·fræge, &
\alst{l}eohteste hǫnd \hld\ \alst{l}ofes tó wyrcenne, &
\alst{h}eortan un·\alst{h}neaweste \hld\ \alst{h}ringa ge·dâles, &
\alst{b}eorhtra \alst{b}éaga, \hld\ \alst{b}earn Éad-wines.\eva

\bvb Likewise was I in Italy with Elfwin; \\
of mankind he had—as far as I have learned— \\
the lightest hand in the winning of praise, \\
the unstingiest heart in the dealing of rings \\
and bright bighs, that child of Edwin.\evb\evg


\bvg\bva Mid \alst{S}ercingum ic wæs \hld\ ǫnd mid \alst{S}eringum; &
mid \alst{C}reacum ic wæs ǫnd mid Finnum \hld\ ǫnd mid \alst{C}âsere, &
sé þe \alst{w}in-burga \hld\ ge·\alst{w}eald áhte, &
\alst{w}iolena ǫnd \alst{w}ilna, \hld\ ǫnd \alst{W}ala rices.\eva

\bvb TODO.\evb\evg


\bvg\bva Mid Scottum ic wæs ǫnd mid Peohtum \hld\ ǫnd mid Scríde-finnum; &
mid Líd-wícingum ic wæs ǫnd mid Léonum \hld\ ǫnd mid Lǫng-beardum, &
mid hæ̂ðnum ǫnd mid hæleþum \hld\ ǫnd mid Hundingum.\eva

\bvb Among Scots I was and among Picts, and among Shride-Finns; \\
among Lid-Wikings I was among Leans, and among Longbeards; \\
among heathens and among heroes and among Hundings.\evb\evg


\bvg\bva Mid Israhelum ic wæs \hld\ ǫnd mid Exsyringum, &
mid Ebreum ǫnd mid Indeum \hld\ ǫnd mid Egyptum. &
Mid Moidum ic wæs ǫnd mid Persum \hld\ ǫnd mid Myrgingum, &
ǫnd Mofdingum \hld\ ǫnd on·gend Myrgingum, &
ǫnd mid Amothingum. \hld\ Mid Éast-þyringum ic wæs &
ǫnd mid Eolum ǫnd mid Istum \hld\ ǫnd Idumingum.\eva

\bvb Among Israelites I was and among Assyrians, \\
among Hebrews and among Indians and among Egyptians. \\
Among the Medes I was and among Persians, and among Mirgings \\
and Mofdings and again the Mirgings \\
and among Amothings. Among East-Thirings I was \\
and among Eals and among Ists, and Idumings.\evb\evg


\bvg\bva Ǫnd ic wæs mid Eorman-ríce \hld\ ealle þráge, &
þær mé Gotena cyning \hld\ góde dohte; &
sé mé béag for·geaf, \hld\ burg-warena fruma, &
ǫn þam siex hund wæs \hld\ smǽtes goldes, &
ge·scyred sceatta \hld\ scilling-ríme; &
þǫne ic Ead-gilse \hld\ ǫn æht sealde, &
mínum hléo-dryhtne, \hld\ þa ic to hâm bi·cwǫm, &
leófum to léane, \hld\ þæs þe hé mé lǫnd for·geaf, &
mínes fæder ǿþel, \hld\ fréa Myrginga.\eva

\bvb And I was with Ermenric for the longest time, \\
where the king of the Gots treated me well. \\
He gave me a bigh—that chief of city-dwellers— \\
in which were reckoned six hundred shats \\
of purest gold in shilling-count. \\
I gave it in the possession of Edgils \\
my dear shelter and lord, when I came home, \\
as repayment for his giving me land, \\
—that lord of Mirgins—my father’s ethel.\evb\evg


\bvg\bva Ǫnd mé þá Ealh-hild \hld\ ȯþerne for·geaf, &
dryht-cwén duguþe, \hld\ dohtor Éad-wines. &
Hyre lof lęngde \hld\ geond lǫnda fela, &
þǫnne ic be sǫnge \hld\ sęcgan sceolde &
hwær ic under swegl \hld\ sélast wisse &
gold-hrodene cwén \hld\ giefe bryttian.\eva

\bvb And then Elhild gave me another, \\
the noble queen of the veterans, daughter of Edwin. \\
Her praise stretched further through a multitude of lands; \\
then I in song should say, \\
where beneath the heaven I know the most blessed \\
gold-adorned queen dispensing gifts.\evb\evg


\bvg\bva Þǫnne wit Scilling \hld\ scíran reorde &
for uncrum sige-dryhtne \hld\ sǫng a·hófan, &
hlúde bí hearpan, \hld\ hleoþor swinsade, &
þǫnne mǫnige męnn, \hld\ módum wlǫnce, &
wordum sprécan, \hld\ þá þe wel cu̇þan, &
þæt hí næ̂fre sǫng \hld\ séllan ne hýrdon.\eva

\bvb Then I and Shilling with clear voices, \\
before our victorious lord raised up a song,
loudly by the harp—the tune rang out. \\
Then many men proud of heart \\
told with words—those who knew well— \\
that they never had heard a better song.\evb\evg

\sectionline

\bvg\bva Ðǫnan ic ealne geond·hwearf \hld\ ǿþel Gotena, &
sóhte ic â síþa \hld\ þá sélestan; &
þæt wæs inn-weorud \hld\ Earman-rices.\eva

\bvb Then I passed through all the ethel of the Gots; \\
TODO.\evb\evg


\bvg\bva Heðcan sóhte ic ǫnd Beadecan \hld\ ǫnd Hęre-lingas, &
Emercan sóhte ic ǫnd Fridlan \hld\ ǫnd Éast-gotan, &
fródne ǫnd gódne \hld\ fæder Un-wenes.\eva

\bvb TODO\evb\evg


\bvg\bva Seccan sóhte ic ǫnd Beccan, \hld\ Seafolan ǫnd Þeód-ríc, &
Heaþo-ríc ǫnd Sifecan, \hld\ Hliþe ǫnd Incgen-þeow. &
Éad-wine sóhte ic ǫnd Elsan, \hld\ Ægel-mund ǫnd Hún-gâr, &
ǫnd þá wlǫncan ge·dryht \hld\ Wiþ-myrginga.\eva

\bvb TODO\evb\evg


\bvg\bva Wulf-hęre sóhte ic ǫnd Wyrm-hęre; \hld\ ful oft þær wíg ne a·læg, &
þǫnne Hræda hęre \hld\ heardum sweordum &
ymb Wistla-wudu \hld\ węrgan sceoldon &
ealdne ǿþel-stól \hld\ Ætlan leódum.\eva

\bvb I sought out Wolfer and Wyrmer—very seldom did the warring there stop, \\
when the Reth-army, with hard swords, \\
in the Wistlewood had to defend \\
the old homeland-seat against Attle’s people.\evb\evg


\bvg\bva Rǽd-hęre sóhte ic ǫnd Rǫnd-hęre, \hld\ Rúm-stân ǫnd Gisl-hęre, &
Wiþer-gield ǫnd Freoþe-ric, \hld\ Wudgan ǫnd Hâman; &
ne wǽran þæt ge·síþa \hld\ þá sǽmestan, &
þéah þe ic hý a·níhst \hld\ nęmnan sceolde.\eva

\bvb TODO.\evb\evg


\bvg\bva Ful oft of þâm héape \hld\ hwínende fléag &
\edtrans{giellende gâr}{a yelling spear}{\Bfootnote{Formulaic.}} \hld\ ǫn grǫme þeóde; &
wræccan þær weoldan \hld\ wundnan golde &
werum ǫnd wífum, \hld\ Wudga ǫnd Hâma.\eva

\bvb Most often from that troop whistling did fly \\
a yelling spear into the fiendish host; \\
there ruled the exiles Woody and Homer \\
twisted gold, men and women.\evb\evg


\bvg\bva Swá ic þæt symle ǫn·fǫnd \hld\ ǫn þæ̂re fęringe, &
þæt sé biþ leófast \hld\ lǫnd-búendum &
sé þe him God syleð \hld\ gumena ríce &
to ge·healdenne, \hld\ þenden hé hér leofað.“\eva

\bvb So I always did find while on that journey, \\
that he is dearest to land-dwellers \ken{men}, \\
whom God grants the realm of men \\
for to hold while here he lives.”\evb\evg

\sectionline

\bvg\bva Swá \alst{sc}ríþende \hld\ ge·\alst{sc}eapum hweorfað &
\alst{g}leó-męnn \alst{g}umena \hld\ geond \alst{g}runda fela, &
\alst{þ}earfe sęcgað, \hld\ \alst{þ}ǫnc-word sprecaþ, &
\alst{s}imle \alst{s}u̇ð oþþe norð \hld\ \alst{s}umne ge·mǿtað &
\alst{g}ydda \alst{g}leawne, \hld\ \alst{g}eofum un·hneawne, &
sé þe fore \alst{d}uguþe wile \hld\ \alst{d}óm a·ræ̂ran, &
\alst{eo}rl-scipe \alst{æ}fnan, \hld\ oþþæt \alst{ea}l scæceð, &
\alst{l}eoht ǫnd \alst{l}if sǫmod; \hld\ \alst{l}of sé ge·wyrceð, &
\alst{h}afað under \alst{h}eofonum \hld\ \alst{h}éah-fæstne dóm.\eva

\bvb So passing through fates they wander, \\
the song-men of mankind, through many lands; \\
they say their needs, speak thoughtful words; \\
whether in the south or north they meet some one, \\
gay in songs, unstingy with gifts, \\
who for the old troop will rear up \inx[C]{doom}, \\
accomplish earlship until all goes away, \\
light and life together.  He who works praise \\
has under the heavens a high, firm doom.\evb\evg

\sectionline
%
%	\bookStart{Walder}[Waldhere]

\begin{flushright}%
\textbf{Dating:} TODO

\textbf{Meter:} \Fornyrdislag%para
\end{flushright}%

A heroic poem preserved in two fragments.  The flyting between the heroes Walder and Guther in fragment 2 is very reminiscent of the dialogue in \Hildebrandslied.

For the manuscript I have inspected the digital facsimile at https://digipal.eu/digipal/page/1072/.

\sectionline

\bvg\bva hyrde hyne georne: &
„Huru Welande... \hld\ worc ne geswiceð &
monna ænigum \hld\ ðara ðe Mimming can &
heardne gehealdan. \hld\ Oft æt hilde gedreas &
swatfag and sweordwund \hld\ secg æfter oðrum. &
ætlan ordwyga, \hld\ ne læt ðin ellen nu gyt &
gedreosan to dæge, \hld\ dryhtscipe &
[nú] is se dæg cumen &
þæt ðu scealt aninga \hld\ oðer twega, &
lif forleosan \hld\ oððe langne dóm &
âgan mid ęldum, \hld\ Ælf-hęres sunu! &
Nalles ic ðé, wine mín, \hld\ wordum cide, &
ðy ic ðé ge·sáwe \hld\ æt ðam sweord-plegan &
ðurh edwit-scype \hld\ æniges mǫnnes &
wíg for·bugan \hld\ oððe on weal fleon, &
líce beorgan, \hld\ ðeah þe lâðra fela &
ðinne byrn-hǫmon \hld\ billum heowun, &
ac ðu symle furðor \hld\ feohtan sóhtest, &
mǽl ofer mearce; \hld\ ðy ic ðe metod on·dréd, &
þæt ðu to fyren-líce \hld\ feohtan sóhtest &
æt ðam æt-stealle \hld\ oðres monnes, &
wíg-rǽdenne. \hld\ Weorða ðe selfne &
gódum dǽdum, \hld\ ðenden ðin god ręcce. &
Ne murn ðu for ði méce; \hld\ ðe wearð mâðma cyst &
gifeðe to geoce, \hld\ mid ðy ðú Gu̇ðhęre scealt &
beot for·bigan, \hld\ ðæs ðe he ðas beaduwe on·gan &
...d unryhte \hld\ ǽrest sécan. &
For-sóc he ðam swurde \hld\ and ðam sync-fatum, &
béaga mænigo, \hld\ nu sceal béaga-léas &
hworfan from ðisse hilde, \hld\ hlâfurd sécan &
ealdne éðel \hld\ oððe hér ǽr swefan, &
gif he ða [...]“\eva

\bvb TODO.\evb\evg

\sectionline

\bvg\bva „...ce bæteran &
b·úton ðam ânum \hld\ ðe ic eac hafa &
on stân-fate \hld\ stille ge·hided. &
Ic wât þæt hit ðóhte \hld\ Ðeodric Widian &
selfum on·sendon, \hld\ and eac sinc micel &
mâðma mid ði méce, \hld\ monig oðres mid him &
golde ge·girwan \hld\ (iulean ge·nam), &
þæs ðe hine of nearwum \hld\ Níðhades mǽg, &
Welandes bearn, \hld\ Widia ut forlet; &
ðurh fifela geweald \hld\ forð on·ętte.“ &
Waldere maðelode, \hld\ wíga ęllen-rof, &
hæfde him on handa \hld\ hilde-frófre, &
gu̇ð-billa gripe, \hld\ gyddode wordum: &
„Hwæt, ðu húru wéndest, \hld\ wine Burgenda, &
þæt me Hagenan hand \hld\ hilde ge·fremede &
and getwæmde ...ðewigges. \hld\ Feta, gyf ðu dyrre, &
æt ðus heaðu-węrigan \hld\ hâre byrnan. &
Standeð me hér on eaxelum \hld\ Ælfheres lâf, &
gód and géap-neb, \hld\ golde ge·weorðod, &
ealles un-scende \hld\ æðelinges réaf &
to habbanne, \hld\ þonne hand węreð &
feorh-hord feondum. \hld\ Ne bið fah wið mé, &
þonne ...... un-mǽgas \hld\ ęft on·gynnað, &
mécum ge·metað, \hld\ swá gé mé dydon. &
Ðeah mæg sige syllan \hld\ se ðe symle byð &
recon and rǽd-fęst \hld\ ryh... ...a ge·hwilces. &
Se ðe him to ðam hâlgan \hld\ helpe ge·lifeð, &
to gode gioce, \hld\ hé þær gearo findeð &
gif ða earnunga \hld\ ǽr ge·ðenceð. &
Þonne moten wlance \hld\ welan britnian, &
æhtum wealdan, \hld\ þæt is [...]“\eva

\bvb TODO.\evb\evg

\sectionline
%
	\bookStart{Deer}[Deor]

\begin{flushright}%
Dating: TODO

Meter: \Fornyrdislag%para
\end{flushright}%

A lamentation from the Exeter Book, filled with numerous references to heroic legend.

\sectionline

\bvg\bva[]%
\alst{W}elund him be \alst{w}urman \hld\ \alst{w}ræces cunnade, &
\alst{â}n-hýdig \alst{eo}rl \hld\ \alst{ea}rfoþa dréag, &
hæfde him tó ge·\alst{s}iþþe \hld\ \alst{s}orge ǫnd lǫngaþ, &
\alst{w}inter-cealde \alst{w}ræce; \hld\ \alst{w}éan oft ǫn·fǫnd, &
siþþan hine \alst{N}íðhad ǫn \hld\ \alst{n}éde lęgde, &
\alst{s}wǫncre \alst{s}eono-bende \hld\ ǫn \alst{s}yllan mǫnn. &
\alst{Þ}æs ofer-eode, \hld\ \alst{þ}isses swá mæg!\eva

\bvb \inx[P]{Wayland} with worms his exile experienced; \\
the one-minded earl hardship did suffer; \\
had him for companions sorrow and longing, \\
winter-cold exile; woes he often found, \\
since \inx[P]{Nithad} on him fetters did lay; \\
heavy sinew-bonds on the better man. \\
\emph{That} passed over; \emph{this} may likewise.\evb\evg


\bvg\bva[][7]%
\alst{B}eadohilde ne wæs \hld\ hyre \alst{b}róþra déaþ &
on \alst{s}efan swá \alst{s}âr \hld\ swá hyre \alst{s}ylfre þing, &
þæt heo \alst{g}earo-líce \hld\ on·\alst{g}ieten hæfde &
þæt heo \alst{é}acen wæs; \hld\ \alst{æ̂}fre ne meahte &
\alst{þ}riste ge·\alst{þ}ęncan, \hld\ hú ymb \alst{þ}æt sceolde. &
\alst{Þ}æs ofer-eode, \hld\ \alst{þ}isses swá mæg!\eva

\bvb For \inx[P]{Beadhild} was not her brothers’ deaths \\
on her heart so sore, as her own thing, \\
that she clearly had understood, \\
that she was pregnant.  Never could she \\
bravely think out what about \emph{that} she should do. \\
\emph{That} passed over; \emph{this} may likewise.\evb\evg


\bvg\bva[][13]%
Wé þæt \alst{M}æðhilde \hld\ \alst{m}ǫnge ge·frugnon &
wurdon \alst{g}rund-léase \hld\ \alst{G}eates frige, &
þæt hi seo \alst{s}org-lufu \hld\ \alst{s}lǽp ealle bi·nǫm. &
\alst{Þ}æs ofer-eode, \hld\ \alst{þ}isses swá mæg!\eva

\bvb That for Mathild many, we have heard, \\
bottomless [troubles] arose, for Geat’s beloved, \\
that the sorrowful love her of sleep all deprived. \\
\emph{That} passed over; \emph{this} may likewise.\evb\evg


\bvg\bva[][17]%
\alst{Þ}eodríc áhte \hld\ \alst{þ}rítig wintra &
\alst{M}ǽringa burg; \hld\ þæt wæs \alst{m}ǫnegum cu̇þ. &
\alst{Þ}æs ofer-eode, \hld\ \alst{þ}isses swá mæg!\eva

\bvb \inx[P]{Thedric} owned for thirty winters \\
the fort of the Meerings; that was to many known. \\
\emph{That} passed over; \emph{this} may likewise.\evb\evg


\bvg\bva[][20]%
Wé ge·\alst{a}scodan \hld\ \alst{Eo}rmanríces &
\alst{w}ylfenne ge·þȯht; \hld\ áhte \alst{w}íde folc &
\alst{G}otena ríces. \hld\ \edtrans{Þæt wæs \alst{g}rim cyning!}{That was a grim king!}{\Bfootnote{Formulaic; cf. \Beowulf\ 11b: \emph{Þæt wæs gód cyning!} ‘That was a good king!’}} &
\alst{S}æt \alst{s}ęcg mǫnig \hld\ \alst{s}orgum ge·bunden, &
\alst{w}éan on \alst{w}énan, \hld\ \alst{w}ýscte ge·neahhe &
þæt þæs \alst{c}yne-ríces \hld\ ofer-\alst{c}umen wǽre. &
\alst{Þ}æs ofer-eode, \hld\ \alst{þ}isses swá mæg!\eva

\bvb We have learned of \inx[P]{Erminric}’s \\
wolven nature; he wielded widely the folk \\
of the realm of the Gots.  That was a grim king! \\
Sat many a man by sorrows bound, \\
woes in his thoughts; wished plenty \\
that the kingdom might be overcome. \\
\emph{That} passed over; \emph{this} may likewise.\evb\evg


\bvg\bva[][27]%
\alst{S}iteð \alst{s}org-céarig, \hld\ \alst{s}ǽlum bi·dæ̂led, &
on \alst{s}efan \alst{s}weorceð, \hld\ \alst{s}ylfum þinceð &
þæt sý \alst{ę}nde-léas \hld\ \alst{ea}rfoda dæ̂l. &
Mæg \alst{þ}ǫnne ge·\alst{þ}ęncan, \hld\ þæt geond \alst{þ}ás woruld &
\alst{w}itig dryhten \hld\ \alst{w}ęndeþ ge·neahhe, &
\alst{eo}rle mǫnegum \hld\ \alst{â}re ge·sceawað, &
\alst{w}ís-licne blǽd, \hld\ sumum \alst{w}éana dæ̂l.\eva

\bvb One may sit grieved with sorrow, of blessings bereft; \\
his heart darkens; to himself he thinks \\
that endless must be his share of hardships. \\
He may then think that throughout this world \\
the Wise Lord is fickle plenty. \\
To many an earl honour he shows, \\
sure success—to another a share of woes.\evb\evg


\bvg\bva[][34]%
Þæt ic bi mé \alst{s}ylfum \hld\ \alst{s}ęcgan wille, &
þæt ic \alst{h}wile wæs \hld\ \alst{H}eodeninga scóp, &
\alst{d}ryhtne \alst{d}ýre— \hld\ mé wæs \alst{D}eor nǫma. &
Áhte ic \alst{f}ela wintra \hld\ \alst{f}olgað tilne, &
\alst{h}oldne \alst{h}laford, \hld\ oþþæt \alst{H}eorrenda nú, &
\alst{l}éoð-cræftig mǫnn \hld\ \alst{l}ǫnd-ryht ge·þáh, &
þæt me \alst{eo}rla hléo \hld\ \alst{æ̂}r ge·sealde. &
\alst{Þ}æs ofer-eode, \hld\ \alst{þ}isses swá mæg!\eva

\bvb This of myself I wish to say, \\
that for a while I was the Headenings’s shop, \\
dear to their lord—Deer was my name. \\
I had for a multitude of winters a good retinue, \\
a \inx[C]{hold} bread-giver, until Harrend now, \\
the song-crafty man the land-right has received, \\
which to \emph{me} the shelter of earls of yore did grant. \\
\emph{That} passed over; \emph{this} may likewise.\evb\evg

\sectionline
%
%	\include{books/Finnsbury.tex}%
%	\include{books/Wolf.tex}%
%	\include{books/Bewolf.tex}% Probably not happening.

% Runic poetry
	\part{Miscellaneous Runic Poetry}

\bookStart{Introuction to Runic Poetry}

Not all poetry preserved in Runic inscriptions is included here; see below under Galders.  The stanza from the Rök runestone will be found under Norse Heroic Poetry, and the Runic version of the \emph{Dream of the Rood} under Christian poetry.

Metrically the poetry is generally in \Fornyrdislag.  A few fragments from Jutland are in \Ljodahattr\ and two from Sweden are in \Drottkvett.

\bookStart{Three Rune Poems}

\section{Introduction to the Rune Poems}

TODO: Acrophonic principle

The order and names of the letters in the Runic alphabets or \emph{futharks} stayed relatively consistent throughout the many centuries and countries in which they were used.  This can probably be ascribed to the \emph{rune poems}—poetic lists of the names of each rune with a short explanation, passed down orally as mnemonic devices to aid early Germanic learners, who were doubtless far more accustomed to learn by heart spoken poems than written letters.

Three such rune poems survive, from three countries: England, Norway, and Iceland.  The English rune poem documents the English \emph{futhorc}, while the Norwegian and Icelandic document the Scandinavian \emph{younger futhark}.

When compared to the Common Germanic \emph{elder futhark}, these two daughter scripts have taken opposing paths.  Whereas the English futhorc has appended several letters for new vowels to the end of the rune row, the Scandinavian futhark has instead done away with numerous runes, namely those for \emph{ng}, plosives \emph{d, g, p}, the semi-vowel \emph{w} and the vowels \emph{o} and \emph{e}, along with the obscure hook-shaped rune (TODO).  That much of this simplification was probably intentional, rather than the result of neglect or language change, is seen from the following facts.

First, several of the lost runes stood for sounds that did not undergo any major sound shifts in the North Germanic languages in the relevant time period.  For instance, all modern Scandinavian dialects still clearly distinguish between the initial consonants in the descendants of \emph{dagʀ} ‘day’ and \emph{Týr} ‘\inx[C]{Tew}’, and most even have the same articulation of these consonants as modern English.

Second, in two archaic runic inscriptions we find clear proof that the names and sound values of some of the lost runes were still remembered and passed down even after the adoption of the simplified younger futhark.  On the Swedish Rök stone (Ög 136), which is mostly composed in the younger futhark, runes of the elder futhark are used in a cipher, which works in the following way: Every younger futhark rune representing two distinct phonemes, where one of those was the sound value of that rune in the elder futhark system, and the other has been assimilated from a lost rune, is replaced by the elder futhark rune whose value it assimilated.  For instance, the \textbf{k} rune, which in the elder futhark stood for only /k/, but which in the younger futhark stands for both /k/ and /g/, is replaced with the old \textbf{g} rune.  A similar instance of two-scriptedness is found on the Ingelsta stone (Ög 43), where the old \textbf{d} rune is used in an otherwise younger futhark inscription, probably standing for its name \emph{dagʀ} ‘day’, which is also attested as a male given name.

Third, there is virtually no regional variation in which runes disappear in the transition from elder to younger futhark.  There is some variation in their shapes, but there is no region which, say, simplifies only the plosive consonants \emph{t/d, k/g, b/p} > \emph{t, k, b}, but retains the written distinction between \emph{o} and \emph{u}—they all go away at once.

These facts point away from neglect or a natural development of the script—they instead suggest deliberate reform.  Since we lack historical sources, the motivations behind such a reform can only be guessed at, but making the script simpler may have been intended to increase literacy by making it easier to learn and faster to write.  If this were the case it was certainly successful: the transition to the simplified younger futhark brings with it a huge increase in inscriptions in Scandinavia, along with interest in various ciphers, and a new tradition of inscribed stones in Denmark, where they were previously unknown.

This new system also quickly gave rise to even more simplified systems, like the “short-stave” runes found already on the C9th Rök stone, or the “staveless” runes known from northern Sweden.  Both of these variants make it even faster to write on materials like wood, wax and bone; the runes also take up less space—very useful for carvers writing on limited surfaces.

In any case, the names of the runes seem to have survived these developments.  Of the 16 runes found in both the English and Icelandic (which appears to be more conservative than the Norwegian) rune poems, 10—\textbf{f, r, h, n, i, j, s, b, m} and \textbf{l}—have etymologically identical names.  Three of the remaining six—\textbf{þ, a} and \textbf{t}—in the Icelandic stand for words with clear Heathen associations—Thurse, Os, and Tew—and so may have been changed deliberately after the conversion of England, rather than lost in the process of oral transmission.  Two more—\textbf{u} and \textbf{k}—have names which agree in form but not in meaning.  Thus it is only for the old \textbf{ʀ}-rune where there is complete disagreement about the original name.  This is easily understood, since the sound which that rune designated was lost in early Old English.

\section{The English Rune Poem}\chapterStart{}

\begin{flushright}%
\textbf{Dating:} 700s–C10th%TODO

\textbf{Meter:} \Fornyrdislag%para
\end{flushright}%

TODO: Introduction.  Preservation only in printed copy.

\sectionline


\bvg\bva%
ᚠ (feoh) byþ \alst{f}rofur \hld\ \alst{f}ira ge·hwylcum. &
Sceal ðeah \alst{m}anna ge·hwylc \hld\ \alst{m}iclun hyt dælan &
gif he wile for \alst{d}rihtne \hld\ \alst{d}ómes hleotan.\eva

\bvb TODO: TRANSLATION.\evb\evg


\bvg\bva%
ᚢ (ur) byþ \alst{â}n-mód \hld\ and \alst{o}fer-hyrned, &
\alst{f}ela-\alst{f}récne deor, \hld\ \alst{f}eohteþ mid hornum, &
\alst{m}ǽre \alst{m}ór-stapa; \hld\ þæt is \alst{m}ódig wuht.\eva

\bvb TODO: TRANSLATION.\evb\evg


\bvg\bva%
ᚦ (ðorn) byþ \alst{ð}earle scearp; \hld\ \alst{ð}egna ge·hwylcum &
\alst{a}n-feng ys \alst{y}fyl, \hld\ \alst{u}n-gemetun reþe &
\alst{m}anna ge·hwylcun \hld\ ðe him \alst{m}id resteð.\eva

\bvb TODO: TRANSLATION.\evb\evg


\bvg\bva%
ᚩ (os) byþ \alst{o}rd-fruma \hld\ \alst{æ}lcre spræce, &
\alst{w}ís-dómes \alst{w}raþu \hld\ and \alst{w}itena frofur, &
and \alst{eo}rla ge·hwam \hld\ \alst{ea}d-nys and to·hiht.\eva

\bvb TODO: TRANSLATION.\evb\evg


\bvg\bva%
ᚱ (rad) byþ on \alst{r}ecyde \hld\ \alst{r}inca ge·hwylcum &
\alst{s}efte, and \alst{s}wiþ-hwæt \hld\ ðam ðe \alst{s}itteþ on ufan &
\alst{m}eare \alst{m}ægen-heardum \hld\ ofer \alst{m}íl-paþas.\eva

\bvb TODO: TRANSLATION.\evb\evg


\bvg\bva%
ᚳ (cen) byþ \alst{c}wicera ge·hwam \hld\ \alst{c}u̇þ on fyre, &
\alst{b}lac and \alst{b}eorht-líc, \hld\ \alst{b}yrneþ oftust &
ðær hí \alst{æ}þelingas \hld\ \alst{i}nne restaþ.\eva

\bvb TODO: TRANSLATION.\evb\evg


\bvg\bva%
ᚷ (gyfu) \alst{g}umena byþ \hld\ \alst{g}leng and herenys, &
\alst{w}raþu and \alst{w}yrþ-scype, \hld\ and \alst{w}ræcna ge·hwam &
\alst{a}r and \alst{æ}twist \hld\ ðe byþ \alst{o}þra leas.\eva

\bvb TODO: TRANSLATION.\evb\evg


\bvg\bva%
ᚹ (wen) ne bruceþ \hld\ ðe can \alst{w}éana lýt, &
\alst{s}âres and \alst{s}orge, \hld\ and him \alst{s}ylfa hæfþ &
\alst{b}lǽd and \alst{b}lysse \hld\ and eac \alst{b}yrga ge·niht.\eva

\bvb TODO: TRANSLATION.\evb\evg


\bvg\bva%
ᚻ (hægl) byþ \alst{h}wítust corna; \hld\ hwyrft hit of \alst{h}eofones lyfte, &
\alst{w}ealcaþ hit \alst{w}indes scura, \hld\ weorþeþ hit to \alst{w}ætere syððan.\eva

\bvb TODO: TRANSLATION.\evb\evg


\bvg\bva%
ᚾ (nyd) byþ \alst{n}earu on breostan, \hld\ weorþeþ hi ðeah oft \alst{n}iþa bearnum &
to \alst{h}elpe and to \alst{h}æle ge·hwæþre, \hld\ gif hí his \alst{h}lystaþ æror.\eva

\bvb TODO: TRANSLATION.\evb\evg


\bvg\bva%
ᛁ (is) byþ \alst{o}fer-ceald, \hld\ \alst{u}n-ge·metum slidor, &
\alst{g}lisnaþ \alst{g}læs-hluttur, \hld\ \alst{g}immum ge·licust, &
\alst{f}lor \alst{f}orste ge·woruht, \hld\ \alst{f}æger an-sýne.\eva

\bvb TODO: TRANSLATION.\evb\evg


\bvg\bva%
ᛄ (ger) byþ \alst{g}umena hiht, \hld\ ðon \alst{G}od læteþ, &
\alst{h}âlig \alst{h}eofones cyning, \hld\ \alst{h}rusan syllan &
\alst{b}eorhte \alst{b}leda \hld\ \alst{b}eornum and ðearfum.\eva

\bvb TODO: TRANSLATION.\evb\evg


\bvg\bva%
ᛇ (eoh) byþ \alst{u}tan \hld\ \alst{u}n-smeþe treow, &
\alst{h}eard, \alst{h}rusan fæst, \hld\ \alst{h}yrde fyres, &
\alst{w}yrt-rumun under·\alst{w}reþyd, \hld\ \alst{w}ynan on éþle.\eva

\bvb TODO: TRANSLATION.\evb\evg


\bvg\bva%
ᛈ (peorð) byþ symble \hld\ \alst{p}lega and hlehter &
{[...]} \alst{w}lancum \hld\ ðar \alst{w}igan sittaþ &
on \alst{b}eor-sele \hld\ \alst{b}líþe æt·somne. \eva

\bvb TODO: TRANSLATION.\evb\evg


\bvg\bva%
ᛉ (eolhx)-secg \alst{ea}rd hæfþ \hld\ \alst{o}ftust on fenne, &
\alst{w}exeð on \alst{w}ature, \hld\ \alst{w}undaþ grimme, &
\alst{b}lode \alst{b}reneð \hld\ \alst{b}eorna ge·hwylcne &
ðe him \alst{æ}nigne \hld\ \alst{o}n-feng ge·deð.\eva

\bvb TODO: TRANSLATION.\evb\evg


\bvg\bva%
ᛋ (sigel) \alst{s}é-mannum \hld\ \alst{s}ymble biþ on hihte, &
ðonn hi hine \alst{f}eriaþ \hld\ ofer \alst{f}isces beþ, &
oþ hí \alst{b}rim-hengest \hld\ \alst{b}ringeþ to lande.\eva

\bvb TODO: TRANSLATION.\evb\evg


\bvg\bva%
ᛏ (tir) biþ \alst{t}âcna sum, \hld\ healdeð \alst{t}rywa wel &
wiþ \alst{æ}þelingas, \hld\ \alst{â} biþ on færylde, &
ofer \alst{n}ihta ge·\alst{n}ipu \hld\ \alst{n}æfre swiceþ.\eva

\bvb TODO: TRANSLATION.\evb\evg


\bvg\bva%
ᛒ (beorc) byþ \alst{b}leda leas, \hld\ \alst{b}ereþ efne swa ðeah &
\alst{t}ânas b·útan \alst{t}udder, \hld\ biþ on \alst{t}elgum wlitig, &
\alst{h}eah on \alst{h}elme \hld\ \alst{h}rysted fægere, &
ge·\alst{l}oden \alst{l}eafum, \hld\ \alst{l}yfte ge·tenge.\eva

\bvb TODO: TRANSLATION.\evb\evg


\bvg\bva%
ᛖ (eh) byþ for \alst{eo}rlum \hld\ \alst{æ}þelinga wyn, &
\alst{h}ors \alst{h}ófum wlanc, \hld\ ðær him \alst{h}æleþe ymb, &
\alst{w}elege on \alst{w}icgum, \hld\ \alst{w}rixlaþ spræce, &
and biþ \alst{u}n-styllum \hld\ \alst{æ}fre frofur.\eva

\bvb TODO: TRANSLATION.\evb\evg


\bvg\bva%
ᛗ (man) byþ on \alst{m}yrgþe \hld\ his \alst{m}agan leof; &
sceal þeah \alst{â}nra ge·hwylc \hld\ \alst{o}ðrum swícan, &
for ðam \alst{d}ryhten wyle \hld\ \alst{d}óme síne &
þæt \alst{ea}rme flæsc \hld\ \alst{eo}rþan be·tæcan.\eva

\bvb TODO: TRANSLATION.\evb\evg


\bvg\bva%
ᛚ (lagu) byþ \alst{l}eodum \hld\ \alst{l}ang-sum ge·þuht, &
gif hí sculun \alst{n}eþun \hld\ on \alst{n}acan tealtum, &
and hi \alst{s}æyþa \hld\ \alst{s}wýþe bregaþ, &
and se \alst{b}rim-hengest \hld\ \alst{b}ridles ne gymeð.\eva

\bvb TODO: TRANSLATION.\evb\evg


\bvg\bva%
ᛝ (ing) wæs \alst{æ}rest \hld\ mid Éast-Dęnum &
ge·\alst{s}ewen \alst{s}ęcgun, \hld\ oþ he \alst{s}iððan est &
ofer \alst{w}ǽg ge·\alst{w}ât, \hld\ \alst{w}æn æfter rann; &
ðus \alst{h}eardingas \hld\ ðone \alst{h}æle nęmdun.\eva

\bvb TODO: TRANSLATION.\evb\evg


\bvg\bva%
ᛟ (eþel) byþ \alst{o}fer-leof \hld\ \alst{æ}g·hwylcum men, &
gif he mot ðær \alst{r}ihtes \hld\ and ge·\alst{r}ysena on &
\alst{b}rúcan on \alst{b}lode \hld\ \alst{b}leadum oftast.\eva

\bvb TODO: TRANSLATION.\evb\evg


\bvg\bva%
ᛞ (dæg) byþ \alst{d}rihtnes sond, \hld\ \alst{d}eore mannum, &
\alst{m}ære \alst{m}etodes leoht, \hld\ \alst{m}yrgþ and to·hiht &
\alst{ea}dgum and \alst{ea}rmum, \hld\ \alst{ea}llum brice.\eva

\bvb TODO: TRANSLATION.\evb\evg


\bvg\bva%
ᚪ (ac) byþ on \alst{eo}rþan \hld\ \alst{ę}lda bearnum &
\alst{f}læsces \alst{f}odor, \hld\ \alst{f}ereþ ge·lome &
ofer \alst{g}anotes bæþ; \hld\ \alst{g}âr-sęcg fandaþ &
hwæþer \alst{â}c hæbbe \hld\ \alst{æ}þele treowe.\eva

\bvb TODO: TRANSLATION.\evb\evg


\bvg\bva%
ᚫ (æsc) biþ \alst{o}fer-heah, \hld\ \alst{ę}ldum dýre, &
\alst{st}iþ on \alst{st}aþule, \hld\ \alst{st}ede rihte hylt, &
ðeah him \alst{f}eohtan on \hld\ \alst{f}iras monige.\eva

\bvb TODO: TRANSLATION.\evb\evg


\bvg\bva%
ᚣ (yr) byþ \alst{æ}þelinga \hld\ and \alst{eo}rla ge·hwæs &
\alst{w}yn and \alst{w}yrþ-mynd, \hld\ byþ on \alst{w}icge fæger, &
\alst{f}æst-lic on \alst{f}ær-elde, \hld\ \alst{f}yrd-geatewa sum.\eva

\bvb TODO: TRANSLATION.\evb\evg


\bvg\bva%
ᛡ (iar, ior) byþ \alst{éa}-fixa, \hld\ and ðeah \alst{á} bruceþ &
\alst{f}ódres on \alst{f}oldan, \hld\ hafaþ \alst{f}ægerne eard, &
\alst{w}ætre be·\alst{w}orpen, \hld\ ðær he \alst{w}ynnum leofaþ.\eva

\bvb TODO: TRANSLATION.\evb\evg


\bvg\bva%
ᛠ (ear) byþ \alst{e}gle \hld\ \alst{e}orla ge·hwylcun, &
ðonn \alst{f}æst-lice \hld\ \alst{f}læsc on·ginneþ, &
\alst{h}raw colian, \hld\ \alst{h}rusan ceosan &
\alst{b}lac to ge·\alst{b}eddan; \hld\ \alst{b}leda ge·dreosaþ, &
\alst{w}ynna ge·\alst{w}itaþ, \hld\ \alst{w}era ge·swicaþ.\eva

\bvb TODO: TRANSLATION.\evb\evg

\sectionline

\section{The Icelandic Rune Poem}\chapterStart{}

\begin{flushright}%
\textbf{Dating:} Medieval.%TODO

\textbf{Meter:} Unclear.
\end{flushright}%

The poem is highly formulaic.  All lines begin with the respective rune’s name, followed by three kennings for it.  It is only attested in late manuscripts which often have major disagreements with each other.

\sectionline

\bvg\bva%
\alst{F}é es \alst{f}rę́nda róg \hld\ ok \alst{f}lǿðar viti &
\ind ok \alst{g}raf-sęiðs \alst{g}ata.\eva

\bvb Wealth is strife of kinsmen and beacon of the sea \\
\ind and grave-saithe’s \ken{serpent’s} street.\evb\evg


\bvg\bva%
Úr es \alst{sk}ýja grátr \hld\ ok \alst{sk}ára þvęrrir &
\ind ok \alst{h}irðis \alst{h}atr.\eva

\bvb Drizzle is weeping of clouds and ... \\
\ind and shepherd’s hatred.\evb\evg


\bvg\bva%
Þurs es \alst{k}venna \alst{k}vǫl \hld\ ok \alst{k}letta í·búi &
\ind ok \alst{v}arð-rúnar \alst{v}err.\eva

\bvb Thurse is women’s torment and indweller of hills \\
\ind and husband of the weird-whisperess \ken{giantess}.\evb\evg


\bvg\bva%
\alst{Ǫ́}ss es \alst{a}ldinn gautr \hld\ ok \alst{Ǫ́}s-garðs jǫfurr, &
\ind ok \alst{V}al-hallar \alst{v}ísi.\eva

\bvb Os is ancient Geat, and Osyard’s chief, \\
\ind and Walhall’s overseer.\evb\evg


\bvg\bva%
Ręið es \alst{s}itjandi \alst{s}ę́la \hld\ ok \alst{s}núðig fęrð &
\ind ok \alst{jó}s \alst{ę}rfiði.\eva

\bvb Chariot is sitting bliss and twirling journey \\
\ind and horse’s heavy work.\evb\evg


\bvg\bva%
Kaun es \alst{b}arna \alst{b}ǫl \hld\ ok \alst{b}ar-dagi &
\ind ok \alst{h}old-fúa \alst{h}ús.\eva

\bvb Boil is children’s curse and TODO \\
\ind and house of flesh-rot.\evb\evg


\bvg\bva%
Hagall es \alst{k}alda \alst{k}orn \hld\ ok \alst{k}nappa drífa &
\ind ok \alst{s}náka \alst{s}ótt.\eva

\bvb Hail is cold kernel and storm of beads \\
\ind and sickness of snakes.\evb\evg


\bvg\bva%
Nauð es \alst{þ}ýjar \alst{þ}rǫ́ \hld\ ok \alst{þ}ungr kostr &
\ind ok \alst{v}ás-samlig \alst{v}erk.\eva

\bvb Need is maidservant’s yearning and scant choice \\
\ind and working in wet-cold weather.\evb\evg


\bvg\bva%
\alst{Í}ss es \alst{á}ar bǫrkr \hld\ ok \alst{u}nnar þękja &
\ind ok \alst{f}ęigra manna \alst{f}ár.\eva

\bvb Ice is river’s bark and wave’s roof \\
\ind and fey men’s danger.\evb\evg


\bvg\bva%
Ár es \alst{g}umna \alst{g}óði \hld\ ok \alst{g}ótt sumar &
\ind \emph{ok} \alst{a}l-gróinn \alst{a}kr.\eva

\bvb Year is men’s boon and good summer \\
\ind (and) all-grown acre.\evb\evg


\bvg\bva%
Sól es \alst{sk}ýja \alst{sk}jǫldr \hld\ ok \alst{sk}ínandi rǫðull &
\ind ok \alst{í}sa \alst{a}ldr-tregi.\eva

\bvb Sun is the shield of clouds and shining wheel \\
\ind and ice-sheets’ life-sorrow.\evb\evg


\bvg\bva%
Týr es \alst{ęi}n-hęndr \alst{ǫ́}ss \hld\ ok \alst{u}lfs lęifar &
\ind ok \alst{h}ofa \alst{h}ilmir.\eva

\bvb Tew is the one-handed Os and the wolf’s leftovers \\
\ind and lord of hoves.\evb\evg


\bvg\bva%
Bjarkan es \alst{l}aufgat \alst{l}im \hld\ ok \alst{l}ítit tré &
\ind ok \alst{u}ng-samligr \alst{v}iðr.\eva

\bvb Birch is leafy branch and little tree \\
\ind and youthful wood.\evb\evg


\bvg\bva%
\alst{M}aðr es \alst{m}anns gaman \hld\ ok \alst{m}oldar auki &
\ind ok \alst{sk}ipa \alst{sk}ręytir.\eva

\bvb Man is man’s joy and the product of dust \\
\ind and adorner of ships.\evb\evg


\bvg\bva%
Lǫgr es \alst{v}ellanda \alst{v}atn \hld\ ok \alst{v}íðr kętill &
\ind ok \alst{g}lǫmmungr \alst{g}rund.\eva

\bvb Liquid is boiling water and wide kettle \\
\ind and TODO.\evb\evg


\bvg\bva%
Ýr es \alst{b}ęndr bogi \hld\ ok \alst{b}rot-gjarnt járn &
\ind ok \alst{f}ęnju \alst{f}lęygir.\eva

\bvb Yew is a bent bow and easily broken iron \\
\ind and arrow’s hurler.\evb\evg

\sectionline

\section{The Norwegian Rune Poem}\chapterStart{}

\begin{flushright}%
\textbf{Dating:} Medieval.%TODO

\textbf{Meter:} Unclear.
\end{flushright}%

The \textbf{Norwegian rune poem} is clearly very closely related to the Icelandic.  With the exception of runes 2 (\emph{úr} ‘slag’) and 4 (\emph{óss} ‘river-mouth’), the names of the runes are identical, as are many of the kennings used to describe them.

Still the language is unmistakably that of mediæval Norway.  As can be seen from the rhymes and alliteration the following uniquely Norwegian sound changes have occurred:
\begin{itemize}
  \item \emph{hl, hn, hr} > \emph{l, n, r} (2 \emph{lęypr} < \emph{hlęypr}; 8 \emph{nęppa} < \emph{hnęppa}; 5 \emph{rossum} < \emph{hrossum}).
  \item \emph{rst} > \emph{st} (5 \emph{vęsta} < \emph{vęrsta})
\end{itemize}

\sectionline

\bvg\bva%
ᚠ \alst{F}é vęldr \alst{f}rę́nda rógi; \hld\ \alst{f}ǿðisk ulfr í skógi.\eva

\bvb Wealth causes the strife of kinsmen; the wolf feeds itself in the wood.\evb\evg


\bvg\bva%
ᚢ \alst{Ú}r ’s af illu jarni; \hld\ \alst{o}pt lęypr ręinn á hjarni.\eva

\bvb TRANSLATION.\evb\evg


\bvg\bva%
ᚦ Þurs vęldr \alst{k}vinna \alst{k}villu; \hld\ \alst{k}átr verðr fár af illu.\eva

\bvb TRANSLATION.\evb\evg


\bvg\bva%
ᚬ Óss er \alst{f}lęstra \alst{f}ęrða \hld\ \alst{f}ǫr, en skalpr er sverða.\eva

\bvb River-mouth is the path of most journeys, and the scabbard-mouth is of swords.\evb\evg


\bvg\bva%
ᚱ \alst{R}ęið kveða \alst{r}ossum vęsta; \hld\ \alst{R}ęginn sló sverðit bęsta.\eva

\bvb Chariot they say is worst for horses; Rein struck the best sword.\evb\evg


\bvg\bva%
ᚴ Kaun er \alst{b}arna \alst{b}ǫlvan; \hld\ \alst{b}ǫl gørvir nán fǫlvan.\eva

\bvb TRANSLATION.\evb\evg


\bvg\bva%
ᚼ Hagall er \alst{k}aldastr \alst{k}orna; \hld\ \alst{K}ristr skóp hęiminn forna.\eva

\bvb Hail is coldest of kernels; Christ created the world of yore.\evb\evg


\bvg\bva%
ᚾ \alst{N}auðr gørir \alst{n}ęppa kosti; \hld\ \alst{n}øktan kęlr í frosti.\eva

\bvb TRANSLATION.\evb\evg


\bvg\bva%
ᛁ Ís kǫllum \alst{b}rú \alst{b}ręiða; \hld\ \alst{b}lindan þarf at lęiða.\eva

\bvb Ice we call a broad bridge; the blind man must be lead.\evb\evg


\bvg\bva%
ᛅ Ár er \alst{g}umna \alst{g}óði; \hld\ \alst{g}et’k at ǫrr var Fróði.\eva

\bvb Year is men’s boon; I recall that Frood was mad.\evb\evg


\bvg\bva%
ᛋ Sól er \alst{l}anda \alst{l}jómi; \hld\ \alst{l}úti’k hęlgum dómi.\eva

\bvb Sun is the light of the lands; I bow in the holy place.\evb\evg


\bvg\bva%
ᛏ Týr er \alst{ę}in-ęndr \alst{á}sa; \hld\ \alst{o}pt verðr smiðr blása.\eva

\bvb Tew is the one-handed of the Eese; the smith must often blow.\evb\evg


\bvg\bva%
ᛒ Bjarkan er \alst{l}auf-grǿnstr \alst{l}íma; \hld\ \alst{L}oki bar flę́rða tíma.\eva

\bvb TRANSLATION.\evb\evg


\bvg\bva%
ᛘ \alst{M}aðr er \alst{m}oldar auki; \hld\ \alst{m}ikil er gręip á hauki.\eva

\bvb Man is the product of dust; mighty is the grip on the hawk.\evb\evg


\bvg\bva%
ᛚ Lǫgr er er \alst{f}ęllr ór \alst{f}jalli \hld\ \alst{f}oss; en gull eru nossir.\eva

\bvb TRANSLATION.\evb\evg


\bvg\bva%
ᛦ Ýr er \alst{v}etr-grǿnstr \alst{v}iða; \hld\ \alst{v}ę́nt ’s, er brennr, at sviða. \eva

\bvb Yew is winter-greenest of trees; ’tis expected, when it burns, to get singed.\evb\evg

\sectionline


% \chapter{Runic Poetry in the Elder Futhark}

% \input{books/runic/Elder Futhark.tex}

\bookStart{Runic Poetry from Sweden and Gotland}

TODO: Introduction to Swedish inscriptions

\sectionline

\section{Sm 16}

\begin{flushright}%
Dating: C11th

Meter: \Fornyrdislag
\end{flushright}%

TODO.

\sectionline

\bvg\bva[]%
Hróstęinn auk \alst{Ęi}lífʀ, \hld\ \alst{Á}ki auk Hǫ́kon, &
ręistu þęiʀ \alst{s}vęinaʀ \hld\ ęptiʀ \alst{s}inn faður &
\alst{k}umbl \alst{k}ęnni-ligt \hld\ ęptiʀ \alst{K}ala dauðan. &
Þý mun \alst{g}óðs manns \hld\ um \alst{g}etit verða, &
með \alst{st}ęinn lifiʀ \hld\ ok \alst{st}afiʀ rúna.\eva

\bvb Rothstan and Anlif, Eke and Hathkin, \\
those lads raised after their father \\
a remarkable monument after the dead Cale. \\
Thus will the good man be spoken of, \\
while the stone lives and the staves of the runes.\evb\evg

\sectionline

\section{Sm 39}

\begin{flushright}%
Dating: C11th

Meter: \Fornyrdislag
\end{flushright}%

A standing stone inscribed on two sides, one of which has a large cross.  The expression is formulaic; cf. Sm 44, Sö 130, U 703, U 739, and U 805.  For “\inx[C]{good of meat}”, which also occurs in \Havamal; see Encyclopedia.  The first line is not poetic.

\sectionline

\bvg\bva[]%
Gunni satti stên þęnna eptiʀ Súna, fǫður sinn, &
\alst{m}ildan orða \hld\ ok \alst{m}ataʀ góðan.\eva

\bvb Guthe set this stone after Sown, his father, \\
mild of words and good of meat.\evb\evg

\sectionline

\section{Sm 44}

\begin{flushright}%
Dating: C11th

Meter: \Fornyrdislag
\end{flushright}%

TODO.  The expression is formulaic; cf. Sm 39, Sö 130, U 703, U 739, and U 805.

\sectionline

\bvg\bva[]%
TODO
mildan við sinna \hld\ ok mataʀ góðan, &
TODO.\eva

\bvb TODO \\
Mild with his men and good of meat. \\
TODO\evb\evg

\sectionline

\section{Sö 34–35 (Tjuvstigen)}

\begin{flushright}%
Dating: C11th–12th

Meter: \Fornyrdislag
\end{flushright}%

Two paired stones standing next to each other.  The last line of Sö 35 is not poetic.

\sectionline

\bvg\bva[Sö 34]%
\alst{St}yrlaugʀ ok Holmbʀ \hld\ \alst{st}ęina ręistu &
at \alst{b}rǿðr sína, \hld\ \alst{b}rautu nę́sta. &
Þęir \alst{ę}ndaðus \hld\ í \alst{au}str-vegi, &
\alst{Þ}órkęll ok Styrbjǫrn, \hld\ \alst{þ}iagnar góðir.\eva

\bvb Sturley and Holm raised the stones, \\
after their brothers, nearest to the road. \\
They were ended in the Eastway, \\
Thurkettle and Sturbern, good thanes.\evb\evg


\bvg\bva[Sö 35]%
Lét \alst{I}ngigęiʀʀ \hld\ \alst{a}nnan ręisa stęin &
at \alst{s}onu \alst{s}ína, \hld\ \alst{s}ýna giǫrði.
Guð hjalpi ǫnd þęira. Þóriʀ hjó.\eva

\bvb Inggar let raise another stone, \\
after his sons made visible. \\
God may help their spirit. Thurer hewed.\evb\evg

\sectionline

\section{Sö 56 (Fyrby)}

\begin{flushright}%
Dating: C11th–12th

Meter: \Fornyrdislag
\end{flushright}%

TODO: INTRODUCTION.

\sectionline

\bvg\bva[] Iak vęit \alst{H}á-stęin \hld\ þá \alst{H}olm-stęin brǿðr &
\alst{m}ęnnr rýnasta \hld\ á \alst{M}ið-garði &
sęttu \alst{st}ęin \hld\ auk \alst{st}afa marga &
eptir \alst{F}ręy-stęin \hld\ \alst{f}ǫður sinn.\eva

\bvb I know Highstan and Holmstan, the brothers, \\
men most rune-cunning in Middenyard, \\
set the stone and many staves, \\
in memory of Freestan, their father.\evb\evg

\sectionline

\section{Sö 65 (Djulefors)}

\begin{flushright}%
Dating: C11th–12th

Meter: \Fornyrdislag\ with hendings in the b-verses
\end{flushright}%

A standing stone inscribed on one side with a large cross.  Already on the earliest depictions the stone was damaged, but an even larger part has now gone missing.  Other stones that mention \inx[L]{Longbeardland} (Lombardy) include TODO...  The meter is highly unusual for runic Swedish poetry, relying on hendings (in line 2 an ethel-hending \emph{arð- : barð-}, in line 3 a shot-hending \emph{land- : ęnd-}).  Line 2b is formulaic; see note.

\sectionline

\bvg\bva[] Inga ręisti stęin þannsi at Ǫ́lęif sinn \textbf{a}... &
Hann austarla \hld\ \edtrans{arði barði}{ploughed with the prow}{\Bfootnote{i.e. “sailed”.  A formulaic poetic expression shared with an anonymous line from the Third Grammatical Treatise, which reads: \emph{sá’s af Íslandi \hld\ arði barði} ‘he who [awawy] from Iceland ploughed with the prow’.}} &
auk ȧ Langbarði- \hld\ landi ęndaðis.\eva

\bvb Inge raised this stone after Anlaf, her ... . \\
Easterly he ploughed with the prow, \\
and on Longbeardland was ended.\evb\evg

\sectionline

\section{Sö 130}

\begin{flushright}%
Dating: C11th–12th

Meter: \Fornyrdislag
\end{flushright}%

A standing stone. TODO.  The expression is formulaic; cf. Sm 39, Sm 44, U 703, U 739, and U 805.

\sectionline

\bvg\bva[] \alst{F}iuriʀ gęrðu \hld\ at \alst{f}ǫður góðan &
\alst{d}ýrð \alst{d}ręngi-la \hld\ at \alst{D}ómara &
\alst{m}ildan orða \hld\ ok \alst{m}ataʀ góðan. &
Þat \dots\eva

\bvb Four men made after a good father, \\
an honour, valiantly, after Doomer, \\
mild of words and good of meat. \\
This \dots\evb\evg

\sectionline

\section{Sö 179 (Gripsholm)}

\begin{flushright}%
Dating: C11th

Meter: \Fornyrdislag
\end{flushright}%

TODO: INTRODUCTION.  The three-line stanza is a biographical addition following a typical prose memorial formula.

\sectionline

\bpg\bpa[] Tóla lét ręisa stęin þennsa at son sinn Harald, bróður Ingvars.\epa

\bpb Tool let raise this stone after his son Harold, brother of Ingwar.\epb\epg

\bvg\bva[] Þęiʀ \alst{f}óru dręngi-la \hld\ \alst{f}iarri at gulli &
ok \alst{au}star-la \hld\ \alst{ę}rni gǫ́fu, &
dóu \alst{s}unnar-la \hld\ á \alst{S}ęrk-landi.\eva

\bvb They journeyed valiantly far for gold, \\
and easterly gave to the eagle; \\
died southerly in Serkland.\evb\evg

\sectionline

\section{U 703}

\begin{flushright}%
Dating: C11th

Meter: \Fornyrdislag
\end{flushright}%

A standing stone inscribed on one side.  There is no cross present, but a large four-legged beast with a long tail.  The stone is heavily damaged, but mostly readable, except for what is here taken to be the half of line 2, which is entirely lost.  The expression is formulaic; cf. Sm 39, Sm 44, Sö 130, U 739, and U 805.  For “\inx[C]{good of meat}”, which also occurs in \Havamal; see Encyclopedia.  The first line is not poetic.

\sectionline

\bvg\bva[U 703]%
Ásvi lét ręisa stęin þennsa at Ǫrnulf, son sinn góðan. &
Hann byggi hér \hld\ ..., &
\alst{m}andr \alst{m}atar goðr \hld\ ok \alst{m}áls risinn.\eva

\bvb Oswye let raise this stone after Arnolf, her good son. \\
He dwelled here ..., \\
a man good of meat and proud of speech.\evb\evg

\sectionline

\section{U 739}

\begin{flushright}%
Dating: C11th

Meter: \Fornyrdislag
\end{flushright}%

A standing stone inscribed on one side, with a large cross present.  There are no major difficulties with the reading.  The expression is formulaic; cf. Sm 39, Sm 44, Sö 130, U 703, and U 805.  “mild of meat” appears to be a variant of “\inx[C]{good of meat}”, which also occurs in \Havamal; see Encyclopedia.  The first line is not poetic.  For other stones raised by someone in memory of themselves, see TODO.

\sectionline

\bvg\bva[U 739]%
Holbjǫrn lét ręisa stęin at sik sjalfan. &
Hann vaʀ \alst{m}ildr \alst{m}ataʀ \hld\ ok \alst{m}áls risinn.\eva

\bvb Holbern let raise this stone after himself. \\
He was mild of meat and proud of speech.\evb\evg

\sectionline

\section{U 805}

\begin{flushright}%
Dating: C11th

Meter: \Fornyrdislag
\end{flushright}%

The stone has been lost, and only survives in old depictions, which makes the reading, especially two of the personal names, uncertain.  My transliteration follows Rundata.

The expression is formulaic; cf. Sm 39, Sm 44, Sö 130, U 703, and U 739.  For “\inx[C]{good of meat}”, which also occurs in \Havamal; see Encyclopedia.  The first line is not poetic.

\sectionline

\bvg\bva[U 805]%
Fylkir lét ręisa st\emph{ęin epti}r \textbf{iel}, bróður sinn, ok Gunnmarr eptir \textbf{menk}, fǫður sinn, &
\alst{b}ónda góðan matar; \hld\ \alst{b}yggi í Víkbý.\eva

\bvb Filch let raise this stone after ..., his brother, and Guthmar after ..., his father, \\
a farmer good of meat; he lived in Wickby.\evb\evg

\sectionline


% \chapter{Younger Runic Poetry from Norway}

% \input{books/runic/Norway.tex}

% \chapter{Younger Runic Poetry from Denmark and Scania}

% \input{books/runic/Denmark.tex}

% \chapter{English Runic Poetry}

% \input{books/runic/England.tex}
% Frow

% Galders: Charms, Spells, and Curses
	Under this section I have gathered sundry \emph{galders} (charms and spells) attested in Old Germanic languages. I have generally only included those with clear Heathen or traditional elements. The Old Saxon baptismal vow, while explicitly anti-Heathen, has also been included due to its mention of Germanic Heathen deities.


\chapter{Continental Germanic spells}

\bookStart{The Two Merseburg Galders}

\begin{flushright}%
Dating: TODO.

Meter: \Fornyrdislag, \Galdralag%
\end{flushright}

These two galders, preserved in a manuscript (TODO) are some of the only surviving examples of genuine Heathen galders from the continent. The two share a common two-part structure, each beginning with an \emph{historiola} (a pseudo-historical account describing the successful effects of the galder in the mythic past), followed by an \emph{imperative}, commanding that the willed effects take place in the present.

The first galder begins with an historiola describing a group of supernatural women in the midst of a battle who placed soldiers in fetters, hindering an army. The imperative then commands that some fetters in the present be destroyed so that captive(s) can escape.

The second galder begins with an historiola describing a group of Gods riding through the woods. Among them is Balder, whose horse sprains its foot. Three Gods are said to have sung (see Note to \emph{bi·guol} below) a healing-galder each over the horse in order to heal it. First sang the goddess Sithguth, then the goddess Sun, and finally the god Weden. The imperative (apparently the same as was sung over Balder’s horse) then commands that a sprain in the present be healed.

\sectionline

\bvg
\bva Ęiris \alst{s}ázun idisi \hld\ \alst{s}ázun hera duo der; &
suma \alst{h}apt \alst{h}ęptidun \hld\ suma \alst{h}ęri lęzidun &
suma \alst{k}lubodun \hld\ umbi \alst{k}uonjo-widi &
\alst{i}n·sprink hapt-bandun \hld\ \alst{i}n·far fígandun &
\edtext{.H.}{\Bfootnote{The meaning of this letter, which is very clear and written in the same hand as the galders, is uncertain. To me, the most convincing suggestion is that it be read as \emph{.N.}, short for Latin \emph{nomen} ‘name’, presumably the name for the person whom the singer wishes to free from the fetters.}}\eva

\bvb Of yore sat dises, sat here, then there: \\
some fastened fetters, some hindered armies, \\
some cleaved shackles (TODO!).— \\
Destroy the fetter-bonds, flee the fiends! \\
.H.\evb
\evg


\bvg
\bva \edtext{\alst{F}ol}{\Afootnote{\emph{Phol} ms.}} ęnde Wódan \hld\ \alst{f}uorun zi holza &
dú wart demo Balderes \alst{f}olon \hld\ sín \alst{f}uoz bi·ręnkit &
þú \edtrans{bi·guol}{begale}{\Bfootnote{third past singular of \emph{bi·galan} ‘begale’, transitive of \emph{galan} ‘gale, sing a galder’. This verb is important as it is the origin of the verbal noun “galder” (literally ‘something galed’), which is thus shown to describe the charm.}} en \edtext{\alst{S}inthgunt}{\lemma{Sinthgunt}\Afootnote{\emph{Sinhtgunt} ms.}} \hld\ \alst{S}unna era swister &
þú bi·guol en \alst{F}rija \hld\ \alst{F}olla era swister &
þú bi·guol en \alst{W}ódan \hld\ só hé \alst{w}ola konda &
só-se \alst{b}èn-ręnkí \hld\ só-se \alst{b}luot-ręnkí \hld\ só-se lidi-ręnkí &
\ind \alst{b}èn zi \alst{b}èna &
\ind \alst{b}luot zi \alst{b}luoda &
\alst{l}id zi ge·\alst{l}iden \hld\ só-se ge·\alst{l}imida sín!\eva

\bvb Phol and Weden journeyed in the woods; \\
then was the foot of Balder’s foal sprained. \\
Then \inx[C]{begale}[begaled] him \inx[P]{Sithguth}, \inx[P]{Sun} her sister; \\
then begaled him \inx[P]{Frie}, \inx[P]{Full} her sister; \\
then begaled him Weden, as he knew well: \\
Like bone-sprain, like blood-sprain, like joint-sprain! \\
Bone to bone, \\
blood to blood, \\
joint to joints, like were they glued together!\evb
\evg



\section{Against worms (Contra vermes)}

\bvg
\bva Gang út, \alst{n}esso, \hld\ mid \alst{n}igun \alst{n}essi-klínon, &
ut fana þemo marge an þat \alst{b}èn, \hld\ fan þemo \alst{b}ène an þat flesg, &
ut fan þemo flesgke an þia \alst{h}úd, \hld\ ut fan þera \alst{h}úd an þesa strála. &
Drohtin, werthe só.\eva

\bvb Go out, Nesse, with nine small Nesses! Out from the marrow onto the bone, from this bone onto the flesh, out from the flesh onto the skin, out from the skin onto these arrows. Lord, may it be so.\evb
\evg


\section{The Old Saxon Baptismal vow}

\bpg
\bpa „For·sachistu diobole?“ \emph{et respondeat:} „ec for·sacho diabole“\epa

\bpb “Forsakest thou the Devil?” \emph{and he should respond:} “I forsake the Devil.”\epb
\epg


\bpg
\bpa „end allum diobol-gelde?“ \emph{respondeat:} „end ec for·sacho allum diobol-gelde.“\epa

\bpb “And all devil-yields?” \emph{he should respond:} “I forsake all devil-yields.”\epb
\epg


\bpg
\bpa „End allum dioboles wercum?“ \emph{respondeat} „end ec for·sacho allum dioboles wercum and wordum, Thuner ende Wóden ende Sax-nòte ende allem them un·holdum the hira ge·nòtas sint.“\epa

\bpb “And all the Devil’s works” \emph{he should respond:} “and I forsake all the works and words of the Devil; Thunder and Weden and Saxneet and all those unhold ones who are their fellows.”\epb
\epg


\bpg
\bpa „Ge·lòbistu in Got ala-męhtigun fader?“ „Ec ge·lòbo in Got ala-męhtigun fader.“\epa

\bpb “Believest thou in God, the almighty father?” “I believe in God, the almighty father.”\epb
\epg


\bpg
\bpa „Ge·lòbistu in Crist Godes suno?“ „Ec ge·lòbo in Crist Gotes suno.“\epa

\bpb “Believest thou in Christ, God’s son?” “I believe in Christ, God’s son.”\epb
\epg


\bpg
\bpa „Ge·lòbistu in hàlogan gàst?“ „Ec ge·lòbo in hàlogan gàst.“\epa

\bpb “Believest thou in the Holy Ghost?” “I believe in the Holy Ghost.”\epb
\epg


\chapter{Old English spells}

%\bookStart{Against Dwarf}[Wið dweorh]

\begin{flushright}%
Dating: TODO

Meter: \Fornyrdislag%
\end{flushright}

TODO: Introduction.

\sectionline

\bpg\bpa Mann sceal niman \emph{seofon} lytle of-lætan swylce mann mid ofrað, ond wrítan þás naman on ælcre oflætan: Maximianus, Malchus, Johannes, Martinianus, Dionisius, Constantinus, Serafion.  Þænne eft þæt galdor þæt hér æfter cweð[eð] mann sceal singan, ærest on þæt wynstre éare, þænne on þæt swíðre éare, þænne búfan þæs mannes moldan; ond gá þænne ân mæden-mann tó, ond hó hit ǫn his sweoran, ond dó mann swá þrý dagas.  Him bið sóna sél.\epa

\bpb One shall take seven small small wafers, such as one offers [during the Mass], and write these names on each wafer: Maximianus, Malchus, Johannes, Martinianus, Dionysius, Constantinus, Seraphion.  After that shall one sing this galder which is henceforth said; first into the left ear, then into the right ear, then over the man’s head; and thereafter a maiden go forth, and hang it on his neck; and one do so for three days.  He will soon be well.\epb\epg


\bvg\bva Hér cóm in·gangan \hld\ in·spiden wiht, &
hæfde him his haman ǫn handa; \hld\ cwæð þæt þú his hæncgest wǽre, &
lęgeþe þé his téage \emph{ǫ}n sweoran; \hld\ ǫn·gunnan him ǫf þæm lande líðan. &
Sóna swá hý ǫf þæm lande cóman \hld\ þá ǫn·gunnan him þá \emph{leomu} cólian.— &
Þá cóm in·gangan \hld\ déores sweostar; &
þá ge·ændode héo \hld\ ond âðas swór, &
þæt næfre þis þæm adlegan \hld\ \emph{egl}ian ne móste &
né þæm þe þis galdor \hld\ be·gýtan mihte &
oððe þe þis galdor \hld\ on·galan cu̇ðe. &
Amen fiað.\eva

\bvb Here came walking in an inspiden wight, \\
had his harness in his hands; said that thou wert his horse, \\
laid his reins on thy neck; then they together began to ride from the land. \\
As soon as they came away from the land, then they together began to cool limbs. \\
Then came walking in the beast’s sister; \\
then she ended [it], and swore oaths, \\
that this never should harm the ailing man, \\
nor him who this galder might get, \\
nor whomever this galder could gale. \\
Amen, let it be.\evb\evg


\bookStart{Against a Sudden Stitch}[Wið fǽr-stice]

\begin{flushright}%
Dating: ?

Meter: \Fornyrdislag%para
\end{flushright}%

Attested in \Lacnunga.

\sectionline

\bvg
\bva \alst{H}lúde wǽran hý, lá, \alst{h}lúde, \hld\ þá hý ofer þone \alst{h}lǽw ridan, &
wǽran \alst{â}n-móde, \hld\ þá hý \alst{o}fer land ridan. &
Scyld þú þé nú, þú þysne \alst{n}íð \hld\ ge·\alst{n}esan móte. &
\alst{Ú}t, lýtel spere, \hld\ gif hér \alst{i}nne síe!\eva

\bvb Loud were they, lo, loud, when they rode over that mound; \\
they were steadfast, when they rode over land. \\
Shield thyself now; thou mayst escape this evil! \\
Out little spear, if here within it be!\evb
\evg


\bvg
\bva Stód under \alst{l}inde, \hld\ under \alst{l}eohtum scylde, &
þær þá \alst{m}ihtigan wíf \hld\ hýra \alst{m}ægen be·rǽddon &
and hý \alst{g}yllende \hld\ \alst{g}âras sændan; &
ic him \alst{ó}ðerne \hld\ \alst{e}ft wille sændan, &
\alst{f}léogende \alst{f}lâne \hld\ \alst{f}orane tó·géanes. &
\alst{Ú}t, lytel spere, \hld\ gif hit her \alst{i}nne sý!\eva

\bvb Stood under the linden \ken{shield}—under the light shield— \\
where those mighty wives their might arrayed, \\
and they yelling spears did send. \\
To them another [projectile] will I send back: \\
a flying arrow, aimed against [them]. \\
Out little spear, if here within it be!\evb
\evg


\bvg
\bva \alst{S}æt \alst{s}mið, \hld\ \alst{s}loh seax, &
lytel \alst{í}serna, \hld\ \alst{w}und swíðe. &
\alst{Ú}t, lytel spere, \hld\ gif her \alst{i}nne sý!\eva

\bvb Sat the smith, struck the sax: \\
a little iron-thing—a great wound. \\
Out little spear, if here within it be!\evb
\evg


\bvg
\bva \alst{S}yx \alst{s}miðas \alst{s}ætan, &
\alst{w}æl-spera \alst{w}orhtan. &
\alst{Ú}t, spere, \hld\ næs \alst{i}n, spere! &
Gif her \alst{i}nne sý \hld\ \alst{í}senes dǽl, &
\alst{h}æg-tessan ge·weorc, \hld\ \alst{h}it sceal ge·myltan.\eva

\bvb Six smiths sat, \\
wrought slaughter-spears. \\
Out, spear! Be not in, spear! \\
If here within be a part of iron, \\
the work of a \inx[C]{hag-tess}—\emph{it} shall melt!\evb
\evg


\bvg
\bva Gif þú wǽre on \alst{f}ell scoten \hld\ oððe wǽre on \alst{f}læsc scoten &
oððe wǽre on blód scoten \hld\ [...] &
oððe wǽre on \alst{l}ið scoten, \hld\ næfre ne sý þín \alst{l}íf atæsed;\eva

\bvb If thou wert shot in the skin, or wert shot in the flesh, \\
or wert shot in the blood, [...], \\
or wert shot in the limb—never be thy life injured.\evb
\evg


\bvg
\bva gif hit wǽre \alst{ė}sa ge·scot \hld\ oððe hit wǽre \alst{y}lfa ge·scot &
oððe hit wǽre \alst{h}æg-tessan ge·scot, \hld\ nú ic wille þín \alst{h}elpan: &
þis þé tó bóte \alst{ė}sa ge·scotes, \hld\ þis þé tó bóte \alst{y}lfa ge·scotes, &
þis þé tó bóte \alst{h}æg-tessan ge·scotes; \hld\ ic þín wille \alst{h}elpan.\eva

\bvb If it were Eese-shot, or it were Elf-shot,\footnoteB{Formulaic; see \inx[F]{Eese and Elves}. That they are held in the same category as the hag-tess—a witch—indicates Christian influence. Among the Germanic peoples the elves and Eese were originally beneficial, as seen by numerous names like Alfred (OE \emph{Ęlf-réd} ‘Elf-counsel’), Oswald (OE \emph{Ós-weald} ‘Os-power’), Elfwin (Lomb. \emph{Alb-oin} ‘Elf-friend’), Oshelm (Lomb. \emph{Anselm} ‘Os-helmet’).}  \\
or it were Hag-tess-shot—now I will help thee! \\
This for thee as cure against Eese-shot; this for thee as cure against Elf-shot;  \\
this for thee as cure against Hag-tess-shot—I will help thee!\evb
\evg


\bvg
\bva \alst{F}leo þær on \hld\ \alst{f}yrgen-hæfde! &
\alst{H}âl wes-tu, \hld\ \alst{h}elpe þín drihten! &
Nim þonne þæt seax, \hld\ ado on wætan.\eva

\bvb TODO. \\
Be thou hale, may the Lord help thee.\evb
\evg


\bookStart{The Nine Herbs Galder}

\begin{flushright}%
Dating: ?

Meter: \Fornyrdislag%para
\end{flushright}%

\sectionline

\bvg
\bva[0]Ge·myne ðú mug-wyrt \hld\ hwæt þú á·meldodest &
hwæt þu renadest \hld\ æt Regen-melde?\eva

\bvb Rememberest thou, Mugwort, what thou madest known,  \\
what thou arrangedest at Reinmeld?\evb
\evg


\bvg\setlinenum{2}
\bva[0]Una þú hàttest \hld\ yldost wyrta &
þú miht wið III \hld\ and wið XXX &
þú miht wiþ attre \hld\ and wið on·flyge &
þú miht wiþ þàm làþan \hld\ ðe geond lond færð\eva

\bvb Un art thou called, oldest of worts; \\
thou availest against three and against thirty; \\
thou availest against the venom and against the onflier; \\
thou availest against the loathsome one that journeys through the lands.\evb
\evg


\bvg\setlinenum{6}
\bva[0]+ Ond þú weg·bráde \hld\ wyrta módor &
éastan opene \hld\ innan mihtigu &
ofer ðy cræte curran \hld\ ofer ðy cwéne réodan &
\ind ofer ðy brýde brýodedon &
\ind ofer ðy fearras fnærdon.\eva

\bvb And thou, Waybroad, mother of worts, open from the east, mighty from within. Over thee TODO.\evb
\evg


\bvg\setlinenum{6}
\bva[0]Eallum þu þon wið·stóde \hld\ and wið·stunedest &
swá ðú wið·stonde attre \hld\ and on·flyge &
and þǽm làðan \hld\ þe geond lond fereð.\eva

\bvb Them all withstoodest thou then, and stoppedst; \\
so may thou withstand the venom and the onflier, \\
and the loathsome one that journeys through the lands.\evb
\evg


\bvg\setlinenum{6}
\bva[0]Stune hætte þéos wyrt, \hld\ héo on stàne ge·weox &
stond héo wið attre, \hld\ stunað héo wærce &
Stiðe héo hatte, \hld\ wið·stunað héo attre &
wreceð héo wràðan, \hld\ weorpeð út attor.\eva

\bvb Stun is this wort called, she grew on stone; \\
she withstands venom, she stops aches. \\
Stithe is she called, she stops the venom; \\
she drives away the wroth one, she casts out the venom.\evb
\evg


\bvg\setlinenum{6}
\bva[0]+ Þis is séo wyrt \hld\ séo wiþ wyrm ge·feaht &
þéos mæg wið attre, \hld\ héo mæg wið on·flyge; &
héo mæg wið ðàm làþan \hld\ ðe geond lond fereþ.\eva

\bvb This is the wort that fought against the Wyrm; \\
this one avails against the venom, she avails against the onflier; \\
she avails against the loathsome one that journeys through the lands.\evb
\evg


\bvg\setlinenum{6}
\bva[0]Fleoh þú nú attor-làðe, \hld\ séo lǽsse ðá màran &
séo màre þá lǽssan, \hld\ oððæt him beigra bót sý!\eva

\bvb TODO\evb
\evg


\bvg\setlinenum{6}
\bva[0]Ge·myne þú, mægðe,\hld\ hwæt þú á·meldodest &
hwæt ðú ge·ændadest \hld\ æt Alor-forda &
þæt nǽfre for ge·floge \hld\ feorh ne ge·sealde &
syþðan him mǫn mægðan \hld\ tú mete ge·gyrede\eva

\bvb TODO\evb
\evg


\bvg\setlinenum{6}
\bva[0]Þis is séo wyrt \hld\ ðe wer-gulu hatte &
ðás on·sænde seolh \hld\ ofer sǽs hrygc &
ondan attres \hld\ óþres tó bóte\eva

\bvb TODO\evb
\evg


\bvg\setlinenum{6}
\bva[0]Ðás VIIII magon \hld\ wið nygon attrum.\eva

\bvb These nine avail against nine venoms.\evb
\evg


\bvg\setlinenum{6}
\bva[0]+ Wyrm cóm snícan, \hld\ to·slàt hé man &
ðá ge·nam Wóden \hld\ VIIII wuldor-tànas &
slóh ðá þá nǽddran \hld\ þæt héo on VIIII tó·fléah &
Þǽr ge·ændade æppel \hld\ and attor &
þæt héo nǽfre ne wolde \hld\ on hús búgan.\eva

\bvb A \inx[C]{Wyrm} came crawling; he tore apart a man. \\
Then took Weden nine glory-twigs, \\
slew then that adder, that it sprung into nine [parts]. \\
There ended apple and venom, \\
that she would never wish to enter a house.\evb
\evg


\bvg\setlinenum{6}
\bva[0]+ Fille and finule, \hld\ fela-mihtigu twá &
þá wyrte ge·sceop \hld\ wítig drihten &
hàlig on heofonum, \hld\ þá hé hongode &
sette and sænde \hld\ on VII worulde &
earmum and éadigum \hld\ eallum tó bóte\eva

\bvb Fill and Fennel, the many-mighty two; \\
those worts shaped the wise lord, \\
holy in heaven, when he hung. \\
He set and sent them into seven worlds, \\
for wretched men and for wealthy, for all men as a cure.\evb
\evg


\bvg\setlinenum{6}
\bva[0]Stond héo wið wærce, \hld\ stunað héo wið attre &
séo mæg \edtext{wið III \hld\ \emph{and} wið XXX}{\lemma{wið III and wið XXX ‘against three and against thirty’}\Bfootnote{Formulaic; an uncountable amount; “snakes” are probably understood. This oral formula appears in many folk ballads, viz. (Child) 4EFG, 18B, 20C, 30, 53BCDEIKM, 63EFH, 73I, 97AC, 100AG, 110BGH, 156G, 185A, 187A, 187C, 190A, 192A, 193B, 203C, 211A, 217GHLN, 244A, 268A, 269C, 281ABC. Things described include horses, heads of cattle, warriors, days, years, winters.}} &
wið [féondes] hond \hld\ and wið fǽr-bregde &
wið malscrunge \hld\ manra wihta\eva

\bvb She stands against ache, she stands against venom;
she avails against three and against thirty;
against \evb
\evg


\bvg\setlinenum{6}
\bva[0]+ Nu magon þás VIIII wyrta \hld\ wið nygon wuldor-ge·flogenum &
wið VIIII attrum \hld\ and wið nygon on·flygnum &
wið ðý réadan attre, \hld\ wið ðý runlan attre &
wið ðý hwitan attre, \hld\ wið ðý [hæwe]nan attre &
wið ðý geolwan attre, \hld\ wið ðý grénan attre &
wið ðý wonnan attre, \hld\ wið ðý wedenan attre &
wið ðý brúnan attre, \hld\ wið ðý basewan attre &
wið wyrm-ge·blæd, \hld\ wið wæter-ge·blæd &
wið þorn-ge·blæd, \hld\ wið þystel-ge·blæd &
wið ýs-ge·blæd, \hld\ wið attor-ge·blæd\eva

\bvb Now these nine worts avail against glory-onfliers: \\
against nine venoms and against nine onfliers; \\
against the red venom; against the TODO venom; \\
against the white venom; against the TODO venom; \\
against the yellow venom; against the green venom; \\
against the TODO venom; against the TODO venom; \\
against the brown venom; against the TODO venom; \\
against worm-TODO; against water-TODO; \\
against thorn-TODO; against thistle-TODO; \\
against ice-TODO; against venom-TODO.\evb
\evg


\bvg\setlinenum{6}
\bva[0]Gif ænig attor cume \hld\ éastan fleógan &
oððe ǽnig norðan cume &
oððe ǽnig westan \hld\ ofer wer-ðeóde\eva

\bvb If any venom should come flying from the east; \\
or any come from the north; \\
or any from the west, over mankind.\evb
\evg


\bvg\setlinenum{6}
\bva[0]+ Críst stód ofer ádle \hld\ ǽngan cundes &
Ic àna wàt \hld\ éa rinnende &
þǽr þá nygon nǽdran \hld\ néan be·healdað\eva

\bvb Christ stood over TODO; \\
I know one river running, \\
there the nine adders TODO.\evb
\evg


\bvg\setlinenum{6}
\bva[0]Motan ealle wéoda \hld\ nu wyrtum á·springan &
sǽs tó·slúpan, \hld\ eal sealt wæter &
ðonne ic þis attor \hld\ of ðé ge·bláwe\eva

\bvb TODO\evb
\evg


\bpg\bpa Mucgwyrt, weg-brade þe eastan open sy, lombes-cyrse, attor-laðan, mageðan, netelan, wudu-sur-æppel, fille and finul, ealde sapan. Ge·wyrc ða wyrta to duste, mængc wiþ þa sapan and wiþ þæs æpples gor.\epa

\bpb TODO.\epb\epg


\bpg\bpa Wyrc slypan of wætere and of axsan, ge·nim finol, wyl on þære slyppan and beþe mid æggemongc, þonne he þa sealfe on do, ge ær ge æfter.\epa

\bpb TODO.\epb\epg


\bpg\bpa Sing þæt galdor on æcre þara wyrta, :III: ær he hy wyrce and on þone æppel eal-swa; ond singe þon men in þone muð and in þa earan buta and on ða wunde þæt ilce gealdor, ær he þa sealfe on do :.\epa

\bpb TODO.\epb\epg



\chapter{Old Norse spells}

\section{Ribe rune charm}

\bvg
\bva[]\alst{Jo}rð bið ak varðę \hld\ ok \alst{u}p-himęn &
\alst{s}ól ok \alst{s}antę María \hld\ ok \alst{s}alfęn Guð dróttęn &
þęt hamn \alst{l}ę́ mik \alst{l}ę́knęs-hand \hld\ ok \alst{l}yf-tungę &
at lyfę \alst{b}ifjandę \hld\ þęr \alst{b}ótę þarf. &
\ind Ór \alst{b}ak ok ór \alst{b}ryst
\ind ór \alst{l}íkę ok ór \alst{l}im &
\ind ór \alst{ǿ}fęn ok ór \alst{ǿ}ręn &
\ind ór \alst{a}llę þé þęr \alst{i}llt kann í \alst{a}t-kumę. &
Svart hètęr \alst{st}ènn \hld\ han \alst{st}ę́r í hafę útę, &
\ind þęr liggęr á þé \alst{n}í\emph{u} \alst{n}auðęr; &
\ind þęr skulę hvęrki \alst{s}ǿtęn \alst{s}ofę; &
\ind ęð \alst{v}armęn \alst{v}akę; &
førr ęn þú þęssa bót biðęr, þęr ak orð at-kvę́ðę ronti.\eva

\bvb I bid earth to ward, and up-heaven, \\
sun and saint Mary—and the very lord God, \\
that he lend me a healing-hand and medicine-tongue, \\
as medicine for the trembling one who needs a cure. \\
Out of back and out of breast; \\
out of body and out of limb; \\
out of eyes and out of ears; \\
out of everything where evil which might come in! \\
Swart is called a stone, he stands out in the ocean: \\
there lie on it nine needs; \\
they will not [let thee] sleep sweetly \\
nor wake warmly— \\
until thou prayest this cure, where I tried the words of the charms.\evb
\evg

\section{Charms from Bryggen}

These charms are found inscribed on medieval pieces of wood found at Bryggen in the city of Bergen, Norway.

\sectionline

A stick with four sides, dated to c. 1335. It is clearly a love-charm and—as seen by the feminine dative adjective \emph{sjalfri} ‘self’ on side C—addressed to a woman. The language is very close to that of \Skirnismal\ 36, wherein Shirner threatens to curse the ettin-woman Gird with \emph{ęrgi} ‘degeneracy’ and \emph{ǿði} ‘madness’ and \emph{óþoli} ‘impatience’ unless she sleep with his master, Free. A crucial difference is of course that this charm is not an Eddic narrative poem; it must have been expected to work. Both of these share a root with the curse-formula seen on the two C7th runic inscriptions from Stentoften and Björketorp (see TODO), wherein the destroyer of the respective monuments will be \emph{hermalausaʀ argjú} ‘restless with degeneracy’, i.e. ‘incessantly randy’. As it would be absurd to think that the poet of \Skirnismal\ should have learned this type of magic from one of the rune-stones, and then passed this onto the carver of the present inscription, we must rather be dealing with a common form of curse magic, wherein the victim is cursed with incessant randiness leading to sexual perversion.

\bvg
\bva[A]\mssnote{\textbf{B257}}Ríst ek \alst{b}ót-rúnar \hld\ ríst ek \alst{b}jarg-rúnar &
\ind \alst{ei}n-falt við \alst{ǫ}lfum &
\ind \alst{t}ví-falt við \alst{t}rollum &
\ind \alst{þ}rí-falt við \alst{þ}u\emph{rsum}\eva

\bvb I carve healing-runes; I carve saving-runes; onefold against elves; twofold against trolls; threefold against thurses.\evb
\evg


\bvg
\bva[B]Við inni \alst{sk}ǿðu \hld\ \alst{sk}ag-val-kyrju &
svá’t \alst{ei} megi \hld\ þó-at \alst{ę́} vili &
\alst{l}ę́-vís kona \hld\ \alst{l}ífi þínu g\emph{randa}.\eva

\bvb Against the scatheful shag-walkirrie, so that she may not—although she ever wishes to, that guile-wise woman—harm thy life.\evb
\evg


\bvg
\bva[C]Ek \alst{s}endir þér \hld\ ek \alst{s}é á þér &
\alst{y}lgjar \alst{e}rgi \hld\ ok \alst{ó}þola; &
á þér hríni \alst{ó}þoli \hld\ ok \alst{jǫ}tuns móð\emph{r}; &
\alst{s}it-tu aldri, \hld\ \alst{s}op-tu aldri.\eva

\bvb I send to thee—I see on thee—a she-wolf’s degeneracy and impatience; on thee stick impatience, and an ettin’s wrath! Sit thou never, sleep thou never!\evb
\evg


\bvg
\bva[D]Ant mér sem sjalfri þér. Beirist rubus rabus et arantabus laus abus rosa gava\eva

\bvb Love me like thy self.\evb
\evg

\sectionline

\bvg
\bva[]\mssnote{\textbf{B380}}\alst{H}ęill sé þú \hld\ ok í \alst{h}ugum góðum; &
\ind \alst{Þ}órr þik \alst{þ}iggi, &
\ind \edtrans{\alst{Ó}ðinn þik \alst{ęi}gi}{‘may Weden own thee’}{\Bfootnote{See note to \Voluspa\ 23.}}.\eva

\bvb Be thou hale, and in good spirits;\footnoteB{A formula also attested in \Hymiskvida\ 41; see there for parallels.} may Thunder receive thee, may Weden own thee.\evb
\evg


\section{Runic plates}
% Assorted spells

\part{Poetry on Christian Subjects}
	\bookStart{Old Saxon Baptismal Vow}

\begin{flushright}%
Dating: ?

Meter: Prose.
\end{flushright}%

While not an alliterative poem in the slightest, this short text is important for its mention of Saxon Heathen Gods, and as I have no section for Miscellanea, I have here set it first among the Christian poetry, in order to give relevant cultural context.  The format of the text is straight-forward and resembles the modern Catholic questions asked to participants during the Sacrament of Confirmation (TODO: reference).  The person to be baptised is to respond positively to three denying and three affirming questions; first to forsake the Devil (P1), all “Devil-yields” (i.e. non-Christian rituals, see note to that word) (P2), and all the Devil’s “works and words” and his followers, among which are listed the three Germanic-Saxon gods Thunder, Weden, and Saxneet (P3); and then to profess belief in each member of the Trinity: God the almighty father (P4), Christ God’s son (P5), and the Holy Ghost (P6).

\sectionline

\bpg
\bpa „For·sachistu diobole?“ et respondeat: „ec for·sacho diabole“\epa

\bpb “Forsakest thou the Devil?” \emph{and he should respond:} “I forsake the Devil.”\epb\epg


\bpg
\bpa „end allum \edtrans{diobol-gelde}{devil-yields}{\Bfootnote{An obvious calque of OE TODO, which means TODO.}}?“ respondeat: „end ec for·sacho allum diobol-gelde.“\epa

\bpb “And all devil-yields?” \emph{he should respond:} “I forsake all devil-yields.”\epb\epg


\bpg
\bpa „End allum dioboles wercum?“ respondeat „end ec for·sacho allum dioboles wercum and wordum, Thuner ende Wóden ende Sax-nôte ende allem them un·holdum the hira ge·nôtas sint.“\epa

\bpb “And all the Devil’s works” \emph{he should respond:} “and I forsake all the works and words of the Devil; Thunder and Weden and Saxneet and all those unhold ones who are their fellows.”\epb\epg


\bpg
\bpa „Ge·lôbistu in Got ala-męhtigun fader?“ „Ec ge·lôbo in Got ala-męhtigun fader.“\epa

\bpb “Believest thou in God, the almighty father?” “I believe in God, the almighty father.”\epb\epg


\bpg
\bpa „Ge·lôbistu in Crist Godes suno?“ „Ec ge·lôbo in Crist Gotes suno.“\epa

\bpb “Believest thou in Christ, God’s son?” “I believe in Christ, God’s son.”\epb\epg


\bpg
\bpa „Ge·lôbistu in hâlogan gâst?“ „Ec ge·lôbo in hâlogan gâst.“\epa

\bpb “Believest thou in the Holy Ghost?” “I believe in the Holy Ghost.”\epb\epg

\sectionline

	\bookStart{Heliand}% TODO. Check and remove all TODO and NOTE tags in the text.
\def\thisBookCode{Heliand}

\begin{flushright}%
\textbf{Dating:} 830s

\textbf{Meter:} \Fornyrdislag%para
\end{flushright}%

\section{Introduction}

The \textbf{Heliand} (\Heliand; OS \emph{Hêljand} ‘Saviour’, cf. OE \emph{Hę̂lend}, German \emph{Heiland}) is an Old Saxon epic poem that narrates the life of Jesus.  It is essentially a verse paraphrase of Tatian’s C2nd gospel harmony, the \emph{Diatessaron}, and is by far the most important source of Old Saxon literature.

A Latin preface is preserved independently of the poem itself.  According to this short text, \Heliand\ was composed at the behest of emperor Ludwig (\emph{Ludowicus}, probably Louis “the Pious” 778–840, son of Charlemagne), who commanded a Saxon man, “who was regarded among his own as a not undistinguished poet” (\emph{qui apud suos non ignobilis vates habebatur}) to render the entirity of the Old and New Testaments into Saxon verse.  Thus, he, “beginning with the creation of the world, and summarizing according to the truth of history the most significant events, at times depicting certain events with a mystical sense where he saw fit, led the interpretation, according to poetic custom and with rather witty eloquence, through to the end of the entire Old and New Testaments.” (\emph{a mundi creatione initium capiens, iuxta historiae veritatem quaeque excellentiora summatim decerpens, interdum quaedam ubi commodum duxit, mystico sensu depingens, ad finem totius Veteris ac Novi Testamenti interpretando more poetico satis faceta eloquentia perduxit.})  According to native custom, the work was divided into fitts (\emph{vitteas}).

There is no reason to doubt the general truth of this account, although it is hard to believe that the poet should have rendered the entirity of the Old and New Testaments, including the prophets and epistles, into alliterative verse.  The rendering of the Old Testament is probably to be identified with \SaxonGenesis, while the New Testament is what we have before us.

At the end of the preface we hear something much less believable, for we are told that “they say that this same poet, while he was still entirely ignorant of this art, was warned in a dream to adapt the precepts of the Sacred Law into song, with a fitting melody in his own language.” (\emph{ferunt eundem Vatem dum adhuc artis huius penitus esset ignarus, in somnis esse admonitum, ut Sacrae Legis praecepta ad cantilenam propriae linguae congrua modulatione coaptaret.})

This narrative is apparently taken from Bede’s account of Cadman (see Cadman’s Hymn below), but whatsoever be the case with the Hymn, it can scarcely be true about \Heliand\ and \SaxonGenesis.  The style of these two poems is very intricate, and the poet was certainly trained in the traditional craft, likely having learned and composed much vernacular heroic poetry before undertaking the task of the Biblical epics.  The preface itself says as much when it says that he “was regarded among his own as a not undistinguished poet”.

Good evidence for his having previous training can be found in his proficient use of such “Beowulfian” type scenes as the great feast in the mead-hall (2005–12, 2736–42) or the stormy sea-voyage (2233–68, 2906–65).  It is just in these episodes that the poetry is most expressive and least repetitive, for it is just here that he can make the most use of his inherited stock of oral poetic formulaic expressions, synonyms, and kennings.  Likewise, the speeches made by Christ’s disciples, with their talk of ever-lasting fame and glory (e.g. Thomas’s speech 3994–4002), and their service as thanes to their lord (drighten) Christ, clearly harken back to those of pagan heroic poetry, as does the constant emphasis on the noble ancestry of Christ and his disciples—these are no commoners, but rather members of a noble, elite warband, much like the presumed audience of the poem.

On the other hand we should not (as some authors have done) make the mistake of taking these traditional elements as proof that the religion of \Heliand\ is some sort of Germanic “warrior Christianity”.  Such elements were unavoidable since they were built into the very essence of the alliterative poetic tradition, but in spite of them the Christian message of pacifism and humility is present throughout.  The Germanic warrior ideology even comes under direct attack by the denigrating of its own vocabulary, as in lines 5040–50, which condemn the boastful pride and strength of a warrior as useless without help from God.

There are other important ways in which \Heliand\ consciously departs from the Germanic heroic tradition.  One is the idea of fate.  In the old pagan tradition hostile fate often plays the key role in driving the narrative, as is the case in \Hildebrandslied\ and the Walsing Cycle.  Although \Heliand\ refers to fated events by what are in all likelihood originally pagan expressions like \emph{regano gi·skapu} ‘Shapes of the Reins’, they are also \emph{godes gi·skapu} ‘God’s Shapes’, indicating that God is the ruler of the destinies of men, not the ambivalent Norns.

Another departure is in the language of war, especially in the disuse of the traditional feminine words for war, \emph{*gu̇ðja} and \emph{hildi}.  In \Hildebrandslied\ and Old Norse and Old English poetry both words are very common, but in \Heliand\ the former is entirely absent, while the latter is only used twice, in both cases disparagingly.  This break becomes especially apparent when one considers how, as mentioned above, \Heliand\ otherwise adheres fairly closely to tradition in the context of the sea-voyage and mead-hall.  The reason seems straightforward enough; these feminine nouns were too closely tied to Weden’s \inx[G]{walkirries} and the associated ferocious celebration of war for the poet to be comfortable using them, and so he instead opted for neuter synonyms like \emph{stríd, ur-lagi, wíg}, and \emph{gi·winn}.  Their presence in earlier Old Saxon language is in any case assured by their use in early OS women’s names and compounds like \emph{gu̇þ-fano} ‘field standard’ and \emph{hildi-skalk} ‘war-servant, warrior’.

\subsection{Orthography}

Notes on the normalization:
  \begin{itemize}
    \item Long vowels are marked by the acute rather than by the circumflex accent or macron. This is both faithful to the original manuscripts and concordant with my practice in normalising other Germanic languages.
    \item Long vowels \emph{ê} and \emph{ô} resulting from monophthongisation of diphthongs \emph{ai} and \emph{au} are, however, written with the circumflex accent. That these were in fact articulated separately is seen by the following circumstance: in the mss. etymological \emph{é} and \emph{ó} are frequently written as \emph{ie} and \emph{uo}, but this is never done for \emph{ê} and \emph{ô}.
    \item If attested in all mss., epenthetic (\emph{svara-bʰaktí}) vowels are marked with an underdot. Otherwise they are deleted.
    \item Unstressed \emph{a}-vowels reduced to \emph{e} in \textbf{C} are reverted back to \emph{a}
    \item Long vowels resulting from nasal assimilation are marked with an overdot. \emph{i} is written as \emph{ï}.
    \item ms. \emph{e} and \emph{i}, when occuring between vowels are written as \emph{j}.
    \item ms. \emph{i}, when word-initial or following \emph{g} and corresponding to etymological \emph{j} is written as \emph{j}
    \item ms. \emph{e} as resulting from \emph{i}-mutation is written as \emph{ę}.
    \item ms. \emph{b} or \emph{ƀ}, when representing the voiced bilabial fricative, is written as \emph{v}.
    \item ms. \emph{th} is written as \emph{þ}.
    \item ms. \emph{uu} is written as \emph{w}.
  \end{itemize}

\subsection{Preservation}
The following is an exhaustive list of source mss. in chronological order.

\begin{small}\begin{longtabu} to \textwidth {|c l c c|}
	\hline
	Siglum & Date & Lines & Full name \\
	\hline\hline\endhead
  \textbf{L} & 840–850 & 5824b–5871a & Thomas 4073 \\
  \textbf{P} & 840–850 & 958–1006a & Berlin DHM R 56/2537 \\
  \textbf{V} & 800–850 & 1279–1358a & Palatini Latini 1447 \\
  \textbf{S} & 850 & \begin{tabular}{@{}c@{}}351b–360a, 368b–384, 393–400a, \\ 492–582a, 675–683a, 693–706, \\ 716b–722a\end{tabular} & BSB Cgm 8840 \\
  \textbf{M} & 850–875 & TODO & BSB Cgm 25 \\
  \textbf{C} & 950–1000 & 1–5970 & Cotton Caligula A VII \\
	\hline
\end{longtabu}\end{small}

The two main mss. are \textbf{M} and \textbf{C}.  Fragments \textbf{L} and \textbf{P} are identical in terms of handwriting and page layout and appear to have originally belonged to the same codex.  \textbf{V} also attests \SaxonGenesis, which suggests a close relation between that text and \Heliand.

\sectionline

The following is very much a work in progress.  The radically normalized orthography has been implemented, as has the marking of alliteration, but the original text has not been critically edited, nor is there any English translation.

\sectionline

\section{Heliand}

\bvg\bva%
\alst{M}anega wáron, \hld\ þe sia iro \alst{m}ód ge·spón, &
þat sia bi·gunnun word godes, &
\alst{r}ękkjan þat gi·\alst{r}úni, \hld\ þat þie \alst{r}íkjo Krist &
undar \alst{m}an-kunnja \hld\ \alst{m}áriða gi·frumida &
mid \alst{w}ordun ęndi mid \alst{w}erkun. \hld\ Þat wolda þȯ \alst{w}ísara filo &
\alst{l}iudo barno \alst{l}ovon, \hld\ \alst{l}êra Kristes, &
\alst{h}êlag word godas, \hld\ ęndi mid iro \alst{h}andon skrívan &
\alst{b}erẹht-líko an \alst{b}uok, \hld\ hwó sia is gi·\alst{b}od-skip skoldin &
\alst{f}rummjan, \alst{f}iriho barn. \hld\ Þan wárun þoh sia \alst{f}iori te þiu &
under þera \alst{m}ęnigo, \hld\ þia habdon \alst{m}aht godes, &
\alst{h}elpa fan \alst{h}imila, \hld\ \alst{h}êlagna gêst, &
\alst{k}raft fan \alst{K}riste; \hld\ sia wurðun gi·\alst{k}orana te þio, &
þat sie þan \alst{É}wangelium \hld\ \alst{ê}nan skoldun &
an \alst{b}uok skrívan \hld\ endo só manag gi·\alst{b}od godes, &
\alst{h}êlag \alst{h}imilisk word: \hld\ sia ne muosta \alst{h}ęliðo þan mêr, &
\alst{f}iriho barno \alst{f}rummjan, \hld\ newan þat sia \alst{f}iori te þio &
þuru \alst{k}raft godas \hld\ ge·\alst{k}orana wurðun, &
\alst{M}atheus ęndi \alst{M}arkus, \hld\ —só wárun þia \alst{m}an hêtana— &
Lukas ęndi \alst{J}ohannes; \hld\ sia wárun \alst{g}ode lieva, &
\alst{w}irðiga ti þem gi·\alst{w}irkje. \hld\ Habda im \alst{w}aldand god, &
þem \alst{h}ęliðon an iro \alst{h}ertan \hld\ \alst{h}êlagna gêst &
\alst{f}asto bi·\alst{f}olhan \hld\ ęndi \alst{f}erạhtan hugi, &
só manag \alst{w}ís-lík \alst{w}ord \hld\ ęndi gi·\alst{w}it mikil, &
þat sea skoldin a·\alst{h}ębbjan \hld\ \alst{h}êlagaro stemnun &
\alst{g}od-spell þat \alst{g}uoda, \hld\ þat ni havit ênigan gi·\alst{g}adon hwęrgin, &
þiu \alst{w}ord an þesaro \alst{w}er-oldi, \hld\ þat io \alst{w}aldand mêr, &
\alst{d}rohtin \alst{d}iurje \hld\ efþo \alst{d}ervi þing, &
\alst{f}irin-werk \alst{f}ęllje \hld\ efþo \alst{f}íundo níð, &
\alst{st}ríd wiðer·\alst{st}ande—, \hld\ hwand hie habda \alst{st}arkan hugi, &
\alst{m}ildjan ęndi guodan, \hld\ þie þe \alst{m}êster was, &
\alst{a}ðal-\alst{o}rd-frumo \hld\ \alst{a}lo-mahtig. &
Þat skoldun sea \alst{f}iori \hld\ þuȯ \alst{f}ingron skrívan, &
\alst{s}ęttjan ęndi \alst{s}ingan \hld\ ęndi \alst{s}ęggjan forð, &
þat sea fan \alst{K}ristes \hld\ \alst{k}rafte þem mikilon &
gi·\alst{s}áhun ęndi gi·hôrdun, \hld\ þes hie \alst{s}elvo gi·sprak, &
gi·\alst{w}ísda ęndi gi·\alst{w}arạhta, \hld\ \alst{w}undạr-líkas filo, &
só \alst{m}anag mid \alst{m}annon \hld\ \alst{m}ahtig drohtin, &
all so hie it fan þem \alst{a}n-ginne \hld\ þuru is \alst{ê}nes kraht, &%NOTE: kraht checked.
\alst{w}aldand gi·sprak, \hld\ þuȯ hie êrist þesa \alst{w}er-old gi·skuop &
ęndi þuȯ \alst{a}ll bi·fieng \hld\ mid \alst{ê}nu wordo, &
\alst{h}imil ęndi erða \hld\ ęndi al þat sea bi·\alst{h}lidan êgun &
gi·\alst{w}arạhtes ęndi gi·\alst{w}ahsanes: \hld\ þat warð þuȯ all mid \alst{w}ordon godas &
\alst{f}asto bi·\alst{f}angan, \hld\ ęndi gi·\alst{f}rumid after þiu, &
hwi-lik þan \alst{l}iud-skępi \hld\ \alst{l}andes skoldi &
\alst{w}ídost gi·\alst{w}aldan, \hld\ efþo hwar þiu \alst{w}er-old-aldar &
\alst{ę}ndon skoldin. \hld\ \alst{Ê}n was iro þuȯ noh þan &
\alst{f}iriho barnun bi·\alst{f}oran, \hld\ ęndi þiu \alst{f}ïvi wárun a·gangan: &
skolda þuȯ þat \alst{s}ehsta \hld\ \alst{s}álig-líko &
kuman þuru \alst{k}raft godes \hld\ ęndi \alst{K}ristas gi·burd, &
\alst{h}êlandero bęstan, \hld\ \alst{h}êlagas gêstes, &
an þesan \alst{m}iddil-gard \hld\ \alst{m}anagon te helpun, &
\alst{f}irjo barnon ti \alst{f}rumon \hld\ wið \alst{f}íundo níð, &
wið \alst{d}ęrnero \alst{d}walm. \hld\ Þan habda þuȯ \alst{d}rohtin god &
\alst{R}ómano-liudjon far·liwan \hld\ \alst{r}íkjo mêsta, &
\alst{h}abda þem \alst{h}ęri-skipje \hld\ \alst{h}erta gi·stęrkid, &
þat sia habdon bi·\alst{þ}wungana \hld\ \alst{þ}iedo gi·hwi-lika, &
habdun fan \alst{R}úmu-burg \hld\ \alst{r}íki gi·wunnan &
\alst{h}elm-gi·trôstjon, \hld\ sáton iro \alst{h}ęri-togon &
an \alst{l}ando gi·hwem, \hld\ habdun \alst{l}iudjo gi·wald, &
\alst{a}llon \alst{ę}li-þeodon. \hld\ \alst{E}rodes was &
an \alst{J}erusalem \hld\ over þat \alst{J}udeono folk &
gi·\alst{k}oran te \alst{k}uninge, \hld\ só ina þie \alst{k}êser þarod, &
fon \alst{R}úmu-burg \hld\ \alst{r}íki þiodan &
\alst{s}atta undar þat gi·\alst{s}ïði. \hld\ Hie ni was þoh mid \alst{s}ibbjon bi·lang &
\alst{a}varon \alst{I}sraheles, \hld\ \alst{ę}ðili-gi·burdi, &
\alst{k}uman fon iro \alst{k}nuosle, \hld\ newan þat hie þuru þes \alst{k}êsures þank &
fan \alst{R}úmu-burg \hld\ \alst{r}íki habda, &
þat im wárun só gi·\alst{h}ôriga \hld\ \alst{h}ildi-skalkos, &
\alst{a}varon \alst{I}sraheles \hld\ \alst{ę}lljan-ruova: &
swíðo un·\alst{w}anda \alst{w}ini, \hld\ þan lang hie gi·\alst{w}ald êhta, &
Erodes þes \alst{r}íkjas \hld\ ęndi \alst{r}ád-burdjon held &
\alst{J}udeo liudi. \hld\ Þan was þár ên gi·\alst{g}amalod mann, &
þat was \alst{f}ruod gomo, \hld\ habda \alst{f}erẹhtan hugi, &
was fan þem \alst{l}iudjon \hld\ \alst{L}ewias kunnes, &
\alst{J}akobas sunjas, \hld\ \alst{g}uodero þiedo: &
\alst{Z}akharias was hie hêtan. \hld\ Þat was só \alst{s}álig man, &
hwand hie simblon \alst{g}erno \hld\ \alst{g}ode þeonoda, &
\alst{w}arạhta after is \alst{w}illjon; \hld\ deda is \alst{w}íf só self &
—was iru gi·\alst{a}ldrod \alst{i}dis: \hld\ ni muosta im \alst{ę}rvi-ward &
an iro \alst{j}uguð-hêdi \hld\ \alst{g}iviðig werðan— &
\alst{l}ibdun im far·úter \alst{l}aster, \hld\ warụhtun \alst{l}of goda, &
wárun só gi·\alst{h}ôriga \hld\ \alst{h}evan-kuninge, &
\alst{d}iuridon u̇san \alst{d}rohtin: \hld\ ni weldun \alst{d}ęrvjas wiht &
under \alst{m}an-kunnje, \hld\ \alst{m}ênes gi·frummjan, &
ne *\alst{s}aka ne \alst{s}undja; \hld\ was im þoh an \alst{s}orgun hugi, &
þat sie \alst{ę}rvi-ward \hld\ \alst{ê}gan ni móstun, &
ak wárun im \alst{b}arno-lôs. \hld\ Þan skolda hé gi·\alst{b}od godes &
þár an \alst{J}erusalem, \hld\ só oft só is gi·\alst{g}ęngi gi·stód, &
þat ina \alst{t}orht-líko \hld\ \alst{t}ídi gi·manodun, &
só skolda hé at þem \alst{w}íha \hld\ \alst{w}aldandes geld &
\alst{h}êlag bi·\alst{h}wervan, \hld\ \alst{h}evan-kuninges, &
\alst{g}odes \alst{j}ungar-skępi: \hld\ \alst{g}ern was hé swíðo, &
þat hé it þurh \alst{f}erhtan hugi \hld\ \alst{f}rummjan mósti.\eva

\bvb TODO.\evb\evg

\bvg\bva[2][94]%
Þȯ warð þiu \alst{t}íd kuman, \hld\ —þat þár gi·\alst{t}ald habdun &
\alst{w}ísa man mid \alst{w}ordun,— \hld\ þat skolda þana \alst{w}íh godes &
\alst{Z}akharias bi·\alst{s}ehan. \hld\ Þȯ warð þár gi·\alst{s}amnod filu &
þár te \alst{J}erusalem \hld\ \alst{J}udeo liudi, &
\alst{w}erodes te þem \alst{w}íha, \hld\ þár sie \alst{w}aldand god &
swíðo \alst{þ}eo-líko \hld\ \alst{þ}iggjan skoldun, &
\alst{h}êrron is \alst{h}uldi, \hld\ þat sie \alst{h}evan-kuning &
\alst{l}êðes a·\alst{l}éti. \hld\ Þea \alst{l}iudi stódun &
umbi þat \alst{h}êlaga \alst{h}ús, \hld\ ęndi géng im þe gi·\alst{h}êrodo man &
an þana \alst{w}íh innan. \hld\ Þat \alst{w}erod ȯðar bêd &
umbi þana \alst{a}lạh \alst{ú}tan, \hld\ \alst{E}breo liudi, &
hwan êr þe \alst{f}ródo man \hld\ gi·\alst{f}rumid habdi &
\alst{w}aldandes \alst{w}illjon. \hld\ Só hé þȯ þana \alst{w}í-rôk dróg, &
\alst{a}ld aftar þem \alst{a}lạha, \hld\ ęndi umbi þana \alst{a}ltari géng &
mid is \alst{r}ôk-fatun \hld\ \alst{r}íkjun þionon, &
—\alst{f}ręmida \alst{f}erht-líko \hld\ \alst{f}râon sínes, &
\alst{g}odes \alst{j}ungar-skępi \hld\ \alst{g}erno swíðo &
mid \alst{h}luttru \alst{h}ugi, \hld\ *só man \alst{h}êrren skal &
\alst{g}erno ful-\alst{g}angan—, \hld\ \alst{g}rurjos kwámun im, &
\alst{ę}gison an þem \alst{a}lạhe: \hld\ hie gi·sah þár aftar þiu ênna \alst{ę}ngil godes &
an þem \alst{w}íhe innan, \hld\ hie sprak im mid is \alst{w}ordun tuo, &
hiet þat \alst{f}ruod gumo \hld\ \alst{f}orọht ni wári, &
hiet þat hie im ni an·\alst{d}riede: \hld\ þína \alst{d}ádi sind“, kwaþ-hie*, &%TODO: and-riede?
„\alst{w}aldanda \alst{w}erðe \hld\ ęndi þín \alst{w}ord só self, &
þín \alst{þ}ionost is im an \alst{þ}anke, \hld\ þat þú su·lika gi·\alst{þ}ȧht haves &
an is \alst{ê}nes kraft. \hld\ Ik is \alst{ę}ngil bium, &
\alst{G}abriel bium ik hêtan, \hld\ þe gio for \alst{g}oda standu, &
\alst{a}nd-ward for þem \alst{a}lo-waldon, \hld\ ne sí þat hé me an is \alst{â}rundi hwarod &
\alst{s}ęndjan willja. \hld\ Nu hiet hé me an þesan \alst{s}ïð faran, &
hiet þat ik þi þoh gi·\alst{k}u̇ðdi, \hld\ þat þi \alst{k}ind gi·boran, &
fon þínera \alst{a}lderu \alst{i}dis \hld\ \alst{ô}dan skoldi &
\alst{w}erðan an þesero \alst{w}er-oldi, \hld\ \alst{w}ordun spáhi. &
Þat ni skal an is \alst{l}iva gio \hld\ \alst{l}íðes an·bítan, &
\alst{w}ínes an is \alst{w}er-oldi: \hld\ só haved im \alst{w}urd-gi·skapu, &
\alst{m}etod gi·\alst{m}arkod \hld\ ęndi \alst{m}aht godes. &
Hét þat ik þi þoh \alst{s}agdi, \hld\ þat it skoldi gi·\alst{s}ïð wesan &
\alst{h}evan-kuninges, \hld\ hét þat git it \alst{h}eldin wel, &
\alst{t}uhin þurh \alst{t}rewa, \hld\ kwað þat hé im \alst{t}íras só filu &
an \alst{g}odes ríkja \hld\ for·\alst{g}evan weldi. &
Hé kwað þat þe \alst{g}ódo \alst{g}umo \hld\ \alst{J}ohannes te namon &
\alst{h}ębbjan skoldi, \hld\ gi·bôd þat git it \alst{h}étin só, &
þat \alst{k}ind, þan it \alst{k}wámi, \hld\ kwað þat it \alst{K}ristes gi·sïð &
an þesaro \alst{w}ídun \alst{w}er-old \hld\ \alst{w}erðan skoldi, &
is \alst{s}elves \alst{s}unjes, \hld\ ęndi kwað þat sie \alst{s}liumo herod &
an is \alst{b}od-skępi \hld\ \alst{b}êðe kwámin.“ &
\alst{Z}akharias þȯ gi·mahạlda \hld\ ęndi wið \alst{s}elvan sprak &
\alst{d}rohtines ęngil, \hld\ ęndi im þero \alst{d}ádjo bi·gan, &
\alst{w}undron þero \alst{w}ordo: \hld\ „hwó mag þat gi·\alst{w}erðan só“, kwað hé, &
„\alst{a}ftar an \alst{a}ldre? \hld\ it is unk \alst{a}l te lat &
só te gi·\alst{w}innanne, \hld\ só þú mid þínun \alst{w}ordun gi·sprikis. &
Hwanda wit habdun \alst{a}ldres \hld\ êr \alst{e}fno twên-tig &
\alst{w}intro an unkro \alst{w}er-oldi, \hld\ êr þan kwámi þit \alst{w}íf te mí; &
þan wárun wit nu at·\alst{s}amna \hld\ ant·\alst{s}ivunta wintro &
gi·\alst{b}ęnkjon ęndi gi·\alst{b}ęddjon, \hld\ sïðor ik sie mí te \alst{b}rúdi ge·kôs. &
Só wit þes an unkro \alst{j}uguði \hld\ gi·\alst{g}irnan ni mohtun, &
þat wit \alst{ę}rvi-ward \hld\ \alst{ê}gan móstin, &
\alst{f}ódjan an unkun \alst{f}lęttja, \hld\ nu wit sus gi·\alst{f}ródod sint &
—havad unk \alst{ę}ldi bi·noman \hld\ \alst{ę}lljan-dádi, &
þat wit sint an unkro \alst{s}iuni gi·\alst{s}lekit \hld\ ęndi an unkun \alst{s}ídun lat; &
\alst{f}lêsk is unk ant·\alst{f}allan, \hld\ \alst{f}el un·skôni, &
is unka \alst{l}ud gi·\alst{l}iðen, \hld\ \alst{l}ík gi·drusnod, &
sind \alst{u}nka \alst{a}nd-bári \hld\ \alst{ȯ}ðar-líkaron, &
\alst{m}ód ęndi \alst{m}ęgin-kraft—, \hld\ só wit giu só \alst{m}anagan dag &
\alst{w}árun an þesero \alst{w}er-oldi, \hld\ só mí þes \alst{w}undạr þunkit, &
hwó it só gi·\alst{w}erðan mugi, \hld\ só þú mid þínun \alst{w}ordun gi·sprikis.\eva

\bvb TODO.\evb\evg

\bvg\bva[3][159]%
Þȯ warð þat \alst{h}evan-kuninges bodon \hld\ \alst{h}arm an is móde, &
þat hé is gi·\alst{w}erkes \hld\ só \alst{w}undron skolda &
ęndi þat ni welda gi·\alst{h}uggjan, \hld\ þat ina mahta \alst{h}êlag god &
só \alst{a}la-jungan, \hld\ só hé fon \alst{ê}rist was, &
\alst{s}elvo gi·wirkjan, \hld\ of hé \alst{s}ó weldi. &
Skęrida im þȯ te \alst{w}ítja, \hld\ þat hé ni mahte ênig \alst{w}ord sprekan, &
gi·\alst{m}ahljen mid is \alst{m}u̇ðu, \hld\ „êr þan þi \alst{m}agu wirðid, &
fon þínero \alst{a}ldero \alst{i}dis \hld\ \alst{e}rl a·fódit, &
\alst{k}ind-jung gi·boran \hld\ \alst{k}unnjes gódes, &
\alst{w}ánum te þesero \alst{w}er-oldi. \hld\ Þan skalt þú eft \alst{w}ord sprekan, &
hębbjan þínaro \alst{st}emna gi·wald; \hld\ ni þarft þú \alst{st}um wesan &
\alst{l}ęngron hwíla.“ \hld\ Þȯ warð it sán gi·\alst{l}êstid só, &
gi·\alst{w}orðan te \alst{w}áron, \hld\ só þár an þem \alst{w}íha gi·sprak &
\alst{ę}ngil þes \alst{a}lo-waldon: \hld\ warð \alst{a}ld gumo &
\alst{sp}ráka bi·lôsit, \hld\ þoh hé \alst{sp}áhan hugi &
\alst{b}ári an is \alst{b}reostun. \hld\ \alst{B}idun allan dag &
þat \alst{w}erod for þem \alst{w}íha \hld\ ęndi \alst{w}undrodun alla, &
bi·hwí hé þár só \alst{l}ango, \hld\ \alst{l}of-sálig man, &
swíðo \alst{f}ród gumo \hld\ \alst{f}râon sínun &
\alst{þ}ionon \alst{þ}orfti, \hld\ só þár êr ênig \alst{þ}egno ni deda, &
þan sie þár at þem \alst{w}íha \hld\ \alst{w}aldandes geld &
\alst{f}olmon \alst{f}rumidun. \hld\ Þȯ kwam \alst{f}ród gumo &
\alst{ú}t fon þem \alst{a}lạha. \hld\ \alst{E}rlos þrungun &
\alst{n}áhor mikilu: \hld\ was im \alst{n}iud mikil, &
hwat hé im \alst{s}ȯð-líkes \hld\ \alst{s}ęggjan weldi, &
\alst{w}ísjan te \alst{w}áron. \hld\ hé ni mohta þȯ ênig \alst{w}ord sprekan, &
gi·\alst{s}ęggjan þem gi·\alst{s}ïðja, \hld\ b·útan þat hé mid is \alst{s}wíðron hand &
\alst{w}ísda þem \alst{w}eroda, \hld\ þat sie u̇ses \alst{w}aldandes &
\alst{l}êra \alst{l}êstin. \hld\ Þea \alst{l}iudi for·stódun, &
þat hé þár habda \alst{g}egnungo \hld\ \alst{g}od-kundes hwat &
for·\alst{s}ehen \alst{s}elvo, \hld\ þoh hé is ni mahti gi·\alst{s}ęggjan wiht, &
gi·\alst{w}ísjan te \alst{w}áron. \hld\ Þȯ habda hé u̇ses \alst{w}aldandes &
\alst{g}eld gi·lêstid, \hld\ al só is gi·\alst{g}ęngi was &
gi·\alst{m}arkod mid \alst{m}annun. \hld\ Þȯ warð sán aftar þiu \alst{m}aht godes, &
gi·\alst{k}u̇ðid is \alst{k}raft mikil: \hld\ warð þiu \alst{k}wán ôkan, &
\alst{i}dis an ira \alst{ę}ldju: \hld\ skolda im \alst{ę}rvi-ward, &
swíðo \alst{g}od-kund \alst{g}umo \hld\ \alst{g}iviðig werðan, &
\alst{b}arn an \alst{b}urgun. \hld\ \alst{B}êd aftar þiu &
þat \alst{w}íf \alst{w}urdi-gi·skapu. \hld\ Skrêd þe \alst{w}intạr forð, &
\alst{g}éng þes \alst{g}ę́res gi·tal. \hld\ \alst{J}ohannes kwam &
an \alst{l}iudjo \alst{l}ioht: \hld\ \alst{l}ík was im skôni, &
was im \alst{f}el \alst{f}agạr, \hld\ \alst{f}ahs ęndi naglos, &
\alst{w}angun wárun im \alst{w}litige. \hld\ Þȯ fórun þár \alst{w}íse man, &
\alst{s}nelle te·\alst{s}amne, \hld\ þea \alst{s}wásostun mêst, &
\alst{w}undrodun þes \alst{w}erkes, \hld\ bi·hwí it gio mahti gi·\alst{w}erðan só, &
þat undar só \alst{a}ldun twêm \hld\ \alst{ô}dan wurði &
\alst{b}arn an gi·\alst{b}urdjon, \hld\ ni wári þat it gi·\alst{b}od godes &
\alst{s}elves wári: \hld\ af·\alst{s}uovun sie garo, &
þat it elkor só \alst{w}án-lík \hld\ \alst{w}erðan ni mahti. &
Þȯ sprak þár ên gi·\alst{f}ródot man, \hld\ þe só \alst{f}ilo konsta &
\alst{w}ísaro \alst{w}ordo, \hld\ habde gi·\alst{w}it mikil, &
frágode \alst{n}iud-líko, \hld\ hwat is \alst{n}amo skoldi &
\alst{w}esan an þesaro \alst{w}er-oldi: \hld\ „mi þunkid an is \alst{w}ísu gi·lík &
iak an is gi·\alst{b}árja, \hld\ þat hé sí \alst{b}ętara þan wi, &
só ik wániu, þat ina u̇s \alst{g}egnungo \hld\ \alst{g}od fon himila &
\alst{s}elvo \alst{s}ęndi“. \hld\ Þȯ sprak \alst{s}án aftar &
þiu \alst{m}ódar þes kindes, \hld\ þiu þana \alst{m}agu habda, &
þat \alst{b}arn an ire \alst{b}arme: \hld\ „hér kwam gi·\alst{b}od godes“, kwað siu, &
„\alst{f}ernun gę́re, \hld\ \alst{f}urmon wordu &
gi·bôd, þat hé \alst{J}ohannes \hld\ bi \alst{g}odes lêrun &
\alst{h}êtan skoldi. \hld\ Þat ik an mínumu \alst{h}ugi ni gi·dar &
\alst{w}ęndjan mid \alst{w}ihti, \hld\ of ik is gi·\alst{w}aldan mót“. &
Þȯ sprak ên \alst{g}êl-hert man, \hld\ þe ira \alst{g}aduling was: &
„ne hét êr \alst{io}·wiht só“, \hld\ kwað hé, „\alst{a}ðal-boranes &
u̇ses \alst{k}unnjes efþo \alst{k}nósles; \hld\ wita \alst{k}iasan im ȯðrana &
\alst{n}iud-samna \alst{n}amon: \hld\ hé \alst{n}iate of hé móti“. &
Þȯ sprak eft þe \alst{f}ródo man, \hld\ þe þár konsta \alst{f}ilo mahljan: &
„ni givu ik þat te \alst{r}áde“, \hld\ kwað hé, „\alst{r}inko neg·ênun, &
þat hé \alst{w}ord godes \hld\ \alst{w}ęndjan bi·ginna; &
ak wita is þana \alst{f}ader \alst{f}rágon, \hld\ þe þár só gi·\alst{f}ródod sitit, &
\alst{w}ís an is \alst{w}ín-sęli: \hld\ þoh hé ni mugi ênig \alst{w}ord sprekan, &
þoh mag hé bi \alst{b}ók-stavon \hld\ \alst{b}réf ge·wirkjan, &
\alst{n}amon gi·skrívan“. \hld\ Þȯ hé \alst{n}áhor géng, &
lęgda im êna \alst{b}ók an \alst{b}arm \hld\ ęndi \alst{b}ad gerno &
\alst{w}rítan \alst{w}ís-líko \hld\ \alst{w}ord-gi·merkjun, &
hwat sie þat \alst{h}êlaga barn \hld\ \alst{h}êtan skoldin. &
Þȯ nam hé þia bók an \alst{h}and \hld\ ęndi an is \alst{h}ugi þȧhte &
swíðo \alst{g}erno te \alst{g}ode: \hld\ \alst{J}ohannes namon &
\alst{w}ís-líko gi·\alst{w}rêt \hld\ ęndi ôk aftar mid is \alst{w}ordu gi·sprak &
swíðo \alst{sp}áh-líko: \hld\ habda im eft is \alst{sp}ráka gi·wald, &
gi·\alst{w}ittjas ęndi \alst{w}ísun. \hld\ Þat \alst{w}íti was þȯ a·gangan, &
\alst{h}ard \alst{h}arm-skare, \hld\ þe im \alst{h}êlag god &
\alst{m}ahtig \alst{m}akode, \hld\ þat hé an is \alst{m}ód-sevon &
\alst{g}odes ni for·\alst{g}áti, \hld\ þan hé im eft sęndi is \alst{j}ungron tó.\eva

\bvb TODO.\evb\evg

\bvg\bva[4][243]%
Þȯ ni was \alst{l}ang aftar þiu, \hld\ ne it al só gi·\alst{l}êstid warð, &
só hé \alst{m}an-kunnja \hld\ \alst{m}anaga hwíla, &
\alst{g}od alo-mahtig \hld\ for·\alst{g}even habda, &
þat hé is \alst{h}imilisk barn \hld\ \alst{h}erod te wer-oldi, &
sí \alst{s}elves \alst{s}unu \hld\ \alst{s}ęndjan weldi, &
te þiu þat hé hér a·\alst{l}ôsdi \hld\ al \alst{l}iud-stamna, &
\alst{w}erod fon \alst{w}ítja. \hld\ Þȯ warð is \alst{w}is-bodo &
an \alst{G}alilea-land, \hld\ \alst{G}abriel kuman, &
\alst{ę}ngil þes \alst{a}lo-waldon, \hld\ þár hé êne \alst{i}dis wisse, &
\alst{m}uni-líka \alst{m}agað: \hld\ \alst{M}aría was siu hêten, &
was iru \alst{þ}iorna gi·\alst{þ}igan. \hld\ Sea ên \alst{þ}egạn habda, &
\alst{J}oseph gi·mahlit, \hld\ \alst{g}ódes kunnjes man, &
þea \alst{D}awides \alst{d}ohter: \hld\ þat was só \alst{d}iur-lík wíf, &
\alst{i}dis \alst{a}nt-hêti. \hld\ Þár sie þe \alst{ę}ngil godes &
an \alst{N}azareth-burg \hld\ bi \alst{n}amon selvo &
\alst{g}rótte \alst{g}ęgin-warde \hld\ ęndi sie fon \alst{g}ode kwędda: &
„\alst{H}êl wis þú, Maria“, \hld\ kwað hé, „þú bist þínun \alst{h}êrron liof, &
\alst{w}aldande \alst{w}irðig, \hld\ hwand þú gi·\alst{w}it haves, &
\alst{i}dis \alst{ę}nstjo fol. \hld\ Þu skalt for \alst{a}llun wesan &
\alst{w}ívun gi·\alst{w}íhit. \hld\ Ne have þú \alst{w}êkan hugi, &
ne \alst{f}orhti þú þínun \alst{f}erhe: \hld\ ne kwam ik þi te ênigun \alst{f}rêson herod, &
ne \alst{d}ragu ik ênig \alst{d}rugi-þing. \hld\ Þu skalt u̇ses \alst{d}rohtines wesan &
\alst{m}ódar mid \alst{m}annun \hld\ ęndi skalt þana \alst{m}agu fódjan, &
þes \alst{h}ôhon \edtext{\alst{h}evan-kuninges}{\Afootnote{so \textbf{M}; \emph{himilcuninges} \textbf{C}}} suno. \hld\ Þe skal \alst{h}êljand te namon &
\alst{ê}gan mid \alst{ę}ldjun. \hld\ Neo \alst{ę}ndi ni kumid, &
þes \alst{w}ídon ríkjas gi·\alst{w}and, \hld\ þe hé gi·\alst{w}aldan skal, &
\alst{m}ári þeodan.“ \hld\ Þȯ sprak im eft þiu \alst{m}agað an·gęgin, &
wið þana \alst{ę}ngil godes \hld\ \alst{i}diso skônjost, &
allaro \alst{w}ívo \alst{w}litigost: \hld\ „hwó mag þat gi·\alst{w}erðen só“, kwað siu, &
„þat ik \alst{m}agu fódje? \hld\ Ne ik gio \alst{m}annes ni warð &
\alst{w}ís an mínera \alst{w}er-oldi.“ \hld\ Þȯ habde eft is \alst{w}ord garu &
\alst{ę}ngil þes \alst{a}lo-waldon \hld\ þero \alst{i}disiu te·gęgnes: &
„an þí skal \alst{h}êlag gêst \hld\ fon \alst{h}evan-wange &
\alst{k}uman þurh \alst{k}raft godes. \hld\ Þanan skal þi \alst{k}ind ôdan &
\alst{w}erðan an þesaro \alst{w}er-oldi; \hld\ \alst{w}aldandes kraft &
skal þi fon þem \alst{h}ôhoston \hld\ \alst{h}evan-kuninge &
\alst{sk}adowan mid \alst{sk}imon. \hld\ Ni warð \alst{sk}ônjera gi·burd, &
ne só \alst{m}ári mid \alst{m}annun, \hld\ hwand siu kumid þurh \alst{m}aht godes &
an þese \alst{w}ídon \alst{w}er-old.“ \hld\ Þȯ warð eft þes \alst{w}íves hugi &
\alst{a}ftar þem \alst{â}rundje \hld\ \alst{a}l gi·hworven &
an \alst{g}odes willjon. \hld\ „Þan ik hér \alst{g}aru standu“, kwað siu, &
„te su·likun \alst{a}mbaht-skępi, \hld\ só hé mi \alst{ê}gan wili. &
Þiu bium ik \alst{þ}eot-godes. \hld\ Nu ik þeses \alst{þ}inges gi·trúon; &
\alst{w}erðe mi aftar þínun \alst{w}ordun, \hld\ al só is \alst{w}illjo sí, &
\alst{h}êrron mínes; \hld\ nis mí \alst{h}ugi twífli, &
ne \alst{w}ord ne \alst{w}ísa.“ \hld\ Só gi·fragn ik, þat þat \alst{w}íf ant·féng &
þat \alst{g}odes ârundi \hld\ \alst{g}erno swíðo &
mid \alst{l}eohtu hugi \hld\ ęndi mid gi·\alst{l}ôvon gódun &
ęndi mid \alst{h}luttrun trewun; \hld\ warð þe \alst{h}êlago gêst, &
þat \alst{b}arn an ira \alst{b}ósma; \hld\ ęndi siu ira \alst{b}reostun for·stód &
iak an ire \alst{s}evon \alst{s}elvo, \hld\ \alst{s}agda þem siu welda, &
þat sie habde gi·\alst{ô}kana \hld\ þes \alst{a}lo-waldon kraft &
\alst{h}êlag fon \alst{h}imile. \hld\ Þȯ warð \alst{h}ugi Josepes, &
is \alst{m}ód gi·worrid, \hld\ þe im êr þea \alst{m}agað habda, &
þea \alst{i}dis \alst{a}nt-hêttja, \hld\ \alst{a}ðal-knósles wíf &
gi·\alst{b}oht im te \alst{b}rúdju. \hld\ hé af·sóf þat siu habda \alst{b}arn undar iru: &
ni \alst{w}ánda þes mid \alst{w}ihti, \hld\ þat iru þat \alst{w}íf habdi &
gi·\alst{w}ardod só \alst{w}aro-líko: \hld\ ni wisse \alst{w}aldandes þȯ noh &
\alst{b}líði gi·\alst{b}od-skępi. \hld\ Ni welda sia imo te \alst{b}rúdi þȯ, &
\alst{h}alon imo te \alst{h}íwon, \hld\ ak bi·gan im þȯ an \alst{h}ugi þęnkjan, &
hwó hé sie só for·\alst{l}éti, \hld\ só iru þár nu wurði \alst{l}êdes wiht, &
\alst{ô}dan \alst{a}rvides. \hld\ Ni welda sie \alst{a}ftar þiu &
\alst{m}eldon for \alst{m}ęnigi: \hld\ antd-réd þat sie \alst{m}anno barn &
\alst{l}ívu bi·námin. \hld\ Só was þan þero \alst{l}iudjo þau &
þurh þen \alst{a}ldon \alst{ê}w, \hld\ \alst{E}breo folkes, &
só hwi-lik só þár an \alst{u}n-reht \hld\ \alst{i}dis gi·híwida, &
þat siu simbla þana \alst{b}ed-skępi \hld\ \alst{b}uggjan skolda, &
\alst{f}rí mid ira \alst{f}erhu: \hld\ ni was gio þiu \alst{f}êmja só gód, &
þat siu mid þem \alst{l}iudun \alst{l}ęng \hld\ \alst{l}ibbjen mósti, &
\alst{w}esan undar þem \alst{w}eroda. \hld\ Bi·gan im þe \alst{w}íso mann, &
swíðo \alst{g}ód \alst{g}umo, \hld\ \alst{J}oseph an is móda &
\alst{þ}ęnkjan þero \alst{þ}ingo, \hld\ hwó hé þea \alst{þ}iornun þȯ &
\alst{l}istjun for·\alst{l}éti. \hld\ Þȯ ni was \alst{l}ang te þiu, &
þat im þár an \alst{d}rôma \hld\ kwam \alst{d}rohtines ęngil, &
\alst{h}evan-kuninges bodo, \hld\ ęndi hét sie ina \alst{h}aldan wel, &
\alst{m}innjon sie an is \alst{m}óde: \hld\ „Ni wis þú“, kwað hé, „\alst{M}ariun wrêð, &
\alst{þ}iornun \alst{þ}ínaro; \hld\ siu is gi·\alst{þ}ungan wíf; &
ne for·\alst{h}ugi þú sie te \alst{h}ardo; \hld\ þú skalt sie \alst{h}aldan wel, &
\alst{w}ardon ira an þesaro \alst{w}er-oldi. \hld\ Lêsti þú inka \alst{w}ini-trewa &
\alst{f}orð só þú dádi, \hld\ ęndi hald inkan \alst{f}riund-skępi wel! &
Ne lát þú sie þi þiu \alst{l}êðaron, \hld\ þoh siu undar ira \alst{l}iðon êgi, &
\alst{b}arn an ira \alst{b}ósma. \hld\ It kumid þurh gi·\alst{b}od godes, &
\alst{h}êlages gêstes \hld\ fon \alst{h}evan-wanga: &
þat is \alst{J}ésu Krist, \hld\ \alst{g}odes êgan barn, &
\alst{w}aldandes sunu. \hld\ Þu skalt sie \alst{w}el haldan, &
\alst{h}êlag-líko. \hld\ Ne lát þú þi þínan \alst{h}ugi twífljen, &
\alst{m}ęrrjan þína \alst{m}ód-gi·þȧht.“ \hld\ Þȯ warð eft þes \alst{m}annes hugi &
gi·\alst{w}ęndid aftar þem \alst{w}ordun, \hld\ þat hé im te þem \alst{w}íva ge·nam, &
te þera \alst{m}agað \alst{m}innja: \hld\ ant·kęnda \alst{m}aht godes, &
\alst{w}aldandes gi·bod; \hld\ was im \alst{w}illjo mikil, &
þat hé sia só \alst{h}êlag-líko \hld\ \alst{h}aldan mósti: &
bi·\alst{s}orgoda sie an is gi·\alst{s}ïðja, \hld\ ęndi siu só \alst{s}úvro dróg &%TODO: dróg or drôg.
al te \alst{h}uldi godes \hld\ \alst{h}êlagna gêst, &
\alst{g}ód-líkan \alst{g}umon, \hld\ ant-þat sie \edtrans{\alst{g}odes gi·skapu}{God’s shapes}{\Bfootnote{TODO: some note about this.}} &
\alst{m}ahtig gi·\alst{m}anodun, \hld\ þat siu ina an \alst{m}anno lioht, &
allaro \alst{b}arno \alst{b}ętst, \hld\ \alst{b}rengjan skolda.\eva

\bvb TODO.\evb\evg

\bvg\bva[5][339]%
Þȯ warð fon \alst{R}úmu-burg \hld\ \alst{r}íkes mannes &
ovar \alst{a}lla þesa \alst{i}rmin-þiod \hld\ \alst{O}ktawiánas &
\alst{b}an ęndi \alst{b}od-skępi \hld\ ovar þea is \alst{b}rêdon gi·wald &
\alst{k}uman fon þem \alst{k}êsure \hld\ \alst{k}uningo gi·hwi-likun, &
\alst{h}êm-sittjandjun, \hld\ só wído só is \alst{h}ęri-togon &
ovar al þat \alst{l}and-skępi \hld\ \alst{l}iudjo gi·weldun. &
Hiet man þat \alst{a}lla þea \alst{ę}li-lęndjun man \hld\ iro \alst{ó}ðil sóhtin, &
\alst{h}ęliðos iro \alst{h}and-mahạl \hld\ an·gegen iro \alst{h}êrron bodon, &
\alst{k}wámi te þem \alst{k}nósla gi·hwe, \hld\ þanan hé \alst{k}unnjas was, &
gi·\alst{b}oran fon þem \alst{b}urgjun. \hld\ Þat gi·\alst{b}od warð gi·lêstid &
ovar þesa \alst{w}ídon \alst{w}er-old; \hld\ \alst{w}erod samnoda &
te allaro \alst{b}urgjo gi·hwem. \hld\ Fórun þea \alst{b}odon ovar all, &
þea fon þem \alst{k}êsura \hld\ \alst{k}umana wárun, &%NOTE: run start S
\alst{b}ók-spáha weros, \hld\ ęndi an \alst{b}réf skrivun &
swíðo \alst{n}iud-líko \hld\ \alst{n}amono gi·hwi-likan, &
ia \alst{l}and ia \alst{l}iudi, \hld\ þat im ni mahti a·\alst{l}ęttjan mann &
\alst{g}umono su·lika \alst{g}ambra, \hld\ só im skolda \alst{g}eldan gi·hwe &
\alst{h}ęliðo fon is \alst{h}ôvda. \hld\ Þȯ gi·wêt im ôk mid is \alst{h}íwiska &
\alst{J}oseph þe \alst{g}ódo, \hld\ só it \alst{g}od mahtig, &
\alst{w}aldand \alst{w}elda: \hld\ sóhta im þiu \alst{w}ánamon hêm, &
þea \alst{b}urg an \alst{B}ethleem, \hld\ þár iro \edtext{\alst{b}ęiðero}{\Afootnote{The diphthong is original and occurs in which manuscripts? TODO. It also occurs at two other places, viz. TODO and TODO.}} was, &
þes \alst{h}ęliðes \alst{h}and-mahạl* \hld\ ęndi ôk þera \alst{h}êlagun þiornun, &
\alst{M}ariun þera gódun. \hld\ Þár was þes \alst{m}árjon stól &
an \alst{ê}r-dagun, \hld\ \alst{a}ðal-kuninges, &
\alst{D}awides þes gódon, \hld\ þan langa þe hé þana \alst{d}ruht-skępi þár, &
\alst{e}rl undar \alst{E}breon \hld\ \alst{ê}gan mósta, &
\alst{h}aldan \alst{h}ôh-gi·setu. \hld\ Sie wárun is \alst{h}íwiskas, &
\alst{k}uman fon is \alst{k}nósla, \hld\ \alst{k}unnjas gódes, &
\alst{b}êðju bi gi·\alst{b}urdjun. \hld\ Þár gi·fragn ik, þat sie þiu \alst{b}erhtun gi·skapu, &
\alst{M}ariun gi·\alst{m}anodun \hld\ *ęndi \alst{m}aht godes, &
þat iru an þem \alst{s}ïða \hld\ \alst{s}unu ôdan warð, &
gi·\alst{b}oran an \alst{B}ethleem \hld\ \alst{b}arno strangost, &
allaro \alst{k}uningo \alst{k}raftigost: \hld\ \alst{k}uman warð þe márjo, &
\alst{m}ahtig an \alst{m}anno lioht, \hld\ só is êr \alst{m}anagan dag &
\alst{b}iliði wárun \hld\ ęndi \alst{b}ôkno filu &
gi·\alst{w}orðen an þesero \alst{w}er-oldi. \hld\ Þȯ was it all gi·\alst{w}árod só, &
só it êr \alst{sp}áha man \hld\ gi·\alst{sp}rokan habdun, &
þurh hwi-lik \alst{ô}d-módi \hld\ hé þit \alst{e}rð-ríki herod &
þurh is \alst{s}elves kraft \hld\ \alst{s}ókjan welda, &
\alst{m}anagaro \alst{m}und-boro. \hld\ Þȯ ina þiu \alst{m}ódar nam, &
bi·\alst{w}and ina mid \alst{w}ádju \hld\ \alst{w}ívo skônjost, &
\alst{f}agạron \alst{f}ratahun, \hld\ ęndi ina mid iro \alst{f}olmon twêm &
\alst{l}ęgda \alst{l}iov-líko \hld\ \alst{l}uttilna man, &
þat \alst{k}ind an êna \alst{k}ribbjun, \hld\ þoh hé habdi \alst{k}raft godes, &
\alst{m}anno drohtin. \hld\ Þár sat þiu \alst{m}ódar bi·foran, &
\alst{w}íf \alst{w}akogjandi, \hld\ \alst{w}ar*doda selvo, &
\alst{h}eld þat \alst{h}êlaga barn: \hld\ ni was ira \alst{h}ugi twífli, &
þera \alst{m}agað ira \alst{m}ód-sevo. \hld\ Þȯ warð þat \alst{m}anagun ku̇ð &
ovar þesa \alst{w}ídon \alst{w}er-old, \hld\ \alst{w}ardos ant·fundun, &
þea þár \alst{e}hu-skalkos \hld\ \alst{ú}ta wárun, &
\alst{w}eros an \alst{w}ahtu, \hld\ \alst{w}iggjo gômjan, &
\alst{f}ehas aftar \alst{f}el*da: \hld\ gi·sáhun \alst{f}inistri an twê &
te·\alst{l}átan an \alst{l}ufte, \hld\ ęndi kwam \alst{l}ioht godes &
\alst{w}ánum þurh þiu \alst{w}olkan \hld\ ęndi þea \alst{w}ardos þár &
bi·\alst{f}éng an þem \alst{f}elda. \hld\ Sie wurðun an \alst{f}orhtun þȯ, &
þea \alst{m}an an ira \alst{m}óda: \hld\ gi·sáhun þár \alst{m}ahtigna &
\alst{g}odes ęngil kuman, \hld\ þe im te·\alst{g}ęgnes sprak, &
hét þat im þea \alst{w}ardos \hld\ \alst{w}iht ne antd-rédin &
\alst{l}êðes fon þem \alst{l}iohta: \hld\ „ik skal eu“, kwað hé, „\alst{l}iovara þing, &
swíðo \alst{w}ár-líko \hld\ \alst{w}illjon sęggjan, &
\alst{k}u̇ðjan \alst{k}raft mikil: \hld\ nu is \alst{K}rist ge·boran &
an þeser*o \alst{s}elvun naht, \hld\ \alst{s}álig barn godes, &
an þera \alst{D}awides burg, \hld\ \alst{d}rohtin þe gódo. &
Þat is \alst{m}ęndislo \hld\ \alst{m}anno kunnjas, &
allaro \alst{f}iriho \alst{f}ruma. \hld\ Þár gí ina \alst{f}ïðan mugun, &
an \alst{B}ethlema-burg \hld\ \alst{b}arno ríkjost: &
hębbjad þat te \alst{t}êkna, \hld\ þat ik eu gi·\alst{t}ęlljan mag &
\alst{w}árun \alst{w}ordun, \hld\ þat hé þár bi·\alst{w}undan ligid, &
þat \alst{k}ind an ênera \alst{k}ribbjun, \hld\ þoh hé sí \alst{k}uning ovar al &
\alst{e}rðun ęndi himiles \hld\ ęndi ovar \alst{ę}ldjo barn, &
\alst{w}er-oldes \alst{w}aldand“. \hld\ Reht só hé þȯ þat \alst{w}ord gi·sprak, &
só warð þár \alst{ę}ngilo te þem \alst{ê}nun \hld\ \alst{u}n-rím kuman, &
\alst{h}êlag \alst{h}ęri-skępi \hld\ fon \alst{h}evan-wanga, &
\alst{f}agạr \alst{f}olk godes, \hld\ ęndi \alst{f}ilu sprákun, &
\alst{l}of-word manag \hld\ \alst{l}iudjo hêrron. &
Af·\alst{h}óvun þȯ \alst{h}êlagna sang, \hld\ þȯ sie eft te \alst{h}evan-wanga &
\alst{w}undun þurh þiu \alst{w}olkan. \hld\ Þea \alst{w}ardos hôrdun, &
hwó þiu \alst{ę}ngilo kraft \hld\ \alst{a}lo-mahtigna god &
swíðo \alst{w}erð-líko \hld\ \alst{w}ordun lovodun: &
„\alst{d}iuriða sí nu“, \hld\ kwáðun sie, „\alst{d}rohtine selvun &
an þem \alst{h}ôhoston \hld\ \alst{h}imilo ríkja &
ęndi \alst{f}riðu an erðu \hld\ \alst{f}iriho barnun, &
\alst{g}ód-willigun \alst{g}umun, \hld\ þem þe \alst{g}od ant·kęnnjad &
þurh \alst{h}luttran \alst{h}ugi.“ \hld\ Þea \alst{h}irdjo for·stódun, &
þat sie \alst{m}ahtig þing \hld\ gi·\alst{m}anod habda, &
\alst{b}líð-lík \alst{b}od-skępi: \hld\ gi·witun im te \alst{B}ethleem þanan &
\alst{n}ahtes sïðon; \hld\ was im \alst{n}iud mikil, &
þat sie \alst{s}elvon Krist \hld\ gi·\alst{s}ehan móstin.\eva

\bvb TODO.\evb\evg

\bvg\bva[6][427]%
Habda im þe \alst{ę}ngil godes \hld\ \alst{a}l gi·wísid &
\alst{t}orhtun \alst{t}êknun, \hld\ þat sie im \alst{t}ó selvun, &
te þem \alst{g}odes barne \hld\ \alst{g}angan mahtun, &
ęndi \alst{f}undun sán \hld\ \alst{f}olko drohtin, &
\alst{l}iudjo hêrron. \hld\ Sagdun þȯ \alst{l}of goda, &
\alst{w}aldande mid iro \alst{w}ordun \hld\ ęndi \alst{w}ído ku̇ðdun &
ovar þea \alst{b}erhtun \alst{b}urg, \hld\ hwi-lik im þár \alst{b}iliði warð &
fon \alst{h}evan-wanga \hld\ \alst{h}êlag gi·tôgit, &
\alst{f}agạr an \alst{f}elde. \hld\ Þat \alst{f}rí al bi·held &
an ira \alst{h}ugi-skęftjun, \hld\ \alst{h}êlag þiorna, &
þiu \alst{m}agað an ira \alst{m}óde, \hld\ só hwat só siu gi·hôrda þea \alst{m}ann sprekan. &
\alst{F}ódda ina þȯ \alst{f}agạro \hld\ \alst{f}rího skânjosta, &
þiu \alst{m}ódar þurh \alst{m}innja \hld\ \alst{m}anagaro drohtin, &
\alst{h}êlag \alst{h}imilisk barn. \hld\ \alst{H}ęliðos gi·sprákun &
an þem \alst{a}htodon daga \hld\ \alst{e}rlos managa, &
swíðo \alst{g}lawa \alst{g}umon \hld\ mid þera \alst{g}odes þiornun, &
þat hé \alst{h}êljand te namon \hld\ \alst{h}ębbjan skoldi, &
só it þe \alst{g}odes ęngil \hld\ \alst{G}abriel gi·sprak &
\alst{w}áron \alst{w}ordun \hld\ ęndi þem \alst{w}íve gi·bôd, &
\alst{b}odo drohtines, \hld\ þȯ siu êrist þat \alst{b}arn ant·féng &
\alst{w}ánum te þesero \alst{w}er-oldi; \hld\ was iru \alst{w}illjo mikil, &
þat siu ina só \alst{h}êlag-líko \hld\ \alst{h}aldan mósti, &
ful-\alst{g}éng im þȯ só \alst{g}erno. \hld\ Þat \alst{g}ę́r furðor skrêd &
unt-þat þat \alst{f}riðu-barn godes \hld\ \alst{f}iar-tig habda &
\alst{d}ago ęndi nahto. \hld\ Þȯ skoldun sie þár êna \alst{d}ád frummjan, &
þat sie ina te \alst{J}erusalem \hld\ for·\alst{g}evan skoldun &
\alst{w}aldanda te þem \alst{w}íha. \hld\ Só was iro \alst{w}ísa þan, &
þero \alst{l}iudjo \alst{l}and-sidu, \hld\ þat þat ni mósta for·\alst{l}átan ne-gên &
\alst{i}dis undar \alst{E}breon, \hld\ ef iru at \alst{ê}rist warð &
\alst{s}unu a·fódit, \hld\ ne siu ina \alst{s}imbla þarod &
te þem \alst{g}odes wíha \hld\ for·\alst{g}evan skolda. &
Gi·witun im þȯ þiu \alst{g}ódun twê, \hld\ \alst{J}oseph ęndi Maria &
\alst{b}êðju fon Bethleem: \hld\ habdun þat \alst{b}arn mid im, &
\alst{h}êlagna Krist, \hld\ sóhtun im \alst{h}ús godes &
an \alst{J}erusalem; \hld\ þár skoldun sie is \alst{g}eld frummjan &
\alst{w}aldanda at þem \alst{w}íha \hld\ \alst{w}ísa lêstjan &
\alst{J}udeo folkes. \hld\ Þár fundun sea ênna \alst{g}ódan man &
\alst{a}ldan at þem \alst{a}lạha, \hld\ \alst{a}ðal-boranan, &
þe habda at þem \alst{w}íha só filu \hld\ \alst{w}intro ęndi sumaro &
gi·\alst{l}ibd an þem \alst{l}iohta: \hld\ oft warhta hé þár \alst{l}of goda &
mid \alst{h}luttru \alst{h}ugi; \hld\ habda im \alst{h}êlagna gêst, &
\alst{s}álig-líkan \alst{s}evon; \hld\ \alst{S}imeon was hé hêtan. &
Im habda gi·\alst{w}ísid \hld\ \alst{w}aldandas kraft &
\alst{l}anga hwíla, \hld\ þat hé ni mósta êr þit \alst{l}ioht a·gevan, &
\alst{w}ęndjan af þesero \alst{w}er-oldi, \hld\ êr þan im þe \alst{w}illjo gi·stódi, &
þat hé \alst{s}elvan Krist \hld\ gi·\alst{s}ehan mósti, &
\alst{h}êlagna \alst{h}evan-kuning. \hld\ Þȯ warð im is \alst{h}ugi swíðo &
\alst{b}líði an is \alst{b}riostun, \hld\ þȯ hé gi·sah þat \alst{b}arn kuman &
an þena \alst{w}íh innan. \hld\ Þuȯ sagda hie \alst{w}aldande þank, &
\alst{a}l-mahtigon gode, \hld\ þes hé ina mid is \alst{ô}gun gi·sah. &
\alst{G}éng im þȯ te·\alst{g}ęgnes \hld\ ęndi ina \alst{g}erno ant·féng &
\alst{a}ld mid is \alst{a}rmun: \hld\ \alst{a}l ant·kęnde &
\alst{b}ôkan ęndi \alst{b}iliði \hld\ ęndi ôk þat \alst{b}arn godes, &
\alst{h}êlagna \alst{h}evan-kuning. \hld\ „Nu ik þi, \alst{h}êrro, skal“, kwað hé, &
„\alst{g}erno biddjan, \hld\ nu ik sus gi·\alst{g}amalod bium, &
þat þú þínan \alst{h}oldan skalk \hld\ nu hinan \alst{h}wervan látas, &
an þína \alst{f}riðu-wára \alst{f}aran, \hld\ þár êr mína \alst{f}orðrun dedun, &
\alst{w}eros fon þesero \alst{w}er-oldi, \hld\ nu mi þe \alst{w}illjo gi·stód, &
\alst{d}ago liovosto, \hld\ þat ik mínan \alst{d}rohtin gi·sah, &
\alst{h}oldan \alst{h}êrron, \hld\ só mi gi·\alst{h}êtan was &
\alst{l}anga hwíla. \hld\ Þú bist \alst{l}ioht mikil &
\alst{a}llun \alst{ę}li-þiodun, \hld\ þea êr þes \alst{a}lo-waldon &
\alst{k}raft ne ant·\alst{k}ęndun. \hld\ Þína \alst{k}umi sindun &
te \alst{d}óma ęndi te \alst{d}iurðon, \hld\ \alst{d}rohtin frô mín, &
\alst{a}varun \alst{I}srahelas, \hld\ \alst{ê}ganumu folke, &
þínun \alst{l}iovun *\alst{l}iudjun.“ \hld\ \alst{L}istjun talde þȯ &
þe \alst{a}ldo man an þem \alst{a}lạha \hld\ \alst{i}dis þero gódun, &
\alst{s}agda \alst{s}ȯð-líko, \hld\ hwó iro \alst{s}unu skolda &
ovar þesan \alst{m}iddil-gard \hld\ \alst{m}anagun werðan &
sumun te \alst{f}alle, sumun te \alst{f}róvru \hld\ \alst{f}iriho barnun, &
þem \alst{l}iudjun te \alst{l}eova, \hld\ þe is \alst{l}êrun gi·hôrdin, &
ęndi þem te \alst{h}arma, \hld\ þe \alst{h}ôrjen ni weldin &
\alst{K}ristas lêron. \hld\ „Þu skalt noh“, kwað hé, „\alst{k}ara þiggjan, &
\alst{h}arm an þínumu \alst{h}erton, \hld\ þan ina \alst{h}ęliðo barn &
\alst{w}ápnun \alst{w}ítnod. \hld\ Þat wirðid þi \alst{w}erk mikil, &
\alst{þ}rim te gi·\alst{þ}olonna.“ \hld\ Þiu \alst{þ}iorna al for·stód &
\alst{w}ísas mannas \alst{w}ord. \hld\ Þȯ kwam þár ôk ên \alst{w}íf gangan &
\alst{a}ld innan þem \alst{a}lạha: \hld\ \alst{A}nna was siu hêtan, &
\alst{d}ohtar Fanueles; \hld\ siu habde ira \alst{d}rohtine wel &
gi·\alst{þ}ionod te \alst{þ}anka, \hld\ was iru gi·\alst{þ}ungan wíf. &
Siu \alst{m}ósta aftar ira \alst{m}agað-hêdi, \hld\ sïðor siu \alst{m}annes warð, &
\alst{e}rles an \alst{ê}hti \hld\ \alst{ę}ðili þiorne, &
só mósta siu mid ira \alst{b}rúdi-gumon \hld\ \alst{b}odlo gi·waldan &
\alst{s}ivun wintạr \alst{s}aman. \hld\ Þȯ gi·fragn ik þat iru þár \alst{s}orga gi·stód &
þat sie þiu \alst{m}ikila \alst{m}aht \hld\ \alst{m}etodes te·dêlda, &
\alst{w}rêð \alst{w}urdi-gi·skapu. \hld\ Þȯ was siu \alst{w}idowa aftar þiu &
at þem \alst{f}riðu-wíha \hld\ \alst{f}ior ęndi ant·ahtoda &
\alst{w}intro an iro \alst{w}er-oldi, \hld\ só siu nia þana \alst{w}íh ni for·lét, &
ak siu þár ira \alst{d}rohtine wel \hld\ \alst{d}ages ęndi nahtes, &
\alst{g}ode þionode. \hld\ Siu kwam þár ôk \alst{g}angan tó &
an þea \alst{s}elvun tíd: \hld\ \alst{s}án ant·kęnde &
þat \alst{h}êlage barn godes \hld\ ęndi þem \alst{h}ęliðon ku̇ðde, &
þem \alst{w}eroda aftar þem \alst{w}íha \hld\ \alst{w}il-spel mikil, &
kwað þat im \alst{n}ęrjandas gi·\alst{n}ist \hld\ gi·\alst{n}áhid wári, &
\alst{h}elpa \alst{h}evan-kuninges: \hld\ „nu is þe \alst{h}êlago Krist, &
\alst{w}aldand selvo \hld\ an þesan \alst{w}íh kuman &
te a·\alst{l}ôsjenne þea \alst{l}iudi, \hld\ þe hér nu \alst{l}ango bidun &
an þesara \alst{m}iddil-gard, \hld\ \alst{m}anaga hwíla, &
\alst{þ}urftig \alst{þ}ioda, \hld\ só nu þes \alst{þ}inges mugun &
\alst{m}ęndjan \alst{m}an-kunni.“ \hld\ \alst{M}anag fagonoda &
\alst{w}erod aftar þem \alst{w}íha: \hld\ gi·hôrdun \alst{w}il-spel mikil &
fon \alst{g}ode sęggjan. \hld\ Þat \alst{g}eld habde þȯ gi·lêstid &
þiu \alst{i}dis an þem \alst{a}lạha, \hld\ al só it im an ira \alst{ê}wa gi·bôd &
ęndi an þera \alst{b}erhtun \alst{b}urg \hld\ \alst{b}ók gi·wísdun, &
\alst{h}êlagaro \alst{h}and-gi·werk. \hld\ Gi·witun im þȯ te \alst{h}ús þanan &
fon \alst{J}erusalem \hld\ \alst{J}oseph ęndi Maria, &
\alst{h}êlag \alst{h}íwiski: \hld\ habdun im \alst{h}evan-kuning &
\alst{s}imbla te gi·\alst{s}ïða, \hld\ \alst{s}unu drohtines, &
\alst{m}anagaro \alst{m}und-boron, \hld\ só it gio \alst{m}ári ni warð &
þan \alst{w}ídor an þesaro \alst{w}er-oldi, \hld\ b·útan só is \alst{w}illjo géng, &
\alst{h}evan-kuninges \alst{h}ugi.\eva

\bvb TODO.\evb\evg

\bvg\bva[7][537]%
\hspace*{100pt} Þoh þár þan gi·hwi-lik \alst{h}êlag man &%NOTE: in cæsura
\alst{K}rist ant·\alst{k}ęndi, \hld\ þoh ni warð it gio te þes \alst{k}uninges hove &
þem \alst{m}annun gi·\alst{m}árid, \hld\ þea im an iro \alst{m}ód-sevon &
\alst{h}olde ni wárun, \hld\ ak was im só bi·\alst{h}alden forð &
mid \alst{w}ordun ęndi mid \alst{w}erkun, \hld\ ant-þat þár \alst{w}eros ôstan, &
swíðo \alst{g}lawa \alst{g}umon \hld\ \alst{g}angan kwámun &
\alst{þ}rea te þero \alst{þ}iodu, \hld\ \alst{þ}egnos snelle, &
an \alst{l}angan weg \hld\ ovar þat \alst{l}and þarod: &
folgodun ênun \alst{b}erhtun \alst{b}ôkne \hld\ ęndi sóhtun þat \alst{b}arn godes &
mid \alst{h}luttru \alst{h}ugi: \hld\ weldun im \alst{h}nígan tó, &
\alst{g}ehan im te \alst{j}ungrun: \hld\ drivun im \alst{g}odes gi·skapu. &
Þȯ sie \edtext{E\alst{r}ódesan}{\Bfootnote{This alliteration also occurs in at least two other lines. TODO.}} þár \hld\ \alst{r}íkjan fundun &
an is \alst{s}ęli \alst{s}ittjen, \hld\ \alst{s}líð-wurdjan kuning, &
\alst{m}ódagna mid is \alst{m}annun: \hld\ —simbla was hé \alst{m}orðes gern— &
þȯ \alst{k}waddun sie ina \alst{k}úsko \hld\ an \alst{k}uning-wísun, &
\alst{f}agạro an is \alst{f}lęttje, \hld\ ęndi hé \alst{f}rágoda sán, &
hwi-lik sie \alst{â}rundi \hld\ \alst{ú}ta gi·brȧhti, &
\alst{w}eros an þana \alst{w}rak-sïð: \hld\ „hweðer lêdjad gí \alst{w}undan gold &
te \alst{g}evu hwi-likun \alst{g}umuno? \hld\ te hwí gí þus an \alst{g}anga kumad, &
gi·\alst{f}aran an \alst{f}óðju? \hld\ Hwat gí n·êt-hwanan \alst{f}erran sind &
\alst{e}rlos fon \alst{ȯ}ðrun þiodun. \hld\ Ik gi·sihu þat gi sind \alst{ę}ðili-gi·burdjun &
\alst{k}unnjes fon \alst{k}nósle gódun: \hld\ nio hér êr su·lika \alst{k}umana ni wurðun &
\alst{é}ri fon \alst{ȯ}ðrun þiodun, \hld\ sïðor ik mósta þesas \alst{e}rlo folkes, &
gi·\alst{w}aldan þesas \alst{w}ídon ríkjas. \hld\ Gí skulun mi te \alst{w}árun sęggjan &
for þesun \alst{l}iudjo folke, \hld\ bi·hwí gí sín te þesun \alst{l}ande kumana“. &
Þȯ sprákun im eft te·\alst{g}ęgnes \hld\ \alst{g}umon ôstr-onja, &
\alst{w}ord-spáhe \alst{w}eros: \hld\ „wí þí te \alst{w}árun mugun“, kwáðun sie, &
„\alst{u̇}se \alst{â}rundi \hld\ \alst{ó}ðo gi·tęlljen, &
gi·\alst{s}ęggjan \alst{s}ȯð-líko, \hld\ bi·hwí wí kwámun an þesan \alst{s}ïð herod &
fon \alst{ô}stan te þesaro \alst{e}rðu. \hld\ Giu wárun þár \alst{a}ðaljes man, &
\alst{g}ód-sprákja \alst{g}umon, \hld\ þea u̇s \alst{g}ódes só filu, &
\alst{h}elpa gi·\alst{h}étun \hld\ fon \alst{h}evan-kuninge &
\alst{w}árum \alst{w}ordun. \hld\ Þan was þár ên gi·\alst{w}ittig man, &
\alst{f}ród ęndi \alst{f}il-wís \hld\ —\alst{f}orn was þat giu—, &
\alst{u̇}se \alst{a}ldiro \alst{ô}star hinan, \hld\ —þár ni warð sïðor \alst{ê}nig man &
\alst{sp}rákono só \alst{sp}áhi—; \hld\ hé mahte rekkjen \alst{sp}el godes, &
hwand im habde for·\alst{l}iwan \hld\ \alst{l}iudjo hêrro, &
þat hé mahte fon \alst{e}rðu \hld\ \alst{u}p gi·hôrjan &
\alst{w}aldandes \alst{w}ord: \hld\ bi·þiu was is gi·\alst{w}it mikil, &
þes \alst{þ}egnes gi·\alst{þ}ȧhti. \hld\ Þȯ hé \alst{þ}anan skolda, &
a·\alst{g}even \alst{g}ardos, \hld\ \alst{g}adulingo gi·mang, &
for·\alst{l}áten \alst{l}iudjo drôm, \hld\ sókjen \alst{l}ioht ȯðar, &
þȯ hé is \alst{j}ungron hét \hld\ \alst{g}angan náhor, &
\alst{ę}rvi-wardos, \hld\ ęndi is \alst{e}rlun þȯ &
\alst{s}agde \alst{s}ȯð-líko: \hld\ —þat al \alst{s}ïðor kwam, &
gi·\alst{w}arð* an þesaro \alst{w}er-oldi—: \hld\ þȯ sagda hé þat hér skoldi kuman ên \alst{w}ís-kuning &
\alst{m}ári ęndi \alst{m}ahtig \hld\ an þesan \alst{m}iddil-gard &
þes \alst{b}ętston gi·\alst{b}urdjes; \hld\ kwað þat it skoldi wesan \alst{b}arn godes, &
kwað þat hé þesero \alst{w}er-oldes \hld\ \alst{w}aldan skoldi &
gio te \alst{ê}wan-daga, \hld\ \alst{e}rðun ęndi himiles. &
Hé kwað þat an þem \alst{s}elvon daga, \hld\ þe ina \alst{s}áligna &
an þesan \alst{m}iddil-gard \hld\ \alst{m}ódar gi·drógi, &
só kwað hé þat \alst{ô}stana \hld\ \alst{ê}n skoldi skínan &
\alst{h}imil-tungạl \alst{h}wít, \hld\ su·lik só wí hér ne \alst{h}abdin êr &
undar·twisk \alst{e}rða ęndi himil \hld\ \alst{ȯ}ðar hwerigin, &
ne su·lik \alst{b}arn ne su·lik \alst{b}ôkan. \hld\ Hét þat þár te \alst{b}edu fórin &
\alst{þ}rea man fon þero \alst{þ}iodu, \hld\ hét sie \alst{þ}ęnkjan wel, &
hwan êr sie gi·\alst{s}áwin ôstana \hld\ up \alst{s}íðogjan, &%TODO: Check síðogjan
þat \alst{g}odes bôkan \alst{g}angan, \hld\ hét sie \alst{g}arwjan sán, &
hét þat wí im \alst{f}olgodin, \hld\ só it \alst{f}uri wurði, &
\alst{w}estạr ovar þesa \alst{w}er-oldi. \hld\ Nu is it al gi·\alst{w}árod só, &
\alst{k}uman þurh \alst{k}raft godes: \hld\ þe \alst{k}uning is gi·fódit, &
gi·\alst{b}oran \alst{b}ald ęndi strang: \hld\ wí gi·sáhun is \alst{b}ôkan skínan &
\alst{h}êdro fon \alst{h}imiles tunglun, \hld\ só ik wêt, þat it \alst{h}êlag drohtin, &
\alst{m}arkoda \alst{m}ahtig selvo; \hld\ wí gi·sáhun \alst{m}orgno gi·hwi-likes &
\alst{b}líkan þana \alst{b}erhton sterron, \hld\ ęndi wí géngun aftar þem \alst{b}ôkna herod &
\alst{w}egas ęndi \alst{w}aldas hwílon. \hld\ Þat wári u̇s allaro \alst{w}illjono mêsta, &
þat wí ina \alst{s}elvon gi·\alst{s}ehan móstin, \hld\ wissin, hwar wí ina \alst{s}ókjan skoldin, &
þana \alst{k}uning an þesumu \alst{k}êsur-dóma. \hld\ Saga u̇s, undar hwi-likumu hé sí þesaro \alst{k}unnjo a·fódit.“ &
Þȯ warð \alst{E}rodesa \hld\ \alst{i}nnan briostun &
\alst{h}arm wið \alst{h}erta, \hld\ bi·gan im is \alst{h}ugi wallan, &
\alst{s}evo mid \alst{s}orgun: \hld\ gi·hôrde \alst{s}ęggjan þȯ, &
þat hé þár \alst{o}var-hôvdon \hld\ \alst{ê}gan skoldi, &
\alst{k}raftagoron \alst{k}uning \hld\ \alst{k}unnjes gódes, &
\alst{s}áligoron undar þem gi·\alst{s}ïðja. \hld\ Þȯ hé \alst{s}amnon hét, &
só hwat só an \alst{J}erusalem \hld\ \alst{g}ódaro manno &
allaro \alst{sp}áhoston \hld\ \alst{sp}rákono wárun &
ęndi an iro \alst{b}rioston \hld\ \alst{b}ók-kraftes mêst &
\alst{w}issun te \alst{w}árun, \hld\ ęndi hé sie mid \alst{w}ordun fragn, &
swíðo \alst{n}iud-líko \hld\ \alst{n}íð-hugdig man, &
\alst{k}uning þero liudjo, \hld\ hwar \alst{K}rist gi·boran &
an \alst{w}er-old-ríkja \hld\ \alst{w}erðan skoldi, &
\alst{f}riðu-gumono bętst. \hld\ Þȯ sprak im eft þat \alst{f}olk an·gęgin, &
þat \alst{w}erod \alst{w}ár-líko, \hld\ kwáðun þat sie \alst{w}issin garo, &
þat hé skoldi an \alst{B}ethleem gi·\alst{b}oran werðan: \hld\ „só is an u̇sun \alst{b}ókun gi·skrivan, &
\alst{w}ís-líko gi·\alst{w}ritan, \hld\ só it \alst{w}ár-sagon, &
swíðo \alst{g}lawa \alst{g}umon \hld\ bi \alst{g}odes krafta &
\alst{f}il-wíse man \hld\ \alst{f}urn gi·sprákun, &
þat skoldi fon \alst{B}ethleem \hld\ \alst{b}urgo hirdi, &
\alst{l}iof \alst{l}andes ward \hld\ an þit \alst{l}ioht kuman, &
\alst{r}íki \alst{r}ád-gevo, \hld\ þe \alst{r}ihtjen skal &
\alst{J}udeono \alst{g}um-skępi \hld\ ęndi is \alst{g}eva wesan &
\alst{m}ildi ovar \alst{m}iddil-gard \hld\ \alst{m}anagun þiodun.“\eva

\bvb TODO.\evb\evg

\bvg\bva[8][630]%
Þȯ gi·fragn ik þat \alst{s}án aftar þiu \hld\ \alst{s}líð-mód kuning &
þero \alst{w}ár-sagono \alst{w}ord \hld\ þem \alst{w}rękkjun sagda, &
þea þár an \alst{ę}li-lęndi \hld\ \alst{e}rlos wárun &
\alst{f}erran gi·\alst{f}arana, \hld\ ęndi hé \alst{f}rágoda aftar þiu, &
hwan sie an \alst{ô}star-wegun \hld\ \alst{ê}rist gi·sáhin &
þana \alst{k}uning-sterron kuman, \hld\ \alst{k}umbal liuhtjen &
\alst{h}êdro fon \alst{h}imile. \hld\ Sie ni weldun is im þȯ \alst{h}elen eo·wiht, &
ak \alst{s}agdun it im \alst{s}ȯð-líko. \hld\ Þȯ hét hé sie an þana \alst{s}ïð faran, &
hét þat sie ira \alst{â}rundi al \hld\ \alst{u}ndar fundin &
umbi þes \alst{k}indes \alst{k}umi, \hld\ ęndi þe \alst{k}uning selvo gi·bôd &
swíðo \alst{h}ard-liko, \hld\ \alst{h}êrro Judeono, &
þem \alst{w}ísun mannun, \hld\ êr þan sie fórin \alst{w}estan forð, &
þat sie im eft gi·\alst{k}u̇ðdin, \hld\ hwar hé þana \alst{k}uning skoldi &
\alst{s}ókjan at is \alst{s}elðon; \hld\ kwað þat hé þár weldi mid is gi·\alst{s}ïðun tó, &
\alst{b}edan te þem \alst{b}arne. \hld\ Þan hogda hé im te \alst{b}anon werðan &
\alst{w}ápnes ęggjun. \hld\ Þan eft \alst{w}aldand god &
\alst{þ}ȧhte wið þem \alst{þ}inga: \hld\ hé mahta a·\alst{þ}ęngjan mêr, &
gi·\alst{l}êstjan an þesum \alst{l}iohte: \hld\ þat is noh \alst{l}ango skín, &
gi·\alst{k}u̇ðid \alst{k}raft godes. \hld\ Þȯ géngun eft þiu \alst{k}umbl forð &
\alst{w}ánum undar \alst{w}olknun. \hld\ Þȯ wárun þea \alst{w}íson man &
\alst{f}u̇sa te \alst{f}aranne: \hld\ gi·witun im \alst{f}orð þanan &
\alst{b}alda an \alst{b}od-skępi: \hld\ weldun þat \alst{b}arn godes &
\alst{s}elvon \alst{s}ókjan. \hld\ Sie ni habdun þanan gi·\alst{s}ïðjas mêr, &
b·útan þat sie \alst{þ}ríe wárun: \hld\ wissun im \alst{þ}ingo gi·skêð, &
wárun im \alst{g}lawe \alst{g}umon, \hld\ þe þea \alst{g}eva lêddun. &
Þan sáhun sie só \alst{w}ís-líko \hld\ undar þana \alst{w}olknes skion, &
up te þem \alst{h}ôhon \alst{h}imile, \hld\ hwó fórun þea \alst{h}wíton sterron &
—ant·\alst{k}ęndun sie þat \alst{k}umbạl godes—, \hld\ þiu wárun þurh \alst{K}rista herod &
gi·\alst{w}arht te þesero \alst{w}er-oldi. \hld\ Þea \alst{w}eros aftar géngun, &
\alst{f}olgodun \alst{f}erạht-líko \hld\ —sie \alst{f}rumide þe mahte— &
ant-þat sie gi·\alst{s}áhun, \hld\ \alst{s}ïð-wórige man, &
\alst{b}erht \alst{b}ôkạn godes, \hld\ \alst{b}lêk an himile &
\alst{st}illo gi·\alst{st}anden. \hld\ Þe \alst{st}erro liohto skên &
\alst{h}wít ovar þem húse, \hld\ þár þat hêlage barn &
\alst{w}onode an \alst{w}illjon \hld\ ęndi ina þat \alst{w}íf bi·held, &
þiu \alst{þ}iorne gi·\alst{þ}iudo. \hld\ Þȯ warð þero \alst{þ}egno hugi &
\alst{b}líði an iro \alst{b}riostun: \hld\ bi þem \alst{b}ôkna for·stódun, &
þat sie þat \alst{f}riðu-barn godes \hld\ \alst{f}unden habdun, &
\alst{h}êlagna \alst{h}evan-kuning. \hld\ Þȯ sie an þat \alst{h}ús innan &
mid iro \alst{g}evun \alst{g}éngun, \hld\ \alst{g}umon ôstr-onja, &
\alst{s}ïð-wórige man: \hld\ \alst{s}án ant·kęndun &
þea \alst{w}eros \alst{w}aldand Krist. \hld\ Þea \alst{w}rękkjon fellun &
te þem \alst{k}inde an \alst{k}neo-beda \hld\ ęndi ina an \alst{k}uning-wísa &
\alst{g}ódan \alst{g}róttun \hld\ ęndi im þea \alst{g}eva drógun, &
\alst{g}old ęndi wíh-rôk \hld\ bi \alst{g}odes têknun &
*ęndi \alst{m}yrra þár \alst{m}id. \hld\ Þea \alst{m}an stódun garowa, &
\alst{h}olde for iro \alst{h}êrron, \hld\ þea it mid iro \alst{h}andun sán &
\alst{f}agạro ant·\alst{f}éngun. \hld\ Þȯ gi·witun im þea \alst{f}erạhton man, &
\alst{s}ęggi te \alst{s}elðon \hld\ \alst{s}ïð-wórige, &
\alst{g}umon an \alst{g}ast-sęli. \hld\ Þár im \alst{g}odes ęngil &
\alst{s}lápandjun an naht \hld\ \alst{s}wevan gi·tôgde, &
gi·\alst{d}rog im an \alst{d}rôme, \hld\ al so it \alst{d}rohtin self, &
\alst{w}aldand \alst{w}elde, \hld\ þat im þúhte þat man im mid \alst{w}ordun gi·budi, &
þat sie im* þanan \alst{ȯ}ðran weg, \hld\ \alst{e}rlos fórin, &
\alst{l}iðodin sie te \alst{l}ande \hld\ ęndi þana \alst{l}êðan man, &
\alst{E}rodesan \hld\ \alst{e}ft ni sóhtin, &
\alst{m}ódagna kuning. \hld\ Þȯ warð \alst{m}organ kuman &
\alst{w}ánum te þesero \alst{w}er-oldi. \hld\ Þȯ bi·gunnun þea \alst{w}íson man &
\alst{s}ęggjan iro \alst{s}wevanos; \hld\ \alst{s}elvon ant·kęndun &
\alst{w}aldandes \alst{w}ord, \hld\ hwand sie gi·\alst{w}it mikil &
\alst{b}árun an iro \alst{b}riostun: \hld\ \alst{b}ádun alo-waldon, &
\alst{h}êron \alst{h}evan-kuning, \hld\ þat sie móstin is \alst{h}uldi forð, &
gi·\alst{w}irkjan is \alst{w}illjon, \hld\ kwáðun þat sea ti im habdin gi·\alst{w}ęndit hugi, &
*iro \alst{m}ód \alst{m}organ gi·hwem. \hld\ Þȯ fórun eft þie \alst{m}an þanan, &
\alst{e}rlos \alst{ô}str-onje, \hld\ al só im þe \alst{ę}ngil godes &
\alst{w}ordun gi·\alst{w}ísde: \hld\ námun im \alst{w}eg ȯðran, &
ful-\alst{g}éngun \alst{g}odes lêrun: \hld\ ni weldun þemu \alst{J}udeo kuninge &
umbi þes \alst{b}arnes gi·\alst{b}urd \hld\ \alst{b}odon ôstr-onje, &
\alst{s}ïð-wórige man \hld\ \alst{s}ęggjan gio·wiht, &
ak \alst{w}endun im eft an iro \alst{w}illjon.\eva

\bvb TODO.\evb\evg

\bvg\bva[9][699]%
\hspace*{100pt} Þȯ warð sán aftar þiu \alst{w}aldandes, &%NOTE: In cæsura
\alst{g}odes ęngil kumen \hld\ \alst{J}osepe te sprákun, &
\alst{s}agde im an \alst{s}wefne \hld\ \alst{s}lápandjum an naht, &
\alst{b}odo drohtines, \hld\ þat þat \alst{b}arn godes &
\alst{s}líð-mód kuning \hld\ \alst{s}ókjan welda, &
\alst{á}htjan is \alst{a}ldres; \hld\ „nu skaltu ine an \alst{A}egypteo &
\alst{l}and ant·\alst{l}êdjan \hld\ ęndi undar þem \alst{l}iudjun wesan &
mid þiu \alst{g}odes barnu \hld\ ęndi mid þeru \alst{g}ódan þior*nan, &
\alst{w}unon undar þemu \alst{w}erode, \hld\ unt-þat þi \alst{w}ord kume &
\alst{h}êrron þínes, \hld\ þat þú þat \alst{h}êlage barn &
eft te þesum \alst{l}and-skępi \hld\ \alst{l}êdjan mótis, &
\alst{d}rohtin þínen.“ \hld\ Þȯ fon þem \alst{d}rôma an·sprang &
\alst{J}oseph an is \alst{g}ęst-sęli, \hld\ ęndi þat \alst{g}odes gi·bod &
\alst{s}án ant·kęnda: \hld\ gi·wêt im an þana \alst{s}ïð þanen &
þe \alst{þ}egạn mid þeru \alst{þ}iornon, \hld\ sóhta im \alst{þ}iod ȯðra &
ovar \alst{b}rêdan \alst{b}erg: \hld\ welda þat \alst{b}arn godes &
\alst{f}íundun ant·\alst{f}órjan. \hld\ *Þȯ gi·\alst{f}rang aftar þiu &%NOTE: gi·frang [sic]
E\alst{r}ódes þe kuning, \hld\ þár hé an is \alst{r}íkja sat, &
þat \alst{w}árun þea \alst{w}íson man \hld\ \alst{w}estan gi·hworvan &
\alst{ô}star an iro \alst{ó}ðil \hld\ ęndi fórun im \alst{ȯ}ðran weg: &
wisse þat sie im þat \alst{â}rundi \hld\ \alst{e}ft ni weldun &
\alst{s}ęggjan an is \alst{s}elðon. \hld\ Þȯ warð im þes an \alst{s}orgun hugi, &
\alst{m}ód \alst{m}ornondi, \hld\ kwað þat it im þie \alst{m}an dedin, &
\alst{h}ęliðos* te \alst{h}ônðun. \hld\ Þȯ hé só \alst{h}riwig sat, &
\alst{b}alg ina an is \alst{b}riostun, \hld\ kwað þat hé is mahti \alst{b}ętaron rád, &
\alst{ȯ}ðran gi·þęnkjen: \hld\ „nu ik is \alst{a}ldar kan, &
\alst{w}êt is \alst{w}inter-gi·talu: \hld\ nu ik gi·\alst{w}innan mag, &
þat hé io ovar þesaro \alst{e}rðu \hld\ \alst{a}ld ni wirðit, &
\alst{h}ér undar þesum \alst{h}ęri-skępi.“ \hld\ Þȯ hé só \alst{h}ardo gi·bôd, &
E\alst{r}ódes ovar is \alst{r}íki, \hld\ hét þȯ is \alst{r}inkos faran &
\alst{k}uning þero liudjo, \hld\ hét þat sie \alst{k}inda só filo &
þurh iro \alst{h}and-magen \hld\ \alst{h}ôvdu bi·námin, &
só manag \alst{b}arn umbi \alst{B}ethleem, \hld\ só filo só þár gi·\alst{b}oran wurði, &
an \alst{t}wêm gêrun a·\alst{t}ogan. \hld\ \alst{T}ionon frumidon &
þes \alst{k}uninges gi·sïðos. \hld\ Þȯ skolda þár só manag \alst{k}indisk man &
\alst{s}weltan \alst{s}undjono lôs. \hld\ Ni warð \alst{s}íð noh êr &
\alst{j}ámar-líkara for·\alst{g}ang \hld\ \alst{j}ungaro manno, &
\alst{a}rm-líkara dôð. \hld\ \alst{I}disi wiopun, &
\alst{m}ódar \alst{m}anaga, \hld\ gi·sáhun iro \alst{m}ęgi spildjan: &
ni mahte siu im nio gi·\alst{f}ormon, \hld\ þoh siu mid iro \alst{f}aðmon twêm &
iro \alst{ê}gan barn \hld\ \alst{a}rmun bi·féngi, &
\alst{l}iof ęndi \alst{l}uttil, \hld\ þoh skolda is simbla þat \alst{l}íf gevan, &
þe \alst{m}agu for þeru \alst{m}ódar. \hld\ \alst{M}ênes ni sáhun, &
\alst{w}ítjes þie \alst{w}am-skaðon: \hld\ \alst{w}ápnes ęggjun &
\alst{f}ręmidun \alst{f}irin-werk mikil. \hld\ \alst{F}ellun managa &
\alst{m}agu-junge \alst{m}an. \hld\ Þia \alst{m}ódar wiopun &
\alst{k}ind-jungaro \alst{k}walm; \hld\ \alst{k}ara was an Bethleem, &
\alst{h}ofno \alst{h}lúdost: \hld\ þoh man im iro \alst{h}erton an twê &
\alst{s}niði mid \alst{s}werdu, \hld\ þoh ni mohta im gio \alst{s}êrara dád &
\alst{w}erðan an þesaro \alst{w}er-oldi, \hld\ \alst{w}ívun managun, &
\alst{b}rúdjun an \alst{B}ethleem: \hld\ gi·sáhun iro \alst{b}arn bi·foran, &
\alst{k}ind-junge man, \hld\ \alst{k}walmu sweltan &
\alst{b}lódag an iro \alst{b}armun. \hld\ Þie \alst{b}anon wítnodun &
un·\alst{sk}uldige \alst{sk}ole: \hld\ ni bi·\alst{sk}rivun gio·wiht &
þea \alst{m}an umbi \alst{m}ên-werk: \hld\ \alst{w}eldun mahtigna, &
\alst{K}rist selvon a·\alst{k}węlljan. \hld\ Þan habde ina \alst{k}raftag god &
gi·\alst{n}ęridan wið iro \alst{n}íðe, \hld\ þat inan \alst{n}ahtes þanan &
an \alst{A}egypteo land \hld\ \alst{e}rlos ant·lêddun, &
\alst{g}umon mid \alst{J}osepe \hld\ an þana \alst{g}rónjon wang, &
an \alst{e}rðono bętstun, \hld\ þár ên \alst{a}ha fliutid, &
\alst{N}íl-strôm mikil \hld\ \alst{n}orð te sêwa, &
\alst{f}lódo \alst{f}agọrosta. \hld\ Þár þat \alst{f}riðu-barn godes &
\alst{w}onoda an \alst{w}illjon, \hld\ ant-þat \alst{w}urd for·nam &
\alst{E}rodes þana kuning, \hld\ þat hé for·lét \alst{ę}ldjo barn, &
\alst{m}ódag \alst{m}anno drôm. \hld\ Þȯ skolda þero \alst{m}arka gi·wald &
\alst{ê}gan is \alst{ę}rvi-ward: \hld\ þe was \alst{A}rkheláus &
\alst{h}êtan, \alst{h}ęri-togo \hld\ \alst{h}elm-berandero: &
þe skolda umbi \alst{J}erusalem \hld\ \alst{J}udeono folkes, &
\alst{w}erodes gi·\alst{w}aldan. \hld\ Þȯ warð \alst{w}ord kuman &
þár an \alst{E}gypti \hld\ \alst{ę}ðiljun manne, &
þat hé þár te \alst{J}osepe, \hld\ \alst{g}odes ęngil sprak, &
\alst{b}odo drohtines, \hld\ hét ina eft þat \alst{b}arn þanan &
\alst{l}êdjen te \alst{l}ande. \hld\ „nu havað þit \alst{l}ioht af·geven“, kwað hé, &
„\alst{E}rodes þe kuning; \hld\ hé welde is \alst{á}htjen giu, &
\alst{f}rêson is \alst{f}erạhas. \hld\ Nu maht þú an \alst{f}riðu lêdjen &
þat \alst{k}ind undar ewa \alst{k}unni, \hld\ nu þe \alst{k}uning ni livod, &
\alst{e}rl \alst{o}var-módig.“ \hld\ \alst{A}l ant·kęnde &
\alst{J}osep \alst{g}odes têkạn: \hld\ \alst{g}ęriwide ina sniumo &
þe \alst{þ}egạn mit þera \alst{þ}iornun, \hld\ þȯ sie \alst{þ}anan weldun &
\alst{b}êðju mid þiu \alst{b}arnu: \hld\ lêstun þiu \alst{b}erhton gi·skapu, &
\alst{w}aldandes \alst{w}illjon, \hld\ al só hé im êr mid is \alst{w}ordun gi·bôd.\eva

\bvb TODO.\evb\evg

\bvg\bva[10][780]%
Gi·witun im þȯ eft an \alst{G}alilea-land \hld\ \alst{J}oseph ęndi Maria, &
\alst{h}êlag \alst{h}íwiski \hld\ \alst{h}evan-kuninges, &
wárun im an \alst{N}azareth-burg. \hld\ Þár þe \alst{n}ęrjondio Krist &
\alst{w}óhs undar þem \alst{w}erode, \hld\ warð gi·\alst{w}ittjes ful, &
\alst{a}n was imu \alst{a}nst godes, \hld\ hé was \alst{a}llun liof &
\alst{m}ódar-\alst{m}águn: \hld\ hé ni was ȯðrun \alst{m}annun gi·lík, &
þe \alst{g}umo an sínera \alst{g}ódi. \hld\ Þȯ hé \alst{g}ę́r-talo &
\alst{t}we-livi habde, \hld\ þȯ warð þiu \alst{t}íd kuman, &
þat sie þár te \alst{J}erusalem, \hld\ \alst{J}uðeo liudi &
iro \alst{þ}iod-gode \hld\ \alst{þ}ionon skoldun, &
\alst{w}irkjan is \alst{w}illjon. \hld\ Þȯ warð þár an þana \alst{w}íh innan &
þár te \alst{J}erusalem \hld\ \alst{J}udeono gi·samnod &
\alst{m}an-kraft \alst{m}ikil. \hld\ Þár \alst{M}aria was &
\alst{s}elf an gi·\alst{s}ïðja \hld\ ęndi iru \alst{s}unu habda, &
\alst{g}odes êgan barn. \hld\ Þȯ sie þat \alst{g}eld habdun, &
\alst{e}rlos an þem \alst{a}lạha, \hld\ só it an iro \alst{ê}wa gi·bôd, &
gi·\alst{l}êstid te iro \alst{l}and-wísun, \hld\ þȯ fórun im eft þie \alst{l}iudi þanan, &
\alst{w}eros an iro \alst{w}illjon \hld\ ęndi þár an þem \alst{w}íha af·stód &
\alst{m}ahtig barn godes, \hld\ só ina þiu \alst{m}ódar þár &
ni \alst{w}issa te \alst{w}áron; \hld\ ak siu wánda þat hé mid þem \alst{w}eroda forð, &
\alst{f}óri mit iro \alst{f}riundun. \hld\ Gi·\alst{f}rang aftar þiu &
eft an \alst{ȯ}ðrun daga \hld\ \alst{a}ðal-kunnjes wíf, &
\alst{s}álig þiorna, \hld\ þat hé undar þem gi·\alst{s}ïðja ni was. &
warð \alst{M}ariun þȯ \hld\ \alst{m}ód an sorgun, &
\alst{h}riwig umbi iro \alst{h}erta, \hld\ þȯ siu þat \alst{h}êlaga barn &
ni \alst{f}and undar þem \alst{f}olka: \hld\ \alst{f}ilu gornoda &
þiu \alst{g}odes þiorna. \hld\ Gi·witun im þȯ eft te \alst{J}erusalem &
iro \alst{s}unu \alst{s}ókjan, \hld\ fundun ina \alst{s}ittjan þár &
an þem \alst{w}íha innan, \hld\ þár þe \alst{w}ísa man, &
swíðo \alst{g}lauwa \alst{g}umon \hld\ an \alst{g}odes êwa &
\alst{l}ásun ęnde \alst{l}ínodun, \hld\ hwó sie \alst{l}of skoldin &%NOTE: the ms spelling <ende> for <endi> only occurs here and 38 lines below
\alst{w}irkjan mid iro \alst{w}ordun þem, \hld\ þe þesa \alst{w}er-old gi·skóp. &
Þár sat undar \alst{m}iddjun \hld\ \alst{m}ahtig barn godes, &
\alst{K}rist alo-waldo, \hld\ só is þea ni mahtun ant·\alst{k}ęnnjan wiht, &
þe þes \alst{w}íhes þár \hld\ \alst{w}ardon skoldun, &
ęndi \alst{f}rágoda sie \hld\ \alst{f}iri-wit-líko &
\alst{w}ísera \alst{w}ordo. \hld\ Sie \alst{w}undradun alle, &
bu-hwí gio só \alst{k}indisk man \hld\ su·lika \alst{k}widi mahti &
mid is \alst{m}u̇ðu gi·\alst{m}ênjan. \hld\ Þár ina þiu \alst{m}ódar fand &
\alst{s}ittjan under þem gi·\alst{s}ïðja \hld\ ęndi iro \alst{s}unu grótta, &
\alst{w}ísan undar þem \alst{w}eroda, \hld\ sprak im mid ira \alst{w}ordun tó: &
„hwí weldes þú þínera \alst{m}ódar, \hld\ \alst{m}anno liovosto, &
gi·\alst{s}idon su·lika \alst{s}orga, \hld\ þat ik þi só \alst{s}êrag-mód, &%TODO:Check "gi·sidon"
\alst{i}dis \alst{a}rm-hugdig \hld\ \alst{ê}skon skolda &
undar þesun \alst{b}urg-liudjun?“ \hld\ Þȯ sprak iru eft þat \alst{b}arn an·gęgin &
\alst{w}ísun \alst{w}ordun: \hld\ „Hwat þú \alst{w}êst garo“, kwað hé, &
„þat ik þár gi·\alst{r}ísu, \hld\ þár ik bi \alst{r}ehton skal &
\alst{w}onon an \alst{w}illjon, \hld\ þár gi·\alst{w}ald havad &%TODO: Check "wonon"
\alst{m}ín \alst{m}ahtig fader.“ \hld\ Þie \alst{m}an ni for·stódun, &
þie \alst{w}eros an þem \alst{w}íha, \hld\ bi·hwí hé só þat \alst{w}ord gi·sprak, &
gi·\alst{m}ênda mid is \alst{m}u̇ðu: \hld\ \alst{M}aria al bi·held, &
gi·\alst{b}arg an ira \alst{b}reostun, \hld\ só hwat só siu gi·hôrda ira \alst{b}arn sprekan &
\alst{w}isaro \alst{w}ordo. \hld\ Gi·\alst{w}itun im þȯ eft þanan &
fon \alst{J}erusalem \hld\ \alst{J}oseph ęndi Maria, &
habdun im te gi·\alst{s}ïðja \hld\ \alst{s}unu drohtines, &
allaro \alst{b}arno \alst{b}ętsta, \hld\ þero þe io gi·\alst{b}oran wurði &
\alst{m}agu fon \alst{m}ódar: \hld\ habdun im þár \alst{m}innja tó &
þurh \alst{h}luttran \alst{h}ugi, \hld\ ęndi hé só gi·\alst{h}ôrig was, &
\alst{g}odes êgan barn \hld\ \alst{g}aduling-mágun &
þurh is \alst{ô}d-módi, \hld\ \alst{a}ldron sínun: &
ni welda an is \alst{k}indiski þȯ noh \hld\ is \alst{k}raft mikil &
\alst{m}annun \alst{m}árjan, \hld\ þat hé su·lik \alst{m}ęgin êhta, &
gi·\alst{w}ald an þesaro \alst{w}er-oldi, \hld\ ak hé im an is \alst{w}illjon bêd &
gi·\alst{þ}iudo undar þero \alst{þ}iodu \hld\ \alst{þ}rí-tig gę́ro, &
êr þan hé þár \alst{t}êkạn ênig \hld\ \alst{t}ôgjan weldi, &
\alst{s}ęggjan þem gi·\alst{s}ïðja, \hld\ þat hé \alst{s}elvo was &
an þesaro \alst{m}iddil-gard \hld\ \alst{m}anno drohtin. &
\alst{H}abda im só bi·\alst{h}alden \hld\ \alst{h}êlag barn godes &
\alst{w}ord ęndi \alst{w}ís-dóm \hld\ ęnde allaro gi·\alst{w}ittjo mêst, &
tulgo \alst{sp}áhan hugi: \hld\ ni mahta man is an is \alst{sp}rákun werðan, &
an is \alst{w}ordun gi·\alst{w}ar, \hld\ þat hé su·lik gi·\alst{w}it êhta, &
\alst{þ}egạn su·lika gi·\alst{þ}ȧhti, \hld\ ak hé im só gi·\alst{þ}iudo bêd &
\alst{t}orhtaro \alst{t}êkno. \hld\ Ni was noh þan þiu \alst{t}íd kuman, &
þat hé ina ovar þesan \alst{m}iddil-gard \hld\ \alst{m}árjan skolda, &
\alst{l}êrjan þie \alst{l}iudi, \hld\ hwó sie skoldin iro gi·\alst{l}ôvon haldan, &
\alst{w}irkjan \alst{w}illjon godes; \hld\ \alst{w}issun þat þoh managa &
\alst{l}iudi aftar þem \alst{l}anda, \hld\ þat hé was an þit \alst{l}ioht kuman, &
þoh sie ina \alst{k}u̇ð-líko \hld\ an·\alst{k}ęnnjan ni mahtin, &
êr þan hé ina \alst{s}elvo \hld\ \alst{s}ęggjan welda.\eva

\bvb TODO.\evb\evg

\bvg\bva[11][859]%
Þan was im \alst{J}ohannes \hld\ fon is \alst{j}uguð-hêdi &
a·\alst{w}ahsan an ênero \alst{w}óstunni; \hld\ þár ni was \alst{w}erodes þan mêr, &
b·útan þat hé þár \alst{ê}n-kora \hld\ \alst{a}lo-waldon gode, &
\alst{þ}egạn \alst{þ}ionoda: \hld\ for·lét \alst{þ}ioda gi·mang, &
\alst{m}anno gi·\alst{m}ênðon. \hld\ Þár warð im \alst{m}ahtig kuman &
an þero \alst{w}óstunni \hld\ \alst{w}ord fon himila, &
\alst{g}ód-lík stemna \alst{g}odes, \hld\ ęndi \alst{J}ohanne gi·bod, &
þat hé \alst{K}ristes \alst{k}umi \hld\ ęndi is \alst{k}raft mikil &
ovar þesan \alst{m}iddil-gard \hld\ \alst{m}árjan skoldi; &
hét ina \alst{w}ár-líko \hld\ \alst{w}ordun sęggjan, &
þat wári \alst{h}evan-ríki \hld\ \alst{h}ęliðo barnun &
an þem \alst{l}and-skępi, \hld\ \alst{l}iudjun gi·náhid, &
\alst{w}elono \alst{w}un-samost. \hld\ Im was þȯ \alst{w}illjo mikil, &
þat hé fon \alst{s}u·likun \alst{s}áldun \hld\ \alst{s}ęggjan mósti. &
Gi·wêt im þȯ \alst{g}angan, \hld\ al só \alst{J}ordan flót, &
\alst{w}atar an \alst{w}illjon, \hld\ ęndi þem \alst{w}eroda allan dag, &
aftar þem \alst{l}and-skępi \hld\ þem \alst{l}iudjun ku̇ðda, &
þat sie mid \alst{f}astunnju \hld\ \alst{f}irin-werk manag, &
iro \alst{s}elvoro \hld\ \alst{s}undja bóttin, &
„þat gí werðan \alst{h}rênja“, \hld\ kwað hé. „\alst{H}evan-ríki is &
gi·náhid \alst{m}anno barnun. \hld\ Nu látad eu an ewan \alst{m}ód-sevon &
ewar \alst{s}elvoro \hld\ \alst{s}undja hrewan, &
\alst{l}êdas þat gí an þesun \alst{l}iohta fręmidun, \hld\ ęndi mínun \alst{l}êrun hôrjad, &
\alst{w}ęndjat aftar mínun \alst{w}ordun. \hld\ Ik eu an \alst{w}atara skal &
gi·\alst{d}ôpjan \alst{d}iur-líko, \hld\ þoh ik ewa \alst{d}ádi ne mugi, &
ewar \alst{s}elvaro \hld\ \alst{s}undja a·látan, &
þat gí þurh mín \alst{h}and-gi·werk \hld\ \alst{h}luttra werðan &
\alst{l}êðaro gi·\alst{l}êsto: \hld\ ak þe is an þit \alst{l}ioht kuman, &
\alst{m}ahtig te \alst{m}annun \hld\ ęndi undar eu \alst{m}iddjun stéd, &
—þoh gí ina \alst{s}elvun \hld\ gi·\alst{s}ehan ni willjan—, &
þe eu gi·\alst{d}ôpjan skal \hld\ an ewes \alst{d}rohtines namon &
an þana \alst{h}âlagon gêst. \hld\ Þat is \alst{h}êrro ovar al: &
hé mag allaro \alst{m}anno gi·hwena \hld\ \alst{m}ên-gi·þȧhtjo, &
\alst{s}undjono \alst{s}ikoron, \hld\ só hwene só só \alst{s}álig mót &
\alst{w}erðen an þesaro \alst{w}er-oldi, \hld\ þat þes \alst{w}illjon havad, &
þat hé só gi·\alst{l}êstja, \hld\ só hé þesun \alst{l}iudjun wili, &
gi·\alst{b}ioden \alst{b}arn godes. \hld\ Ik bium an is \alst{b}od-skępi herod &
an þesa \alst{w}er-old kumen \hld\ ęndi skal im þana \alst{w}eg rúmjen, &
\alst{l}êrjan þesa \alst{l}iudi, \hld\ hwó sea skulin iro gi·\alst{l}ôvon haldan &
þurh \alst{h}luttran \alst{h}ugi, \hld\ ęndi þat sie an \alst{h}ęllja ni þurvin, &
\alst{f}aran an \alst{f}ern þat hêta. \hld\ Þes wirðid só \alst{f}agan an is móde &
man te só \alst{m}anagaro stundu, \hld\ só hwe só þat \alst{m}ên for·látid, &
\alst{g}erno þes \alst{g}ramon an-busni, \hld\ —só mag im þes \alst{g}ódon gi·wirkjan, &
\alst{h}uldi \alst{h}evan-kuninges,— \hld\ só hwe só havad \alst{h}luttra trewa &
up te þem \alst{a}lo-mahtigon gode.“ \hld\ \alst{E}rlos managa &
bi þem \alst{l}êrun þȯ, \hld\ \alst{l}iudi wándun, &
\alst{w}eros \alst{w}ár-líko, \hld\ þat þat \alst{w}aldand Krist &
\alst{s}elbo wári, \hld\ hwanda hé só filu \alst{s}ȯðes gi·sprak, &
\alst{w}ároro \alst{w}ordo. \hld\ Þȯ warð þat só \alst{w}ído ku̇ð &
ovar þat for·\alst{g}evana land \hld\ \alst{g}umono gi·hwi-likum, &
\alst{s}ęggjun at iro \alst{s}elðun: \hld\ þȯ kwámun ina \alst{s}ókjan þarod &
fon \alst{J}erusalem \hld\ \alst{J}udeo liudjo &
\alst{b}odon fon þeru \alst{b}urgi \hld\ ęndi frágodun, ef hé wári þat \alst{b}arn godes, &
„þat hér \alst{l}ango giu“, \hld\ kwaðun sie, „\alst{l}iudi sagdun, &
\alst{w}eros \alst{w}ár-líko, \hld\ þat hé skoldi an þesa \alst{w}er-old kuman“. &
\alst{J}ohannes þȯ gi·mahạlde \hld\ ęndi te·\alst{g}ęgnes sprak &
þem \alst{b}odun \alst{b}ald-líko: \hld\ „ni bium ik“, kwað hé, „þat \alst{b}arn godes, &
\alst{w}ár \alst{w}aldand Krist, \hld\ ak ik skal im þana \alst{w}eg rúmjen, &
\alst{h}êrron \alst{m}ínumu.“ \hld\ Þea \alst{h}ęliðos frugnun, &
þea þár an þem \alst{â}rundje \hld\ \alst{e}rlos wárun, &
\alst{b}odon fon þero \alst{b}urgi: \hld\ „ef þú nú ni bist þat \alst{b}arn godes, &
bist þú þan þoh \alst{E}lias, \hld\ þe hér an \alst{ê}r-dagun &
\alst{w}as undar þesumu \alst{w}erode? \hld\ hé is \alst{w}is-kumo &
eft an þesan \alst{m}iddil-gard. \hld\ Saga u̇s hwat þú \alst{m}anno sís! &
Bist þú \alst{ê}nig þero, \hld\ þe hér \alst{ê}r wári &
\alst{w}ísaro \alst{w}ár-saguno? \hld\ Hwat skulun wí þem \alst{w}erode fon þi &
\alst{s}ęggjan te \alst{s}ȯðon? \hld\ Neo hér êr \alst{s}u·lik ni warð &
an þesun \alst{m}iddil-gard \hld\ \alst{m}an ȯðar kuman &
\alst{d}ádjun só mári. \hld\ Bi·hwí þú hér \alst{d}ôpisli &
\alst{f}ręmis undar þesumu \alst{f}olke, \hld\ ef þú þaro \alst{f}ora·sagono &
\alst{ê}n-hwi-lik ni bist?“ \hld\ Þȯ habde \alst{e}ft garo &
\alst{J}ohannes þe \alst{g}ódo \hld\ glau \alst{a}nd-wordi: &
„Ik bium \alst{f}ora-bodo \hld\ \alst{f}râon mínes, &
\alst{l}ioves hêrron; \hld\ ik skal þit \alst{l}and rekon, &
þit \alst{w}erod aftar is \alst{w}illjon. \hld\ Ik hębbju fon is \alst{w}orde mid mí &
\alst{st}ranga \alst{st}emna, \hld\ þoh sie hér ni willje for·\alst{st}andan filo &
\alst{w}erodes an þesaro \alst{w}óstunni. \hld\ Ni bium ik mid \alst{w}ihti gi·lík &
\alst{d}rohtine mínumu: \hld\ hé is mid is \alst{d}ádjun só strang, &
só \alst{m}ári ęndi só \alst{m}ahtig \hld\ —þat wirðid \alst{m}anagun ku̇ð, &
\alst{w}erun aftar þesaro \alst{w}er-oldi— \hld\ þat ik þes \alst{w}irðig ni bium, &
þat ik móti an is gi·\alst{sk}uoha, \hld\ þoh ik sí is \alst{sk}alk êgan, &
an só \alst{r}íkjumu drohtine, \hld\ þea \alst{r}eomon ant·bindan: &
só mikilu is hé \alst{b}ętara þan ik. \hld\ Nis þes \alst{b}odon gi·mako &
\alst{ê}nig ovar \alst{e}rðu, \hld\ ne nu \alst{a}ftar ni skal &
\alst{w}erðan an þesaro \alst{w}er-oldi. \hld\ Hębbjad ewan \alst{w}illjon þarod, &
\alst{l}iudi ewan gi·\alst{l}ôvon: \hld\ þan eu \alst{l}ango skal &
wesan ewa \alst{h}ugi \alst{h}rómag; \hld\ þan gi \alst{h}ęlli-gi·þwing, &
for·\alst{l}átad \alst{l}êðaro drôm \hld\ ęndi sókjad eu \alst{l}ioht godes, &
\alst{u}p-\alst{ô}des hêm, \hld\ \alst{ê}wig ríki, &
\alst{h}ôhan \alst{h}evan-wang. \hld\ Ne látad ewan \alst{h}ugi twífljen!“\eva

\bvb TODO.\evb\evg

\bvg\bva[12][949]%
Só sprak þȯ \alst{j}ung \alst{g}umo \hld\ bi \alst{g}odes lêrun &
\alst{m}annun te \alst{m}árðu. \hld\ \alst{M}anag samnoda &
þár te \alst{B}ethania \hld\ \alst{b}arn Israheles; &
\alst{k}wámun þár te Johannese \hld\ \alst{k}uningo gi·sïðos, &
\alst{l}iudi te \alst{l}êrun \hld\ ęndi iro gi·\alst{l}ôvon ant·féngun. &
Hé \alst{d}ôpte sie \alst{d}ago gi·hwi-likes \hld\ ęndi im iro \alst{d}ádi lóg, &%TODO: check lóg
\alst{w}rêðaro \alst{w}illjon, \hld\ ęndi lovode im \alst{w}ord godes, &
\alst{h}êrron sínes: \hld\ „\alst{h}evan-ríki wirðid“, kwað hé, &
„\alst{g}aru \alst{g}umono só hwem, \hld\ só ti \alst{g}ode þęnkid &
ęndi an þana \alst{h}êljand *wili \hld\ \alst{h}luttro gi·lôvjan, &%NOTE ms. -- wili] P 1r.
\alst{l}êstjan is \alst{l}êra“. \hld\ Þȯ ni was \alst{l}ang te þiu, &
þat im fon \alst{G}alilea gi·wêt \hld\ \alst{g}odes êgan barn, &
*\alst{d}iur-lík \alst{d}rohtines sunu, \hld\ \alst{d}ôpi suokjan. &
\alst{w}as im þuȯ an is \alst{w}astme \hld\ \alst{w}aldandes barn*, &
al só hé mid þero \alst{þ}iodu \hld\ \alst{þ}rí-tig habdi &
\alst{w}intro an is \alst{w}er-oldi. \hld\ Þȯ hé an is \alst{w}illjon kwam, &
þár \alst{J}ohannes \hld\ an \alst{J}ordana strôme &
allan \alst{l}angan dag \hld\ \alst{l}iudi manage &
\alst{d}ôpte \alst{d}iur-líko. \hld\ Reht só hé þȯ is \alst{d}rohtin gi·sah, &
\alst{h}oldan \alst{h}êrron, \hld\ só warð im is \alst{h}ugi blíði, &
þes im þe \alst{w}illjo gi·stód, \hld\ ęndi sprak im þȯ mid is \alst{w}ordun tó, &
swíðo \alst{g}ód \alst{g}umo, \hld\ \alst{J}ohannes te Kriste: &
„nu kumis þú te mínero \alst{d}ôpi, \hld\ \alst{d}rohtin frô mín, &
\alst{þ}iod-gumono bętsto: \hld\ só skolde ik te \alst{þ}ínero duan, &
hwand þú bist allaro \alst{k}uningo \alst{k}raftigost.“ \hld\ \alst{K}rist selvo gi·bôd, &
\alst{w}aldand \alst{w}ár-líko, \hld\ þat hé ni spráki þero \alst{w}ordo þan mêr: &
„wêst þú, þat u̇s só gi·\alst{r}ísid“, \hld\ kwað hé, „allaro \alst{r}ehto gi·hwi-lik &
te gi·\alst{f}ulljanne \hld\ \alst{f}orð-wardes nu &
an \alst{g}odes willjon“. \hld\ \alst{J}ohannes stód, &
\alst{d}ôpte allan \alst{d}ag \hld\ \alst{d}ruht-folk mikil, &
\alst{w}erod an \alst{w}atere \hld\ ęndi ôk \alst{w}aldand Krist, &
\alst{h}êran \alst{h}evan-kuning \hld\ \alst{h}andun sínun &
an allaro \alst{b}aðo þem \alst{b}ętston \hld\ ęndi im þár te \alst{b}edu gi·hnêg &
an \alst{k}neo \alst{k}raftag. \hld\ \alst{K}rist up gi·wêt &
\alst{f}agạr fon þem \alst{f}lóde, \hld\ \alst{f}riðu-barn godes, &
\alst{l}iof \alst{l}iudjo ward. \hld\ Só hé þȯ þat \alst{l}and af·stóp, &%TODO: check af·stóp
só ant·\alst{h}lidun þȯ \alst{h}imiles doru, \hld\ ęndi kwam þe \alst{h}êlago gêst &
fon þem \alst{a}lo-waldon \hld\ \alst{o}vane te Kriste: &
—was im an gi·\alst{l}ík-nissje \hld\ \alst{l}ungras fugles, &
\alst{d}iur-líkara \alst{d}úvun— \hld\ ęndi sat im uppan u̇ses \alst{d}rohtines ahslu, &
\alst{w}onoda im ovar þem \alst{w}aldandes barne. \hld\ Aftar kwam þár \alst{w}ord fon himile, &
\alst{h}lúd fon þem \alst{h}ôhon radura \hld\ ęndi grótta þane \alst{h}êljand selvon, &
\alst{K}rista, allaro \alst{k}uningo bętston, \hld\ kwað þat hé ina gi·\alst{k}orana habdi &
\alst{s}elvo fon \alst{s}ínun ríkja, \hld\ kwað þat im þe \alst{s}unu líkodi &
\alst{b}ętst allaro gi·\alst{b}oranaro manno, \hld\ kwað þat hé im wári allaro \alst{b}arno liovost. &
Þat móste \alst{J}ohannes þȯ, \hld\ al só it \alst{g}od welde, &
gi·\alst{s}ehan ęndi gi·hôrjan. \hld\ hé gi·deda it \alst{s}án aftar þiu &
\alst{m}annun \alst{m}ári, \hld\ þat sie þár \alst{m}ahtigna &
\alst{h}êrron \alst{h}abdun: \hld\ „Þit is“, kwað hé, „\alst{h}evan-kuninges sunu, &
\alst{ê}n \alst{a}lo-waldand: \hld\ þesas willjo ik \alst{u}r-kundjo &
\alst{w}esan an þesaro \alst{w}er-oldi, \hld\ hwand it sagda mí \alst{w}ord godes, &
\alst{d}rohtines stemne, \hld\ þȯ hé mi \alst{d}ôpjan hét &
\alst{w}eros an \alst{w}atare, \hld\ só hwar só ik gi·sáwi \alst{w}ár-líko &
þana \alst{h}êlagon gêst \hld\ *fan \alst{h}evan-wange &
an þesan \alst{m}iddil-gard \hld\ ênigan \alst{m}an waron, &
\alst{k}uman mid \alst{k}raftu; \hld\ þat kwað, þat skoldi \alst{K}rist wesan, &
\alst{d}iur-lík \alst{d}rohtines suno. \hld\ Hie \alst{d}ôpjan skal &
an þana \alst{h}êlagan gêst \hld\ ęndi \alst{h}êljan managa &%NOTE ms. -- þana] P end.
\alst{m}anno \alst{m}ên-dádi. \hld\ hé havad \alst{m}aht fon gode, &
þat hé a·\alst{l}átan mag \hld\ \alst{l}iudjo gi·hwi-likun &
\alst{s}aka ęndi \alst{s}undja. \hld\ Þit is \alst{s}elvo Krist, &
\alst{g}odes êgan barn, \hld\ \alst{g}umono bętsto, &
\alst{f}riðu wið \alst{f}íundun. \hld\ Wala þat eu þes mag \alst{f}râh-mód hugi &
\alst{w}esan an þesaro \alst{w}er-oldi, \hld\ þes eu þe \alst{w}illjo gi·stód, &
þat gí só \alst{l}ibbjanda \hld\ þana \alst{l}andes ward &
\alst{s}elvon gi·\alst{s}áhun. \hld\ Ní mót sliumo \alst{s}undjono lôs &
manag \alst{g}êst faran \hld\ an \alst{g}odes willjon &
\alst{t}ionon a·\alst{t}ómid, \hld\ þe mid \alst{t}rewon wili &%TODO: check a·tómid
wið is \alst{w}ini \alst{w}irkjan \hld\ ęndi an \alst{w}aldand Krist &
\alst{f}asto gi·lôvjan. \hld\ Þat skal te \alst{f}rumun werðen &
\alst{g}umono só hwi-likun, \hld\ só þat \alst{g}erno dót“.\eva

\bvb TODO.\evb\evg

\bvg\bva[13][1020]%
Só ge·fragn ik þat \alst{J}ohannes þȯ \hld\ \alst{g}umono gi·hwi-likun, &
\alst{l}ovoda þem \alst{l}iudjun \hld\ \alst{l}êra Kristes, &
\alst{h}êrron sínes, \hld\ ęndi \alst{h}evan-ríki &
te gi·\alst{w}innanne, \hld\ \alst{w}elono þane mêston, &
\alst{s}álig \alst{s}in-líf. \hld\ Þȯ hé im \alst{s}elvo gi·wêt &
aftar þem \alst{d}ôpislja, \hld\ \alst{d}rohtin þe gódo, &
an êna \alst{w}óstunnja, \hld\ \alst{w}aldandes sunu; &
was im þár an þero \alst{ê}n-\alst{ô}di \hld\ \alst{e}rlo drohtin &
\alst{l}ange hwíla; \hld\ ne habda \alst{l}iudjo þan mêr, &
\alst{s}ęggjo te gi·\alst{s}ïðun, \hld\ al só hé im \alst{s}elvo gi·kôs: &
welda is þár látan \alst{k}oston \hld\ \alst{k}raftiga wihti, &
\alst{s}elvon \alst{S}atanasan, \hld\ þe gio an \alst{s}undja spęnit, &
\alst{m}an an \alst{m}ên-werk: \hld\ hé konsta is \alst{m}ód-sevon, &
\alst{w}rêðan \alst{w}illjon, \hld\ hwó hé þesa \alst{w}er-old êrist, &
an þem \alst{a}n-ginnja \hld\ \alst{i}rmin-þioda &
bi·\alst{s}wêk mit \alst{s}undjun, \hld\ þȯ hé þiu \alst{s}in-híun twê, &
\alst{Á}daman ęndi \alst{É}wan, \hld\ þurh \alst{u}n-trewa &
for·\alst{l}êdda mid \alst{l}uginun, \hld\ þat \alst{l}iudo barn &
aftar iro \alst{h}in-fęrdi \hld\ \alst{h}ęllja sóhtun, &
\alst{g}umono \alst{g}êstos. \hld\ Þȯ welda þat \alst{g}od mahtig, &
\alst{w}aldand \alst{w}ęndjan \hld\ ęndi welda þesum \alst{w}erode for·geven &
\alst{h}ôh \alst{h}imil-ríki: \hld\ be·þiu hé herod \alst{h}êlagna bodon, &
is \alst{s}unu \alst{s}ęnda. \hld\ Þat was \alst{S}atanase &
tulgo \alst{h}arm an is \alst{h}ugi: \hld\ afonsta \alst{h}evan-ríkjes &
\alst{m}anno kunnje: \hld\ welda þȯ \alst{m}ahtigna &
mid þem \alst{s}elvon \alst{s}akun \hld\ \alst{s}unu drohtines, &
þem hé \alst{Á}daman \hld\ an \alst{ê}r-dagun &
\alst{d}arnungo bi·\alst{d}róg, \hld\ þat hé warð is \alst{d}rohtine lêð, &
bi·\alst{s}wêk ina mid \alst{s}undjun \hld\ —só welda hé þȯ \alst{s}elvan dón &
\alst{h}êlandjan Krist. \hld\ Þan habda hé is \alst{h}ugi fasto &
wið þana \alst{w}am-skaðon, \hld\ \alst{w}aldandes barn, &
\alst{h}erte só gi·\alst{h}ęrdid: \hld\ welda \alst{h}evan-ríki &
\alst{l}iudjun gi·\alst{l}êstjan. \hld\ Was im þes \alst{l}andes ward &
an \alst{f}astunnja \hld\ \alst{f}ior-tig nahto, &
\alst{m}anno drohtin, \hld\ só hé þár \alst{m}ates ni ant·bêt; &
þan langa ni gi·\alst{d}orstun \hld\ im \alst{d}ęrnja wihti, &
\alst{n}íð-hugdig fíund, \hld\ \alst{n}áhor gangan, &
\alst{g}rótjan ina \alst{g}ęgin-warðan: \hld\ wánde þat hé \alst{g}od ên-fald, &
for·útar \alst{m}an-kunnjes wiht \hld\ \alst{m}ahtig wári, &
\alst{h}êleg \alst{h}imiles ward. \hld\ Só hé ina þȯ ge·\alst{h}ungrjan lét, &
þat ina bi·gan bi þero \alst{m}ęnnisko \hld\ \alst{m}óses lustjan &
aftar þem \alst{f}iuwar-tig dagun, \hld\ þe \alst{f}íund náhor géng, &
\alst{m}irki \alst{m}ên-skaðo: \hld\ wánda þat hé \alst{m}an ên-fald &
\alst{w}ári \alst{w}issungo, \hld\ sprak im þȯ mid is \alst{w}ordun tó, &
\alst{g}rótta ina þe \alst{g}êr-fíund: \hld\ „ef þú sís \alst{g}odes sunu“, kwað hé, &
„be·hwí ni hêtis þú þan \alst{w}erðan, \hld\ ef þú gi·\alst{w}ald haves, &
allaro \alst{b}arno \alst{b}ętst, \hld\ \alst{b}rôd af þesun stênun? &
Ge·\alst{h}êli þínna \alst{h}ungạr!“ \hld\ Þȯ sprak eft þe \alst{h}êlago Krist: &
„ni mugun \alst{ę}ldi-barn“, \hld\ kwað hé, „\alst{ê}n-faldes brôdes, &
\alst{l}iudi \alst{l}ibbjen, \hld\ ak sie skulun þurh \alst{l}êra godes &
\alst{w}esan an þesero \alst{w}er-oldi \hld\ ęndi skulun þiu \alst{w}erk frummjen, &
þea þár werðad a·\alst{h}lúdid \hld\ fon þero \alst{h}êlogun tungun, &
fon þem \alst{g}alme \alst{g}odes: \hld\ þat is \alst{g}umono líf &
\alst{l}iudjo só hwi-likon, \hld\ só þat \alst{l}êstjan wili, &
þat fon \alst{w}aldandes \hld\ \alst{w}orde ge·biudid.“ &
Þȯ bi·gan eft \alst{n}iuson \hld\ ęndi \alst{n}áhor géng &
\alst{u}n-hiuri fíund \hld\ \alst{ȯ}ðru sïðu, &
\alst{f}andoda is \alst{f}rôhan. \hld\ Þat \alst{f}riðu-barn þolode &
\alst{w}rêðes \alst{w}illjon \hld\ ęndi im gi·\alst{w}ald for·gaf, &
þat hé umbi is \alst{k}raft mikil \hld\ \alst{k}oston mósti, &
\alst{l}ét ina þȯ \alst{l}êdjan \hld\ þana \alst{l}iud-skaðon, &
þat hé ina an \alst{J}erusalem \hld\ te þem \alst{g}odes wíha, &
\alst{a}lles \alst{o}van-wardan, \hld\ \alst{u}p gi·sętta &
an allaro \alst{h}úso \alst{h}ôhost, \hld\ ęndi \alst{h}osk-wordun sprak, &
þe \alst{g}ramo þurh \alst{g}elp mikil: \hld\ „ef þú sís \alst{g}odes sunu“, kwað hé, &
„\alst{sk}ríd þi te erðu hinan. \hld\ Ge·\alst{sk}rivan was it giu lango, &
an \alst{b}ókun ge·writen, \hld\ hwó gi·\alst{b}oden havad &
is \alst{ę}ngilun \hld\ \alst{a}lo-mahtig fader, &
þat sie þi at \alst{w}ege ge·hwem \hld\ \alst{w}ardos sinðun, &
\alst{h}aldad þi undar iro \alst{h}andun. \hld\ Hwat þú \alst{h}wargin ni þarft &
mid þínun \alst{f}ótun \hld\ an \alst{f}elis be·spurnan, &
an \alst{h}ardan stên.“ \hld\ Þȯ sprak eft þe \alst{h}êlago Krist, &
allaro \alst{b}arno \alst{b}ętst: \hld\ „só is ôk an \alst{b}ókun ge·skrivan“, kwað hé, &
„þat þú te \alst{h}ardo ni skalt \hld\ \alst{h}êrran þínes, &
\alst{f}andon þínes \alst{f}rôhan: \hld\ þat nis þí allaro \alst{f}rumono neg·ên.“ &
Lét ina þȯ an þana \alst{þ}riddjan sïð \hld\ þana \alst{þ}iod-skaðon &
gi·\alst{b}rengen uppan ênan \alst{b}erg þen hôhon: \hld\ þár ina þe \alst{b}alo-wíso &
lét \alst{a}l \alst{o}var-sehan \hld\ \alst{i}rmin-þiode, &
\alst{w}onod-saman \alst{w}elon \hld\ ęndi \alst{w}er-old-ríki &
ęndi all su·lik \alst{ô}des, \hld\ só þius \alst{e}rða bi·havad &
\alst{f}agọroro \alst{f}rumono, \hld\ ęndi sprak im þȯ þe \alst{f}íund an·gęgin, &
kwað þat hé im þat al só \alst{g}ód-lík \hld\ for·\alst{g}even weldi, &
\alst{h}ôha \alst{h}ęri-dómos, \hld\ „ef þú wilt \alst{h}nígan te mí, &
\alst{f}allan te mínun \alst{f}ótun \hld\ ęndi mí for \alst{f}rôhan havas, &
\alst{b}edos te mínun \alst{b}arma. \hld\ Þan látu ik þí \alst{b}rúkan wel &
\alst{a}lles þes \alst{ô}d-welon, \hld\ þes ik þí hębbju gi·\alst{ô}git hír.“ &
Þȯ ni welda þes \alst{l}êðan word \hld\ \alst{l}ęngeron hwíle &
\alst{h}ôrjan þe \alst{h}êlago Krist, \hld\ ak hé ina fon is \alst{h}uldi for·drêf, &
\alst{S}atanasan for·\alst{s}wêp, \hld\ ęndi \alst{s}án aftar sprak &
allaro \alst{b}arno \alst{b}ętst, \hld\ kwað þat man \alst{b}edon skoldi &
\alst{u}p te þem \alst{a}lo-mahtigon gode \hld\ ęndi im \alst{ê}num þionon &
swíðo \alst{þ}io-liko \hld\ \alst{þ}egnos managa, &
\alst{h}ęliðos aftar is \alst{h}uldi: \hld\ „þár ist þiu \alst{h}elpa ge·lang &
\alst{m}anno ge·hwi-likun.“ \hld\ Þȯ gi·wêt im þe \alst{m}ên-skaðo, &
\alst{s}wíðo \alst{s}êrag-mód \hld\ \alst{S}atanas þanan, &
\alst{f}íund undar \alst{f}ern-dalu. \hld\ Warð þár \alst{f}olk mikil &
fon þem \alst{a}lo-waldan \hld\ \alst{o}vana te Kriste &
\alst{g}odes ęngilo kumen, \hld\ þie im sïðor \alst{j}ungar-dóm, &
skoldun \alst{a}mbaht-skępi \hld\ \alst{a}ftar lêstjen, &
\alst{þ}ionon \alst{þ}io-líko: \hld\ só skal man \alst{þ}iod-gode, &
\alst{h}êrron aftar \alst{h}uldi, \hld\ \alst{h}evan-kuninge.\eva

\bvb TODO.\evb\evg

\bvg\bva[14][1121]%
Was im an þem \alst{s}in-węldi \hld\ \alst{s}álig barn godes &
\alst{l}ange hwíle, \hld\ unt-þat im þȯ \alst{l}iovora warð, &
þat hé is \alst{k}raft mikil \hld\ \alst{k}u̇ðjen wolda &
\alst{w}eroda te \alst{w}illjon. \hld\ Þȯ for·lét hé \alst{w}aldes hleo, &%NOTE: Not hlêo.
\alst{ê}n-ôdjes \alst{a}rd \hld\ ęndi sóhte im eft \alst{e}rlo ge·mang, &
\alst{m}ári \alst{m}ęgin-þiode \hld\ ęndi \alst{m}anno drôm, &
géng im þȯ bi \alst{J}ordanes staðe: \hld\ þár ina \alst{J}ohannes ant·fand, &
þat \alst{f}riðu-barn godes, \hld\ \alst{f}rôhan sínan, &
\alst{h}êlagana \alst{h}evan-kuning, \hld\ ęndi þem \alst{h}ęliðun sagda, &
\alst{J}ohannes is \alst{j}ungurun, \hld\ þȯ hé ina \alst{g}angan ge·sah: &
„þit is þat \alst{l}amb godes, \hld\ þat þár \alst{l}ôsjan skal &
af þesaro \alst{w}ídon \alst{w}er-old \hld\ \alst{w}rêða sundja, &
\alst{m}an-kunnjas \alst{m}ên, \hld\ \alst{m}ári drohtin, &
\alst{k}uningo \alst{k}raftigost.“ \hld\ \alst{K}rist im forð gi·wêt &
an \alst{G}alileo land, \hld\ \alst{g}odes êgan barn, &
\alst{f}ór im te þem \alst{f}riundun, \hld\ þár hé a·\alst{f}ódit was, &
\alst{t}ír-líko a·\alst{t}ogan, \hld\ ęndi \alst{t}alda mid wordun &
\alst{K}rist undar is \alst{k}unnje, \hld\ \alst{k}uningo ríkjost, &
hwó sie \alst{s}koldin iro \alst{s}elvoro \hld\ \alst{s}undja bótjan, &
hét þat sie im iro \alst{h}arm-werk manag \hld\ \alst{h}rewan létin, &
\alst{f}eldin iro \alst{f}irin-dádi: \hld\ „nu is it all ge·\alst{f}ullot só, &
só hír \alst{a}lde man \hld\ \alst{ê}r hwanna sprákun, &
ge·\alst{h}étun eu te \alst{h}elpu \hld\ \alst{h}evan-ríki: &
nu is it giu gi·\alst{n}áhid þurh þes \alst{n}ęrjandan kraft: \hld\ þes mótun gí \alst{n}eotan forð, &
só hwe só \alst{g}erno wili \hld\ \alst{g}ode þeonogjan, &
\alst{w}irkjan aftar is \alst{w}illjon.“ \hld\ Þȯ warð þes \alst{w}erodes filu, &
þero \alst{l}iudjo an \alst{l}ustun: \hld\ wurðun im þea \alst{l}êra Kristes, &
só \alst{s}wótja þem gi·\alst{s}ïðja. \hld\ hé bi·gan im \alst{s}amnon þȯ &
\alst{g}umono te \alst{j}ungoron, \hld\ \alst{g}ódoro manno, &
\alst{w}ord-spáha \alst{w}eros. \hld\ Géng im þȯ bi ênes \alst{w}atares staðe, &
þat þár habda \alst{J}ordan \hld\ a·nevan \alst{G}alileo land &
ênna \alst{s}ê ge·warhtan. \hld\ Þár hé \alst{s}ittjan fand &
\alst{A}ndreas ęndi Petrus \hld\ bi þem \alst{a}ha-strôme, &
\alst{b}êðja þea ge·\alst{b}róðar, \hld\ þár sie an \alst{b}rêd watar &
swíðo \alst{n}iud-líko \hld\ \alst{n}ętti þenidun, &
\alst{f}iskodun im an þem \alst{f}lóde. \hld\ Þár sie þat \alst{f}riðu-barn godes &
bi þes \alst{s}êes staðe \hld\ \alst{s}elvo grótta, &
hét þat sie im \alst{f}olgodin, \hld\ kwað þat hé im só \alst{f}ilu woldi &
\alst{g}odes ríkjas for·\alst{g}even; \hld\ „al só git hír an \alst{J}ordanes strôme &
\alst{f}iskos \alst{f}ȧhat, \hld\ só skulun git noh \alst{f}iriho barn &
\alst{h}alon te inkun \alst{h}andun, \hld\ þat sie an \alst{h}evan-ríki &
þurh inka \alst{l}êra \hld\ \alst{l}íðan mótin, &
\alst{f}aran \alst{f}olk manag.“ \hld\ Þȯ warð \alst{f}rô-mód hugi &
\alst{b}êðjun þem gi·\alst{b}róðrun: \hld\ ant·kęndun þat \alst{b}arn godes, &
\alst{l}iovan hêrron: \hld\ for·\alst{l}étun al saman &
\alst{A}ndreas ęndi Petrus, \hld\ só hwat só sie bi þeru \alst{a}hu habdun, &
ge·\alst{w}unstes bi þem \alst{w}atare: \hld\ was im \alst{w}illjo mikil, &
þat sie mid þem \alst{g}odes barne \hld\ \alst{g}angan móstin, &
\alst{s}amad an is gi·\alst{s}ïðja, \hld\ skoldun \alst{s}álig-líko &
\alst{l}ôn ant·fȧhan: \hld\ só dót \alst{l}iudjo so hwi-lik, &
só þes \alst{h}êrran wili \hld\ \alst{h}uldi gi·þionon, &
ge·\alst{w}irkjan is \alst{w}illjon. \hld\ Þȯ sie bi þes \alst{w}atares staðe &
\alst{f}urðor kwámun, \hld\ þȯ fundun sie þár ênna \alst{f}ródan man &
\alst{s}ittjan bi þem \alst{s}êwa \hld\ ęndi is \alst{s}uni twêne, &
\alst{J}akobus ęndi \alst{J}ohannes: \hld\ wárun im \alst{j}unga man. &
\alst{S}átun im þá ge·\alst{s}un-fader \hld\ an ênumu \alst{s}ande uppen, &
\alst{b}rugdun ęndi \alst{b}óttun \hld\ \alst{b}êðjum handun &
þiu \alst{n}ętti \alst{n}iud-líko, \hld\ þea sie habdun \alst{n}ahtes êr &
for·\alst{s}liten an þem \alst{s}êwa. \hld\ Þár sprak im \alst{s}elvo tó &
\alst{s}álig barn godes, \hld\ hét þat sie an þana \alst{s}ïð mid im, &
\alst{J}akobus ęndi \alst{J}ohannes, \hld\ \alst{g}éngin bêðje, &
\alst{k}ind-junge man. \hld\ Þȯ wárun im \alst{K}ristes word &
só \alst{w}irðig an þesaro \alst{w}er-oldi, \hld\ þat sie bi þes \alst{w}atares staðe &
iro \alst{a}ldan fader \hld\ \alst{ê}nna for·létun, &
\alst{f}ródan bi þem \alst{f}lóde, \hld\ ęndi al þat sie þár \alst{f}ehas êhtun, &
\alst{n}ęttju ęndi \alst{n}ęglit-skipu, \hld\ ge·kurun im þana \alst{n}ęrjandan Krist, &
\alst{h}êlagna te \alst{h}êrron, \hld\ was im is \alst{h}elpono þarf &
te gi·\alst{þ}iononne: \hld\ só is allaro \alst{þ}egno ge·hwem, &
\alst{w}ero an þesero \alst{w}er-oldi. \hld\ Þȯ gi·wêt im þe \alst{w}aldandes sunu &
mid þem \alst{f}iuwarjun \alst{f}orð, \hld\ ęndi im þȯ þana \alst{f}ïfton gi·kôs &
\alst{K}rist an ênero \alst{k}ôp-stędi, \hld\ \alst{k}uninges jungoron, &
\alst{m}ód-spáhana man: \hld\ \alst{M}attheus was hé hêtan, &
was im \alst{a}mbahtjo \hld\ \alst{ę}ðilero manno, &
skolda þár te is \alst{h}êrron \hld\ \alst{h}andun ant·fȧhan &
\alst{t}ins ęndi \alst{t}olna; \hld\ \alst{t}rewa habda hé góda, &
\alst{a}ðal-and·bári: \hld\ for·lét \alst{a}l saman &
\alst{g}old ęndi silụvar \hld\ ęndi \alst{g}eva managa, &
\alst{d}iurje mêðmos, \hld\ ęndi warð im u̇ses \alst{d}rohtines man; &
\alst{k}ôs im þe \alst{k}uninges þegn \hld\ \alst{K}rist te hêrran, &
\alst{m}ilderan \alst{m}êðọm-gevon, \hld\ þan êr is \alst{m}an-drohtin &
\alst{w}ári an þesero \alst{w}er-oldi: \hld\ féng im \alst{w}óðera þing, &
\alst{l}ang-samoron rád. \hld\ Þȯ warð it allun þem \alst{l}iudjun ku̇ð, &
fon allaro \alst{b}urgo gi·hwem, \hld\ hwó þat \alst{b}arn godes &
\alst{s}amnode ge·\alst{s}ïðos \hld\ ęndi \alst{s}elvo ge·sprak &
só manag \alst{w}ís-lík \alst{w}ord \hld\ ęndi \alst{w}áres só filu, &
\alst{t}orhtes gi·\alst{t}ôgde \hld\ ęndi \alst{t}êkạn manag &
ge·\alst{w}arhte an þesero \alst{w}er-oldi. \hld\ Was þat an is \alst{w}ordun skín &
iak an is \alst{d}ádjun só same, \hld\ þat hé \alst{d}rohtin was, &
\alst{h}imilisk \alst{h}êrro \hld\ ęndi te \alst{h}elpu kwam &
an þesan \alst{m}iddil-gard \hld\ \alst{m}anno barnun, &
\alst{l}iudjun te þesun \alst{l}iohta. \hld\ Oft ge·deda hé þat an þem \alst{l}ande skín, &
þan hé þár \alst{t}orht-líko \hld\ só manag \alst{t}êkạn gi·warhte, &
þár hé \alst{h}êlde mid is \alst{h}andun \hld\ \alst{h}alte ęndi blinde, &
\alst{l}ôsde af þeru \alst{l}éf-hêdi \hld\ \alst{l}iudi manage, &
af \alst{s}u·likun \alst{s}uhtjun, \hld\ só þan allaro \alst{s}wároston &
an \alst{f}iriho barn \hld\ \alst{f}íund bi·wurpun, &
tulgo \alst{l}ang-sam \alst{l}egar.\eva

\bvb TODO.\evb\evg

\bvg\bva[15][1217]%
\hspace*{100pt} Þȯ fórun þár þie \alst{l}iudi tó &%NOTE: In cæsura.
allaro \alst{d}ago ge·hwi-likes, \hld\ þár u̇sa \alst{d}rohtin was &
\alst{s}elvo undar þem gi·\alst{s}ïðje, \hld\ unt-þat þár ge·\alst{s}amnod warð &
\alst{m}ęgin-folk \alst{m}ikil \hld\ \alst{m}anagero þiodo, &
þoh sie þár alle be ge·\alst{l}íkumu \hld\ ge·\alst{l}ôvon ni kwámin. &
\alst{w}eros þurh ênan \alst{w}illjon: \hld\ sume sóhtun sie þat \alst{w}aldandes barn, &
\alst{a}rmoro manno filu \hld\ —was im \alst{á}tes þarf—, &
þat sie im þár at þeru \alst{m}ęnigi \hld\ \alst{m}ates ęndi drankes, &
\alst{þ}igidin at þeru \alst{þ}iodu; \hld\ hwand þár was manag \alst{þ}egạn só gód, &
þie ira \alst{a}lamosnje \hld\ \alst{a}rmun mannun &
\alst{g}erno \alst{g}ávun. \hld\ Sume wárun sie im eft \alst{J}udeono kunnjes, &
\alst{f}êgni \alst{f}olk-skępi: \hld\ wárun þár ge·\alst{f}arana te þiu, &
þat sie u̇ses \alst{d}rohtines \hld\ \alst{d}ádjo ęndi wordo &
\alst{f}áron woldun, \hld\ habdun im \alst{f}êgnjen hugi, &
\alst{w}rêðen \alst{w}illjon: \hld\ woldun \alst{w}aldand Krist &
a·\alst{l}êdjen þem \alst{l}iudjun, \hld\ þat sie is \alst{l}êron ni hôrdin, &
ne \alst{w}ęndin aftar is \alst{w}illjon. \hld\ Suma wárun sie im eft só \alst{w}íse man, &
wárun im \alst{g}lawe \alst{g}umon \hld\ ęndi \alst{g}ode werðe, &
a·\alst{l}esane undar þem \alst{l}iudjun, \hld\ kwámun im þarod be þem \alst{l}êron Kristes, &
þat sie is \alst{h}êlag word \hld\ \alst{h}ôrjen móstin, &
\alst{l}ínon ęndi \alst{l}êstjen: \hld\ habdun mid iro ge·\alst{l}ôvon te im &
\alst{f}asto ge·\alst{f}angen, \hld\ habdun im \alst{f}erhten hugi, &
wurðun is \alst{þ}egnos te þiu, \hld\ þat hé sie an \alst{þ}iod-welon &
\alst{a}ftar iro \alst{ê}n-dagon \hld\ \alst{u}p ge·brȧhti, &
an \alst{g}odes ríki. \hld\ hé só \alst{g}erno ant·féng &
\alst{m}an-kunnjes \alst{m}anag \hld\ ęndi \alst{m}und-burd gi·hét &
te \alst{l}angaru hwílu, \hld\ ęndi mahta só gi·\alst{l}êstjen wel. &
Þȯ warð þár \alst{m}ęgin só \alst{m}ikil \hld\ umbi þana \alst{m}árjon Krist, &
\alst{l}iudjo ge·samnod: \hld\ þȯ gi·sah hé fon allun \alst{l}andun kuman, &
fon allun \alst{w}ídun \alst{w}egun \hld\ \alst{w}erod te·samne &
\alst{l}ungro \alst{l}iudjo: \hld\ is \alst{l}of was só wído &
\alst{m}anagun ge·\alst{m}árid. \hld\ Þȯ gi·wêt im \alst{m}ahtig self &
an ênna \alst{b}erg uppan, \hld\ \alst{b}arno ríkjost, &
\alst{s}undạr ge·\alst{s}ittjen, \hld\ ęndi im \alst{s}elvo ge·kôs &
\alst{t}we-livi ge·\alst{t}alda, \hld\ \alst{t}rew-hafta man, &
\alst{g}ódoro \alst{g}umono, \hld\ þea hé im te \alst{j}ungoron forð &
allaro \alst{d}ago ge·hwi-likes, \hld\ \alst{d}rohtin welda &
an is ge·\alst{s}ïð-skępja \hld\ \alst{s}imblon hębbjan. &
\alst{N}ęmnida sie þȯ bi \alst{n}aman \hld\ ęndi hét sie im þȯ \alst{n}áhor gangan, &
\alst{A}ndreas ęndi Petrus \hld\ \alst{ê}rist sána, &
ge·\alst{b}róðar twêne, \hld\ ęndi \alst{b}êðje mid im, &
\alst{J}akobus ęndi \alst{J}ohannes: \hld\ sie wárun \alst{g}ode werðe; &
\alst{m}ildi was hé im an is \alst{m}óde; \hld\ sie wárun ênes \alst{m}annes suni &
\alst{b}êðje bi ge·\alst{b}urdjun; \hld\ sie kôs þat \alst{b}arn godes &
\alst{g}óde te \alst{j}ungoron \hld\ ęndi \alst{g}umono filu, &
\alst{m}árjero \alst{m}anno: \hld\ \alst{M}attheus ęndi Þomas, &
\alst{J}udasas twêna \hld\ ęndi \alst{J}akob ȯðran, &
is \alst{s}elves \alst{s}wiri: \hld\ sie wárun fon gi·\alst{s}ustruonjon twêm &
\alst{k}nósles \alst{k}umana, \hld\ \alst{K}rist ęndi Jakob, &
\alst{g}óde \alst{g}adulingos. \hld\ Þȯ habda þero \alst{g}umono þár &
þe \alst{n}ęrjendo Krist \hld\ \alst{n}iguni ge·talde, &%TODO: check niguni
\alst{t}rew-hafte man: \hld\ þȯ hét hé ôk þana \alst{t}e·handon gangan &
\alst{s}elvo mid þem gi·\alst{s}ïðun: \hld\ \alst{S}ímon was hé hêtan; &
hét ôk \alst{B}artholomeus \hld\ an þana \alst{b}erg uppan &
\alst{f}aran fan þem \alst{f}olke áðrum \hld\ ęndi \alst{Ph}ilippus mid im, &
\alst{t}rew-hafte man. \hld\ Þȯ géngun sie \alst{t}we-livi samad, &
\alst{r}inkos te þeru \alst{r}únu, \hld\ þár þe \alst{r}ádand sat, &
\alst{m}anagoro \alst{m}und-boro, \hld\ þe allumu \alst{m}an-kunnje &
wið \alst{h}ęllje ge·þwing \hld\ \alst{h}elpan welde, &
\alst{f}ormon wið þem \alst{f}erne, \hld\ só hwem só \alst{f}rummjen wili &
só \alst{l}iov-líka \alst{l}êra, \hld\ só hé þem \alst{l}iudjun þár &
þurh is gi·\alst{w}it mikil \hld\ \alst{w}ísjan hogda.\eva

\bvb TODO.\evb\evg

\bvg\bva[16][1279]%
Þȯ umbi þana \alst{n}ęrjandon Krist \hld\ \alst{n}áhor géngun &%NOTE ms. -- Þó] V 1 (27r)
\alst{s}u-lika ge·\alst{s}ïðos, \hld\ só hé im \alst{s}elvo ge·kôs, &
\alst{w}aldand undar þem \alst{w}erode. \hld\ Stódun \alst{w}ísa man, &
\alst{g}umon umbi þana \alst{g}odes sunu \hld\ \alst{g}erno swíðo, &
\alst{w}eros an \alst{w}illjon: \hld\ was im þero \alst{w}ordo niud, &
\alst{þ}ȧhtun ęndi \alst{þ}agodun, \hld\ hwat im þero \alst{þ}iodo drohtin, &
\alst{w}eldi \alst{w}aldand self \hld\ \alst{w}ordun ku̇ðjan &
þesum \alst{l}iudjun te \alst{l}iove. \hld\ Þan sat im þe \alst{l}andes hirdi &
\alst{g}ęgin-ward for þem \alst{g}umun, \hld\ \alst{g}odes êgan barn: &
welda mid is \alst{sp}rákun \hld\ \alst{sp}áh-word manag &
\alst{l}êrjan þea \alst{l}iudi, \hld\ hwó sie \alst{l}of gode &
an þesum \alst{w}er-old-ríkja \hld\ \alst{w}irkjan skoldin. &
\alst{S}at im þȯ ęndi \alst{s}wígoda \hld\ ęndi \alst{s}ah sie an lango, &
was im \alst{h}old an is \alst{h}ugi \hld\ \alst{h}êlag drohtin, &
\alst{m}ildi an is \alst{m}óde, \hld\ ęndi þȯ is \alst{m}und ant·lôk, &
\alst{w}ísde mid \alst{w}ordun \hld\ \alst{w}aldandes sunu &
\alst{m}anag \alst{m}ár-lík þing \hld\ ęndi þem \alst{m}annum sagde &
\alst{sp}áhun wordun, \hld\ þem þe hé te þeru \alst{sp}ráku þarod, &
\alst{K}rist alo-waldo, \hld\ ge·\alst{k}oran habda, &
hwi-like wárin \alst{a}llaro \hld\ \alst{i}rmin-manno &
\alst{g}ode werðoston \hld\ \alst{g}umono kunnjes; &
\alst{s}agde im þȯ te \alst{s}ȯðan, \hld\ kwað þat þie \alst{s}áliga wárin, &
\alst{m}an an þesoro \alst{m}iddil-gardun, \hld\ þie hér an iro \alst{m}óde wárin &
\alst{a}rme þurh \alst{ô}d-módi: \hld\ „þem is þat \alst{ê}wana ríki, &
swíðo \alst{h}êlag-lík \hld\ an \alst{h}evan-wange &
\alst{s}in-líf far·geven.“ \hld\ Kwað þat ôk \alst{s}álige wárin &
\alst{m}áð-mundje \alst{m}an: \hld\ „þie mótun þie \alst{m}árjon erðe, &
of·\alst{s}ittjen þat \alst{s}elve ríki.“ \hld\ Kwað þat ôk \alst{s}álige wárin, &
þie hír \alst{w}iopin iro \alst{w}ammun dádi; \hld\ „þie mótun eft \alst{w}illjon ge·bídan, &
\alst{f}rófre an iro \alst{f}râhon ríkja. \hld\ Sálige sind ôk, þe sie hír \alst{f}rumono gi·lustid, &
\alst{r}inkos, þat sie \alst{r}ehto a·dómjen. \hld\ Þes mótun sie werðan an þem \alst{r}íkja drohtines &
gi·\alst{f}ullit þurh iro \alst{f}erhton dádi: \hld\ su-líkoro mótun sie \alst{f}rumono bi·knégan &
þie \alst{r}inkos, þie hír \alst{r}ehto a·dómjad, \hld\ ne willjad an \alst{r}únun be·swíkan &
\alst{m}an, þár sie at \alst{m}ahle sittjad. \hld\ Sálige sind ôk þem hír \alst{m}ildi wirðit &
\alst{h}ugi an \alst{h}ęliðo briostun: \hld\ þem wirðit þe \alst{h}êlego drohtin, &
\alst{m}ildi \alst{m}ahtig selvo. \hld\ Sálige sind ôk undar þesaro \alst{m}anagon þiodu, &
þie hębbjad iro \alst{h}erta gi·\alst{h}rênod: \hld\ þie mótun þane \alst{h}evanes waldand &
\alst{s}ehan an \alst{s}ínum ríkja.“ \hld\ Kwað þat ôk \alst{s}álige wárin, &
„þie þe \alst{f}riðu-samo undar þesumu \alst{f}olke libbjod \hld\ ęndi ni willjad êniga \alst{f}ehta ge·wirken, &
\alst{s}aka mid iro \alst{s}elvoro dádjun: \hld\ þie mótun wesan \alst{s}uni drohtines ge·nęmnide, &
hwande hé im wil ge·\alst{n}ádig werðen; \hld\ þes mótun sie \alst{n}iotan lango &
\alst{s}elvon þes \alst{s}ínes ríkjes.“ \hld\ Kwað þat ôk \alst{s}álige wárin &
þie \alst{r}inkos, þe \alst{r}ehto weldin, \hld\ „ęndi þurh þat þolod \alst{r}íkjoro manno &
\alst{h}ęti ęndi \alst{h}arm-kwidi: \hld\ þem is ôk an \alst{h}imile eft &
\alst{g}odes wang for·\alst{g}even \hld\ ęndi \alst{g}êst-lík \edtext{líf}{\Afootnote{Last word of \textbf{V} 27r; text continues on 32v.}} &
\alst{a}ftar te \alst{ê}wan-dage, \hld\ só is io \alst{ę}ndi ni kumit, &%NOTE ms. -- aftar] V 2 (32v)
\alst{w}elan \alst{w}un-sames.“ \hld\ Só habde þȯ \alst{w}aldand Krist &
for þem \alst{e}rlom þár \hld\ \alst{a}hto ge·talda &
\alst{s}álda ge·\alst{s}agda; \hld\ mid þem skal \alst{s}imbla gi·hwe &
\alst{h}imil-ríki ge·\alst{h}alon, \hld\ ef hé it \alst{h}ębbjan wili, &
eþþo hé skal te \alst{ê}wan-daga \hld\ \alst{a}ftar þarvon &
\alst{w}elon ęndi \alst{w}illjon, \hld\ sïðor hé þese \alst{w}er-old a·givid, &
\alst{e}rð-lívi-gi·skapu, \hld\ ęndi sókit im \alst{ȯ}ðar lioht &
só \alst{l}iof só \alst{l}êð, \hld\ só hé mid þesun \alst{l}iudjun hér &
gi·\alst{w}erkod an þesoro \alst{w}er-oldi, \hld\ al só it þár þȯ mid is \alst{w}ordun sagde &
\alst{K}rist alo-waldo, \hld\ \alst{k}uningo ríkjost &
\alst{g}odes êgan barn \hld\ \alst{j}ungorun sínun: &
„Ge werðat ôk só \alst{s}álige“, \hld\ kwað hé, „þes iu \alst{s}aka biodat &
\alst{l}iudi aftar þeson \alst{l}ande \hld\ ęndi \alst{l}êð sprekat, &
\alst{h}ębbjad iu te \alst{h}oska \hld\ ęndi \alst{h}armes filu &
ge·\alst{w}irkjad an þesoro \alst{w}er-oldi \hld\ ęndi \alst{w}íti ge·frummjad, &
\alst{f}ęlgjad iu \alst{f}irin-spráka \hld\ ęndi \alst{f}íund-skępi, &
\alst{l}âgnjad iuwa \alst{l}êra, \hld\ dót iu \alst{l}êðes filu, &
\alst{h}armes þurh iuwan \alst{h}êrron. \hld\ Þes látad gi iuwan \alst{h}ugi simbla, &
\alst{l}íf an \alst{l}ustun, \hld\ hwand iu þat \alst{l}ôn stęndit &
an \alst{g}odes ríkja \alst{g}aru, \hld\ \alst{g}ódo ge·hwi-likes, &
\alst{m}ikil ęndi \alst{m}anag-fald: \hld\ þat is iu te \alst{m}édu far·gevan, &
hwand gí hér \alst{ê}r bi·foran \hld\ \alst{a}rvid þolodun, &
\alst{w}íti an þesoro \alst{w}er-oldi. \hld\ \alst{W}irs is þem ȯðrum, &
\alst{g}iviðig \alst{g}rimmora þing, \hld\ þem þe hér \alst{g}ód êgun, &
\alst{w}ídan \alst{w}orold-\alst{w}elon: \hld\ þie for·slítat iro \alst{w}unnja hér; &
ge·\alst{n}iudot sie ge·\alst{n}óges, \hld\ skulun eft \alst{n}arowaro þing &
aftar iro \alst{h}in-fęrdi \hld\ \alst{h}ęliðos þolojan. &
Þan \alst{w}ópjan þár \alst{w}an-skęfti, \hld\ þie hér êr an \alst{w}unnjon sín, &
\alst{l}ibbjad an allon \alst{l}ustun, \hld\ ne willjad þes far·\alst{l}átan wiht, &
\alst{m}êni-gi·þȧhtjo, \hld\ þes sie an iro \alst{m}ód spęnit, &
\alst{l}êðoro gi·\alst{l}êstjo. \hld\ Þan im þat \alst{l}ôn kumid, &
\alst{u}vil \alst{a}rved-sam, \hld\ þan sie is þane \alst{ę}ndi skulun &
\alst{s}orgondi ge·\alst{s}ehan. \hld\ Þan wirðid im \alst{s}êr hugi, &
þes sie þesero \alst{w}er-oldes só filu \hld\ \alst{w}illjan ful-géngun, &%NOTE ms. -- sie] V end.
\alst{m}an an iro \alst{m}ód-sevon. \hld\ Nú skulun gí im þat \alst{m}ên lahan, &
\alst{w}ęrjan mid \alst{w}ordun, \hld\ al só ik giu nú ge·\alst{w}ísjan mag, &
\alst{s}ęggjan \alst{s}ȯð-líko, \hld\ ge·\alst{s}ïðos míne, &
\alst{w}árun \alst{w}ordun, \hld\ þat gí þesoro \alst{w}er-oldes nú forð &
skulun \alst{s}alt wesan, \hld\ \alst{s}undigero manno, &
\alst{b}ótjan iro \alst{b}alu-dádi, \hld\ þat sie an \alst{b}ętara þing, &
\alst{f}olk far·\alst{f}ȧhan \hld\ ęndi for·látan \alst{f}íundes gi·werk, &
\alst{d}iuvales ge·\alst{d}ádi, \hld\ ęndi sókjan iro \alst{d}rohtines ríki. &
Só skulun gí mid iuwon \alst{l}êrun \hld\ \alst{l}iud-folk manag &
\alst{w}ęndjan aftar mínon \alst{w}illjon. \hld\ Ef iuwar þan a·\alst{w}irðid hwi-lik, &
far·\alst{l}átid þea \alst{l}êra, \hld\ þea hé \alst{l}êstjan skal, &
þan is im só þem \alst{s}alte, \hld\ þe man bi \alst{s}êes staðe &
\alst{w}ído te·\alst{w}irpit: \hld\ þan it te \alst{w}ihti ni dôg, &
ak it \alst{f}iriho barn \hld\ \alst{f}ótun spurnat, &
\alst{g}umon an \alst{g}reote. \hld\ Só wirðid þem, þe þat \alst{g}odes word skal &
\alst{m}annum \alst{m}árjan: \hld\ ef hé im þan látid is \alst{m}ód twehon, &
þat hí ne willja mid \alst{h}luttro \alst{h}ugi \hld\ te \alst{h}evan-ríkja &
\alst{sp}anen mid is \alst{sp}ráku \hld\ ęndi sęggjan \alst{sp}el godes, &
ak \alst{w}ęnkid þero \alst{w}ordo, \hld\ þan wirðid im \alst{w}aldand gram, &
\alst{m}ahtig \alst{m}ódag, \hld\ ęndi só samo \alst{m}anno barn; &
wirðid \alst{a}llun þan \hld\ \alst{i}rmin-þiodun, &
\alst{l}iudjun a·\alst{l}êðid, \hld\ ef is \alst{l}êra ni dugun.“\eva

\bvb TODO.\evb\evg

\bvg\bva[17][1381]%
Só \alst{sp}rak hé þȯ \alst{sp}áh-líko \hld\ ęndi sagda \alst{sp}el godes, &
\alst{l}êrde þe \alst{l}andes ward \hld\ \alst{l}iudi síne &
mid \alst{h}luttru \alst{h}ugju. \hld\ \alst{H}ęliðos stódun, &
\alst{g}umon umbi þana \alst{g}odes sunu \hld\ \alst{g}erno swíðo, &
\alst{w}eros an \alst{w}illjon: \hld\ was im þero \alst{w}ordo niud, &
\alst{þ}ȧhtun ęndi \alst{þ}agodun, \hld\ gi·hôrdun þero \alst{þ}iodo drohtin &
sęggjan \alst{ê}w godes \hld\ \alst{ę}ldi-barnun; &
gi·\alst{h}ét im \alst{h}evan-ríki \hld\ ęndi te þem \alst{h}ęliðun sprak: &
„Ôk mag ik iu \alst{s}ęggjan, \hld\ ge·\alst{s}ïðos mína, &
\alst{w}árun \alst{w}ordun, \hld\ þat gí þesoro \alst{w}er-oldes nú forð &
skulun \alst{l}ioht wesan \hld\ \alst{l}iudjo barnun, &
\alst{f}agạr mid \alst{f}irihun \hld\ ovar \alst{f}olk manag, &
\alst{w}litig ęndi \alst{w}un-sam: \hld\ ni mugun iuwa \alst{w}erk mikil &
bi·\alst{h}olan werðan, \hld\ mid hwi-liko gi sea \alst{h}ugi ku̇ðjat: &
þan mêr þe þiu \alst{b}urg ni mag, \hld\ þiu an \alst{b}erge stáð, &
\alst{h}ôh \alst{h}olm-klivu, \hld\ bi·\alst{h}olen werðen, &
\alst{w}risi-lík gi·\alst{w}erk, \hld\ ni mugun iuwa \alst{w}ord þan mêr &
an þesoro \alst{m}iddil-gard \hld\ \alst{m}annum werðen, &
iuwa \alst{d}ádi bi·\alst{d}ęrnit. \hld\ \alst{D}ót, só ik iu lêrju: &
\alst{l}átad iuwa \alst{l}ioht mikil \hld\ \alst{l}iudjun skínan, &
\alst{m}anno barnun, \hld\ þat sie far·standan iuwan \alst{m}ód-sevon, &
iuwa \alst{w}erk ęndi iuwan \alst{w}illjon, \hld\ ęndi þes \alst{w}aldand god &
mid \alst{h}luttro \alst{h}ugju, \hld\ \alst{h}imiliskan fader, &
\alst{l}ovon an þesumu \alst{l}iohte, \hld\ þes hé iu su·lika \alst{l}êra far·gaf. &
Ni skal neoman \alst{l}ioht, þe it havad, \hld\ \alst{l}iudjun dęrnjan, &
te \alst{h}ardo be·\alst{h}węlvjan, \hld\ ak hé it \alst{h}ôho skal &
an \alst{s}ęli \alst{s}ęttjan, \hld\ þat þea ge·\alst{s}ehan mugin &
\alst{a}lla ge·líko, \hld\ þea þár \alst{i}nna sind, &
\alst{h}ęliðos an \alst{h}allu. \hld\ Þan hald ni skulun gi iuwa \alst{h}êlag word &
an þesumu \alst{l}and-skępa \hld\ \alst{l}iudjun dęrnjen, &
\alst{h}ęlið-kunnje far·\alst{h}elan, \hld\ ak ge it \alst{h}ôho skulun &
\alst{b}rêdjan, þat gi·\alst{b}od godes, \hld\ þat it allaro \alst{b}arno ge·hwi-lik, &
ovar al þit \alst{l}and-skępi \hld\ \alst{l}iudi far·standan &
ęndi só ge·\alst{f}rummjen, \hld\ só it an \alst{f}orn-dagun &
tulgo \alst{w}íse man \hld\ \alst{w}ordun ge·sprákun, &
þan sie þana \alst{a}ldan \alst{ê}w \hld\ \alst{e}rlos heldun, &
ęndi ôk \alst{s}u·liku \alst{s}wíðor, \hld\ só ik iu nu \alst{s}ęggjan mag, &
alloro \alst{g}umono ge·hwi-lik \hld\ \alst{g}ode þionojan, &
þan it þár an þem \alst{a}ldom \hld\ \alst{ê}wa ge·beode. &
Ni \alst{w}ánjat gi þes mit \alst{w}ihtju, \hld\ þat ik bi þiu an þesa \alst{w}er-old kwámi, &
þat ik þana \alst{a}ldan \alst{ê}w \hld\ \alst{i}rrjen willje, &
\alst{f}ęlljan undar þesumu \alst{f}olke \hld\ efþo þero \alst{f}ora-sagono &
\alst{w}ord \alst{w}iðar-\alst{w}erpen, \hld\ þea hér só gi·\alst{w}árja man &
\alst{b}ar-líko ge·\alst{b}udun. \hld\ Êr skal \alst{b}êðju te·faran, &
\alst{h}imil ęndi erðe, \hld\ þiu nu bi·\alst{h}lidan standat, &
êr þan þero \alst{w}ordo \hld\ \alst{w}iht bi·líva &
un·\alst{l}êstid an þesumu \alst{l}iohte, \hld\ þea sie þesum \alst{l}iudjun hér &
\alst{w}ár-líko ge·budun. \hld\ Ni kwam ik an þesa \alst{w}er-old te þiu, &
þat ik \alst{f}eldi þero \alst{f}ora-sagono word, \hld\ ak ik siu \alst{f}ulljen skal, &
\alst{ô}kjon ęndi nígjan \hld\ \alst{ę}ldi-barnum, &
þesumu \alst{f}olke te \alst{f}rumu. \hld\ Þat was \alst{f}orn ge·skrivan &
an þem \alst{a}ldon \alst{ê}o \hld\ —ge hôrdun it \alst{o}ft sprekan &
\alst{w}ord-\alst{w}íse man—: \hld\ só hwe só þat an þesoro \alst{w}er-oldi gi·dót, &
þat hé \alst{ȧ}ðrana \hld\ \alst{a}ldru bi·neote, &
\alst{l}ívu bi·\alst{l}ôsje, \hld\ þem skulun \alst{l}iudjo barn &
\alst{d}ôd a·\alst{d}êljan. \hld\ Þan willjo ik it iu \alst{d}iopor nu, &
\alst{f}urður bi·\alst{f}ȧhan: \hld\ só hwe só ina þurh \alst{f}íund-skępi, &
\alst{m}an wiðar ȯðrana \hld\ an is \alst{m}ód-sevon &
\alst{b}ilgit an is \alst{b}reostun \hld\ —hwand sie alle ge·\alst{b}róðar sint, &
\alst{s}álig folk godes, \hld\ \alst{s}ibbjon bi·tengja, &%TODO: Check etymology of bi·tengja.
\alst{m}an mid \alst{m}ág-skępi—, \hld\ þan wirðit þoh hwe ȯðrumu an is \alst{m}óde só gram, &
\alst{l}íbes weldi ina bi·\alst{l}ôsjen, \hld\ of hé mahti gi·\alst{l}êstjen só: &
þan is hé sán a·\alst{f}éhit \hld\ ęndi is þes \alst{f}erạhas skolo, &
\alst{a}l su·likes \alst{u}r-dêljes \hld\ só þe \alst{ȯ}ðar was, &
þe þurh is \alst{h}and-męgin \hld\ \alst{h}ôvdo bi·lôsde &
\alst{e}rl \alst{ȯ}ðarna. \hld\ Ôk is an þem \alst{ê}o ge·skrivan &
\alst{w}árun \alst{w}ordun, \hld\ só gí \alst{w}iton alle, &
þan man is \alst{n}áhiston \hld\ \alst{n}iud-líko skal &
\alst{m}innjan an is \alst{m}óde, \hld\ wesen is \alst{m}águn hold, &
\alst{g}adulingun \alst{g}ód, \hld\ wesen is \alst{g}eva mildi, &
\alst{f}râhon is \alst{f}riunda ge·hwane, \hld\ ęndi skal is \alst{f}íund hatan, &
wiðer·\alst{st}anden þem mid \alst{st}rídu \hld\ ęndi mid \alst{st}arku hugi, &
\alst{w}ęrjan wiðar \alst{w}rêðun. \hld\ Þan sęggjo ik iu te \alst{w}áron nu, &
\alst{f}ul-líkur for þesumu \alst{f}olke, \hld\ þat gí iuwa \alst{f}íund skulun &
\alst{m}innjon an iuwomu \alst{m}óde, \hld\ só samo só gí iuwa \alst{m}ágos dót, &
an \alst{g}odes namon. \hld\ Dót im \alst{g}ódes filu, &
tôgjat im \alst{h}luttran \alst{h}ugi, \hld\ \alst{h}olda trewa, &
\alst{l}iof wiðar ira \alst{l}êðe. \hld\ Þat is \alst{l}ang-sam rád &
\alst{m}anno só hwi-likumu, \hld\ só is \alst{m}ód te þiu &
ge·\alst{f}líhit wiðar is \alst{f}íunde. \hld\ Þan mótun gí þea \alst{f}ruma êgan, &
þat gí mótun \alst{h}êten \hld\ \alst{h}evan-kuninges suni, &
is \alst{b}líði \alst{b}arn. \hld\ Ne mugun gí iu \alst{b}ętaran rád &
ge·\alst{w}innan an þesoro \alst{w}er-oldi. \hld\ Þan sęggjo ik iu te \alst{w}áron ôk, &
\alst{b}arno ge·hwi-likum, \hld\ þat gí ne mugun mid gi·\alst{b}olgono hugi &
iuwas \alst{g}ódes wiht \hld\ te \alst{g}odes húsun &
\alst{w}aldande far·gevan, \hld\ þat it imu \alst{w}irðig sí &
te ant·\alst{f}ȧhanne, \hld\ só lango só þú \alst{f}íund-skępjes wiht, &
wiðer \alst{ȯ}ðran man \hld\ \alst{i}n-wid hugis. &
Êr skalt þú þi \alst{s}imbla ge·\alst{s}ónjen \hld\ wið þana \alst{s}ak-waldand, &
ge·\alst{m}ódi gi·\alst{m}ahljan: \hld\ sïðor maht þú \alst{m}êðmos þína &
te þem \alst{g}odes altere a·\alst{g}evan: \hld\ þan sind sie þemu \alst{g}ódan werðe, &
\alst{h}evan-kuninge. \hld\ Mér skulun gi aftar is \alst{h}uldi þionon, &
\alst{g}odes willjon ful·\alst{g}án, \hld\ þan ȯðra \alst{J}udeon duon, &
ef gí willjat \alst{ê}gan \hld\ \alst{ê}wan ríki, &
\alst{s}in-líf \alst{s}ehan. \hld\ Ôk skal ik iu \alst{s}ęggjan noh, &
hwó it þár an þem \alst{a}ldon \hld\ \alst{ê}o ge·biudid, &
þat \alst{ê}nig \alst{e}rl \alst{ȯ}ðres \hld\ \alst{i}dis ni bi·swíka, &
\alst{w}íf mid \alst{w}ammu. \hld\ Þan sęggjo ik iu te \alst{w}áron ôk, &
þat þár man is \alst{s}iuni mugun \hld\ \alst{s}wíðo far·lêdjan &
an \alst{m}irki \alst{m}ên, \hld\ ef hi ina látid is \alst{m}ód spanen, &
þat hé be·\alst{g}inna þero \alst{g}irnjan, \hld\ þiu imu ge·\alst{g}angan ni skal. &
Þan haved hé an imu \alst{s}elvon \alst{s}án \hld\ \alst{s}undja ge·warhta, &
ge·\alst{h}ęftid an is \alst{h}ertan \hld\ \alst{h}ęlli-wíti. &
Ef þan þana man is \alst{s}iun wili \hld\ eþþa is \alst{s}wíðare hand &
far·\alst{l}êdjen is \alst{l}iðo hwi-lik \hld\ an \alst{l}êðan weg, &
þan is \alst{e}rlo ge·hwem \hld\ \alst{ȯ}ðar bętara, &
\alst{f}iriho barno, \hld\ þat hé ina \alst{f}ram werpa &
ęndi þana \alst{l}ið \alst{l}ôsje \hld\ af is \alst{l}ík-hamon &
ęndi ina \alst{á}no kuma \hld\ \alst{u}p te himile, &
þan hé só mid \alst{a}llun \hld\ te þem \alst{I}nferne, &
\alst{h}werve mid só \alst{h}êlun \hld\ an \alst{h}ęlli-grund. &
Þan mênid þiu \alst{l}éf-hêd, \hld\ þat ênig \alst{l}iudjo ni skal &
far·\alst{f}olgan is \alst{f}riunde, \hld\ ef hé ina an \alst{f}irina spanit, &
\alst{s}wás man an \alst{s}aka: \hld\ þan ne sí hé imu eo só swíðo an \alst{s}ibbjun bi·lang, &
ne iro \alst{m}ág-skępi só \alst{m}ikil, \hld\ ef hé ina an \alst{m}orð spęnit, &
\alst{b}édid \alst{b}alu-werko; \hld\ \alst{b}ętera is imu þan ȯðar, &
þat hé þana \alst{f}riund fan imu \hld\ \alst{f}er far·werpa, &
\alst{m}íðe þes \alst{m}áges \hld\ ęndi ni hębbja þár êniga \alst{m}innja tó, &
þat hé móti \alst{ê}no \hld\ \alst{u}p ge·stígan &
\edtext{\alst{h}ôh}{\Afootnote{TODO: Critical note (ms. apparently has hô)}} \alst{h}imil-ríki, \hld\ þan sie \alst{h}ęlli-ge·þwing, &
\alst{b}rêd \alst{b}alu-wíti \hld\ \alst{b}êðja gi·sókjan, &
\alst{u}vil \alst{a}rvidi.\eva

\bvb TODO.\evb\evg

\bvg\bva[18][1502]%
\hspace*{100pt} Ôk is an þem \alst{ê}o ge·skrivan &%NOTE: In cæsura.
\alst{w}árun \alst{w}ordun, \hld\ só gí \alst{w}itun alle, &
þat \alst{m}íðe \alst{m}ên-êðos \hld\ \alst{m}an-kunnjes ge·hwi-lik, &
ni for·\alst{s}węrje ina \alst{s}elvon, \hld\ hwand þat is \alst{s}undje te mikil, &
far·\alst{l}êdid \alst{l}iudi \hld\ an \alst{l}êðan weg. &
Þan willjo ik iu eft \alst{s}ęggjan, \hld\ þan sán ni \alst{s}węrja neo-man &
\alst{ê}nigan \alst{ê}ð-staf \hld\ \alst{ę}ldi-barno, &
ne bi \alst{h}imile þemu \alst{h}ôhon, \hld\ hwand þat is þes \alst{h}êrron stól, &
ne bi \alst{e}rðu þár \alst{u}ndar, \hld\ hwand þat is þes \alst{a}lo-waldon &
\alst{f}agạr \alst{f}ót-skamel, \hld\ nek ênig \alst{f}iriho barno &
ne \alst{s}węrja bi is \alst{s}elves hôvde, \hld\ hwand hé ni mag þár ne \alst{s}wart ne hwít &
ênig \alst{h}ár ge·wirkjan, \hld\ b·útan só it þe \alst{h}êlago god, &
ge·\alst{m}arkode \alst{m}ahtig; \hld\ be·þiu skulun \alst{m}íðan filu &
\alst{e}rlos \alst{ê}ð-wordo. \hld\ Só hwe só it \alst{o}fto dót, &
só \alst{w}irðid is simbla \alst{w}irsa, \hld\ hwand hé imu gi·\alst{w}ardon ni mag. &
Bi·þiu skal ik iu nu te \alst{w}árun \hld\ \alst{w}ordun gi·beodan, &
þat gi neo ne \alst{s}węrjen \hld\ \alst{s}wíðoron êðos, &
\alst{m}éron \alst{m}et \alst{m}annun, \hld\ b·útan só ik iu mid \alst{m}ínun hér &
swíðo \alst{w}ár-liko \hld\ \alst{w}ordun ge·biudu: &
ef man hwemu \alst{s}aka \alst{s}ókja, \hld\ bi·\alst{s}ęggja þat wáre, &
kweðe \alst{j}á, gef it sí, \hld\ \alst{g}eha þes þár wár is, &
kweðe \alst{n}ên, af it \alst{n}is, \hld\ láta im ge·\alst{n}óg an þiu; &
só hwat só is \alst{m}êr ovar þat \hld\ \alst{m}an ge·frummjad, &
só kumid it \alst{a}l fan \alst{u}vile \hld\ \alst{ę}ldi-barnun, &
þat \alst{e}rl þurh \alst{u}n-trewa \hld\ \alst{ȯ}ðres ni wili &
\alst{w}ordo ge·lôvjan. \hld\ Þan sęggjo ik iu te \alst{w}áron ôk, &
hwó it þár an þem \alst{a}ldon \hld\ \alst{ê}o ge·biudit: &
só hwe só \alst{ô}gon ge·nimid \hld\ \alst{ȯ}ðres mannes, &
\alst{l}ôsid af is \alst{l}ík-haman, \hld\ eþþa is \alst{l}iðo hwi-likan, &
þat hé it eft mid is \alst{s}elves skal \hld\ \alst{s}án ant·gelden &
mid ge·\alst{l}íkun \alst{l}iðjon. \hld\ Þan willjo ik iu \alst{l}êrjan nu, &
þat gí só ni \alst{w}rekan \hld\ \alst{w}rêða dádi, &
ak þat gí þurh \alst{ô}d-módi \hld\ \alst{a}l ge·þologjan &
\alst{w}ítjes ęndi \alst{w}ammes, \hld\ só hwat só man iu an þesoro \alst{w}er-oldi ge·dóe. &
Dóe \alst{a}lloro \alst{e}rlo ge·hwi-lik \hld\ \alst{ȯ}ðrom manne &
\alst{f}rume ęndi ge·\alst{f}óri, \hld\ só hé willje, þat im \alst{f}iriho barn &
\alst{g}ódes an·\alst{g}ęgin dóen. \hld\ Þan wirðit im \alst{g}od mildi, &
\alst{l}iudjo só hwi-likum, \hld\ só þat \alst{l}êstjen wili. &
\alst{Ê}rod gí \alst{a}rme man, \hld\ dêljad iuwan \alst{ô}d-welon &
undar þero \alst{þ}urftigon \alst{þ}iodu; \hld\ ne rókjad, hweðar gí is ênigan \alst{þ}ank ant·fȧhan &
efþo lôn an þesoro \alst{l}êhnjon wer-oldi, \hld\ ak huggjat te iuwomu \alst{l}eovon hêrran &
þero \alst{g}evono te \alst{g}elde, \hld\ þat sie iu \alst{g}od lôno, &
\alst{m}ahtig \alst{m}und-boro, \hld\ só hwat só gi is þurh is \alst{m}innes gi·dót. &
Ef þú þan \alst{g}evogjan wili \hld\ \alst{g}ódun mannun &
\alst{f}agạre \alst{f}eho-skattos, \hld\ þár þú eft \alst{f}rumono hugis &
\alst{m}êr ant·fȧhan, \hld\ te hwí havas þú þes êniga \alst{m}éda fon gode &
eþþa \alst{l}ôn an þemu is \alst{l}iohte? \hld\ hwand þat is \alst{l}êhni feho. &
Só is þes \alst{a}lles ge·hwat, \hld\ þe þú \alst{ȯ}ðrun ge·duos &
\alst{l}iudjon te \alst{l}eove, \hld\ þár þú hugis eft ge·\alst{l}ík neman &
þero \alst{w}ordo ęndi þero \alst{w}erko: \hld\ te hwí wêt þi þes u̇sa \alst{w}aldand þank, &
þes þú þín só bi·\alst{f}ilhis \hld\ ęndi ant·\alst{f}áhis eft þan þú wili? &
\alst{i}uwan \alst{ô}ð-welon \hld\ gevan gi þem \alst{a}rmun mannun, &
þe ina iu an þesoro \alst{w}er-oldi ne lônon \hld\ ęndi rómot te iuwes \alst{w}aldandes ríkja. &
Te \alst{h}lúd ni dó þú it, \hld\ þan þú mid þínun \alst{h}andun bi·felhas &
þína \alst{a}lamosna þemu \alst{a}rmon manne, \hld\ ak dó im þurh \alst{ô}d-módjen &
\alst{g}erno þurh \alst{g}odes þank: \hld\ þan móst þú eft \alst{g}eld niman, &
swíðo \alst{l}iof-lík \alst{l}ôn, \hld\ þár þú is \alst{l}ango bi·þarft, &
\alst{f}agạroro \alst{f}rumono. \hld\ Só hwat só þú is só þurh \alst{f}erhtan hugi &
\alst{d}arno ge·\alst{d}êljas, \hld\ —so is u̇sumu \alst{d}rohtine werð— &
ne \alst{g}alpo þú far þínun \alst{g}evun te swíðo, \hld\ noh ênig \alst{g}umono ne skal, &
þat siu im þurh \alst{í}dale hróm \hld\ \alst{e}ft ni werðe &
\alst{l}êð-líko far·\alst{l}oren. \hld\ Þanna þú skalt \alst{l}ôn nemen &
fora \alst{g}odes ôgun \hld\ \alst{g}ódero werko. &
Ôk skal ik iu ge·\alst{b}eodan, \hld\ þan gi willjad te \alst{b}edu hnígan &
ęndi willjad te iuwomu \alst{h}êrron \hld\ \alst{h}elpono biddjan, &
þat hé iu a·\alst{l}áte \hld\ \alst{l}êðes þinges, &
þero \alst{s}akono ęndi þero \alst{s}undjono, \hld\ þea gi iu \alst{s}elvon hír &
\alst{w}rêða ge·\alst{w}irkjad, \hld\ þat gi it þan for ȯðrumu \alst{w}erode ni duad: &
ni \alst{m}árjad it far \alst{m}ęnigi, \hld\ þat iu þes \alst{m}an ni lovon, &
ni \alst{d}iurjan þero \alst{d}ádjo, \hld\ þat gi iuwes \alst{d}rohtines gi·bed &
þurh þat \alst{í}dala hróm \hld\ \alst{a}l ne far·leosan. &
Ak þan gí willjan te iuwomo \alst{h}êrron \hld\ \alst{h}elpono biddjan, &
\alst{þ}iggjan \alst{þ}eo-líko, \hld\ —þes iu is \alst{þ}arf mikil— &
þat iu \alst{s}igi-drohtin \hld\ \alst{s}undjono tómja, &
þan \alst{d}ót gi þat só \alst{d}arno: \hld\ þoh wêt it iuwe \alst{d}rohtin self &
\alst{h}êlag an \alst{h}imile, \hld\ hwand imu nis bi·\alst{h}olan n·eo·wiht &
ne \alst{w}ordo ne \alst{w}erko. \hld\ hé látid it þan al ge·\alst{w}erðan só, &
só gi ina þan \alst{b}iddjad, \hld\ þan gi te þero \alst{b}edo hnígad &
mid \alst{h}luttru \alst{h}ugi.“ \hld\ \alst{H}ęliðos stódun, &
\alst{g}umon umbi þana \alst{g}odes sunu \hld\ \alst{g}erno swíðo, &
\alst{w}eros an \alst{w}illjon: \hld\ was im þero \alst{w}ordo niud, &
\alst{þ}ȧhtun ęndi \alst{þ}agodun, \hld\ was im \alst{þ}arf mikil, &
þat sie þat eft ge·\alst{h}ogdin, \hld\ þat im þat \alst{h}êlaga barn &
an þana \alst{f}orman sïð \hld\ \alst{f}ilu mid wordun &
\alst{t}orhtes ge·\alst{t}alde. \hld\ Þȯ sprak im eft ên þero \alst{t}we-livjo an·gęgin, &
\alst{g}lauworo \alst{g}umono, \hld\ te þem \alst{g}odes barne:\eva

\bvb TODO.\evb\evg

\bvg\bva[19][1588]%
„\alst{H}êrro þe gódo“, \hld\ kwað hé, „u̇s is þínoro \alst{h}uldi þarf, &
te gi·\alst{w}irkenne þínna \alst{w}illjon, \hld\ ęndi ôk þínoro \alst{w}ordo só self, &
allaro \alst{b}arno \alst{b}ętst, \hld\ þat þú u̇s \alst{b}edon lêres, &
\alst{j}ungoron þíne, \hld\ só \alst{J}ohannes duot, &
\alst{d}iur-lík \alst{d}ôperi, \hld\ \alst{d}ago ge·hwi-likas &
is \alst{w}erod mid \alst{w}ordun, \hld\ hwí sie \alst{w}aldand skulun, &
\alst{g}ódan \alst{g}rótjan. \hld\ Dó þína \alst{j}ungorun só self: &
ge·\alst{r}ihti u̇s þat ge·\alst{r}úni.“ \hld\ Þȯ habda eft þe \alst{r}íkjo garu &
\alst{s}án aftar þiu, \hld\ \alst{s}unu drohtines, &
\alst{g}ód word an·\alst{g}ęgin: \hld\ „Þan gi \alst{g}od willjan“, kwað hé, &
„\alst{w}eros mid iuwon \alst{w}ordun \hld\ \alst{w}aldand grótjan, &
allaro \alst{k}uningo \alst{k}raftigostan, \hld\ þan \alst{k}weðad gi, só ik iu lêrju: &
‚\alst{F}adar u̇sa \hld\ \alst{f}iriho barno, &
þú bist an þem \alst{h}ôhon \hld\ \alst{h}imila ríkja, &
ge·\alst{w}íhid sí þín namo \hld\ \alst{w}ordo ge·hwi-liko. &
\alst{K}uma þín \hld\ \alst{k}raftag ríki. &
\alst{W}erða þín \alst{w}illjo \hld\ ovar þesa \alst{w}er-old alla, &
só sama an erðo, \hld\ só þár \alst{u}ppa ist &
an þem \alst{h}ôhon \hld\ \alst{h}imilo ríkja. &
Gef u̇s \alst{d}ago ge·hwi-likes rád, \hld\ \alst{d}rohtin þe gódo, &
þína \alst{h}êlaga \alst{h}elpa, \hld\ ęndi a·lát u̇s, \alst{h}evanes ward, &
\alst{m}anagoro \alst{m}ên-skuldjo, \hld\ al só we ȯðrum \alst{m}annum dóan. &
Ne lát u̇s far·\alst{l}êdjan \hld\ \alst{l}êða wihti &
só forð an iro \alst{w}illjon, \hld\ só wí \alst{w}irðige sind, &
ak help u̇s wiðar \alst{a}llun \hld\ \alst{u}vilon dádjun.‘ &
Só skulun gí \alst{b}iddjan, \hld\ þan gi te \alst{b}ede hnígad &
\alst{w}eros mid iuwom \alst{w}ordun, \hld\ þat iu \alst{w}aldand god &
\alst{l}êðes a·\alst{l}áte \hld\ an \alst{l}eut-kunnja. &
Ef gi þan willjad a·\alst{l}átan \hld\ \alst{l}iudjo ge·hwi-likun &
þero \alst{s}akono ęndi þero \alst{s}undjono, \hld\ þe sie wið iu \alst{s}elvon hír &
\alst{w}rêða ge·\alst{w}irkjat, \hld\ þan a·látid iu \alst{w}aldand god, &
\alst{f}adar ala-mahtig \hld\ \alst{f}irin-werk mikil, &
\alst{m}anagoro \alst{m}ên-skuldjo. \hld\ Ef iu þan wirðid iuwa \alst{m}ód te stark, &
þat gi ne wiljat \alst{ȯ}ðrun \hld\ \alst{e}rlun a·látan, &
\alst{w}eron \alst{w}am-dádi, \hld\ þan ne wil iu ôk \alst{w}aldand god &
\alst{g}rim-werk far·\alst{g}evan, \hld\ ak gi skulun is \alst{g}eld niman, &
swíðo \alst{l}êð-lik \alst{l}ôn \hld\ te \alst{l}anguru hwílu, &
\alst{a}lles þes \alst{u}n-rehtes, \hld\ þes gi \alst{ȯ}ðrum hír &
gi·\alst{l}êstjad an þesumu \alst{l}iohte \hld\ ęndi þan wið \alst{l}iudjo barn &
þea \alst{s}aka ni gi·\alst{s}ónjad, \hld\ êr gi an þana \alst{s}ïð faran, &
\alst{w}eros fon þesoro \alst{w}er-oldi. \hld\ Ok skal ik iu te \alst{w}árun sęggjan, &
hwó gi \alst{l}êstjan skulun \hld\ \alst{l}êra mína: &
þan gi iuwa \alst{f}astonnja \hld\ \alst{f}rummjan willjan, &
\alst{m}inson iuwa \alst{m}ên-dádi, \hld\ þan ni duad gi þat te \alst{m}anagom ku̇ð, &
ak \alst{m}íðad is far ȯðrum \alst{m}annun: \hld\ þoh wêt \alst{m}ahtig god, &
\alst{w}aldand iuwan \alst{w}illjan, \hld\ þoh iu \alst{w}erod ȯðar, &
\alst{l}iudjo barn ne \alst{l}ovon. \hld\ hé gildid is iu \alst{l}ôn aftar þiu, &
iuwa \alst{h}êlag fadar \hld\ an \alst{h}imil-ríkja, &
þes ge im mid su·likum \alst{ô}d-módja, \hld\ \alst{e}rlos þeonod, &
só \alst{f}erht-líko undar þesumu \alst{f}olke. \hld\ Ne willjat \alst{f}eho winnan &
\alst{e}rlos an \alst{u}n-reht, \hld\ ak wirkjad \alst{u}p te gode &
\alst{m}an aftar \alst{m}édu: \hld\ þat is \alst{m}êra þing, &
þan man hír an \alst{e}rðu \hld\ \alst{ô}dag libbja, &
\alst{w}er-old-skattes ge·\alst{w}ono. \hld\ Ef gi willjad mínun \alst{w}ordun hôrjan, &
þan ne \alst{s}amnod gi hír \alst{s}ink mikil \hld\ \alst{s}ilọvres ne goldes &
an þesoro \alst{m}iddil-gard, \hld\ \alst{m}êðọm-hordes, &
hwand it \alst{r}otat hír an \alst{r}oste, \hld\ ęndi \alst{r}ęgin-þeovos far·stelad, &
\alst{w}urmi a·\alst{w}ardjad, \hld\ wirðid þat gi·\alst{w}ádi far·slitan, &
ti·\alst{g}angid þe \alst{g}old-welo. \hld\ Lêstjad iuwa \alst{g}ódon werk, &
samnod iu an \alst{h}imile \hld\ \alst{h}ord þat méra, &
\alst{f}agạra \alst{f}eho-skattos: \hld\ þat ni mag iu ênig \alst{f}íund be·niman, &
ne-\alst{w}iht an·\alst{w}ęndjan, \hld\ hwand þe \alst{w}elo standid &
\alst{g}aru iu te·\alst{g}ęgnes, \hld\ só hwat só gí \alst{g}ódes þarod, &
an þat \alst{h}imil-ríki \hld\ \alst{h}ordes ge·samnod, &
\alst{h}ęliðos þurh iuwa \alst{h}and-geva, \hld\ ęndi hębbjad þarod iuwan \alst{h}ugi fasto; &
hwand þár ist alloro \alst{m}anno gi·hwes \hld\ \alst{m}ód-ge·þȧhti, &
\alst{h}ugi ęndi \alst{h}erta, \hld\ þár is \alst{h}ord ligid, &
\alst{s}ink ge·\alst{s}amnod. \hld\ Nis eo só \alst{s}álig man, &
þat mugi an þesoro \alst{b}rêdon wer-old \hld\ \alst{b}êðju ant·hengjan, &
ge þat hí an þesoro \alst{e}rðo \hld\ \alst{ô}dag libbja, &
an allun \alst{w}er-old-lustun \alst{w}esa, \hld\ ge þoh \alst{w}aldand gode &
te \alst{þ}anke ge·\alst{þ}eono: \hld\ ak hé skal alloro \alst{þ}ingo gi·hwes &
simbla \alst{ȯ}ðar-hweðar \hld\ \alst{ê}n far·látan &
eþþo \alst{l}usta þes \alst{l}ík-hamon \hld\ eþþo \alst{l}íf êwig. &
Be·þiu ni \alst{g}ornot gi umbi iuwa ge·\alst{g}aruwi, \hld\ ak huggjad te \alst{g}ode fasto, &
ne \alst{m}ornont an iuwomu \alst{m}óde, \hld\ hwat gi eft an \alst{m}organ skulin &
\alst{e}tan efþo drinkan \hld\ eþþo \alst{a}n hębbjan &
\alst{w}eros te ge·\alst{w}ę́dja: \hld\ it wêt al \alst{w}aldand god, &
hwes þea bi·\alst{þ}urvun, \hld\ þea im hír \alst{þ}ionod wel, &
\alst{f}olgod iro \alst{f}rôhan willjon. \hld\ Hwat gi þat bi þesun \alst{f}uglun mugun &
\alst{w}ár-líko undar·\alst{w}itan, \hld\ þea hír an þesoro \alst{w}er-oldi sint, &
\alst{f}arad an \alst{f}eðar-hamun: \hld\ sie ni kunnun ênig \alst{f}eho winnan, &
þoh givid im \alst{d}rohtin god \hld\ \alst{d}ago ge·hwi-likes &
\alst{h}elpa wiðar \alst{h}ungre. \hld\ Ôk mugun gi an iuwom \alst{h}ugi markon, &
\alst{w}eros umbi iuwa ge·\alst{w}ádi, \hld\ hwó þie \alst{w}urti sint &
\alst{f}agọro ge·\alst{f}ratohot, \hld\ þea hír an \alst{f}elde stád, &
\alst{b}erht-líko ge·\alst{b}lóid: \hld\ ne mahta þe \alst{b}urges ward, &
\alst{S}alomon þe \alst{s}uning, \hld\ þe habda \alst{s}ink mikil, &
\alst{m}êðọm-hordas \alst{m}êst, \hld\ þero þe ênig \alst{m}an êhti, &
\alst{w}elono ge·\alst{w}unnan \hld\ ęndi allaro ge·\alst{w}ádjo kust,— &
þoh ni mohte hé an is \alst{l}íve, \hld\ þoh hé habdi alles þeses \alst{l}andes ge·wald, &
a·\alst{w}innan su·lik ge·\alst{w}ádi, \hld\ só þiu \alst{w}urt havad, &
þiu hír an \alst{f}elde stád \hld\ \alst{f}agọro ge·gariwit, &
\alst{l}illi mid só \alst{l}iof-líku blómon: \hld\ ina wádit þe \alst{l}andes waldand &
hér fan \alst{h}evanes wange. \hld\ Mér is im þoh umbi þit \alst{h}ęliðo kunni, &
\alst{l}iudi sint im \alst{l}iovoron mikilu, \hld\ þea hé im an þesumu \alst{l}ande ge·warhte, &
\alst{w}aldand an \alst{w}illjon sínan. \hld\ Be·þiu ne þurvon gi umbi iuwa ge·\alst{w}ádi sorgon, &
ne \alst{g}ornot gi umbi iuwa ge·\alst{g}ariwi te swíðo: \hld\ \alst{g}od wili is alles rádan, &
\alst{h}elpan fan \alst{h}evanes wange, \hld\ ef gi willjad aftar is \alst{h}uldi þeonon. &
\alst{G}erot gi simbla êrist þes \alst{g}odes ríkjas, \hld\ ęndi þan duat aftar þem is \alst{g}ódun werkun, &
\alst{r}ómod gi \alst{r}ehtoro þingo: \hld\ þan wili iu þe \alst{r}íkjo drohtin &
\alst{g}evon mid alloro \alst{g}ódu ge·hwi-liku, \hld\ ef gi im þus ful·\alst{g}angan willjad, &
só ik iu te \alst{w}árun hír \hld\ \alst{w}ordun sęggjo.\eva

\bvb TODO.\evb\evg

\bvg\bva[20][1691]%
Ne skulun gí \alst{ê}nigumu manne \hld\ \alst{u}n-rehtes wiht, &
\alst{d}ęrvjes a·\alst{d}êljan, \hld\ hwand þe \alst{d}óm eft kumid &
ovar þana \alst{s}elvon man, \hld\ þár it im te \alst{s}orgon skal, &
\alst{w}erðan þem te \alst{w}ítja, \hld\ þe hír mid is \alst{w}ordun ge·sprikid &
\alst{u}n-reht \alst{ȯ}ðrum. \hld\ Neo þat iuwar \alst{ê}nig ne dua &
\alst{g}umono an þesom \alst{g}ardon \hld\ \alst{g}eldes eþþo kôpes, &
þat hi \alst{u}n-reht gi·met \hld\ \alst{ȯ}ðrumu manne &
\alst{m}ên-ful \alst{m}ako, \hld\ hwand it simbla \alst{m}ótjan skal &
\alst{e}rlo ge·hwi-likomu, \hld\ su·lik só hé it \alst{ȯ}ðrumu ge·dód, &
só kumid it im eft te·\alst{g}ęgnes, \hld\ þár hé \alst{g}erno ne wili &
ge·\alst{s}ehan is \alst{s}undjon. \hld\ Ôk skal ik iu \alst{s}ęggjan noh, &
hwar gi iu \alst{w}ardon skulun \hld\ \alst{w}ítjo mêsta, &
\alst{m}ên-werk \alst{m}anag: \hld\ te hwí skalt þú ênigan \alst{m}an be·sprekan, &
\alst{b}róðar þínan, \hld\ þat þú undar is \alst{b}ráhon ge·sehas &
\alst{h}alm an is ôgon, \hld\ ęndi ge·\alst{h}uggjan ni wili &
þana \alst{s}wáran balkon, \hld\ þe þú an þínoro \alst{s}iuni havas, &
\alst{h}ard trio ęndi \alst{h}ęvig. \hld\ Lát þi þat an þínan \alst{h}ugi fallan, &
hwó þú þana êrist a·\alst{l}ôsjas: \hld\ þan skínid þi \alst{l}ioht be·foran, &
\alst{ô}gun werðad þi ge·\alst{o}ponot; \hld\ þan maht þú \alst{a}ftar þiu &
\alst{s}wáses mannes ge·\alst{s}iun \hld\ \alst{s}ïðor ge·bótjan, &
ge·\alst{h}êljan an is \alst{h}ôvde. \hld\ Só mag þat an is \alst{h}ugi méra &
an þesoro \alst{m}iddil-gard \hld\ \alst{m}anno ge·hwi-likumu, &
\alst{w}esan an þesoro \alst{w}er-oldi, \hld\ þat hi hír \alst{w}ammas ge·duot, &
þan hi \alst{a}htogja \hld\ \alst{ȯ}ðres mannes &
\alst{s}aka ęndi \alst{s}undja, \hld\ ęndi havad im \alst{s}elvo mêr &
\alst{f}irin-werko ge·\alst{f}rumid. \hld\ Ef hé wili is \alst{f}ruma lêstjan, &
þan skal hí ina \alst{s}elvon êr \hld\ \alst{s}undjono a·tómjan, &
\alst{l}êð-werko \alst{l}ôson: \hld\ sïðor mag hí mid is \alst{l}êrun werðan &
\alst{h}ęliðun te \alst{h}elpu, \hld\ sïðor hí ina \alst{h}luttran wêt, &
\alst{s}undjono \alst{s}ikoran. \hld\ Ne skulun gí \alst{s}wínum te·foran &
iuwa \alst{m}ęre-gríton makon \hld\ eþþo \alst{m}êðmo ge·striuni, &
\alst{h}êlag \alst{h}als-męni, \hld\ hwand siu it an \alst{h}oru spurnat, &
\alst{s}ulwjad an \alst{s}ande: \hld\ ne witun \alst{s}úvrjas ge·skêð, &
\alst{f}agạroro \alst{f}ratoho. \hld\ Su-lik sint hír \alst{f}olk manag, &
þe iuwa \alst{h}êlag word \hld\ \alst{h}ôrjan ne willjad, &
ful-gangan \alst{g}odes lêrun: \hld\ ne witun \alst{g}ódes ge·skêð, &
ak sind im \alst{l}ári word \hld\ \alst{l}eovoron mikilu, &
umbi·\alst{þ}arvi \alst{þ}ing, \hld\ þanna \alst{þ}eot-godes &
\alst{w}erk ęndi \alst{w}illjo. \hld\ Ne sind sie \alst{w}irðige þan, &
þat sie ge·\alst{h}ôrjan iuwa \alst{h}êlag word, \hld\ ef sie is ne willjad an iro \alst{h}ugi þęnkjan, &
ne \alst{l}ínon ne \alst{l}êstjan. \hld\ Þem ni sęggjan gi iuworo \alst{l}êron wiht, &
þat gi þea \alst{sp}ráka godes \hld\ ęndi \alst{sp}el managu &
ne far·\alst{l}eosan an þem \alst{l}iudjun, \hld\ þea þár ne willjan gi·\alst{l}ôvjan tó, &
\alst{w}ároro \alst{w}ordo. \hld\ Ôk skulun gí iu \alst{w}ardon filu &
\alst{l}istjun undar þesun \alst{l}iudjun, \hld\ þár gí aftar þesumu \alst{l}ande farad, &
þat iu þea \alst{l}uggjon ne mugin \hld\ \alst{l}êron be·swíkan &
ni mid \alst{w}ordun ni mid \alst{w}erkun. \hld\ Sie kumad an su·likom ge·\alst{w}ádjon te iu, &
\alst{f}agọron \alst{f}ratohon: \hld\ þoh hębbjad sie \alst{f}êknan hugi: &
þea mugun gí sán ant·\alst{k}ęnnjan, \hld\ só gí sie \alst{k}uman ge·sehad: &
sie sprekad \alst{w}ís-lík \alst{w}ord, \hld\ þoh iro \alst{w}erk ne dugin, &
þero \alst{þ}egno ge·\alst{þ}ȧhti. \hld\ Hwand gí witun, þat eo an \alst{þ}ornjun ne skulun &
\alst{w}ín-beri \alst{w}esan \hld\ efþa \alst{w}elon eo·wiht, &
\alst{f}agọroro \alst{f}ruhtjo, \hld\ nek ôk \alst{f}ígun ne lesad &
\alst{h}ęliðos an \alst{h}iopon. \hld\ Þat mugun gi undar·\alst{h}uggjan wel, &
þat \alst{e}o þe \alst{u}vilo bôm, \hld\ þár hé an \alst{e}rðu stád, &
\alst{g}óden wastum ne \alst{g}ivid, \hld\ nek it ôk \alst{g}od ni ge·skóp, &
þat þe \alst{g}ódo bôm \hld\ \alst{g}umono barnun &
\alst{b}ári \alst{b}ittres wiht, \hld\ ak kumid fan alloro \alst{b}âmo ge·hwi-likumu &
su·lik \alst{w}astom te þesero \alst{w}er-oldi, \hld\ só im fan is \alst{w}urtjon ge·dregid, &
eþþa \alst{b}erht eþþa \alst{b}ittar. \hld\ Þat mênid þoh \alst{b}reost-hugi, &
\alst{m}anagoro \alst{m}ód-sevon \hld\ \alst{m}anno kunnjes, &
hwó \alst{a}lloro \alst{e}rlo ge·hwi-lik \hld\ \alst{ô}git selvo, &
\alst{m}eldod mid is \alst{m}u̇ðu, \hld\ hwi-likan hé \alst{m}ód havad, &
\alst{h}ugi umbi is \alst{h}erte: \hld\ þes ni mag hé far·\alst{h}elan eo·wiht, &
ak kumad fan þem \alst{u}vilan man \hld\ \alst{i}n-wid-rádos, &
\alst{b}ittara \alst{b}alu-spráka, \hld\ su·lik só hi an is \alst{b}reostun havad &
ge·\alst{h}ęftid umbi is \alst{h}erte: \hld\ simbla is \alst{h}ugi ku̇ðid, &
is \alst{w}illjon mid is \alst{w}ordun, \hld\ ęndi farad is \alst{w}erk aftar þiu. &
Só kumad fan þemu \alst{g}ódan manne \hld\ \alst{g}lau and-wordi, &
\alst{w}ís-lík fan is ge·\alst{w}ittja, \hld\ þat hi simbla mid is \alst{w}ordu ge·sprikid, &
\alst{m}an mid is \alst{m}íðu su·lik, \hld\ só hé an is \alst{m}óde havad &
\alst{h}ord umbi is \alst{h}erte. \hld\ Þanan kumad þea \alst{h}êlagan lêra, &
swíðo \alst{w}un-sam \alst{w}ord, \hld\ ęndi skulun is \alst{w}erk aftar þiu &
\alst{þ}eodu ge·\alst{þ}íhan, \hld\ \alst{þ}egnun managun &
\alst{w}erðan te \alst{w}illjon, \hld\ al só it \alst{w}aldand self &
\alst{g}ódun mannun far·\alst{g}ivid, \hld\ \alst{g}od alo-mahtig, &
\alst{h}imilisk \alst{h}êrro, \hld\ hwand sie áno is \alst{h}elpa ni mugun &
ne mid \alst{w}ordun ne mid \alst{w}erkun \hld\ \alst{w}iht a·þęngjan &
\alst{g}ódes an þesun \alst{g}ardun. \hld\ Be·þiu skulun \alst{g}umono barn &
an is \alst{ê}nes kraft \hld\ \alst{a}lle gi·lôvjan.\eva

\bvb TODO.\evb\evg

\bvg\bva[21][1771]%
Ôk skal ik iu \alst{w}ísjan, \hld\ hwó hír \alst{w}egos twêna &
\alst{l}iggjad an þesumu \alst{l}iohte, \hld\ þea farad \alst{l}iudjo barn, &
\alst{a}l \alst{i}rmin-þiod. \hld\ Þero is \alst{ȯ}ðar sán &
\alst{w}íd stráta ęndi brêd, \hld\ —farid sie \alst{w}erodes filu, &
\alst{m}an-kunnjes \alst{m}anag, \hld\ hwand sie þarod iro \alst{m}ód spęnit, &
\alst{w}er-old-lusta \alst{w}eros— \hld\ þiu an þea \alst{w}irson hand &
\alst{l}iudi \alst{l}êdid, \hld\ þár sie te far·\alst{l}ora werðad, &
\alst{h}ęliðos an \alst{h}ęllju, \hld\ þár is \alst{h}êt ęndi swart, &
\alst{ę}gis-lík an \alst{i}nnan: \hld\ \alst{ó}ði ist þarod te faranne &
\alst{ę}ldi-barnun, \hld\ þoh it im at þemu \alst{ę}ndje ni dugi. &
Þan ligid \alst{e}ft \alst{ȯ}ðar \hld\ \alst{ę}ngira mikilu &
\alst{w}eg an þesoro \alst{w}er-oldi, \hld\ fęrid ina \alst{w}erodes lút, &
\alst{f}áho \alst{f}olk-skępi: \hld\ ni willjad ina \alst{f}iriho barn &
\alst{g}erno \alst{g}angan, \hld\ þoh hé te \alst{g}odes ríkja, &
an þat \alst{ê}wiga líf, \hld\ \alst{e}rlos lêdja. &
Þan nimad gí iu þana \alst{ę}ngjan: \hld\ þoh hé só \alst{ó}ði ne sí &
\alst{f}irihon te \alst{f}aranne, \hld\ þoh skal hi te \alst{f}rumu werðan &
só hwemu só ina þurh·\alst{g}ęngid, \hld\ só skal is \alst{g}eld niman, &
swíðo \alst{l}ang-sam \alst{l}ôn \hld\ ęndi \alst{l}íf êwig, &
\alst{d}iur-líkan \alst{d}rôm. \hld\ Eo gi þes \alst{d}rohtin skulun, &
\alst{w}aldand biddjen, \hld\ þat gi þana \alst{w}eg mótin &
\alst{f}an foran ant·\alst{f}ȧhan \hld\ ęndi \alst{f}orð þurh gi·gangan &
an þat \alst{g}odes ríki. \hld\ hé ist \alst{g}aru simbla &
wiðar þiu te \alst{g}evanne, \hld\ þe man ina \alst{g}erno bidid, &
\alst{f}ergot \alst{f}iriho barn. \hld\ Sókjad \alst{f}adar iuwan &
\alst{u}p te þemu \alst{ê}winom ríkja: \hld\ þan mótun gi ina \alst{a}ftar þiu &
te iuworu \alst{f}rumu \alst{f}ïðan. \hld\ Ku̇ðjad iuwa \alst{f}ard þarod &
at iuwas \alst{d}rohtines \alst{d}urun: \hld\ þan werðad iu an·\alst{d}ón aftar þiu, &
\alst{h}imil-portun ant·\alst{h}lidan, \hld\ þat gi an þat \alst{h}êlage lioht, &
an þat \alst{g}odes ríki \hld\ \alst{g}angan mótun, &
\alst{s}in-líf \alst{s}ehan. \hld\ Ôk skal ik iu \alst{s}ęggjan noh &
far þesumu \alst{w}erode allun \hld\ \alst{w}ár-lík biliði, &
þat alloro \alst{l}iudjo só hwi-lik, \hld\ só þesa mína \alst{l}êra wili &
ge·\alst{h}aldan an is \alst{h}erton \hld\ ęndi wil iro an is \alst{h}ugi a·þęnkjan, &
\alst{l}êstjan sea an þesumu \alst{l}ande, \hld\ þe gi·\alst{l}íko duot &
\alst{w}ísumu manne, \hld\ þe gi·\alst{w}it havad, &
\alst{h}orska \alst{h}ugi-skęfti, \hld\ ęndi \alst{h}ús-stędi kiusid &
an \alst{f}astoro \alst{f}oldun \hld\ ęndi an \alst{f}elisa uppan &
\alst{w}égos \alst{w}irkid, \hld\ þár im \alst{w}ind ni mag, &
ne wág ne \alst{w}atares strôm \hld\ \alst{w}ihtju ge·tiunjan, &
ak mag im þár wið \alst{u}n-gi·widerjon \hld\ \alst{a}llun standan &
an þemu \alst{f}elise uppan, \hld\ hwand it só \alst{f}asto warð &
gi·\alst{st}ellit an þemu \alst{st}êne: \hld\ ant·havad it þiu \alst{st}ędi niðana, &
\alst{w}ręðid wiðar \alst{w}inde, \hld\ þat it \alst{w}íkan ni mag. &
Só duot eft \alst{m}anno só hwi-lik, \hld\ só þesun \alst{m}ínun ni wili &
\alst{l}êrun hôrjen \hld\ ne þero \alst{l}êstjen wiht; &
só duot þe \alst{u}n-wíson \hld\ \alst{e}rla ge·líko, &
un-ge·\alst{w}ittigon \alst{w}ere, \hld\ þe im be \alst{w}atares staðe &
an \alst{s}ande wili \hld\ \alst{s}ęli-hús wirkjan, &
þár it \alst{w}estrani \alst{w}ind \hld\ ęndi \alst{w}ágo strôm, &
\alst{s}êes u̇ðjon te·\alst{s}láad; \hld\ ne mag im \alst{s}and ęndi greot &
ge·\alst{w}ręðjan wið þemu \alst{w}inde, \hld\ ak wirðid te·\alst{w}orpan þan, &
te·\alst{f}allen an þemu \alst{f}lóde, \hld\ hwand it an \alst{f}astoro nis &
\alst{e}rðu ge·timbrod. \hld\ Só skal allaro \alst{e}rlo ge·hwes &
\alst{w}erk ge·þïhan \alst{w}iðar þiu, \hld\ þe hi þius mín \alst{w}ord frumid, &
\alst{h}aldid \alst{h}êlag ge·bod.“ \hld\ Þȯ bi·gunnun an iro \alst{h}ugi wundron &
\alst{m}ęgin-folk \alst{m}ikil: \hld\ ge·hôrdun \alst{m}ahtiges godes &
\alst{l}iof-líka \alst{l}êra; \hld\ ne wárun an þemu \alst{l}ande ge·wuno, &
þat sie eo fan \alst{s}u·likun êr \hld\ \alst{s}ęggjan ge·hôrdin &
\alst{w}ordun eþþo \alst{w}erkun. \hld\ Far·stódun \alst{w}íse man, &
þat hé só \alst{l}êrde, \hld\ \alst{l}iudjo drohtin, &
\alst{w}árun \alst{w}ordun, \hld\ só hé ge·\alst{w}ald habde, &
\alst{a}llun þem \alst{u}n-ge·líko, \hld\ þe þár an \alst{ê}r-dagun &
undar þem \alst{l}iud-skępja \hld\ \alst{l}êrjon wárun &
a·\alst{k}oran undar þemu \alst{k}unnje: \hld\ ne habdun þiu \alst{K}ristes word &
ge·\alst{m}akon mid \alst{m}annun, \hld\ þe hé far þero \alst{m}ęnigi sprak, &
ge·\alst{b}ôd uppan þemu \alst{b}erge.\eva

\bvb TODO.\evb\evg

\bvg\bva[22][1837]%
\hspace*{100pt} Hé im þȯ \alst{b}êðju be·falh &
te ge·\alst{s}ęggennja \hld\ \alst{s}ínom wordun, &
hwó man \alst{h}imil-ríki \hld\ ge·\alst{h}alon skoldi, &
\alst{w}íd-brêdan \alst{w}elan, \hld\ gia hé im ge·\alst{w}ald far·gaf, &
þat sie móstin \alst{h}êljan \hld\ \alst{h}alte ęndi blinde, &
\alst{l}iudjo \alst{l}éf-hêdi, \hld\ \alst{l}egar-będ manag, &
\alst{s}wára \alst{s}uhti, \hld\ giak hé im \alst{s}elvo ge·bôd, &
þat sie at ênigumu \alst{m}anne \hld\ \alst{m}éde ne námin, &
\alst{d}iurje mêðmos: \hld\ „ge·huggjad gi“, kwað hé, —„hwand iu is þiu \alst{d}ád kuman, &
þat ge·\alst{w}it ęndi þe \alst{w}ís-dóm, \hld\ ęndi iu þea ge·\alst{w}ald far·givid &
alloro \alst{f}iriho \alst{f}adar, \hld\ só gi sie ni þurvun mid ênigo \alst{f}eho kôpon, &
\alst{m}édjan mid ênigun \alst{m}êðmun,— \hld\ só wesat gi iro \alst{m}annun forð &
an iuwon \alst{h}ugi-skęftjun \hld\ \alst{h}elpono mildja, &
\alst{l}êrjad gi \alst{l}iudjo barn \hld\ \alst{l}ang-samna rád, &
\alst{f}ruma \alst{f}orð-wardes; \hld\ \alst{f}irin-werk lahad, &
\alst{s}wára \alst{s}undjon. \hld\ Ne látad iu \alst{s}ilọvar nek gold &
\alst{w}ihti þes \alst{w}irðig, \hld\ þat it eo an iuwa ge·\alst{w}ald kuma, &
\alst{f}agạra \alst{f}eho-skattos: \hld\ it ni mag iu te ênigoro \alst{f}rumu hwęrgin, &
\alst{w}erðan te ênigumu \alst{w}illjon. \hld\ Ne skulun gi ge·\alst{w}ádjas þan mêr &
\alst{e}rlos \alst{ê}gan, \hld\ b·útan só gi þan \alst{a}n hębbjan, &
\alst{g}umon te \alst{g}arewja, \hld\ þan gi \alst{g}angan skulun &
an þat gi·\alst{m}ang innan. \hld\ Neo gi umbi iuwan \alst{m}ęti ni sorgot, &
\alst{l}ęng umbi iuwa \alst{l}íf-nare, \hld\ hwand þene \alst{l}êrjand skulun &
\alst{f}ódjan þat \alst{f}olk-skępi: \hld\ þes sint þea \alst{f}ruma werða, &
\alst{l}eov-líkes \alst{l}ônes, \hld\ þe hi þem \alst{l}iudjun sagad. &
\alst{w}irðig is þe \alst{w}urhtjo, \hld\ þat man ina \alst{w}el fódja, &
þana \alst{m}an mid \alst{m}ósu, \hld\ þe só \alst{m}anagoro skal &
\alst{s}eola bi·\alst{s}organ \hld\ ęndi an þana \alst{s}ïð spanen, &
\alst{g}êstos an \alst{g}odes wang. \hld\ Þat is \alst{g}rôtara þing, &
þat man bi·\alst{s}orgon skal \hld\ \alst{s}eolun managa, &
hwó man þea ge·\alst{h}alde \hld\ te \alst{h}evan-ríkja, &
þan man þene \alst{l}ík-hamon \hld\ \alst{l}iudi-barno &
\alst{m}ósu bi·\alst{m}orna. \hld\ Be·þiu \alst{m}an skulun &
\alst{h}aldan þene \alst{h}old-líko, \hld\ þe im te \alst{h}evan-ríkja &
þene \alst{w}eg wísit \hld\ ęndi sie \alst{w}am-skaðun, &
\alst{f}eondun wit·\alst{f}áhit \hld\ ęndi \alst{f}irin-werk lahid, &
\alst{s}wára \alst{s}undjon. \hld\ Nu ik iu \alst{s}ęndjan skal &
aftar þesumu \alst{l}and-skępje \hld\ só \alst{l}amb undar wulvos: &
só skulun gi undar iuwa \alst{f}íund \alst{f}aren, \hld\ undar \alst{f}ilu þeodo, &
undar \alst{m}is-líke \alst{m}an. \hld\ Hębbjad iuwan \alst{m}ód wiðar þem &
só \alst{g}lawan te·\alst{g}ęgnes, \hld\ só samo só þe \alst{g}elwo wurm, &
\alst{n}ádra þiu féha, \hld\ þár siu iro \alst{n}íð-skępjes, &
\alst{w}itodes \alst{w}ánit, \hld\ þat man iu undar þemu \alst{w}erode ne mugi &
be·\alst{s}wíkan an þemu \alst{s}ïðe. \hld\ Far þiu gi \alst{s}orgon skulun, &
þat iu þea \alst{m}an ni \alst{m}ugin \hld\ \alst{m}ód-ge·þȧhti, &
\alst{w}illjan a·\alst{w}ardjen. \hld\ Wesat iu so \alst{w}ara wiðar þiu, &
wið iro \alst{f}êknjon dádjun, \hld\ só man wiðar \alst{f}íundun skal. &
Þan wesat gí eft an iuwon \alst{d}ádjun \hld\ \alst{d}úvon ge·líka, &
hębbjad wið \alst{e}rlo ge·hwene \hld\ \alst{ê}n-faldan hugi, &
\alst{m}ildjan \alst{m}ód-sevon, \hld\ þat þár \alst{m}an neg·ên &
þurh iuwa \alst{d}ádi \hld\ be·\alst{d}rogan ne werðe, &
be·\alst{s}wikan þurh iuwa \alst{s}undja. \hld\ Nu skulun gí an þana \alst{s}ïð faran, &
an þat \alst{â}rundi: \hld\ þár skulun gí \alst{a}rvidjes só filu &
ge·\alst{þ}olon undar þeru \alst{þ}iod \hld\ ęndi ge·\alst{þ}wing só samo &
\alst{m}anag ęndi \alst{m}is-lík, \hld\ hwand gi an \alst{m}ínumu namon &
þea \alst{l}iudi \alst{l}êrjat. \hld\ Be·þiu skulun gi þár \alst{l}êðes filu &
fora \alst{w}er-old-kuningun, \hld\ \alst{w}ítjas ant·fȧhan. &
Oft skulun gi þár for \alst{r}íkja \hld\ þurh þius mín \alst{r}ehtun word &
ge·\alst{b}undane standen \hld\ ęndi \alst{b}êðju ge·þologjan, &
ge \alst{h}osk ge \alst{h}arm-kwidi: \hld\ umbi þat ne látad gi iuwan \alst{h}ugi twíflon, &
\alst{s}evon \alst{s}wíkandjan: \hld\ gi ni þurvun an ênigun \alst{s}orgun wesan &
an iuwomu \alst{h}ugi \alst{h}węrgin, \hld\ þan man iu for þea \alst{h}êri forð &
an þene \alst{g}ast-sęli \hld\ \alst{g}angan hêtid, &
hwat gi im þan te·\alst{g}ęgnes skulin \hld\ \alst{g}ódoro wordo, &
\alst{sp}áh-líkoro ge·\alst{sp}rekan, \hld\ hwand iu þiu \alst{sp}ód kumid, &
\alst{h}elpe fon \alst{h}imile, \hld\ ęndi sprikid þe \alst{h}êlogo gêst, &
\alst{m}ahtig fon iuwomu \alst{m}unde. \hld\ Be·þiu ne and-rádad gi iu þero \alst{m}anno níð &
ne \alst{f}orhtjat iro \alst{f}íund-skępi: \hld\ þoh sie hębbjan iuwas \alst{f}erạhes ge·wald, &
þat sie mugin þene \alst{l}ík-hamon \hld\ \alst{l}ívu be·neotan, &
a·\alst{s}lahan mid \alst{s}werde, \hld\ þoh sie þeru \alst{s}eolun ne mugun &
\alst{w}iht a·\alst{w}ardjan. \hld\ Ant-drádad iu \alst{w}aldand god, &
\alst{f}orhtjad \alst{f}ader iuwan, \hld\ \alst{f}rummjad gerno &
is ge·\alst{b}od-skępi, \hld\ hwand hi havad \alst{b}êðjes gi·wald, &
\alst{l}iudjo \alst{l}íves \hld\ ęndi ôk iro \alst{l}ík-hamon &
gek þero \alst{s}eolon só \alst{s}elf: \hld\ ef gi iuwa an þem \alst{s}ïðe þarod &
far·\alst{l}iosat þurh þesa \alst{l}êra, \hld\ þan mótun gi sie eft an þemu \alst{l}iohte godes &
be·\alst{f}oran \alst{f}ïðan, \hld\ hwand sie \alst{f}ader iuwa, &
\alst{h}aldid \alst{h}êlag god \hld\ an \alst{h}imil-ríkja.\eva

\bvb TODO.\evb\evg

\bvg\bva[23][1915]%
Ne kumat þea alle te \alst{h}imile, \hld\ þea þe hér \alst{h}rópat te mí &
\alst{m}anno te \alst{m}und-burd. \hld\ \alst{M}anaga sind þero, &
þea willjad alloro \alst{d}ago ge·hwi-likes \hld\ te \alst{d}rohtine hnígan, &
\alst{h}rópad þár te \alst{h}elpu \hld\ ęndi \alst{h}uggjad an ȯðar, &
\alst{w}irkjad \alst{w}am-dádi: \hld\ ne sind im þan þiu \alst{w}ord fruma, &
ak þea mótun \alst{h}wervan \hld\ an þat \alst{h}imiles lioht, &
\alst{g}angan an þat \alst{g}odes ríki, \hld\ þea þes \alst{g}erne sint, &
þat sie hír ge·\alst{f}rummjen \hld\ \alst{f}ader ala-waldan &
\alst{w}erk ęndi \alst{w}illjon. \hld\ Þea ni þurvun mid \alst{w}ordun só fílu &
\alst{h}rópan te \alst{h}elpu, \hld\ hwanda þe \alst{h}êlogo god &
wêt alloro \alst{m}anno ge·hwes \hld\ \alst{m}ód-ge·þȧhti, &
\alst{w}ord ęndi \alst{w}illjon, \hld\ ęndi gildid im is \alst{w}erko lôn. &
Be·þiu skulun gí \alst{s}orgon, \hld\ þan gí an þene \alst{s}ïð farad, &
hwó gi þat \alst{â}rundi \hld\ ti \alst{ę}ndja be·brengen. &
Þan gí \alst{l}íðan skulun \hld\ aftar þesumu \alst{l}and-skępja, &
\alst{w}ído aftar þesoro \alst{w}er-oldi, \hld\ al só iu \alst{w}egos lêdjad, &
\alst{b}rêd stráta te \alst{b}urg, \hld\ simbla sókjad gi iu þene \alst{b}ętston sán &
\alst{m}an undar þeru \alst{m}ęnegi \hld\ ęndi ku̇ðjad imu iuwan \alst{m}óð-sevon &
\alst{w}árun \alst{w}ordun. \hld\ Ef sie þan þes \alst{w}irðige sint, &
þat sie iuwa \alst{g}ódun werk \hld\ \alst{g}erno ge·lêstjen &
mid \alst{h}luttru \alst{h}ugi, \hld\ þan gi an þemu \alst{h}úse mid im &
\alst{w}onod an \alst{w}illjon \hld\ ęndi im wel \alst{l}ônod, &
\alst{g}eldad im mid \alst{g}ódu \hld\ ęndi sie te \alst{g}ode selvon &
\alst{w}ordun ge·\alst{w}íhad \hld\ ęndi sęggjad im \alst{w}issan friðu, &
\alst{h}êlaga \alst{h}elpa \hld\ \alst{h}evan-kuninges. &
Ef sie þan só \alst{s}áliga \hld\ þurh iro \alst{s}elvoro dád &
\alst{w}erðan ni mótun, \hld\ þat sie iuwa \alst{w}erk frummjen, &
\alst{l}êstjen iuwa \alst{l}êra, \hld\ þan gi fan þem \alst{l}iudjun sán, &
\alst{f}arad fan þemu \alst{f}olke, \hld\ —þe iuwa \alst{f}riðu hwirvid &
eft an iuworo \alst{s}elvoro \alst{s}ïð,— \hld\ ęndi látad sie mid \alst{s}undjun forð, &
mid \alst{b}alu-werkun \alst{b}úan \hld\ ęndi sókjad iu \alst{b}urg ȯðra, &
\alst{m}ikil \alst{m}an-werod, \hld\ ęndi ne látad þes \alst{m}elmes wiht &
\alst{f}olgan an iuwom \alst{f}ótun, \hld\ þanan þe man iu ant·\alst{f}ȧhan ne wili, &
ak \alst{sk}uddjat it fan iuwon \alst{sk}óhun, \hld\ þat it im eft te \alst{sk}amu werðe, &
þemu \alst{w}erode te ge·\alst{w}it-skępje, \hld\ þat iro \alst{w}illjo ne dôg. &
Þan sęggjo ik iu te \alst{w}árun, \hld\ só hwan só þius \alst{w}er-old ęndjad &
ęndi þe \alst{m}árjo dag \hld\ ovar \alst{m}an farid, &
þat þan \alst{S}odomo-burg, \hld\ þiu hír þurh \alst{s}undjon warð &
an \alst{a}f-grundi \hld\ \alst{ê}ldes kraftu, &
\alst{f}iuru bi·\alst{f}allen, \hld\ þat þiu þan havad \alst{f}riðu méran, &
\alst{m}ildiran \alst{m}und-burd, \hld\ þan þea \alst{m}an êgin, &
þe iu hír \alst{w}iðar-\alst{w}erpat \hld\ ęndi ne willjad iuwa \alst{w}ord frummjen. &
Só hwe só iu þan ant·\alst{f}áhit \hld\ þurh \alst{f}erhtan hugi, &
þurh \alst{m}ildjan \alst{m}ód, \hld\ só havad \alst{m}ínan forð &
\alst{w}illjon ge·\alst{w}arhten \hld\ ęndi ôk \alst{w}aldand god, &
ant·\alst{f}angan \alst{f}ader iuwan, \hld\ \alst{f}iriho drohtin, &
\alst{r}íkjan \alst{r}ád-gevon, \hld\ þene þe al \alst{r}eht bi·kan. &
\alst{w}êt \alst{w}aldand self, \hld\ ęndi \alst{w}illjan lônot &
\alst{g}umono ge·hwi-likumu, \hld\ só hwat só hi hír \alst{g}ódes ge·duot, &
þoh hi þurh \alst{m}innja godes \hld\ \alst{m}anno hwi-likumu &
\alst{w}illjandi far·geve \hld\ \alst{w}atares drinkan, &
þat hi \alst{þ}urftigumu manne \hld\ \alst{þ}urst ge·hêlje, &
\alst{k}aldes brunnan. \hld\ Þesa \alst{k}widi werðad wára, &
þat eo ne bi·\alst{l}ívid, \hld\ ne hi þes \alst{l}ôn skuli, &
fora \alst{g}odes ôgun \hld\ \alst{g}eld ant·fȧhan, &
\alst{m}éda \alst{m}anag-falde, \hld\ só hwat só hi is þurh mína \alst{m}innja ge·duot. &
Só hwe só mín þan far·\alst{l}ôgnid \hld\ \alst{l}iudi-barno, &
\alst{h}ęliðo for þesoro \alst{h}ęrju, \hld\ só dóm ik is an \alst{h}imile só self &
þár \alst{u}ppe far þem \alst{a}lo-waldan fader \hld\ ęndi for allumu is \alst{ę}ngilo krafte, &
far þeru \alst{m}ikilon \alst{m}ęnigi. \hld\ Só hwi-lik só þan eft \alst{m}anno barno &
an þesoro \alst{w}er-oldi ne wili \hld\ \alst{w}ordun míðan, &
ak \alst{g}ihit far \alst{g}um-skępi, \hld\ þat hé mín \alst{j}ungoro sí, &
þene willju ek \alst{e}ft \alst{ó}gjan \hld\ far \alst{ô}gun godes, &
fora alloro \alst{f}iriho \alst{f}ader, \hld\ þár \alst{f}olk manag &
for þene \alst{a}lo-waldon \hld\ \alst{a}lla gangad &
\alst{r}eðinon wið þene \alst{r}íkjon. \hld\ Þár willju ik imu an \alst{r}eht wesan &
\alst{m}ildi \alst{m}und-boro, \hld\ só hwemu só \alst{m}ínun hír &
\alst{w}ordun hôrid \hld\ ęndi þiu \alst{w}erk frumid, &
þea ik hír an þesumu \alst{b}erge uppan \hld\ ge·\alst{b}oden hębbju.“ &
Habda þȯ te \alst{w}árun \hld\ \alst{w}aldandes sunu &
ge·\alst{l}êrid þea \alst{l}iudi, \hld\ hwó sie \alst{l}of gode &
\alst{w}irkjan skoldin. \hld\ Þȯ lét hi þat \alst{w}erod þanan &
an alloro \alst{h}alva ge·hwi-lika, \hld\ \alst{h}ęri-skępi manno &
\alst{s}ïðon te \alst{s}elðon. \hld\ Habdun \alst{s}elves word, &
ge·\alst{h}ôrid \alst{h}evan-kuninges \hld\ \alst{h}êlaga lêra, &
só eo te \alst{w}er-oldi sint \hld\ \alst{w}ordo ęndi dádjo, &
\alst{m}an-kunnjes \alst{m}anag \hld\ ovar þesan \alst{m}iddil-gard &
\alst{sp}rákono þiu \alst{sp}áhiron, \hld\ só hwe só þiu \alst{sp}el ge·frang, &
þea þár an þemu \alst{b}erge ge·sprak \hld\ \alst{b}arno ríkjast.\eva

\bvb TODO.\evb\evg

\bvg\bva[24][1994]%
Ge·wêt imu þȯ umbi \alst{þ}rea naht aftar þiu \hld\ þesoro \alst{þ}iodo drohtin &
an \alst{G}alileo land, \hld\ þár hé te ênum \alst{g}ômum warð, &
ge·\alst{b}edan þat \alst{b}arn godes: \hld\ þár skolda man êna \alst{b}rúd gevan, &
\alst{m}una-líka \alst{m}agað. \hld\ Þár \alst{M}aria was, &
mid iro \alst{s}uni \alst{s}elvo, \hld\ \alst{s}álig þiorna, &
\alst{m}ahtiges \alst{m}óder. \hld\ \alst{M}anagoro drohtin &
\alst{g}éng imu þȯ mid is \alst{j}ungoron, \hld\ \alst{g}odes êgan barn, &
an þat \alst{h}ôha \alst{h}ús, \hld\ þár þe \alst{h}ęri drank, &
þea \alst{J}udeon an þemu \alst{g}ast-sęli: \hld\ hé im ôk at þem \alst{g}ômun was, &
giak hi þár ge·\alst{k}u̇ðde, \hld\ þat hi habda \alst{k}raft godes, &
\alst{h}elpa fan \alst{h}imil-fader, \hld\ \alst{h}êlagna gêst, &
\alst{w}aldandes \alst{w}ís-dóm. \hld\ \alst{W}erod blíðode, &
wárun þár an \alst{l}uston \hld\ \alst{l}iudi at·samne, &
\alst{g}umon \alst{g}lad-módje. \hld\ \alst{G}éngun ambaht-man, &
\alst{sk}ęnkjon mid \alst{sk}álun, \hld\ drógun \alst{sk}írjane wín &
mid \alst{o}rkun ęndi mid \alst{a}lo-fatun; \hld\ was þár \alst{e}rlo drôm &
\alst{f}agạr an \alst{f}lęttja, \hld\ þȯ þár \alst{f}olk undar im &
an þem \alst{b}ęnkjon só \alst{b}ętst \hld\ \alst{b}líðsja af·hóvun, &
\alst{w}árun þár an \alst{w}unnjun. \hld\ Þȯ im þes \alst{w}ínes brast, &
þem \alst{l}iudjun þes \alst{l}íðes: \hld\ is ni was far·\alst{l}êvid wiht &
\alst{h}węrgin an þemu \alst{h}úse, \hld\ þat for þene \alst{h}ęri forð &
\alst{sk}ęnkjon drógin, \hld\ ak þiu \alst{sk}apu wárun &
\alst{l}íðes a·\alst{l}árid. \hld\ Þȯ ni was \alst{l}ang te þiu, &
þat it sán ant·\alst{f}unda \hld\ \alst{f}río skônjosta, &
\alst{K}ristes móder: \hld\ géng wið iro \alst{k}ind sprekan, &
wið iro \alst{s}unu \alst{s}elvon, \hld\ \alst{s}agda im mid wordun, &
þat þea \alst{w}erdos þȯ mêr \hld\ \alst{w}ínes ne habdun &
þem \alst{g}ęstjun te \alst{g}ômun. \hld\ Siu þȯ \alst{g}erno bad, &
þat is þe \alst{h}êlogo Krist \hld\ \alst{h}elpa ge·riedi &
þemu \alst{w}erode te \alst{w}illjon. \hld\ Þȯ habda eft is \alst{w}ord garu &
\alst{m}ahtig barn godes \hld\ ęndi wið is \alst{m}óder sprak: &
„Hwat ist \alst{m}í ęndi þí“, \hld\ kwað hé, „umbi þesoro \alst{m}anno lið, &
umbi þeses \alst{w}erodes \alst{w}ín? \hld\ Te hwí sprikis þú þes, \alst{w}íf, só filu, &
\alst{m}anos mi far þesoro \alst{m}ęnigi? \hld\ Ne sint \alst{m}ína noh &
\alst{t}ídi kumana.“ \hld\ Þan þoh gi·\alst{t}rúoda siu wel &
an iro \alst{h}ugi-skęftjun, \hld\ \alst{h}êlag þiorne, &
þat is aftar þem \alst{w}ordun \hld\ \alst{w}aldandes barn, &
\alst{h}êljandoro bętst \hld\ \alst{h}elpan weldi. &
Hét þȯ þea \alst{a}mbaht-man \hld\ \alst{i}diso skônjost, &
\alst{sk}ęnkjon ęndi \alst{sk}ap-wardos, \hld\ þea þár skoldun þero \alst{sk}olu þionon, &
þat sie þes ne \alst{w}ord ne \alst{w}erk \hld\ \alst{w}iht ne far·létin, &
þes sie þe \alst{h}êlogo Krist \hld\ \alst{h}êtan weldi &
\alst{l}êstjan far þem \alst{l}iudjun. \hld\ \alst{L}árja stódun þár &
\alst{st}ên-fatu sehsi. \hld\ Þȯ só \alst{st}illo ge·bôd &
\alst{m}ahtig barn godes, \hld\ só it þár \alst{m}anno filu &
ne \alst{w}issa te \alst{w}árun, \hld\ hwó hé it mid is \alst{w}ordu ge·sprak; &
hé hét þea \alst{sk}ęnkjon \hld\ þȯ \alst{sk}írjas watares &
þiu \alst{f}atu \alst{f}ulljen, \hld\ ęndi hi þár mid is \alst{f}ingrun þȯ, &
\alst{s}egnade \alst{s}elvo \hld\ \alst{s}ínun handun, &
\alst{w}arhte it te \alst{w}íne \hld\ ęndi hét is an ên \alst{w}êgi hlaðen, &
\alst{sk}ęppjen mid ênoro \alst{sk}álon, \hld\ ęndi þȯ te þem \alst{sk}ęnkjon sprak, &
hét is þero \alst{g}ęstjo, \hld\ þe at þem \alst{g}ômun was &
þemu \alst{h}êroston \hld\ an \alst{h}and gevan, &
\alst{f}ul mid \alst{f}olmun, \hld\ þemu þe þes \alst{f}olkes þár &
ge·\alst{w}eld aftar þemu \alst{w}erde. \hld\ Reht só hi þes \alst{w}ínes ge·drank, &
só ni \alst{m}ahte hé be·\alst{m}íðan, \hld\ ne hi far þeru \alst{m}ęnigi sprak &
te þemu \alst{b}rúdi-gumon, \hld\ kwað þat simbla þat \alst{b}ętste líð &
\alst{a}lloro \alst{e}rlo ge·hwi-lik \hld\ \alst{ê}rist skoldi &
\alst{g}evan at is \alst{g}ômun: \hld\ „undar þiu wirðid þero \alst{g}umono hugi &
a·\alst{w}ękid mid \alst{w}ínu, \hld\ þat sie \alst{w}el blíðod, &
\alst{d}runkan \alst{d}rômjad. \hld\ Þan mag man þár \alst{d}ragan aftar þiu &
\alst{l}íht-\alst{l}íkora \alst{l}íð: \hld\ só ist þesoro \alst{l}iudjo þau. &
Þan havas þú nu \alst{w}undẹr-líko \hld\ \alst{w}erd-skępi þínan &
ge·\alst{m}arkod far þesoro \alst{m}ęnigi: \hld\ hétis far þit \alst{m}anno folk &
alles þínes \alst{w}ínes \hld\ þat \alst{w}irsiste &
þíne \alst{a}mbaht-man \hld\ \alst{ê}rist brengjan, &
\alst{g}evan at þínun \alst{g}ômun. \hld\ Nu sint þína \alst{g}ęsti sade, &
sint þíne \alst{d}ruhtingos \hld\ \alst{d}runkane swíðo, &
is þit \alst{f}olk \alst{f}rô-mód: \hld\ nu hétis þú hír \alst{f}orð dragan &
alloro \alst{l}íðo \alst{l}of-samost, \hld\ þero þe ik eo an þesumu \alst{l}iohte ge·sah &
\alst{h}węrgin \alst{h}ębbjan. \hld\ Mid þius skoldis þú u̇s \alst{h}in-dag êr &
\alst{g}evon ęndi \alst{g}ômjan: \hld\ þan it alloro \alst{g}umono ge·hwi-lik &
ge·\alst{þ}igedi te \alst{þ}anke.“ \hld\ Þȯ warð þár \alst{þ}egạn manag &
ge·\alst{w}ar aftar þem \alst{w}ordun, \hld\ sïðor sie þes \alst{w}ínes ge·drunkun, &
þat þár þe \alst{h}êlogo Krist \hld\ an þemu \alst{h}úse innan &
\alst{t}êkạn warhte: \hld\ \alst{t}rúodun sie sïðor &
þiu \alst{m}êr an is \alst{m}und-burd, \hld\ þat hi habdi \alst{m}aht godes, &
ge·\alst{w}ald an þesoro \alst{w}er-oldi. \hld\ Þȯ warð þat só \alst{w}ído ku̇ð &
ovar \alst{G}alileo land \hld\ \alst{J}udeo liudjun, &
hwó þár \alst{s}elvo ge·deda \hld\ \alst{s}unu drohtines &
\alst{w}ater te \alst{w}íne: \hld\ þat warð þár \alst{w}undro êrist, &
þero þe hi þár an \alst{G}alilea \hld\ \alst{J}udeo liudjon, &
\alst{t}êkno ge·\alst{t}ôgdi. \hld\ Ne mag þat ge·\alst{t}ęlljan man, &
ge·\alst{s}ęggjan te \alst{s}ȯðan, \hld\ hwat þár \alst{s}ïðor warð &
\alst{w}undres undar þemu \alst{w}erode, \hld\ þár \alst{w}aldand Krist &
an \alst{g}odes namon \hld\ \alst{J}udeo liudjon &
allan \alst{l}angan dag \hld\ \alst{l}êra sagde, &
gi·\alst{h}ét im \alst{h}evan-ríki \hld\ ęndi \alst{h}ęlljo ge·þwing &
\alst{w}ęride mid \alst{w}ordun, \hld\ hét sie \alst{w}ara godes, &
\alst{i}n-líf \alst{s}ókjan: \hld\ þár is \alst{s}eolono lioht, &
\alst{d}rôm \alst{d}rohtines \hld\ ęndi \alst{d}ag-skímon, &
\alst{g}ód-lík-nissja godes; \hld\ þár \alst{g}êst manag &
\alst{w}unod an \alst{w}illjan, \hld\ þe hír \alst{w}el þęnkid, &
þat hé hír bi·\alst{h}alde \hld\ \alst{h}evan-kuninges ge·bod.\eva

\bvb TODO.\evb\evg

\bvg\bva[25][2088]%
Ge·wêt imu þȯ mid is \alst{j}ungoron \hld\ fan þem \alst{g}ômun forð &%TODO: check gômun.
\alst{K}ristus te \alst{K}apharnaum, \hld\ \alst{k}uningo ríkjost, &
te þeru \alst{m}árjon burg. \hld\ \alst{M}ęgin samnode, &
\alst{g}umon imu te·\alst{g}ęgnes, \hld\ \alst{g}ódoro manno &
\alst{s}álig ge·\alst{s}ïði: \hld\ weldun þiu is \alst{s}wótjan word &
\alst{h}êlag \alst{h}ôrjen. \hld\ Þár im ên \alst{h}unno kwam, &
ên \alst{g}ód man an·\alst{g}ęgin \hld\ ęndi ina \alst{g}erno bad &
\alst{h}elpan \alst{h}êlagne, \hld\ kwað þat hi undar is \alst{h}íwiskja &
ênna \alst{l}efna \alst{l}amon \hld\ \alst{l}ango habdi, &
\alst{s}eokan an is \alst{s}elðon: \hld\ „só ina ênig \alst{s}ęggjo ne mag &
\alst{h}andun ge·\alst{h}êljen. \hld\ Nu is im þínoro \alst{h}elpono þarf, &
\alst{f}rô mín þe gódo.“ \hld\ Þȯ sprak im eft þat \alst{f}riðu-barn godes &
\alst{s}án aftar þiu \hld\ \alst{s}elvo te·gęgnes, &
kwað þat hé þár \alst{k}wámi \hld\ ęndi þat \alst{k}ind weldi &
\alst{n}ęrjan af þeru \alst{n}ôdi. \hld\ Þȯ im \alst{n}áhor géng &
þe man far þeru \alst{m}ęnigi \hld\ wið só \alst{m}ahtigna &
\alst{w}ordun \alst{w}ehslan: \hld\ „ik þes \alst{w}irðig ne bium,“ kwað hé, &
„\alst{h}êrro þe gódo, \hld\ þat þú an mín \alst{h}ús kumes, &
\alst{s}ókjas mína \alst{s}ęliða, \hld\ hwand ik bium só \alst{s}undig man &
mid \alst{w}ordun ęndi mid \alst{w}erkun. \hld\ Ik ge·lôvju þat þú ge·\alst{w}ald havas, &
þat þú ina \alst{h}inana maht \hld\ \alst{h}êlan ge·wirkjan, &
\alst{w}aldand frô mín: \hld\ ef þú it mid þínun \alst{w}ordun ge·sprikis, &
þan is sán þiu \alst{l}éf-hêd \alst{l}ôsot \hld\ ęndi wirðid is \alst{l}ík-hamo &
\alst{h}êl ęndi \alst{h}rêni, \hld\ ef þú im þína \alst{h}elpa far·givis. &
Ik bium mi \alst{a}mbaht-man, \hld\ hębbju mi \alst{ô}des ge·nóg, &
\alst{w}elono ge·\alst{w}unnen: \hld\ þoh ik undar ge·\alst{w}ęldi sí &
\alst{a}ðal-kuninges, \hld\ þoh hębbju ik \alst{e}rlo ge·trôst, &
\alst{h}olde \alst{h}ęri-rinkos, \hld\ þea mi só ge·\alst{h}ôriga sint, &
þat sie þes ne \alst{w}ord ne \alst{w}erk \hld\ \alst{w}iht ne far·látad, &
þes ik sie an þesumu \alst{l}and-skępje \hld\ \alst{l}êstjan héte, &
ak sie \alst{f}arad ęndi \alst{f}rummjad \hld\ ęndi eft te iro \alst{f}rôhan kumad, &
\alst{h}olde te iro \alst{h}êrron. \hld\ Þoh ik at mínumu \alst{h}ús êgi &
\alst{w}íd-brêdene \alst{w}elon \hld\ ęndi \alst{w}erodes ge·nóg, &
\alst{h}ęliðos \alst{h}ugi-dęrvje, \hld\ þoh ni gi·dar ik þi só \alst{h}êlagna &
\alst{b}iddjen, \alst{b}arn godes, \hld\ þat þú an mín \alst{b}ú gangas, &
\alst{s}ókjas mína \alst{s}ęliða, \hld\ hwand ik só \alst{s}undig bium, &
\alst{w}êt mína far·\alst{w}urhti.“ \hld\ Þȯ sprak eft \alst{w}aldand Krist, &
þe \alst{g}umo wið is \alst{j}ungoron, \hld\ kwað þat hi an \alst{J}udeon hwęrgin &
\alst{u}ndar \alst{I}sraheles \hld\ \alst{a}voron ne fundi &
ge·\alst{m}akon þes \alst{m}annes, \hld\ þe io \alst{m}êr te gode &
an þemu \alst{l}and-skępi \hld\ ge·\alst{l}ôvon habdi, &
þan \alst{h}luttron te \alst{h}imile: \hld\ „nu látu ik iu þár \alst{h}ôrjen tó, &
þár ik it iu te \alst{w}árun hír \hld\ \alst{w}ordun sęggjo, &
þat noh skulun \alst{ę}li-þeoda \hld\ \alst{ô}stane ęndi westane, &
\alst{m}an-kunnjes kuman \hld\ \alst{m}anag te·samne, &
\alst{h}êlag folk godes \hld\ an \alst{h}evan-ríki: &
þea motun þár an \alst{A}brahames \hld\ ęndi an \alst{I}saakes só self &
ęndi ôk an \alst{J}akobes, \hld\ \alst{g}ódoro manno, &
\alst{b}armun restjen \hld\ ęndi \alst{b}êðju ge·þologjan, &
\alst{w}elon ęndi \alst{w}illjon \hld\ ęndi \alst{w}onod-sam líf, &
\alst{g}ód lioht mid \alst{g}ode. \hld\ Þan skal \alst{J}udeono filu, &
þeses \alst{r}íkjas suni \hld\ be·\alst{r}ôvode werðen, &
be·\alst{d}êlide su·likoro \alst{d}iurðo, \hld\ ęndi skulun an \alst{d}alun þiustron &
an þemu alloro \alst{f}erristan \hld\ \alst{f}erne liggen. &
Þár mag man ge·\alst{h}ôrjen \hld\ \alst{h}ęliðos kwíðjan, &
þár sie iro \alst{t}orn manag \hld\ \alst{t}andon bítad; &
þár ist \alst{g}rist-grimmo \hld\ ęndi \alst{g}rádag fiur, &
\alst{h}ard \alst{h}ęlljo ge·þwing, \hld\ \alst{h}êt ęndi þiustri, &
\alst{s}wart \alst{s}in-nahti \hld\ \alst{s}undja te lône, &
\alst{w}rêðoro ge·\alst{w}urhtjo, \hld\ só hwemu só þes \alst{w}illjon ne havad, &
þat hé ina a·\alst{l}ôsje, \hld\ êr hi þit \alst{l}ioht a·geve, &
\alst{w}ęndje fan þesoro \alst{w}er-oldi. \hld\ Nu maht þú þi an þínan \alst{w}illjon forð &
\alst{s}ïðon te \alst{s}elðun; \hld\ þan findis þú ge·\alst{s}undan at hús &
\alst{m}ago-jungan \alst{m}an: \hld\ \alst{m}ód is imu an luston, &
þat \alst{b}arn is ge·hêlid, \hld\ só þú \alst{b}édi te mi: &
it wirðid al só ge·\alst{l}êstid, \hld\ só þú ge·\alst{l}ôvon havas &
an þínumu \alst{h}ugi \alst{h}ardo.“ \hld\ Þȯ sagde \alst{h}evan-kuninge, &
þe \alst{a}mbaht-man \hld\ \alst{a}lo-waldon gode &
\alst{þ}ank for þero \alst{þ}iodo, \hld\ þes hé imu at su·likun \alst{þ}arvun halp. &
Habda þo gi·\alst{â}rundid, \hld\ \alst{a}l só hé welde, &
\alst{s}álig-líko: \hld\ gi·wêt imu an þana \alst{s}ïð þanan, &
\alst{w}ende an is \alst{w}illjan, \hld\ þár hé \alst{w}elon êhte, &
\alst{b}ú ęndi \alst{b}odlos: \hld\ fand þat \alst{b}arn ge·sund, &
\alst{k}ind-jungan man. \hld\ \alst{K}ristes wárun þȯ &
\alst{w}ord ge·fullot: \hld\ hi ge·\alst{w}ald habda &
te \alst{t}ôgjanna \alst{t}êkạn, \hld\ só þat ni mag gi·\alst{t}ęlljen man, &
ge·\alst{a}hton ovar þesoro \alst{e}rðu, \hld\ hwat hé þurh is \alst{ê}nes kraft &
an þesaro \alst{m}iddil-gard \hld\ \alst{m}áriða ge·frumide, &
\alst{w}undres ge·\alst{w}arhte, \hld\ hwand al an is ge·\alst{w}ęldi stád, &
\alst{h}imil ęndi erðe.\eva

\bvb TODO.\evb\evg

\bvg\bva[26][2167]%
\hspace*{100pt} Þȯ ge·wêt imu þe \alst{h}êlogo Krist &%NOTE: In cæsura.
\alst{f}orð-wardes \alst{f}aren, \hld\ \alst{f}ręmide alo-mahtig &
alloro \alst{d}ago ge·hwi-likes, \hld\ \alst{d}rohtin þe gódo, &
\alst{l}iudjo barnum \alst{l}eof, \hld\ \alst{l}êrde mid wordun &
\alst{g}odes willjon \alst{g}umun, \hld\ habda imu \alst{j}ungorono filu &
\alst{s}imbla te gi·\alst{s}ïðun, \hld\ \alst{s}álig folk godes, &
\alst{m}anno \alst{m}ęgin-kraft, \hld\ \alst{m}anagoro þeodo, &
\alst{h}êlag \alst{h}ęri-skępi, \hld\ was is \alst{h}elpono gód, &
\alst{m}annun \alst{m}ildi. \hld\ Þȯ hi mid þeru \alst{m}ęnigi kwam, &
mid þiu \alst{b}rahtmu þat \alst{b}arn godes \hld\ te \alst{b}urg þeru hôhon, &
þe \alst{n}ęrjendo te \alst{N}aim: \hld\ þár skolde is \alst{n}amo werðen &
\alst{m}annun ge·\alst{m}árid. \hld\ Þȯ géng \alst{m}ahtig tó &
\alst{n}ęrjendo Krist, \hld\ an-tat hé gi·\alst{n}áhid was, &
\alst{h}êljandero bętst: \hld\ þȯ sáhun sie þár ên \alst{h}rêo dragan, &
ênan \alst{l}íf-lôsan \alst{l}ík-hamon \hld\ þea \alst{l}iudi fórjen, &
\alst{b}eran an ênaru \alst{b}áru \hld\ út at þera \alst{b}urges dore, &
\alst{m}agu-jungan \alst{m}an. \hld\ Þiu \alst{m}óder aftar géng &
an iro \alst{h}ugi \alst{h}riwig \hld\ ęndi \alst{h}andun slóg, &
\alst{k}arode ęndi \alst{k}úmde \hld\ iro \alst{k}indes dôð, &
\alst{i}dis \alst{a}rm-skapan; \hld\ it was ira \alst{ê}nag barn: &
siu was iru \alst{w}idowa, \hld\ ne habda \alst{w}unnja þan mêr, &
bi·úten te þemu \alst{ê}nagun sunje \hld\ \alst{a}l ge·láten &
\alst{w}unnja ęndi \alst{w}illjan, \hld\ ant-tat ina iru \alst{w}urd be·nam, &
\alst{m}ári \alst{m}etodo-ge·skapu. \hld\ \alst{M}ęgin folgode, &
\alst{b}urg-liudjo ge·\alst{b}rak, \hld\ þár man ina an \alst{b}áru dróg, &
\alst{j}ungan man te \alst{g}rave. \hld\ Þár warð imu þe \alst{g}odes sunu, &
\alst{m}ahtig \alst{m}ildi \hld\ ęndi te þeru \alst{m}óder sprak, &
hét þat þiu \alst{w}idowa \hld\ \alst{w}óp far·léti, &
\alst{k}ara aftar þemu \alst{k}inde: \hld\ „þú skalt hír \alst{k}raft sehan, &
\alst{w}aldandes gi·\alst{w}erk: \hld\ þi skal hír \alst{w}illjo ge·standen, &
\alst{f}rófra far þesumu \alst{f}olke: \hld\ ne þarft þú \alst{f}erạh karon &
\alst{b}arnes þínes.“ \hld\ *Þuȯ hie ti þero \alst{b}áron géng &
iak hie ina \alst{s}elvo ant·hrên, \hld\ \alst{s}uno drohtines, &
\alst{h}êlagon \alst{h}andon, \hld\ ęndi ti þem \alst{h}ęliðe sprak, &
hiet ina só \alst{a}la-jungan \hld\ \alst{u}p a·standan, &
a·\alst{r}ísan fan þeru \alst{r}estun. \hld\ Þie \alst{r}ink up a·sat, &
þat \alst{b}arn an þero \alst{b}árun: \hld\ warð im eft an is \alst{b}riost kuman &
þie \alst{g}êst þuru \alst{g}odes kraft, \hld\ ęndi hie te·\alst{g}ęgnes sprak, &
þe \alst{m}an wið is \alst{m}ágos. \hld\ Þuȯ ina eft þero \alst{m}uoder bi·falạh &
\alst{h}êlandi Krist an \alst{h}and: \hld\ \alst{h}ugi warð iro te frovra, &
þes \alst{w}íves an \alst{w}unnjon, \hld\ hwand iro þár su·lik \alst{w}illjo gi·stuod. &
\alst{F}éll siu þȯ te \alst{f}uotun Kristes \hld\ ęndi þena \alst{f}olko drohtin &
\alst{l}ovoda for þero \alst{l}iudjo męnigi, \hld\ hwand hie iro at só \alst{l}iobes ferạhe &
\alst{m}undoda wiðer \alst{m}etodi-gi·skęftje: \hld\ far·stuod siu þat hie was þie \alst{m}ahtigo drohtin, &
þie \alst{h}êlago, þie \alst{h}imiles gi·waldid, \hld\ ęndi þat hie mahti gi·\alst{h}elpan managon, &
\alst{a}llon \alst{i}rmin-þiedon. \hld\ Þuȯ bi·gunnun þat \alst{a}hton managa, &
þat \alst{w}undẹr, þat under þem \alst{w}eroda gi·burida, \hld\ kwáðun þat \alst{w}aldand selvo, &
\alst{m}ahtig kwámi þarod is \alst{m}ęnigi wíson, \hld\ ęndi þat hie im só \alst{m}árjan sandi &
\alst{w}ár-sagon an þero \alst{w}er-oldes ríki, \hld\ þie im þár su·likan \alst{w}illjon frumidi. &
warð þár þuȯ \alst{e}rl manag \hld\ \alst{ę}gison bi·fangan, &
þat \alst{f}olk warð an \alst{f}orọhton: \hld\ gi·sáhun þena is \alst{f}erạh êgan, &
\alst{d}ages lioht sehan, \hld\ þena þe êr \alst{d}ôð for·nam, &
an \alst{s}uht-będdjon \alst{s}walt: \hld\ þuȯ was im eft gi·\alst{s}und after þiu, &
\alst{k}ind-jung a·\alst{k}wikot. \hld\ Þuȯ warð þat \alst{k}u̇ð obar all &
\alst{a}varon \alst{I}sraheles. \hld\ Reht só þuȯ \alst{á}vand kwam, &
\alst{s}ó warð þár all gi·\alst{s}amnod \hld\ \alst{s}eokora manno, &
\alst{h}altaro ęndi \alst{h}ávaro, \hld\ só hwat só þár \alst{h}węrgin was, &
þia \alst{l}évun under þem \alst{l}iudjon, \hld\ ęndi wurðun þár gi·\alst{l}êdit tuo, &
\alst{k}umana te \alst{K}riste, \hld\ þár hie im þuru is \alst{k}raft mikil &
\alst{h}alp ęndi sie \alst{h}êlda, \hld\ ęndi liet sia eft gi·\alst{h}aldana þanan &
\alst{w}endan an iro \alst{w}illjon. \hld\ Be·þiu skal man is \alst{w}erk lovon, &
\alst{d}iuran is \alst{d}ádi, \hld\ hwand hie is \alst{d}rohtin self, &
\alst{m}ahtig \alst{m}und-boro \hld\ \alst{m}anno kunnje, &
\alst{l}iudjo só hwi-likon, \hld\ só þár gi·\alst{l}ôbit tuo &
an is \alst{w}ord ęndi an is \alst{w}erk.\eva

\bvb TODO.\evb\evg

\bvg\bva[27][2231]%
\hspace*{100pt} Þuȯ was þár \alst{w}erodes só filo &%NOTE: In cæsura.
\alst{a}llaro \alst{ę}li-þiodo kuman \hld\ te þem \alst{ê}ron Kristes, &
te só \alst{m}ahtiges \alst{m}und-burd. \hld\ Þuȯ welda hie þár êna \alst{m}ęri líðan, &
þie \alst{g}odes suno mid is \alst{j}ungron \hld\ a·nevan \alst{G}alilea-land, &
\alst{w}aldand ênna \alst{w}ágo strôm. \hld\ Þuȯ hiet hie þat \alst{w}erod ȯðar &
\alst{f}orð-werdes \alst{f}aran, \hld\ ęndi hie gi·wêt im \alst{f}ahora sum &
an ênna \alst{n}akon innan, \hld\ \alst{n}ęrjendi Krist, &
\alst{s}lápan \alst{s}ïð-wórig. \hld\ \alst{S}egel up dádun &
\alst{w}eder-wísa \alst{w}eros, \hld\ lietun \alst{w}ind after &
\alst{m}anon ovar þena \alst{m}ęri-strôm, \hld\ unþat hie te \alst{m}iddjan kwam, &
\alst{w}aldand mid is \alst{w}erodu. \hld\ Þuȯ bi·gan þes \alst{w}edares kraft, &
\alst{u̇}st up stígan, \hld\ \alst{u̇}ðjun wahsan; &
\alst{s}wang gi·\alst{s}werk an gi·mang: \hld\ þie \alst{s}êw warð an hruoru, &
wan \alst{w}ind ęndi \alst{w}ater; \hld\ \alst{w}eros sorọgodun, &
þiu \alst{m}ęri warð só \alst{m}uodag, \hld\ ni wánda þero \alst{m}anno nig·ên &
\alst{l}ęngron \alst{l}íves. \hld\ Þuȯ sia \alst{l}andes ward &
\alst{w}ękidun mid iro \alst{w}ordon \hld\ ęndi sagdun im þes \alst{w}edares kraft, &
bádun þat im gi·\alst{n}áðig \hld\ \alst{n}ęrjendi Krist &
\alst{w}urði wið þem \alst{w}atare: \hld\ „efþa wí skulun hier te \alst{w}undẹr-kwálu &
\alst{s}weltan an þeson \alst{s}êwe.“ \hld\ \alst{S}elf up a·rês &
þie \alst{g}uodo \alst{g}odes suno \hld\ ęndi te is \alst{j}ungron sprak, &
hiet þat sia im \alst{w}edares gi·\alst{w}in \hld\ \alst{w}iht ni and-rédin: &
„te hwí sind gi só \alst{f}orhta?“ \hld\ kwaþ-hie. „Nis iu noh \alst{f}ast hugi, &
gi·\alst{l}ôvo is iu te \alst{l}uttil. \hld\ Nis nú \alst{l}ang te þiu, &
þat þia \alst{st}rômos skulun \hld\ \alst{st}ilrun werðan &
gi þit *\alst{w}edar \alst{w}un-sam.“ \hld\ Þo hi te þem \alst{w}inde sprak &
ge te þemu \alst{s}êwa só \alst{s}elf \hld\ ęndi sie \alst{s}multro hét &
\alst{b}êðja ge·\alst{b}árjan. \hld\ Sie gi·\alst{b}od lêstun, &
\alst{w}aldandes \alst{w}ord: \hld\ \alst{w}eder stillodun, &
\alst{f}agạr warð an \alst{f}lóde. \hld\ Þȯ bi·gan þat \alst{f}olk undar im, &
\alst{w}erod \alst{w}undrajan, \hld\ ęndi suma mid iro \alst{w}ordun sprákun, &
hwi-lik þat só \alst{m}ahtigoro \hld\ \alst{m}anno wári, &
þat imu só þe \alst{w}ind ęndi þe \alst{w}ág \hld\ \alst{w}ordu hôrdin, &
\alst{b}êðja is gi·\alst{b}od-skępjes. \hld\ Þȯ habda sie þat \alst{b}arn godes &
gi·\alst{n}ęrid fan þeru \alst{n}ôdi: \hld\ þe \alst{n}ako furðor \edtext{skręid}{\Afootnote{See note to line TODO (bęiðero) above.}}, &
\alst{h}ôh-\alst{h}urnid skip; \hld\ \alst{h}ęliðos kwámun, &
\alst{l}iudi te \alst{l}ande, \hld\ sagdun \alst{l}of gode, &
\alst{m}áridun is \alst{m}ęgin-kraft. \hld\ Kwam þár \alst{m}anno filu &
an·\alst{g}ęgin þemu \alst{g}odes sunje; \hld\ hé sie \alst{g}erno ant·féng, &
só hwene só þár mid \alst{h}luttru \alst{h}ugi \hld\ \alst{h}elpa sóhte; &
\alst{l}êrde sie iro gi·\alst{l}ôvon \hld\ ęndi iro \alst{l}ík-hamon &
\alst{h}andun \alst{h}êlde: \hld\ nio þe man só \alst{h}ardo ni was &
gi·\alst{s}êrit mid \alst{s}uhtjun: \hld\ þoh ina \alst{S}atanases &
\alst{f}êknja jungoron \hld\ \alst{f}íundes kraftu &
\alst{h}abdin undar \alst{h}andun \hld\ ęndi is \alst{h}ugi-skęfti, &
gi·\alst{w}it a·\alst{w}ardid, \hld\ þat hé \alst{w}ódjendi &
\alst{f}óri undar þemu \alst{f}olke, \hld\ þoh im simbla \alst{f}erh far·gaf &
\alst{h}êlandjo Krist, \hld\ ef hé te is \alst{h}andun kwam, &
\alst{d}rêf þea \alst{d}iuvlas þanan \hld\ \alst{d}rohtines kraftu, &
\alst{w}árun \alst{w}ordun, \hld\ ęndi im is ge·\alst{w}it far·gaf, &
lét ina þan \alst{h}êlan \hld\ wiðer \alst{h}ęttjandun, &
gaf im wið þie \alst{f}íund \alst{f}riðu, \hld\ ęndi im \alst{f}orð gi·wêt &
an só hwi-lik þero \alst{l}ando, \hld\ só im þan \alst{l}eovost was.\eva

\bvb TODO.\evb\evg

\bvg\bva[28][2284]%
Só deda þe \alst{d}rohtines sunu \hld\ \alst{d}ago ge·hwi-likes &
\alst{g}ód werk mid is \alst{j}ungeron, \hld\ só neo \alst{J}udeon umbi þat &
an þea is \alst{m}ikilun kraft \hld\ þiu \alst{m}êr ne ge·lôvdun, &
þat hé \alst{a}lo-waldo \hld\ \alst{a}lles wári, &
\alst{l}andes ęndi \alst{l}iudjo: \hld\ þes sie noh \alst{l}ôn nimat, &
\alst{w}ídana \alst{w}rak-sïð, \hld\ þes sie þár þat ge·\alst{w}in drivun &
wið \alst{s}elvan þene \alst{s}unu drohtines. \hld\ Þȯ hé im mid is ge·\alst{s}ïðon gi·wêt &
eft an \alst{G}alilaeo land, \hld\ \alst{g}odes êgan barn, &
\alst{f}ór im te þem \alst{f}riundun, \hld\ þár hé a·\alst{f}ódid was &
ęndi al undar is \alst{k}unnje \hld\ \alst{k}ind-jung a·wóhs, &
þe \alst{h}êlago \alst{h}êljand. \hld\ Umbi ina \alst{h}ęri-skępi, &
\alst{þ}eoda \alst{þ}rungun; \hld\ þár was \alst{þ}egạn manag &
só \alst{s}álig undar þem ge·\alst{s}ïðe. \hld\ Þár drógun ênna \alst{s}eokan man &
\alst{e}rlos an iro \alst{a}rmun: \hld\ weldun ina for \alst{ô}gun Kristes, &
\alst{b}rengjan for þat \alst{b}arn godes \hld\ —was im \alst{b}ótono þarf, &
þat ina ge·\alst{h}êldi \hld\ \alst{h}evanes waldand, &
\alst{m}anno \alst{m}und-boro—, \hld\ þe was êr só \alst{m}anagan dag &
\alst{l}iðu-wastmon bi·\alst{l}amod, \hld\ ni mahte is \alst{l}ík-hamon &
\alst{w}iht ge·\alst{w}aldan. \hld\ Þan was þár \alst{w}erodes só filu, &
þat sie ina fora þat \alst{b}arn godes \hld\ \alst{b}rengjan ni mahtun, &
ge·\alst{þ}ringan þurh þea \alst{þ}ioda, \hld\ þat sie só \alst{þ}urftiges &
\alst{s}unnja ge·\alst{s}agdin. \hld\ Þȯ gi·wêt imu an ênna \alst{s}ęli innan &
\alst{h}êljando Krist; \hld\ \alst{h}warf warð þár umbi, &
\alst{m}ęgin-þeodo ge·\alst{m}ang. \hld\ Þȯ bi·gunnun þea \alst{m}an spreken, &
þe þene \alst{l}éfna \alst{l}amon \hld\ \alst{l}ango fórdun, &
\alst{b}árun mid is \alst{b}ęddju, \hld\ hwó sie ina ge·drógin fora þat \alst{b}arn godes, &
an þat \alst{w}erod innan, \hld\ þár ina \alst{w}aldand Krist &
\alst{s}elvo gi·\alst{s}áwi. \hld\ Þȯ géngun þea ge·\alst{s}ïðos tó, &
\alst{h}óvun ina mid iro \alst{h}andun \hld\ ęndi uppan þat \alst{h}ús stigun, &
\alst{s}litun þene \alst{s}ęli ovana \hld\ ęndi ina mid \alst{s}élun létun &
an þene \alst{r}akud innan, \hld\ þár þe \alst{r}íkjo was, &
\alst{k}uningo \alst{k}raftigost. \hld\ Reht só hé ina þȯ \alst{k}uman gi·sah &
þurh þes \alst{h}úses \alst{h}róst, \hld\ só hé þȯ an iro \alst{h}ugi far·stód, &
an þero \alst{m}anno \alst{m}ód-sevon, \hld\ þat sie \alst{m}ikilana te imu &
ge·\alst{l}ôvon habdun, \hld\ þȯ hé for þen \alst{l}iudjun sprak, &
kwað þat hé þene \alst{s}iakon man \hld\ \alst{s}undjono tómjan &
\alst{l}átan weldi. \hld\ Þȯ sprákun im eft þea \alst{l}iudi an·gęgin, &
\alst{g}ram-harde \alst{J}udeon, \hld\ þea þes \alst{g}odes barnes &
\alst{w}ord aftar \alst{w}arodun, \hld\ kwáðun þat þat ni mahti gi·\alst{w}erðen só, &
\alst{g}rim-werk far·\alst{g}even, \hld\ bi·útan \alst{g}od êno, &
\alst{w}aldand þesaro \alst{w}er-oldes. \hld\ Þȯ habda eft is \alst{w}ord garu &
\alst{m}ahtig barn godes: \hld\ „ik gi·dón þat“, kwað hé, „an þesumu \alst{m}anne skín, &
þe hír só \alst{s}iak ligid \hld\ an þesumu \alst{s}ęli innan, &
te \alst{w}undron gi·\alst{w}êgid, \hld\ þat ik ge·\alst{w}ald hębbju &
\alst{s}undja te far·\alst{g}evanne \hld\ ęndi ôk \alst{s}eokan man &
te ge·\alst{h}êljanne, \hld\ só ik ina \alst{h}rínan ni þarf.“ &
\alst{M}anoda ina þȯ \hld\ þe \alst{m}árjo drohtin, &
\alst{l}iggjandjan \alst{l}amon, \hld\ hét ina far þem \alst{l}iudjun a·standan &
up \alst{a}lo-hêlan \hld\ ęndi hét ina an is \alst{a}hslun niman, &
is \alst{b}ęd-gi·wádi te \alst{b}aka; \hld\ hé þat gi·\alst{b}od lêste &
\alst{s}niumo for þemu gi·\alst{s}ïðja \hld\ ęndi géng imu eft ge·\alst{s}und þanan, &
\alst{h}êl fan þemu \alst{h}úse. \hld\ Þȯ þes só manag \alst{h}êðin man, &
\alst{w}eros \alst{w}undradun, \hld\ kwáðun þat imu \alst{w}aldand self, &
\alst{g}od alo-mahtig \hld\ far·\alst{g}evan habdi &
\alst{m}éron \alst{m}ahti \hld\ þan elkor ênigumu \alst{m}annes sunje, &
\alst{k}raft ęndi \alst{k}usti; \hld\ sie ni weldun ant·\alst{k}ęnnjan þoh, &
\alst{J}udeo liudi, \hld\ þat hé \alst{g}od wári, &
ne ge·\alst{l}ôvdun is \alst{l}êran, \hld\ ak habdun im \alst{l}êðan stríd, &
\alst{w}unnun wiðar is \alst{w}ordun: \hld\ þes sie \alst{w}erk hlutun, &
\alst{l}êð-lík \alst{l}ôn-geld, \hld\ ęndi só noh \alst{l}ango skulun, &
þes sie ni weldun \alst{h}ôrjen \hld\ \alst{h}evan-kuninges, &
\alst{K}ristes lêrun, \hld\ þea hé \alst{k}u̇ðde ovar al, &
\alst{w}ído aftar þesaro \alst{w}er-oldi, \hld\ ęndi lét sie is \alst{w}erk sehan &
allaro \alst{d}ago ge·hwi-likes, \hld\ is \alst{d}ádi skawon, &
\alst{h}ôrjen is \alst{h}êlag word, \hld\ þe hé te \alst{h}elpu ge·sprak &
\alst{m}anno barnun, \hld\ ęndi só manag \alst{m}ahtig-lík &
\alst{t}êkạn ge·\alst{t}ôgda, \hld\ þat sie gi·\alst{t}rúodin þiu bet, &
gi·\alst{l}ôvdin an is \alst{l}êra. \hld\ hé só managan \alst{l}ík-hamon &
\alst{b}alu-suhtjo ant·\alst{b}and \hld\ ęndi \alst{b}óta ge·skęride, &
far·gaf \alst{f}êgjun \alst{f}erạh, \hld\ þem þe \alst{f}u̇sid was &
\alst{h}ęlið an \alst{h}ęl-sïð: \hld\ þan gi·deda ina þe \alst{h}êland self, &
\alst{K}rist þurh is \alst{k}raft mikil \hld\ \alst{k}wikan aftar dôða, &
lét ina an þesaro \alst{w}er-oldi forð \hld\ \alst{w}unnjono neotan.\eva

\bvb TODO.\evb\evg

\bvg\bva[29][2357]%
Só \alst{h}êlde hé þea \alst{h}altun man \hld\ ęndi þea \alst{h}ávon só self, &
\alst{b}ótta þem þár \alst{b}linde wárun, \hld\ lét sie þat \alst{b}erhte lioht, &
\alst{s}in-skôni \alst{s}ehan, \hld\ \alst{s}undja lôsda, &
\alst{g}umono \alst{g}rim-werk. \hld\ Ni was gio \alst{J}udeono be·þiu, &
\alst{l}êðes \alst{l}iud-skępjes \hld\ gi·\alst{l}ôvo þiu bętara &
an þene \alst{h}êlagon Krist, \hld\ ak habdun im \alst{h}ardene mód, &
swíðo \alst{st}arkan \alst{st}ríd, \hld\ far·\alst{st}andan ni weldun, &
þat sie habdun for·\alst{f}angan \hld\ \alst{f}íundun an willjan, &
\alst{l}iudi mid iro ge·\alst{l}ôvun. \hld\ Ni was gio þiu \alst{l}atoro be·þiu &
\alst{s}unu drohtines, \hld\ ak hé \alst{s}agde mid wordun, &
hwó sie skoldin ge·\alst{h}alon \hld\ \alst{h}imiles ríki, &
\alst{l}êrde aftar þemu \alst{l}ande, \hld\ habde imu þero \alst{l}iudjo só filu &
gi·\alst{w}enid mid is \alst{w}ordun, \hld\ þat im \alst{w}erod mikil, &
\alst{f}olk \alst{f}olgoda, \hld\ ęndi hé im \alst{f}ilu sagda, &
be \alst{b}iliðjun þat \alst{b}arn godes, \hld\ þes sie ni mahtun an iro \alst{b}reostun far·standan, &
undar·\alst{h}uggjan an iro \alst{h}erton, \hld\ êr it im þe \alst{h}êlago Krist &
ovar þat \alst{e}rlo folk \hld\ \alst{o}ponun wordun &
þurh is \alst{s}elves kraft \hld\ \alst{s}ęggjan welda, &
\alst{m}árjan hwat hé \alst{m}ênde. \hld\ Þár ina \alst{m}ęgin umbi, &
\alst{þ}ioda \alst{þ}rungun: \hld\ was im \alst{þ}arf mikil &
te gi·\alst{h}ôrjenne \hld\ \alst{h}evan-kuninges &
\alst{w}ár-fastun \alst{w}ord. \hld\ hé stód imu þȯ bi ênes \alst{w}atares staðe, &
ni welde þȯ bi þemu ge·\alst{þ}ringe \hld\ ovar þat \alst{þ}egno folk &
an þemu \alst{l}ande uppan \hld\ þea \alst{l}êra ku̇ðjan, &
ak \alst{g}éng imu þȯ þe \alst{g}ódo \hld\ ęndi is \alst{j}ungaron mid imu, &
\alst{f}riðu-barn godes, \hld\ þemu \alst{f}lóde náhor &
an ên \alst{sk}ip innan, \hld\ ęndi it \alst{sk}alden hét &
\alst{l}ande rúmur, \hld\ þat ina þea \alst{l}iudi só filu, &
\alst{þ}ioda ni \alst{þ}rungi. \hld\ Stód \alst{þ}egạn manag, &
\alst{w}erod bi þemu \alst{w}atare, \hld\ þár \alst{w}aldand Krist &
ovar þat \alst{l}iudjo folk \hld\ \alst{l}êra sagde: &
„Hwat ik iu \alst{s}ęggjan mag“, \hld\ kwað hé, „ge·\alst{s}ïðos míne, &
hwó imu \alst{ê}n \alst{e}rl bi·gan \hld\ an \alst{e}rðu sájan &
\alst{h}rên-korni mid is \alst{h}andun. \hld\ Sum it an \alst{h}ardan stên &
\alst{o}van-wardan fel, \hld\ \alst{e}rðon ni habda, &
þat it þár mahti \alst{w}ahsan \hld\ efþa \alst{w}urtjo gi·fȧhan, &
\alst{k}ínan efþa bi·\alst{k}líven, \hld\ ak warð þat \alst{k}orn far·loren, &
þat þár an þeru \alst{l}éian gi·\alst{l}ag. \hld\ Sum it eft an \alst{l}and bi·fel, &
an \alst{e}rðun \alst{a}ðal-kunnjes: \hld\ bi·gan imu \alst{a}ftar þiu &
\alst{w}ahsen \alst{w}án-líko \hld\ ęndi \alst{w}urtjo fȧhan, &
\alst{l}ód an \alst{l}ustun: \hld\ was þat \alst{l}and só gód, &
\alst{f}ránisko gi·\alst{f}ehod. \hld\ Sum it eft bi·\alst{f}allen warð &
an êna \alst{st}arka \alst{st}rátun, \hld\ þár \alst{st}ópon géngun, &
\alst{h}rosso \alst{h}óf-slaga \hld\ ęndi \alst{h}ęliðo tráda; &
warð imu þár an \alst{e}rðu \hld\ ęndi eft \alst{u}p gi·géng, &
bi·gan imu an þemu \alst{w}ege \alst{w}ahsen; \hld\ þȯ it eft þes \alst{w}erodes far·nam, &
þes \alst{f}olkes \alst{f}ard mikil \hld\ ęndi \alst{f}uglos a·lásun, &
þat is þemu \alst{é}ksan wiht \hld\ \alst{a}ftar ni móste &%TODO: check éksan
\alst{w}erðan te \alst{w}illjan, \hld\ þes þár an þene \alst{w}eg bi·fel. &
Sum warð it þan bi·\alst{f}allen, \hld\ þár só \alst{f}ilu stódun &
\alst{þ}ikkero \alst{þ}orno \hld\ an \alst{þ}emu dage; &
warð imu þár an \alst{e}rðu \hld\ ęndi eft \alst{u}p gi·géng, &%TODO: is the repeated line due to an error?
\alst{k}én imu þár ęndi \alst{k}livode. \hld\ Þȯ slógun þár eft \alst{k}rúd an gi·mang, &
\alst{w}ęridun imu þene \alst{w}astom: \hld\ habda it þes \alst{w}aldes hlea &
\alst{f}orana ovar-\alst{f}angan, \hld\ þat it ni mahte te ênigaro \alst{f}rumu werðen, &
ef it þea \alst{þ}ornos \hld\ só \alst{þ}ringan móstun.“ &
Þȯ \alst{s}átun ęndi \alst{s}wígodun \hld\ ge·\alst{s}ïðos Kristes, &
\alst{w}ord-spáha \alst{w}eros: \hld\ was im \alst{w}undạr mikil, &
be hwi-likun \alst{b}iliðjun \hld\ þat \alst{b}arn godes &
su·lik \alst{s}ȯð-lík spel \hld\ \alst{s}ęggjan bi·gunni. &
Þȯ bi·gan is þero \alst{e}rlo \hld\ \alst{ê}n frágojan &
\alst{h}oldan \alst{h}êrron, \hld\ \alst{h}nêg imu te·gęgnes &
tulgo \alst{w}erð-liko: \hld\ „Hwat þú ge·\alst{w}ald havas“, kwað hé, &
„ia an \alst{h}imile ia an erðu, \hld\ \alst{h}êlag drohtin, &
\alst{u}ppa ęndi niðara, \hld\ bist þú \alst{a}lo-waldo &
\alst{g}umono \alst{g}êsto, \hld\ ęndi wí þíne \alst{j}ungaron sind, &
an u̇sumu \alst{h}ugi \alst{h}olde. \hld\ \alst{H}êrro þe gódo, &
ef it þín \alst{w}illjo sí, \hld\ lát u̇s þínaro \alst{w}ordo þár &
\alst{ę}ndi gi·hôrjen, \hld\ þat wí it \alst{a}ftar þi &
ovar al \alst{K}ristin-folk \hld\ \alst{k}u̇ðjan mótin. &
wí \alst{w}itun þat þínun \alst{w}ordun \hld\ \alst{w}ár-lík biliði &
\alst{f}orð \alst{f}olgojad, \hld\ ęndi u̇s is \alst{f}irinun þarf, &
þat wí þín \alst{w}ord ęndi þín \alst{w}erk, \hld\ —hwand it fan su·likumu ge·\alst{w}ittja kumid— &
þat wí it an þesumu \alst{l}ande \hld\ at þi \alst{l}ínon mótin.“\eva

\bvb TODO.\evb\evg

\bvg\bva[30][2431]%
Þȯ im eft te·\alst{g}ęgnes \hld\ \alst{g}umono bętsta &
\alst{a}nd-wordi ge·sprak: \hld\ „ni mênde ik \alst{e}lkor wiht“, kwað hé, &
„te bi·\alst{d}ęrnjenne \hld\ \alst{d}ádjo mínaro, &
\alst{w}ordo efþa \alst{w}erko; \hld\ þit skulun gí \alst{w}itan alle, &
\alst{j}ungaron míne, \hld\ hwand iu far·\alst{g}even havad &
\alst{w}aldand þesaro \alst{w}er-oldes, \hld\ þat gí \alst{w}itan mótun &
an iuwom \alst{h}ugi-skęftjun \hld\ \alst{h}imilisk ge·rúni; &
þem ȯðrun skal man be \alst{b}iliðjun \hld\ þat gi·\alst{b}od godes &
\alst{w}ordun \alst{w}ísjen. \hld\ Nu willju ik iu te \alst{w}árun hier &
\alst{m}árjen, hwat ik \alst{m}ênde, \hld\ þat gí \alst{m}ína þiu bet &
ovar al þit \alst{l}and-skępi \hld\ \alst{l}êra far·standan. &
Þat \alst{s}ád, þat ik iu \alst{s}agda, \hld\ þat is \alst{s}elves word, &
þiu \alst{h}êlaga lêra \hld\ \alst{h}evan-kuninges, &
hwó \alst{m}an þea \alst{m}árjen skal \hld\ ovar þene \alst{m}iddil-gard, &
\alst{w}ído aftar þesaro \alst{w}er-oldi. \hld\ \alst{W}eros sind im gi·hugide, &
\alst{m}an \alst{m}is-líko: \hld\ sum su·likan \alst{m}ód dręgid, &
\alst{h}arda \alst{h}ugi-skęfti \hld\ ęndi \alst{h}rêan sevon, &
þat ina ni ge·\alst{w}erðod, \hld\ þat hé it be iuwon \alst{w}ordun due, &
þat hé þesa mína \alst{l}êra forð \hld\ \alst{l}êstjen willje, &
ak werðad þár só far·\alst{l}orana \hld\ \alst{l}êra mína, &
\alst{g}odes ambusni \hld\ ęndi iuwaro \alst{g}umono word &
an þemu \alst{u}vilon manne, \hld\ só ik iu \alst{ê}r sagda, &
þat þat \alst{k}orn far·warð, \hld\ þat þár mid \alst{k}íðun ni mahte &
an þemu \alst{st}êne uppan \hld\ \alst{st}ędi-haft werðan. &
Só wirðid \alst{a}l far·loran \hld\ \alst{ę}ðilero spráka, &
\alst{â}rundi godes, \hld\ só hwat só man þemu \alst{u}vilon manne &
\alst{w}ordun ge·\alst{w}ísid, \hld\ ęndi hé an þea \alst{w}irson hand, &
undar \alst{f}íundo \alst{f}olk \hld\ \alst{f}ard ge·kiusid, &
an \alst{g}odes un-wiljan \hld\ ęndi an \alst{g}ramono hróm &
ęndi an \alst{f}iures \alst{f}arm. \hld\ \alst{F}orð skal hé hêtjan &
mid is \alst{b}reost-hugi \hld\ \alst{b}rêda logna. &
Nio gi an þesumu \alst{l}ande þiu \alst{l}és \hld\ \alst{l}êra mína &
\alst{w}ordun ni \alst{w}ísjad: \hld\ is þeses \alst{w}erodes só filu, &
\alst{e}rlo aftar þesaro \alst{e}rðun: \hld\ bi·stéd þár \alst{ȯ}ðar man, &
þe is imu \alst{j}ung ęndi \alst{g}lau, \hld\ —ęndi havad imu \alst{g}ódan mód—, &
\alst{sp}rákono \alst{sp}áhi \hld\ ęndi wêt iuwaro \alst{sp}ello gi·skêð, &
\alst{h}ugid is þan an is \alst{h}erton \hld\ ęndi \alst{h}ôrid þár mid is ôrun tó &
swíðo \alst{n}iud-líko \hld\ ęndi \alst{n}áhor stéd, &
an is \alst{b}reost hlędid \hld\ þat gi·\alst{b}od godes, &
\alst{l}ínod ęndi \alst{l}êstid: \hld\ is is gi·\alst{l}ôvo só gód, &
talod imu, hwó hé \alst{ȯ}ðrana \hld\ \alst{e}ft gi·hwervje &
\alst{m}ên-dádigan \alst{m}an, \hld\ þat is \alst{m}ód draga &
\alst{h}luttra trewa \hld\ te \alst{h}evan-kuninge. &
Þan \alst{b}rêdid an þes \alst{b}reostun \hld\ þat gi·\alst{b}od godes, &
þie \alst{l}uvigo gi·\alst{l}ôbo, \hld\ só an þemu \alst{l}ande duod &
þat \alst{k}orn mid \alst{k}íðun, \hld\ þár it gi·\alst{k}und havad &
ęndi imu þiu \alst{w}urð bi·hagod \hld\ ęndi \alst{w}ederes gang, &
\alst{r}ęgin ęndi sunne, \hld\ þat it is \alst{r}eht havad. &
Só duod þiu \alst{g}odes lêra \hld\ an þemu \alst{g}ódun manne &
\alst{d}ages ęndi nahtes, \hld\ ęndi gangid imu \alst{d}iuval fer, &
\alst{w}rêða \alst{w}ihti \hld\ ęndi þe \alst{w}ard godes &
\alst{n}áhor mikilu \hld\ \alst{n}ahtes ęndi dages, &
ant-tat sie ina \alst{b}rengjad, \hld\ þat þár \alst{b}êðju wirðid &
ia þiu \alst{l}êra te frumu \hld\ \alst{l}iudjo barnun, &
þe fan is \alst{m}u̇ðe kumid, \hld\ iak wirðid þe \alst{m}an gode; &
havad só gi·\alst{w}ehslod \hld\ te þesaro \alst{w}er-old-stundu &
mid is \alst{h}ugi-skęftjun \hld\ \alst{h}imil-ríkjas gi·dêl, &
\alst{w}elono þene mêstan: \hld\ farid imu an gi·\alst{w}ald godes, &
\alst{t}ionuno \alst{t}ómig. \hld\ \alst{T}rewa sind só góda &
\alst{g}umono ge·hwi-likumu, \hld\ só nis \alst{g}oldes hord &
ge·\alst{l}ík su·likumu gi·\alst{l}ôvon. \hld\ Wesad iuwaro \alst{l}êrono forð &
\alst{m}an-kunnje \alst{m}ildje; \hld\ sie sind só \alst{m}is-líka, &
\alst{h}ęliðos ge·\alst{h}ugda: \hld\ sum havad iro \alst{h}ardan stríd, &
\alst{w}rêðan \alst{w}illjan, \hld\ \alst{w}ankolna hugi, &
is imu \alst{f}êknes \alst{f}ul \hld\ ęndi \alst{f}irin-werko. &
\alst{Þ}an bi·ginnid imu \alst{þ}unkjan, \hld\ þan hé undar þeru \alst{þ}iodu stád &
ęndi þár gi·\alst{h}ôrid \hld\ ovar \alst{h}lust mikil &
þea \alst{g}odes lêra, \hld\ þan þunkid imu, þat hé sie \alst{g}erno forð &
\alst{l}êstjen willje; \hld\ þan bi·ginnid imu þiu \alst{l}êra godes &
an is \alst{h}ugi \alst{h}afton, \hld\ ant-tat imu þan eft an \alst{h}and kumid &
\alst{f}eho te gi·\alst{f}órja \hld\ ęndi \alst{f}ręmiði skat. &
Þan far·\alst{l}êdjad ina \hld\ \alst{l}êða wihti, &
þan hé imu far·\alst{f}áhid \hld\ an \alst{f}eho-giri, &
a·\alst{l}ęskid þene gi·\alst{l}ôbon: \hld\ þan was imu þat \alst{l}uttil fruma, &
þat hé it gio an is \alst{h}ertan ge·\alst{h}ugda, \hld\ ef hé it \alst{h}alden ne wili. &
Þat is só þe \alst{w}astom, \hld\ þe an þemu \alst{w}ege be·gan, &
\alst{l}iodan an þemu \alst{l}ande: \hld\ þȯ far·nam ina eft þero \alst{l}iudjo fard. &
Só duot þea \alst{m}ęgin-sundjon \hld\ an þes \alst{m}annes hugi &
þea \alst{g}odes lêra, \hld\ ef hé is ni \alst{g}ômid wel; &
elkor bi·\alst{f}ęlljad sia ina \hld\ \alst{f}erne te boðme, &
an þene \alst{h}êtan \alst{h}ęl, \hld\ þár hé \alst{h}evan-kuninge &
ni wirðid \alst{f}urður te \alst{f}rumu, \hld\ ak ina \alst{f}íund skulun &
\alst{w}ítju gi·\alst{w}aragjan. \hld\ Simla gí mid \alst{w}ordun forð &
\alst{l}êrjad an þesumu \alst{l}ande: \hld\ *ik kan þesaro \alst{l}iudjo hugi, &
só \alst{m}is-líkan \alst{m}uod-sevon \hld\ \alst{m}anno kunnjes, &
só \alst{w}anda \alst{w}ísa \hld\ {[...]} &
Sum havit all te þiu is \alst{m}uod gi·látan \hld\ ęndi \alst{m}êr sorọgot, &
hwó hie þat \alst{h}ord bi·\alst{h}alde, \hld\ þan hwó hie \alst{h}evan-kuninges &
\alst{w}illjon gi·\alst{w}irkje. \hld\ Be·þiu þár \alst{w}ahsan ni mag &
þat \alst{h}êlaga gi·bod godes, \hld\ þoh it þár a·\alst{h}afton mugi, &
\alst{w}urtjon bi·\alst{w}erpan, \hld\ hwand it þie \alst{w}elo þringit. &
Só samo só þat \alst{k}rúd ęndi þie þorn \hld\ þat \alst{k}orn ant·fȧhat, &
\alst{w}ęrjat im þena \alst{w}astom, \hld\ só duot þie \alst{w}elo manne: &
gi·\alst{h}ęftid is \alst{h}erta, \hld\ þat hie it gi·\alst{h}uggjan ni muot, &
þie \alst{m}an an is \alst{m}uode, \hld\ þes hie \alst{m}êst bi·þarf, &
hwó hie þat gi·\alst{w}irkje, \hld\ þan lang þie hie an þesaro \alst{w}er-oldi sí, &
þat hie ti \alst{ê}won-dage \hld\ \alst{a}fter muoti &
\alst{h}ębbjan þuru is \alst{h}êrren þank \hld\ \alst{h}imiles ríki, &
só \alst{ę}ndi-lôsan welon, \hld\ só þat ni mag \alst{ê}nig man &
\alst{w}itan an þesaro \alst{w}er-oldi. \hld\ Nio hie só \alst{w}ído ni kan &
te gi·\alst{þ}ęnkjanne, \hld\ \alst{þ}egạn an is muode, &
þat it bi·\alst{h}aldan mugi \hld\ \alst{h}erta þes mannes, &
þat hie þat ti \alst{w}áron witi, \hld\ hwat \alst{w}aldand god havit &
\alst{g}uodes gi·\alst{g}ęrewid, \hld\ þat all \alst{g}ęgin-werd stéð &
\alst{m}anno só hwi-likon, \hld\ só ina hier \alst{m}innjot wel &
ęndi \alst{s}elvo te þiu \hld\ is \alst{s}eola gi·haldit, &
þat hie an \alst{l}ioht godes \hld\ \alst{l}íðan muoti.“\eva

\bvb TODO.\evb\evg

\bvg\bva[31][2538]%
Só \alst{w}ísda hie þuȯ mid \alst{w}ordon, \hld\ stuod \alst{w}erod mikil &
umbi þat \alst{b}arn godes, \hld\ ge·hôrdun ina bi \alst{b}iliðon filo &
umbi þesaro \alst{w}er-oldes gi·\alst{w}and \hld\ \alst{w}ordon tęlljan; &
kwað þat im ôk \alst{ê}n \alst{a}ðales man \hld\ an is \alst{a}kker sáidi &
\alst{h}luttạr \alst{h}rên-korni \hld\ \alst{h}andon sínon: &
\alst{w}olda im þár só \alst{w}un-sames \hld\ \alst{w}astmes tiljan, &
\alst{f}agạres \alst{f}ruhtes. \hld\ Þuȯ géng þár is \alst{f}íond aftar &
þuru \alst{d}ęrnjan hugi, \hld\ ęndi it all mid \alst{d}urðu ovar-séu, &%TODO: séu unclear.
mid \alst{w}eodo \alst{w}irsiston. \hld\ Þuȯ \alst{w}óhsun sia bêðju, &
ge þat \alst{k}orn ge þat \alst{k}rúd. \hld\ Só \alst{k}wámun gangan &
is \alst{h}aga-stoldos te \alst{h}ús, \hld\ iro \alst{h}êrren sagdun, &
\alst{þ}egnos iro \alst{þ}iodne \hld\ \alst{þ}rístjon wordon: &
„Hwat þú sáidos \alst{h}luttạr korn, \hld\ \alst{h}êrro þie guodo, &
\alst{ê}n-fald an þínon \alst{a}kkar: \hld\ nú ni gi·sihit ênig \alst{e}rlo þan mêr &
\alst{w}eodes \alst{w}ahsan. \hld\ Hwí mohta þat gi·\alst{w}erðan só?“ &
Þuȯ sprak eft þie \alst{a}ðales man \hld\ þem \alst{e}rlon te·gęgnes, &
\alst{þ}iodan wið is \alst{þ}egnos, \hld\ kwað þat hie it mahti undar·\alst{þ}ęnkjan wel, &
þat im þár \alst{u}n-hold man \hld\ \alst{a}ftar sáida, &
\alst{f}íond \alst{f}êkni krúd: \hld\ „ne gionsta mi þero \alst{f}ruhtjo wel, &
a·\alst{w}erda mi þena \alst{w}astom.“ \hld\ Þuȯ þár eft \alst{w}ini sprákun, &
is \alst{j}ungron te·\alst{g}ęgnes, \hld\ kwáðun þat sia þár weldin \alst{g}angan tuo, &
\alst{k}uman mid \alst{k}raftu \hld\ ęndi lôsjan þat \alst{k}rúd þanan, &
\alst{h}alon it mid iro \alst{h}andon. \hld\ Þuȯ sprak im eft iro \alst{h}êrro an·gęgin: &
„ne \alst{w}ęlljo ik, þat gí it \alst{w}iodon“, \hld\ kwaþ-hie, „hwand gi bi·\alst{w}ardon ni mugun, &
gi·\alst{g}ômjan an iuwon \alst{g}ange, \hld\ þoh gí it \alst{g}erno ni duan, &
ni gí þes \alst{k}ornes te filo, \hld\ \alst{k}íðo a·węrdjat, &
\alst{f}ęlljat under iuwa \alst{f}uoti. \hld\ Láte man sia \alst{f}orð hinan &
\alst{b}êðju wahsan, \hld\ und êr \alst{b}ewod kume &
ęndi an þem \alst{f}elde sind \hld\ \alst{f}ruhti rípja, &
\alst{a}roa an þem \alst{a}kkare: \hld\ þan faran wí þár \alst{a}lla tuo, &
\alst{h}alon it mid u̇ssan \alst{h}andon \hld\ ęndi þat \alst{h}rên-kurni lesan &
\alst{s}úvro te·\alst{s}amne \hld\ ęndi it an mínon \alst{s}ęli duojan, &
\alst{h}ębbjan it þár gi·\alst{h}aldan, \hld\ þat it \alst{h}węrgin ni mugi &
\alst{w}iht a·\alst{w}ęrdjan, \hld\ ęndi þat \alst{w}iod niman, &
\alst{b}indan it te \alst{b}urðinnjon \hld\ ęndi werpan it an \alst{b}ittạr fiur, &
láton it þár \alst{h}alojan \hld\ \alst{h}êta logna, &
\alst{ạ}ld \alst{u}n-fuodi.“ \hld\ Þuȯ stuod \alst{e}rl manag, &
\alst{þ}egnos \alst{þ}agjandi, \hld\ hwat \alst{þ}iod-gomo, &
*\alst{m}ári \alst{m}ahtig Krist \hld\ \alst{m}ênjan weldi, &
\alst{b}ôknjen mid þiu \alst{b}iliðju \hld\ \alst{b}arno ríkjost. &
Bádun þȯ só \alst{g}erno \hld\ \alst{g}ódan drohtin &
ant·\alst{l}úkan þea \alst{l}êra, \hld\ þat sia móstin þea \alst{l}iudi forð, &
\alst{h}êlaga \alst{h}ôrjan. \hld\ Þȯ sprak im eft iro \alst{h}êrro an·gęgin, &
\alst{m}ári \alst{m}ahtig Krist: \hld\ „þat is“, kwað hé, „\alst{m}annes sunu: &
ik \alst{s}elvo bium, þat þár \alst{s}áiu, \hld\ ęndi sind þesa \alst{s}áliga man &
þat \alst{h}luttra \alst{h}rên-korni, \hld\ þea mí hér \alst{h}ôrjad wel, &
\alst{w}irkjad mínan \alst{w}illjan; \hld\ þius \alst{w}er-old is þe akkar, &
þit \alst{b}rêda \alst{b}ú-land \hld\ \alst{b}arno man-kunnjes; &
\alst{S}atanas \alst{s}elvo is, \hld\ þat þár \alst{s}áid aftar &
só \alst{l}êð-líka \alst{l}êra: \hld\ havad þesaro \alst{l}iudjo só filu, &
\alst{w}erodes a·\alst{w}ardid, \hld\ þat sie \alst{w}am frummjad, &
\alst{w}irkjad aftar is \alst{w}illjon; \hld\ þoh skulun sie hér \alst{w}ahsen forð, &
þea for·\alst{g}riponon \alst{g}umon, \hld\ só samo só þea \alst{g}ódun man, &
ant-tat \alst{M}úd-spelles mę9gin \hld\ ovar \alst{m}an fęrid, &
\alst{ę}ndi þesaro wer-oldes. \hld\ Þan is allaro \alst{a}kkaro ge·hwi-lik &
ge·\alst{r}ípod an þesumu \alst{r}íkja: \hld\ skulun iro \alst{r}egan-gi·skapu &
\alst{f}rummjen \alst{f}iriho barn. \hld\ Þan te·\alst{f}arid erða: &
þat is allaro \alst{b}ewo \alst{b}rêdost; \hld\ þan kumid þe \alst{b}erhto drohtin &
\alst{o}vana mid is \alst{ę}ngilo kraftu, \hld\ ęndi kumad \alst{a}lle te·samne &
\alst{l}iudi, þe io þit \alst{l}ioht gi·sáun, \hld\ ęndi skulun þan \alst{l}ôn ant·fȧhan &
\alst{u}viles ęndi gódes. \hld\ Þan gangad \alst{ę}ngilos godes, &
\alst{h}êlage \alst{h}evan-wardos, \hld\ ęndi lesat þea \alst{h}luttron man &
\alst{s}undọr te·\alst{s}amne, \hld\ ęndi duat sie an \alst{s}in-skôni, &
\alst{h}ôh \alst{h}imiles lioht, \hld\ ęndi þea ȯðra an \alst{h}ęllja grund, &
\alst{w}erpad þea far·\alst{w}arhton \hld\ an \alst{w}allandi fiur; &
þár skulun sie gi·\alst{b}undene \hld\ \alst{b}ittra logna, &
\alst{þ}rá-werk \alst{þ}olon, \hld\ ęndi þea ȯðra \alst{þ}iod-welon &
an \alst{h}evan-ríkja, \hld\ \alst{h}wítaro sunnon &
\alst{l}iohtjan ge·\alst{l}íko. \hld\ Su-lik \alst{l}ôn nimad &
\alst{w}eros \alst{w}al-dádjo. \hld\ Só hwe só gi·\alst{w}it êgi, &
ge·\alst{h}ugdi an is \alst{h}ertan, \hld\ eþþa gi·\alst{h}ôrjen mugi, &
\alst{e}rl mid is \alst{ô}run, \hld\ só láta imu þit an \alst{i}nnan sorga, &
an is \alst{m}ód-sevon, \hld\ hwó hé skal an þemu \alst{m}árjon dage &
wið þene \alst{r}íkjon god \hld\ an \alst{r}ęðju standen &%TODO: check reðju
\alst{w}ordo ęndi \alst{w}erko allaro, \hld\ þe hé an þesaro \alst{w}er-oldi gi·duod. &
Þat is \alst{ę}gis-líkost \hld\ \alst{a}llaro þingo, &
\alst{f}orht-líkost \alst{f}iriho barnun, \hld\ þat sie skulun wið iro \alst{f}râhon mahljen, &
\alst{g}umon wið þene \alst{g}ódan drohtin: \hld\ þan weldi \alst{g}erno ge·hwe wesan, &
allaro \alst{m}anno ge·hwi-lik \hld\ \alst{m}ênes tómig, &
\alst{s}líðero \alst{s}akono. \hld\ Aftar þiu skal \alst{s}orgon êr &
allaro \alst{l}iudjo ge·hwi-lik, \hld\ êr hé þit \alst{l}ioht af·geve, &
þe þan \alst{ê}gan wili \hld\ \alst{a}lungan tír, &
\alst{h}ôh \alst{h}evan-ríki \hld\ ęndi \alst{h}uldi godes.“\eva

\bvb TODO.\evb\evg

\bvg\bva[32][2621]%
Só gi·fragn ik þat þȯ \alst{s}elvo \hld\ \alst{s}unu drohtines, &
allaro \alst{b}arno \alst{b}ętst \hld\ \alst{b}iliðjo sagda, &
hwi-lik þero \alst{w}ári \hld\ an \alst{w}er-old-ríkja &
undar \alst{h}ęlið-kunnje \hld\ \alst{h}imil-ríkje ge·lík; &
kwað þat oft \alst{l}uttiles hwat \hld\ \alst{l}iohtora wurði, &
só \alst{h}ôho af·\alst{h}uovi, \hld\ „so duot \alst{h}imil-ríki: &
þat is simla \alst{m}êra, \hld\ þan is \alst{m}an ênig &
\alst{w}ánje an þesaro \alst{w}er-oldi. \hld\ Ôk is imu þat \alst{w}erk ge·lík, &
þat man an \alst{s}êo innan \hld\ \alst{s}ęgina wirpit, &
\alst{f}isk-nęt an \alst{f}lód \hld\ ęndi \alst{f}áhit bêðju, &
\alst{u}vile ęndi góde, \hld\ tiuhid \alst{u}p te staðe, &
\alst{l}iðod sie te \alst{l}ande, \hld\ \alst{l}isit aftar þiu &
þea \alst{g}ódun an \alst{g}reote \hld\ ęndi látid þea ȯðra eft an \alst{g}rund faran, &
an \alst{w}ídan \alst{w}ág. \hld\ Só duod \alst{w}aldand god &
an þemu \alst{m}árjon dage \hld\ \alst{m}ęnniskono barn: &
brengid \alst{i}rmin-þiod, \hld\ \alst{a}lle te·samne, &
lisit imu þan þea \alst{h}luttron \hld\ an \alst{h}evan-ríki, &
látid þea far·\alst{g}riponon \hld\ an \alst{g}rund faren &
\alst{h}ęllje fiures. \hld\ Ni wêt \alst{h}ęliðo man &
þes \alst{w}ítjes \alst{w}iðar-lága, \hld\ þes þár \alst{w}eros þiggjat, &
an þemu \alst{I}nferne \hld\ \alst{i}rmin-þioda. &
Þan hald ni mag þera \alst{m}édan \alst{m}an \hld\ gi·\alst{m}akon fïðen, &
ni þes \alst{w}elon ni þes \alst{w}illjon, \hld\ þes þár \alst{w}aldand skerid, &%NOTE: skerid uncertain; skeran or skęrjan?
\alst{g}ildid \alst{g}od selvo \hld\ \alst{g}umono só hwi-likumu, &
só ina \alst{h}ér gi·\alst{h}aldid, \hld\ þat hé an \alst{h}evan-ríki, &
an þat \alst{l}ang-same \alst{l}ioht \hld\ \alst{l}íðan móti.“ &
Só \alst{l}êrda hé þȯ mid \alst{l}istjun. \hld\ Þan fórun þár þea \alst{l}iudi tó &
ovar al \alst{G}alilaeo land \hld\ þat \alst{g}odes barn sehan: &
dádun it bi þemu \alst{w}undre, \hld\ hwanen imu mahti su·lik \alst{w}ord kumen, &
só \alst{sp}áh-líko gi·\alst{sp}rokan, \hld\ þat hé \alst{sp}el godes &
gio só \alst{s}ȯð-líko \hld\ \alst{s}ęggjan konsti, &
só \alst{k}raftig-líko gi·\alst{k}weðen: \hld\ „Hé is þeses \alst{k}unnjes hinen“, kwáðun sie, &
„þe man þurh \alst{m}ág-skępi: \hld\ hér is is \alst{m}óder mid u̇s, &
\alst{w}íf undar þesumu \alst{w}erode. \hld\ Hwat wí þe hér \alst{w}itun alle, &
só \alst{k}u̇ð is u̇s is \alst{k}uni-burd \hld\ ęndi is \alst{k}nósles ge·hwat; &
a·\alst{w}óhs al undar þesumu \alst{w}erode: \hld\ hwanen skoldi imu su·lik ge·\alst{w}it kuman, &
\alst{m}éron \alst{m}ahti, \hld\ þan hér ȯðra \alst{m}an êgin?“ &
Só far·\alst{m}unste ina þat \alst{m}anno folk \hld\ ęndi sprákun im gi·\alst{m}êd-lik word, &
far·\alst{h}ogdun ina só \alst{h}êlagna, \hld\ \alst{h}ôrjen ni weldun &
is gi·\alst{b}od-skępjes. \hld\ Ni hé þár ôk \alst{b}iliðjo filu &
þurh iro \alst{u}n-gi·lôvon \hld\ \alst{ó}gjan ni welde, &
\alst{t}orhtero \alst{t}êkno, \hld\ hwand hé wisse iro \alst{t}wífljan hugi, &
iro \alst{w}rêðan \alst{w}illjan, \hld\ þat ni wárun \alst{w}eros ȯðra &
só \alst{g}rimme under \alst{J}udeon, \hld\ só wárun umbi \alst{G}alilaeo land, &
só \alst{h}ardo ge·\alst{h}ugide: \hld\ só þár was þe \alst{h}êlago Krist, &
gi·\alst{b}oren þat \alst{b}arn godes, \hld\ si ni weldun is gi·\alst{b}od-skępi þoh &
ant·\alst{f}ȧhan \alst{f}erht-líko, \hld\ ak bi·gan þat \alst{f}olk undar im, &
\alst{r}inkos \alst{r}ádan, \hld\ hwó sie þene \alst{r}íkjon Krist &
\alst{w}êgdin te \alst{w}undron. \hld\ Hétun þȯ iro \alst{w}erod kumen, &
ge·\alst{s}ïði te·\alst{s}amne: \hld\ \alst{s}undja weldun &
an þene \alst{g}odes sunu \hld\ \alst{g}erno gi·tęlljen &
\alst{w}rêðes \alst{w}illjon; \hld\ ni was im is \alst{w}ordo niud, &
\alst{sp}áharo \alst{sp}ello, \hld\ ak sie bi·gunnun \alst{sp}rekan undar im, &
hwó sie ina só \alst{k}raftagne \hld\ fan ênumu \alst{k}live wurpin, &
ovar ênna \alst{b}erges wal: \hld\ weldun þat \alst{b}arn godes &
\alst{l}ivu bi·\alst{l}ôsjen. \hld\ Þȯ hé imu mid þem \alst{l}iudjun samad &
\alst{f}rô-líko \alst{f}ór: \hld\ ni was imu \alst{f}orạht hugi, &
—wisse þat imu ni \alst{m}ahtun \hld\ \alst{m}ęnniskono barn, &
bi þeru \alst{g}od-kundi \hld\ \alst{J}udeo liudi &
êr is \alst{t}ídjun wiht \hld\ \alst{t}eonon gi·frummjen, &
\alst{l}êðaro gi·\alst{l}êsto—, \hld\ ak hé imu mid þem \alst{l}iudjun samad &
\alst{st}êg uppen þene \alst{st}ên-holm, \hld\ ant-þat sie te þeru \alst{st}ędi kwámun, &
þár sie ine fan þemu \alst{w}alle niðer \hld\ \alst{w}erpen hugdun, &
\alst{f}ęlljen te \alst{f}oldu, \hld\ þat hé wurði is \alst{f}erhes lôs, &
is \alst{a}ldres at \alst{ę}ndje. \hld\ Þȯ warð þero \alst{e}rlo hugi, &
an þemu \alst{b}erge uppen \hld\ \alst{b}ittra gi·þȧhti &
\alst{J}uðeono te·\alst{g}angen, \hld\ þat iro ênig ni habde só \alst{g}rimmon sevon &
ni só \alst{w}rêðen \alst{w}illjon, \hld\ þat sie mahtin þene \alst{w}aldandes sunu, &
\alst{K}rist ant·\alst{k}ęnnjen; \hld\ hé ni was iro \alst{k}u̇ð ênigumu, &
þat sie ina þȯ undar·\alst{w}issin. \hld\ Só mahte hé undar ira \alst{w}erode standen &
ęndi an iro gi·\alst{m}ange \hld\ \alst{m}iddjumu gangen, &
\alst{f}aren undar iro \alst{f}olke. \hld\ hé dede imu þene \alst{f}riðu selvo, &
\alst{m}und-burd wið þeru \alst{m}ęnegi \hld\ ęndi gi·wêt imu þurh \alst{m}iddi þanan &
þes \alst{f}íundo \alst{f}olkes, \hld\ \alst{f}ór imu þȯ, þár hé welde, &
an êne \alst{w}óstunnje \hld\ \alst{w}aldandes sunu, &
\alst{k}uningo \alst{k}raftigost: \hld\ habde þero \alst{k}ustes gi·wald, &
hwar imu an þemu \alst{l}ande \hld\ \alst{l}eovost wári &
te \alst{w}esanne an þesaru \alst{w}er-oldi.\eva

\bvb TODO.\evb\evg

\bvg\bva[33][2698]%
\hspace*{100pt} Þan fór imu an \alst{w}eg ȯðran &%NOTE: In cæsura.
\alst{J}ohannes mid is \alst{j}ungarun, \hld\ \alst{g}odes ambaht-man, &
\alst{l}êrde þea \alst{l}iudi \hld\ \alst{l}ang-samane rád, &
hét þat sie \alst{f}rume fręmidin, \hld\ \alst{f}irina far·létin, &
\alst{m}ên ęndi \alst{m}orð-werk. \hld\ hé was þár \alst{m}anagumu liof &
\alst{g}ódaro \alst{g}umono. \hld\ hé sóhte imu þȯ þene \alst{J}udeono kuning, &
þene \alst{h}ęri-togon at \alst{h}ús, \hld\ þe \alst{h}êten was &
\alst{E}rodes aftar is \alst{ę}ldiron, \hld\ \alst{o}var-módig man: &
\alst{b}úide imu be þeru \alst{b}rúdi, \hld\ þiu êr sínes \alst{b}róðer was, &
\alst{i}dis an \alst{ê}hti, \hld\ ant-tat hé \alst{ę}lljor skók, &
\alst{w}er-old \alst{w}eslode. \hld\ Þȯ imu þat \alst{w}íf gi·nam &
þe \alst{k}uning te \alst{k}wenun; \hld\ êr wárun iro \alst{k}ind ôdan, &
\alst{b}arn be is \alst{b}róðer. \hld\ Þȯ bi·gan imu þea \alst{b}rúd lahan &
\alst{J}ohannes þe \alst{g}ódo, \hld\ kwað þat it \alst{g}ode wári, &
\alst{w}aldande \alst{w}iðer-mód, \hld\ þat it ênig \alst{w}ero frumidi, &
þat \alst{b}róðer \alst{b}rúd \hld\ an is \alst{b}ęd námi, &
\alst{h}ębbje sie imu te \alst{h}íwun. \hld\ „Ef þú mi \alst{h}ôrjen wili, &
gi·\alst{l}ôvjen mínun \alst{l}êrun, \hld\ ni skalt þú sie \alst{l}ęng êgan, &
ak míð ire an þínumu \alst{m}óde: \hld\ ni hava þár su·lika \alst{m}innja tó, &
ni \alst{s}undjo þi te \alst{s}wíðo.“ \hld\ Þȯ warð an \alst{s}orgun hugi &
þes \alst{w}íves aftar þem \alst{w}ordun; \hld\ and-réd þat hé þene \alst{w}er-old-kuning &
\alst{sp}rákono ge·\alst{sp}óni \hld\ ęndi \alst{sp}áhun wordun, &
þat hé sie far·\alst{l}éti. \hld\ Be·gan siu imu þȯ \alst{l}êðes filu &
\alst{r}áden an \alst{r}únon, \hld\ ęndi ine \alst{r}inkos hét, &
\alst{u}n-sundigane \hld\ \alst{e}rlos fȧhan &
ęndi ine an ênumu \alst{k}arkerja \hld\ \alst{k}lústar-bęndjun, &
\alst{l}iðo-kospun bi·\alst{l}úkan: \hld\ be þem \alst{l}iudjun ne gi·dorstun &
ine \alst{f}erạhu bi·lôsjen, \hld\ hwand sie wárun imu \alst{f}riund alle, &
wissun ine só \alst{g}óden \hld\ ęndi \alst{g}ode werðen, &
habdun ina for \alst{w}ár-sagon, \hld\ só sia \alst{w}ela mahtun. &
Þȯ wurðun an þemu \alst{g}ę́r-tale \hld\ \alst{J}udeo kuninges &
\alst{t}ídi kumana, \hld\ só þár gi·\alst{t}ald habdun &
\alst{f}róde \alst{f}olk-weros, \hld\ þȯ hé gi·\alst{f}ódid was, &
an \alst{l}ioht kuman. \hld\ Só was þero \alst{l}iudjo þau, &
þat þat \alst{e}rlo ge·hwi-lik \hld\ \alst{ó}vjan skolde, &
\alst{J}udeono mid \alst{g}ômun. \hld\ Þȯ warð þár an þene \alst{g}ast-sęli &
\alst{m}ęgin-kraft \alst{m}ikil \hld\ \alst{m}anno ge·samnod, &
\alst{h}ęri-togono an þat \alst{h}ús, \hld\ þár iro \alst{h}êrro was &
an is \alst{k}uning-stóle. \hld\ \alst{K}wámun managa &
\alst{J}udeon an þene \alst{g}ast-sęli; \hld\ warð im þár \alst{g}lad-mód hugi, &
\alst{b}líði an iro \alst{b}reostun: \hld\ gi·sáhun iro \alst{b}âg-gevon &
\alst{w}esen an \alst{w}unnjon. \hld\ Dróg man \alst{w}ín an flęt &
\alst{sk}íri mid \alst{sk}álun, \hld\ \alst{sk}ęnkjon hwurvun, &
\alst{g}éngun mid \alst{g}old-fatun: \hld\ \alst{g}aman was þár inne &
\alst{h}lúd an þero \alst{h}allu, \hld\ \alst{h}ęliðos drunkun. &
Was þes an \alst{l}ustun \hld\ \alst{l}andes hirdi, &
hwat hé þemu \alst{w}erode mêst \hld\ te \alst{w}unnjun gi·fręmidi. &
Hét hé þȯ \alst{g}angen forð \hld\ \alst{g}êla þiornun, &
is \alst{b}róder \alst{b}arn, \hld\ þár hé an is \alst{b}ęnki sat &
\alst{w}ínu gi·\alst{w}lęnkid, \hld\ ęndi þȯ te þemu \alst{w}íve sprak; &
\alst{g}rótte sie fora þemu \alst{g}um-skępje \hld\ ęndi \alst{g}erno bad, &
þat siu þár fora þem \alst{g}astjun \hld\ \alst{g}aman af·hóvi &
\alst{f}agạr an \alst{f}lęttje: \hld\ „lát þit \alst{f}olk sehan, &
hwó þú ge·\alst{l}ínod havas \hld\ \alst{l}iudjo męnegi &
te \alst{b}líðsjanne an \alst{b}ęnkjun; \hld\ ef þú mi þera \alst{b}ede tugiðos, &
mín \alst{w}ord for þesumu \alst{w}erode, \hld\ þan willju ik it hér te \alst{w}árun ge·kweðen, &
\alst{l}iahto fora þesun \alst{l}iudjun \hld\ ęndi ôk gi·\alst{l}êstjen só, &
þat ik þí þan \alst{a}ftar þiu \hld\ \alst{ê}ron willju, &
só hwes só þú mí \alst{b}idis \hld\ for þesun mínun \alst{b}âg-winjun: &
þoh þú mí þesaro \alst{h}ęri-dómo \hld\ \alst{h}alvaro fergos, &
\alst{r}íkjas mínes, \hld\ þoh gi·dón ik, þat it ênig \alst{r}inko ni mag &
\alst{w}ordun gi·\alst{w}ęndjen, \hld\ ęndi it skal gi·\alst{w}erðen só.“ &
Þȯ warð þera \alst{m}agað aftar þiu \hld\ \alst{m}ód gi·hworven, &
\alst{h}ugi aftar iro \alst{h}êrron, \hld\ þat siu an þemu \alst{h}úse innen, &
an þemu \alst{g}ast-sęli \hld\ \alst{g}amen up a·huof, &
al só þero \alst{l}iudjo \hld\ \alst{l}and-wíse gi·dróg, &
þero \alst{þ}iodo \alst{þ}au. \hld\ Þiu \alst{þ}iorne spilode &
\alst{h}rór aftar þemu \alst{h}úse: \hld\ \alst{h}ugi was an lustun, &%NOTE: hrór checked.
\alst{m}anagaro \alst{m}ód-sevo. \hld\ Þȯ þiu \alst{m}agað habda &
gi·\alst{þ}ionod te \alst{þ}anke \hld\ \alst{þ}iod-kuninge &
ęndi \alst{a}llumu þemu \alst{e}rl-skępje, \hld\ þe þár \alst{i}nne was &
\alst{g}ódaro \alst{g}umono, \hld\ siu welde þȯ ira \alst{g}eva êgan, &
þiu \alst{m}agað for þeru \alst{m}ęnegi: \hld\ géng þȯ wið iro \alst{m}ódar sprekan &
ęndi \alst{f}rágode sie \hld\ \alst{f}iri-wit-líko, &
hwes siu þene \alst{b}urges ward \hld\ \alst{b}iddjen skoldi. &
Þȯ \alst{w}ísde siu aftar iro \alst{w}illjon, \hld\ hét þat siu \alst{w}ihtes þan êr &
ni \alst{g}ęrodi for þemu \alst{g}um-skępje, \hld\ bi·útan þat man iru \alst{J}ohannes &
an þeru \alst{h}allu innan \hld\ \alst{h}ôvid gávi &
a·\alst{l}ôsid af is \alst{l}ík-hamon. \hld\ Þat was allun þem \alst{l}iudjun harm, &
þem \alst{m}annun an iro \alst{m}óde, \hld\ þȯ sie þat gi·hôrdun þea \alst{m}agað sprekan; &
só was it ôk þemu \alst{k}uninge: \hld\ hé ni mahte is \alst{k}widi liagan, &
is \alst{w}ord \alst{w}ęndjen: \hld\ hét þȯ is \alst{w}ę́pan-berand &
\alst{g}angen fan þemu \alst{g}ast-sęli \hld\ ęndi hét þene \alst{g}odes man &
\alst{l}ívu bi·\alst{l}ôsjen. \hld\ Þȯ ni was \alst{l}ang te þiu, &
þat man an þea \alst{h}alla \hld\ \alst{h}ôvid brȧhte &
\alst{þ}es \alst{þ}iod-gumon, \hld\ ęndi it þár þeru \alst{þ}iornun far·gaf, &
\alst{m}agað for þeru \alst{m}ęnegi: \hld\ siu dróg it þeru \alst{m}óder forð. &
Þȯ was \alst{ê}n-dago \hld\ \alst{a}llaro manno &
þes \alst{w}ísoston, \hld\ þero þe gio an þesa \alst{w}er-old kwámi, &
þero þe \alst{k}wene ênig \hld\ \alst{k}ind gi·bári, &
\alst{i}dis fan \alst{e}rle, \hld\ lét man simla þen \alst{ê}non bi·foran, &
þe þiu \alst{þ}iorne gi·dróg, \hld\ þe gio \alst{þ}egnes ni warð &
\alst{w}ís an iro \alst{w}er-oldi, \hld\ bi·útan só ine \alst{w}aldand god &
fan \alst{h}evan-wange \hld\ \alst{h}êlages gêstes &
gi·\alst{m}arkode \alst{m}ahtig: \hld\ þe ni habde ênigan gi·\alst{m}akon hwęrgin &
\alst{ê}r nek \alst{a}ftar. \hld\ \alst{E}rlos hwurvun, &%NOTE: nek checked.
\alst{g}umon umbi \alst{J}ohannen, \hld\ is \alst{j}ungaron managa, &
\alst{s}álig ge·\alst{s}ïði, \hld\ ęndi ine an \alst{s}ande bi·gróvun, &
\alst{l}eoves \alst{l}ík-hamon: \hld\ wissun þat hé \alst{l}ioht godes, &
\alst{d}iur-líkan \alst{d}rôm \hld\ mid is \alst{d}rohtine samad, &
\alst{u}p-\alst{ô}das hêm \hld\ \alst{ê}gan móste, &
\alst{s}álig \alst{s}ókjan.\eva

\bvb TODO.\evb\evg

\bvg\bva[34][2799]%
\hspace*{100pt} Þȯ ge·witun im þea ge·\alst{s}ïðos þanen, &%NOTE: In cæsura.
\alst{J}ohannes \alst{j}ungaron \hld\ \alst{j}ámer-móde, &
\alst{h}êlag-ferạha: \hld\ was im iro \alst{h}êrron dôð &
\alst{s}wíðo an \alst{s}orgun. \hld\ Ge·witun im \alst{s}ókjan þȯ &
an þeru \alst{w}óstunni \hld\ \alst{w}aldandes sunu, &
\alst{k}raftigana \alst{K}rist \hld\ ęndi imu \alst{k}u̇ð gi·dedun &
\alst{g}ódes mannes for·\alst{g}ang, \hld\ hwó habde þe \alst{J}udeono kuning &
\alst{m}anno þene \alst{m}árjostan \hld\ \alst{m}ákjas ęggjun &
\alst{h}ôvdu bi·\alst{h}auwan: \hld\ hé ni welde is ênigen \alst{h}arm spreken, &
\alst{s}unu drohtines; \hld\ hé wisse þat þiu \alst{s}eole was &
\alst{h}êlag gi·\alst{h}alden \hld\ \alst{w}iðer hęttjandjon, &
an \alst{f}riðe wiðer \alst{f}íundun. \hld\ Þȯ só gi·\alst{f}rági warð &
aftar þem \alst{l}and-skępjun \hld\ \alst{l}êrjandero bętst &
an þeru \alst{w}óstunni: \hld\ \alst{w}erod samnode, &
\alst{f}ór \alst{f}olkun tó: \hld\ was im \alst{f}iri-wit mikil &
\alst{w}ísaro \alst{w}ordo; \hld\ imu was ôk \alst{w}illjo só samo, &
\alst{s}unje drohtines, \hld\ þat hé su·lik ge·\alst{s}ïðo folk &
an þat \alst{l}ioht godes \hld\ \alst{l}aðojan mósti, &
\alst{w}ęnnjen mid \alst{w}illjon. \hld\ \alst{W}aldand lêrde &
allan \alst{l}angan dag \hld\ \alst{l}iudi managa, &
\alst{ę}li-þeodige man, \hld\ ant-tat an \alst{á}vand sêg &
\alst{s}unne te \alst{s}edle. \hld\ Þȯ géngun is ge·\alst{s}ïðos twe-livi, &
\alst{g}umon te þemu \alst{g}odes barne \hld\ ęndi sagdun iro \alst{g}ódumu hêrron, &
mid hwi-liku \alst{a}rvedju þár þea \alst{e}rlos livdin, \hld\ kwáðun þat sie is \alst{ê}ra bi·þorftin, &
\alst{w}eros an þemu \alst{w}óstjon lande: \hld\ „sie ni mugun sie hér mid \alst{w}ihti ant·hębbjen, &
\alst{h}ęliðos bi \alst{h}ungres ge·þwinge. \hld\ Nu lát þú sie, \alst{h}êrro þe gódo, &
\alst{s}ïðon, þár sie \alst{s}ęliða fïðen. \hld\ Náh sind hér ge·\alst{s}etana burgi &
\alst{m}anaga mid \alst{m}ęgin-þiodun: \hld\ þár fïðad sie \alst{m}ęti te kôpe, &
\alst{w}eros aftar þem \alst{w}íkjon.“ \hld\ Þȯ sprak eft \alst{w}aldand Krist, &
\alst{þ}ioda drohtin, \hld\ kwað þat þes êniga \alst{þ}urụfti ni wárin, &
„þat sie þurh \alst{m}ęti-lôsi \hld\ \alst{m}ína far·látan &
\alst{l}eov-líka \alst{l}êra. \hld\ Gevad gi þesun \alst{l}iudjun gi·nóg, &
\alst{w}ęnnjad sie hér mid \alst{w}illjon.“ \hld\ Þȯ habde eft is \alst{w}ord garu &
\alst{Ph}ilippus \alst{f}ród gumo, \hld\ kwað þat þár só \alst{f}ilu wári &
\alst{m}anno \alst{m}ęnigi: \hld\ „þoh wí hér te \alst{m}ęti habdin &
\alst{g}aru im te \alst{g}evanne, \hld\ só wí mahtin far·\alst{g}elden mêst, &
ef wí hér gi·\alst{s}aldin \hld\ \alst{s}ilụver-skatto &
\alst{t}wê hund samad, \hld\ \alst{t}weho wári is noh þan, &
þat iro \alst{ê}nig þár \hld\ \alst{ê}nes gi·námi: &
só \alst{l}uttik wári þat þesun \alst{l}iudjun.“ \hld\ Þȯ sprak eft þe \alst{l}andes ward &%NOTE: luttik checked.
ęndi \alst{f}rágode sie \hld\ \alst{f}iri-wit-líko, &
\alst{m}anno drohtin, \hld\ hwat sie þár te \alst{m}ęti habdin &
\alst{w}istes ge·\alst{w}unnin. \hld\ Þȯ sprak imu eft mid is \alst{w}ordun an·gęgin &
\alst{A}ndreas fora þem \alst{e}rlun \hld\ ęndi þemu \alst{a}lo-waldon &
\alst{s}elvumu \alst{s}agde, \hld\ þat sie an iro gi·\alst{s}ïðje þan mêr &
\alst{g}arowes ni habdin, \hld\ „bi·útan \alst{g}irstin brôd &
\alst{f}ïvi an u̇saru \alst{f}ęrdi \hld\ ęndi \alst{f}iskos twêne. &
Hwat mag þat þoh þesaru \alst{m}ęnigi?“ \hld\ Þȯ sprak imu eft \alst{m}ahtig Krist, &
þe \alst{g}ódo \alst{g}odes sunu, \hld\ ęndi hét þat \alst{g}umono folk &
\alst{sk}ęrjen ęndi \alst{sk}êðen \hld\ ęndi hét þea \alst{sk}ola sęttjen, &
\alst{e}rlos aftar þeru \alst{e}rðu, \hld\ \alst{i}rmin-þioda &
an \alst{g}rase \alst{g}ruonimu, \hld\ ęndi þȯ te is \alst{j}ungarun sprak, &
allaro \alst{b}arno \alst{b}ętst, \hld\ hét imu þiu \alst{b}rôd halon &
ęndi þea \alst{f}iskos \alst{f}orð. \hld\ Þat \alst{f}olk stillo bêd, &
\alst{s}at ge·\alst{s}ïði mikil; \hld\ undar þiu hé þurh is \alst{s}elves kraft, &
\alst{m}anno drohtin, \hld\ þene \alst{m}ęti wíhide, &
\alst{h}êlag \alst{h}evan-kuning, \hld\ ęndi mid is \alst{h}andun brak, &
\alst{g}af it is \alst{j}ungarun forð, \hld\ ęndi it sie undar þemu \alst{g}um-skępje hét &
\alst{d}ragan ęndi \alst{d}êljen. \hld\ Sie lêstun iro \alst{d}rohtines word, &
is \alst{g}eva \alst{g}erno drógun \hld\ \alst{g}umono gi·hwemu, &
\alst{h}êlaga \alst{h}elpa. \hld\ It undar iro \alst{h}andun wóhs, &
\alst{m}ęti \alst{m}anno gi·hwemu: \hld\ þeru \alst{m}ęgin-þiodu warð &
\alst{l}íf an \alst{l}ustun, \hld\ þea \alst{l}iudi wurðun alle, &
\alst{s}ade \alst{s}álig folk, \hld\ só hwat só þár gi·\alst{s}amnod was &
fan allun \alst{w}ídun \alst{w}egun. \hld\ Þȯ hét \alst{w}aldand Krist &
\alst{g}angen is \alst{j}ungaron \hld\ ęndi hét sie \alst{g}ômjen wel, &
þat þiu \alst{l}éva þár \hld\ far·\alst{l}oren ni wurði; &
hét \alst{s}ie þȯ \alst{s}amnon, \hld\ þȯ þár \alst{s}ade wárun &
\alst{m}an-kunnjes \alst{m}anag. \hld\ Þár \alst{m}óses warð, &
\alst{b}rôdes te lévu, \hld\ þat man \alst{b}irilos gi·las &
\alst{t}we-livi fulle: \hld\ þat was \alst{t}êkạn mikil, &
\alst{g}rôt kraft \alst{g}odes, \hld\ hwand þár was \alst{g}umono gi·tald &
áno \alst{w}íf ęndi kind, \hld\ \alst{w}erodes at·samme &
\alst{f}ïf þúsundig. \hld\ Þat \alst{f}olk al far·stód, &
þea \alst{m}an an iro \alst{m}óde, \hld\ þat sie þár \alst{m}ahtigna &
\alst{h}êrron \alst{h}abdun. \hld\ Þȯ sie \alst{h}evan-kuning, &
þea \alst{l}iudi \alst{l}ovodun, \hld\ kwáðun þat gio ni wurði an þit \alst{l}ioht kuman &
\alst{w}ísaro \alst{w}ár-sago, \hld\ efþa þat hé gi·\alst{w}ald mid gode &
an þesaru \alst{m}iddil-gard \hld\ \alst{m}éron habdi, &
\alst{ê}n-faldaran hugi. \hld\ \alst{A}lle gi·sprákun, &
þat hé \alst{w}ári \alst{w}irðig \hld\ \alst{w}elono ge·hwi-likes, &
þat hé \alst{e}rð-ríki \hld\ \alst{ê}gan mósti, &
\alst{w}ídene \alst{w}er-old-stól, \hld\ „nu hé su·lik ge·\alst{w}it havad, &
só \alst{g}rôte kraft mid \alst{g}ode.“ \hld\ Þea \alst{g}umon alle gi·warð, &
þat sie ine gi·\alst{h}óvin \hld\ te \alst{h}êrosten, &
gi·\alst{k}urin ine te \alst{k}uninge: \hld\ þat \alst{K}riste ni was &
\alst{w}ihtes \alst{w}irðig, \hld\ hwand hé þit \alst{w}er-old-ríki, &
\alst{e}rðe ęndi \alst{u}p-himil \hld\ þurh is \alst{ê}nes kraft &
\alst{s}elvo gi·warhte \hld\ ęndi \alst{s}ïðor gi·held, &
\alst{l}and ęndi \alst{l}iud-skępi, \hld\ —þoh þes ênigan gi·\alst{l}ôvon ni dedin &
\alst{w}rêðe \alst{w}iðer-sakon— \hld\ þat al an is gi·\alst{w}alde stád, &
\alst{k}uning-ríkjo \alst{k}raft \hld\ ęndi \alst{k}êsur-dómes, &
\alst{m}ęgin-þiodo \alst{m}ahal. \hld\ Be·þiu ni welde hé þurh þero \alst{m}anno spráka &
\alst{h}ębbjan ênigan \alst{h}êr-dóm, \hld\ \alst{h}êlag drohtin, &
\alst{w}er-old-kuninges namon; \hld\ ni hé þȯ mid \alst{w}ordun stríd &
ni af·hóf wið þat \alst{f}olk \alst{f}urður, \hld\ ak \alst{f}ór imu þȯ, þár hé welde, &
an ên ge·\alst{b}irgi uppan: \hld\ flóh þat \alst{b}arn godes &
\alst{g}êlaro \alst{g}elp-kwidi \hld\ ęndi is \alst{j}ungaron hét &
ovar ênne \alst{s}êo \alst{s}ïðon \hld\ ęndi im \alst{s}elvo gi·bôd, &
hwar sie im eft te·\alst{g}ęgnes \hld\ \alst{g}angen skoldin.\eva

\bvb TODO.\evb\evg

\bvg\bva[35][2899]%
Þȯ te·\alst{l}ét þat \alst{l}iud-werod \hld\ aftar þemu \alst{l}ande allumu, &
te·\alst{f}ór \alst{f}olk mikil, \hld\ sïðor iro \alst{f}râho gi·wêt &
an þat ge·\alst{b}irgi uppan, \hld\ \alst{b}arno ríkjost, &
\alst{w}aldand an is \alst{w}illjon. \hld\ Þȯ te þes \alst{w}atares staðe &
\alst{s}amnodun þea ge·\alst{s}ïðos Kristes, \hld\ þe hé imu habde \alst{s}elvo gi·korane, &
sie \alst{t}welivi þurh iro \alst{t}rewa góda: \hld\ ni was im \alst{t}weho nigijan, &
nevu sie an þat \alst{g}odes þionost \hld\ \alst{g}erno weldin &
ovar þene \alst{s}êo \alst{s}ïðon. \hld\ Þȯ létun sie \alst{s}wíðjan strôm, &
\alst{h}ôh \alst{h}urnid-skip \hld\ \alst{h}luttron u̇ðjon, &
\alst{sk}êðan \alst{sk}ír water. \hld\ \alst{Sk}rêd lioht dages, &
\alst{s}unne warð an \alst{s}edle; \hld\ þe \alst{s}êo-líðandjan &
\alst{n}aht \alst{n}evulo bi·warp; \hld\ \alst{n}áðidun erlos &
\alst{f}orð-wardes an \alst{f}lód; \hld\ warð þiu \alst{f}iorðe tíd &
þera \alst{n}ahtes kuman \hld\ —\alst{n}ęrjendo Krist &
\alst{w}arode þea \alst{w}ág-líðand—: \hld\ þȯ warð \alst{w}ind mikil, &
\alst{h}ôh wedẹr af·\alst{h}aven: \hld\ \alst{h}lamodun u̇ðjon, &
\alst{st}rôm an \alst{st}amne; \hld\ \alst{st}rídjun fęridun &
þea \alst{w}eros wiðer \alst{w}inde, \hld\ was im \alst{w}rêð hugi, &
\alst{s}evo \alst{s}orgono ful: \hld\ \alst{s}elvon ni wándun &
\alst{l}agu-\alst{l}íðandja \hld\ an \alst{l}and kumen &
þurh þes \alst{w}ederes ge·\alst{w}in. \hld\ Þȯ gi·sáhun sie \alst{w}aldand Krist &
an þemu \alst{s}êe uppan \hld\ \alst{s}elvun gangan, &
\alst{f}aran an \alst{f}áðjon: \hld\ ni mahte an þene \alst{f}lód innan, &
an þene \alst{s}êo \alst{s}inkan, \hld\ hwand ine is \alst{s}elves kraft &
\alst{h}êlag ant·\alst{h}abde. \hld\ \alst{H}ugi warð an forhtun, &
þero \alst{m}anno \alst{m}ód-sevo: \hld\ and-rédun þat it im \alst{m}ahtig fíund &
te gi·\alst{d}roge \alst{d}ádi. \hld\ Þȯ sprak im iro \alst{d}rohtin tó, &
\alst{h}êlag \alst{h}evan-kuning, \hld\ ęndi sagde im þat hé iro \alst{h}êrro was &
\alst{m}ári ęndi \alst{m}ahtig: \hld\ „nu gí \alst{m}ódes skulun &
\alst{f}astes \alst{f}áhen; \hld\ ne sí iu \alst{f}orht hugi, &
gi·\alst{b}árjad gi \alst{b}ald-líko: \hld\ ik \alst{b}ium þat barn godes, &
is \alst{s}elves \alst{s}unu, \hld\ þe iu wið þesumu \alst{s}êe skal, &
\alst{m}undon wið þesan \alst{m}ęri-strôm.“ \hld\ Þȯ sprak imu ên þero \alst{m}anno an·gęgin &
ovar \alst{b}ord skipes, \hld\ \alst{b}ar-wirðig gumo, &
\alst{P}etrus þe gódo \hld\ —ni welde \alst{p}íne þolon, &
\alst{w}atares \alst{w}íti—: \hld\ „ef þú it \alst{w}aldand sís“, kwað hé, &
„\alst{h}êrro þe gódo, \hld\ só mi an mínumu \alst{h}ugi þunkit, &
hêt mí þan þarod \alst{g}angan te þí \hld\ ovar þesen \alst{g}evenes strôm, &
\alst{d}rokno ovar \alst{d}iap water, \hld\ ef þú mín \alst{d}rohtin sís, &
\alst{m}anagoro \alst{m}und-boro.“ \hld\ Þȯ hét ine \alst{m}ahtig Krist &
\alst{g}angan imu te·\alst{g}ęgnes. \hld\ hé warð \alst{g}aru sáno, &
\alst{st}ôp af þemu \alst{st}amne \hld\ ęndi \alst{st}rídjun géng &
\alst{f}orð te is \alst{f}rôjan. \hld\ Þiu \alst{f}lód ant·habde &
þene \alst{m}an þurh \alst{m}aht godes, \hld\ an-tat hé imu an is \alst{m}óde bi·gan &
and-ráden \alst{d}iap water, \hld\ þȯ hé \alst{d}ríven gi·sah &
þene \alst{w}ég mid \alst{w}indu: \hld\ \alst{w}undun ina u̇ðjon, &
\alst{h}ôh strôm umbi·\alst{h}ring. \hld\ Reht só hé þȯ an is \alst{h}ugi twehode, &
só \alst{w}êk imu þat \alst{w}ater under, \hld\ ęndi hé an þene \alst{w}ág innan, &
\alst{s}ank an þene \alst{s}êo-strôm, \hld\ ęndi hé hriop \alst{s}án aftar þiu &
\alst{g}áhon te þemu \alst{g}odes sunje \hld\ ęndi \alst{g}erno bad, &
þat hé ine þȯ ge·\alst{n}ęridi, \hld\ þȯ hé an \alst{n}ôdjun was, &
\alst{þ}egạn an ge·\alst{þ}winge. \hld\ \alst{Þ}iodo drohtin &
ant·\alst{f}éng ine mid is \alst{f}aðmun \hld\ ęndi \alst{f}rágode sána, &
te hwí hé þȯ ge·\alst{t}wehodi: \hld\ „Hwat þú mahtes ge·\alst{t}rúojan wel, &
\alst{w}iten þat te \alst{w}árun, \hld\ þat þi \alst{w}atares kraft &
an þemu \alst{s}êe innen \hld\ þínes \alst{s}ïðes ni mahte, &
\alst{l}agu-strôm gi·\alst{l}ęttjen, \hld\ só lango só þú habdes ge·\alst{l}ôvon te mi &
an þínumu \alst{h}ugi \alst{h}ardo. \hld\ Nu willju ik þi an \alst{h}elpun wesen, &
\alst{n}ęrjen þi an þesaru \alst{n}ôdi“. \hld\ Þȯ \alst{n}am ine alo-mahtig, &
\alst{h}êlag bi \alst{h}andun: \hld\ þȯ warð imu eft \alst{h}lutter water &
\alst{f}ast under \alst{f}ótun, \hld\ ęndi sie an \alst{f}áði samad &
\alst{b}êðja géngun, \hld\ an-tat sie ovar \alst{b}ord skipes &
\alst{st}ópun fan þemu \alst{st}rôme, \hld\ ęndi an þemu \alst{st}amne ge·sat &
allaro \alst{b}arno \alst{b}ętst. \hld\ Þȯ warð \alst{b}rêd water, &
\alst{st}rômos ge·\alst{st}illid, \hld\ ęndi sie te \alst{st}aðe kwámun, &
\alst{l}agu-\alst{l}íðandja \hld\ an \alst{l}and samen &
þurh þes \alst{w}ateres ge·\alst{w}in, \hld\ sagdun þo \alst{w}aldande þank, &
\alst{d}iurden iro \alst{d}rohtin \hld\ \alst{d}ádjun ęndi wordun, &
\alst{f}ellun imu te \alst{f}ótun \hld\ ęndi \alst{f}ilu sprákun &
\alst{w}ísaro \alst{w}ordo, \hld\ kwáðun þat sie \alst{w}issin garo, &
þat hé wári \alst{s}elvo \hld\ \alst{s}unu drohtines &
\alst{w}ár an þesaru \alst{w}er-oldi \hld\ ęndi ge·\alst{w}ald habdi &
ovar \alst{m}iddil-gard, \hld\ ęndi þat hé mahti allaro \alst{m}anno gi·hwes &
\alst{f}erạhe gi·\alst{f}ormon, \hld\ al só hé im an þemu \alst{f}lóde dede &
wið þes \alst{w}atares ge·\alst{w}in.\eva

\bvb TODO.\evb\evg

\bvg\bva[36][2973]%
\hspace*{100pt} Þȯ gi·wêt imu \alst{w}aldand Krist &%NOTE: in cæsura
\alst{s}ïðon fan þemu \alst{s}êe, \hld\ \alst{s}unu drohtines, &
\alst{ê}nag barn godes. \hld\ \alst{Ę}li-þioda kwam imu, &
\alst{g}umon te·\alst{g}ęgnes: \hld\ wárun is \alst{g}ódun werk &
\alst{f}erran ge·\alst{f}rági, \hld\ þat hé só \alst{f}ilu sagde &
\alst{w}ároro \alst{w}ordo: \hld\ imu was \alst{w}illjo mikil, &
þat hé su·lik \alst{f}olk-skępi \hld\ \alst{f}rummjen mósti, &
þat sie simla \alst{g}erno \hld\ \alst{g}ode þionodin, &
wárin ge·\alst{h}ôrige \hld\ \alst{h}evan-kuninge &
\alst{m}an-kunnjes \alst{m}anag. \hld\ Þȯ gi·wêt hé imu over þea \alst{m}arka Judeono, &
\alst{s}óhte imu \alst{S}idono burg, \hld\ habde ge·\alst{s}ïðos mid imu, &
\alst{g}óde \alst{j}ungaron. \hld\ Þár imu te·\alst{g}ęgnes kwam &
ên \alst{i}dis fan \alst{ȧ}ðrom þiodun; \hld\ siu was iru \alst{a}ðali-ge·burdjo, &
\alst{k}unnjes fan \alst{K}ananeo lande; \hld\ siu bad þene \alst{k}raftagan drohtin, &
\alst{h}êlagna, þat hé iru \alst{h}elpe ge·rédi, \hld\ kwað þat iru wári \alst{h}arm gi·standen, &
\alst{s}orọga at iru \alst{s}elvaru dohter, \hld\ kwað þat siu wári mid \alst{s}uhtjun bi·fangen: &
„be·\alst{d}rogan habbjad sie \alst{d}ęrnja wihti. \hld\ Nú is iro \alst{d}ôd at hęndi, &
þea \alst{w}rêðon habbjad sie ge·\alst{w}ittju be·numane. \hld\ Nu biddju ik þi, \alst{w}aldand frô min, &
\alst{s}elvo \alst{s}unu Dawides, \hld\ þat sie af su·likum \alst{s}uhtjun a·tómjes, &
þat þú sie só \alst{a}rma \hld\ \alst{ê}-gróht-fullo &
\alst{w}am-skaðon bi·\alst{w}eri.“ \hld\ Ni gaf iru þȯ noh \alst{w}aldand Krist &
\alst{ê}nig \alst{a}nd-wordi; \hld\ siu imu \alst{a}ftar géng, &
\alst{f}olgode \alst{f}ruokno, \hld\ an-tat siu te is \alst{f}ótun kwam, &
\alst{g}rótte ina \alst{g}reatandi. \hld\ \alst{J}ungaron Kristes &
\alst{b}ádun iro \alst{h}êrron, \hld\ þat hé an is \alst{h}ugja mildi &
\alst{w}urði þemu \alst{w}íve. \hld\ Þȯ habde eft is \alst{w}ord garu &
\alst{s}unu drohtines \hld\ ęndi te is ge·\alst{s}ïðun sprak: &
„\alst{ê}rist skal ik \alst{I}sraheles \hld\ \alst{a}voron werðen, &
\alst{f}olk-skępi te \alst{f}rumu, \hld\ þat sie \alst{f}erhtan hugi &
\alst{h}ębbjan te iro \alst{h}êrron: \hld\ im is \alst{h}elpono þarf, &
þea \alst{l}iudi sind far·\alst{l}orane, \hld\ far·\alst{l}áten habbjad &
\alst{w}aldandes \alst{w}ord, \hld\ þat \alst{w}erod is ge·twíflid, &
\alst{d}rívad im \alst{d}ęrnjan hugi, \hld\ ne willjad iro \alst{d}rohtine hôrjen &
\alst{I}srahelo \alst{e}rl-skępi, \hld\ \alst{u}n-gi·lôviga sind &
\alst{h}ęliðos iro \alst{h}êrron: \hld\ þoh skal þanen \alst{h}elpe kumen &
\alst{a}llun \alst{ę}li-þiodun.“ \hld\ \alst{A}galêto bad &
þat \alst{w}íf mid iro \alst{w}ordun, \hld\ þat iru \alst{w}aldand Krist &
an is \alst{m}ód-sevon \hld\ \alst{m}ildi wurði, &
þat siu iro \alst{b}arnes forð \hld\ \alst{b}rúkan mósti, &
\alst{h}ębbjan sie \alst{h}êle. \hld\ Þȯ sprak iru \alst{h}êrro an·gęgin, &
\alst{m}ári ęndi \alst{m}ahtig: \hld\ „nis þat“, kwað hé, „\alst{m}annes reht, &
\alst{g}umono nig·ênum \hld\ \alst{g}ód te gi·frummjenne &
þat hé is \alst{b}arnun \hld\ \alst{b}rôdes af·tíhe, &
\alst{w}ęrnje im ovar \alst{w}illjon, \hld\ láte sie \alst{w}íti þoljan, &
\alst{h}ungạr \alst{h}ęti-grimmen, \hld\ ęndi fódje is \alst{h}undos mid þiu.“ &
„\alst{W}ár is þat, \alst{w}aldand“, \hld\ kwað siu, „þat þú mid þínun \alst{w}ordun sprikis, &
\alst{s}ȯð-líko \alst{s}agis: \hld\ Hwat þoh oft an \alst{s}ęli innen &
undar iro \alst{h}êrron diske \hld\ \alst{h}welpos \alst{h}wervad &%NOTE: unusual alliteration
\alst{b}rosmono fulle \hld\ þero fan þemu \alst{b}iode niðer &
ant·\alst{f}allat iro \alst{f}rôjan.“ \hld\ Þȯ gi·hôrde þat \alst{f}riðu-barn godes &
\alst{w}illjan þes \alst{w}íves \hld\ ęndi sprak iru mid is \alst{w}ordun tó: &
„\alst{w}ela þat þú \alst{w}íf haves \hld\ \alst{w}illjan góden! &
\alst{M}ikil is þín gi·lôvo \hld\ an þea \alst{m}aht godes, &
an þene \alst{l}iudjo drohtin. \hld\ Al wirðid gi·\alst{l}êstid só &
umbi þínes \alst{b}arnes líf, \hld\ só þú \alst{b}ádi te mi.“ &
Þȯ warð siu sán gi·\alst{h}êlid, \hld\ só it þe \alst{h}êlago ge·sprak &
\alst{w}ordun \alst{w}ár-fastun: \hld\ þat \alst{w}íf fagonode, &
þes siu iro \alst{b}arnes forð \hld\ \alst{b}rúkan móste; &
\alst{h}abde iru gi·\alst{h}olpen \hld\ \alst{h}êljando Krist, &
habde sie far·\alst{f}angane \hld\ \alst{f}íundo kraftu, &
\alst{w}am-skaðun bi·\alst{w}ęrid. \hld\ Þȯ gi·wêt imu \alst{w}aldand forð, &
\alst{b}arno þat \alst{b}ętste, \hld\ sóhte imu \alst{b}urg ȯðre, &
þiu só \alst{þ}ikko was \hld\ mid þeru \alst{þ}iodu Judeono, &
mid \alst{s}u̇ðar-liudjun gi·\alst{s}eten. \hld\ Þár gi·fragn ik þat hé is ge·\alst{s}ïðos grótte, &
þe \alst{j}ungaron þe hé imu habde be is \alst{g}óde gi·korane, \hld\ þat sie mid imu \alst{g}erno ge·wunodun, &
\alst{w}eros þurh is \alst{w}íson spráka: \hld\ „alle skal ik iu“, kwað hé, „mid \alst{w}ordun frágon, &
\alst{j}ungaron míne: \hld\ hwat kweðat þese \alst{J}udeo liudi, &
\alst{m}ári \alst{m}ęgin-þioda, \hld\ hwat ik \alst{m}anno sí?“ &
Imu and-wordidun \alst{f}rô-líko \hld\ is \alst{f}riund an·gęgin, &
\alst{j}ungaron síne: \hld\ „nis þit \alst{J}udeono folk, &
\alst{e}rlos \alst{ê}n-wordje: \hld\ sum sagad þat þú \alst{E}lias sís, &
\alst{w}ís \alst{w}ár-sago, \hld\ þe hér giu \alst{w}as lango, &
\alst{g}ód undar þesumu \alst{g}um-skępje, \hld\ sum sagad þat þú \alst{J}ohannes sís, &
\alst{d}iur-lík \alst{d}rohtines bodo, \hld\ þe hér \alst{d}ôpte iu &
\alst{w}erod an \alst{w}atere; \hld\ alle sie mid \alst{w}ordun sprekad, &
þat þú \alst{ê}n-hwi-lik sís \hld\ \alst{ę}ðilero manno, &
þero \alst{w}ár-sagono, \hld\ þe hér mid \alst{w}ordun giu &
\alst{l}êrdun þese \alst{l}iudi, \hld\ ęndi þat þú sís eft an þit \alst{l}ioht kumen &
te \alst{w}ísjanne þesumu \alst{w}erode.“ \hld\ Þȯ sprak eft \alst{w}aldand Krist: &
„hwe kweðad \alst{g}i, þat ik sí“, \hld\ kwað hé, „\alst{j}ungaron míne, &
\alst{l}iovon \alst{l}iud-weros?“ \hld\ Þȯ te \alst{l}at ni warð &
\alst{S}ímon Petrus: \hld\ sprak \alst{s}án an·gęgin &
\alst{ê}no for im \alst{a}llun \hld\ —habde imu \alst{ę}lljen gód, &
\alst{þ}rístja gi·\alst{þ}ȧhti, \hld\ was is \alst{þ}eodone hold—:\eva

\bvb TODO.\evb\evg

\bvg\bva[37][3057]%
„Þú bist þe \alst{w}áro \hld\ \alst{w}aldandes sunu, &
\alst{l}ibbjendes godes, \hld\ þe þit \alst{l}ioht gi·skóp, &
\alst{K}rist \alst{k}uning êwig: \hld\ só willjad wí \alst{k}weðen alle, &
\alst{j}ungaron þíne, \hld\ þat þú sís \alst{g}od selvo, &
\alst{h}êljandero bętst.“ \hld\ Þȯ sprak imu eft is \alst{h}êrro an·gęgin: &
„\alst{s}álig bist þú \alst{S}ímon“, kwað hé, „\alst{s}unu Jonases; \hld\ ni mahtes þú þat \alst{s}elvo ge·huggjan, &
gi·\alst{m}arkon an þínun \alst{m}ód-gi·þȧhtjun, \hld\ ne it ni mahte þi \alst{m}annes tunge &
\alst{w}ordun ge·\alst{w}ísjen, \hld\ ak dede it þi \alst{w}aldand selvo, &
\alst{f}ader allaro \alst{f}iriho barno, \hld\ þat þú só \alst{f}orð gi·spráki, &
só \alst{d}iapo bi \alst{d}rohtin þínen. \hld\ \alst{D}iur-líko skalt þú þes lôn ant·fáhen, &
\alst{h}luttro havas þú an þínan \alst{h}êrron gi·lôvon, \hld\ \alst{h}ugi-skęfti sind þíne stêne ge·líka, &
só \alst{f}ast bist þú só \alst{f}elis þe hardo; \hld\ hêten skulun þi \alst{f}iriho barn &
\alst{s}ankte Péter: \hld\ ovar þemu stêne skal man mínen \alst{s}ęli wirkjan, &
\alst{h}êlag \alst{h}ús godes; \hld\ þár skal is \alst{h}íwiski tó &
\alst{s}álig \alst{s}amnon: \hld\ ni mugun wið þem þínun \alst{s}wíðjun krafte &
an·þebbjen \alst{h}ęllje portun. \hld\ Ik far·givu þi \alst{h}imil-ríkjas slutilas, &%TODO: Etymology an·þebbjen
þat þú móst \alst{a}ftar mi \hld\ \alst{a}llun gi·waldan &
\alst{k}ristinum folke; \hld\ \alst{k}umad alle te þi &
\alst{g}umono \alst{g}êstos; \hld\ þú have \alst{g}rôte gi·wald, &
hwene þú hér an \alst{e}rðu \hld\ \alst{ę}ldi-barno &
ge·\alst{b}inden willjes: \hld\ þemu is \alst{b}êðju gi·duan, &
\alst{h}imil-ríki bi·loken, \hld\ ęndi \alst{h}ęllje sind imu opana, &
\alst{b}rinnandi fiur; \hld\ só hwene só þú eft ant·\alst{b}inden wili, &
an-þeftjen is \alst{h}ęndi, \hld\ þemu is \alst{h}imil-ríki, &
ant·\alst{l}oken \alst{l}iohto mêst \hld\ ęndi \alst{l}íf êwig, &
\alst{g}róni \alst{g}odes wang. \hld\ Mid su·likaru ik þi \alst{g}evu willju &
\alst{l}ônon þínen gi·\alst{l}ôvon. \hld\ Ni willju ik, þat gí þesun \alst{l}iudjun noh, &
\alst{m}árjen þesaru \alst{m}ęnigi, \hld\ þat ik bium \alst{m}ahtig Krist, &
\alst{g}odes êgan barn. \hld\ Mi skulun \alst{J}udeon noh, &
\alst{u}n·skuldigna \hld\ \alst{e}rlos binden, &
\alst{w}êgjan mi te \alst{w}undrun \hld\ —dót mi \alst{w}ítjes filo— &
innan \alst{J}erusalem \hld\ \alst{g}êres ordun, &
\alst{á}htjen mínes \alst{a}ldres \hld\ \alst{ę}ggjun skarpun, &
bi·\alst{l}ôsjen mi \alst{l}ívu. \hld\ Ik an þesumu \alst{l}iohte skal &
þurh u̇ses \alst{d}rohtines kraft \hld\ fan \alst{d}ôde a·standen &
an \alst{þ}riddjumu dage“. \hld\ Þȯ warð \alst{þ}egno bętst &
\alst{s}wíðo an \alst{s}orgun, \hld\ \alst{S}ímon Petrus, &
warð imu \alst{h}ugi \alst{h}riwig, \hld\ ęndi te is \alst{h}êrron sprak &
\alst{r}ink an \alst{r}únun: \hld\ „ni skal þat \alst{r}íki god“, kwað hé, &
„\alst{w}aldand \alst{w}illjen, \hld\ þat þú eo su·lik \alst{w}íti mikil &
gi·\alst{þ}olos undar þesaru \alst{þ}iod: \hld\ nis þes \alst{þ}arf nigijan, &%TODO: Check nigijan
\alst{h}êlag drohtin.“ \hld\ Þȯ sprak imu eft is \alst{h}êrro an·gęgin, &
\alst{m}ári \alst{m}ahtig Krist \hld\ —was imu an is \alst{m}óde hold—: &
„Hwat þú nú \alst{w}iðer-\alst{w}ard bist“, \hld\ kwað hé, „\alst{w}illjon mínes, &
\alst{þ}egno bętsto! \hld\ Hwat þú þesaro \alst{þ}iodo kanst &
\alst{m}ęnniskan sidu: \hld\ þú ni wêst þe \alst{m}aht godes, &
þe ik gi·\alst{f}rummjen skal. \hld\ Ik mag þi \alst{f}ilu sęggjan &
\alst{w}árun \alst{w}ordun, \hld\ þár hér undar þesumu \alst{w}erode standad &
ge·\alst{s}ïðos míne, \hld\ þea ni mótun \alst{s}welten êr, &
\alst{h}werven an \alst{h}inen-fard \hld\ êr sie \alst{h}imiles lioht, &
\alst{g}odes ríki sehat.“ \hld\ Kôs imu \alst{j}ungarono þȯ &
\alst{s}án aftar þiu \hld\ \alst{S}ímon Petrus, &
\alst{J}akob ęndi \alst{J}ohannes, \hld\ ea \alst{g}umon twêne, &
\alst{b}êðja þea gi·\alst{b}róðer, \hld\ ęndi imu þȯ uppen þene \alst{b}erg gi·wêt &
\alst{s}under mid þem ge·\alst{s}ïðun, \hld\ \alst{s}álig barn godes, &
mid þem \alst{þ}egnun \alst{þ}rim, \hld\ \alst{þ}iodo drohtin, &
\alst{w}aldand þesaro \alst{w}er-oldes: \hld\ welde im þár \alst{w}undres filu, &
\alst{t}êkno \alst{t}ôgjan, \hld\ þat sie gi·\alst{t}rúodin þiu bet, &
þat hé \alst{s}elvo was \hld\ \alst{s}unu drohtines, &
\alst{h}êlag \alst{h}evan-kuning. \hld\ Þȯ sie an \alst{h}ôhan wall &
\alst{st}igun \alst{st}ên ęndi berg, \hld\ an-tat sie te þeru \alst{st}ędi kwámun, &
\alst{w}eros wiðer \alst{w}olkan, \hld\ þár \alst{w}aldand Krist, &
\alst{k}uningo \alst{k}raftigost \hld\ gi·\alst{k}oren habde, &
þat hé is \alst{g}od-kundi \hld\ \alst{j}ungarun sínun &
þurh is \alst{ê}nes kraft \hld\ \alst{ó}gjan welde, &
\alst{b}erht-lík \alst{b}iliði.\eva

\bvb TODO.\evb\evg

\bvg\bva[38][3122]%
\hspace*{100pt}Þȯ imu þár te \alst{b}edu gi·hnêg, &
þȯ warð imu þár \alst{u}ppe \hld\ \alst{ȯ}ðar-líkora &
\alst{w}liti ęndi gi·\alst{w}ádi: \hld\ wurðun imu is \alst{w}angun liohte, &
\alst{b}líkandi só þiu \alst{b}erhte sunne: \hld\ só skên þat \alst{b}arn godes, &
\alst{l}iuhte is \alst{l}ík-hamo: \hld\ \alst{l}iomon stódun &
\alst{w}ánamo fan þemu \alst{w}aldandes barne; \hld\ warð is ge·\alst{w}ádi só hwít &
só \alst{s}nêw te \alst{s}ehanne. \hld\ Þȯ warð þár \alst{s}eld-lík þing &
gi·\alst{ô}gid aftar þiu: \hld\ \alst{E}lias ęndi Moyses &
\alst{k}wámun þár te \alst{K}riste \hld\ wið só \alst{k}raftagne &
\alst{w}ordun \alst{w}ehsljan. \hld\ Þár warð só \alst{w}un-sam spráka, &
só \alst{g}ód word undar \alst{g}umun, \hld\ þár þe \alst{g}odes sunu &
wið þea \alst{m}árjan \alst{m}an \hld\ \alst{m}ahljen welde, &
só \alst{b}líði warð uppan þemu \alst{b}erge: \hld\ skên þat \alst{b}erhte lioht, &
was þár \alst{g}ard \alst{g}ód-lík \hld\ ęndi \alst{g}róni wang, &
\alst{P}aradíse ge·lík. \hld\ \alst{P}etrus þȯ gi·mahạlde, &
\alst{h}ęlið \alst{h}ard-módig \hld\ ęndi te is \alst{h}êrron sprak, &
\alst{g}rótte þene \alst{g}odes sunu: \hld\ „\alst{g}ód is it hér te wesanne, &
ef þú it gi·\alst{k}iosan wili, \hld\ \alst{K}rist alo-waldo, &
þat man þí \alst{h}ér an þesaru \alst{h}ôhe \hld\ ên \alst{h}ús ge·wirkja, &
\alst{m}ár-líko ge·\alst{m}ako \hld\ ęndi \alst{M}oysese ȯðer &
ęndi \alst{E}liase þriddja: \hld\ þit is \alst{ô}das hêm, &
\alst{w}elono \alst{w}un-samost.“ \hld\ Reht só hé þȯ þat \alst{w}ord ge·sprak, &
só ti·\alst{l}ét þiu \alst{l}uft an twê: \hld\ \alst{l}ioht wolkan skên, &
\alst{g}lítandi \alst{g}límo, \hld\ ęndi þea \alst{g}ódun man &
\alst{w}liti-skôni be·\alst{w}arp. \hld\ Þȯ fan þemu \alst{w}olkne kwam &
\alst{h}êlag stemne godes, \hld\ ęndi þem \alst{h}ęliðun þár &
\alst{s}elvo \alst{s}agde, \hld\ þat þat is \alst{s}unu wári, &
\alst{l}ibbjendero \alst{l}iovost: \hld\ „an þemu mí \alst{l}íkod wel &
an mínun \alst{h}ugi-skęftjun. \hld\ Þemu gí \alst{h}ôrjen skulun, &
ful·\alst{g}angad imu \alst{g}erno.“ \hld\ Þȯ ni mahtun þea \alst{j}ungaron Kristes &
þes \alst{w}olknes \alst{w}liti \hld\ ęndi \alst{w}ord godes, &
þea is \alst{m}ikilon \alst{m}aht \hld\ þea \alst{m}an ant·standen, &
ak sie bi·\alst{f}ellun þȯ \alst{f}orð-wardes: \hld\ \alst{f}erhes ni wándun, &
\alst{l}ęngiron \alst{l}íves. \hld\ Þȯ géng im tó þe \alst{l}andes ward, &
be·\alst{h}rên sie mid is \alst{h}andun \hld\ \alst{h}êljandero bętst, &
hét þat sie im ni an·\alst{d}rédin: \hld\ „ni skal iu hér \alst{d}erjen eo·wiht, &
þes gí hér \alst{s}eld-líkes \hld\ gi·\alst{s}ehen habbjad, &
\alst{m}érjaro þingo.“ \hld\ Þȯ eft þem \alst{m}annun warð &
\alst{h}ugi at iro \alst{h}erton \hld\ ęndi gi·\alst{h}êlid mód, &
gi·\alst{b}ade an iro \alst{b}reostun: \hld\ gi·sáhun þat \alst{b}arn godes &
\alst{ê}nna standen, \hld\ was þat \alst{ȯ}ðer þȯ, &
be·\alst{h}liden \alst{h}imiles lioht. \hld\ Þȯ gi·wêt imu þe \alst{h}êlago Krist &
fan þemu \alst{b}erge niðer; \hld\ gi·\alst{b}ôd aftar þiu &
\alst{j}ungarun sínun, \hld\ þat sie ovar \alst{J}udeono folk &
ni \alst{s}agdin þea gi·\alst{s}ioni: \hld\ „er þan ik \alst{s}elvo hér &
swíðo \alst{d}iur-líko \hld\ fan \alst{d}ôðe a·stande, &
a·\alst{r}íse fan þeru \alst{r}estu: \hld\ sïðor mugun gí it \alst{r}ękkjen forð, &
\alst{m}árjen ovar \alst{m}iddil-gard \hld\ \alst{m}anagun þiodun &
\alst{w}ído aftar þesaru \alst{w}er-oldi.“\eva

\bvb TODO.\evb\evg

\bvg\bva[39][3170]%
\hspace*{100pt}Þȯ gi·wêt imu \alst{w}aldand Krist &
eft an \alst{G}alileo land, \hld\ sóhte is \alst{g}adulingos, &
\alst{m}ahtig is \alst{m}ágo hêm, \hld\ sagde þár \alst{m}anages hwat &
\alst{b}erhtero \alst{b}iliðjo, \hld\ ęndi þat \alst{b}arn godes &
þem is \alst{s}áligun ge·\alst{s}ïðun \hld\ \alst{s}org-spell ni for·hal, &
ak hé im \alst{o}pen-líko \hld\ \alst{a}llun sagde, &
þem is \alst{g}ódun \alst{j}ungarun, \hld\ hwó ine skolde þat \alst{J}udeono folk &
\alst{w}êgjan te \alst{w}undrun. \hld\ Þes wurðun þár \alst{w}íse man &
\alst{s}wíðo an \alst{s}orgun, \hld\ warð im \alst{s}êr hugi, &
\alst{h}riwig umbi iro \alst{h}erte: \hld\ gi·hôrdun iro \alst{h}êrron þȯ, &
\alst{w}aldandes sunu \hld\ \alst{w}ordun tęlljen, &
hwat hé undar þeru \alst{þ}iodu \hld\ \alst{þ}olojan skolde, &
\alst{w}illjendi undar þemu \alst{w}erode. \hld\ Þȯ gi·wêt imu \alst{w}aldand Krist, &
\alst{g}umo fan \alst{G}alilea, \hld\ sóhte imu \alst{J}udeono burg, &
\alst{k}wámun im te \alst{K}afarnaum. \hld\ Þár fundun sie ênan \alst{k}uninges þegạn &
\alst{w}lankan undar þemu \alst{w}erode: \hld\ kwað þat hé wári gi·\alst{w}ęldig bodo &
\alst{a}ðal-kêsures; \hld\ hé grótte \alst{a}ftar þiu &
\alst{S}ímon Petrusen, \hld\ kwað þat hé wári gi·\alst{s}ęndid þarod, &
þat hé þár gi·\alst{m}anodi \hld\ \alst{m}anno ge·hwi-liken &
þero \alst{h}ôvid-skatto, \hld\ þe sie te þemu \alst{h}ove skoldin &
\alst{t}insi gelden: \hld\ „nis þes \alst{t}weho ênig &
\alst{g}umono ni-gj·ênumu, \hld\ ne sie ina far·\alst{g}elden sán &
\alst{m}êðmo kustjon, \hld\ bi·úten iuwe \alst{m}êster êno &
havad it far·\alst{l}áten. \hld\ Ni skal þat \alst{l}íkon wel &
mínumu \alst{h}êrron, \hld\ só man it imu at is \alst{h}ove ku̇ðid, &
\alst{a}ðal-kêsure.“ \hld\ Þȯ géng \alst{a}ftar þiu &
\alst{S}ímon Petrus, \hld\ welde it \alst{s}ęggjan þȯ &
\alst{h}êrron sínumu: \hld\ hé was is an is \alst{h}ugi iu þan, &%TODO: Check sínumu.
gi·\alst{w}aro \alst{w}aldand Krist: \hld\ —imu ni mahte \alst{w}ord ênig &
bi·\alst{h}olen werðen, \hld\ hé wisse \alst{h}ugi-skęfti &
\alst{m}anno ge·hwi-likes—: \hld\ hét þȯ þene is \alst{m}árjan þegạn, &
\alst{S}ímon Petrus \hld\ an þene \alst{s}êo innen &
\alst{a}ngul werpen: \hld\ „su·liken só þú þár \alst{ê}rist mugis &
\alst{f}isk gi·\alst{f}áhen“, \hld\ kwað hé, „só teoh þú þene fan þemu \alst{f}lóde te þi, &
ant·\alst{k}lęmmi imu þea \alst{k}inni: \hld\ þár maht þú undar þem \alst{k}aflon nimen &
\alst{g}uldine skattos, \hld\ þat þú far·\alst{g}elden maht &
þemu \alst{m}anne te gi·\alst{m}ódja \hld\ \alst{m}ínen ęndi þínen &
\alst{t}insjo só hwi-likan, \hld\ só hé u̇s \alst{t}ó sókid.“ &
Hé ni þorfte imu þȯ \alst{a}ftar þiu \hld\ \alst{ȯ}ðaru wordu &
\alst{f}urður gi·bioden: \hld\ géng \alst{f}iskari gód, &
\alst{S}ímon Petrus, \hld\ warp an þene \alst{s}êo innen &
\alst{a}ngul an \alst{u̇}ðjon \hld\ ęndi \alst{u}p gi·tôh &
\alst{f}isk an \alst{f}lóde \hld\ mid is \alst{f}olmun twêm, &
te·\alst{k}lóf imu þea \alst{k}inni \hld\ ęndi undar þem \alst{k}aflun nam &
\alst{g}uldine skattos: \hld\ dede al, só imu þe \alst{g}odes sunu &
\alst{w}ordun ge·\alst{w}ísde. \hld\ Þár was þȯ \alst{w}aldandes &
\alst{m}ęgin-kraft gi·\alst{m}árid, \hld\ hwó skal allaro \alst{m}anno ge·hwi-lik &
swíðo \alst{w}illjendi \hld\ is \alst{w}er-old-hêrron &
\alst{sk}uldi ęndi \alst{sk}attos, \hld\ þea imu gi·\alst{sk}ęride sind, &
\alst{g}erno \alst{g}elden: \hld\ ni skal ine far·\alst{g}úmon eo·wiht, &
ni far·\alst{m}uni ine an is \alst{m}óde, \hld\ ak wese imu \alst{m}ildi an is hugi, &
\alst{þ}iono imu \alst{þ}io-líko: \hld\ an þiu mag hé \alst{þ}iod-godes &
\alst{w}illjan ge·\alst{w}irkjan \hld\ ęndi ôk is \alst{w}er-old-hêrron &
\alst{h}uldi \alst{h}abbjen.\eva

\bvb TODO.\evb\evg

\bvg\bva[40][3223]%
Só lêrde þe \alst{h}êlago Krist &
þea is \alst{g}ódon \alst{j}ungaron: \hld\ „ef ênig \alst{g}umono wið iu“, kwað hé, &
„\alst{s}undja ge·wirkja, \hld\ þan nim þú ina \alst{s}undạr te þi, &
þene \alst{r}ink an \alst{r}úna \hld\ ęndi imu is \alst{r}ád saga, &
\alst{w}ísi imu mid \alst{w}ordun. \hld\ Ef imu þan þes \alst{w}erð ne sí, &
þat hé þí gi·\alst{h}ôrje, \hld\ \alst{h}ala þí þár ȯðara tó &
\alst{g}ódaro \alst{g}umono, \hld\ ęndi lah imu is \alst{g}rimmun werk, &
\alst{s}ak ina \alst{s}ȯð-wordun. \hld\ Ef imu þan is \alst{s}undja aftar þiu, &
\alst{l}ôs-werk ni \alst{l}êðon, \hld\ gi·duo it ȯðrun \alst{l}iudjun ku̇ð, &
\alst{m}ári it þan for \alst{m}ęnegi \hld\ ęndi lát \alst{m}anno filu &
\alst{w}iten is far·\alst{w}urhti: \hld\ óðo be·ginnad imu þan is \alst{w}erk tregan, &
an is \alst{h}ugi \alst{h}rewen, \hld\ þan hé it gi·hôrid \alst{h}ęliðo filu, &
\alst{a}hton \alst{ę}ldi-barn \hld\ ęndi imu is \alst{u}vilon dád &
\alst{w}ęrjad mid \alst{w}ordun. \hld\ Ef hé þan ôk \alst{w}ęndjen ne wili, &
ak far·\alst{m}ódat su·lika \alst{m}ęnegi, \hld\ þan lát þú þene \alst{m}an faren, &
\alst{h}ava ina þan far \alst{h}êðinen \hld\ ęndi lát ina þi an þínumu \alst{h}ugi lêðen, &
\alst{m}íð is an þínumu \alst{m}óde, \hld\ ne sí þat imu eft \alst{m}ildi god, &
\alst{h}êr \alst{h}evan-kuning \hld\ \alst{h}elpe far·líhe, &
\alst{f}ader allaro \alst{f}iriho barno.“ \hld\ Þȯ \alst{f}rágode Petrus, &
allaro \alst{þ}egno bętst \hld\ \alst{þ}eodan sínan: &
„hwó oft skal ik þem \alst{m}annun, \hld\ þe wið \alst{m}í habbjad &
\alst{l}êð-werk gi·duan, \hld\ \alst{l}eovo drohtin, &
skal ik im \alst{s}ivun \alst{s}ïðun \hld\ iro \alst{s}undja a·láten, &
\alst{w}rêðaro \alst{w}erko, \hld\ êr þan ik is êniga \alst{w}réka frummje, &%TODO: Check wréka.
\alst{l}êðes te \alst{l}ône?“ \hld\ Þȯ sprak eft þe \alst{l}andes ward, &
an·gęgin þe \alst{g}odes sunu \hld\ \alst{g}ódumu þegne: &
„ni \alst{s}ęggju ik þi fan \alst{s}ivunjun, \hld\ só þú \alst{s}elvo sprikis, &
\alst{m}ahlis mid þínu \alst{m}u̇ðu, \hld\ ik duom þi \alst{m}êra þár tó: &
\alst{s}ivun \alst{s}ïðun \alst{s}ivun-tig \hld\ só skalt þú \alst{s}undja ge·hwemu, &
\alst{l}êðes a·\alst{l}áten: \hld\ só willju ik þi te \alst{l}êrun geven &
\alst{w}ordun \alst{w}ár-fastun. \hld\ Nu ik þí su·lika gi·\alst{w}ald far·gaf, &
þat þú mínes \alst{h}íwiskes \hld\ \alst{h}êrost wáris, &
\alst{m}anages \alst{m}ann-kunnjes, \hld\ nu skalt þú im \alst{m}ildi wesen, &
\alst{l}iudjun \alst{l}íði.“ \hld\ Þȯ þár te þemu \alst{l}êrjande kwam &
ên \alst{j}ung man an·\alst{g}ęgin \hld\ ęndi frágode \alst{J}esu Krist: &
„\alst{m}êster þe gódo“, \hld\ kwað hé, „hwat skal ik \alst{m}anages duan, &
an þiu þe ik \alst{h}evan-ríki \hld\ ge·\alst{h}alan móti?“ &
Habde imu \alst{ô}d-welon \hld\ \alst{a}llen ge·wunnen, &
\alst{m}êðọm-hord \alst{m}anag, \hld\ þoh hé \alst{m}ildjan hugi &
\alst{b}ári an is \alst{b}reostun. \hld\ Þȯ sprak imu þat \alst{b}arn godes: &
„hwat kwiðis þú umbi \alst{g}ódon? \hld\ nis þat \alst{g}umono ênig &
bi·útan þe \alst{ê}no, \hld\ þe þár \alst{a}l ge·skóp, &
\alst{w}er-old ęndi \alst{w}unnja. \hld\ Ef þú is \alst{w}illjan havas, &
þat þú an \alst{l}ioht godes \hld\ \alst{l}íðan mótis, &
þan skalt þú bi·\alst{h}alden \hld\ þea \alst{h}êlagon lêra, &
þe þár an þemu \alst{a}ldon \hld\ \alst{ê}wa ge·biudid, &
þat þú \alst{m}an ni slah, \hld\ ni þú \alst{m}ênes ni sweri, &
far·\alst{l}egar-nessi far·\alst{l}át \hld\ ęndi \alst{l}uggi ge·wit-skępi, &
\alst{st}ríd ęndi \alst{st}ulina; \hld\ ne wis þú te \alst{st}ark an hugi, &
ne \alst{n}íðin ne hatul, \hld\ ni \alst{n}ôd-róf ni fręmi; &
\alst{a}v-unst \alst{a}lla far·lát; \hld\ wis þínun \alst{ę}ldirun gód, &
\alst{f}ader ęndi móder, \hld\ ęndi þínun \alst{f}riundun hold, &
þem \alst{n}áhistun gi·\alst{n}áðig. \hld\ Þan þú þi gi·\alst{n}iodon móst &
\alst{h}imilo ríkjas, \hld\ ef þú it bi·\alst{h}alden wili, &
ful-\alst{g}angan \alst{g}odes lêrun.“ \hld\ Þȯ sprak eft þe \alst{j}ungo man &
„al hębbju ik só gi·\alst{l}êstid“, \hld\ kwað hé, „só þú mi \alst{l}êris nu, &
\alst{w}ordun \alst{w}ísis, \hld\ só ik is eo \alst{w}iht ni far·lét &
fan mínero \alst{k}indiski.“ \hld\ Þȯ bi·gan ina \alst{K}rist sehan &
\alst{a}n mid is \alst{ô}gun: \hld\ „\alst{ê}n is þár noh nu“, kwað hé, &
„\alst{w}an þero \alst{w}erko: \hld\ ef þú is \alst{w}illjon havas, &
þat þú \alst{þ}urh-fręmid \hld\ \alst{þ}ionon mótis &
\alst{h}êrron þínumu, \hld\ þan skalt þú þat þín \alst{h}ord nimen, &
skalt þínan \alst{ô}d-welon \hld\ \alst{a}llan far·kôpjen, &
\alst{d}iurje mêðmos, \hld\ ęndi \alst{d}êljen hét &
\alst{a}rmun mannun: \hld\ þan havas þú \alst{a}ftar þiu &
\alst{h}ord an \alst{h}imile; \hld\ kum þi þan gi·\alst{h}alden te mi, &
\alst{f}olgo þi mínaro \alst{f}ęrdi: \hld\ þan havas þú \alst{f}riðu sïður.“ &
Þȯ wurðun \alst{K}ristes word \hld\ \alst{k}ind-jungumu manne &
\alst{s}wíðo an \alst{s}orgun, \hld\ was imu \alst{s}êr hugi, &
\alst{m}ód umbi herte: \hld\ habde \alst{m}êðmo filu, &
\alst{w}elono ge·\alst{w}unnen; \hld\ \alst{w}ęnde imu eft þanen, &
was imu \alst{u}n-\alst{ó}ðo \hld\ \alst{i}nnan breostun, &
an is \alst{s}evon \alst{s}wáro. \hld\ \alst{S}ah imu aftar þȯ &
\alst{K}rist alo-waldo, \hld\ \alst{k}wað it þȯ, þár hé welde, &
te þem is \alst{j}ungarun \alst{g}ęgin-wardun, \hld\ þat wári an \alst{g}odes ríki &
\alst{u}n-óði \alst{ô}dagumu manne \hld\ \alst{u}p te kumanne: &
„\alst{ó}ður mag man \alst{o}lvundjon, \hld\ þoh hé sí \alst{u}n-met grôt, &%TODO: check ódur
þurh \alst{n}áðlan gat, \hld\ þoh it sí \alst{n}aru swíðo, &
\alst{s}áftur þurh·\alst{s}lópjen, \hld\ þan mugi kuman þiu \alst{s}iole te himile &
þes \alst{ô}dagan mannes, \hld\ þe hér \alst{a}l havad &
gi·\alst{w}ęndid an þene \alst{w}er-old-skat \hld\ \alst{w}illjon sínen, &
\alst{m}ód-gi·þȧhti, \hld\ ęndi ni hugid umbi þie \alst{m}aht godes.“\eva

\bvb TODO.\evb\evg

\bvg\bva[41][3305]%
Imu \alst{a}nd-wordjade \hld\ \alst{ê}r-þungan gumo, &
\alst{S}ímon Petrus, \hld\ ęndi \alst{s}ęggjan bad &
\alst{l}eovan hêrron: \hld\ „Hwat skulun wí þes te \alst{l}ône nimen“, kwað hé, &
„\alst{g}ódes te \alst{g}elde, \hld\ þes wí þurh þín \alst{j}ungar-dóm &
\alst{ê}gan ęndi \alst{ę}rvi \hld\ \alst{a}l far·létun &
\alst{h}ovos ęndi \alst{h}íwiski \hld\ ęndi þi te \alst{h}êrron gi·kurun, &
\alst{f}olgodun þínaru \alst{f}ęrdi: \hld\ hwat skal u̇s þes te \alst{f}rumu werðen, &
\alst{l}anges te \alst{l}ône?“ \hld\ \alst{L}iudjo drohtin &
\alst{s}agde im þȯ \alst{s}elvo: \hld\ „Þan ik \alst{s}ittjen kumu“, kwað hé, &
„an þie \alst{m}ikilan \alst{m}aht \hld\ an þemu \alst{m}árjan dage, &
þár ik \alst{a}llun skal \hld\ \alst{i}rmin-þiodun &
\alst{d}ómos a·\alst{d}êljen, \hld\ þan mótun gi mid iuwomu \alst{d}rohtine þár &
\alst{s}elvon \alst{s}ittjen \hld\ ęndi mótun þera \alst{s}aka waldan: &
mótun gí \alst{I}srahelo \hld\ \alst{ę}ðili-folkun &
a·\alst{d}êljen aftar iro \alst{d}ádjun: \hld\ só mótun gi þár gi·\alst{d}iuride wesen. &
Þan sęggju ik iu te \alst{w}áran: \hld\ só hwe só þat an þesaru \alst{w}er-oldi gi·duot, &
þat hé þurh \alst{m}ína \alst{m}innja \hld\ \alst{m}ágo ge·sidli &
\alst{l}iof far·\alst{l}étid, \hld\ þes skal hi hér \alst{l}ôn niman &
\alst{t}ehan sïðun \alst{t}ehin-fald, \hld\ ef hé it mid \alst{t}rewon duot, &
mid \alst{h}luttru \alst{h}ugi. \hld\ Ovar þat havad hé ôk \alst{h}imiles lioht, &
\alst{o}pen \alst{ê}wig líf.“ \hld\ Bi·gan imu þȯ \alst{a}ftar þiu &
allaro \alst{b}arno \alst{b}ętst \hld\ ên \alst{b}iliði sęggjan, &
kwað þat þár \alst{ê}n \alst{ô}dag man \hld\ an \alst{ê}r-dagun &
\alst{w}ári undar þemu \alst{w}erode: \hld\ þe habde \alst{w}elono ge·nóg, &
\alst{s}inkas gi·\alst{s}amnod \hld\ ęndi imu \alst{s}imlun was &
\alst{g}aru mid \alst{g}oldu \hld\ ęndi mid \alst{g}odo-wębbju, &
\alst{f}agạrun \alst{f}ratahun \hld\ ęndi imu so \alst{f}ilu habde &
\alst{g}ódes an is \alst{g}ardun \hld\ ęndi imu at \alst{g}ômun sat &
allaro \alst{d}ago ge·hwi-likes: \hld\ habde imu \alst{d}iur-lík líf, &
\alst{b}líðsja an is \alst{b}ęnkjun. \hld\ Þan was þár eft ên \alst{b}iddjendi man, &
gi·\alst{l}évod an is \alst{l}ík-hamon, \hld\ \alst{L}azarus was hé hêten, &
lag imu \alst{d}ago ge·hwi-likes \hld\ at þem \alst{d}urun foren, &
þár hé þene \alst{ô}dagan man \hld\ \alst{i}nne wisse &
an is \alst{g}ęst-sęli \hld\ \alst{g}ôme þiggjan, &
\alst{s}ittjen at \alst{s}umble, \hld\ ęndi hé \alst{s}imlun bêd &
gi·\alst{a}rmod þár \alst{ú}te: \hld\ ni móste þár \alst{i}n kuman, &
ne hé ni mahte ge·\alst{b}iddjen, \hld\ þat man imu þes \alst{b}rôdes þarod &
gi·\alst{d}ragan weldi, \hld\ þes þár fan þemu \alst{d}iske niðer &
ant·\alst{f}el undar iro \alst{f}óti: \hld\ ni mahte imu þár ênig \alst{f}ruma werðen &
fan þemu \alst{h}êroston, þe þes \alst{h}úses gi·weld, \hld\ bi·útan þat þár géngun is \alst{h}undos tó, &
\alst{l}ikkodun is \alst{l}ík-wundon, \hld\ þár hé \alst{l}iggjandi &
\alst{h}ungạr þolode; \hld\ ni kwam imu þár te \alst{h}elpu wiht &
fan þemu \alst{r}íkjon manne. \hld\ Þȯ gi·fragn ik þat ina is \alst{r}egano gi·skapu, &
þene \alst{a}rmon man \hld\ is \alst{ê}n-dago &
gi·\alst{m}anoda \alst{m}ahtjun swíð, \hld\ þat hé \alst{m}anno drôm &
a·\alst{g}even skolde. \hld\ \alst{G}odes ęngilos &
ant·\alst{f}éngun is \alst{f}erh \hld\ ęndi lêddun ine \alst{f}orð þanen, &
þat sie an \alst{A}brahames barm \hld\ þes \alst{a}rmon mannes &
\alst{s}iole gi·\alst{s}ęttun: \hld\ þár móste hé \alst{s}imlun forð &
\alst{w}esen an \alst{w}unnjun. \hld\ Þȯ kwámun ôk \alst{w}urde-gi·skapu, &
þemu \alst{ô}dagan man \hld\ \alst{o}r-lag-hwíle, &
þat hé þit \alst{l}ioht far·\alst{l}ét: \hld\ \alst{l}êða wihti &
be·\alst{s}inkodun is \alst{s}iole \hld\ an þene \alst{s}warton hęl, &
an þat \alst{f}ern innen \hld\ \alst{f}íundun te willjan, &
be·\alst{g}róvun ine an \alst{g}ramono hêm. \hld\ Þanen mahte hé þene \alst{g}ódan skawon, &
\alst{A}braham ge·sehen, \hld\ þár hé \alst{u}ppe was &
\alst{l}íves an \alst{l}ustun, \hld\ ęndi \alst{L}azarus sat &
\alst{b}líði an is \alst{b}arme, \hld\ berht \alst{l}ôn ant·féng &
\alst{a}llaro is \alst{a}rm-ódjo, \hld\ ęndi lag þe \alst{ô}dago man &
\alst{h}êto an þeru \alst{h}ęllju, \hld\ \alst{h}riop up þanen: &
„\alst{f}ader Abraham“, \hld\ kwað hé, „mí is \alst{f}irinun þarf, &
þat þú \alst{m}í an þínumu \alst{m}ód-sevon \hld\ \alst{m}ildi werðes, &
\alst{l}íði an þesaru \alst{l}ognu: \hld\ sęndi mi \alst{L}azarus herod, &
þat hé mí ge·\alst{f}órja \hld\ an þit \alst{f}ern innan &
\alst{k}aldes wateres. \hld\ Ik hér \alst{k}wik brinnu &
\alst{h}êto an þesaru \alst{h}ęllju: \hld\ nu is mi þínaro \alst{h}elpono þarf, &
þat hé mí a·\alst{l}ęskje \hld\ mid is \alst{l}uttikon fingru &
\alst{t}ungon míne, \hld\ nu siu \alst{t}êkạn havad, &
\alst{u}vil \alst{a}rvedi. \hld\ \alst{I}nwid-rádo, &
\alst{l}êðaro spráka, \hld\ alles is mi nu þes \alst{l}ôn kumen.“ &
Imu \alst{a}nd-wordjade þȯ \alst{A}braham \hld\ —þat was \alst{a}ld-fader—: &
„ge·\alst{h}ugi þú an þínumu \alst{h}erton“, \hld\ kwað hé, „hwat þú \alst{h}abdes iu &
\alst{w}elono an \alst{w}er-oldi. \hld\ Hwat þú þár alle þíne \alst{w}unnja far·sliti, &
\alst{g}ódes an \alst{g}ardun, \hld\ só hwat só þi \alst{g}iviðig forð &
\alst{w}erðen skolde. \hld\ \alst{W}íti þolode &
\alst{L}azarus an þemu \alst{l}iohte, \hld\ habde þár \alst{l}êðes filu, &
\alst{w}ítjas an \alst{w}er-oldi. \hld\ Be·þiu skal hé nu \alst{w}elon êgan, &
\alst{l}ibbjen an \alst{l}ustun: \hld\ þú skalt þea \alst{l}ogna þolan, &
\alst{b}rinnendi fiur: \hld\ ni mag is þi ênig \alst{b}óte kumen &
\alst{h}inana te \alst{h}ęllju: \hld\ it havad þe \alst{h}êlago god &
só gi·\alst{f}astnod mid is \alst{f}aðmun: \hld\ ni mag þár \alst{f}aren ênig &
\alst{þ}egno þurh þat \alst{þ}iustri: \hld\ it is hér só \alst{þ}ikki undar u̇s.“ &
Þȯ sprak eft \alst{A}brahame \hld\ þe \alst{e}rl te·gęgnes &
fan þeru \alst{h}êtan \alst{h}ęll \hld\ ęndi \alst{h}elpono bad, &
þat hé \alst{L}azarus \hld\ an \alst{l}iudjo drôm &
\alst{s}elvon \alst{s}andi: \hld\ „þat hé ge·\alst{s}ęggja þár &
\alst{b}róðarun mínun, \hld\ hwó ik hér \alst{b}rinnendi &
\alst{þ}rá-werk \alst{þ}olon; \hld\ si þár undar þeru \alst{þ}iodu sind, &
si \alst{f}ïvi undar þemu \alst{f}olke: \hld\ ik an \alst{f}orhtun bium, &
þat sie im þár far·\alst{w}irkjen, \hld\ þat sie skulin ôk an þit \alst{w}íti te mi, &
an só \alst{g}rádag fiur.“ \hld\ Þȯ imu eft te·\alst{g}ęgnes sprak &
\alst{A}braham \alst{a}ld-fader, \hld\ kwað þat sie þár \alst{ê}o godes &
an þemu \alst{l}and-skępi, \hld\ \alst{l}iudi habdin, &
\alst{M}oyseses gi·bôd \hld\ ęndi þár \alst{m}anagaro tó &
\alst{w}ár-saguno \alst{w}ord: \hld\ „ef sie is \alst{w}illige sind, &
þat sie þat bi·\alst{h}alden, \hld\ þan ni þurvun sie an þea \alst{h}ęll innen, &
an þat \alst{f}ern \alst{f}aren, \hld\ ef sie ge·\alst{f}rummjad só, &
só þea ge·\alst{b}iodad, \hld\ þe þea \alst{b}ók lesat &
þem \alst{l}iudjun te \alst{l}êrun. \hld\ Ef sie þes þan ni willjad \alst{l}êstjen wiht, &
þanne ni \alst{h}ôrjad sie ôk \hld\ þemu þe \alst{h}inan a·stád, &
\alst{m}an fan dôðe. \hld\ Láte man sie an iro \alst{m}ód-sevon &
\alst{s}elvon keosen, \hld\ hweðer im \alst{s}wótjera þunkje &
te gi·\alst{w}innanne, \hld\ só lango só sie an þesaru \alst{w}er-oldi sind, &
þat sie \alst{e}ft \alst{u}vil eþþa gód \hld\ \alst{a}ftar habbjen.“\eva

\bvb TODO.\evb\evg

\bvg\bva[42][3409]%
Só \alst{l}êrde hé þȯ þea \alst{l}iudi \hld\ \alst{l}iohton wordon, &
allaro \alst{b}arno \alst{b}ętst, \hld\ ęndi \alst{b}iliði sagde &
\alst{m}anag \alst{m}an-kunnje \hld\ \alst{m}ahtig drohtin, &
kwað þat imu ên \alst{s}álig gumo \hld\ \alst{s}amnon bi·gunni &
\alst{m}an an \alst{m}orgen, \hld\ „ęndi im \alst{m}éda gi·hét, &
þe \alst{h}êrosto þes \alst{h}íwiskjas, \hld\ swíðo *\alst{h}old-lík lôn“, &
kwað þat hie iro \alst{a}llaro gi·hwem \hld\ \alst{ê}nna gávi &
\alst{s}ilọvrinna skat. \hld\ „Þuȯ \alst{s}amnodun managa &
\alst{w}eros an is \alst{w}ín-gardon, \hld\ —ęndi hie im \alst{w}erk bi·falạh— &
\alst{á}dro an \alst{ú}htan. \hld\ Sum kwam þár ôk an \alst{u}ndorn tuo, &
sum kwam þár an \alst{m}iddjan dag, \hld\ \alst{m}an te þem werke, &
sum kwam þár te \alst{n}ónu, \hld\ þuȯ was þiu \alst{n}iguða tíd &
\alst{s}umar-langes dages; \hld\ sum þár ôk \alst{s}ïðor kwam &
an þia \alst{ę}lliftun tíd. \hld\ Þuȯ géng þár \alst{á}vand tuo, &
\alst{s}unna ti \alst{s}edle. \hld\ Þuȯ hie \alst{s}elvo gi·bôd &
\alst{i}s \alst{a}mbahtjon, \hld\ \alst{e}rlo drohtin, &
þat \alst{m}an þero \alst{m}anno gi·hwem \hld\ is \alst{m}eoda for·guldi, &
þem \alst{e}rlon \alst{a}rvid-lôn; \hld\ hiet þiem at \alst{ê}rist gevan. &
þia þár at \alst{l}ętst wárun, \hld\ \alst{l}iudi kumana, &
\alst{w}eros te þem \alst{w}erke, \hld\ ęndi mid is \alst{w}ordon gi·bôd, &
þat man þem \alst{m}annon iro \hld\ \alst{m}ieda for·guldi &
\alst{a}lles at \alst{a}ftan, \hld\ þem þár kwámun at \alst{ê}rist tuo &
\alst{w}illendi te þem \alst{w}erke. \hld\ \alst{W}ándun sia swíðo, &
þat man im \alst{m}êra lôn \hld\ gi·\alst{m}akod habdi &
wið iro \alst{a}rạvedje: \hld\ þan man im \alst{a}llon gaf, &
þem \alst{l}iudjon gi·\alst{l}íko. \hld\ \alst{L}êð was þat swíðo, &
\alst{a}llon þem \alst{a}ndo, \hld\ þem þár kwámun at \alst{ê}rist tuo: &
„wí kwámun hier an \alst{m}orạgan“, \hld\ kwáðun sia, „ęndi þolodun hier \alst{m}anag te dage &
\alst{a}rạvid-werko, \hld\ hwílon \alst{u}n-met hét, &
\alst{sk}ínandja sunna: \hld\ nu ni givis þú u̇s \alst{sk}attes þan mêr, &
þie þú þem \alst{ȯ}ðron duos, \hld\ þia hier \alst{ê}na hwíla &
\alst{w}áron an þínon \alst{w}erke.“ \hld\ Þuȯ habda eft is \alst{w}ord garo &
þie \alst{h}êrosto þes \alst{h}íwiskes, \hld\ kwað þat hie im ni habdi gi·\alst{h}êtan þan mêr &
\alst{w}erðes wið iro \alst{w}erke: \hld\ „Hwat ik gi·\alst{w}ald hębbju“, kwaþ-hie, &
„þat ik iu allon gi·\alst{l}íko \hld\ muot \alst{l}ôn for·geldan, &
iuwes \alst{w}erkes \alst{w}erð.“ \hld\ Þan \alst{w}aldandi Krist &
\alst{m}ênda im þoh \alst{m}éra þing, \hld\ þoh hie ovar þat \alst{m}anno folk &
fan þem \alst{w}ín-gardon só \hld\ \alst{w}ordon spráki, &
hwó þár \alst{u}n-\alst{e}fno \hld\ \alst{e}rlos kwámun, &
\alst{w}eros te þem \alst{w}erke. \hld\ Só skulun fan þero \alst{w}er-oldi duon &
\alst{m}ann-kunnjes barn \hld\ an þat \alst{m}árjo lioht, &
\alst{g}umon an \alst{g}odes wang: \hld\ sum bi·ginnit ina \alst{g}iriwan sán &
an is \alst{k}indiski, \hld\ havit im gi·\alst{k}oranan muod, &
\alst{w}illjon guodan, \hld\ \alst{w}er-old-saka míðit, &
far·\alst{l}átit is \alst{l}usta; \hld\ ni mag ina is \alst{l}ík-hamo &
an un·\alst{sp}uod for·\alst{sp}anan: \hld\ \alst{sp}áhiða línot, &
\alst{g}odes êw, \hld\ \alst{g}ramono for·látit, &
\alst{w}rêðaro \alst{w}illjon, \hld\ duot im só te is \alst{w}er-oldi forð, &
\alst{l}êstit só an þeson \alst{l}iohte, \hld\ ant-þat im is \alst{l}íves kumit, &
\alst{a}ldres \alst{á}vand; \hld\ gi·wítit im þan \alst{u}p-wegos: &
þár wirðit im is \alst{a}rạvedi \hld\ \alst{a}ll gi·lônot, &
far·\alst{g}oldan mid \alst{g}uodu \hld\ an \alst{g}odes ríkje. &
Þat mêndun þia \alst{w}urụhtjon, \hld\ þia an þem \alst{w}ín-gardon &
\alst{á}dro an \alst{ú}hta \hld\ \alst{a}rvid-líko &
\alst{w}erk bi·gunnun \hld\ ęndi þuru·\alst{w}onodun forð, &
\alst{e}rlos unt \alst{á}vand. \hld\ Sum þár ôk an \alst{u}ndern kwam, &
habda þuȯ far·\alst{m}ęrrid, \hld\ þia \alst{m}orạgan-stunda &
þes \alst{d}ag-werkes for·\alst{d}uolon; \hld\ só duot \alst{d}oloro filo, &
gi·\alst{m}êdaro \alst{m}anno: \hld\ drívit im \alst{m}is-lík þing &
\alst{g}erno an is \alst{j}uguði, \hld\ —havit im \alst{g}elp-kwidi &
\alst{l}êða gi·\alst{l}ínot \hld\ ęndi \alst{l}ôs-word manag—, &
ant-þat is \alst{k}indiski \hld\ far·\alst{k}uman wirðit, &
þat ina after is \alst{j}uguði \hld\ \alst{g}odes anst manot &
\alst{b}líði an is \alst{b}rioston; \hld\ fáhit im te \alst{b}ęteron þan &
\alst{w}ordon ęndi \alst{w}erkon, \hld\ lêdit im is \alst{w}er-old mid þiu, &
is \alst{a}ldar ant þena \alst{ę}ndi: \hld\ kumit im \alst{a}lles lôn &
an \alst{g}odes ríkje, \hld\ \alst{g}ódaro werko. &
Sum \alst{m}ann þan \alst{m}id-firi \hld\ \alst{m}ên far·látid, &
\alst{s}wára \alst{s}undjun, \hld\ fáhit im an \alst{s}álig þing, &
bi·\alst{g}innit im þuru \alst{g}odes kraft \hld\ \alst{g}uodaro werko, &
\alst{b}uotit \alst{b}alo-spráka, \hld\ látit im is \alst{b}ittrun dád &
an is \alst{h}ugje \alst{h}rewan; \hld\ kumit im þiu \alst{h}elpa fon gode, &
þat im gi·\alst{l}êstid þie gi·\alst{l}ôvo, \hld\ só lango só im is \alst{l}íf warod; &
\alst{f}arit im \alst{f}orð mid þiu, \hld\ ant·\alst{f}áhit is mieda, &
\alst{g}uod lôn at \alst{g}ode; \hld\ ni sindun êniga \alst{g}eva bęteran. &
Sum bi·ginnit þan ôk \alst{f}urðor, \hld\ þan hie ist \alst{f}ruodot mêr, &
is \alst{a}ldares af·hęldit, \hld\ —þan bi·ginnat im is \alst{u}vilon werk &
\alst{l}êðon an þeson \alst{l}iohte, \hld\ þan ina \alst{l}êra godes &
gi·\alst{m}anod an is \alst{m}uode: \hld\ wirðit im \alst{m}ildera hugi, &
þuru·\alst{g}ęngit im mid \alst{g}uodu \hld\ ęndi \alst{g}eld nimit, &
\alst{h}ôh \alst{h}imil-ríki, \hld\ þan hie \alst{h}inan węndit, &
wirðit im is \alst{m}ieda só sama, \hld\ só þem \alst{m}an *nun warð, &
þea þár te \alst{n}ónu dages, \hld\ an þea \alst{n}igunda tíd, &
an þene \alst{w}ín-gardon \hld\ \alst{w}irkjan kwámun. &
Sum wirðid þan só \alst{s}wíðo ge·fródot, \hld\ só hé ni wili is \alst{s}undja bótjen, &
ak hé \alst{ô}kid sie mid \alst{u}vilu ge·hwi-liku, \hld\ an-tat imu is \alst{á}vand náhid, &
is \alst{w}er-old ęndi is \alst{w}unnja far·slítid; \hld\ þan be·ginnid hé imu \alst{w}íti and-réden, &
is \alst{s}undjon werðad imu \alst{s}orga an móde: \hld\ ge·hugid hwat hé \alst{s}elvo ge·frumide &
\alst{g}rimmes þan lango, þe hé móste is \alst{j}uguðjo neoten; \hld\ ni mag þan mid ȯðru \alst{g}ódu gi·bótjen &
þea \alst{d}ádi, þea hé só \alst{d}ęrvja ge·frumide, \hld\ ak hé slęhit allaro \alst{d}ago ge·hwi-likes &
an is \alst{b}reost mid \alst{b}êðjun handun \hld\ ęndi wópit sie mid \alst{b}ittrun trahnun, &
\alst{h}lúdo hé sie mid \alst{h}ofnu kúmid, \hld\ bidid þene \alst{h}êlagon drohtin &
\alst{m}ahtigne, þat hé imu \alst{m}ildi werðe: \hld\ ni látid imu sïðor is \alst{m}ód gi·twífljen; &
só \alst{ê}-gróht-ful is, þe þár \alst{a}lles ge·węldid: \hld\ hé ni wili ênigumu \alst{i}rmin-manne &
far·\alst{w}ęrnjen \alst{w}illjan sínes; \hld\ far·givid imu \alst{w}aldand selvo &
\alst{h}êlag \alst{h}imil-ríki: \hld\ þan is imu gi·\alst{h}olpen sïður. &
\alst{A}lle skulun sie þár \alst{ê}ra ant·fáhen, \hld\ þoh sie þarod te \alst{ê}naru tídi &
ni \alst{k}umen, þat \alst{k}unni manno, \hld\ þoh wili imu þe \alst{k}raftigo drohtin, &
gi·\alst{l}ônon allaro \alst{l}iudjo só hwi-likumu, \hld\ só hér is gi·\alst{l}ôvon ant·fáhit: &
\alst{ê}n himil-ríki \hld\ givid hé \alst{a}llun þeodun, &
\alst{m}annun te \alst{m}édu. \hld\ Þat mênde \alst{m}ahtig Krist, &
\alst{b}arno þat \alst{b}ętste, \hld\ þȯ hé þat \alst{b}iliði sprak, &
hwó þár te þem \alst{w}ín-gardun \hld\ \alst{w}urhtjon kwámin, &
\alst{m}an \alst{m}is-líko: \hld\ þoh nam is \alst{m}éde ge·hwe &
\alst{f}ulle te is \alst{f}rôjan. \hld\ Só skulun \alst{f}iriho barn &
at \alst{g}ode selvumu \hld\ \alst{g}eld ant·fáhen, &
swíðo \alst{l}eov-lík \alst{l}ôn, \hld\ þoh sie sume só \alst{l}ate werðan.\eva

\bvb TODO.\evb\evg

\bvg\bva[43][3516]%
Hét imu þȯ þea is \alst{g}ódan \hld\ \alst{j}ungaron náhor &
\alst{t}we-livi gangan \hld\ —þea wárun imu \alst{t}riuwiston &
\alst{m}an ovar erðu—, \hld\ sagde im \alst{m}ahtig selvo &
\alst{ȯ}ðer-sïðu, \hld\ hwi-lik imu þár \alst{a}rvedi &
\alst{t}ó-ward wárun: \hld\ „þes ni mag ênig \alst{t}weho werðen“, kwað hé; &
kwað þat sie þȯ te \alst{J}erusalem \hld\ an þat \alst{J}udeono folk &
\alst{l}íðan skoldin: \hld\ „þár wirðid all gi·\alst{l}êstid só, &
ge·\alst{f}rumid undar þemu \alst{f}olke, \hld\ só it an \alst{f}urn-dagun &
\alst{w}íse man be mí \hld\ \alst{w}ordun ge·sprákun. &
Þár skulun mi far·\alst{k}ôpon \hld\ undar þea \alst{k}raftigon þiod, &
\alst{h}ęliðos te þeru \alst{h}êri; \hld\ þár werðat mína \alst{h}ęndi ge·bundana, &
\alst{f}aðmos werðad mi þár ge·\alst{f}astnod; \hld\ \alst{f}ilu skal ik þár gi·þolojan, &
\alst{h}oskes gi·\alst{h}ôrjen \hld\ ęndi \alst{h}arm-kwidi, &
\alst{b}ismer-spráka \hld\ ęndi \alst{b}i·hêt-word manag; &
sie \alst{w}êgjat mi te wundron \hld\ \alst{w}ápnes ęggjun, &
bi·\alst{l}ôsjad mi \alst{l}ívu: \hld\ ik te þesumu \alst{l}iohte skal &
þurh \alst{d}rohtines kraft \hld\ fan \alst{d}ôðe a·standen &
an \alst{þ}riddjon dage. \hld\ Ni kwam ik undar þesa \alst{þ}eoda herod &
te þiu, þat mín \alst{ę}ldi-barn \hld\ \alst{a}rved habdin, &
þat mi \alst{þ}ionodi þius \alst{þ}iod: \hld\ ni willju ik is sie \alst{þ}iggjen nu, &
\alst{f}ergon þit \alst{f}olk-skępi, \hld\ ak ik skal imu te \alst{f}rumu werðen, &
\alst{þ}eonon imu \alst{þ}eo-líko \hld\ ęndi for alla þesa \alst{þ}eoda geven &
\alst{s}eole míne. \hld\ Ik willju sie \alst{s}elvo nu &
\alst{l}ôsjen mid mínu \alst{l}ívu, \hld\ þea hér \alst{l}ango bidun, &
\alst{m}an-kunnjes \alst{m}anag, \hld\ \alst{m}ínara helpa.“ &
\alst{F}ór imu þȯ \alst{f}orð-wardes \hld\ —habde imu \alst{f}asten hugi, &
\alst{b}líðjan an is \alst{b}reostun \hld\ \alst{b}arn drohtines— &
welda im te \alst{J}erusalem \hld\ \alst{J}udeo folkes &
\alst{w}illjon \alst{w}ísan: \hld\ hé konste þes \alst{w}erodes só garo &
\alst{h}ęti-grimmen \alst{h}ugi \hld\ ęndi \alst{h}ardan stríd, &
\alst{w}rêðan \alst{w}illjon. \hld\ \alst{W}erod sïðode &
furi \alst{J}erikho-burg; \hld\ was þe \alst{g}odes sunu, &
\alst{m}ahtig undar þero \alst{m}ęnigi. \hld\ Þár sátun twênje \alst{m}an bi wege, &
\alst{b}linde wárun sie \alst{b}êðje: \hld\ was im \alst{b}ótono þarf, &
þat sie ge·\alst{h}êldi \hld\ \alst{h}evanes waldand, &
hwand sie só \alst{l}ango \hld\ \alst{l}iohtes þolodun, &
\alst{m}anaga hwíla. \hld\ Sie gi·hôrdun þȯ þat \alst{m}ęgin faren &
ęndi \alst{f}rágodun sán \hld\ \alst{f}iri-wit-líko &
\alst{r}ęgini-blindun, \hld\ hwi-lik þár \alst{r}íki man &
undar þemu \alst{f}olk-skępi \hld\ \alst{f}urista wári, &
\alst{h}êrost an \alst{h}ôvid. \hld\ Þȯ sprak im ên \alst{h}ęlið an·gęgin, &
kwað þat þár \alst{J}esu Krist \hld\ fan \alst{G}alilea-lande, &
\alst{h}êljandero bętst \hld\ \alst{h}êrost wári, &
\alst{f}óri mid is \alst{f}olku. \hld\ Þȯ warð \alst{f}ráh-mód hugi &
\alst{b}êðjun þem \alst{b}lindun mannun, \hld\ þȯ sie þat \alst{b}arn godes &
\alst{w}issun under þemu \alst{w}erode: \hld\ hreopun im þȯ mid iro \alst{w}ordun tó, &
\alst{h}lúdo te þemu \alst{h}êlagon Kriste, \hld\ bádun þat hé im \alst{h}elpe ge·rédi: &
„\alst{d}rohtin \alst{D}awides sunu: \hld\ wis u̇s mid þínun \alst{d}ádjun mildi, &
\alst{n}ęri u̇s af þesaru \alst{n}ôdi, \hld\ só þú gi·\alst{n}óge dós &
\alst{m}anno kunnjes: \hld\ þú bist \alst{m}anagun gód, &
\alst{h}ilpis ęndi \alst{h}êlis.“ \hld\ Þo bi·gan im þat \alst{h}ęliðo folk &
\alst{w}ęrjen mid \alst{w}ordun, \hld\ þat sie an \alst{w}aldand Krist &
só \alst{h}lúdo ni \alst{h}riopin. \hld\ Si ni weldun im \alst{h}ôrjen te þiu, &
ak sie simla \alst{m}êr ęndi \alst{m}êr \hld\ ovar þat \alst{m}anno folk &
\alst{h}lúdo \alst{h}reopun. \hld\ \alst{H}éljand ge·stód, &
allaro \alst{b}arno \alst{b}ętst, \hld\ hét sie þȯ \alst{b}rengjen te imu, &
\alst{l}êdjen þurh þea \alst{l}iudi, \hld\ sprak im \alst{l}istjun tó &
\alst{m}ild-líko for þeru \alst{m}ęnegi: \hld\ „hwat willjad git \alst{m}ínaro hér“, kwað hé, &
„\alst{h}elpono \alst{h}abbjen?“ \hld\ Sie bádun ina \alst{h}êlagna, &
þat hé im ira \alst{ô}gon \hld\ \alst{o}pana gi·dádi, &
far·\alst{l}iwi þeses \alst{l}iohtes, \hld\ þat sie \alst{l}iudjo drôm, &
\alst{s}wigle \alst{s}unnun skín \hld\ gi·\alst{s}ehen móstin, &
\alst{w}liti-skônje \alst{w}er-old. \hld\ \alst{W}aldand frumide, &
\alst{h}rên sie þȯ mid is \alst{h}andun, \hld\ dede is \alst{h}elpe þár tó, &
þat þem \alst{b}lindun þȯ \hld\ \alst{b}êðjum wurðun &
\alst{ô}gon gi·\alst{o}ponod, \hld\ þat sie \alst{e}rðe ęndi himil &
þurh \alst{k}raft godes \hld\ ant·\alst{k}iennjen mahtun, &
\alst{l}ioht ęndi \alst{l}iudi. \hld\ Þȯ sagdun sie \alst{l}of gode, &
\alst{d}iurdun u̇san \alst{d}rohtin, \hld\ þes sie \alst{d}ages liohtes &
\alst{b}rúkan móstun: \hld\ ge·witun im \alst{b}êðje mid imu, &
\alst{f}olgodun is \alst{f}ęrdi: \hld\ was im þiu \alst{f}ruma giviðig, &
ęndi ôk \alst{w}aldandes \alst{w}erk \hld\ \alst{w}ído ge·ku̇ðid, &
\alst{m}anagun gi·\alst{m}árid.\eva

\bvb TODO.\evb\evg

\bvg\bva[44][3588]%
\hspace*{100pt}Þár was só \alst{m}ahtig-lík &
\alst{b}iliði gi·\alst{b}ôknid, \hld\ þár þe \alst{b}lindon man &
bi þemu \alst{w}ege sátun, \hld\ \alst{w}íti þolodun, &
\alst{l}iohtes \alst{l}ôse: \hld\ þat mênid þoh \alst{l}iudjo barn, &
al \alst{m}an-kunni, \hld\ hwó sie \alst{m}ahtig god &
an þemu \alst{a}na·ginne \hld\ þurh is \alst{ê}nes kraft &
\alst{s}in-híun twê \hld\ \alst{s}elvo gi·warhte, &
\alst{Á}dam ęndi \alst{É}wan: \hld\ far·gaf im \alst{u}p-wegos, &
\alst{h}imilo ríki; \hld\ ak þȯ warð im þe \alst{h}atola te náh, &
\alst{f}íund mid \alst{f}êknu \hld\ ęndi mid \alst{f}irin-werkun, &
bi·\alst{s}wêk sie mid \alst{s}undjun, \hld\ þat sie \alst{s}in-skôni, &
\alst{l}ioht far·\alst{l}étun: \hld\ wurðun an \alst{l}êðaron stędi, &
an þesen \alst{m}iddil-gard \hld\ \alst{m}an far·worpen, &
\alst{þ}olodun hér an \alst{þ}iustrju \hld\ \alst{þ}iod-arvedi, &
\alst{w}unnun \alst{w}rak-sïðos, \hld\ \alst{w}elon þarvodun: &
far·\alst{g}átun \alst{g}odes ríkjes, \hld\ \alst{g}ramon þeonodun, &
\alst{f}íundo barnun; \hld\ sie guldun is im mid \alst{f}iuru lôn &
an þeru \alst{h}êton \alst{h}ęllju. \hld\ Be·þiu wárun siu an iro \alst{h}ugi blinda &
an þesaru \alst{m}iddil-gard, \hld\ \alst{m}ęnniskono barn, &
hwand siu ine ni ant·\alst{k}iendun, \hld\ \alst{k}raftagne god, &
\alst{h}imilisken \alst{h}êrron, \hld\ þene þe sie mid is \alst{h}andun gi·skóp, &
gi·\alst{w}arhte an is \alst{w}illjon. \hld\ Þius \alst{w}er-old was þȯ só far·hwervid, &
bi·\alst{þ}wungen an \alst{þ}iustrje, \hld\ an \alst{þ}iod-arvidi, &
an \alst{d}ôðes \alst{d}alu: \hld\ sátun im þȯ bi þeru \alst{d}rohtines strátun &
\alst{j}ámar-móde, \hld\ \alst{g}odes helpe bidun: &
siu ni mahte im þȯ êr \alst{w}erðen, \hld\ êr þan \alst{w}aldand god &
an þesan \alst{m}iddil-gard, \hld\ \alst{m}ahtig drohtin, &
is \alst{s}elves \alst{s}unu \hld\ \alst{s}ęndjen weldi &
þat hé \alst{l}ioht ant·\alst{l}uki \hld\ \alst{l}iudjo barnun, &
\alst{o}ponodi im \alst{ê}wig líf, \hld\ þat sie þene \alst{a}lo-waldon &
mahtin ant·\alst{k}ęnnjen wel, \hld\ \alst{k}raftagna god. &
Ôk mag ik giu gi·\alst{t}ęlljen, \hld\ of gí þár \alst{t}ó willjad &
\alst{h}uggjen ęndi \alst{h}ôrjen, \hld\ þat gí þes \alst{h}êljandes mugun &
\alst{k}raft ant·\alst{k}ęnnjen, \hld\ hwó is \alst{k}umi wurðun &
an þesaru \alst{m}iddil-gard \hld\ \alst{m}anagun te helpu, &
ia hwat hé mid þem \alst{d}ádjun \hld\ \alst{d}rohtin selvo &
\alst{m}anages \alst{m}ênde, \hld\ ia be·hwiu þiu \alst{m}árje burg &
\alst{J}erikho hêtid, \hld\ þiu þár an \alst{J}udeon stád &
gi·\alst{m}akod mid \alst{m}úrun: \hld\ þiu is aftar þemu \alst{m}ánen gi·nęmnid, &
aftar þemu \alst{t}orhten \alst{t}ungle: \hld\ hé ni mag is \alst{t}ídi be·míðen, &
ak hé \alst{d}ago ge·hwi-likes \hld\ \alst{d}uod ȯðer-hweðer, &
\alst{w}anod ohþo \alst{w}ahsid. \hld\ Só dód an þesaro \alst{w}er-oldi hér, &%TODO: check ohþo
an þesaru \alst{m}iddil-gard \hld\ \alst{m}ęnniskono barn: &
\alst{f}arad ęndi \alst{f}olgod, \hld\ \alst{f}róde stervad, &
werðad \alst{e}ft junga \hld\ \alst{a}ftar kumane, &
\alst{w}eros a·\alst{w}ahsane, \hld\ unt-tat sie eft \alst{w}urd far·nimid. &
Þat mênde þat \alst{b}arn godes, \hld\ þȯ hé fon þeru \alst{b}urgi fór, &
þe \alst{g}ódo fan \alst{J}erikho, \hld\ þat ni mahte êr werðen \alst{g}umono barnun &
þiu \alst{b}lindja gi·\alst{b}ótid, \hld\ þat sie þat \alst{b}erhte lioht, &
gi·\alst{s}áhin \alst{s}in-skôni, \hld\ êr þan hé \alst{s}elvo hér &
an þesaru \alst{m}iddil-gard \hld\ \alst{m}ęnniski ant·féng, &
\alst{f}lêsk ęndi lík-hamon. \hld\ Þȯ wurðun þes \alst{f}iriho barn &
gi·\alst{w}ar an þesaru \alst{w}er-oldi, \hld\ þe hér an \alst{w}ítje êr, &
\alst{s}átun an \alst{s}undjun \hld\ gi·\alst{s}iunjes lôse, &
\alst{þ}olodun an \alst{þ}iustrje, \hld\ —sie af·sóvun þat was þesaru \alst{þ}iod kuman &
\alst{h}êljand te \alst{h}elpu \hld\ fan \alst{h}evan-ríkje, &
\alst{K}rist allaro \alst{k}uningo bęst; \hld\ sie mahtun is ant·\alst{k}ęnnjen sán, &
gi·\alst{f}óljen is \alst{f}ardjo. \hld\ Þȯ sie só \alst{f}ilu hriopun, &
þe \alst{m}an te þemu \alst{m}ahtigon gode, \hld\ þat im \alst{m}ildi aftar þiu &
\alst{w}aldand \alst{w}urði. \hld\ Þan \alst{w}ęridun im swíðo &
þia \alst{s}wárun \alst{s}undjon, \hld\ þe sie im êr \alst{s}elvon gi·dádun, &
\alst{l}ettun sie þes gi·\alst{l}ôbon. \hld\ Sie ni mahtun þem \alst{l}iudjun þoh &%TODO: check "lettun"
bi·\alst{w}ęrjen iro \alst{w}illjon, \hld\ ak sie an \alst{w}aldand god &
\alst{h}lúdo \alst{h}riopun, \hld\ an-tat hé im iro \alst{h}êli far·gaf, &
þat sie \alst{s}in-líf \hld\ gi·\alst{s}ehen móstin, &
\alst{o}pen \alst{ê}wig lioht \hld\ ęndi \alst{a}n faren &
an þiu \alst{b}erhtun \alst{b}ú. \hld\ Þat mêndun þea \alst{b}lindun man, &
þe þár bi \alst{J}erikho-burg \hld\ te þemu \alst{g}odes barne &
\alst{h}lúdo \alst{h}riopun, \hld\ þat hé im iro \alst{h}êli far·lihi, &
\alst{l}iohtes an þesumu \alst{l}íve: \hld\ þan im þea \alst{l}iudi só filu &
\alst{w}ęridun mid \alst{w}ordun, \hld\ þea þár an þemu \alst{w}ege fórun &
bi·\alst{f}oren ęndi bi·hinden: \hld\ só dót þea \alst{f}irin-sundjon &
an þesaru \alst{m}iddil-gard \hld\ \alst{m}an-kunnje. &
hôrjad nu hwó þie \alst{b}lindun, \hld\ sïður im gi·\alst{b}ótid warð, &
þat sie \alst{s}unnun lioht \hld\ ge·\alst{s}ehen móstun, &
hwó si þȯ \alst{d}ádun: \hld\ ge·witun im mid iro \alst{d}rohtine samad, &
\alst{f}olgodun is \alst{f}ęrdi, \hld\ sprákun \alst{f}ilu wordo &
þemu \alst{l}andes hirdje te \alst{l}ove: \hld\ só dód im noh \alst{l}iudjo barn &
\alst{w}ído aftar þesaru \alst{w}er-oldi, \hld\ sïður im \alst{w}aldand Krist &
ge·\alst{l}iuhte mid is \alst{l}êrun \hld\ ęndi im \alst{l}íf êwig, &
\alst{g}odes ríki far·\alst{g}af \hld\ \alst{g}ódun mannun, &
\alst{h}ôh \alst{h}imiles lioht \hld\ ęndi is \alst{h}elpe þár tó, &
só hwemu só þat gi·\alst{w}erkod, \hld\ þat hé móti þemu is \alst{w}ege folgon.\eva

\bvb TODO.\evb\evg

\bvg\bva[45][3671]%
Þȯ \alst{n}áhide \hld\ \alst{n}ęrjendo Krist, &
þe \alst{g}ódo te \alst{J}erusalem. \hld\ Kwam imu þár te·\alst{g}ęgnes filu &
\alst{w}erodes an \alst{w}illjon \hld\ \alst{w}el huggendjes, &
ant·\alst{f}éngun ina \alst{f}agạro \hld\ ęndi imu bi·\alst{f}oren stręidun &%NOTE: ęi is original.
þene \alst{w}eg mid iro gi·\alst{w}ádjun \hld\ ęndi mid \alst{w}urtjun só same, &
mid \alst{b}erhtun \alst{b}lómun \hld\ ęndi mid \alst{b}ômo tógun, &
þat \alst{f}eld mid \alst{f}agạron palmun, \hld\ al só is \alst{f}ard ge·buride, &
þat þe \alst{g}odes sunu \hld\ \alst{g}angan welde &
te þeru \alst{m}árjan burg. \hld\ Hwarf ina \alst{m}ęgin umbi &
\alst{l}iudjo an \alst{l}ustun, \hld\ ęndi \alst{l}of-sang a·hóf &
þat \alst{w}erod an \alst{w}illjon: \hld\ sagdun \alst{w}aldande þank, &
þes þár \alst{s}elvo kwam \hld\ \alst{s}unu Dawides &
\alst{w}íson þes \alst{w}erodes. \hld\ Þȯ ge·sah \alst{w}aldand Krist &
þe \alst{g}ódo te \alst{J}erusalem, \hld\ \alst{g}umono bętsta, &
\alst{b}líkan þene \alst{b}urges wal \hld\ ęndi \alst{b}ú Judeono, &
\alst{h}ôha \alst{h}orn-sęli \hld\ ęndi ôk þat \alst{h}ús godes, &
allaro \alst{w}ího \alst{w}un-samost. \hld\ Þȯ \alst{w}el imu an innen &
\alst{h}ugi wið is \alst{h}erte: \hld\ þȯ ni mahte þat \alst{h}êlage barn &
\alst{w}ópu a·\alst{w}ísjen, \hld\ sprak þȯ \alst{w}ordo filu &
\alst{h}riwig-líko \hld\ —was imu is \alst{h}ugi sêreg—: &
„\alst{w}ê warð þí, Jerusalem“, \hld\ kwað hé, „þes þú te \alst{w}árun ni wêst &
þea \alst{w}urde-gi·skęfti, \hld\ þe þí noh gi·\alst{w}erðen skulun, &
\alst{h}wó þú noh wirðis be·\alst{h}abd \hld\ \alst{h}ęrjes kraftu &
ęndi þí bi·\alst{s}ittjad \hld\ \alst{s}líð-móde man, &
\alst{f}íund mid \alst{f}olkun. \hld\ Þan ni havas þú \alst{f}riðu hwęrgin, &
\alst{m}und-burd mid \alst{m}annun: \hld\ lêdjad þi hér \alst{m}anage tó &
\alst{o}rdos ęndi \alst{ę}ggja, \hld\ \alst{o}r-legas word, &
far·\alst{f}ioþ þín \alst{f}olk-skępi \hld\ \alst{f}iures liomon, &
þese \alst{w}íki a·\alst{w}óstjad, \hld\ \alst{w}allos hôha &
\alst{f}ęlljad te \alst{f}oldun: \hld\ ni af·stád is \alst{f}elis nígijan, &
\alst{st}ên ovar ȯðrumu, \hld\ ak werðad þesa \alst{st}ędi wóstja &
umbi \alst{J}erusalem \hld\ \alst{J}udeo liudjo, &
hwand sie ni ant·\alst{k}ęnnjad, \hld\ þat im \alst{k}umana sind &
iro \alst{t}ídi \alst{t}ó-wardes, \hld\ ak sie habbjad im \alst{t}wífljen hugi, &
ni \alst{w}itun þat iro \alst{w}ísad \hld\ \alst{w}aldandes kraft.“ &
Gi·wêt imu þȯ mid þeru \alst{m}ęnegi \hld\ \alst{m}anno drohtin &
an þea \alst{b}erhton \alst{b}urg. \hld\ Só þȯ þat \alst{b}arn godes &
innan \alst{J}erusalem \hld\ mid þiu \alst{g}umono folku, &
\alst{s}êg mid þiu ge·\alst{s}ïðu, \hld\ þȯ warð þár allaro \alst{s}ango mêst, &
\alst{h}lúd stemnje af·\alst{h}aven \hld\ \alst{h}êlagun wordun, &
\alst{l}ovodun þene \alst{l}andes ward \hld\ \alst{l}iudjo męnegi, &
\alst{b}arno þat \alst{b}ętste; \hld\ þiu \alst{b}urg warð an hróru, &
þat \alst{f}olk warð an \alst{f}orhtun \hld\ ęndi \alst{f}rágodun sán, &
hwe þat \alst{w}ári, \hld\ þat þár mid þiu \alst{w}erodu kwam, &
mid þeru \alst{m}ikilon \alst{m}ęnegi. \hld\ Þȯ sprak im ên \alst{m}an an·gęgin, &
kwað þat þár \alst{J}esu Krist \hld\ fan \alst{G}alileo lande, &
fan \alst{N}azareth-burg \hld\ \alst{n}ęrjand kwámi, &
\alst{w}itig \alst{w}ár-sago \hld\ þemu \alst{w}erode te helpu. &
Þȯ was þem \alst{J}udiun, \hld\ þe imu êr \alst{g}rame wárun, &
un·\alst{h}olde an \alst{h}ugi, \hld\ \alst{h}arm an móde, &
þat imu þea \alst{l}iudi só filu \hld\ \alst{l}of-sang warhtun, &
\alst{d}iurdun iro drohtin. \hld\ Þȯ géngun \alst{d}ol-móde, &
þat sie wið \alst{w}aldand Krist \hld\ \alst{w}ordun sprákun, &
bádun þat hé þat ge·\alst{s}ïði \hld\ \alst{s}wígon héti, &
\alst{l}etti þea \alst{l}iudi, \hld\ þat sie imu \alst{l}of só filu &
\alst{w}ordun ni \alst{w}arhtin: \hld\ „it is þesumu \alst{w}erode lêð“, kwáðun sie, &
„þesun \alst{b}urg-liudjun.“ \hld\ Þȯ sprak eft þat \alst{b}arn godes: &
„ef gi sie a·\alst{m}ęrrjad“, \hld\ kwað hé, „þat hér ni mótin \alst{m}anno barn &
\alst{w}aldandes kraft \hld\ \alst{w}ordun diurjen, &
þan skulun it \alst{h}rópen þoh \hld\ \alst{h}arde stênos &
for þesumu \alst{f}olk-skępi, \hld\ \alst{f}elisos starka, &
êr þan it eo be·\alst{l}íve, \hld\ nevo man is \alst{l}of spreke &
\alst{w}ído aftar þesaru \alst{w}er-oldi.“ \hld\ Þȯ hé an þene \alst{w}íh innen, &
\alst{g}éng an þat \alst{g}odes hús: \hld\ fand þár \alst{J}udeono filu, &
\alst{m}is-líke \alst{m}an, \hld\ \alst{m}anage at·samne, &
þea im þár \alst{k}ôp-stędi \hld\ gi·\alst{k}oran habdun, &
\alst{m}angodun im þár mid \alst{m}anages hwí: \hld\ \alst{m}unitęrjas sátun &
an þemu \alst{w}íhe innan, \hld\ habdun iro \alst{w}esl gi·dago &
\alst{g}aru te \alst{g}evanne. \hld\ Þat was þemu \alst{g}odes barne &
\alst{a}l an \alst{a}ndun: \hld\ drêf sie \alst{ú}t þanen &
\alst{r}úmo fan þemu \alst{r}akude, \hld\ kwað þat wári \alst{r}ehtara dád, &
þat þár te \alst{b}edu fórin \hld\ \alst{b}arn Israheles &%TODO: check bedu
„ęndi an þesumu mínumu \alst{h}úse \hld\ \alst{h}elpono biddjan, &
þat sia \alst{s}igi-drohtin \hld\ \alst{s}undjono tuomje, &
þan hér \alst{þ}eovas \hld\ an \alst{þ}ing-stędi halden, &
þea far·\alst{w}arhton \alst{w}eros \hld\ \alst{w}ehsal drívan, &
\alst{u}n-reht \alst{ê}n-fald. \hld\ Ne gi êniga \alst{ê}ra ni witun &
þeses \alst{g}odes húses, \hld\ \alst{J}udeo liudi.“ &
Só \alst{r}úmde hé þȯ ęndi \alst{r}ekode, \hld\ \alst{r}íki drohtin, &
þat \alst{h}êlaga hús \hld\ ęndi an \alst{h}elpun was &
\alst{m}anagumu \alst{m}an-kunnje, \hld\ þem þe is \alst{m}ikilon kraft &
\alst{f}errene ge·\alst{f}rugnun \hld\ ęndi þár gi·\alst{f}aran kwámun &
ovar \alst{l}angan weg. \hld\ Warð þár \alst{l}éf so manag, &
\alst{h}alt gi·\alst{h}êlid \hld\ ęndi \alst{h}áf só same, &
\alst{b}lindun gi·\alst{b}ótid. \hld\ Só dede þat \alst{b}arn godes &
\alst{w}illjendi þemu \alst{w}erode, \hld\ hwand al an is gi·\alst{w}ęldi stéd &
umbi þesaro \alst{l}iudjo \alst{l}íf \hld\ ęndi ôk umbi þit \alst{l}and só same.\eva

\bvb TODO.\evb\evg

\bvg\bva[46][3758]%
Stód imu þȯ fora þemu \alst{w}íhe \hld\ \alst{w}aldandjo Krist, &
\alst{l}iof \alst{l}andes ward, \hld\ ęndi imu þero \alst{l}iudjo hugi, &
iro \alst{w}illjon aftar·\alst{w}arode: \hld\ gi·sah \alst{w}erod mikil &
an þat \alst{m}árje hús \hld\ \alst{m}êðmos fórjen, &
\alst{g}evon mid \alst{g}oldu \hld\ ęndi mid \alst{g}odu-wębbju, &
\alst{d}iurjun fratahun. \hld\ Þat al \alst{d}rohtin Krist &
\alst{w}arode \alst{w}ís-líko. \hld\ Þȯ kwam þár ôk ên \alst{w}idowa tó, &
\alst{i}dis \alst{a}rm-skapen, \hld\ ęndi te þemu \alst{a}lạha géng &
ęndi siu an þat \alst{t}resur-hús \hld\ \alst{t}wêne lęgde &
\alst{ê}ríne skattos: \hld\ was iru \alst{ê}n-fald hugi, &
\alst{w}illjan gódes. \hld\ Þȯ sprak \alst{w}aldand Krist, &
þe \alst{g}umo wið is \alst{j}ungaron, \hld\ kwað þat siu þár \alst{g}eva brȧhti &
\alst{m}êron \alst{m}ikilu þan ęlkor \hld\ ênig \alst{m}annes sunu: &
„ef hér \alst{ô}daga man“, \hld\ kwað hé, „\alst{ê}ra brȧhtun, &
\alst{m}êðọm-hord \alst{m}anag, \hld\ sie létun im \alst{m}êr at hús &
\alst{w}elona ge·\alst{w}unnen. \hld\ Ni dede þius \alst{w}idowa só, &
ak siu te þesumu \alst{a}lạhe gaf \hld\ \alst{a}l þat siu habde &
\alst{w}elono ge·\alst{w}unnen, \hld\ só siu iru \alst{w}iht ni far·lét &
\alst{g}ódes an iro \alst{g}ardun. \hld\ Be·þiu sind ira \alst{g}eva mêron, &
\alst{w}aldande \alst{w}erða, \hld\ hwand siu it mid su·likumu \alst{w}illjon dede &
te þesumu \alst{g}odes húse. \hld\ Þes skal siu \alst{g}eld niman, &
swíðo \alst{l}ang-sam \alst{l}ôn, \hld\ þes siu su·likan gi·\alst{l}ôvon havad.“ &
Só gi·fragn ik þat þár an þemu \alst{w}íhe \hld\ \alst{w}aldandjo Krist &
allaro \alst{d}ago ge·hwi-likes, \hld\ \alst{d}rohtin manno, &
\alst{w}ísde mid \alst{w}ordun. \hld\ Stód ine \alst{w}erod umbi, &
\alst{g}rôt folk \alst{J}udeono, \hld\ gi·hôrdun is \alst{g}ódan word, &
\alst{s}wótja \alst{s}ęggjan. \hld\ Sum só \alst{s}álig warð &
\alst{m}anno undar þeru \alst{m}ęnegi, \hld\ þat it bi·gan an is \alst{m}ód hladen; &
\alst{l}ínodun im þea \alst{l}êra, \hld\ þe þe \alst{l}andes ward &
al be \alst{b}iliðjun sprak, \hld\ \alst{b}arn drohtines. &
Sumun wárun eft so \alst{l}êða \hld\ \alst{l}êra Kristes, &
\alst{w}aldandes \alst{w}ord: \hld\ was im \alst{w}iðer-mód hugi &
allun þem, þe an þemu \alst{h}ęri-skępi \hld\ \alst{h}êrost wárun, &
\alst{f}uriston an þemu \alst{f}olke: \hld\ \alst{f}áres hugdun &
\alst{w}rêða mid iro \alst{w}ordun \hld\ —habdun im \alst{w}iðer-sakon &
gi·\alst{h}aloden te \alst{h}elpu, \hld\ þes \alst{h}êroston man, &
\alst{E}rodeses þegạn, \hld\ þe þár \alst{a}nd-ward stód &
\alst{w}rêðes \alst{w}illjan, \hld\ þat hé iro \alst{w}ord ovar-hôrdi— &
ef sie ina for·\alst{f}éngin, \hld\ þat sie ina þan \alst{f}eteros an, &
þea \alst{l}iudi \alst{l}iðo-bęndi \hld\ \alst{l}ęggjen móstin, &
\alst{s}undja lôsan. \hld\ Þȯ géngun im þea ge·\alst{s}ïðos tó &
\alst{b}ittra gi·hugde, \hld\ þat sie wið þat \alst{b}arn godes, &
\alst{w}rêða \alst{w}iðer-sakon \hld\ \alst{w}ordun sprákun: &
„Hwat þú bist \alst{ê}o-sago“, \hld\ kwáðun sie, „\alst{a}llun þiodun, &
\alst{w}ísis wáres só \alst{f}ilu: \hld\ nis þi \alst{w}erð eo·wiht &
te bi·\alst{m}íðanne \hld\ \alst{m}anno ni-ênumu &
umbi is \alst{r}íki-dóm, \hld\ nevo þú simlun þat \alst{r}eht sprikis &
ęndi an þene \alst{g}odes weg \hld\ \alst{g}umono ge·sïði &
\alst{l}êdis mid þinun \alst{l}êrun: \hld\ ni mag þi \alst{l}aster man &
\alst{f}ïðan undar þesumu \alst{f}olke. \hld\ Nu wí þi \alst{f}rágon skulun. &
\alst{r}íki þiodan, \hld\ hwi-lik \alst{r}eht havad &
þe \alst{k}êsur fan Rúmu, \hld\ þe imu te þesumu \alst{k}unnje herod &
\alst{t}insi sókid \hld\ ęndi gi·\alst{t}ald havad, &
hwat wí imu \alst{g}elden skulin \hld\ \alst{g}ę́ro ge·hwi-likes &
\alst{h}ôvid-skatto. \hld\ Saga hwat þi þes an þínumu \alst{h}ugi þunkja: &
is it \alst{r}eht þe nis? \hld\ \alst{R}ád for þínun &
\alst{l}and-mę́gun wel: \hld\ u̇s is þínaro \alst{l}êrono þarf.“ &
Sie weldun þat hé it ant·\alst{k}wáði: \hld\ þan mahte hé þoh ant·\alst{k}ęnnjen wel &
iro \alst{w}rêðon \alst{w}illjon: \hld\ „te hwí gi \alst{w}ár-logon“, kwað hé, &
„\alst{f}andot mín só \alst{f}rókno? \hld\ Ni skal iu þat te \alst{f}rumu werðen, &
þat gi \alst{d}reogerjas \hld\ \alst{d}arnungo nu &
willjad mi far·\alst{f}áhen.“ \hld\ Hét hé þȯ \alst{f}orð dragan &
te \alst{sk}awonne þe \alst{sk}attos, \hld\ „þe gí \alst{sk}uldige sind &
an þat \alst{g}eld \alst{g}even.“ \hld\ \alst{J}udeon drógun &
ênna \alst{s}ilụvrinna forð: \hld\ \alst{s}áhun manage tó, &
hwó hé was ge·\alst{m}unitod: \hld\ was an \alst{m}iddjen skín &
þes \alst{k}êsures biliði \hld\ —þat mahtun sie ant·\alst{k}ęnnjen wel—, &
iro \alst{h}êrron \alst{h}ôvid-mál. \hld\ Þȯ frágode sie þe \alst{h}êlago Krist, &
aftar hwemu þiu ge·\alst{l}ík-nessi \hld\ gi·\alst{l}egid wári. &
Sie kwáðun þat it \alst{w}ári \hld\ \alst{w}er-old-kêsures &
fan \alst{R}úmu-burg, \hld\ „þes þe alles þeses \alst{r}íkes havad &
ge·\alst{w}ald an þesaru \alst{w}er-oldi.“ \hld\ „Þan willju ik iu te \alst{w}árun hér“, kwað hé, &
„\alst{s}elvo \alst{s}ęggjan, \hld\ þat gí imu \alst{s}ín gevad, &
\alst{w}er-old-hêrron is ge·\alst{w}unst, \hld\ ęndi \alst{w}aldand gode &
\alst{s}ęlljad, þat þár \alst{s}ín ist: \hld\ þat skulun iuwa \alst{s}eolon wesen, &
\alst{g}umono \alst{g}êstos.“ \hld\ Þȯ warð þero \alst{J}udeono hugi &
ge·\alst{m}insod an þemu \alst{m}ahle: \hld\ ni mahtun þe \alst{m}ên-skaðon &
\alst{w}ordun ge·\alst{w}innen, \hld\ só iro \alst{w}illjo géng, &
þat sie ina far·\alst{f}éngin, \hld\ hwand imu þat \alst{f}riðu-barn godes &
\alst{w}ardode wið þe \alst{w}rêðon \hld\ ęndi im \alst{w}ár an·gęgin, &
\alst{s}ȯð-spel \alst{s}agde, \hld\ þoh sie ni wárin só \alst{s}álige te þiu, &
þat sie it só far·\alst{f}éngin, \hld\ só it iro \alst{f}ruma wári.\eva

\bvb TODO.\evb\evg

\bvg\bva[47][3840]%
Sie ni weldun it þoh far·\alst{l}áten, \hld\ ak hétun þár \alst{l}êdjen forð &
ên \alst{w}íf for þemu \alst{w}erode, \hld\ þiu habde \alst{w}am ge·frumid, &
\alst{u}n-reht \alst{ê}n-fald: \hld\ þiu \alst{i}dis was bi·fangen &
an far·\alst{l}egar-nessi, \hld\ was iro \alst{l}íves skolo, &
þat sie \alst{f}iriho barn \hld\ \alst{f}erạhu bi·námin, &
\alst{ê}htin iro \alst{a}ldres: \hld\ só was an iro \alst{ê}w ge·skriven. &
Sie bi·gunnun ina þȯ \alst{f}rágon, \hld\ \alst{f}ruokne liudi, &
\alst{w}rêða mid iro \alst{w}ordun, \hld\ hwat sie skoldin þemu \alst{w}íve duan, &
hweðer sie sie \alst{k}węlidin, \hld\ þe sie sie \alst{k}wika létin, &
þe hwat hé umbi su·lika \alst{d}ádi \hld\ a·\alst{d}êljen weldi: &
„þú wêst, hwó þesaru \alst{m}ęnegi“, \hld\ kwáðun sie, „\alst{M}oyses gi·bôd &
\alst{w}árun \alst{w}ordun, \hld\ þat allaro \alst{w}ívo ge·hwi-lik &
an far·\alst{l}egar-nessi \hld\ \alst{l}íves far·warhti &
ęndi þat sie þan a·\alst{w}urpin \hld\ \alst{w}eros mid handun, &
\alst{st}arkun \alst{st}ênun: \hld\ nu maht þú sie sehan \alst{st}anden hér &
an \alst{s}undjun bi·fangan: \hld\ \alst{s}aga hwat þú is willjes.“ &
\alst{w}eldun ine þea \alst{w}iðer-sakon \hld\ \alst{w}ordun far·fáhen, &
ef hé þat gi·\alst{k}wáði, \hld\ þat sie sie \alst{k}wika létin, &
\alst{f}riðodi ira \alst{f}erạhe, \hld\ þan weldi þat \alst{f}olk Judeono &
kweðen, þat hé iro \alst{a}ldiron \hld\ \alst{ê}o wiðer-sagdi, &
þero \alst{l}iudjo \alst{l}and-reht; \hld\ ef hé sie þan héti \alst{l}ívu bi·nimen, &
þea \alst{m}agað fur þeru \alst{m}ęnegi, \hld\ þan weldin sie kweðen, þat hé só \alst{m}ildjene hugi &
ni \alst{b}ári an is \alst{b}reostun, \hld\ só skoldi habbjen \alst{b}arn godes: &
weldun sie só \alst{h}weðeres \hld\ \alst{h}êlagne Krist &
þero \alst{w}ordo ge·\alst{w}ítnon, \hld\ só hé þár for þemu \alst{w}erode ge·spráki, &
a·\alst{d}êldi te \alst{d}óme. \hld\ Þan wisse \alst{d}rohtin Krist &
þero \alst{m}anno só garo \hld\ \alst{m}ód-gi·þȧhti, &
iro \alst{w}rêðon \alst{w}illjon; \hld\ þȯ hé te þemu \alst{w}erode sprak, &
te \alst{a}llun þem \alst{e}rlun: \hld\ „só hwi-lik só iuwar \alst{á}no sí“, kwað hé, &
„\alst{s}líðja \alst{s}undjon, \hld\ só ganga iru \alst{s}elvo tó &
ęndi sie at \alst{ê}rist \hld\ \alst{e}rl mid is handun &
\alst{st}ên ana werpe.“ \hld\ Só \alst{st}ódun Judeon, &
\alst{þ}ȧhtun ęndi \alst{þ}agodun: \hld\ ni mahte \alst{þ}egạn nigijan &
wið þem \alst{w}ord-kwidi \hld\ \alst{w}iðer-saka finden: &
ge·hugde \alst{m}anno ge·hwi-lik \hld\ \alst{m}ên-gi·þȧhti, &
is \alst{s}elves \alst{s}undja: \hld\ ni was iro só \alst{s}ikur ênig, &
þat hé bi þemu \alst{w}orde \hld\ þemu \alst{w}íve ge·dorsti &
\alst{st}ên an werpen, \hld\ ak létun sie \alst{st}anden þár &
\alst{ê}nan þár inne \hld\ ęndi im \alst{ú}t þanen &
\alst{g}éngun \alst{g}ram-harde \hld\ \alst{J}udeo liudi, &
\alst{ê}n aftar \alst{ȯ}ðrumu, \hld\ an-tat iro þár \alst{ê}nig ni was &
þes \alst{f}íundo \alst{f}olkes, \hld\ þe iro \alst{f}erhes þȯ, &
þeru \alst{i}dis \alst{a}ldar-lago \hld\ \alst{á}htjen weldi. &
Þȯ gi·\alst{f}ragn ik þat sie \alst{f}rágode \hld\ \alst{f}riðu-barn godes, &
allaro \alst{g}umono bętst: \hld\ „hwar kwámun þit \alst{J}udeono folk“, kwað hé, &
„þine \alst{w}iðer-sakon, \hld\ þea þi hér \alst{w}rógdun te mi? &
Ne sie þi \alst{h}iudu wiht \hld\ \alst{h}armes ne gi·dádun, &
þea \alst{l}iudi \alst{l}êðes, \hld\ þe þi weldun \alst{l}ívu be·niman, &
\alst{w}êgjan te \alst{w}undrun?“ \hld\ Þȯ sprak imu eft þat \alst{w}íf an·gęgin, &
kwað þat iru þár \alst{n}io·man \hld\ þurh þes \alst{n}ęrjandan &
\alst{h}êlaga \alst{h}elpa \hld\ \alst{h}arm ne gi·frumidi &
\alst{w}ammes te lône. \hld\ Þȯ sprak eft \alst{w}aldand Krist, &
\alst{d}rohtin manno: \hld\ „ne ik þi geþ ni \alst{d}ęrju n·eo·wiht“, kwað hé, &
„ak gang þí \alst{h}êl \alst{h}inen, \hld\ lát þi an þínumu \alst{h}ugi sorga, &
þat þú nio \alst{s}ïð aftar þius \hld\ \alst{s}undig ni werðes.“ &
\alst{H}abde iru þȯ gi·\alst{h}olpen \hld\ \alst{h}êlag barn godes, &
ge·\alst{f}riðot iro \alst{f}erạhe. \hld\ Þan stód þat \alst{f}olk Judeono &
\alst{u}viles \alst{a}n-mód \hld\ só fan \alst{ê}ristan, &
\alst{w}rêðes \alst{w}illjan, \hld\ hwó sie \alst{w}ord-hęti &
wið þat \alst{f}riðu-barn godes \hld\ \alst{f}rummjen móstin. &
Habdun þea \alst{l}iudi an twê \hld\ mid iro gi·\alst{l}ôvon gi·fangan: &
was þiu \alst{s}male þioda \hld\ \alst{s}ínes willjan &
\alst{g}ernora mikilu, \hld\ þes \alst{g}odes barnes word &
te ge·\alst{f}rummjenne, \hld\ só im iro \alst{f}râho gi·bôd: &
\alst{r}ómodun te \alst{r}ehta \hld\ bet þan þie \alst{r}íkjon man, &
\alst{h}abdun ina far iro \alst{h}êrron \hld\ ia far \alst{h}evan-kuning, &
ful·\alst{g}éngun imu \alst{g}erno. \hld\ Þȯ gi·wêt imu þe \alst{g}odes sunu &
an þene \alst{w}íh innan: \hld\ hwarf ina \alst{w}erod umbi, &
\alst{m}ęgin-þiodo gi·\alst{m}ang. \hld\ hé an \alst{m}iddjen stód, &
\alst{l}êrde þea \alst{l}iudi \hld\ \alst{l}iohtun wordun, &
\alst{h}lúdero stemnun: \hld\ was \alst{h}lust mikil, &
\alst{þ}agode \alst{þ}egạn manag, \hld\ ęndi hé þeru \alst{þ}iod gi·bôd, &
só hwe só þár mid \alst{þ}urstu \hld\ bi·\alst{þ}wungan wári, &
„só ganga imu herod \alst{d}rinkan te mi“, \hld\ kwað hé, „\alst{d}ago ge·hwi-likes &
\alst{s}wótjes brunnan. \hld\ Ik mag \alst{s}ęggjan iu, &
só hwe só hér gi·\alst{l}ôvid te mi \hld\ \alst{l}iudjo barno &
\alst{f}asto undar þesumu \alst{f}olke, \hld\ þat imu þan \alst{f}lioten skulun &
fan is \alst{l}ík-hamon \hld\ \alst{l}ibbjendi flód, &
\alst{i}rnandi water, \hld\ \alst{a}ho-spring mikil, &
\alst{k}umad þanen \alst{k}wika brunnon. \hld\ Þesa \alst{k}widi werðad wára, &
\alst{l}iudjun gi·\alst{l}êstid, \hld\ só hwemu só hér gi·\alst{l}ôvid te mi.“ &
Þan mênde mid þiu \alst{w}ataru \hld\ \alst{w}aldandjo Krist, &
\alst{h}êr \alst{h}evan-kuning \hld\ \alst{h}êlagna gêst, &
hwó þene \alst{f}iriho barn \hld\ ant·\alst{f}áhen skoldin, &
\alst{l}ioht ęndi \alst{l}isti \hld\ ęndi \alst{l}íf êwig, &
\alst{h}ôh \alst{h}evan-ríki \hld\ ęndi \alst{h}uldi godes.\eva

\bvb TODO.\evb\evg

\bvg\bva[48][3926]%
Wurðun þȯ þea \alst{l}iudi \hld\ umbi þea \alst{l}êra Kristes, &
umbi þiu \alst{w}ord an ge·\alst{w}inne: \hld\ stódun \alst{w}lanka man, &
\alst{g}êl-móde Judeon, \hld\ sprákun \alst{g}elp mikil, &
\alst{h}abdun it im te \alst{h}oska, \hld\ kwaðun þat sie mahtin gi·\alst{h}ôrjen wel, &
þat imu \alst{m}ahlidin fram \hld\ \alst{m}ódaga wihti, &
\alst{u}n-holde \alst{ú}t: \hld\ „nu hé an \alst{a}vu lêrid“, kwáðun sie, &
„\alst{w}ordu ge·hwi-liku.“ \hld\ Þȯ sprak eft þat \alst{w}erod ȯðar: &
„ni þurvun gi þene \alst{l}êrjand \alst{l}ahan“, \hld\ kwáðun sie: „kumad \alst{l}íves word &
\alst{m}ahtig fan is \alst{m}u̇de; \hld\ hé wirkid \alst{m}anages hwat, &
\alst{w}undres an þesaru \alst{w}er-oldi: \hld\ nis þat \alst{w}rêðaro dád, &
\alst{f}íundo kraftes: \hld\ nio it þan te su·likaru \alst{f}rumu ni wurði, &
ak it \alst{g}egnungo \hld\ fan \alst{g}ode alo-waldon, &
\alst{k}umid fan is \alst{k}rafte. \hld\ Þat mugun gi ant·\alst{k}ęnnjen wel &
an þem is \alst{w}árun \alst{w}ordun, \hld\ þat hé gi·\alst{w}ald havad &
\alst{a}lles ovar \alst{e}rðu.“ \hld\ Þȯ weldun ina þe \alst{a}nd-sakon þár &
an \alst{st}ędi fáhen \hld\ efþa \alst{st}ên ana werpen, &
ef sie im þero \alst{m}anno \hld\ \alst{m}ęnigi ni and-rédin, &
ni \alst{f}orhtodin þat \alst{f}olk-skępi. \hld\ Þȯ sprak þat \alst{f}riðu-barn godes: &
„ik tôgju iu \alst{g}ódes só filu“, \hld\ kwað hé, „fan \alst{g}ode selvumu, &
\alst{w}ordo ęndi \alst{w}erko: \hld\ nu willjad gi mi \alst{w}ítnon hér &
þurh iuwan \alst{st}arkan hugi, \hld\ \alst{st}ên ana werpen, &
bi·\alst{l}ôsjen mi \alst{l}ívu.“ \hld\ Þȯ sprákun imu eft þea \alst{l}iudi an·gęgin, &
\alst{w}rêða \alst{w}iðer-sakon: \hld\ „ne wí it be þínun \alst{w}erkun ni duat“, kwáðun sia, &
„þat wí þí \alst{a}ldres \hld\ tó \alst{á}htjen willjad, &
ak wí duat it be þínun \alst{w}ordun, \hld\ hwand þú su·lik \alst{w}áh sprikis, &
*hwand þú þik só \alst{m}áris \hld\ ęndi su·lik \alst{m}ên sagis, &
\alst{g}ihis for þeson \alst{J}udeon, \hld\ þat þú sís \alst{g}od selvo, &
\alst{m}ahtig drohtin, \hld\ ęndi bist þi þoh \alst{m}an só wi, &
\alst{k}uman fan þeson \alst{k}unnje.“ \hld\ \alst{K}rist alo-waldo &
ne wolda þero \alst{J}udeono þuȯ lęng \hld\ \alst{g}elpes hôrjan, &
\alst{w}rêðaro \alst{w}illjon, \hld\ ak hie im af þem \alst{w}íhe fuor &
ovar \alst{J}ordanes strôm; \hld\ habda \alst{j}ungron mid im, &
þia is \alst{s}áligun gi·\alst{s}ïðos, \hld\ þia im \alst{s}imlon mid im &
\alst{w}illjon \alst{w}onodun: \hld\ suohta \alst{w}erod ȯðer, &
deda þár só hie gi·\alst{w}onoda, \hld\ \alst{d}rohtin selvo, &
\alst{l}êrda þia \alst{l}iudi: \hld\ gi·\alst{l}ôvda þie wolda &
an is hêlagun word. \hld\ Þat skolda sinnon wel &%NOTE: alliteration is indeed missing.
\alst{m}anno só hwi-likon, \hld\ só þat an is \alst{m}uod gi·nam. &
Þuȯ gi·frang ik þat þár te \alst{K}riste \hld\ \alst{k}umana wurðun &%NOTE: gi·frang] Checked according to C.
\alst{b}odon fan \alst{B}ethaniu \hld\ ęndi sagdun þem \alst{b}arne godes, &
þat sia an þat \alst{â}rundi þarod \hld\ \alst{i}disi sęndin, &
\alst{M}aria ęndi \alst{M}artha, \hld\ \alst{m}agað frí-líka, &
swíðo \alst{w}un-sama \alst{w}íf; \hld\ þia \alst{w}issa hie bêðja, &
wárun im gi·\alst{s}wester twá, \hld\ þia hie \alst{s}elvo êr &
\alst{m}innjoda an is \alst{m}uode \hld\ þuru iro \alst{m}ildjan hugi, &
þiu \alst{w}íf þuru iro \alst{w}illjon guodan. \hld\ Sia im te \alst{w}áron þuȯ &
an·\alst{b}udun fon \alst{B}ethaniu, \hld\ þat iro \alst{b}ruoðer was &
\alst{L}azarus \alst{l}egar-fast \hld\ ęndi þat sia is \alst{l}íves ni wándun; &
bádun þat þarod \alst{k}wámi \hld\ \alst{K}rist alo-waldo &
\alst{h}êlag te \alst{h}elpu. \hld\ Reht só hie sia gi·\alst{h}ôrda þuȯ &
\alst{s}ęggjan fan só \alst{s}iekon, \hld\ só sprak hie \alst{s}án an·gęgin, &
kwað þat \alst{L}azaruses \hld\ \alst{l}egar ni wári &
gi·\alst{d}uan im te \alst{d}ôðe, \hld\ „ak þár skal \alst{d}rohtines lof“, kwaþ-hie, &
„gi·\alst{f}rumid werðan: \hld\ nis it im te ȯðron \alst{f}rêson gi·duan.“ &
was im þár þuȯ \alst{s}elvo \hld\ \alst{s}uno drohtines &
\alst{t}wá naht ęndi dagas. \hld\ Þiu \alst{t}íd was þuȯ ge·náhit, &
þat hie eft te \alst{J}erusalem \hld\ \alst{J}udeo liudjo &
\alst{w}íson \alst{w}elda, \hld\ só hie gi·\alst{w}ald habda. &
\alst{S}agda þuȯ is gi·\alst{s}ïðon \hld\ \alst{s}uno drohtines, &
þat hie eft ovar \alst{J}ordan \hld\ \alst{J}udeo liudi &
\alst{s}uokjan welda. \hld\ Þuȯ sprákun im \alst{s}án an·gęgin &
\alst{j}ungron sína: \hld\ „te hwí bist þú só \alst{g}ern þarod“, kwaðun sia, &
„\alst{f}rô mín, te \alst{f}aranne? \hld\ Ni þat nu \alst{f}urn ni was, &
þat sia þik þínero \alst{w}ordo \hld\ \alst{w}ítnon hogdun, &
weldun þi mid \alst{st}ênon \alst{st}arkan a·werpan? \hld\ nu þú eft undar þia \alst{st}rídigun þioda &
\alst{f}undos te \alst{f}aranne, \hld\ þár ist \alst{f}íondo gi·nuog, &
\alst{e}rlos \alst{o}var-muoda?“ \hld\ Þuȯ \alst{ê}n þero twe-livjo, &
\alst{Þ}uomas gi·málda \hld\ —was im gi·\alst{þ}ungan mann, &
\alst{d}iur-lík \alst{d}rohtines þegạn—: \hld\ „ne skulun wí im þia \alst{d}ád lahan“, kwaþ-hie, &
„ni \alst{w}ęrnjan wí im þes \alst{w}illjen, \hld\ ak wita im \alst{w}onjan mid, &%TODO: check "wonjan"
\alst{þ}uolojan mid u̇sson \alst{þ}iodne: \hld\ þat ist \alst{þ}egnes kust, &
þat hie mid is \alst{f}râhon samad \hld\ \alst{f}asto gi·stande, &
\alst{d}ôje mid im þár an \alst{d}uome. \hld\ \alst{D}uan u̇s alla só, &
\alst{f}olgon im te þero \alst{f}ęrdi: \hld\ ni látan u̇se \alst{f}erạh wið þiu &
\alst{w}ihtes \alst{w}irðig, \hld\ neva wí an þem \alst{w}erode mid im, &
\alst{d}ôjan mid u̇son \alst{d}rohtine. \hld\ Þan lêvot u̇s þoh \alst{d}uom after, &
\alst{g}uod word for \alst{g}umon.“ \hld\ Só wurðun þuȯ \alst{j}ungron Kristes, &
\alst{e}rlos \alst{a}ðal-borana \hld\ an \alst{ê}n-falden hugje, &
\alst{h}êrren te willjen. \hld\ Þuȯ sagda \alst{h}êlag Krist &
\alst{s}elvo is gi·\alst{s}ïðon \hld\ þat a·\alst{s}lápan was &
\alst{L}azarus fan þem \alst{l}egare, \hld\ „havit þit \alst{l}ioht a·gevan, &
an·\alst{s}wevit ist an \alst{s}elmon. \hld\ Nu wí an þena \alst{s}ïð faran &
ęndi ina a·\alst{w}ękkjan, \hld\ þat hie muoti eft þesa \alst{w}er-old sehan, &
\alst{l}ibbjandi \alst{l}ioht: \hld\ þan wirðit iuwa gi·\alst{l}ôvo after þiu &
\alst{f}orð-werd gi·\alst{f}ęstid.“ \hld\ Þuȯ gi·wêt hie im ovar þia \alst{f}luod þanan, &
þie \alst{g}uodo \alst{g}odes suno, \hld\ an-þat hie mid is \alst{j}ungron kwam &
þár te \alst{B}ithaniu, \hld\ \alst{b}arn drohtines &
\alst{s}elvo mid is gi·\alst{s}ïðon, \hld\ þár þia gi·\alst{s}wester twá, &
\alst{M}aria ęndi \alst{M}artha \hld\ an \alst{m}uod-karon &
\alst{s}êraga \alst{s}átun. \hld\ Was þár gi·\alst{s}amnot filo &
fan \alst{J}erusalem \hld\ \alst{J}udeo liudo, &
þia þiu *\alst{w}íf \alst{w}eldun \hld\ \alst{w}ordun fruovrjan, &
þat sie só ni \alst{k}arodin \hld\ \alst{k}ind-jungas dôð, &
\alst{L}azaruses far·\alst{l}ust. \hld\ Só þȯ þe \alst{l}andes ward &
\alst{g}éng an þiu \alst{g}ardos, \hld\ só wurðun þes \alst{g}odes barnes &
\alst{k}umi þár gi·\alst{k}u̇ðid, \hld\ þat hé só \alst{k}raftig was &
bi þeru \alst{b}urg úten. \hld\ Þȯ im \alst{b}êðjun was, &
þem \alst{w}ívun su·lik \alst{w}illjo, \hld\ þat sie im \alst{w}aldand tó, &
þat \alst{f}riðu-barn godes, \hld\ \alst{f}arandjen wissun.\eva

\bvb TODO.\evb\evg

\bvg\bva[49][4025]%
Þȯ þem \alst{w}ívun was \hld\ \alst{w}illjono mêsta &
\alst{k}umi drohtines \hld\ ęndi \alst{K}ristes word &
te gi·\alst{h}ôrjenne. \hld\ \alst{H}eovandi géng &
\alst{M}artha \alst{m}ód-karag \hld\ wið só \alst{m}ahtigne &
\alst{w}ordun \alst{w}ehslan \hld\ ęndi wið \alst{w}aldand sprak &
an iro \alst{h}ugi \alst{h}riwig: \hld\ „Þár þú mí, \alst{h}êrro mín“, kwað siu, &
„\alst{n}ęrjendero bętst, \hld\ \alst{n}áhor wáris, &
\alst{h}êljand þe gódo, \hld\ þan ni þorfti ik nú su·lik \alst{h}arm þolon, &
\alst{b}ittra \alst{b}reost-kara, \hld\ þan ni wári nú mín \alst{b}róðer dôd, &
\alst{L}azarus fan þesumu \alst{l}iohte, \hld\ ak hé imu mahti \alst{l}ibbjen forð &
\alst{f}erạhes ge·\alst{f}ullid. \hld\ Ik þoh, \alst{f}rô mín, te þí &
\alst{l}iohto gi·\alst{l}ôvju, \hld\ \alst{l}êrjandero bętst, &
só hwes só þú \alst{b}iddjen wili \hld\ \alst{b}erhton drohtin, &
þat hé it þi sán far·\alst{g}ivid, \hld\ \alst{g}od alo-mahtig, &
gi·\alst{w}erðot þínan \alst{w}illjan.“ \hld\ Þȯ sprak eft \alst{w}aldand Krist &
þeru \alst{i}dis \alst{a}nd-wordi: \hld\ „Ni lát þú þí an \alst{i}nnan þes“, kwað hé, &
„þínan \alst{s}evon \alst{s}werkan: \hld\ ik þí \alst{s}ęggjan mag &
\alst{w}árun \alst{w}ordun, \hld\ þat þes nis gi·\alst{w}and ênig, &
nevu þín \alst{b}róðer skal \hld\ þurh gi·\alst{b}od godes, &
þurh \alst{d}rohtines kraft \hld\ fan \alst{d}ôðe a·standen &
an is \alst{l}ík-hamon.“ \hld\ „All hębbju ik gi·\alst{l}ôvon só“, kwað siu, &
„þat it só gi·\alst{w}erðen skal, \hld\ só hwan só þius \alst{w}er-old ęndjod &
ęndi þe \alst{m}árjo dag \hld\ ovar \alst{m}an fęrid, &
þat hé þan fan \alst{e}rðu skal \hld\ \alst{u}p a·standen &
an þemu \alst{d}ómes \alst{d}aga, \hld\ þan werðad fan \alst{d}ôðe kwika &
þurh \alst{m}aht godes \hld\ \alst{m}an-kunnjes ge·hwi-lik, &
a·\alst{r}ísad fan \alst{r}estu.“ \hld\ Þȯ sagde \alst{r}íkjo Krist &
þeru \alst{i}dis \alst{a}lo-mahtig \hld\ \alst{o}ponun wordun, &
þat hé \alst{s}elvo was \hld\ \alst{s}unu drohtines, &
bêðju ia \alst{l}íf ia \alst{l}ioht \hld\ \alst{l}iudjo barnon &
te a·\alst{st}andanne: \hld\ „nio þe \alst{st}erven ni skal, &
\alst{l}íf far·\alst{l}iosen, \hld\ þe hér gi·\alst{l}ôvid te mi: &
þoh ina \alst{ę}ldi-barn \hld\ \alst{e}rðu bi·þękkjen, &
\alst{d}iapo bi·\alst{d}elven, \hld\ nis hé \alst{d}ôd þiu mêr: &
þat \alst{f}lêsk is bi·\alst{f}olhen, \hld\ þat \alst{f}erạh is gi·halden, &
is þiu \alst{s}iola gi·\alst{s}und.“ \hld\ Þȯ sprak imu eft \alst{s}án an·gęgin &
þat \alst{w}íf mid iro \alst{w}ordun: \hld\ „ik gi·lôvju þat þú þe \alst{w}áro bist“, kwað siu, &
„\alst{K}rist godes sunu: \hld\ þat mag man ant·\alst{k}ęnnjen wel, &
\alst{w}iten an þínun \alst{w}ordun, \hld\ þat þú gi·\alst{w}ald haves &
þurh þiu \alst{h}êlagon gi·skapu \hld\ \alst{h}imiles ęndi erðun.“ &
Þȯ ge·fragn ik þat þár þero \alst{i}disjo kwam \hld\ \alst{ȯ}ðar gangan &
\alst{M}aria \alst{m}ód-karag: \hld\ géngun iro \alst{m}anaga aftar &
\alst{J}udeo liudi. \hld\ Þȯ siu þemu \alst{g}odes barne &
\alst{s}agde \alst{s}êrag-mód, \hld\ hwat iru te \alst{s}orgun gi·stód &
an iro \alst{h}ugi \alst{h}armes: \hld\ \alst{h}ofnu kúmde &
\alst{L}azaruses far·\alst{l}ust, \hld\ \alst{l}iaves mannes, &
\alst{g}riat \alst{g}ornundi, \hld\ an-tat þemu \alst{g}odes barne &
\alst{h}ugi warð gi·\alst{h}rórid: \hld\ \alst{h}ête trahni &
\alst{w}ópu a·\alst{w}ellun, \hld\ ęndi þȯ te þem \alst{w}ívun sprak, &
hét ina þȯ \alst{l}êdjen, \hld\ þár \alst{L}azarus was &
\alst{f}oldu bi·\alst{f}olhen. \hld\ Lag þár ên \alst{f}elis bi·ovan, &
\alst{h}ard stên be·\alst{h}liden. \hld\ Þȯ hét þe \alst{h}êlago Krist &
ant·\alst{l}úkan þea \alst{l}éia, \hld\ þat hé mósti þat \alst{l}ík sehan, &
\alst{h}rêo skawojen. \hld\ Þȯ ni mahte an iro \alst{h}ugi míðan &
\alst{M}arþa for þeru \alst{m}ęnegi, \hld\ wið \alst{m}ahtigne sprak: &
„\alst{f}rô mín þe gódo“, \hld\ kwað siu, „ef man þene \alst{f}elis nimid, &
þene \alst{st}ên ant·lúkid, \hld\ þan wániu ik þat þanen \alst{st}ank kume, &
un·\alst{s}wóti \alst{s}wek, \hld\ hwand ik þi \alst{s}ęggjan mag &
\alst{w}árun \alst{w}ordun, \hld\ þat þes nis gi·\alst{w}and ênig, &
þat hé þár nu bi·\alst{f}olhen was \hld\ \alst{f}iuwar naht ęndi dagos &
an þemu \alst{e}rð-grave.“ \hld\ \alst{A}nd-wordi gaf &
\alst{w}aldand þemu \alst{w}íve: \hld\ „Hhwat ni sagde ik þí te \alst{w}árun êr“, kwað hé, &
„ef þú gi·\alst{l}ôvjen wili, \hld\ þan nis nu \alst{l}ang te þiu, &
þat þú hér ant·\alst{k}ęnnjen skalt \hld\ \alst{k}raft drohtines, &
þe \alst{m}ikilon \alst{m}aht godes?“ \hld\ Þȯ géngun \alst{m}anage tó, &
af·\alst{h}óvun \alst{h}arden stên. \hld\ Þȯ sah þe \alst{h}êlago Krist &
\alst{u}p mid is \alst{ô}gun, \hld\ \alst{á}-lát sagde &
þemu þe þese \alst{w}er-old gi·skóp, \hld\ „þes þú mín \alst{w}ord gi·hôris“, kwað hé, &
\alst{„}sigi-drohtin \alst{s}elvo; \hld\ ik wêt þat þú só \alst{s}imlun duos, &
ak ik duom it be þesumu \alst{g}rôton \hld\ \alst{J}udeono folke, &
þat sie þat te \alst{w}árun \alst{w}itin, \hld\ þat þú mi an þese \alst{w}er-old sęndes &
þesun \alst{l}iudjun te \alst{l}êrun.“ \hld\ Þȯ hé te \alst{L}azaruse hriop &
\alst{st}arkaru \alst{st}emnju \hld\ ęndi hét ina \alst{st}anden up &
ia fan þemu \alst{g}rave \alst{g}angan. \hld\ Þȯ warð þe \alst{g}êst kumen &
an þene \alst{l}ík-hamon: \hld\ hé bi·gan is \alst{l}iði hrórjen, &
ant·\alst{w}arp undar þemu gi·\alst{w}ę́dje: \hld\ was imo só be·\alst{w}unden þȯ noh, &
an \alst{h}rêo-będdjon bi·\alst{h}elid. \hld\ Hét imu \alst{h}elpen þȯ &
\alst{w}aldandjo Krist. \hld\ \alst{W}eros géngun tó, &
ant·\alst{w}undun þat ge·\alst{w}ádi. \hld\ \alst{W}ánum up a·rês &
\alst{L}azarus te þesumu \alst{l}iohte: \hld\ was imu is \alst{l}íf far·geven, &
þat hé is \alst{a}ldar-lagu \hld\ \alst{ê}gan mósti, &
\alst{f}riðu \alst{f}orð-wardes. \hld\ Þȯ \alst{f}agonadun bêðja, &
\alst{M}aria ęndi \alst{M}artha: \hld\ ni mag þat \alst{m}an ȯðrumu &
gi·\alst{s}ęggjan te \alst{s}ȯðe, \hld\ hwó þea ge·\alst{s}wester twó &
\alst{m}ęndjodun an iro \alst{m}óde. \hld\ \alst{M}aneg wundrode &
\alst{J}udeo liudjo, \hld\ þȯ sie ina fan þemu \alst{g}rave sáhun &
\alst{s}ïðon ge·\alst{s}unden, \hld\ þene þe êr \alst{s}uht far·nam &
ęndi sie bi·\alst{d}ulvun \hld\ \alst{d}iapo undar erðu &
\alst{l}íves \alst{l}ôsen: \hld\ þȯ móste imu \alst{l}ibbjen forð &
\alst{h}êl an \alst{h}êmun. \hld\ Só mag \alst{h}evan-kuninges, &
þiu \alst{m}ikile \alst{m}aht godes \hld\ \alst{m}anno ge·hwi-likes &
\alst{f}erạhe gi·\alst{f}ormon \hld\ ęndi wið \alst{f}íundo níð &
\alst{h}êlag \alst{h}elpen, \hld\ só hwemu só hé is \alst{h}uldi far·givid.\eva

\bvb TODO.\evb\evg

\bvg\bva[50][4118]%
Þȯ warð þár só \alst{m}anagumu \alst{m}anne \hld\ \alst{m}ód aftar Kriste, &
gi·\alst{h}worven \alst{h}ugi-skęfti, \hld\ sïðor sie is \alst{h}êlagon werk &
\alst{s}elvon gi·\alst{s}áhun, \hld\ hwand eo êr \alst{s}u·lik ni warð &
\alst{w}undẹr an \alst{w}er-oldi. \hld\ Þan was eft þes \alst{w}erodes só filu, &
só \alst{m}ód-starke \alst{m}an: \hld\ ni weldon þe \alst{m}aht godes &
ant·\alst{k}ęnnjen \alst{k}u̇ð-líko, \hld\ ak sie wið is \alst{k}raft mikil &
\alst{w}unnun mid iro \alst{w}ordun: \hld\ \alst{w}árun im waldandes &
\alst{l}êra so \alst{l}êða: \hld\ sóhtun im \alst{l}iudi ȯðra &
an \alst{J}erusalem, \hld\ þár \alst{J}udeono was &
\alst{h}êri \alst{h}and-mahạl \hld\ ęndi \alst{h}ôvid-stędi, &
\alst{r}ôt \alst{g}um-skępi \hld\ \alst{g}rimmaro þioda. &
Sie \alst{k}u̇ðdun im þȯ \alst{K}ristes werk, \hld\ kwáðun þat sie \alst{k}wikan sáhin &
þene \alst{e}rl mid iro \alst{ô}gun, \hld\ þe an \alst{e}rðu was, &
\alst{f}oldu bi·\alst{f}olhen \hld\ \alst{f}iuwar naht ęndi dagos, &
\alst{d}ôd bi·\alst{d}olven, \hld\ an-tat hé ina mid is \alst{d}ádjun selvo, &
mid is wordun a·\alst{w}ękide, \hld\ þat hé mósti þese \alst{w}er-old sehan. &
Þȯ was þat só \alst{w}iðer-\alst{w}ard \hld\ \alst{w}lankun mannun, &
\alst{J}udeo liudjun: \hld\ hétun iro \alst{g}um-skępi þȯ, &
\alst{w}erod samnojan \hld\ ęndi \alst{w}arvos fáhen, &
\alst{m}ęgin-þioda gi·\alst{m}ang, \hld\ an \alst{m}ahtigna Krist &
\alst{r}iedun an \alst{r}únun: \hld\ „nis þat \alst{r}ád ênig“, kwáðun sie, &
„þat wí þat gi·\alst{þ}olojan: \hld\ wili þesaro \alst{þ}ioda te filu &
gi·\alst{l}ôvjen aftar is \alst{l}êrun. \hld\ Þan u̇s \alst{l}iudi farad, &
an \alst{e}o-rid-folk, \hld\ werðat u̇sa \alst{o}var-hôvdun &
\alst{r}inkos fan \alst{R}úmu. \hld\ Þan wí þeses \alst{r}íkjes skulun &
\alst{l}ôse \alst{l}ibbjen \hld\ efþa wí skulun u̇ses \alst{l}íves þolon, &
\alst{h}ęliðos u̇saro \alst{h}ôvdo.“ \hld\ Þȯ sprak þár ên gi·\alst{h}êrod man &
ovar \alst{w}arf \alst{w}ero, \hld\ þe was þes \alst{w}erodes þȯ &
an þeru \alst{b}urg innan \hld\ \alst{b}iskop þero liudjo &
—\alst{K}aiphas was hé hêten; \hld\ habdun ina gi·\alst{k}oranen te þiu &
an þeru \alst{g}ę́r-talu \hld\ \alst{J}udeo liudi, &
þat hé þes \alst{g}odes húses \hld\ \alst{g}ômjen skoldi, &
\alst{w}ardon þes \alst{w}íhes—: \hld\ „Mí þunkid \alst{w}undẹr mikil“, kwað hé, &
„\alst{m}ári þioda, \hld\ —gí kunnun \alst{m}anages gi·skêð— &
hwí gí þat te \alst{w}árun ni \alst{w}itin, \hld\ \alst{w}erod Judeono, &
þat hér is \alst{b}ętera rád \hld\ \alst{b}arno ge·hwi-likumu, &
þat man hér \alst{ê}nne man \hld\ \alst{a}ldru bi·lôsje &
ęndi þat hé þurh iuwa \alst{d}ádi \hld\ \alst{d}rôreg sterve, &
for þesumu \alst{f}olk-skępi \hld\ \alst{f}erạh far·láte, &
þan al þit \alst{l}iud-werod \hld\ far·\alst{l}oren werðe.“ &
Ni was it þoh is \alst{w}illjan, \hld\ þat hé só \alst{w}ár ge·sprak, &
só \alst{f}orð for þemu \alst{f}olke, \hld\ \alst{f}rume man-kunnjes &
gi·\alst{m}ênde for þeru \alst{m}ęnegi, \hld\ ak it kwam imu fan þeru \alst{m}aht godes &
þurh is \alst{h}êlagan \alst{h}êd, \hld\ hwand hé þat \alst{h}ús godes &
þár an \alst{J}erusalem \hld\ bi·\alst{g}angan skolde, &
\alst{w}ardon þes \alst{w}íhes: \hld\ be·þiu hé só \alst{w}ár gi·sprak, &
\alst{b}iskop þero liudjo, \hld\ hwó skoldi þat \alst{b}arn godes &
\alst{a}lla \alst{i}rmin-þiod \hld\ mid is \alst{ê}nes ferhe, &
mid is \alst{l}ívu a·\alst{l}ôsjen: \hld\ þat was allaro þesaro \alst{l}iudjo rád, &
\alst{h}wand hé gi·\alst{h}alode \hld\ mid þiu \alst{h}êðina liudi, &
\alst{w}eros an is \alst{w}illjon \hld\ \alst{w}aldandio Krist. &
Þȯ wurðun \alst{ê}n-wordje \hld\ \alst{o}var-módje man, &
\alst{w}erod Judeono, \hld\ ęndi an iro \alst{w}arve gi·sprákun, &
\alst{m}ári þioda, \hld\ þat sie im ni létin iro \alst{m}ód twehon: &
só hwe só ina undar þemu \alst{f}olke \hld\ \alst{f}inden mahti, &
þat ina sán gi·\alst{f}éngi \hld\ ęndi \alst{f}orð brȧhti &
an þero \alst{þ}iodo \alst{þ}ing; \hld\ kwáðun þat sie ni mahtin gi·\alst{þ}olojan lęng, &
þat sie þe \alst{ê}no man \hld\ só \alst{a}lla weldi, &
\alst{w}erod far·winnen. \hld\ Þan wisse \alst{w}aldand Krist &
þero \alst{m}anno só garo \hld\ \alst{m}ód-gi·þȧhti, &
\alst{h}ęti-grimmon \alst{h}ugi, \hld\ hwand imu ni was bi·\alst{h}olen eo·wiht &
an þesaru \alst{m}iddil-gard: \hld\ hé ni welde þȯ an þie \alst{m}ęnigi innen &
sïður \alst{o}pen-líko, \hld\ under þat \alst{e}rlo folk, &
\alst{g}angan under þea \alst{J}udeon: \hld\ bêd þe \alst{g}odes sunu &
þero \alst{t}orọhtjon \alst{t}íd, \hld\ þe imu \alst{t}ó-ward was, &
þat hé far \alst{þ}esa \alst{þ}ioda \hld\ \alst{þ}olojan welde, &
far þit \alst{w}erod \alst{w}íti: \hld\ \alst{w}isse imu selvo &
þat \alst{d}ag-þingi garo. \hld\ Þȯ gi·wêt imu u̇se \alst{d}rohtin forð &
ęndi imu þȯ an \alst{E}ffrem \hld\ \alst{a}lo-waldo Krist &
an þeru \alst{h}ôhon burg \hld\ \alst{h}êlag drohtin &
\alst{w}unode mid is \alst{w}erodu, \hld\ an-tat hé an is \alst{w}illjan hwarf &
eft te \alst{B}ethania \hld\ \alst{b}rahtmu þiu mikilun, &
mid þiu is \alst{g}ódum \alst{g}um-skępi. \hld\ \alst{J}udeon bi·sprákun þat &
\alst{w}ordu ge·hwi-liku, \hld\ þȯ sie imu su·lik \alst{w}erod mikil &
\alst{f}olgon gi·sáhun: \hld\ „nis \alst{f}rume ênig“, kwáðun sie, &
„u̇ses \alst{r}íkjes gi·\alst{r}ádi, \hld\ þoh wí \alst{r}eht sprekan, &
ni \alst{þ}íhit u̇ses \alst{þ}inges wiht: \hld\ þius \alst{þ}iod wili &
\alst{w}ęndjen after is \alst{w}illjan; \hld\ imu all þius \alst{w}er-old folgot, &
\alst{l}iudi bi þem is \alst{l}êrun, \hld\ þat wí imu \alst{l}êðes wiht &
for þesumu \alst{f}olk-skępi \hld\ gi·\alst{f}rummjen ni mótun.“\eva

\bvb TODO.\evb\evg

\bvg\bva[51][4198]%
Gi·wêt imu þȯ þat \alst{b}arn godes \hld\ innan \alst{B}ethania &
\alst{s}ehs nahtun êr, \hld\ þan þiu \alst{s}amnunga &
þár an \alst{J}erusalem \hld\ \alst{J}udeo liudjo &
an þem \alst{w}íh-dagun \hld\ \alst{w}erðen skolde, &
þat sie skoldun \alst{h}aldan \hld\ þea \alst{h}êlagon tídi, &
\alst{J}udeono paskha. \hld\ Béd þe \alst{g}odes sunu, &
\alst{m}ahtig under þeru \alst{m}ęnegi: \hld\ was þár \alst{m}anno kraft, &
\alst{w}erodes bi þem is \alst{w}ordun. \hld\ Þár géngun ina twê \alst{w}íf umbi, &
\alst{M}aria ęndi \alst{M}artha, \hld\ mid \alst{m}ildju hugi, &
\alst{þ}ionodun imu \alst{þ}eo-líko. \hld\ \alst{Þ}iodo drohtin &
gaf im \alst{l}ang-sam \alst{l}ôn: \hld\ lét sea \alst{l}êðes gi·hwes, &
\alst{s}undjono \alst{s}ikora, \hld\ ęndi \alst{s}elvo gi·bôd, &
þat sea an \alst{f}riðe \alst{f}órin \hld\ wiðer \alst{f}íundo níð, &
þea \alst{i}disa mid is \alst{o}rlovu gódu: \hld\ habdun iro \alst{a}mbaht-skępi &
bi·\alst{w}ęndid an is \alst{w}illjon. \hld\ Þȯ gi·wêt imu \alst{w}aldand Krist &
\alst{f}orð mid þiu \alst{f}olku, \hld\ \alst{f}iriho drohtin, &
innan \alst{J}erusalem, \hld\ þár \alst{J}udeono was &
\alst{h}ęte-lík \alst{h}ard-buri, \hld\ þár sie þea \alst{h}êlagon tíd &
\alst{w}arodun at þemu \alst{w}íhe; \hld\ was þár \alst{w}erodes só filu, &
\alst{k}raftigaro \alst{k}unnjo, \hld\ þie ni weldun \alst{K}ristes word &
\alst{g}erno hôrjen \hld\ ni te þemu \alst{g}odes barne &
an iro \alst{m}ód-sevon \hld\ \alst{m}innje ni habdun, &
ak \alst{w}árun im só \alst{w}rêða \hld\ \alst{w}lanka þioda, &
\alst{m}ódeg \alst{m}an-kunni, \hld\ habdun im \alst{m}orð-hugi, &
\alst{i}n-wid an \alst{i}nnan: \hld\ an \alst{a}vuh far·féngun &
\alst{K}ristes lêre, \hld\ weldun ina \alst{k}raftigna &
\alst{w}ítnon þero \alst{w}ordo; \hld\ ak was þár \alst{w}erodes só filu, &
umbi \alst{e}rl-skępi \hld\ \alst{a}nt-langana dag, &
habde ine þiu \alst{s}male þiod \hld\ þurh is \alst{s}wótjun word &
\alst{w}erodu bi·\alst{w}orpen, \hld\ þat ine þie \alst{w}iðer-sakon &
under þemu \alst{f}olk-skępi \hld\ \alst{f}áhen ne gi·dorstun, &
ak \alst{m}iðun is bi þeru \alst{m}ęnegi. \hld\ Þan stód \alst{m}ahtig Krist &
an þemu \alst{w}íhe innan, \hld\ sagde \alst{w}ord manag &
\alst{f}iriho barnun te \alst{f}rumu. \hld\ Was þár \alst{f}olk umbi &
allan \alst{l}angan dag, \hld\ an-tat þiu \alst{l}iohte gi·wêt &
\alst{s}unne te \alst{s}edle. \hld\ Þȯ te \alst{s}ęliðun fór &
\alst{m}an-kunnjes \alst{m}anag. \hld\ Þan was þár ên \alst{m}ári berg &
bi þeru \alst{b}urg úten, \hld\ þe was \alst{b}rêd ęndi hôh, &
\alst{g}róni ęndi skôni: \hld\ hétun ina \alst{J}udeo liudi &
\alst{O}liueti bi namon. \hld\ Þár imu \alst{u}p gi·wêt &
\alst{n}ęrjendjo Krist, \hld\ só ina þiu \alst{n}aht bi·féng, &
was imu þár mid is \alst{j}ungarun, \hld\ só ine þár \alst{J}udeono ênig &
ni \alst{w}isse ti \alst{w}árun, \hld\ hwand hé an þemu \alst{w}íhe stód, &
\alst{l}iudjo drohtin, \hld\ só \alst{l}ioht ôstene kwam, &
ant·\alst{f}éng þat \alst{f}olk-skępi \hld\ ęndi im \alst{f}ilu sagde &
\alst{w}ároro \alst{w}ordo, \hld\ só nis an þesaru \alst{w}er-oldi ênig, &
an þesaru \alst{m}iddil-gard \hld\ \alst{m}anno só spáhi, &
\alst{l}iudjo barno nig·ên, \hld\ þat þero \alst{l}êrono mugi &
\alst{ę}ndi gi·tęlljen, \hld\ þe hé þár an þemu \alst{a}lạhe gi·sprak, &
\alst{w}aldand an þemu \alst{w}íhe, \hld\ ęndi simlun mid is \alst{w}ordun gi·bôd, &
þat sie sie \alst{g}ęrewidin \hld\ te \alst{g}odes ríkje, &
allaro \alst{m}anno ge·hwi-lik, \hld\ þat sie móstin an þemu \alst{m}árjon daga &
iro \alst{d}rohtines \hld\ \alst{d}iuriða ant·fáhen. &
Sagde im hwat sie it \alst{s}undjun frumidun \hld\ ęndi \alst{s}imlun gi·bôd, &
þat sie þea a·\alst{l}ęskidin; \hld\ hét sie \alst{l}ioht godes &
\alst{m}innjon an iro \alst{m}óde, \hld\ \alst{m}ên far·láten, &
\alst{a}voha \alst{o}var-hugdi, \hld\ \alst{ô}d-módi niman, &
\alst{h}laðen þat an iro \alst{h}ertan; \hld\ kwað þat im þan wári \alst{h}evan-ríki, &
\alst{g}aru \alst{g}ódo mêst. \hld\ Þȯ warð þár \alst{g}umono só filu &
gi·\alst{w}ęndid aftar is \alst{w}illjon, \hld\ sïður sie þat \alst{w}ord godes &
\alst{h}êlag gi·\alst{h}ôrdun, \hld\ \alst{h}evan-kuninges, &
ant·\alst{k}ęndun \alst{k}raft mikil, \hld\ \alst{k}umi drohtines, &
\alst{h}êrron \alst{h}elpe, \hld\ ia þat \alst{h}evan-ríki was, &
\alst{n}ęrjendi gi·\alst{n}áhid \hld\ ęndi \alst{n}áða godes &
\alst{m}anno barnun. \hld\ Sum só \alst{m}ódeg was &
\alst{J}udeo folkes, \hld\ habdun \alst{g}rimman hugi, &
\alst{s}líð-móden \alst{s}evon \hld\ {[...]}, &
ni weldun is \alst{w}orde gi·lôvjen, \hld\ ak habdun im ge·\alst{w}in mikil &
wið þea \alst{K}ristes kraft: \hld\ \alst{k}umen ni móstun &
þea \alst{l}iudi þurh \alst{l}êðen stríd, \hld\ þat sie gi·\alst{l}ôvon te imu &
\alst{f}asto gi·\alst{f}éngin; \hld\ ni was im þiu \alst{f}rume giviðig, &
þat sie \alst{h}evan-ríki \hld\ \alst{h}abbjen móstin. &
\alst{G}éng imu þȯ þe \alst{g}odes sunu \hld\ ęndi is \alst{j}ungaron mid imu, &
\alst{w}aldand fan þemu \alst{w}íhe, \hld\ all só is \alst{w}illjo géng, &
iak imu uppen þene \alst{b}erg gi·stêg \hld\ \alst{b}arn drohtines: &
\alst{s}at imu þár mid is ge·\alst{s}ïðun \hld\ ęndi im \alst{s}agde filu &
\alst{w}ároro \alst{w}ordo. \hld\ Sí bi·gunnun im þȯ umbi þene \alst{w}íh sprekan, &
þie \alst{g}umon umbi þat \alst{g}odes hús, \hld\ kwáðun þat ni wári \alst{g}ód-líkora &
\alst{a}lạh ovar \alst{e}rðu \hld\ þurh \alst{e}rlo hand, &
þurh \alst{m}annes gi·werk \hld\ mid \alst{m}ęgin-kraftu &
\alst{r}akud a·\alst{r}ihtid. \hld\ Þȯ þe \alst{r}íkjo sprak, &
\alst{h}êr \alst{h}evan-kuning \hld\ —\alst{h}ôrdun þe ȯðra—: &
„ik mag iu gi·\alst{t}ęlljen“, \hld\ kwað hé, „þat noh wirðid þiu \alst{t}íd kumen, &
þat is af·\alst{st}anden ni skal \hld\ \alst{st}ên ovar ȯðrumu, &
ak it \alst{f}allid ti \alst{f}oldu \hld\ ęndi \alst{f}iur nimid, &
\alst{g}rádag logna, \hld\ þoh it nu só \alst{g}ód-lík sí, &
só \alst{w}ís-líko gi·\alst{w}arht, \hld\ ęndi só dód all þesaro \alst{w}er-oldes gi·skapu, &
te·\alst{g}lídid \alst{g}róni wang.“ \hld\ Þȯ géngun imu is \alst{j}ungaron tó, &
frágodun ina só \alst{st}illo: \hld\ „hwó lango skal \alst{st}anden noh“, kwáðun sie, &
„þius \alst{w}er-old an \alst{w}unnjun, \hld\ êr þan þat gi·\alst{w}and kume, &
þat þe \alst{l}asto dag \hld\ \alst{l}iohtes skíne &
þurh \alst{w}olkan-skion, \hld\ efþo hwan is þín eft \alst{w}án kumen &
an þene \alst{m}iddil-gard, \hld\ \alst{m}anno kunnje &
te a·\alst{d}êljenne, \hld\ \alst{d}ôdun ęndi kwikun? &
\alst{f}rô mín þe gódo, \hld\ u̇s is þes \alst{f}iri-wit mikil, &
\alst{w}aldandjo Krist, \hld\ hwan þat gi·\alst{w}erðen skuli.“\eva

\bvb TODO.\evb\evg

\bvg\bva[52][4294]%
Þȯ im \alst{a}nd-wordi \hld\ \alst{a}lo-waldo Krist &
\alst{g}ód-lík far·\alst{g}af \hld\ þem \alst{g}umun selvo: &
„þat havad só bi·\alst{d}ęrnid“, \hld\ kwað hé, „\alst{d}rohtin þe gódo, &
iak só \alst{h}ardo far·\alst{h}olen \hld\ \alst{h}imil-ríkjes fader, &
\alst{w}aldand þesaro \alst{w}er-oldes, \hld\ só þat \alst{w}iten ni mag &
ênig \alst{m}annisk barn, \hld\ hwan þiu \alst{m}árje tíd &
gi·\alst{w}irðid an þesaru \alst{w}er-oldi, \hld\ ne it ôk te \alst{w}áran ni kunnun &
\alst{g}odes ęngilos, \hld\ þie for imu \alst{g}ęgin-warde &
\alst{s}imlun \alst{s}indun: \hld\ sie it ôk gi·\alst{s}ęggjan ni mugun &
te \alst{w}áran mid iro \alst{w}ordun, \hld\ hwan þat gi·\alst{w}erðen skuli, &
þat hé willje an þesan \alst{m}iddil-gard, \hld\ \alst{m}ahtig drohtin, &
\alst{f}iriho \alst{f}andon. \hld\ \alst{F}ader wêt it êno &
\alst{h}êlag fan \alst{h}imile: \hld\ elkur is it bi·\alst{h}olen allun, &
\alst{k}wikun ęndi dôdun, \hld\ hwan is \alst{k}umi werðad. &
Ik mag iu þoh gi·\alst{t}ęlljen, \hld\ hwi-lik hér \alst{t}êkạn bi·foran &
gi·\alst{w}erðad \alst{w}undẹr-lík, \hld\ êr þan hé an þese \alst{w}er-old kume &
an þemu \alst{m}árjon daga: \hld\ þat wirðid hér êr an þemu \alst{m}ánon skín &
iak an þeru \alst{s}unnon só \alst{s}ame; \hld\ gi·\alst{s}werkad siu bêðju, &
mid \alst{f}inistre werðad bi·\alst{f}angan; \hld\ \alst{f}allad sterron, &
\alst{h}wít \alst{h}evan-tungạl, \hld\ ęndi \alst{h}risid erðe, &
\alst{b}ivod þius \alst{b}rêde wer-old \hld\ —wirðid su·likaro \alst{b}ôkno filu—: &
\alst{g}rimmid þe \alst{g}rôto sêo, \hld\ wirkid þie \alst{g}evenes strôm &
\alst{ę}gison mid is \alst{u̇}ðjun \hld\ \alst{e}rð-búandjun. &
Þan \alst{þ}orrot þiu \alst{þ}iod \hld\ þurh þat ge·\alst{þ}wing mikil, &
\alst{f}olk þurh þea \alst{f}orhta: \hld\ þan nis \alst{f}riðu hwęrgin, &
ak \alst{w}irðid \alst{w}íg só maneg \hld\ ovar þese \alst{w}er-old alla &
\alst{h}ęte-lík af·\alst{h}aben, \hld\ ęndi \alst{h}ęri lêdid &
\alst{k}unni ovar ȯðar: \hld\ wirðid \alst{k}uningo gi·win, &
\alst{m}ęgin-fard \alst{m}ikil: \hld\ wirðid \alst{m}anagoro kwalm, &
\alst{o}pen \alst{u}r-lagi \hld\ —þat is \alst{ę}gis-lík þing, &
þat io su·lik \alst{m}orð \hld\ skulun \alst{m}an af·hębbjen—, &
\alst{w}irðid \alst{w}ól só mikil \hld\ ovar þese \alst{w}er-old alle, &
\alst{m}an-stervono \alst{m}êst, \hld\ þero þe gio an þesaru \alst{m}iddil-gard &
\alst{s}wulti þurh \alst{s}uhti: \hld\ liggjad \alst{s}eoka man, &
\alst{d}riosat ęndi \alst{d}ôjat \hld\ ęndi iro \alst{d}ag ęndjad, &
\alst{f}ulljad mid iro \alst{f}erạhu; \hld\ \alst{f}ęrid un·met grôt &
\alst{h}ungạr \alst{h}ęti-grim \hld\ ovar \alst{h}ęliðo barn, &
\alst{m}ęti-gêdjono \alst{m}êst: \hld\ nis þat \alst{m}inniste &
þero \alst{w}ítjo an þesaru \alst{w}er-oldi, \hld\ þe hér gi·\alst{w}erðen skulun &
êr \alst{d}ómes \alst{d}age. \hld\ Só hwan só gi þea \alst{d}ádi gi·sehan &
gi·\alst{w}erðen an þesaru \alst{w}er-oldi, \hld\ só mugun gi þan te \alst{w}áran far·standen, &
þat þan þe \alst{l}atsto dag \hld\ \alst{l}iudjun náhid &
\alst{m}ári te \alst{m}annun \hld\ ęndi \alst{m}aht godes, &
\alst{h}imil-kraftes \alst{h}róri \hld\ ęndi þes \alst{h}êlagon kumi, &
\alst{d}rohtines mid is \alst{d}iuriðun. \hld\ Hwat gí þesaro \alst{d}ádjo mugun &
bi þesun \alst{b}ômun \hld\ \alst{b}iliði ant·kęnnjen: &
þan sie \alst{b}rustjad ęndi \alst{b}lójat \hld\ ęndi \alst{b}ladu tôgjat, &
\alst{l}ôf ant·\alst{l}úkad, \hld\ þan witun \alst{l}iudjo barn, &
þat þan is \alst{s}án after þiu \hld\ \alst{s}umer gi·náhid &
\alst{w}arm ęndi \alst{w}un-sam \hld\ ęndi \alst{w}edẹr skôni. &
Só witin gi ôk bi þesun \alst{t}êknun, \hld\ þe ik iu \alst{t}alde hér, &
hwan þe \alst{l}atsto dag \hld\ \alst{l}iudjun náhid. &
Þan sęggjo ik iu te \alst{w}áran, \hld\ þat êr þit \alst{w}erod ni mót, &
te·\alst{f}aran þit \alst{f}olk-skępi, \hld\ êr þan werðe ge·\alst{f}ullid só, &
mínu \alst{w}ord gi·\alst{w}árod. \hld\ Noh gi·\alst{w}and kumid &
\alst{h}imiles ęndi erðun, \hld\ ęndi stéid mín \alst{h}êlag word &
\alst{f}ast \alst{f}orð-wardes \hld\ ęndi wirðid al ge·\alst{f}ullod só, &
gi·\alst{l}êstid an þesumu \alst{l}iohte, \hld\ só ik for þesun \alst{l}iudjun ge·spriku. &
\alst{w}akot gí \alst{w}ar-líko: \hld\ iu is \alst{w}is-kumo &
\alst{d}uom-\alst{d}ag þe márjo \hld\ ęndi iuwes \alst{d}rohtines kraft, &
þiu \alst{m}ikilo \alst{m}ęgin-strengi \hld\ ęndi þiu \alst{m}árje tíd, &
gi·\alst{w}and þesaro \alst{w}er-oldes. \hld\ Fora þiu gi \alst{w}ardon skulun, &
þat hé iu \alst{s}lápandje \hld\ an \alst{s}wef-restu &
\alst{f}árungo ni bi·\alst{f}áhe \hld\ an \alst{f}irin-werkun, &
\alst{m}ênes fulle. \hld\ \alst{M}út-spelli kumit &
an \alst{þ}iustrja naht, \hld\ al só \alst{þ}iof fęrid &
\alst{d}arno mid is \alst{d}ádjun, \hld\ só kumid þe \alst{d}ag mannun, &
þe \alst{l}atsto þeses \alst{l}iohtes, \hld\ só it êr þese \alst{l}iudi ni witun, &
só samo só þiu \alst{f}lód deda \hld\ an \alst{f}urn-dagun, &
þe þár mid \alst{l}agu-strômun \hld\ \alst{l}iudi far·tęride &
bi \alst{N}óeas tídjun, \hld\ bi·útan þat ina \alst{n}ęride god &
mid is \alst{h}íwiskja, \hld\ \alst{h}êlag drohtin, &
wið þes \alst{f}lódes \alst{f}arm: \hld\ só warð ôk þat \alst{f}iur kuman &
\alst{h}êt fan \alst{h}imile, \hld\ þat þea \alst{h}ôhon burgi &
umbi \alst{S}odomo land \hld\ \alst{s}wart logna bi·féng &
\alst{g}rim ęndi \alst{g}rádag, \hld\ þat þár n·ênig \alst{g}umono ni gi·nas &
bi·útan \alst{L}oth êno: \hld\ ina ant·\alst{l}êddun þanen &
\alst{d}rohtines ęngilos \hld\ ęndi is \alst{d}ohter twá &
an ênan \alst{b}erg uppen: \hld\ þat ȯðar al \alst{b}rinnandi fiur, &
ia \alst{l}and ia \alst{l}iudi \hld\ \alst{l}ogna far·tęride: &
só \alst{f}árungo warð þat \alst{f}iur kumen, \hld\ só warð êr þe \alst{f}lód só samo: &
só wirðid þe \alst{l}atsto dag. \hld\ For þiu skal allaro \alst{l}iudjo ge·hwi-lik &
\alst{þ}ęnkjan fora þemu \alst{þ}inge; \hld\ þes is \alst{þ}arf mikil &
\alst{m}anno ge·hwi-likumu: \hld\ be·þiu látad iu an iuwan \alst{m}ód sorga.\eva

\bvb TODO.\evb\evg

\bvg\bva[53][4378]%
Hwand só hwan só þat ge·\alst{w}irðid, \hld\ þat \alst{w}aldand Krist, &
\alst{m}ári \alst{m}annes sunu \hld\ mid þeru \alst{m}aht godes, &
\alst{k}umit mid þiu \alst{k}raftu \hld\ \alst{k}uningo ríkjost &
\alst{s}ittjan an is \alst{s}elves maht \hld\ ęndi \alst{s}amod mid imu &
\alst{a}lle þea \alst{ę}ngilos, \hld\ þe þár \alst{u}ppa sind &
\alst{h}êlaga an \alst{h}imile, \hld\ þan skulun þarod \alst{h}ęliðo barn, &
\alst{ę}li-þeoda kuman \hld\ \alst{a}lla te·samne &
\alst{l}ibbjandero \alst{l}iudjo, \hld\ só hwat só io an þesumu \alst{l}iohte warð &
\alst{f}iriho a·\alst{f}ódid. \hld\ Þár hé þemu \alst{f}olke skal, &
allumu \alst{m}an-kunnje \hld\ \alst{m}ári drohtin &
a·\alst{d}êljen aftar iro \alst{d}ádjun. \hld\ Þan skêðid hé þea far·\alst{d}uanan man, &
þea far·\alst{w}arhton \alst{w}eros \hld\ an þea \alst{w}inistron hand: &
só duot hé ôk þea \alst{s}áligon \hld\ an þea \alst{s}wíðeron half; &
\alst{g}rótid hé þan þea \alst{g}ódun \hld\ ęndi im te·\alst{g}ęgnes sprikid: &
„\alst{K}umad gí“, kwiðid hé, „þea þár gi·\alst{k}orene sindun, \hld\ ęndi ant·fȧhad þit \alst{k}raftiga ríki, &
þat \alst{g}óde, þat þár gi·\alst{g}ęrewid stęndid, \hld\ þat þár warð \alst{g}umono barnun &
gi·\alst{w}arht fan þesaro \alst{w}er-oldes ęndje: \hld\ iu havad ge·\alst{w}íhid selvo &
\alst{f}ader allaro \alst{f}iriho barno: \hld\ gí mótun þesaro \alst{f}rumono neotan, &
ge·\alst{w}aldon þeses \alst{w}ídon ríkjas, \hld\ hwand gí oft mínan \alst{w}illjon frumidun, &
ful·\alst{g}éngun mí \alst{g}erno \hld\ ęndi wárun mí iuwaro \alst{g}evo mildje, &
\alst{þ}an ik bi·\alst{þ}wungan was \hld\ \alst{þ}urstu ęndi hungru, &
\alst{f}rostu bi·\alst{f}angan \hld\ efþo an \alst{f}eteron lag, &
bi·\alst{k}lęmmid an \alst{k}arkare: \hld\ oft wurðun mí \alst{k}umana þarod &
\alst{h}elpa fan iuwun \alst{h}andun: \hld\ gí wárun mí an iuwomu \alst{h}ugi mildje, &
\alst{w}ísodun mín \alst{w}erð-liko.“ \hld\ Þan sprikid imu eft þat \alst{w}erod an·gęgin: &
„\alst{F}rô mín þe gódo“, \hld\ kweðat sie, „hwan wári þú bi·\alst{f}angan só, &
be·\alst{þ}wungan an su·likun \alst{þ}arạvun, \hld\ só þú fora þesaru \alst{þ}iod tęlis, &
\alst{m}ahtig \alst{m}ênis? \hld\ Hwan gi·sah þí \alst{m}an ênig &
be·\alst{þ}wungen an su·likun \alst{þ}arạvun? \hld\ Hwat þú haves allaro \alst{þ}iodo gi·wald &
iak só samo þero \alst{m}êðmo, \hld\ þero þe io \alst{m}anno barn &
ge·\alst{w}unnun an þesaro \alst{w}er-oldi.“ \hld\ Þan sprikid im eft \alst{w}aldand god: &
„só hwat só gí \alst{d}ádun“, \hld\ kwiðit hé, „an iuwes \alst{d}rohtines namon, &
\alst{g}ódes far·\alst{g}ávun \hld\ an \alst{g}odes êra &
þem \alst{m}annun, þe hér \alst{m}inniston sindun, \hld\ þero nu undar þesaru \alst{m}ęnegi standad &
ęndi þurh \alst{ô}d-módi \hld\ \alst{a}rme wárun &
\alst{w}eros, hwand sie mínan \alst{w}illjon fręmidun \hld\ —só hwat só gí im iuwaro \alst{w}elono far·gávun, &
gi·\alst{d}ádun þurh \alst{d}iuriða, \hld\ þat ant·féng iuwa \alst{d}rohtin selvo, &
þiu \alst{h}elpe kwam te \alst{h}evan-kuninge. \hld\ Be·þiu wili iu þe \alst{h}êlago drohtin &
\alst{l}ônon iuwan gi·\alst{l}ôvon: \hld\ givid iu \alst{l}íf êwig.“ &
\alst{W}ęndid ina þan \alst{w}aldand \hld\ an þea \alst{w}inistron hand, &
\alst{d}rohtin te þem far·\alst{d}uanun mannun, \hld\ sagad im þat sie skulin þea \alst{d}ád ant·gelden, &
þea \alst{m}an iro \alst{m}ên-gi·werk: \hld\ „nu gí fan \alst{m}í skulun“, kwiðit hé, &
„\alst{f}aran só for·\alst{f}lókane \hld\ an þat \alst{f}iur êwig, &
þat þár gi·\alst{g}arewid warð \hld\ \alst{g}odes and-sakun, &
\alst{f}íundo \alst{f}olke \hld\ be \alst{f}irin-werkun, &
\alst{h}wand gí mí ni \alst{h}ulpun, \hld\ þan mí \alst{h}unger ęndi þurst &
\alst{w}êgde te \alst{w}undrun \hld\ efþa ik ge·\alst{w}ádjes lôs &
\alst{g}éng \alst{j}ámer-mód, \hld\ was mí \alst{g}rôtun þarf, &
þan ni habde ik þár ênige \alst{h}elpe, \hld\ þan ik ge·\alst{h}ęftid was, &
an \alst{l}iðo-kospun bi·\alst{l}okan, \hld\ efþa mi \alst{l}egar bi·féng, &
\alst{s}wára \alst{s}uhti: \hld\ þan ni weldun gí mín \alst{s}iokes þár &
\alst{w}íson mid \alst{w}ihti: \hld\ ni was iu \alst{w}erð eo·wiht, &
þat gí mín ge·\alst{h}ugdin. \hld\ Be·þiu gí an \alst{h}ęllje skulun &
\alst{þ}olon an \alst{þ}iustre.“ \hld\ Þan sprikid imu eft þiu \alst{þ}iod an·gęgin: &
„\alst{W}ola \alst{w}aldand god“, \hld\ kweðad sie, „hwí wilt þú só wið þit \alst{w}erod sprekan, &
\alst{m}ahljen wið þese \alst{m}ęnegi? \hld\ Hwan was þí io \alst{m}anno þarf, &
\alst{g}umono \alst{g}ódes? \hld\ Hwat sie it al be þínun \alst{g}evun êgun, &
\alst{w}elon an þesaro \alst{w}er-oldi“. \hld\ Þan sprikid eft \alst{w}aldand god: &
„þan gí þea \alst{a}rmostun“, \hld\ kwiðid hé, „\alst{ę}ldi-barno, &
\alst{m}anno þea \alst{m}inniston \hld\ an iuwomu \alst{m}ód-sevon &
\alst{h}ęliðos far·\alst{h}ugdun, \hld\ létun sea iu an iuwomu \alst{h}ugi lêðe, &
be·\alst{d}êldun sie iuwaro \alst{d}iurða, \hld\ þan dádun gí iuwana \alst{d}rohtin só sama, &
gi·\alst{w}ęrnidun imu iuwaro \alst{w}elono: \hld\ be·þiu ni wili iu \alst{w}aldand god, &
ant·\alst{f}áhen \alst{f}ader iuwa, \hld\ ak gí an þat \alst{f}iur skulun, &
an þene \alst{d}iopun \alst{d}ôð, \hld\ \alst{d}iuvlun þionon, &
\alst{w}rêðun \alst{w}iðer-sakun, \hld\ hwand gí só \alst{w}arhtun bi·foran.“ &
Þan aftar þem \alst{w}ordun skêðit \hld\ þat \alst{w}erod an twê, &
þea \alst{g}ódun ęndi þea uvilon: \hld\ farad þea far·\alst{g}riponon man &
an þea \alst{h}êtan \alst{h}ęl \hld\ \alst{h}riwig-móde, &
þea far·\alst{w}arhton \alst{w}eros, \hld\ \alst{w}íti ant·fȧhat, &
\alst{u}vil \alst{ę}ndi-lôs. \hld\ Lêdid \alst{u}p þanen &
\alst{h}êr \alst{h}evan-kuning \hld\ þea \alst{h}luttạron þeoda &
an þat \alst{l}ang-same \alst{l}ioht: \hld\ þár is \alst{l}íf êwig, &
gi·\alst{g}arewid \alst{g}odes ríki \hld\ \alst{g}ódaro þiado.“\eva

\bvb TODO.\evb\evg

\bvg\bva[54][4452]%
Só ge·fragn ik þat þem \alst{r}inkun þȯ \hld\ \alst{r}íki drohtin &
umbi þesaro \alst{w}er-oldes gi·\alst{w}and \hld\ \alst{w}ordun talde, &
hwó þiu \alst{f}orð \alst{f}ęrid, \hld\ þan lango þe sie \alst{f}iriho barn &
\alst{a}rdon mótun, \hld\ ia hwó siu an þemu \alst{ę}ndje skal &
te·\alst{g}líden ęndi te·\alst{g}angen. \hld\ hé sagde ôk is \alst{j}ungarun þár &
\alst{w}árun \alst{w}ordun: \hld\ „Hwat gí \alst{w}itun alle“, kwað hé, &
„þat nu ovar \alst{t}wá naht \hld\ sind \alst{t}ídi kumana, &
\alst{J}udeono paskha, \hld\ þat sie skulun iro \alst{g}ode þionon, &
\alst{w}eros an þemu \alst{w}íhe. \hld\ Þes nis ge·\alst{w}and ênig, &
þat þár wirðid \alst{m}annes sunu \hld\ te þeru \alst{m}ęgin-þiodu &
\alst{k}raftag far·\alst{k}ôpot \hld\ ęndi an \alst{k}rúke a·slagan, &
\alst{þ}olod \alst{þ}iad-kwála.“ \hld\ Þȯ warð þár \alst{þ}egạn manag &
\alst{s}líð-mód gi·\alst{s}amnod, \hld\ \alst{s}u̇ðar-liudjo, &
\alst{J}udeono \alst{g}um-skępi, \hld\ þár sie skoldun iro \alst{g}ode þionon. &
wurðun \alst{ê}o-sagon \hld\ \alst{a}lle kumane, &
an \alst{w}arf \alst{w}eros, \hld\ þe sie þȯ \alst{w}ísostun &
undar þeru \alst{m}ęnegi \hld\ \alst{m}anno taldun, &
\alst{k}raftag \alst{k}uni-burd. \hld\ Þár \alst{K}aiphas was, &
\alst{b}iskop þero liudjo. \hld\ Sie rédun þȯ an þat \alst{b}arn godes, &
hwó sie ina a·\alst{s}luogin \hld\ \alst{s}undja lôsan, &
kwáðun þat sie ina an þemu \alst{h}êlagon daga \hld\ \alst{h}rínen ni skoldin &
undar þero \alst{m}anno \alst{m}ęnegi, \hld\ „þat ni werðe þius \alst{m}ęgin-þioda, &
\alst{h}ęliðos an \alst{h}róru, \hld\ hwand ina þit \alst{h}ęri-skępi wili &
far·\alst{st}anden mid \alst{st}rídu. \hld\ Wí só \alst{st}illo skulun &
\alst{f}rêson is \alst{f}erạhes, \hld\ þat þit \alst{f}olk Judeono &
an þesun \alst{w}íh-dagun \hld\ \alst{w}róht ni af·hębbjen.“ &
Þȯ géng imu þár \alst{J}údas forð, \hld\ \alst{j}ungaro Kristes, &
\alst{ê}n þero twe-livjo, \hld\ þár þat \alst{a}ðali sat, &
\alst{J}udeono \alst{g}um-skępi; \hld\ kwað þat hé is im \alst{g}ódan rád &
\alst{s}ęggjan mahti: \hld\ „hwat willjad gí mí \alst{s}ęlljen hér“, kwað hé, &
„\alst{m}êðmo te \alst{m}édu, \hld\ ef ik iu þene \alst{m}an givu &
áno \alst{w}íg ęndi áno \alst{w}róht?“ \hld\ Þȯ warð þes \alst{w}erodes hugi, &
þero \alst{l}iudjo an \alst{l}ustun: \hld\ „ef þú wili gi·\alst{l}êstjen só“, kwáðun sie, &
„þín \alst{w}ord gi·\alst{w}áron, \hld\ þan þú gi·\alst{w}ald haves, &
hwat þú at \alst{þ}esaru \alst{þ}iodu \hld\ \alst{þ}iggjan willjes &
\alst{g}ódaro mêðmo.“ \hld\ Þȯ gi·hét imu þat \alst{g}um-skępi þár &
an is \alst{s}elves dóm \hld\ \alst{s}ilụvar-skatto &
\alst{þ}rí-tig at·samne, \hld\ ęndi hé te þeru \alst{þ}iodu gi·sprak &
\alst{d}ęrẹvjun wordun, \hld\ þat hé gávi is \alst{d}rohtin wið þiu. &
\alst{w}ende ina þȯ fan þemu \alst{w}erode: \hld\ was im \alst{w}rêð hugi, &
\alst{t}alode im só \alst{t}reu-lôs, \hld\ hwan êr wurði imu þiu \alst{t}íd kuman, &
þat hé ina mahti far·\alst{w}ísjen \hld\ \alst{w}rêðaro þiodo, &
\alst{f}íundo \alst{f}olke. \hld\ Þan wisse þat \alst{f}riðu-barn godes, &
\alst{w}ár \alst{w}aldand Krist, \hld\ þat hé þese \alst{w}er-old skolde, &
a·\alst{g}even þese \alst{g}ardos \hld\ ęndi sókjen imu \alst{g}odes ríki, &
gi·\alst{f}aren is \alst{f}ader-óðil. \hld\ Þȯ ni gi·sah ênig \alst{f}iriho barno &
\alst{m}êron \alst{m}innje, \hld\ þan hé þȯ te þem \alst{m}annun gi·nam, &
te þem is \alst{g}ódun \alst{j}ungaron: \hld\ \alst{g}ôme warhte, &
\alst{s}ętte sie \alst{s}wás-líko \hld\ ęndi im \alst{s}agde filu &
\alst{w}ároro \alst{w}ordo. \hld\ Skrêd \alst{w}estẹr dag, &
\alst{s}unne te \alst{s}edle. \hld\ Þȯ hé \alst{s}elvo gi·bôd, &
\alst{w}aldand mid is \alst{w}ordun, \hld\ hét im \alst{w}ater dragan &
\alst{h}luttạr te \alst{h}andun, \hld\ ęndi rês þȯ þe \alst{h}êlago Krist, &
þe \alst{g}ódo at þem \alst{g}ômun \hld\ ęndi þár is \alst{j}ungarono þwóg &
\alst{f}óti mid is \alst{f}olmun \hld\ ęndi swarf sie mid is \alst{f}anon aftar, &
\alst{d}ruknide sie \alst{d}iur-líka. \hld\ Þȯ wið is \alst{d}rohtin sprak &
\alst{S}ímon Petrus: \hld\ „Ni þunkid mí þit \alst{s}ómi þing“, kwað hé, &
„\alst{f}rô mín þe gódo, \hld\ þat þú míne \alst{f}óti þwahes &
mid þem þínun \alst{h}êlagun \alst{h}andun.“ \hld\ Þȯ sprak imu eft is \alst{h}êrro an·gęgin, &
\alst{w}aldand mid is \alst{w}ordun: \hld\ „Ef þú is \alst{w}illjan ni haves“, kwað hé, &
„te ant·\alst{f}ȧhanne, \hld\ þat ik þíne \alst{f}óti þwahe &
þurh su·lika \alst{m}innja, \hld\ só ik þesun ȯðrun \alst{m}annun hér &
\alst{d}óm þurh \alst{d}iurða, \hld\ þan ni haves þú ênigan \alst{d}êl mid mí &
an \alst{h}evan-ríkja.“ \hld\ \alst{H}ugi warð þȯ gi·węndid &
\alst{S}ímon Petruse: \hld\ „Þú hava þí \alst{s}elvo gi·wald“, kwað hé, &
„\alst{f}rô mín þe gódo, \hld\ \alst{f}óto ęndi hando &
ęndi mínes \alst{h}ôvdes só sama, \hld\ \alst{h}andun þínun, &
\alst{þ}iadan, te \alst{þ}wahanne, \hld\ te þiu þak ik móti \alst{þ}ína forð &
\alst{h}uldi \alst{h}ębbjan \hld\ ęndi \alst{h}evan-ríkjes &
su·lik gi·\alst{d}êli, \hld\ só þú mí, \alst{d}rohtin, wili &
far·\alst{g}even þurh þína \alst{g}ódi.“ \hld\ \alst{J}ungaron Kristes, &
þene \alst{a}mbaht-skępi \hld\ \alst{e}rlos þolodun, &
\alst{þ}egnos mid gi·\alst{þ}uldjon, \hld\ só hwat só im iro \alst{þ}iodan dede, &
\alst{m}ahtig þurh þea \alst{m}innja, \hld\ ęndi mênde imu al \alst{m}éra þing &
\alst{f}irihon te gi·\alst{f}rummjenne.\eva

\bvb TODO.\evb\evg

\bvg\bva[55][4526]%
\hspace*{100pt}\alst{F}riðu-barn godes &
géng imu þȯ eft gi·\alst{s}ittjen \hld\ under þat ge·\alst{s}ïðo folk &
ęndi im sagda filu \alst{l}ang-samna rád. \hld\ Warð eft \alst{l}ioht kuman, &
\alst{m}orgen te \alst{m}annun. \hld\ \alst{M}ahtigne Krist &
\alst{g}róttun is \alst{j}ungaron ęndi frágodun, \hld\ hwar sie is \alst{g}ôma þȯ &
an þemu \alst{w}íh-dage \hld\ \alst{w}irkjen skoldin, &
\alst{h}war hé weldi \alst{h}alden \hld\ þea \alst{h}êlagon tídi &
\alst{s}elvo mid is ge·\alst{s}ïðun. \hld\ Þȯ hé sie \alst{s}ókjen hét, &
þea \alst{g}umon \alst{J}erusalem: \hld\ „só gí þan \alst{g}angan kumad“, kwað hé, &
„an þea \alst{b}urg innan \hld\ —þár is \alst{b}raht mikil, &
\alst{m}ęgin-þiodo gi·\alst{m}ang—, \hld\ þár mugun gí ênan \alst{m}an sehan &
an is \alst{h}andun dragen \hld\ \alst{h}luttres watares &
\alst{f}ul mid \alst{f}olmun. \hld\ Þemu gí \alst{f}olgon skulun &
an só hwi-like \alst{g}ardos, \hld\ só gí ina \alst{g}angan gi·sehat, &
ia gí þan þemu \alst{h}êrron, \hld\ þe þie \alst{h}ovos êgi, &
\alst{s}elvon \alst{s}ęggjad, \hld\ þat ik iu \alst{s}ęnde þarod &
te gi·\alst{g}aruwenne mína \alst{g}ôma. \hld\ Þan tôgid hé iu ên \alst{g}ód-lík hús, &
\alst{h}ôhan sóleri, \hld\ þe is bi·\alst{h}angen al &
\alst{f}agạrun \alst{f}ratahun. \hld\ Þár gí \alst{f}rummjen skulun &
\alst{w}erd-skępi mínan. \hld\ Þár bium ik \alst{w}is-kumo &
\alst{s}elvo mid mínun ge·\alst{s}ïðun.“ \hld\ Þȯ wurðun \alst{s}án aftar þiu &
þár te \alst{J}erusalem \hld\ \alst{j}ungaron Kristes &
\alst{f}orð-ward an \alst{f}ęrdi, \hld\ \alst{f}undun all só hé sprak &
\alst{w}ord-têkạn \alst{w}ár: \hld\ ni was þes gi·\alst{w}and ênig. &
Þár \alst{g}ęrẹwidun sie þea \alst{g}ôma. \hld\ Warð þe \alst{g}odes sunu, &
\alst{h}êlag drohtin \hld\ an þat \alst{h}ús kuman, &
þár sie þe \alst{l}and-wíse \hld\ \alst{l}êstjen skoldun, &
ful·\alst{g}angan \alst{g}odes gi·bode, \hld\ al só \alst{J}udeono was &
\alst{ê}o ęndi \alst{a}ld-sidu \hld\ an \alst{ê}r-dagun. &
Gi·wêt imu þȯ an þemu \alst{á}vande \hld\ \alst{a}lo-waldand Krist &
an þene \alst{s}ęli \alst{s}ittjen; \hld\ hét þár is ge·\alst{s}ïðos te imu &
\alst{t}we-livi gangan, \hld\ þea im gi·\alst{t}riwiston &
an iro \alst{m}ód-sevon \hld\ \alst{m}anno wárun &
bi \alst{w}ordun ęndi bi \alst{w}ísun: \hld\ \alst{w}isse imu selvo &
iro \alst{h}ugi-skęfti \hld\ \alst{h}êlag drohtin. &
\alst{G}rótte sie þȯ ovar þem \alst{g}ômun: \hld\ „\alst{G}ern bium ik swíðo“, kwað hé, &
„þat ik \alst{s}amad mid iu \hld\ \alst{s}ittjen móti, &
\alst{g}ômono neoten, \hld\ \alst{J}udeono paskha &
\alst{d}êljen mid iu só \alst{d}iurjun. \hld\ Nu ik iu iuwes \alst{d}rohtines skal &
\alst{w}illjon sęggjan, \hld\ þat ik an þesaro \alst{w}er-oldi ni mót &
mid \alst{m}annun \alst{m}êr \hld\ \alst{m}óses an·bíten &
\alst{f}urður mid \alst{f}irihun, \hld\ êr þan gi·\alst{f}ullod wirðid &
\alst{h}imilo ríki. \hld\ Mí is an \alst{h}andun nú &
\alst{w}íti ęndi \alst{w}undẹr-kwále, \hld\ þea ik for þesumu \alst{w}erode skal, &
\alst{þ}olon for þesaru \alst{þ}iodu.“ \hld\ Só hé þȯ só te þem \alst{þ}egnun sprak, &
\alst{h}êlag drohtin, \hld\ só warð imu is \alst{h}ugi dróvi, &
warð imu gi·\alst{s}worken \alst{s}evo, \hld\ ęndi eft te þem ge·\alst{s}ïðun sprak, &
þe \alst{g}ódo te þem is \alst{j}ungarun: \hld\ „Hwat ik iu \alst{g}odes ríki“, kwað hé, &
„gi·\alst{h}ét \alst{h}imiles lioht, \hld\ ęndi gí mí \alst{h}old-líko &
iuwan \alst{þ}egạn-skępi. \hld\ Nú ni willjat gí a·\alst{þ}ęngjan só, &
ak \alst{w}ęnkjat þero \alst{w}ordo. \hld\ Nú sęggju ik iu te \alst{w}áran hér, &
þat wili iuwar \alst{t}we-livjo ên \hld\ \alst{t}rewana swíkan, &
wili mi far·\alst{k}ôpon \hld\ undar þit \alst{k}unni Judeono, &
gi·\alst{s}ęlljen wiðer \alst{s}ilụvre, \hld\ ęndi wili imu þár \alst{s}ink niman, &
\alst{d}iurje mêðmos, \hld\ ęndi geven is \alst{d}rohtin wið þiu, &
\alst{h}oldan \alst{h}êrran. \hld\ Þat imu þoh te \alst{h}arme skal, &
\alst{w}erðan te \alst{w}ítje; \hld\ be þat hé þea \alst{w}urdi far·sihit &
ęndi hé þes \alst{a}rvedjes \hld\ \alst{ę}ndi skawot, &
þan \alst{w}êt hé þat te \alst{w}áran, \hld\ þat imu wári \alst{w}óðjera þing, &
\alst{b}ętera mikilu, \hld\ þat hé gio gi·\alst{b}oran ni wurði &
\alst{l}ibbjendi te þesumu \alst{l}iohte, \hld\ þan hé þat \alst{l}ôn nimid, &
\alst{u}vil \alst{a}rvedi \hld\ \alst{i}n-wid-rádo.“ &
Þȯ bi·gan þero \alst{e}rlo ge·hwi-lik \hld\ te \alst{ȯ}ðrumu skawon, &
\alst{s}orgondi \alst{s}ehan; \hld\ was im \alst{s}êr hugi, &
\alst{h}riwig umbi iro \alst{h}erta: \hld\ gi·hôrdun iro \alst{h}êrron þȯ &
\alst{g}orn-word sprekan. \hld\ Þea \alst{g}umon sorgodun, &
hwi-likan hé þero \alst{t}we-livjo \hld\ te þiu \alst{t}ęlljen weldi, &
\alst{sk}uldigna \alst{sk}aðon, \hld\ þat hé habdi þea \alst{sk}attos þár &
ge·\alst{þ}ingod at þeru \alst{þ}iod. \hld\ Ni was þero \alst{þ}egno ênigumu &
su·likes \alst{i}n-widdjes \hld\ \alst{ó}ði te gehanne, &
\alst{m}ên-gi·þȧhtjo \hld\ —ant·suok þero \alst{m}anno ge·hwi-lik—, &
wurðun alle an \alst{f}orhtun, \hld\ \alst{f}rágon ne gi·dorstun, &
êr þan þȯ ge·\alst{b}ôknide \hld\ \alst{b}ar-wirðig gumo, &
\alst{S}ímon Petrus \hld\ —ne gi·dorste it \alst{s}elvo sprekan— &
te \alst{J}ohanne þemu \alst{g}ódon: \hld\ hé was þemu \alst{g}odes barne &
an \alst{þ}em dagun \hld\ \alst{þ}egno liovost, &
\alst{m}êst an \alst{m}innjun \hld\ ęndi móste þár þȯ an þes \alst{m}ahtiges Kristes &
\alst{b}arme restjen \hld\ ęndi an is \alst{b}reostun lag, &
\alst{h}linode mid is \alst{h}ôvdu: \hld\ þár nam hé só manag \alst{h}êlag ge·rúni, &
\alst{d}iapa gi·þȧhti, \hld\ ęndi þȯ te is \alst{d}rohtine sprak, &
be·gan ina þȯ \alst{f}rágon: \hld\ „hwe skal þat, \alst{f}rô mín, wesen“, kwað hé, &
„þat þi far·\alst{k}ôpon wili, \hld\ \alst{k}uningo ríkjost, &
undar þínaro \alst{f}íundo \alst{f}olk? \hld\ Ús wári þes \alst{f}iri-wit mikil, &
\alst{w}aldand, te \alst{w}itanne.“ \hld\ Þȯ habde eft is \alst{w}ord garu &
\alst{h}êljando Krist: \hld\ „seh þi, hwemu ik hér an \alst{h}and geve &
mínes \alst{m}óses for þesun \alst{m}annun: \hld\ þe haved \alst{m}ên-gi·þȧht, &
\alst{b}irid \alst{b}ittran hugi; \hld\ þe skal mi an \alst{b}anono ge·wald, &
\alst{f}íundun bi·\alst{f}elhen, \hld\ þár man mínes \alst{f}erhes skal, &
\alst{a}ldres \alst{á}htjen.“ \hld\ Nam hé þȯ \alst{a}ftar þiu &
þes \alst{m}óses for þem \alst{m}annun \hld\ ęndi gaf is þemu \alst{m}ên-skaðen, &
\alst{J}udase an hand \hld\ ęndi imu te·\alst{g}ęgnes sprak &
\alst{s}elvo for þem is ge·\alst{s}ïðun \hld\ ęndi ina \alst{s}niumo hét &
\alst{f}aran fan þemu is \alst{f}olke: \hld\ „\alst{f}rumi só þú þęnkis“, kwað hé, &
„\alst{d}ó þat þú \alst{d}uan skalt: \hld\ þú ni maht bi·\alst{d}ęrnjen lęng &
\alst{w}illjon þínan. \hld\ Þiu \alst{w}urd is at handun, &
þea \alst{t}ídi sind nu gi·náhid.“ \hld\ Só þȯ þe \alst{t}reu-logo &
þat \alst{m}ós ant·féng \hld\ ęndi mid is \alst{m}u̇ðu an·bêt, &
só af·\alst{g}af ina þȯ þiu \alst{g}odes kraft, \hld\ \alst{g}ramon in ge·witun &
an þene \alst{l}ík-hamon, \hld\ \alst{l}êða wihti, &
warð imu \alst{S}atanas \hld\ \alst{s}êro bi·tęngi, &
\alst{h}ardo umbi is \alst{h}erte, \hld\ sïður ine þiu \alst{h}elpe godes &
far·\alst{l}ét an þesumu \alst{l}iohte. \hld\ Só is þena \alst{l}iudjo wê, &
þe só undar þesumu \alst{h}imile skal \hld\ \alst{h}êrron wehslon.\eva

\bvb TODO.\evb\evg

\bvg\bva[56][4629]%
Gi·wêt imu þȯ \alst{ú}t þanen \hld\ \alst{i}n-widjas gern &
\alst{J}udas \alst{g}angan: \hld\ habde imu \alst{g}rimmen hugi &
\alst{þ}egạn wið is \alst{þ}iodan. \hld\ Was þȯ iu \alst{þ}iustri naht, &
\alst{s}wíðo gi·\alst{s}worken. \hld\ \alst{S}unu drohtines &
was ima at þem \alst{g}ômun forð \hld\ ęndi is \alst{j}ungarun þár &
\alst{w}aldand \alst{w}ín ęndi brôd \hld\ \alst{w}íhide bêðju, &
\alst{h}êlagode \alst{h}evan-kuning, \hld\ mid is \alst{h}andun brak, &
\alst{g}af it undar þem is \alst{j}ungarun \hld\ ęndi \alst{g}ode þankode, &
sagde þem \alst{á}-lát, \hld\ þe þár \alst{a}l gi·skóp, &
\alst{w}er-old ęndi \alst{w}unnja, \hld\ ęndi sprak \alst{w}ord manag: &
„gi·\alst{l}ôvjot gí þes \alst{l}iohto“, \hld\ kwað hé, „þat þit is mín \alst{l}ík-hamo &
ęndi mín \alst{b}lód só same: \hld\ givu ik iu hér \alst{b}êðju samad &
\alst{e}tan ęndi drinkan. \hld\ Þit ik an \alst{e}rðu skal &
\alst{g}evan ęndi \alst{g}eotan \hld\ ęndi iu te \alst{g}odes ríkje &
\alst{l}ôsjen mid mínu \alst{l}ík-hamen \hld\ an \alst{l}íf êwig, &
an þat \alst{h}imiles lioht. \hld\ Gi·\alst{h}uggjat gí simlun, &
þat \alst{g}í þiu ful·\alst{g}angan, \hld\ þiu ik an þesun \alst{g}ômun dón; &
\alst{m}árjad þit for \alst{m}ęnegi: \hld\ þit is \alst{m}ahtig þing, &
mid þius skulun gí iuwomu \alst{d}rohtine \hld\ \alst{d}iuriða frummjen, &
\alst{h}abbjad þit mín te gi·\alst{h}ugdjun, \hld\ \alst{h}êlag biliði, &
þat it \alst{ę}ldi-barn \hld\ \alst{a}ftar lêstjen, &
\alst{w}aron an þesaru \alst{w}er-oldi, \hld\ þat þat \alst{w}itin alle, &
\alst{m}an ovar þesan \alst{m}iddil-gard, \hld\ þat it is þurh mína \alst{m}innja gi·duan &
\alst{h}êrron te \alst{h}uldi. \hld\ Ge·\alst{h}uggjad gí simlun, &
hweo ik iu hér ge·\alst{b}iudu, \hld\ þat gí iuwan \alst{b}róðer-skępi &
\alst{f}asto \alst{f}rummjad: \hld\ habbjad \alst{f}erhtan hugi, &
\alst{m}innjod iu an iuwomu \alst{m}óde, \hld\ þat þat \alst{m}anno barn &
\alst{o}var \alst{i}rmin-þiod \hld\ \alst{a}lle far·standen, &
þat \alst{g}í sind \alst{g}egnungo \hld\ \alst{j}ungaron míne. &
Ôk skal ik iu \alst{k}u̇ðjen, \hld\ hwó hér wili \alst{k}raftag fíund, &
\alst{h}ęttjand \alst{h}eru-grim, \hld\ umbi iuwan \alst{h}ugi niusjen, &
\alst{S}atanas \alst{s}elvo: \hld\ hé kumid iuwaro \alst{s}eolono herod &
\alst{f}rókno \alst{f}rêson. \hld\ Simlun gí \alst{f}asto te gode &
\alst{b}erad iuwa \alst{b}reost-gi·þȧht: \hld\ ik skal an iuwaru \alst{b}edu standen, &
þat iu ni \alst{m}ugi þe \alst{m}ên-skaðo \hld\ \alst{m}ód ge·twífljan; &
ik \alst{f}ul-lêstju iu wiðer þemu \alst{f}íunde. \hld\ Ôk kwam hé herod giu \alst{f}rêson mín, &
þoh imu is \alst{w}illjon hér \hld\ \alst{w}iht ne gi·stódi, &
\alst{l}ioves an þemu mínumu \alst{l}ík-hamon. \hld\ Nu ni willju ik iu \alst{l}ęng helen, &
hwat iu hér nú \alst{s}niumo skal \hld\ te \alst{s}orgu gi·standen: &
gí skulun mí ge·\alst{s}wíkan, \hld\ ge·\alst{s}ïðos míne, &
iuwes \alst{þ}egạn-skępjes, \hld\ êr þan þius \alst{þ}iustrje naht &
\alst{l}iudi far·\alst{l}íða \hld\ ęndi eft \alst{l}ioht kume, &
\alst{m}organ te \alst{m}annun.“ \hld\ Þȯ warð \alst{m}ód gumon &
\alst{s}wíðo gi·\alst{s}worken \hld\ ęndi \alst{s}êr hugi, &
\alst{h}riwig umbi iro \alst{h}erte \hld\ ęndi iro \alst{h}êrron word &
\alst{s}wíðo an \alst{s}orgun. \hld\ \alst{S}ímon Petrus þȯ, &
\alst{þ}egạn wið is \alst{þ}iodan \hld\ \alst{þ}ríst-wordun sprak &
bí \alst{h}uldi *wið is \alst{h}êrron: \hld\ „þoh þí all þit \alst{h}ęliðo folk“, kwaþ-hie, &
„gi·\alst{s}wíkan þína gi·\alst{s}ïðos, \hld\ þoh ik \alst{s}innon mid þí &
at allon \alst{þ}arạvon \hld\ \alst{þ}olojan willju. &
Ik biun \alst{g}aro sinnon, \hld\ ef mi \alst{g}od látið, &
þat ik an þínon \alst{f}ul-lêstje \hld\ \alst{f}asto gi·stande; &
þoh sia þi an \alst{k}arkarjes \hld\ \alst{k}lústron hardo, &
þesa \alst{l}iudi bi·\alst{l}úkan, \hld\ þoh ist mi \alst{l}uttil tweho, &
ne ik an þem \alst{b}ęndjon mid þi \hld\ \alst{b}ídan willje, &
\alst{l}iggjan mid þi só \alst{l}ieven; \hld\ ef sia þínes \alst{l}íves þan &
þuru \alst{ę}ggja níð \hld\ \alst{á}htjan willjad, &
\alst{f}rô mín þie guodo, \hld\ ik givu mín \alst{f}erạh furi þik &
an \alst{w}ápno spil: \hld\ nis mi \alst{w}erð iowiht &
te bi·\alst{m}íðanne, \hld\ só lango só mi \alst{m}ín warod &
\alst{h}ugi ęndi \alst{h}and-kraft.“ \hld\ Þuȯ sprak im eft is \alst{h}êrro an·gęgin: &
„Hwat þú þik bi·\alst{w}ánis“, \hld\ kwaþ-hie, „\alst{w}issaro trewono, &
\alst{þ}rístero \alst{þ}ingo: \hld\ þú havis \alst{þ}egnes hugi, &
\alst{w}illjon guodan. \hld\ Ik mag þi sęggjan, hwó it þoh gi·\alst{w}erðan skal, &
þat þú \alst{w}irðis só \alst{w}êk-muod, \hld\ þoh þú nu ni \alst{w}ánjes só, &
þat þú þínes \alst{þ}iadnes te naht \hld\ \alst{þ}ríwo far·lôgnis &
êr \alst{h}ano-krádi ęndi kwiðis, \hld\ þak ik þín \alst{h}êrro ni sí, &
ak þú far·\alst{m}anst mína \alst{m}und-burd.“ \hld\ Þuȯ sprak eft þie \alst{m}an an·gęgin: &
„ef it gio an \alst{w}er-oldi“, \hld\ kwaþ-hie, „gi·\alst{w}erðan muosti, &
þat ik \alst{s}amad midi þi \hld\ \alst{s}weltan muosti, &
\alst{d}ôjan \alst{d}iur-líko, \hld\ þan ne wurði gio þie \alst{d}ag kuman, &
þat ik þín far·\alst{l}ôgnidi, \hld\ \alst{l}ievo drohtin, &
\alst{g}erno for þeson \alst{J}uðeon.“ \hld\ Þuȯ kwáðun alla þia \alst{j}ungron só, &
þat sia þár an þem \alst{þ}ingon mid im \hld\ \alst{þ}oljan weldin\eva

\bvb TODO.\evb\evg

\bvg\bva[57][4703]%
Þuȯ im eft mid is \alst{w}ordon gi·bôd \hld\ \alst{w}aldand selvo, &
\alst{h}êr \alst{h}evan-kuning, \hld\ þat sia im ni lietin iro \alst{h}ugi twífljan, &
hiet þat sia ni weldin {[...]} \hld\ \alst{d}iopa gi·þȧhti: &
„Ne \alst{d}ruovje iuwa herta \hld\ þuru iuwes \alst{d}rohtines word, &
ne \alst{f}orọhtjat te filo: \hld\ ik skal \alst{f}ader u̇san &
\alst{s}elvan \alst{s}uokjan \hld\ ęndi iu \alst{s}ęndjan skal &
fan \alst{h}evan-ríkje \hld\ \alst{h}êlagna gêst: &
þie skal iu eft gi·\alst{f}ruofrjan \hld\ ęndi te \alst{f}rumu werðan, &
\alst{m}anon iu þero \alst{m}ahlo, \hld\ þie ik iu \alst{m}anag hębbju &
\alst{w}ordon gi·\alst{w}ísid. \hld\ Hie givit iu gi·\alst{w}it an briost, &
\alst{l}ust-sama \alst{l}êra, \hld\ þat gi \alst{l}êstjan forð &
þiu \alst{w}ord ęndi þiu \alst{w}erk, \hld\ þia ik iu an þesaro \alst{w}er-oldi gi·bôd.“ &
A·\alst{r}ês im þuȯ þe \alst{r}íkjo \hld\ an þemo \alst{r}akode innan, &
\alst{n}ęrjendo Krist \hld\ ęndi gi·wêt im \alst{n}ahtes þanan &
\alst{s}elvo mid is gi·\alst{s}ïðon: \hld\ \alst{s}êrago géngun &
swíðo \alst{g}ornondja \hld\ \alst{j}ungron Kristes, &
\alst{h}riwig-muoda. \hld\ Þuȯ hie im an þena \alst{h}ôhan gi·wêt &
\alst{O}liueti-berg: \hld\ þár was hie \alst{u}p gi·wuno &
\alst{g}angan mid is \alst{j}ungron. \hld\ Þat wissa \alst{J}udas wel, &
\alst{b}alo-hugdig man, \hld\ hwand hie was oft an þem \alst{b}erẹge mid im. &
Þár \alst{g}ruotta þie \alst{g}odes suno \hld\ \alst{j}u̇gron sína: &
„Gí sind nú só \alst{d}ruovja“, \hld\ kwaþ-hie, „nú gí mínan \alst{d}ôð witun; &
nu \alst{g}ornonð gí ęndi \alst{g}riotand, \hld\ ęndi þesa \alst{J}uðeon sind an luston, &
\alst{m}ęndit þius \alst{m}ęnigi, \hld\ sindun an iro \alst{m}uode fráha, &
þius \alst{w}er-old ist an \alst{w}unnjon. \hld\ Þes wirðit þoh gi·\alst{w}and kuman &
\alst{s}niumo tulgo: \hld\ þan wirðit im \alst{s}êr hugi, &
þan \alst{m}ornjat sia an iro \alst{m}óde, \hld\ ęndi gi \alst{m}ęndjan skulun &
\alst{a}fter te \alst{ê}won-dage, \hld\ hwand gio \alst{ę}ndi ni kumið, &
iuwes \alst{w}el-líves gi·\alst{w}and: \hld\ be·þiu ne þurvun iu þius \alst{w}erk tregan, &
\alst{h}rewan mín \alst{h}in-fard, \hld\ hwand þanan skal þiu \alst{h}elpa kuman &
\alst{g}umono barnon.“ \hld\ Þuȯ hiet hie is \alst{j}ungron þár &
\alst{b}ídan uppan þemo \alst{b}erge, \hld\ kwað þat hie ti \alst{b}edu weldi &
an þiu \alst{h}olm-klivu \hld\ \alst{h}ôhor stígan; &
hiet þuȯ \alst{þ}ria mid im \hld\ \alst{þ}egnos gangan, &
\alst{J}akobe ęndi \alst{J}ohannese \hld\ ęndi þena \alst{g}uodan Petruse, &
\alst{þ}ríst-muodjan \alst{þ}egạn. \hld\ Þuȯ sia mid iro \alst{þ}iedne samad &
\alst{g}erno \alst{g}éngun. \hld\ Þuȯ hiet sia þie \alst{g}odes suno &
an \alst{b}erge uppan \hld\ te \alst{b}edu hnígan, &
hiet sia \alst{g}od \alst{g}ruotjan, \hld\ *\alst{g}erno biddjan, &
þat hé im þero \alst{k}ostondero \hld\ \alst{k}raft far·stódi, &
\alst{w}rêðaro \alst{w}illjon, \hld\ þat im þe \alst{w}iðer-sako, &
ni \alst{m}ahti þe \alst{m}ên-skaðo \hld\ \alst{m}ód gi·twífljan, &
iak imu þȯ \alst{s}elvo gi·hnêg \hld\ \alst{s}unu drohtines &
\alst{k}raftag an \alst{k}nio-beda, \hld\ \alst{k}uningo ríkjost, &
\alst{f}orð-ward te \alst{f}oldu: \hld\ \alst{f}ader alo-þiado &
\alst{g}ódan \alst{g}rótte, \hld\ \alst{g}orn-wordun sprak &
\alst{h}riwig-líko: \hld\ was imu is \alst{h}ugi dróvi, &
bi þeru \alst{m}ęnniski \hld\ \alst{m}ód gi·hrórid, &
is \alst{f}lêsk was an \alst{f}orhtun: \hld\ \alst{f}ellun imo trahni, &
\alst{d}rôp is \alst{d}iur-lík swêt, \hld\ al só \alst{d}rôr kumid &
\alst{w}allan fan \alst{w}undun. \hld\ Was an ge·\alst{w}inne þȯ &
an þemu \alst{g}odes barne \hld\ þe \alst{g}êst ęndi þe lík-hamo: &
ȯðar was \alst{f}u̇sid \hld\ an \alst{f}orð-wegos, &
þe \alst{g}êst an \alst{g}odes ríki, \hld\ ȯðar \alst{j}ámar stód, &
\alst{l}ík-hamo Kristes: \hld\ ni welde þit \alst{l}ioht a·geven, &
ak \alst{d}róvde for þemu \alst{d}ôðe. \hld\ Simla hé hreop te \alst{d}rohtine forð &
þiu \alst{m}êr aftar þiu \hld\ \alst{m}ahtigna grótte, &
\alst{h}ôhan \alst{h}imil-fader, \hld\ \alst{h}êlagna god, &
\alst{w}aldand mid is \alst{w}ordun: \hld\ „ef nu \alst{w}erðen ni mag“, kwað hé, &
„\alst{m}an-kunni ge·nęrid, \hld\ ne sí þat ik \alst{m}ínan geve &
\alst{l}iovan \alst{l}ík-hamon \hld\ for \alst{l}iudjo barn &
te \alst{w}êgjanne te \alst{w}undrun, \hld\ it sí þan þín \alst{w}illjo só, &
ik willju is þan gi·\alst{k}oston: \hld\ ik nimu þene \alst{k}ęlik an hand, &
\alst{d}rinku ina þi te \alst{d}iurðu, \hld\ \alst{d}rohtin frô mín, &
\alst{m}ahtig \alst{m}und-boro. \hld\ Ni seh þú \alst{m}ínes hér &
\alst{f}lêskes gi·\alst{f}órjes. \hld\ Ik \alst{f}ullon skal &
\alst{w}illjon þínen: \hld\ þú haves ge·\alst{w}ald ovar al.“ &
Gi·wêt imu þȯ \alst{g}angen, \hld\ þár hé êr is \alst{j}ungaron lét &
\alst{b}ídan uppan þemu \alst{b}erge; \hld\ fand sie þat \alst{b}arn godes &
\alst{s}lápen \alst{s}organdje: \hld\ was im \alst{s}êr hugi, &
þes sie fan iro \alst{d}rohtine \hld\ \alst{d}êljen skoldun. &
Só sind þat \alst{m}ód-þraka \hld\ \alst{m}anno ge·hwi-likumu, &
þat hé far·\alst{l}áten skal \hld\ \alst{l}iavane hêrron, &
af·\alst{g}even þene só \alst{g}ódene. \hld\ Þȯ hé te is \alst{j}ungarun sprak, &
\alst{w}ahte sie \alst{w}aldand \hld\ ęndi \alst{w}ordun grótte: &
„Hwí willjad gi \alst{s}ó \alst{s}lápen?“ \hld\ kwað hé; „ni mugun \alst{s}amad mid mí &
\alst{w}akon êne tíd? \hld\ Þiu \alst{w}urd is at handun, &
þat it só gi·\alst{g}angen skal, \hld\ só it \alst{g}od fader &
gi·\alst{m}arkode \alst{m}ahtig. \hld\ Mí nis an mínumu \alst{m}óde tweho: &
mín \alst{g}êst is \alst{g}aru \hld\ an \alst{g}odes willjan, &
\alst{f}u̇s te \alst{f}aranne: \hld\ mín \alst{f}lêsk is an sorgun, &
\alst{l}ętid mik mín \alst{l}ík-hamo: \hld\ \alst{l}êð is imu swíðo &
\alst{w}íti te þolonne. \hld\ Ik þoh \alst{w}illjan skal &
mínes \alst{f}ader ge·\alst{f}rummjen; \hld\ hębbjad gi \alst{f}asten hugi.“ &
Gi·wêt imu þȯ \alst{e}ft þanan \hld\ \alst{ȯ}ðer-sïðu &
an þene \alst{b}erg uppen \hld\ te \alst{b}edu gangan, &
\alst{m}ári drohtin, \hld\ ęndi þár só \alst{m}anag gi·sprak &
\alst{g}ódoro wordo. \hld\ \alst{G}odes ęngil kwam &
\alst{h}êlag fan \alst{h}imile, \hld\ is \alst{h}ugi fastnode, &
\alst{b}ęldide te þem \alst{b}ęndjun. \hld\ hé was an þeru \alst{b}edu simla &
\alst{f}orð an \alst{f}líte \hld\ ęndi is \alst{f}ader grótte, &
\alst{w}aldand mid is \alst{w}ordun: \hld\ „ef it nu \alst{w}esen ni mag“, kwað hé, &
„\alst{m}ári drohtin, \hld\ nevu ik for þit \alst{m}anno folk &
\alst{þ}iod-kwále \alst{þ}oloje, \hld\ ik an \alst{þ}ínan skal &
\alst{w}illjan \alst{w}onjan.“ \hld\ Gi·\alst{w}êt imu þȯ eft þanen &
\alst{s}ókjan is ge·\alst{s}ïðos: \hld\ fand sie \alst{s}lápandje, &
\alst{g}rótte sie \alst{g}áhun. \hld\ \alst{G}éng imu eft þanen &
\alst{þ}riddjon sïðu te bedu \hld\ ęndi sprak \alst{þ}iod-kuning &
al þiu \alst{s}elvon word, \hld\ \alst{s}unu drohtines, &
te þemu \alst{a}lo-waldon fader, \hld\ só hé \alst{ê}r dede, &
\alst{m}anode \alst{m}ahtigna \hld\ \alst{m}anno frumana &
swíðo \alst{n}iud-líko \hld\ \alst{n}ęrjando Krist, &
\alst{g}éng imu þȯ eft te þem is \alst{j}ungarun, \hld\ \alst{g}rótte sie sáno: &
„\alst{s}lápad gí ęndi ręstjad“, \hld\ kwað hé, „nú wirðid \alst{s}niumo herod &
\alst{k}uman mid \alst{k}raftu, \hld\ þe mi far·\alst{k}ôpot havad, &
\alst{s}undja lôsan gi·\alst{s}ald.“ \hld\ Ge·\alst{s}ïðos Kristes &
\alst{w}akodun þȯ aftar þem \alst{w}ordun \hld\ ęndi gi·sáhun þȯ þat \alst{w}erod kuman &
an þene \alst{b}erg uppen \hld\ \alst{b}rahtmu þiu mikilon, &
\alst{w}rêða \alst{w}ápạn-berand.\eva

\bvb TODO.\evb\evg

\bvg\bva[58][4811]%
\hspace*{100pt} \alst{W}ísde im Judas, &
\alst{g}ram-hugdig man; \hld\ \alst{J}udeon aftar sigun, &
\alst{f}íundo \alst{f}olk-skępi; \hld\ dróg man \alst{f}iur an gi·mang, &
\alst{l}ogna an \alst{l}ioht-fatun, \hld\ \alst{l}êdde man faklon &
\alst{b}rinnandja fan \alst{b}urg, \hld\ þár sie an þene \alst{b}erg uppan &
\alst{st}igun mid \alst{st}rídu. \hld\ Þea \alst{st}ędi wisse Judas wel, &
hwar hé þea \alst{l}iudi \hld\ tó \alst{l}êdjan skolde. &
Sagde imu þȯ te \alst{t}êkne, \hld\ þȯ sie þár \alst{t}ó fórun &
þemu \alst{f}olke bi·\alst{f}oran, \hld\ te þiu þat sie ni far·\alst{f}éngin þár, &
\alst{e}rlos \alst{ȯ}ðren man: \hld\ „ik gangu imu at \alst{ê}rist tó“, kwað hé, &
„\alst{k}ussju ine ęndi \alst{k}waddju: \hld\ þat is \alst{K}rist selvo. &
Þene gi \alst{f}áhen skulun \hld\ \alst{f}olko kraftu, &
\alst{b}inden ina uppan þemu \alst{b}erge \hld\ ęndi ina te \alst{b}urg hinan &
\alst{l}êdjen undar þea \alst{l}iudi: \hld\ hé is \alst{l}íves havad &
mid is \alst{w}ordun far·\alst{w}erkod.“ \hld\ \alst{W}erod sïðode þȯ, &
an-tat sie te \alst{K}riste \hld\ \alst{k}umane wurðun, &
\alst{g}rim folk \alst{J}udeono, \hld\ þár hé mid is \alst{j}ungarun stód, &
\alst{m}ári drohtin: \hld\ bêd \alst{m}etodo-gi·skapu, &
\alst{t}orhtero \alst{t}ídjo. \hld\ Þȯ géng imu \alst{t}reu-lôs man, &
\alst{J}udas te·\alst{g}ęgnes \hld\ ęndi te þemu \alst{g}odes barne &
\alst{h}nêg mid is \alst{h}ôvdu \hld\ ęndi is \alst{h}êrron kwędde, &
\alst{k}uste ina \alst{k}raftagne \hld\ ęndi is \alst{k}widi lêste, &
\alst{w}ísde ina þemu \alst{w}erode, \hld\ al só hé êr mid \alst{w}ordun ge·hét. &
Þat \alst{þ}olode al mid gi·\alst{þ}uldjun \hld\ \alst{þ}iodo drohtin, &
\alst{w}aldand þesara \alst{w}er-oldes \hld\ ęndi sprak imu mid is \alst{w}ordun tó, &
\alst{f}rágode ine \alst{f}rókno: \hld\ „be·hwí kumis þú só mid þius \alst{f}olku te mí, &
be·hwí \alst{l}êdis þú mí só þese \alst{l}iudi tó \hld\ ęndi mi te þesare \alst{l}êðan þiode sprekan, &
far·\alst{k}ôpos mid þínu \alst{k}ussu \hld\ under þit \alst{k}unni Judeono, &
\alst{m}eldos mi te þesaru \alst{m}ęnegi?“ \hld\ Géng imu þȯ wið þea \alst{m}an &
wið þat \alst{w}erod ȯðar \hld\ ęndi sie mid is \alst{w}ordun fragn, &
hwene sie mid þiu ge·\alst{s}ïðju \hld\ \alst{s}ókjan kwámin &
só \alst{n}iud-liko an \alst{n}aht, \hld\ „so gí willjan \alst{n}ôd frummjen &
\alst{m}anno hwi-likumu.“ \hld\ Þȯ sprak imu eft þiu \alst{m}ęnegi an·gęgin, &
kwáðun þat im \alst{h}êljand \hld\ þár an þemu \alst{h}olme uppan &
ge·\alst{w}ísid \alst{w}ári, \hld\ „þe þit gi·\alst{w}er frumid &
\alst{J}udeo liudjun \hld\ ęndi ina \alst{g}odes sunu &
\alst{s}elvon hêtid. \hld\ Ina kwámun wí \alst{s}ókjan herod, &
weldin ina \alst{g}erno bi·\alst{g}eten: \hld\ hé is fan \alst{G}alileo lande, &
fan \alst{N}azareth-burg.“ \hld\ Só im þȯ þe \alst{n}ęrjendjo Krist &
\alst{s}agde te \alst{s}ȯðan, \hld\ þat hé it \alst{s}elvo was, &
só wurðun þȯ an \alst{f}orhtun \hld\ \alst{f}olk Judeono, &
wurðun under·\alst{b}adode, \hld\ þat sie under \alst{b}ak fellun &
\alst{a}lle \alst{e}fno sán, \hld\ \alst{e}rðe gi·sóhtun, &
wiðer·\alst{w}ardes þat \alst{w}erod: \hld\ ni mahte þat \alst{w}ord godes, &
þie \alst{st}emnje ant·\alst{st}andan: \hld\ wárun þoh só \alst{st}rídige man, &
a·\alst{h}liopun eft up an þemu \alst{h}olme, \hld\ \alst{h}ugi fastnodun, &
\alst{b}undun \alst{b}riost-gi·þȧht, \hld\ gi·\alst{b}olgane géngun &
\alst{n}áhor mid \alst{n}íðu, \hld\ ant-tat sie þene \alst{n}ęrjendjon Krist &
\alst{w}erodo bi·\alst{w}urpun. \hld\ Stódun \alst{w}íse man, &
swíðo \alst{g}ornundje \hld\ \alst{j}ungaron Kristes &
bi·foran þeru \alst{d}ęrẹvjon \alst{d}ádi \hld\ ęndi te iro \alst{d}rohtine sprákun: &
„\alst{w}ári it nu þín \alst{w}illjo“, \hld\ kwáðun sie, „\alst{w}aldand frô mín, &
þat sie u̇s hér an \alst{sp}eres ordun \hld\ \alst{sp}ildjen móstin &
\alst{w}ápnun \alst{w}unde, \hld\ þan ni wári u̇s \alst{w}iht só gód, &
só þat wí hér for u̇sumu \alst{d}rohtine \hld\ \alst{d}óan móstin &
\alst{b}ęniðjun \alst{b}lêka“. \hld\ Þȯ gi·\alst{b}olgan warð &
\alst{s}nel \alst{s}werd-þegạn, \hld\ \alst{S}ímon Petrus, &
\alst{w}ell imu innan hugi, \hld\ þat hé ni mahte ênig \alst{w}ord sprekan: &
só \alst{h}arm warð imu an is \alst{h}ertan, \hld\ þat man is \alst{h}êrron þár &
\alst{b}inden welde. \hld\ Þȯ hé gi·\alst{b}olgan géng, &
swíðo \alst{þ}ríst-mód \alst{þ}egạn \hld\ for is \alst{þ}iodan standen, &
\alst{h}ard for is \alst{h}êrron: \hld\ ni was imu is \alst{h}ugi twífli, &
\alst{b}lóð an is \alst{b}reostun, \hld\ ak hé is \alst{b}il a·tôh, &
\alst{s}werd bi \alst{s}ídu, \hld\ \alst{s}lóg imu te·gęgnes &
an þene \alst{f}uriston \alst{f}íund \hld\ \alst{f}olmo krafto, &
þat þȯ \alst{M}alkhus warð \hld\ \alst{m}ákjas ęggjun, &
an þea \alst{s}wíðaron half \hld\ \alst{s}werdu gi·málod: &
þiu \alst{h}lust warð imu far·\alst{h}awan, \hld\ hé warð an þat \alst{h}ôvid wund, &
þat imu \alst{h}eru-drôrag \hld\ \alst{h}lear ęndi ôre &
\alst{b}ęni-wundun \alst{b}rast: \hld\ \alst{b}lód aftar sprang, &
\alst{w}ell fan \alst{w}undun. \hld\ Þȯ was an is \alst{w}angun skard &
þe \alst{f}uristo þero \alst{f}íundo. \hld\ Þȯ stód þat \alst{f}olk an rúm: &
an-drédun im þes \alst{b}illes \alst{b}iti. \hld\ Þȯ sprak þat \alst{b}arn godes &
\alst{s}elvo te \alst{S}ímon Petruse, \hld\ hét þat hé is \alst{s}werd dedi &
\alst{sk}arp an \alst{sk}êðja: \hld\ „ef ik wið þesa \alst{sk}ola weldi“, kwað hé, &
„wið þeses \alst{w}erodes ge·\alst{w}in \hld\ \alst{w}íg-saka frummjen, &
þan \alst{m}anodi ik þene \alst{m}árjon \hld\ \alst{m}ahtigne god, &
\alst{h}êlagne fader \hld\ an \alst{h}imil-ríkja, &
þat hé mi só managan \alst{ę}ngil herod \hld\ \alst{o}vana sandi &
\alst{w}íges só \alst{w}ísen, \hld\ só ni mahtin iro \alst{w}ápạn-þręki &
\alst{m}an a·dôgjan: \hld\ iro ni stódi gio su·lik \alst{m}ęgin samad, &
\alst{f}olkes gi·\alst{f}astnod, \hld\ þat im iro \alst{f}erh aftar þiu &
\alst{w}erðen mahti. \hld\ Ak it havad \alst{w}aldand god, &
\alst{a}lo-mahtig fader \hld\ an \alst{ȯ}ðar gi·markot, &
þat wí gi·\alst{þ}olojan skulun, \hld\ só hwat só u̇s þius \alst{þ}ioda tó &
\alst{b}ittres \alst{b}rengit: \hld\ ni skulun u̇s \alst{b}elgan wiht, &
\alst{w}rêðjan wið iro ge·\alst{w}inne; \hld\ hwand só hwe só \alst{w}ápno níð, &
\alst{g}rimman \alst{g}êr-hęti wili \hld\ \alst{g}erno frummjen, &
hé \alst{s}wiltit imu \hld\ eft \alst{s}werdes ęggjun, &
\alst{d}óit im bi·\alst{d}rôregan: \hld\ wí mid u̇sun \alst{d}ádjun ni skulun &
\alst{w}iht a·\alst{w}ęrdjan.“ \hld\ Géng hé þȯ te þemu \alst{w}undon manne, &
\alst{l}ęgde mid \alst{l}istjun \hld\ \alst{l}ík te·samne, &
\alst{h}ôvid-wundon, \hld\ þat siu sán gi·\alst{h}êlid warð, &
þes \alst{b}illes \alst{b}iti, \hld\ ęndi sprak þat \alst{b}arn godes &
wið þat \alst{w}rêðe \alst{w}erod: \hld\ „mí þunkid \alst{w}undẹr mikil“, kwað hé, &
„ef gí mí \alst{l}êðes wiht \hld\ \alst{l}êstjen weldun, &
hwí gí mí þȯ ni \alst{f}éngun, \hld\ þan ik undar iuwomu \alst{f}olke stód, &
an þemu \alst{w}íhe innan \hld\ ęndi þár \alst{w}ord manag &
\alst{s}ȯð-lík \alst{s}agde. \hld\ Þan was \alst{s}unnon skín, &
\alst{d}iur-lik \alst{d}ages lioht, \hld\ þan ni weldun gí mí \alst{d}óan eo·wiht &
\alst{l}êðes an þesumu \alst{l}iohte, \hld\ ęndi nu lêdjad mí iuwa \alst{l}iudi tó &
an \alst{þ}iustrje naht, \hld\ al só man \alst{þ}iove dót, &
þan man þene \alst{f}ȧhan wili \hld\ ęndi hé is \alst{f}erhes havad &
far·\alst{w}erkot, \alst{w}am-skaðo.“ \hld\ \alst{w}erod Judeono &
\alst{g}ripun þȯ an þene \alst{g}odes sunu, \hld\ \alst{g}rimma þioda, &
\alst{h}atandjero \alst{h}óp, \hld\ \alst{h}wurvun ina umbi &
\alst{m}ódag \alst{m}anno folk \hld\ —\alst{m}ênes ni sáhun—, &
\alst{h}ęftun \alst{h}eru-bęndjun \hld\ \alst{h}andi te·samne, &
\alst{f}aðmos mid \alst{f}iterjun. \hld\ Im ni was su·likaro \alst{f}irin-kwála &
\alst{þ}arf te gi·\alst{þ}olonne, \hld\ \alst{þ}iod-arvedjes, &
te \alst{w}innanne su·lik \alst{w}íti, \hld\ ak hé it þurh þit \alst{w}erod deda, &
hwand hé \alst{l}iudjo barn \hld\ \alst{l}ôsjen welda, &
\alst{h}alon fan \alst{h}ęllju \hld\ an \alst{h}imil-ríki, &
an þene \alst{w}ídon \alst{w}elon: \hld\ be·þiu hé þes \alst{w}iht ne bi·sprak, &
þes sie imu þurh \alst{i}n-wid-níð \hld\ \alst{ó}gjan weldun.\eva%TODO: ógjan or ôgjan?

\bvb TODO.\evb\evg

\bvg\bva[59][4926]%
Þȯ wurðun þes só \alst{m}alske \hld\ \alst{m}ódag folk Judeono, &
þiu \alst{h}êri warð þes só \alst{h}rómeg, \hld\ þes sie þena \alst{h}êlagon Krist &
an \alst{l}iðo-bęndjon \hld\ \alst{l}êdjan muostun, &
\alst{f}órjan an \alst{f}iterjun. \hld\ Þie \alst{f}íund eft ge·witun &
fan þemu \alst{b}erge te \alst{b}urg. \hld\ Géng þat \alst{b}arn godes &
undar þemu \alst{h}ęri-skępi \hld\ \alst{h}andun ge·bunden, &
\alst{d}rúvondi te \alst{d}ale. \hld\ Wárun imu þea is \alst{d}iurjon þȯ &
ge·\alst{s}ïðos ge·\alst{s}wikane, \hld\ al só hé im êr \alst{s}elvo gi·sprak: &
ni was it þoh be ênigaru \alst{b}lóði, \hld\ þat sie þat \alst{b}arn godes, &
\alst{l}ioven far·\alst{l}étun, \hld\ ak it was só \alst{l}ango bi·foren &
\alst{w}ár-sagono \alst{w}ord, \hld\ þat it skoldi gi·\alst{w}erðen só: &
be·þiu ni \alst{m}ahtun sie is be·\alst{m}íðan. \hld\ Þan aftar þeru \alst{m}ęnegi géngun &
\alst{J}ohannes ęndi Petrus, \hld\ þie \alst{g}umon twêne, &
\alst{f}olgodun \alst{f}errane: \hld\ was im \alst{f}iri-wit mikil, &
hwat þea \alst{g}rimmon \alst{J}udeon \hld\ þemu \alst{g}odes barne, &
weldin iro \alst{d}rohtine \alst{d}óen. \hld\ Þȯ sie te \alst{d}ale kwámun &
fan þemu \alst{b}erge te \alst{b}urg, \hld\ þár iro \alst{b}iskop was, &
iro \alst{w}íhes \alst{w}ard, \hld\ þár lêddun ina \alst{w}lanke man, &
\alst{e}rlos undar \alst{e}deros. \hld\ Þár was \alst{ê}ld mikil, &
\alst{f}iur an \alst{f}ríd-hove \hld\ þemu \alst{f}olke te·gęgnes, &
ge·\alst{w}arht for þemu \alst{w}erode: \hld\ þár géngun sie im \alst{w}ęrmjen tó, &
\alst{J}udeo liudi, \hld\ létun þene \alst{g}odes sunu &
\alst{b}ídon an \alst{b}ęndjun. \hld\ Was þár \alst{b}raht mikil, &
\alst{g}êl-módigaro \alst{g}alm. \hld\ \alst{J}ohannes was êr &
þemu \alst{h}êroston \alst{k}u̇ð: \hld\ be·þiu móste hé an þene \alst{h}of innan &
\alst{þ}ringan mid þeru \alst{þ}ioda. \hld\ Stód allaro \alst{þ}egno bętsto, &
\alst{P}etrus þár úte: \hld\ ni lét ina þe \alst{p}ortun ward &
\alst{f}olgon is \alst{f}rôen, \hld\ êr it at is \alst{f}riunde a·bad, &
\alst{J}ohannes at ênumu \alst{J}udeon, \hld\ þat man ina \alst{g}angan lét &
\alst{f}orð an þene \alst{f}ríd-hof. \hld\ Þár kwam im ên \alst{f}êkni wíf &
\alst{g}angan te·\alst{g}ęgnes, \hld\ þiu ênas \alst{J}udeon was, &
iro \alst{þ}eodanes \alst{þ}iw, \hld\ ęndi þȯ te þemu \alst{þ}egne sprak &
\alst{m}agað un·wán-lík: \hld\ „Hwat þú mahtis \alst{m}an wesan“, kwað siu, &
„\alst{j}ungaro fan \alst{G}alilea, \hld\ þes þe þár \alst{g}enower stéd &
\alst{f}aðmun gi·\alst{f}astnod.“ \hld\ Þȯ an \alst{f}orhtun warð &
\alst{S}ímon Petrus \alst{s}án, \hld\ \alst{s}lak an is móde, &
kwað þat hé þes \alst{w}íves \hld\ \alst{w}ord ni bi·konsti &
ni þes \alst{þ}eodanes \hld\ \alst{þ}egạn ni wári: &
\alst{m}êð is þȯ for þeru \alst{m}ęnegi, \hld\ kwað þat hé þena \alst{m}an ni ant·kęndi: &
„ni sind mí þíne \alst{k}widi \alst{k}u̇ðe“, \hld\ kwað hé; was imu þiu \alst{k}raft godes, &
þe \alst{h}ęrdislo fan þemu \alst{h}ertan. \hld\ \alst{H}warạvondi géng &
\alst{f}orð undar þemu \alst{f}olke, \hld\ an-tat hé te þemu \alst{f}iure kwam; &
gi·\alst{w}êt ina þȯ \alst{w}armjen. \hld\ Þár im ôk ên \alst{w}íf bi·gan &
\alst{f}ęlgjan \alst{f}irin-spráka: \hld\ „hér mugun gí“, kwað siu, „an iuwan \alst{f}íund sehan: &
þit is \alst{g}egnungo \hld\ \alst{j}ungaro Kristes, &
is \alst{s}elves ge·\alst{s}ïð.“ \hld\ Þȯ géngun imu \alst{s}án aftar þiu &
\alst{n}áhor \alst{n}íð-hwata \hld\ ęndi ina \alst{n}iud-líko &
\alst{f}rágodun \alst{f}íundo barn, \hld\ hwi-likes hé \alst{f}olkes wári: &
“ni bist þú þesoro \alst{b}urg-liudjo“, \hld\ kwáðun sie; „þat mugun wí an þínumu gi·\alst{b}árje gi·sehan, &
an þínun \alst{w}ordun ęndi an þínaru \alst{w}íson, \hld\ þat þú þeses \alst{w}erodes ni bist, &
ak þú bist \alst{g}aliléisk man.“ \hld\ hé ni welda þes þȯ \alst{g}ehan eo·wiht, &
ak \alst{st}ód þȯ ęndi \alst{st}rídda \hld\ ęndi \alst{st}arkan êð &
\alst{s}wíð-líko ge·\alst{s}wór, \hld\ þat hé þes ge·\alst{s}ïðes ni wári. &
Ni habda is \alst{w}ordo ge·\alst{w}ald: \hld\ it skolde gi·\alst{w}erðen só, &
só it þe ge·\alst{m}arkode, \hld\ þe \alst{m}an-kunnjes &
far·\alst{w}ardot an þesaru \alst{w}er-oldi. \hld\ Þȯ kwam imu ôk an þemu \alst{w}arve tó &
þes \alst{m}annes \alst{m}ág-wini, \hld\ þe hé êr mid is \alst{m}ákjo gi·héw, &
\alst{s}werdu þiu skarpon, \hld\ kwað þat hé ina \alst{s}áhi þár &
an þemu \alst{b}erge uppan, \hld\ „þár wí an þemu \alst{b}ôm-gardon &
\alst{h}êrron þínumu \hld\ \alst{h}ęndi bundun, &
\alst{f}astnodun is \alst{f}olmos.“ \hld\ Hé þȯ þurh \alst{f}orhtan hugi &
for·\alst{l}ôgnide þes is \alst{l}ioves hêrron, \hld\ kwað þat hé weldi wesan þes \alst{l}íves skolo, &
ef it mahti \alst{ê}nig þár \hld\ \alst{i}rmin-manno &
gi·\alst{s}ęggjan te \alst{s}ȯðan, \hld\ þat hé þes ge·\alst{s}ïðes wári, &
\alst{f}olgodi þeru \alst{f}ęrdi. \hld\ Þȯ warð an þena \alst{f}ormon sïð &
\alst{h}ano-krád af·\alst{h}aven. \hld\ Þȯ sah þe \alst{h}êlago Krist, &
\alst{b}arno þat \alst{b}ętste, \hld\ þár hé ge·\alst{b}unden stóð, &
\alst{s}elvo te \alst{S}ímon Petruse, \hld\ \alst{s}unu drohtines &
te þemu \alst{e}rle ovar is \alst{a}hsla. \hld\ Þȯ warð imu an \alst{i}nnan sán, &
\alst{S}ímon Petruse \hld\ \alst{s}êr an is móde, &
\alst{h}arm an is \alst{h}ertan \hld\ ęndi is \alst{h}ugi dróvi, &
\alst{s}wíðo warð imu an \alst{s}orgun, \hld\ þat hé êr \alst{s}elvo ge·sprak: &
gi·hugde þero \alst{w}ordo þȯ, \hld\ þe imu êr \alst{w}aldand Krist &
\alst{s}elvo \alst{s}agda, \hld\ þat hé an þeru \alst{s}wartan naht &
êr \alst{h}ano-krádi \hld\ is \alst{h}êrron skoldi &
\alst{þ}ríwo far·lôgnjen. \hld\ Þes \alst{þ}ram imu an innan mód &
\alst{b}ittro an is \alst{b}reostun, \hld\ ęndi géng imu þȯ gi·\alst{b}olgan þanen &
þe \alst{m}an fan þeru \alst{m}ęnigi \hld\ an \alst{m}ód-karu, &
\alst{s}wíðo an \alst{s}orgun, \hld\ ęndi is \alst{s}elves word, &
\alst{w}am-skęfti \alst{w}eop, \hld\ an-tat imu \alst{w}allan kwámun &
þurh þea \alst{h}ert-kara \hld\ \alst{h}ête trahni, &
\alst{b}lódage fan is \alst{b}reostun. \hld\ hé ni wánde þat hé is mahti gi·\alst{b}ótjen wiht, &
\alst{f}irin-werko \alst{f}urður \hld\ efþa te is \alst{f}râhon kuman, &
\alst{h}êrron \alst{h}uldi: \hld\ nis ênig \alst{h}ęliðo só ald, &
þat io \alst{m}annes sunu \hld\ \alst{m}êr gi·sáhi &
is \alst{s}elves word \hld\ \alst{s}êrur hrewan, &%NOTE: sêrur hrewan checked.
\alst{k}aron efþa \alst{k}úmjen: \hld\ „wola \alst{k}rafteg god“, kwað hé, &
þat ik hębbju mi só for·\alst{w}erkot, \hld\ só ik mínaro \alst{w}er-oldes ni þarf &
\alst{ó}-lát sęggjan. \hld\ Ef ik nu te \alst{a}ldre skal &
\alst{h}uldjo þínaro \hld\ ęndi \alst{h}evan-ríkjas, &
\alst{þ}eoden, \alst{þ}olojan, \hld\ þan ni þarf mi þes ênig \alst{þ}ank wesan, &
\alst{l}iovo drohtin, \hld\ þat ik io te þesumu \alst{l}iohte kwam. &
Ni bium ik nu þes \alst{w}irðig, \hld\ \alst{w}aldand frô mín, &
þat ik under þíne \alst{j}ungaron \hld\ \alst{g}angan móti, &
þus \alst{s}undig under þíne ge·\alst{s}ïðos: \hld\ ik iro \alst{s}elvo skal &
\alst{m}íðan an mínumu \alst{m}óde, \hld\ nu ik mi su·lik \alst{m}ên ge·sprak.“ &
Só \alst{g}ornode \hld\ \alst{g}umono bętsta, &
\alst{h}rau im só \alst{h}ardo, \hld\ þat hé habde is \alst{h}êrren þȯ &
\alst{l}eoves far·\alst{l}ôgnid. \hld\ Þan ni þurvun þes \alst{l}iudjo barn, &
\alst{w}eros \alst{w}undrojan, \hld\ be·hwí it \alst{w}eldi god, &
þat só \alst{l}ioven man \hld\ \alst{l}êð gi·stódi, &
þat hé só \alst{h}ôn-líko \hld\ \alst{h}êrron sínes &
þurh þera \alst{þ}iwun word, \hld\ \alst{þ}egno snellost, &
far·\alst{l}ôgnide só \alst{l}ioves: \hld\ it was al bi þesun \alst{l}iudjun gi·duan, &
\alst{f}iriho barnun te \alst{f}rumu. \hld\ hé welde ina te \alst{f}uriston dóan, &
\alst{h}êrost ovar is \alst{h}íwiski, \hld\ \alst{h}êlag drohtin: &
lét ina ge·\alst{k}unnon, \hld\ hwi-like \alst{k}raft havet &
þe \alst{m}ęnniska \alst{m}ód \hld\ áno þe \alst{m}aht godes; &
lét ina ge·\alst{s}undjon, \hld\ þat hé \alst{s}ïðor þiu bet &
\alst{l}iudjun gi·\alst{l}ôvdi, \hld\ hwó \alst{l}iof is þár &
\alst{m}anno gi·hwi-likumu, \hld\ þan hé \alst{m}ên ge·frumit, &
þat man ina a·\alst{l}áte \hld\ \alst{l}êðes þinges, &
\alst{s}akono ęndi \alst{s}undjono, \hld\ só im þȯ \alst{s}elvo dede &
\alst{h}evan-ríki god \hld\ \alst{h}arm-ge·wurhti.\eva

\bvb TODO.\evb\evg

\bvg\bva[60][5040]%
Be þiu nis \alst{m}annes bág \hld\ \alst{m}ikilun bi·þęrvi, &
\alst{h}agu-staldes \alst{h}róm: \hld\ ef imu þiu \alst{h}elpe godes &
ge·\alst{s}wíkid þurh is \alst{s}undjon, \hld\ þan is imu \alst{s}án aftar þiu &
\alst{b}reost-hugi \alst{b}lóðora, \hld\ þoh hé êr \alst{b}i·hêt spreka, &
\alst{h}rómje fan is \alst{h}ildi \hld\ ęndi fan is \alst{h}and-krafti, &
þe \alst{m}an fan is \alst{m}ęgine. \hld\ Þat warð þár an þemu \alst{m}árjon skín, &
\alst{þ}egno bętston, \hld\ þȯ imu is \alst{þ}iodanes gi·swêk &
\alst{h}êlag \alst{h}elpe. \hld\ Be·þiu ni skoldi \alst{h}rómjen man &
te \alst{s}wíðo fan imu \alst{s}elvon, \hld\ hwand imu þár \alst{s}wíkid oft &
\alst{w}án ęndi \alst{w}illjo, \hld\ ef imu \alst{w}aldand god, &
\alst{h}êr \alst{h}evan-kuning \hld\ \alst{h}erte ni stęrkit. &
Þan bêd allaro \alst{b}arno \alst{b}ętst, \hld\ \alst{b}ęndi þolode &
þurh \alst{m}an-kunni. \hld\ Hwurvun ina \alst{m}anaga umbi &
\alst{J}udeono liudi, \hld\ sprákun \alst{g}elp mikil, &
\alst{h}abdun ina te \alst{h}oska, \hld\ þár hé gi·\alst{h}ęftid stód, &
\alst{þ}olode mid ge·\alst{þ}uldjun, \hld\ só hwat só imu þiu \alst{þ}iod deda, &
\alst{l}iudi \alst{l}êðes. \hld\ Þȯ warð eft \alst{l}ioht kuman, &
\alst{m}organ te \alst{m}annun. \hld\ \alst{M}anag samnoda &
\alst{h}ęri Judeono: \hld\ \alst{h}abdun im \alst{h}ugi wulvo, &
\alst{i}n-wid an \alst{i}nnan. \hld\ Warð þár \alst{ê}o-sago &
an \alst{m}organ-tíd \hld\ \alst{m}anag gi·samnod &
\alst{i}rri ęndi \alst{ê}n-hard, \hld\ \alst{i}n-widjas gern, &
\alst{w}rêðes \alst{w}illjan. \hld\ Géngun im an \alst{w}arf samad &
\alst{r}inkos an \alst{r}úna, \hld\ bi·gunnun im \alst{r}ádan þȯ, &
hwó sie ge·\alst{w}ísadin \hld\ mid \alst{w}ár-lôsun, &
\alst{m}annun \alst{m}ên-ge·witun \hld\ an \alst{m}ahtigna Krist &
te gi·\alst{s}ęggjanne \alst{s}undja \hld\ þurh is \alst{s}elves word, &
þat sie ina þan te \alst{w}undẹr-kwálu \hld\ \alst{w}êgjan móstin, &
a·\alst{d}êljen te \alst{d}ôðe. \hld\ Sie ni mahtun an þemu \alst{d}age finden &
só \alst{w}rêð ge·\alst{w}it-skępi, \hld\ þat sie imu \alst{w}íti be·þiu &
a·\alst{d}êljen gi·\alst{d}orstin \hld\ efþa \alst{d}ôð frummjen, &
\alst{l}ívu bi·\alst{l}ôsjen. \hld\ Þȯ kwámun þár at \alst{l}atstan forð &
an þena \alst{w}arf \alst{w}ero \hld\ \alst{w}ár-lôse man &
\alst{t}wêne gangan \hld\ ęndi bi·gunnun im \alst{t}ęlljen an, &
kwáðun þat sie ina \alst{s}elvon \hld\ \alst{s}ęggjan gi·hôrdin, &
þat hé mahti te·\alst{w}erpen \hld\ þena \alst{w}íh godes, &
allaro \alst{h}úso \alst{h}ôhost \hld\ ęndi þurh is \alst{h}and-męgin, &
þurh is \alst{ê}nes kraft \hld\ \alst{u}p a·rihtjen &
an \alst{þ}riddjon daga, \hld\ só is elkor ni þorfti be·\alst{þ}íhan man. &
Hé \alst{þ}agoda ęndi \alst{þ}oloda: \hld\ ni sprak imu io þiu \alst{þ}iod só filu, &
þea \alst{l}iudi mid \alst{l}uginun, \hld\ þat hé it mid \alst{l}êðun an·gęgin &
\alst{w}ordun \alst{w}ráki. \hld\ Þȯ þár undar þemu \alst{w}erode a·rês &
\alst{b}alu-hugdig man, \hld\ \alst{b}iskop þero liudjo, &
þe \alst{f}uristo þes \alst{f}olkes \hld\ ęndi \alst{f}rágode Krist &
iak ina be imu \alst{s}elvon bi·\alst{s}wór \hld\ \alst{s}wíðon êðun, &
\alst{g}rótte ina an \alst{g}odes namon \hld\ ęndi \alst{g}erno bad, &
þat hé im þat gi·\alst{s}agdi, \hld\ ef hé \alst{s}unu wári &
þes \alst{l}ibbjendjes godes: \hld\ „þes þit \alst{l}ioht ge·skóp, &
\alst{K}rist \alst{k}uning êwig. \hld\ Wí ni mugun is ant·\alst{k}ięnnjen wiht &
ne an þínun \alst{w}ordun ni an þínun \alst{w}erkun.“ \hld\ Þȯ sprak imu eft þe \alst{w}áro an·gęgin, &
þe \alst{g}ódo \alst{g}odes sunu: \hld\ „þú kwiðis it for þesun \alst{J}udeon nu, &
\alst{s}ȯð-líko \alst{s}ęgis, \hld\ þat ik it \alst{s}elvo bium. &
Þes ni gi·\alst{l}ôvjad mí þese \alst{l}iudi: \hld\ ni willjad mi for·\alst{l}átan be·þiu; &
ni sind im mín \alst{w}ord \alst{w}irðig. \hld\ Nu sęggju ik iu te \alst{w}árun þoh, &
þat gí noh skulun \alst{s}ittjen gi·\alst{s}ehan \hld\ an þe \alst{s}wíðaron half godes &
\alst{m}árjan \alst{m}annes sunu, \hld\ an \alst{m}ęgin-krafte &
þes \alst{a}lo-walden fader, \hld\ ęndi þanan \alst{e}ft kuman &
an \alst{h}imil-wolknun \alst{h}erod \hld\ ęndi allumu \alst{h}ęliðo kunnje &
mid is \alst{w}ordun a·dêljen, \hld\ al só iro ge·\alst{w}urhti sind.“ &
Þo \alst{b}alg ina þe \alst{b}iskop, \hld\ habde \alst{b}ittren hugi, &
\alst{w}rêðida wið þemu \alst{w}orde \hld\ ęndi is gi·\alst{w}ádi slêt, &
\alst{b}rak for is \alst{b}reostun: \hld\ „Nú ni þurvun gí \alst{b}ídan lęng“, kwað hé, &
„þit \alst{w}erod ge·\alst{w}it-skępjes, \hld\ nu im su·lik \alst{w}ord farad, &
\alst{m}ên-spráka fan is \alst{m}u̇ðe. \hld\ Þat gi·hôrid hér nu \alst{m}anno filu, &
\alst{r}inko an þesumu \alst{r}akude, \hld\ þat hé ina só \alst{r}íkjan telit, &
\alst{g}ihid þat hé \alst{g}od sí. \hld\ Hwat willjad gí \alst{J}udeon þes &
a·\alst{d}êljen te \alst{d}óme? \hld\ Is hé \alst{d}ôðes nú &
\alst{w}irðig be su·likun \alst{w}ordun?“\eva

\bvb TODO.\evb\evg

\bvg\bva[61][5107]%
\hspace*{100pt} Þat \alst{w}erod al ge·sprak, &
\alst{f}olk Judeono, \hld\ þat hé wári þes \alst{f}erhes skolo, &
\alst{w}ítjes só \alst{w}irðig. \hld\ Ni was it þoh be is ge·\alst{w}urhtjun gi·dóen, &
þat ine þár an \alst{J}erusalem \hld\ \alst{J}udeo liudi, &
\alst{s}unu drohtines \hld\ \alst{s}undja lôsen &
a·\alst{d}êldun te \alst{d}ôðe. \hld\ Þȯ was þero \alst{d}ádjo hróm &
\alst{J}udeo liudjun, \hld\ hwat sie þemu \alst{g}odes barne mahtin &
só \alst{h}aftemu mêst, \hld\ \alst{h}armes ge·frummjen. &
Be·\alst{w}urpun ina þȯ mid \alst{w}erodu \hld\ ęndi ina an is \alst{w}angon slógun, &
an is \alst{h}leor mid iro \alst{h}andun \hld\ —al was imu þat te \alst{h}oske gi·dóen—, &
\alst{f}ęlgidun imu \alst{f}irin-word \hld\ \alst{f}íundo męnegi, &
\alst{b}ismer-spráka. \hld\ Stód þat \alst{b}arn godes &
\alst{f}ast under \alst{f}íundun: \hld\ wárun imu is \alst{f}aðmos ge·bundene, &
\alst{þ}olode mid gi·\alst{þ}uldjun, \hld\ só hwat só imu þiu \alst{þ}ioda tó &
\alst{b}ittres \alst{b}ráhte: \hld\ ni \alst{b}alg ina n·eo·wiht &
wið þes \alst{w}erodes ge·\alst{w}in. \hld\ Þȯ námon ina \alst{w}rêðe man &
só gi·\alst{b}undanan, \hld\ þat \alst{b}arn godes, &
ęndi ina þȯ \alst{l}êddun, \hld\ þár þero \alst{l}iudjo was, &
þere \alst{þ}iade \alst{þ}ing-hús. \hld\ Þár \alst{þ}egạn manag &
\alst{h}wurvun umbi iro \alst{h}ęri-togon. \hld\ Þár was iro \alst{h}êrron bodo &
fan \alst{R}úmu-burg, \hld\ þes þe þȯ þes \alst{r}íkjas gi·weld: &
\alst{k}umen was hé fan þemu \alst{k}êsure, \hld\ gi·sęndid was hé undar þat \alst{k}unni Judeono &
te \alst{r}ihtjenne þat \alst{r}íki, \hld\ was þár \alst{r}ád-gevo: &
\alst{P}ilatus was hé hêten; \hld\ hé was fan \alst{P}onteo lande &
\alst{k}nósles \alst{k}ęnnit. \hld\ Habde imu \alst{k}raft mikil, &
an þemu \alst{þ}ing-húse \hld\ \alst{þ}iod gi·samnod, &
an \alst{w}arf \alst{w}eros; \hld\ \alst{w}ár-lôse man &
a·\alst{g}ávun þȯ þena \alst{g}odes sunu, \hld\ \alst{J}udeo liudi, &
under \alst{f}íundo \alst{f}olk, \hld\ kwáðun þat hé wári þes \alst{f}erhes skolo, &
þat man ina \alst{w}ítnodi \hld\ \alst{w}ápnes ęggjun, &
\alst{sk}arpun \alst{sk}úrun. \hld\ Ni welde þiu \alst{sk}ole Judeono &
\alst{þ}ringan an þat \alst{þ}ing-hús, \hld\ ak þiu \alst{þ}iod úte stód, &
\alst{m}ahlidun þanen wið þea \alst{m}ęnegi: \hld\ ni weldun an þat gi·\alst{m}ang faren, &
an \alst{ę}li-landige man, \hld\ þat sie þár \alst{u}n-reht word, &
an þemu \alst{d}age \alst{d}ęrvjes wiht \hld\ a·\alst{d}êljan ne gi·hôrdin, &
ak kwáðun þat sie im só \alst{h}luttro \hld\ \alst{h}êlaga tídi, &
weldin iro \alst{p}askha halden. \hld\ \alst{P}ilatus ant·féng &
at þem \alst{w}am-skaðun \hld\ \alst{w}aldandes barn, &
\alst{s}undja lôsen. \hld\ Þȯ an \alst{s}orgun warð &
\alst{J}udases hugi, \hld\ þȯ hé a·\alst{g}evan gi·sah &
is \alst{d}rohtin te \alst{d}ôðe, \hld\ þȯ bi·gan imu þiu \alst{d}ád aftar þiu &
an is \alst{h}ugja \alst{h}rewan, \hld\ þat hé habde is \alst{h}êrron êr &
\alst{s}undja lôsen gi·\alst{s}ald. \hld\ Nam imu þȯ þat \alst{s}ilụvar an hand, &
\alst{þ}rí-tig skatto, \hld\ þat man imu êr wið is \alst{þ}iodane gaf, &
\alst{g}éng imu þȯ te þem \alst{J}udiun \hld\ ęndi im is \alst{g}rimmon dád, &
\alst{s}undjon \alst{s}agde, \hld\ ęndi im þat \alst{s}ilụvar bôd &
\alst{g}erno te a·\alst{g}evanne: \hld\ „ik hębbju it só \alst{g}rio-líko“, kwað hé, &
„mínes \alst{d}rohtines \hld\ \alst{d}rôru gi·kôpot, &
só ik wêt þat it mi ni \alst{þ}íhit.“ \hld\ \alst{Þ}iod Judeono &
ni weldun it þȯ ant·\alst{f}ȧhan, \hld\ ak hétun ina \alst{f}orð aftar þiu &
umbi \alst{s}u·lika \alst{s}undja \hld\ \alst{s}elvon ahton, &
hwat hé wið is \alst{f}râhon \hld\ ge·\alst{f}rumid habdi: &
„Þú \alst{s}áhi þi \alst{s}elvo þes“, \hld\ kwaðun sie; „hwat wili þú þes nu \alst{s}óken te u̇s? &
Ne \alst{w}ít þú þat þesumu \alst{w}erode!“ \hld\ Þȯ gi·\alst{w}êt imu eft þanan &
\alst{J}udas \alst{g}angan \hld\ te þemu \alst{g}odes wíhe &
\alst{s}wíðo an \alst{s}orgun \hld\ ęndi þat \alst{s}ilụvar warp &
an þena \alst{a}lạh innan, \hld\ ne gi·dorste it \alst{ê}gan lęng; &
\alst{f}ór imu þȯ só an \alst{f}orhtun, \hld\ só ina \alst{f}íundo barn &
\alst{m}ódage \alst{m}anodun: \hld\ habdun þes \alst{m}annes hugi &
\alst{g}ramon under·\alst{g}ripanen, \hld\ was imu \alst{g}od a·bolgan, &
þat hé imu \alst{s}elvon þȯ \hld\ \alst{s}ímon warhte, &
\alst{h}nêg þȯ an \alst{h}eru-sêl \hld\ an \alst{h}inginna, &
\alst{w}arạg an \alst{w}urgil \hld\ ęndi \alst{w}íti ge·kôs, &
\alst{h}ard \alst{h}ęllje ge·þwing, \hld\ \alst{h}êt ęndi þiustri, &
\alst{d}iap \alst{d}ôðes \alst{d}alu, \hld\ hwand hé êr umbi is \alst{d}rohtin swêk.\eva

\bvb TODO.\evb\evg

\bvg\bva[62][5172]%
Þan bêd þat \alst{b}arn godes \hld\ —\alst{b}ęndi þolode &
an þemu \alst{þ}ing-húse—, \hld\ hwan êr þiu \alst{þ}iod under im, &
\alst{e}rlos \alst{ê}n-wordje \hld\ \alst{a}lle wurðin, &
hwat sie imu þan te \alst{f}erạh-kwálu \hld\ \alst{f}rummjan weldin. &
Þȯ þár an þem \alst{b}ęnkjun a·rês \hld\ \alst{b}odo kêsures &
fan \alst{R}úmu-burg \hld\ ęndi géng imu wið þat \alst{r}íki Judeono &
\alst{m}ódag \alst{m}ahljen, \hld\ þár þiu \alst{m}ęnigi stód &
aftar þemu \alst{h}ove \alst{h}warvon: \hld\ ni weldun an þat \alst{h}ús kuman &
an þemu \alst{p}askha-dage. \hld\ \alst{P}ilatus bi·gan &
\alst{f}rókno \alst{f}rágon \hld\ ovar þat \alst{f}olk Judeono, &
mid hwiu þe \alst{m}an habdi \hld\ \alst{m}orðes gi·skuldit, &
\alst{w}ítjes gi·\alst{w}erkot: \hld\ „be hwí gi imu só \alst{w}rêðe sind, &
an iuwomu \alst{h}ugja \alst{h}ótje?“ \hld\ Sie kwáðun þat hé im habdi \alst{h}armes só filu, &
\alst{l}êðes gi·\alst{l}êstid: \hld\ „ni gávin ina þesa \alst{l}iudi þi, &
þár sie ina \alst{ê}r bi·foran \hld\ \alst{u}vilan ni wissin, &
\alst{w}ordun far·\alst{w}arhten. \hld\ hé havat þeses \alst{w}erodes só filu &
far·\alst{l}êdid mid is \alst{l}êrun \hld\ —ęndi þesa \alst{l}iudi męrrid, &
dóit im iro \alst{h}ugi twífljen—, \hld\ þat wí ni mótun te þemu \alst{h}ove kêsures &
\alst{t}insi gelden; \hld\ þat mugun wí ina gi·\alst{t}ęlljen an &
mid \alst{w}áru ge·\alst{w}it-skępi. \hld\ hé sprikid ôk \alst{w}ord mikil, &
\alst{k}wiðit þat hé \alst{K}rist sí, \hld\ \alst{k}uning ovar þit ríki, &
be·\alst{g}ihit ina só \alst{g}rôtes.“ \hld\ Þȯ im eft te·\alst{g}ęgnes sprak &
\alst{b}odo kêsures: \hld\ „ef hé só \alst{b}ar-líko“, kwað hé, &
„under þesaru \alst{m}ęnigi \hld\ \alst{m}ên-werk frumid, &
ant·\alst{f}ȧhad ina þan eft under iuwe \alst{f}olk-skępi, \hld\ ef hé sí is \alst{f}erhes skolo, &
ęndi imu só a·\alst{d}êljad, \hld\ ef hé sí \alst{d}ôðes werð, &
só it an \alst{i}uwaro \alst{a}ldrono \hld\ \alst{ê}o ge·biode.“ &
Sie kwáðun þȯ, þat sie ni \alst{m}óstin \hld\ \alst{m}anno nig·ênumu &
an þea \alst{h}êlagon tíd \hld\ te \alst{h}and-banon, &
\alst{w}erðen mid \alst{w}ápnun \hld\ an \alst{þ}emu wíh-dage. &
Þȯ \alst{w}ęnde ina fan þemu \alst{w}erode \hld\ \alst{w}rêð-hugdig man, &
\alst{þ}egạn kêsures, \hld\ þe ovar þea \alst{þ}ioda was &
\alst{b}odo fan Rúmu-burg—: \hld\ hét imu þȯ þat \alst{b}arn godes &
\alst{n}áhor gangan \hld\ ęndi ina \alst{n}iud-líko, &
\alst{f}rágoda \alst{f}rókno, \hld\ ef hé ovar þat \alst{f}olk kuning &
þes \alst{w}erodes \alst{w}ári. \hld\ Þȯ habde eft is \alst{w}ord garu &
\alst{s}unu drohtines: \hld\ „hweðer þú þat fan þi \alst{s}elvumu sprikis“, kwað hé, &
„þe it þi \alst{ȯ}ðre hér \hld\ \alst{e}rlos sagdun, &
\alst{k}wáðun umbi mínan \alst{k}uning-duom?“ \hld\ Þȯ sprak eft þe \alst{k}êsures bodo &
\alst{w}lank ęndi \alst{w}rêð-mód, \hld\ þár hé wið \alst{w}aldand Krist &
\alst{r}eðjode an þem \alst{r}akude: \hld\ „ni bium ik þeses \alst{r}íkjes hinan“, kwað hé, &
„\alst{J}udeo liudjo, \hld\ ni \alst{g}adoling þín, &
þesaro \alst{m}anno \alst{m}ág-wini, \hld\ ak mí þí þius \alst{m}ęnigi bi·falạh, &
a·\alst{g}ávun þí þína \alst{g}adulingos mí, \hld\ \alst{J}udeo liudi, &
\alst{h}aftan te \alst{h}andun. \hld\ Hwat havas þú \alst{h}armes gi·duan, &
þat þú só \alst{b}ittro skalt \hld\ \alst{b}ęndi þolojan, &
\alst{k}walm undar þínumu \alst{k}unnje?“ \hld\ Þȯ sprak imu eft \alst{K}rist an·gęgin, &
\alst{h}êlendero bętst, \hld\ þár hé gi·\alst{h}ęftid stód &
an þemu \alst{r}akude innan: \hld\ „nis mín \alst{r}íki hinan“, kwað hé, &
„fan þesaru \alst{w}er-old-stundu. \hld\ Ef it þoh \alst{w}ári só, &
þan wárin só \alst{st}ark-móde \hld\ wiðer \alst{st}ríd-hugi, &
wiðer \alst{g}rama þioda \hld\ \alst{j}ungaron míne, &
só man mi ni \alst{g}ávi \hld\ \alst{J}udeo liudjun, &
\alst{h}ęttendjun an \alst{h}and \hld\ an \alst{h}eru-bęndjun &
te \alst{w}êgjanne te \alst{w}undrun. \hld\ Te þiu warð ik an þesaru \alst{w}er-oldi gi·boran, &
þat ik ge·\alst{w}it-skępi giu \hld\ \alst{w}áres þinges &
mid mínun \alst{k}umjun \alst{k}u̇ðdi. \hld\ Þat mugun ant·\alst{k}ęnnjen wel &
þe \alst{w}eros, þe sind fan \alst{w}áre kumane: \hld\ þe mugun mín \alst{w}ord far·standen, &
gi·\alst{l}ôvjen mínun \alst{l}êrun.“ \hld\ Þȯ ni mahte \alst{l}asteres wiht &
an þem \alst{b}arne godes \hld\ \alst{b}odo kêsures, &
\alst{f}indan \alst{f}êknja word, \hld\ þat hé is \alst{f}erhes be·þiu &
\alst{sk}uldig wári. \hld\ Þȯ géng hé im eft wið þea \alst{sk}ola Judeono &
\alst{m}ódag \alst{m}ahljen \hld\ ęndi þeru \alst{m}ęnigi sagde &
ovar \alst{h}lust mikil, \hld\ þat hé an þemu \alst{h}afton manne &
su·lika \alst{f}irin-spráka \hld\ \alst{f}inden ni mahti &
for þem \alst{f}olk-skipje, \hld\ só hé wári is \alst{f}erhes skolo, &
\alst{d}ôðes wirðig. \hld\ Þan stódun \alst{d}ol-móde &
\alst{J}udeo liudi \hld\ ęndi þane \alst{g}odes sunu &
\alst{w}ordun \alst{w}rógdun: \hld\ kwáðun þat hé gi·\alst{w}er êrist &
be·\alst{g}unni an \alst{G}alileo lande, \hld\ „ęndi ovar \alst{J}udeon fór &
\alst{h}erod-wardes þanan, \hld\ \alst{h}ugi twíflode, &
\alst{m}anno \alst{m}ód-sevon, \hld\ só hé is \alst{m}orðes werð, &
þat man ina \alst{w}ítnoje \hld\ \alst{w}ápnes ęggjun, &
ef eo man mid su·likun \alst{d}ádjun mag \hld\ \alst{d}ôðes ge·skuldjen.“\eva

\bvb TODO.\evb\evg

\bvg\bva[63][5246]%
Só \alst{w}rógdun ina mid \alst{w}ordun \hld\ \alst{w}erod Judeono &
þurh \alst{h}ótjan \alst{h}ugi. \hld\ Þȯ þe \alst{h}ęri-togo, &
\alst{s}líð-módig man \hld\ \alst{s}ęggjan gi·hôrde, &
fan hwi-likumu \alst{k}unnje was \hld\ \alst{K}rist a·fódid, &
\alst{m}anno þe bętsto: \hld\ hé was fan þeru \alst{m}árjan þiadu, &
þe \alst{g}ódo fan \alst{G}alilea-lande; \hld\ þár was \alst{g}um-skępi &
\alst{ę}ðiljero manno; \hld\ \alst{E}rodes bi·held þár &
\alst{k}raftagne \alst{k}uning-dóm, \hld\ só ina imu þe \alst{k}êsur far·gaf, &
þe \alst{r}íkjo fan \alst{R}úmu, \hld\ þat hé þár \alst{r}ehto ge·hwi-lik &
ge·\alst{f}rumidi undar þemu \alst{f}olke \hld\ ęndi \alst{f}riðu lêsti, &
\alst{d}ómos a·\alst{d}êldi. \hld\ hé was ôk an þemu \alst{d}age selvo &
an \alst{J}erusalem \hld\ mid is \alst{g}um-skępi, &
mid is \alst{w}erode at þemu \alst{w}íhe: \hld\ só was iro \alst{w}íse þan, &
þat sie þár þia \alst{h}êlagun tíd \hld\ \alst{h}aldan skoldun, &
\alst{p}askha Judeono. \hld\ \alst{P}ilatus gi·bôd þȯ, &
þat þena \alst{h}afton man \hld\ \alst{h}ęliðos námin &
só gi·\alst{b}undanan, \hld\ þat \alst{b}arn godes, &
hét þat sie ina \alst{E}rodese, \hld\ \alst{e}rlos brȧhtin &
\alst{h}aften te \alst{h}andun, \hld\ hwand hé fan is \alst{h}ęri-skępi was, &
fan is \alst{w}erodes ge·\alst{w}ald. \hld\ \alst{W}ígand frumidun &
iro \alst{h}êrron word: \hld\ \alst{h}êlagne Krist &
\alst{f}órdun an \alst{f}iterjun \hld\ for þena \alst{f}olk-togun, &
allaro \alst{b}arno \alst{b}ętst, \hld\ þero þe io gi·\alst{b}oren wurði &
an \alst{l}iudjo \alst{l}ioht; \hld\ an \alst{l}iðu-bęndjun géng, &
an-tat sie ina \alst{b}ráhtun, \hld\ þár hé an is \alst{b}ęnkja sat, &
\alst{k}uning Erodes: \hld\ umbi·hwarf ina \alst{k}raft wero, &
\alst{w}lanke \alst{w}ígandos: \hld\ was im \alst{w}illjo mikil, &
þat sie þár \alst{s}elvon Krist \hld\ gi·\alst{s}ehan móstin: &
wándun þat hé im sum \alst{t}êkạn \hld\ þár \alst{t}ôgjan skoldi, &
\alst{m}ári ęndi \alst{m}ahtig, \hld\ só hé \alst{m}anagun dede &
þurh is \alst{g}od-kundi \hld\ \alst{J}udeo *liudjon. &
\alst{F}rágoda ina þuȯ þie \alst{f}olk-kuning \hld\ \alst{f}iri-wit-líko &
\alst{m}anagon wordon, \hld\ wolda is \alst{m}uod-sevon &
\alst{f}orð undar·\alst{f}indan, \hld\ hwat hie te \alst{f}rumu mohti &
\alst{m}annon gi·\alst{m}arkon. \hld\ Þan stuod \alst{m}ahtig Krist, &
\alst{þ}agoda ęndi \alst{þ}oloda: \hld\ ne wolda þem \alst{þ}ied-kuninge, &
\alst{E}rodese ne is \alst{e}rlon \hld\ \alst{a}nt-swór gevan &
\alst{w}ordo nig·ênon. \hld\ Þan stuod þiu \alst{w}rêða þiod, &
\alst{J}udeo liudi \hld\ ęndi þena \alst{g}odes suno &
\alst{w}urrun ęndi \alst{w}ruogdun, \hld\ anþat im warð þie \alst{w}er-old-kuning &
an is \alst{h}uge \alst{h}uoti \hld\ ęndi all is \alst{h}ęri-skipi, &
far·\alst{m}uonstun ina an iro \alst{m}uode: \hld\ ne ant·kęndun \alst{m}aht godes, &
\alst{h}imiliskan \alst{h}êrron, \hld\ ak was im iro \alst{h}ugi þiustri, &
\alst{b}aluwes gi·\alst{b}landan. \hld\ \alst{B}arn drohtines &
iro \alst{w}rêðun \alst{w}erk, \hld\ \alst{w}ord ęndi dádi &
þuru \alst{ô}d-muodi \hld\ \alst{a}ll gi·þoloda, &
só hwat só sia im \alst{t}ionono þuȯ \hld\ \alst{t}uogjan woldun. &
Sia \alst{h}ietun im þuȯ te \alst{h}oske \hld\ \alst{h}wít gi·wádi &
umbi is \alst{l}iði \alst{l}ęggjan, \hld\ þiu mêr hie wurði þem \alst{l}iudjon þár, &
\alst{j}ungron te \alst{g}amne. \hld\ \alst{J}udeon faganodun, &
þuȯ sia ina te \alst{h}oske \hld\ \alst{h}ębbjan gi·sáhun, &
\alst{e}rlos \alst{o}var-muoda. \hld\ Þuȯ sęnda ina \alst{e}ft þanan &
\alst{E}rodes se kuning \hld\ an þat \alst{ȯ}ðer folk; &
a·\alst{l}êdjan hiet ina \alst{l}ungra mann, \hld\ ęndi \alst{l}astar sprákun, &
\alst{f}elgidun im \alst{f}irin-word, \hld\ þár hie an \alst{f}eteron géng &
bi·\alst{h}lagan mid \alst{h}osku: \hld\ ni was im \alst{h}ugi twífli, &
neva hie it þuru \alst{ô}d-muodi \hld\ \alst{a}ll gi·þoloda; &
ne welda iro \alst{u}vilun word \hld\ \alst{i}dug-lônon, &
\alst{h}osk ęndi \alst{h}arm-kwidi. \hld\ Þuȯ brȧhtun sia ina eft an þat \alst{h}ús innan, &
an þia \alst{p}alenkja uppan, \hld\ þár \alst{P}ilatus was &
an þero \alst{þ}ing-stędi. \hld\ \alst{Þ}egnos a·gávun &
\alst{b}arno þat \alst{b}ęsta \hld\ \alst{b}anon te handon &
\alst{s}undi-lôsjan, \hld\ só hie \alst{s}elvo gi·kôs: &
welda \alst{m}anno barn \hld\ \alst{m}orðes a·tuomjan, &
\alst{n}ęrjan af \alst{n}ôdi. \hld\ Stuodun \alst{n}íð-hwata, &
\alst{J}udeon far þem \alst{g}ast-sęlje: \hld\ habdun sia \alst{g}ramono barn, &
þia \alst{sk}ola far·\alst{sk}undid, \hld\ þat sia ne be·\alst{sk}rivun iowiht &
\alst{g}rimmera dádjo. \hld\ Þuȯ gi·wêt im \alst{g}angan þarod &
\alst{þ}egạn kêsures \hld\ wið þia \alst{þ}iod sprekan, &
\alst{h}ard \alst{h}ęri-togo: \hld\ „Hwat gí mí þesan \alst{h}aftan mann“, kwaþ-hie, &
„an þesan \alst{s}ęli \alst{s}ęndun \hld\ ęndi \alst{s}elvon an·budun, &
þat hie iuwes \alst{w}erodes só filo \hld\ a·\alst{w}erdit habdi, &
far·\alst{l}êdid mid is \alst{l}êron. \hld\ Nu ik mid þeson \alst{l}iudon ni mag, &
\alst{f}indan mid þius \alst{f}olku, \hld\ þat hie is \alst{f}erạhes sí &
furi þesaro \alst{sk}olu \alst{sk}uldig. \hld\ \alst{Sk}ín was þat hiudu: &
\alst{E}rodes mohta, \hld\ þie iuwan \alst{ê}o bi·kan, &
iuwaro \alst{l}iudo \alst{l}and-reht, \hld\ hie ni mahta is \alst{l}íves gi·frêson, &
þat hie hier þuru êniga \alst{s}undja te dage \hld\ \alst{s}weltan skoldi, &
\alst{l}íf far·\alst{l}átan. \hld\ Nu willju ik ina for þeson \alst{l}iudjon hier &
gi·\alst{þ}róon mid \alst{þ}ingon, \hld\ \alst{þ}rístjon wordun, &
\alst{b}uotjan im is \alst{b}riost-hugi, \hld\ látan ina \alst{b}rúkan forð &
\alst{f}erạhes mid \alst{f}irjon.“ \hld\ \alst{F}olk Judeono &
\alst{h}reopun þuȯ alla samad \hld\ \alst{h}lúdero stemnu, &
hietun \alst{f}lít-líko \hld\ \alst{f}erạhes áhtjan &
\alst{K}rist mid \alst{k}walmu \hld\ ęndi an \alst{k}rúki slahan, &
\alst{w}êgjan te \alst{w}undron: \hld\ „hie mid is \alst{w}ordon havit &
\alst{d}ôðes gi·skuldid: \hld\ sagit þat hie \alst{d}rohtin sí, &
\alst{g}egnungo \alst{g}odes suno. \hld\ Þat hie a·\alst{g}eldan skal, &
\alst{i}n-wid-spráka, \hld\ só is an u̇son \alst{ê}we gi·skrivan, &
þat man su·lika \alst{f}irin-kwidi \hld\ \alst{f}erạhu kôpo.“\eva

\bvb TODO.\evb\evg

\bvg\bva[64][5336]%
Þuȯ warð þie an \alst{f}orạhton, \hld\ þie þes \alst{f}olkes gi·weld, &
\alst{m}ikilon an is \alst{m}uode, \hld\ þuȯ hie gi·hôrda þia \alst{m}an sprekan, &
þat sia ina \alst{s}elvon \hld\ \alst{s}ęggjan gi·hôrdin, &
\alst{g}ehan fur þem \alst{g}um-skipe, \hld\ þat hie wári \alst{g}odes suno. &
Þuȯ hwarf im eft þie \alst{h}ęri-togo \hld\ an þat \alst{h}ús innan &
te þero \alst{þ}ing-stędi, \hld\ \alst{þ}rístjon wordon &
\alst{g}ruotta þena \alst{g}odes suno \hld\ ęndi frágoda, hwat hie \alst{g}umono wári: &
„hwat bist þú \alst{m}anno?“ \hld\ kwaþ-hie. „Te hwí þú mí só þínan \alst{m}uod hilis, &
\alst{d}ęrnis \alst{d}iop-gi·þȧht? \hld\ Wêst þú þat it all an mínon \alst{d}uome stéd &%TODO: Check stéd.
umbi þínes \alst{l}íves gi·\alst{l}agu? \hld\ Mí þi hębbjat þesa \alst{l}iudi far·gevan, &
\alst{w}erod Judeono, \hld\ þat ik gi·\alst{w}aldan muot &
só þik te \alst{sp}ildjanne \hld\ an \alst{sp}eres orde, &
só ti \alst{k}węlljanne an \alst{k}rúkjum, \hld\ só \alst{k}wikan látan, &
só hweðer sí mi \alst{s}elvon \hld\ \alst{s}uotera þunkit &
te gi·\alst{f}rummjanne mid mínu \alst{f}olku.“ \hld\ Þuȯ sprak eft þat \alst{f}riðu-barn godes: &
„\alst{W}êst þú þat te \alst{w}áron“, \hld\ kwaþ-hie, „þat þú gi·\alst{w}ald ovar mik &
\alst{h}ębbjan ni mohtis, \hld\ ne wári þat it þi \alst{h}êlag god &
\alst{s}elvo far·gávi? \hld\ Ôk hębbjat þia \alst{s}undjono mêr, &
þia mik þi bi·\alst{f}ulhun \hld\ þuru \alst{f}íond-skipi, &
gi·\alst{s}aldun an \alst{s}ímon haftan.“ \hld\ Þuȯ welda ina \alst{s}ïð after þiu &
\alst{g}ram-hugdig man \hld\ \alst{g}erno far·látan, &
\alst{þ}egạn kêsures, \hld\ þár hie is havdi for þero \alst{þ}ioda gi·wald; &
ak sia \alst{w}ęridun im þena \alst{w}illjon \hld\ \alst{w}ordu gi·hwi-liku, &
\alst{k}unni Judeono: \hld\ „ne bist þú“, kwáðun sia, „þes \alst{k}êsures friund, &
þínon \alst{h}êrren \alst{h}old, \hld\ ef þú ina \alst{h}inan látis &
\alst{s}ïðon gi·\alst{s}undon: \hld\ þat þi noh te \alst{s}orạgan mag, &
\alst{w}erðan te \alst{w}íte, \hld\ hwand só hwe só su·lik \alst{w}ord sprikit, &
a·\alst{h}avið ina só \alst{h}ôho, \hld\ kwiðit þat hie \alst{h}ębbjan mugi &
\alst{k}uning-duomes namon, \hld\ ne sí þat ina im þie \alst{k}êsur geve, &
hie \alst{w}irrid im is \alst{w}er-uld-ríki \hld\ ęndi is \alst{w}ord far·hugid, &
far·\alst{m}an ina an is \alst{m}uode. \hld\ Be·þiu skalt þú su·lik \alst{m}ên wrekan, &
\alst{h}osk-word manag, \hld\ ef þú umbi þínes \alst{h}êrren ruokis, &
umbi þínes \alst{f}rôhon \alst{f}riund-skipi, \hld\ þan skalt þú ina þiu \alst{f}erhu be·niman.“ &
Þuȯ gi·\alst{h}ôrda þie \alst{h}ęri-togo \hld\ þia \alst{h}êri Juðeono &
\alst{þ}rêgjan fan is \alst{þ}iodne; \hld\ þuȯ hie far þero \alst{þ}ing-stędi géng &
\alst{s}elvo gi·\alst{s}ittjan, \hld\ þár gi·\alst{s}amnod was &
só mikil \alst{w}arf \alst{w}erodes, \hld\ hiet \alst{w}aldand Krist &
\alst{l}êdjan for þia \alst{l}iudi. \hld\ \alst{L}angoda Judeon, &
hwan êr sia þat \alst{h}êlaga barn \hld\ \alst{h}angon gi·sáwin, &
\alst{k}węlan an \alst{k}rúkje; \hld\ sia kwáðun þat sia \alst{k}uning ȯðran &
ne \alst{h}avdin undar iro \alst{h}ęri-skipje, \hld\ nevan þena \alst{h}êran kêsar &
fan \alst{R}úmu-burg: \hld\ „þie havit hier \alst{r}íki over u̇s. &
Be·þiu ni skalt þú þesan far·\alst{l}átan; \hld\ hie havit u̇s só filo \alst{l}êðes gi·sprokan, &
far·\alst{d}uan havit hie im mid is \alst{d}ádjon. \hld\ Hie skal \alst{d}ôð þolon, &
\alst{w}íti ęndi \alst{w}undạr-kwála.“ \hld\ \alst{W}erod Judeono &
só \alst{m}anag \alst{m}is-lík þing \hld\ an \alst{m}ahtigna Krist &
\alst{s}agdun te \alst{s}undjun. \hld\ Hie \alst{s}wígondi stuod &
þuru \alst{ô}ð-muodi, \hld\ ne \alst{a}nt-wordida n·io·wiht &
wið iro \alst{w}rêðun \alst{w}ord: \hld\ wolda þesa \alst{w}er-old alla &
\alst{l}ôsjan mid is \alst{l}ívu: \hld\ bi·þiu liet hie ina þia \alst{l}êðun þiod &
\alst{w}êgjan te \alst{w}undron, \hld\ all só iro \alst{w}illjo géng: &
ni wolda im \alst{o}pan-líko \hld\ \alst{a}llon ku̇ðjan &
\alst{J}udeo liudjon, \hld\ þat hie was \alst{g}od selvo; &
hwand \alst{w}issin sia þat te \alst{w}áron, \hld\ þat hie su·lika gi·\alst{w}ald havdi &
ovar þeson \alst{m}iddil-gard, \hld\ þan wurði im iro \alst{m}uod-sevo &
gi·\alst{b}lôðit an iro \alst{b}rioston: \hld\ þan ne gi·dorstin sia þat \alst{b}arn godes &
\alst{h}andon ant·\alst{h}rínan: \hld\ þan ni wurði \alst{h}evan-ríki, &
ant·\alst{l}okan \alst{l}iohto mêst \hld\ \alst{l}iudjo barnon. &
Be·þiu \alst{m}êð hie is só an is \alst{m}uode, \hld\ ne lét þat \alst{m}anno folk &
\alst{w}itan, hwat sia \alst{w}arạhtun. \hld\ Þiu \alst{w}urd náhida þuȯ, &
\alst{m}ári \alst{m}aht godes \hld\ ęndi \alst{m}iddi dag, &
þat sia þia \alst{f}erạh-kwála \hld\ \alst{f}rummjan skoldun. &
Þan lag þár ôk an \alst{b}ęndjon \hld\ an þero \alst{b}urg innan &
ên \alst{r}uof \alst{r}ęgin-skaðo, \hld\ þie habda under þem \alst{r}íke só filo &
\alst{m}orðes gi·rádan \hld\ ęndi \alst{m}an-slahta gi·frumid, &
was \alst{m}ári \alst{m}ęgin-þiof: \hld\ ni was þár is gi·\alst{m}ako hwęrgin; &
was þár ôk bi \alst{s}ínon \hld\ \alst{s}undjon gi·hęftid, &
\alst{B}arrabas was hie hêtan; \hld\ hie after þem \alst{b}urgjon was &
þuru is \alst{m}ên-dádi \hld\ \alst{m}anogon gi·ku̇ðid. &
Þan was \alst{l}and-wísa \hld\ \alst{l}iudjo Judeono, &
þat sia \alst{j}áro gi·hwen \hld\ an \alst{g}odes minnja &
an þem \alst{h}êlagon dage \hld\ ênna \alst{h}aftan mann &
a·\alst{b}iddjan skoldun, \hld\ þat im iro \alst{b}urges ward, &
iro \alst{f}olk-togo \hld\ \alst{f}erạh far·gávi. &
Þuȯ bi·gan þie \alst{h}ęri-togo \hld\ þia \alst{h}êri Judeono, &
þat \alst{f}olk \alst{f}rágojan, \hld\ þár sia im \alst{f}ora stuodun, &
hweðeron sia þero \alst{t}wejo \hld\ \alst{t}uomjan weldin, &
\alst{f}erạhes biddjan: \hld\ „þia hier an \alst{f}eteron sind &
\alst{h}aft undar þeson \alst{h}ęri-skipje?“ \hld\ Þiu \alst{h}êri Judeono &
habdun þuȯ þia \alst{a}rạmun man \hld\ \alst{a}lla gi·spanana, &
þat sia þemo \alst{l}and-skaðen \hld\ \alst{l}íf a·bádin, &
gi·\alst{þ}ingodin þem \alst{þ}iove, \hld\ þie oft an \alst{þ}iustrja naht &
\alst{w}am gi·\alst{w}arạhta, \hld\ ęndi \alst{w}aldand Krist &
\alst{k}węlidin an \alst{k}rúkje. \hld\ Þuȯ warð þat \alst{k}u̇ð ovar all, &
hwó þiu þiod havda \alst{d}uomos a·\alst{d}êlid. \hld\ Þuȯ skoldun sia þia \alst{d}ád frummjan, &
\alst{h}ȧhan þat \alst{h}êlaga barn. \hld\ Þat warð þem \alst{h}ęri-togen &
\alst{s}ïðor te \alst{s}orgon, \hld\ þat hie þia \alst{s}aka wissa, &
þat sia þuru \alst{n}íð-skipi \hld\ \alst{n}ęrjendon Krist, &
\alst{h}atoda þiu \alst{h}êri, \hld\ ęndi hie im \alst{h}ôrda te þiu, &
\alst{w}arạhta iro \alst{w}illjon: \hld\ þes hie \alst{w}íti ant·féng, &
\alst{l}ôn an þeson \alst{l}iohte \hld\ ęndi \alst{l}ang after, &
\alst{w}ói sïðor \alst{w}ann, \hld\ sïðor hie þesa \alst{w}er-old a·gaf.\eva%NOTE: wói checked.

\bvb TODO.\evb\evg

\bvg\bva[65][5428]%
Þuȯ warð þas þie \alst{w}rêðo gi·\alst{w}aro, \hld\ \alst{w}am-skaðono mêst, &
\alst{S}atanas \alst{s}elvo, \hld\ þuȯ þiu \alst{s}eola kwam &
\alst{J}udases an \alst{g}rund \hld\ \alst{g}rimmaro hęlljun— &
þuȯ \alst{w}issa hie te \alst{w}áren, \hld\ þat þat was \alst{w}aldand Krist, &
\alst{b}arn drohtines, \hld\ þat þár gi·\alst{b}undan stuod; &
\alst{w}issa þuȯ te \alst{w}áron, \hld\ þat hie welda þesa \alst{w}er-old alla &
mid is \alst{h}ęnginnja \hld\ \alst{h}ęllja gi·þwinges, &
\alst{l}iudi a·\alst{l}ôsjan \hld\ an \alst{l}ioht godes. &
Þat was \alst{S}atanase \hld\ \alst{s}êr an muode, &
\alst{t}ulgo harm an is \alst{h}ugje: \hld\ welda is \alst{h}elpan þuȯ, &
þat im \alst{l}iudjo barn \hld\ \alst{l}íf ne bi·námin, &
ne \alst{k}węlidin an \alst{k}rúkje, \hld\ ak hie welda, þat hie \alst{k}wik livdi, &
te þiu þat \alst{f}iriho barn \hld\ \alst{f}ernes ne wurðin, &
\alst{s}undjono \alst{s}ikura. \hld\ \alst{S}atanas gi·wêt im þuȯ, &
þár þes \alst{h}ęri-togen \hld\ \alst{h}íwiski was &
an þero \alst{b}urg innan. \hld\ Hie þero is \alst{b}rúdi bi·gann, &
þera idis \alst{o}pan-líko \hld\ \alst{u}n-hiuri fíond &
\alst{w}undẹr tôgjan, \hld\ þat sia an \alst{w}ord-helpon &
\alst{K}riste wári, \hld\ þat hie muosti \alst{k}wik libbjan, &
\alst{d}rohtin manno \hld\ —hie was iu þan te \alst{d}ôðe gi·skęrid— &
\alst{w}issa þat te \alst{w}áron, \hld\ þat hie im skoldi þia gi·\alst{w}ald bi·niman, &
þat hie sia ovar þesan \alst{m}iddil-gard \hld\ só \alst{m}ikila ni havdi, &
ovar \alst{w}ída \alst{w}er-old. \hld\ Þat \alst{w}íf warð þuȯ an forạhton, &
\alst{s}wíðo an \alst{s}orọgon, \hld\ þuȯ iru þiu gi·\alst{s}iuni kwámun &
þuru þes \alst{d}ęrnjen \alst{d}ád \hld\ an \alst{d}ages liohte, &
an \alst{h}ęlið-helme bi·\alst{h}elid. \hld\ Þuȯ siu te iru \alst{h}êrren an·bôd, &
þat \alst{w}íf mid iro \alst{w}ordon \hld\ ęndi im te \alst{w}áren hiet &
\alst{s}elvon \alst{s}ęggjan, \hld\ hwat iro þár te gi·\alst{s}iunjon kwam &
þuru þena \alst{h}êlagan mann, \hld\ ęndi im \alst{h}elpan bad, &
\alst{f}ormon is \alst{f}erhe: \hld\ „ik hębbju hier só \alst{f}ilo þuru ina &
\alst{s}eld-líkes gi·\alst{s}ewan, \hld\ só ik wêt, þat þia \alst{s}undjun skulun &
\alst{a}llaro \alst{e}rlo gi·hwem \hld\ \alst{u}vilo gi·þíhan, &
só im \alst{f}ruokno tuo \hld\ \alst{f}erạhes áhtið.“ &
Þie \alst{s}ęgg warð þuȯ an \alst{s}ïðe, \hld\ an-tat hie \alst{s}ittjan fand &
þena \alst{h}ęri-togon \hld\ an \alst{h}warạve innan &
an þem \alst{st}ên-wege, \hld\ þár þiu \alst{st}ráta was &
\alst{f}elison gi·\alst{f}uogid. \hld\ Þár hie te is \alst{f}rôhon géng, &
sagda im þes \alst{w}íves \alst{w}ord. \hld\ Þuȯ warð im \alst{w}rêð hugi, &
þem \alst{h}ęri-togen, \hld\ —\alst{h}warạvoda an innan—, &
gi·\alst{b}lôðit \alst{b}riost-gi·þȧht: \hld\ was im \alst{b}êðjes wê, &
gie þat sea ina \alst{s}luogin \hld\ \alst{s}undja lôsan, &
gie it bi þem \alst{l}iudjon þuȯ \hld\ for·\alst{l}átan ne gi·dorsta &
þuru þes \alst{w}erodes word. \hld\ Warð im gi·\alst{w}ęndid þuȯ &
\alst{h}ugi an \alst{h}erten \hld\ after þero \alst{h}êri Judeono, &
te \alst{w}erkjanne iro \alst{w}illjon: \hld\ ne \alst{w}ardoda im nie-wiht &
þia \alst{s}wárun \alst{s}undjun, \hld\ þia hie im þár þuȯ \alst{s}elvo gi·deda. &
Hiet im þuȯ te is \alst{h}andon dragan \hld\ \alst{h}luttran brunnjon, &
\alst{w}atar an \alst{w}égje, \hld\ þár hie furi þem \alst{w}erode sat, &
\alst{þ}wóg ina þár for þero \alst{þ}ioda \hld\ \alst{þ}egạn kêsures, &
\alst{h}ard \alst{h}ęri-togo \hld\ ęndi þuȯ fur þero \alst{h}êri sprak, &
kwað þat hie ina þero \alst{s}undjono þár \hld\ \alst{s}ikoran dádi, &
\alst{w}rêðero \alst{w}erko: \hld\ „ne willju ik þes \alst{w}ihtes plegan“, kwaþ-hie, &
„umbi þesan \alst{h}êlagan mann, \hld\ ak \alst{h}leotad gi þes alles, &
gie \alst{w}ordo gie \alst{w}erko, \hld\ þes gi im hér te \alst{w}ítje gi·duan.“ &
Þuȯ \alst{h}reop all saman \hld\ \alst{h}ęri-skipi Judeono, &
þiu \alst{m}ikila \alst{m}ęnigi, \hld\ kwáðun þat sia weldin umbi þena \alst{m}an plegan &
\alst{d}ęrạvoro \alst{d}ádjo: \hld\ „fare is \alst{d}rôr ovar u̇s, &
is \alst{b}luod ęndi is \alst{b}aneði \hld\ ęndi ovar u̇sa \alst{b}arn só samo, &
ovar \alst{u̇}sa \alst{a}varon þár after \hld\ —wí willjat is \alst{a}lles plegan“, kwaðun sia, &
„umbi þena \alst{s}lęgi \alst{s}elvon,— \hld\ ef wí þár êniga \alst{s}undja gi·duan!“ &
A·\alst{g}evan warð þár þuȯ furi þem \alst{J}udeon \hld\ allaro \alst{g}umono bęsta &
\alst{h}ęttendjon an \alst{h}and, \hld\ an \alst{h}eru-bęndjon &
\alst{n}arạwo gi·\alst{n}ôdid, \hld\ þár ina \alst{n}íð-hwata, &
\alst{f}íond ant·\alst{f}éngun: \hld\ \alst{f}olk ina umbi·hwarf, &
\alst{m}ên-skaðono \alst{m}ęgin. \hld\ \alst{M}ahtig drohtin &
\alst{þ}oloda gi·\alst{þ}uldjon, \hld\ só hwat só im þiu \alst{þ}ioda deda. &
Sia hietun ina þuȯ \alst{f}illjan, \hld\ êr þan sia im \alst{f}erạhes tuo, &
\alst{a}ldres \alst{á}htin, \hld\ ęndi im undar is \alst{ô}gun spiwun, &
dedun im þat te \alst{h}oske, \hld\ þat sia mid iro \alst{h}andon slógun, &
\alst{w}eros an is \alst{w}angun \hld\ ęndi im is gi·\alst{w}ádi bi·námun, &
\alst{r}ôvodun ina þia \alst{r}ęgin-skaðon, \hld\ \alst{r}ôdes lakanes &
dedun im eft \alst{ȯ}ðer \alst{a}n \hld\ þuru \alst{u}n·huldi; &
\alst{h}ietun þuȯ \alst{h}ôvid-band \hld\ \alst{h}ardaro þorno &
\alst{w}undron \alst{w}indan \hld\ ęndi an \alst{w}aldand Krist &
\alst{s}elvon \alst{s}ęttjan, \hld\ ęndi géngun im þia gi·\alst{s}ïðos tuo, &
\alst{k}węddun ina an \alst{k}uning-wísu \hld\ ęndi þár an \alst{k}nio fellun, &
\alst{h}nigun im mid iro \alst{h}ôvdu: \hld\ all was im þat te \alst{h}oske gi·duan, &
þoh hie it all gi·\alst{þ}olodi, \hld\ \alst{þ}iodo drohtin, &
\alst{m}ahtig þuru þia \alst{m}innja \hld\ \alst{m}anno kunnjes. &
Hietun sia þuȯ \alst{w}irkjan \hld\ \alst{w}ápnes ęggjon &
\alst{h}ęliðos mid iro \alst{h}andon \hld\ \alst{h}ardes bômes &
\alst{k}raftiga \alst{k}rúki \hld\ ęndi hietun sia \alst{K}ristan þuȯ, &
\alst{s}álig barn godes \hld\ \alst{s}elvon fuorjan, &
\alst{d}ragan hietun sia u̇san \alst{d}rohtin, \hld\ þár hie be·\alst{d}rôragad skolda &
\alst{s}weltan \alst{s}undjono lôs. \hld\ \alst{S}íðodun Judeon, &
\alst{w}eros an \alst{w}illon, \hld\ lêddun \alst{w}aldand Krist, &
\alst{d}rohtin te \alst{d}ôðe. \hld\ Þár mohta man þuȯ \alst{d}erẹvi þing &
\alst{h}arm-lík gi·\alst{h}ôrjan: \hld\ \alst{h}iovandi þár after &
géngun \alst{w}íf mid \alst{w}ópu, \hld\ \alst{w}eros gnornodun, &
þia fan \alst{G}alilea mid im \hld\ \alst{g}angan kwámun, &
\alst{f}olgodun ovar \alst{f}err-wegos: \hld\ was im iro \alst{f}rôhon dôð &
\alst{s}wíðo an \alst{s}orạgan. \hld\ Þuȯ hie \alst{s}elvo sprak, &
\alst{b}arno þat \alst{b}ęsta \hld\ ęndi under \alst{b}ak be·sah, &
hiet þat sia ni \alst{w}épin: \hld\ „ni þarf iu \alst{w}iht tregan“, kwaþ-hie, &
„mínero \alst{h}in-fęrdjo, \hld\ ak gí mid \alst{h}ofnu mugun &
iuwa \alst{w}rêðan \alst{w}erk \hld\ \alst{w}ópu kúmjan, &
\alst{t}ornon \alst{t}rahnon. \hld\ Noh wirðið þiu \alst{t}íd kuman, &
þat þia \alst{m}uoder þes \hld\ \alst{m}ęndendja sind, &%TODO: check męndendja
\alst{b}rúdi Judeono, \hld\ þem gio \alst{b}arn ni warð &
\alst{ô}dan an \alst{a}ldre. \hld\ Þan gí iuwa \alst{i}n-wid skulun &
\alst{g}rimmo an·\alst{g}eldan; \hld\ þan gí só \alst{g}erna sind, &
þat iu \alst{h}ier bi·\alst{h}lídan \hld\ \alst{h}ôha bergos, &
\alst{d}iopo be·\alst{d}elvan; \hld\ \alst{d}ôð wári iu þan allon &
\alst{l}iovera an þeson \alst{l}ande \hld\ þan su·lik \alst{l}iudjo kwalm &
te gi·\alst{þ}oljanne, \hld\ só hier þan þesaro \alst{þ}ioda kumid.“\eva

\bvb TODO.\evb\evg

\bvg\bva[66][5533]%
Þuȯ sia þár an \alst{g}riete \hld\ \alst{g}algon rihtun, &
an þem \alst{f}elde uppan \hld\ \alst{f}olk Judeono, &
\alst{b}ôm an \alst{b}erẹge, \hld\ ęndi þár an þat \alst{b}arn godes &
\alst{k}węlidun an \alst{k}rúkje: \hld\ slógun \alst{k}ald ísarn, &
\alst{n}iwa \alst{n}aglos \hld\ \alst{n}íðon skarpa &
\alst{h}ardo mid \alst{h}amuron \hld\ þuru is \alst{h}ęndi ęndi þuru is fuoti, &
\alst{b}ittra \alst{b}ęndi: \hld\ is \alst{b}lód ran an erða, &
\alst{d}rôr fan u̇son \alst{d}rohtine. \hld\ Hie ni welda þoh þia \alst{d}ád wrekan &
\alst{g}rimma an þem \alst{J}udeon, \hld\ ak hie þes \alst{g}od fader &
\alst{m}ahtigna bad, \hld\ þat hie ni wári þem \alst{m}anno folke, &
þem \alst{w}erode þiu \alst{w}rêðra: \hld\ „hwand sia ni \alst{w}itun, hwat sia duot“, kwaþ-hie. &
Þuȯ þia \alst{w}ígandos \hld\ gi·\alst{w}ádi Kristes, &
\alst{d}rohtines \alst{d}êldun, \hld\ \alst{d}ęrẹvja mann, &
þes \alst{r}íken gi·\alst{r}ôbi. \hld\ Þia \alst{r}inkos ni mahtun &
umbi þena \alst{s}elvon {[...]} \hld\ \alst{s}am-wurdi gi·sprekan, &
êr sia an iro \alst{h}warạve \hld\ \alst{h}lôtos wurpun, &
\alst{h}wi-lik iro skoldi \alst{h}ębbjan \hld\ þia \alst{h}êlagun pêda, &
allaro gi·\alst{w}ádjo \alst{w}un-samost. \hld\ Þes \alst{w}erodes hirdi &
\alst{h}iet þuȯ, þe \alst{h}ęri-togo, \hld\ ovar þem \alst{h}ôvde selves &
\alst{K}ristes an \alst{k}rúke skrívan, \hld\ þat þat wári \alst{k}uning Judeono, &
Jesus fan \alst{N}azareth-burh, \hld\ þie þár \alst{n}ęglid stuod &
an \alst{n}iwon galgon \hld\ þuru \alst{n}íð-skipi, &
an \alst{b}ômin treo. \hld\ Þuȯ \alst{b}ádun þia liudi &
þat \alst{w}ord \alst{w}ęndjan, \hld\ kwáðun þat hie im só an is \alst{w}illjon spráki, &
\alst{s}elvo \alst{s}agdi, \hld\ þat hie habdi þes gi·\alst{s}ïðes gi·wald, &
\alst{k}uning wári ovar Judeon. \hld\ Þuȯ sprak eft þie \alst{k}êsures bodo, &
\alst{h}ard \alst{h}ęri-togo: \hld\ „it ist iu só ovar is \alst{h}ôvde gi·skrivan, &
\alst{w}ís-líko gi·\alst{w}ritan, \hld\ só ik it nu \alst{w}ęndjan ni mag.“ &
Dádun þuȯ þár te \alst{w}ítje \hld\ \alst{w}erod Judeono &
\alst{t}wêna far·\alst{t}alda man \hld\ an \alst{t}wá halva &
\alst{K}ristes an \alst{k}rúki: \hld\ lietun sia \alst{k}walm þolon &
an þem \alst{w}arạg-trewe \hld\ \alst{w}erko te lône, &
\alst{l}êðaro dádjo. \hld\ Þia \alst{l}iudi sprákun &
\alst{h}osk-word manag \hld\ \alst{h}êlagon Kriste, &
\alst{g}rottun ina mid \alst{g}elpu: \hld\ sáwun allaro \alst{g}umono þen bęston &
\alst{k}węlan an þemo \alst{k}rúkje: \hld\ „ef þú sís \alst{k}uning ovar all“, kwáðun sia, &
„\alst{s}uno drohtines, \hld\ só þú havis \alst{s}elvo gi·sprokan, &
\alst{n}ęri þik fan þero \alst{n}ôdi \hld\ ęndi \alst{n}íðes a·tuomi, &
gang þi \alst{h}êl \alst{h}erod; \hld\ þan węlljat an þik \alst{h}ęliðo barn, &
þesa \alst{l}iudi gi·\alst{l}ôvjan.“ \hld\ Sum imo ôk \alst{l}astar sprak &
swíðo \alst{g}êl-hert \alst{J}udeo, \hld\ þár hie fur þem \alst{g}algon stuod: &
„\alst{W}ah warð þesaro \alst{w}er-oldi“, \hld\ kwaþ-hie, „ef þú iro skoldis gi·\alst{w}ald êgan. &
Þú sagdas þat þú mahtis an \alst{ê}non dage \hld\ \alst{a}ll te·werpan &
þat \alst{h}ôha \alst{h}ús \hld\ \alst{h}evan-kuninges, &
\alst{st}ên-werko mêst \hld\ ęndi eft \alst{st}andan gi·duon &
an \alst{þ}riddjon dage, \hld\ só is elkor ni þorfti bi·\alst{þ}íhan mann &
þeses \alst{f}olkes \alst{f}urðor. \hld\ Sínu hwó þú nu gi·\alst{f}astnod stés, &
\alst{s}wíðo gi·\alst{s}êrid: \hld\ ni maht þi \alst{s}elvon wiht &
\alst{b}alowes gi·\alst{b}uotjan.“ \hld\ Þuȯ þár ôk an þem \alst{b}ęndjon sprak &
þero \alst{þ}eovo ȯðer, \hld\ all só hie þia \alst{þ}ioda gi·hôrda, &
\alst{w}rêðon \alst{w}ordon \hld\ —ne was is \alst{w}illjo guod, &
þes \alst{þ}egnes gi·\alst{þ}ȧht—: \hld\ „ef þú sís \alst{þ}iod-kuning“, kwaþ-hie, &
„\alst{K}rist, godes suno, \hld\ gang þi þan fan þem \alst{k}rúke niðer, &
\alst{s}lópi þi fan þem \alst{s}ímon \hld\ ęndi u̇s \alst{s}amad allon &
\alst{h}ilp ęndi \alst{h}êli. \hld\ Ef þú sís \alst{h}evan-kuning, &
\alst{w}aldand þesaro \alst{w}er-oldes, \hld\ gi·duo it þan an þínon \alst{w}erkon skín, &
\alst{m}ári þik fur þesaro \alst{m}ęnigi.“ \hld\ Þuȯ sprak þero \alst{m}anno ȯðer &
an þero \alst{h}ęnginna, \hld\ þár hie gi·\alst{h}ęftid stuod, &
\alst{w}an \alst{w}undẹr-kwála: \hld\ „Be·hwí wilt þú su·lik \alst{w}ord sprekan, &
\alst{g}ruotis ina mid \alst{g}elpu? \hld\ Stés þí hier an \alst{g}algen haft, &
gi·\alst{b}rokan an \alst{b}ôme. \hld\ Wit hier \alst{b}êðja þolod &
\alst{s}êr þuru unka \alst{s}undjun: \hld\ is unk unkero \alst{s}elvero dád &
\alst{w}orðan te \alst{w}ítje. \hld\ Hie stéd hier \alst{w}ammes lôs, &
allaro \alst{s}undjono \alst{s}ikur, \hld\ só hie \alst{s}elvo gio &
\alst{f}irina ni gi·\alst{f}rumida, \hld\ botan þat hie þuru þeses \alst{f}olkes nið &
\alst{w}illendi an þesaro \alst{w}er-uldi \hld\ \alst{w}íti ant·fáhid. &
Ik willju þár gi·\alst{l}ôvjan tuo“, \hld\ kwaþ-hie, „ęndi willju þena \alst{l}andes ward, &
þena \alst{g}odes suno \hld\ \alst{g}erno biddjan, &
þat þú mín gi·\alst{h}uggjes \hld\ ęndi an \alst{h}elpun sís, &
\alst{r}ádendero bęst, \hld\ þan þú an þín \alst{r}íki kumis: &
wes mi þan gi·\alst{n}áðig.“ \hld\ Þuȯ sprak im eft \alst{n}ęrjendo Krist &
\alst{w}ordon te·gęgnes: \hld\ „Ik sęggju þí te \alst{w}áron hier“, kwaþ-hie, &
„þat þú noh \alst{h}iu-du móst \hld\ an \alst{h}imil-ríke &
mid mí \alst{s}amad \hld\ \alst{s}ehan lioht godes, &
an þemo \alst{P}aradýse, \hld\ þoh þú nu an su·likoro \alst{p}ínu sís.“ &
Þan stuod þár ôk \alst{M}aria, \hld\ \alst{m}uoder Kristes, &
\alst{b}lêk under þem \alst{b}ôme, \hld\ gi·sah iro \alst{b}arn þolon, &
\alst{w}innan \alst{w}undẹr-kwála. \hld\ Ôk wárun þár \alst{w}íf mid iro &
an só \alst{m}ahtiges \hld\ \alst{m}innja kumana— &
þan stuod þár ôk \alst{J}ohannes, \hld\ \alst{j}ungro Kristes, &
\alst{h}riwi undar is \alst{h}êrren, \hld\ was im is \alst{h}ugi sêrag— &
\alst{d}rúvodun fur þem \alst{d}ôðe. \hld\ Þár sprak \alst{d}rohtin Krist &
\alst{m}ahtig te þero \alst{m}uoder: \hld\ „nu ik þí hier \alst{m}ínemo skal &
\alst{j}ungron be·felhan, \hld\ þem þí hier \alst{g}ęgin-ward stéd: &
wis þí an is gi·\alst{s}ïðje \alst{s}amad: \hld\ þú skalt ina furi \alst{s}uno hębbjan.“ &
\alst{G}rótta hie þuȯ \alst{J}ohannes, \hld\ hiet þat hie iru ful-\alst{g}éngi wel, &
\alst{m}innjodi sia só \alst{m}ildo, \hld\ só man is \alst{m}uoder skal, &
\alst{i}dis \alst{u}n·wamma. \hld\ Þuȯ hie sia an is \alst{ê}ra ant·féng &
þuru \alst{h}luttran \alst{h}ugi, \hld\ só im is \alst{h}êrro gi·bôd.\eva

\bvb TODO.\evb\evg

\bvg\bva[67][5622]%
Þuȯ warð þár an \alst{m}iddjan dag \hld\ \alst{m}ahtig têkạn, &
\alst{w}undạr-lík gi·\alst{w}arạht \hld\ ovar þesan \alst{w}er-old allan, &
þuȯ man þena \alst{g}odes suno \hld\ an þena \alst{g}algon huof, &
\alst{K}rist an þat \alst{k}rúki: \hld\ þuȯ warð it \alst{k}u̇ð ovar all, &
hwó þiu \alst{s}unna warð gi·\alst{s}workan: \hld\ ni mahta \alst{s}wigli lioht &
\alst{sk}ôni gi·\alst{sk}ínan, \hld\ ak sia \alst{sk}ado far·féng, &
\alst{þ}imm ęndi \alst{þ}iustri \hld\ ęndi só gi·\alst{þ}rusmod neval. &
Warð allaro \alst{d}ago \alst{d}ruovost, \hld\ \alst{d}unkar swíðo &
ovar þesan \alst{w}ídun \alst{w}er-uld, \hld\ só lango só \alst{w}aldand Krist &
\alst{k}wal an þemo \alst{k}rúkje, \hld\ \alst{k}uningo ríkost, &
ant \alst{n}uon dages. \hld\ Þuȯ þie \alst{n}eval ti·skrêd, &
þat gi·\alst{s}werk warð þuȯ te·\alst{s}wungan, \hld\ bi·gan \alst{s}unnun lioht &
\alst{h}êdron an \alst{h}imile. \hld\ Þuȯ \alst{h}reop up te gode &
allaro \alst{k}uningo \alst{k}raftigost, \hld\ þuȯ hie an þemo \alst{k}rúkje stuod &
\alst{f}aðmon gi·\alst{f}astnot: \hld\ „\alst{f}ader alo-mahtig“, kwaþ-hie, &
„te hwí þú mik só far·\alst{l}ieti, \hld\ \alst{l}ievo drohtin, &
\alst{h}êlag \alst{h}evan-kuning, \hld\ ęndi þína \alst{h}elpa dedos, &
\alst{f}ullisti só \alst{f}err? \hld\ Ik standu under þeson \alst{f}íondon hier &
\alst{w}undron gi·\alst{w}êgid.“ \hld\ \alst{W}erod Judeono &
\alst{h}lógun is im þuȯ te \alst{h}oske: \hld\ gi·\alst{h}ôrdun þena hêlagun Krist, &
\alst{d}rohtin furi þem \alst{d}ôðe \hld\ \alst{d}rinkan biddjan, &
kwað þat ina \alst{þ}urstidi. \hld\ Þiu \alst{þ}ioda ne latta, &
\alst{w}rêða \alst{w}iðar-sakon: \hld\ was im \alst{w}illjo mikil, &
hwat sia im \alst{b}ittres tuo \hld\ \alst{b}ringan mahtin. &
Habdun im \alst{u}n·swóti \hld\ \alst{ę}kid ęndi galla &
gi·\alst{m}ęngid þia \alst{m}ên-hwaton; \hld\ stuod ên \alst{m}ann garo, &
swíðo \alst{sk}uldig \alst{sk}aðo, \hld\ þena habdun sia gi·\alst{sk}ęrid te þiu, &
far·\alst{sp}anan mid \alst{sp}rákon, \hld\ þat hie sia en êna \alst{sp}unsja nam, &
\alst{l}íðo þes \alst{l}êðosten, \hld\ druog it an ênon \alst{l}angan skafte, &
gi·\alst{b}undan an ênon \alst{b}ôme \hld\ ęndi deda it þem \alst{b}arne godes, &
\alst{m}ahtigon te \alst{m}u̇ðe. \hld\ Hie an·kęnda iro \alst{m}irkjun dádi, &
gi·\alst{f}uolda iro \alst{f}égnes: \hld\ \alst{f}urðor ni welda &%TODO: check fégnes
is só \alst{b}ittres an·\alst{b}ítan, \hld\ ak hreop þat \alst{b}arn godes &
\alst{h}lúdo te þem \alst{h}imiliskon fader: \hld\ „ik an þina \alst{h}ęndi be·filhu“, kwaþ-hie, &
„mínon \alst{g}êst an \alst{g}odes willjon; \hld\ hie ist nu \alst{g}aro te þiu, &
\alst{f}u̇s te \alst{f}aranne.“ \hld\ \alst{F}iriho drohtin &
gi·\alst{h}nêgida þuȯ is \alst{h}ôvid, \hld\ \alst{h}êlagon áðom &
\alst{l}iet fan þemo \alst{l}ík-hamen. \hld\ Só þuȯ þie \alst{l}andes ward &
\alst{s}walt an þem \alst{s}ímon, \hld\ só warð \alst{s}án after þiu &
\alst{w}undạr-têkạn gi·\alst{w}arạht, \hld\ þat þár \alst{w}aldandes dôð &
un·\alst{k}weðandes só filo \hld\ ant·\alst{k}ęnnjan skolda, &
þiadnes \alst{ê}n-dagon: \hld\ \alst{e}rða bivoda, &
\alst{h}risidun þia \alst{h}ôhun bergos, \hld\ \alst{h}arda stênos kluvun, &
\alst{f}elisos after þem \alst{f}elde, \hld\ ęndi þat \alst{f}êha lakan te·brast &
an \alst{m}iddjon an twê, \hld\ þat êr \alst{m}anagan dag &
an þemo \alst{w}íhe innan \hld\ \alst{w}undron gi·striunid &
\alst{h}êl \alst{h}angoda \hld\ —ni muostun \alst{h}ęliðo barn, &
þia \alst{l}iudi skawon, \hld\ hwat under þemo \alst{l}akane was &
\alst{h}êlages be·\alst{h}angan: \hld\ þuȯ mohtun an þat \alst{h}orð sehan &
\alst{J}udeo liudi— \hld\ \alst{g}ravu wurðun gi·opanod &
\alst{d}ôdero manno, \hld\ ęndi sia þuru \alst{d}rohtines kraft &
an iro \alst{l}ík-hamon \hld\ \alst{l}ibbjandi a·stuodun &
\alst{u}p fan \alst{e}rðu \hld\ ęndi wurðun gi·\alst{ô}gida þár &
\alst{m}annon te \alst{m}árðu. \hld\ Þat was só \alst{m}ahtig þing, &
þat þár \alst{K}ristes dôð \hld\ ant·\alst{k}ęnnjan skoldun, &
só \alst{f}ilo þes gi·\alst{f}uoljan, \hld\ þie gio mid \alst{f}irihon ne sprak &
\alst{w}ord an þesaro \alst{w}er-oldi. \hld\ \alst{W}erod Judeono &
\alst{s}áwun \alst{s}eld-lík þing, \hld\ ak was im iro \alst{s}líði hugi &
só far·\alst{h}ardod an iro \alst{h}erten, \hld\ þat þár io só \alst{h}êlag ni warð &
\alst{t}êkạn gi·\alst{t}ôgid, \hld\ þat sia \alst{t}rúodin þiu bat &
an þia \alst{K}ristes \alst{k}raft, \hld\ þat hie \alst{k}uning ovar all, &
þes \alst{w}erodes \alst{w}ári. \hld\ Suma sia þár mid iro \alst{w}ordon gi·sprákun, &
þia þes \alst{h}rêwes þár \hld\ \alst{h}uodjan skoldun, &
þat þat \alst{w}ári te \alst{w}áren \hld\ \alst{w}aldandes suno, &
\alst{g}odes \alst{g}egnungo, \hld\ þat þár an þem \alst{g}algon swalt, &
\alst{b}arno þat \alst{b}ęsta. \hld\ Slógun an iro \alst{b}riost filo &
\alst{w}ópjandero \alst{w}ívo: \hld\ was im þiu \alst{w}undẹr-kwála &
\alst{h}arm an iro \alst{h}erten \hld\ ęndi iro \alst{h}êrren dôð &
\alst{s}wíðo an \alst{s}orọgon. \hld\ Þan was \alst{s}ido Judeono, &
þat sia þia \alst{h}aftun þuru þena \alst{h}êlagon dag \hld\ \alst{h}angon ni lietin &
\alst{l}ęngerun \alst{h}wíla, \hld\ þan im þat \alst{l}íf skriði, &
þiu \alst{s}eola be·\alst{s}unki: \hld\ \alst{s}líð-muoda mann &
géngun im mid \alst{n}íð-skipju \alst{n}áhor, \hld\ þár só be·\alst{n}ęglida stuodun &
\alst{þ}eovos twêna, \hld\ \alst{þ}olodun bêðja &
\alst{k}wála bi \alst{K}riste: \hld\ wárun im \alst{k}wika noh þan, &
unt-þat sia þia \alst{g}rimmun \hld\ \alst{J}udeo liudi &
\alst{b}ênon be·\alst{b}rákon, \hld\ þat sia \alst{b}êðja samad &
\alst{l}íf far·\alst{l}ietun, \hld\ suohtun im \alst{l}ioht ȯðer. &
Sia ni þorftun \alst{d}rohtin Krist \hld\ \alst{d}ôðes bêdjan &
\alst{f}urðor mid ênigon \alst{f}irinon: \hld\ fundun ina gi·\alst{f}aranan þuȯ iu: &
is \alst{s}eola was gi·\alst{s}ęndid \hld\ an \alst{s}uȯðan weg, &
an \alst{l}ang-sam \alst{l}ioht, \hld\ is \alst{l}iði kuolodun; &
þat \alst{f}erạh was af þem \alst{f}lêske. \hld\ Þuȯ géng im ên þero \alst{f}íondo tuo &
an \alst{n}íð-hugi, \hld\ druog \alst{n}ęgilid sper &
\alst{h}ard an is \alst{h}andon, \hld\ mid \alst{h}eru-þrummjon stak, &
liet \alst{w}ápnes ord \hld\ \alst{w}undum sníðan, &
þat an \alst{s}elves warð \hld\ \alst{s}ídu Kristes &
ant·\alst{l}okan is \alst{l}ík-hamo. \hld\ Þia \alst{l}iudi gi·sáwun, &
þat þanan \alst{b}luod ęndi water \hld\ \alst{b}êðju sprungun, &
\alst{w}ellun fan þero \alst{w}undun, \hld\ all só is \alst{w}illjo géng &
ęndi hie habda gi·\alst{m}arkod êr \hld\ \alst{m}anno kunnje, &
\alst{f}iriho barnon te \alst{f}rumu: \hld\ þuȯ was it all gi·\alst{f}ullid só.\eva

\bvb TODO.\evb\evg

\bvg\bva[68][5714]%
Só þuȯ gi·\alst{s}êgid warð \hld\ \alst{s}edle náhor &
\alst{h}êdra sunna \hld\ mid \alst{h}evan-tunglon &
an þem \alst{d}ruoven \alst{d}age, \hld\ þuȯ géng im u̇ses \alst{d}rohtines þegạn &
—was im \alst{g}lau \alst{g}umo, \hld\ \alst{j}ungro Kristes &
\alst{m}anaga hwíla, \hld\ só it þár \alst{m}anno filo &
ne \alst{w}issa te \alst{w}áron, \hld\ hwand hie it mid is \alst{w}ordon hal &
\alst{J}uðeono \alst{g}um-skipje: \hld\ \alst{J}oseph was hie hêtan, &
\alst{d}arnungo was hie u̇ses \alst{d}rohtines jungro: \hld\ hie ni welda þero far·\alst{d}uanun þiod &
\alst{f}olgon te ênigon \alst{f}irin-werkon, \hld\ ak hie bêd im under þem \alst{f}olke Judeono, &
\alst{h}êlag \alst{h}imilo ríkjes— \hld\ hie géng im þuȯ wið þena \alst{h}ęri-togon mahljan, &
\alst{þ}ingon wið þena \alst{þ}egạn kêsures, \hld\ \alst{þ}igida ina gerno, &
þat hie muosti a·\alst{l}ôsjan \hld\ þena \alst{l}ík-hamon &
\alst{K}ristes fan þemo \alst{k}rúkje, \hld\ þie þár gi·\alst{k}węlmid stuod, &
þes \alst{g}uoden fan þem \alst{g}algen \hld\ ęndi an \alst{g}raf lęggjan, &
\alst{f}oldu bi·\alst{f}elạhan. \hld\ Im ni welda þie \alst{f}olk-togo þuȯ &
\alst{w}ęrnjan þes \alst{w}illjen, \hld\ ak im gi·\alst{w}ald far·gaf, &
þat hie só muosti gi·\alst{f}rummjan. \hld\ Hie gi·wêt im þuȯ \alst{f}orð þanan &
\alst{g}angan te þem \alst{g}algon, \hld\ þár hie wissa þat \alst{g}odes barn, &
\alst{h}rêo \alst{h}angondi \hld\ \alst{h}êrren sínes, &
nam ina þuȯ an þero \alst{n}iwun ruodun \hld\ ęndi ina fan \alst{n}aglon a·tuomda, &
ant·\alst{f}éng ina mid is \alst{f}aðmon, \hld\ só man is \alst{f}rôhon skal, &
\alst{l}ioves \alst{l}ík-hamon, \hld\ ęndi ina an \alst{l}íne bi·wand, &
\alst{d}ruog ina \alst{d}iur-líko \hld\ —só was þie \alst{d}rohtin werð—, &
þár sia þia \alst{st}ędi havdun \hld\ an ênon \alst{st}êne innan &
\alst{h}andon gi·\alst{h}auwan, \hld\ þár gio \alst{h}ęliðo barn &
\alst{g}umon ne bi·\alst{g}ruovon. \hld\ Þár sia þat \alst{g}odes barn &
te iro \alst{l}and-wísu, \hld\ \alst{l}íko hêlgost &
\alst{f}oldu bi·\alst{f}ulhun \hld\ ęndi mid ênu \alst{f}elisu be·lukun &
allaro \alst{g}ravo \alst{g}uod-líkost. \hld\ \alst{G}riotandi sátun &
\alst{i}disi \alst{a}rm-skapana, \hld\ þia þat \alst{a}ll for·sáwun, &
þes \alst{g}umen \alst{g}rimman dôð. \hld\ Gi·witun im þuȯ \alst{g}angan þanan &
\alst{w}ópjandi \alst{w}íf \hld\ ęndi \alst{w}ara námun, &
hwó sia eft te þem \alst{g}rave \hld\ \alst{g}angan mahtin: &
havdun im far·\alst{s}ewana \hld\ \alst{s}orọga gi·nuogja, &
\alst{m}ikila \alst{m}uod-kara: \hld\ \alst{M}aria wárun sia hêtana, &
\alst{i}disi \alst{a}rm-skapana. \hld\ Þuȯ warð \alst{á}vand kuman, &
\alst{n}aht mid \alst{n}eflu. \hld\ \alst{N}íð-folk Judeono &
warð an \alst{m}orạgan eft, \hld\ \alst{m}ęnigi gi·samnod, &
\alst{r}ękidun an \alst{r}únon: \hld\ „Hwat þú wêst, hwó þit \alst{r}íki was &
þuru þesan \alst{ê}nan man \hld\ \alst{a}ll gi·twíflid, &
\alst{w}erod gi·\alst{w}orran: \hld\ nu ligid hie \alst{w}undon siok, &
\alst{d}iopa bi·\alst{d}olvan. \hld\ Hie sagda simnen, þat hie skoldi fan \alst{d}ôðe a·standan &
an \alst{þ}riddjan dage. \hld\ Þius \alst{þ}iod gi·lôvit te filo, &
þit \alst{w}erod after is \alst{w}ordon. \hld\ Nu þú hier \alst{w}ardon hét, &
ovar þem \alst{g}rave \alst{g}ômjan, \hld\ þat ina is \alst{j}ungron þár &
ne far·\alst{st}elan an þemo \alst{st}êne \hld\ ęndi sęggjan þan, þat hie a·\alst{st}andan sí, &
\alst{r}íki fan \alst{r}aston: \hld\ þan wirðit þit \alst{r}inko folk &
\alst{m}êr gi·\alst{m}ęrrid, \hld\ ef sia it bi·ginnat \alst{m}árjan hier.“ &
Þuȯ wurðun þár gi·\alst{sk}ęrida \hld\ fan þero \alst{sk}olu Judeono &
\alst{w}eros te þero \alst{w}ahtu: \hld\ gi·witun im mid iro gi·\alst{w}ápnjon þarod &
te þem \alst{g}rave \alst{g}angan, \hld\ þár sia skoldun þes \alst{g}odes barnes &
\alst{h}rêwes \alst{h}uodjan. \hld\ Warð þie \alst{h}êlago dag &
\alst{J}udeono far·\alst{g}angan. \hld\ Sia ovar þemo \alst{g}rave sátun, &
\alst{w}eros an þero \alst{w}ahtun \hld\ \alst{w}annom nahton, &
\alst{b}idun undar iro \alst{b}ordon, \hld\ hwan êr þie \alst{b}erẹhto dag &
ovar \alst{m}iddil-gard \hld\ \alst{m}annon kwámi, &
\alst{l}iudon te \alst{l}iohte. \hld\ Þuȯ ni was \alst{l}ang te þiu, &
þat þár warð þie \alst{g}êst kuman \hld\ be \alst{g}odes krafte, &
\alst{h}âlag áðom \hld\ undar þena \alst{h}ardon stên &
an þena \alst{l}ík-hamon. \hld\ \alst{L}ioht was þuȯ gi·opanod &
\alst{f}iriho barnon te \alst{f}rumu: \hld\ was \alst{f}erkal manag &
ant·\alst{h}ęftid fan \alst{h}ęll-doron \hld\ ęndi te \alst{h}imile weg &
gi·\alst{w}arạht fan þesaro \alst{w}er-oldi. \hld\ \alst{W}ánom up a·stuod &
\alst{f}riðu-barn godes, \hld\ \alst{f}uor im þuȯ þár hie welda, &
só þia \alst{w}ardos þes \hld\ \alst{w}iht ni af·swovun, &
\alst{d}ęrvja liudi, \hld\ hwan hie fan þem \alst{d}ôðe a·stuod, &
a·\alst{r}ês fan þero \alst{r}astun. \hld\ \alst{R}inkos sátun &
umbi þat \alst{g}raf útan, \hld\ \alst{J}udeo liudi, &
\alst{sk}ola mid iro \alst{sk}ildjon. \hld\ \alst{Sk}rêd forð-wardes &
\alst{s}wigli \alst{s}unnun lioht. \hld\ \alst{S}íðodun idisi &
te þem \alst{g}rave \alst{g}angan, \hld\ \alst{g}um-kunnjes wíf, &
\alst{M}ariun \alst{m}uni-líka: \hld\ habdun \alst{m}êðmo filo &
gi·\alst{s}ald wiðer \alst{s}alvum, \hld\ \alst{s}ilụvres ęndi goldes, &
\alst{w}erðes wiðer \alst{w}urtjon, \hld\ só sia mahtun a·\alst{w}innan mêst, &
þat sia þena \alst{l}ík-hamon \hld\ \alst{l}ioves hêrren, &
\alst{s}uno drohtines, \hld\ \alst{s}alvon muostin, &
\alst{w}undun \alst{w}ritanan. \hld\ Þiu \alst{w}íf sorạgodun &
an iro \alst{s}evon \alst{s}wíðo, \hld\ ęndi \alst{s}uma sprákun, &
hwie im þena \alst{g}rôtan stên \hld\ fan þemo \alst{g}rave skoldi &
gi·\alst{h}węrevjan an \alst{h}alva, \hld\ þe sia ovar þat \alst{h}rêo sáwun &
þia \alst{l}iudi \alst{l}ęggjan, \hld\ þuȯ sia þena \alst{l}ík-hamon þár &
be·\alst{f}ulhun an þemo \alst{f}elise. \hld\ Só þiu \alst{f}rí havdun &
ge·\alst{g}angan te þem \alst{g}ardon, \hld\ þat sia te þem \alst{g}rave mahtun &
gi·\alst{s}ehan \alst{s}elvon, \hld\ þuȯ þár \alst{s}wógan kwam &
\alst{ę}ngil þes \alst{a}lo-waldon \hld\ \alst{o}vana fan radure, &
\alst{f}aran an \alst{f}eðer-hamon, \hld\ þat all þiu \alst{f}olda an skian, &
þiu \alst{e}rða dunida \hld\ ęndi þia \alst{e}rlos wurðun &
an \alst{w}êkan hugje, \hld\ \alst{w}ardos Juðeono, &
bi·\alst{f}ellun bi þem \alst{f}orạhton: \hld\ ne wándun ira \alst{f}erạh êgan, &
\alst{l}íf \alst{l}angerun hwíl.\eva

\bvb TODO.\evb\evg

\bvg\bva[69][5803]%
\hspace*{100pt} \alst{L}águn þá wardos, &
þia gi·\alst{s}ïðos \alst{s}ám-kwika: \hld\ \alst{s}án up a·hlâd &
þie \alst{g}rôto stên fan þem \alst{g}rave, \hld\ só ina þie \alst{g}odes ęngil &
gi·\alst{h}węrịvida an \alst{h}alva, \hld\ ęndi im uppan þem \alst{h}lêwe gi·sat &
\alst{d}iur-lík \alst{d}rohtines bodo. \hld\ Hie was an is \alst{d}ádjon ge·lík, &
an is \alst{a}n-siunjon, \hld\ só hwem só ina muosta undar is \alst{ô}gon skawon, &
só \alst{b}erẹht ęndi só \alst{b}líði \hld\ all só \alst{b}liksmun lioht; &
was im is gi·\alst{w}ádi \hld\ \alst{w}intạr-kaldon &
\alst{s}nêwe gi·líkost. \hld\ Þuȯ sáwun sia ina \alst{s}ittjan þár, &
þiu \alst{w}íf uppan þem gi·\alst{w}ęndidan stêne, \hld\ ęndi im fan þem \alst{w}litje kwámun, &
þem \alst{i}dison su·lika \alst{ę}gison te·gęgnes: \hld\ \alst{a}ll wurðun fan þem grurje &
þiu \alst{f}rí an \alst{f}orạhton mikilon, \hld\ \alst{f}urðor ne gi·dorstun &
te þemo \alst{g}rave \alst{g}angan, \hld\ êr sia þie \alst{g}odes ęngil, &
\alst{w}aldandes bodo \hld\ \alst{w}ordon gruotta, &
kwað þat hie iro \alst{â}rundi \hld\ \alst{a}ll bi·kunsti, &
\alst{w}erk ęndi \alst{w}illjon \hld\ ęndi þero \alst{w}ívo hugi, &
hiet þat sia im ne an·\alst{d}rédin: \hld\ „ik wêt þat gí iuwan \alst{d}rohtin suokat, &
\alst{n}ęrjendon Krist \hld\ fan \alst{N}azareth-burg, &
þena þi hier \alst{k}węlidun \hld\ ęndi an \alst{k}rúki slógun &
\alst{J}udeo liudi \hld\ ęndi an \alst{g}raf lagdun &
\alst{s}undi-lôsjan. \hld\ Nu nist hie \alst{s}elvo hier, &
ak hie ist a·\alst{st}andan iu, \hld\ ęndi sind þesa \alst{st}ędi lárja, &%NOTE ms. -- a·standan] L 1r.
þit \alst{g}raf an þeson \alst{g}riote. \hld\ Nú mugun gí \alst{g}angan herod &
\alst{n}áhor mikilu \hld\ —ik wêt þat is iu ist \alst{n}iud sehan &
an þeson \alst{st}êne innan—: \hld\ hier sind noh þia \alst{st}ędi skína, &
þár is \alst{l}ík-hamo \alst{l}ag.“ \hld\ \alst{L}ungra féngun &
gi·\alst{b}ada an iro \alst{b}rioston \hld\ \alst{b}lêka idisi, &
\alst{w}liti-skôni \alst{w}íf: \hld\ was im \alst{w}il-spell mikil &
te gi·\alst{h}ôrjanne, \hld\ þat im fan iro \alst{h}êrren sagda &
\alst{ę}ngil þes \alst{a}lo-walden. \hld\ Hiet sia \alst{e}ft þanan &
fan þem \alst{g}rave \alst{g}angan ęndi faran \hld\ te þem \alst{j}ungron Kristes, &
\alst{s}ęggjan þem is gi·\alst{s}ïðon \hld\ \alst{s}uoðon wordon, &
þat iro \alst{d}rohtin was \hld\ fan \alst{d}ôðe a·standan. &
Hiet ôk an \alst{s}undron \hld\ \alst{S}ímon Petruse &
\alst{w}ill-spell mikil \hld\ \alst{w}ordon ku̇ðjan, &
\alst{k}umi drohtines, \hld\ gie þat \alst{K}rist selvo &
was an \alst{G}alileo land, \hld\ „þár ina eft is \alst{j}ungron skulun, &
gi·\alst{s}ehan is gi·\alst{s}ïðos, \hld\ só hie im êr \alst{s}elvo gi·sprak &
\alst{w}árom \alst{w}ordon.“ \hld\ Reht só þuȯ þiu \alst{w}íf þanan &
\alst{g}angan weldun, \hld\ só stuodun im te·\alst{g}ęgnes þár &
\alst{ę}ngilos twêna \hld\ an \alst{a}la-hwíton &
\alst{w}ánamon gi·\alst{w}ádjom \hld\ ęndi sprákun im mid iro \alst{w}ordon tuo &
\alst{h}êlag-líko: \hld\ \alst{h}ugi warð gi·blôðid &
þen \alst{i}dison an \alst{ę}gison: \hld\ ne mahtun an þia \alst{ę}ngilos godes &
bi þemo \alst{w}lite skawon: \hld\ was im þiu \alst{w}ánami te strang, &%NOTE ms. -- strang] L 1v.
te \alst{s}wíði te \alst{s}ehanne. \hld\ Þuȯ sprákun \edtext{im \alst{s}án}{\Afootnote{so C; om. L}} an·gęgin &
\alst{w}aldandes bodun \hld\ ęndi þiu \alst{w}íf frágodun, &
te hwí sia \alst{K}ristan þarod \hld\ \alst{k}wikan mid dôdon, &
\alst{s}uno drohtines \hld\ \alst{s}uokjan kwámin &
\alst{f}erạhes \alst{f}ullan; \hld\ „nu gí ina ni \alst{f}indat hier &
an þeson \alst{st}ên-grave, \hld\ ak hie ist a·\alst{st}andan nu &
an is \alst{l}ík-hamon: \hld\ þes gí gi·\alst{l}ôvjan skulun &
ęndi gi·huggjan þero \alst{w}ordo, \hld\ þe hie iu te \alst{w}áron oft &
\alst{s}elvo \alst{s}agda, \hld\ þan hie an iuwon ge·\alst{s}ïðja was &
an \alst{G}alilea-lande, \hld\ hwó hie skoldi gi·\alst{g}evan werðan, &
gi·\alst{s}ald \alst{s}elvo \hld\ an \alst{s}undigaro manno, &
\alst{h}ęttjandero hand, \hld\ \alst{h}êlag drohtin, &
þat sea ina \alst{k}węlidin \hld\ ęndi an \alst{k}rúki slógin, &
\alst{d}ôdan gi·\alst{d}ádin \hld\ ęndi þat hie skoldi þuruh \alst{d}rohtines kraft &
an \alst{þ}riddjon dage \hld\ \alst{þ}ioda te willjan &
\alst{l}ibbjandi a·standan. \hld\ Nu havat hie all gi·\alst{l}êstid só, &
ge·\alst{f}rumid mid \alst{f}irihon: \hld\ íljat gi nu \alst{f}orð hinan, &
\alst{g}angat \alst{g}áh-líko \hld\ ęndi duot it þem is \alst{j}ungron ku̇ð.\eva

\bvb TODO.\evb\evg

\bvg\bva[70][5866]%
Hie havat sia iu fur·\alst{f}arana \hld\ ęndi ist im \alst{f}orð hinan &
an \alst{G}alileo land, \hld\ þár ina eft is \alst{j}ungron skulun, &
gi·\alst{s}ehan is ge·\alst{s}ïðos.“ \hld\ Þuȯ warð \edtext{\alst{s}án}{\Afootnote{so L; om. C}} after þiu &
þem \alst{w}ívon an \alst{w}illjon, \hld\ þat sia gi·hôrdun su·lik \alst{w}ord sprekan, &
\alst{k}u̇ðjan þia \alst{k}raft godes \hld\ —wárun im só a·\alst{k}umana þuȯ noh &
gie só \alst{f}orạhta ge·\alst{f}rumida—: \hld\ gi·witun im \alst{f}orð þanan &%NOTE ms. -- forahta] L end.
fan þem \alst{g}rave \alst{g}angan \hld\ ęndi sagdun þem \alst{j}ungron Kristes &
\alst{s}eld-lík gi·\alst{s}iuni, \hld\ þár sia \alst{s}orọgondi &
\alst{b}idun su·likero \alst{b}uota. \hld\ Þuȯ wurðun ôk an þia \alst{b}urg kumana &
\alst{J}udeono wardos, \hld\ þia ovar þemo \alst{g}rave sátun &
alla \alst{l}anga naht \hld\ ęndi þes \alst{l}ík-hamen þár, &
\alst{h}uodun þes \alst{h}rêwes. \hld\ Sia sagdun þero \alst{h}êri Judeono, &
hwi-lika im þár \alst{a}nd-warda \hld\ \alst{ę}gison kwámun, &
\alst{s}eld-lík gi·\alst{s}iuni, \hld\ \alst{s}agdun mid wordon, &
al só it gi·\alst{d}uan was \hld\ an þero \alst{d}rohtines kraft, &
ni \alst{m}iðun an iro \alst{m}uode. \hld\ Þuȯ budun im \alst{m}êðmo filo &
\alst{J}udeo liudi, \hld\ \alst{g}old ęndi silụvar, &
\alst{s}aldun im \alst{s}ink manag, \hld\ te þiu þat sia it ni \alst{s}agdin forð, &
ne \alst{m}áridin þero \alst{m}ęnigi: \hld\ „ak kweðat þat iu \alst{m}óði hugi &
an·\alst{s}wevidi mid \alst{s}lápu \hld\ ęndi þat þár kwámin is gi·\alst{s}ïðos tuo, &
far·\alst{st}álin ina an þem \alst{st}êne. \hld\ Simnen wesat gí an \alst{st}ríde mid þiu, &
\alst{f}orð an \alst{f}líte: \hld\ ef it wirðit þem \alst{f}olk-togen ku̇ð, &
wí gi·\alst{h}elpat iu wið þena \alst{h}êrosten, \hld\ þat hie iu \alst{h}armes wiht, &
\alst{l}êðes ni gi·\alst{l}êstid.“ \hld\ Þuȯ námun sia an þem \alst{l}iudon filo &
\alst{d}iurero mêðmo, \hld\ \alst{d}ádun all só sia bi·gunnun &
—ne gi·\alst{w}eldun iro \alst{w}illjon— \hld\ dádun só \alst{w}ído ku̇ð &
þem \alst{l}iudon after þem \alst{l}ande, \hld\ þat sia su·lika \alst{l}ugina woldun &
a·\alst{h}ębbjan be þan \alst{h}êlagan drohtin. \hld\ Þan was eft gi·\alst{h}êlid hugi &
\alst{j}ungron Kristes, \hld\ þuȯ sia gi·hôrdun þiu \alst{g}uodun wíf &
\alst{m}árjan þia \alst{m}aht godes; \hld\ þuȯ wárun sia an iro \alst{m}uode fráha, &
gie im te þem \alst{g}rave bêðja, \hld\ \alst{J}ohannes ęndi Petrus &
runnun \alst{o}vast-líko: \hld\ warð \alst{ê}r kuman &
\alst{J}ohannes þie \alst{g}uodo, \hld\ ęndi im ovar þem \alst{g}rave gi·stuod, &
ant-at þár \alst{s}án after kwam \hld\ \alst{S}ímon Petrus, &
\alst{e}rl \alst{ę}llan-ruof \hld\ ęndi im þár \alst{i}n gi·wêt &
an þat \alst{g}raf \alst{g}angan: \hld\ gi·sah þár þes \alst{g}odes barnes, &
\alst{h}rêo-gi·wádi \hld\ \alst{h}êrren sínes &
\alst{l}ínin \alst{l}iggjan, \hld\ mid þiu was êr þie \alst{l}ík-hamo &
\alst{f}agạro bi·\alst{f}angan; \hld\ lag þie \alst{f}ano sundạr, &
mit þem was þat \alst{h}ôvid bi·\alst{h}elid \hld\ \alst{h}êlages Kristes, &
\alst{r}íkjes drohtines, \hld\ þan hie an þesaro \alst{r}astu was. &
Þuȯ \alst{g}éng im ôk \alst{J}ohannes \hld\ an þat \alst{g}raf innan &
\alst{s}ehan \alst{s}eld-lík þing; \hld\ warð im \alst{s}án after þiu &
ant·\alst{l}okan is gi·\alst{l}ôvo, \hld\ þat hie wissa, þat skolda eft an þit \alst{l}ioht kuman &
is \alst{d}rohtin diur-líko, \hld\ fan \alst{d}ôðe a·standan &
\alst{u}p fan \alst{e}rðu. \hld\ Þuȯ gi·witun im \alst{e}ft þanan &
\alst{J}ohannes ęndi Petrus, \hld\ ęndi kwámun þia \alst{j}ungron Kristes, &
þia gi·\alst{s}ïðos te·\alst{s}amne. \hld\ Þan stuod \alst{s}êrag-muod &
\alst{ê}n þera \alst{i}diso \hld\ \alst{ȯ}ðer-sïðu &
\alst{g}riotandi ovar þem \alst{g}rave, \hld\ was iro \alst{j}ámar muod— &
\alst{M}aria was þat \alst{M}agdalena—, \hld\ was iro \alst{m}uod-gi·þȧht, &
\alst{s}evo mit \alst{s}orọgon gi·blandan, \hld\ ne wissa hwarod siu \alst{s}ókjan skolda &
þena \alst{h}êrron, þár iro wárun at þia \alst{h}elpa gi·langa. \hld\ Siu ni mohta þuȯ \alst{h}ofnu a·wísan, &
þat \alst{w}íf ni mahta \alst{w}óp for·látan: \hld\ ne wissa hwarod siu sia \alst{w}ęndjan skolda; &
gi·\alst{m}ęrrid wárun iro þes \alst{m}uod-gi·þȧhti. \hld\ Þuȯ gi·sah siu þena \alst{m}ahtigan þár &
\alst{K}riste standan, \hld\ þuoh siu ina \alst{k}u̇ð-líko &
ant·\alst{k}ęnnjan ni mohti, \hld\ êr þan hie ina \alst{k}u̇ðjan welda, &
\alst{s}ęggjan þat hie it \alst{s}elvo wári. \hld\ Hie frágoda hwat siu só \alst{s}êro bi·wiepi, &
só \alst{h}armo mid \alst{h}êton trahnin. \hld\ Siu kwað, þat siu umbi iro \alst{h}êrron ni wissi &
te \alst{w}áren, hwarod hie \alst{w}erðan skoldi: \hld\ „ef þú ina mí gi·\alst{w}ísan mohtis, &
\alst{f}rô mín, ef ik þik \alst{f}rágon gi·dorsti, \hld\ ef þú ina hier an þeson \alst{f}elise gi·námis, &
\alst{w}ísi ina mí mid \alst{w}ordon þínon: \hld\ þan wári mí allaro \alst{w}illjono mêsta, &
þat ik ina \alst{s}elvo gi·\alst{s}áhi.“ \hld\ Sia ni wissa, þat sia þie \alst{s}uno drohtines &
\alst{g}ruotta mid \alst{g}ódaro sprákun: \hld\ siu wánda þat it þie \alst{g}ardari wári, &
\alst{h}of-ward \alst{h}êrren sínes. \hld\ Þuȯ gruotta sia þie \alst{h}êlago drohtin, &
bi \alst{n}amen \alst{n}ęrjendero bęst: \hld\ siu géng im þuȯ \alst{n}áhor sniumo, &
þat \alst{w}íf mid \alst{w}illjon guodan, \hld\ ant·kęnda iro \alst{w}aldand selvan, &
\alst{m}íðan siu is þuru þia \alst{m}innja ni wissa: \hld\ welda ina mid iro \alst{m}undon grípan, &
þiu \alst{f}êhmja an þena \alst{f}olko drohtin, \hld\ novan þat iro \alst{f}riðu-barn godes &
\alst{w}ęrida mid \alst{w}ordon sínon, \hld\ kwað þat siu ina mid \alst{w}ihti ni mósti &
\alst{h}andon ant·\alst{h}rínan: \hld\ „ik ni stêg noh“, kwaþ-hie, „te þem \alst{h}imiliskon fader; &
ak \alst{í}li þú nu \alst{o}fst-líko \hld\ ęndi þem \alst{e}rlon ku̇ði, &
\alst{b}ruoðron mínon, \hld\ þat ik u̇ser \alst{b}êðero fader &
\alst{a}la-waldan, \hld\ \alst{i}uwan ęndi mínan &
\alst{s}uȯð-fastan god \hld\ \alst{s}uokjan willju.“\eva

\bvb TODO.\evb\evg

\bvg\bva[71][5941]%
Þat \alst{w}íf warð þuȯ an \alst{w}unnon, \hld\ þat siu muosta su·likan \alst{w}illjon ku̇ðjan, &
\alst{s}ęggjan fan im gi·\alst{s}undon: \hld\ warð \alst{s}án garo &
þiu \alst{i}dis an þat \alst{â}rundi \hld\ ęndi þem \alst{e}rlon brȧhta, &
\alst{w}ill-spel \alst{w}eron, \hld\ þat siu \alst{w}aldand Krist &
gi·\alst{s}undan gi·\alst{s}áwi, \hld\ ęndi sagda hwó hé iru \alst{s}elvo gi·bôd &
\alst{t}orọhtero \alst{t}êkno. \hld\ Sia ni weldun gi·\alst{t}rúojan þuȯ noh &
þes \alst{w}íves \alst{w}ordon, \hld\ þat siu su·lik \alst{w}ill-spel brȧhte &
\alst{g}egnungo fan þemo \alst{g}odes suno, \hld\ ak sia sátun im \alst{j}ámor-muoda, &
\alst{h}ęliðos \alst{h}riwonda. \hld\ Þuȯ warð þie \alst{h}êlago Krist &
eft \alst{o}pan-líko \hld\ \alst{ȯ}ðer-sïðu, &
\alst{d}rohtin gi·tôgid, \hld\ sïðor hie fan \alst{d}ôðe a·stuod, &
þan \alst{w}ívon an \alst{w}illjon, \hld\ þat hie im þár an \alst{w}ege muotta. &
\alst{k}wędda sia \alst{k}u̇ð-líko, \hld\ ęndi sia te is \alst{k}neohon hnigun, &
\alst{f}ellun im tó \alst{f}uoton. \hld\ Hie hét þat sia \alst{f}orạhtan hugi &
ne \alst{b}árin an iro \alst{b}rioston: \hld\ „ak gí mínon \alst{b}ruoðron skulun &
þesa \alst{k}widi \alst{k}u̇ðjan, \hld\ þat sia \alst{k}uman after mi &
an \alst{G}alileo land; \hld\ þár ik im eft te·\alst{g}ęgnes biun.“ &
Þan fuorun im ôk fan \alst{J}erusalem \hld\ þero \alst{j}ungrono twêna &
an þem \alst{s}elvon daga \hld\ \alst{s}án an morgan, &
\alst{e}rlos an iro \alst{â}rundi: \hld\ weldun im te \alst{E}maus &
þat \alst{k}astel suokan. \hld\ Þuȯ bi·gunnun im \alst{k}widi managa &
under þem \alst{w}eron \alst{w}ahsan, \hld\ þár sia after þem \alst{w}ege fuorun, &
þem \alst{h}ęliðon umbi iro \alst{h}êrron. \hld\ Þuȯ kwam im þár þie \alst{h}êlago tuo &
\alst{g}angandi \alst{g}odes suno. \hld\ Sia ni mahtun ina \alst{g}aro-líko &
ant·\alst{k}ęnnan \alst{k}raftigna: \hld\ hie ni welda ina þuȯ noh \alst{k}u̇ðjan te im; &
was im þoh an iro gi·\alst{s}ïðje \alst{s}amad \hld\ ęndi frágoda, umbi hwi-lika sia \alst{s}aka sprákin: &
„hwí \alst{g}angat gí só \alst{g}ornondja?“ \hld\ kwaþ-hie; „Ist ink \alst{j}ámer hugi, &
\alst{s}evo \alst{s}orạgono full.“ \hld\ Sia sprákun im \alst{s}án an·gęgin, &
þia \alst{e}rlos \alst{a}nd·wurdi: \hld\ „te hwí þú þes \alst{ê}skos só“, kwáðun sia; &
„bist þí fan \alst{J}erusalem \hld\ \alst{J}udeono folkas &
\skipnumbering{[...]}“\eva

\bvb TODO.\evb\evg

\bvg\bva[][5971]%
\skipnumbering„{[...]} &
\alst{h}êlagumu gêste \hld\ fan \alst{h}evan-wange, &
mid þem \alst{g}rôtun \alst{g}odes kraft.“ \hld\ Nam is \alst{j}ungaron þȯ, &
\alst{e}rlos góde, \hld\ lêdda sie \alst{ú}t þanan, &
an-tat hé sie \alst{b}rȧhte \hld\ an \alst{B}ethanía; &
þár \alst{h}óf hé is \alst{h}ęndi up \hld\ ęndi \alst{h}êlegoda sie alle, &
\alst{w}íhida sie mid is \alst{w}ordun. \hld\ Gi·\alst{w}êt imo up þanan, &
sóhta imo þat \alst{h}ôha \alst{h}imilo ríki \hld\ ęndi þena is \alst{h}êlagon stól: &
\alst{s}itit imo þár \hld\ an þea \alst{s}wíðron half godes, &
\alst{a}lo-mahtiges fader \hld\ ęndi þanan \alst{a}ll ge·sihit &
\alst{w}aldandjo Krist, \hld\ só hwat só þius \alst{w}er-old be·havet. &
Þȯ an þeru \alst{s}elvon stędi \hld\ ge·\alst{s}ïðos góde &
te \alst{b}edu fellun \hld\ ęndi im eft te \alst{b}urg þanan &
þár te \alst{J}erusalem \hld\ \alst{j}ungaron Kristes &
\alst{f}órun \alst{f}aganondi: \hld\ was im \alst{f}ráh-mód hugi, &
\alst{w}árun im þár at þemu \alst{w}íhe. \hld\ \alst{W}aldandes kraft &
{[...]}\eva

\bvb TODO.\evb\evg

\sectionline
%
	\bookStart{Muspilli}

\begin{flushright}%
\textbf{Dating:} 800s

\textbf{Meter:} \Fornyrdislag%para
\end{flushright}%

Found in the margins of a single theological manuscript from the 820s, \emph{CLM 14098}.

The second sound shift is applied consistently.  That this was the case at composition is seen by the alliteration between Latin words starting with \emph{p-} and Germanic words which originally began with \emph{b-}:

\begin{itemize}
  \item l. 16: Germanic \emph{pú} (= OE, ON \emph{bú}) with borrowed \emph{pardísu} (< Latin \emph{paradīsum}),
  \item l. 21: Germanic \emph{piutit} (= OE \emph{bíett}, ON \emph{býðr}) with borrowed \emph{pehhes} (< Latin \emph{pix}) and \emph{pína} (< Latin \emph{poena}),
  \item l. 25: Germanic \emph{prinnan} (= OE \emph{biernan}, ON \emph{brinna}), \emph{palw-} (= OE \emph{bealu}, ON \emph{bǫlv-}) with borrowed \emph{pehhe} (see above).
\end{itemize}

\sectionline

\bvg\bva Sín \alst{t}ak pi·kweme, \hld\ daz er \alst{t}ouwan skal. &
Wanta \alst{s}ár só sih diu \alst{s}êla \hld\ in den \alst{s}ind ar·hęvit, &
ęnti si den \alst{l}íh-hamun \hld\ \alst{l}ikkan lázzit, &
só kwimit ęin \alst{h}ęri \hld\ fona \alst{h}imil-zungalon; &
daz andar fona \alst{p}ehhe: \hld\ dár \alst{p}ágant siu umpi. &
\alst{S}orgén mak diu \alst{s}êla, \hld\ unzi diu \alst{s}uona ar·gét, &
za wederemo \alst{h}ęrje \hld\ si gi·\alst{h}alót werde. &
Wanta ipu sia daz \alst{S}atanazses \hld\ ki·\alst{s}indi ki·winnit, &
daz \alst{l}ęitit sia sár \hld\ dár iru \alst{l}ęid wirdit, &
in \alst{f}uir ęnti in \alst{f}instrí: \hld\ daz ist rehto \alst{v}irin-líh ding. &
Upi sia avar ki·\alst{h}alónt die \hld\ die dár fona \alst{h}imile kwemant, &
ęnti si dero \alst{ę}ngilo \hld\ \alst{ęi}gan wirdit, &
die pringent sia sár úf in himilo ríhi: &
dár ist \alst{l}íp áno tôd, \hld\ \alst{l}ioht áno finstrí, &
\alst{s}ęlida áno \alst{s}orgun: \hld\ dár n·ist neo-man \alst{s}iuh. &
Denne der man in \alst{p}ardísu \hld\ \alst{p}ú ki·winnit, &
\alst{h}ús in \alst{h}imile, \hld\ dár kwimit imo \alst{h}ilfa ki·nuok. &
Pi·diu ist durft mihhil allero \alst{m}anno we-líhemo, \hld\ daz in es sín \alst{m}uot ki·spane, &%NOTE: daz] 119v
daz er \alst{k}otes willun \hld\ \alst{k}erno tuoo &
ęnti \alst{h}ęlla fuir \hld\ \alst{h}arto wíse, &
\alst{p}ehhes \alst{p}ína: \hld\ dár \alst{p}iutit der Satanasz altist &
\alst{h}ęizzan lauk. \hld\ Só mak \alst{h}ukkan za diu, &%NOTE: za] 120r
\alst{s}orgén dráto, \hld\ der sih \alst{s}untigen węiz. &
Wê demo in \alst{v}instrí skal \hld\ síno \alst{v}iriná stúén, &
\alst{p}rinnan in \alst{p}ehhe: \hld\ daz ist rehto \alst{p}alwík dink, &
daz der man \alst{h}arét ze gote \hld\ ęnti imo \alst{h}ilfa ni kwimit. &
\alst{W}ánit sih ki·náda \hld\ diu \alst{w}ênaga sêla: &%NOTE: wánit] 120v
ni ist in ki·\alst{h}uktin \hld\ \alst{h}imiliskin gote, &
wanta hiar in \alst{w}er-olti \hld\ after ni \alst{w}erkóta. &
Só denne der \alst{m}ahtigo khunink \hld\ daz \alst{m}ahal ki·pannit, &
dara skal \alst{k}weman \hld\ \alst{kh}unno ki·líhaz: &
denne ni ki·tar \alst{p}arno nohhęin \hld\ den \alst{p}an furi·sizzan, &
ni allero \alst{m}anno we-líh \hld\ ze demo \alst{m}ahale skuli. &
Dár skal er vora demo \alst{r}íhhe \hld\ az \alst{r}ahhu stantan, &
pí daz er in \alst{w}er-olti eo \hld\ ki·\alst{w}erkót hapéta. &
Daz hôrt’ ih \alst{r}ahhón \hld\ dia wer-olt-\alst{r}eht-wíson, &
daz skuli der \alst{a}nti-khristo \hld\ mit \alst{E}líase págan. &
Der \alst{w}arkh ist ki·\alst{w}áfanit, \hld\ denne wirdit untar in \alst{w}ík ar·hapan. &
\alst{Kh}ęnfun sint só \alst{k}ręftík; \hld\ diu \alst{k}ósa ist só mihhil. &
\alst{E}lías strítit \hld\ pí den \alst{ê}wigon líp, &
wili dén \alst{r}eht-kernón \hld\ daz \alst{r}íhhi ki·starkan: &
pi·diu skal imo \alst{h}elfan \hld\ der \alst{h}imiles ki·waltit. &
Der \alst{A}nti-khristo \hld\ stét pí demo \alst{a}lt-fíante, &
stét pí demo \alst{S}atanase, \hld\ der inan var·\alst{s}enkan skal: &
pi·diu skal er in deru \alst{w}ík-stęti \hld\ \alst{w}unt pi·vallan &
ęnti in demo \alst{s}inde \hld\ \alst{s}iga-lôs werdan. &
Doh wánit des vilo got-manno, &
daz Elías in demo \alst{w}íge \hld\ ar·\alst{w}artit werde. &
Só daz \alst{E}líases pluot \hld\ in \alst{e}rda ki·triufit, &
só in·\alst{p}rinnant die \edtext{\alst{p}erga, \hld\ \alst{p}oum}{\lemma{perga \dots\ poum ‘mountains \dots woods’}\Bfootnote{Formulaic word-pair; see note to \Muspilli\ 3.}} ni ki·stęntit &
\alst{ê}nihk in \alst{e}rdu, \hld\ \alst{a}há ar·truknént, &
muor var·\alst{s}wilhit sih, \hld\ \alst{s}wilizót lougiu der himil, &
\alst{m}áno vallit, \hld\ prinnit \alst{m}ittila-gart, &
\alst{st}ên ni ki·\alst{st}ęntit, \hld\ vęrit denne \alst{st}úa-tago in lant, &
\alst{v}ęrit mit diu \alst{v}uiru \hld\ \alst{v}iriho wísón: &
dár ni mak denae \alst{m}ák andremo \hld\ helfan vora demo \alst{M}úspille. &
Denne daz \alst{p}ręita wasal \hld\ allaz var·\alst{p}rinnit, &
ęnti vuir ęnti luft \hld\ iz allaz ar·furpit. &
Wár ist denne diu \alst{m}arha, \hld\ dár man dár eo mit sínén \alst{m}ágon piehk? &
Diu marha ist far·prunnan, \hld\ diu sêla stét pi·dungan, &
ni węiz mit wiu puaze: \hld\ só vęrit sí za wíze. &
Pi·diu ist demo \alst{m}anne só guot, \hld\ denner ze demo \alst{m}ahale kwimit, &
daz er \alst{r}ahóno we-líha \hld\ \alst{r}ehto ar·tęile. &
Denne ni darf er \alst{s}orgén, \hld\ denne er ze deru \alst{s}uonu kwimit. &
Ni \alst{w}ęiz der \alst{w}ênago man, \hld\ wie-líhan \alst{w}artil er habét, &
denner mit den \alst{m}iatón \hld\ \alst{m}arrit daz rehta, &
daz der \alst{t}iuval dár pí \hld\ ki·\alst{t}arnit stęntit. &
Der hapét in \alst{r}uovu \hld\ \alst{r}ahóno we-líha, &
daz der man \alst{ê}r ęnti síd \hld\ \alst{u}piles ki·frumita, &
daz er iz allaz ki·\alst{s}agét, \hld\ denne er ze deru \alst{s}uonu kwimit; &
ni skolta síd \alst{m}anno nohhęin \hld\ \alst{m}iatun int·fáhan. &
Só daz \alst{h}imiliska \alst{h}orn \hld\ \edtrans{ki·\alst{h}lútit}{sounds}{\Afootnote{\emph{kilutit} ms.}\Bfootnote{Restoration of the cluster \emph{hl-} is required by the alliteration.}} wirdit, &
ęnti sih der \alst{s}uanari \hld\ ana den \alst{s}ind ar·hęvit &
der dár suannan skal \hld\ tôten ęnti lepentén, &
denne \alst{h}ęvit sih mit imo \hld\ \alst{h}ęrjo męista, &
daz ist allaz só \alst{p}ald, \hld\ daz imo nio-man ki·\alst{p}ágan ni mak. &
Denne vęrit er ze deru \alst{m}ahal-stęti, \hld\ deru dár ki·\alst{m}arkhót ist: &
dár wirdit diu \alst{s}uona, \hld\ dia man dár io \alst{s}agéta. &
Denne varant \alst{ę}ngila \hld\ \alst{u}per dio marha, &
\alst{w}ękhant deota, \hld\ \alst{w}íssant ze dinge. &
Denne skal \alst{m}anno gi·líh \hld\ fona deru \alst{m}oltu ar·stén, &
\alst{l}ôssan sih ar dero \alst{l}éwo vazzón: \hld\ skal imo avar sín \alst{l}íp pi·kweman, &
daz er sín \alst{r}eht allaz \hld\ ki·\alst{r}ahhón muozzi, &
ęnti imo after sínén \alst{t}átin \hld\ ar·\alst{t}ęilit werde. &
Denne der gi·\alst{s}izzit, \hld\ der dár \alst{s}uonnan skal &
ęnti ar·\alst{t}ęillan skal \hld\ \alst{t}ôtén ęnti kwekkhén, &
denne stét dár \alst{u}mpi \hld\ \alst{ę}ngilo męnigí, &
\alst{g}uotero \alst{g}omóno: \hld\ \alst{g}art ist só mihhil: &
dara kwimit ze deru \alst{r}ihtungu só vilo \hld\ dia dár ar \alst{r}ęstí ar·stént. &
Só dár \alst{m}anno nohhęin \hld\ wiht pi·\alst{m}ídan ni mak, &
dár skal denne \alst{h}ant sprehhan, \hld\ \alst{h}oupit sagén, &
allero \alst{l}ido we-líhk \hld\ unzi in den \alst{l}uzígun vinger, &
waz er untar desen \alst{m}annun \hld\ \alst{m}ordes ki·frumita. &
Dár ni ist eo só \alst{l}istík man \hld\ der dár io·wiht ar·\alst{l}iugan męgi, &
daz er ki·\alst{t}arnan męgi \hld\ \alst{t}áto dehhęina, &
niz al fora demo \alst{kh}uninge \hld\ ki·\alst{kh}undit werde, &
\alst{ú}zzan er iz \hld\ mit \alst{a}lamusanu furi·męgi &
ęnti mit \alst{f}astún \hld\ dio \alst{v}iriná ki·puazti. &
Denne der \alst{p}aldét \hld\ der gi·\alst{p}uazzit hapét, &
denner ze deru suonu kwimit. &
Wirdit denne \alst{f}uri ki·tragan \hld\ daz \alst{f}rôno khrúki, &
dár der \alst{h}êligo Khrist \hld\ ana ar·\alst{h}angan ward. &
Denne augit er dio \alst{m}ásún, \hld\ dio er in deru \alst{m}ęnniskí an·fénk, &
dio er duruh desse \alst{m}an-kunnes \hld\ \alst{m}inna far·doléta.\eva

\bvb TODO: Split into multiple parts. Translate.\evb\evg

\sectionline

	\bookStart{The Wessobrunner Hymn}

\begin{flushright}%
Dating: late C8th

Meter: \Fornyrdislag%para
\end{flushright}%

This text can be split into two parts, the “poem” and the “prayer”. Following my principle of including sources rather than excluding (TODO: see Introduction), I here present both.

The first part is a short alliterative poem describing the earliest beginning of the world. The poet describes “the greatest of wonders”, namely that the universe began as a void, where neither earth nor heaven existed. In this void was, however, the almighty God, along with his many spirits (presumably the Heavenly Host or the Angels). While the cosmogony expressed is clearly Jewish-Christian rather than Germanic, the poem does contain two word-pairs also found in Norse Heathen stanzas about the creation of the world (see Notes to ll. 2, 3.), which may point toward a repurposing of older Heathen motifs and expressions in the new, Christian context.

The second part is a thoroughly Christian prayer. The author first thanks God for creating the earth and heaven, this is presumably why the poem was included, and for giving good things to mankind. He then asks for faith, strength and wisdom to “withstand devils”, “reproach degeneracy” and “work [God’s] will”.

\sectionline

\bvg
\bva[0]Dat ga·\alst{f}regin ih mit \alst{f}irahim · \alst{f}iri·wizzó męista, &
dat \edtext{\edtext{\alst{e}rdo}{\Afootnote{\emph{ero} ms.}} ni was · noh \alst{ú}f-himil}{\lemma{erdo \dots\ úf-himil ‘earth \dots\ up-heaven’}\Bfootnote{A formulaic merism attested across the Germanic world, expressing the totality of the universe. Cf. especially \Vafthrudnismal\ 21, where the god Weden asks the ettin Webthrithner about the origin of “earth and up-heaven”, and \Voluspa\ 3/3, where it is said, about the time before the World existed, that “earth and up-heaven” never existed.}} &
noh \edtext{\alst{p}aum · noh \alst{p}erek}{\lemma{paum \dots\ perek ‘forest \dots\ mountain’}\Bfootnote{The same word-pair is found in \Grimnismal\ 40, describing the creation of the world from Yimer’s body by the Gods.}} ni was &
ni [...] nohh-ęinig · noh sunna ni skęin &
noh \alst{m}áno ni liuhta · noh der \alst{m}árjo sèo. &
Dó dar ni·\alst{w}iht ni \alst{w}as · ęntjó ni \alst{w}ęntjó, &
ęnti dó was der \alst{ęi}no · \alst{a}l-mahtiko kot, &
\alst{m}anno \alst{m}iltisto, · ęnti dar wárun auh \alst{m}anaké mit inan &
\alst{k}ót-líhhé \alst{g}ęistá, · ęnti \alst{k}ot hęilak.\eva

\bvb I learned among men that greatest of wonders, \\
that earth was not, nor up-heaven, \\
nor a forest, nor a mountain was not, \\
nor any [...]; nor did the sun shine, \\
nor the moon give off light, nor the glittering sea. \\
Then nothing was there, neither of limit nor infinity (TODO: Translation),— \\
and then was the One Almighty God: \\
the mildest of men \ken*{= Christ}, and there were also many with Him: \\
good ghosts, and Holy God.\evb
\evg


\bpg
\bpa Kot al-mahtiko, dú himil ęnti erda ga·worahtós, ęnti dú mannun só manak kót for·gápi,
for·gip mir in dína ga·náda rehta ga·laupa, ęnti kótan willjon; wís-tóm ęnti spáhida, ęnti kraft tiuflun za widar·stantanne, ęnti ark za pi·wísanne, ęnti dínan willjon za ga·wurkhanne.\epa

\bpb O God almighty, thou didst work heaven and earth, and thou didst give men so many good things.
Give me in thy mercy the right belief and good will, wisdom and prophecy, and power to withstand devils and to reproach degeneracy and to work thy will.\epb
\epg

%	\include{books/Sun.tex}% Leed of the Sun. I very much want to do this one.

%	Giga-index at the end
	\part{Encyclopedia (INCOMPLETE!)}

NOTE: This encyclopedia is both incomplete and inconsistently formatted. New entries will be added, and old ones be corrected and expanded in the future.

\section{Cultural and religious expressions (C)}
\begin{itemize}

\inxitem[C]{ape} (ON \emph{api}, OE \emph{apa}, OS \emph{apo}, OHG \emph{affo}, PNWGmc. \emph{*apó})
  In the Old Norse the word seems to mean ‘fool, buffoon’, in the other old languages apparently ‘monkey’, though this sense should be a later development of the former; why would the early Germanic tribes have a word for an animal that they had never encountered?

\inxitem[C]{aught} (ON \emph{ę́tt}, OE \emph{ǽht} ‘possession, property’)
  The Nordic (paternal) clan or family line.

\inxitem[C]{begale} (OHG \emph{bi-galan})
  To affect, bewitch something using \inx[C]{galder}[galders]. See also \inx[C]{gale}.

\inxitem[C]{bigh} (ON \emph{baugr}, OE \emph{béag}, OHG \emph{boug})
  Armlets used as currency during the Migration Period. — The giving of rings and armlets in exchange for loyalty (\inx[C]{holdness} being the word used for a warrior’s loyalty towards his lord, and of a lord’s grace towards his servants) was common across all of Germanic Europe, as seen in the many poetic ruler-kennings of the type “breaker of rings” (e.g. \emph{béaga brytta} ‘the breaker of bighs’ in \Beowulf\ ll. 35, 352, 1487). An illustrative example of this is \Hildebrandslied\ 33–35.
  This is also connected with the oath-ring, and the famous ring-swords. TODO? reference some literature on this.

\inxitem[C]{bloot} (ON \emph{blót}, OE \emph{blót}, OHG \emph{bluoz})
  A sacrifice or a sacrificial feast, one of the best attested Germanic pagan practices. The animals would be sacrificed by the host, cooked in large kettles and eaten communally.

\inxitem[C]{bloot-kettle}
  The large pots used for cooking the bloot-stew.

\inxitem[C]{Doom} (ON \emph{dómr}, OE \emph{dóm})
  Commonly ‘judgement, verdict’ (whence Doomsday, ‘Judgement Day’), in the Norse and Anglo-Saxon poetry often specifically referring to one’s fame or good reputation (that is, how others will judge one’s character and deeds), especially after death. It is clear that this verdict was of utmost importance to the ancient Germanic people. The clearest examples are \Havamal\ 77 (see there): \emph{I know one that never dies: the \textbf{Doom} o’er each man dead.} and \Beowulf\ 1384-1389, where Beewolf consols king Rothgar after Grendle’s mother has slain his trusted advisor Asher (\emph{Æschere}):
  \bvg\bva[] \emph{Ne sorga, snotor guma! \hld\ Sélre bið ǽghwǽm, //
  þæt hé his fréond wrece, \hld\ þonne hé fela murne. //
  Úre ǽghwylc sceal \hld\ ende gebídan //
  worolde lífes; \hld\ wyrce sé þe móte //
  \textbf{dómes} ǽr déaþe; \hld\ þæt bið drihtguman //
  unlifgendum \hld\ æfter sélest.}\eva
  \bvb ‘Sorrow not, wise man! ’Tis better for each one that he avenge his friend, than that he mourn much. Each one of us shall suffer the end of worldly life—win he who might \textbf{Doom} before death: that is for the warrior, unliving, afterwards the best.’\evb\evg
  Other illustrative examples in \Beowulf\ include 884b–887a: \emph{[...] Sigemunde gesprong // æfter déaðdæge \hld\ \textbf{dóm} unlýtel // syþðan wíges heard \hld\ wyrm ácwealde // hordes hyrde [...]} ‘For \inx[P]{Syemund} sprang up after his day of death an unlittle \ken*{= great} \textbf{Doom}, since hard in conflict he defeated the \inx[C]{Wyrm}, the herder of the hoard.’
  and 953b–955a: \emph{[...] þú þé self hafast // dę́dum gefremed \hld\ þæt þín \textbf{dóm} lyfað // áwa tó aldre [...]} ‘Thou hast for thyself by deeds accomplished that thy \textbf{Doom} lives for ever and ever.’

\inxitem[C]{fee} (ON \emph{fé}, OE \emph{féoh})
  Originally ‘cattle’, however also used in a broader sense to refer to one’s mobile wealth. For this cf. particularly \Havamal\ TODO.

\inxitem[C]{many-cunning} (ON \emph{fjǫl-kunnigr})
  Literally ‘much-cunning, cunning in many ways’. Skilled with sorcery.

\inxitem[C]{fey} (ON \emph{fęigr}, OE \emph{fǽge}, OHG \emph{feigi} ‘cowardly’)
  Being doomed or fated to die, with a sense of predestination and inevitability. Its earliest use is on the Rök stone: \textbf{aft uamuþ stąnta runaʀ þaʀ ᛭ n uarin faþi faþiʀ aft} faikiąn \textbf{sunu} \emph{Apt Vámóð standa rúnaʀ þáʀ, en Varinn fáði, faðir aft \textbf{fęigjan} sonu} ‘After Woemood (\emph{Vámóðr}) stand these \inx[C]{rune}[runes], but Warren (\emph{Varinn}) painted, the father after the \textbf{fey} son.’ It was believed that one’s TODO. See \textciteshorttitle{PCRN-HS} II:35, p. 928 ff. (TODO)

\inxitem[C]{feyness} (ON \emph{fęigð})
  The state of being \inx[C]{fey}.

\inxitem[C]{fimble-} (ON \emph{fimbul-})
  The ultimate, final, greatest. See \inx[P]{Fimblethyle}, \inx[L]{Fimble-winter}.

\inxitem[C]{five days} (ON \emph{fimm dagar})
  That the old Scandinavian week was \textbf{five days} long is well attested. According to the \Gulatingslog\ there were six weeks in a month, and the expression \textbf{five days} is used as the equivalent of \emph{week} in \Havamal\ 51 and 74, in the second of which it is contrasted with \emph{month}. Related to this is the legal term \emph{fifth} (ON \emph{fimmt}, OSw. \emph{fæmt}), a meeting or gathering set to be held at a five-day notice. See \emph{fimt} in \CV, \textcite{LMNL} for further discussion.

\inxitem[C]{galder} (ON \emph{galdr}, OE \emph{gealdor}, OHG \emph{galdar})
  A magical spell or song. See the Merseburg charms (TODO?) for examples. See also \inx[C]{gale}.

\inxitem[C]{gale} (ON \emph{gala}, OE \emph{galan}, OHG \emph{galan})
  To sing \inx[C]{galder}[galders].

\inxitem[C]{gand} (ON \emph{gandr}, Latin \emph{gandus})
  A witch’s familiar, a spirit sent out to do her bidding. See \textciteshorttitle{PCRN-HS} I:17, p. 361 and II:26, p. 656. TODO

\inxitem[C]{gid} (ON \emph{goði}, OE \emph{Gydda} masc. nom. prop.)
  A heathen priest or master of ceremonies.

\inxitem[C]{gidden} (ON \emph{gyðja}, OE \emph{gyden} ‘goddess’)
  The feminine equivalent of \inx[C]{gid}.

\inxitem[C]{yin-} (ON \emph{ginn-})
  A rare augmentative prefix. TODO.

\inxitem[C]{yin-holy} (ON \emph{ginn-hęilagr})
  High holy, sacrosanct. Used of the gods in the formula \emph{ginn-hęilǫg goð}.

\inxitem[C]{good of meat} (ON \emph{matar góðr})
   An old expression, appearing not just in \Havamal\ 39 (“I found not a generous man, or so \textbf{good of meat}, that a gift were not accepted;”) but also several Viking Age Runic inscriptions, such as Sm 39: \emph{mildan orða \hld\ ok mataʀ góðan} ‘mild of words and \textbf{good of meat}’, U 805: \emph{bónda góðan matar} ‘a farmer \textbf{good of meat}’, U 703: \emph{mandr matar góðr \hld\ auk máls risinn} ‘a man \textbf{good of meat} and proud in speech™; compare also U 739: \emph{hann vaʀ mildr mataʀ \hld\ auk máls risinn} ‘he was \textbf{mild of meat} and proud in speech’. — See \inx[C]{meat-nithing} for its opposite.

\inxitem[C]{hame} (ON \emph{hamr})
  A skin, shape. Individuals can through magic “shift hames” (ON \emph{skipta hǫmum}), and leave their human \emph{hames} behind, instead entering into the shapes of wolves, bears, birds. During this process the original hame would be sleeping in a vulnerable state, as described in the Saw of the Walsings, chap. TODO: . See also \inx[P]{feather-hame}, \inx[C]{town-riders}, \inx[C]{evening-riders}.

\inxitem[C]{harrow} (ON \emph{hǫrgr}, OE \emph{hearg}, PNWGmc. \emph{*harugaʀ})
  A cairn constructed for ritual purposes. \Hyndluljod\ 10 describes one: “A \inx[C]{harrow} he made for me, loaded with stones; now that stone-pile is become into glass. He reddened [it] in fresh blood of oxen; \inx[P]{Oughter} ever trusted on the \inx[G]{Ossens}.” See also \inx[C]{wigh}.

\inxitem[C]{hold} (ON \emph{hollr}, OE \emph{hold}, OS \emph{hold}, OHG \emph{hold})
  %TODO Mention: unhold wights, Old Saxon baptismal formula.
  ‘Favourable, loyal, gracious’, often of a ruler towards his subject (in the sense of ‘gracious, benevolent’) or the reverse (in the sense of ‘loyal, devoted’). Mirroring these earthly relations, it is likewise often used to refer to divine grace, both of the Christian God—thus in the \emph{Ecclesiastical Laws of King Cnut} \textciteshorttitle[372]{ALIE1}: \emph{Ðam byþ witodlíce God hold þe bið his hláforde rihtlíce hold} ‘Indeed God is \textbf{hold} towards him who is rightly \textbf{hold} towards his lord’—but in the oldest Scandinavian material likewise of the Heathen gods.
  Thus \Lokasenna\ 4: \emph{holl ręgin} ‘\textbf{hold} \inx[G]{Reins}’, and \Oddrunargratr\ 10 (TODO: Numbering is very uncertain): \\ \emph{Svá hjalpi þér \hld\ hollar véttir, \\ Frigg ok Fręyja \hld\ ok flęiri goð} \\ ‘So help thee \textbf{hold} \inx[C]{wights}; \inx[P]{Frie} and \inx[P]{Frow}, and more gods [...]’.

  The word is also used in this way several medieval oath-formulæ, for instance in the Elder West-Geatish Law: \emph{Svá sé mér goð holl} ‘So may the gods(!) be \textbf{hold} towards me,’ in medieval Norwegian laws (\textciteshorttitle{NGL2}[197,397]) and Grey-Goose (TODO: cite): \emph{Guð sé mér hollr ef ek satt segi, gramr ef ek lýg} ‘God be \textbf{hold} towards me if I speak truly, wroth if I lie,’ in Grey-Goose (TODO) also: \emph{Sé guð hollr þeim er heldr griðum, en gramr þeim er grið rýfr} ‘God be \textbf{hold} towards him who keeps the truce, but wroth against him who breaks the truce’. I refer to \textcite{Läffler1895} for further discussion on these formulæ.

  \inxitem[C]{holdness} Closely connected to this is of course the abstract noun \textbf{holdness} (ON \emph{hylli}, OE \emph{hyldu}, OHG \emph{huldí}) ‘favour, loyalty, grace,’ with the same semantics as the adjective.

  Notably, this word appears three times in connection with the grace of gods in the poetry, namely in \Grimnismal\ 43, where (according to my interpretation) the preparer of food at the bloot is said to earn the “\textbf{holdness} of \inx[P]{Woulder} and of all the gods;” and \Grimnismal\ 53 where the disgraced king Garfrith is said to have been bereft of “my [= Weden’s] support; of all the Ownharriers (see note to the v.), and of Weden’s \textbf{holdness}”. Weden’s holdness (\emph{Óðins hylli}; the phrase is identical in both occurences) is also mentioned in a stanza by Hallfred (edited as Hfr Lv 7 by Diana Whaley in \Skp\ V) where the scold states that: ‘The whole race of man has wrought songs to win the \textbf{holdness} of Weden; I recall the fully rewarded works of our kinsmen/ancestors.’

  From all these citations the Germanic view on divine favour is clear: the gods are \textbf{hold} towards those who do good works, which in the aforementioned instances include swearing true oaths, faithfully observing truces, partaking in the bloot, following rules of hospitality and composing poetry—and \inx[C]{gram} ‘wroth’ towards those who do the opposite.

\inxitem[C]{Home} (ON \emph{hęimr}, OE \emph{hám}, PNWGmc. \emph{*haimaʀ})
  In the Norse often referring to a realm in the cosmology (\Voluspa\ 2: “I remember nine \textbf{Homes}”, \Vafthrudnismal\ TODO: “From the runes of the \inx[G]{Ettins} and of all the gods I can speak truly, for I have come into each \textbf{Home}”). Thus \inx[L]{Ettinham} is the ‘\textbf{Home}/realm of the ettins’. When used alone the term simply means ‘the world (that we inhabit)’. See also \inx[L]{nine Homes}, \inx[L]{Thrithham}.

\inxitem[C]{leat} (ON \emph{hlaut})
  Sacrificial blood (that is, taken from the animal), especially when used for auguries.

\inxitem[C]{leat-twig} (ON \emph{hlauttęinn})
  A twig used to sprinkle the \inx[C]{leat} in auguries (presumably the pattern of the blood would then be inspected).

\inxitem[C]{leed} (ON \emph{ljóð}, OE \emph{léod})
  A magical chant or incantation. See also \inx[C]{galder}, \inx[C]{gale}, \inx[C]{begale}.

\inxitem[C]{manwit} (ON \emph{manvit})
  Practical/common sense and wisdom, situational awareness.

\inxitem[C]{nithe} (ON \emph{níð}, OE \emph{níþ}, OHG \emph{níd})
  Originally probably ‘hatred, emnity’, in the Norse a sort of ritual libel that brought great dishonor.

\inxitem[C]{orlay} (ON \emph{ørlǫg}, OE \emph{orlæg})
  One’s predetermined fate, destiny, purpose as decreed by the \inx[G]{Norns}.

\inxitem[C]{rest} (ON \emph{rǫst})
  The distance between two rest-stops, a geographical mile (about 1850 metres). See especially \CV.

\inxitem[C]{rune} (ON \emph{rún}, OE \emph{rún}, OS \emph{rúna}, OHG \emph{rúna}, Got. \emph{rúna}, PNWGmc. \emph{rūnu})
  An (esoteric) secret message or formula. That this—rather than ‘letter (of a Runic alphabet)’—is the original and proper sense is apparent from among others the Finnish borrowing \emph{runo} ‘poem; poetry; a division of a poem (specifically of the \emph{Kalevala})’, and its use in the singular in the earliest Runic inscriptions (e.g. Noleby Vg 63, which contains the linguistically indecipherable string of letters {ᚢᚾᚨᚦᛟᚢᛊᚢᚺᚢᚱᚨᚺᛊᚢᛊᛁᚺ[--]ᚨᛁ\rotatebox[origin=c]{180}{ᛏ}ᛁᚾ}, a \emph{rune} in the proper sense or the recently discovered Svingerud fragment.) Thus, Weden’s taking of the \emph{runes} should not be interpreted as merely a myth for the invention of profane writing, but rather the origin of esoteric incantations, not at all unlike Indian \emph{mantras}.
  The word for letter was instead \inx[C]{stave}, see also there.

\inxitem[C]{scold} (ON \emph{skald})
  A Scandinavian poet. The name probably comes from their ability to slander with words.

\inxitem[C]{simble} (ON \emph{sumbl}, OE \emph{symbol})
  A banquet.

\inxitem[C]{soo} (ON \emph{sóa})
  To ritually waste, to slay (especially in a sacrificial context).

\inxitem[C]{thill} (ON \emph{þylja})
  To chant poetry or lists (so called \inx[C]{thule}[thules]) acquired by rote memorization. See \inx[C]{thyle}.

\inxitem[C]{Thing} (ON, OE \emph{þing}, OS \emph{thing}, OHG \emph{ding})
  The legal assembly and gathering place where matters would be settled and the law recited.

\inxitem[C]{thule} (ON \emph{þula})
  A poetic list, typically of various items of a category (e.g. gods, legendary horses) or poetic synonyms (e.g. for swords, men, Weden). Degoratively also a ditty, poorly composed poem. See \inx[C]{thyle}.

\inxitem[C]{thyle} (ON \emph{þulr}, OE \emph{þyle}, PNWGmc. \emph{*þuliʀ})
  A sage who through rote learning has acquired a large amount of mythological lore (cf. \inx[C]{thule} ‘a list in poetic form; a ditty, bad poem’ and \inx[C]{thill} ‘to recite, to chant’). Thus \inx[P]{Weden} is the \inx[P]{Fimblethyle}, being the unbeaten master of lore, as can be seen in his wisdom contests (like \Vafthrudnismal). Runic inscription DR 248 (Snoldelev) suggests the thyle may have tied to a specific place, and in \Beowulf\ it seems to have been a court position, with the poet Unferth being described (l. 1456) as the “thyle of Rothgar”.

\inxitem[C]{wale} (ON \emph{vǫlr})
  The staff or sceptre, especially of a wallow. TODO: archeological finds, mention Sutton Hoo.

\inxitem[C]{wallow} (ON \emph{vǫlva}, OE \emph{*wealwe} (cf. ON \emph{svǫlva}, OE \emph{swealwe} ‘swallow’))
  A sibyl, seeress, oracle. The word derives from the \inx[C]{wale}, a staff or sceptre probably used for ritual purposes.

\inxitem[C]{wigh} (ON \emph{vé}, OE \emph{wéoh}, \emph{wíh}, PNWGmc. \emph{*wīhą})
  A holy shrine or sanctuary. It seems that where the \inx[C]{harrow} was a pile of stones or cairn used for carrying out rituals, the \textbf{wigh} was an enclosed space. The earliest Norse attestation is the runic inscription Ög N288 (Oklunda), which reads: “Guther <= Gunnarr> painted these runes, and he fled, guilty. Sought this wigh, and he fled into this clearing. And he bound. [...]” The implication seems to be that the wigh was considered so sacred that Guther could not be apprehended or punished for his crime while in it. — In OE the word means ‘pagan idol’. It is not immediately clear which meaning is the original one, but in the present edition the Norse sense has been adopted, since the Anglo-Saxon sources are all of a Christian nature. The \Beowulf\ name \emph{Wighstone} (\emph{Wīh-} or \emph{Wēohstān}) in any case suggests it is the Norse meaning, since ‘idol-stone’ makes little sense.

\inxitem[C]{wode} (ON \emph{óðr}, OE \emph{wód}, PNWGmc. \emph{*wóþuʀ})
  \inx[P]{Heener}’s gift to men, though the name would suggest it be from \inx[P]{Weden}. The word has several related meanings: ‘poetic inspiration, madness, rage’.
\end{itemize}

\section{Persons and objects (P)}
\begin{itemize}

\inxitem[P]{Attle} (\emph{Attila}, ON \emph{Atli}, OE \emph{Ætla}, MHG. \emph{Etzel}, PNWGmc. \emph{*Attiló})
  The ruler of the \inx[G]{Huns} (historically from 434–453). Husband of \inx[P]{Guthrun}, and with her father of \inx[P]{Earp and Oatle}. and murderer of
  I HHb 54, SiL 11, I Gr 23, ShS 28, 29, 33, 37, 54, 56, 57, II Gr 26, 38, 45, III Gr 1, 9, BnOr 0, OdW A, 2, 22, 23, 25, 26, 30, 31, AtD 0, AtL 1, 3, 15, 17, 18, 27, 31, 32, 34, 36, 37, 38, 41, 43, B, AtS 2, 4, 21, 22, 44, 52, 60, 64, 71, 73, 77, 80, 86, 87, 97, 98, 108, 113, 117, FGr 0, GrB 12, Ham 6.

\inxitem[P]{Balder} (ON \emph{Baldr}, OE \emph{Bældæg} (not directly cognate), OHG \emph{Balter}, PWGmc. \emph{*Baldraʀ})
  The beautiful son of \inx[P]{Weden}, slayed by his brother \inx[P]{Hath}, avenged by his other brother \inx[P]{Wonnel}.

\inxitem[P]{Earp and Oatle} (ON \emph{Erpr ok Ęitill})
  The sons of \inx[P]{Attle} and \inx[P]{Guthrun}.

\inxitem[P]{Earth} (ON \emph{jǫrð}, OE \emph{eorþe}, OHG \emph{erda}, PNWGmc. \emph{*erþu}, PGmc. \emph{*erþó})
  The personified Earth. Through \inx[P]{Weden} the mother of \inx[P]{Thunder}.

\inxitem[P]{feather-hame} (ON \emph{fjaðr-hamr}, OE \emph{feðer-hama}, OS \emph{feðar-}, \emph{feðer-hamo})
  An object by which the wearer may fly like a bird. One is owned by Frow and used by Lock to fly between the homes. In the Heliand \textbf{feather-hames} are donned by angels who fly from heaven to earth. See also \inx[C]{hame}.

\inxitem[P]{Free} (ON \emph{Fręyr}, OE \emph{fréa} ‘lord’, PNWGmc. \emph{*Frawjaʀ})
  Son of \inx[P]{Nearth}, brother of \inx[P]{Frow}. See also \inx[P]{Ing}.

\inxitem[P]{Frie} (ON \emph{Frigg}, OE \emph{*Frige}, OHG \emph{Frija}, PNWGmc. \emph{*Frijju})
  Wife of \inx[P]{Weden}, mother of \inx[P]{Balder}. Related to \inx[P]{Full}, who is either her sister (Second Merseburg Charm, though this may be metaphorical, as in \Hyndluljod\ 1) or her maid-servant (the Norse sources).

\inxitem[P]{Frow} (ON \emph{Fręyja})
  Cat-goddess, daughter of \inx[P]{Nearth}, sister of \inx[P]{Free}, wife of \inx[P]{Wode}. Promised to the Ettin. Possibly = Easter?

\inxitem[P]{Full} (ON \emph{Fulla}, OHG \emph{Folla})
  Maid-servant (or sister?) of \inx[P]{Frie}; see there.

\inxitem[P]{Guthrun} (ON \emph{Guðrún})
  Daughter of king \inx[P]{Yivick}, sister of \inx[P]{Guther} and \inx[P]{Hain}. The wife of \inx[P]{Attle}.

\inxitem[P]{Hain}[Hain 1] (ON \emph{Hǫgni}, OE \emph{Haguna}, \emph{Hagena}, OHG \emph{Hagano}, Ger. \emph{Hagen}, PNWGmc. \emph{*Hagunó})
  A \inx[G]{Nivlings}[Nifling] and \inx[G]{Yivickings}[Yivicking], son of king \inx[P]{Yivick}, brother of \inx[P]{Guther} and \inx[P]{Guthrun}. In \emph{AtL} he defeats seven warriors before being captured by \inx[P]{Attle}, who has his heart cut out at the request of Guther.

\inxitem[P]{Hain 2}[2]
  A petty king of \inx[L]{East Geatland}, contemporary with \inx[P]{Granmer}, the king of \inx[L]{Southmanland} and Ingeld Illred, the \inx[G]{Inglings}[Ingling] king of \inx[L]{Upland}.

\inxitem[P]{Hath} (ON \emph{Hǫðr})
  The blind son of \inx[P]{Weden}, the slayer of his brother \inx[P]{Balder}.

\inxitem[P]{Heener} (ON \emph{Hǿnir}, PNWGmc. \emph{Hónijaʀ} ‘the little swan(?)’)
  An obscure god. \textcite{Rydberg1886}[552] has convincingly argued that he is connected with the stork, connecting his name with the Greek \textgreek{κύκνος} ‘swan’ and Sanskrit \emph{śakuna} ‘bird of omen’, and noting that his epithets \emph{langi fótr} ‘long foot’ and \emph{aurkonungr} ‘mud-king’ (both found in \Skaldskaparmal\ 22) accurately describe the stork. He gives \inx[C]{wode} TODO.

\inxitem[P]{Hindle} (ON \emph{Hyndla})
  A witch awoken by \inx[P]{Frow} in \Hyndluljod.

\inxitem[P]{Homedall} (ON \emph{Hęimdallr}, OE \emph{*Hámdall})
  Ward of the gods, whitest of the \inx[G]{Ease}.

\inxitem[P]{Hymer} (ON \emph{Hymir})
  \inx[P]{Tew}’s father according to \Hymiskvida.

\inxitem[P]{Ing} (ON \emph{Yngvi}, OE \emph{Ing})
  Probably an older name of \inx[P]{Free}. The legendary ancestor of the \inx[G]{Inglings}. Cf. the Old English Rune Poem.

\inxitem[P]{Lother} (ON \emph{Lóðurr}, OS \emph{Logaþore}, PNWGmc. \emph{*Logaþorjaʀ} ‘Flame-darer(?)’)
  Gives three gifts to man. The Old-Saxon attestation is a bit uncertain.

\inxitem[P]{Millner} (ON \emph{Mjǫllnir}, OE \emph{*Meldne}, PNWGmc. \emph{*Meldunjaʀ})
  Powerful hammer owned by Thunder.

\inxitem[P]{Nearth} (ON \emph{Njǫrðr})
  The father of \inx[P]{Free} and \inx[P]{Frow} by \inx[P]{Shede}.

\inxitem[P]{Nithad} (ON \emph{Níðuðr}, OE \emph{*Hámdall})
  The Swedish king that imprisons \inx[P]{Wayland} in \Volundarkvida. Father of \inx[P]{Beadhild}.

\inxitem[P]{Oughter} (ON \emph{Óttarr}, OE \emph{Óhthere}, PNWGmc. \emph{*Óhtaharjaʀ})
  Legendary Swedish king.

\inxitem[P]{Rotholf} (ON \emph{Hrólfr kraki}, OE \emph{Hróþulf}, PNWGmc. \emph{*Hróþiwulfaʀ})
  A king of the \inx[G]{Shieldings} (see family tree). As foreshadowed in \Beowulf\ 1017–9, 1180–90, he betrays the sons of \inx[P]{Rothgar}, his cousins \inx[P]{Rethrich and Rothmund}, in order to take the throne for himself. In the later Icelandic tradition this has been forgotten, and he is consistently portrayed as a heroic king.

\inxitem[P]{Rothgar} (ON \emph{Hróarr}, OE \emph{Hróþgár}, PNWGmc. \emph{*Hróþigaiʀaʀ})
  A king of the \inx[G]{Shieldings} (see family tree), one of the main characters in \Beowulf.

\inxitem[P]{Shield} (ON \emph{Skjǫldr}, OE \emph{Scyld})
  Legendary Danish king, founder of the \inx[G]{Shieldings}.

\inxitem[P]{Syemund} (ON \emph{Sigmundr}, OE \emph{Sigemund}, MHG. \emph{Siegmund})
  A hero of the \inx[G]{Walsings}, in \Beowulf\ attested as the slayer of the dragon along with his nephew \inx[P]{Sinfittle}. In the Norse tradition however, it is his half-brother \inx[P]{Siward} that slays the dragon instead.

\inxitem[P]{Sithguth} (OHG \emph{Sinthgunt}, PNWGmc. \emph{*Sinþagunþiz})
  Only known from \MerseburgTwo\ as the sister of \inx[C]{Sun}.

\inxitem[P]{Sun} (ON \emph{Sól}, OHG \emph{Sunna})
  The personified sun (see also \inx[P]{Moon}). In \MerseburgTwo, described as the sister of \inx[C]{Sithguth}.

\inxitem[P]{Thrim} (ON \emph{Þrymr})
  The ettin responsible for stealing Thunder’s hammer in \Thrymskvida.

\inxitem[P]{Thunder} (ON \emph{Þórr}, OE \emph{Þunor}, OHG \emph{Donar}, PNWGmc. \emph{*Þonaraʀ})
  Son of \inx[P]{Weden} and \inx[P]{Earth}.

\inxitem[P]{Tew} (ON \emph{Týr}, OE \emph{Tíw})
  Son of \inx[P]{Hymer}. One-handed god. TODO.

\inxitem[P]{Webthrithner} (ON \emph{Vafþrúðnir})
  The ettin defeated by Weden in the wisdom contest in \Vafthrudnismal.

\inxitem[P]{Weden} (rhymes with \emph{leaden}; ON \emph{Óðinn}, OE \emph{Wóden}, \emph{Wéden}, OHG \emph{Wuotan}, PNWGmc. \emph{*Wódanaʀ})
  Chief of the \inx[G]{Ease}, his name is clearly related to \inx[C]{wode}, referring to his role as the patron of \inx[C]{scold}[scolds] and \inx[C]{bearserk}[bearserks]. Husband of \inx[P]{Frie}, and by her father of \inx[P]{Balder}. Also father of \inx[P]{Thunder} by \inx[P]{Earth}. Brother of \inx[P]{Heener} and \inx[P]{Lother}.

\inxitem[P]{Wider} (ON \emph{Víðarr}, OE \emph{*Wídhere})
  A son of \inx[P]{Weden}, who avenges him at the \inx[L]{Rakes of the Reins}.

\inxitem[P]{Wode} (ON \emph{Óðr}, OE \emph{Wód})
  Husband of \inx[P]{Frow}. His name looks to be the same word as \inx[C]{wode}.

\inxitem[P]{Wonnel} (ON \emph{Váli}, OE \emph{*Wonela}, PNWGmc. \emph{*Wanilô} ‘the little \inx[G]{Wanes}[Wane]?’)
  The son of \inx[P]{Weden}, who one-night old avenged his brother \inx[P]{Balder} through slaying \inx[P]{Hath}, his half-brother.

\inxitem[P]{Woulder} (ON \emph{Ullr}, \emph{*Wuldor}, PNWGmc. \emph{*Wulþuz})
  A rather obscure god. He is mentioned in connection with oath-rings (TODO) and the setting of ritual fires (\Grimnismal\ TODO). These obscure references are likely related to the interesting finds at Lilla Ullevi (‘the small \inx[C]{wigh} of Woulder’) in Upland, Sweden, consisting of several dozen fire striker-shaped iron amulet rings dating to 660–780 (for a detailed description see \parencite{afEdholm2009}).

\inxitem[P]{Yimer} (ON \emph{Ymir}, OE \emph{*Yime})
  The first ettin, probably equivalent to \inx[P]{Earyelmer}.

\inxitem[P]{Yivick} (ON \emph{Gjúki}, OE \emph{Gifica}, OHG \emph{Gibicho}, MHG. \emph{Gibeche})
  King of the \inx[G]{Burgends} (historically from late 300s–407) of the Nifling dynasty, ancestor of the \inx[G]{Yivickings}. Father of \inx[P]{Guthrun}, \inx[P]{Guther} and \inx[P]{Hain}.
\end{itemize}

\section{Groups and tribes (G)}
TODO: Map of rough tribal areas. Geneaologies.

\begin{itemize}

\inxitem[G]{Danes} (ON \emph{danir}, OE \emph{dene}, PNWGmc. \emph{*daníʀ})
  A tribe in eastern modern-day Denmark and southern Sweden. They probably originated in Scania in southern Sweden, before moving westwards into the Danish isles and eventually Jutland, driving out the \inx[G]{Earls} and \inx[G]{Jutes}.
  Noted members: TODO
  Attestations: TODO

\inxitem[G]{Dwarfs} (ON \emph{dvergar}, OE \emph{dweorgas}, OHG \emph{twerca}, PNWGmc. \emph{*dwergóʀ})
  Earthly (chthonic) supernatural beings, often referred to as living in rocks and mountains.
  Noted members: TODO
  Attestations: TODO

\inxitem[G]{Ease} (rhyming with \emph{geese}; ON \emph{ę́sir}, OE \emph{ése}, PNWGmc. \emph{*ansiwiʀ}; sg. \emph{os}, ON \emph{áss}, OE \emph{ós}, PNWGmc. \emph{*ansuʀ})
  A group of Gods, though the word can also refer to all the Gods. See \inx[G]{Gods}, \inx[G]{Tews}, \inx[G]{Wanes}, \inx[G]{Reins}.
  Noted members: \inx[P]{Weden}, \inx[P]{Thunder}, \inx[P]{Frie}, \inx[P]{Hath} and \inx[P]{Balder}
  Attestations: TODO

\inxitem[G]{Elves} (ON \emph{alfar}, OE \emph{ielfe}, PNWGmc. \emph{*alβíʀ})
  Earthly (chthonic) supernatural beings. Possibly ancestral spirits?
  Noted members: TODO
  Attestations: TODO

\inxitem[G]{Ettins} (ON \emph{jǫtnar}, OE \emph{eotenas}, PNWGmc. \emph{*etunóʀ})
  The fundamental enemies of the Gods, the agents of chaos and disorder. See \inx[G]{Rises}, \inx[G]{Thurses}.
  Noted members: \inx[P]{Hymer}, \inx[P]{Thrim}, \inx[P]{Webthrithner}, \inx[P]{Yimer}
  Attestations: TODO

\inxitem[G]{Geats} (ON \emph{gautar}, OE \emph{géatas}, PNWGmc. \emph{*gautóʀ} from \emph{*geut-} ‘to pour’, perhaps ‘the libators’)
  A tribe in what is today southern-central Sweden. See also \inx[L]{Geatland}, \inx[G]{Swedes}.
  Noted members: TODO
  Attestations: TODO

\inxitem[G]{yin-Reins} (ON \emph{ginn-ręgin})
  \inx[C]{yin-} + \inx[G]{Reins}. The sacrosanct, highest divine powers.

\inxitem[G]{Gods} (ON \emph{goð}, OE \emph{godu}, OHG \emph{gota}, PNWGmc. \emph{*godu})
  TODO.
  Noted members: TODO
  Attestations: TODO

\inxitem[G]{Huns} (ON \emph{húnir}, OE \emph{Húne}, OHG \emph{Húni}, \emph{Hunni}, PNWGmc. \emph{*húníʀ})
  An invading Asiatic tribe in the Migration Period. In the legendary material their cultural and ethnic foreignness is not seen.
  Noted members: TODO
  Attestations: TODO

\inxitem[G]{Inglings} (ON \emph{ynglingar}, PNWGmc. \emph{*ingwalingóʀ} ‘the descendants of \inx[P]{Ing}’)
  Difference between this term and \inx[G]{Shelvings} is a bit unclear. They seem to be used synonymously in the Norse sources, whereas the English only use the later.

\inxitem[G]{Nears} (ON \emph{níarar} \char`~ \emph{njárar})
  A Swedish tribe, only mentioned in \Volundarkvida, where it is ruled by king \inx[P]{Nithad}. The name and location may allow us to connect them with the Swedish province of Närke, cf. Old Swedish: \emph{Nærikiar} ‘inhabitants of Närke’, \emph{Nærisker} ‘belonging to Närke; Nearish’, in which case the Old Swedish stem \emph{nær-} (with unclear vowel length, though it is probably long) would be a reduced form of \emph{níar-}, \emph{njár-}.

\inxitem[G]{Norns} (ON \emph{nornir})
  A group of supernatural women responsible for declaring the fates of men.

\inxitem[G]{Ossens} (ON \emph{ǫ́synjur})
  The women of the \inx[G]{Ease}, see there.

\inxitem[G]{Ownharriers} (ON \emph{ęinhęrjar}, OE \emph{*ánhergas})
  Earthly (chthonic) supernatural beings, often referred to as living in rocks and mountains.
  Noted members: TODO
  Attestations: TODO

\inxitem[G]{Reins} (ON \emph{rǫgn}, \emph{ręgin})
  The divine powers. Based on \Vafthrudnismal\ (TODO) the term may be more closely associated with the \inx[G]{Wanes} than the \inx[G]{Ease}.

\inxitem[G]{Saxons} (ON \emph{saxar}, OE \emph{Seaxan}, \emph{Seaxe})
  TODO.
  Noted members: TODO
  Attestations: TODO

\inxitem[G]{Shieldings} (ON \emph{skjǫldungar}, OE \emph{Scyldingas}, PNWGmc. \emph{*skeldungóʀ})
  The descendants of \inx[P]{Shield}; the legendary \inx[G]{Danes}[Danish] royal dynasty. With \inx[P]{Harward}’s death after his slaying of \inx[P]{Rotholf} their rule ended. TODO
  Noted members: TODO
  Attestations: TODO

\inxitem[G]{Shelvings} (ON \emph{skilfingar}, OE \emph{scilfingas}, PNWGmc. \emph{*skilβingóʀ})
  The descendants of \inx[P]{Shelf}; the legendary \inx[G]{Swedes}[Swedish] royal dynasty. The exact difference between the terms Shelvings and \inx[G]{Inglings} is unclear, but the first may have referred to the old royal family in Sweden, while the latter to the Norwegian branch which claimed descent from the former. TODO
  Noted members: TODO
  Attestations: \Hyndluljod\ 15, 20

\inxitem[G]{Swedes} (ON \emph{svíar}, OE \emph{swéon}, PNWGmc. \emph{*swihaníʀ})
  The tribe around the Mälar valley in eastern Sweden.
  Noted members: TODO
  Attestations: TODO

\inxitem[G]{Thurses} (sg. Thurse; ON \emph{þurs}, OE \emph{þyrs}, OS \emph{thuris}, OHG \emph{duris}, PNWGmc. \emph{*þurisaʀ})
  Possibly a poetic synonym for \inx[G]{Ettins}. See also \inx[G]{Rime-Thurses}.
  Noted members: TODO
  Attestations: Wal 8, Shr 31, 35, 36, Hyme 17, Thr 5, 10, 21, 24, 29, 30, Alw 2, I HHb 40, HHw 27.

\inxitem[G]{Tews} (ON \emph{tívar}, PNWGmc. \emph{*tíwóʀ})
  A poetic synonym for \inx[G]{Gods}.
  Attestations: TODO

\inxitem[G]{Wanes} (ON \emph{vanir}, OE \emph{wan-?})
  A subgroup or tribe of the gods, associated with fertility, harvests and fishing.
  Noted members: \inx[P]{Nearth}, \inx[P]{Ing}, \inx[P]{Frow}
  Attestations: TODO

\inxitem[G]{Yivickings} (ON \emph{gjúkungar})
  The descendants of \inx[P]{Yivick}, including \inx[P]{Guther}, \inx[P]{Guthrun} and \inx[P]{Hain}.
  Attestations: TODO
\end{itemize}

\section{Place names, locations and events (L)}
\begin{itemize}

\inxitem[L]{Eastern Way} (ON \emph{Austrvegr})
  The eastern lands of the \inx[G]{Ettins} (probably identical in meaning to \inx[L]{Ettinham}), whither \inx[P]{Thunder} goes to fight.

\inxitem[L]{Ettinham} (ON \emph{Jǫtunhęimr}, \emph{Jǫtnahęimr})
  The ‘\inx[G]{Ettins}[Ettin]-\inx[C]{Home}’ or ‘home of the Ettins’; the eastern realm of chaotic and inhospitable beings. See also \inx[L]{Eastern Way}, \inx[L]{Outyards}.

\inxitem[L]{Fimble-winter} (ON \emph{fimbulvetr})
  The great winter, which kills all humans apart from \inx[P]{Life and Lifethrasher}.

\inxitem[L]{Hell} (ON \emph{hęl}, PNWGmc. \emph{*halju}, Got. \emph{halja})
  The underworld, personfied as and formally identical with \inx[P]{Hell}. After Christianity the word came to refer to the Christian hell (= Gehenna), as is the case in all attested languages apart from the Old Norse. See also \inx[L]{Nivelhell}.

\inxitem[L]{Middenyard} (ON \emph{Mið-garðr}, OE \emph{Middangeard}, OS \emph{Middilgard}, OHG \emph{Mittilgart}, Got. \emph{midjungards})
  The ‘middle enclosure’; the realm of men. See also \inx[L]{Osyard}, \inx[L]{Outyards}.

\inxitem[L]{Nivelhell} (ON \emph{niflhęl})
  ‘Mist-Hell’, from the poetic evidence it seems like it may originally have been a synonym for \inx[L]{Hell}. In poetry it is attested in \Vafthrudnismal\ TODO: \emph{níu kom’k hęima |hld\ fyr Niflhel neðan, \\ hinig deyja ór helju halir. } ‘into nine homes I came, beneath Nivelhell; thither die men out of Hell’, the second by \Baldrsdraumar\ 2: \emph{ręið niðr þaðan |hld\ niflhęljar til; \\ mǿtti hvelpi, |hld\ þęim’s ór hęlju kom.} ‘[Weden] rode down thence to Nivel-hell; met the whelp that out of Hell came.’ Possibly the distinction was held by the first poet but not the second.

\inxitem[L]{Osyard} (ON \emph{Ásgarðr})
  The ‘enclosure of the \inx[G]{Ease}’; the heavenly realm. See also \inx[L]{Middenyard}, \inx[L]{Outyards}.

\inxitem[L]{Outyards} (ON \emph{Útgarðar})
  Not eddic. The ‘outer enclosures’, described in \Gylfaginning. See also \inx[L]{Ettinham}, \inx[L]{Middenyard}, \inx[L]{Osyard}.

\inxitem[L]{Rakes of the Reins} (ON \emph{ragna rǫk})
  The ‘fates of the \inx[G]{Reins}’, euphemism for the destruction of the world.

\inxitem[L]{Rakes of the Tews} (ON \emph{tíva rǫk})
  The \inx[L]{Rakes of the Reins}.

\inxitem[L]{Up-heaven} (ON \emph{Upphiminn}, OE \emph{Upheofon}, OS \emph{Upphimil}, OHG \emph{úfhimil})
  Highest heaven. See also \inx[F]{Earth and Up-heaven}.

\inxitem[L]{Walhall} (ON \emph{Valhǫll}, OE \emph{Wælheall})
  The hall of the slain, held by \inx[P]{Weden} and inhabited by the \inx[G]{Ownharriers}.
\end{itemize}

\section{Poetic formulæ (F)}
All formulæ are given in English translation, their attested forms and a Proto-Germanic rendition. For those consisting of two words bound together by a conjunction, \& is written in its place.

\begin{itemize}
\inxitem[F]{Earth and Up-heaven} (ON \emph{jǫrð \& upphiminn}, OE \emph{eorþe \& upheofon}, PGmc. \emph{*erþō \& uphiminaz})
  ON: Ribe charm \Voluspa\ 3, \Vafthrudnismal\ 20, \Thrymskvida\ 2, \Oddrunargratr\ 17, OE: Acreboot

\inxitem[F]{Ease and Elves} (ON \emph{ę́sir \& alfar}, OE \emph{ése \& ielfe}, PNWGmc. \emph{*alβíʀ \& ansiwiʀ})
  A merism; both heavenly and earthly spiritual beings. Notably the two words always occur in this order (never ‘Elves and Ease’), even in OE.

\inxitem[F]{words and works} (ON \emph{orð \& verk}, OE \emph{word \& weorc}, PGmc. \emph{*wurdó \& werkó})
  \Beowulf\ 289, 1100, 1833

\end{itemize}
%
%

\end{document}
