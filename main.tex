% This file should be compiled with XeLaTeX.

\documentclass[]{memoir}

% Font and typesetting
\usepackage[final]{microtype}

\usepackage{fontspec}
\setmainfont{Junicode}[
	Extension=.ttf,
	BoldFont=*-Bold,
	ItalicFont=*-Italic,
	BoldItalicFont=*-BoldItalic]

% TODO: Underline that does not skip descender?
% Should be called \nsunderline

% Packages
\usepackage{xparse}% For better document commands
\usepackage{graphicx}% For rotating characters (used when citing runic inscriptions)
\usepackage{hyperref}% For index links
\usepackage{geometry}% For margins.
\geometry{b5paper}% Book size
\usepackage{longtable}% Long tables.

% Critical edition
\usepackage{reledmac}

% Headers
\usepackage{fancyhdr}
\pagestyle{fancy}
\fancyhead{}% Clear
\fancyfoot{}% Clear
\fancyhead[OR,EL]{\thepage}% Page numbering on both odd and even
\fancyhead[EC]{The Anglish Edda}% Title of project (TODO: probably going to change!)
\fancyhead[OC]{\booktitle}% Title of current book

% Define verse counters
\newcounter{versea}
\newcounter{verseb}

\begin{document}

% Book and chapter commands
	\NewDocumentCommand{\chapterStart}{o O{Chap}}{% Command at the start of chapter
		\setcounter{versea}{0}%
		\setcounter{verseb}{0}%
		\stepcounter{chapter}%
		\IfNoValueF{#1}{%
			\begin{center}%
			\textbf{#2. \arabic{chapter}} \\
			{#1}\end{center}%
		}%
	}

	\NewDocumentCommand{\bookStart}{m o}{% Command at the start of book
		% arg 1 (mandatory): English title
		% arg 2 (optional):  Original title
	  \IfValueTF{#2}{%
	  	\book*{#1 \emph{(#2)}}%
			\def\booktitle{#1 \emph{(#2)}}%
		}{%
			\book*{#1}%
			\def\booktitle{#1}%
		}%
		\addcontentsline{toc}{book}{\booktitle}
		\setcounter{chapter}{0}% Set chapter count to zero.
		\chapterStart{}%
	}

% Verse format commands
	\NewDocumentCommand{\bvg}{o}{% Begin verse group
		\begin{ledgroup}%
		\beginnumbering%
		\setcounter{footnoteB}{0}%
	}

	\NewDocumentCommand{\bva}{o}{% Begin verse a
		\begin{large}\begin{stanza}% Begin stanza
		\IfNoValueTF{#1}{% Add verse number according to counter
			\stepcounter{versea}% Step verse counter
			\flagstanza{\textbf{\arabic{versea}}}%
		}{\IfEq{#1}{0}{}{% If optional verse number is NOT 0 we show it.
			\flagstanza{\textbf{#1}}%
		}}%
	}
	\NewDocumentCommand{\eva}{o}{% End verse a
		\& \end{stanza}\end{large}% End reledmac stanza
		\vspace{1.5mm}% Vertical space
	}

	\NewDocumentCommand{\bvb}{o}{% Begin verse b (see above)
		\IfNoValueTF{#1}{%
			\stepcounter{verseb}%
%			\textbf{\arabic{verseb} }%
		}{\IfEq{#1}{0}{}{%
%			\textbf{\textbf{#1}}
		}}
	}
	\NewDocumentCommand{\evb}{o}{% End verse b
		% Nothing (for now?)
	}

	\NewDocumentCommand{\evg}{o}{% End verse group
		\endnumbering\end{ledgroup}% End numbering and ledgroup
		\vspace{1cm}%
	}

% Note formatting
	% Sidenote margin
	\setlength{\ledlsnotesep}{2 \ledlsnotesep}

	% Make A footnotes paragraphs
	\Xarrangement[A]{paragraph}

	% Make B footnotes roman
	\renewcommand*{\thefootnoteB}{\alph{footnoteB}}

% Poem formatting
	% First line number at 3
	\firstlinenum{2}
	\linenumincrement{2}

	% Stanza indentation (required by reledmac)
	\setstanzaindents{5, 2, 2}
	\setcounter{stanzaindentsrepetition}{2}

	% Line numbers directly under verse number (kind of a hack)
	\setlength{\linenumsep}{-1.62pc}

	% Mark cæsura.
	\newcommand{\hld}{ · }%

	% Indent lines (in Ljóðaháttr or Galdralag).
	\newcommand{\ind}{%
		\hspace{1.5em}%
	}

	% Mark alliteration. This might not be present in the final version.
	\NewDocumentCommand{\alst}{m}{%
		\underline{#1}%
	}

	% Mark kennings.
	\NewDocumentCommand{\ken}{sm}{%
		% No small caps; text inside the brackets.
		\IfBooleanTF{#1}{% No small caps
			{[{#2}]}% EXAMPLE: [= Wooden]
		}{%
			\textsc{[{#2}]}% EXAMPLE: [ETTIN]
		}%
	}%

	% Mark names with angular brackets.
	\NewDocumentCommand{\name}{m}{%
	% Text inside the brackets.
		<{#1}>% EXAMPLE: <ettin>
	}%

% Index commands
	\NewDocumentCommand{\inx}{m o O{#1}}{% Index link (link, type, optional alt display)
		\IfNoValueTF{#2}{% TODO: phase this out
			{#3}\textsuperscript{†}%
		}{% Proper noun
			\hyperref[#2:#1]{{#3}\textsuperscript{#2}}%
		}%
	}

	\NewDocumentCommand{\inxitem}{m o O{#1}}{% Index label (word, type, alt display)
		\item[\textbf{#3}]%
		\phantomsection\label{#2:#1}%
	}

% Sigla
	% Meters
	\newcommand{\Fornyrdislag}{% Law of ancient speeches
		\emph{Law of Ancient Utterings}%
	}

	\newcommand{\Snorri}{% Should probably not be used
		Snorri%
	}

	%Modern books and editions
	\newcommand{\CV}{% Cleasby-Vigfússon dictionary of Old Norse
		\emph{C-V}%
	}
	\newcommand{\FaulkesEdda}{% Faulkes translation of the Prose Edda
		\emph{SnE} (2005)%
	}
	\newcommand{\FinnurEdda}{% Finnur Jónsson’s edition of the Poetic Edda
		Finnur Jónsson (1932)%
	}
	\newcommand{\FGTHaugen}{% First Grammatical Treatise (Einar Haugen 1950)
		\emph{1GT} (1950)%
	}
	\newcommand{\GudniEdda}{% Guðni Jónsson’s edition of the Poetic Edda
		Guðni Jónsson (1954)%
	}
	\newcommand{\LaFarge}{% Glossary to the Poetic Edda
		La Farge and Tucker (1992)%
	}
	\newcommand{\Larrington}{% Carolyne Larrington translation of the Poetic Edda
		\emph{Larrington} (2005)%
	}
	\newcommand{\LMNL}{% Lexicon of Medieval Nordic Law
		\emph{LMNL} (2020)%
	}
	\newcommand{\ONP}{% Dictionary of Old Norse Prose
		\emph{ONP}%
	}
	\newcommand{\PettitEdda}{% Edward Pettit Edda
		\emph{Pettit} (2023)%
	}
	\newcommand{\Skp}{% Skaldic Poetry of the Scandinavian Middle Ages
		\emph{Skp}%
	}

	% Manuscripts
	\newcommand{\Regius}{% Codex Regius (of the poetic edda)
		\emph{R}%
	}
	\newcommand{\AM}{% AM 748 I a 4to
		\emph{A}%
	}
	\newcommand{\Hauksbok}{% Hauksbok
		\emph{H}%
	}
	\newcommand{\GylfMS}{% For referring to Gylfaginning manuscripts when verses are attested there.
		\emph{G}%
	}
	\newcommand{\RegiusProse}{% Codex Regius of the Prose Edda
		\emph{S}%
	}
	\newcommand{\Trajectinus}{% Codex Trajectinus
		\emph{T}%
	}
	\newcommand{\Wormianus}{% Codex Wormianus
		\emph{W}%
	}
	\newcommand{\Upsaliensis}{% Codex Upsaliensis
		\emph{U}%
	}
	\newcommand{\HildMS}{% For referring to the Hildebrandslied manuscript.
		\emph{Hild ms.}%
	}
	\newcommand{\Hickes}{% George Hickes
		\emph{Hickes}%
	}

	% Old texts
	% The command codes must be as close to the original language titles as possible.
	\newcommand{\Allvismal}{% Speeches of Allwise
		\emph{Allw}%
	}
	\newcommand{\Baldrsdraumar}{% The Dreams of Balder
		\emph{Dreams}%
	}
	\newcommand{\Beowulf}{% Beewolf
		\emph{Bee}%
	}
	\newcommand{\Deor}{% Dear
		\emph{Dear}%
	}
	\newcommand{\Gulatingslog}{% Law of the Gole-thing
		\emph{GolL}%
	}
	\newcommand{\FGT}{% First Grammatical Treatise
		\emph{1GT}%
	}
	\newcommand{\FostrbroedhraSaga}{% Saw of the Foster-brothers
		\emph{FbrS}%
	}
	\newcommand{\FraLoka}{% From Lock
		\emph{FrL}%
	}
	\newcommand{\Grimnismal}{% Speeches oF Grimner
		\emph{Grim}%
	}
	\newcommand{\Gylfaginning}{% The Guiling of Yilfer; for referring to Gylfaginning as a text
		\emph{Yilf}%
	}
	\newcommand{\Haleygjatal}{% Tally of the Hallowlendings
		\emph{Hal}%
	}
	\newcommand{\Harbardsljod}{% Leed of Hoarbeard
		\emph{Hoar}%
	}
	\newcommand{\Havamal}{% Speeches of the High One
		\emph{High}%
	}
	\newcommand{\HervararSaga}{% Saw of Harware
		\emph{HarS}%
	}
	\newcommand{\Hildebrandslied}{% Speeches of Hildbrand
		\emph{Hild}%
	}
	\newcommand{\Hymiskvida}{% Lay of Hymer
		\emph{Hym}%
	}
	\newcommand{\Hyndluljod}{% Leed of Hindle
		\emph{Hindle}%
	}
	\newcommand{\Rigsthula}{% Thule of Righ
		\emph{Righ}%
	}
	\newcommand{\Sigrdrifumal}{% Speeches of Sighdrive
		\emph{Sigh}%
	}
	\newcommand{\Skaldskaparmal}{% The Matter of Scoldship
		\emph{Scold}%
	}
	\newcommand{\Skirnismal}{% Speeches of Shirner
		\emph{Shirn}%
	}
	\newcommand{\ThidreksSaga}{% Saw of Thedrich
		\emph{ThedS}%
	}
	\newcommand{\Vafthrudnismal}{% Speeches of Webthrithner
		\emph{Web}%
	}
	\newcommand{\VolsungaSaga}{% Saw of the Walsings
		\emph{WalsS}%
	}
	\newcommand{\Volundarkvida}{% Lay of Wayland
		\emph{Way}%
	}
	\newcommand{\Voluspa}{% Spae of the Wallow
		\emph{Spae}%
	}
	\newcommand{\Waldere}{% Walder
		\emph{Walder}%
	}
	\newcommand{\YnglingaSaga}{% Saw of the Inglings
		\emph{IngS}%
	}
	\newcommand{\Ynglingatal}{% Tally of the Inglings
		\emph{IngT}%
	}

% Books

% Introduction, bibliography and abbreviations
	\frontmatter
	\title{%
  \Huge The \textsc{Old Germanic Scoldship}, \\
  \huge\emph{or, \\
  \textsc{Scandinavian, English} and \textsc{German Mythic} and \textsc{Heroic Alliterative Poetry, Newly Translated, Edited} and \textsc{Commented upon}} \\
  \emph{by} \\
  \Huge \textsc{Konrad Olof Lennart Rosenberg}; \\ \emph{also \textsc{Including} a \textsc{List} of \textsc{Poetic Formulæ}, and \textsc{Several Essays} on the \textsc{Ancient \\ Common Germanic Culture} \\ and \textsc{Worldview}.}}

\maketitle

\newpage\thispagestyle{empty}

\begin{center} The following people have been especially helpful in giving corrections and general feedback: Ęinarr, Nikhilasurya Dwibhashyam, Joseph S. Hopkins, John Newman, Trevor L. Payne, Thibault.\end{center}

\begin{center} \emph{\alst{V}ęl kęypts hlutar \hld\ hęf’k \alst{v}ęl notit; \\
\alst{f}ás es \alst{f}róðum vant; \\
því-at \alst{Ó}ð-rǿrir \hld\ es nú \alst{u}pp kominn \\
á \alst{a}lda vés \alst{ja}ðar} \\
(\emph{Háva mǫ́l} 106)\end{center}

\newpage\thispagestyle{empty}

\tableofcontents

\newpage

\thispagestyle{empty}\section{Abbreviations}
  \begin{itemize}% Manuscript sigla
    \item \AM\ = AM 748 I a 4° (https://handrit.is/manuscript/view/da/AM04-0748-I-a)
    \item \AMb\ = AM 748 I b 4° (https://handrit.is/manuscript/view/is/AM04-0748-Ib)
    \item \EddaBms\ = AM 757 a 4° (https://handrit.is/manuscript/view/is/AM04-0757a)
    \item \FlatMS\ = Flatsęyjarbók, GKS 1005 fol. (https://handrit.is/manuscript/view/is/GKS02-1005)
    \item \Hauksbok\ = Hauksbók, AM 544 4° (https://handrit.is/manuscript/view/en/AM04-0544)
    \item \VolsungaMS\ = NKS 1824 b 4° (https://onp.ku.dk/onp/onp.php?m9641)
    \item \Regius\ = Codex Regius of the Poetic Edda, GKS 2365 4° (https://eae.ku.dk/q.php?p=cr/poems)
    \item \RegiusProse\ = Codex Regius of the Prose Edda, GKS 2367 4° (https://handrit.is/manuscript/view/is/GKS04-2367)
    \item \Trajectinus\ = Codex Trajectinus, Traj 1374ˣ
    \item \Upsaliensis\ = Codex Upsaliensis, DG 11
    \item \Wormianus\ = Codex Wormianus, AM 242 fol. (https://clarino.uib.no/menota/text/menota/AM-242-fol)
  \end{itemize}

  \begin{itemize}% Languages
    \item Eng. = Modern English
    \item Ger. = Modern German
    \item Got. = Gotnish (or Gothic)
    \item Lomb. = Lombardic
    \item MHG = Middle High German
    \item OE = Old English
    \item OF = Old Frisian
    \item OHG = Old High German
    \item ON = Old Norse
    \item OS = Old Saxon
    \item OSwe. = Old Swedish
    \item PGmc. = Proto-Germanic
    \item PN = Proto-Norse
    \item PNWGmc. = Proto-North-West Germanic
  \end{itemize}

  \begin{itemize}% Grammar
    \item 1st = first-person
    \item 2nd = second-person
    \item 3rd = third-person
    \item acc. = accusative case
    \item cpd = compound
    \item dat. = dative case
    \item gen. = genitive case
    \item imper. = imperative mood
    \item ind. = indicative mood
    \item instr. = instrumental case
    \item nom. = nominative case
    \item pl. = plural number
    \item sg. = singular number
    \item subj. = subjunctive mood
  \end{itemize}

  \begin{itemize}% Other abbreviations
    \item cert. = certainly
    \item c. = circa
    \item cf. = \emph{confere}; compare
    \item corr. = corrected in the ms.
    \item e. = excerpt (not the whole stanza)
    \item ed. = edition, edited (by)
    \item e.g. = \emph{exemplio gratia}; for instance
    \item emend. = emendation, emended (by)
    \item fol., foll. = folio, folios
    \item i.e. = \emph{id est}; that is
    \item l., ll. = line, lines
    \item lit. = literally
    \item metr. emend. = emended based on (secure) metrical criteria
    \item ms., mss. = manuscript, manuscripts
    \item norm. = normalised from the ms. spelling
    \item om. = omitted by
    \item p., pp. = page, pages
    \item tr. = translation, translated (by)
    \item sens. emend. = emended based on sense
    \item st., sts. = stanza, stanzas
    \item viz. = \emph{vidēlicet}; namely, to wit
    \item wo. = without
    \item wrt. = with regard to
  \end{itemize}

\newpage

\bookStart{Introduction (INCOMPLETE!)}

\section{Introduction to Eddic poetry}
  Don't go too indepth on individual poems! Each one will have its own introduction.
  \subsection{Metrics and conventions}
    Alliteration
    Kennings
  \subsection{How can we know the age of the Eddic poems?}
    Linguistic criteria
    Archeological evidence
    Comparison with known Christian texts (Sólarljóð, Hugsvinnsmál)
    Snorri thought they were old
    Saxo had access to them
    Many of them clearly describe non-Icelandic surroundings
      Especially Hávamál is clearly Norwegian

\section{Ancient Germanic cult(ure)}
  \subsection{Economy (fee)}
  \subsection{Morals}
    Honour, personal integrity
    Notes on the terms \emph{argr} and \emph{ergi}
  \subsection{Religious conceptions}
    Cosmic cycles
    Reincarnation
    Analogies with other Indo-European traditions

\section{Notes to English translation}
  Point about literal translation for use by scholars of comparative mythology
    The “guiding star” of this translation effort has been literality and consistency. All previous translations (to my knowledge) have such issues as: rendering identically repeated phrases differently at various places; covering up or obscuring technical and cultural terminology; simplifying kennings and other expressions—and this often without notes, to a point where the original meaning is, at times, unrecognizable.
    While I wholly encourage all readers of sufficient interest to study Old Norse (and other ancient Germanic languages!), perhaps even using the present edition as a tool, I also realize that this is a demanding ask which not all interested students and scholars of comparative mythology, anthropology, literature, religion and other fields will be able to fulfill. I therefore want these groups to be able to have a text that is as close to the original as possible, at the very least when it regards sense and expression.
  \subsection{Anglish proper nouns}
    One of the most idiosyncratic parts of the present edition will be its handling of proper nouns. I have opted to render all cultural and religious terms, names of places, heroes, gods, and other entities by their English cognates (thus \emph{Thunder} for Old Norse \emph{Þórr}) and where such do not exist, their philologically expected English (\emph{Anglish}) forms (e.g. \emph{wallow} for Old Norse \emph{vǫlva}).
    One reason for this is ideological. I believe that these myths and poems are a common Germanic or Northern European heritage, and should be treated as such. The English once knew gods such as Weden and Thunder, and called them by names naturally evolved in their language. So too did the Germans and Scandinavians, of course, and I would hope that any translators into those languages would follow this spirit and render the names in their natural forms there as well.\footnote{For instance in German perhaps Wuten, Donner, Froh, in Swedish Oden, Tor, Frö.}
    Another is philological. Forms like Odin and Thor are, while now commonly accepted, debased. They do not even represent the Old Norse pronunciation as accurate as would be possible (for instance, Odin would be better anglicized as Othin; the dental fricative still survives in English!), and many are difficult for English speakers to pronounce. I shudder when hearing a word like \emph{ę́sir} pronounced /aɪˈsɪ:ɹ/

\section{Notes to critical edition}
  My goal with the critical editing of the texts has been to produce something as close to the original mss. as possible, without excessive emendation to the preserved recension(s). There are texts in three languages in the present edition, namely Old Norse, Old English and Old High German. Old Norse texts have been normalized according to roughly the same orthography as \textcite{FinnurEdda}. On the other hand the Old High German and Old English texts have only been lightly normalized, correcting obvious errors and marking vowel length with acute accents.

  \subsection{Normalization}
    The general principle in normalizing texts has been to strive for a uniform orthography across languages, where the same sound is written with the same character. This of course means disregarding local manuscript traditions and philological tradition, but I see this as justified. My goal is to render the texts themselves in a manner that gives as much information to the reader as possible—not to present a facsimile edition for students of paleography. Anyway, such obvious aspects of the original manuscripts as the long \emph{ſ}, arbitrary punctuation, arbitrary spelling, and lack of line breaks are almost never reproduced in modern editions of Old Germanic poetry.

    \subsubsection{Normalization of poetry}
    \begin{enumerate}
      \item Lines are broken at each long-line, not each half-line. This follows traditional practice for the publication of West Germanic poetry, while departing from that of Old Norse poetry.
      \item Cæsuræ are represented with the interpunct (·).
      \item Alliterations are marked with red colour.
    \end{enumerate}

    \subsubsection{Normalization of Old West Norse}
    The orthography is inspired by \textcite{FinnurEdda} in that it strives for a more archaic form than that of the surviving mss., one that instead represents the poetry as it may (in many cases, must) originally have looked. For this reason, it often has more in common with the proposed orthography of the First Grammatical Treatise than with the standard Old Icelandic orthography seen in most editions. The following list describes the differences from the standard orthography.

    \begin{enumerate}
    \item I distinguish short \emph{e} (from etymological short \emph{e}) and short \emph{ę} (from etymological short \emph{a} + \emph{i}-umlaut).
    \item I distinguish long \emph{á} and \emph{ǫ́}, as done by the First Grammatical Treatise.
    \item I use \emph{ǿ} and \emph{ę́} rather than the traditional \emph{œ} and \emph{æ}, to represent the vowels descended from Proto-Norse \emph{ō} and \emph{ā} after \emph{i}-umlaut (cf. the short \emph{ø, ę} < \emph{o, a} + \emph{i}-umlaut).
    \item I distinguish long nasal \emph{ȧ, ė, ï, ȯ, u̇} from long oral \emph{á, é, í, ó, ú}, as done by the First Grammatical Treatise.
    \item I restore the old \emph{s}—which in modern Scandinavian and even in most Old Norse manuscripts has become \emph{r}, but which is found consistently in old manuscripts such as AM 237 a fol (c. 1150), and fossilized in forms like \emph{þaz} (i.e. \emph{þat’s}) in \Regius—in the words \emph{es} ‘which, that, where, when’, and in inflections of \emph{vesa} (later \emph{vera}) such as \emph{es} ‘is’ (3rd sg. pres. ind.) and \emph{vas} (3rd sg. pret. ind.). The following forms retain the \emph{r}, as it is there the result of Verner’s law, and not of this (much younger) sound change: the pl. pres. ind. (\emph{erum} \&c.), the pl. pret. ind. (\emph{vǫ́rum} \&c.), and the pl. pret. subj. (\emph{vę́rim} \&c.)
    \item When metrically benefactory, I contract \emph{ek} ‘I’, \emph{eru} ‘are’, and \emph{es} ‘which; is’ to \emph{’k}, \emph{’ru} and \emph{’s}, respectively.
    \item I use \textcite{FinnurEdda}’s way of distinguishing between the relative particle \emph{es} and the verb \emph{es}: the first is appended to the previous word with only an apostrophe (e.g. \emph{hann’s} ‘he who’), while the second is separated by a space (e.g. \emph{hann ’s} ‘he is’).
    \end{enumerate}

    \subsubsection{Normalization of Old English}

    \subsubsection{Normalization of Old High German}

  \subsection{Manuscripts}

    \subsubsection{Eddic poetry}
    There are two surviving ancient mss. which contain full Eddic poems.

    The first and most important is GKS 2365 4to, here \Regius. It dates to the 1270s and has 45 surviving leaves, containing TODO poems. Of these 10 are mythological, and the rest heroic, dealing with legends mostly of the Migration Period. Notably, following fol. 32, there is a large gap of missing pages. This occurs in the heroic section, specifically cutting off \Sigrdrifumal. It is unclear how many leaves and poems went missing.
    \Regius\ is not just a compilation of poems, it shows editorial input as well. Several of the mythological poems are separated by short prose sections, which tie them together into a loose frame narrative, though it is clear from their style and composition that they are originally separate works. When it comes to the heroic poems long prose sections occur both within and between them, creating a \inx[C]{saw}-like narrative where the prose in many cases holds up the poetry, rather than the reverse. For further literature see TODO.

    The second ms. is AM 748 I a 4to, here \AM. It dates to the 1300s and is but a fragment, consisting of just 6 leaves. It contains only mythological poems, and in a different order from \Regius; unlike it there is no trace of a frame narrative. On the first two leaves are contained the final stanzas of \Harbardsljod\ (1r–v), the complete \Baldrsdraumar\ (1v–2r), and the first verses of \Skirnismal, after which a single leaf has been lost. The next four leaves follow eachother and contain the second half of \Vafthrudnismal, the complete \Grimnismal\ and \Hymiskvida, and the beginning of the prose introduction to \Volundarkvida. \AM\ is the only medieval manuscript attesting \Baldrsdraumar, and its variants of the poems attested in \Regius\ are clearly not copied from it, but rather derive from a common ancestor. This makes it very valuable for textual criticism. For further literature see TODO.

    Several Eddic poems are quoted in \Gylfaginning, namely (TODO): \Voluspa, \Vafthrudnismal, \Grimnismal. The text also quotes a few fragmentary verses of Eddic character (possibly from lost Eddic poems), which have here been edited together with their surrounding prose passages. For \Gylfaginning\ I have relied on the following four main mss.:\begin{enumerate}
	   \item The Codex Regius of the Prose Edda \RegiusProse\ (GKS 2367 4to; 1300-1350)
     \item The Codex Trajectinus \Trajectinus\ (Traj 1374; a c. 1595 paper copy of a ms. closely related to \RegiusProse.)
     \item The Codex Wormianus \Wormianus\ (AM 242 fol.; 1340–70)
     \item The Codex Upsaliensis \Upsaliensis\ (DG 11; 1300–25)\end{enumerate}

    For discussion on their internal stemmatics and origins I refer to \textcite{Haukur2017}. When all employed witness mss. of \Gylfaginning\ agree on a reading the siglum \GylfMS\ is used in the critical apparatus, which is thus equivalent to \RegiusProse\Trajectinus\Wormianus\Upsaliensis.

    A few other Eddic poems have also been edited. One of them, \Rigsthula, only survives in \Wormianus, though it is sadly incomplete (see its Introduction). Other Eddic poems survive only in younger paper mss., namely: TODO. While I have not consulted these paper mss. for poems attested in medieval mss., I have had to rely on them for these poems. Their exclusive survival there does not necessarily prove them to be late antiquarian works, as is clearly shown by \Baldrsdraumar, which among medieval mss. is only attested in the fragmentary \AM. It thus cannot be excluded that some of these poems would have existed in other lost medieval mss., perhaps even in the lost pages of \Regius\ or \AM.

    \subsubsection{West Germanic poetry}

    As none of the West Germanic poems edited here (TODO: Will we be editing other poems than Hildebrandslied?) survive in more than one copy, the specific details of their transmission is discussed in their individual Introductions.

  \printbibliography% Does it work?


% Theology, mythology, generally independent order
	\mainmatter
	\bookStart{The Spae of the Wallow}[Vǫluspǫ́]

% Introduction.

The \textbf{\inx{Spae} of the \inx{Wallow}} is the most comprehensive mythological text surviving from Heathen times. It takes the form of the monologue of a \inx{wallow}[C], summoned by Weden in order to reveal mythological exposition. In this it fits closely with \Vafthrudnismal, \Grimnismal, \Sigrdrifumal\ and \Allvismal, but differs from them in several ways: there is no format of a dialogue (this it shares with \Grimnismal) or competition; the meter is in \Fornyrdislag; and it gives an overview of the mythological chronology in an otherwise unparalleled way.

Many events are related in a very allusive fashion, and not all of them are clear. There are also some likely gaps, possibly the result of misplaced verses. The poem begins with a bid for silence (v. 1), and the wallow reckoning her earliest memories (v. 2). She then recounts the ordering of the cosmos by the gods (vv. 3–6) and the earliest golden age (vv. 7–8), which however is interrupted by the intrusion of three unidentified ettin maidens (v. 8, and see note there). After this follow two verses about the shaping of the dwarfs (9–10), and then several independent \emph{dwarf-tallies} (vv. 11–15), which are undoubtedly later inserts. We then return to the gods, specifically the creation of man (vv. 16–17). Judging from the end of verse 8 and the beginning of verse 16, it seems likely that these various dwarf-related verses have taken the place of some other verse. After this we get a description of the great tree Ugdrassle (v. 18), and the three norns living under it (v. 19).

This is where our two full recensions diverge. We have here followed the order of \Regius\ due to the age of its text, but whether it is the original is hard to say. In \Regius\ the wallow recounts the earliest war in the world

The poem is attested in full in two independent recensions. The first is \Regius\ (GKS 2365 4to; 1270s), where it is the first poem, found on folios 1r–3r. Second is Hawksbook, \Hauksbok\ (AM 544 4to; 1300–75), where it is found at 20r–21r in the middle of a large collection of saws and Catholics works. Many verses are also cited in \Gylfaginning, which here has the general siglum \GylfMS—to avoid confusion, it is only used when all employed witness mss. agree. See further the General Introduction. %TODO: elaborate

Order of verses by manuscript, compared to this edition. As most verses in \GylfMS\ are quoted on their own, and have little relation to the original order, these are simply marked with plus signs. When verses are quoted in a series, they are preceded by an alphabetically incrementing letter denoting which series they belong to. When there is a major difference in a ms. relative to the ed., such as in v. 10 where \GylfMS\ omits the first two lines, it is then marked with a star. The verses beginning with \emph{Þȧ gingu ręgin ǫll ...} are represented by the following sentence.
\begin{longtable}{|c c c c c c|}
	\hline
	\multicolumn{2}{|c}{\emph{Current ed.}} & \Regius & \Hauksbok & \RegiusProse\Trajectinus\Wormianus & \Upsaliensis \\ [0.5ex]
	\hline\hline
	1 & Hljóðs bið’k allar hęlgar kindir & 1 & 1 & − & − \\
	2 & Ek man jǫtna ár of borna & 2 & 2 & − & − \\
	3 & Ár vas alda þar’s Ymir byggði & 3 & 3 & + & + \\
	4 & Áðr Burs synir bjǫðum of ypðu & 4 & 4 & − & − \\
	5 & Sól varp sunnan sinni mȧna & 5 & 5 & +* & +* \\
	6 & ... nǫ́tt ok niðjum nǫfn of gǫ́fu & 6 & 6 & − & − \\
	7 & Hittusk ę̇sir ȧ Iðavęlli & 7 & 7 & − & − \\
	8 & Tęflðu í túni, tęitir vǫ́ru & 8 & 8 & − & − \\
	9 & ... hvęrr skyldi dverga drótt of skępja & 9 & 9 & B1 & B1 \\
	10 & Þar vas Móðsognir mę́ztr of orðinn & 10 & 10 & B2* & B2* \\
	− & \emph{Dwarf-tallies} & 11–15 & 11–16 & + & + \\
	16 & Unz þrír kvǫ̇mu ór því liði & 16 & 17 & − & − \\
	17 & Ǫnd þau né ǫ́ttu, óð þau né hǫfðu & 17 & 18 & − & − \\
	18 & Ask vęit’k standa hęitir Yggdrasill & 18 & 19 & + & + \\
	19 & Þaðan koma męyjar margs vitandi & 19–20 & 20–21 & − & − \\
	20 & Þat man hǫ̇n folkvíg fyrst í hęimi & 21–22 & 27 & − & − \\
	21 & Hęiði hétu, hvar’s til húsa kom & 23 & 28 & − & − \\
	22 & ... hvárt skyldu ę̇sir afráð gjalda & 24 & 29 & − & − \\
	23 & Flęygði Óðinn ok í folk of skaut; & 25 & 30 & − & − \\
	24 & ... hvęrr hęfði lopt alt lę́vi blandit  & 26 & 22 & C1 & C1 \\
	25 & Þȯrr ęinn þar vá þrunginn móði & 27 & 23 & C2* & C2* \\
	26 & Vęit hǫ̇n Hęimdallar hljóð of folgit & 28 & 24 & − & − \\
	27 & Ęin sat hǫ̇n úti, þȧ’s hinn aldni kom & 29 & − & − & − \\
	28 & Alt vęit’k, Óðinn, hvar auga falt & 29 & − & + & + \\
	29 & Valði hęnni Hęrfǫðr hringa ok męn & 30 & − & − & − \\
	30 & Sá hǫ̇n valkyrjur vítt of komnar & 31 & − & − & − \\
	31 & Ek sá Baldri, blóðgum tívi & 32 & − & − & − \\
	32 & Varð af męiði, þęim’s mę́r sýndisk & 33 & − & − & − \\
	33 & Þó hann ę́va hęndr né hǫfuð kęmbði & 34 & − & − & − \\
	34 & Þȧ kná Váli vígbǫnd snúa & − & 31 & − & − \\
	35 & Hapt sá hǫ̇n liggja und Hveralundi & 35 & 32* & − & − \\
	36 & Ǫ́ fęllr austan of ęitrdala & 36 & − & − & − \\
	37 & Stóð fyr norðan ȧ Niðavǫllum & 36 & − & − & − \\
	38 & Sal sá hǫ̇n standa sólu fjarri & 37 & 36 & E1 & E1 \\
	39 & Sér hǫ̇n þar vaða þunga strauma & 38 & 37 & E2* & E2* \\
	40 & Austr býr hin aldna í Járnviði & 39 & 25 & A1 & A1 \\
	41 & Fyllisk fjǫrvi fęigra manna & 40 & 26 & A2 & A2 \\
	42 & Sat þar ȧ haugi ok sló hǫrpu & 41 & 34 & − & − \\
	43 & Gól of ǫ̇sum Gollinkambi & 42 & 35 & − & − \\
	44, 49, 57 & Gęyr Garmr mjǫk fyr Gnipahęlli & 43, 46, 55 & 33, 38, 43, 48, 51 & − & − \\
	45 & Brǿðr munu bęrjask ok at bǫnum verðask, & 44 & 39 & − & − \\
	46 & Lęika Míms synir, ęn mjǫtuðr kyndisk & 45 & 40 & D1* & D1* \\
	47 & Skęlfr Yggdrasils askr standandi & 45* & 41 & D1* & D1* \\
	48 & Hvat ’s með ǫ̇sum? hvat ’s með ǫlfum? & 49 & 42 & D2 & D2* \\
	50 & Hrymr ękr austan, hęfsk lind fyrir & 47 & 44 & D3 & − \\
	51 & Kjóll fęrr austan koma munu Múspells & 48 & 45 & D4 & − \\
	52 & Surtr fęrr sunnan með sviga lę́vi & 50 & 46 & +, D5 & + \\
	53 & Þȧ kømr Hlínar harmr annarr framm & 51 & 47 & D6 & − \\
	54 & Þȧ kømr hinn mikli mǫgr Sigfǫður & 52 & − & D7 & − \\
	55 & Gínn lopt yfir lindi jarðar & − & 48 & — & − \\
	56 & Þȧ kømr hinn mę́ri mǫgr Hlǫðynjar & 53* & 49* & C8 & − \\
	57 & Sól tér sortna, søkkr fold í mar & 54 & 50 & C9 & − \\
	59 & Sér hǫ̇n upp koma ǫðru sinni & 56 & 52 & − & − \\
	60 & Finnask ę̇sir ȧ Iðavęlli & 57* & 53 & − & − \\
	61 & Þar munu ęptir undrsamligar & 58 & 54 & − & − \\
	62 & Munu ȯsánir akrar vaxa & 59 & 55 & − & − \\
	63 & Þȧ kná Hø̇nir hlautvið kjósa & 60 & 56 & − & − \\
	64 & Sal sér hǫ̇n standa sólu fęgra & 61 & 57 & + & + \\
	65 & Þar kømr hinn dimmi dręki fljúgandi & 62 & 59 & − & − \\
	X & Þȧ kømr hinn ríki at ręgindȯmi & − & 58 & − & − \\ [1ex]
	\hline
\end{longtable}

\bvg {\small Greeting to the audience, bidding of Weden.}
\bva\ledleftnote{\Regius\Hauksbok}\alst{H}ljóðs bið’k allar \hld\ \edtext{\alst{h}ęlgar}{\lemma{hęlgar}\Afootnote{\emph{om.} \Regius}} kindir, &
\alst{m}ęiri ok \alst{m}inni \hld\ \alst{m}ǫgu Hęimdallar; &
\alst{v}ildu at, \alst{V}alfǫðr, \hld\ \alst{v}ęl fram tęlja’k &
\alst{f}orn spjǫll \alst{f}ira, \hld\ þau’s \alst{f}ręmst of man?\eva

\bvb For hearing I ask all holy kindreds, greater and lesser, sons of \inx{Homedall}!\footnoteB{Cf. \Rigsthula, wherein Righ, identified by the prose as Homedall, sires three classes of men (namely earls, churls and thralls).—The wallow has been summoned to recite, and asks for all beings present to be silent.} Wilt thou, Father of the Slain \ken{Weden}[1], that I well tell forth the ancient sayings of men, those I foremost recall?\footnoteB{Cf. \Vafthrudnismal\ 34, 35 with very similar phrasing.}\evb
\evg


\bvg {\small Wallow reckons what she recalls; the creation and ordering of the world.}
\bva\ledleftnote{\Regius\Hauksbok}Ek man jǫtna \hld\ ár of borna, &
þȧ es forðum \hld\ mik fǿdda hǫfðu; &
níu man’k hęima, \hld\ níu \edtext{íviðjur}{\Afootnote{\emph{Previously read} íviði, \emph{but closer study of} \Regius\ \emph{has disproven this. See Stefán Karlsson 1979.}}}, &
mjǫtvið mę́ran \hld\ fyr mold neðan.\eva

\bvb I recall \inx{Ettins}, born of yore, those who anciently had nourished me. Nine \inx{Homes} I recall, nine \inx{Inwithies}; the renowned \inx{Metwood} beneath the soil.\footnoteB{Certainly Ugdrassle, “beneath the soil” likely referring to it still being a seed.}\evb
\evg


\bvg
\bva\ledleftnote{\Regius\Hauksbok\GylfMS}Ár vas alda \hld\ \edtext{þar’s Ymir byggði}{\lemma{þar’s ... byggði “there ... dwelled”}\Afootnote{þat’s ekki vas “that which nothing was” \GylfMS}}, &
vas-a sandr né sę́r, \hld\ né svalar unnir; &
jǫrð fansk ę́va \hld\ né upphiminn; &
gap vas ginnunga, \hld\ ęn gras \edtext{hvęrgi}{\Afootnote{ekki \Hauksbok}}.\eva

\bvb It was the beginning of \inx{elds}, there where Yimer dwelled; was there not sand nor sea, nor cool waves. The earth was never found, nor \inx{Up-heaven}; a gap was of ginnings,\footnoteB{\emph{ginnungr} (of which \emph{ginnunga} would be the genitive plural) means ‘hawk’ in the Scoldish poetry, but that meaning hardly makes sense here, unless it is taken as an obscure sky-kenning referring to the primeval void.} but grass nowhere.\evb
\evg


\bvg
\bva\ledleftnote{\Regius\Hauksbok}Áðr Burs synir \hld\ bjǫðum of ypðu, &
þęir es Miðgarð \hld\ mę́ran skópu; &
sól skęin sunnan \hld\ ȧ salar stęina; &
þȧ vas grund gróin \hld\ grø̇num lauki.\eva

\bvb Before the sons of Bur the flatlands did upwards lift, they who shaped the renowned Middenyard. Sun shone from the south on the stones of the hall; then was the ground grown with green leek.\footnoteB{The sons of Bur, that is Weden, Will and Wigh (cf. \Gylfaginning\ TODO), lift the lands out of the primordial chaos (the Gap of Ginnings).}\evb
\evg


\bvg
\bva\ledleftnote{\Regius\Hauksbok\GylfMS}\edtext{Sól varp sunnan, \hld\ sinni mȧna, &
hęndi hinni hǿgri \hld\ \edtext{of himinjǫður}{\Afootnote{vm himin iodyr \Regius\ of ioður \Hauksbok}}}{\lemma{Sól ... himinjǫður}\Afootnote{\emph{om.} \GylfMS}}; &
sól þat né vissi, \hld\ hvar hǫ̇n sali átti; &
\edtext{stjǫrnur þat né vissu, \hld\ hvar þę́r staði ǫ́ttu}{\lemma{stjǫrnur ... ǫ́ttu}\Bfootnote{In \GylfMS\ follows 5, so that order is sun, moon, stars.}}; &
mȧni þat né vissi, \hld\ hvat hann męgins átti.\eva

\bvb Sun cast from the south—the companion of Moon\footnoteB{At times translated as “its moon”; this cannot be correct, as \emph{mȧni} ‘moon’ is masculine, while \emph{sinni}, dative singular of \emph{sínn} ‘its (reflexive)’ is feminine.}—her right hand over heaven’s rim;\footnoteB{The sun heaved herself up over the horizon and rose for the first time.} Sun knew not, where halls she owned; stars knew not, where steads they owned; Moon knew not, what sort of might he owned.\evb
\evg


\bvg
\bva\ledleftnote{\Regius\Hauksbok}Þȧ gingu ręgin ǫll \hld\ ȧ rǫkstóla, &
ginnhęilǫg goð, \hld\ ok umb þat gę́ttusk. &
Nǫ́tt ok niðjum \hld\ nǫfn of gǫ́fu, &
morgin hétu \hld\ ok miðjan dag, &
undurn ok aptan, \hld\ ǫ́rum at tęlja.\eva

\bvb Then went the Powers all onto the rake-seats\footnoteB{Judgment-seats; first element \emph{rǫk} defined by \CV\ as ‘reason, ground, origin’.}: the gin-holy gods, and from each other took counsel about that.\footnoteB{10, 23, 25 (TODO) would suggest two lines be missing here.}—To night and the moon-phases names did they give; morning they called, and middle day; afternoon and evening, the years for to tally.\footnoteB{Cf. \emph{Web} 23, 25.}\evb
\evg


\bvg
\bva\ledleftnote{\Regius\Hauksbok}Hittusk ę̇sir \hld\ ȧ Iðavęlli, &
\edtext{þęir’s hǫrg ok hof \hld\ hǫ́ timbruðu}{\lemma{þęir’s ... timbruðu “they ... timbered”}\Afootnote{afls kostuðu \hld\ allz freistuðu “[their] strength they tried; all they tempted” \Hauksbok}}; &
afla lǫgðu, \hld\ auð smíðuðu, &
tangir skópu \hld\ ok tól gęrðu.\eva

\bvb The Ease found each other on the \inx{Idewolds}, they who \inx{harrows} and \inx{hoves} high timbered: hearths they laid, wealth they smithed, tongs they shaped, and tools they made.\evb
\evg


\bvg
\bva\ledleftnote{\Regius\Hauksbok}Tęflðu í túni, \hld\ tęitir vǫ́ru, &
vas þęim véttugis \hld\ vant ór golli, &
unz þríar kvǫ̇mu \hld\ þursa męyjar, &
ȧmátkar mjǫk, \hld\ ór Jǫtunhęimum.\eva

\bvb They played \inx{Tavel} in the yards, joyous were they: was for them no lack of gold\footnoteB{Cf. v. 59.}—until three came, maidens of \inx{thurses}, greatly loathsome, out of \inx{Ettinham}.\footnoteB{These are immediately forgotten and not again mentioned (unless they are taken to be the norns in v. 21, but they would then be introduced twice).—There seems to be something missing between here, perhaps giving further information of the three thurse-maidens, or detailing the reason for the creation of dwarfs?}\evb
\evg


\bvg {\small Creation of dwarfs.}
\bva\ledleftnote{\Regius\Hauksbok\GylfMS}Þȧ gingu ręgin ǫll \hld\ ȧ rǫkstóla, &
ginnhęilǫg goð, \hld\ ok umb þat gę́ttusk: &
\edtext{hvęrr skyldi dverga}{\lemma{hvęrr skyldi dverga “Who would ... of dwarfs”}\Afootnote{\emph{thus} \Regius\Wormianus\Upsaliensis; at skyldi dverga “That they would ... of dwarfs” \RegiusProse\Trajectinus; hverir skyldu dvergar “Which dwarfs would [shape the people]” \Hauksbok}} \hld\ \edtext{drótt of}{\Afootnote{\emph{thus} \GylfMS; drotin (\emph{late definite wo. doubt not original}) \Regius; dróttir “the people” \Hauksbok}} \edtext{skępja}{\Afootnote{spekia “soothe [the troop]” \Upsaliensis}} &
\edtext{ór \edtext{brimi blóðgu}{\lemma{brimi blóðgu “bloody surf”}\Afootnote{\emph{thus} \Hauksbok\RegiusProse\Wormianus\Upsaliensis; Brimis blóði “the blood of Brimmer” \Regius\Trajectinus}} \hld\ ok ór \edtext{blǫ́um lęggjum}{\lemma{blǫ́um lęggjum “blue-black legs”}\Afootnote{\emph{metr. emend}; ‘blám leggiom’ “id.” \Regius; Bláins lęggjum “the legs of Blown” \Hauksbok\Wormianus; Bláms lęggjum (\emph{wo. doubt corrupt form of former}) \RegiusProse\Trajectinus\Upsaliensis}}}{\lemma{ór brimi ... lęggjum}\Bfootnote{I think that the poem simply telling of “the bloody surf” and “the blue-black legs” fits better with its general allusive style, but this choice may be somewhat controversial.}}?\eva

\bvb — Then went the Powers all onto the rake-seats: the gin-holy gods, and from each other took counsel about that: Who would shape the troops of \inx{dwarfs}, out of the bloody surf, and out of the blue-black legs\footnoteB{Gurevich (\emph{Skp} 2017, p. 693) (employing the translation of \FaulkesEdda\ p. 16) interprets the “legs of Blown (\emph{a dwarf})” as a kenning for ‘stone’, but this disagrees with the prose in \Gylfaginning\ (TODO), which states that the dwarfs first originated as maggots in Yimer’s rotting corpse.}?\evb
\evg


\bvg
\bva\ledleftnote{\Regius\Hauksbok\GylfMS}\edtext{\edtext{Þar vas Móðsognir}{\Afootnote{\emph{thus} \Hauksbok; ‘Þar mótſognir vitnir’ “there Mootsowner wolf” (\emph{wo. doubt corrupt}) \Regius\ — \emph{The prose of} \Gylfaginning\ \emph{confirms reading Móðsognir.}}} \hld\ mę́ztr of orðinn &
dverga allra, \hld\ ęn Durinn annarr;}{\lemma{Þar ... annarr “There ... second”}\Afootnote{\emph{om.} \GylfMS}} &
\edtext{\edtext{þęir manlíkun \hld\ mǫrg of gęrðu,}{\lemma{þęir ... gęrðu “They ... many”}\Afootnote{\emph{thus} \Regius\Hauksbok\Upsaliensis; þar manlíkun / mǫrg of gęrðusk (\emph{norm.}) “There man-likenesses many were made” \RegiusProse\Trajectinus\Wormianus}} &
dvergar \edtext{í}{\lemma{ór “out of”}\Afootnote{\emph{thus} \Regius\ í “in” \RegiusProse\Trajectinus\Wormianus\Upsaliensis\Hauksbok}} jǫrðu, \hld\ \edtext{sęm Durinn sagði}{\lemma{sęm Durinn sagði “as Dorn said”}\Afootnote{\emph{thus} \Regius\Hauksbok\RegiusProse\Wormianus; sem dur menn sagdi “as door-men said” \Trajectinus; sem þeim dyrinn kendi “as the animals taught them” \Upsaliensis}}.}{\lemma{þęir ... sagði “They ... said.”}\Bfootnote{There are two conflicting forms of the verse. Either the dwarfs were created on their own; this is supported by the prose of \Gylfaginning\ (see note to last v.) and by the form of its verse. On the other hand, both \Regius\ and \Hauksbok\ have the “worthiest” dwarfs Moodsowner and Dorn shaping “man-likenesses” out of soil. I have gone with the latter reading, but both should be considered.}}\eva

\bvb There was Moodsowner become the worthiest of all dwarfs, but Dorn [was] second. They made man-likenesses many; dwarfs out of the earth, as Dorn said.\evb
\evg

%TODO: move these verses to appendix.
\bvg {\small Two lists of dwarfs. That both belonged to the original poem is impossible, since several names (Oakenshield, Great-grandfather) appear in both. The three following verses seem to belong together, since there is no repetition of names. From the last line of the middle one, it seems that it should have been placed at the end of the group.}
\bva\ledleftnote{\Regius\Hauksbok\GylfMS}Nýi ok Niði, \hld\ Norðri, Suðri, &
Austri, Vestri, \hld\ Alþjófr, Dvalinn, &
Bívurr, Bávurr, \hld\ Bǫmburr, Nóri, &
Ȧnn ok Ȧnarr, \hld\ Ái, Mjǫðvitnir.\eva

\bvb — New and Nithe, Norther and Suther, Easter and Wester, Allthief, Dwollen, Bewer, Bower, Bamber, Noor, Own and Owner, Great-grandfather, Meadwitner.\evb
\evg


\bvg
\bva\ledleftnote{\Regius\Hauksbok\GylfMS}Vęigr ok Gandalfr, \hld\ Vindalfr, Þráinn, &
Þękkr ok Þorinn, \hld\ Þrór, Vitr ok Litr, &
Nár ok Nýráðr, \hld\ nú hęf’k dverga, &
Ręginn ok Ráðsviðr, \hld\ rétt of talða.\eva

\bvb Wey and Gandelf, Windelf, Thrown, Thetch and Thorn, Throo, Wit and Lit, Nee and Newred—now have I the dwarfs—Rain and Redswith—rightly tallied.\evb
\evg


\bvg {\small Second list.}
\bva\ledleftnote{\Regius\Hauksbok\GylfMS}Fíli, Kíli, \hld\ Fundinn, Náli, &
Hępti, Víli, \hld\ Hannarr, Svíurr, &
Frár, Hornbori, \hld\ Frę́gr ok Lȯni, &
Aurvangr, Jari, \hld\ Ęikinskjaldi.\eva

\bvb Filer, Chiler, Found and Needler, Hefter, Wiler, Hanner, Swigher, Fraw, Hornborer, Fray and Looner, Earwong, Earer, Oakenshield.\evb
\evg


\bvg
\bva\ledleftnote{\Regius\Hauksbok\GylfMS}Mál es dverga \hld\ í Dvalins liði &
ljȯna kindum \hld\ til Lofars tęlja, &
\edtext{þęir}{\Afootnote{þeim \Hauksbok}} es sóttu \hld\ frȧ salar stęini &
aurvanga sjǫt \hld\ til Jǫruvalla.\eva

\bvb — ’Tis time to tally the dwarfs in Dwollen’s host [back] to Loffer, for the kindreds of men;\footnoteB{A standard genealogical introduction (compare \Haleygjatal\ 1). The line of dwarfs is to be counted to their progenitor, Loffer. This possibly disagrees with the earlier introduction (“There was ...”), where Moodsown is said to be the foremost of the dwarfs, and Loffer is not mentioned.} they who sought, from the stone of the hall, the abode of \inx{Earwongs}\footnoteB{\CV\ \emph{aurvangr} ‘a loamy field’, and indeed this fits etymologically.} to the \inx{Erwolds}.\footnoteB{\Gylfaginning\ (TODO): “But these came from Swornshigh (\emph{Svarinshaugr}) to the Earwongs on the Erwolds, and thence Lofer is come; these are their names: Sherper (\emph{Skirpir}), Werper (\emph{Virpir}), Showfind, Great-grandfather, Elf and Ing (\emph{Ingi}), Oakenshield, Fale (\emph{Falr}), Frost, Finn, Ginner.”}\evb
\evg


\bvg
\bva\ledleftnote{\Regius\Hauksbok\GylfMS}Þar vas Draupnir \hld\ ok Dolgþrasir, &
Hár, Haugspori, \hld\ Hlévangr, Glói, &
Skirfir, Virfir, \hld\ Skáfiðr, Ái, &
Alfr ok Yngvi, \hld\ Ęikinskjaldi, &
Fjalarr ok Frosti, \hld\ Finnr ok Ginnarr; &
Þat mun \edtext{ę́}{\Afootnote{\emph{om.} \Regius}} uppi, \hld\ meðan ǫld lifir, &
langniðja-tal \hld\ \edtext{til}{\Afootnote{\emph{om.} \Hauksbok}} Lofars hafat.\eva

\bvb There was Dreepen and Dollowthrasher, High, Highspurer, Leewong, Glower, Sherver, Werver, Showfind, Great-grandfather, Elf and Ing, Oakenshield, Feller and Frost, Finn and Ginner: That will ever be remembered, while the \inx{eld} lives\footnoteB{Two archaic formulae. The first literally “that will ever up above”, cf. \Hervarar\ TODO: “We two are cursed, brother, thy bane am I become! That will ever be remembered (\emph{þat mun ę́ uppi}, but both mss. \emph{þat mun enn uppi}), evil is the doom of the norns!”. The second is found in a runic inscription, U 323 (980–1015): “Ever will lie, while the eld lives (\textbf{meþ + altr + lifiʀ} \emph{með aldr lifir}), the hard-hammered bridge, broad, after a good man.”}, the tally of descendants, heaved to Lofer.\evb
\evg


\bvg {\small Creation of first men.}
\bva\ledleftnote{\Regius\Hauksbok}Unz \edtext{þrír}{\Afootnote{\emph{gramm. emend.} þrjár (\emph{norm.}) \Regius\Hauksbok}} kvǫ̇mu \hld\ \edtext{ór því liði}{\Afootnote{þussa brúðir “brides of thurses” (\emph{wo. doubt corrupt}) \Hauksbok}} &
\edtext{ǫflgir ok ȧstkir}{\Afootnote{ȧstkir ok ǫflgir \Hauksbok}} \hld\ ę̇sir at húsi; &
fundu ȧ landi \hld\ lítt męgandi &
Ask ok Emblu \hld\ ørlǫglausa.\eva

\bvb — Until three came out of that host: strong and lovely Ease along the houses; they found on land the little availing Ash and Emble, \inx{orlay}-less.\footnoteB{For, according to \Gylfaginning\ (TODO: reference), they were pieces of driftwood.}\evb
\evg


\bvg
\bva\ledleftnote{\Regius\Hauksbok}Ǫnd þau né ǫ́ttu, \hld\ óð þau né hǫfðu, &
lǫ́ né lę́ti \hld\ né litu góða; &
ǫnd gaf Óðinn, \hld\ óð gaf Hø̇nir, &
lǫ́ gaf Lóðurr \hld\ ok litu góða.\eva

\bvb Breath they owned not, \inx{wode} they had not, not craft nor sound, nor good complexion. Breath gave Weden, wode gave Heen, craft gave Lother, and good complexion.\evb
\evg


\bvg {\small The ash of Ugdrassle and its three norns.}
\bva\ledleftnote{\Regius\Hauksbok\GylfMS}Ask vęit’k \edtext{standa}{\lemma{standa “stand[ing]”}\Afootnote{\emph{thus} \Regius\Hauksbok\Upsaliensis; ausinn “[is] poured” \RegiusProse\Trajectinus\Wormianus}}, \hld\ hęitir \edtext{Yggdrasill}{\Afootnote{Yggdrasils \RegiusProse}}, &
hǫ́r \edtext{baðmr}{\lemma{baðmr “beam”}\Afootnote{borinn “born” (\emph{wo. doubt corrupt}) \Upsaliensis}}, \edtext{ausinn}{\lemma{ausinn “poured”}\Afootnote{hęilagr (\emph{norm.}) “holy” \GylfMS}} \hld\ hvíta auri; &
þaðan koma dǫggvar \hld\ \edtext{þę́r’s}{\Afootnote{er “which” \RegiusProse\Trajectinus}} í dala falla; &
\edtext{stęndr}{\Afootnote{\emph{add.} hann \RegiusProse\Trajectinus}} \edtext{ę́}{\Afootnote{\emph{om.} \Upsaliensis}} yfir \edtext{grø̇nn}{\Afootnote{‘grvnn’ \RegiusProse; ‘grein’ \Upsaliensis}} \hld\ Urðar brunni.\eva

\bvb — An ash I know standing, \inx{Ugdrassle} ’tis called: a high beam\footnoteB{Tree.}, poured with white mud\footnoteB{Compare perhaps with the Indian ritual pouring of beverages onto the \emph{lingam}.—For the whole passage compare 27.}. Thence come the dew-drops which in the dales fall; it stands ever green over the \inx{Well of Weird}.\evb
\evg


\bvg
\bva\ledleftnote{\Regius\Hauksbok}Þaðan koma męyjar \hld\ margs vitandi &
þríar ór þęim \edtext{sę́}{\lemma{sę́ “lake”}\Afootnote{sal “hall” \Hauksbok}}, \hld\ es \edtext{und}{\lemma{und “beneath”}\Afootnote{ȧ “on” \Hauksbok}} þolli stęndr; &
Urð hétu ęina, \hld\ aðra Verðandi, &
skǫ́ru ȧ skíði, \hld\ Skuld hina þriðju &
þę́r lǫg lǫgðu, \hld\ þę́r líf køru, &
alda bǫrnum, \hld\ ørlǫg \edtext{sęggja}{\lemma{sęggja “of men”}\Afootnote{at segia “to say” \Hauksbok}}.\eva

\bvb Thence come maidens, much knowing: three out of that lake, which stands beneath the pine\footnoteB{But here simply meaning ‘tree’; perhaps the same applies for “ash” earlier.}: Weird they called one, the other Worthing—carved they on boards—Shild the third. Laws they laid, lives they chose: for the children of mortals, the \inx{orlay}[C] of men.\evb
\evg


\bvg {\small The origin of the Wallow.}
\bva\ledleftnote{\Regius\Hauksbok}Þat man hǫ̇n folkvíg \hld\ fyrst í hęimi, &
es Gollvęigu \hld\ gęirum studdu &
ok í hǫll Háars \hld\ hȧna bręnndu, &
\edtext{þrysvar bręnndu}{\Afootnote{\emph{repeated twice} \Hauksbok}} \hld\ þrysvar borna, &
opt ȯsjaldan, \hld\ þó hǫ̇n ęnn lifir.\eva

\bvb — That troop-war she recalls\footnoteB{While appealing to read \emph{folk-víg} ‘troop-war’ as meaning ‘ethnic conflict’, thus describing the war between the Ease and Wanes, \emph{folk} almost certainly here carries its earlier meaning of ‘troop, group of warriors’.}, the first in the \inx{home}, as Goldwey with spears they goaded, and in the hall of \inx{Higher} <= Weden>\ken{Walhall}[1] burned her: thrice they burned the thrice born; often unseldom, though she yet lives.\footnoteB{Very cryptic. TODO: double check Snorri. Goldwey was apparently burned three times “often unseldom” (in short succession?) by the Ease, which yet did not kill her?}\evb
\evg


\bvg
\bva\ledleftnote{\Regius\Hauksbok}Hęiði hétu, \hld\ hvar’s til húsa kom, &
\edtext{vǫlu}{\Afootnote{ok vǫlu \Hauksbok}} \edtext{velspáa}{\Afootnote{\emph{metr. emend.}; ‘uel spá’ \Regius; ‘vel spa’ \Hauksbok}}, \hld\ vitti hǫ̇n ganda; &
sęið \edtext{hvar’s kunni}{\Afootnote{hon kvnni \Regius; hon hvars hvn kunni \Hauksbok}}, \hld\ sęið \edtext{hug lęikinn}{\Afootnote{hon leikinn \Regius; hon hugleikin \Hauksbok}}; &
ę́ vas hǫ̇n angan \hld\ illrar brúðar.\eva

\bvb Heath they called her, where to houses she came: a well-spaeing\footnoteB{Gifted at soothsaying.} \inx{wallow}, she bewitched \inx{gands}. She soth\footnoteB{Past tense of sithe (ON \emph{síða}) ‘to enchant, bewitch’.} where she could, she soth deluded minds; ever was she the love of an evil bride.\evb
\evg


\bvg {\small War between Ease and Wanes.}
\bva\ledleftnote{\Regius\Hauksbok}Þȧ gingu ręgin ǫll \hld\ ȧ rǫkstóla, &
ginnhęilǫg goð, \hld\ ok umb þat gę́ttusk: &
hvárt skyldu ę̇sir \hld\ afráð gjalda, &
eða skyldu goð ǫll \hld\ gildi ęiga?\eva

\bvb Then went the Powers all onto the rake-seats: the gin-holy gods, and from each other took counsel about that: whether the Ease should tribute yield, or should the gods all a banquet hold?\evb
\evg


\bvg
\bva\ledleftnote{\Regius\Hauksbok}Flęygði Óðinn \hld\ ok í folk of skaut; &
þat vas ęnn folkvíg \hld\ fyrst í hęimi; &
brotinn vas borðvęggr \hld\ borgar ȧsa, &
knǫ́ttu vanir vígspǫ́ \hld\ vǫllu sporna.\eva

\bvb Weden flung [a spear], and into the opposing army did shoot; that was yet the first folk-war\footnoteB{\emph{folk} probably in its earlier sense, ‘troop’, though reading it as ‘people, folk’ is attractive, since it would give \emph{folkvíg} the meaning ‘ethnic conflict’.} in the \inx{home}. Broken was the board-wall\footnoteB{Wall made of planks.} of the fortification of the Ease; the Wanes did by \inx{wigh-spae} tread the fields.\footnoteB{The Wanes used magic spells to defeat the Ease.}\evb
\evg


\bvg {\small Building of the wall by the ettin.}
\bva\ledleftnote{\Regius\Hauksbok\GylfMS}Þȧ gingu ręgin ǫll \hld\ ȧ rǫkstóla, &
ginnhęilǫg goð, \hld\ ok umb þat gę́ttusk: &
hvęrr hęfði lopt alt \hld\ lę́vi blandit &
eða ę́tt jǫtuns \hld\ Óðs męy gefna.\eva

\bvb Then went the Powers all onto the rake-seats: the gin-holy gods, and from each other took counsel about that: Who had the air all with treason blended, or to the ettin’s \inx{aught} given \inx{Wode}’s maiden\footnoteB{That is, promised Frie to the ettin NAME. TODO: relate with what Snorri writes about the building of the wall.}?\evb
\evg


\bvg {\small Thunder slays him.}
\bva\ledleftnote{\Regius\Hauksbok\GylfMS}\edtext{Þȯrr ęinn \edtext{þar vá}{\lemma{þar vá “fought there”}\Afootnote{\emph{thus} \Hauksbok\Trajectinus\Upsaliensis; þar var “was there” \Regius; þat vann “performed it" \RegiusProse; þat ua “fought it” \Wormianus}} \hld\ þrunginn móði, &
hann sjaldan sitr, \hld\ es slíkt of fregn; &
\edtext{ȧ gingusk ęiðar, \hld\ orð ok sǿri, &
mǫ́l ǫll męginlig, \hld\ es ȧ meðal \edtext{fóru}{\Afootnote{voru “[between them] were” \Hauksbok\Trajectinus}}.}{\lemma{ȧ ... fóru.}\Afootnote{\emph{om.} \Wormianus}}}{\lemma{Þȯrr ... fóru.}\Bfootnote{In \GylfMS\ the two helmings (\emph{Þȯrr ... fregn;} \emph{ȧ ... fóru}) come in reverse order of \Regius\Hauksbok, which is here followed.}}\eva

\bvb Thunder alone fought there, pressed by wrath; he seldom sits, when of such\footnoteB{Oath-breaking, lies and deception.} he learns. Trampled were oaths, speeches and vows; the mighty treaties all, which between them had gone.\evb
\evg


\bvg {\small Homedall’s hearing hidden beneath Ugdrassle.}
\bva\ledleftnote{\Regius\Hauksbok}Vęit hǫ̇n Hęimdallar \hld\ hljóð of folgit &
und hęiðvǫnum \hld\ hęlgum baðmi; &
ȧ sér hǫ̇n ausask \hld\ aurgum forsi &
af veði Valfǫðrs. \hld\ Vituð ér ęnn eða hvat?\eva

\bvb — Knows she the hearing of Homedall hidden, ’neath a shady\footnoteB{\emph{hęiðvanr}, literally ‘clear-, bright-less’.}, hallowed beam\footnoteB{The tree must be Ugdrassle.}. On it she sees being poured a muddy torrent\footnoteB{Literally “on she sees being poured with a muddy torrent”, which should be the same mud as in v. 19. However, if ms. \emph{ȧ} is read as \emph{ǫ́} ‘river’, it would mean “A river she sees being fed by a muddy waterfall, from ...”}, from the pledge of the \inx{Father of the Slain}—know ye yet, or what?\footnoteB{“Do ye (Weden) know enough now, or what?”—repeated in 28, 33, 34, 38, 40, 47, 60, 61.}”\evb
\evg


\bvg {\small Weden sought out the wallow.—The following two verses are written together as one in \Regius.}
\bva\ledleftnote{\Regius}Ęin sat hǫ̇n úti, \hld\ þȧ’s hinn aldni kom &
yggjungr ȧsa \hld\ ok í augu lęit; &
hvęrs fregnið mik? \hld\ hví fręistið mín?\eva

\bvb — Lone sat she outside, when the old one came: the Terrifier of the Ease\footnoteB{Weden.}, and into [her] eyes looked. “Why inquirest thou me? Why temptest thou me?\footnoteB{The Wallow speaks.}\evb
\evg

\bvg
\bva\ledleftnote{\Regius\GylfMS}Alt vęit’k, Óðinn, \hld\ hvar auga falt &
\edtext{í hinum mę́ra}{\Afootnote{\emph{thus} \Wormianus; þitt (\emph{with points marking as error}) i enom męra \Regius í þęim hinum meira (“id.”) (\emph{norm.}) \Trajectinus\Upsaliensis; vr þeim envm mę́ra “out of the renowned” \RegiusProse}} \hld\ Mímis brunni; &
drekkr mjǫð Mímir \hld\ morgin hvęrjan &
af \edtext{veði}{\lemma{veði “pledge”}\Afootnote{veiþi “hunting”}} Valfǫðrs. \hld\ Vituð ér ęnn eða hvat?\eva

\bvb I know it all, Weden; where thine eye thou hidst: in the renowned \inx{Well of Mime}, [there] drinks Mime mead every morning, from the pledge of the \inx{Father of the Slain}—know ye yet, or what?”\evb
\evg


\bvg
\bva\ledleftnote{\Regius}Valði hęnni Hęrfǫðr \hld\ hringa ok męn; &
\edtext{féspjǫll spaklig}{\lemma{“wise wealth-spells”}\Bfootnote{By some authors (see Haukur 2020, p. 51 ff.) emended to \emph{fekk spjǫll spaklig} “he (= Weden) received wise tidings”}} \hld\ ok spáganda; &
sá hǫ̇n vítt ok umb vítt \hld\ of verǫld hvęrja.\eva

\bvb Host-father chose for her, rings and necklaces, wise wealth-spells, and spae-gands\footnoteB{The meaning of a \emph{gand} not fully clear. In this verse perhaps staffs used in ritual?}; saw she widely and widely about, o’er every world.\evb
\evg


\bvg {\small The Walkirries.}
\bva\ledleftnote{\Regius}Sá hǫ̇n valkyrjur \hld\ vítt of komnar, &
gǫrvar at ríða \hld\ til goðþjóðar. &
\edtext{Skuld hęlt skildi, \hld\ ęn Skǫgul ǫnnur, &
Gunnr, Hildr, Gǫndul \hld\ ok Gęirskǫgul; &
nú eru talðar \hld\ nǫnnur Hęrjans, &
gǫrvar at ríða \hld\ grund valkyrjur.}{\lemma{Skuld ... valkyrjur}\Bfootnote{These four lines, especially from the out-of-place ending (\emph{nú eru talðar}), seem to be a latter insert from a \emph{thule} counting the walkirries.}}\eva

\bvb Saw she walkirries, widely come, ready to ride to \inx{Godthede}. Shild held a shield, and Shagle another; Guth, Hild, Gandle, and Goreshagle; now are tallied the women of the Lord of Hosts: \inx{walkirries} ready to ride the ground.\evb
\evg


\bvg {\small The fate of Balder.}
\bva\ledleftnote{\Regius}Ek sá Baldri, \hld\ blóðgum tívi, &
Óðins barni, \hld\ ørlǫg folgin; &
stóð of vaxinn \hld\ vǫllum hę́ri &
mjór ok mjǫk fagr \hld\ mistiltęinn.\eva

\bvb — I saw Balder’s, the bloody tue’s, the child of Weden’s, \inx{orlay} sealed\footnoteB{Notably, \emph{fela} ‘hide, conceal’ is used to describe burial in mounds, as in \Ynglingatal\ 24, Öl 1 (900s): “hidden (\textbf{fulkin} \emph{folginn}) in this mound lies he whom the greatest deeds followed...”}; grown did stand, higher than the fields, slender and greatly fair, the mistletoe.\footnoteB{Told allusively in the following three verses is the death of Balder at the hands of his blind brother Hath. \Gylfaginning\ TODO}\evb
\evg


\bvg
\bva\ledleftnote{\Regius}Varð af męiði, \hld\ þęim’s mę́r sýndisk, &
harmflaug hę́ttlig, \hld\ Hǫðr nam skjóta. &
Baldrs bróðir vas \hld\ of borinn snimma, &
sá nam, Óðins sonr, \hld\ ęinnę́ttr vega;\eva

\bvb Became of that beam, which meager seemed, a baneful harm-flier; Hath began to shoot. Balder’s brother was born early; that one began, Weden’s son, one night old, to slay.\evb
\evg


\bvg
\bva\ledleftnote{\Regius}þó hann ę́va hęndr \hld\ né hǫfuð kęmbði, &
áðr ȧ bál of bar \hld\ Baldrs andskota. &
Ęn Frigg of grét \hld\ í Fęnsǫlum &
vǫ́ Valhallar. \hld\ Vituð ér ęnn eða hvat?\eva

\bvb Washed he never hands, nor head combed, before onto the pyre he did bear Balder’s opponent. But Frie did lament, in the Fenhalls, the woe of Walhall—know ye yet, or what?\evb
\evg


\bvg
\bva\ledleftnote{\Hauksbok}\edtext{Þȧ kná Váli \hld\ vígbǫnd snúa &
hęldr vǫ́ru harðgǫr \hld\ hǫpt ór þǫrmum.}{\lemma{Þȧ ... þǫrmum.}\Bfootnote{Only attested in \Hauksbok\, where it is combined with the last two lines of the next v. (\emph{þar ... hvat?}).}}\eva

\bvb Then did Woal the war-bonds turn; were they rather sturdy, fetters made of intestines.\evb
\evg


\bvg {\small The imprisoned Locke.}
\bva\ledleftnote{\Regius\Hauksbok}\edtext{Hapt sá hǫ̇n liggja \hld\ und Hveralundi &
lę́gjarnlíki \hld\ Loka ȧþękkjan;}{\lemma{Hapt ... ȧþękkjan}\Afootnote{\emph{om.} \Hauksbok}} &
þar sitr Sigyn \hld\ þęygi of sínum &
veri velglýjuð. \hld\ Vitud ér ęnn eða hvat?\eva

\bvb A captive she saw lying, ’neath Wharlund: the guileful form of similar Locke. There sits Sighyn, not at all cheerful, above her husband;\footnoteB{See \FraLoka.}—know ye yet, or what?\evb
\evg


\bvg
\bva\ledleftnote{\Regius}Ǫ́ fęllr austan \hld\ of ęitrdala &
sǫxum ok sverðum, \hld\ Slíðr hęitir sú.\eva

\bvb A river falls from the east, above the venom-dales, with saxes and swords; Slide is that one called.\evb
\evg


\bvg {\small Two halls.}
\bva\ledleftnote{\Regius}Stóð fyr norðan \hld\ ȧ Niðavǫllum &
salr ór golli \hld\ Sindra ę́ttar, &
ęn annarr stóð \hld\ ȧ Ȯkólni, &
bjórsalr jǫtuns, \hld\ ęn sá Brimir hęitir.\eva

\bvb Stood to the north, on the Nithewolds, a hall out of gold, of the \inx{aught} of Sinder; but another one stood, on Uncoalner, the beer-hall of an ettin, and Brimmer ’tis called.\evb
\evg


\bvg {\small The worst hall.}
\bva\ledleftnote{\Regius\Hauksbok\GylfMS}Sal sá hǫ̇n standa \hld\ sólu fjarri &
Nástrǫndu ȧ, \hld\ norðr horfa dyrr; &
falla ęitrdropar \hld\ inn umb ljóra, &
sá ’s undinn salr \hld\ orma hryggjum.\eva

\bvb A hall she saw standing, far from the sun, on Nawstrand, north face the doors; fall venom-drops in through the smoke-vent, that hall is wound by the spines of snakes.\evb
\evg


\bvg
\bva\ledleftnote{\Regius\Hauksbok\GylfMS}\edtext{Sá hǫ̇n}{\lemma{Sá hǫ̇n “she saw”}\Afootnote{\emph{thus} \Regius; ser hon “she sees” \Hauksbok; skulu “shall” \GylfMS}} þar vaða \hld\ þunga strauma &
męnn męinsvara \hld\ ok morðvarga &
ok þann’s annars glępr \hld\ ęyrarúnu. &
Þar \edtext{saug}{\lemma{saug “sucked”}\Afootnote{\emph{thus} \Hauksbok; súg (\emph{corrupt form of} saug) \Regius; kvęlr “torments”}} Níðhǫggr \hld\ nái framgingna; &
slęit vargr vera. \hld\ Vituð ér ęnn eða hvat?\eva

\bvb There she saw wade, through heavy streams, oath-breaking men and murderwargs, and the one who confounds another’s understanding\footnoteB{Literally “who confounds another’s ear-rune;” false counsellors.}. There sucked Nithehew from corpses passed-on; the warg tore men asunder—know ye yet, or what?\evb
\evg


\bvg {\small The hag nourishes the destroyers in Ironwood.}
\bva\ledleftnote{\Regius\Hauksbok\GylfMS}Austr \edtext{býr}{\Afootnote{\emph{Thus} \Hauksbok\GylfMS\ sat “stayed [the old]” \Regius}} hin \edtext{aldna}{\Afootnote{arma “the wretched woman” \Upsaliensis}} \hld\ í \edtext{Járnviði}{\Afootnote{jarnuidiom “[in] Ironwoods” \Trajectinus}} &
ok \edtext{fǿðir}{\Afootnote{\emph{Thus} \Hauksbok\GylfMS; fǿddi “nourished” \Regius}} þar \hld\ Fęnris kindir; &
verðr \edtext{af}{\Afootnote{ór “out of [them] \Trajectinus\RegiusProse}} þęim ǫllum \hld\ ęinna nøkkurr &
tungls \edtext{tjúgari}{\lemma{tjúgari}\Afootnote{tuigan \Trajectinus\ \emph{wo. doubt corrupt}; tregari “griever [of the moon]” \Upsaliensis\ — As the young agentive suffix \emph{-ari} is found only here in the poem, it is possible that this word is corrupt. In that case, it must have occurred quite early in the transmission, as reflexes of \emph{*tiugari} are found in all surviving mss.}} \hld\ í trolls hami.\eva

\bvb In the east dwells the old woman, in \inx{Ironwood}, and nourishes there the kindreds of \inx{Fenner}; from them all becomes one most particular: a seizer of the moon, in the \inx{hame} of a troll.\footnoteB{The old hag raises the offspring of the wolf Fenner, of which one will swallow the moon (and according to \Gylfaginning\ TODO the other the sun). See note to the next v.}\evb
\evg


\bvg
\bva\ledleftnote{\Regius\Hauksbok\GylfMS}Fyllisk fjǫrvi \hld\ fęigra manna, &
rýðr ragna sjǫt \hld\ rauðum dręyra, &
svǫrt verða sólskin \hld\ umb sumur ęptir, &
veðr ǫll válynd. \hld\ Vituð ér ęnn eða hvat?\eva

\bvb He\footnoteB{The wolf.} fills himself with the life of \inx{fey} men; he reddens the abode of the \inx{Powers} with red gore. Black becomes the sunshine about the summers afterwards\footnoteB{After the sun is swallowed. But since the wallow does not tell us that this is a different wolf (it seems rather it be one and the same), it may reflect an earlier version of the myth, where one son of Fenner swallowed both the sun and moon. Yet, according to \Vafthrudnismal\ 36-37 it is Fenner himself who will swallow the sun (and thus likely the moon as well,) unless it there be taken as a general \inx{hote} for ‘wolf’ (which undoubtedly is its original meaning). TODO}; the storms all woeful—know ye yet, or what?\evb
\evg


\bvg {\small Edgethew struck harp; a fair-red cock crowed.}
\bva\ledleftnote{\Regius\Hauksbok}Sat þar ȧ haugi \hld\ ok sló hǫrpu &
gýgjar hirðir, \hld\ glaðr Ęggþér; &
gól of hǫ̇num \hld\ í Gaglviði &
fagrrauðr hani, \hld\ sá’s Fjalarr hęitir.\eva

\bvb Sat there on the \inx{high} and struck the harp, the troll-woman’s herdsman, glad \inx{Edgethew}. Above him crowed, in Galewood\footnoteB{\emph{gagl} ‘wild goose’, maybe here referring to carrion-eating ravens? Possibly the same as Ironwood.}, a fair-red cock, that one who Feller is called.\evb
\evg


\bvg {\small A golden cock crowed in Osyard; a soot-red in Hell.}
\bva\ledleftnote{\Regius\Hauksbok}Gól of ǫ̇sum \hld\ Gollinkambi, &
sá vękr hǫlða \hld\ at Hęrjafǫðrs, &
ęn annarr gęlr \hld\ fyr jǫrð neðan &
sótrauðr hani \hld\ at sǫlum Hęljar.\eva

\bvb Above the Ease crowed Goldencombe: he wakes men at the Father of Hosts’s [estate]; but another one crows beneath the earth: a soot-red cock, at the halls of Hell.\evb
\evg


\bvg
\bva\ledleftnote{\Regius\Hauksbok}Gęyr Garmr mjǫk \hld\ fyr Gnipahęlli, &
fęstr mun slitna, \hld\ ęn Freki rinna; &
fjǫlð vęit hǫ̇n frǿða, \hld\ framm sé’k lęngra &
of ragna rǫk, \hld\ rǫmm sigtíva.\eva

\bvb Barks Garm loudly before the Gnip-caverns; the rope will tear, and Freck run. Much she knows of learning, forth I see yet further; about the mighty Rakes of the Powers, of the victory-tues.\evb
\evg


\bvg {\small Degeneration of man.}
\bva\ledleftnote{\Regius\Hauksbok\GylfMS}Brǿðr munu bęrjask \hld\ ok at bǫnum verðask, &
munu \edtext{systrungar}{\lemma{systrungar “sister’s sons”}\Afootnote{stystrungar (\emph{wo. doubt corrupt}) \Trajectinus}} \hld\ sifjum spilla; &
hart ’s \edtext{í hęimi}{\lemma{í hęimi “in the home”}\Afootnote{\emph{thus} \Regius\Hauksbok\Upsaliensis; með hǫlðum “among men” \RegiusProse\Trajectinus\Wormianus}}, \hld\ hórdȯmr mikill, &
skęggǫld, skalmǫld, \hld\ \edtext{skildir}{\lemma{skildir “shields”}\Afootnote{\emph{add.} ró “are” \Regius}} \edtext{klofnir}{\lemma{klofnir “cloven”}\Afootnote{klofna “become cloven" \Upsaliensis}}, &
\edtext{vindǫld}{\lemma{vindǫld “wind-eld”}\Bfootnote{In \Hauksbok\ capitalized, marking as new verse.}}, vargǫld, \hld\ \edtext{áðr}{\lemma{áðr “before”}\Afootnote{unz (\emph{norm.}) “until” \Upsaliensis}} verǫld \edtext{stęypisk}{\lemma{stęypisk “tumbles down”}\Bfootnote{After this word \Hauksbok\ has a line not found in \Regius\ or \GylfMS: \emph{grundir gjalla / gífr fljúgandi} (\emph{norm.}) “foundations shrill, fiends flying”}} &
\edtext{mun \edtext{ęngi}{\Afootnote{enn (\emph{wo. doubt corrupt}) \Upsaliensis}} maðr \hld\ ǫðrum þyrma.}{\lemma{mun ... þyrma “before ... spare.”}\Bfootnote{\emph{om.} \RegiusProse\Trajectinus\Wormianus}}\eva

\bvb Brothers will fight, and become each other’s slayers; sister’s sons will spill their kinship.\footnoteB{Whether through incest or treachery. TODO: literary evidence of the phrase \emph{spilla sifjum}.} ’Tis hard in the Home, whoredom great: axe-eld, sword-eld—shields are rent—wind-eld, warg-eld; before the world\footnoteB{\emph{ver-ǫld} ‘world’ is literally ‘man-eld’, ‘the eld of man’.} tumbles down, no man will another spare.\evb
\evg


\bvg {\small Prophesied events come to pass.}
\bva\ledleftnote{\Regius\Hauksbok\GylfMS}\edtext{Lęika Míms synir, \hld\ ęn mjǫtuðr kyndisk &
at hinu galla \hld\ Gjallarhorni; &
hǫ́tt blę́ss Hęimdallr, \hld\ horn ’s ȧ lopti; &
\edtext{mę́lir}{\lemma{mę́lir “speaks”}\Afootnote{mey \RegiusProse; nie \Trajectinus\ \emph{both wo. doubt corrupt}}} Óðinn \hld\ við Míms hǫfuð.}{\lemma{Lęika ... hǫfuð.}\Bfootnote{In \GylfMS\ ll. 1–2 (\emph{Lęika ... Gjallarhorni;} “Play ... Horn of Yell.”) are missing, and ll. 3–4 (\emph{hǫ́tt ... hǫfuð.} “High ... head [of Mime.]”) are instead paired with the first two lines of the next v. (Skęlfr ... losnar;)}}\eva

\bvb Play the sons of Mime, and the Metted is kindled, at [the sounding of] the shrill Horn of Yell. Loudly blows Homedall; the horn is aloft; Weden speaks with the head of Mime.\evb
\evg


\bvg
\bva\ledleftnote{\Regius\Hauksbok\GylfMS}\edtext{Skęlfr Yggdrasils \hld\ askr standandi, &
ymr it aldna tré, \hld\ ęn jǫtunn losnar;}{\lemma{Skęlfr ... losnar “Quakes ... loosens.”}\Bfootnote{thus \Hauksbok\GylfMS; in \Regius\ the two lines are reversed.}} &
\edtext{hrę́ðask allir \hld\ ȧ hęlvegum &
áðr Surtar þann \hld\ sefi of glęypir.}{\lemma{hrę́ðask ... glęypir “[All] are frightened ... devour [it.]”}\Bfootnote{only in \Hauksbok}} \eva

\bvb Quakes the ash of Ugdrassle, standing; groans the old tree, and the ettin loosens. All are frightened on the Hell-ways, before Surt’s kinsman does devour it.\evb
\evg


\bvg
\bva\ledleftnote{\Regius\Hauksbok\GylfMS}Hvat ’s með ǫ̇sum? \hld\ hvat ’s með \edtext{ǫlfum}{\lemma{ǫlfum “Elves”}\Afootnote{asynivm “Osennies” \Upsaliensis}}? &
\edtext{gnýr allr Jǫtunhęimr, \hld\ ę̇sir ’ro ȧ þingi,}{\lemma{gnýr ... þingi}\Afootnote{\emph{om.} \Upsaliensis}} &
stynja dvergar \hld\ fyr \edtext{stęindurum}{\Afootnote{steins \Upsaliensis — -dyrum \Hauksbok\Wormianus\Upsaliensis}} &
\edtext{\edtext{vęggbergs}{\lemma{vęggbergs “wedge-rock”}\Afootnote{vegbergs “way-rock” \Hauksbok\Trajectinus\Wormianus}} vísir}{\Afootnote{\emph{om.} \Upsaliensis}} — \hld\ vituð ér ęnn eða hvat?\eva

\bvb — What is with the Ease? What is with the Elves? Roars all Ettinham, the Ease are at the Thing. Dwarfs groan before gates of stone, the princes of the wedge-rock—know ye yet, or what?\evb
\evg


\bvg
\bva\ledleftnote{\Regius\Hauksbok}Gęyr nú Garmr mjǫk \hld\ fyr Gnipahęlli, &
fęstr mun slitna, \hld\ ęn Freki rinna; &
fjǫlð vęit hǫ̇n frǿða, \hld\ framm sé’k lęngra &
of ragna rǫk, \hld\ rǫmm sigtíva.\eva

\bvb Barks now Garm loudly before the Gnip-caverns; the rope will tear, and Freck run. Much she knows of learning, forth I see yet further; about the mighty Rakes of the Powers, of the victory-tues.\evb
\evg


\bvg {\small The enemies of the gods assemble.}
\bva\ledleftnote{\Regius\Hauksbok\RegiusProse\Trajectinus\Wormianus}Hrymr ękr austan, \hld\ hęfsk lind fyrir, &
snýsk Jǫrmungandr \hld\ í jǫtunmóði; &
ormr knýr unnir, \hld\ \edtext{ęn ari hlakkar}{\lemma{ęn ari hlakkar “but the eagle screams”}\Afootnote{ǫrn mun hlakka “the eagle will scream” \RegiusProse\Trajectinus}}, &
slítr nái neffǫlr; \hld\ Naglfar losnar.\eva

\bvb Rim drives from the east, holding his shield before himself; Ermingand writhes about in ettin’s wrath. The worm propels the waves, but the eagle screams: the pale-beak tears corpses; Nailfare loosens.\evb
\evg


\bvg
\bva\ledleftnote{\Regius\Hauksbok\RegiusProse\Trajectinus\Wormianus}Kjóll fęrr austan \hld\ koma munu Múspells &
of lǫg lýðir, \hld\ ęn Loki stýrir; &
fara fíflmęgir \hld\ með Freka allir, &
þęim es bróðir \hld\ Býlęists í fǫr.\eva

\bvb A ship travels from the east—come will Muspell’s subjects by sea—but Locke steers it. Travel the warlocks all with Freck; with them comes the brother of Bylest \ken{Locke}[1] along.\evb
\evg


\bvg {\small Surt comes; the final battle begins.}
\bva\ledleftnote{\Regius\Hauksbok\GylfMS}\edtext{Surtr}{\Afootnote{Svartr \Upsaliensis}} fęrr sunnan \hld\ með sviga lę́vi, &
skínn af sverði \hld\ sól valtíva; &
grjótbjǫrg gnata, \hld\ ęn \edtext{gífr rata}{\Afootnote{guðar hrata “[but] the gods stagger” (\emph{wo. doubt corrupt, young masc. pl. is proof enough.}) \Upsaliensis}}, &
troða halir hęlveg, \hld\ ęn himinn klofnar.\eva

\bvb Surt comes from the south, with the switch-bane\footnoteB{According to \CV\ ‘fire’.}; from the sword shines the sun of the slain-tues; boulders clash, but the fiends reel; men march on the \inx{Hell-ways}, but heaven is sundered.\evb
\evg


\bvg {\small Weden falls to the Wolf and Free to Surt.}
\bva\ledleftnote{\Regius\Hauksbok\RegiusProse\Trajectinus\Wormianus}Þȧ kømr Hlínar \hld\ harmr annarr framm, &
es Óðinn fęrr \hld\ við ulf vega, &
ęn bani Bęlja \hld\ bjartr at Surti; &
þȧ mun Friggjar \hld\ falla \edtext{angan}{\Afootnote{angantyr \Regius}}.\eva

\bvb Then comes \inx{Line}’s second sorrow to pass, as Weden goes to strike against the wolf; but the bane of \inx{Bellow}\footnoteB{\inx{Free}.}, bright, [goes] against Surt; then will Frie’s beloved\footnoteB{Weden, her husband.} fall.\evb
\evg


\bvg {\small Wider avenges Weden and slays the Wolf.}
\bva\ledleftnote{\Regius\RegiusProse\Trajectinus\Wormianus}\edtext{Þȧ kømr hinn mikli \hld\ mǫgr Sigfǫður}{\lemma{Þȧ kømr ... Sigfǫður “Then ... Sighfather”}\Afootnote{Gęngr Óðins sonr / við ulf vega “Goes Weden’s son against the wolf to fight” \GylfMS}}, &
Víðarr \edtext{vega}{\Afootnote{of veg \GylfMS}} \hld\ at valdýri; &
lę́tr hann męgi Hveðrungs \hld\ mund of standa &
hjǫr til hjarta; \hld\ þȧ ’s hefnt fǫður.\eva

\bvb Then comes the great lad of \inx{Sighfather}, Wider, to strike at the murderous beast; he lets his hand plunge the sword into the heart of \inx{Whethring}’s lad\footnoteB{The son of Locke; the wolf.}; then is the father avenged.\evb
\evg


\bvg
\bva\ledleftnote{\Hauksbok}\edtext{Gínn lopt yfir \hld\ lindi jarðar, &
gapa ýgs kjaptar \hld\ orms í hę́ðum; &
mun Óðins son \hld\ \edtext{ęitri}{\lemma{ęitri “venom”}\Afootnote{ormi “the worm” \Hauksbok, \emph{cf. the prose of} \Gylfaginning: \emph{“Thunder bears the bane-word from the Middenyardsworm and thence strides away nine paces. Then he falls dead to the earth by the \textbf{venom} \emph{(ęitri)} which the Worm blows on him.”}}} mǿta &
vargs at \edtext{dauða}{\Afootnote{da... \Hauksbok}} \hld\ Víðars niðja.}{\lemma{Gínn ... niðja.}\Bfootnote{Reading taken from Jón Helgason 1971, pp. 13, 44ff.}}\eva

\bvb Yawns over the air the girdle of the earth \ken{the Middenyardsworm}[1]; gape the jaws of the fierce worm in the heights. The venom of the beast will meet Weden’s son \ken{Thunder}[1], after the deaths of Wider’s kinsmen \ken{the Ease}[1].\evb
\evg


\bvg {\small Thunder and the Worm kill each other.}
\bva\ledleftnote{\Regius\Hauksbok\RegiusProse\Trajectinus\Wormianus}\edtext{Þȧ kømr}{\Afootnote{Gęngr \GylfMS}} hinn mę́ri \hld\ mǫgr Hlǫðynjar &
\edtext{gęngr Óðins sonr \hld\ við orm vega.}{\lemma{gęngr ... vega}\Afootnote{\emph{Only in} \Regius}} &
\edtext{Drepr af móði \hld\ Miðgarðs véurr; &
munu halir allir \hld\ hęimstǫð ryðja; &
gęngr fet níu \hld\ Fjǫrgynjar burr &
nęppr frȧ naðri, \hld\ níðs ȯkvíðnum.}{\lemma{Drepr ... ȯkviðnum}\Afootnote{neppr af naðri / niðs ȯkvíðnum / munu halir allir / hęimstǫð ryðja, / es af móði drepr / Miðgarðs véurr “[Goes the renowned lad of Lathyn,] pained, away from the loathsome adder. All men will empty their homesteads, when Middenyard’s wigh-ward strikes out of wrath.” \GylfMS}}\eva

\bvb Then comes the renowned lad of Lathyn: the son of Weden goes the \inx{worm} to meet. Middenyard’s wigh-ward strikes out of wrath; all men will their homesteads empty.\footnoteB{It seems likely that the order found in \Gylfaginning\ is original. After Thunder dies, farming becomes impossible, and thus men must leave their homes.} The son of Firgyn goes nine paces, pained, away from the loathsome adder.\footnoteB{Thunder, mortally wounded, struggles nine steps away from the Worm before he falls. See note to previous verse.}\evb
\evg


\bvg {\small Culmination.}
\bva\ledleftnote{\Regius\Hauksbok\GylfMS}Sól tér sortna, \hld\ \edtext{søkkr fold í mar}{\lemma{søkkr ... mar}\Bfootnote{This line is very similar to a line of v. 24 in Arnthur ‘earl-scold’ Thurthson’s Drape of Thurfinn (\Skp: Arn \emph{Þorfdr} 24\textsuperscript{II}): \emph{søkkr fold í mar døkkvan} “sinks the fold into the dark sea”. For this reason, \emph{søkkr} ‘sinks’ \RegiusProse\Trajectinus\Wormianus has been chosen over \emph{sígr} ‘descends’ \Regius\Hauksbok\Upsaliensis.}}, &
hverfa af himni \hld\ hęiðar stjǫrnur; &
gęisar ęimi \hld\ við aldrnara; &
lęikr hǫ́r hiti \hld\ við himin sjalfan.\eva

\bvb The sun does blacken, sinks the fold into the sea; disappear off heaven the clear stars. Rages smoke from the nourisher of life\footnoteB{Fire.}; licks the high heat heaven itself.\evb
\evg


\bvg
\bva\ledleftnote{\Regius\Hauksbok}Gęyr nú Garmr mjǫk \hld\ fyr Gnipahęlli, &
fęstr mun slitna, \hld\ ęn Freki rinna; &
fjǫlð vęit hǫ̇n frǿða, \hld\ framm sé’k lęngra &
of ragna rǫk, \hld\ rǫmm sigtíva.\eva

\bvb Barks now Garm loudly before the Gnip-caverns; the rope will tear, and Freck run. Much she knows of learning, forth I see yet further; about the mighty Rakes of the Powers, of the victory-tues.\evb
\evg


\bvg {\small The world is reborn.}
\bva\ledleftnote{\Regius\Hauksbok}Sér hǫ̇n upp koma \hld\ ǫðru sinni &
jǫrð ór ę́gi \hld\ iðjagrø̇na; &
falla forsar, \hld\ flýgr ǫrn yfir, &
sá’s ȧ fjalli \hld\ fiska vęiðir.\eva

\bvb Sees she come up, a second time: the earth out of the sea, ever green anew. Torrents fall; flies an eagle above, the one who on the fells fish does catch.\evb
\evg


\bvg
\bva\ledleftnote{\Regius\Hauksbok}Finnask ę̇sir \hld\ ȧ Iðavęlli &
ok umb moldþinur \hld\ mǫ́tkan dø̇ma, &
ok minnask þar \hld\ ȧ męgindȯma &
ok ȧ Fimbultýs \hld\ fornar rúnar.\eva

\bvb The Ease find each other on the Idewolds, and about the mighty earth-strip\footnoteB{The Middenyardsworm.} converse, and remember there mighty judgements, and Fimbletue’s <= Weden’s> ancient runes.\evb
\evg

\bvg {\small A new golden age.}
\bva\ledleftnote{\Regius\Hauksbok}Þar munu ęptir \hld\ undrsamligar &
gollnar tǫflur \hld\ í grasi finnask, &
þę́r’s í árdaga \hld\ áttar hǫfðu.\eva

\bvb There will afterwards wondrous golden Tavel-bricks in the grass be found: those which in days of yore they had owned.\footnoteB{Cf. v. 9. The rediscovering of the golden game pieces symbolizes a new golden age.}\evb
\evg


\bvg
\bva\ledleftnote{\Regius\Hauksbok}Munu ȯsánir \hld\ akrar vaxa; &
bǫls mun alls batna \hld\ mun Baldr koma; &
búa Hǫðr ok Baldr \hld\ Hropts sigtoptir &
(vęl valtívar, \hld\ Vituð ér ęnn eða hvat?)\eva

\bvb Unsown will fields grow: evil will all be bettered: Balder will come. Bedwell Hath and Balder the victory-plots of Roft <= Weden>, happily, the slain Tues—know ye yet, or what?\evb
\evg


\bvg
\bva\ledleftnote{\Regius\Hauksbok}Þȧ kná Hø̇nir \hld\ hlautvið kjósa &
ok burir byggva \hld\ brǿðra Tvęggja &
vindhęim víðan. \hld\ Vituð ér ęnn eða hvat?\eva

\bvb Then does Heen choose the \inx{leat}-wood\footnoteB{Restore the bloot and practice divination.}, and the sons of the brothers of Tway <= Weden> settle the wide wind-home\ken{Sky.}\footnoteB{Will and Wigh? Who their sons are is unknown.}—know ye yet, or what?\evb
\evg


\bvg
\bva\ledleftnote{\Regius\Hauksbok\GylfMS}Sal \edtext{sér hǫ̇n}{\lemma{sér hǫ̇n “she sees”}\Afootnote{vęit’k (\emph{norm.}) “I know” \GylfMS}} standa \hld\ sólu fęgra, &
golli \edtext{þakðan}{\lemma{þakðan “thatched”}\Afootnote{betra “better [than gold]” \RegiusProse\Trajectinus}}, \hld\ ȧ \edtext{Gimléi}{\Afootnote{\emph{metr. emend.} Gimlé (\emph{norm.}) \Regius\Hauksbok\GylfMS}}; &
\edtext{þar}{\lemma{þar “there”}\Afootnote{þann “it [shall dutiful men bedwell]” \Trajectinus\Wormianus}} skulu dyggvar \hld\ dróttir byggva &
ok umb aldrdaga \hld\ ynðis njóta.\eva

\bvb A hall she sees standing, fairer than the sun: thatched with gold, on Gemlee; there dutiful men shall dwell, and in their life-days delights enjoy.\evb
\evg


\bvg {\small The dragon still lives; the wallow descends.}
\bva\ledleftnote{\Regius\Hauksbok}Þar kømr hinn dimmi \hld\ dręki fljúgandi, &
naðr frȧnn neðan \hld\ frȧ Niðafjǫllum; &
berr sér í fjǫðrum \hld\ —flýgr vǫll yfir— &
Níðhǫggr nái; \hld\ nú mun hǫ̇n søkkvask.\eva

\bvb — Then comes the shadowy dragon flying; the gleaming adder down below from the \inx{Nithfells}. Nithehew bears in his feathers—flying over the field—corpses.” — Now she will sink!\footnoteB{The wallow, referring to herself in third person, descends back down into her grave, whence Weden woke her.}\evb
\evg


\bvg {\small Spurious verse from \Hauksbok.}
\bva[X]\ledleftnote{\Hauksbok}\edtext{Þȧ kømr hinn ríki \hld\ at ręgindȯmi &
ǫflugr ofan \hld\ sá’s ǫllu rę́ðr.}{\lemma{Þȧ ... rę́ðr.}\Bfootnote{This verse is found only in \Hauksbok, in between the last two vv. It is without doubt a late, Christian addition.}}\eva

\bvb[X] — Then comes the mighty one, for the great judgement; strong from above, the one who over all things wields.\evb
\evg
% Weden
	\bookStart{The Speeches of Webthrithner}[Vafþrúðnismǫ́l]

\bvg {\small [Weden quoth:]}
\bva Ráð mér nú \alst{F}rigg \hld\ alls mik \alst{f}ara tíðir &
\ind at \alst{v}itja \alst{V}afþrúðnis; &
\alst{f}orvitni mikla \hld\ kveð’k mér á \alst{f}ornum stǫfum &
\ind við þann hinn \alst{a}lsvinna \alst{jǫ}tun.\eva

\bvb \inx[P]{Weden} quoth: “Counsel me now, \inx[P]{Frie}, as I desire to travel to visit \inx[P]{Webthrithner}; greatly curious am I of ancient staves\footnoteB{Ancient (pieces of) lore; cf. v. 55.} by that all-wise \inx[G]{Ettins}[ettin].”\evb
\evg


\bvg {\small [Frie quoth:]}
\bva \alst{H}ęima lętja \hld\ mynda’k \alst{H}ęrjafǫðr &
\ind í \alst{g}ǫrðum \alst{g}oða; &
\alst{ę}ngi \alst{jǫ}tun \hld\ hugða’k \alst{ja}fnramman &
\ind sęm \alst{V}afþrúðni \alst{v}esa.\eva

\bvb “I would encourage \inx[P]{Harryfather} \name{= Weden} to [stay at] home in the yards of the gods, for no ettin I thought to be even-strong with Webthrithner.”\evb
\evg


\bvg {\small [Weden quoth:]}
\bva Fjǫlð ek fór, \hld\ fjǫlð fręistaða’k, &
\ind fjǫlð ek ręynda ręgin; &
hitt vil’k vita, \hld\ hvé Vafþrúðnis &
\ind salakynni séi.\eva

\bvb “Much I travelled, much I tempted, much I tested the \inx[G]{Reins}. This I wish to know, how the condition of the halls of Webthrithner might be?”\evb
\evg


\bvg {\small [Frie quoth:]}
\bva Hęill þú farir, \hld\ hęill þú aptr komir, &
\ind hęill á sinnum séir; &
ǿði þér dugi \hld\ hvar’s skalt, Aldafǫðr, &
\ind orðum mę́la jǫtun.\eva

\bvb “Whole travel thou, whole come thou back, whole be thou on thy paths! Thy wisdom suffice thee, where thou shalt, \inx[P]{Eldfather} \name{= Weden}, words with the ettin exchange.”\evb
\evg


\bvg
\bva Fór þá Óðinn \hld\ at fręista orðspęki &
\ind þess hins alsvinna jǫtuns; &
at hǫllu hann kom, \hld\ \edtext{es}{\Afootnote{ok \Regius}} átti \edtext{Hymis}{\Afootnote{\emph{metr. emend. after} \textcite{FinnurEdda}; Íms \Regius}} faðir; &
\ind inn gekk Yggr þegar.\eva

\bvb Then went Weden, to try the word-wisdom of that all-wise ettin. To a hall he came, which the father of \inx[P]{Hymer} \ken{= Webthrithner} owned; shortly \inx[P]{Ug} \name{= Weden} walked in.\evb
\evg


\bvg {\small [Weden quoth:]}
\bva Hęill þú nú, Vafþrúðnir, \hld\ nú em’k í hǫll kominn &
\ind á þik sjalfan séa; &
hitt vil’k fyrst vita, \hld\ ef fróðr séir &
\ind eða alsviðr, jǫtunn.\eva

\bvb “Hail thee now, Webthrithner; now am I come into the hall, to gaze upon thy self! This I wish first to know, if learned thou be, or all-wise, ettin.”\evb
\evg


\bvg {\small [Webthrithner quoth:]}
\bva Hvat’s þat manna, \hld\ es í mínum sal &
\ind verpumk orði á? &
út þú né kømr \hld\ órum hǫllum frá. &
\ind nema þú inn snotrari séir.\eva

\bvb “What sort of man is that, who in my hall throws words at me? Out comest thou not from our halls, unless thou be the cleverer.”\evb
\evg


\bvg {\small [Weden quoth:]}
\bva \edtext{Gagnráðr}{R’s \emph{Gagnráðr} ‘Gainred,’ is attested as Gangráðr ‘Journey-adviser’ in \emph{Gylf}.} hęiti’k, \hld\ nú em’k af gǫngu kominn, &
\ind þyrstr til þinna sala; &
laðar þurfi \hld\ hęf’k lęngi farit &
\ind ok þinna andfanga, jǫtunn.\eva

\bvb “\inx[P]{Gainred} I am called, now am I come from walking, thirsty, to thy halls. In need of reception I have travelled for long, and of thy hospitality, ettin!”\evb
\evg


\bvg {\small [Webthrithner quoth:]}
\bva Hví þú þá, Gagnráðr, \hld\ mę́lisk af golfi fyrir? &
\ind far þú í sess í sal; &
þá skal fręista, \hld\ hvárr flęira viti, &
\ind gęstr eða hinn gamli þulr.\eva

\bvb “Why then, Gainred, speakest thou from the floor before me? Take a seat in the hall! Then it shall be tried, which of the two might know more; the guest, or the old \inx[C]{thyle}.”\evb
\evg


\bvg {\small [Gainred quoth:]}
\bva Óauðigr maðr, \hld\ es til auðigs kømr, &
\ind mę́li þarft eða þęgi; &
ofrmę́lgi mikil \hld\ hygg’k at illa geti &
\ind hvęim’s við kaldrifjaðan kømr.\eva

\bvb “An unwealthy man, who to a wealthy one comes, ought to speak the needful or be silent.\footnoteB{Last line identical to \Havamal\ 18.} Great over-speaking, I judge, will bring evil for him who to a cold-ribbed\footnoteB{i.e. ‘cold-hearted, cunning’.} man comes.”\evb
\evg


\bvg {\small [Webthrithner quoth:]}
\bva Sęg mér, Gagnráðr, \hld\ alls á golfi vill &
\ind þíns of fręista frama, &
hvé hęstr hęitir, \hld\ sá’s hvęrjan dręgr &
\ind dag of dróttmǫgu.\eva

\bvb “Say to me, Gainred, since on the floor I will to try thy fame: What is the horse called, which pulls each \emph{day} above the sons of the retinue \ken{Men}?”\evb
\evg


\bvg {\small [Gainred quoth:]}
\bva Skinfaxi hęitir, \hld\ es hinn skíra dręgr &
\ind dag of dróttmǫgu; &
hęsta baztr \hld\ þykkir með Hręiðgotum; &
\ind ęy lýsir mǫn af mari.\eva

\bvb “\inx[P]{Shinefax} is called he who pulls the bright day above the sons of the retinue. The best of horses he seems among the \inx[G]{Reth-Gots}; the mane of that stallion ever shines.”\evb
\evg


\bvg {\small [Webthrithner quoth:]}
\bva Sęg þat, Gagnráðr, \hld\ alls á golfi vill &
\ind þíns of fręista frama, &
hvé jór hęitir, \hld\ sá’s austan dręgr &
\ind nótt of nýt ręgin.\eva

\bvb “Say this, Gainred, since on the floor I will to try thy fame: What is the steed called, which from the east pulls night above the useful \inx[G]{Reins}?”\evb
\evg


\bvg {\small [Gainred quoth:]}
\bva Hrímfaxi hęitir, \hld\ es hvęrja dręgr &
\ind nótt of nýt ręgin; &
méldropa fęllir \hld\ morgin hvęrjan; &
\ind þaðan kømr dǫgg of dala.\eva

\bvb “\inx[P]{Rimefax}\ he is called, who pulls each night above the useful Reins. Every morning he lets foam fall from his bit\footnoteB{lit. “he fells bit-drops”.}; thence comes dew in the dales.\footnoteB{For another explanation of the origin of dew, see}”\evb
\evg


\bvg {\small [Webthrithner quoth:]}
\bva Sęg þat, Gagnráðr, \hld\ alls á golfi vill &
\ind þíns of fręista frama, &
hvé ǫ́ hęitir, \hld\ sú’s dęilir með jǫtna sonum &
\ind grund ok með goðum.\eva

\bvb “Say this, Gainred, since on the floor I will to try thy fame; How the river is called, which divides the ground between the sons of ettins and the gods?”\evb
\evg


\bvg {\small [Gainred quoth:]}
\bva Ífing hęitir ǫ́, \hld\ es dęilir með jǫtna sonum &
\ind grund ok með goðum; &
opin rinna \hld\ hón skal um aldrdaga; &
\ind verðr-at íss á ǫ́.\eva

\bvb “\inx[L]{Iving} the river is called, which divides the ground between the sons of ettins and the gods. Throughout [her] life-days she shall flow open; ice forms not on the river.”\evb
\evg


\bvg {\small [Webthrithner quoth:]}
\bva Sęg þat, Gagnráðr, \hld\ alls á golfi vill &
\ind þíns of fręista frama, &
hvé vǫllr hęitir, \hld\ es finnask vigi at &
\ind Surtr ok hin svǫ́su goð.\eva

\bvb “Say this, Gainred, since on the floor I will to try thy fame: How that plain is called, where \inx[P]{Surt} and the excellent gods find each other at war?”\evb
\evg


\bvg {\small [Gainred quoth:]}
\bva Vígríðr hęitir vǫllr, \hld\ es finnask vígi at &
\ind Surtr ok hin svǫ́su goð; &
hundrað rasta \hld\ hann’s á hvęrjan veg; &
\ind sá ’s þęim vǫllr vitaðr.\eva

\bvb “\inx[L]{Wighride} is the plain called, where Surt and the cheerful gods find each other at war. A hundred \inx[C]{rest}[rests] it stretches in each direction; for them that plain is marked out.”\evb
\evg


\bvg {\small [Webthrithner quoth:]}
\bva Fróðr est nú gęstr, \hld\ far á bękk jǫtuns, &
\ind ok mę́lumk í sessi saman; &
hǫfði vęðja \hld\ vit skulum hǫllu í &
\ind gęstr, of gęðspęki.\eva

\bvb “Learned art thou now, guest, sit down on the ettin’s bench and let us speak on the seat together. Wager a head, shall we two in the hall, guest, over god-wisdom.”\evb
\evg


\bvg {\small [Gainred quoth:]}
\bva Sęg þat hit ęina, \hld\ ef þitt \edtext{ǿði}{\Bfootnote{The first word on fol. 3r. of \AM; from this point we have the poem in both manuscripts.}} dugir &
\ind ok þú Vafþrúðnir vitir, &
hvaðan jǫrð of kom \hld\ eða upphiminn &
\ind fyrst, hinn fróði jǫtunn.\eva

\bvb “Say the one, if thy wisdom suffices, and thou, Webthrithner, knowest: Whence Earth did come, or \inx[L]{Up-heaven}, first, learned ettin.”\evb
\evg


\bvg {\small [Webthrithner quoth:]}
\bva Ór Ymis holdi \hld\ vas jǫrð of skǫpuð, &
\ind ęn ór bęinum bjǫrg, &
himinn ór hausi \hld\ hins hrimkalda jǫtuns, &
\ind ęn ór svęita sę́r.\eva

\bvb “Out of \inx[P]{Yimer}’s hull was the earth created, but out of his bones the crags; heaven out of the skull of the rime-cold ettin, but out of his blood the sea.\footnoteB{\emph{svęiti} ‘sweat’ is often used to refer to blood. — This v. closely resembles \Grimnismal\ 40–41 TODO.}”\evb
\evg


\bvg {\small [Gainred quoth:]}
\bva Sęg þat annat, \hld\ ef þitt ǿði dugir &
\ind ok þú Vafþrúðnir vitir, &
hvaðan Máni of kom, \hld\ svá’t fęrr menn yfir, &
\ind eða Sól hit sama.\eva

\bvb “Say the other, if thy wisdom suffices, and thou, Webthrithner, knowest: Whence Moon did come, he that travels over men, or likewise Sun?”\evb
\evg


\bvg {\small [Webthrithner quoth:]}
\bva Mundilfari hęitir, \hld\ hann’s Mána faðir &
\ind ok svá Solar hit sama; &
himin hverfa \hld\ þau skulu hvęrjan dag &
\ind ǫldum at ártali.\eva

\bvb “\inx[P]{Mundelfare} is he called; he is the father of the Moon, and likewise of the Sun. They shall circle in the heavens every day, for men to tally years.”\evb
\evg


\bvg {\small [Gainred quoth:]}
\bva Sęg þat þriðja, \hld\ alls þik svinnan kveða &
\ind ok þú Vafþrúðnir vitir, &
hvaðan dagr of kom, \hld\ sá’s fęrr drótt yfir, &
\ind eða nótt með niðum.\eva

\bvb “Say the third, as they call thee wise, and thou, Webthrithner, knowest: Whence the day came, the one that travels over the retinue, or night with the moon-phases?”\evb
\evg


\bvg {\small [Webthrithner quoth:]}
\bva Dęllingr hęitir, \hld\ hann’s Dags faðir, &
\ind ęn Nótt vas Nǫrvi borin; &
ný ok nið \hld\ skópu nýt ręgin &
\ind ǫldum at ártali.\eva

\bvb “\inx[P]{Delling} is called; he is the father of \inx[P]{Day}, but \inx[P]{Night} was born to \inx[P]{Narrow}. The waxing and waning,\footnoteB{i.e. the phases of the moon.} did the useful Reins create, for men to tally years.”\evb
\evg


\bvg {\small [Gainred quoth:]}
\bva Sęg þat fjórða, \hld\ alls þik fróðan kveða, &
\ind ok þú Vafþrúðnir vitir, &
hvaðan vetr of kom \hld\ eða varmt sumar &
\ind fyrst með fróð ręgin.\eva

\bvb “Say the fourth, as they call thee learned, and thou, Webthrithner, knowest: Whence winter did come, or the warm summer, first among the learned Reins?”\evb
\evg


\bvg {\small [Webthrithner quoth:]}
\bva Vindsvalr hęitir, \hld\ hann’s Vetrar faðir, &
\ind ęn Svǫ́suðr Sumars.\footnotemark[15]\eva
\footnotetext[15]{Second half of the v. seems missing.}

\bvb “\inx[P]{Windswoll}\ he is called, he is the father of \inx[P]{Winter}; but \inx[P]{Sosuth}\ of \inx[P]{Summer}.”\evb
\evg


\bvg {\small [Gainred quoth:]}
\bva Sęg þat fimta, \hld\ alls þik fróðan kveða, &
\ind ok þú Vafþrúðnir vitir, &
hvęrr ása ęlztr \hld\ eða Ymis niðja &
\ind yrði í árdaga.\eva

\bvb “Say the fifth, as they call thee learned, and thou, Webthrithner, knowest: Who in days of yore became the eldest of the \inx[G]{Ease}, or of the kinsmen of Yimer \ken{ettins}?\footnoteB{Cf. the question on the 9th c. Malt Stone (DR NOR1988;5): \textbf{huaʀisi : alistiąsa}, perhaps \emph{Hvaʀ es inn ęlisti ása?} ‘Who is the eldest of the Ease?’}”\evb
\evg


\bvg {\small [Webthrithner quoth:]}
\bva Ørófi vetra \hld\ áðr vę́ri jǫrð of skǫpuð, &
\ind þá vas Bergęlmir borinn, &
Þrúðgęlmir \hld\ vas þess faðir, &
\ind ęn Aurgęlmir afi.\eva

\bvb “Uncountable winters before the earth would be created, then \inx[P]{Bearyelmer} was born. \inx[P]{Thrithyelmer} was that one’s father, and \inx[P]{Earyelmer} the grandfather.”\evb
\evg


\bvg {\small [Gainred quoth:]}
\bva Sęg þat sétta, \hld\ alls þik svinnan kveða, &
\ind ok þú Vafþrúðnir vitir, &
hvaðan Aurgęlmir kom \hld\ með jǫtna sonum &
\ind fyrst, hinn fróði jǫtunn.\eva

\bvb “Say the sixth, as they call thee wise, and thou, Webthrithner, knowest: Whence Earyelmer came among the sons of ettins, first, learned ettin?”\evb
\evg


\bvg {\small [Webthrithner quoth:]}
\bva Ór Élivǫ́gum \hld\ stukku ęitrdropar, &
\ind svá óx unz ór varð jǫtunn; &
órar ę́ttir \hld\ kómu þar allar saman; &
\ind því’s þat ę́ alt til atalt.\footnotemark[20]\eva
\footnotetext[20]{Lines 3–4 missing in R and 748, but quoted in \emph{Gylf}.}

\bvb “Out of the \inx[L]{Ilewaves} splashed venom-drops; thus grew until an ettin emerged. Our kindreds came there all together, therefore they are ever wholly fierce.\footnoteB{Over aeons splashing venom-drops combined into a sentient being, Yimer, the ancestor of all Ettins. The account of this poem is quite different from that of \Gylfaginning.}”\evb\evg


\bvg {\small [Gainred quoth:]}
\bva Sęg þat sjaunda, \hld\ alls þik svinnan kveða, &
\ind ok þú Vafþrúðnir vitir, &
hvé sá bǫrn gat \hld\ hinn \edtext{baldni}{\Afootnote{\emph{thus} \AM; aldni ‘the aged, old’ \Regius \emph{breaks alliteration}}} jǫtunn, &
\ind es hann hafði-t gýgjar gaman.\eva

\bvb “Say the seventh, as they call thee wise, and thou, Webthrithner, knowest: How did that one, the defiant ettin, beget children, when he did not enjoy the pleasure of a troll-woman?”\evb
\evg


\bvg {\small [Webthrithner quoth:]}
\bva Und hęndi vaxa \hld\ kvǫ́ðu hrímþursi &
\ind męy ok mǫg saman; &
fótr við fǿti \hld\ gat hins fróða jǫtuns &
\ind sexhǫfðaðan son.\eva

\bvb “Neath the arm\footnoteB{lit. ‘hand’.} on the \inx[G]{Rime-Thurses}[rime-thurse], they said that a maiden and lad grew together. A foot against a foot begot, of the learned ettin, a six-headed son.”\evb
\evg


\bvg {\small [Gainred quoth:]}
\bva Sęg þat áttunda, \hld\ alls þik fróðan kveða, &
\ind ok þú Vafþrúðnir vitir, &
hvat fyrst of mant \hld\ eða fręmst of vęizt, &
\ind þú est alsviðr jǫtunn.\eva

\bvb “Say the eigth, as they call thee learned, and thou, Webthrithner, knowest: What thou first rememberest, or foremost knowest? Thou art all-wise, ettin.”\evb
\evg


\bvg {\small [Webthrithner quoth:]}
\bva \edtext{Ørófi vetra \hld\ áðr vę́ri jǫrð of skǫpuð, &
\ind þá vas Bergęlmir borinn; &
þat fyrst of man’k, \hld\ es hinn fróði jǫtunn &
\ind á vas lúðr of lagiðr.}{\lemma{Ørófi ... lagiðr}\Bfootnote{The whole verse is quoted in \Gylfaginning.}}\eva

\bvb “Uncountable winters before the earth would be created, then Bearyelmer was born. That I first remember, when the learned ettin on the tree-trunk was laid.\footnoteB{The reference here is obscure. According to the prose of \Gylfaginning\, after the sons of \inx[P]{Byre} (that is, \inx[P]{Weden}, \inx[P]{Will} and \inx[P]{Wigh}) slew Yimer, so much blood flew from his wounds that all the race of Ettins were drowned, save for  Bearyelmer and his family, who survived by getting up on his \emph{lúðr}. In regular prose, \emph{lúðr} usually means ‘trumpet’, but it can also refer to a hollow tree-trunk. Considering the transitive nature of Bearyelmer being laid (\emph{of lagiðr}) on it, it could rather be interpreted as describing a boat burial, in which case the first thing Webthrithner remembers would be Bearyelmer’s funeral.}”\evb
\evg


\bvg {\small [Gainred quoth:]}
\bva Sęg þat níunda, \hld\ alls þik svinnan kveða, &
\ind ok þú Vafþrúðnir vitir, &
hvaðan vindr of kømr \hld\ svá’t fęrr vág yfir, &
\ind ę́ męnn hann sjalfan of séa.\eva

\bvb “Say the ninth, as they call thee wise, and thou, Webthrithner, knowest: Whence the wind comes, it that travels over the wave; ever men see hisself.\footnoteB{Almost certainly a negation has been lost here, men can of course not see the wind.}”\evb
\evg


\bvg {\small [Webthrithner quoth:]}
\bva Hrę́svęlgr hęitir, \hld\ es sitr á himins ęnda, &
\ind jǫtunn í arnar ham; &
af hans vę́ngjum \hld\ kveða vind koma &
\ind alla męnn yfir.\eva

\bvb “\inx[P]{Rawswallower} he is called, who sits at the end of the heavens; an ettin in an eagle’s \inx[C]{hame}. From his wings, they say that the wind comes over all men.”\evb
\evg


\bvg {\small [Gainred quoth:]}
\bva Sęg þat tíunda, \hld\ alls þú tíva rǫk &
\ind ǫll Vafþrúðnir vitir, &
hvaðan Njǫrðr of kom \hld\ með ása sonum. &
Hofum ok hǫrgum \hld\ hann rę́ðr hundmǫrgum &
\ind ok varð-at hann ǫ́sum alinn.\eva

\bvb “Say the tenth, since thou of the \inx[P]{Rakes of the Tues} all, Webthrithner, knowest: Whence \inx[P]{Nearth} did come among sons of the \inx[G]{Ease}? Of \inx[C]{hoves} and \inx[C]{harrows} he rules a hound-many,\footnoteB{Cf. \Grimnismal\ 16.} and he was not among the Ease begotten.”\evb
\evg


\bvg {\small [Webthrithner quoth:]}
\bva Í Vanahęimi \hld\ skópu hann vís ręgin &
\ind ok sęldu at gíslingu goðum, &
í aldar rǫk \hld\ hann mun aptr koma &
\ind hęim með vísum vǫnum.\eva

\bvb “In \inx[L]{Waneham}, created him the wise \inx[G]{Reins}\footnoteB{While \emph{ręgin} ‘Reins’ is usually just a synonym of \emph{goð} ‘gods’, it seems here to refer specifically to the Wanes, in contrast with the \inx[G]{Ease}.} created him, and sold him as a hostage to the gods. In the rake of the \inx[C]{eld}\footnoteB{i.e. the \inx[P]{Rakes of the Reins}.} he will come back, home among the wise \inx[G]{Wanes}.”\evb
\evg


\bvg {\small [Gainred quoth:]}
\bva Sęg þat ęllipta, \hld\ hvar ýtar túnum í &
\ind hǫggvask hvęrjan dag; &
val þęir kjósa \hld\ ok ríða vígi frá, &
\ind sitja męir of sáttir saman.\footnoteB{This and the next v. are damaged in both R and 748; R has only this verse, but splits it in two (the 2nd starting with \emph{val}), while 748 has 40:1 (Ms.: \emph{S. þ. e. XI}) and then jumps to the answer v. 41. They have here been reconstructed, but it is possible some lines are still missing. TODO: use edtext instead}\eva

\bvb “Say the eleventh: Where men in yards hew away at each other each day. The slain they choose, and ride from the battle; sit they more content together.”\evb
\evg


\bvg {\small [Webthrithner quoth:]}
\bva Allir ęinhęrjar \hld\ Óðins túnum í &
\ind hǫggvask hvęrjan dag, &
val þeir kjósa \hld\ ok ríða vígi frá, &
\ind sitja męir of sáttir saman.\eva

\bvb “All the \inx[G]{One-harriers} in Weden’s yards hew away at each other every day. The slain they choose, and ride from the battle; sit they more content together.”\evb
\evg


\bvg {\small [Gainred quoth:]}
\bva Sęg þat tolpta, \hld\ hví þú tíva rǫk &
\ind ǫll Vafþrúðnir vitir, &
frá jǫtna rúnum \hld\ ok allra goða &
\ind þú hit sannasta sęgir, &
\ind hinn alsvinni jǫtunn.\eva

\bvb “Say the twelfth: Why thou, the rakes of the Tues all, Webthrithner, might know? From the \inx[C]{rune}[runes] of the ettins and of all the gods speakest thou the truest, all-wise ettin.”\evb
\evg


\bvg {\small [Webthrithner quoth:]}
\bva Frá jǫtna rúnum \hld\ ok allra goða &
\ind ek kann sęgja satt, &
\ind því’t hvęrn hęf’k heim of komit, &
níu kom’k hęima \hld\ fyr niflhęl neðan; &
\ind hinig dęyja ór hęlju halir.\eva

\bvb “From the runes of the ettins and of all the gods I can speak truly, for I have come into each \inx[C]{Home}. Into nine Homes I came beneath \inx[L]{Nivelhell}; that way die men out of \inx[L]{Hell}.\footnoteB{Presumably lower underworlds, more severe than the ‘normal’ one. \textcite{FinnurEdda}\ considers \emph{ór hęlju} “out of Hell” a later interpolation, presumably for metric reasons, but there is no textual support for it.}”\evb
\evg


\bvg {\small [Gainred quoth:]}
\bva Fjǫlð ek fór, \hld\ fjǫlð fręistaða’k, &
\ind fjǫlð ek ręynda ręgin; &
hvat lifir manna, \hld\ þá’s hinn mę́ra líðr &
\ind fimbulvetr með firum?\eva

\bvb “Much I travelled, much I tempted, much I tested the Reins.\footnoteB{Cf. v. 3.} What remains of men, when the renowned \inx[P]{Fimble-winter} among them passes?}”\evb
\evg


\bvg {\small [Webthrithner quoth:]}
\bva Líf ok Lífþrasir, \hld\ ęn þau lęynask munu &
\ind í holti Hoddmímis; &
morgindǫggvar \hld\ þau sér at mat hafa; &
\ind þaðan af aldir alask.\eva

\bvb “\inx[P]{Life} and \inx[P]{Lifethrasher}, but they will hide themselves in \inx[P][Hoardmimer}’s wood.\footnoteB{Perhaps in the hollowed-out Uggdrassle.} Morning-dew [will] they have as their food; thence generations [will] be bred.”\evb
\evg


\bvg {\small [Gainred quoth:]}
\bva Fjǫlð ek fór, \hld\ fjǫlð fręistaða’k, &
\ind fjǫlð ek ręynda ręgin; &
hvaðan kømr sól \hld\ á hinn slétta himin, &
\ind es þessa hęfr Fęnrir farit?\eva

\bvb “Much I travelled, much I tempted, much I tested the Reins. Whence comes Sun onto the smooth heaven, when \inx[P]{Fenrer} has this one\footnoteB{i.e. the current incarnation of the sun, as explained in the next v.} slain?”\evb
\evg


\bvg {\small [Webthrithner quoth:]}
\bva Ęina dóttur \hld\ berr alfrǫðull, &
\ind áðr hana Fęnrir fari; &
sú skal ríða, \hld\ þá’s ręgin dęyja, &
\ind móður brautir mę́r.\eva

\bvb “One daughter the elf-wheel \ken{sun}[1] bears before Fenner might slay her. She shall ride—when the Reins die—a maiden her mother’s paths.”\evb
\evg


\bvg {\small [Gainred quoth:]}
\bva Fjǫlð ek fór, \hld\ fjǫlð fręistaða’k, &
\ind fjǫlð ek ręynda ręgin; &
hvęrjar ’ro męyjar, \hld\ es líða mar yfir, &
\ind fróðgęðjaðar fara.\eva

\bvb “Much I travelled, much I tempted, much I tested the Reins. Which are the maidens that pass over the ocean; learned-minded they go?”\evb
\evg


\bvg {\small [Webthrithner quoth:]}
\bva Þríar þjóðár \hld\ falla þorp yfir &
\ind męyja Mǫgþrasis; &
hamingjur ęinar \hld\ þę́r’s í hęimi eru, &
\ind þó þę́r með jǫtnum alask.\eva

\bvb “Three great rivers fall over the settlement of the maidens of Maythrasher; the only Hamings are they in the Home,\footnoteB{In Ettinham, or in the entire world?} though they are among the ettins begotten.”\evb
\evg


\bvg {\small [Gainred quoth:]}
\bva Fjǫlð ek fór, \hld\ fjǫlð fręistaða’k, &
\ind fjǫlð ek ręynda ręgin; &
hvęrir ráða ę́sir \hld\ ęignum goða, &
\ind þá’s sloknar Surta logi?\eva

\bvb “Much I travelled, much I tempted, much I tested the Reins. Which Ease rule the estates of the gods, when the flame of \inx[P]{Surt} goes out?”\evb
\evg


\bvg {\small [Webthrithner quoth:]}
\bva Víðarr ok Váli \hld\ byggva vé goða, &
\ind þá’s sloknar Surtalogi; &
Móði ok Magni \hld\ skulu Mjǫlni hafa &
\ind Vingnis at vígþroti.\eva

\bvb “\inx[P]{Wider} and \inx[P]{Wonnel} inhabit the \inx[C]{wigh}[wighs] of the gods, when the flame of Surt goes out. \inx[P]{Mood} and \inx[P]{Main} shall own \inx[P]{Millner}, when \inx[P]{Wingner} is too tired to fight.\footnoteB{lit. ‘at Wingner’s fight-exhaustion,’ referring to his death.}”\evb
\evg


\bvg {\small [Gainred quoth:]}
\bva Fjǫlð ek fór, \hld\ fjǫlð fręistaða’k, &
\ind fjǫlð ek ręynda ręgin; &
hvat verðr Óðni \hld\ at aldrlagi, &
\ind þá’s rjúfask ręgin?\eva

\bvb “Much I travelled, much I tempted, much I tested the Reins. What brings Weden’s life to an end, when the Reins are broken?\footnoteB{Cf. the formulation in \Baldrsdraumar\ 14: \emph{es lauss Loki · líðr ór bǫndum // ok ragna rǫk · rjúfęndr koma.} ‘when loose Lock passes out of his bonds, and at the \inx[P]{Rakes of the Reins}, the breakers come.’}”\evb
\evg


\bvg {\small [Webthrithner quoth:]}
\bva Ulfr glęypa \hld\ mun Aldafǫðr, &
\ind þess mun Víðarr vreka; &
kalda kjapta \hld\ hann klyfja mun &
\ind vitnis vígi at.\eva

\bvb “The wolf will devour \inx[P]{Eldfather} \name{= Weden}; that will Wider avenge. The cold jaws he will cleave, of the Wolf at the battle.”\evb
\evg


\bvg {\small [Gainred quoth:]}
\bva Fjǫlð ek fór, \hld\ fjǫlð fręistaða’k, &
\ind fjǫlð ek ręynda ręgin; &
hvat mę́lti Óðinn, \hld\ áðr á bál stigi, &
\ind sjalfr í ęyra syni?\eva

\bvb “Much I travelled, much I tempted, much I tested the Reins. What spoke Weden, before he would step onto the pyre,\footnoteB{Weden did not burn on the pyre, and so the sense must be ‘before he set the pyre alight’.} himself in the ear of the son?”\evb
\evg


\bvg {\small [Webthrithner quoth:]}
\bva Ęy \edtext{manngi}{\Afootnote{manni \Regius\AM\ \emph{is impossible; a nominative is needed}}} vęit, \hld\ hvat þú í árdaga &
\ind sagðir í ęyra syni; &
fęigum munni \hld\ mę́lta’k mína forna stafi &
\ind ok of ragna rǫk. &
Nú við Óðin \hld\ dęilda’k mína orðspęki; &
\ind þú est ę́ vísastr vera.\eva

\bvb “Ever no man knows, what thou in days of yore saidst in the ear of the son. With \inx[C]{fey}\footnoteB{Webthrithner realizes that he was bound to die (\emph{fęigr} ‘fey’, a word with strong fatalistic connotations) from the moment he proposed the wager (v. 19), as no being can outwit Weden.} mouth I spoke my ancient \inx[C]{stave}[staves], and of the Rakes of the Reins. Now with Weden I shared my word-wisdom\footnoteB{The same word-wisdom Weden in v. 5 set out to try.}; thou art ever wisest of beings.\footnoteB{\emph{verr} literally means ‘husband, man,’ but here surely in the broader sense of ‘(male) being’. For other instances of gods being called men, see TODO.}”\evb
\evg
% Weden
%	\include{books/Speeches of Allwise.tex}% Wisdom poem
	\bookStart{The Speeches of the High One}[Hávamǫ́l]

%Introduction.

The \textbf{Speeches of the High One} is the second poem of \Regius, which is also the only ancient manuscript in which it is attested. Several verses are however cited in other places, such as Eyv \emph{Hák} (TODO: formatting) 21 and \FostrbroedhraSaga\ TODO.

The poem as it currently comes down to us hardly seems like a single composition, much rather like a grab bag of traditional verses and poems associated with the god Weden. It combines two separate advice-poems with verses concerning Weden’s love adventures, runes and spells. Little unites these various strands other than their speaker.

Following previous authors, I identify several such strands, excepting various lone insert-verses. In this edition each of them is given a separate, short introduction:

\begin{itemize}
  \item 1–79 The Guest-strand, containing practical life advice placed within a frame narrative of a guest arriving at a homestead.
  \item 81–89 Other verses of advice, mostly composed in \Fornyrdislag.
  \item 90–109 Weden’s love adventures, advice for love and seduction.
  \item 110–135 The Speeches of Loddfathomer (\emph{Loddfáfnismǫ́l}), advice given to Loddfathomer.
  \item 136–144 The Rune-tally (\emph{Rúnatal}), various verses relating to runes.
  \item 145–163 The Leed-tally (\emph{Ljóðatal}), Weden’s listing of 18 spells.
  \item 164 Final verse, composed when the poem as we have it was assembled.
\end{itemize}

Whatever their origins, it is clear from the final verse that they have been thought of as a single work, but it is notable that this verse, which also contains the title \emph{Hávamǫ́l} ‘Speeches of the High One’, is highly metrically irregular. It has likely been composed by the person who assembled the disparate elements listed above into one text.

\sectionline

\section{The Guest-strand}

The Guest-Strand (Old Norse: \emph{Gęstaþáttr}) is possibly the finest work in Norse poetry. Sadly, its structure has been obscured by various inserted and possibly displaced verses. My hope is to shed some light on the original vision behind the poem, while as usual not changing the order of verses as they appear in the only surviving witness manuscript.

The poem moves through many elements of life, but in a poetically almost seamless way. To move from one topic to another, the poet often employs transitions where a verse recalls the structure of the previous one, but with a new subject. This is particularly evident in verses 4–5 and 10–11.

The strand begins with a verse encouraging travellers to be wary of entering strange houses without first spying out who is inside (1), after which a voice inside of a farmstead (possibly Weden?) announces that a guest is waiting to be let in (2). The same speaker then lists several things which the newly arrived guest needs from the host, namely: fire, food and clothes (3), water, a towel, a great welcome, a good reception, an opportunity to speak and silence in return (4).

After this focus shifts to the conduct of the wanderer, with an introductory verse explaining that he needs wit (specifically \inx[C]{manwit} (\emph{manvit}); see Encyclopedia), lest he become a laughing-stock (5). He should be silent but attentive, and choose his words carefully (6–7). He should be confident in himself and his own decisions, and not rely too much on the opinions of others (8–9), since there is nothing better one may bring along on the journey than much manwit (10).

Here the advice moves to the subject alcohol. Where the best thing one may bring along on the journey is manwit, the worst is too much ale (11). It is not as good as men call it (12) since it “robs [them] of their senses”; it is even personified as a “heron of forgetfulness” (13). A drinking round is best when the participants do not drink too much, but rather regain their senses afterwards (14).

Verse 15 contains some general advice; a royal child should be silent, thoughtful and bold in battle, and all men should stay happy, until they die.

TODO.

\sectionline

\bvg
\bva \alst{G}áttir allar \hld\ áðr \alst{g}angi framm &
\ind \edtext{of \alst{sk}oðask \alst{sk}yli,}{\lemma{of skoðask skyli}\Bfootnote{\emph{om.} \GylfMS}} &
\ind of \alst{sk}yggnask \alst{sk}yli; &
því’t ó\alst{v}íst ’s at \alst{v}ita, \hld\ hvar ó\alst{v}inir &
\ind sitja á \alst{f}lęti \alst{f}yrir.\eva

\bvb All doorways—before one might go forth—should be watched, should be spied at; for uncertain ’tis to know, where enemies sit on the benches inside.\evb
\evg


\bvg
\bva \alst{G}efęndr hęilir, \hld\ \alst{g}ęstr ’s inn kominn, &
\ind hvar skal \alst{s}itja \alst{s}já? &
mjǫk es \alst{b}ráðr \hld\ sá’s á \alst{b}rǫndum skal &
\ind síns of \alst{f}ręista \alst{f}rama.\eva

\bvb Hail the givers,\footnoteB{The hosts.} a guest is come in! Where shall this one sit? Very impatient is he, who on the fires shall try his distinction.\footnoteB{Possibly referring a Norwegian folk custom, wherein a guest would sit down on the wood-pile outside of the door, waiting until being let in. See further TODO SOME ARTICLE on this custom. The speaker thus announces to the hosts that a frozen, wet and tired guest has arrived and currently sits impatiently on the wood-pile, and ought to be taken in.}\evb
\evg


\bvg
\bva \alst{Ę}lds es þǫrf \hld\ þęim’s \alst{i}nn es kominn &
\ind ok á \alst{k}néi \alst{k}alinn, &
\alst{m}atar ok váða \hld\ es \alst{m}anni þǫrf, &
\ind þęim’s hęfr of \alst{f}jall \alst{f}arit.\eva

\bvb Of fire is there need for the one who is come in, and cold about the knees; of food and of clothing is there need for the one who over the fell has fared.\evb
\evg


\bvg
\bva \alst{V}ats es þǫrf \hld\ þęim’s til \alst{v}erðar kømr, &
\ind \alst{þ}ęrru ok \alst{þ}jóðlaðar, &
\alst{g}óðs of ǿðis, \hld\ —ef sér \alst{g}eta mę́tti— &
\ind \alst{o}rðs ok \alst{ę}ndrþǫgu.\eva

\bvb Of water is there need for the one who comes for a meal; of a towel and of a great welcome; of a good reception—if he might get one—of speech, and of silence in return.\footnoteB{There is a well thought-out linear progression throughout this verse. The guest must first wash himself, then dry himself with a towel, then be welcomed to sit and eat at the table and speak with the host. The host has done his part, and now it is the guest’s turn. This nicely leads the transition to the following verses, where the proper conduct of the guest (first in speech, and then in various other areas) is discussed.}\evb
\evg


\bvg
\bva \alst{V}its es þǫrf \hld\ þęim’s \alst{v}íða ratar; &
\ind dę́lt es \alst{h}ęima \alst{h}vat; &
at \alst{au}gabragði \hld\ verðr sá’s \alst{ę}kki kann &
\ind ok með \alst{s}notrum \alst{s}itr.\eva

\bvb Of wit is there need for the one who widely roams; everything is easy at home. A laughing-stock\footnoteB{An idiom, \emph{augabragð} lit. ‘twinkling of an eye, moment’.} becomes he who nothing knows, and among the clever sits.\evb
\evg


\bvg
\bva At \alst{h}yggjandi sinni \hld\ skyli-t maðr \alst{h}rǿsinn vesa, &
\ind hęldr \alst{g}ę́tinn at \alst{g}ęði, &
þá’s \alst{h}orskr ok þǫgull \hld\ kømr \alst{h}ęimisgarða til, &
\ind sjaldan verðr \alst{v}íti \alst{v}ǫrum. &
\edtext{því’t \alst{ó}brigðra vin \hld\ fę́r \alst{a}ldrigi, &
\ind an \alst{m}anvit \alst{m}ikit.}{\lemma{því \dots\ mikit}\Bfootnote{The shift in person from third to second, along with the abnormal verse length (six lines instead of four), indicates that this is an insertion.}}\eva

\bvb Of his thinking should man not be boastful; rather guarding of his senses, when sharp and silent he comes to a homestead; sudden injury seldom strikes the wary, (for thou gettest never an unfickler friend, than much \inx[C]{manwit}.)\evb
\evg


\bvg
\bva Hinn \alst{v}ari gęstr, \hld\ es til \alst{v}erðar kømr, &
\ind \alst{þ}unnu hljóði \alst{þ}ęgir; &
\alst{ęy}rum hlýðir, \hld\ ęn \alst{au}gum skoðar, &
\ind svá nýsisk \alst{f}róðra hvęrr \alst{f}yrir.\eva

\bvb The wary guest—when for a meal he comes—with thin heed shuts up.\footnoteB{i.e. is in attentive silence.} With ears he heeds, but with eyes observes; so pries each learned man about.\evb
\evg


\bvg
\bva Hinn es \alst{s}ę́ll, \hld\ es \alst{s}ér of getr &
\ind \alst{l}of ok \alst{l}íknstafi; &
\alst{ó}dę́lla es við þat, \hld\ es \alst{ęi}ga skal &
\ind \alst{a}nnars brjóstum \alst{í}.\eva

\bvb The one is blessed, who for himself gets praise and staves of grace. ’Tis uneasy regarding that which one shall own in another’s breast.\evb
\evg


\bvg
\bva \alst{S}á es \alst{s}ę́ll, \hld\ es \alst{s}jalfr of á &
\ind \alst{l}of ok vit meðan \alst{l}ifir; &
því’t \alst{i}ll rǫ́ð \hld\ hęfr maðr \alst{o}pt þęgit &
\ind \alst{a}nnars brjóstum \alst{ó}r.\eva

\bvb That one is blessed, whose self owns praise and wits while he lives; for ill counsels has man oft taken out of another’s breast.\evb
\evg


\bvg
\bva \alst{B}yrði \alst{b}ętri \hld\ berr-at maðr \alst{b}rautu at, &
\ind an sé \alst{m}anvit \alst{m}ikit; &
\alst{au}ði bętra \hld\ þykkir þat í \alst{ó}kunnum stað; &
\ind slíkt es \alst{v}álaðs \alst{v}era.\eva

\bvb A better burden bears man not on the road than much manwit. In an unknown place it seems better than wealth; such is the shelter of the impoverished.\evb
\evg

% TODO: NEW SECTION (Alcohol)

\bvg
\bva \alst{B}yrði \alst{b}ętri \hld\ berr-at maðr \alst{b}rautu at, &
\ind an sé \alst{m}anvit \alst{m}ikit; &
\alst{v}egnest \alst{v}erra \hld\ \alst{v}egr-a \alst{v}ęlli at, &
\ind an sé \alst{o}fdrykkja \alst{ǫ}ls.\eva

\bvb A better burden bears man not on the road than much manwit. Worse way-provision he drags not along in the field\footnoteB{\emph{vǫllr} ‘plain, (uncultivated) field’ is repeated in vv. 38 and 49. It is easily understood that the heaths and plains of Iron Age Norway were particularly unsafe places, where a traveller needed to keep his wits with him, lest he fall victim to robbers or murderers.} than a too great drink of ale.\evb
\evg


\bvg
\bva Es-a svá \alst{g}ótt, \hld\ sęm \alst{g}ótt kveða, &
\ind \alst{ǫ}l \alst{a}lda sonum; &
því’t \alst{f}ę́ra vęit, \hld\ es \alst{f}lęira drekkr, &
\ind síns til \alst{g}ęðs \alst{g}umi.\eva

\bvb ’Tis not so good, as good they say, ale for the sons of men; for the less he knows, as the more he drinks, man of his own senses.\evb
\evg


\bvg
\bva \alst{Ó}minnishegri hęitir, \hld\ sá’s yfir \alst{ǫ}lðrum þrumir, &
\ind hann stelr \alst{g}ęði \alst{g}uma; &
þess \alst{f}ogls \alst{f}jǫðrum \hld\ ek \alst{f}jǫtraðr vas’k &
\ind í \alst{g}arði \alst{G}unnlaðar.\eva

\bvb The heron of forgetfulness is called he who above ale-feasts hovers; he robs men of their senses.\footnoteB{Here drunkenness is personified as a bird, a “heron of forgetfulness”.} With that bird’s feathers I was fettered in the yards of \inx[P]{Guthlathe}.\evb
\evg


\bvg
\bva \alst{Ǫ}lr ek varð, \hld\ varð \alst{o}frǫlvi, &
\ind at hins \alst{f}róða \alst{F}jalars; &
því es \alst{ǫ}lðr bazt, \hld\ at \alst{a}ptr of hęimtir &
\ind hvęrr sitt \alst{g}ęð \alst{g}umi.\eva

\bvb Drunk I became—I became the drunkest by far—at the learned Fealer’s [home]. Thus is an ale-feast best, as each man takes his senses back home.\evb
\evg

% TODO: NEW SECTION (War)

\bvg
\bva \alst{Þ}agalt ok hugalt \hld\ skyli \alst{þ}jóðans barn &
\ind ok \alst{v}ígdjarft \alst{v}esa; &
\alst{g}laðr ok ręifr \hld\ skyli \alst{g}umna hvęrr, &
\ind unz sinn \alst{b}íðr \alst{b}ana.\eva

\bvb Silent and thoughtful should the ruler’s child be, and battle-bold. Glad and cheerful should each man be, until he suffer his bane.\evb
\evg


\bvg
\bva \alst{Ó}snjallr maðr \hld\ hyggsk munu \alst{ę}y lifa, &
\ind ef við \alst{v}íg \alst{v}arask; &
ęn \alst{ę}lli gefr hǫ́num \hld\ \alst{ę}ngi frið, &
\ind þótt hǫ́num \alst{g}ęirar \alst{g}efi.\eva

\bvb The unvalorous man thinks he will forever live, if he of war is wary; but old age gives him no peace, although spears might give him.\footnoteB{He might have been spared by the spears, but death will still find him. The underlying meaning seems to be that since death is unavoidable it is better to live bravely, even if one risks dying in battle, than to live cowardly and die of old age. This verse connects well to the ancient view of the ‘straw-death’.}\evb
\evg


\bvg
\bva \alst{K}ópir afglapi, \hld\ es til \alst{k}ynnis kømr, &
\ind \alst{þ}ylsk hann umb eða \alst{þ}rumir; &
alt es \alst{s}ęnn, \hld\ ef \alst{s}ylg of getr, &
\ind uppi es þá \alst{g}ęð \alst{g}uma.\eva

\bvb Gapes the oaf when to visit he comes; he mumbles about or loiters. All at once—if a sip he gets—are the senses of the man exposed.\evb
\evg


\bvg
\bva Sá ęinn \alst{v}ęit, \hld\ es \alst{v}íða ratar &
\ind ok hęfr \alst{f}jǫlð of \alst{f}arit, &
hvęrju \alst{g}ęði \hld\ stýrir \alst{g}umna hvęrr, &
\ind sá es \alst{v}itandi ’s \alst{v}its.\eva

\bvb He alone knows, who widely roams, and has travelled much: his own senses does each man control, who is aware of his wits.\evb
\evg


\bvg
\bva \alst{H}aldi-t maðr á kęri, \hld\ drekki þó at \alst{h}ófi mjǫð, &
\ind mę́li \alst{þ}arft eða \alst{þ}ęgi; &
\alst{ó}kynnis þess \hld\ váar þik \alst{ę}ngi maðr, &
\ind at gangir \alst{s}nimma at \alst{s}ofa.\eva

\bvb Man ought not to hold onto the cask, yet drink a fitting serving of mead; he ought to speak the needful or shut up.\footnoteB{Identical to a certain verse in \Vafthrudnismal\ TODO: which one} For that uncouthness will no man blame thee, that thou go early to sleep.\evb
\evg


\bvg
\bva \alst{G}rǫ́ðugr halr, \hld\ nema \alst{g}ęðs viti, &
\ind \alst{e}tr sér \alst{a}ldrtrega; &
opt fę́r \alst{h}lǿgis, \hld\ es með \alst{h}orskum kømr, &
\ind \alst{m}anni hęimskum \alst{m}agi.\eva

\bvb The gluttonous man—unless he know his sense—eats himself a life-sorrow. Oft the belly—when among the sharp he comes—brings a foolish man ridicule.\evb
\evg


\bvg
\bva \alst{H}jarðir þat vitu, \hld\ nę́r \alst{h}ęim skulu, &
\ind ok \alst{g}anga þá af \alst{g}rasi; &
ęn \alst{ó}sviðr maðr \hld\ kann \alst{ę́}vagi &
\ind síns of \alst{m}ál \alst{m}aga.\eva

\bvb Herds know when homewards they shall [turn], and then part from the grass; but an unwise man never knows the measure of his own belly.\evb
\evg


\bvg
\bva \alst{V}esall maðr \hld\ ok \alst{i}lla skapi &
\ind \alst{h}lę́r at \alst{h}vívetna; &
hitki hann \alst{v}ęit, \hld\ es \alst{v}ita þyrpti, &
\ind at hann es-a \alst{v}amma \alst{v}anr.\eva

\bvb The wretched man, and the ill-spirited, laughs at whatever. This he knows not, which he might need to know: he is not free of blemishes.\evb
\evg


\bvg
\bva \alst{Ó}sviðr maðr \hld\ vakir umb \alst{a}llar nę́tr &
\ind ok \alst{h}yggr at \alst{h}vívetna; &
þá es \alst{m}óðr, \hld\ es at \alst{m}orni kømr; &
\ind alt es \alst{v}íl sęm \alst{v}as.\eva

\bvb The unwise man is awake for all nights, and thinks of whatever. Then he is weary when the morning comes; [his] trouble is all as it was.\evb
\evg


\bvg
\bva \alst{Ó}snotr maðr \hld\ hyggr sér \alst{a}lla vesa &
\ind \alst{v}iðrhlę́jęndr \alst{v}ini; &
hit-ki hann \alst{f}iðr, \hld\ þótt þęir of hann \alst{f}ár lesi, &
\ind ef með \alst{s}notrum \alst{s}itr.\eva

\bvb The unclever man thinks all who laugh with him friends. This he finds not, that they find flaws in him, if among the clever he sits.\evb
\evg


\bvg
\bva \alst{Ó}snotr maðr \hld\ hyggr sér \alst{a}lla vesa &
\ind \alst{v}iðhlę́jęndr \alst{v}ini; &
\alst{þ}á þat fiðr \hld\ es at \alst{þ}ingi kømr, &
\ind at á \alst{f}ormę́lęndr \alst{f}áa.\eva

\bvb The unclever man thinks all who laugh with him friends. Then he finds—when to the \inx[C]{Thing} he comes—that he has spokesmen few.\footnoteB{Repeated in v. 62. He has few who are ready to take his side and speak up for him; the sense is that true friends are proven in conflict, not in talking. The Thing (see Encyclopeda) was the old Germanic legal assembly, and so the specific reference here is legal disputes, but it should be kept in mind that they could easily turn into deadly feuds.}\evb
\evg


\bvg
\bva \alst{Ó}snotr maðr \hld\ þykkisk \alst{a}lt vita, &
\ind ef á sér i \alst{v}ǫ́ \alst{v}eru; &
hitki hann \alst{v}ęit, \hld\ hvat hann skal \alst{v}ið kveða, &
\ind ef hans \alst{f}ręista \alst{f}irar.\eva

\bvb The unclever man seems to know everything if he takes shelter in a nook. This he knows not, what he shall say in return if men test him.\evb
\evg


\bvg
\bva \alst{Ó}snotr maðr, \hld\ es með \alst{a}ldir kømr, &
\ind \alst{þ}at ’s bazt at hann \alst{þ}ęgi; &
\alst{ę}ngi þat vęit, \hld\ at hann \alst{ę}kki kann, &
\ind nema hann \alst{m}ę́li til \alst{m}art. &
\alst{v}ęit-a maðr, \hld\ hinn’s \alst{v}ę́tki vęit, &
\ind þótt hann \alst{m}ę́li til \alst{m}art.\eva

\bvb The unclever man, when among people he comes, ’tis best that he shut up. None knows that he nothing knows, unless he speak too much. Man knows not, who nothing knows, although he speak too much.\footnoteB{That is, mindless speech will not make him any wiser.}\evb
\evg


\bvg
\bva \alst{F}róðr sá þykkisk, \hld\ es \alst{f}regna kann, &
\ind ok \alst{s}ęgja hit \alst{s}ama, &
\alst{ęy}vitu lęyna \hld\ męgu \alst{ý}ta synir &
\ind því es \alst{g}ęngr of \alst{g}uma.\eva

\bvb Learned seems he who can ask and answer the same. Naught may the sons of men conceal of that\footnoteB{Rumours and gossip.} which goes about a man.\evb
\evg


\bvg
\bva \alst{Ǿ}rna mę́lir, \hld\ sá’s \alst{ę́}va þęgir, &
\ind \alst{st}aðlausu \alst{st}afi; &
\alst{h}raðmę́lt tunga, \hld\ nema \alst{h}aldęndr ęigi, &
\ind opt sér ó\alst{g}ótt of \alst{g}ęlr.\eva

\bvb Quite enough speaks he—who never shuts up—utterings of absurdity. A quick-spoken tongue—unless it be held in place\footnoteB{lit. ‘unless holders own it’ or ‘unless it own holders’. The ‘holders’ may perhaps refer to the teeth holding the tongue in places.}—oft sings evil [into being] for itself.\evb
\evg


\bvg
\bva At \alst{au}gabragði \hld\ skal-a maðr \alst{a}nnan hafa, &
\ind \edtext{þótt}{\lemma{þótt “although”}\Bfootnote{Perhaps an error? \emph{es} ‘when’ would surely work better in context.}} til \alst{k}ynnis \alst{k}omi; &
margr \alst{f}róðr þykkisk, \hld\ ef \alst{f}reginn es-at &
\ind ok nái \alst{þ}urrfjallr \alst{þ}ruma.\eva

\bvb As a laughing-stock shall man not have another, although he come to visit. Many a one seems learned if he is not asked, and manages to loiter about dry-skinned.\footnoteB{This sense of \emph{fjall} is apparently almost non-existent in Old Norse literature, but compare Swedish \emph{fjäll} ‘scale (on fish and reptiles)’. The meaning is in any case figurative, equivalent to the English “get one’s feet wet”.}\evb
\evg


\bvg
\bva \alst{F}róðr þykkisk \hld\ sá’s \edtrans{\alst{f}lótta}{flee}{\Bfootnote{Emended to \emph{flátta} ‘mock’ by \textcite{Athugasemdir1929}}} tękr &
\ind \alst{g}ęstr at \alst{g}ęst hę́ðinn; &
\alst{v}ęit-a gǫrla \hld\ sá’s of \alst{v}erði glissir, &
\ind þótt með \alst{g}rǫmum \alst{g}lami.\eva

\bvb Learned seems he who takes to flee\footnoteB{Probably not literally, rather ‘pulls back, does not take part’.} when a guest at a guest is scoffing. He knows not clearly, who grins above the food, that he with fiends be prattling.\evb
\evg


\bvg
\bva \alst{G}umnar margir \hld\ erusk \alst{g}agnhollir, &
\ind ęn at \alst{v}irði \alst{v}rekask; &
\alst{a}ldar róg \hld\ þat mun \alst{ę́} vesa; &
\ind órir \alst{g}ęstr við \alst{g}ęst.\eva

\bvb Many men are \inx[C]{hold} to each other, but over a meal drive each other away. The strife of mankind will that ever be; guest raves against guest.\evb
\evg


\bvg
\bva \alst{Á}rliga verðar \hld\ skyli maðr \alst{o}pt fáa, &
\ind nema til \alst{k}ynnis \alst{k}omi; &
\alst{s}itr ok \alst{s}nópir, \hld\ lę́tr sęm \alst{s}olginn sé, &
\ind ok kann \alst{f}regna at \alst{f}ǫ́u.\eva

\bvb An early meal should man oft get, unless he come to visit: he sits and idles haplessly, makes as if starved, and can ask about little.\evb
\evg


\bvg
\bva \alst{A}fhvarf mikit \hld\ es til \alst{i}lls vinar, &
\ind þótt á \alst{b}rautu \alst{b}úi, &
ęn til \alst{g}óðs vinar \hld\ liggja \alst{g}agnvegir, &
\ind þótt hann sé \alst{f}irr \alst{f}arinn.\eva

\bvb A great detour ’tis to a wicked friend, although he on the highway live; but to a good friend lie the shortest ways, although he far gone be.\evb
\evg


\bvg
\bva \alst{G}anga skal, \hld\ skal-a \alst{g}ęstr vesa &
\ind \alst{ęy} í \alst{ęi}num stað; &
\edtext{\alst{l}júfr verðr \alst{l}ęiðr}{\lemma{ljúfr verðr lęiðr ‘the loved becomes loathed’}\Bfootnote{}}, \hld\ ef \alst{l}ęngi sitr &
\ind \alst{a}nnars flętjum \alst{á}.\eva

\bvb One shall go; shall not be a guest forever in one place. The loved becomes loathed if for long he sits on another’s benches.\evb
\evg


\bvg
\bva \alst{B}ú es \alst{b}ętra, \hld\ þótt lítit sé, &
\ind \alst{h}alr es \alst{h}ęima \alst{h}vęrr; &
þótt \alst{t}vę́r gęitr ęigi \hld\ ok \alst{t}augręptan sal, &
\ind þat es þó \alst{b}ętra an \alst{b}ǿn.\eva

\bvb A dwelling is better, though small it be: each is a warrior at home. Though two goats he own, and a cord-roofed hall, that is yet better than begging.\evb
\evg


\bvg
\bva \alst{B}ú es \alst{b}ętra, \hld\ þótt lítit sé, &
\ind \alst{h}alr es \alst{h}ęima \alst{h}vęrr; &
\alst{b}lóðugt es hjarta \hld\ þęim’s \alst{b}iðja skal &
\ind sér í \alst{m}ál hvęrt \alst{m}atar.\eva

\bvb A dwelling is better, though small it be: each is a warrior at home. Bloody is the heart of the one who shall beg for himself each meal of food.\evb
\evg


\bvg
\bva \alst{V}ǫ́pnum sínum \hld\ skal-a maðr \alst{v}ęlli á &
\ind \edtext{\alst{f}eti ganga \alst{f}ramarr}{\lemma{feti ganga framarr ‘take one step further’}\Bfootnote{Cf. \Lokasenna\ 1: \emph{svát ęinugi feti gangir framarr,} ‘so that thou not take one step further’.}}; &
því’t ó\alst{v}íst ’s at \alst{v}ita, \hld\ nę́r verðr á \alst{v}egum úti &
\ind \alst{g}ęirs of þǫrf \alst{g}uma.\eva

\bvb From his weapons shall man in the field not take one step further; for uncertain ’tis to know, when on the ways outside, man comes in need of a spear.\evb
\evg


\bvg
\bva Fann’k-a \alst{m}ildan mann \hld\ eða svá \edtext{\alst{m}atar góðan}{\lemma{matar góðan ‘good of meat’}\Bfootnote{A Viking Age expression; see Encyclopedia.}}, &
\ind at vę́ri-t \alst{þ}iggja \alst{þ}ęgit; &
eða \alst{s}íns féar \hld\ \alst{s}vági \edtext{[...]}{\Bfootnote{It is doubtless that a word has been lost here; the meter and sense require it. \textcite{FinnurEdda}\ suggests \emph{gløggvan} ‘miserly, stingy’, giving a litotes ‘so not stingy’, i.e., ‘so generous’.}}, &
\ind at \alst{l}ęið sé \alst{l}aun, ef þegi.\eva

\bvb I found not a generous man, or one so \inx[C]{good of meat}, that a gift were not accepted; or one of his \inx[C]{fee} so not [...], that the rewards were loathed, if he accepted [them].\footnoteB{No man is so generous that he would refuse a gift presented to him, nor loathe receiving a favour as thanks for his generosity.}\evb
\evg


\bvg
\bva \alst{F}éar síns, \hld\ es \alst{f}ęngit hęfr, &
\ind skyli-t maðr \alst{þ}ǫrf \alst{þ}ola; &
opt sparir \alst{l}ęiðum \hld\ þat’s hęfr \alst{l}júfum hugat; &
\ind mart gęngr \alst{v}err an \alst{v}arir.\eva

\bvb Of his own \inx[C]{fee}, which he has earned, should man not suffer need. Oft one saves for the loathed what was meant for the loved; many a thing goes worse than one expects.\evb
\evg


\bvg
\bva \alst{V}ǫ́pnum ok \alst{v}ǫ́ðum \hld\ skulu \alst{v}inir glęðjask; &
\ind þat ’s á \alst{s}jǫlfum \alst{s}ýnst; &
\alst{v}iðrgefęndr ok ęndrgefęndr \hld\ erusk \alst{v}inir lęngst, &
\ind ef þat bíðr at \alst{v}erða \alst{v}ęl.\eva

\bvb With weapons and garments shall friends gladden each other; that is most seen on oneself.\footnoteB{i.e. in one’s own lived experience.} Mutual givers and return-givers are friends for the longest, if it\footnoteB{The friendship.} is to last long.\evb
\evg


\bvg
\bva \alst{V}in sínum \hld\ skal maðr \alst{v}inr \alst{v}esa, &
\ind ok \alst{g}jalda \alst{g}jǫf við \alst{g}jǫf; &
\alst{h}látr við \alst{h}látri \hld\ skyli \alst{h}ǫlðar taka, &
\ind ęn \alst{l}ausung við \alst{l}ygi.\eva

\bvb With his friend shall man be a friend, and reward gift against gift; laughter against laughter should men take, but duplicity against lie.\evb
\evg


\bvg
\bva \alst{V}in sínum \hld\ skal maðr \alst{v}inr vesa, &
\ind \alst{þ}ęim ok \alst{þ}ess vin; &
ęn \alst{ó}vinar síns \hld\ skyli \alst{ę}ngi maðr &
\ind \alst{v}inar \alst{v}inr \alst{v}esa.\eva

\bvb With his friend shall man be a friend, with him and his friend; but with his enemy’s, should no man, friend’s friend be.\evb
\evg


\bvg
\bva \alst{V}ęizt, ef \alst{v}in átt, \hld\ þann’s \alst{v}ęl trúir &
\ind ok vilt af hǫ́num \alst{g}ótt \alst{g}eta, &
\alst{g}ęði skalt við þann \hld\ ok \alst{g}jǫfum skipta, &
\ind \alst{f}ara at \alst{f}inna opt.\eva

\bvb Know, if thou have a friend, one on which thou well trust, and wilt receive good from: mind and gifts shalt thou share with him; journey to find him oft.\footnoteB{This verse is closely related to 117, which seems like an abridged version of this one.}\evb
\evg


\bvg
\bva Ef þú \alst{á}tt \alst{a}nnan, \hld\ þann’s þú \alst{i}lla trúir, &
\ind vilt af hǫ́num þó \alst{g}ótt \alst{g}eta, &
\alst{f}agrt skalt mę́la, \hld\ ęn \alst{f}látt hyggja &
\ind ok gjalda \alst{l}ausung við \alst{l}ygi.\eva

\bvb If thou have another, one on which thou badly trust, and wilt yet receive good from: fairly shalt thou speak, but falsely think, and pay duplicity against lie.\evb
\evg


\bvg
\bva Þat ’s \alst{ę}nn umb þann, \hld\ es þú \alst{i}lla trúir &
\ind ok þér es \alst{g}runr at \alst{g}ęði, &
\alst{h}lę́ja skalt við þęim \hld\ ok of \alst{h}ug mę́la; &
\ind \alst{g}lík skulu \alst{g}jǫld \alst{g}jǫfum.\eva

\bvb ’Tis yet regarding that one, on which thou badly trustest, and who causes thy senses doubt:\footnoteB{lit. “and for thee is doubt in senses”.} laugh shalt thou with him, and speak with care; rewards shall be equal to gifts.\footnoteB{Equivalent to the last line of the previous v. (“reward duplicity against lie”).}\evb
\evg


\bvg
\bva Ungr vas’k \alst{f}orðum, \hld\ \alst{f}ór’k ęinn saman, &
\ind þá varð’k \alst{v}illr \alst{v}ega; &
\alst{au}ðigr þóttumk, \hld\ es \alst{a}nnan fann’k, &
\ind \alst{m}aðr es \alst{m}anns gaman.\eva

\bvb Young was I once, I travelled alone; then I became lost about the ways. Wealthy I thought myself when another one I found; man is the pleasure of man.\evb
\evg


\bvg
\bva \alst{M}ildir frǿknir \hld\ \alst{m}ęnn bazt lifa, &
\ind \alst{s}jaldan \alst{s}út ala; &
\alst{ó}snjallr maðr \hld\ \alst{u}ggir hvatvetna, &
\ind sýtir ę́ \alst{g}løggr við \alst{g}jǫfum.\eva

\bvb Generous, bold men live the best; seldom they nourish grief. The unvalorous man is frightened by whatever; ever the stingy man grieves a gifts.\footnoteB{Refer back to v. 39; after receiving a gift, one was culturally obliged to give something back.}\evb
\evg


\bvg
\bva \alst{V}áðir mínar \hld\ gaf’k \alst{v}ęlli at &
\ind \alst{t}vęim \alst{t}rémǫnnum; &
\alst{r}ekkar þat þóttusk, \hld\ es \alst{r}ipt hǫfðu; &
\ind \alst{n}ęiss es \alst{n}ǫkkviðr halr.\eva

\bvb My garments I gave in the field, to two tree-men. Champions they seemed when cloaks they had; shameful is the naked warrior.\footnoteB{One of the hardest verses in the poem. After much thought I consider the probable sense to be that the clothes make the warrior; under expensive gear a thin tree-man might be hiding, and likewise even a strong man (I see the choice of the word \emph{halr} ‘warrior’ rather than the more neutral \emph{maðr} ‘man, person’ as intentional) when naked and facing a heavily armoured opponent becomes as vulnerable as the ‘tree-man’ on a plain.}\evb
\evg


\bvg
\bva Hrørnar \alst{þ}ǫll, \hld\ sú’s stęndr \alst{þ}orpi á, &
\ind hlýrat hęnni \alst{b}ǫrkr né \alst{b}arr; &
svá es \alst{m}aðr, \hld\ sá’s \alst{m}anngi ann; &
\ind hvat skal hann \alst{l}ęngi \alst{l}ifa?\eva

\bvb Wilters the pine that stands on the yard; shields her not bark nor needle. So is the man who loves none; for what shall he live for long?\evb
\evg


\bvg
\bva \alst{Ę}ldi hęitari \hld\ brinnr með \alst{i}llum vinum &
\ind \alst{f}riðr \alst{f}imm daga, &
ęn þá \alst{sl}oknar, \hld\ es hinn \alst{s}étti kømr, &
\ind ok \alst{v}ersnar allr \alst{v}inskapr.\eva

\bvb Hotter than fire burns peace among bad friends, for \inx[C]{five days};\footnoteB{A reference to the five-day week (see also v. 74); the number is symbolic. See further Encyclopedia.} but then goes out when the sixth one comes, and all the friendship worsens.\evb
\evg


\bvg
\bva \alst{M}ikit ęitt \hld\ skal-a \alst{m}anni gefa; &
\ind opt kaupir sér í \alst{l}ítlu \alst{l}of, &
með \alst{h}ǫlfum \alst{h}lęif \hld\ ok með \alst{h}ǫllu kęri &
\ind \alst{f}ekk ek mér \alst{f}élaga.\eva

\bvb Much at once shall one not give a man; oft one buys oneself praise for little. With half a loaf and an awry cask, I got me a companion.\evb
\evg


\bvg
\bva \alst{L}ítilla sanda, \hld\ \alst{l}ítilla sę́va, &
\ind lítil eru \alst{g}ęð \alst{g}uma; &
því’t \alst{a}llir męnn \hld\ \alst{u}rðu-t jafnspakir; &
\ind \alst{h}ǫlf es ǫld \alst{h}var.\eva

\bvb Of small sands, of small seas; small are the senses of man. For all have not become evenly knowing; half is every man.\footnoteB{The genitive “of small sands, of small seas” is probably a partitive; man’s horizons are small, the universe is far greater than he, and always will be. On the meaning of the second half of the verse I find that of \textcite{Athugasemdir1929} most convincing, namely that everybody has both strengths and weaknesses. As nobody can excel at everything, nobody is complete; every person is half. This fits particularly closely with v. 71 and 131.}\evb
\evg


\bvg
\bva \alst{M}eðalsnotr \hld\ skyli \alst{m}anna hvęrr, &
\ind ę́va til \alst{s}notr \alst{s}é; &
þęim es \alst{f}yrða \hld\ \alst{f}ęgrst at lifa, &
\ind es \alst{v}ęl mart \alst{v}itu.\eva

\bvb Middle-clever should each man be; never too clever. For those men ’tis fairest to live, who know well enough.\evb
\evg


\bvg
\bva \alst{M}eðalsnotr \hld\ skyli \alst{m}anna hvęrr, &
\ind ę́va til \alst{s}notr \alst{s}é; &
\alst{s}notrs manns hjarta \hld\ verðr \alst{s}jaldan glatt, &
\ind ef sá ’s \alst{a}lsnotr es \alst{á}.\eva

\bvb Middle-clever should each man be; never too clever. The clever man’s heart is seldom gladdened, if he is all-clever that owns [it].\evb
\evg


\bvg
\bva \alst{M}eðalsnotr \hld\ skyli \alst{m}anna hvęrr, &
\ind ę́va til \alst{s}notr \alst{s}é; &
\alst{ø}rlǫg sín \hld\ viti \alst{ę}ngi fyr; &
\ind þęim es \alst{s}orgalausastr \alst{s}efi.\eva

\bvb Middle-clever should each man be; never too clever. His own \inx[C]{orlay} ought none to know ahead; his is the most sorrowless mind.\footnoteB{Who knows not his fate. It is fitting that Weden would say this, having knowledge of the inevitable destruction of the world and hisself.}\evb
\evg


\bvg
\bva \alst{B}randr af \alst{b}randi \hld\ \alst{b}rinnr unz \alst{b}runninn es, &
\ind \alst{f}uni kvęykisk af \alst{f}una; &
\alst{m}aðr af \alst{m}anni \hld\ verðr at \alst{m}áli kuðr; &
\ind ęn til \alst{d}ǿlskr af \alst{d}ul.\eva

\bvb Fire by fire burns until it burnt is; flame is kindled from flame. Man by man becomes known for speech, but the too dull by his delusion.\evb
\evg


\bvg
\bva \alst{Á}r skal rísa, \hld\ sá’s \alst{a}nnars vill &
\ind \alst{f}é eða \alst{f}jǫr hafa; &
sjaldan \alst{l}iggjandi ulfr \hld\ \alst{l}ę́r of getr, &
\ind né \alst{s}ofandi maðr \alst{s}igr.\eva

\bvb Early shall rise he who another’s \inx[C]{fee} or life will have. Seldom does the lying wolf get a thigh, or the sleeping man victory.\evb
\evg


\bvg
\bva \alst{Á}r skal rísa, \hld\ sá’s á \alst{y}rkjęndr fáa, &
\ind ok ganga síns \alst{v}erka á \alst{v}it; &
\alst{m}art of dvęlr \hld\ þann’s umb \alst{m}orgin sefr, &
\ind \alst{h}alfr es auðr und \alst{h}vǫtum.\eva

\bvb Early shall rise he who owns workers few, and go his work to meet. Much is kept back from him who in the morning sleeps; half the wealth is due to the brisk.\footnoteB{Half of a man’s wealth is due to his briskness.}\evb
\evg


\bvg
\bva \alst{Þ}urra skíða \hld\ ok \alst{þ}akinna nę́fra, &
\ind þess kann \alst{m}aðr \alst{m}jǫt, &
ok þess \alst{v}iðar, \hld\ es \alst{v}innask męgi &
\ind \alst{m}ál ok \alst{m}issęri.\eva

\bvb Of dry planks and of thatching birch bark: thereof man knows the measure—and of that firewood which may be used for a season and half-year.\footnoteB{Over the winter.}\evb
\evg


\bvg
\bva \alst{Þ}vęginn ok męttr \hld\ ríði maðr \alst{þ}ingi at, &
\ind þótt hann sé-t \alst{v}ę́ddr til \alst{v}ęl; &
\alst{sk}úa ok bróka \hld\ \alst{sk}ammisk ęngi maðr &
\ind né \alst{h}ęsts in \alst{h}ęldr, &
\ind \edtext{þótt hann \alst{h}afi’t góðan}{\lemma{þótt \dots\ góðan “although \dots\ good one”}\Bfootnote{As \textcite{FinnurEdda}\ points out, surely a late insertion. Whoever made it was not aware of the rules of the \Ljodahattr, interpreting the c-line as a \Fornyrdislag\ a-line, and then insreting the supposed b-line.}}.\eva

\bvb Washed and filled ought man to ride to the Thing, although he might not be dressed too well; of his shoes and breeches ought no man to be ashamed, nor indeed of his horse, (although he might not have a good one.)\evb
\evg


\bvg
\bva \alst{S}napir ok gnapir, \hld\ es til \alst{s}ę́var kømr, &
\ind \alst{ǫ}rn á \alst{a}ldinn mar; &
svá es \alst{m}aðr, \hld\ es með \alst{m}ǫrgum kømr &
\ind ok á \alst{f}ormę́lęndr \alst{f}áa.\eva

\bvb Shuffles and stoops—when to the sea it comes—the eagle on the aged ocean. So is the man, as among the many comes, and has spokesmen few.\footnoteB{Cf. v. 25.}\evb
\evg


\bvg
\bva \alst{F}regna ok sęgja \hld\ skal \alst{f}róðra hvęrr, &
\ind sá’s vill \alst{h}ęitinn \alst{h}orskr; &
\alst{ęi}nn vita \hld\ né \alst{a}nnarr skal, &
\ind \alst{þ}jóð vęit ef \alst{þ}rír ’ró.\eva

\bvb Ask and speak shall each learned man, who wishes to be called sharp; one shall know, but not another: thirty\footnoteB{\emph{þjóð} lit. ‘people, nation’; cf. \Skaldskaparmal\ (TODO): \emph{þjóð eru þrír tigir} “thirty are a \emph{people}”.} know if there are three.\evb
\evg


\bvg
\bva \alst{R}íki sitt \hld\ skyli \alst{r}áðsnotra &
\ind hvęrr í \alst{h}ófi \alst{h}afa; &
þá hann þat \alst{f}innr, \hld\ es með \alst{f}rǿknum kømr, &
\ind at \alst{ę}ngi es \alst{ęi}nna hvatastr.\eva

\bvb His power should each counsel-clever man use in moderation; then he finds it—when among the bold he comes—that none is the briskest of all.\footnoteB{i.e., every man has his match. For the expression compare particularly \VolsungaSaga\ TODO \emph{þviat hverr sa, er med maurgum kemr, ma þat finna eitthvert sinn, at einge er einna hvataztr} “for each one who comes among the many must at some point find that none is the briskest of all.”}\evb
\evg


\bvg
\bva \alst{O}rða þęira, \hld\ es maðr \alst{ǫ}ðrum sęgir, &
\ind opt hann \alst{g}jǫld of \alst{g}etr.\eva

\bvb For those words which man to another says, he oft gets recompense.\evb
\evg


\bvg
\bva \alst{M}ikilsti snimma \hld\ kom’k í \alst{m}arga staði, &
\ind ęn til \alst{s}íð í \alst{s}uma; &
\alst{ǫ}l vas drukkit, \hld\ sumt vas \alst{ó}lagat; &
\ind sjaldan hittir \alst{l}ęiðr í \alst{l}ið.\eva

\bvb Much too early I came to many places, and too late to some. The ale was drunk, at other times yet unbrewed;\footnoteB{lit. “some [of it] was unbrewed”} seldom finds the loathsome man his place.\evb
\evg


\bvg
\bva \alst{H}ér ok \alst{h}var \hld\ myndi mér \alst{h}ęim of boðit, &
\ind ef þyrpta’k at \alst{m}ǫ́lungi \alst{m}at, &
eða \alst{t}vau lę́r hęngi \hld\ at hins \alst{t}ryggva vinar, &
\ind þar’s ek hafða \alst{ęi}tt \alst{e}tit.\eva

\bvb Here and there would I to a home be invited, if at no meal-time I needed food; or [if] two hams would hang at the trusty friend’s [home], where I one had eaten.\footnoteB{Not everyone is hospitable, especially with regards to food, which was valuable and had to be closely counted among subsistence farmers. The poet notes that even a “trusty friend” (might be sarcastic) would invite him to eat at his house more often if he brought more food than he ate.}\evb
\evg


\bvg
\bva \alst{Ę}ldr es baztr \hld\ með \alst{ý}ta sonum &
\ind ok \alst{s}ólar \alst{s}ýn, &
\alst{h}ęilyndi sitt, \hld\ ef \alst{h}afa náir, &
\ind án við \alst{l}ǫst at \alst{l}ifa.\eva

\bvb Fire is best among the sons of men, and the sight of the sun; one’s good health—if thou manage to keep it—and living without vice.\evb
\evg


\bvg
\bva Es-at maðr \alst{a}lls vesall, \hld\ þótt sé \alst{i}lla hęill, &
\ind \alst{s}umr es af \alst{s}onum \alst{s}ę́ll, &
\alst{s}umr af frę́ndum, \hld\ \alst{s}umr af fé ǿrnu, &
\ind sumr af \alst{v}erkum \alst{v}ęl.\eva

\bvb Man is not all wretched, though he of poor health be: someone is blessed by sons, someone by kinsmen, someone by ample \inx[C]{fee}, someone by works done well.\evb
\evg


\bvg
\bva Bętra ’s \alst{l}ifðum, \hld\ ok sę́l\alst{l}ifðum, &
\ind ęy getr \alst{k}vikr \alst{k}ú; &
\alst{ę}ld sá’k \alst{u}pp brinna \hld\ \alst{au}ðgum manni fyr, &
\ind ęn úti vas \alst{d}auðr fyr \alst{d}urum.\eva

\bvb ’Tis better with the living, and the blessed living: ever gets the quick\footnoteB{i.e. the living.} a cow.\footnoteB{A reference to the cattle-based economy (see also v. 76), the cow being used as a metonym. The meaning is that new opportunities always present themselves.} A fire\footnoteB{His funeral-pyre.} I saw burning high for a wealthy man, but outside he was dead before the door.\evb
\evg


\bvg
\bva \alst{H}altr ríðr \alst{h}rossi, \hld\ \alst{h}jǫrð rekr \alst{h}andarvanr, &
\ind daufr \alst{v}egr ok \alst{d}ugir; &
\alst{b}lindr es \alst{b}ętri, \hld\ an \alst{b}ręndr séi; &
\ind \alst{n}ýtr manngi \alst{n}ás.\eva

\bvb A halt man rides a horse; a handless drives a herd; a deaf fights and avails. Blind is better than be burnt; no man has use for a corpse.\evb
\evg


\bvg
\bva \alst{S}onr es bętri, \hld\ þótt sé \alst{s}íð of alinn &
\ind ęptir \alst{g}inginn \alst{g}uma; &
sjaldan \alst{b}autarstęinar \hld\ standa \alst{b}rautu nę́r, &
\ind nema ręisi \alst{n}iðr at \alst{n}ið.\eva

\bvb A son is better, although he late be born after a passed-on man\footnoteB{i.e. after the father is dead.}; seldom beat-stones\footnoteB{Large menhirs raised as memorial stones, later and especially in Upland decorated with Runic inscriptions.} near the highway stand, save by kinsman after kinsman raised.\evb
\evg


\bvg
\bva \edtext{\alst{T}vęir ’ru ęins hęrjar, \hld\ \alst{t}unga es hǫfuðs bani; &
mér ’s í \alst{h}eðin \alst{h}vęrn \hld\ \alst{h}andar vę́ni.}{\lemma{Tvęir \dots\ vę́ni}\Bfootnote{Whole v. undoubtedly a later insertion, the divergent meter is proof enough.}}\eva

\bvb Two are of one host;\footnoteB{\emph{hęrjar} gen. sg. of \emph{hęrr} ‘host, army’ may alternatively be read as the nom. pl. meaning ‘harriers, raiders,’ present in \emph{ęinhęrjar} (\inx[G]{Ownharriers}). Thus ‘two are the destroyers of one (i.e. the person)’.} the tongue is the head’s bane;\footnoteB{The tongue and the head are part of the same body and need each other, yet the former often leads to the demise of the latter. — For this phrase cf. especially the Old Swedish Heathen Law \parencite{Läffler1879}: \emph{Faldr þan orð havr giuit · Glöpr orða værstr · Tunga houuðbani · Liggi i vgildum acri} “Falls the one who has given the word—wickedness is the worst of words; the tongue the head’s bane-man—may he lie in an unpaid field (i.e. no weregild will be paid for him).”} in every cloak I expect a hand.\evb
\evg


\bvg
\bva \alst{N}ǫ́tt verðr fęginn, \hld\ sá’s \alst{n}esti trúir, &
\ind \alst{sk}ammar ’ru \alst{sk}ips ráar, &
\ind \alst{h}verf es \alst{h}austgríma; &
\alst{f}jǫlð of viðrir \hld\ á \alst{f}imm dǫgum, &
\ind ęn \alst{m}ęir á \alst{m}ánaði.\eva

\bvb At night he rejoices, who trusts on his provisions; short are the ship’s sailyards;\footnoteB{TODO: Write about the varying interpretations (Finnur, Cleasby, Skp) of this line.} ever-shifting is the autumn night. The weather shifts much in \inx[C]{five days},\footnoteB{See note to v. 51 and Encyclopedia.} but more in a month.\evb
\evg


\bvg
\bva \alst{V}ęit-a hinn, \hld\ es \alst{v}ę́tki \alst{v}ęit, &
\ind margr verðr \edtext{af \alst{au}rum}{\lemma{af aurum}\Afootnote{‘aflꜹðrom’ \emph{ms.}}} \alst{a}pi; &
maðr es \alst{au}ðigr, \hld\ annarr \alst{ó}auðigr, &
\ind skyli-t þann \alst{v}ítka \alst{v}áar.\eva

\bvb The one knows not, who nothing knows: many a man becomes by treasures the fool.\footnoteB{For \emph{api}, here “fool”, see \inx[C]{ape}.} A man is wealthy, another not wealthy; one oughtn’t to curse him for his woe.\evb
\evg


\bvg
\bva \alst{D}ęyr fé, \hld\ \alst{d}ęyja frę́ndr, &
\ind dęyr \alst{s}jalfr hit \alst{s}ama; &
ęn \alst{o}rðstírr \hld\ dęyr \alst{a}ldrigi &
\ind hvęim’s sér \alst{g}óðan \alst{g}etr.\eva

\bvb \inx[C]{fee}[Fee] dies, kinsmen die, oneself dies the same;\footnoteB{The power of this succinct merism may be less clear to the modern reader. In Germanic Iron Age society a man’s wealth was reckoned by how many heads of cattle (for which compare particularly English \emph{chattel} ‘tangible, movable property’ and the etymology of \emph{capital}) he owned, and his social power by the number of able male relatives ready to side with him in conflict. The meaning is thus: all your power will pass away, and so too must you. — For poetic analogues, see \textcite[99\psqq]{West2007}.} but a word-glory never dies, for whomever gets himself a good one.\evb
\evg


\bvg
\bva \alst{D}ęyr fé, \hld\ \alst{d}ęyja frę́ndr, &
\ind dęyr \alst{s}jalfr hit \alst{s}ama; &
\alst{e}k vęit \alst{ęi}nn \hld\ at \alst{a}ldrigi dęyr: &
\ind \alst{d}ómr of \alst{d}auðan hvęrn.\eva

\bvb Fee dies, kinsmen die, oneself dies the same. I know one that never dies: the \inx[C]{Doom} o’er each man dead.\evb
\evg


\bvg
\bva \alst{F}ullar grindr \hld\ sá’k fyr \alst{F}itjungs sonum, &
\ind nú bera þęir \alst{v}ánar \alst{v}ǫl; &
svá es \alst{au}ðr \hld\ sęm \alst{au}gabragð, &
\ind hann es \alst{v}altastr \alst{v}ina.\eva

\bvb Full pens I saw for the sons of Fitting; now they carry the staff of hope.\footnoteB{A beggar’s staff.} So is wealth like the twinkling of an eye; it is the ficklest of friends.\evb
\evg


\bvg
\bva \alst{Ó}snotr maðr, \hld\ es \alst{ęi}gnask getr &
\ind \alst{f}é eða \alst{f}ljóðs munuð; &
\alst{m}etnaðr hǫ́num þróask, \hld\ ęn \alst{m}anvit aldrigi; &
\ind framm gęngr hann \alst{d}rjúgt í \alst{d}ul.\eva

\bvb The unclever man, if he gets to own fee or a girl’s grace: his conceit flourishes, but never his manwit; far he goes forth in delusion.\evb
\evg

\sectionline

\section{A stand-alone insert verse. It would fit better later on.}

\bvg
\bva Þat es þá \alst{r}ęynt, \hld\ es þú at \alst{r}únum spyrr \hld\ \edtext{hinum \alst{r}ęginkunnum}{\lemma{hinum ręginkunnum ‘the ones born of the Reins’}\Bfootnote{This expression also appears on the Noleby stone. TODO}}, &
\ind þęim’s \alst{g}ęrðu \alst{g}innręgin &
\ind ok \alst{f}áði \alst{f}imbulþulr; &
\ind þá hęfr hann bazt, ef hann þęgir.\eva

\bvb Then that is proven of which thou inquires the runes, the ones born of the Reins, those which the \inx[G]{gin-Reins} made, and the Fimblethyle \name{= Weden} painted. (Then he has it best, if he shuts up.)\evb
\evg

\sectionline

\section{Verses of practical advice, mostly in \Fornyrdislag.}

\bvg
\bva At \alst{k}veldi skal dag lęyfa, \hld\ \alst{k}onu es bręnnd es, &
\alst{m}ę́ki es ręyndr es, \hld\ \alst{m}ęy es gefin es, &
\alst{í}s es \alst{y}fir kømr, \hld\ \alst{ǫ}l es drukkit es.\eva

\bvb At evening shall one praise day, a woman when she is burned, a sword when it is tried, a maiden when she is given,\footnoteB{i.e. in marriage.} ice when one crosses over, ale when it is drunk.\evb
\evg


\bvg
\bva Í \alst{v}indi skal \alst{v}ið hǫggva, \hld\ \alst{v}eðri á sę́ róa, &
\alst{m}yrkri við \alst{m}an spjalla, \hld\ \alst{m}ǫrg eru dags augu, &
á \alst{sk}ip skal \alst{sk}riðar orka, \hld\ ęn á \alst{sk}jǫld til hlífar, &
\alst{m}ę́ki til hǫggs, \hld\ ęn \alst{m}ęy til kossa.\eva

\bvb In wind shall one cut wood, in storm row on the sea, in darkness meet with a maiden; many are the eyes of day. A ship shall one have for its speed, but a shield for shelter; a sword for striking, but a maiden for her kisses.\evb
\evg


\bvg
\bva Við \alst{ę}ld skal \alst{ǫ}l drekka, \hld\ ęn á \alst{í}si skríða, &
\alst{m}agran \alst{m}ar kaupa, \hld\ ęn \alst{m}ę́ki saurgan, &
\alst{h}ęima \alst{h}ęst fęita, \hld\ ęn \alst{h}und á búi.\eva

\bvb By fire shall one drink ale, and on the ice skate; buy a meager stallion, and a rusty sword; fatten the horse at home, and the hound in the household.\evb
\evg


\bvg
\bva \alst{M}ęyjar orðum \hld\ skyli \alst{m}anngi trúa, &
\ind né því’s \alst{k}veðr \alst{k}ona; &
\edtext{\edtext{þvít}{\Afootnote{\emph{om.} \FostrbroedhraSaga}} á \alst{h}verfanda \alst{h}véli \hld\ \edtext{vǫ́ru}{\Afootnote{er \FostrbroedhraSaga}} þęim \edtext{\alst{h}jǫrtu skǫpuð}{\lemma{hjǫrtu skǫpuð}\Afootnote{hjarta skapat \FostrbroedhraSaga}}, &
\ind \edtext{\alst{b}rigð}{\lemma{brigð}\Afootnote{ok brigð \FostrbroedhraSaga}} í \alst{b}rjóst of \edtext{lagið}{\Afootnote{‘laginn’ \FostrbroedhraSaga}}.}{\lemma{þvít \dots\ lagið}\Bfootnote{Quoted in slightly divergent form in \FostrbroedhraSaga\ (Thott 1768 4°\textsuperscript{x}, fol. 210r): \emph{“And then he remembered the ditty which had been composed about loose women: [...]”}}}\eva

\bvb The words of a maiden should no man believe, nor that which a woman sings. For on a spinning wheel were their hearts shaped; fickleness in their breasts was laid.\evb
\evg


\bvg
\bva \alst{B}restanda \alst{b}oga, \hld\ \alst{b}rinnanda loga, &
\alst{g}ínanda ulfi, \hld\ \alst{g}alandi krǫ́ku, &
\alst{r}ýtanda svíni, \hld\ \alst{r}ótlausum viði, &
\alst{v}axanda \alst{v}ági, \hld\ \alst{v}ellanda katli,\eva

\bvb The bursting bow, the burning flame, the gaping wolf, the crowing crow, the roaring swine, the rootless tree, the waxing wave, the swelling kettle,\evb
\evg


\bvg
\bva \alst{f}ljúganda \alst{f}lęini, \hld\ \alst{f}allandi bǫ́ru, &
\alst{í}si \alst{ęi}nnę́ttum, \hld\ \alst{o}rmi hringlęgnum, &
\alst{b}rúðar \alst{b}ęðmǫ́lum \hld\ eða \alst{b}rotnu sverði, &
\alst{b}jarnar lęiki \hld\ eða \alst{b}arni konungs, &
\alst{s}júkum kalfi, \hld\ \alst{s}jalfráða þrę́li, &
\alst{v}ǫlu \alst{v}ilmę́li, \hld\ \alst{v}al nýfęldum.\eva

\bvb the flying spear, the falling billow, the one-night old ice, the coiled-up serpent, the bed-speeches of a bride, or the broken sword, the play of a bear, or the child of a king, the sick calf, the freed slave, the pleasing speech of a wallow, newly felled corpses,\evb
\evg

In \Regius the following two verses come in the opposite order, but it is clear that 88 should conclude the old list of things not to trust. It is clear from its meter that 87 is a separate composition; it was probably inserted in between 86 and 88 by an inattentive scribe.

\bvg
\bva[88]\alst{b}róðurbana sínum \hld\ þótt á \alst{b}rautu mǿti, &
\alst{h}úsi \alst{h}alfbrunnu, \hld\ \alst{h}ęsti alskjótum, &
þá ’s \alst{jó}r \alst{ó}nýtr, \hld\ ef \alst{ęi}nn fótr brotnar; &
verðr-it maðr svá \alst{t}ryggr \hld\ at þessu \alst{t}rúi ǫllu.\eva

\bvb his brother’s bane-man—though on the highway they meet—a half-burned house, an all-fleet horse: then is the steed useless, if one foot breaks. There may be no man so trusting, that he trust in all this.\evb
\evg\stepcounter{stanza}


\bvg
\bva[87]\alst{A}kri \alst{á}rsǫ́num \hld\ trúi \alst{ę}ngi maðr, &
\ind né til \alst{s}nimma \alst{s}yni; &
\alst{v}eðr rę́ðr akri, \hld\ ęn \alst{v}it syni; &
\ind \alst{h}ę́tt es þęira \alst{h}várt.\eva

\bvb An early sown field ought no man to trust, nor too early\footnoteB{i.e. in life.} a son. The weather rules the field, but the wits the son; there is risk to both of them.\evb
\evg\stepcounter{stanza}

\sectionline

\section{Advice on love and Weden’s failed seduction of Billing’s maiden.}

\bvg
\bva Svá ’s \alst{f}riðr kvinna \hld\ þęira’s \alst{f}látt hyggja, &
sęm aki \alst{jó} \alst{ó}bryddum \hld\ á \alst{í}si hǫ́lum &
\alst{t}ęitum, \alst{t}vévetrum \hld\ ok sé \alst{t}amr illa, &
eða í \alst{b}yr óðum \hld\ \alst{b}ęiti stjórnlausu, &
eða skyli \alst{h}altr \alst{h}ęnda \hld\ \alst{h}ręin í þáfjalli.\eva

\bvb So is the peace of women—those who falsely think—like one rode an unshod horse on slippery ice—a merry one, two winters old, and badly tamed; or in mad wind tacked a rudderless [ship], or [as] should a halt man catch a reindeer on a thawing mountain.\evb
\evg


\bvg
\bva \alst{B}ert nú mę́li’k, \hld\ því-at \alst{b}ę́ði vęit’k, &
\ind brigðr es \alst{k}arla hugr \alst{k}onum, &
þá \alst{f}ęgrst mę́lum, \hld\ es \alst{f}lást hyggjum; &
\ind þat tę́lir \alst{h}orska \alst{h}ugi.\eva

\bvb Plainly I now speak, for I know both [sides]: fickle is men’s thought towards women. We then speak the most fairly, when the most falsely we think; that entices sharp minds.\evb
\evg


\bvg
\bva \alst{F}agrt skal mę́la \hld\ ok \alst{f}é bjóða, &
\ind sá’s vill \alst{f}ljóðs ǫ́st \alst{f}áa, &
\alst{l}íki \alst{l}ęyfa \hld\ hins \alst{l}jósa mans, &
\ind sá \alst{f}ę́r, es \alst{f}ríar.\eva

\bvb Fairly shall speak, and offer \inx[C]{fee}, he who will earn a girl’s love; [he shall] praise the body of the light maiden; he gets, who woos.\footnoteB{That is, ‘he who woos her gets her’.}\evb
\evg


\bvg
\bva \alst{Á}star firna \hld\ skyli \alst{ę}ngi maðr &
\ind \alst{a}nnan \alst{a}ldrigi; &
opt fáa á \alst{h}orskan, \hld\ es á \alst{h}ęimskan né fáa, &
\ind \alst{l}ostfagrir \alst{l}itir.\eva

\bvb For [his] love should no man ever blame another; oft lust-fair looks seize the sharp one, when they seize not the foolish one.\evb
\evg


\bvg
\bva \alst{Ęy}vitar firna, \hld\ es maðr \alst{a}nnan skal, &
\ind þess es of margan \alst{g}ęngr \alst{g}uma; &
\alst{h}ęimska ór \alst{h}orskum \hld\ gęrir \alst{h}ǫlða sonu &
\ind sá hinn \alst{m}átki \alst{m}unr.\eva

\bvb For nothing shall man ever blame another, which happens to many a man; fools out of sharp ones makes—among the sons of men—that mighty delight \ken{love}.\evb
\evg


\bvg
\bva \alst{H}ugr ęinn þat vęit, \hld\ es býr \alst{h}jarta nę́r, &
\ind ęinn es hann \alst{s}ér of \alst{s}efa; &
øng es \alst{s}ótt verri \hld\ hvęim \alst{s}notrum manni &
\ind an sér \alst{ø}ngu at \alst{u}na.\eva

\bvb The thought alone knows what dwells close to the heart; he is alone with his mind. No ailment is worse for any clever man, than to be content with nothing.\evb
\evg


\bvg
\bva Þat þá \alst{r}ęyndak, \hld\ es í \alst{r}ęyri sat’k, &
\ind ok vę́tta’k \alst{m}íns \alst{m}unar, &
\alst{h}old ok \alst{h}jarta \hld\ vas mér hin \alst{h}orska mę́r, &
\ind þęygi hana at \alst{h}ęldr \alst{h}ęf’k.\eva

\bvb That I then discovered, as I sat in the reed, and awaited my pleasure. My flesh and heart that sharp maiden was; I have her none the more.\evb
\evg


\bvg
\bva \alst{B}illings męy \hld\ ek fann \alst{b}ęðjum á &
\ind \alst{s}ólhvíta \alst{s}ofa; &
\alst{ja}rls \alst{y}nði \hld\ þótti mér \alst{ę}kki vesa &
\ind nema við þat \alst{l}ík at \alst{l}ifa.\eva

\bvb Billing’s maiden I found on the beds, sun-white, sleeping. An earl’s pleasure seemed me naught to be, save for living alongside that body.\evb
\evg


\bvg {\small [Billing’s maiden:]}
\bva „\alst{Au}k nę́r \alst{a}ptni \hld\ skalt-u \alst{Ó}ðinn koma, &
\ind ef vilt þér \alst{m}ę́la \alst{m}an, &
\alst{a}lt eru \alst{ó}skǫp, \hld\ nema \alst{ęi}n vitim &
\ind \alst{s}likan lǫst \alst{s}aman.“\eva

\bvb “And by evening shalt thou, Weden, come, if thou wilt for thee have the maiden \ken*{= me}; all is misshapen, if we might not know one such vice together.”\evb
\evg


\bvg
\bva \alst{A}ptr ek hvarf \hld\ ok \alst{u}nna þóttumk &
\ind \alst{v}ísum \alst{v}ilja frá; &
\alst{h}itt ek \alst{h}ugða, \hld\ at \alst{h}afa mynda’k &
\ind \alst{g}ęð hęnnar alt ok \alst{g}aman.\eva

\bvb Back I turned—and thought myself to love [her]—away from my wise will; this I thought, that I would own her senses all and pleasure.\evb
\evg


\bvg
\bva Svá kom’k \alst{n}ę́st, \hld\ at hin \alst{n}ýta vas &
\ind \alst{v}ígdrótt ǫll of \alst{v}akin; &
með \alst{b}rinnǫndum ljósum \hld\ ok \alst{b}ornum viði, &
\ind svá vas mér \alst{v}ílstígr of \alst{v}itaðr.\eva

\bvb So I came next, as was the useful\footnoteB{Sarcastic.} battle-people all awake; with burnings lights and carried wood;\footnoteB{They were presumably armed with sticks.} so was for me a miserable path\footnoteB{Ambiguous whether it refers to the beating he would have received at the hands of the men had he entered, or to his walk of shame away from the hall.} marked out.\evb
\evg


\bvg
\bva \alst{Au}k nę́r morni, \hld\ es vas’k \alst{ę}nn of kominn, &
\ind þá vas \alst{s}aldrótt of \alst{s}ofin; &
\alst{g}ręy ęitt þá fann’k \hld\ hinnar \alst{g}óðu konu &
\ind \alst{b}undit \alst{b}ęðjum á.\eva

\bvb And by morning, when I was come again, then was the hall-people asleep. A lone bitch I then found, owned by the good woman, bound on the beds.\evb
\evg


\bvg
\bva Mǫrg es \alst{g}óð mę́r, \hld\ ef \alst{g}ǫrva kannar, &
\ind \alst{h}ugbrigð við \alst{h}ali; &
þá þat \alst{r}ęynda’k, \hld\ es hit \alst{r}áðspaka &
\ind tęygða’k á \alst{f}lę́rðir \alst{f}ljóð. &
\alst{h}ǫ́ðungar \alst{h}vęrrar \hld\ lęitaði mér hit \alst{h}orska man &
\ind ok hafða’k þess \alst{v}ę́tki \alst{v}ífs.\eva

\bvb Many a good maiden—if one knows her clearly—is heart-fickle towards men; that I learned when into sins I lured that counsel-clever woman. All sorts of disgraces that sharp girl sought out for me, and I had naught of that wife.\evb
\evg

\sectionline

\section{Weden’s obtaining of the mead of poetry}

This story is told in \Gylfaginning. Weden under the name Baleworker used a drill named \inx[P]{Rate} in order to drill into the mountains. TODO.

\bvg
\bva Hęima \alst{g}laðr \alst{g}umi \hld\ ok við \alst{g}ęsti ręifr, &
\ind \alst{s}viðr skal of \alst{s}ik vesa; &
\alst{m}innigr ok \alst{m}ǫ́lugr, \hld\ ef vill \alst{m}argfróðr vesa; &
\ind opt skal \alst{g}óðs \alst{g}eta; &
\alst{f}imbul\alst{f}ambi hęitir, \hld\ sá’s \alst{f}átt kann sęgja; &
\ind þat es \alst{ó}snotrs \alst{a}ðal.\eva

\bvb At home shall man be glad, and cheerful towards a guest; wise about himself. Remembering and speaking, if he wishes to be many-learned; oft shall he speak of good. A fimble-fool is called he who can say little; that is an unclever man’s nature.\evb
\evg


\bvg
\bva Hinn \alst{a}ldna \alst{jǫ}tun sóttak, \hld\ nú em’k \alst{a}ptr of kominn; &
\ind fátt gat’k \alst{þ}ęgjandi \alst{þ}ar; &
\alst{m}ǫrgum orðum \hld\ \alst{m}ę́lta’k í minn frama &
\ind í \alst{S}uttungs \alst{s}ǫlum.\eva

\bvb The old ettin I sought, now am I come back; I got little silence there. Many words I spoke to my furtherance, in the halls of Sutting.\evb
\evg


\bvg
\bva \alst{G}unnlǫð mér of \alst{g}af \hld\ \alst{g}ollnum stóli á &
\ind \alst{d}rykk hins \alst{d}ýra mjaðar; &
\alst{i}ll \alst{i}ðgjǫld \hld\ lét’k hana \alst{ę}ptir hafa &
\ind síns hins \alst{h}ęila \alst{h}ugar. &
\ind (síns hins \alst{s}vára \alst{s}efa).\eva

\bvb \inx[P]{Guthlathe} did give me, on the golden chair, a drink of the dear mead; evil recompense I let her have afterwards, for her whole heart; for her severe affection.\evb
\evg


\bvg
\bva \alst{R}ata munn \hld\ létumk \alst{r}úms of fáa &
\ind ok of \alst{g}rjót \alst{g}naga; &
\alst{y}fir ok \alst{u}ndir \hld\ stóðumk \alst{jǫ}tna vegir, &
\ind svá \alst{h}ę́tta’k \alst{h}ǫfði til.\eva

\bvb Rate’s mouth I let bring me room, and gnaw away at the rubble. Over and under me stood the roads of the ettins \ken{mountains}; so I risked my head.\evb
\evg


\bvg
\bva \alst{V}ęl kęypts hlutar \hld\ hęf’k \alst{v}ęl notit; &
\ind \alst{f}ás es \alst{f}róðum vant; &
því’t \alst{Ó}ðrerir \hld\ nú \alst{u}pp ’s kominn &
\ind á \alst{a}lda vés \alst{ja}rðar.\eva

\bvb The well purchased thing \ken{mead of poetry} I have used well; little is lacking for the learned—for Woderearer is now come up onto the earths of the \inx[C]{wigh} of men \ken*{Middenyard}.\footnoteB{Weden says that he has made good use of the mead of poetry, since it can now be tapped and served by wise humans.}\evb
\evg


\bvg
\bva \alst{I}fi es mér á, \hld\ at vę́ra’k \alst{ę}nn kominn &
\ind \alst{jǫ}tna gǫrðum \alst{ó}r, &
ef \alst{G}unnlaðar né nyta’k, \hld\ hinnar \alst{g}óðu konu, &
\ind es lǫgðumk \alst{a}rm \alst{y}fir.\eva

\bvb There is doubt in me, that I were still come out of the yards of the Ettins if Guthlathe I had not used: that good woman, whom I laid my arm over.\evb
\evg


\bvg
\bva Hins \alst{h}indra dags \hld\ gingu \alst{h}rímþursar &
\ind \alst{H}áva ráðs at fregna, &
\ind (\alst{H}áva \alst{h}ǫllu í,) &
at \alst{B}ǫlverki spurðu, \hld\ ef vę́ri með \alst{b}ǫndum kominn &
\ind eða hęfði hǫ́num \alst{S}uttungr of \alst{s}óit.\eva

\bvb The other day went the Rime-Thurses to ask for the counsel of the High One; in the hall of High One. About Baleworker \name{= Weden} \ken*{me} they asked, if he \ken*{I} were come among the bonds \name{gods}, or if Suttung had slain him.\evb
\evg


\bvg
\bva Baugęið \alst{Ó}ðinn \hld\ hygg at \alst{u}nnit hafi, &
\ind hvat skal hans \alst{t}ryggðum \alst{t}rúa? &
\alst{S}uttung \alst{s}vikvinn \hld\ hann lét \alst{s}umbli frá &
\ind ok \alst{g}rǿtta \alst{G}unnlǫðu.\eva

\bvb A \inx[C]{bigh-oath} I ween that Weden has sworn; how shall one trust his truces? He let Sutting walk betrayed from the feast, and Guthlathe made to weep.\evb
\evg

\sectionline

\section{The Speeches of Loddfathomer}

\emph{Loddfáfnismǫ́l}. Advice given to Loddfathomer. In \Regius\ this section is marked out with a large initial, like the beginnings of separate poems.

\sectionline

\bvg
\bva Mál ’s at \alst{þ}ylja \hld\ \alst{þ}ular stóli á; &
\ind \alst{U}rðar brunni \alst{a}t &
\alst{s}á’k ok þagða’k, \hld\ \alst{s}á’k ok hugða’k, &
\ind hlýdda’k á \alst{m}anna \alst{m}ál; &
of \alst{r}únar hęyrða’k dǿma, \hld\ né umb \alst{r}ǫ́ðum þǫgðu &
\ind \alst{H}áva \alst{h}ǫllu at, &
\ind \alst{H}áva \alst{h}ǫllu í &
\ind hęyrða’k \alst{s}ęgja \alst{s}vá:\eva

\bvb ’Tis time to \inx[C]{thill}, upon the chair of the \inx[C]{thyle}. At the well of Weird, I saw and I shut up: I saw and I thought: I heeded the matters of men. Of runes I heard them speak, nor about counsels were they silent, at the hall of the High One \name{= Weden} \ken*{= Walhall}, in the hall of the High One, I heard [them] say thus:\footnoteB{The speaker, describing himself as a thyle (\emph{þulr} ‘sage, chanter of memorized poetry’), says that he will relate what he has heard said at the hall of the High One \name{= Weden} \ken*{= Walhall}. Considering the location, it seems almost certain that the giver of this advice was \inx[P]{Weden}. The receiver of the advice, \inx[P]{Loddfathomer} (see Encyclopedia for etymologies), is otherwise unknown.}\evb
\evg


\bvg
\bva \alst{R}ǫ́ðumk þér Loddfáfnir, \hld\ at þú \alst{r}ǫ́ð nemir, &
\ind \alst{n}jóta munt ef \alst{n}emr, &
\ind þér munu \alst{g}óð ef \alst{g}etr: &
\alst{n}ǫ́tt þú rís-at, \hld\ nema á \alst{n}jósn séir, &
\ind eða lęitir þér \alst{i}nnan \alst{ú}t staðar.\eva

\bvb I counsel thee Loddfathomer, that thou learn the counsels; thou wilt benefit if thou learnest; they will be good for thee if thou gettest: At night thou rise not, unless at scouting thou be, or thou art forced out from within a place.\footnoteB{Very difficult phrase. Possibly a euphemism for needing to relieve oneself?}\evb
\evg


\bvg
\bva \alst{R}ǫ́ðumk þér Loddfáfnir, \hld\ at þú \alst{r}ǫ́ð nemir, &
\ind \alst{n}jóta munt ef \alst{n}emr, &
\ind þér munu \alst{g}óð ef \alst{g}etr: &
\alst{f}jǫlkunnigri konu \hld\ skal-at-tu í \alst{f}aðmi sofa, &
\ind svá’t hon \alst{l}yki þik \alst{l}iðum. &
Hón svá \alst{g}ęrir \hld\ at þú \alst{g}áir ęigi &
\ind \alst{þ}ings né \alst{þ}jóðans máls; &
\alst{m}at þú vill-at \hld\ né \alst{m}anskis gaman &
\ind fęrr þú \alst{s}orgafullr at \alst{s}ofa.\eva

\bvb I counsel thee Loddfathomer, that thou learn the counsels; thou wilt benefit if thou learnest; they will be good for thee if thou gettest: In the bosom of a \inx[C]{feal-cunning} woman shalt thou never sleep, so that she might lock you in [her?] limbs. She makes it so that thou heed not the \inx[C]{Thing}, nor the ruler’s speech; food wilt thou not [have], nor any man’s pleasure; thou farest sorrowful to sleep.\evb
\evg


\bvg
\bva \alst{R}ǫ́ðumk þér Loddfáfnir, \hld\ at þú \alst{r}ǫ́ð nemir, &
\ind \alst{n}jóta munt ef \alst{n}emr, &
\ind þér munu \alst{g}óð ef \alst{g}etr: &
\alst{a}nnars konu \hld\ tęyg þér \alst{a}ldrigi &
\ind \alst{ęy}rarúnu \alst{a}t.\eva

\bvb I counsel thee Loddfathomer, that thou learn the counsels; thou wilt benefit if thou learnest; they will be good for thee if thou gettest: Never lure another man’s woman into [becoming] thy ear-whisperer \ken{lover}.\evb
\evg


\bvg
\bva \alst{R}ǫ́ðumk þér Loddfáfnir, \hld\ at þú \alst{r}ǫ́ð nemir, &
\ind \alst{n}jóta munt ef \alst{n}emr, &
\ind þér munu \alst{g}óð ef \alst{g}etr: &
á \alst{f}jalli eða \alst{f}irði, \hld\ ef þik \alst{f}ara tíðir, &
\ind fásk-tu at \alst{v}irði \alst{v}ęl.\eva

\bvb I counsel thee Loddfathomer, that thou learn the counsels; thou wilt benefit if thou learnest; they will be good for thee if thou gettest: on the fell or firth—if thou desire to travel—get thyself a good meal.\evb
\evg


\bvg
\bva \alst{R}ǫ́ðumk þér Loddfáfnir, \hld\ at þú \alst{r}ǫ́ð nemir, &
\ind \alst{n}jóta munt ef \alst{n}emr, &
\ind þér munu \alst{g}óð ef \alst{g}etr: &
\alst{i}llan mann \hld\ lát \alst{a}ldrigi &
\ind \edtext{\alst{ó}hǫpp at þér \alst{v}ita}{\Bfootnote{Excluding some corrpution (but there hardly seems to be any) this line is probably one the few undisputed cases of \emph{v-} alliterating with a vowel.}}. &
af \alst{i}llum manni \hld\ fę́r \alst{a}ldrigi &
\ind \alst{g}jǫld hins \alst{g}óða hugar.\eva

\bvb I counsel thee Loddfathomer, that thou learn the counsels; thou wilt benefit if thou learnest; they will be good for thee if thou gettest: An evil man let thou never know of thy misfortunes. From an evil man receivest thou never recompense for thy good heart.\evb
\evg


\bvg
\bva \alst{O}farla bíta \hld\ sá’k \alst{ęi}num hal &
\ind \alst{o}rð \alst{i}llrar konu, &
\alst{f}lárǫ́ð tunga \hld\ varð hǫ́num at \alst{f}jǫrlagi &
\ind ok þęygi of \alst{s}anna \alst{s}ǫk.\eva

\bvb Biting I saw, high up on one man, the words of an evil woman; a deceit-counseling tongue brought his life to end, and in no way over a truthful charge.\evb
\evg


\bvg
\bva \alst{R}ǫ́ðumk þér Loddfáfnir, \hld\ at þú \alst{r}ǫ́ð nemir, &
\ind \alst{n}jóta munt ef \alst{n}emr, &
\ind þér munu \alst{g}óð ef \alst{g}etr: &
\alst{v}ęizt, ef \alst{v}in átt, \hld\ þann’s \alst{v}ęl trúir, &
\ind \alst{f}ar þú at \alst{f}inna opt; &
\edtext{því’t \alst{h}rísi vęx \hld\ ok \alst{h}ǫ́u grasi}{\lemma{hrísi vęx ok hǫ́u grasi ‘with brushwood and with tall grass grows’}\Bfootnote{Identical with \Grimnismal\ 17/1.}} &
\ind \alst{v}egr, es \alst{v}ę́tki trøðr,\eva

\bvb I counsel thee Loddfathomer, that thou learn the counsels; thou wilt benefit if thou learnest; they will be good for thee if thou gettest: Know, if thou have a friend, one on which thou well trust, journey to find him oft; for with brushwood and tall grass grows the way which no man treads.\evb
\evg


\bvg
\bva \alst{R}ǫ́ðumk þér Loddfáfnir, \hld\ at þú \alst{r}ǫ́ð nemir, &
\ind \alst{n}jóta munt ef \alst{n}emr, &
\ind þér munu \alst{g}óð ef \alst{g}etr: &
\alst{g}óðan mann \hld\ tęyg þér at \alst{g}amanrúnum &
\ind ok nem \alst{l}íknargaldr meðan \alst{l}ifir.\eva

\bvb I counsel thee Loddfathomer, that thou learn the counsels; thou wilt benefit if thou learnest; they will be good for thee if thou gettest: Lure a good man to thee through pleasure-runes,\footnoteB{Pleasurable conversation. Cf. 128.} and learn healing-galders while thou livest.\evb
\evg


\bvg
\bva \alst{R}ǫ́ðumk þér Loddfáfnir, \hld\ at þú \alst{r}ǫ́ð nemir, &
\ind \alst{n}jóta munt ef \alst{n}emr, &
\ind þér munu \alst{g}óð ef \alst{g}etr: &
\alst{v}in þínum \hld\ \alst{v}es aldrigi &
\ind \alst{f}yrri at \alst{f}laumslitum. &
\alst{s}org etr hjarta, \hld\ ef þú \alst{s}ęgja né náir &
\ind \alst{ęi}nhvęrjum \alst{a}llan hug.\eva

\bvb I counsel thee Loddfathomer, that thou learn the counsels; thou wilt benefit if thou learnest; they will be good for thee if thou gettest: With thy friend be thou never the first to tear apart the company. Sorrow eats thy heart if thou cannot speak to anyone thy whole mind.\footnoteB{cf. v. 122.}\evb
\evg


\bvg
\bva \alst{R}ǫ́ðumk þér Loddfáfnir, \hld\ at þú \alst{r}ǫ́ð nemir, &
\ind \alst{n}jóta munt ef \alst{n}emr, &
\ind þér munu \alst{g}óð ef \alst{g}etr: &
orðum \alst{sk}ipta \hld\ \alst{sk}alt aldrigi &
\ind við \alst{ó}svinna \alst{a}pa.\eva

\bvb I counsel thee Loddfathomer, that thou learn the counsels; thou wilt benefit if thou learnest; they will be good for thee if thou gettest: Words shalt thou never exchange with unwise apes.\evb
\evg


\bvg
\bva Því’t af illum \alst{m}anni \hld\ \alst{m}unt aldrigi &
\ind \alst{g}óðs laun of \alst{g}eta, &
ęn \alst{g}óðr maðr \hld\ mun þik \alst{g}ęrva męga &
\ind \edtext{\alst{l}íknfastan}{\lemma{líknfastan ‘health-firm’}\Bfootnote{A cpd. from \emph{líkn} \ONP: ‘mercy, compassion, relief, comfort, help’ and \emph{fastr} ‘fast, firm’. \textcite{LaFargeGlossary} give a tentative ‘assured of favour’, while \CV\ gives ‘fast in goodwill, beloved’. I read it as literally as possible, since the word \emph{líkn} has some connections with healing.}} at \alst{l}ofi.\eva

\bvb For from an evil man wilt thou never get a reward for thy goodness, but a good man will know make thee health-firm by [his] praise.\evb
\evg


\bvg
\bva \alst{S}ifjum ’s þá blandit \hld\ hvęrr es \alst{s}ęgja rę́ðr &
\ind \alst{ęi}num \alst{a}llan hug; &
alt es \alst{b}ętra \hld\ an sé \alst{b}rigðum at vesa: &
\ind es-a sá \alst{v}inr es \alst{v}ilt ęitt sęgir.\eva

\bvb Kinship is then blended,\footnoteB{cf. v. 44.} when any man decides to speak to one man his whole mind. Everything is better than to be among the fickle; he is no friend, who speaks that which is wanted alone.\evb
\evg


\bvg
\bva \alst{R}ǫ́ðumk þér Loddfáfnir, \hld\ at þú \alst{r}ǫ́ð nemir, &
\ind \alst{n}jóta munt ef \alst{n}emr, &
\ind þér munu \alst{g}óð ef \alst{g}etr: &
þrimr orðum sęnna \hld\ skal-at-tu þér við verra mann, &
\ind opt hinn \alst{b}ętri \alst{b}ilar. &
\ind þá’s hinn \alst{v}erri \alst{v}egr.\eva

\bvb I counsel thee Loddfathomer, that thou learn the counsels; thou wilt benefit if thou learnest; they will be good for thee if thou gettest: With three words shalt thou not flyte with a worse man;\footnoteB{i.e. ‘not even with three words’.} oft the better one breaks when the worse one strikes.\evb
\evg


\bvg
\bva \alst{R}ǫ́ðumk þér Loddfáfnir, \hld\ at þú \alst{r}ǫ́ð nemir, &
\ind \alst{n}jóta munt ef \alst{n}emr, &
\ind þér munu \alst{g}óð ef \alst{g}etr: &
\alst{sk}ósmiðr þú verir \hld\ né \alst{sk}ęptismiðr, &
\ind nema \alst{s}jǫlfum þér \alst{s}éir. &
\alst{Sk}ór ’s \alst{sk}apaðr illa\hld\ eða \alst{sk}apt sé vrangt, &
\ind þá ’s þér \alst{b}ǫls \alst{b}eðit.\eva

\bvb I counsel thee Loddfathomer, that thou learn the counsels; thou wilt benefit if thou learnest; they will be good for thee if thou gettest: Thou ought not to be a shoe-maker nor shaft-maker, unless thou be one for thyself. [If] the shoe is shaped badly or the shaft be crooked, then for thee a \inx[C]{bale} is bidden.\footnoteB{i.e. ‘the customer will put a curse you’.}\evb
\evg


\bvg
\bva \alst{R}ǫ́ðumk þér Loddfáfnir, \hld\ at þú \alst{r}ǫ́ð nemir, &
\ind \alst{n}jóta munt ef \alst{n}emr, &
\ind þér munu \alst{g}óð ef \alst{g}etr: &
hvars þú \alst{b}ǫl kant, \hld\ kveð þér \alst{b}ǫlvi at &
\ind ok gefat þínum \alst{f}jǫ́ndum \alst{f}rið.\eva

\bvb I counsel thee Loddfathomer, that thou learn the counsels; thou wilt benefit if thou learnest; they will be good for thee if thou gettest: Where thou a bale knowest, declare it to be a bale, and give not thy enemies peace.\footnoteB{i.e. ‘if somebody puts a curse on you, do not ignore it, but respond forcefully’, though it should be noted that the verse has often been interpreted as a command to call out evil, even when done towards somebody else, and there is nothing in it that goes against that reading.}\evb
\evg


\bvg
\bva \alst{R}ǫ́ðumk þér Loddfáfnir, \hld\ at þú \alst{r}ǫ́ð nemir, &
\ind \alst{n}jóta munt ef \alst{n}emr, &
\ind þér munu \alst{g}óð ef \alst{g}etr: &
\alst{i}llu fęginn \hld\ ves þú \alst{a}ldrigi, &
\ind \edtext{ęn lát þér at \alst{g}óðu \alst{g}etit}{\lemma{ęn lát þér at góðu getit ‘but rather let thyself be pleased by good’}\Bfootnote{This construction is equivalent to the sense ACC. A. IV. in \CV.}}.\eva

\bvb I counsel thee Loddfathomer, that thou learn the counsels; thou wilt benefit if thou learnest; they will be good for thee if thou gettest: Gladdened by evil be thou never, but let thyself be pleased by good.\evb
\evg


\bvg
\bva \alst{R}ǫ́ðumk þér Loddfáfnir, \hld\ at þú \alst{r}ǫ́ð nemir, &
\ind \alst{n}jóta munt ef \alst{n}emr, &
\ind þér munu \alst{g}óð ef \alst{g}etr: &
\alst{u}pp líta \hld\ skal-at-tu í \alst{o}rrostu; &
\alst{g}jalti \alst{g}líkir \hld\ verða \alst{g}umna synir &
\ind síðr þitt of \alst{h}ęilli \alst{h}alir.\eva

\bvb I counsel thee Loddfathomer, that thou learn the counsels; thou wilt benefit if thou learnest; they will be good for thee if thou gettest: Up shalt thou not look in battle—alike to a madman become the sons of men—lest men bewitch thy [sense/life/face].\footnoteB{A very difficult verse. \CV\ explains \emph{gjalti} as an old dative of \emph{gǫltr} ‘boar, hog’, and thus sees the closely related phrase \emph{verða at gjalti} as “‘to be turned into a hog’, i.e. ‘to turn mad with terror’, esp. in a fight”. The vowel breaking is however unexpected here, since \emph{gǫltr} (< Proto-Norse \emph{*galtuʀ}) is an u-stem, which makes the stem-vowel in the dat. sg. \emph{gęlti} (< \emph{*galtiu}, cf. \textbf{kunimudiu}, dat. sg. of \emph{*Kunimunduʀ}, on the Tjurkö 1 bracteate) the result of i-umlaut rather than an original short \emph{*e}.

\textcite{LaFargeGlossary} instead explains the word as a borrowing from Old Irish \emph{geilt} ‘insane, mad’. \textcite{PettitEdda} follows this, and arguess that the whole theme of the verse probably be of Celtic origin, giving several examples from Celtic literature of warriors going mad upon looking up into the sky during battle. In this case the men (\emph{halir}, which word seems to have an association with warriors; cf. 36–37, 49) would be to quote Pettit some sort of “supernatural sky warriors”, in my opinion most likely the \inx[G]{Ownharriers}.}\evb
\evg


\bvg
\bva \alst{R}ǫ́ðumk þér Loddfáfnir, \hld\ at þú \alst{r}ǫ́ð nemir, &
\ind \alst{n}jóta munt ef \alst{n}emr, &
\ind þér munu \alst{g}óð ef \alst{g}etr: &
Ef vilt þér góða \alst{k}onu \hld\ \alst{k}vęðja at \edtext{gamanrúnum}{\lemma{gamanrúnum ‘pleasure-runes’}\Bfootnote{While easily interpreted as ‘intercourse’, the word is used in 118 with a decidedly non-sexual meaning. It probably just means ‘good, light-hearted conversation’.}} &
\ind ok \alst{f}á \alst{f}ǫgnuð af, &
\alst{f}ǫgru skalt hęita \hld\ ok láta \alst{f}ast vesa; &
\ind lęiðisk manngi \alst{g}ótt ef \alst{g}etr.\eva

\bvb I counsel thee Loddfathomer, that thou learn the counsels; thou wilt benefit if thou learnest; they will be good for thee if thou gettest: If thou wilt for thee welcome a good woman to pleasure-runes, and receive good cheer from [her]; fair things shalt thou promise, and let it be fast; none loathes a good thing if one gets it.\evb
\evg


\bvg
\bva \alst{R}ǫ́ðumk þér Loddfáfnir, \hld\ at þú \alst{r}ǫ́ð nemir, &
\ind \alst{n}jóta munt ef \alst{n}emr, &
\ind þér munu \alst{g}óð ef \alst{g}etr: &
\alst{v}aran bið’k þik \alst{v}esa \hld\ ok ęigi of\alst{v}aran, &
ves þú við \alst{ǫ}l varastr, \hld\ ok við \alst{a}nnars konu &
ok við \alst{þ}at hit \alst{þ}riðja, \hld\ at \alst{þ}jófar né lęiki.\eva

\bvb I counsel thee Loddfathomer, that thou learn the counsels; thou wilt benefit if thou learnest; they will be good for thee if thou gettest: Wary I ask thee to be, and not over-wary; be wariest with ale, and with another man’s woman, and with the third, that thieves do not outplay [thee].\evb
\evg


\bvg
\bva \alst{R}ǫ́ðumk þér Loddfáfnir, \hld\ at þú \alst{r}ǫ́ð nemir, &
\ind \alst{n}jóta munt ef \alst{n}emr, &
\ind þér munu \alst{g}óð ef \alst{g}etr: &
at \alst{h}áði né \alst{h}látri \hld\ \alst{h}af aldrigi &
\ind \alst{g}ęst né \alst{g}anganda.\eva

\bvb I counsel thee Loddfathomer, that thou learn the counsels; thou wilt benefit if thou learnest; they will be good for thee if thou gettest: In mockery or laughter have thou never a guest nor wanderer.\evb
\evg


\bvg
\bva \alst{O}pt vitu \alst{ó}gǫrla, \hld\ þęir’s sitja \alst{i}nni fyr, &
\ind hvęrs þęir ’ru \alst{k}yns es \alst{k}oma; &
es-at maðr svá \alst{g}óðr \hld\ at \alst{g}alli né fylgi, &
\ind né svá \alst{i}llr at \alst{ęi}nu-gi dugi.\eva

\bvb They oft hardly know, who sit inside, of what sort those men are who come; no man is so good that no flaw follows him, nor so evil that he to nothing avails.\evb
\evg


\bvg
\bva \alst{R}ǫ́ðumk þér Loddfáfnir, \hld\ at þú \alst{r}ǫ́ð nemir, &
\ind \alst{n}jóta munt ef \alst{n}emr, &
\ind þér munu \alst{g}óð ef \alst{g}etr: &
at \alst{h}ǫ́rum þul \hld\ \alst{h}lę́ aldrigi, &
\ind opt ’s \alst{g}ótt þat’s \alst{g}amlir kveða, &
opt ór \alst{sk}ǫrpum bęlg \hld\ \alst{sk}ilin orð koma &
\ind þęim’s \alst{h}angir með \alst{h}ǫ́um &
\ind ok \alst{sk}ollir með \alst{sk}rǫ́um, &
\ind ok \alst{v}áfir með \alst{v}ílmǫgum.\eva

\bvb I counsel thee Loddfathomer, that thou learn the counsels; thou wilt benefit if thou learnest; they will be good for thee if thou gettest: At a hoary thyle laugh thou never; oft ’tis good, that which the old sing. Oft out of a scorched leather discerning words come; out of that one that hangs with hides, and dangles with dry skins, and sways among lads of toil \ken{thralls}.\footnoteB{TODO: Some note on this. \emph{vilmǫgum} meaning ‘veal-stomachs’? Cf. Crawford’s video on this.}\evb
\evg


\bvg
\bva \alst{R}ǫ́ðumk þér Loddfáfnir, \hld\ at þú \alst{r}ǫ́ð nemir, &
\ind \alst{n}jóta munt ef \alst{n}emr, &
\ind þér munu \alst{g}óð ef \alst{g}etr: &
\alst{g}ęst þú né \alst{g}ęyj-a \hld\ né á \alst{g}rind hrę́kir; &
\ind get þú \alst{v}ǫ́luðum \alst{v}ęl.\eva

\bvb I counsel thee Loddfathomer, that thou learn the counsels; thou wilt benefit if thou learnest; they will be good for thee if thou gettest: Bark not at a guest, nor spit at the gate;\footnoteB{Behind which the guest stands, waiting for the farmer to open.} furnish the impoverished well.\evb
\evg


\bvg
\bva \alst{R}amt es þat tré, \hld\ es \alst{r}íða skal &
\ind \alst{ǫ}llum at \alst{u}pploki; &
\alst{b}aug þú gef \hld\ eða þat \alst{b}iðja mun &
\ind þér \alst{l}ę́s hvęrs á \alst{l}iðu.\eva

\bvb Strong is that wood which shall swing to open for all;\footnoteB{i.e. the beam of the gate in front of the farm.} give a bigh, or it will bid thee every kind of deceit onto thy limbs.\evb
\evg


\bvg
\bva \alst{R}ǫ́ðumk þér Loddfáfnir, \hld\ at þú \alst{r}ǫ́ð nemir, &
\ind \alst{n}jóta munt ef \alst{n}emr, &
\ind þér munu \alst{g}óð ef \alst{g}etr: &
hvar’s \alst{ǫ}l drekkir \hld\ kjós þér \alst{ja}rðar męgin, &
því’t \alst{jǫ}rð tękr við \alst{ǫ}lðri, \hld\ ęn \alst{ę}ldr við sóttum, &
\alst{ęi}k við \alst{a}bbindi, \hld\ \alst{a}x við fjǫlkyngi, &
\alst{h}ǫll við \alst{h}ýrógi; \hld\ \alst{h}ęiptum skal mána kvęðja, &
\alst{b}ęiti við \alst{b}itsóttum, \hld\ ęn við \alst{b}ǫlvi rúnar; &
\ind \alst{f}old skal við \alst{f}lóði taka.\eva

\bvb I counsel thee Loddfathomer, that thou learn the counsels; thou wilt benefit if thou learnest; they will be good for thee if thou gettest: Wherever thou ale drinkest, choose for thee the might of the earth; for earth takes against drunkenness, but fire against sickness; oak against dysentery, the ear [of wheat] against sorcery, bearded rye against hernia—in conflicts shall one invoke Moon\footnoteB{According to \Voluspa\ 5, the moon has some sort of power, and based on \Lokasenna\ P3 \emph{kvęðja} ‘greet, call’ seems to be the word used for invoking in prayer.}—heather against bite-sicknesses; but \inx[C]{rune}[runes] against \inx[C]{bale};\footnoteB{cf. v. 124, 149.} the fold \ken{earth} must take against the flood.\evb
\evg

\sectionline

\section{The Rune-Tally}

These verses are labelled as \emph{Rúnatals þáttr} (The strand of the Runecount) in younger Eddic paper manuscripts. Many give an archaic, pagan impression. It is as if they were drawn from the lips of an Odinic priest.

\bvg
\bva\alst{V}ęit’k at ek hekk \hld\ \alst{v}indga męiði á &
\ind \alst{n}ę́tr allar \alst{n}íu, &
\alst{g}ęiri undaðr \hld\ ok \alst{g}efinn Óðni, &
\ind \alst{s}jalfr \alst{s}jǫlfum mér, &
á þęim \alst{m}ęiði, \hld\ es \alst{m}anngi vęit, &
\ind hvęrs af \alst{r}ótum \alst{r}innr.\eva

\bvb I know that I hung on the windy beam, for all of nine nights; wounded by spear and given to Weden—myself to myself—on that beam, which no man knows, of whose roots it runs.\evb
\evg


\bvg
\bva Við \alst{h}lęifi mik sę́ldu-t \hld\ né við \alst{h}orni-gi; &
\alst{n}ýsta’k \alst{n}iðr, \hld\ \alst{n}am’k upp rúnar, &
\alst{ǿ}pandi nam, \hld\ fell’k \alst{a}ptr þaðan.\eva

\bvb With loaf they gladdened me not, nor with any horn. I peered down, I took up the runes, screaming I took; I fell back thence.\evb
\evg


\bvg
\bva \alst{F}imbulljóð níu \hld\ nam’k af hinum \alst{f}rę́gja syni &
\ind \alst{B}ǫlþorns, \alst{B}ęstlu fǫður, &
ok ek \alst{d}rykk of gat \hld\ hins \alst{d}ýra mjaðar &
\ind \alst{au}sinn \alst{Ó}ðreri.\eva

\bvb Nine \inx[C]{fimble-leeds} I learned from the famous son of \inx[P]{Balethorn}, the father of \inx[P]{Bestle}—and a drink I got, of that dear mead poured to \inx[P]{Woderearer}.\footnoteB{This verse fits poorly here and seems like an insert. It mentions \emph{ljóð} ‘leeds; (magical) songs, incantations’ rather than runes, and has nothing to do with Weden’s hanging on the tree. Bestle was Weden’s mother and Balethorn his maternal grandfather. The famous son of Balethorn would then be his maternal uncle. The custom of sending sons away to be fostered by their maternal uncles or grandfathers (which seems to be what is going on here) was quite common in Germanic society, cf. TODO.}\evb
\evg


\bvg
\bva Þá nam’k \alst{f}rę́vask \hld\ ok \alst{f}róðr vesa &
\ind ok \alst{v}axa ok \alst{v}ęl hafask; &
\alst{o}rð mér af \alst{o}rði \hld\ \alst{o}rðs lęitaði &
\ind \alst{v}erk mér af \alst{v}erki \alst{v}erks.\eva

\bvb Then I took to thrive, and be learned, and grow and have myself well. A word for me of a word a word sought out; a work for me of a work a work.\footnoteB{Each good word and deed was followed by another.}\evb
\evg


\bvg
\bva \alst{R}únar munt finna \hld\ ok \alst{r}áðna stafi, &
\ind mjǫk \alst{st}óra \alst{st}afi, &
\ind mjǫk \alst{st}inna \alst{st}afi, &
\ind es \alst{f}áði \alst{f}imbulþulr &
\ind ok \alst{g}ęrðu \alst{g}innręgin &
\ind ok \alst{r}ęist Hroptr \edtext{\alst{r}agna}{\lemma{ragna ‘of the Reins’}\Afootnote{‘rǫgna’ \Regius}}.\eva

\bvb \inx[C]{rune}[Runes] wilt thou find, and interpreted staves: very large staves, very stiff staves, which \inx[P]{Fimblethyle} \name{= Weden} painted, and the \inx[G]{gin-Reins} made, and Roft \name{= Weden} of the Reins carved.\evb
\evg


\bvg
\bva \alst{Ó}ðinn með \alst{ǫ́}sum, \hld\ ęn fyr \alst{ǫ}lfum Dáinn, &
\ind \alst{D}valinn \alst{d}vergum fyr, &
\ind \alst{Á}sviðr \alst{jǫ}tnum fyr, &
\ind ek ręist \alst{s}jalfr \alst{s}umar.\eva

\bvb \inx[P]{Weden} among the \inx[G]{Ease}, but for the \inx[G]{Elves} \inx[P]{Dowen}; \inx[P]{Dwollen} for the \inx[G]{Dwarfs}; \inx[P]{Onswith} for the Ettins; I myself carved some.\footnoteB{The identity of the speaker is not clear.}\evb
\evg


\bvg
\bva Vęizt, hvé \alst{r}ísta skal? \hld\ Vęizt, hvé \alst{r}áða skal? &
Vęizt, hvé \alst{f}áa skal? \hld\ Vęizt, hvé \alst{f}ręista skal? &
Vęizt, hvé \alst{b}iðja skal? \hld\ Vęizt, hvé \alst{b}lóta skal? &
Vęizt, hvé \alst{s}ęnda skal? \hld\ Vęizt, hvé \alst{s}óa skal?\eva

\bvb Knowest thou how one shall carve? Knowest thou how one shall read? Knowest thou how one shall paint? Knowest thou how one shall try? Knowest thou how one shall bid? Knowest thou how one shall \inx[C]{bloot}? Knowest thou one shall send? Knowest thou how one shall \inx[C]{soo}?\footnoteB{A symmetric structure would be attained if the first four verbs refer to \inx[C]{rune}[runes]—carving, interpreting, painting (with blood?), and divining—while the latter four refer to sacrifice—praying, sacrificing, sending (the sacrifice or the prayer; making sure the gods receive it), and slaying the victim. See further relevant Encyclopedia entries. The meter of the v. is unusual, but bears some resemblance to Vg 216 (the Högstena galder). TODO: Elaborate.}\evb
\evg


\bvg
\bva \alst{B}ętra ’s ó\alst{b}eðit \hld\ an sé of\alst{b}lótit, &
\ind ęy sér til \alst{g}ildis \alst{g}jǫf; &
bętra ’s ó\alst{s}ęnt \hld\ an sé of\alst{s}óit; &
\edtext{[...]}{\Bfootnote{Last line probably missing here; the meter and sense require it.}}\eva

\bvb ’Tis better unbid than over\inx[C]{bloot}[blooted]; a gift always sees recompense. ’Tis better unsent than over\inx[C]{soo}[sooed]; [...].\footnoteB{Identical wording (\emph{biðja} ‘to bid; to pray’ : \emph{blóta} ‘to bloot; to sacrifice’; \emph{senda} ‘to send’ : \emph{sóa} ‘to soo; to slay’) suggests a close relation to the previous verse. — The sense seems to be that it is better not to sacrifice at all than to sacrifice in excess, since even a small gift (to the gods) will be rewarded. This mechanistic system of gifts and rewards between man and the gods is also seen in other Indo-European pagan literatures. Compare the Sanskrit \emph{Dehí me, dádāmi te} ‘Give to me; I give to thee’ or Latin \emph{dō ut dēs} ‘I give that thou might give’.}\evb
\evg


\bvg
\bva Svá \alst{Þ}undr of ręist \hld\ fyr \alst{þ}jóða rǫk &
þar’s \alst{u}pp of ręis, \hld\ es \alst{a}ptr of kom.\eva

\bvb Thus \inx[P]{Thound} \name{= Weden} carved for the rakes of nations, where up he rose as back he came.\footnoteB{A very cryptic v. TODO.}\evb
\evg

\sectionline

\section{The Leed-Tally}

This final section of the poem has fittingly been called the Leed-Tally (\emph{Ljóðatal}). The speaker (certainly Weden) recounts eighteen spells, aristocratic and Odinic in character; they deal with such things as healing (2, 12), battle (3, 4, 5, 8, 11, 13), countering sorcery (6, 10), stilling the elements (7, 9), and seduction (16, 17).

In particular the fourth spell bears a strong likeness to the first Merseburg charm.


\bvg
\bva Ljóð \alst{þ}au kann’k, \hld\ es kann-at \alst{þ}jóðans kona &
\ind ok \alst{m}anskis \alst{m}ǫgr. &
\alst{H}jǫlp hęitir ęitt, \hld\ þat þér \alst{h}jalpa mun &
\ind við \alst{s}orgum ok \alst{s}ǫkum, \hld\ ok \alst{s}útum gǫrvǫllum.\eva

\bvb Those \inx[C]{leed}[leeds] I know, as knows not the ruler’s woman, and no man’s lad. Help is called one, it will help thee against sorrows and sakes,\footnoteB{Legal proceedings.} and all kinds of griefs.\footnoteB{TODO: elaborate on translatioon}\evb
\evg


\bvg
\bva Þat kann’k \alst{a}nnat, \hld\ es þurfu \alst{ý}ta synir, &
\ind þęir’s vilja \alst{l}ę́knar \alst{l}ifa.\eva

\bvb I know another, which the sons of men need;\footnoteB{Identical wording to 164/2.} they who wish to live as healers.\evb
\evg


\bvg
\bva Þat kann’k \alst{þ}riðja, \hld\ ef mér verðr \alst{þ}ǫrf mikil &
\ind \alst{h}apts við mína \alst{h}ęiptmǫgu, &
\alst{ę}ggjar dęyfi’k \hld\ minna \alst{a}ndskota, &
\ind bíta-t þęim \alst{v}ǫ́pn né \alst{v}élir.\eva

\bvb I know the third, if I come in great need of hindrance against my conflict-lads \ken{enemies}; I dull the edges of my opponents; for them neither weapons nor wiles bite.\evb
\evg


\bvg
\bva Þat kann’k \alst{f}jórða, \hld\ ef mér \alst{f}yrðar bera &
\ind \alst{b}ǫnd at \alst{b}oglimum, &
svá ek \alst{g}ęl, \hld\ at \alst{g}anga má’k, &
\ind sprettr mér af \alst{f}ótum \alst{f}jǫturr. &
\ind ęn af \alst{h}ǫndum \alst{h}apt.\eva

\bvb I know the fourth, if men bear bonds onto my bow-limbs: so I gale that I may walk; from my feet spring the fetters off, but from my hands the bonds.\evb
\evg


\bvg
\bva Þat kann’k \alst{f}imta, \hld\ ef sé’k af \alst{f}ári skotinn &
\ind \alst{f}lęin í \alst{f}olki vaða, &
flýgr-a svá \alst{st}int, \hld\ at \alst{st}ǫðvi’g-a’k, &
\ind ef hann \alst{s}jónum of \alst{s}é’k.\eva

\bvb I know the fifth, if I see a dangerously shot arrow wading in the troop; it flies not so stiffly that I may not hinder it, if I see it with my sights.\evb
\evg


\bvg
\bva Þat kann’k \alst{s}étta, \hld\ ef mik \alst{s}ę́rir þegn &
\ind á \alst{r}ótum \alst{r}ás viðar. &
þann \alst{h}al, \hld\ es mik \alst{h}ęipta kvęðr, &
\ind þann eta \alst{m}ęin hęldr an \alst{m}ik.\eva

\bvb I know the sixth, if a thane injures me on the roots of a green tree;\footnoteB{Presumably by carving runes into it.} that man who sings hatred against me, him the harms eat rather than me.\evb
\evg


\bvg
\bva Þat kann’k \alst{s}jaunda, \hld\ ef \alst{s}é’k hǫ́van loga &
\ind \alst{s}al umb \alst{s}essmǫgum, &
\alst{b}rinnr-at svá \alst{b}ręitt, \hld\ at hǫ́num \alst{b}jargi’g-a’k; &
\ind þann kann’k \alst{g}aldr at \alst{g}ala.\eva

\bvb I know the seventh, if I see a high flame around a hall with seat-lads \ken{feasting warriors}, it burns not so broad that I do not rescue it—that galder I can gale.\evb
\evg


\bvg
\bva Þat kann’k \alst{á}tta, \hld\ es \alst{ǫ}llum es &
\ind \alst{n}ytsamligt at \alst{n}ema, &
\alst{h}var’s \alst{h}atr vęx \hld\ með \alst{h}ildings sonum, &
\ind þat má’k \alst{b}ǿta \alst{b}rátt.\eva

\bvb I know the eighth, which for all is useful to learn: wherever hatred grows among the sons of princes, it I may shortly mend.\evb
\evg


\bvg
\bva Þat kann’k \alst{n}íunda, \hld\ ef mik \alst{n}auðr of stęndr &
\ind at bjarga \alst{f}ari mínu á \alst{f}loti, &
\alst{v}ind ek kyrri \hld\ \alst{v}ági á &
\ind ok \alst{s}vę́fi’k allan \alst{s}ę́.\eva

\bvb I know the ninth, if need requires me to rescue my friend (TODO) on a floater \ken{ship}. The wind I calm on the wave, and put all the sea asleep.\evb
\evg


\bvg
\bva Þat kann’k \alst{t}íunda, \hld\ ef sé’k \alst{t}únriður &
\ind \alst{l}ęika \alst{l}opti á, &
ek svá \alst{v}inn’k, \hld\ at \edtrans{þę́r \alst{v}illar fara}{they (\emph{feminine}) journey lost}{\Bfootnote{emend.; \emph{þęir villir fara} ‘they (\emph{masculine}) journey lost’ \Regius}} &
\ind sinna \alst{h}ęim-\alst{h}ama &
\ind sinna \alst{h}ęim-\alst{h}uga.\eva

\bvb I know the tenth, if I see \inx[G][town-riders] playing aloft; I accomplish it so that they journey lost of their home-\inx[C]{hame}[hames]; of their home-minds.\footnoteB{The \emph{riður} ‘(female) riders’ were witches who were thought to leave their hames (\emph{hamir} ‘skins, shapes’) in a form of astral projection in order to fly around in the air, tormenting villagers. Their original bodies would of course be lying in a comatose state, and with the bodies their original minds; their humanness. Weden was through his second sight able to see these riders, and could use his superior magical abilities in order to confuse them so that they were not able to return to their original hames or minds; a cruel fate. — Weden likewise brags about tricking \emph{riders} in \Harbardsljod\ 20.}\evb
\evg


\bvg
\bva Þat kann’k \alst{ę}llipta, \hld\ ef skal’k til \alst{o}rrostu &
\ind \alst{l}ęiða \alst{l}angvini, &
und \alst{r}andir gęl’k, \hld\ ęn þęir með \alst{r}íki fara, &
\ind \alst{h}ęilir \alst{h}ildar til, &
\ind \alst{h}ęilir \alst{h}ildi frá, &
\ind koma þęir \alst{h}ęilir \alst{h}vaðan.\eva

\bvb I know the eleventh, if I shall lead old friends into battle: beneath the shields I gale, and they go with power healthy to the conflict; healthy from the conflict; they return healthy from wherever.\evb
\evg


\bvg
\bva Þat kann’k \alst{t}olpta, \hld\ ef sé’k á \alst{t}ré uppi &
\ind \alst{v}áfa \alst{v}irgilná, &
svá ek \alst{r}íst \hld\ ok í \alst{r}únum fá’k, &
\ind at sá \alst{g}ęngr \alst{g}umi. &
\ind ok \alst{m}ę́lir við \alst{m}ik.\eva

\bvb I know the twelfth, if I see high up on a tree a gallow-corpse waving: so I carve, and paint into runes, that that man walks and speaks with me.\evb
\evg


\bvg
\bva Þat kann’k \alst{þ}rettánda \hld\ ef skal’k \alst{þ}egn ungan &
\ind \alst{v}erpa \alst{v}atni á, &
mun-at hann \alst{f}alla, \hld\ þótt í \alst{f}olk komi, &
\ind \alst{h}nígr-a sá \alst{h}alr fyr \alst{h}jǫrum.\eva

\bvb I know the thirteenth, if I shall upon a young thane throw water;\footnoteB{Describing the pagan ritual of pouring water on a newborn child. Cf. \Rigsthula\ 7, 21, 34.} he will not fall, although he comes into battle; that man does not sink down before swords.\evb
\evg


\bvg
\bva Þat kann’k \alst{f}jǫgurtánda, \hld\ ef skal’k \alst{f}yrða liði &
\ind \alst{t}ęlja \alst{t}íva fyr, &
\alst{á}sa ok \alst{a}lfa \hld\ ek kann \alst{a}llra skil, &
\ind fár kann ó\alst{s}notr \alst{s}vá.\eva

\bvb I know the fourteenth, if I shall count the Tues before the retinue of men. Of all the Ease and Elves I know distinctions; few unwise men can do so.\evb
\evg


\bvg
\bva \alst{Þ}at kann’k fimtánda, \hld\ es gól \alst{Þ}jóðrørir &
\ind \alst{d}vergr fyr \alst{D}ęllings \alst{d}urum, &
\alst{a}fl gól \alst{ǫ́}sum, \hld\ ęn \alst{ǫ}lfum frama, &
\ind \alst{h}yggju \alst{H}roptatý.\eva

\bvb I know the fifteenth, which Thedrearer galed, the dwarf before Delling’s doors. Power he galed for the Ease, but for the Elves fame; thought for Roft-Tue \name{= Weden}.\evb
\evg


\bvg
\bva Þat kann’k \alst{s}extánda, \hld\ ef vil’k hins \alst{s}vinna mans &
\ind hafa \alst{g}ęð alt ok \alst{g}aman, &
\alst{h}ugi \alst{h}vęrfi’k \hld\ \alst{h}vitarmri konu &
\ind ok \alst{s}ný’k hęnnar ǫllum \alst{s}efa.\eva

\bvb I know the sixteenth, if I will from the wise girl have her whole sense and pleasure; the heart I change of the white-armed woman, and I turn her whole affection.\evb
\evg


\bvg
\bva Þat kann’k \alst{s}jautjánda \hld\ at mik \alst{s}ęint mun firrask &
\ind hit \alst{m}anunga \alst{m}an.\eva

\bvb I know the seventeenth, that the girl-young girl will lately shun me.\evb
\evg


\bvg
\bva \alst{L}jóða þessa \hld\ munt \alst{L}oddfáfnir &
\ind lengi \alst{v}anr \alst{v}esa; &
\ind þó sé þér \alst{g}óð ef \alst{g}etr, &
\ind \alst{n}ýt ef \alst{n}emr, &
\ind \alst{þ}ǫrf ef \alst{þ}iggr.\eva

\bvb Of these leeds wilt thou, Loddfathomer, long be deprived, although they might be good for thee if thou gettest, beneficial if thou learnest, needful if thou acceptest.\evb
\evg


\bvg
\bva Þat kann’k \alst{á}tjánda, \hld\ es \alst{ę́}va kęnni’k &
\ind \alst{m}ęy né \alst{m}anns konu, &
(\alst{a}lt es bętra \hld\ es \alst{ęi}nn of kann, &
\ind þat fylgir \alst{l}jóða \alst{l}okum,) &
nema þęiri \alst{ęi}nni, \hld\ es mik \alst{a}rmi vęrr, &
\ind eða mín \alst{s}ystir \alst{s}é.\eva

\bvb I know the eighteenth, which I will never teach to a maiden nor man’s woman—everything is better when one alone can do it; that follows the end of the leeds—save for that one alone, who wraps me in her arm,\footnoteB{This interesting expression is also used \Volundarkvida\ 2. — The one who wraps Weden in her arm may be his wife, Frie. He has no known sister.} or who be my sister.\evb
\evg


\bvg
\bva Nú eru \alst{H}áva mǫ́l kveðin \hld\ \alst{H}áva \alst{h}ǫllu í; &
\ind \alst{a}llþǫrf \alst{ý}ta sonum, &
\ind \alst{ó}þǫrf \edtext{\alst{jǫ}tna}{\lemma{jǫtna}\Afootnote{ýta \emph{corrected in margin} \Regius}} sonum; &
hęill sá’s \alst{k}vað, \hld\ hęill sá’s \alst{k}ann, &
\ind \alst{n}jóti sá’s \alst{n}am, &
\ind \alst{h}ęilir þęir’s \alst{h}lýddu.\eva

\bvb Now are the speeches of the High One sung, in the hall of the High One; of great need for the sons of men, of harm for the sons of ettins! Hail he who sang [them]; hail he who knows [them]; may he benefit who learned [them]; hail those who heeded [them]!\evb
\evg
% Weden
	\bookstart{From the Sons of king Reeding}[Frá sonum Hrauðungs konungs]

BPA Hrauðungr konungr átti tvá sonu. Hét annarr Agnarr, enn annarr Geirrøðr.
BPA Agnarr var tíu vetra enn Geirrøðr átta vetra. Þeir reru tveir á báti með dorgar sínar at smáfiski.
BPA Vindr rak þá í haf út. Í náttmyrkri brutu þeir við land ok gingu upp; fundu kotbónda einn.
BPA Þar vǫ́ru þeir um vetrinn. Kerling fostraði Agnar enn karl Geirrøð.
BPA At vári fekk karl þeim skip. Enn er þau kerling leiddu þá til strandar, þá mælti karl einmæli við Geirrøð.
BPA Þeir fengu byr ok kvǫ́mu til stǫðva fǫður síns. Geirrøðr var fram í skipi.
BPA Hann hljóp upp á land enn hratt út skipinu, ok mælti: ”Far þú þar er smyl hafi þik.”
BPA Skipit rak út. Enn Geirrøðr gekk út til bǿjar; hánum var vel fagnat;
BPA þá var faðir hans andaðr. Var þá Geirrøðr til konungs tekinn, ok varð maðr ágætr.

BPB King Reeding owned two sons. One was called Eynhere, and the other Garred.
BPB Eynhere was ten winters old, and Garred eight winters. The two were rowing in a boat with their trolling-lines for small fishing.
BPB Wind then drove them out into the sea. In the darkness of night they crashed into land and walked up; they found a lone cottage-farmer.
BPB There they were about the winter. The wife fostered Eynhere, but the husband Garred.
BPB At spring the man got them ships. But when the woman led them to the shore, the husband spoke privately with Garred.
BPB They got favourable wind, and came to their father's harbour. Garred was in the front of the ship.
BPB He leapt up onto land and pushed out the ship, and spoke: ”Go thou where the \textbf{smil} might have thee.”
BPB The ship drove out. But Garred walked towards the farm; he was welcomed well;
BPB his father had by then drawn his final breath. Then was Garred taken as king, and became an excellent man.

BPA Óðinn ok Frigg sátu í Hliðskjǫlfu ok sá um heima alla.
BPA Óðinn mælti: Sér þú Agnar fóstra þinn, hvar hann elr bǫrn við gýgi í hellinum?
BPA En Geirrøðr, fóstri minn, er konungr ok sitr nú at landi.
BPA Frigg segir: Hann er matníðingr sá at hann kvelr gesti sína ef hánum þykkja ofmargir koma.
BPA Óðinn segir at þat er in mesta lygi. Þau veðja um þetta mál.
BPA Frigg sendi eskismey sína, Fullu, til Geirrøðar. Hon bað konung varask at eigi fyrgerði hánum fjǫlkunnigr maðr sá er þar var kominn í land ok sagði þat mark á at engi hundr var svá olmr at á hann myndi hlaupa.
BPA En þat var inn mesti hégómi at Geirrøðr væri eigi matgóðr ok þó lætr hann handtaka þann mann er eigi vildu hundar á ráða.
BPA Sá var í feldi blám ok nefndisk Grímnir ok sagði ekki fleira frá sér þótt hann væri atspurðr.
BPA Konungr lét hann pína til sagna ok setja milli elda tveggja ok sat hann þar átta nætr.
BPA Geirrøðr konungr átti son tíu vetra gamlan ok hét Agnarr eftir bróður hans.
BPA Agnarr gekk at Grímni ok gaf hánum horn fullt at drekka, sagði að konungr gerði illa er hann lét pína hann saklausan.
BPA Grímnir drakk af. Þá var eldrinn svá kominn at feldrinn brann af Grímni. Hann kvað:

BPB Weden and Frie sat in Litheshelf and looked about all the Homes.
BPB Weden spoke: Seest thou Eynhere thy foster-son, where he begets children with the troll-woman in the cave?
BPB But Garred, my foster-son, is king and now sits at land.
BPB Frie says: He is such a meat-nithing that he tortures his guests if he thinks there are too many of them.
BPB Weden says that this is the greatest lie; they make a bet about this matter.
BPB Frie sent her handmaid Full to Garred's. She asked the king to be wary, that he might not be ended by that fealcunning man who was come in the land, and said that his mark was that no hound were so fierce that he would leap onto him.
BPB But that was the greatest vainglory that Garred would not be meat-good, and yet he has that man seized, whom the dogs would not touch.
BPB He was clad in a blue cloak, and called himself Grimen, and did not tell any more about himself, even though he was interrogated.
BPB The king had him tortured so that he would speak, and set him between two fires, and he remained there for eight nights.
BPB King Garred had a son ten winters old, and he was named Eynhere after his brother.
BPB Eynhere walked up to Grimen, and gave him a full horn to drink, saying that the king did ill as he had him tortured without cause.
BPB Grimen drank from it; then the fire had come such that the cloak burned on Grimen. He quoth:
% Weden
	\bookStart{The Speeches of Grimner}[Grímnismǫ́l]

% Introduction


The \textbf{Speeches of Grimner} are preserved whole in both \Regius\ and \AM.

The structure of the poem is mostly clear; the first three verses set the stage, repeating some of what we got in the prose. It is certain that Weden is the speaker. After this various lore is touched on, not always clearly. In this the poem aligns closely with ones such as \Vafthrudnismal\, \Sigrdrifumal\ and \Allvismal.

First are listed the halls of the gods (4–17), though the numbering does not seem to agree with the count of locations mentioned. Then the conditions and surroundings of Weden’s animals and hall are elaborated on (18–23). Mentioned are the preparation of food (18), his wolves (19) and ravens (20), the river through which dead men have to wade (21), the gate through which they have to pass (22), the count of doors in the hall (23) and the two animals who gnaw on the branches of the tree (25–26). We then have a long list of rivers (28–30) and horses ridden by the gods (31). Then is told of the conditions and animals of Ugdrassle (32–36).

Thereafter follow several discordant verses. A list of Walkirries (37), the progression of the sun and moon (38–40), the creation of the world from Yimer’s body (41–42), the significance of \inx{bloot-kettles}[C] (43), the creation of the ship Shidebladner (44) and finally the noblest of several categories of things and groups (45).

After all of this Weden utters an unclear verse invoking the gods (46), before listing many of his names and the circumstances in which they were used (47–50). He then turns to Garfrith, disappointed by the inhospitality and poor conduct of his former protégé, and predicts his imminent death (51–53). He finally reveals himself by his true name, daring Garfrith to face him (53). After this he repeats several of his names (54), and the poetry ends.

In the final prose section we are told that Garfrith tripped and fell on his sword, after which is son Eyner rules for a long time.


\bvg
\bva Hęitr est hripuðr \hld\ ok hęldr til mikill, &
\ind gǫngumk firr funi. &
Loði sviðnar, \hld\ þótt á lopt bera'k; &
\ind brinnumk feldr fyrir.\eva

\bvb Hot art thou, flame, and rather too large; go far from me, fire! The woolen cape is singed though I hold it aloft; the cloak burns before me.\evb
\evg


\bvg
\bva Átta nę́tr satk \hld\ milli ęlda hér, &
\ind svát mér mangi \hld\ mat né bauð &
nema ęinn Agnarr, \hld\ es ęinn skal ráða, &
Gęirrøðar sonr, \hld\ Gotna landi.\eva

\bvb For eight nights sat I between the fires here, while no man offered me food; save for lone Eyner, who lone shall rule—that son of Garfrith—the land of the Gots!\evb
\evg


\bvg
\bva Hęill skalt, Agnarr, \hld\ alls hęilan biðr &
\ind þik Veratýr vesa; &
ęins drykkjar \hld\ þú skalt aldrigi &
\ind bętri gjǫld geta.\eva

\bvb Hale shalt thou be, Eyner, as hale thee Weretue <= Weden> bids be; for one drink shalt thou never get a better recompense.\footnoteB{The recompense being the esoteric lore.}\evb
\evg


\bvg
\bva Land es hęilagt, \hld\ es liggja sé’k &
\ind ǫ́sum ok ǫlfum nę́r; &
ęn í Þrúðhęimi \hld\ skal Þórr vesa &
\ind unz of rjúfask ręgin.\eva

\bvb The land is holy, which I see lying close to the Ease and elves; but in Thrithham shall Thunder be, until the Reins are rent.\footnoteB{Thrithham is not}\evb
\evg


\bvg
\bva Ýdalir hęita, \hld\ þar’s Ullr of hęfr &
\ind sér of gǫrva sali; &
Alfhęim Fręy \hld\ gǫ́fu í árdaga &
\ind tívar at tannféi.\eva

\bvb Yewdales are called where Woulder has made himself a hall. Elfham to Free in days of yore the Tues as a tooth-gift\footnoteB{The gift that a child receives when he gets his first tooth.} gave.
\evg


\bvg
\bva Bǿr ’s hinn þriði, \hld\ es blíð ręgin &
\ind silfri þǫkðu sali; &
Valaskjǫlf hęitir, \hld\ es vélti sér &
\ind ǫ́ss í árdaga.\eva

\bvb Bower is the third, where the blithe Reins with silver thatched a hall. Waleshelf is called, where tricked himself, the os in days of yore.\evb
\evg


\bvg
\bva Søkkvabękkr hęitir hinn fjórði, \hld\ ęn þar svalar knegu &
\ind unnir glymja yfir; &
þar þau Óðinn ok Sága \hld\ drekka umb alla daga &
\ind glǫð ór gollnum kęrum.\eva

\bvb Sinkbench is called the fourth, but there cool waves do clash above; there Weden and Sey drink all days, gladly out of golden vats.\evb
\evg


\bvg
\bva Glaðshęimr hęitir hinn fimti \hld\ þar’s hin gollbjarta &
\ind Valhǫll víð of þrumir; &
ęn þar Hroptr \hld\ kýss hvęrjan dag &
\ind vápndauða vera.\eva

\bvb Gladsham is called the fifth, where the gold-bright Walhall, wide, stands fast; but there Roft <= Weden> chooses every day weapon-dead men.\evb
\evg


\bvg
\bva Mjǫk ’s auðkęnt \hld\ þęim’s til Óðins koma &
\ind salkynni at séa, &
skǫptum ’s rann rępt, \hld\ skjǫldum ’s salr þakiðr, &
\ind brynjum of bękki stráat.\eva

\bvb Very easily recognized, for those who to Weden come, is the hall to see: With shafts is the house roofed; with shields is the hall thatched; with byrnies the benches strewn.\evb
\evg


\bvg
\bva Mjǫk ’s auðkęnt \hld\ þęim’s til Óðins koma &
\ind salkynni at séa, &
vargr hangir \hld\ fyr vestan dyrr &
\ind ok drúpir ǫrn yfir.\eva

\bvb Very easily recognized, for those who to Weden come, is the hall to see: A wolf hangs before the western door, and an eagle droops over.\evb
\evg


\bvg
\bva Þrymhęimr hęitir hinn sétti, \hld\ es Þjazi bjó, &
\ind sá hinn ámátki jǫtunn; &
ęn nú Skaði byggvir, \hld\ skír brúðr goða, &
\ind fornar toptir fǫður.\eva

\bvb Thrimham is called the sixth, where Thedse dwelled, that terrifying ettin; but now Scathe bedwells—pure bride of the gods—the ancient plots of her father.\evb
\evg


\bvg
\bva Bręiðablik eru hin sjaundu, \hld\ ęn þar Baldr hęfir &
\ind sér of gǫrva sali, &
á því landi \hld\ es liggja vęit’k &
\ind fę́sta fęiknstafi.\eva

\bvb Broadblicks are the seventh, and there Balder has made for himself a hall; on that land, where I know lie the fewest staves of treachery.\footnoteB{Evil deeds.}\evb
\evg


\bvg
\bva Himinbjǫrg eru in ǫ́ttu \hld\ ęn þar Hęimdall &
\ind kveða valda véum. &
þar vǫrðr goða \hld\ drękkr í vę́ru ranni &
\ind glaðr góða mjǫð.\eva

\bvb Heavenbarrows are the eighth, and there Homedall, they say, wields over wighs. There in the tranquil house the ward of the gods \ken{Homedall}[1] drinks glad the good mead.\evb
\evg


\bvg
\bva Folkvangr es inn níundi \hld\ en þar Fręyja rę́ðr &
\ind sessa kostum í sal; &
halfan val \hld\ hon kýss hvęrjan dag &
\ind ęn halfan Óðinn á.\eva

\bvb Folkwong is the ninth, and there Frow wields the choice of seats in the hall; half of the slain she chooses each day, but half Weden owns.\evb
\evg


\bvg
\bva Glitnir es inn tíundi; \hld\ hann es gulli studdr &
\ind ok silfri þakðr it sama; &
ęn þar Forseti \hld\ byggir flęstan dag &
\ind ok svę́fir allar sakir.\eva

\bvb Glitner is the tenth, it is studded by gold, and thatched by silver the same; but there Forset dwells most of the day, and resolves\footnoteB{Puts to sleep,} all [legal] matters.\evb
\evg


\bvg
\bva Nóatún eru in ęlliftu \hld\ ęn þar Njǫrðr hęfir &
\ind sér um gǫrva sali, &
manna þęngill \hld\ inn męinsvani &
\ind hátimbruðum hǫrgi rę́ðr.\eva

\bvb Nowetowns are the tenth, and there Nearth has made himself a hall. The prince of men, the guileless one, rules the high-timbered \inx{harrow}.\evb
\evg


\bvg
\bva Hrísi vęx \hld\ ok há grasi &
\ind Víðars land, viði, &
ęn þar mǫgr of lę́zk \hld\ af mars baki &
\ind frǿkn at hęfna fǫður.\eva

\bvb TO-DO.\evb
\evg


\bvg
\bva Andhrímnir \hld\ lę́tr í Ęldhrímni &
\ind Sę́hrímni soðinn, &
flęska bęzt, \hld\ ęn þat fáir vitu &
\ind við hvat ęinhęrjar alask.\eva

\bvb Andrimner lets in Eldrimner Sowrimner be boiled. The best of pork, but few know it, by what the Ownharriers are nourished.\footnoteB{The cook Andrimner ‘face-sooty’ has the boar Sowrimner ‘sow-sooty’ boiled in the cauldron Eldrimner ‘fire-sooty’; by this meat are the Ownharriers nouished.}\evb
\evg


\bvg
\bva Gera ok Freka \hld\ sęðr gunntamiðr, &
\ind hróðigr Hęrjafǫðr, &
ęn við vín ęitt \hld\ vápngǫfugr &
\ind Óðinn ę́ lifir.\eva

\bvb The battle-accustomed, glorious Father of Hosts \ken{Weden}[1] feeds Gerr and Freck; but by wine alone, the weapon-worshipful Weden ever lives.\evb
\evg


\bvg
\bva Huginn ok Muninn \hld\ fljúga hvęrjan dag &
\ind jǫrmungrund yfir; &
óumk of Hugin, \hld\ at aptr né komit; &
\ind þó séumk męir of Munin.\eva

\bvb Highen and Minden fly every day over the \inx{ermin-ground}[C]. I fear for Highen, that he come not back; yet I worry more for Minden.\evb
\evg


\bvg
\bva Þýtr Þund, \hld\ unir Þjóðvitnis &
\ind fiskr flóði í; &
áarstraumr \hld\ þykkir ofmikill &
\ind valglaumi at vaða.\eva

\bvb \inx{Thound}[P] roars; dwells Thedwitner’s fish\footnoteB{A very difficult kenning to interpret, but see TODO.} in the flood; the river-stream seems far too great, for the noisy slain host to wade through.\footnoteB{Presumably describing the river which surrounds Walhall, and which the dead have to pass on their way to it.}\evb
\evg


...


\bvg
\bva Ór Ymis holdi \hld\ vas jǫrð of skǫpuð, &
\ind ęn ór svęita sę́r, &
bjǫrg ór bęinum, \hld\ baðmr ór hári, &
\ind ęn ór hausi himinn.\eva

\bvb Out of Yimer’s hull was the earth shaped, but out of his blood the seas; crags out of his bones, trees from his hair, but heaven out of his skull.\evb
\evg


\bvg
\bva En ór hans brǫ́um \hld\ gęrðu blíð ręgin &
\ind Miðgarð manna sonum, &
ęn ór hans hęila \hld\ vǫ́ru þau hin harðmóðgu &
\ind ský ǫll of skǫpuð.\eva

\bvb But out of his brows the blithe Reins made Midyard for the sons of men; but out of his brains were the hard-stirred skies all shaped.\evb
\evg


\bvg
\bva Ullar hylli \hld\ hęfr ok allra goða &
\ind hvęrr ’s tękr fyrstr á funa, &
því’t opnir hęimar \hld\ verða of ása sonum,
\ind þá’s hęfja af hvera.\eva

\bvb The favour of Woulder, and of all the Ease, has each who first touches the fire; for the Homes become open about the sons of the Ease, when the cauldrons are lifted off.\evb
\evg


\bvg
\bva Ívalda synir \hld\ gingu í árdaga &
\ind Skíðblaðni at skapa, &
skipa bazt \hld\ skírum Fręy, &
nýtum Njarðar bur.\eva

\bvb The sons of Iwald went, in days of yore, Shidebladner to shape; the best of ships for the pure Free, for the useful son of Nearth \ken{Free}[1].\evb
\evg


\bvg
\bva Askr Yggdrasils, \hld\ hann es ǿztr viða &
\ind ęn Skíðblaðnir skipa, &
Óðinn ása \hld\ ęn jóa Slęipnir, &
Bilrǫst brúa \hld\ ęn Bragi skalda, &
Hábrók hauka \hld\ ęn hunda Garmr.\eva

\bvb The ash of Ugdrassle, that is the noblest of trees, but Shidebladner of ships; Weden of the Ease, but of horses Slopner; Bilrest of bridges, but Bray of scolds; Highbrook of hawks, but of hounds Garm.\evb
\evg


...


\bvg
\bva Ǫlr est Geirrøðr, \hld\ hęfr þú of drukkit; &
miklu est hnugginn, \hld\ es þú est mínu gęngi, &
ǫllum ęinhęrjum \hld\ ok Óðins hylli. \eva

\bvb Worse for ale art thou, Garfrith, hast thou too much drunk. Of much art thou bereft, as thou art of my support, of all the Ownharriers, and of Weden’s favour.\evb
\evg


\bvg
\bva Fjǫlð þér sagðak, \hld\ ęn þú fátt of mant, &
\ind of þik véla vinir;
mę́ki liggja \hld\ sé’k míns vinar &
\ind allan í dręyra drifinn.\eva

\bvb Much I told thee, but thou recallest little; ’tis friends that deal with thee. The sword I see, of my friend, lying all drenched in gore.\footnoteB{Weden predicts Garfrith’s imminent death.}\evb
\evg


\bvg
\bva Ęggmóðan val \hld\ nú mun Yggr hafa, &
\ind þitt vęitk líf of liðit; &
varar ro dísir, \hld\ nú knátt Óðin séa; &
\ind nálgask mik ef þú męgir.\eva

\bvb An edge-tired corpse will Ug now have; I know thy life to be passed. Wary are the dises; now thou dost see Weden—approach me, if thou mayst!\evb
\evg


\bvg
\bva Óðinn nú hęiti’k, \hld\ Yggr áðan hét’k,
\ind hétumk Þundr fyr þat,
Vakr ok Skilfingr, \hld\ Vǫ́fuðr ok Hroptatýr
\ind Gautr ok Jalkr með goðum.
Ófnir ok Sváfnir \hld\ hygg at orðnir sé
\ind allir at ęinum mér.\eva

\bvb Weden I am now called, Ug was I called earlier; I called myself Thound before that. Wacker and Shelfing, Waved and Roft-Tue, Geat and Gelding among the gods. Ofner and Sweefner, I ween, are become all for me alone.\evb
\evg


Geirröðr konungr sat ok hafði sverð um kné sér ok brugðit til miðs. En er hann heyrði at Óðinn var þar kominn stóð hann upp ok vildi taka Óðin frá eldinum. Sverðit slapp ór hendi hánum; vissu hjöltin niðr. Konungr drap fę́ti ok steyptiz áfram en sverðit stóð í gögnum hann ok fekk {hann}{þar af \AM} bana. {Óðinn hvarf þá.}{\emph{om.} \AM} En Agnarr {var þar}{varð \AM} konungr {lengi síðan.}{\emph{om.} \AM}

King Garfrith sat and had a sword about his knee, and it was brandished half-way up. But when he heard that Weden was come there, he stood up and would take Weden from the fire. The sword slipped out of his hand; the hilt pointed downwards. The king tripped and threw himself forth, but the sword pierced him, and he received his bane. Weden then disappeared, but Eyner was there king for a long while thence.
% Weden
	\bookStart{The Thule of Righ}[Rígsþula]

\begin{flushright}%
Dating \parencite{Sapp2022}: early C11th (0.240), late C11th (0.204), late C12th (0.195), C13th (0.280)

Meter: \Fornyrdislag%
\end{flushright}

Dumezil hypothesis. Irish influence? Many interesting things to write here!

\sectionline

\bpg
\bpa Svá sęgja menn í fornum sǫgum, at ęinnhvęrr af ǫ́sum, sá es Hęimdallr hét, fór fęrðar sinnar ok framm með sjóvarstrǫndu nǫkkurri, kom at ęinum húsabǿ ok nęfndisk Rígr; ęptir þęiri sǫgu es kvę́ði þetta.\epa

\bpb Thus say men in ancient \inx[C]{saw}[saws], that one of the \inx[G]{Ease}—he who was called \inx[P]{Homedall}—went on his journey forth along some lakeshore, came upon a lone homestead and called himself Righ. According to that saw is this poem.\epb
\epg


\bvg
\bva Ár kvǫ́ðu ganga \hld\ grǿnar brautir &
ǫflgan ok aldinn \hld\ ǫ́s kunnigan, &
ramman ok rǫskvan \hld\ Ríg stíganda.\eva

\bvb Of yore they said did walk the green paths, a mighty and aged \inx[G]{Ease}[os], cunning; the strong and brisk Righ, striding.\evb
\evg


\bvg
\bva Gekk hann męir at þat \hld\ miðrar brautar, &
kom hann at húsi, \hld\ hurð vas á gę́tti; &
inn nam at ganga, \hld\ ęldr vas á golfi, &
hjón sǫ́tu þar \hld\ hǫ́r at arni, &
Ái ok Ędda \hld\ aldinfalda.\eva

\bvb Went he further at that, on the middle of the road; came he to a house; the door was wide open. He took to go inside; fire was on the floor. A couple sat there, hoary by the hearth: Great Grandfather and Great Grandmother, old-fashioned.\evb
\evg


\bvg
\bva Rigr kunni þęim \hld\ rǫ́ð at sęgja; &
męir sęttisk hann \hld\ miðra flętja &
en á hlið hvára \hld\ hjón salkynna.\eva

\bvb Righ knew to tell them counsels; he further set himself down on the middle of the floor-bench, but on either side [sat] the couple of the hall.\evb
\evg


\bvg
\bva Þá tók Ędda \hld\ økkvinn hlęif, &
þungan ok þykkvan, \hld\ þrunginn sǫ́ðum, &
bar hǫ́n męir at þat \hld\ miðra skutla, &
soð vas í bolla \hld\ sętti á bjóð; &
vas kalfr soðinn \hld\ krása bęztr; &
ręis hann upp þaðan, \hld\ réðsk at sofna;\eva

\bvb Then took Great Grandmother a lumpy loaf, heavy and thick, stuffed with chaff. She carried it further at that on the middle of a trencher—broth was in the bowl—she set it on a plate; a cooked calf was the best dainty; he rose up thence, resolved to sleep.\evb
\evg


\bvg
\bva Rigr kunni þęim \hld\ rǫ́ð at sęgja; &
męir lagðisk hann \hld\ miðrar rękkju, &
en á hlið hvára \hld\ hjón salkynna.\eva

\bvb Righ knew to tell them counsels; he further laid himself down in the middle of the bed, but on either side [lay] the couple of the hall.\evb
\evg


\bvg
\bva Þar var hann at þat \hld\ þríar nę́tr saman; &
gekk hann męirr at þat \hld\ miðrar brautar; &
liðu męirr at þat \hld\ mǫ́nuðr níu.\eva

\bvb There was at that for nine nights in all; went he further at that, on the middle of the road; passed further at that nine months.\evb
\evg


\bvg
\bva Jóð ól Ędda, \hld\ jósu vatni &
\edtext{hǫrundsvartan}{\lemma{hǫrundsvartan ‘swarthy of skin’}\Afootnote{\emph{emend.}; hǫrvi svartan ‘swarthy with flax’ \Wormianus}}, \hld\ hétu Þrę́l.\eva

\bvb Great Grandmother begot a child, they poured it with water\footnoteB{A reference to the Heathen naming ceremony, wherein water would be poured on a newborn (quite similar to the Christian baptism). Cf. \Havamal\ 156.}—swarthy of skin—they called him Thrall.\evb
\evg


\bvg
\bva Hann nam at vaxa \hld\ ok vęl dafna; &
vas þar á hǫndum \hld\ hrokkit skinn, &
kropnir knúar, \hld\ [...] &
fingr digrir, \hld\ fúlligt andlit, &
lotr hryggr, \hld\ langir hę́lar.\eva

\bvb He took to grow, and thrive well; there on his hands was skin wrinkled, knuckles crooked, [...], fingers thick, a face foul, back stooping, heels long.\evb
\evg


\bvg
\bva Nam hann męirr at þat \hld\ magns of kosta, &
bast at binda, \hld\ byrðar gørva; &
bar hann hęim at þat \hld\ hrís gęrstan dag.\eva

\bvb Took he further at that to try his power; bast to bind, burdens to make; he carried home at that brushwood on a dismal day.\footnote{The thrall had to work in even the most hostile weather.}\evb
\evg


\bvg
\bva Þar kom at garði \hld\ \edtext{gęngilbęina}{\lemma{‘gangleboned woman’}\Bfootnote{Derogatory term for somebody that only travels on their legs.}}, &
aurr vas á iljum, \hld\ armr sólbrunninn, &
niðrbjúgt vas nęf, \hld\ nęfndisk \edtext{Þír}{\lemma{Þír ‘Thew’}\Bfootnote{The name probably means ‘maid-servant’ or ‘female slave’. Unlike Thrall, it is not attested in any prose texts, but probably corresponds to OS \emph{thiwi} ‘maid(-servant)’, being further root-related to \emph{þéa \char`~ þjá} ‘to enthral’, Proto-Norse \textbf{þewaʀ} ‘servant’, OE \emph{þéow} ‘slave, servant’,.}}.\eva

\bvb There came to the farm a gangleboned woman; mud was on her footsoles, her arm sunburnt; downturned was her face; she called herself Thew.\evb
\evg


\bvg
\bva Męir settisk hǫ́n \hld\ miðra flętja, &
sat hjá hęnni \hld\ sonr húss, &
rǿddu ok rýndu, \hld\ rękkju gęrðu &
Þrę́ll ok Þír \hld\ þrungin dǿgr.\eva

\bvb She further set himself down on the middle of the floor-bench; beside her sat the son of the house \ken*{= Thrall}. They spoke and whispered; made a bed—Thrall and Thew—on hard-pressed days.\evb
\evg


\bvg
\bva Bǫrn ólu þau, \hld\ bjuggu ok unðu; &
hygg’k at héti \hld\ Hręimr ok Fjósnir, &
Klúrr ok Klęggi, \hld\ Kęfsir, Fúlnir, &
Drumbr, Digraldi, \hld\ Drǫttr ok Hǫsvir, &
Lútr ok Leggjaldi; \hld\ lǫgðu garða, &
akra tǫddu, \hld\ unnu at svínum, &
gęita gę́ttu, \hld\ grófu torf.\eva

\bvb Children they begot, they settled and were content. I judge that they were called Ream and Feasner, Clour and Cleg, Chafser, Foulner, Drumber, Dighrald, Draught and Hazer, Lout and Leggald, they laid yard-fences, dunged fields, fed swine, tended to goats, dug turf.\evb
\evg


\bvg
\bva Dǿtr vǫ́ru þę́r \hld\ Drumba ok Kumba, &
Økkvinkalfa \hld\ ok Arinnęfja, &
Ysja ok Ambǫ́tt, \hld\ Ęikintjasna, &
Tǫtrughypja \hld\ ok Trǫnubęina; &
þaðan eru komnar \hld\ þræla ættir.\eva

\bvb The daughters were Drumb and Cumb; .\evb
\evg


\bvg
\bva VERSE.\eva

\bvb Translation.\evb
\evg


\bvg
\bva VERSE.\eva

\bvb Translation.\evb
\evg


\bvg
\bva VERSE.\eva

\bvb Translation.\evb
\evg


\bvg
\bva VERSE.\eva

\bvb Translation.\evb
\evg


\bvg
\bva VERSE.\eva

\bvb Translation.\evb
\evg


\bvg
\bva VERSE.\eva

\bvb Translation.\evb
\evg


\bvg
\bva VERSE.\eva

\bvb Translation.\evb
\evg


\bvg
\bva VERSE.\eva

\bvb Translation.\evb
\evg


\bvg
\bva VERSE.\eva

\bvb Translation.\evb
\evg


\bvg
\bva VERSE.\eva

\bvb Translation.\evb
\evg


\bvg
\bva VERSE.\eva

\bvb Translation.\evb
\evg


\bvg
\bva VERSE.\eva

\bvb Translation.\evb
\evg


\bvg
\bva VERSE.\eva

\bvb Translation.\evb
\evg


\bvg
\bva VERSE.\eva

\bvb Translation.\evb
\evg


\bvg
\bva VERSE.\eva

\bvb Translation.\evb
\evg


\bvg
\bva VERSE.\eva

\bvb Translation.\evb
\evg


\bvg
\bva VERSE.\eva

\bvb Translation.\evb
\evg


\bvg
\bva VERSE.\eva

\bvb Translation.\evb
\evg


\bvg
\bva VERSE.\eva

\bvb Translation.\evb
\evg


\bvg
\bva VERSE.\eva

\bvb Translation.\evb
\evg


\bvg
\bva VERSE.\eva

\bvb Translation.\evb
\evg


\bvg
\bva VERSE.\eva

\bvb Translation.\evb
\evg


\bvg
\bva VERSE.\eva

\bvb Translation.\evb
\evg


\bvg
\bva VERSE.\eva

\bvb Translation.\evb
\evg


\bvg
\bva VERSE.\eva

\bvb Translation.\evb
\evg


\bvg
\bva VERSE.\eva

\bvb Translation.\evb
\evg


\bvg
\bva VERSE.\eva

\bvb Translation.\evb
\evg


\bvg
\bva VERSE.\eva

\bvb Translation.\evb
\evg


\bvg
\bva VERSE.\eva

\bvb Translation.\evb
\evg


\bvg
\bva VERSE.\eva

\bvb Translation.\evb
\evg


\bvg
\bva VERSE.\eva

\bvb Translation.\evb
\evg


\bvg
\bva VERSE.\eva

\bvb Translation.\evb
\evg


\bvg
\bva VERSE.\eva

\bvb Translation.\evb
\evg


\bvg
\bva VERSE.\eva

\bvb Translation.\evb
\evg


\bvg
\bva VERSE.\eva

\bvb Translation.\evb
\evg


\bvg
\bva VERSE.\eva

\bvb Translation.\evb
\evg


\bvg
\bva VERSE.\eva

\bvb Translation.\evb
\evg


\bvg
\bva VERSE.\eva

\bvb Translation.\evb
\evg


\bvg
\bva VERSE.\eva

\bvb Translation.\evb
\evg


\bvg
\bva VERSE.\eva

\bvb Translation.\evb
\evg


\bvg
\bva VERSE.\eva

\bvb Translation.\evb
\evg


\bvg
\bva VERSE.\eva

\bvb Translation.\evb
\evg


\bvg
\bva VERSE.\eva

\bvb Translation.\evb
\evg


\bvg
\bva VERSE.\eva

\bvb Translation.\evb
\evg


\bvg
\bva VERSE.\eva

\bvb Translation.\evb
\evg


\bvg
\bva VERSE.\eva

\bvb Translation.\evb
\evg


\bvg
\bva VERSE.\eva

\bvb Translation.\evb
\evg


\bvg
\bva VERSE.\eva

\bvb Translation.\evb
\evg


\bvg
\bva VERSE.\eva

\bvb Translation.\evb
\evg


\bvg
\bva VERSE.\eva

\bvb Translation.\evb
\evg


\bvg
\bva VERSE.\eva

\bvb Translation.\evb
\evg


\bvg
\bva VERSE.\eva

\bvb Translation.\evb
\evg


\bvg
\bva VERSE.\eva

\bvb Translation.\evb
\evg


\bvg
\bva VERSE.\eva

\bvb Translation.\evb
\evg
% Weden
	\bookStart{The Leed of Hoarbeard}[Hárbarðsljóð]

In my opinion the poem can be seen as an allegory on class relations, namely between the self-owning Norwegian and later Icelandic farmers, and the warlike Norwegian earls.

Of all Eddic poems this one is probably the strangest in terms of form. Verse length varies greatly, and many of the lines (see especially the final verse) are of an obscene length reminiscent of late continental Germanic poems like the Heliand; some simply have no metrical qualities at all. The young clitic definite is (uniquely) employed frequently throughout the poem. These criteria would seem to point towards a late origin for the poem (though not later than the late 13th century, when \Regius\ was written).

Against this late origin speaks the presence of rare words (e.g. \emph{ǫgurr} v. 13) and a thorough understanding of the personalities of the two gods which would seem unlikely to stem from several centuries after the conversion of Iceland. The model devised by Sapp gives the poem a 57.8\% likelihood of being from the early 11th century, and a 37.7\% likelihood of being from the late 11th. These scores are most similar to those obtained by \Gripisspa, a poem that on the surface seems much more archaic.

What could we then be dealing with? It may of course be that the poet is heavily corrupt, but there is really no good evidence for this (apart from the above-mentioned irregularities). Most lines are readily understandable and fit well within their respective context and the poem as a whole. I think a better solution to this problem is that the poem has been acted out as a sort of carnivalesque theatre, with two masked actors, each playing one of the gods. This would explain the variations in meter and line length, and the prose; some lines were simply shouted out, and the lack of alliteration in these still gives a powerful, discordant effect when read aloud.

This is shown also by uses of the word ‘here’ in vv. 9 and 14. TODO: mention concept of "double scene" by Lars Lönnroth?


\sectionline


\bpg
\bpa Þórr fór ór austrvegi ok kom at sundi einu. Ǫðrum megum sundsins var ferjukarlinn með skipit. Þórr kallaði:\epa

\bpb Thunder journeyed out of the eastern ways and came to a sound. At the other side of the sound was the ferryman with the ship. Thunder called out:\epb
\epg


\bvg
\bva „Hvęrr ’s sá svęinn svęina \hld\ es stęndr fyr sundit handan?“\eva

\bvb “Who is that swain of swains, that stands across the sound?”\evb
\evg


\bvg
\bva Hann svaraði:
„Hvęrr ’s sá karl karla \hld\ es kallar of váginn?“\eva

\bvb He answered:
“Who is that churl of churls, that calls out over the wave?”\evb
\evg


\bvg
\bva „Fęr þú mik of sundit, \hld\ fǿði’k þik á morgun; &
męis hęfi’k á baki, \hld\ verðr-a matrinn bętri. &
Át ek í hvíld \hld\ áðr ek hęiman fór, &
síldr ok hafra; \hld\ saðr em’k ęnn þęss.“\eva

\bvb [Thunder:]
“Ferry me over the sound, I feed thee in the morning! A basket I have on my back; the food does not get better.\footnoteB{i.e. ‘you will not get better food than that.’} I ate for a while before I journeyed from home, herring and hegoats; I am still full from that.”\evb
\evg


\bvg
\bva „Árligum verkum \hld\ hrósar þú vęrðinum; &
\ind vęizt-at-tu fyr gǫrla, &
dǫpr ’ru þín hęimkynni, \hld\ dauð hygg’k at þín móðir sé.“\eva

\bvb “[In place of] early works boastest thou of thy eating! Thou knowest not the future clearly; dismal is the state of thy home, dead I think thy mother might be.”\evb
\evg


\bvg
\bva „Þat sęgir þú nú \hld\ es hvęrjum þikkir &
męst at vita— \hld\ at mín móðir dauð sé.“\eva

\bvb “Thou now sayest that which to each man seems most important to know: that my mother might be dead!”\evb
\evg


\bvg
\bva „Þęygi ’s sem þú \hld\ þrjú bú ęigir góð; &
bęrbęinn þú stęndr \hld\ ok hęfir brautinga gørvi, &
\ind þat-ki at þú hafir brę́kr þínar.“\eva

\bvb “’Tis hardly as if thou might own three good homesteads; bare-legged thou standest, and hast the gear of a tramp; ’tis not even as if thou have thy own breeches!”\evb
\evg


\bvg
\bva „Stýrðu hingat ęikjunni, \hld\ ek mun þér stǫðna kęnna &
eða hvęrr á skipit \hld\ es þú hęldr við landit?“\eva

\bvb “Steer hither the boat! I will show thee to the harbour—or who owns the ship which thou holdest by the shore?”\evb
\evg


\bvg
\bva „Hildólfr sá hęitir \hld\ es mik halda bað, &
rekkr inn ráðsvinni \hld\ es býr í Ráðsęyjarsundi; &
bað-at hann hlęnnimęnn flytja \hld\ eða hrossaþjófa, &
góða ęina \hld\ ok þá’s ek gørva kunna; &
sęg-ðu til nafns þíns \hld\ ef þú vill of sundit fara.“\eva

\bvb “Hildolf is called he who asked me to hold it, the counsel-wise man who lives in Redeseysound. He did not bid me to carry thief-men, nor horse-thiefs; good men only, and those whom I know well—state thy name if thou wilt fare o’er the sound!”\evb
\evg


\bvg
\bva „Sęgja mun’k til nafns míns \hld\ þótt ek sękr sjá’k &
ok til alls øðlis: \hld\ Ek em Óðins sonr, &
Męila bróðir \hld\ ęn Magna faðir, &
þrúðvaldr goða \hld\ við Þór knátt-u hér dǿma!
Hins vil’k nú spyrja \hld\ hvat þú hęitir?“\eva

\bvb “I will state my name—[and would] even if I were outlawed—and all my origin: I am Weden’s son, Male’s brother and Main’s father, the strength-wielder of the Gods; with Thunder thou here speakest! This I will now ask, what thou art called?”\evb
\evg


\bvg
\bva „Hárbarðr ek hęiti, \hld\ hyl’k of nafn sjaldan.“\eva

\bvb “Hoarbeard I am called, seldom I conceal my name.”\evb
\evg


\bvg
\bva „Hvat skalt-u of nafn hylja \hld\ nema þú sakar ęigir?“\eva

\bvb “Why shalt thou conceal thy name, unless thou be guilty of crime?”\evb
\evg


\bvg
\bva „En þótt ek sakar ęiga \hld\ fyr slíkum sem þú est &
þá mun’k forða fjǫrvi mínu \hld\ nema ek fęigr sé.“\eva

\bvb “Even though I were guilty of crime, for such a one as thou art I would still protect by life, unless I be \inx[C]{fey}.”\evb
\evg


\bvg
\bva „Harm ljótan mér þikkir í því &
at vaða of váginn til þín \hld\ ok vę́ta \edtrans{ǫgur}{burden}{\Bfootnote{The sense of this word is not clear, though it is probably the same as the first element of the compound \emph{ǫgurstund} ‘burdensome hour’, found in \Volundarkvida\ 42. Some authors have read it as a crude euphemism for ‘penis’, which would not be out of character for this poem. I however consider the best interpretation to be that of an author whose name I've forgotten (TODO!), namely that Thunder is referring to the food he carries on his back (cf. v. 3).}} minn; &
skylda’k launa kǫgursveini þínum kanginyrði \hld\ ef ek komumk yfir sundit.“\eva

\bvb “An ugly harm it seems to me to wade o’er the wave to thee, and wet my burden. I would repay thee, swaddle-swain, for thy mocking words if myself I could bring over the sound.”\evb
\evg


\bvg
\bva „Hér mun ek standa \hld\ ok þín heðan bíða; &
fannt-a-tu mann inn harðara \hld\ at Hrungni dauðan.“\eva

\bvb “Here I will stand, and hence await thee; thou foundest not a harder man since the death of \inx[P]{Rungner}!\footnoteB{Rungner was an ettin slain by Thunder, TODO. Hoarbeard’s mentioning of him sets off a long interchange, wherein the two boast of their deeds, and ask what the other one was doing meanwhile.}”\evb
\evg


\bvg
\bva „Hins vilt-u nú geta \hld\ es vit Hrungnir dęildum, &
sá inn stórúðgi jǫtunn, \hld\ es ór stęini vas hǫfuðit á, &
þó lét’k hann falla \hld\ ok fyr hníga; &
\ind hvat vannt-u þá meðan, Hárbarðr?“\eva

\bvb “This wilt thou now mention, of when I and Rungner dealt with each other; that great-minded ettin on which the head was made of stone. Yet I let him fall, and sink down before [me]—what didst thou then meanwhile, Hoarbeard?”\evb
\evg


\bvg
\bva „Vas’k með Fjǫlvari \hld\ fimm vetr alla &
í ęy þeiri \hld\ er Algrǿn hęitir; &
vega vér þar knǫ́ttum \hld\ ok val fęlla, &
margs at fręista, \hld\ mans at kosta.“\eva

\bvb “I was with Felwar for five winters all in that island which Allgreen is called. There we knew to fight, and fell corpses; many to tempt, a girl to win.\footnoteB{I read \emph{margs} ‘many a’ as modifying \emph{mans} ‘girl’, thus giving ‘(we knew) to tempt and to win many a girl’.}”\evb
\evg


\bvg
\bva „Hversu snúnuðu yðr konur yðrar?“\eva

\bvb “How did your women pleasure (TODO!!!) you?.\footnoteB{Seemingly a prose line; see Introduction.}”\evb
\evg


\bvg
\bva „Sparkar ǫ́ttum vér konur \hld\ ef oss at spǫkum yrði; &
horskar ǫ́ttum vér konur \hld\ ef oss hollar vę́ri, &
þę́r ór sandi \hld\ síma undu &
\ind ok ór dali djúpum &
\ind grund of grófu; &
varð’k þęim ęinn ǫllum \hld\ øfri at rǫ́ðum; &
\ind hvílda’k hjá systrum sjau &
\ind ok hafða’k gęð þęira allt ok gaman;
\ind hvat vannt-u þá meðan, Þórr?“\eva

\bvb “We \ken*{I} owned frisky women, if they were pleasing towards us \ken*{me}; we \ken*{I} owned wise women, if they were \inx[C]{hold} towards us \ken*{me}; out of the sand a rope they wound, and out of a deep dale dug up the ground; I alone became superior to all of them in counsels; I rested by those sisters seven, and had their senses all, and pleasure—what didst thou then meanwhile, Thunder?”\evb
\evg


\bvg
\bva „Ek drap Þjaza, \hld\ hinn þrúðmóðga jǫtun, &
upp ek varp augum \hld\ Allvalda sonar &
\ind á þann hinn hęiða himin; &
þau ’ru męrki męst \hld\ minna verka, &
\ind þau’s allir męnn síðan of sé; &
\ind hvat vannt-u þá meðan, Hárbarðr?“\eva

\bvb “I slew \inx[C]{Thedse}, the strength-minded ettin; up I threw the eyes of the son of Allwald \ken*{= Thedse} onto that clear heaven; those are the greatest marks of my works, those that all men since do see\footnoteB{We here have a rare example of native Germanic star-lore. Is the exact constellation identifiable? TODO.}—what didst thou then meanwhile, Hoarbeard?”\evb
\evg


\bvg
\bva „Miklar manvélar \hld\ hafða’k við myrkriður &
\ind þá’s ek vélta þę́r frá verum; &
harðan jǫtun \hld\ hugða’k Hlébarð vesa; &
\ind gaf hann mér gambantęin &
\ind en ek vélta hann ór viti.“\eva

\bvb “Great girl-tricks I used against \inx[C]{murkriders}, when I tricked them away from their husbands.\footnoteB{Alternatiely ‘away from men’. The \emph{riður} ‘(female) riders’ were witches thought to torment people and cause disease and suffering. See \Havamal\ 154 for a more detailed explanation.} A hard ettin I judged Leebeard to be; he gave me a \inx[C]{gombentoe}, but I tricked him out of his wits.”\evb
\evg


\bvg
\bva „Illum huga launaðir þú þá góðar gjafar.“\eva

\bvb “With an evil mind rewardedst thou that good gift.”\evb
\evg


\bvg
\bva „Þat hęfir ęik \hld\ es af annarri skęfr; &
\ind umb sik es hvęrr í slíku; &
\ind hvat vannt-u þá meðan, Þórr?“\eva

\bvb “An oak has that which it scrapes from another; each is for himself in such [a matter]—what didst thou then meanwhile, Thunder?”\evb
\evg


\bvg
\bva „Ek vas austr \hld\ ok jǫtna barða’k &
brúðir bǫlvísar \hld\ es til bjargs gengu; &
mikil myndi ę́tt jǫtna \hld\ ef allir lifði, &
vę́tr myndi manna \hld\ undir Miðgarði; &
hvat vannt-u þá meðan, Hárbarðr?\eva

\bvb “I was in the east, and ettins I fought; bale-wise brides who walked to the mountain. Great would the lineage of ettins be if all lived; naught would remain of men within Middenyard—what didst thou then meanwhile, Hoarbeard?”\evb
\evg


\bvg
\bva „Vas’k á Vallandi \hld\ ok vígum fylgða’k, &
atta ek jǫfrum \hld\ en aldrigi sę́tta’k; &
Óðinn á jarla \hld\ þá’s í val falla &
\ind en Þórr á þrę́la kyn.“\eva

\bvb “I was in \inx[L]{Walland} and followed conflicts; I incited princes, and never reconciled them. Weden owns the earls which fall among the slain, but Thunder owns the kin of thralls.\footnoteB{We see here a sort of aristocratic, Odinic disregard for lower life and life as a good in itself; where Thunder boasts of saving men, Weden sarcastically responds that he caused the deaths of men so that he could have them for himself.}”\evb
\evg


\bvg
\bva „Ójafnt skipta \hld\ es þú myndir með ǫ́sum liði &
\ind ef þú ę́ttir vilgi mikils vald.“\eva

\bvb “Translation.”\evb%TODO: There’s something very weird going on here.
\evg


\bvg
\bva „Þórr á afl ǿrit \hld\ en ękki hjarta; &
af hrę́ðslu ok hugblęyði \hld\ þér vas í hanzka troðit &
\ind ok þóttisk-a þú þá Þórr vesa; &
hvárki þú þá þorðir \hld\ fyr hrę́ðslu þinni &
hnjósa né físa \hld\ svá’t Fjalarr hęyrði.“\eva

\bvb “Thunder owns ample strength, but no heart; out of fear and mind-softness didst thou tread into a glove, and then seemedest thou not to be Thunder. Thou daredest neither—for thy fear—to sneeze nor to fart so that Feller might hear [it].\footnoteB{This story is also referenced in \Lokasenna\ 60. It is elaborated heavily on in \Gylfaginning\ 45: Thunder, Lock, and the siblings Thelve and Wrash had travelled east for a long time when they discovered a large hall, with an opening on one end, as wide as the building. They took rest inside, but in the middle of the night there was a great earthquake and the ground beneath them trembled. Thunder rose and led the party to a side-room to the right in the middle of the hall. He sat closest to the opening with his hammer ready, while the others sat terrified further inside. At daybreak they left the hall and found a huge ettin named \emph{Skrymir} (\inx[P]{Shrimer}) sleeping next to them. His snoring had caused the earth-quakes, and the hall was his mitten; the side-room was the thumb-part.}”\evb
\evg


\bvg
\bva „Hárbarðr hinn ragi, \hld\ munda’k þik í Hęl drepa &
\ind ef mę́tta’k sęilask of sund.“\eva

\bvb “Hoarbeard the \inx[C]{degenerate}, I would strike thee into \inx[L]{Hell}, if I might sail o’er the sound!”\evb
\evg


\bvg
\bva „Hvat skyldir of sund sęilask \hld\ es sakir ’ru allz øngar? &
\ind hvat vannt-u þá meðan, Þórr? “\eva

\bvb “Why should thou sail o’er the sound when there are no offenses?—what didst thou then meanwhile, Thunder?”\evb
\evg


\bvg
\bva „Ek vas austr \hld\ ok ána varða’k &
þá’s mik sóttu \hld\ þęir Svárangs synir; &
grjóti mik bǫrðu, \hld\ gagni urðu þó lítt fęgnir, &
þó urðu mik fyrri \hld\ friðar at biðja. &
\ind hvat vannt-u þá meðan, Hárbarðr?“\eva

\bvb “I was in the east, and warded the river, when the sons of Sweering attacked me. With rocks they fought me, yet they rejoiced little in victory; yet they earlier had to beg me for peace—what didst thou then meanwhile, Hoarbeard?”\evb
\evg


\bvg
\bva „Ek var austr \hld\ ok við ęinhvęrja dǿmða’k, &
lék’k við ina lindhvítu \hld\ ok lǫng þing háða’k, &
gladda’k ina gullbjǫrtu, \hld\ gamni mę́r unði.“\eva

\bvb “I was in the east, and with a certain woman conversed; I played with the linen-white one, and held long \inx[C]{Thing}[Things]; I gladdened the gold-bright one; the maiden enjoyed pleasure.”\evb
\evg


\bvg
\bva „Góð ǫ́ttu þęir mankynni þar þá.“\eva

\bvb “Then they had good girl-visits there.”\evb
\evg


\bvg
\bva „Liðs þíns vę́ra’k þá þurfi, Þórr, \hld\ at hęlda’k þęiri inni línhvítu męy.“\eva

\bvb “Of thy help I might have been in need then, Thunder, that I might hold that linen-white maiden.”\evb
\evg


\bvg
\bva „Ek mynda þér þá þat vęita \hld\ ef ek viðr of kę́misk.“\eva

\bvb “I would then have granted thee that, if I were able.”\evb
\evg


\bvg
\bva „Ek mynda þér þá trúa, \hld\ nema mik í tryggð véltir.“\eva

\bvb “I would then have trusted thee, unless thou betrayed my trust.”\evb
\evg


\bvg
\bva „Em’k-at ek sá hę́lbítr \hld\ sem húðskór forn á vár.“\eva

\bvb “I am not such a heel-biter as an old hide-shoe in spring.\footnoteB{Proverbial (a heel-biter being someone who betrays his companions); the leather of a shoe would become very stiff and chafing over the winter.}”\evb
\evg


\bvg
\bva „Brúðir bersęrkja \hld\ barða’k í Hléseyju; &
þę́r hǫfðu vęrst unnit, \hld vélta þjóð alla.“\eva

\bvb “The brides of bearserks I fought in Leesie; they had done the worst: deceived a whole people.”\evb
\evg


\bvg
\bva „Klę́ki vannt-u þá, Þórr, \hld\ es þú á konum barðir.“\eva

\bvb “A great disgrace didst thou then, Thunder, when thou foughtst women.”\evb
\evg


\bvg
\bva „Vargynjur vǫ́ru þę́r \hld\ en varla konur, &
skęlldu skip mitt \hld\ es ek skorðat hafða’k, &
ǿgðu mér járnlurki \hld\ en ęltu Þjálfa. &
hvat vannt-u þá meðan, Hárbarðr?“\eva

\bvb “She-wolves were they, but hardly women; they knocked my ship which I had propped; frightened me with an iron-cudgel, but chased Thelve around—what didst thou then meanwhile, Hoarbeard?”\evb
\evg


\bvg
\bva „Ek vas’k í hęrnum \hld\ es hingat gjǫrðisk &
gnę́fa gunnfana, \hld\ gęir at rjóða.“\eva

\bvb “I was in the army, as hence it made ready to raise the war-standard; to redden the spear.”\evb
\evg


\bvg
\bva „Þess vilt-u nú geta, es þú fórt oss \edtext{óljúfan}{\footnoteB{oliyfan \AM; †olubann† \Regius}} at bjóða.“\eva

\bvb “This wilt thou now mention, as thou wentest to bid us \ken*{= the Ease} hatred!”\evb
\evg


\bvg
\bva „Bǿta skal þér þat þá \hld\ munda baugi &
sem jafnęndr unnu \hld\ þęir’s okkr vilja sę́tta.“\eva

\bvb “I will then restore thee for that with a hand-bigh, like the settlers [have] considered, those who wish to reconcile us.”\evb
\evg


\bvg
\bva „Hvar namt þęssi \hld\ in hnǿfiligu orð &
es ek hęyrða aldrigi \hld\ hnǿfiligri?“\eva

\bvb “Where learnedest thou these sarcastic words, as I never heard more sarcastic ones?”\evb
\evg


\bvg
\bva „Nam’k at mǫnnum þęim inum aldrǿnum es búa í hęimisskógum.“\eva

\bvb “I learned them from the old men who dwell in the home-forests.”\evb
\evg


\bvg
\bva „Þó gefr þú gótt nafn dysjum, es þú kallar þat hęimisskóga.“\eva

\bvb “Yet thou givest a good name to poor cairns,\footnoteB{cf. his waking the dead in various poems TODO.} as thou callest them home-forests.”\evb
\evg


\bvg
\bva „Svá dǿmi’k of slíkt far.“\eva

\bvb “So I speak about such things.”\evb
\evg


\bvg
\bva „Orðkringi þín \hld\ mun þér illa koma &
\ind ef ek rę́ð á vág at vaða; &
ulfi hę́ra \hld\ hygg’k at ǿpa mynir &
ef hlýtr af hamri hǫgg.“\eva

\bvb “Thy word-glibness will bring thee evil, if I resolve to wade on the wave; higher than a wolf I think that thou wilt scream, if thou suffer a strike from the hammer.”\evb
\evg


\bvg
\bva „Sif á hó hęima, \hld\ hans munt fund vilja, &
þann munt þręk drýgja, \hld\ þat ’s þér skyldara.“\eva

\bvb “Sib has a whoremonger at home, him wilt thou wish to meet; then shalt thou use thy strength, that is thee more befitting!”\evb
\evg


\bvg
\bva „Mę́lir þú at munns ráði \hld\ svá’t mér skyldi vęrst þikkja, &
halr inn hugblauði, \hld\ hygg’k at þú ljúgir.“\eva

\bvb “Thou speakest to the counsel of thy mouth that which would seem me the worst; heart-soft man, I think that thou liest!”\evb
\evg


\bvg
\bva „Satt hygg’k mik sęgja, \hld\ sęinn est at fǫr þinni, &
langt myndir nú kominn, Þórr, \hld\ ef þú \edtrans{litum fǿrir}{brought thy colours}{\Bfootnote{Very unclear expression. \emph{fǿra litum} TODO.}}.“\eva

\bvb “I think myself to speak truly: late art thou in thy journey; far would thou now be come, Thunder, if thou had brought thy colours.”\evb
\evg


\bvg
\bva „Hárbarðr inn ragi, \hld\ hęldr hęfir nú mik dvalðan!“\eva

\bvb “Hoarbeard the degenerate; thou hast now delayed me greatly!”\evb
\evg


\bvg
\bva „Ása-Þórs \hld\ hugða’k aldrigi myndu &
\ind glępja féhirði farar.“\eva

\bvb “The journey of Thunder of the Ease I never thought that a shepherd \ken*{= I} would divert.”\evb
\evg


\bvg
\bva „Ráð mun’k þér nú ráða: \hld\ Ró þú hingat bátinum, &
hę́ttum hǿtingi, \hld\ hitt fǫður Magna!“\eva

\bvb “I will now counsel thee a counsel: Row hither the boat; seize with the taunting; come to the father of Main \ken*{= Thunder = me}!”\evb
\evg


\bvg
\bva „Far þú firr sundi, \hld\ þér skal fars synja!“\eva

\bvb “Go far from the sound; the ferry shall be denied thee!”\evb
\evg


\bvg
\bva „Vísa þú mér nú lęiðina \hld\ allz þú vill mik ęigi of váginn fęrja!“\eva

\bvb “Show me now the path, as thou wilt not ferry me o’er the wave!”\evb
\evg


\bvg
\bva „Lítit ’s at synja, \hld\ langt ’s at fara; &
stund ’s til stokksins, \hld\ ǫnnur til stęinsins, &
halt svá til vinstra vegsins \hld\ unz þú hittir Verland; &
þar mun Fjǫrgyn \hld\ hitta Þór, son sinn,
ok mun hǫ́n kęnna hǫ́num ǫ́ttunga brautir \hld\ til Óðins landa.“\eva

\bvb “’Tis little to deny, ’tis long to journey: an hour to the log, another to the stone; hold thus to the left road, until thou findest Wereland; there will Firgyn find Thunder, her son, and she will teach him the highways of her ancestors, to Weden’s lands \ken*{= Osyard}.”\evb
\evg


\bvg
\bva „Mun’k taka þangat í dag?“\eva

\bvb “Will I come thither today?”\evb
\evg


\bvg
\bva „Taka við víl ok ęrfiði \hld\ at uppverandi sólu &
es ek get þána.“\eva

\bvb “[Thou wilt] come with toil and hardship at the rising of the sun, as I think it might thaw.”\evb
\evg


\bvg
\bva „Skammt mun nú mál okkat vesa, \hld\ allz þú mér skǿtingu ęinni svarar; &
launa mun ek þér farsynjun \hld\ ef vit finnumk í sinn annat. &
Far þú nú þar’s þik hafi allan gramir!“\eva

\bvb “Short will now our speech be, as thou answerest me with scoffing alone; I will reward thee for this ferry-denial if we meet another time. Now go, whither the fiends may have all of thee!”\evb
\evg
% Weden, Thunder
	\bookStart{The Lay of Thrim}[Þrymskviða]

% Introduction

Compare \Haustlong, \Hymiskvida, other poems and refer to the SkP intro to one of the big Thunder poems. TODO.

\bvg
\bva \edtext{\alst{V}ręiðr}{\lemma{Vręiðr}\Afootnote{TODO: Note about ambiguity of alliteration.}} vas þá Ving-Þórr \hld\ es hann vaknaði &
ok síns hamars \hld\ of saknaði, &
skegg nam at hrista, \hld\ skǫr nam at dýja, &
réð Jarðar burr \hld\ umb at þreifask.\eva

\bvb Wroth was then Wing-Thunder when he woke, and of his hammer was bereaved. His beard he took to shake, his locks he took to pull; resolved the son of Earth to look about.\evb
\evg


\bvg
\bva Ok hann þat orða \hld\ allz fyrst of kvað: &
“Hęyrðu nú, Loki, \hld\ hvat ek nú mę́li &
es ęigi vęit \hld\ jarðar hvęrgi &
né upphimins: \hld\ áss es stolinn hamri!”\eva

\bvb And he that word first of all did speak: “Hear thou now, Lock, what I now speak, which nowhere is known, not on earth nor \inx[L]{Up-heaven}:\footnoteB{A common Germanic poetic formula, see Index: \inx[L]{Earth and Up-heaven}.} the \inx[G]{Ease}[os] \ken*{= Thunder = I} has been robbed of his hammer!”\evb
\evg


\bvg
\bva Gengu þęir fagra \hld\ Fręyju túna &
ok hann þat orða \hld\ allz fyrst of kvað: &
“Muntu mér, Fręyja, \hld\ fjaðrhams ljá &
ef ek mínn hamar \hld\ mę́tta’k hitta?”\eva

\bvb Went they to the fair yards of \inx[P]{Frow}, and he that word, first of all did speak: “Wilt thou me, Frow, the \inx[P]{feather-hame} lend, if I my hammer might find?”\evb
\evg


\bvg {\small [Frow quoth:]}
\bva “Þó mynda’k gefa þér \hld\ þótt ór gulli vę́ri &
ok þó sęlja \hld\ at vę́ri ór silfri.”\eva

\bvb “I would yet give it to thee, though it were out of gold, and yet offer\footnoteB{\emph{sęlja} ‘sell’ here has its earlier meaning, cf. Gothic \emph{saljan} ‘\emph{opfern}; θύειν’ (Streitberg 1910:116).} it to thee, as it were out of silver.”\footnoteB{Regaining the hammer is of such importance to the gods (cf. v. 17; without it the Ease stand powerless against the \inx[G]{Ettins}), that Frow would lend the feather-hame to the greedy and untrusty Lock, even if it were made out of solid gold or silver.}\evb
\evg

\bvg
\bva Fló þá Loki, \hld\ fjaðrhamr dunði, &
unz fyr útan kom \hld\ ása garða &
ok fyr innan kom \hld\ jǫtna hęima.\eva

\bvb Flew then Lock\footnoteB{Though Thunder is the one asking for the hame (“if I \emph{my} hammer might find”), Lock is the one that takes off flying.}—the feather-hame rustled—until outside he came of the \inx[L]{Osyard}[yards of the Ease], and inside he came of the \inx[L]{Ettinham}[homes of the Ettins].\evb
\evg


\bvg
\bva Þrymr sat á haugi, \hld\ þursa dróttinn, &
gręyjum sínum \hld\ gullbǫnd snøri &
ok mǫrum sínum \hld\ mǫn jafnaði.\eva

\bvb Thrim sat on the howe, the lord of \inx[G]{Thurses}: on his greyhounds the golden leashes he twirled, and on his mares the manes he cut even.\evb
\evg


\bvg
\bva „Hvat es með ǫ́sum? \hld\ Hvat es með ǫlfum? &
Hví estu ęinn kominn \hld\ í jǫtunhęima?“ &
„Illt es með ǫ́sum, \hld\ \edtext{illt es með ǫlfum!}{\Bfootnote{Inserted in analogy with the first pair, regardless it is needed for metrical reasons.}} &
Hęfir þú Hlórriða \hld\ hamar of folginn?“\eva

\bvb [Thrim quoth:] “What is with the Ease? What is with the elves? Why art thou alone come into the \inx[L]{Ettinham}[Ettin-homes]?” — [Lock quoth:] “’Tis ill with the Ease, ’tis ill with the elves! Hast thou the hammer of Loride \name{= Thunder} hidden?”\evb
\evg


\bvg {\small [Thrim quoth:]}
\bva „Ek hęfi Hlórriða \hld\ hamar of folginn &
átta rǫstum \hld\ fyr jǫrð neðan; &
hann ęngi maðr \hld\ aptr of hęimtir &
nęma fǿri mér \hld\ Fręyju at kvę́n.“\eva

\bvb “I have the hammer of Loride hidden, eight \inx[C]{rest}[rests] beneath the earth; it no man will fetch again, unless he bring me Frow as wife.”\evb
\evg


\bvg
\bva Fló þá Loki, \hld\ fjaðrhamr dunði, &
unz fyr útan kom \hld\ jǫtna hęima &
ok fyr innan kom \hld\ ása garða; &
mǿtti hann Þór \hld\ miðra garða &
ok þat hann orða \hld\ allz fyrst of kvað:\eva

\bvb Flew then Lock—the feather-hame rustled—until outside he came of the homes of the Ettins, and inside he came of the yards of the Ease. He met Thunder in the middle of the yards, and he \ken*{= Thunder} that word first of all did say:\evb
\evg


\bvg {\small [Thunder quoth:]}
\bva „Hęfir þú ørendi \hld\ sem ęrfiði? &
Segðu á lopti \hld\ lǫng tíðendi! &
Opt sitjanda \hld\ sǫgur of fallask &
ok liggjandi \hld\ lygi of bęllir.“\eva

\bvb “Hast thou an errand of hardship?\footnoteB{lit. “Hast thou an errand, as hardship?” Thunder asks Lock if he has bad news.} Say thou aloft, the long tidings! Often sitting, tales fail each other, and lying down, lies are dealt.”\footnoteB{Proverbial. If one sits down and thinks too much over bad news, details will be left out, excuses thought up. Thus it is best that Lock immediately tell Thunder what he has learned.}\evb
\evg


\bvg {\small [Lock quoth:]}
\bva „Hefi ek ørindi \hld\ erfiði ok: &
Þrymr hęfir þinn hamar, \hld\ þursa dróttinn; &
hann ęngi maðr \hld\ aptr of hęimtir &
nęma hǫ́num fǿri \hld\ Fręyju at kvę́n.“\eva

\bvb “I have an errand, hardship also: Thrim has thy hammer, the lord of Thurses; it no man will fetch again, unless he bring him Frow as wife.”\evb
\evg


\bvg
\bva Ganga þęir fagra \hld\ Fręyju at hitta &
ok hann þat orða \hld\ allz fyrst of kvað: &
„Bittu þik, Fręyja, \hld\ brúðar líni! &
Vit skulum aka tvau \hld\ í jǫtunhęima.“\eva

\bvb Go they the fair Frow to find, and he\footnoteB{Unclear. Possibly Lock, since he was the speaker of the last verse.} that word, first of all did say: “Bind thee, Frow, with a bride’s linen\footnoteB{A linen band tied around the bride’s head. TODO: Reference this note.}! We two shall drive into the Ettin-homes.”\evb
\evg


\bvg
\bva Vręið varð þá Fręyja \hld\ ok fnasaði, &
allr ása salr \hld\ undir bifðisk, &
stǫkk þat it mikla \hld\ męn Brísinga: &
„Mik vęizt verða \hld\ vergjarnasta &
ef ek ęk með þér \hld\ í jǫtunhęima.“\eva

\bvb Wroth became then Frow, and snorted—the whole hall of the Ease trembled below—threw she off the great necklace of the Brisings: “Thou knowest that I will become the most man-eager,\footnoteB{Either Frow is speaking out of self-awareness of her own lust, or the sense is that she will be accused of being lustful by the other gods, but there is no verb here corresponding to ‘accuse’.} if I drive with thee into the Ettin-homes.”\evb
\evg


\bvg
\bva Sęnn vǫ́ru ę́sir \hld\ allir á þingi &
ok ǫ́synjur \hld\ allar á máli, &
ok of þat réðu \hld\ ríkir tívar: &
hvé þęir Hlórriða \hld\ hamar of sǿtti.\eva

\bvb Soon were the \inx[G]{Ease} all at the \inx[C]{Thing}, and the \inx[C]{Ossens} all at speech, and of this counseled the mighty \inx[G]{Tues}:\footnoteB{Identical to \Baldrsdraumar\ 1.} how they the hammer of Loride would seek out.\evb
\evg


\bvg
\bva Þá kvað þat Heimdallr, \hld\ hvítastr ása, &
vissi hann vel framm \hld\ sęm vanir aðrir: &
„Bindu vér Þór þá \hld\ brúðar líni; &
hafi hann it mikla \hld\ męn Brísinga!\eva

\bvb Then quoth that \inx[P]{Homedall}, the whitest of the Ease; he knew well forth,\footnoteB{\emph{vita framm} ‘to know forward’ i.e. to know the future. Compare \emph{framvíss} ’forth-wise; prescient.’} like the other \inx[G]{Wanes}: “Let us bind Thunder with the bride’s linen; may he have the great \inx[P]{necklace of the Brisings}.\evb
\evg


\bvg
\bva Lǫ́tum und hǫ́num \hld\ hrynja lukla &
ok kvenváðir \hld\ umb kné falla &
en á brjósti \hld\ bręiða stęina &
ok hagliga \hld\ umb hǫfuð typpum!“\eva

\bvb Let us place by his side keys to jingle, and women’s garments to fall down about his knees, and on the breast broad stones, and skillfully let us tip his head!\footnoteB{This verse contains an interesting description of Viking age bridal dress: As the everyday manager of the household, keys were the mark of a respectable married woman. The “broad stones” on the breast are probably tortoise brooches, while the tipping of the head refers to some sort of bridal hat (TODO: Literature). Breast-brooches are also mentioned in \Volundarkvida\ 25, 36.}”\evb
\evg


\bvg
\bva Þá kvað þat Þórr, \hld\ þrúðugr áss: &
„Mik munu ę́sir \hld\ argan kalla &
ef ek bindask lę́t \hld\ brúðar líni!“\eva

\bvb Then quoth that Thunder, the mighty os: “Me would the Ease call \inx[C]{degenerate}, if I let myself be bound with bride’s linen!”\evb
\evg


\bvg
\bva Þá kvað þat Loki \hld\ Laufęyjar sonr: &
„Þęgi þú, Þórr, \hld\ þęira orða! &
Þegar munu jǫtnar \hld\ Ásgarð búa &
nęma þú þinn hamar \hld\ þér of hęimtir.“\eva

\bvb Then quoth that Lock, the son of Leafie: “Shut thou, Thunder, those words up! Shortly the Ettins will settle Osyard, unless thou thy hammer for thyself dost fetch!”\evb
\evg


\bvg
\bva Bundu þęir Þór þá \hld\ brúðar líni &
ok inu mikla \hld\ męni Brísinga, &
létu und hǫ́num \hld\ hrynja lukla &
ok kvenváðir \hld\ umb kné falla &
ęn á brjósti \hld\ bręiða stęina &
ok hagliga \hld\ of hǫfuð typpðu.\eva

\bvb Bound they Thunder then, with bride’s linen, and with the great necklace of the Brisings. They placed by his side keys to jingle, and women’s garments to fall down about his knees, and on the breast broad stones, and skillfully they tipped his head.\evb
\evg


\bvg
\bva Þá kvað þat Loki \hld\ Laufęyjar sonr: &
„Mun ek ok með þér \hld\ ambǫ́tt vesa, &
vit skulum aka tvau \hld\ í jǫtunhęima.“\eva

\bvb Then quoth that Lock, the son of Leafie: “I will also with thee be a handmaid; we two\footnoteB{The form used, \emph{tvau}, is the neuter plural, ie. one of the pair is female and the other male. This is either an error due to mindless copying of v. 11, or a backhanded insult against Thunder.} shall drive into the Ettin-homes.”\evb
\evg


\bvg
\bva Sęnn vǫ́ru hafrar \hld\ hęim of vreknir, &
skyndir at skǫklum, \hld\ skyldu vel renna; &
bjǫrg brotnuðu, \hld\ brann jǫrð loga; &
ók Óðins sonr \hld\ í jǫtunhęima.\eva

\bvb Soon \inx[C]{he-goats}\footnoteB{Thunder’s cart was driven by he-goats, and he is likewise called “the lord of he-goats” in \Hymiskvida\ 20, 31. See Index.} were driven home, hasted onto the cart-poles; they were to run well. Crags burst, the earth burned with flame; the son of Weden \ken*{= Thunder} drove into the Ettin-homes.\footnoteB{A very similar but more detailed description of Thunder driving is found in Thedwolf’s \Haustlong\ 14–16. In both poems his wagon is drawn by he-goats, causing great cosmic disturbance: crags (\emph{bjǫrg} in both) are rent asunder and fires rage before him. See also \Baldrsdraumar\ 3 for a related description of Weden riding.}\evb
\evg


\bvg
\bva Þá kvað þat Þrymr, \hld\ þursa dróttinn: &
„Standið upp, jǫtnar, \hld\ ok stráið bękki! &
Nú fǿrið mér \hld\ Fręyju at kván, &
Njarðar dóttur \hld\ ór Nóatúnum.\eva

\bvb Then quoth that Thrim, the lord of Thurses: “Stand ye up, ettins, and strew the benches! Now bring me Frow as wife; the daughter of \inx[P]{Nearth} of the \inx[L]{Nowetowns}.\evb
\evg


\bvg
\bva Ganga hér at garði \hld\ gullhyrnðar kýr, &
øxn alsvartir, \hld\ jǫtni at gamni, &
fjǫlð á’k męiðma, \hld\ fjǫlð á’k męnja; &
ęinnar mér Fręyju \hld\ ávant þykkir.“\eva

\bvb Here march to the estate golden-horned cows, all-black oxen, to the enjoyment of the ettin \ken*{= me}. A great deal I own of treasures, a great deal I own of necklaces; of Frow alone methinks is missing.”\evb
\evg


\bvg
\bva Vas þar at kveldi \hld\ of komit snimma &
ok fyr jǫtna \hld\ ǫl framm borit. &
Ęinn át oxa, \hld\ átta laxa, &
krásir allar, \hld\ þę́r’s konur skyldu, &
drakk Sifjar verr \hld\ sáld þrjú mjaðar.\eva

\bvb There was the evening come quickly, and before the ettins ale brought forth. Ate he \ken*{= Thunder} one ox, eight salmons, and all the delicacies which were meant for the women; drank the husband of Sib \ken*{= Thunder} three sieves of mead.\footnoteB{Compare \Hymiskvida\ 15 for a similar description of Thunder’s great eating.}\evb
\evg


\bvg
\bva Þá kvað þat Þrymr, \hld\ þursa dróttinn: &
„Hvar sáttu brúðir \hld\ bíta hvassara? &
Sá’k-a brúðir \hld\ bíta ęnn bręiðara &
né ęnn męira mjǫð \hld\ męy of drekka!“\eva

\bvb Then quoth that Thrim, the lord of Thurses: “Where sawest thou brides bite sharper? Saw I never brides bite yet broader, nor yet more mead a maiden drink.”\evb
\evg


\bvg
\bva Sat in alsnotra \hld\ ambǫ́tt fyr &
es orð of fann \hld\ við jǫtuns máli: &
„Át vę́tr Fręyja \hld\ átta nóttum, &
svá vas hón óðfús \hld\ í jǫtunhęima.“\eva

\bvb Sat the allclever maid-servant \ken*{= Lock} in front, when she a word did find against the speech of the ettin: “Ate Frow naught, for eight nights; so madly was she longing for the Ettin-homes.”\evb
\evg


\bvg
\bva Laut und línu, \hld\ lysti at kyssa, &
ęn hann útan stǫkk \hld\ ęndlangan sal: &
„Hví eru ǫndótt \hld\ augu Fręyju? &
Þykki mér ór \hld\ augum brenna!“\eva

\bvb He looked ’neath the linen, he lusted for a kiss, but he from the outside leapt back, across the length of the hall: “Why are the eyes of Frow fiery? Methinks there is flame coming out of the eyes!\footnoteB{Lit. “Methinks out of the eyes burn.”}”\evb
\evg


\bvg
\bva Sat in alsnotra \hld\ ambǫ́tt \edtext{fyrir}{\Afootnote{‘ſ.’ \emph{add.} \Regius \emph{possibly a lost word}}} &
es orð of fann \hld\ við jǫtuns máli: &
„Svaf vę́tr Fręyja \hld\ átta nóttum, &
svá vas hón óðfús \hld\ í jǫtunhęima.“\eva

\bvb Sat the allclever maid-servant \ken*{= Lock} in front, when she a word did find against the speech of the ettin: “Slept Frow naught, for eight nights; so madly was she longing for the Ettin-homes.”\evb
\evg


\bvg
\bva Inn kom in arma \hld\ jǫtna systir, &
hin es brúðfjár \hld\ biðja þorði: &
„Láttu þér af hǫndum \hld\ hringa rauða &
ef þú ǫðlask vill \hld\ ástir mínar, &
ástir mínar, \hld\ alla hylli!“\eva

\bvb In came the wretched sister of the ettins, the one who for the bride-price had dared ask: “Take off from thy hands the red rings, if thou wilt win my loves; my loves, [and] all [my] \inx[C]{holdness}.”\footnoteB{The sister, who already asked for the hammer, now has the audacity to ask Thunder (still disguised as Frow) to give her the very rings on his hands.}\evb
\evg


\bvg
\bva Þá kvað þat Þrymr, \hld\ þursa dróttinn: &
„Berið inn hamar \hld\ brúði at vígja, &
leggið Mjǫllni \hld\ í męyjar kné, &
vígið okkr saman \hld\ Várar hęndi!“\eva

\bvb Then quoth that Thrim, the lord of Thurses: “Bear ye in the hammer, the bride to bless; lay Millner in the maiden’s knee, bless us two together by the hand of \inx[P]{Ware}!\footnoteB{A minor goddess presumably presiding over marriage. See Index.}”\evb
\evg


\bvg
\bva Hló Hlórriða \hld\ hugr í brjósti &
es harðhugaðr \hld\ hamar of þękkði; &
Þrym drap hann fyrstan, \hld\ þursa dróttin, &
ok ę́tt jǫtuns \hld\ alla lamði.\eva

\bvb The heart of Loride laughed in his breast, when, hard-hearted, he recognized the hammer. Thrim he slew first, the lord of Thurses, and all the lineage of the ettin he thrashed.\evb
\evg


\bvg
\bva Drap hann ina ǫldnu \hld\ jǫtna systur, &
hin es brúðfjár \hld\ of beðit hafði; &
hón skell of hlaut \hld\ fyr skillinga &
en hǫgg hamars \hld\ fyr hringa fjǫlð.\eva

\bvb He slew the old sister of the ettins, the one who for the bride-price had asked; she received a smiting before shillings, and a strike of the hammer before a multitude of rings.\evb
\evg


\bvg
\bva Svá kom Óðins sonr \hld\ ęndr at hamri.\eva

\bvb Thus Weden’s son regained his hammer.\evb
\evg
% Thunder
	\bookStart{\emph{Hymiskviða} — The Lay of Hymer.}

% Introduction.
Attested in two manuscripts, \Regius\ and \AM. The two are surprisingly consistent.

Þórr dró Miðgarðsorm. % TO-DO: Format as header.

Thunder pulled up the Middenyardsworm.\footnotetext{This is the only title the poem has in \Regius. \AM\ has the proper title \emph{Hymiskviða} instead.}


\bvg
\bva Ár valtívar \hld vęiðar nǫ́mu &
ok sumblsamir \hld áðr saðir yrði, &
hristu tęina \hld ok á hlaut sǫ́u, &
fundu þęir at Ægis \hld ørkost hvera.\eva

\bvb Of yore the slaughter-Tues had caught game\footnoteB{Lit. ‘took game’}, and banqueting before they might eat\footnoteB{Lit. ‘might become sated’}, they shook the twigs and looked at the \inx{leat}; they found at Eyer’s a great choice of cauldrons.\footnoteB{The gods sprinkled the leat (sacrificial blood) of the beasts and interpreted the pattern; they found it most auspicious to feast at Eyer’s.}\evb
\evg


\bvg
\bva Sat bergbúi \hld barntęitr fyr, &
mjǫk glíkr męgi \hld Miskorblinda, &
lęit í augu \hld Yggs barn í þrá: &
„þú skalt ǫ́sum \hld opt sumbl \edtext{gęra}{\lemma{gęra “host”}\Afootnote{gefa “give” \AM}}!“\eva

\bvb — Sat the mountain-dweller \ken{Eyer}[1] there, joyous like a child, much like the lad of Misherblind\footnoteB{A reference to a lost myth? Unless Misherblind is an alternative name for Firneet, Eyer’s father.}; into his eyes looked the child of Ug <= Weden> \ken{Thunder}[1] in defiance: “Thou shalt for the Ease oft’ host banquets!”\footnoteB{Having seen that Eyer has a great store of cauldrons, Thunder orders him to host future banquets for the Ease.}\evb
\evg


\bvg
\bva Ǫnn fekk jǫtni \hld orðbæginn halr, &
hugði at hefndum \hld hann næst við goð, &
bað hann Sifjar ver \hld sér fǿra hver, &
„þann’s ek ǫllum ǫl \hld yðr of hęita.“\eva

\bvb Great toil for the ettin the word-peevish man \ken{Thunder}[1] caused; thought he of revenge, soon, against the god: asked he Sib’s husband to bring him a cauldron, “that one with which I for you all ale might brew.”\footnoteB{Eyer asks Thunder to find a single cauldron which can hold enough ale to supply all the Ease.}
\evg


\bvg
\bva Né þat mǫ́ttu \hld mærir tívar &
ok ginnręgin \hld of geta hvęrgi, &
unz af tryggðum \hld Týr Hlórriða &
ástráð mikit \hld ęinum sagði:\eva

\bvb But that might the renowned Tues and the \inx{Gin-Reins} nowhere get ahold of, until out of loyalty, a great word of loving advice Tue to Loride <= Thunder> alone did say:\evb
\evg


\bvg
\bva „Býr fyr austan \hld Élivága &
hundvíss Hymir \hld at himins ęnda, &
á minn faðir \hld móðugr kętil, &
\edtext{rúmbrugðinn}{\Afootnote{‘rumbrygðan’ \AM}} hver \hld rastar djúpan.“\eva

\bvb “Lives to the east of the Ilewaves the houndwise Hymer, at the end of heaven. Owns my father\footnoteB{Hymer being Tue’s father.}, fierce, a kettle; a size-renowned cauldron one \inx{rest} deep.”\evb
\evg


\bvg
\bva „Veiztu, ef þiggjum \hld þann lǫgvelli?“ &
„Ef, vinr, vélar \hld vit gørvum til!“\eva

\bvb “Knowest thou if we will receive that ale-boiler?” — “If, friend, we two make use of wiles!”\footnoteB{The speakers are not indicated, but it is most sensible that Thunder asks and Tue answers.}\evb
\evg

\bvg
\bva Fóru drjúgum \hld \edtext{dag þann framan}{\lemma{dag þann framan “from the beginning of the day”}\Afootnote{\emph{Emendation from Finnur 1932}; dag þann fram “on that day forth” \Regius; dag fráliga “swiftly at day” \AM}} &
Ásgarði frá \hld unz til \edtext{Ęgils}{\lemma{Ęgils “Agle’s”}\Afootnote{\emph{thus} \Regius; Ægis “Eyer’s” \AM; — \AM\ \emph{reading possibly from confusion with Eyer described earlier in the poem, but or the shepherd did share his name.}}} kvǫ́mu. &
Hirði hann hafra \hld horngǫfgasta; &
hurfu at hǫllu \hld es Hymir átti.\eva

\bvb — They travelled with great strides from the beginning of the day, from Osyard, until to Agle’s they came—he herded bucks with the noblest of horns—they turned to the hall which Hymer owned.\evb
\evg


\bvg
\bva Mǫgr fann ǫmmu, \hld mjǫk lęiða sér, &
hafði hǫfða \hld hundruð níu. &
ęn ǫnnur gekk \hld algollin framm &
brúnhvít bera \hld bjórvęig syni.\eva

\bvb The lad found his grandmother greatly loathsome; she had of heads nine hundred. But another woman, all-golden, stepped forth: white-browed, she carried a beer-draught for the son \ken{Tue}[1].\evb
\evg


\bvg
\bva „Áttniðr jǫtna \hld ek vilja’k ykr &
hugfulla tvá \hld und hvera sętja; &
es mínn \edtext{fríi}{\lemma{fríi “lover”}\Afootnote{\emph{thus} \Regius; faðir “father” \AM}} \hld mǫrgu sinni &
gløggr við gęsti \hld gǫrr ills hugar.“\eva

\bvb “Kinsman of ettins! I would wish to set you high-mettled two under the cauldrons; my lover has many a time been stingy against guests, quick to ill temper.”\footnoteB{Tue’s mother (the all-golden woman in previous v.) wishes to hide him and Thunder, lest her husband (Hymer) find them.}\evb
\evg


\bvg
\bva Ęn váskapaðr \hld varð \edtext{síðbúinn}{\Afootnote{\emph{om.} \AM}}, &
harðráðr Hymir, \hld hęim af vęiðum; &
gekk inn í sal, \hld glumðu jǫklar, &
vas karls, es kom, \hld kinnskógr frørinn.\eva

\bvb But the misshapen one was come late—the hard-minded Hymer—home from the hunt. He entered the hall—icicles clattered—frozen was the cheek-forest \ken{beard} of the churl who came.\evb
\evg


\bvg
\bva „Ves þú hęill, Hymir, \hld í hugum góðum! &
Nú ’s sonr kominn \hld til sala þinna, &
sá’s vit vættum \hld af vęgi lǫngum; &
fylgir hǫ́num \hld Hróðrs andskoti, &
vinr verliða; \hld Véurr hęitir sá.\eva

\bvb “Be thou hale, Hymer, in good spirits!\footnoteB{Formula identically mirrored in runic inscription N B380: \emph{Heill sé þú / ok í hugum góðum. / Þórr þik þiggi, / Óðinn þik eigi.} “May thou be hale, and in good spirits! May Thunder receive thee, may Weden own thee.” Cf. also \Beowulf\ l. 407: \emph{Wæs þú Hróðgár hál!} “Be thou, Rothgar, hale!”} Now the son is come to thy halls, the one whom we two have been expecting, from a long way off. Follows him the opponent of Rooder <ettin> \ken{Thunder}[1], the friend of manly retinues \ken{Thunder}[1]; Wighward he is called.\evb
\evg


\bvg
\bva Sé þú hvar sitja \hld und salar gafli, &
svá \edtext{forða sér}{\Afootnote{forðask \AM}}, \hld stęndr \edtext{súl}{\Afootnote{‘sol’ \AM}} fyrir.“ &
Sundr stǫkk súla \hld fyr sjón jǫtuns, &
ęn \edtext{allr}{\Afootnote{áðr \Regius\AM TODO: elaborate, mention Finnur}} í tvau \hld áss brotnaði.\eva

\bvb See where they sit, ’neath the hall’s gable: thus they hide themselves—a pillar stands before them!” The pillars sprang asunder before the sight of the ettin, but all in two the beam was broken.\evb
\evg


\bvg
\bva Stukku átta, \hld ęn ęinn af þęim &
hverr harðslęginn \hld hęill af þolli; &
framm gingu þęir, \hld ęn forn jǫtunn &
sjónum lęiddi \hld sinn andskota.\eva

\bvb Eight\footnoteB{Eight kettles.} sprung apart, but one of them, a hard-forged kettle, [came] whole off its peg\footnoteB{Presumably the one in which Tue and Thunder were hiding.}. Forth went they, but the ancient ettin with his sight beheld\footnoteB{Literally “led with his sight”.} his opponent.\evb
\evg


\bvg
\bva Sagðit hǫ́num \hld hugr vęl þá’s sá &
gýgjar \edtext{grǿti}{\lemma{grǿti “distresser”}\Afootnote{gæti “keeper, warder” \AM}} \hld á golf kominn, &
þar vǫ́ru þjórar \hld þrír of tęknir, &
bað \edtext{sęnn}{\Afootnote{‘sun’ \AM}} jǫtunn \hld sjóða ganga.\eva

\bvb His heart was not pleased then, when he saw the distresser of Gows <ettin-women> \ken{Thunder}[1] come on the floor. There were three bulls taken: the ettin at once bade them be cooked.\evb
\evg


\bvg
\bva Hvęrn létu þęir \hld hǫfði skęmra &
ok á sęyði \hld síðan bǫ́ru, &
át Sifjar verr \hld áðr sofa gingi, &
ęinn með ǫllu \hld øxn tvá Hymis.\eva

\bvb Each one they let shorten by a head, and onto the fire-pit then carried: ate the husband of Sib \ken{Thunder}[1], before he might go to sleep, alone all together two of Hymer’s oxen.\evb
\evg


\bvg
\bva Þótti hǫ́rum \hld Hrungnis spjalla &
verðr Hlórriða \hld vęl fullmikill, &
„munum at aptni \hld ǫðrum verða &
við vęiðimat \hld vér þrír lifa.“\eva

\bvb To the hoary friend of Rungner \ken{Hymer}[1] seemed Loride’s eating far too large; “we must this evening eat something else: by hunted game we three shall live.\Bfootnote{A clear example of inhospitality, illustrating the otherness of the Ettins. See introduction to the poem.}”\evb
\evg
% Thunder, Tue
%	\bookStart{Dreams of Balder}[Baldrs draumar]

In ancient manuscripts only preserved in \AM, but the poem also survives in later manuscripts in longer form.

\bvg
\bva Sęnn vǫ́ru ę́sir \hld\ allir á þingi &
ok ǫ́synjur \hld\ allar á máli, &
ok of þat réðu \hld\ ríkir tívar, &
hví vę́ri Baldri \hld\ ballir draumar.\eva

\bvb Soon were the \inx{Ease}[G] all at the \inx{Thing}[C], and the \inx{Ossens}[G] all at speech, and of this counseled the mighty \inx{tues}: why for Balder were baleful dreams.\evb
\evg


\bvg
\bva Upp reis Óðinn, \hld\ aldinn gautr, &
ok hann á Slęipni \hld\ sǫðul of lagði, &
ręið niðr þaðan \hld\ niflhęljar til; &
mǿtti hvelpi, \hld\ þęim’s ór hęlju kom.\eva

\bvb Up rose Weden—the aged Geat—and he on \inx{Slapner}[P] the saddle did lay; rode down thence to \inx{Nivel-hell}[G]; met the whelp that out of Hell came.\evb
\evg


\bvg
\bva Sá vas blóðugr \hld\ of brjóst framan, &
ok galdrs fǫður \hld\ gól oflęngi, &
framm ręið Óðinn, \hld\ foldvegr dunði, &
hann kom at hǫ́u \hld\ Hęljar ranni.\eva

\bvb That one was bloody on the front of the chest, and at the father of \inx{galder}[C] for a long time bayed.—Forth rode Weden, the fold-way resounded;\footnoteB{The verb \emph{dynja} ‘to resound’ also occurs in connection with the movement of a god in Thedwolf’s Harvest-long (Þjóð \emph{Haustl}, edited by Margaret Clunies Ross in \emph{SkP} III) 14–16, particularly: \emph{Ók at ísarnleiki · Jarðar sunr, en dunði \\ — móðr svall Meila blóða — · mána vegr und hônum.} ‘Drove to the iron-play \ken{battle} the son of Earth \ken*{= Thunder}, but the path—the wrath of Mole’s brother swelled—of the moon \ken{heaven} resounded beneath him.’} he came to the high house of Hell.\evb
\evg


\bvg
\bva Þá ręið Óðinn \hld\ fyr austan dyrr, &
þar’s hann vissi \hld\ vǫlu lęiði; &
nam hann vittugri \hld\ valgaldr kveða, &
unz nauðug ręis, \hld\ nás orð of kvað:\eva

\bvb Then rode Weden east of the door, there as he knew the wallow’s grave; he took to sing a \inx{slain-galder}[C] for the witchcraft-skilled woman, until forced she rose, a corpse’s words did speak:\evb
\evg


\bvg
\bva „Hvat ’s manna þat \hld\ mér ókunnra, &
es mér hęfr aukit \hld\ ęrfitt sinni; &
vas’k snifin snę́vi, \hld\ ok slęgin regni &
ok drifin dǫggu, \hld\ dauð vas’k lęngi.“\eva

\bvb “What sort of man is that, unknown to me, who has caused for me this toilsome walk?\footnoteB{i.e. out of the grave.} I was snowed by snow and struck by rain, and sprayed with dew;\footnoteB{Cf. Thou alone causest Hallow etc. TODO} dead was I for long.”\evb
\evg


\bvg
\bva „Vegtamr hęiti’k, \hld\ em’k Valtams sonr, &
sęg mér ór hęlju, \hld\ ek ór hęimi mun, &
hvęim eru bękkir \hld\ baugum sánir? &
flęt fagrliga \hld\ flóuð eru golli.“\eva

\bvb “Waytame I am called, I am Waltame’s son. Tell me about Hell—I will [tell] about the world; for whom are the benches sown with \inx{bigh}[C]{bighs}; the fair rooms are flooded with gold.”\evb
\evg


\bvg
\bva „Hér stęndr Baldri \hld\ of brugginn mjǫðr, &
skírar vęigar, \hld\ liggr skjǫldr yfir, &
ęn ásmęgir \hld\ í ofvę́ni; &
nauðug sagða’k, \hld\ nú mun’k þęgja.“\eva

\bvb “Here stands brewed for Balder mead, pure draughts—a shield lies over;\footnoteB{Shields covering casks of mead is a common trope.} but the os-lads \ken*{Ease} [stand] in great suspense; forced I spoke, now I will be silent.”\evb
\evg


\bvg
\bva „Þęgjat vǫlva, \hld\ þik vil’k fregna, &
unz ’s alkunna, \hld\ vil’k ęnn vita, &
hvęrr mun Baldri \hld\ at bana verða, &
ok Óðins son \hld\ aldri rę́na?“\eva

\bvb “Be not silent, wallow! Thee I wish to ask; until all is known I wish to know further: Who will for Balder become the bane, and Weden’s son \ken*{= Balder} rob of life?”\evb
\evg


\bvg
\bva „Hǫðr berr hǫ́van \hld\ hróðrbaðm þinig, &
hann mun Baldri \hld\ at bana verða, &
ok Óðins son \hld\ aldri rę́na; &
nauðug sagða’k, \hld\ nú mun’k þęgja.“\eva

\bvb “\inx{Hath}[P] bears the high, renowned beam \ken{mistletoe} thither; he will for Balder become the bane, and Weden’s son \ken*{= Balder} rob of life; forced I spoke, now I will be silent.”\evb
\evg


\bvg
\bva „Þęgjat vǫlva, \hld\ þik vil’k fregna, &
unz ’s alkunna, \hld\ vil’k ęnn vita, &
hvęrr mun hęipt Hęði \hld\ hęfnt of vinna, &
eða Baldrs bana \hld\ á bál vega?“\eva

\bvb “Be not silent, wallow! Thee I wish to ask; until all is known I wish to know further: Who will for the evil-doing get revenge on Hath, or bear onto the pyre Balder’s bane \ken*{= Hath}?”\evb
\evg


\bvg
\bva „Rindr berr Vála \hld\ í vestrsǫlum, &
sá mun Óðins sonr \hld\ einnę́ttr vega, &
hǫnd of þvę́rat \hld\ né hǫfuð kęmbir, &
áðr á bál of berr \hld\ Baldrs andskota; &
nauðug sagðak, \hld\ nú munk þęgja.“\eva

\bvb “Rind bears \inx{Wonnel}[P] in the western halls; that one will, Weden’s son, one night old, fight. His hand he washes not, nor head combs, before onto the pyre he bears Balder’s opponent \ken*{= Hath}; forced I spoke, now I will be silent.\footnoteB{Note the similarity with \Voluspa\ 34–35 and the irregularity of the verse length, which may suggest that a line (most likely 2) has been inserted.}”\evb
\evg


\bvg
\bva „Þęgjat vǫlva, \hld\ þik vil’k fregna, &
unz ’s alkunna, \hld\ vil’k ęnn vita, &
hvęrjar ’ró męyjar, \hld\ es at muni gráta &
ok á himin verpa \hld\ halsaskautum?“\eva

\bvb “Be not silent, wallow! Thee I wish to ask; until all is known I wish to know further: Which are the maidens that weep as they wish, and onto heaven throw their throat-corners?\footnoteB{Wat mean...}”\evb
\evg


\bvg
\bva „Estat Vegtamr, \hld\ sem ek hugða, &
hęldr est Óðinn, \hld\ aldinn gautr.“ &
„Estat vǫlva \hld\ né vís kona, &
hęldr est þriggja \hld\ þursa móðir.\eva

\bvb “Thou art not Waytame, as I thought; rather art thou Weden, the aged Geat!”—“Thou art not a wallow, nor a wise woman; rather art thou of three \inx{thurses}[G] the mother!”\evb
\evg


\bvg
\bva „Hęim ríð Óðinn \hld\ ok hróðigr ves, &
svá komit manna \hld\ męirr aptr á vit, &
es lauss Loki \hld\ líðr ór bǫndum &
ok ragna rǫk \hld\ rjúfęndr koma.“\eva

\bvb “Ride home Weden, and be renowned!\footnoteB{A sarcastic statement, the sense being: “Your renown, Weden, will not save you.”} So may no other man come again to visit [me], when loose Lock passes out of his bonds, and at the \inx{rakes of the Reins}, the breakers come.\footnoteB{Cf. wording of \Vafthrudnismal\ TODO: \emph{þá’s rjúfask ręgin} ‘when the Reins are broken’.}”\evb
\evg


Late verses in paper manuscripts? TODO
% Balder, Weden
	\bookStart{The Leed of Hindle}[Hyndluljóð]

\bvg
\bva „Vaki mę́r męyja, \hld\ vaki mín vina, &
Hyndla systir, \hld\ es í hęlli býr; &
nú ’s røkr røkra, \hld\ ríða vit skulum &
til Valhallar \hld\ ok til vés hęilags.\eva

\bvb Frow quoth:
“Wake maiden of maidens, wake my friend, sister Hindle, who lives in the rock-face. Now is the twilight of twilights, we two shall ride to Walhall, and to the holy wigh†!\evb
\evg


\bvg
\bva Biðjum Hęrjafǫðr \hld\ í hugum sitja, &
hann geldr ok gefr \hld\ gull \edtext{verðugum}{\Afootnote{verðungu ‘to the retinue’ \emph{emend.} \FinnurEdda\ \GudniEdda}}, &
gaf hann Hęrmóði \hld\ hjalm ok brynju, &
ęn Sigmundi \hld\ sverð at þiggja.\eva

\bvb Let us bid the Father of Hosts \ken{Weden}[1] to be in his favour; he rewards and gives gold to the worthy. Gave he to Heremood helmet and byrnie, but Sighmund a sword to receive.\evb
\evg


\bvg
\bva Gefr hann sigr sumum\footnotetext[1], \hld\ ęn sumum\footnotetext[2] aura, &
mę́lsku mǫrgum \hld\ ok manvit firum, &
byri gefr brǫgnum, \hld\ ęn brag skǫldum, &
gefr hann mannsęmi \hld\ mǫrgum rekki. &
\footnotetext[1] ms. \emph{sonum}
\footnotetext[2] ms. \emph{suinnum}\eva

\bvb He gives victory to some, but to some silver\footnotemark[1]; speech to many, and manwit to men. Fair wind he gives to noble ones, and poetry to scolds†; he gives valour to many a champion.
\footnotemark[1] Lit. "ounces".\evb
\evg


\bvg
\bva Þór munk blóta, \hld\ þess munk biðja, &
at hann ę́ við þik \hld\ einart láti; &
þó ’s hǫ́num ótítt \hld\ við jǫtuns brúðir.\eva

\bvb To Thunder I will bloot†, of this I will bid, that he always show friendliness to thee, though he is prejudiced against the brides of the ettins\footnotemark[1].
\footnotetext[1] Lit. “though [it] is to him infrequent with ettin's brides”.\evb
\evg


\bvg
\bva Nú taktu ulf þinn \hld\ ęinn af stalli, &
lát hann rinna \hld\ með runa mínum.“ &
[Hyndla kvað:] „Sęinn es gǫltr þinn \hld\ goðveg troða, &
vilkat mar minn \hld\ mę́tan hlǿða.\eva

\bvb Now take thy single wolf from the stable; let him run with my boar.” [Hindle quoth:] “Slow is thy boar, to tread the Godways; I wish not lade my dear steed.”\evb
\evg


\bvg
\bva Flǫ́ est Fręyja, \hld\ es fręistar mín, &
visar þú augum \hld\ á oss þannig, &
es hafir ver þinn \hld\ í valsinni &
Óttar unga \hld\ Innsteins bur.“\eva

\bvb Deicitful art thou, Frow, as thou temptest me; thou showest thy eyes on us this way, as thou hast thy man on the Walways: the young Oughthere, Instone's offspring.”\evb
\evg


\bvg
\bva Fręyja kvað:
„Dulið est Hyndla, \hld\ draums ę́tlak þér, &
es kveðr ver minn \hld\ í valsinni.\eva

\bvb Frow quoth:
Thou art foolish, Hindle, I think thee dreamy, who sayest that my man is on the Walways.\evb
\evg


\bvg
\bva Þar’s gǫltr glóar \hld\ Gullinbursti, &
Hildisvíni, \hld\ es mér hagir gęrðu, &
dvergar tvęir \hld\ Dáinn ok Nabbi.\eva

\bvb Where the boar glows, Goldenbristle; the hildswine\footnotemark[1], which the skillful for me made: the two dwarves Dowen and Nab.
\footnotemark[1] \emph{Hildisvíni} 'battle-swine', in this case probably an alternative name for Goldenbristle.\evb
\evg


\bvg
\bva Sęnn í sǫðlum \hld\ sitja vit skulum &
ok of jǫfra \hld\ ę́ttir dǿma, &
gumna þęira, \hld\ es frá goðum kómu.\eva

\bvb Soon in the saddles we two shall sit, and judge about the aughts† of princes, of those men who came from the gods.\evb
\evg


\bvg
\bva Þęir hafa vęðjat \hld\ Vala malmi &
Óttarr ungi \hld\ ok Angantýr; &
skylt ’s at vęita, \hld\ svá’t skati hinn ungi & &
fǫðurlęifð hafi \hld\ ępt frę́ndr sína.\eva

\bvb They have wagered the Welsh ore [GOLD], young Oughter and Ongenthew; it is required to grant, so that the young prince might have the fatherly inheritance left behind by his kinsmen.\footnotemark[1]
\footnotemark[1] Lit. 'the father-remains after his kinsmen'. — Happening seems to be that Oughthere and Ongenthew each lay claim the inheritance. In order to settle the matter (in Oughthere's favour) Hindle must (\emph{skylt es} “it is required, obligated”) divulge (\emph{vęita} ‘to grant, to give away’) what she knows about his lineage.\evb
\evg


\bvg
\bva Hǫrg hann mér gęrði \hld\ hlaðinn stęinum; &
nú ’s grjót þat \hld\ at glęri orðit; &
rauð hann í nýju \hld\ nauta blóði; &
ę́ trúði Óttarr \hld\ á ǫ́synjur.\footnotemark[1]
\footnotemark[1] Frow argues yet further in favour of Oughthere, bringing up his piety shown towards the godesses.\eva

\bvb A harrow† he made for me, loaded with stones; now that stone-pile is become into glass. He reddened [it] in fresh blood of oxen; Oughthere ever trusted on the osennies†.\evb
\evg


\bvg
\bva Nú lát-tu forna \hld\ niðja talða &
ok uppbornar \hld\ ę́ttir manna &
hvat ’s Skjǫldunga, \hld\ hvat ’s Skilfinga, &
hvat ’s Ǫðlinga \hld\ hvat ’s Ylfinga & &
hvat ’s hǫldborit, \hld\ hvat ’s hęrsborit &
męst manna val \hld\ und Miðgarði?“\eva

\bvb Now let be recounted the ancient lines of kinsmen, and the upborn\footnote[1] aughts† of men: What is of the Shieldings? What is of the Shilvings? What is of the Athlings? What is of the Wolvings? What is born of hero? What is born of chief, the mightiest choice of men in Midyard?”
\footnote[1] Noble.\evb
\evg


\bvg
\bva „Þú est Óttarr \hld\ borinn Innstęini, &
ęn Innstęinn vas \hld\ Alfi inum gamla, &
Alfr vas Ulfi, \hld\ Ulfr Sę́fara, &
ęn Sę́fari \hld\ Svan inum rauða.\eva

\bvb Hindle quoth:
“Thou\footnote[1] art, Oughthere, born to Instone, but Instone was born to Elf the old, Elf to Wolf, Wolf to Seafare, but Seafare to Swan the red.
\footnote[1] Hindle, apparently in a trance-like state, speaks straight to Oughthere.\evb
\evg


\bvg
\bva Móður átti faðir þinn \hld\ męnjum gǫfga, &
hygg at héti \hld\ Hlédís gyðja, &
Fróði vas faðir þęirar, \hld\ ęn Fríund\footnotemark[1] móðir; &
ǫll þótti ę́tt sú \hld\ með yfirmǫnnum.\eva
\footnotemark[1] Emended from the meaningless ms. reading \emph{friaut}.

\bvb Thy father had thy mother, beautiful with neck-rings, I think that she was called Leedise yidde†. Frood was her father, but Friend her mother; all her aught seemed to be among overmen.\evb
\evg


\bvg
\bva Auði vas áðr \hld\ ǫflgastr manna, &
Halfdanr fyrri \hld\ hę́str Skjǫldunga, &
frę́g vǫ́ru folkvíg, \hld\ þaus framir gęrðu, &
hvarfla þóttu verk \hld\ með himins skautum.\eva

\bvb Ed was before [that] the most powerful of men, Halfdane earlier the highest of Shieldings. Renowned were the troop-battles which the famous ones performed; his <= Halfdane's> works seemed to travel around the corners of heaven.\evb
\evg


\bvg
\bva Ęflðisk við Ęymund \hld\ ǿztan manna &
ęn vá Sigtrygg \hld\ með svǫlum ęggjum, &
ęiga gekk Almvęig, \hld\ ǿzta kvinna, &
ólu þau ok ǫ́ttu \hld\ átján sonu.\eva

\bvb He <= Halfdane> became the in-law of Iemund\footnotemark[1], the noblest of men, but he slew Sightrue with cool edges. He went on to have Elmwey, the noblest of women; they begot and had eighteen sons.
\footnotemark[1] Lit. "[he] was strengthened by". Parallelism of "noblest of men/women" makes the meaning yet clearer. Elmwey was Iemund's daughter or sister.\evb
\evg


\bvg
\bva Þaðan eru Skjǫldungar, \hld\ þaðan eru Skilfingar, &
þaðan eru Ǫðlingar, \hld\ þaðan eru Ynglingar, &
þaðan es hǫldborit, \hld\ þaðan es hęrsborit, &
mest mannaval \hld\ und Miðgarði; &
alt ’s þat ę́tt þín, \hld\ Óttarr heimski.\eva

\bvb Thereof are the Shieldings! Thereof are the Shilvings! Thereof are the Inglings!\footnotemark[1] Thereof is born of hero! Thereof is born of chief, the mightiest choice of men in Midyard! That is all thy aught†, foolish Oughthere!”
\footnotemark[1] Note the contradiction with v. 12. Since the Inglings have already been mentioned (under the name Shilvings, of the difference between the two see the index), it seems likely that Wolvings is the original reading.\evb
\evg


\bvg
\bva Vas Hildigunnr \hld\ hęnnar móðir, &
Svǫ́fu barn \hld\ ok sę́konungs; &
alt ’s þat ę́tt þín, \hld\ Óttarr hęimski. &
varðar\footnote[1] at viti svá, \hld\ viltu ęnn lęngra?\eva
\footnote[1] Emended from ms. \emph{varði}.

\bvb Hildguth was her mother, the child of Swabe and Seaking; that is all thy aught†, foolish Oughthere!—It is meaningful that one might know thus; wilt thou [go] yet further?\evb
\evg


\bvg
\bva Dagr átti Þóru \hld\ dręngjamóður, &
ólusk í ę́tt þar \hld\ ǿztir kappar, &
Fraðmarr ok Gyrðr \hld\ ok Frekar báðir, &
Ámr ok Jǫsurmarr, \hld\ Alfr hinn gamli. &
varðar at viti svá, \hld\ viltu ęnn lęngra?\eva

\bvb Day had Thure, the mother of valiant men; in that aught were begotten the noblest champions: Fradmer and Yird, and both Frecks; Ame and Essirmer; Elf the old.—It is meaningful that one might know thus; wilt thou [go] yet further?\evb
\evg


\bvg
\bva Kętill hét vinr þęira \hld\ Klypps arfþęgi, &
vas hann móðurfaðir \hld\ móður þinnar; &
þar vas Fróði \hld\ fyrr ęnn Kári, &
ęn Hildi vas \hld\ Hóalfr of getinn.\eva

\bvb Kettle, the inheritor of Clip, was their friend; he was the father of thy mother's mother. There was Frood, yet earlier Keer, but Highelf was by Hild begotten.\evb
\evg

... %TODO More dialogue
% Frow
	\bookStart{Book of Galders}


Old High German galders

\section{The two Merseburg charms}

\bvg
\bva Eiris sázun idísi \hld\ sázun hera duoder; &
suma hapt heptidun \hld\ suma heri lezidun &
suma clubodun \hld\ umbi cuoniowidi &
insprinc haptbandun \hld\ infar fígandun .H.\eva

\bvb Of yore stayed dises, stayed here and there: some fastened fetters, some hindered hosts, some cleaved shackles.—Break the fetter-bonds, flee the fiends! .H.\footnoteB{TODO: note about this strange mark in the ms.}\evb
\evg


\bvg
\bva Phol ende Wódan \hld\ fuórun zi holza &
dú wart demo Balderes folon \hld\ sín fuóz birenkit &
thú biguól en Sinthgunt \hld\ Sunna era swister &
thú biguól en Fríja \hld\ Folla era swister &
thú biguól en Wódan \hld\ só hé wola conda &
sóse bénrenkí \hld\ sóse bluótrenkí \hld\ sóse lidirenkí &
bén zi béna \hld\ bluót zi bluóda &
lid zi geliden \hld\ sóse gelimida sín\eva

\bvb Phol and Weden journeyed to the woods; then was the foot of Balder’s foal sprained. Then \inx[C]{begale}[begaled] him \inx[P]{Sithguth}, \inx[P]{Sun} her sister; then begaled him \inx[P]{Frie}, \inx[P]{Full} her sister; then begaled him Weden, as he well knew: “Like bone-sprain, like blood-sprain, like joint-sprain! Bone to bone, blood to blood, joint to joints, like were they glued together!”\evb
\evg


\section{Against worms (Contra vermes)}

Nessi mid nigon nessiklínun


Old English galders


\section{Against a dwarf}


Old Norse galders


\section{Charms from Bergen}

\bvg {\small N B380 (~1185CE)}
\bva \alst{H}ęill sé þú \hld ok í \alst{h}ugum góðum; &
\ind \alst{Þ}órr þik \alst{þ}iggi, &
\ind \alst{Ó}ðinn þik \alst{ęi}gi.\eva

\bvb Mayst thou be hale, and in good spirits. May Thunder receive thee, may Weden own thee.\evb
\evg
% Assorted spells
	\bookStart{Eddic fragments from Snorre’s Edda}
%TODO: Further discussion on the fragments.

A number of Eddic lines, stanzas and groups of stanzas are quoted in Snorre’s Edda.  The majority of them are taken from longer Eddic poems preserved in full in other manuscripts (primarily \Regius\ and \AM), but a few are found nowhere else.  These fragments will be edited in the present section.

The fragments have some things in common: they are generally pieces of spoken dialogue quoted in the context of longer narrative prose sections, and are, with one exception (Homedal’s galder, see below), not introduced by reference to their source but rather with phrases like \emph{þá kvað hann} ‘then he quoth’.

\sectionline

\section{A lost riddle-poem}

This half-stanza is quoted in \Gylfaginning\ 2, being the second Eddic verse in the text, following \Havamal\ 1 in the same chapter, which is uttered by Yilfer himself when he enters the hall of the Eese. The whole section is clearly referencing other Eddic mythic wisdom contests and particularly reminiscent of \Vafthrudnismal.

\bpg\bpa Hann sá þrjú há-sę́ti ok hvert upp frá ǫðru, ok sátu þrír menn sinn í hverju. Þá spurði hann, hvert nafn hǫfðingja þeira vę́ri. Sá svarar, er hann leiddi inn, at sá, er í inu neðsta hásę́ti sat, var konungr, ok heitir Hárr, en þar nę́st sá, er heitir Jafnhárr, en sá ofast, er Þriði heitir. Þá spyrr Hárr komandann, hvárt fleira er erendi hans, en heimill er matr ok drykkr honum sem ǫllum þar í Háva hǫll. Hann segir, at fyrst vill hann spyrja, ef nǫkkurr er fróðr maðr inni. Hárr segir, at hann komi eigi heill út, nema hann sé fróðari,\epa

\bpb He [= Yilfer] saw three high-seats and each higher than the other, and three men sat there, each in his own seat. Then he asked what the names of those chieftains were. He who led him in answers that the one who sat in the lowest high-seat was a king called High, and next to him he who is called Evenhigh, and uppermost he who is called Third. Then High asks the guest whether he has any other errands, but food and drink will be freely offered him, like all men there in the High One’s hall. He [= Yilfer] asks whether anyone within is a learned man.  High says that he will not come out whole unless he be more learned [than he],\epb\epg

\bvg\bva „ok statt-u \alst{f}ramm \hld\ meðan þú \alst{f}regn &
\ind \alst{s}itja skal \alst{s}á es \alst{s}ęgir.“\eva

\bvb “and stand forth while thou askest; \\
\ind sit shall he who speaks!”\evb\evg

\sectionline

\section{Nearth and Shede}

The following passage is almost the whole of \Gylfaginning\ 23, excepting at the very end \emph{svá er sagt} ‘so it is said’, after which is quoted \Grimnismal\ 11.
Notably, the two stanzas cited here are also found translated in \textcite{Saxo}[1.8.18--19], where they are said to have been spoken by Hadding and Rainhild, respectively.  For discussion \textcite{Hopkins2021}.

\sectionline

\bpg\bpa Inn þriði áss er sá, er kallaðr er Njǫrðr. Hann býr á himni, þar sem heitir Nóatún. Hann rę́ðr fyrir gǫngu vinds ok stillir sjá ok eld. Á hann skal heita til sę́-fara ok til veiða. Hann er svá auðigr ok fé-sę́ll, at hann má gefa þeim auð, landa eða lausa-fjár. Á hann skal til þess heita. Eigi er Njǫrðr ása ę́ttar. Hann var upp fǿddr í Vana-heimi, en Vanir gísluðu hann goðunum ok tóku í mót at gíslingu þann, er Hǿnir heitir. Hann varð at sę́tt með goðum ok Vǫnum. Njǫrðr á þá konu, er Skaði heitir, dóttir Þjatsa jǫtuns. Skaði vill hafa bú-stað þann, er átt hafði faðir hennar, þat er á fjǫllum nǫkkurum, þar sem heitir Þrym-heimr, en Njǫrðr vill vera nę́r sę́. Þau sę́ttust á þat, at þau skyldu vera níu nę́tr í Þrym-heimi, en þá aðrar níu at Nóa-túnum. En er Njǫrðr kom aftr til Nóatúna af fjallinu, þá kvað hann þetta:\epa

\bpb The third Os is that one who is called Nearth. He lives in the heaven in the place called Nowetowns. He rules the course of the wind, and stills sea and fire. On him shall one call for sea-faring and for hunting. He is so wealthy and blessed with money that he may give them a wealth of lands or loose property; on him shall one call for that sake. Nearth is not of the lineage of the Eese. He was brought up in Wanehome, but the Wanes gave him as a hostage to the gods, and in return got as hostage that one who is called Heener. He was used to reconcile the gods and the Wanes. Nearth has that woman who is called Shede, the daughter of the ettin Thedse. Shede wishes to have the dwelling which her father had owned, which lies on some fells in the place called Thrimham—but Nearth wishes to live by the sea. They agreed with each other that they would live for nine nights in Thrimham, but the other nine at Nowetowns. But when Nearth came back to the Nowetowns from the fell, he quoth this:\epb\epg

\bvg\bva „\alst{L}ęið erumk fjǫll, \hld\ vas’k-a \alst{l}ęngi á, &
\ind \alst{n}ę́tr ęinar \alst{n}íu; &
\alst{u}lfa þytr \hld\ mér þótti \alst{i}llr vesa &
\ind hjá \alst{s}ǫngvi \alst{s}vana.“\eva

\bvb “The fells are loathsome to me; I was not long thereon— \\
\ind only for nine nights. \\
The howling of the wolves thought me evil, \\
\ind compared to the song of swans.”\evb\evg

\bpg\bpa Þá kvað Skaði þetta:\epa

\bpb Then Shede quoth this:\epb\epg

\bvg\bva „\alst{S}ofa né mát’k-a’k \hld\ \alst{s}ę́var bęðjum á &
\ind \alst{f}ugls jarmi \alst{f}yrir; &
sá mik \alst{v}ękr \hld\ es af \alst{v}íði kømr &
\ind \alst{m}orgun hvęrjan \alst{m}ár.“\eva

\bvb “I could not sleep on the beds of the sea \\
\ind for the bleating of the bird. \\
He awakes me, when from the wide sea he comes, \\
\ind every morning, the mew.”\evb\evg

\bpg\bpa Þá fór Skaði upp á fjall ok byggði í Þrym-heimi, ok ferr hon mjǫk á skíðum ok með boga ok skýtr dýr. Hon heitir ǫndur-goð eða ǫndur-dís.\epa

\bpb Then Shede went up to the fells and dwelled in Thrimham, and she often goes on skis with her bow and shoots beasts. She is called ski-god or ski-dise.\epb\epg

\sectionline

\section{Homedal’s Galder (\emph{Hęimdallargaldr})}

This mysterious fragment is quoted in \Gylfaginning\ 27, the chapter describing Homedal, which is here reproduced in full. The fragment consists of two c-lines and appears to be the end of a stanza in the fitting meter \Galdralag.

The same poem is mentioned again in \Skaldskaparmal\ 15: \emph{Heimdallar hǫfuð heitir sverð. Svá er sagt, at hann var lostinn manns hǫfði í gegnum. Um þat er kveðit í Heimdallar-galdri, ok er síðan kallat hǫfuð mjǫtuðr Heimdallar} ‘A sword is called Homedal’s head. So is said that he was run through with a man’s head.  About that it is sung in Homedal’s galder, and henceforth the head is called Homedal’s bane.’

\sectionline

\bpg\bpa Heimdallr heitir einn. Hann er kallaðr hvíti áss; hann er mikill ok heilagr. Hann báru at syni meyjar níu ok allar systr; hann heitir ok Hallinskíði ok Gullintanni; tennr hans váru af gulli. Hestr hans heitir Gulltoppr. Hann býr þar er heitir Himinbjǫrg við Bifrǫst; hann er vǫrðr goða ok sitr þar við himins enda at gę́ta brúarinnar fyrir berg-risum. Hann þarf minna svefn en fugl. Hann sér jafnt nótt sem dag hundrað rasta frá sér; hann heyrir ok þat, er gras vex á jǫrðu eða ull á sauðum, ok allt þat er hę́ra lę́tr. Hann hefir lúðr þann er Gjallar-horn heitir, ok heyrir blástr hans í alla heima. Heimdallar sverð er kallat hǫfuð manns. Hér er svá sagt:\epa

\bpb Homedal one is named.  He is called the White Os; he is great and holy.  He was born as the son of nine maidens, sisters all.  He is also named Haldenshid and Goldentooth; his tooth were of gold.  His horse is called Goldtop.  He lives at the place called the Heavenbarrows near Bivrest.  He is the Watchman of the Gods and sits there at Heaven’s end to guard the bridge against barrow-risers.  He needs less sleep than a bird.  Both night and day he sees a hundred rests away from him; he also hear when grass grows on the ground or wool on sheep, and everything which sounds louder. He has the basoon called the Horn of Yell, and his blowing can be heard in all realms.  Homedal’s sword is called a man’s head.  Here it says so:\epb\epg

\sectionline

(Here the text cites \Grimnismal\ 13; see there.)

\sectionline

\bpg\bpa Ok enn segir hann sjalfr í Heimdallar-galdri:\epa

\bpb And further he himself says in Homedal’s Galder:\epb\epg


\bvg\bva „Níu em’k \edtrans{\alst{m}ǿðra}{mothers}{\Afootnote{so \RegiusProse\Trajectinus\Wormianus; \emph{męyja} ‘maidens’ \Upsaliensis}} \alst{m}ǫgr, &
níu em’k \alst{s}ystra \edtrans{\alst{s}onr}{son}{\Afootnote{om. \Trajectinus}}.“\eva

\bvb “Of nine mothers I’m the lad, \\
of nine sisters I’m the son.”\evb\evg

\sectionline

\section{Gna and the Wanes}

The following passage is from \Gylfaginning\ 35, which lists the \inx[G]{Ossens}.

\sectionline

\bpg\bpa Fjórtánda Gná, hana sendir Frigg í ymsa heima at ørindum sínum. Hon á þann hest, er renn lopt ok lǫg, er heitir Hóf-varpnir. Þat var eitt sinn, er hon reið, at vanir nǫkkvǫrir sá reið hennar í loptinu. Þa mę́lti einn:\epa

\bpb The fourteenth is Gna; Frie sends her into every home to do her errands. She owns the horse who runs through air and sea, and is called Hoofwarpner. It was one time when she rode that some Wanes saw her riding in the air. Then one spoke:\epb\epg

\bvg\bva „Hvat þar \alst{f}lýgr, \hld\ hvat þar \alst{f}ęrr, &
\ind eða at \alst{l}opti \alst{l}íðr?“\eva

\bvb “What flies there, what fares there, \\
\ind or passes through the air?”\evb\evg


\bpg\bpa Hon svarar:\epa

\bpb She answers:\epb\epg


\bvg\bva „Né ek \alst{f}lýg, \hld\ þó ek \alst{f}ęr &
\ind ok at \alst{l}opti \alst{l}ið’k &
á \alst{H}óf-varpni, \hld\ þęim’s \alst{H}am-skęrpir &
\ind \alst{g}at við \alst{G}arð-rofu.“\eva

\bvb “I fly not, though I fare, \\
\ind and pass through the air, \\
on Hoofwarpner, whom Hamsherper \\
\ind begot with Yardrove.”\evb\evg


\bpg\bpa Af Gnár nafni er svá kallat, at þat gnę́far, er hátt ferr:\epa

\bpb From Gna’s name it is so called that something which fares high up \emph{protrudes}.\epb\epg

\sectionline

\section{Balder’s Death}

\Gylfaginning\ 49 contains the narrative of Balder’s death, beginning with his ominous dreams, and ending with the Eese failing to “weep him out of Hell” (for a summary and discussion of the myth and its attestations, see the introduction to \Voluspa\ 31–33). At the end of the chapter, a single \Ljodahattr\ speech-stanza is quoted.

\sectionline

\bpg\bpa Því nę́st sendu ę́sir um allan heim ørind-reka at biðja, at Baldr vę́ri grátinn ór Helju, en allir gerðu þat, menninir ok kykvendin ok jǫrðin ok steinarnir ok tré ok allr málmr, svá sem þú munt sét hafa, at þessir lutir gráta, þá er þeir koma ór frosti ok í hita. Þá er sendi-menn fóru heim ok hǫfðu vel rekit sín ørindi, finna þeir í helli nǫkkvǫrum, hvar gýgr sat; hon nefndist Þǫkk. Þeir biðja hana gráta Baldr ór helju, hon segir:\epa

\bpb Next after that the Eese sent an errand-runner through all the \inx[C]{Home}, to ask that Balder be wept out of hell. And all did that, the men and the beasts and the earth and the stones and trees and all bedrock, as thou must have seen, that these things weep when they come out of cold and into heat. When the messengers journeyed home, and had ran their errand well, they find in a certain cave that a \inx[C]{gow} sat there; she called herself Thanks. They ask her to weep Balder out of hell. She says:\epb\epg


\bvg\bva „\alst{Þ}ǫkk mun gráta \hld\ \alst{þ}urrum tǫ́rum &
\ind \alst{B}aldrs \alst{b}ál-farar; &
\alst{k}yks né dauðs \hld\ naut’k-a \alst{K}arls sonar &
\ind \alst{h}afi \alst{H}ęl því’s \alst{h}ęfir.“\eva

\bvb “Thanks will weep–with dry tears \\
\ind for Balder’s pyre-journey \ken{death}. \\
Neither alive nor dead did I benefit from Churl’s son \ken*{= Balder}; \\
\ind let Hell have what she has!”\evb\evg


\bpg\bpa En þess geta menn, at þar hafi verit Loki Laufeyjarson, er flest hefir illt gørt með ásum.\epa

\bpb But men guess that this must have been Lock, Leafy’s son, who has done the most evil among the Eese.\epb\epg

\sectionline

\section{Thunder’s Journey to Garfrith}

\Skaldskaparmal\ 26, here edited in part, is the only surviving retelling of Thunder’s journey to the ettin Garfrith, and his following fight with, and slaying of, him and his two daughters, Yelp and Grope. This was apparently a well-known story, and is also mentioned in Vetrl Lv 1/1b (quoted in \Skaldskaparmal\ 11, which lists kennings for Thunder): \emph{stétt of Gjǫlp dauða} ‘thou didst step over the dead Yelp’.
The prose of \Skaldskaparmal\ 26 seems to be based on an earlier, now-lost poem in \Ljodahattr, from which it quotes two stanzas. The first is found in all four main manuscripts, while the second is found only in \Upsaliensis. Both are spoken by Thunder and closely resemble each other stylistically, which is why they most likely come from the same poem.

\sectionline

\bpg\bpa Þá fór Þórr til ár þeirar, er Vimur heitir, allra á mest. Þá spennti hann sik megin-gjǫrðum ok studdi for-streymis Gríðar-vǫl, en Loki helt undir megin-gjarðar. Ok þá er Þórr kom á miðja ána, þá óx svá mjǫk áin, at uppi braut á ǫxl honum. Þá kvað Þórr þetta:\epa

\bpb Then Thunder journeyed to that river which is called Wimbre, greatest of all rivers. Then he wrapped his might-girdle around himself and leaned upon Grith’s stave against the stream, and Lock held up the might-girdle. And when Thunder came to the middle of the river, then it waxed so great that it broke over his shoulders. Then Thunder quoth this:\epb\epg


\bvg\bva „\alst{V}ax-at-tu nú, \alst{V}imur, \hld\ alls mik þik \alst{v}aða tíðir &
\ind \alst{jǫ}tna garða \alst{í}; &
\alst{v}ęitst, ef þú \alst{v}ęx \hld\ at þá \alst{v}ęx mér ǫ́s-męgin &
\ind jafn-\alst{h}átt upp sem \alst{h}iminn.“\eva

\bvb “Wax not now, O Wimbre, as I wish to wade through thee \\
\ind into the yards of the ettins. \\
Thou knowest, if thou waxest, then my os-might waxes \\
\ind up as high as the heaven.”\evb\evg


\bpg\bpa Þá sér Þórr uppi í gljúfrum nǫkkurum, at Gjálp, dóttir Geirrøðar \edtrans{stóð þar tveim megin árinnar, ok gerði hon ár-vǫxtinn.}{stood on both sides of the river, and she caused the river’s growth}{\Bfootnote{She stood with her legs spread and befouled the river.}} Þá tók Þórr upp ór ánni stein mikinn ok kastaði at henni ok mę́lti svá: „At ósi skal á stemma.“ Eigi missti hann, þar er hann kastaði til, ok í því bili bar hann at landi ok fekk tekit reyni-runn nǫkkurn ok steig svá ór ánni. Því er þat orð-tak haft, at reynir er bjǫrg Þórs.\epa

\bpb Then Thunder sees that up in some certain gorges Yelp, daughter of Garfrith, stood on both sides of the river, and she caused the river’s growth. Then Thunder took up from the river a great stone and threw it at her and spoke so: “At its source shall the river be dammed.” He did not miss his target, and in that moment he threw himself towards land and got hold of a certain rowan shrub, and thus stepped out of the river. From this comes the saying that the rowan is Thunder’s deliverance.\epb\epg


\bpg\bpa En er Þórr kom til Geirrøðar, þá var þeim fé-lǫgum vísat fyrst í geita-hús til her-bergis, ok var þar einn stóll til sę́tis, ok sat Þórr þar. Þá varð hann þess varr, at stóllinn fór undir honum upp at rę́fri. Hann stakk Gríðar-veli upp í raftana ok lét sígast fast á stólinn. Varð þá brestr mikill, ok fylgði skrę́kr. Þar hǫfðu verit undir stólinum dǿtr Geirrøðar, Gjálp ok Greip, ok hafði hann brotit hrygginn í báðum. Þa kvað Þórr:\epa

\bpb And when Thunder came to Garfrith’s home the fellows were first shown into a goathouse for lodgings, and therein one chair was for sitting, and Thunder sat down on it. Then he noticed that the chair beneath him was moving up toward the roof. He thrusted Grith’s stave up against the rafters and made it push firm onto the chair. Then there was a great crack, followed by a shriek; there beneath the chair had been the daughters of Garfrith, Yelp and Grope, and he had broken both their backs. Then Thunder quoth:\epb\epg

\bvg\bva „\alst{Ęi}nu \edtrans{\emph{sinni}}{time}{\Bfootnote{metr. and sens. emend.; om. \Upsaliensis}} \hld\ nęytta’k \alst{a}lls męgins &
\ind \alst{jǫ}tna gǫrðum \alst{í} &
þá’s \alst{G}jǫlp ok \alst{G}ręip, \hld\ dǿtr \alst{G}ęir-raðar, &
\ind vildu \alst{h}ęfja mik til \alst{h}imins.“\eva

\bvb “Only one time I used all my might \\
\ind in the yards of the ettins, \\
when Yelp and Grope, daughters of Garfrith, \\
\ind would lift me to the heaven.”\evb\evg

\sectionline

\section{On the Making of Glapner}

The following stanza about the making of Glapner, the fetter used to bind the Fenrerswolf, is found in the short work on kennings today called the \emph{Little Scalda} (\emph{Lítla skálda}), which text was probably used as a source by Snorre; see further \textcite[129--47]{Males2020}.  A variant of this stanza is transparently paraphrased in \Gylfaginning\ 28: \emph{Hann var gǫrr af sex hlutum: af dyn kattarins ok af skeggi konunnar ok af rótum bjargsins ok af sinum bjarnarins ok af anda fisksins ok af fogls hráka.} ‘It [Glapner] was made of six things: of the cat’s din and of the woman’s beard and of the mountain’s root and of the bear’s sinews and of the fish’s breath and of the fowl’s spittle.’  The two differences—\emph{hráka} ‘spittle’ for \emph{mjǫlk} ‘milk’, and the inverted order of lines 2 and 3—suggest that Snorre had access to a somewhat different version.  It is not attributed to any named poem.

\sectionline

\bvg\bva Ór \alst{k}attar dyn \hld\ ok ór \alst{k}onu skeggi, &
ór \alst{f}isks anda \hld\ ok ór \alst{f}ugla mjǫlk, &
ór \alst{b}ergs rótum \hld\ ok \alst{b}jarnar sinum, &
\ind ór því vas hann \alst{G}leipnir \alst{g}ǫrr.\eva

\bvb “From cat’s din and from woman’s beard; \\
from fish’s breath and from fowls’ milk; \\
from mountain’s roots and bear’s sinews; \\
\ind from this was Glapner made.”\evb\evg

\sectionline
% Eddic fragments

% Heroic poems, in order of the Codex Regius'
	\book{The Lay of Wayland. (\emph{Vǫlundarkviða})}\bookStart

\bva Níðuðr hét konungr í Svíþjóð.
\bva Hann átti tvá sonu ok eina dóttur. Hon hét Böðvildr.
\bva Bræðr váru þrír, synir Finnakonungs.
\bva Hét einn Slagfiðr, annarr Egill, þriði Völundr.
\bva Þeir skriðu ok veiddu dýr. Þeir kómu í Úlfdali ok gerðu sér þar hús.
\bva Þar er vatn, er heitir Úlfsjár.
\bva Snemma of morgin fundu þeir á vatnsströndu konur þrjár, ok spunnu lín.
\bva Þar váru hjá þeim álftarhamir þeira. Þat váru valkyrjur.
\bva Þar váru tvær dætr Hlöðvés konungs, Hlaðguðr svanhvít ok Hervör alvitr, in þriðja var Ölrún Kjársdóttir af Vallandi.
\bva Þeir höfðu þær heim til skála með sér. Fekk Egill Ölrúnar, en Slagfiðr Svanhvítrar, en Völundr Alvitrar.
\bva Þau bjuggu sjau vetr. Þá flugu þær at vitja víga ok kómu eigi aftr.
\bva Þá skreið Egill at leita Ölrúnar, en Slagfiðr leitaði Svanhvítrar, en Völundr sat í Úlfdölum.
\bva Hann var hagastr maðr, svá at menn viti, í fornum sögum.
\bva Níðuðr konungr lét hann höndum taka, svá sem hér er um kveðit: \\%E

\bvb Nithad was named a king in Sweden.
\bvb He owned two sons and one daughter, she was called Beadhild.
\bvb There were three brothers, the sons of a Finnish king.
\bvb The first was called Beatfinn, the second Egil, the third Wayland.
\bvb They travelled on skis and hunted wild animals. They came into Wolfdale and made for themselves houses there.
\bvb There is a water there, called Wolfsea.
\bvb Early in the morning they found on the lake-shore three women, and they were spinning linen.
\bvb By them were their swan-\textbf{hames}; those were \textbf{Walkyrries}.
\bvb Two of them were the daughters of King Latheway, Loadguth Swanwhite and Hereware Allwit; the third was Alerune Kear's daughter, from \textbf{Walland}.
\bvb The [brothers] brought the [women] with them to their halls. Egil got Alerune, but Beatfinn Swanwhite, but Wayland Allwit.
\bvb They lived there for seven winters, then they left to attend battles, and did not return.
\bvb Then Egil left on skis to seek out Alerune, but Beatfinn sought out Swanwhite, but Wayland stayed in Wolfdale.
\bvb He was the most skillful man, which men have known in ancient tales.
\bvb King Nithad had him taken, about which this has been sung: \\

\chapterStart

\begin{verse}
\bva Męyjar flugu sunnan \hld Myrkvið í gǫgnum
alvitr ungar, \hld ørlǫg drýgja; \\%E
\end{verse}

\bvb Maidens flew from the south through \textbf{Mirkwood}, young allwits†, to fulfill orlay†. \\

\begin{verse}
\bva þær á sævarstrǫnd \hld sęttusk at hvílask
drósir suðrǿnar, \hld dýrt lín spunnu. \\%E
\end{verse}

\bvb They on the sea-shore set down to rest, the southern ladies, expensive linen they span. \\

\begin{verse}
\bva Ęin nam þęira \hld Ęgil at vęrja
fǫgr mær fira \hld faðmi ljósum.
Ǫnnur vas Svanhvít, \hld svanfjaðrar dró,
ęn hin þriðja \hld þęira systir
varði hvítan \hld hals Vǫlundar. \\%E
\end{verse}

\bvb One of them took to ward Egil, the wise maiden of men by the light bosom; the second was Swanwhite, her swan-feathers she pulled, but the third of the sisters warded the white neck of Wayland. \\

\begin{verse}
\bva Sǫ́tu síðan \hld sjau vetr at þat,
ęn hinn átta \hld allan þrǫ́ðu,
ęn hinn níunda \hld nauðr of skilði,
męyjar fýstusk \hld á myrkvan við,
alvitr ungar \hld ørlǫg drýgja. \\%E
\end{verse}

\bvb Then they remained for seven winters after that, — and all the eighth, they yearned, — and on the ninth, need divorced them: the maidens longed for the mirky wood; the young allwits, to fulfill orlay.\footnotemark[1] \\
\footnotetext[1]{The swan-maidens long to return to Mirkwood (the ravaged lands of the Gots and Huns) to judge battles for Weden, which as walkirries is their orlay (fate and duty).}

\begin{verse}
\bva Kom þar af vęiði \hld veðręygr skyti
Vǫlundr líðandi \hld of langan veg,
Slagfiðr ok Ęgill, \hld sali fundu auða,
gingu út ok inn \hld ok umb sǫ́usk. \\%E
\end{verse}

\bvb Came there from the hunt, the weather-eyed shooter, Wayland passing from a long journey. Slayfinn and Egil found the halls deserted, they walked out and in, and about them looked. \\

\begin{verse}
\bva Austr skręið Ęgill \hld at Ǫlrúnu,
ęn suðr Slagfiðr \hld at Svanhvítu,
ęn ęinn Vǫlundr \hld sat í Ulfdǫlum. \\%E
\end{verse}

\bvb East travelled Egil for Alerune, but southwards Slayfinn for Swanwhite, but alone Wayland remained, in the Wolfdales. \\

\begin{verse}
\bva Hann sló goll rautt \hld við gim fastan,
lukði hann alla \hld lindbaugum vel;
svá bęið hann \hld sinnar ljóssar
kvánar, ef hǫ́num \hld of koma gęrði. \\%E
\end{verse}

\bvb He struck the red gold by fastened gemstone, enclosed he all the armrings well; thus awaited he his bright wife, if to him she might come. \\

\begin{verse}
\bva Þat spyrr Níðuðr, \hld Níara dróttinn,
at ęinn Vǫlundr \hld sat í Ulfdǫlum;
nóttum fóru sęggir, \hld nęglðar vǫ́ru brynjur,
skildir bliku þęira \hld við hinn skarða mána. \\%E
\end{verse}

\bvb Nithad learns, Lord of the Nears, that alone Wayland remained in Wolfdales; at nights travelled warriors — nailed were their byrnies — their shields gleamed by the waning moon. \\

\begin{verse}
\bva Stigu ór sǫðlum \hld at salar gafli,
gingu inn þaðan \hld ęndlangan sal,
sǫ́u þęir á bast \hld bauga dręgna,
sjau hundruð allra, \hld es sá sęggr átti. \\%E
\end{verse}

\bvb They stepped out of the saddles, towards the hall’s gables, they walked inside thence across the length of the hall. They saw on a bast-rope, armrings drawn, seven hundred in all, which that man owned. \\

\begin{verse}
\bva Ok þęir af tóku \hld ok þęir á létu
fyr ęinn útan, \hld es af létu;
kom þar af vęiði \hld veðręygr skyti
Vǫlundr líðandi \hld of langan veg. \\%E
\end{verse}

\bvb And they took off, and they put back on, but for one, which away they put. — Came there from the hunt, the weather-eyed shooter, Wayland passing from a long journey. \\

\begin{verse}
\bva Gekk brúnni \hld beru hold stęikja,
ár brann hrísi \hld allþurru fura,
viðr hinn vindþurri, \hld fyr Vǫlundi. \\%E
\end{verse}

\bvb He went, the brown she-bear’s hull to roast; early burned the brushwood, the all-dry pine, the wind-dry wood, before Wayland.

\begin{verse}
\bva Sat á berfjalli, \hld bauga talði,
alfa ljóði \hld ęins saknaði.
hugði at hęfði \hld Hlǫðvés dóttir,
Alvitr unga, \hld væri aptr komin.
 
\bvb Sat he on the bare mountain, his rings counted — the prince of elves was missing one! He thought that the daughter of Ladwigh might have it, that the young Allwit might be come again.

\begin{verse}
\bva Sat hann svá lęngi, \hld at hann sofnaði,
ok hann vaknaði \hld viljalauss;
vissi sér á hǫndum \hld hǫfgar nauðir,
ęn á fótum \hld fjǫtur of spęntan. \\%E
\end{verse}

\bvb He sat so long, that asleep he fell, and he awoke, powerless; he knew on his hands tortuous restraints, and on his feet fetters tightened.

\begin{verse}
(Vǫlundr kvað) \\
\bva Hvęrir ’ró jǫfrar \hld þęir’s á lǫgðu
bęstisíma \hld ok bundu mik? \\%E
\end{verse}

\bvb Wayland quoth: \\
“Who are those princes, that laid on thick bast-ropes, and bound me?”

\begin{verse}
\bva (Kallaði Níðuðr, \hld Níara dróttinn):
“Hvar gazt Vǫlundr, \hld vísi alfa,
óra aura, \hld í Ulfdǫlum?” \\%E

\end{verse}

\bvb Nithad called, Lord of the Nears: “Where got thou Wayland, leader of Elves, our ounces\footnotemark[1], in Wolfdales?”
\footnotetext[1]{Of gold.}

\begin{verse}
(Vǫlundr kvað) \\
\bva Goll vas þar ęigi \hld á Grana lęiðu,
fjarri hugða’k várt land \hld fjǫllum Rínar.
Man’k at męiri \hld mæti ǫ́ttum,
es vér hęil hjú \hld hęima vǫ́rum. \\%E
Hlaðguðr ok Hervǫr \hld borin vas Hlǫðvé,
kunn vas Ǫlrún \hld Kíars dóttir.  \\%E
\end{verse}

\bvb Wayland quoth: \\
\bvb “Gold was there not on Grane’s path, far I judge our land from the mountains of the Rhine. I remembered that we owned a more precious thing, when we a healthy household were at home. Ladguth and Harware were born to Ladwigh, known was Alerune, Keer’s daughter.”

\begin{verse}
\bva Úti stóð kunnig \hld kvǫ́n Níðaðar,
hón inn of gekk \hld ęndlangan sal,
stóð á golfi, \hld stilti rǫddu:
es-a sá nú hýrr, \hld es ór holti fęrr. \\%E
\end{verse}

\bvb Outside stood the cunning wife of Nithad; she walked inside across the length of the hall; she stood on the floor, steered her voice: “He is not happy, who now comes out of the wood.\footnotemark[1]”
\footnotetext[1]{The abducted Wayland.}

\begin{verse}
\bva Tęnn hǫ́num tęygjask \hld es hǫ́num’s tét sverð
ok hann Bǫðvildar \hld baug of þękkir.
Ǫ́mun eru augu \hld ormi hinum frána,
tęnn hǫ́num tęygjask, \hld es tét es sverð
ok hann Bǫðvildar \hld baug of þękkir,
sníðið hann sina \hld sinna magni,
sętið hann síðan \hld í Sævarstǫð. \\%E
\end{verse}

\bvb The teeth on him are bared, when he is shown the sword, and he recognizes the Beadhild’s arm-ring; like are the eyes to those of the bright snake. — Cut ye from him the might of his sinews, and place him then on Seastead!

\begin{verse}
\bva Svá var gǫrt, at skornar váru sinar í knésfótum ok settr í holm einn, er þar var fyrir landi, er hét Sævarstaðr. Þar smíðaði hann konungi allskyns gǫrsimar; engi maðr þorði at fara til hans, nema konungr einn. Vǫlundr kvað: \\%E
\end{verse}

\bvb Thus was done, that the sinews in his houghs were cut, and he was placed on an alone islet, which there lay by the land, and was called Seastead. There he smithed for the king all manner of jewels; no man dared to travel to him, but the king alone. Wayland quoth: \\

\begin{verse}
\bva Sé’k Níðaði \hld sverð á linda,
þat’s ek hvęsta \hld sęm hagast kunna’k
ok ek hęrða’k \hld sęm hǿgst þótti;
sá ’s mér fránn mækir \hld æ fjarri borinn.
sé’kk-a þann Vǫlundi \hld til smiðju borinn. \\%E
\end{verse}

\bvb I see a sword on Nithad’s belt, the one I sharped as I most handily knew, and I hardened as to me most easily seemed; now that gleaming sword is ever carried far from me, I see it not carried to Wayland’s smithy.

\begin{verse}
\bva Nú berr Bǫðvildr \hld brúðar minnar,
bíð’k-a þess bót, \hld bauga rauða. \\%E
\end{verse}

\bvb Now Beadhild bears the red armrings—I get no recompense for that—of my bride.

\begin{verse}
\bva Sat hann né svaf ávalt \hld ok sló hamri;
vél gęrði hęldr \hld hvatt Níðaðí;
drifu ungir tvęir \hld á dýr séa
synir Níðaðar \hld í Sævarstǫð. \\%E
\end{verse}

\bvb He sat nor slept always, and struck the hammer, rather he keenly planned treachery for Nithad; two young ones were hurrying to look at the precious things, the sons of Nithad, towards Seastead.

\begin{verse}
\bva Kvǫ́mu til kistu, \hld krǫfðu lukla,
opin vas illúð, \hld es í sǫ́u,
fjǫlð vas þar męina, \hld es mǫgum sýndisk
at væri goll rautt \hld ok gǫrsimar. \\%E
\end{verse}

\bvb They came to the chest, demanded the keys, open was the evil, when inside they looked; in there was a multitude of harm, which to the lads seemed like it were red gold and jewels.

\begin{verse}
\bva Komið ęinir tvęir, \hld komið annars dags;
ykkr læt’k þat goll \hld of gefit verða;
sęgið-a męyjum \hld né salþjóðum,
manni ęngum, \hld at mik fyndið. \\%E
\end{verse}

\bvb Come alone ye two, come another day! To you I will have that gold be given; tell not maidens, nor the people of the hall, not any man, that ye met me.

\begin{verse}
\bva Snimma kallaði \hld sęggr á annan,
bróðir á bróður: \hld gǫngum baug séa.
Kómu til kistu, \hld krǫfðu lukla,
opin vas illúð \hld es í litu. \\%E
\end{verse}

\bvb Early called one man to another, brother to brother: “Let us go see the rings!”. They came to the chest, demanded the keys, open was the evil, when inside they looked.

\begin{verse}
\bva Snęið af hǫfuð \hld húna þęira
ok und fęn fjǫturs \hld fǿtr of lagði,
ęn þær skálar, \hld es und skǫrum vǫ́ru,
svęip útan silfri, \hld sęldi Níðaði. \\%E
\end{verse}

\bvb He sliced off the heads of those bear-cubs\footnotemark[1], and under the fetter’s fen their feet did lay, — but the bowls, which under their locks were\footnotemark[2], he coated with silver, and gave to Nithad.
\footnotetext[1]{The sons of Nithad.}
\footnotetext[2]{Their skulls.}

\begin{verse}
\bva Ęn ór augum \hld jarknastęina
sęndi kunnigri \hld kvǫ́n Níðaðar;
ęn ór tǫnnum \hld tvęggja þęira
sló brjóstkringlur, \hld sęndi Bǫðvildi. \\%E
\end{verse}

\bvb But out of the eyes earkenstones, he sent to the cunning wife of Nithad, but out of the teeth of the two, he struck breast-brooches, [which he] sent to Beadhild.

\begin{verse}
\bva Þá nam Bǫðvildr \hld baugi at hrósa
[...] \hld es brotit hafði,
“þori’k-a’k sęgja, \hld nema þér ęinum.”  \\%E
\end{verse}

\bvb Then Beadhild began to praise the ring, [...] which she had broken, “I dare not tell anyone, but thee alone.”\footnotemark[1]
\footnotetext[1]{Clearly the verse is incomplete. Beadhild breaks a ring she has been given, but does not dare tell anybody but Wayland.}

\begin{verse}
\bva “Ek bǿti svá \hld brest á golli,
at fęðr þínum \hld fęgri þykkir,
ok mǿðr þinni \hld miklu bętri,
ok sjalfri þér \hld at sama hófi.”  \\%E
\end{verse}

\bvb “I will mend so the crack on the gold, that to thy father it will seem fairer, and to thy mother much better, and to thyself just the same.”

\begin{verse}
\bva Bar hann hána bjóri, \hld þvíat hann bętr kunni,
svát hón í sessi \hld of sofnaði.
»Nú hęfk hęfnt \hld harma minna
allra nema ęinna \hld íviðgjǫrnum.” \\%E
\end{verse}

\bvb He overcame her with beer — for he was more cunning — so that she in the seat asleep did fall. “Now I have avenged my injustices, — all but one, — on the insidious ones\footnotemark[1].”
\footnotetext[1]{King Nithad and his wife.}

\begin{verse}
\bva “Vęl ek, kvað Vǫlundr, \hld verða’k á fitjum,
þęim’s mik Níðaðar \hld nǫ́mu rekkar.«”
Hlæjandi Vǫlundr \hld hófsk at lopti,
grátandi Bǫðvildr \hld gekk ór ęyju.
tregði fǫr friðils \hld ok fǫður vreiði. \\%E
\end{verse}

\bvb “Well I”, quoth Wayland, “fall on my paddles, those which Nithad’s men bereaved me of!”\footnotemark[1] Laughing Wayland threw himself in the air; weeping Beadhild went from the island, grieved the flight of the lover, and the fury of the father.
\footnotetext[1]{\emph{C-V}: \emph{fit} ‘the webbed foot of water-birds’, the reader may picture for himself. Wayland has crafted wings in stead of his feet, of which use Nithad’s men deprived him.}

\begin{verse}
\bva Úti stóð kunnig \hld kvǫ́n Níðaðar,
ok hón inn of gekk \hld ęndlangan sal,
— ęn hann á salgarð \hld sęttisk at hvílask —,
»Vakir þú Níðuðr, \hld Níara dróttinn?«  \\%E
\end{verse}

\bvb Outside stood the cunning wife of Nithad; she walked inside across the length of the hall, — but he, on the courtyard, set down to rest —, “Art thou awake, Nithad, Lord of the Nears?”

\begin{verse}
\bva “Vaki’k ávalt \hld viljalauss,
sofna’k minst, \hld síz sonu dauða,
kęll mik í hǫfuð, \hld kǫld erumk rǫ́ð þín,
vilnumk þess nú, \hld at við Vǫlund dǿma’k.” \\%E
\end{verse}

\bvb “I wake always, powerless; I sleep the least, since the death of my sons. My head freezes, cold are thy counsels; I wish now but that, to speak with Wayland.”

\begin{verse}
\bva “Sęg mér þat Vǫlundr, \hld vísi alfa,
af hęilum hvat varð \hld húnum mínum?” \\%E
\end{verse}

\bvb “Say that to me, Wayland, leader of Elves, what became of my healthy bear-cubs?”

\begin{verse}
\bva Ęiða skalt mér áðr \hld alla vinna,
at skips borði \hld ok at skjaldar rǫnd,
at mars bǿgi \hld ok at mækis ęgg
at þú kvęlj-at \hld kvǫ́n Vǫlundar,
né brúði minni \hld at bana verðir,
þótt kvǫ́n ęigim, \hld þá’s ér kunnið,
eða jóð ęigim \hld innan hallar. \\%E
\end{verse}

\bvb “Before that shalt thou swear me all oaths, — by the deck of the ship and the rim of the shield, by the bough of the steed and the edge of the sword, — that thou wilt not torment the wife of Wayland, nor of my bride become the bane, though we might own a wife, which ye might know, or own a baby inside the hall.\footnotemark[1]
\footnotetext[1]{Wayland makes Nithad swear an oath that he will not harm Beadhild (“a wife, which ye might know”), nor their (yet unborn) child.}

\begin{verse}
\bva Gakk til smiðju, \hld es gęrðir þú,
þar fiðr þú bęlgi \hld blóði stokna,
snęið’k af hǫfuð \hld húna þinna
ok und fęn fjǫturs \hld fǿtr of lagða’k. \\%E
\end{verse}

\bvb Go to the smithy, which thou made; there thou wilt find bellows, with blood sprinkled. I sliced off the heads of those bear-cubs, and under the fetter’s fen their feet did I lay.

\begin{verse}
\bva Ęn þær skálar, \hld es und skǫrum vǫ́ru,
svęip’k útan silfri, \hld sęlda’k Níðaði,
ęn ór augum \hld jarknastęina,
sęnda'k kunnigri \hld kvǫ́n Níðaðar. \\%E
\end{verse}

\bvb But the bowls, which under their locks were, I coated with silver, and gave to Nithad. But out of the eyes arkenstones, I sent to the cunning wife of Nithad.

\begin{verse}
\bva Ęn ór tǫnnum \hld tvęggja þęira
sló’k brjóstkringlur, \hld sęnda’k Bǫðvildi;
nú gęngr Bǫðvildr \hld barni aukin,
ęingadóttir \hld ykkur bęggja.” \\%E
\end{verse}

\bvb But out of the teeth of the two, I struck breast-brooches, [which] I sent to Beadhild; now goes Beadhild pregnant with child, the only daughter of you both.”

\begin{verse}
\bva “Mæltir-a þú þat mál, \hld es mik męir tregi,
né þik vilja’k Vǫlundr \hld verr of níta;
es-at svá maðr hǫ́r, \hld at þik af hęsti taki,
né svá ǫflugr, \hld at þik neðan skjóti.
þar’s þú skollir \hld við ský uppi.” \\%E
\end{verse}

\bvb “Thou could not have spoken that speech, which might grieve me more, nor could I wish worse, Wayland, to deny thee. There is no man so high, that he might from a horse take thee, nor so mighty, that he might shoot thee down, there where thou jeers, high against the clouds.”

\begin{verse}
\bva Hlæjandi Vǫlundr \hld hófsk at lopti,
ęn ókátr Níðuðr \hld þá ęptir sat. \\%E
\end{verse}

\bvb Laughing Wayland threw himself in the air, but bleak Nithad then afterwards remained.

\begin{verse}
\bva “Upp rís Þakkráðr, \hld þræll minn bazti,
bið Bǫðvildi, \hld męy hina bráhvítu,
gangi fagrvarið \hld við fǫður rǿða.” \\%E
\end{verse}

\bvb “Rise up Thankred, my best thrall; ask Beadhild, — the brow-white maiden, — to go fair-clothed, with her father to counsel.”

\begin{verse}
\bva “Es þat satt Bǫðvildr, \hld es sǫgðu mér,
sǫ́tuð it Vǫlundr \hld saman í holmi?” \\%E
\end{verse}

\bvb “Is it true, Beadhild, what they said to me; sat thou and Wayland, together on the island?” \\

\begin{verse}
\bva Satt ’s þat Níðuðr \hld es sagði þér.
Sǫ́tum vit Vǫlundr \hld saman í holmi
ęina ǫgurstund, \hld æva skyldi;
ek vætr hǫ́num \hld vinna kunna’k,
ek vætr hǫ́num \hld vinna mátta’k.
\end{verse}

\bvb “It is true, Nithad, what \emph{he} said\footnotemark[1] to thee; I and Wayland sat together on the island, for one grave moment — it should never have been. I \emph{knew} nought struggle against him, I \emph{could} nought struggle against him.\footnotemark[2]”
\footnotetext[1]{Beadhild, knowing that the only one who is aware of what happened is Wayland, changes the person from her father’s general plural, to the specific singular.}
\footnotetext[2]{She was both mentally (\emph{C-V}: \emph{kunna} ‘know, understand’) and physically (\emph{C-V}: \emph{mega} ‘to have strength to do, avail’) incapable of struggling against him. As Finnur comments, an unsurpassed ending.}
%
	\bookStart{First Lay of Hallow Hundingsbane}[Helgakviða Hundingsbana fyrsta]

\bvg
\bva Ár vas alda \hld\ þat’s arar gullu &
hnigu hęilǫg vǫtn \hld\ af Himinfjǫllum; &
þá hafði Hęlga \hld\ inn hugumstóra &
Borghildr borit \hld\ í Brálundi.\eva

\bvb It was the beginning of \inx[C]{eld}[elds], as eagles shrieked; holy waters poured down from the Heavenfells; then Burhild in Browlund gave birth to Hallow the Great-hearted.\evb
\evg


\bvg
\bva Nótt varð í bǿ, \hld\ nornir kvǫ́mu, &
þę́r’s ǫðlingi \hld\ aldr of skópu; &
þann bǫ́ðu fylki \hld\ frę́gstan verða &
ok buðlunga \hld\ bęztan þykkja.\eva

\bvb Night came in the settlement; norns came, those who did shape the prince’s life; that marshaller <= Hallow> they declared would become most renowned, and of kings seem the foremost.\evb
\evg


\bvg
\bva Sneru þę́r af afli \hld\ ørlǫgþǫ́ttu &
þá’s borgir braut \hld\ í Brálundi; &
þę́r um gręiddu \hld\ gullinsímu &
ok und mána sal \hld\ miðjan fęstu.\eva

\bvb They turned with their might the strands of \inx[C]{orlay}, as he broke cities in Browlund; they arranged golden bands, and under the moon's hall fastened [them in] the middle.\evb
\evg
%
	\bookStart{The Lay of Hallow Harwardson}[Hęlgakviða Hjǫrvarðssonar]

Frá Hjǫrvarði ok Sigrlinn

Hjǫrvarðr hét konungr. Hann átti fjórar konur. Ein hét Alfhildr; sonr þeira hét Heðinn. Ǫnnur hét Sę́reiþr; þeira sonr hét Humlungr. In þriðja hét Sinrjóð; þeira sonr hét Hymlingr. Hjǫrvarðr konungr hafði þess heit strengt at eiga þá konu er hann vissi vę́nsta. Hann spurði at Sváfnir konungr átti dóttur allra\footnote[‘vęnallra’ \emph{corr.} \Regius] fegrsta; sú hét Sigrlinn. Iðmundr hét jarl hans; Atli var hans sonr er fór at biðja Sigrlinnar til handa konungi. Hann dvalðisk vetrlangt með Sváfni konungi. Fránmarr hét þar jarl, fóstri Sigrlinnar; dóttir hans hét Álǫf. Jarlinn réð, at meyjar var synjat, ok fór jarlinn heim. Atli jarls sonr stóð einn dag við lund nǫkkurn, en fugl sat í limunum uppi yfir hánum ok hafði heyrt til, at hans menn kǫlluðu vę́nstar konur þę́r, er Hjǫrvarðr konungr átti. Fuglinn kvakaði, en Atli hlýddi, hvat hann sagði. Hann kvað:

Regarding Harward and Sighlind


\bvg
\bva Sáttu Sigrlinn, \hld\ Sváfnis dóttur, &
męyna fęgrstu \hld\ î munarhęimi? &
Þó hagligar \hld\ Hjǫrvarðs konur &
gumnum þykkja \hld\ at Glasislundi.\eva

\bvb 1\evb
\evg


\bvg
\bva „Mundu við Atla \hld\ Iðmundar son &
fugl fróðhugaðr \hld\ flęira mę́la?“ &
„Mun’k ef mik buðlungr \hld\ blóta vildi &
ok kýs’k þat’s ek vil \hld\ ór konungs garði.“\eva

\bvb 2\evb
\evg


\bvg
\bva 3\eva

\bvb 3\evb
\evg


\bvg
\bva 4\eva

\bvb 4\evb
\evg


\bvg
\bva 5\eva

\bvb 5\evb
\evg


\bvg
\bva 6\eva

\bvb 6\evb
\evg


\bvg
\bva 7\eva

\bvb 7\evb
\evg


\bvg
\bva Sverð vęit’k liggja \hld\ î Sigarsholmi, &
fjórum fę́ra \hld\ enn fimm tǫgu; &
ęitt es þęira \hld\ ǫllum bętra &
vígnesta bǫl \hld\ ok varið golli.\eva

\bvb Swords I know lying, in Sigharsholm, four less than fifty. One of them is better than all—the bale of war-needles\footnoteB{The kenning \emph{vígnest} also appears in} \ken{spears?}—and inlaid with gold.\evb
\evg


\bvg
\bva Hringr ’s î hjalti, \hld\ hugr ’s î miðju, &
ógn ’s î oddi, \hld\ þęim’s ęiga getr; &
liggr með ęggju \hld\ ormr dręyrfáiðr &
en ȧ valbǫstu \hld\ verpr naðr hala.\eva

\bvb A ring is in the hilt; courage is in the middle; fear is in the point, for the one who gets to own it; along the blade lies a serpent painted in blood, but on the walbast\footnoteB{An unclear part of the sword-hilt; see \Sigrdrifumal\ 7.} an adder chases its tail.\evb
\evg

	\bookStart{Second Lay of Hallow Hundingsbane}[Helgakviða Hundingsbana aðra]

BPG
BPA Helgi fekk Sigrúnar ok áttu þau sonu; var Helgi eigi gamall. Dagr Hǫgna sonr blótaði Óðin til fǫðurhefnda. Óðinn léði Dag geirs síns. Dagr fann Helga, mág sinn, þar sem heitir at Fjǫturlundi. Hann lagði í gǫgnum Helga með geirnum. Þar fell Helgi en Dagr reið til fjalla ok sagði Sigrúnu tíðindi:

BPB Hallow got Sighrun, and they owned sons; Hallow was not old. Day, son of Hain, blooted† to Weden to take revenge for his father. Weden lent Day his spear. Day found Hallow, his brother-in-law, at a place called Fetterlund; he laid the spear through Hallow. There fell Hallow, but Day rode to the fells and told Sighrun the news:
EPB


\bvg
\bva „Trauðr em ek, systir, \hld\ trega þér at sęgja &
þvíat ek hęfi nauðigr \hld\ nipti grę́tta: &
Fell í morgun \hld\ und Fjǫturlundi &
buðlungr sá’s vas \hld\ bęztr í hęimi &
ok hildingum \hld\ á halsi stóð.“\eva

\bvb “Regretful am I, sister, to grieve thee by saying—for, forced must I cause my kinswoman to cry: This morning fell, ’neath Fetterlund, that prince who was in the world the best, and on the throats of rulers stood.”\evb
\evg

...

\bvg
\bva „Fyrr vil’k kyssa \hld\ konung ólifðan &
an þú blóðugri \hld\ brynju kastir; &
hár es þitt, Helgi, \hld\ hélu þrungit, &
allr es vísi \hld\ valdǫgg slęginn, &
hęndr úrsvalar \hld\ Hǫgna mági; &
hvé skal’k þér, buðlungr, \hld\ þess bót of vinna?“\eva

\bvb “Sooner would I kiss the unliving king, than thou the bloody byrnie mightst cast away. Thy hair is, Hallow, with hoarfrost thick: the prince is all with corpse-dew whipped:\footnoteB{For the formulation cf. \Baldrsdraumar\ 5.} the hands wet-cold on the kinsman of Hain. How shall I for thee, lord, remedy that?”\evb
\evg


\bvg
\bva „Ęin vęldr þú, Sigrún \hld\ frá Sefafjǫllum, &
es Hęlgi es \hld\ harmdǫgg slęginn: &
Grę́tr þú, gullvarit, \hld\ grimmum tǫ́rum, &
sólbjǫrt suðrǿn, \hld\ áðr þú sofa gangir, &
hvęrt fęllr blóðugt \hld\ á brjóst grami, &
úrsvalt, innfjalgt \hld\ ękka þrungit.“\eva

\bvb “Thou alone causest, Sighrun from the Sevefells, that Hallow be by harm-dew whipped; thou criest, gold-covered, bitter tears, sun-bright southern lady, before thou to sleep mightst go. Each one falls bloody on the breast of the ruler, wet-cold and stifled, pressed forth by grief.”\evb
\evg
%
%	\include{books/Spae of Griper.tex}%
%	\include{books/Speeches of Reyn.tex}%
	\bookStart{The Speeches of Fathomer}[Fáfnismǫ́l]

Frá dauða Fáfnis

From the death of Fathomer

\bvg {\small [Fathomer quoth:]}
\bva „Svęinn ok svęinn! \hld\ Hvęrjum estu svęini of borinn? &
\ind Hvęrra estu manna mǫgr? &
es þú á Fáfni rautt \hld\ þínn hinn frána mę́ki; &
\ind stǫndumk til hjarta hjǫrr!“\eva

\bvb “Swain and swain! To which swain art thou born; of which men art thou the son? As thou on Fathomer hast reddened thy gleaming blade, the sword stands to my the heart!”\evb
\evg


BPG
BPA Sigurðr dulði nafns síns fyr því at þat var trúa þeira í forneskju at orð feigs manns mę́tti mikit ef hann bǫlvaði óvin sínum með nafni. Hann kvað:EPA

BPB Siward concealed his name, because it was their belief in ancient times that the word of a \inx[C]{fey} man could do much if he cursed his enemy by his name. He \ken*{= Siward} quoth:EPB
EPG


\bvg
\bva „Gǫfugt dýr ek hęiti \hld\ en ek gęngit hef’k &
\ind hinn móðurlausi mǫgr, &
fǫður ek á’kk-a \hld\ sem fira synir, &
\ind gęng ek ęinn saman.“\eva

\bvb “Noble beast I am called, but I have walked as the motherless lad. A father I own not, like the sons of men do; I walk alone.”\evb
\evg


\bvg {\small [Fathomer quoth:]}
\bva „Vęizt, ef fǫður né átt-at \hld\ sem fira synir, &
\ind af hvęrju vastu undri alinn?“\eva

\bvb “Knowest thou, if thou haddest not a father like the sons of men, by which wonder thou wast born?”\evb
\evg


\bvg {\small [Siward quoth:]}
\bva „Ę́tterni mitt \hld\ kveð’k þér ókunnigt vesa &
\ind ok mik sjalfan hit sama: &
Sigurðr ek hęiti \hld\ Sigmundr hét minn faðir &
\ind es hęf’k þik vápnum vegit.“\eva

\bvb “My lineage I say is unknown to thee, and my self the same.\footnoteB{The meaning is that Fathomer would not recognize Siward’s lineage (i.e. his father) or name, since he is an orphan who up until this point has not won any glory. He is not saying that he is lineage is unknown even to himself, since \emph{sjalfan mik} ‘my self’ is accusative, not dative.} Siward I am called—Sighmund was called my father—who with weapons have struck thee.”\evb
\evg


\bvg {\small [Fathomer quoth:]}
\bva „Hvęrr þik hvatti, \hld\ hví hvętjask lézt, &
\ind mínu fjǫrvi at fara? &
Hinn fránęygi svęinn, \hld\ þú áttir fǫður bitran, &
\ind ábornu skjór á skęið.“\eva

\bvb “Who goaded thee, why didst thou let thyself be goaded, my life for to destroy? Gleaming-eyed swain, thou haddest a sharp father; inborn traits show quickly.\footnoteB{The original is unclear. \emph{á skęið} means roughly ‘rapidly, quickly’; thus \emph{ríða á skęið} \CV: ‘to ride at full speed’, but apart from that the words are exceptionally unclear. \LaFarge\ read ‘your innate qualities show quickly’, suggesting two unattested words: an adjective \emph{*áborinn} ‘innate, inborn’ and a verb \emph{skjóa} ‘to show’. Yet the lack of i-umlaut in the supposed 3rd sg. pres. ind. \emph{skjór} is difficult. We would expect \emph{**skýr}, as in \emph{skjóta} ‘to shoot,’ with 2nd/3rd sg. pres. ind \emph{skýtr}. A solution here would be reading a 2nd sg. pres. subj. \emph{skjóir}, with a vowel TODO}”\evb
\evg

TODO: More verses...
%
	\bookStart{The Speeches of Sighdrive}[Sigrdrífumǫ́l]

\begin{flushright}%
Dating \parencite{Sapp2022}: C10th (0.961)

Meter: \Ljodahattr%
\end{flushright}

% Introduction

Many of the verses are quoted in \VolsungaSaga, but notably the two prayer-verses are missing; possibly an instance of Christian censorship. TODO

\sectionline

\bvg {\small [Sighdrive quoth:]}
\bva „Lęngi ek svaf, \hld\ lęngi ek sofnuð vas, &
\ind lǫng eru lýða lę́; &
Óðinn því vęldr \hld\ es ęigi mátta’k &
\ind bregða blundstǫfum.“\eva

\bvb “Long I slept, long was I asleep, long are the deceits”\evb
\evg

\bpg
\bpa Sigurðr sęttisk niðr ok spyrr hana nafns. Hón tók þá horn fullt mjaðar ok gaf hǫ́num minnisvęig.\epa

\bpb Siward set himself down, asking for her name. Then she took a horn full of mead, and gave him a mind-draught:\epb
\epg


\bvg
\bva Hęill \alst{D}agr, \hld\ hęilir \alst{D}ags synir, &
\ind hęil \alst{N}ǫ́tt ok \alst{n}ipt! &
\alst{Ó}ręiðum \alst{au}gum \hld\ lítið \alst{o}kkr þinig &
\ind ok gefið \alst{s}itjǫndum \alst{s}igr!\eva

\bvb “Hail \inx[P]{Day}! Hail the sons of Day!\footnoteB{TODO. Who?} Hail Night and [her] kinswoman \ken*{= Earth}!\footnoteB{According to \Gylfaginning\ 10 Earth is the daughter of Night and \inx[P]{Aner}.} With unwrathful eyes look ye upon us two, and give the sitting ones \ken*{= us} victory.\evb
\evg


\bvg
\bva Hęilir \alst{ę́}sir, \hld\ hęilar \alst{ǫ́}synjur, &
\ind hęil sjá in \alst{f}jǫlnýta \alst{f}old! &
\alst{M}ál ok \alst{m}anvit \hld\ gefið okkr \alst{m}ę́rum tvęim &
\ind ok \alst{l}ę́knishęndr meðan \alst{l}ifum!\eva

\bvb Hail the \inx[G]{Ease}! Hail the \inx[G]{Ossens}! Hail this bountiful fold \ken{earth}! Speech and \inx[C]{manwit} give ye us renowned two, and \inx[C]{healing-hands}\footnoteB{Hands with the power to heal (perhaps supernaturally). The singular form \emph{lę́knishǫnd} occurs in the semi-Christianized prayer on a c. 1300 stick from Ribe, Denmark (signum DR EM85;493).} while we live.”\evb
\evg


BPG
BPA Hon nefndisk Sigrdrífa ok var valkyrja. Hon sagði, at tveir konvngar bǫrðusk. Hét annarr Hjalmgunnarr; hann var þá gamall ok inn mesti hermaðr, ok hafði Óðinn hánum sigri heitit.
En \alst{a}nnarr hét \alst{A}gnarr, \hld\ \alst{Au}ðu bróðir // er \alst{v}ę́tr engi \hld\ \alst{v}ildi þiggja.
Sigrdrífa felldi Hjalmgunnar í orrostunni. En Óðinn stakk hana svefnþorni í hefnd þess ok kvað hana aldri skyldu síðan sigr vega í orrostu, ok kvað hana giftask skyldu, „en sagða’k hánum at strengða’k heit þar í mót, at giptask øngom þeim manni er hrę́ðask kynni.“ Hann segir ok biðr hana kenna sér speki ef hon\footnoteA{\emph{hánom} ms.} vissi tíðendi ór ǫllum heimum. Sigrdrífa kvað:EPA

BPB She called herself Sighdrive and was a walkirrie. She said that two kings fought. One of them was called Helmguther; he was then old and the greatest harrier, and Weden had promised him victory.
But another one was called Eyner, Eade’s brother, who in no way wished to accept.\footnoteB{i.e. ‘wished to lose’ TODO}
Sighdrive felled Helmguther in the battle, but Weden pierced her with the sleeping-thorn as revenge for that, and said that she would never thenceforth win victory in battle, and said that she must marry, “but I told him that I made a vow against that, to marry no man who could be frightened.” He [= Siward] speaks and asks her to teach him wisdom, if she knew any tidings out of all the \inx[C]{Home}[Homes]. Sighdrive quoth: EPB
EPG


\bvg
\bva „Bjór fǿri’k þér, \hld\ brynþings apaldr, &
magni blandinn \hld\ ok męgintíri, &
fullr ’s hann ljóða \hld\ ok líknstafa, &
góðra galdra \hld\ ok gamanrúna.\eva

\bvb Beer I bring thee—apple-tree of the byrnie-\inx[C]{Thing} \ken{battle > warrior}!—mixed with might, and might-glory; it is full of \inx[C]{leed}[leeds], and grace-staves, of good \inx[C]{galder}[galders], and pleasure-\inx[C]{rune}[runes].\evb
\evg


\bvg
\bva Sigrúnar skalt kunna, \hld\ ef vilt sigr hafa, &
\ind ok rísta á hjalti hjǫrs, &
sumar á véttrimum, \hld\ sumar á valbǫstum, &
\ind ok nęfna tysvar Tý.\eva

\bvb Victory-runes shalt thou know, if thou wilt have victory, and carve on the hilt of the sword; some on weight-rims;\footnoteB{Unclear.} some on walbasts\footnoteB{Possibly the sword-pommel, the word also occurs in \HelgakvidaHjorvardssonar\ 9.}, and name \inx[P]{Tue} twice.\evb
\evg


\bvg
\bva Ǫlrúnar skalt kunna \hld\ ef þu vilt annars kvę́n &
\ind vęli t þik i trygd ef þú trúir. &
á horni skal þér rísta \hld\ ok á handar baki &
\ind ok merkia a nagli nꜹþ.\eva

\bvb Ale-runes shalt thou know, if TODO\evb
\evg


\bvg
\bva Full skal signa \hld\ ok við fári séa &
\ind ok verpa lauki í lǫg; &
\edtext{þá þat vęitk, \hld\ at þér verðr aldri &
męini blandinn mjǫðr.}{\lemma{þá \dots\ mjǫðr}\Bfootnote{\emph{thus} \VolsungaSaga, \emph{om.} \Regius}}\eva

\bvb TODO\evb
\evg

...


\bvg
\bva Þá mę́lti \hld\ Míms hǫfuð &
\ind fróðligt it fyrsta orð, &
\ind ok sagði sanna stafi.\eva

\bvb Then spoke the head of Mime learnedly the first word, and said true staves:\evb
\evg


\bvg
\bva Á skildi kvað ristnar \hld\ þęim’s stęndr fyr skínanda goði, &
á ęyra Árvakrs, \hld\ ok á Alsvinns hófi, &
á því hvéli es snýz \hld\ undir ręið Hrungnis, &
á Slęipnis tǫnnum \hld\ ok á slęða fjǫtrum, &
á bjarnar hrammi \hld\ ok á Braga tungu, &
á ulfs klóm \hld\ ok á arnar nęfi, &
á blóðgum vę́ngjum \hld\ ok á brúar sporði, &
á lausnar lófa \hld\ ok á líknar spori, &
á glęri ok á gulli \hld\ ok á gumna hęillum, &
í víni ok virtri \hld\ ok vilisessi. &
Á Gungnis oddi \hld\ ok á Grana brjósti, &
á nornar nagli \hld\ ok á nęfi uglu;\eva

\bvb On a shield it said were carved [runes]—[the shield] that stands before the shining god\footnoteB{According to \Grimnismal\ 39 the sun is covered by a shield, protecting the earth from its heat. Without it, the whole world would burn up.} \ken{sun}—[also] on the ear of Yorewaker, on the hoof of Allswith,\footnoteB{The two horses that pull the sun across the heavens; cf. \Grimnismal\ 38.} on that wheel which turns beneath the chariot of Rungner, on the teeth of Slopner, and on the fetters of sleds, on the paw of the bear, and on the tongue of Bray, on the claws of the wolf, and on the beak of the eagle, on bloody wings, and on the supports of the bridge, on the palm of release, and the track of grace, on glass and on gold, and on the good healths of men, in wine and beerwort, and on the comfortable seat, on the point of Gungner, and on the breast of Grane, on the nail of a norn, and on the beak of an owl.\evb
\evg


\bvg
\bva Allar vǫ́ru af skafnar, \hld\ þę́r es vǫ́ru á ristnar, &
\ind ok hvęrfðar við inn hęlga mjǫð &
\ind ok sęndar á víða vega.\eva

\bvb All were shaven off—those that were carved on—and thrown into the holy mead, and sent on wide ways:\evb
\evg


\bvg
\bva Þę́r ’ru með ǫ́sum, \hld\ þę́r ’ru með ǫlfum, &
\ind sumar með vísum vǫnum, &
\ind sumar hafa męnskir męnn.\eva

\bvb They are among the Ease, they are among the Elves; some among wise Wanes; some manly men have.\evb
\evg

...


\bvg {\small [Sighdrive quoth:]}
\bva ...\eva

\bvb “Now shalt thou choose, as the choice is offered to thee, maple-tree of sharp weapons \ken{warrior}! Speech or silence have thou in thy own heart; all the harms are measured [by the Norns].”\evb
\evg


\bvg {\small [Siwrd quoth:]}
\bva ...\eva

\bvb “I shall not flee, although thou know me to be fey; I am not born with softness.\footnoteB{Note about this common heroic expression.} Thy loving counsels all will I have, for as long as I live.”\evb
\evg


\bvg {\small [Sighdrive quoth:]}
\bva ...\eva

\bvb “That I counsel thee first: that thou against thy kinsmen defend thyself faultlessly. Late ought thou to take revenge, although they incur charges; that they say befits the dead.\evb
\evg


\bvg
\bva Þat rę́ð’k þér annat, \hld\ at ęið né svęrir, &
\ind nema þann ’s saðr séi, &
grimmar simar \hld\ ganga at tryggðrofi; &
\ind armr es vára vargr.\eva

\bvb That I counsel thee second: that thou not swear an oath, save for that one which is true. Grim strands befall the troth-breaker; wretched is the outlaw of vows.\evb
\evg


\bvg
\bva ...\eva

\bvb That I counsel thee third: that thou on the Thing bandy not with foolish men; for an unwise man often lets be spoken worse words than he ought to know.\evb
\evg


\bvg
\bva ...\eva

\bvb All is missing if thou shut up towards it; then thou seemest born with softness, or truthfully accused. Risky is the verdict of neighbours, unless one gets himself a good one.\evb
\evg


\bvg
\bva ...\eva

\bvb At another day make his breath go away, and thus repay the people for the lie.\evb
\evg
%

% TODO: Summarize contents of "Great Lacuna" with excerpts from relevant Saws

%	\include{books/Fragmented Lay of Siward.tex}%
%	\include{books/From the Death of Siward.tex}%
%	\include{books/First Lay of Guthrun.tex}%
%	\include{books/Short Lay of Siward.tex}%
%	\include{books/Hell-ride of Byrnhild.tex}%
%	\include{books/Slaying of the Niflings.tex}%
%	\include{books/Second Lay of Guthrun.tex}%
	\bookStart{The Third Lay of Guthrun}[Guðrúnarkviða þriðja]

BPG
BPA Herkja hét ambǫ́tt Atla; hón hafði verit frilla hans. Hón sagði Atla at hón hefði sét Þjóðrek ok Guðrúnu bę́ði saman. Atli var þá allókátr. Þá kvað Guðrún: EPA

BPB Hark was named the female thrall of Attle; she had been his concubine. She told Attle that she had seen Thederick and Guthrun both together. Attle was then wholly displeased. Then Guthrun quoth: EPB
EPG


\bvg
\bva “Hvat es þér, Atli? \hld ę́, Buðla sonr, &
es þér hryggt í hug; \hld hví hlę́r þú ę́va? &
Hitt myndi ǿðra \hld jǫrlum þykkja &
at við menn mę́ltir \hld ok mik sę́ir.”\eva

\bvb What is with thee, Attle? Always, son of Bodle, art thou sad at heart; why laughest thou never? TO-DO\evb
\evg


\bvg
\bva “Tregr mik þat, Guðrún, \hld Gjúka dóttir, &
mér í hǫllu \hld Hęrkja sagði &
at þit Þjóðrekr \hld undir þaki svę́fið &
ok léttliga \hld líni vęrðið.”\eva

\bvb It troubles me, Guthrun, Yivick’s daughter, which in the hall Hark has said me: that thou and Thederick beneath thatched roof slept, and ye lightly warded the linen.\footnoteB{i.e., they threw off their clothes and slept together.}\evb
\evg


\bvg
\bva “Þér mun’k alls þęss \hld ęiða vinna &
at inum hvíta \hld helga stęini. &
at ek við Þjóðmar \hld þat-ki átta’k &
es vǫrðr né verr \hld vinna knátti.\eva

\bvb GAGAGGAGAG\evb
\evg


\bvg
\bva Nema ek halsaða \hld hęrja stilli, &
jǫfur óneisinn, \hld ęinu sinni; &
aðrar vǫ́ru \hld okkrar spękjur &
es við hǫrmug tvau \hld hnigum at rúnum.\eva

\bvb TESTETET STET T\evb
\evg


\bvg
\bva Hér kom Þjóðrekr \hld með þrjá tǫgu, &
lifa þęir né ęinir, \hld þriggja tega manna; &
hrinktu mik at brǿðrum \hld ok at brynjuðum, &
hrinktu mik at ǫllum \hld á hǫfuðniðjum.\eva

\bvb TESTE TEST EST TES\evb
\evg


\bvg
\bva Sęntu at Saxa, \hld sunnmanna gram; &
hann kann hęlga \hld hver vellanda;” &
sjau hundruð manna \hld í sal gengu &
áðr kvę́n konungs \hld í kętil tǿki.\eva

\bvb Send for Saxe, the prince of southmen; he knows how to hallow a swelling cauldron!” — Seven hundred men went into the hall, before the wife of the king might touch the kettle.\evb
\evg


\bvg
\bva “Kęmr-a nú Gunnarr, \hld kalli’k-a Hǫgna,
sé’k-a síðan \hld svása brǿðr;
sverði myndi Hǫgni \hld slíks harms reka,
nú verð’k sjǫlf fyr mik \hld synja lýta.”\eva

\bvb “Now Guthhere comes not, I call not on Hain; I see not hence [my] sweet brothers. With sword would Hain drive away such an affront; now I will for myself disprove the slanders.”\evb
\evg


\bvg
\bva Brá hón til botns \hld bjǫrtum lófa &
ok hón upp of tók \hld jarknastęina: &
Sé nú sęggir \hld sykn em ek orðin &
hęilagliga— \hld hvé sjá hverr velli.\eva

\bvb Brought she the bright palms to the bottom, and she up did take the earkenstones: “See now, men—I am proven innocent, through holy means—how this cauldron boils!”\evb
\evg


\bvg
\bva Hló þá Atla \hld hugr í brjósti &
es hann hęilar sá \hld hęndr Guðrúnar: &
Nú skal Hęrkja \hld til hvers ganga, &
sú er Guðrúnu \hld grandi vę́nti. \eva

\bvb Then the heart of Attle laughed in his breast, when he saw the hands of Guthrun unscathed: “Now shall Hark go to the cauldron, she who to Guthrun hoped to cause harm.”\evb
\evg


\bvg
\bva Sá-at maðr armligt, \hld hvęrr es þat sá at, &
hvé þar á Hęrkju \hld hęndr sviðnuðu; &
lęiddu þá męy \hld í mýri fúla, &
svá þá Guðrún \hld sinna harma.\eva

\bvb Each man saw not something so pitiful, who saw that: how there on Hark the hands were scorched. Led they the maiden into the foul bog; thus was Guðrún reconstituted for her affronts.\evb
\evg
%
%	\include{books/Weeping of Oddrun.tex}%
	\bookStart{The Lay of Attle}[Atlakviða]

BPG %TODO prose formatting
Dauði Atla.

Guðrún Gjúkadóttir hefndi brǿðra sinna, svá sem frę́gt er orðit. Hon drap fyrst sonu Atla, en eptir drap hon Atla ok brendi hǫllina ok hirðina alla; um þetta er sjá kviða ort.

The Death of Attle

Guthrun Yivicksdaughter avenged her brothers, as has become famous. She first killed the sons of Attle, and after that she killed Attle, and burned the hall and the whole hird. Regarding that this lay is wrought.

\bvg
\bva Atli sęndi \hld\ ár til Gunnars &
kunnan sęgg at ríða, \hld\ Knéfrøðr vas sá hęitinn; &
at gǫrðum kom hann Gjúka \hld\ ok at Gunnars hǫllu, &
bękkjum aringręypum \hld\ ok at bjóri svǫ́sum.\eva

\bvb Attle sent early to Guther a well-known messenger to ride; Kneefred that one was called. To the estates of Yivick he came, and to the hall of Guther; to the hearth-surrounding benches, and to the lovely beer.\evb
\evg


\bvg
\bva Drukku þar dróttmęgir \hld\ —ęn \edtext{dyljęndr}{\lemma{dyljęndr ‘concealed ones’}\Bfootnote{\textcite{FinnurEdda} reasonably interprets this as referring to Attle’s spies at Guther’s court.}} þǫgðu— &
vín í \edtext{valhǫllu}{\lemma{valhǫllu ‘the walhall’}\Bfootnote{The interpretation of this compound is difficult in context. The first element \emph{val-} could be (1) \emph{valr} ‘falcon’, referring to the aristocratic hunting practice; (2) \emph{valr} ‘\inx[G]{Wales}[Wale]’, cognate with ‘Welsh’ but in ON referring to the French or Romans, stressing the southern location or appearance of the hall; or (3) \emph{valr} ‘(collective) the battle-slain’, foreshadowing the inevitable death (\inx[C]{feyness}) of the \inx[G]{Yivickings}. In this case it is linguistically identical to \inx[L]{Walhall}, Weden’s hall, whither the battle-slain go.}}, \hld\ vręiði sǫ́usk þęir Húna; &
kallaði þá Knéfrøðr \hld\ kaldri rǫddu, &
sęggr inn suðrǿni \hld\ sat hann á bękk hǫ́m:\eva

\bvb There the dright-lads drank—but the concealed ones were silent—wine in the walhall; wary were they of the wrath of the Huns. Then Kneefred, the southern man, called with cold voice; he sat on a high bench:\evb
\evg


\bvg
\bva “Atli mik hingat sęndi \hld\ ríða øręndi, &
mar inum mélgręypa, \hld\ Myrkvið inn ókunna &
at biðja yðr, Gunnarr, \hld\ at it á bękk kǿmið &
með hjǫlmum aringręypum \hld\ at sǿkja hęim Atla.\eva

\bvb “Attle me hither sent to ride an errand, with the bit-champing horse through the uncharted Mirkwood, to ask you, Guther, that ye two on the bench might come, with hearth-surrounding helmets, to seek the home of Attle.\evb
\evg


\bvg
\bva Skjǫldu kneguð þar vęlja \hld\ ok skafna aska, &
hjalma gullroðna \hld\ ok Húna męngi, &
silfrgyllt sǫðulklę́ði, \hld\ sęrki valrauða, &
dafar, darraða, \hld\ drǫsla mélgręypa.\eva

\bvb There ye might choose shields, and smooth ash-spears, helmets gold-reddened, and the multitude of the Huns, silver-gilt saddle-cloth, walred serks, dafs, standards, bit-champing steeds.\evb
\evg


\bvg
\bva Vǫll lézk ykkr ok myndu gefa \hld\ víðrar Gnitahęiðar &
af gęiri gjallanda \hld\ ok af gylltum stǫfnum, &
stórar męiðmar \hld\ ok staði Danpar, &
hrís þat it mę́ra \hld\ es meðr Myrkvið kalla.\eva

\bvb GAGAGA\evb
\evg


\bvg
\bva Hǫfði vatt þá Gunnarr \hld\ ok Hǫgna til sagði: &
Hvat rę́ðr þú okkr, sęggr inn ǿri, \hld\ allz vit slíkt hęyrum? &
Gull vissa ek ekki \hld\ á Gnitahęiði, &
þat es vit ę́ttim-a \hld\ annat slíkt.\eva

\bvb His head turned Guther then, and to Hain said: “What counselest thou we two do, younger man, as we such things hear? I knew of no gold on the Gnitheath, that we did not own as much of.\evb
\evg


\bvg
\bva Sjau ęigu vit salhús \hld\ sverða full, &
hvęrju eru þęira \hld\ hjǫlt ór gulli; &
mínn vęit ek mar bęztan \hld\ ęn mę́ki hvassastan, &
boga bękksǿma \hld\ ęn brynjur ór gulli.\eva

\bvb We own seven hallhouses, filled with swords—on each of them is a golden hilt; I know my horse to be the best, and my sword the sharpest; my bow bench-fit, and my byrnies of gold.\evb
\evg


\bvg
\bva Hjalm ok skjǫld hvítastan, \hld\ kominn ór hǫll Kjárs; &
ęinn es mínn bętri \hld\ ęn sé allra Húna.\eva

\bvb A helmet and the whitest shield, taken out of the hall of Chear; alone is mine better, than that of all of the Huns.”\evb
\evg


\bvg
\bva Hvat hyggr þú brúði bęndu \hld\ þá es hón okkr baug sęndi, &
varinn váðum hęiðingja? \hld\ Hykk at hón vǫrnuð byði! &
Hár fann ek hęiðingja \hld\ riðit í hring rauðum; &
ylfskr es vegr okkarr \hld\ at ríða øręndi.\eva

\bvb “What does thou think the bride meant, when she us two an armlet sent, wrapped with the cloth of a heath-dweller \ken{wolf}? I think that she bid us a warning! I found the hair of a heath-dweller wrapped round the red ring; wolven is our way, to ride that errand.”\evb
\evg


\bvg
\bva Niðjar-gi hvǫttu Gunnar \hld\ né náungr annarr, &
rýnęndr né ráðęndr, \hld\ né þęir es ríkir vǫ́ru; &
kvaddi þá Gunnarr \hld\ sęm konungr skyldi, &
mę́rr í mjǫðranni \hld\ af móði stórum:\eva

\bvb No kinsmen urged Guther, nor any other close one, nor counselors nor advisors, nor those who mighty were. Guther then announced—as a king should, renowned in the mead-house—out of great courage:\evb
\evg


\bvg
\bva Rís-tu nú, Fjǫrnir, \hld\ lát-tu á flęt vaða &
gręppa gullskálir \hld\ með gumna hǫndum!\eva

\bvb “Rise now, Ferner; let on the floorboards wade forth the golden bowls of warriors, along the hands of men!\evb
\evg


\bvg
\bva Ulfr mun ráða \hld\ arfi Niflunga, &
gamlir granvarðir, \hld\ ef Gunnars missir, &
birnir blakkfjallir \hld\ bíta þreftǫnnum, &
gamna gręystóði, \hld\ ef Gunnarr né kømr-at.\eva

\bvb The wolf will rule the inheritance of the Niflings: the old grey guardians, if Guther is missing. Bears black-furred bite with wrangling teeth, amusing the pack of bitches, if Guther comes not.”\evb
\evg


\bvg
\bva Lęiddu landrǫgni \hld\ lýðar ónęisir, &
grátęndr, gunnhvatan, \hld\ ór garði Húna; &
þá kvað þat inn ǿri \hld\ ęrfivǫrðr Hǫgna: &
Hęilir farið nú ok horskir \hld\ hvar’s ykkr hugr tęygir!\eva

\bvb GAGAGA\evb
\evg


\bvg
\bva Fetum létu frǿknir \hld\ um fjǫll at þyrja &
marina mélgręypu, \hld\ Myrkvið inn ókunna; &
hristisk ǫll Húnmǫrk \hld\ þar es harðmóðgir fóru, &
vrǫ́ku þęir vannstyggva \hld\ vǫllu algrǿna.\eva

\bvb GAGAGA\evb
\evg


\bvg
\bva Land sǫ́u þęir Atla \hld\ ok liðskjalfar djúpar &
Bikka greppar standa \hld\ á borg inni há &
sal of suðrþjóðum, \hld\ slęginn sessmęiðum, &
bundnum rǫndum, \hld\ blęikum skjǫldum,\eva

\bvb The land of Attle saw they, TODO\evb
\evg


\bvg
\bva dafar, darraða; \hld\ ęn þar drakk Atli &
vín í valhǫllu; \hld\ vęrðir sǫ́tu úti &
at varða þęim Gunnari \hld\ ef þęir hér vitja kǿmi &
með gęiri gjallanda \hld\ at vękja gram hildi.\eva

\bvb but there drank Attle wine in the wale-hall\footnoteB{TODO: this is not Weden’s hall, rather ‘the Roman hall’.} ... \evb
\evg


\bvg
\bva Systir fann þęira snemmst \hld\ at þęir í sal kvǫ́mu, &
brǿðr hęnnar báðir, \hld\ bjóri var hón lítt drukkin: &
Ráðinn ert-u nú, Gunnarr, \hld\ hvat munt-u, ríkr, vinna &
við Húna harmbrǫgðum? \hld\ Hǫll gakk þú ór snemma!\eva

\bvb Their sister found earliest they they had come into the hall, both of her brothers—on beer was she lightly drunk—“Betrayed art thou now, Guther; why wilt thou, mighty one, struggle against Hunnish harm-tricks? Go early out of the hall!\footnoteB{Before anything evil might happen.}”\evb
\evg


\bvg
\bva Bętr hęfðir þú, bróðir, \hld\ at þú í brynju fǿrir, &
sęm hjǫlmum aringręypum \hld\ at séa hęim Atla; &
sę́tir þú í sǫðlum \hld\ sólhęiða daga, &
nái nauðfǫlva \hld\ létir nornir gráta.\eva

\bvb Better hadst thou, brother, if thou in byrnie travelled, and with hearth-surrounding helmets, to see the home of Attle.\evb
\evg


\bvg
\bva Húna skjaldmęyjar \hld\ hęrfi kanna &
ęn Atla sjalfan \hld\ létir þú í ormgarð koma; &
nú es sá ormgarðr \hld\ ykkr of folginn.\eva

\bvb GAGAGA\evb
\evg


\bvg
\bva Sęinað es nú, systir, \hld\ at samna Niflungum, &
langt es at lęita \hld\ lýða sinnis til, &
of rosmufjǫll Rínar, \hld\ rekka ónęissa.\eva

\bvb GAGAGA\evb
\evg


\bvg
\bva Fengu þęir Gunnar \hld\ ok í fjǫtur sęttu, &
vinir Borgunda, \hld\ ok bundu fastla; &
sjau hjó Hǫgni \hld\ sverði hvǫssu &
ęn inum átta hratt hann \hld\ í ęld hęitan.\eva

\bvb Caught they Guther, and in fetters set him—the friends of the Burgends—and bound them tightly. Seven Hain hewed down with sharp sword, and the eighth one threw he into the hot fire.\evb
\evg


\bvg
\bva \edtext{Svá skal frǿkn \hld\ fjándum vęrjask;}{\lemma{Svá ... vęrjask}\Bfootnote{Line moved from the last verse to this one since it seems to connect semantically with the immediately following line, and also creates a regular line distribution of 4-4 instead of 5-3.}} &
Hǫgni varði \hld\ hęndr Gunnars. &
frǫ́gu frǿknan \hld\ ef fjǫr vildi &
Gotna þjóðann \hld\ gulli kaupa.\eva

\bvb Thus shall the bold against fiends ward himself; Hain warded the hands of Guther. They asked the bold one if to buy he wished—the ruler of the Gots—his life with gold.\footnoteB{The Huns ask Guther (it is clear that “ruler of the Gots” refers to him, cf. 1, 3, 10) if he wishes to ransom Hain. He instead responds with the following:}\evb
\evg


\bvg {\small [Guther quoth:]}
\bva “Hjarta skal mér Hǫgna \hld\ í hęndi liggja &
blóðugt, ór brjósti \hld\ skorit baldriða, &
saxi slíðrbęitu, \hld\ syni þjóðans.”\eva

\bvb “The heart of Hain shall lie me in the hands: bloody from the breast—cut from the bold rider with a slide-biting sax\footnoteB{i.e. a short-sword with a blade so sharp that it draws blood when one slides the finger across it.}—of the son of the sovereign.”\evb
\evg


\bvg
\bva Skǫ́ru þęir hjarta \hld\ Hjalla ór brjósti &
blóðugt ok á bjóð lǫgðu \hld\ ok bǫ́ru þat fyr Gunnar.\eva

\bvb They cut the heart of Helle out of the breast; bloody on a platter they laid it, and carried it before Guther.\evb
\evg


\bvg
\bva Þá kvað þat Gunnarr, \hld\ gumna dróttinn: &
Hér hęfi ek hjarta \hld\ Hjalla ins blauða, &
ólíkt hjarta \hld\ Hǫgna ins frǿkna, &
es mjǫk bifask \hld\ es á bjóði liggr; &
bifðisk hǫlfu męirr \hld\ es í brjósti lá!\eva

\bvb Then quoth that Guther, the lord of men: “Here have I the heart of Helle the soft—unlike the heart of Hain the bold!—which much trembles, when on the platter it lies; it trembled twice as much, when in the breast it lay.”\evb
\evg


\bvg
\bva Hló þá Hǫgni \hld\ es til hjarta skǫ́ru &
kvikvan kumblasmið \hld\ kløkkva hann sízt hugði; &
blóðugt þat á bjóð lǫgðu \hld\ ok bǫ́ru fyr Gunnar.\eva

\bvb Hain laughed then, when to the heart they cut on the living wound-smith \ken{warrior}; he thought least of sobbing. Bloody on a platter they laid it, and carried it before Guther.\evb
\evg


\bvg
\bva Mę́rr kvað þat Gunnarr, \hld\ Gęir-Niflungr: &
Hér hęfi ek hjarta \hld\ Hǫgna ins frǿkna, &
ólíkt hjarta \hld\ Hjalla ins blauða, &
es lítt bifask \hld\ es á bjóði liggr; &
bifðisk svági mjǫk \hld\ þá’s í brjósti lá!\eva

\bvb Renowned quoth that Guther, the Gore-Nifling: “Here have I the heart of Hain the bold—unlike the heart of Helle the soft!—which little trembles, when on the platter it lies; it trembled not as much, when in the breast it lay.\evb
\evg


\bvg
\bva Svá skaltu, Atli, \hld\ augum fjarri &
sęm munt \hld\ męnjum verða; &
es und ęinum mér \hld\ ǫll of folgin &
hodd Niflunga: \hld\ Lifir-a nú Hǫgni!\eva

\bvb Thus shalt thou, Attle, be as far from the eyes, as thou wilt from the neck-rings. ’Tis by me alone all concealed, the hoard of the Niflings—now Hain lives not!\evb
\evg


\bvg
\bva Ęy vas mér týja \hld\ meðan vit tvęir lifðum, &
nú es mér ęngi \hld\ es ęinn lifi’k; &
Rín skal ráða \hld\ rógmalmi skatna, &
svinn, ǫ́skunna \hld\ arfi Niflunga.\eva

\bvb I was ever in doubt when we two lived; now I am not when alone I live. The Rhine shall rule the strife-ore of princes \ken{gold}, swift, the os-born inheritance of the Niflings.\evb
\evg


\bvg
\bva Í veltanda vatni \hld\ lýsask valbaugar &
hęldr an á hǫndum gull \hld\ skíni Húna bǫrnum.\eva

\bvb In tumbling water the Welsh bighs gleam, rather than gold might shine on the hands of the children of Huns.”\evb
\evg

...

\bvg
\bva Ęldi gaf hón alla \hld\ es inni vǫ́ru &
ok frá morði þęira Gunnars \hld\ komnir vǫ́ru ór Myrkhęimi; &
forn timbr fellu, \hld\ fjarghús ruku, &
bǿr Buðlunga, \hld\ brunnu ok skjaldmęyjar, &
inni aldrstamar, \hld\ hnigu í ęld hęitan.\eva

\bvb To the fire she gave all those who were inside, who from their murder of Guther were come out of Mirkham. Ancient timbers fell, great houses smoked—the settlement of the Buthlungs—burned the shield–maidens likewise; inside aged trunks bowed into hot fire.\evb
\evg


\bvg
\bva Fullrǿtt’s umb þetta; \hld\ fęrr ęngi svá síðan &
brúðr í brynju \hld\ brǿðra at hęfna; &
hón hęfir þriggja \hld\ þjóðkonunga &
banorð borið, \hld\ bjǫrt, áðr sylti.\eva

\bvb ’Tis fully told of this; none hence fares so, a bride in byrnie, her brothers to avenge. She has of three great kings borne the bane-word,\footnoteB{i.e. ‘She has slain three great kings.’ This expression and its Germanic and Indo-European relatives is discussed in detail in \textcite{Watkins1995}[417--422].} bright woman, before she may die.\evb
\evg


\bvg
\bva Enn segir gleggra í Atlamálum inum grǿnlenskum.\eva

\bvb Yet this is told more clearly in the Greenlendish Speeches of Attle.\evb
\evg
%
%	\include{books/Greenlendish Speeches of Attle.tex}%
%	\include{books/Instigation of Guthrun.tex}%
%	\include{books/Speeches of Hamthew.tex}%

% Additional heroic poems
	\bookStart{The Lay of Hildbrand}

% Introduction

For the text of original poem I generally present the manuscript text. I have found it impossible to produce a normalization without too heavily distorting the received text, being as it is, a blend of several dialects. I have, however, added acute accents to signify long vowels, capitalized proper names, consistently replaced \emph{ƿ} (wynn) and \emph{uu} with \emph{w}, and made minor corrections where the manuscript is clearly in error—these are noted in the critical apparatus. The punctuation of the original, entirely consisting of interpuncts, at times representing line breaks and cæsuræ and at others sporadically placed, has not been retained.

Where they appear in cæsuræ, the words \emph{quad Hiltibrant} ‘Hildbrand quoth’ (found in ll., 30, 49, and 58) replace the usual interpunct. I had originally planned to remove these as hypermetrical, instead indicating the speaker above the verse, but after comparison with \Reginsmal\ 3, wherein the words \emph{kvað Loki} ‘Lock quoth’ appear in the first cæsura of the verse, I have come to believe that these represent an ancient oral indication, seemingly going back as far as the Migration Period (as it seems incredulous to think that the scribe of \HildMS\ would have influenced the scribe of \Regius\ four centuries later in such a minor point.)


\bvg
\bva[0]Ik gihórta dat seggen &
dat sih \alst{u}rhettun \hld\ aenon muotín &
\alst{H}iltibrant enti \alst{H}adubrant \hld\ untar \alst{h}eriun twém &
\alst{s}unufatarungo \hld\ iro \alst{s}aro rihtun &
\alst{g}arutun se iro \alst{g}údhamun \hld\ \alst{g}urtun sih iro swert ana &
\alst{h}elidos ubar \edtext{\alst{h}ringa}{\lemma{hringa}\Afootnote{ringa \HildMS}} \hld\ dó sie to dero \alst{h}iltiu ritun\eva

\bvb[0]I heard it said, that two contenders alone did meet: Hildbrand and Hathbrand, under two hosts.\footnoteB{i.e. each man was a champion of his respective army.} Son and father ordered their armour, readied their war-cloth, girded their swords on, the heroes over the mail, when to that battle they rode.\evb
\evg


\bvg\setlinenum{6}
\bva[0]\alst{H}iltibrant \edtext{gimahalta}{\Afootnote{\emph{add.} heribrantes sunu “Harbrand’s son” \HildMS}} \hld\ her was \alst{h}éróro man &
\alst{f}erahes \alst{f}rótóro \hld\ her \alst{f}rágén gistuont &
\alst{f}óhém wortum \hld\ \edtext{hwer}{\Afootnote{wer \HildMS}} sín \alst{f}ater wári &
\alst{f}ireo in \alst{f}olche \hld\ {[...]} &
{[...]} \hld\ „eddo \edtext{hwelíhhes}{\Afootnote{welihhes \HildMS}} \alst{c}nuosles dú sís &
ibu dú mí \alst{é}nan sagés \hld\ ik mí de \alst{o}dre wét &
\alst{ch}ind in \edtext{\alst{ch}unincríche}{\lemma{chunincríche}\Afootnote{chunnincriche \HildMS}} \hld\ \alst{ch}úd ist mín al irmindeot“\eva

\bvb[0]Hildbrand spoke—he was the hoarier man, more learned in life—he began to ask, with few words, who his father might be, of men in the troop, [...] “or of which lineage thou be; if thou me one say, I the others will know; child, in the kingdom, known to me are all great men.”\evb
\evg


\bvg\setlinenum{13}
\bva[0]Hadubrant gimahalta \hld\ Hiltibrantes sunu &
\edtext{„dat sagetun mí \hld\ úsere liuti}{\lemma{dat ... liuti}\Bfootnote{this l. breaks no rhythmic rules (cf. l. 42), but the needed alliteration is missing.}} &
\alst{a}lte anti fróte \hld\ dea \alst{é}rhina wárun &
dat \alst{H}iltibrant haetti mín fater \hld\ ih heittu \alst{H}adubrant &
forn her \alst{ó}star \edtext{giweit}{\Afootnote{gihueit \HildMS}} \hld\ flóh her \alst{Ó}tachres níd &
hina miti \alst{Th}eotríhhe \hld\ enti sínero \alst{d}egano filu &
her fur\alst{l}aet in \alst{l}ante \hld\ \alst{l}uttila sitten &
\edtext{\alst{b}rút}{\lemma{brút}\Afootnote{prut \HildMS}} in \alst{b}úre \hld\ \alst{b}arn unwahsan &
\alst{a}rbeolaosa \hld\ \edtext{her raet}{\Afootnote{heraet \HildMS}} \alst{ó}star hina &
det síd \alst{D}etríhhe \hld\ \alst{d}arba gistuontum &
\edtext{\alst{f}ateres}{\lemma{fateres}\Afootnote{fatereres \HildMS}} mínes \hld\ dat was só \alst{f}riuntlaos man &
her was \alst{Ó}tachre \hld\ \alst{u}mmet tirri &
\alst{d}egano \alst{d}echisto \hld\ unti \edtext{\alst{D}eotríchhe}{\lemma{Deotríchhe}\Afootnote{\emph{add.} darba gistontun \HildMS}} &
her was eo \alst{f}olches at ente \hld\ imo was eo \edtext{\alst{f}ehta}{\lemma{fehta}\Afootnote{peheta \HildMS}} ti leop &
\alst{ch}úd was her \hld\ \edtext{\alst{ch}óném}{\lemma{chóném}\Afootnote{chonnem \HildMS}} mannum &
ni wániu ih iu líb habbe“\eva

\bvb[0]Hathbrand spoke, Hildbrand’s son: “It told me our people, the old and learned, those who earlier lived, that Hildbrand was called my father — I am called Hathbrand. Long ago he hurried east — he fled Edwaker’s hate — thither with Thedrich, and his great many thanes. He left in the land a little one to stay, a bride in the bower, a bairn ungrown, without inheritance; he rode east thither, as Thedrich was in great need of my father; — that was so friendless a man. He was to Edwaker exceptionally hostile, the dearest of thanes under Thedrich. He was ever at the front of the troop, ever did the fight gladden him, known was he among keen men; I ween not that he have life.”\evb
\evg


\bvg\setlinenum{29}
\bva[0]„wettu \alst{i}rmingot {\small (quad Hiltibrant)} \alst{o}bana ab \edtext{hebane}{\Afootnote{heuane \HildMS}} &
dat dú neo dana halt mit sus sippan man &
dinc ni gileitós“ &
\alst{w}ant her dó ar arme \hld\ \alst{w}untane bauga &
\alst{ch}eisuringu gitán \hld\ so imo sie der \alst{ch}uning gap &
\alst{h}uneo truhtin \hld\ „dat ih dir it nú bí \alst{h}uldí gibu“\eva

\bvb[0]“I call on Ermin-god as witness, above in heaven, that thou never with such a close man once more lead dispute.” Unwound he then from his arm some twisted bighs\footnoteA{Armlets used as currency during the Migration Period; ON \emph{baugr}, OE \emph{béag}. — The giving of rings and armlets in exchange for loyalty was common across all of Germanic Europe, as seen in the many ruler-kennings of the type “breaker of rings” (like \emph{béaga brytta} “the breaker of bighs” \Beowulf\ ll. 35, 352, 1487.) This is also connected with the oath-ring, and the famous ring-swords. TODO? reference some literature on this.}, made from imperial coin, which the king once gave him, the lord of the Huns—“This I now give thee as pledge.”\evb
\evg


\bvg\setlinenum{35}
\bva[0]\alst{H}adubrant gimahalta \hld\ \alst{H}iltibrantes sunu &
„mit \alst{g}éru scal man \hld\ \alst{g}eba infáhan &
\alst{o}rt widar \alst{o}rte \hld\ [...] &
dú bist dir \alst{a}ltér hun \hld\ \alst{u}mmet spáhér &
\alst{sp}enis mih mit díném wortun \hld\ wili mih dínu \alst{sp}eru werpan &
\edtext{bist}{\Afootnote{pist \HildMS}} alsó gialtét man \hld\ só dú éwín inwit fórtós &
dat \alst{s}agetun mí \hld\ \alst{s}éolídante &
\alst{w}estar ubar \alst{W}entilséo \hld\ dat man \alst{w}íc furnam &
tót ist \alst{H}iltibrant \hld\ \alst{H}eribrantes suno“\eva

\bvb[0]Hathbrand spoke, Hildbrand’s son: “With spear shall one earn gifts, point against point! Thou art, old Hun, exceptionally clever; thou lurest me with thy words, wilt thou at me thy spear hurl! Thou art thus old, though thou ever deceit didst work. — It told me seafarers, heading west o’er the Wendle-sea\footnoteA{The Mediterranean, referring to the Vandals in North Africa.}, that war took that man: — dead is Hildbrand, Harbrand’s son!”\evb
\evg


\bvg\setlinenum{44}
\bva[0]\alst{H}iltibrant gimahalta \hld\ \alst{H}eribrantes suno &
„wela gisihu ih \hld\ in díném hrustim &
dat dú \alst{h}abés \alst{h}éme \hld\ \alst{h}érron góten &
dat dú noh bí desemo \alst{r}íche \hld\ \alst{r}eccheo ni wurti“\eva

\bvb[0]Hildbrand spoke, Harbrand’s son: “I see well on thy equipment, that thou hast a good lord at home, that thou still in this reign didst not become an exile.”\evb
\evg


\bvg\setlinenum{48}
\bva[0]„\alst{w}elaga nú \alst{w}altant got {\small (quad Hiltibrant)} \alst{w}éwurt skihit &
ih wallóta \alst{s}umaro enti wintro \hld\ \alst{s}ehstic ur lante &
dar man mih eo \alst{sc}erita \hld\ in folc \alst{sc}eotantero &
só man mir at \alst{b}urc énigeru \hld\ \alst{b}anun ni gifasta &
nú scal mih \alst{s}wásat chind \hld\ \alst{s}wertu hauwan &
\alst{b}retón mit sínu \alst{b}illiu \hld\ eddo ih imo ti \alst{b}anin werdan &
doh maht dú nú \alst{ao}dlíhho \hld\ ibu dir dín \alst{e}llen taoc &
in sus \alst{h}éremo man \hld\ \alst{h}rusti giwinnan &
\alst{r}auba \edtext{bi\alst{r}ahanen}{\lemma{birahanen}\Afootnote{bihrahanen \HildMS}} \hld\ ibu dú dar éníg \alst{r}eht habés“\eva

\bvb[0]“Well now, wielding God! woeful Weird\footnoteA{The personification of fate, in this case most likely just a noun. OE \emph{Wyrd} (\Beowulf\ 455: \emph{Gǽð á Wyrd swá hío scel} “Ever goes Weird as she must”), ON \emph{Urðr} ‘one of the norns’.} comes to pass. I wallowed for summers and winters sixty out of the land, where one ever set me in the troop of shooters; thus one at no fortress my bane did inflict. Now shall my own child hew at me with sword; beat down with his blade, or I his bane become. Yet canst thou now easily—if thy zeal avail thee—from such a hoary man win the equipment; bear away the booty, if thou thereto have any right.”\evb
\evg


\bvg\setlinenum{57}
\bva[0]„der sí doh nú \alst{a}rgósto {\small (quad Hiltibrant)} \alst{ó}starliuto &
der dir nú \alst{w}íges \alst{w}arne \hld\ nú dih es só \alst{w}el lustit &
gúdea gi\alst{m}einun \hld\ niuse de \alst{m}ótti &
\edtext{hwedar}{\Afootnote{werdar \HildMS}} sih \edtext{\alst{h}iutu déro}{\lemma{hiutu déro}\Afootnote{\emph{metr. emend.}; dero hiutu \HildMS}} \alst{h}regilo \hld\ \edtext{\alst{h}ruomen}{\lemma{hruomen}\Afootnote{hrumen \HildMS}} muotti &
\edtext{eddo}{\Afootnote{erdo \HildMS}} desero \alst{b}runnóno \hld\ \alst{b}édero waltan“\eva

\bvb[0]“He be now the weakest of the eastern peoples, who refuse thee the fight, when thou so greatly cravest to struggle together; — try he who might, which of us today of these garments may boast, or both of these byrnies wield!”\evb
\evg


\bvg\setlinenum{62}
\bva[0]dó lettun se \alst{ae}rist \hld\ \alst{a}sckim scrítan &
\alst{sc}arpén \alst{sc}úrim \hld\ dat in dem \alst{sc}iltim stónt &
dó \alst{st}óptun tosamane \hld\ \alst{st}aimbort \edtext{hlúdun}{\Afootnote{chludun \HildMS}} &
\alst{h}ewun harmlícco \hld\ \alst{h}wítte scilti &
unti imo iro \alst{l}intún \hld\ \alst{l}uttilo wurtun &
gi\alst{w}igan miti \alst{w}ábnum \hld\ [...]\eva

\bvb[0]Then let they first their ash-spears glide, in harsh torrents, that in the shields they stuck. Then charged they into each other—the war-boards \ken{shields} resounded—struck they bitterly the white shields, until for them their lindens \ken{shields} became little, worn down by the weapons, [...]\evb
\evg
%
%	\include{books/Fight at Finnsbury.tex}
%	\include{books/Beewolf.tex}% Probably not happening.

% Giga-index at the end
	\part{Encyclopedia (INCOMPLETE!)}

NOTE: This encyclopedia is both incomplete and inconsistently formatted. New entries will be added, and old ones be corrected and expanded in the future.

\section{Cultural and religious expressions (C)}
\begin{itemize}

\inxitem[C]{ape} (ON \emph{api}, OE \emph{apa}, OS \emph{apo}, OHG \emph{affo}, PNWGmc. \emph{*apó})
  In the Old Norse the word seems to mean ‘fool, buffoon’, in the other old languages apparently ‘monkey’, though this sense should be a later development of the former; why would the early Germanic tribes have a word for an animal that they had never encountered?

\inxitem[C]{aught} (ON \emph{ę́tt}, OE \emph{ǽht} ‘possession, property’)
  The Nordic (paternal) clan or family line.

\inxitem[C]{begale} (OHG \emph{bi-galan})
  To affect, bewitch something using \inx[C]{galder}[galders]. See also \inx[C]{gale}.

\inxitem[C]{bigh} (ON \emph{baugr}, OE \emph{béag}, OHG \emph{boug})
  Armlets used as currency during the Migration Period. — The giving of rings and armlets in exchange for loyalty (\inx[C]{holdness} being the word used for a warrior’s loyalty towards his lord, and of a lord’s grace towards his servants) was common across all of Germanic Europe, as seen in the many poetic ruler-kennings of the type “breaker of rings” (e.g. \emph{béaga brytta} ‘the breaker of bighs’ in \Beowulf\ ll. 35, 352, 1487). An illustrative example of this is \Hildebrandslied\ 33–35.
  This is also connected with the oath-ring, and the famous ring-swords. TODO? reference some literature on this.

\inxitem[C]{bloot} (ON \emph{blót}, OE \emph{blót}, OHG \emph{bluoz})
  A sacrifice or a sacrificial feast, one of the best attested Germanic pagan practices. The animals would be sacrificed by the host, cooked in large kettles and eaten communally.

\inxitem[C]{bloot-kettle}
  The large pots used for cooking the bloot-stew.

\inxitem[C]{Doom} (ON \emph{dómr}, OE \emph{dóm})
  Commonly ‘judgement, verdict’ (whence Doomsday, ‘Judgement Day’), in the Norse and Anglo-Saxon poetry often specifically referring to one’s fame or good reputation (that is, how others will judge one’s character and deeds), especially after death. It is clear that this verdict was of utmost importance to the ancient Germanic people. The clearest examples are \Havamal\ 77 (see there): \emph{I know one that never dies: the \textbf{Doom} o’er each man dead.} and \Beowulf\ 1384-1389, where Beewolf consols king Rothgar after Grendle’s mother has slain his trusted advisor Asher (\emph{Æschere}):
  \bvg\bva[] \emph{Ne sorga, snotor guma! \hld\ Sélre bið ǽghwǽm, //
  þæt hé his fréond wrece, \hld\ þonne hé fela murne. //
  Úre ǽghwylc sceal \hld\ ende gebídan //
  worolde lífes; \hld\ wyrce sé þe móte //
  \textbf{dómes} ǽr déaþe; \hld\ þæt bið drihtguman //
  unlifgendum \hld\ æfter sélest.}\eva
  \bvb ‘Sorrow not, wise man! ’Tis better for each one that he avenge his friend, than that he mourn much. Each one of us shall suffer the end of worldly life—win he who might \textbf{Doom} before death: that is for the warrior, unliving, afterwards the best.’\evb\evg
  Other illustrative examples in \Beowulf\ include 884b–887a: \emph{[...] Sigemunde gesprong // æfter déaðdæge \hld\ \textbf{dóm} unlýtel // syþðan wíges heard \hld\ wyrm ácwealde // hordes hyrde [...]} ‘For \inx[P]{Syemund} sprang up after his day of death an unlittle \ken*{= great} \textbf{Doom}, since hard in conflict he defeated the \inx[C]{Wyrm}, the herder of the hoard.’
  and 953b–955a: \emph{[...] þú þé self hafast // dę́dum gefremed \hld\ þæt þín \textbf{dóm} lyfað // áwa tó aldre [...]} ‘Thou hast for thyself by deeds accomplished that thy \textbf{Doom} lives for ever and ever.’

\inxitem[C]{fee} (ON \emph{fé}, OE \emph{féoh})
  Originally ‘cattle’, however also used in a broader sense to refer to one’s mobile wealth. For this cf. particularly \Havamal\ TODO.

\inxitem[C]{many-cunning} (ON \emph{fjǫl-kunnigr})
  Literally ‘much-cunning, cunning in many ways’. Skilled with sorcery.

\inxitem[C]{fey} (ON \emph{fęigr}, OE \emph{fǽge}, OHG \emph{feigi} ‘cowardly’)
  Being doomed or fated to die, with a sense of predestination and inevitability. Its earliest use is on the Rök stone: \textbf{aft uamuþ stąnta runaʀ þaʀ ᛭ n uarin faþi faþiʀ aft} faikiąn \textbf{sunu} \emph{Apt Vámóð standa rúnaʀ þáʀ, en Varinn fáði, faðir aft \textbf{fęigjan} sonu} ‘After Woemood (\emph{Vámóðr}) stand these \inx[C]{rune}[runes], but Warren (\emph{Varinn}) painted, the father after the \textbf{fey} son.’ It was believed that one’s TODO. See \textciteshorttitle{PCRN-HS} II:35, p. 928 ff. (TODO)

\inxitem[C]{feyness} (ON \emph{fęigð})
  The state of being \inx[C]{fey}.

\inxitem[C]{fimble-} (ON \emph{fimbul-})
  The ultimate, final, greatest. See \inx[P]{Fimblethyle}, \inx[L]{Fimble-winter}.

\inxitem[C]{five days} (ON \emph{fimm dagar})
  That the old Scandinavian week was \textbf{five days} long is well attested. According to the \Gulatingslog\ there were six weeks in a month, and the expression \textbf{five days} is used as the equivalent of \emph{week} in \Havamal\ 51 and 74, in the second of which it is contrasted with \emph{month}. Related to this is the legal term \emph{fifth} (ON \emph{fimmt}, OSw. \emph{fæmt}), a meeting or gathering set to be held at a five-day notice. See \emph{fimt} in \CV, \textcite{LMNL} for further discussion.

\inxitem[C]{galder} (ON \emph{galdr}, OE \emph{gealdor}, OHG \emph{galdar})
  A magical spell or song. See the Merseburg charms (TODO?) for examples. See also \inx[C]{gale}.

\inxitem[C]{gale} (ON \emph{gala}, OE \emph{galan}, OHG \emph{galan})
  To sing \inx[C]{galder}[galders].

\inxitem[C]{gand} (ON \emph{gandr}, Latin \emph{gandus})
  A witch’s familiar, a spirit sent out to do her bidding. See \textciteshorttitle{PCRN-HS} I:17, p. 361 and II:26, p. 656. TODO

\inxitem[C]{gid} (ON \emph{goði}, OE \emph{Gydda} masc. nom. prop.)
  A heathen priest or master of ceremonies.

\inxitem[C]{gidden} (ON \emph{gyðja}, OE \emph{gyden} ‘goddess’)
  The feminine equivalent of \inx[C]{gid}.

\inxitem[C]{yin-} (ON \emph{ginn-})
  A rare augmentative prefix. TODO.

\inxitem[C]{yin-holy} (ON \emph{ginn-hęilagr})
  High holy, sacrosanct. Used of the gods in the formula \emph{ginn-hęilǫg goð}.

\inxitem[C]{good of meat} (ON \emph{matar góðr})
   An old expression, appearing not just in \Havamal\ 39 (“I found not a generous man, or so \textbf{good of meat}, that a gift were not accepted;”) but also several Viking Age Runic inscriptions, such as Sm 39: \emph{mildan orða \hld\ ok mataʀ góðan} ‘mild of words and \textbf{good of meat}’, U 805: \emph{bónda góðan matar} ‘a farmer \textbf{good of meat}’, U 703: \emph{mandr matar góðr \hld\ auk máls risinn} ‘a man \textbf{good of meat} and proud in speech™; compare also U 739: \emph{hann vaʀ mildr mataʀ \hld\ auk máls risinn} ‘he was \textbf{mild of meat} and proud in speech’. — See \inx[C]{meat-nithing} for its opposite.

\inxitem[C]{hame} (ON \emph{hamr})
  A skin, shape. Individuals can through magic “shift hames” (ON \emph{skipta hǫmum}), and leave their human \emph{hames} behind, instead entering into the shapes of wolves, bears, birds. During this process the original hame would be sleeping in a vulnerable state, as described in the Saw of the Walsings, chap. TODO: . See also \inx[P]{feather-hame}, \inx[C]{town-riders}, \inx[C]{evening-riders}.

\inxitem[C]{harrow} (ON \emph{hǫrgr}, OE \emph{hearg}, PNWGmc. \emph{*harugaʀ})
  A cairn constructed for ritual purposes. \Hyndluljod\ 10 describes one: “A \inx[C]{harrow} he made for me, loaded with stones; now that stone-pile is become into glass. He reddened [it] in fresh blood of oxen; \inx[P]{Oughter} ever trusted on the \inx[G]{Ossens}.” See also \inx[C]{wigh}.

\inxitem[C]{hold} (ON \emph{hollr}, OE \emph{hold}, OS \emph{hold}, OHG \emph{hold})
  %TODO Mention: unhold wights, Old Saxon baptismal formula.
  ‘Favourable, loyal, gracious’, often of a ruler towards his subject (in the sense of ‘gracious, benevolent’) or the reverse (in the sense of ‘loyal, devoted’). Mirroring these earthly relations, it is likewise often used to refer to divine grace, both of the Christian God—thus in the \emph{Ecclesiastical Laws of King Cnut} \textciteshorttitle[372]{ALIE1}: \emph{Ðam byþ witodlíce God hold þe bið his hláforde rihtlíce hold} ‘Indeed God is \textbf{hold} towards him who is rightly \textbf{hold} towards his lord’—but in the oldest Scandinavian material likewise of the Heathen gods.
  Thus \Lokasenna\ 4: \emph{holl ręgin} ‘\textbf{hold} \inx[G]{Reins}’, and \Oddrunargratr\ 10 (TODO: Numbering is very uncertain): \\ \emph{Svá hjalpi þér \hld\ hollar véttir, \\ Frigg ok Fręyja \hld\ ok flęiri goð} \\ ‘So help thee \textbf{hold} \inx[C]{wights}; \inx[P]{Frie} and \inx[P]{Frow}, and more gods [...]’.

  The word is also used in this way several medieval oath-formulæ, for instance in the Elder West-Geatish Law: \emph{Svá sé mér goð holl} ‘So may the gods(!) be \textbf{hold} towards me,’ in medieval Norwegian laws (\textciteshorttitle{NGL2}[197,397]) and Grey-Goose (TODO: cite): \emph{Guð sé mér hollr ef ek satt segi, gramr ef ek lýg} ‘God be \textbf{hold} towards me if I speak truly, wroth if I lie,’ in Grey-Goose (TODO) also: \emph{Sé guð hollr þeim er heldr griðum, en gramr þeim er grið rýfr} ‘God be \textbf{hold} towards him who keeps the truce, but wroth against him who breaks the truce’. I refer to \textcite{Läffler1895} for further discussion on these formulæ.

  \inxitem[C]{holdness} Closely connected to this is of course the abstract noun \textbf{holdness} (ON \emph{hylli}, OE \emph{hyldu}, OHG \emph{huldí}) ‘favour, loyalty, grace,’ with the same semantics as the adjective.

  Notably, this word appears three times in connection with the grace of gods in the poetry, namely in \Grimnismal\ 43, where (according to my interpretation) the preparer of food at the bloot is said to earn the “\textbf{holdness} of \inx[P]{Woulder} and of all the gods;” and \Grimnismal\ 53 where the disgraced king Garfrith is said to have been bereft of “my [= Weden’s] support; of all the Ownharriers (see note to the v.), and of Weden’s \textbf{holdness}”. Weden’s holdness (\emph{Óðins hylli}; the phrase is identical in both occurences) is also mentioned in a stanza by Hallfred (edited as Hfr Lv 7 by Diana Whaley in \Skp\ V) where the scold states that: ‘The whole race of man has wrought songs to win the \textbf{holdness} of Weden; I recall the fully rewarded works of our kinsmen/ancestors.’

  From all these citations the Germanic view on divine favour is clear: the gods are \textbf{hold} towards those who do good works, which in the aforementioned instances include swearing true oaths, faithfully observing truces, partaking in the bloot, following rules of hospitality and composing poetry—and \inx[C]{gram} ‘wroth’ towards those who do the opposite.

\inxitem[C]{Home} (ON \emph{hęimr}, OE \emph{hám}, PNWGmc. \emph{*haimaʀ})
  In the Norse often referring to a realm in the cosmology (\Voluspa\ 2: “I remember nine \textbf{Homes}”, \Vafthrudnismal\ TODO: “From the runes of the \inx[G]{Ettins} and of all the gods I can speak truly, for I have come into each \textbf{Home}”). Thus \inx[L]{Ettinham} is the ‘\textbf{Home}/realm of the ettins’. When used alone the term simply means ‘the world (that we inhabit)’. See also \inx[L]{nine Homes}, \inx[L]{Thrithham}.

\inxitem[C]{leat} (ON \emph{hlaut})
  Sacrificial blood (that is, taken from the animal), especially when used for auguries.

\inxitem[C]{leat-twig} (ON \emph{hlauttęinn})
  A twig used to sprinkle the \inx[C]{leat} in auguries (presumably the pattern of the blood would then be inspected).

\inxitem[C]{leed} (ON \emph{ljóð}, OE \emph{léod})
  A magical chant or incantation. See also \inx[C]{galder}, \inx[C]{gale}, \inx[C]{begale}.

\inxitem[C]{manwit} (ON \emph{manvit})
  Practical/common sense and wisdom, situational awareness.

\inxitem[C]{nithe} (ON \emph{níð}, OE \emph{níþ}, OHG \emph{níd})
  Originally probably ‘hatred, emnity’, in the Norse a sort of ritual libel that brought great dishonor.

\inxitem[C]{orlay} (ON \emph{ørlǫg}, OE \emph{orlæg})
  One’s predetermined fate, destiny, purpose as decreed by the \inx[G]{Norns}.

\inxitem[C]{rest} (ON \emph{rǫst})
  The distance between two rest-stops, a geographical mile (about 1850 metres). See especially \CV.

\inxitem[C]{rune} (ON \emph{rún}, OE \emph{rún}, OS \emph{rúna}, OHG \emph{rúna}, Got. \emph{rúna}, PNWGmc. \emph{rūnu})
  An (esoteric) secret message or formula. That this—rather than ‘letter (of a Runic alphabet)’—is the original and proper sense is apparent from among others the Finnish borrowing \emph{runo} ‘poem; poetry; a division of a poem (specifically of the \emph{Kalevala})’, and its use in the singular in the earliest Runic inscriptions (e.g. Noleby Vg 63, which contains the linguistically indecipherable string of letters {ᚢᚾᚨᚦᛟᚢᛊᚢᚺᚢᚱᚨᚺᛊᚢᛊᛁᚺ[--]ᚨᛁ\rotatebox[origin=c]{180}{ᛏ}ᛁᚾ}, a \emph{rune} in the proper sense or the recently discovered Svingerud fragment.) Thus, Weden’s taking of the \emph{runes} should not be interpreted as merely a myth for the invention of profane writing, but rather the origin of esoteric incantations, not at all unlike Indian \emph{mantras}.
  The word for letter was instead \inx[C]{stave}, see also there.

\inxitem[C]{scold} (ON \emph{skald})
  A Scandinavian poet. The name probably comes from their ability to slander with words.

\inxitem[C]{simble} (ON \emph{sumbl}, OE \emph{symbol})
  A banquet.

\inxitem[C]{soo} (ON \emph{sóa})
  To ritually waste, to slay (especially in a sacrificial context).

\inxitem[C]{thill} (ON \emph{þylja})
  To chant poetry or lists (so called \inx[C]{thule}[thules]) acquired by rote memorization. See \inx[C]{thyle}.

\inxitem[C]{Thing} (ON, OE \emph{þing}, OS \emph{thing}, OHG \emph{ding})
  The legal assembly and gathering place where matters would be settled and the law recited.

\inxitem[C]{thule} (ON \emph{þula})
  A poetic list, typically of various items of a category (e.g. gods, legendary horses) or poetic synonyms (e.g. for swords, men, Weden). Degoratively also a ditty, poorly composed poem. See \inx[C]{thyle}.

\inxitem[C]{thyle} (ON \emph{þulr}, OE \emph{þyle}, PNWGmc. \emph{*þuliʀ})
  A sage who through rote learning has acquired a large amount of mythological lore (cf. \inx[C]{thule} ‘a list in poetic form; a ditty, bad poem’ and \inx[C]{thill} ‘to recite, to chant’). Thus \inx[P]{Weden} is the \inx[P]{Fimblethyle}, being the unbeaten master of lore, as can be seen in his wisdom contests (like \Vafthrudnismal). Runic inscription DR 248 (Snoldelev) suggests the thyle may have tied to a specific place, and in \Beowulf\ it seems to have been a court position, with the poet Unferth being described (l. 1456) as the “thyle of Rothgar”.

\inxitem[C]{wale} (ON \emph{vǫlr})
  The staff or sceptre, especially of a wallow. TODO: archeological finds, mention Sutton Hoo.

\inxitem[C]{wallow} (ON \emph{vǫlva}, OE \emph{*wealwe} (cf. ON \emph{svǫlva}, OE \emph{swealwe} ‘swallow’))
  A sibyl, seeress, oracle. The word derives from the \inx[C]{wale}, a staff or sceptre probably used for ritual purposes.

\inxitem[C]{wigh} (ON \emph{vé}, OE \emph{wéoh}, \emph{wíh}, PNWGmc. \emph{*wīhą})
  A holy shrine or sanctuary. It seems that where the \inx[C]{harrow} was a pile of stones or cairn used for carrying out rituals, the \textbf{wigh} was an enclosed space. The earliest Norse attestation is the runic inscription Ög N288 (Oklunda), which reads: “Guther <= Gunnarr> painted these runes, and he fled, guilty. Sought this wigh, and he fled into this clearing. And he bound. [...]” The implication seems to be that the wigh was considered so sacred that Guther could not be apprehended or punished for his crime while in it. — In OE the word means ‘pagan idol’. It is not immediately clear which meaning is the original one, but in the present edition the Norse sense has been adopted, since the Anglo-Saxon sources are all of a Christian nature. The \Beowulf\ name \emph{Wighstone} (\emph{Wīh-} or \emph{Wēohstān}) in any case suggests it is the Norse meaning, since ‘idol-stone’ makes little sense.

\inxitem[C]{wode} (ON \emph{óðr}, OE \emph{wód}, PNWGmc. \emph{*wóþuʀ})
  \inx[P]{Heener}’s gift to men, though the name would suggest it be from \inx[P]{Weden}. The word has several related meanings: ‘poetic inspiration, madness, rage’.
\end{itemize}

\section{Persons and objects (P)}
\begin{itemize}

\inxitem[P]{Attle} (\emph{Attila}, ON \emph{Atli}, OE \emph{Ætla}, MHG. \emph{Etzel}, PNWGmc. \emph{*Attiló})
  The ruler of the \inx[G]{Huns} (historically from 434–453). Husband of \inx[P]{Guthrun}, and with her father of \inx[P]{Earp and Oatle}. and murderer of
  I HHb 54, SiL 11, I Gr 23, ShS 28, 29, 33, 37, 54, 56, 57, II Gr 26, 38, 45, III Gr 1, 9, BnOr 0, OdW A, 2, 22, 23, 25, 26, 30, 31, AtD 0, AtL 1, 3, 15, 17, 18, 27, 31, 32, 34, 36, 37, 38, 41, 43, B, AtS 2, 4, 21, 22, 44, 52, 60, 64, 71, 73, 77, 80, 86, 87, 97, 98, 108, 113, 117, FGr 0, GrB 12, Ham 6.

\inxitem[P]{Balder} (ON \emph{Baldr}, OE \emph{Bældæg} (not directly cognate), OHG \emph{Balter}, PWGmc. \emph{*Baldraʀ})
  The beautiful son of \inx[P]{Weden}, slayed by his brother \inx[P]{Hath}, avenged by his other brother \inx[P]{Wonnel}.

\inxitem[P]{Earp and Oatle} (ON \emph{Erpr ok Ęitill})
  The sons of \inx[P]{Attle} and \inx[P]{Guthrun}.

\inxitem[P]{Earth} (ON \emph{jǫrð}, OE \emph{eorþe}, OHG \emph{erda}, PNWGmc. \emph{*erþu}, PGmc. \emph{*erþó})
  The personified Earth. Through \inx[P]{Weden} the mother of \inx[P]{Thunder}.

\inxitem[P]{feather-hame} (ON \emph{fjaðr-hamr}, OE \emph{feðer-hama}, OS \emph{feðar-}, \emph{feðer-hamo})
  An object by which the wearer may fly like a bird. One is owned by Frow and used by Lock to fly between the homes. In the Heliand \textbf{feather-hames} are donned by angels who fly from heaven to earth. See also \inx[C]{hame}.

\inxitem[P]{Free} (ON \emph{Fręyr}, OE \emph{fréa} ‘lord’, PNWGmc. \emph{*Frawjaʀ})
  Son of \inx[P]{Nearth}, brother of \inx[P]{Frow}. See also \inx[P]{Ing}.

\inxitem[P]{Frie} (ON \emph{Frigg}, OE \emph{*Frige}, OHG \emph{Frija}, PNWGmc. \emph{*Frijju})
  Wife of \inx[P]{Weden}, mother of \inx[P]{Balder}. Related to \inx[P]{Full}, who is either her sister (Second Merseburg Charm, though this may be metaphorical, as in \Hyndluljod\ 1) or her maid-servant (the Norse sources).

\inxitem[P]{Frow} (ON \emph{Fręyja})
  Cat-goddess, daughter of \inx[P]{Nearth}, sister of \inx[P]{Free}, wife of \inx[P]{Wode}. Promised to the Ettin. Possibly = Easter?

\inxitem[P]{Full} (ON \emph{Fulla}, OHG \emph{Folla})
  Maid-servant (or sister?) of \inx[P]{Frie}; see there.

\inxitem[P]{Guthrun} (ON \emph{Guðrún})
  Daughter of king \inx[P]{Yivick}, sister of \inx[P]{Guther} and \inx[P]{Hain}. The wife of \inx[P]{Attle}.

\inxitem[P]{Hain}[Hain 1] (ON \emph{Hǫgni}, OE \emph{Haguna}, \emph{Hagena}, OHG \emph{Hagano}, Ger. \emph{Hagen}, PNWGmc. \emph{*Hagunó})
  A \inx[G]{Nivlings}[Nifling] and \inx[G]{Yivickings}[Yivicking], son of king \inx[P]{Yivick}, brother of \inx[P]{Guther} and \inx[P]{Guthrun}. In \emph{AtL} he defeats seven warriors before being captured by \inx[P]{Attle}, who has his heart cut out at the request of Guther.

\inxitem[P]{Hain 2}[2]
  A petty king of \inx[L]{East Geatland}, contemporary with \inx[P]{Granmer}, the king of \inx[L]{Southmanland} and Ingeld Illred, the \inx[G]{Inglings}[Ingling] king of \inx[L]{Upland}.

\inxitem[P]{Hath} (ON \emph{Hǫðr})
  The blind son of \inx[P]{Weden}, the slayer of his brother \inx[P]{Balder}.

\inxitem[P]{Heener} (ON \emph{Hǿnir}, PNWGmc. \emph{Hónijaʀ} ‘the little swan(?)’)
  An obscure god. \textcite{Rydberg1886}[552] has convincingly argued that he is connected with the stork, connecting his name with the Greek \textgreek{κύκνος} ‘swan’ and Sanskrit \emph{śakuna} ‘bird of omen’, and noting that his epithets \emph{langi fótr} ‘long foot’ and \emph{aurkonungr} ‘mud-king’ (both found in \Skaldskaparmal\ 22) accurately describe the stork. He gives \inx[C]{wode} TODO.

\inxitem[P]{Hindle} (ON \emph{Hyndla})
  A witch awoken by \inx[P]{Frow} in \Hyndluljod.

\inxitem[P]{Homedall} (ON \emph{Hęimdallr}, OE \emph{*Hámdall})
  Ward of the gods, whitest of the \inx[G]{Ease}.

\inxitem[P]{Hymer} (ON \emph{Hymir})
  \inx[P]{Tew}’s father according to \Hymiskvida.

\inxitem[P]{Ing} (ON \emph{Yngvi}, OE \emph{Ing})
  Probably an older name of \inx[P]{Free}. The legendary ancestor of the \inx[G]{Inglings}. Cf. the Old English Rune Poem.

\inxitem[P]{Lother} (ON \emph{Lóðurr}, OS \emph{Logaþore}, PNWGmc. \emph{*Logaþorjaʀ} ‘Flame-darer(?)’)
  Gives three gifts to man. The Old-Saxon attestation is a bit uncertain.

\inxitem[P]{Millner} (ON \emph{Mjǫllnir}, OE \emph{*Meldne}, PNWGmc. \emph{*Meldunjaʀ})
  Powerful hammer owned by Thunder.

\inxitem[P]{Nearth} (ON \emph{Njǫrðr})
  The father of \inx[P]{Free} and \inx[P]{Frow} by \inx[P]{Shede}.

\inxitem[P]{Nithad} (ON \emph{Níðuðr}, OE \emph{*Hámdall})
  The Swedish king that imprisons \inx[P]{Wayland} in \Volundarkvida. Father of \inx[P]{Beadhild}.

\inxitem[P]{Oughter} (ON \emph{Óttarr}, OE \emph{Óhthere}, PNWGmc. \emph{*Óhtaharjaʀ})
  Legendary Swedish king.

\inxitem[P]{Rotholf} (ON \emph{Hrólfr kraki}, OE \emph{Hróþulf}, PNWGmc. \emph{*Hróþiwulfaʀ})
  A king of the \inx[G]{Shieldings} (see family tree). As foreshadowed in \Beowulf\ 1017–9, 1180–90, he betrays the sons of \inx[P]{Rothgar}, his cousins \inx[P]{Rethrich and Rothmund}, in order to take the throne for himself. In the later Icelandic tradition this has been forgotten, and he is consistently portrayed as a heroic king.

\inxitem[P]{Rothgar} (ON \emph{Hróarr}, OE \emph{Hróþgár}, PNWGmc. \emph{*Hróþigaiʀaʀ})
  A king of the \inx[G]{Shieldings} (see family tree), one of the main characters in \Beowulf.

\inxitem[P]{Shield} (ON \emph{Skjǫldr}, OE \emph{Scyld})
  Legendary Danish king, founder of the \inx[G]{Shieldings}.

\inxitem[P]{Syemund} (ON \emph{Sigmundr}, OE \emph{Sigemund}, MHG. \emph{Siegmund})
  A hero of the \inx[G]{Walsings}, in \Beowulf\ attested as the slayer of the dragon along with his nephew \inx[P]{Sinfittle}. In the Norse tradition however, it is his half-brother \inx[P]{Siward} that slays the dragon instead.

\inxitem[P]{Sithguth} (OHG \emph{Sinthgunt}, PNWGmc. \emph{*Sinþagunþiz})
  Only known from \MerseburgTwo\ as the sister of \inx[C]{Sun}.

\inxitem[P]{Sun} (ON \emph{Sól}, OHG \emph{Sunna})
  The personified sun (see also \inx[P]{Moon}). In \MerseburgTwo, described as the sister of \inx[C]{Sithguth}.

\inxitem[P]{Thrim} (ON \emph{Þrymr})
  The ettin responsible for stealing Thunder’s hammer in \Thrymskvida.

\inxitem[P]{Thunder} (ON \emph{Þórr}, OE \emph{Þunor}, OHG \emph{Donar}, PNWGmc. \emph{*Þonaraʀ})
  Son of \inx[P]{Weden} and \inx[P]{Earth}.

\inxitem[P]{Tew} (ON \emph{Týr}, OE \emph{Tíw})
  Son of \inx[P]{Hymer}. One-handed god. TODO.

\inxitem[P]{Webthrithner} (ON \emph{Vafþrúðnir})
  The ettin defeated by Weden in the wisdom contest in \Vafthrudnismal.

\inxitem[P]{Weden} (rhymes with \emph{leaden}; ON \emph{Óðinn}, OE \emph{Wóden}, \emph{Wéden}, OHG \emph{Wuotan}, PNWGmc. \emph{*Wódanaʀ})
  Chief of the \inx[G]{Ease}, his name is clearly related to \inx[C]{wode}, referring to his role as the patron of \inx[C]{scold}[scolds] and \inx[C]{bearserk}[bearserks]. Husband of \inx[P]{Frie}, and by her father of \inx[P]{Balder}. Also father of \inx[P]{Thunder} by \inx[P]{Earth}. Brother of \inx[P]{Heener} and \inx[P]{Lother}.

\inxitem[P]{Wider} (ON \emph{Víðarr}, OE \emph{*Wídhere})
  A son of \inx[P]{Weden}, who avenges him at the \inx[L]{Rakes of the Reins}.

\inxitem[P]{Wode} (ON \emph{Óðr}, OE \emph{Wód})
  Husband of \inx[P]{Frow}. His name looks to be the same word as \inx[C]{wode}.

\inxitem[P]{Wonnel} (ON \emph{Váli}, OE \emph{*Wonela}, PNWGmc. \emph{*Wanilô} ‘the little \inx[G]{Wanes}[Wane]?’)
  The son of \inx[P]{Weden}, who one-night old avenged his brother \inx[P]{Balder} through slaying \inx[P]{Hath}, his half-brother.

\inxitem[P]{Woulder} (ON \emph{Ullr}, \emph{*Wuldor}, PNWGmc. \emph{*Wulþuz})
  A rather obscure god. He is mentioned in connection with oath-rings (TODO) and the setting of ritual fires (\Grimnismal\ TODO). These obscure references are likely related to the interesting finds at Lilla Ullevi (‘the small \inx[C]{wigh} of Woulder’) in Upland, Sweden, consisting of several dozen fire striker-shaped iron amulet rings dating to 660–780 (for a detailed description see \parencite{afEdholm2009}).

\inxitem[P]{Yimer} (ON \emph{Ymir}, OE \emph{*Yime})
  The first ettin, probably equivalent to \inx[P]{Earyelmer}.

\inxitem[P]{Yivick} (ON \emph{Gjúki}, OE \emph{Gifica}, OHG \emph{Gibicho}, MHG. \emph{Gibeche})
  King of the \inx[G]{Burgends} (historically from late 300s–407) of the Nifling dynasty, ancestor of the \inx[G]{Yivickings}. Father of \inx[P]{Guthrun}, \inx[P]{Guther} and \inx[P]{Hain}.
\end{itemize}

\section{Groups and tribes (G)}
TODO: Map of rough tribal areas. Geneaologies.

\begin{itemize}

\inxitem[G]{Danes} (ON \emph{danir}, OE \emph{dene}, PNWGmc. \emph{*daníʀ})
  A tribe in eastern modern-day Denmark and southern Sweden. They probably originated in Scania in southern Sweden, before moving westwards into the Danish isles and eventually Jutland, driving out the \inx[G]{Earls} and \inx[G]{Jutes}.
  Noted members: TODO
  Attestations: TODO

\inxitem[G]{Dwarfs} (ON \emph{dvergar}, OE \emph{dweorgas}, OHG \emph{twerca}, PNWGmc. \emph{*dwergóʀ})
  Earthly (chthonic) supernatural beings, often referred to as living in rocks and mountains.
  Noted members: TODO
  Attestations: TODO

\inxitem[G]{Ease} (rhyming with \emph{geese}; ON \emph{ę́sir}, OE \emph{ése}, PNWGmc. \emph{*ansiwiʀ}; sg. \emph{os}, ON \emph{áss}, OE \emph{ós}, PNWGmc. \emph{*ansuʀ})
  A group of Gods, though the word can also refer to all the Gods. See \inx[G]{Gods}, \inx[G]{Tews}, \inx[G]{Wanes}, \inx[G]{Reins}.
  Noted members: \inx[P]{Weden}, \inx[P]{Thunder}, \inx[P]{Frie}, \inx[P]{Hath} and \inx[P]{Balder}
  Attestations: TODO

\inxitem[G]{Elves} (ON \emph{alfar}, OE \emph{ielfe}, PNWGmc. \emph{*alβíʀ})
  Earthly (chthonic) supernatural beings. Possibly ancestral spirits?
  Noted members: TODO
  Attestations: TODO

\inxitem[G]{Ettins} (ON \emph{jǫtnar}, OE \emph{eotenas}, PNWGmc. \emph{*etunóʀ})
  The fundamental enemies of the Gods, the agents of chaos and disorder. See \inx[G]{Rises}, \inx[G]{Thurses}.
  Noted members: \inx[P]{Hymer}, \inx[P]{Thrim}, \inx[P]{Webthrithner}, \inx[P]{Yimer}
  Attestations: TODO

\inxitem[G]{Geats} (ON \emph{gautar}, OE \emph{géatas}, PNWGmc. \emph{*gautóʀ} from \emph{*geut-} ‘to pour’, perhaps ‘the libators’)
  A tribe in what is today southern-central Sweden. See also \inx[L]{Geatland}, \inx[G]{Swedes}.
  Noted members: TODO
  Attestations: TODO

\inxitem[G]{yin-Reins} (ON \emph{ginn-ręgin})
  \inx[C]{yin-} + \inx[G]{Reins}. The sacrosanct, highest divine powers.

\inxitem[G]{Gods} (ON \emph{goð}, OE \emph{godu}, OHG \emph{gota}, PNWGmc. \emph{*godu})
  TODO.
  Noted members: TODO
  Attestations: TODO

\inxitem[G]{Huns} (ON \emph{húnir}, OE \emph{Húne}, OHG \emph{Húni}, \emph{Hunni}, PNWGmc. \emph{*húníʀ})
  An invading Asiatic tribe in the Migration Period. In the legendary material their cultural and ethnic foreignness is not seen.
  Noted members: TODO
  Attestations: TODO

\inxitem[G]{Inglings} (ON \emph{ynglingar}, PNWGmc. \emph{*ingwalingóʀ} ‘the descendants of \inx[P]{Ing}’)
  Difference between this term and \inx[G]{Shelvings} is a bit unclear. They seem to be used synonymously in the Norse sources, whereas the English only use the later.

\inxitem[G]{Nears} (ON \emph{níarar} \char`~ \emph{njárar})
  A Swedish tribe, only mentioned in \Volundarkvida, where it is ruled by king \inx[P]{Nithad}. The name and location may allow us to connect them with the Swedish province of Närke, cf. Old Swedish: \emph{Nærikiar} ‘inhabitants of Närke’, \emph{Nærisker} ‘belonging to Närke; Nearish’, in which case the Old Swedish stem \emph{nær-} (with unclear vowel length, though it is probably long) would be a reduced form of \emph{níar-}, \emph{njár-}.

\inxitem[G]{Norns} (ON \emph{nornir})
  A group of supernatural women responsible for declaring the fates of men.

\inxitem[G]{Ossens} (ON \emph{ǫ́synjur})
  The women of the \inx[G]{Ease}, see there.

\inxitem[G]{Ownharriers} (ON \emph{ęinhęrjar}, OE \emph{*ánhergas})
  Earthly (chthonic) supernatural beings, often referred to as living in rocks and mountains.
  Noted members: TODO
  Attestations: TODO

\inxitem[G]{Reins} (ON \emph{rǫgn}, \emph{ręgin})
  The divine powers. Based on \Vafthrudnismal\ (TODO) the term may be more closely associated with the \inx[G]{Wanes} than the \inx[G]{Ease}.

\inxitem[G]{Saxons} (ON \emph{saxar}, OE \emph{Seaxan}, \emph{Seaxe})
  TODO.
  Noted members: TODO
  Attestations: TODO

\inxitem[G]{Shieldings} (ON \emph{skjǫldungar}, OE \emph{Scyldingas}, PNWGmc. \emph{*skeldungóʀ})
  The descendants of \inx[P]{Shield}; the legendary \inx[G]{Danes}[Danish] royal dynasty. With \inx[P]{Harward}’s death after his slaying of \inx[P]{Rotholf} their rule ended. TODO
  Noted members: TODO
  Attestations: TODO

\inxitem[G]{Shelvings} (ON \emph{skilfingar}, OE \emph{scilfingas}, PNWGmc. \emph{*skilβingóʀ})
  The descendants of \inx[P]{Shelf}; the legendary \inx[G]{Swedes}[Swedish] royal dynasty. The exact difference between the terms Shelvings and \inx[G]{Inglings} is unclear, but the first may have referred to the old royal family in Sweden, while the latter to the Norwegian branch which claimed descent from the former. TODO
  Noted members: TODO
  Attestations: \Hyndluljod\ 15, 20

\inxitem[G]{Swedes} (ON \emph{svíar}, OE \emph{swéon}, PNWGmc. \emph{*swihaníʀ})
  The tribe around the Mälar valley in eastern Sweden.
  Noted members: TODO
  Attestations: TODO

\inxitem[G]{Thurses} (sg. Thurse; ON \emph{þurs}, OE \emph{þyrs}, OS \emph{thuris}, OHG \emph{duris}, PNWGmc. \emph{*þurisaʀ})
  Possibly a poetic synonym for \inx[G]{Ettins}. See also \inx[G]{Rime-Thurses}.
  Noted members: TODO
  Attestations: Wal 8, Shr 31, 35, 36, Hyme 17, Thr 5, 10, 21, 24, 29, 30, Alw 2, I HHb 40, HHw 27.

\inxitem[G]{Tews} (ON \emph{tívar}, PNWGmc. \emph{*tíwóʀ})
  A poetic synonym for \inx[G]{Gods}.
  Attestations: TODO

\inxitem[G]{Wanes} (ON \emph{vanir}, OE \emph{wan-?})
  A subgroup or tribe of the gods, associated with fertility, harvests and fishing.
  Noted members: \inx[P]{Nearth}, \inx[P]{Ing}, \inx[P]{Frow}
  Attestations: TODO

\inxitem[G]{Yivickings} (ON \emph{gjúkungar})
  The descendants of \inx[P]{Yivick}, including \inx[P]{Guther}, \inx[P]{Guthrun} and \inx[P]{Hain}.
  Attestations: TODO
\end{itemize}

\section{Place names, locations and events (L)}
\begin{itemize}

\inxitem[L]{Eastern Way} (ON \emph{Austrvegr})
  The eastern lands of the \inx[G]{Ettins} (probably identical in meaning to \inx[L]{Ettinham}), whither \inx[P]{Thunder} goes to fight.

\inxitem[L]{Ettinham} (ON \emph{Jǫtunhęimr}, \emph{Jǫtnahęimr})
  The ‘\inx[G]{Ettins}[Ettin]-\inx[C]{Home}’ or ‘home of the Ettins’; the eastern realm of chaotic and inhospitable beings. See also \inx[L]{Eastern Way}, \inx[L]{Outyards}.

\inxitem[L]{Fimble-winter} (ON \emph{fimbulvetr})
  The great winter, which kills all humans apart from \inx[P]{Life and Lifethrasher}.

\inxitem[L]{Hell} (ON \emph{hęl}, PNWGmc. \emph{*halju}, Got. \emph{halja})
  The underworld, personfied as and formally identical with \inx[P]{Hell}. After Christianity the word came to refer to the Christian hell (= Gehenna), as is the case in all attested languages apart from the Old Norse. See also \inx[L]{Nivelhell}.

\inxitem[L]{Middenyard} (ON \emph{Mið-garðr}, OE \emph{Middangeard}, OS \emph{Middilgard}, OHG \emph{Mittilgart}, Got. \emph{midjungards})
  The ‘middle enclosure’; the realm of men. See also \inx[L]{Osyard}, \inx[L]{Outyards}.

\inxitem[L]{Nivelhell} (ON \emph{niflhęl})
  ‘Mist-Hell’, from the poetic evidence it seems like it may originally have been a synonym for \inx[L]{Hell}. In poetry it is attested in \Vafthrudnismal\ TODO: \emph{níu kom’k hęima |hld\ fyr Niflhel neðan, \\ hinig deyja ór helju halir. } ‘into nine homes I came, beneath Nivelhell; thither die men out of Hell’, the second by \Baldrsdraumar\ 2: \emph{ręið niðr þaðan |hld\ niflhęljar til; \\ mǿtti hvelpi, |hld\ þęim’s ór hęlju kom.} ‘[Weden] rode down thence to Nivel-hell; met the whelp that out of Hell came.’ Possibly the distinction was held by the first poet but not the second.

\inxitem[L]{Osyard} (ON \emph{Ásgarðr})
  The ‘enclosure of the \inx[G]{Ease}’; the heavenly realm. See also \inx[L]{Middenyard}, \inx[L]{Outyards}.

\inxitem[L]{Outyards} (ON \emph{Útgarðar})
  Not eddic. The ‘outer enclosures’, described in \Gylfaginning. See also \inx[L]{Ettinham}, \inx[L]{Middenyard}, \inx[L]{Osyard}.

\inxitem[L]{Rakes of the Reins} (ON \emph{ragna rǫk})
  The ‘fates of the \inx[G]{Reins}’, euphemism for the destruction of the world.

\inxitem[L]{Rakes of the Tews} (ON \emph{tíva rǫk})
  The \inx[L]{Rakes of the Reins}.

\inxitem[L]{Up-heaven} (ON \emph{Upphiminn}, OE \emph{Upheofon}, OS \emph{Upphimil}, OHG \emph{úfhimil})
  Highest heaven. See also \inx[F]{Earth and Up-heaven}.

\inxitem[L]{Walhall} (ON \emph{Valhǫll}, OE \emph{Wælheall})
  The hall of the slain, held by \inx[P]{Weden} and inhabited by the \inx[G]{Ownharriers}.
\end{itemize}

\section{Poetic formulæ (F)}
All formulæ are given in English translation, their attested forms and a Proto-Germanic rendition. For those consisting of two words bound together by a conjunction, \& is written in its place.

\begin{itemize}
\inxitem[F]{Earth and Up-heaven} (ON \emph{jǫrð \& upphiminn}, OE \emph{eorþe \& upheofon}, PGmc. \emph{*erþō \& uphiminaz})
  ON: Ribe charm \Voluspa\ 3, \Vafthrudnismal\ 20, \Thrymskvida\ 2, \Oddrunargratr\ 17, OE: Acreboot

\inxitem[F]{Ease and Elves} (ON \emph{ę́sir \& alfar}, OE \emph{ése \& ielfe}, PNWGmc. \emph{*alβíʀ \& ansiwiʀ})
  A merism; both heavenly and earthly spiritual beings. Notably the two words always occur in this order (never ‘Elves and Ease’), even in OE.

\inxitem[F]{words and works} (ON \emph{orð \& verk}, OE \emph{word \& weorc}, PGmc. \emph{*wurdó \& werkó})
  \Beowulf\ 289, 1100, 1833

\end{itemize}
%

\end{document}
