% This file should be compiled with XeLaTeX.

\documentclass[]{memoir}

% Font
\usepackage{fontspec}
\setmainfont{Junicode}[
	Extension=.ttf,
	BoldFont=*-Bold,
	ItalicFont=*-Italic,
	BoldItalicFont=*-BoldItalic]

% Underline that does not skip descender
% Should be called \nsunderline

% Packages
\usepackage{xparse}
\usepackage{geometry} %For margins.
\usepackage{longtable} %Long tables.

% Formatting
\usepackage{reledmac}

% Headers
\usepackage{fancyhdr}
\pagestyle{fancy}
\fancyhead[OR,EL]{\thepage}
\fancyhead[EC]{The Poetic Edda}
\fancyhead[OC]{\booktitle}

% Define verse counters
\newcounter{versea}
\newcounter{verseb}

\begin{document}

% Book and chapter commands
	\NewDocumentCommand{\chapterStart}{o O{Chap}}{% Command at the start of chapter
		\setcounter{versea}{0}%
		\setcounter{verseb}{0}%
		\stepcounter{chapter}%
		\IfNoValueF{#1}{%
			\begin{center}%
			\textbf{#2. \arabic{chapter}} \\
			{#1}\end{center}%
		}%
	}

	\NewDocumentCommand{\bookStart}{m}{% Command at the start of book
	  \book{#1}
		\def\booktitle{#1}
		\setcounter{chapter}{0} % Set chapter count to zero.
		\chapterStart{}%
	}

% Verse format commands
	\NewDocumentCommand{\bvg}{o}{% Begin verse group
		\begin{ledgroup}%
		\beginnumbering%
	}

	\NewDocumentCommand{\bva}{o}{% Begin verse a
		\begin{large}\begin{stanza}% Begin stanza
		\IfNoValueTF{#1}{% Add verse number according to counter
			\stepcounter{versea}% Step verse counter
			\flagstanza{\textbf{\arabic{versea}}}
		}{% Add verse number specified as optional argument
			\flagstanza{\textbf{#1}}
		}
	}
	\NewDocumentCommand{\eva}{o}{% End verse a
		\& \end{stanza}\end{large}\endnumbering% End reledmac numbering and stanza
		\vspace{1.5mm}% Vertical space
	}

	\NewDocumentCommand{\bvb}{o}{% Begin verse b
		\stepcounter{verseb}%
		\IfNoValueT{#1}{%
			\textbf{\arabic{verseb} }%
		}%
	}
	\NewDocumentCommand{\evb}{o}{% End verse b
		% Nothing (for now)
	}

	\NewDocumentCommand{\evg}{o}{% End verse group
		\end{ledgroup}%
		\vspace{1cm}%
	}

% Note formatting
	% Side note margin
	\setlength{\ledlsnotesep}{2 \ledlsnotesep}

	% Make A foot notes paragraphs
	\Xarrangement[A]{paragraph}

% Poem formatting
	% First line number at 3
	\firstlinenum{2}
	\linenumincrement{2}

	% Stanza indentation (required for \astanza to work)
	\setstanzaindents{5, 2, 2}
	\setcounter{stanzaindentsrepetition}{2}

	% Mark cæsura.
	\newcommand{\hld}{\hspace{5mm} }%\leavevmode\unskip\quad\ignorespaces}

	% Indent lines (in Ljóðaháttr or Galdralag).
	\newcommand{\ind}{%
		\hspace{1.5em}%
	}

	% Mark alliteration. This might not be present in the final version.
	\NewDocumentCommand{\alst}{m}{%
		\underline{#1}%
	}

	% Mark kennings.
	\NewDocumentCommand{\ken}{m o}{%
		\IfNoValueTF{#2}{% Not proper noun
			\textsc{[{#1}]}% [STONE]
		}{% Proper noun
			{[= {#1}]}% [= Thunder]
		}%
	}

% Index link command
%	{#1}\textsuperscript{†}%	Dagger at the end
%	\nsunderline{#1}%	Underline

	\NewDocumentCommand{\inx}{m o}{%
		{#1}\textsuperscript{#2}%
	}

% Sigla
	% Authors
	\newcommand{\Finnur}{%
		Finnur%
	}
	\newcommand{\Snorri}{%
		Snorri%
	}
	\newcommand{\CV}{%
		Cleasby-Vigfússon%
	}
	\newcommand{\Skp}{% Skaldic Poetry of the Scandinavian Middle Ages
		\emph{Skp}%
	}

	%Modern books
	\newcommand{\FaulkesEdda}{%
		\emph{SnE} 2005%
	}

	% Manuscripts
	\newcommand{\Regius}{% Codex Regius (of the poetic edda)
		\emph{R}%
	}
	\newcommand{\AM}{% AM 748 I a 4to
		\emph{A}%
	}
	\newcommand{\Hauksbok}{% Hauksbok
		\emph{H}%
	}
	\newcommand{\GylfMS}{% For referring to Gylfaginning manuscripts when verses are attested there.
		\emph{G}%
	}
	\newcommand{\RegiusProse}{% Codex Regius of the Prose Edda
		\emph{S}%
	}
	\newcommand{\Trajectinus}{% Codex Trajectinus
		\emph{T}%
	}
	\newcommand{\Wormianus}{% Codex Wormianus
		\emph{W}%
	}
	\newcommand{\Upsaliensis}{% Codex Upsaliensis
		\emph{U}%
	}
	\newcommand{\HildMS}{% For referring to the Hildebrandslied manuscript.
		\emph{Hild ms.}%
	}
	\newcommand{\Hickes}{% George Hickes
		\emph{Hickes}%
	}

	% Texts
	\newcommand{\Beowulf}{% Beewolf
		\emph{Bee}%
	}
	\newcommand{\Fostrbroedhra}{% Saw of the Foster-brothers
		\emph{FbrS}%
	}
	\newcommand{\FraLoka}{% From Lock
		\emph{FrL}%
	}
	\newcommand{\Gylfaginning}{% For referring to Gylfaginning as a text
		\emph{Yilf}% F
	}
	\newcommand{\Haleygjatal}{% Tally of the Hallowlendings
		\emph{Hal}%
	}
	\newcommand{\Hervarar}{% Saw of Harware
		\emph{HarS}%
	}
	\newcommand{\Hildebrandslied}{% Speeches of Hildbrand
		\emph{Hild}%
	}
	\newcommand{\Rigsthula}{% Thule of Righ
		\emph{Righ}%
	}
	\newcommand{\Skaldskaparmal}{ % The Matter of Scoldship
		\emph{Scold}%
	}
	\newcommand{\Vafthrudnismal}{% Speeches of Webthrithner
		\emph{Web}%
	}
	\newcommand{\Volsungasaga}{% Saw of the Walsings
		\emph{WalsS}%
	}
	\newcommand{\Ynglingatal}{% Tally of the Inglings
		\emph{Ing}%
	}

% Books

% Introduction, bibliography and abbreviations
% \title{%
  \Huge The \textsc{Old Germanic Scoldship}, \\
  \huge\emph{or, \\
  \textsc{Scandinavian, English} and \textsc{German Mythic} and \textsc{Heroic Alliterative Poetry, Newly Translated, Edited} and \textsc{Commented upon}} \\
  \emph{by} \\
  \Huge \textsc{Konrad Olof Lennart Rosenberg}; \\ \emph{also \textsc{Including} a \textsc{List} of \textsc{Poetic Formulæ}, and \textsc{Several Essays} on the \textsc{Ancient \\ Common Germanic Culture} \\ and \textsc{Worldview}.}}

\maketitle

\newpage\thispagestyle{empty}

\begin{center} The following people have been especially helpful in giving corrections and general feedback: Ęinarr, Nikhilasurya Dwibhashyam, Joseph S. Hopkins, John Newman, Trevor L. Payne, Thibault.\end{center}

\begin{center} \emph{\alst{V}ęl kęypts hlutar \hld\ hęf’k \alst{v}ęl notit; \\
\alst{f}ás es \alst{f}róðum vant; \\
því-at \alst{Ó}ð-rǿrir \hld\ es nú \alst{u}pp kominn \\
á \alst{a}lda vés \alst{ja}ðar} \\
(\emph{Háva mǫ́l} 106)\end{center}

\newpage\thispagestyle{empty}

\tableofcontents

\newpage

\thispagestyle{empty}\section{Abbreviations}
  \begin{itemize}% Manuscript sigla
    \item \AM\ = AM 748 I a 4° (https://handrit.is/manuscript/view/da/AM04-0748-I-a)
    \item \AMb\ = AM 748 I b 4° (https://handrit.is/manuscript/view/is/AM04-0748-Ib)
    \item \EddaBms\ = AM 757 a 4° (https://handrit.is/manuscript/view/is/AM04-0757a)
    \item \FlatMS\ = Flatsęyjarbók, GKS 1005 fol. (https://handrit.is/manuscript/view/is/GKS02-1005)
    \item \Hauksbok\ = Hauksbók, AM 544 4° (https://handrit.is/manuscript/view/en/AM04-0544)
    \item \VolsungaMS\ = NKS 1824 b 4° (https://onp.ku.dk/onp/onp.php?m9641)
    \item \Regius\ = Codex Regius of the Poetic Edda, GKS 2365 4° (https://eae.ku.dk/q.php?p=cr/poems)
    \item \RegiusProse\ = Codex Regius of the Prose Edda, GKS 2367 4° (https://handrit.is/manuscript/view/is/GKS04-2367)
    \item \Trajectinus\ = Codex Trajectinus, Traj 1374ˣ
    \item \Upsaliensis\ = Codex Upsaliensis, DG 11
    \item \Wormianus\ = Codex Wormianus, AM 242 fol. (https://clarino.uib.no/menota/text/menota/AM-242-fol)
  \end{itemize}

  \begin{itemize}% Languages
    \item Eng. = Modern English
    \item Ger. = Modern German
    \item Got. = Gotnish (or Gothic)
    \item Lomb. = Lombardic
    \item MHG = Middle High German
    \item OE = Old English
    \item OF = Old Frisian
    \item OHG = Old High German
    \item ON = Old Norse
    \item OS = Old Saxon
    \item OSwe. = Old Swedish
    \item PGmc. = Proto-Germanic
    \item PN = Proto-Norse
    \item PNWGmc. = Proto-North-West Germanic
  \end{itemize}

  \begin{itemize}% Grammar
    \item 1st = first-person
    \item 2nd = second-person
    \item 3rd = third-person
    \item acc. = accusative case
    \item cpd = compound
    \item dat. = dative case
    \item gen. = genitive case
    \item imper. = imperative mood
    \item ind. = indicative mood
    \item instr. = instrumental case
    \item nom. = nominative case
    \item pl. = plural number
    \item sg. = singular number
    \item subj. = subjunctive mood
  \end{itemize}

  \begin{itemize}% Other abbreviations
    \item cert. = certainly
    \item c. = circa
    \item cf. = \emph{confere}; compare
    \item corr. = corrected in the ms.
    \item e. = excerpt (not the whole stanza)
    \item ed. = edition, edited (by)
    \item e.g. = \emph{exemplio gratia}; for instance
    \item emend. = emendation, emended (by)
    \item fol., foll. = folio, folios
    \item i.e. = \emph{id est}; that is
    \item l., ll. = line, lines
    \item lit. = literally
    \item metr. emend. = emended based on (secure) metrical criteria
    \item ms., mss. = manuscript, manuscripts
    \item norm. = normalised from the ms. spelling
    \item om. = omitted by
    \item p., pp. = page, pages
    \item tr. = translation, translated (by)
    \item sens. emend. = emended based on sense
    \item st., sts. = stanza, stanzas
    \item viz. = \emph{vidēlicet}; namely, to wit
    \item wo. = without
    \item wrt. = with regard to
  \end{itemize}

\newpage

\bookStart{Introduction (INCOMPLETE!)}

\section{Introduction to Eddic poetry}
  Don't go too indepth on individual poems! Each one will have its own introduction.
  \subsection{Metrics and conventions}
    Alliteration
    Kennings
  \subsection{How can we know the age of the Eddic poems?}
    Linguistic criteria
    Archeological evidence
    Comparison with known Christian texts (Sólarljóð, Hugsvinnsmál)
    Snorri thought they were old
    Saxo had access to them
    Many of them clearly describe non-Icelandic surroundings
      Especially Hávamál is clearly Norwegian

\section{Ancient Germanic cult(ure)}
  \subsection{Economy (fee)}
  \subsection{Morals}
    Honour, personal integrity
    Notes on the terms \emph{argr} and \emph{ergi}
  \subsection{Religious conceptions}
    Cosmic cycles
    Reincarnation
    Analogies with other Indo-European traditions

\section{Notes to English translation}
  Point about literal translation for use by scholars of comparative mythology
    The “guiding star” of this translation effort has been literality and consistency. All previous translations (to my knowledge) have such issues as: rendering identically repeated phrases differently at various places; covering up or obscuring technical and cultural terminology; simplifying kennings and other expressions—and this often without notes, to a point where the original meaning is, at times, unrecognizable.
    While I wholly encourage all readers of sufficient interest to study Old Norse (and other ancient Germanic languages!), perhaps even using the present edition as a tool, I also realize that this is a demanding ask which not all interested students and scholars of comparative mythology, anthropology, literature, religion and other fields will be able to fulfill. I therefore want these groups to be able to have a text that is as close to the original as possible, at the very least when it regards sense and expression.
  \subsection{Anglish proper nouns}
    One of the most idiosyncratic parts of the present edition will be its handling of proper nouns. I have opted to render all cultural and religious terms, names of places, heroes, gods, and other entities by their English cognates (thus \emph{Thunder} for Old Norse \emph{Þórr}) and where such do not exist, their philologically expected English (\emph{Anglish}) forms (e.g. \emph{wallow} for Old Norse \emph{vǫlva}).
    One reason for this is ideological. I believe that these myths and poems are a common Germanic or Northern European heritage, and should be treated as such. The English once knew gods such as Weden and Thunder, and called them by names naturally evolved in their language. So too did the Germans and Scandinavians, of course, and I would hope that any translators into those languages would follow this spirit and render the names in their natural forms there as well.\footnote{For instance in German perhaps Wuten, Donner, Froh, in Swedish Oden, Tor, Frö.}
    Another is philological. Forms like Odin and Thor are, while now commonly accepted, debased. They do not even represent the Old Norse pronunciation as accurate as would be possible (for instance, Odin would be better anglicized as Othin; the dental fricative still survives in English!), and many are difficult for English speakers to pronounce. I shudder when hearing a word like \emph{ę́sir} pronounced /aɪˈsɪ:ɹ/

\section{Notes to critical edition}
  My goal with the critical editing of the texts has been to produce something as close to the original mss. as possible, without excessive emendation to the preserved recension(s). There are texts in three languages in the present edition, namely Old Norse, Old English and Old High German. Old Norse texts have been normalized according to roughly the same orthography as \textcite{FinnurEdda}. On the other hand the Old High German and Old English texts have only been lightly normalized, correcting obvious errors and marking vowel length with acute accents.

  \subsection{Normalization}
    The general principle in normalizing texts has been to strive for a uniform orthography across languages, where the same sound is written with the same character. This of course means disregarding local manuscript traditions and philological tradition, but I see this as justified. My goal is to render the texts themselves in a manner that gives as much information to the reader as possible—not to present a facsimile edition for students of paleography. Anyway, such obvious aspects of the original manuscripts as the long \emph{ſ}, arbitrary punctuation, arbitrary spelling, and lack of line breaks are almost never reproduced in modern editions of Old Germanic poetry.

    \subsubsection{Normalization of poetry}
    \begin{enumerate}
      \item Lines are broken at each long-line, not each half-line. This follows traditional practice for the publication of West Germanic poetry, while departing from that of Old Norse poetry.
      \item Cæsuræ are represented with the interpunct (·).
      \item Alliterations are marked with red colour.
    \end{enumerate}

    \subsubsection{Normalization of Old West Norse}
    The orthography is inspired by \textcite{FinnurEdda} in that it strives for a more archaic form than that of the surviving mss., one that instead represents the poetry as it may (in many cases, must) originally have looked. For this reason, it often has more in common with the proposed orthography of the First Grammatical Treatise than with the standard Old Icelandic orthography seen in most editions. The following list describes the differences from the standard orthography.

    \begin{enumerate}
    \item I distinguish short \emph{e} (from etymological short \emph{e}) and short \emph{ę} (from etymological short \emph{a} + \emph{i}-umlaut).
    \item I distinguish long \emph{á} and \emph{ǫ́}, as done by the First Grammatical Treatise.
    \item I use \emph{ǿ} and \emph{ę́} rather than the traditional \emph{œ} and \emph{æ}, to represent the vowels descended from Proto-Norse \emph{ō} and \emph{ā} after \emph{i}-umlaut (cf. the short \emph{ø, ę} < \emph{o, a} + \emph{i}-umlaut).
    \item I distinguish long nasal \emph{ȧ, ė, ï, ȯ, u̇} from long oral \emph{á, é, í, ó, ú}, as done by the First Grammatical Treatise.
    \item I restore the old \emph{s}—which in modern Scandinavian and even in most Old Norse manuscripts has become \emph{r}, but which is found consistently in old manuscripts such as AM 237 a fol (c. 1150), and fossilized in forms like \emph{þaz} (i.e. \emph{þat’s}) in \Regius—in the words \emph{es} ‘which, that, where, when’, and in inflections of \emph{vesa} (later \emph{vera}) such as \emph{es} ‘is’ (3rd sg. pres. ind.) and \emph{vas} (3rd sg. pret. ind.). The following forms retain the \emph{r}, as it is there the result of Verner’s law, and not of this (much younger) sound change: the pl. pres. ind. (\emph{erum} \&c.), the pl. pret. ind. (\emph{vǫ́rum} \&c.), and the pl. pret. subj. (\emph{vę́rim} \&c.)
    \item When metrically benefactory, I contract \emph{ek} ‘I’, \emph{eru} ‘are’, and \emph{es} ‘which; is’ to \emph{’k}, \emph{’ru} and \emph{’s}, respectively.
    \item I use \textcite{FinnurEdda}’s way of distinguishing between the relative particle \emph{es} and the verb \emph{es}: the first is appended to the previous word with only an apostrophe (e.g. \emph{hann’s} ‘he who’), while the second is separated by a space (e.g. \emph{hann ’s} ‘he is’).
    \end{enumerate}

    \subsubsection{Normalization of Old English}

    \subsubsection{Normalization of Old High German}

  \subsection{Manuscripts}

    \subsubsection{Eddic poetry}
    There are two surviving ancient mss. which contain full Eddic poems.

    The first and most important is GKS 2365 4to, here \Regius. It dates to the 1270s and has 45 surviving leaves, containing TODO poems. Of these 10 are mythological, and the rest heroic, dealing with legends mostly of the Migration Period. Notably, following fol. 32, there is a large gap of missing pages. This occurs in the heroic section, specifically cutting off \Sigrdrifumal. It is unclear how many leaves and poems went missing.
    \Regius\ is not just a compilation of poems, it shows editorial input as well. Several of the mythological poems are separated by short prose sections, which tie them together into a loose frame narrative, though it is clear from their style and composition that they are originally separate works. When it comes to the heroic poems long prose sections occur both within and between them, creating a \inx[C]{saw}-like narrative where the prose in many cases holds up the poetry, rather than the reverse. For further literature see TODO.

    The second ms. is AM 748 I a 4to, here \AM. It dates to the 1300s and is but a fragment, consisting of just 6 leaves. It contains only mythological poems, and in a different order from \Regius; unlike it there is no trace of a frame narrative. On the first two leaves are contained the final stanzas of \Harbardsljod\ (1r–v), the complete \Baldrsdraumar\ (1v–2r), and the first verses of \Skirnismal, after which a single leaf has been lost. The next four leaves follow eachother and contain the second half of \Vafthrudnismal, the complete \Grimnismal\ and \Hymiskvida, and the beginning of the prose introduction to \Volundarkvida. \AM\ is the only medieval manuscript attesting \Baldrsdraumar, and its variants of the poems attested in \Regius\ are clearly not copied from it, but rather derive from a common ancestor. This makes it very valuable for textual criticism. For further literature see TODO.

    Several Eddic poems are quoted in \Gylfaginning, namely (TODO): \Voluspa, \Vafthrudnismal, \Grimnismal. The text also quotes a few fragmentary verses of Eddic character (possibly from lost Eddic poems), which have here been edited together with their surrounding prose passages. For \Gylfaginning\ I have relied on the following four main mss.:\begin{enumerate}
	   \item The Codex Regius of the Prose Edda \RegiusProse\ (GKS 2367 4to; 1300-1350)
     \item The Codex Trajectinus \Trajectinus\ (Traj 1374; a c. 1595 paper copy of a ms. closely related to \RegiusProse.)
     \item The Codex Wormianus \Wormianus\ (AM 242 fol.; 1340–70)
     \item The Codex Upsaliensis \Upsaliensis\ (DG 11; 1300–25)\end{enumerate}

    For discussion on their internal stemmatics and origins I refer to \textcite{Haukur2017}. When all employed witness mss. of \Gylfaginning\ agree on a reading the siglum \GylfMS\ is used in the critical apparatus, which is thus equivalent to \RegiusProse\Trajectinus\Wormianus\Upsaliensis.

    A few other Eddic poems have also been edited. One of them, \Rigsthula, only survives in \Wormianus, though it is sadly incomplete (see its Introduction). Other Eddic poems survive only in younger paper mss., namely: TODO. While I have not consulted these paper mss. for poems attested in medieval mss., I have had to rely on them for these poems. Their exclusive survival there does not necessarily prove them to be late antiquarian works, as is clearly shown by \Baldrsdraumar, which among medieval mss. is only attested in the fragmentary \AM. It thus cannot be excluded that some of these poems would have existed in other lost medieval mss., perhaps even in the lost pages of \Regius\ or \AM.

    \subsubsection{West Germanic poetry}

    As none of the West Germanic poems edited here (TODO: Will we be editing other poems than Hildebrandslied?) survive in more than one copy, the specific details of their transmission is discussed in their individual Introductions.

  \printbibliography% Does it work?


% Theology, mythology, independent order
	\book{The Spae of the Wallow. (Vǫluspǫ́)}\bookStart

% Introduction.

\small{\emph{Vǫluspǫ́}, or the "Spae of the Wallow\footnotemark[1]", is the first poem of the R. manuscript.}
\footnotetext[1]{The Eng. equivalent of ON. \emph{vǫlva} 'seeress'; 'prophetess'. See index.}

% Greeting to the sons of Homedale, asking of Weden.

\bva Hljóðs bið'k allar \hld hęlgar kindir, \\%M
męiri ok minni \hld mǫgu Hęimdallar; \\%M
vildu at, Valfǫðr, \hld vęl fram tęlja'k \\%M
forn spjǫll fira, \hld þau's fręmst of man?

\bvb Of silence I bid all holy kins\footnotemark[1], [the] greater and lesser sons of Homedale\footnotemark[2]. Wilt thou, Leader of the Slain}\footnotemark[3], that I well tell forth the ancient sayings of firs\footnotemark[4], those I foremost recall?\footnotemark[5] \\
\footnotetext[1]{The 'Holy kins', according to FJ referring to the gods, but it might "simply" be an allusion to the divine origin of men.}
\footnotetext[2]{Cf. with \emph{Rþ}, wherein Rig (Homedale) sires the \emph{greater and lesser} human races (\emph{earls}, \emph{churls} and \emph{thralls}). The Wallow (speaking through the poet) addresses not just the human audience, but also the gods.}
\footnotetext[3]{FJ believes the name (for Weden) is chosen intentionally, as it refers to the final fight at the \textbf{Twilight of the Powers.}}
\footnotetext[4]{"Of men".}
\footnotetext[5]{Cf. \emph{Web} 34, 35 with very similar phrasing.}\\%E

\bva Ek man jǫtna \hld ár of borna, \\%M
þá es forðum \hld mik fǿdda hǫfðu; \\%M
níu man'k hęima, \hld níu íviðjur\footnotemark[4], \\%M
mjǫtvið mæran \hld fyr mold neðan.
\footnotetext[5]{Previously read \emph{íviði}, but closer study of R has disproven this. See \emph{Gripla} 3, pp. 227–28.}\\%E

\bvb I recall ettins, born of yore, those who earlier had nourished me. Nine Homes I recall, nine Inwithies; the famous Metwood, beneath the earth.

\bva Ár vas alda \hld þar’s Ymir byggði,\footnotemark[6] \\%M
vas-a sandr né sær, \hld né svalar unnir; \\%M
jǫrð fansk æva \hld né upphiminn; \\%M
gap vas ginnunga, \hld ęn gras hvęrgi.
\footnotetext[6]{\emph{Gylf} has \emph{þat’s ekki vas}.}\\%E

\bvb It was beginning of elds, where Yime dwelled\footnotemark[8]; there was not sand nor sea, nor cool waves. The earth was never found, nor up-heaven; a gap was of ginnings, but grass nowhere. 
\footnotemark[8]{Gylf. “that nothing was”.}

\bva Áðr Burs synir \hld bjǫðum of ypðu, \\%M
þęir es Miðgarð \hld mæran skópu; \\%M
sól skein sunnan \hld á salar stęina; \\%M
þá vas grund gróin \hld grǿnum lauki.\\%E

\bvb Ere Bur's sons did lift the flatlands, they who shaped the renowned Midyard; sun shone from the south on the stones of the hall, then the ground was grown with green leek.

\bva Sól varp sunnan, \hld sinni mána,\footnotemark[10] \\%M
hęndi hinni hǿgri \hld um himinjǫður; \\%M
sól þat né vissi, \hld hvar hon sali átti; \\%M
(stjǫrnur þat né vissu, \hld hvar þær staði ǫ́ttu); \\%M
máni þat né vissi, \hld hvat hann męgins átti.
\footnotetext[10]{At times translated as "its moon". This cannot be correct, as \emph{máni} 'moon' is masculine, while \emph{sinni}, dative singular of \emph{sínn} 'its (reflexive)' is feminine.}\\%E

\bvb Sun, the companion of Moon, cast from the south her right hand about heaven's rim; Sun knew not where she had halls, stars knew not what places they had, Moon knew not what he of power had.

\bva Þá gingu ręgin ǫll \hld á rǫkstóla, \\%M
ginnhęilǫg goð, \hld ok gættusk of þat.\footnotemark[20] \\%M
\footnotetext[20]{Cf. 9:1–4, 23:1–4, 25:1–4; two long-lines containing a question seem to be missing here.}\\%E

\bvb Then the Powers† all went onto the rake-seats†, the gin-holy† gods, and together took counsel of this:

\bva Nótt ok niðjum \hld nǫfn of gǫ́fu, \\%M
morgin hétu \hld ok miðjan dag, \\%M
undurn ok aptan, \hld ǫ́rum at tęlja.\footnotemark[22]
\footnotetext[22]{Cf. \emph{Web} 23, 25.}\\%E

\bvb To night and [her?] descendants they gave names; morning they named, and mid-day, undern (= mid-afternoon) and evening, for to reckon the years.

\bva Hittusk æsir \hld á Iðavęlli, \\%M
þęir's hǫrg ok hof \hld hótimbruðu; \\%M
afla lǫgðu, \hld auð smíðuðu, \\%M
tangir skópu \hld ok tól gęrðu.\\%E

\bvb The \textbf{Eses} met on the \textbf{Idewald}, they who harrows and hofs timbered up high; forges [they] laid, wealth forged, tongs shaped, and tools made.

\bva Tęflðu í túni, \hld tęitir vǫ́ru, \\%M
vas þeim véttugis \hld vant ór golli, \\%M
unz þríar kvǫ́mu \hld þursa męyjar, \\%M
ámátkar mjǫk, \hld ór Jǫtunhęimum.

\bvb They played \textbf{tables} in the yards, joyous were they, for them was no lack of gold. Until three\footnotemark[21] came, maidens of \emph{thurses}, much terrifying, out of \textbf{Ettinhome}.
\footnotetext[21]{These three \emph{thurse-maidens} are immediately forgotten and never again mentioned (unless they are taken to be the norns in v. 21 — but they would then be introduced twice). — Clearly there is something missing between this verse and the next, detailing the creation of dwarves.}\\%E

\bva Þá gingu ręgin ǫll \hld á rǫkstóla, \\%M
ginnhęilǫg goð, \hld ok gættusk of þat, \\%M
hverr skyldi dverga \hld dróttir skępja \\%M
ór Brimis blóði \hld ok ór Bláins lęggjum.\footnotemark[22]
\footnotetext[22]{The final two long-line vary substantially. \emph{R} has \emph{hverr scyldi duerga drotin scepia or brimis bloði oc or blám leggiom.} \emph{H} has \emph{huerer skylldu duergar drottir skepia or brimi bloðgv ok or Blains leggivm.}}\\%E

\bvb Then the Powers† all went onto the rake-seats†, the gin-holy† gods, and together took counsel of this: Who would shape the multitudes\footnotemark[23] of dwarves, out of the blood of Brime, and out of the legs of Blown?
\footnotetext[23]{alt. "the lord"}

\bva Þar vas Móðsognir\footnotemark[25] \hld mæztr of orðinn \\%M
dverga allra, \hld en Durinn annarr; \\%M
þęir manlíkun \hld mǫrg of gęrðu, \\%M
dvergar í jǫrðu, \hld sęm Durinn sagði.\\%E
\footnotetext[25]{R. \emph{mótsognir}, H. \emph{móðsognir}}

\bvb There did Moodsown become the worthiest of all dwarves, but Dorn [was] second; they made men-likenesses many, dwarves out of the earth, as Dorn said.\footnotemark[25]
\footnotetext[25]{A cryptic verse; \emph{manlíkan} 'man-likeness' is a hapax. It seems to imply that the lower dwarves were shaped out of soil or stone, by the mightiest dwarves, Moodsown and Dorn, themselves shaped from the blood of Brime and the legs of Blown (probably alternative names for \textbf{Yime}). \emph{sęm Durinn sagði} 'as Dorn said' implies that Dorn did not shape the dwarves himself; perhaps, he and Moodsown shaped the first lower dwarves out of stone, and then commanded these to finish the creation?}\\%E

\bva Nýi ok Niði, \hld Norðri, Suðri, \\%M
Austri, Vestri, \hld Alþjófr, Dvalinn, \\%M
Bívurr, Bávurr, \hld Bǫmburr, Nóri, \\%M
Ánn ok Ánarr, \hld Ái, Mjǫðvitnir.\\%E\footnotemark[28]
\footnotetext[28]{The three following verses seem to belong together, since there is no repetition of names. From the last verse of the middle one, it seems that it should have been placed at the end of the list.}

\bva Vęigr ok Gandalfr, \hld Vindalfr, Þráinn, \\%M
Þękkr ok Þorinn, \hld Þrór, Vitr ok Litr, \\%M
Nár ok Nýráðr, \hld nú hęf'k dverga, \\%M
Ręginn ok Ráðsviðr, \hld rétt of talða.\\%E

\bva Fíli, Kíli, \hld Fundinn, Náli, \\%M
Hęptifíli, \hld Hannarr, Svíurr, \\%M
Frár, Hornbori, \hld Frægr ok Lóni, \\%M
Aurvangr, Jari, \hld Ęikinskjaldi.\\%E

\bva Mál es dverga \hld í Dvalins liði \\%M
ljóna kindum \hld til Lofars tęlja, \\%M
þęir es sóttu \hld frá salar stęini \\%M
Aurvanga sjǫt \hld til Jǫruvalla.\\%E\footnotemark[30]
\footnotemark[30]{From the repeated names (Ęikinskjaldi, Ái), and the out-of-place introduction (\emph{mál es dverga...} 'a speech is of dwarves...'), it is clear that this verse and the following are originally separate from the previous three, and are a late (and redundant) addition to the \emph{Wale's Spae}.}

\bva Þar vas Draupnir \hld ok Dolgþrasir, \\%M
Hár, Haugspori, \hld Hlévangr, Glói, \\%M
Skirfir, Virfir, \hld Skáfiðr, Ái, \\%M
Alfr ok Yngvi, \hld Ęikinskjaldi, \\%M
Fjalarr ok Frosti, \hld Finnr ok Ginnarr; \\%M
Þat mun æ uppi, \hld meðan ǫld lifir, \\%M
langniðja-tal \hld til Lofars hafat.\\%E

\bva Unz þrír kvǫ́mu \hld ór því liði \\%M
ǫflgir ok ástkir \hld æsir at húsi, \\%M
fundu á landi \hld lítt męgandi \\%M
Ask ok Emblu \hld ørlǫglausa.\\%E

Until three came out of that host: the mighty and loving Ease at house. They found on land the little availing Ash and Emble, lacking orlay†.

\bva Ǫnd þau né ǫ́ttu, \hld óð þau né hǫfðu, \\%M
lǫ́ né læti \hld né litu góða; \\%M
ǫnd gaf Óðinn, \hld óð gaf Hǿnir, \\%M
lǫ́ gaf Lóðurr \hld ok litu góða.\\%E

\bva Breath they owned not, wode† they had not; [neither] craft nor sound, nor good complexion. Breath gave Weden, wode gave Hean, craft gave Lother, and good complexion.

\bva Ask veit'k standa, \hld hęitir Yggdrasill, \\%M
hǫ́r baðmr, ausinn \hld hvíta auri; \\%M
þaðan koma dǫggvar \hld þær's í dala falla; \\%M
stęndr æ yfir grǿnn \hld Urðar brunni.\\%E

I know an ash stands, called Ugdrassle, a high tree, sprinkled with white mud; thence come the dew-drops which fall in the dales; it stands evergreen over the well of Weird.

\bva Þaðan koma męyjar \hld margs vitandi \\%M
þríar ór þeim sæ, \hld es und þolli stendr; \\%M
Urð hétu ęina, \hld aðra Verðandi, \\%M
skǫ́ru á skíði, \hld Skuld hina þriðju \\%M
þær lǫg lǫgðu, \hld þær líf køru, \\%M
alda bǫrnum, \hld ørlǫg sęggja.\\%E

Thence come maidens, much knowing, three out of the lake which stands beneath the tree\footnote[1]: Weird they call one, the other Werthing—they carved on wooden boards—Shild the third. They laid laws, they chose lives, for the children of men, the orlay† of mortals.
\footnote[1] Lit. “pine”

\bva Þat man hon folkvíg \hld fyrst í hęimi, \\%M
es Gollvęigu \hld gęirum studdu \\%M
ok í hǫll Háars \hld hána bręndu, \\%M
þrysvar bręndu \hld þrysvar borna, \\%M
(opt ósjaldan, \hld þó hon ęnn lifir).\\%E

\bva Hęiði hétu, \hld hvar's til húsa kom, \\%M
vǫlu vęlspáa, \hld vitti hon ganda; \\%M
sęið, hvars kunni, \hld sęið hug lęikinn; \\%M
æ vas hon angan \hld illrar brúðar.\\%E

\bva Þá gingu ręgin ǫll \hld á rǫkstóla, \\%M
ginnhęilǫg goð, \hld ok gættusk of þat, \\%M
hvárt skyldi æsir \hld afráð gjalda, \\%M
eða skyldi goð ǫll \hld gildi ęiga.\\%E

\bvb Then the Powers† all went onto the rake-seats†, the gin-holy† gods, and together took counsel of this:

\bva Flęygði Óðinn \hld ok í folk of skaut; \\%M
þat vas ęnn folkvíg \hld fyrst í hęimi; \\%M
brotinn vas borðvęggr \hld borgar ása, \\%M
knǫ́ttu vanir vígspǫ́ \hld vǫllu sporna.\\%E

\bva Þá gingu ręgin ǫll \hld á rǫkstóla, \\%M
ginnhęilǫg goð, \hld ok gættusk um þat, \\%M
hvęrr hęfði lopt alt \hld lævi blandit \\%M
eða ætt jǫtuns \hld Óðs męy gefna.\\%E

\bvb Then the Powers† all went onto the rake-seats†, the gin-holy† gods, and together took counsel of this:

\bva Þórr ęinn þar vá \hld þrunginn móði, \\%M
hann sjaldan sitr, \hld es slíkt of fregn; \\%M
á gingusk ęiðar, \hld orð ok sǿri, \\%M
mǫ́l ǫll męginlig, \hld es á meðal fóru.\\%E

\bva Vęit hon Hęimdallar \hld hljóð of folgit \\%M
und hęiðvǫnum \hld hęlgum baðmi; \\%M
á sér hon ausask \hld aurgum forsi \\%M
af veði Valfǫðrs. \hld Vituð ér ęnn eða hvat?\\%E

\bva Ęin sat hon úti, \hld þá's hinn aldni kom \\%M
yggjungr ása \hld ok í augu lęit — \\%M
»hvęrs fregnið mik? \hld hví fręistið mín? \\%M
Alt vęit'k, Óðinn, \hld hvar auga falt \\%M
í hinum mæra \hld Mímis brunni;« \\%M
drekkr mjǫð Mímir \hld morgin hvęrjan \\%M
af veði Valfǫðrs. \hld Vituð ér ęnn eða hvat?\\%E

\bvb Lone she sat outside, when the old one came, the \textbf{Ose of Terror}, and looked into [her] eyes. "Why inquirest me? Why temptest me? All I know, Weden, where thine eye thou hidst: in the renowned \textbf{well of Mime}. Mime drinks mead every morning, from the pledge of the \textbf{Leader of the Slain}. Know ye yet, or what?"

\bva Valði hęnni Hęrfǫðr \hld hringa ok męn; \\%M
fekk spjǫll spaklig \hld ok spáganda; \\%M
sá vítt ok of vítt \hld of verǫld hvęrja.\\%E

\bva Sá hon valkyrjur \hld vítt of komnar, \\%M
gǫrvar at ríða \hld til goðþjóðar. \\%M
Skuld hęlt skildi, \hld ęn Skǫgul ǫnnur, \\%M
Gunnr, Hildr, Gǫndul \hld ok Gęirskǫgul; \\%M
nú eru talðar \hld nǫnnur Hęrjans, \\%M
gǫrvar at ríða \hld grund valkyrjur.\\%E

\bva Ek sá Baldri, \hld blóðgum tívur, \\%M
Óðins barni, \hld ørlǫg folgin; \\%M
stóð of vaxinn \hld vǫllum hæri \\%M
mjór ok mjǫk fagr \hld mistiltęinn.\\%E

\bvb I saw \textbf{Balder}'s, the bloody \textbf{tue}'s, the child of Weden's hidden \textbf{orlay}; grown did stand, higher than the meadows, the slender and much fair mistletoe.

\bva Varð af męiði, \hld þęim's mær sýndisk, \\%M
harmflaug hættlig, \hld Hǫðr nam skjóta. \\%M
Baldrs bróðir vas \hld of borinn snimma, \\%M
sá nam, Óðins sonr, \hld ęinnættr vega;\\%E

\bvb Of that tree, which looked slender, became a dangerous harm-flier; Had began to shoot. Balder's brother was born early; that son of Weden, one night old, began to kill.

\bva þó hann æva hęndr \hld né hǫfuð kęmbði, \\%M
áðr á bál of bar \hld Baldrs andskota. \\%M
Ęn Frigg of grét \hld í Fęnsǫlum \\%M
vǫ́ Valhallar. \hld Vituð ér ęnn eða hvat?\\%E

\bvb Hands he never washed, nor head combed, before onto the pyre he did bear the opponent of Balder. But Frigg did lament, in the Fenhalls, the woe of Walhall; know ye yet, or what?

\bva Hapt sá hon liggja \hld und Hveralundi \\%M
lægjarns líki \hld Loka áþękkjan; \\%M
þar sitr Sigyn \hld þęygi of sínum \\%M
veri vęl glýjuð. \hld Vitud ér ęnn eða hvat?\\%E

\bva Ǫ́ fęllr austan \hld of ęitrdala \\%M
sǫxum ok sverðum, \hld Slíðr heitir sú.\\%E

\bva Stóð fyr norðan \hld á Niðavǫllum \\%M
salr ór golli \hld Sindra ættar, \\%M
ęn annarr stóð \hld á Ókólni, \\%M
bjórsalr jǫtuns, \hld ęn sá Brimir hęitir.\\%E

\bva Sal sá hon standa \hld sólu fjarri \\%M
Nástrǫndu á, \hld norðr horfa dyrr; \\%M
falla ęitrdropar \hld inn um ljóra, \\%M
sá ’s undinn salr \hld orma hryggjum.\\%E

\bvb A hall she saw stand, far from the sun, on Corpsestrand; the doors face to the north. Drops of venom fell in through the smoke-vent, that hall is wound by the spines of snakes.

\bva Sér hon þar vaða \hld þunga strauma \\%M
męnn męinsvara \hld ok morðvarga \\%M
ok þanns annars glępr \hld ęyrarúnu. \\%M
Þar sýgr Níðhǫggr \hld nái framgingna; \\%M
slítr vargr vera. \hld Vituð ér ęnn eða hvat?\\%E

\bvb There she sees wade through heavy streams, oath-breaking men and murderwargs, TODO. There sucks Nithehew from corpses passed-on; the warg pulls weres [MEN] apart. Know ye yet, or what?

\bva Austr sat hin aldna \hld í Járnviði \\%M
ok fǿddi þar \hld Fęnris kindir; \\%M
verðr af þeim ǫllum \hld ęinna nøkkurr \\%M
tungls tjúgari \hld í trolls hami.\\%E

\bvb East sat the old woman, in Ironwood, and there nourished the kin of Fenner; TODO

\bva Fyllisk fjǫrvi \hld fęigra manna, \\%M
rýðr ragna sjǫt \hld rauðum dręyra, \\%M
svǫrt var þá sólskin \hld um sumur ęptir, \\%M
veðr ǫll válynd. \hld Vituð ér ęnn eða hvat?\\%E

\bvb It fills itself with the life-force of fey men; reddens the seat of the Powers with red gore. Black become the sunrays in the summers afterwards; the weather all hostile. Know ye yet, or what?

\bva Sat þar á haugi \hld ok sló hǫrpu \\%M
gýgjar hirðir, \hld glaðr Ęggþér; \\%M
gól of hǫ́num \hld í gaglviði \\%M
fagrrauðr hani, \hld sá's Fjalarr hęitir.\\%E

\bvb Sat there on the mound, and struck the harp, the troll-woman's keeper, glad Edgethew; by him crowed, in Gallowwood, a fair-red cock, he who is called Fealer.

\bva Gól of ǫ́sum \hld Gollinkambi, \\%M
sá vękr hǫlða \hld at Hęrjafǫðrs, \\%M
ęn annarr gęlr \hld fyr jǫrð neðan \\%M
sótrauðr hani \hld at sǫlum Hęljar.\\%E

\bvb By the Eses crowed Goldencombe, he who wakes men, at the \textbf{Father of Armies}', but another crows for below the earth, a soot-red cock, at the halls of Hell.

\bva Gęyr Garmr mjǫk \hld fyr Gnipahęlli, \\%M
fęstr mun slitna, \hld ęn freki rinna, \\%M
fjǫlð vęit'k frǿða, \hld framm sé'k lęngra \\%M
of ragna rǫk, \hld rǫmm sigtíva.\\%E

\bvb TODO. Much she knows of wisdom, forth I see yet further; about the fates of the Powers, the mighty ones of the victory-tues.

\bva Brǿðr munu bęrjask \hld ok at bǫnum verða, \\%M
munu systrungar \hld sifjum spilla, \\%M
hart ’s í hęimi, \hld hórdómr mikill, \\%M
skęggǫld, skalmǫld, \hld skildir 'ro klofnir, \\%M
vindǫld, vargǫld, \hld áðr verǫld stęypisk, \\%M
mun ęngi maðr \hld ǫðrum þyrma.\\%E

\bvb Brothers will fight one another, and become [one another's] bane; sister's sons will waste their in-laws. 'Tis hard in the Home, great whoredom: halberd-\textbf{eld}, short-sword-eld; shields are split. Wind-eld, outlaw-eld, before the world\footnotetext[4] is overthrown, no man will spare another.
\footnotetext[4]{\emph{ver-ǫld} 'world' might perhaps be better translated as 'man-eld', 'the eld of man' with the other elds preceding it.}

\bva Lęika Míms synir, \hld ęn mjǫtuðr kyndisk \\%M
at hinu galla \hld Gjallarhorni \\%M
hótt blæss Hęimdallr, \hld horn ’s á lopti; \\%M
mælir Óðinn \hld við Míms hǫfuð.\\%E

\bvb The \textbf{sons of Mime} play, but the \textbf{Metted} is kindled, at [the sound of] the shrill \textbf{Yeller-horn}. Homedall blows loudly; the horn is in the air. Weden speaks with the head of Mime.

\bva Skęlfr Yggdrasils \hld askr standandi, \\%M
ymr aldit tré, \hld ęn jǫtunn losnar; \\%M
hræðask allir \hld á hęlvegum \\%M
áðr Surtar þann \hld sevi of glęypir.\\%E

\bvb The ash of Ugdrassel shakes standing; the old tree groans, and the ettin is loosened. All are frightened on the \textbf{Hell-ways}, before \textbf{Surt's kinsman} does devour it.

\bva Hvat ’s með ǫ́sum? \hld hvat ’s með ǫlfum? \\%M
gnýr allr Jǫtunhęimr, \hld æsir ’ro á þingi, \\%M
stynja dvergar \hld fyr stęindurum \\%M
vęggbergs vísir — \hld vituð ér ęnn eða hvat?\\%E

\bvb What is with Eses? What is with Elves? All Ettinhome roars, Eses are at the Thing. Dwarves groan before gates of stone, the princes of the mountain-walls. Know ye yet, or what?

\bva Gęyr nú Garmr mjǫk \hld fyr Gnipahęlli, \\%M
fęstr mun slitna, \hld ęn freki rinna, \\%M
fjǫlð vęit'k frǿða, \hld framm sé'k lęngra \\%M
of ragna rǫk, \hld rǫmm sigtíva.\\%E

\bva Hrymr ękr austan, \hld hęfsk lind fyrir, \\%M
snýsk Jǫrmungandr \hld í jǫtunmóði; \\%M
ormr knýr unnir, \hld ęn ari hlakkar, \\%M
slítr nái niðfǫlr; \hld Naglfar losnar.\\%E

\bva Kjóll fęrr austan \hld koma munu Múspells \\%M
of lǫg lýðir, \hld ęn Loki stýrir; \\%M
fara fíflmęgir \hld með freka allir, \\%M
þęim es bróðir \hld Býlęists í fǫr.\\%E

\bva Surtr\footnotemark[19] fęrr sunnan \hld með sviga lævi, \\%M
skínn af sverði \hld sól valtíva; \\%M
grjótbjǫrg gnata, \hld ęn gífr\footnotemark[20] rata, \\%M
troða halir hęlveg, \hld ęn himinn klofnar.
\footnotetext[19]{SnE: \emph{Svartr}}
\footnotetext[20]{SnE: \emph{guðar} 'gods'.}\\%E

\bva Þá kømr Hlínar \hld harmr annarr framm, \\%M
es Óðinn fęrr \hld við ulf vega, \\%M
ęn bani Bęlja \hld bjartr at Surti; \\%M
þá mun Friggjar \hld falla angan.\\%E

\bva Þá kømr hinn mikli \hld mǫgr Sigfǫður, \\%M
Víðarr vega \hld at valdýri; \\%M
lætr hann męgi Hveðrungs \hld mund of standa \\%M
hjǫr til hjarta; \hld þá ’s hefnt fǫður.\\%E

\bva Þá kømr hinn mæri \hld mǫgr Hlǫðynjar \\%M
gęngr Óðins sonr \hld ormi mǿta. \\%M
Drepr af móði \hld Miðgarðs véurr; \\%M
munu halir allir \hld hęimstǫð ryðja; \\%M
gęngr fet níu \hld Fjǫrgynjar burr \\%M
nęppr frá naðri, \hld níðs ókvíðinn.\\%E

\bva Sól tér sortna, \hld søkkr fold í mar, \\%M
hverfa af himni \hld hęiðar stjǫrnur; \\%M
gęisar ęimi \hld við aldrnara; \\%M
lęikr hór hiti \hld við himin sjalfan.\\%E

\bva Gęyr Garmr mjǫk \hld fyr Gnipahęlli, \\%M
fęstr mun slitna, \hld ęn freki rinna, \\%M
fjǫlð vęit'k frǿða, \hld framm sé'k lęngra \\%M
of ragna rǫk, \hld rǫmm sigtíva.\\%E

\bva Sér hon upp koma \hld ǫðru sinni \\%M
jǫrð ór ægi \hld iðjagrǿna —; \\%M
falla forsar, \hld flýgr ǫrn yfir, \\%M
sás á fjalli \hld fiska vęiðir.\\%E

\bva Finnask æsir \hld á Iðavęlli \\%M
ok of moldþinur \hld mǫ́tkan dǿma, \\%M
ok minnask þar \hld á męgindóma \\%M
ok á Fimbultýs \hld fornar rúnar\\%E

\bvb The Ease are found on the Idewald

\bva Þar munu ęptir \hld undrsamligar \\%M
gollnar tǫflur \hld í grasi finnask, \\%M
þærs í árdaga \hld áttar hǫfðu.\\%E

\bva Munu ósánir \hld akrar vaxa; \\%M
bǫls mun alls batna \hld mun Baldr koma; \\%M
búa Hǫðr ok Baldr \hld Hropts sigtoptir \\%M
(vęl valtívar, \hld Vituð ér ęnn eða hvat?)\\%E

\bvb Fields will grow unsown, all evil be bettered, Balder will come. The \textbf{wal-tues}, Had and Balder, will well inhabit the building-plots of Roft. Know ye yet or what?

\bva Þá kná Hǿnir \hld hlautvið kjósa \\%M
ok burir byggva \hld brǿðra Tvęggja \\%M
vindhęim víðan. \hld Vituð ér ęnn eða hvat?\\%E

\bva Sal sér hon standa \hld sólu fęgra, \\%M
golli þakðan, \hld á Gimléi; \\%M
þar munu dyggvar \hld dróttir byggva \\%M
ok of aldrdaga \hld ynðis njóta.[3]\\%E

\bvb A hall she sees stand, fairer than the sun, thatched with gold, on Gimlee; there the dutiful \texbf{drights} will dwell, and in their \textbf{alder}-days enjoy delight.

\bva Þar kømr hinn dimmi \hld dręki fljúgandi, \\%M
naðr fránn neðan \hld frá Niðafjǫllum; \\%M
berr sér í fjǫðrum \hld — flýgr vǫll yfir — \\%M
Níðhǫggr nái; \hld nú mun hón søkkvask.

\bvb Then comes the shadowy dragon flying; the gleaming serpent down below from the \textbf{Nithfells}. He, Nithehew, carries in his feathers—flying over the plain—corpses." Now she will be sunk!\footnote[1]
\footnote[1] The Wale, referring to herself in third person, sinks back down into her grave, whence Weden woke her.
% — Weden
%	\book{The Speeches of Webthrithner. (Vafþrúðnismǫ́l)}\bookStart

\bvg {\small (Óðinn kvað:)}
\bva Ráð mér nú \alst{F}rigg \hld\ alls mik \alst{f}ara tíðir &
\ind at \alst{v}itja \alst{V}afþrúðnis; &
\alst{f}orvitni mikla \hld\ kveð'k mér á \alst{f}ornum stǫfum &
\ind við þann hinn \alst{a}lsvinna \alst{jǫ}tun.\eva

\bvb \inx{Weden} quoth: “Counsel me now, \inx{Frie}, as I desire to travel to visit \inx{Webthrithner}; greatly curious am I of ancient staves\footnotemark[1] by that all-wise \inx{ettin}."\evb
\footnotetext[1]{Ancient (pieces of) lore; cf. v. 55. — Meaning (from \emph{great} onwards) is clear, but form is very confused.}
\evg


\bvg {\small (Frigg kvað:)}
\bva \alst{H}ęima lętja \hld\ mynda'k \alst{H}ęrjafǫðr &
\ind í \alst{g}ǫrðum \alst{g}oða; &
\alst{ę}ngi \alst{jǫ}tun \hld\ hugða'k \alst{ja}fnramman &
\ind sęm \alst{V}afþrúðni \alst{v}esa.\eva

\bvb Frie quoth: “I would encourage the \inx{Leader of Armies} to [stay at] home in the yards of the gods, for I've judged no ettin be as strong as\footnotemark[3] Webthrithner."\evb
\footnotetext[3]{Lit. ‘equal-strong'.}
\evg


\bvg {\small (Óðinn kvað:)}
\bva Fjǫlð ek fór, \hld\ fjǫlð fręistaða'k, &
\ind fjǫlð ek ręynda ręgin; &
hitt vil'k vita, \hld\ hvé Vafþrúðnis &
\ind salakynni séi.\eva

\bvb Weden quoth: “Much I travelled, much I tried, much I tested the \inx{Reins}\footnotemark[4]. \emph{This} I want to know, how the condition of the halls of Webthrithner might be?"\evb
\footnotetext[4]{The gods.}
\evg


\bvg {\small (Frigg kvað:)}
\bva Hęill þú farir, \hld\ hęill þú aptr komir, &
\ind hęill á sinnum séir; &
ǿði þér dugi \hld\ hvar's skalt, Aldafǫðr, &
\ind orðum mæla jǫtun.\eva

\bvb Frie quoth: “Whole may thou travel, whole may thou return, whole may thou be on thy paths! May thy wisdom suffice, \inx{Leader of Men}, when thou go to exchange words with the ettin."\evb
\evg


\bvg
\bva Fór þá Óðinn \hld\ at fręista orðspęki &
\ind þess hins alsvinna jǫtuns; &
at hǫllu hann kom, \hld\ es\footnotemark[1] átti Íms faðir; &
\ind inn gekk Yggr þegar.\eva
\footnotetext[1]{Ms. \emph{ok} corrected to \emph{es}. Alliteration is lacking in this line, for which reason FJ emends \emph{Íms} to \emph{Hymis}.}

\bvb Then went Weden, to try the word-wisdom of that all-wise ettin. To the hall he came, which the father of \inx{Ime}\footnotemark[5] owned; shortly the \inx{Frightener}\footnotemark[6] walked in.\evb
\footnotetext[5]{Webthrithner.}
\footnotetext[6]{Weden.}
\evg


\bvg {\small (Óðinn kvað:)}
\bva Hęill þú nú, Vafþrúðnir, \hld\ nú em'k í hǫll kominn &
\ind á þik sjalfan séa; &
hitt vilk fyrst vita, \hld\ ef fróðr séir &
\ind eða alsviðr, jǫtunn.\eva

\bvb Weden quoth: “Hail thee now, Webthrithner; now I have come into the hall, to see thee thyself. \emph{This} I want to know first, if knowing thou might be, or all-wise, ettin!"\evb
\evg


\bvg {\small (Vafþrúðnir kvað:)}
\bva Hvat's þat manna, \hld\ es í mínum sal &
\ind verpumk orði á? &
út þú né kømr \hld\ órum hǫllum frá. &
\ind nema þú inn snotrari séir.\eva

\bvb Webthrithner quoth: “What is that of men\footnotemark[10], that in \emph{my} hall throws words at me? Thou will not come \emph{out}, from \emph{our}\footnotemark[11] halls, unless thou be the wiser [of us two]."\evb
\footnotetext[10]{Ie., ‘what man is that'. The use of the neuter pronoun \emph{hvat} by Web-str. may be seen as an insult or a way of belittling the guest.}
\footnotetext[11]{Prob. again meaning ‘my', unless Web-str. has allies present in the hall, but no such indication is given.}
\evg


\bvg {\small (Óðinn kvað:)}
\bva Gagnráðr\footnotemark[5] hęiti'k, \hld\ nú em'k af gǫngu kominn, &
\ind þyrstr til þinna sala; &
laðar þurfi \hld\ hęf'k lęngi farit &
\ind ok þinna andfanga, jǫtunn.\eva
\footnotetext[5]{R's \emph{Gagnráðr} ‘Gainred', is attested as Gangráðr ‘Journey-adviser' in \emph{Gylf}.}

\bvb Weden quoth: “\inx{Gainred} I am called, I am come from the journey, thirsty to thy halls. I have travelled for a long time in need of hospitality, and of thy reception, ettin!"\evb
\evg


\bvg \bvg {\small (Vafþrúðnir kvað:)}
\bva Hví þú þá, Gagnráðr, \hld\ mælisk af golfi fyrir? &
\ind far þú í sess í sal; &
þá skal fręista, \hld\ hvárr flęira viti, &
\ind gęstr eða hinn gamli þulr.\eva

\bvb Webthrithner quoth: “Why then, Gainred, art thou speaking from the floor before [me]? Take a seat in the hall! Then it shall be proven, which of the two might know more; the guest, or the old \inx{thyle}."\evb
\evg


\bvg {\small (Gagnráðr kvað:)}
\bva Óauðigr maðr, \hld\ es til auðigs kømr, &
\ind mæli þarft eða þęgi; &
ofrmælgi mikil \hld\ hygg at illa geti &
\ind hvęim's við kaldrifjaðan kømr.\eva

\bvb Gainred quoth: “An unwealthy man, who comes to a wealthy [one], ought to speak what is needed, or be silent.\footnotemark[14] Much over-speaking\footnotemark[15], I judge, will be bad for the one who comes to a cold-ribbed\footnotemark[16] [man]."\evb
\footnotetext[14]{Line identical to \emph{High} 18/2. The whole verse strongly reminds of verses from the \emph{Guest-thread} portion of said poem.}
\footnotetext[15]{“Speaking too much".}
\footnotetext[16]{That is, ‘cold-hearted', ‘cunning'.}
\evg


\bvg \bvg {\small (Vafþrúðnir kvað:)}
\bva Sęg mér, Gagnráðr, \hld\ alls á golfi vill &
\ind þíns of fręista frama, &
hvé hęstr hęitir, \hld\ sá's hvęrjan dręgr &
\ind dag of dróttmǫgu.\eva

\bvb Webthrithner quoth: “Say to me, Gainred, since on the floor I will to try thy fame: What is the horse called, which pulls each \emph{day} above the sons of the retinue \ken{Men}?"\evb
\evg


\bvg {\small (Gagnráðr kvað:)}
\bva Skinfaxi hęitir, \hld\ es hinn skíra dręgr &
\ind dag of dróttmǫgu; &
hęsta baztr \hld\ þykkir með Hręiðgotum; &
\ind ęy lýsir mǫn af mari.\eva

\bvb Gainred quoth: “\inx{Shining-fax} [that one] is called, who pulls the bright day above the sons of the retinue. The best of horses he seems among the \inx{Rode-goths}; the mane of that stallion ever shines."\evb
\evg


\bvg (Vafþrúðnir kvað:) &
\bva Sęg þat, Gagnráðr, \hld\ alls á golfi vill &
\ind þíns of fręista frama, &
hvé jór hęitir, \hld\ sá's austan dręgr &
\ind nótt of nýt ręgin.\eva

\bvb Webthrithner quoth: “Say this, Gainred, since on the floor I will to try thy fame: What is the horse called, which from the east pulls night above the useful \inx{Reins}?"\evb
\evg


\bvg {\small (Gagnráðr kvað:)}
\bva Hrímfaxi hęitir, \hld\ es hvęrja dręgr &
\ind nótt of nýt ręgin; &
méldropa fęllir \hld\ morgin hvęrjan; &
\ind þaðan kømr dǫgg of dala.\eva

\bvb Gainred quoth: “\inx{Frost-fax} [that one] is called, who pulls each night above the useful Reins. Every morning he lets foam fall from his bit\footnotemark[26]; thence comes dew in the valleys."\evb
\footnotetext[26]{Lit. “he fells bit-drops".}\evg


\bvg {\small (Vafþrúðnir kvað:)}
\bva Sęg þat, Gagnráðr, \hld\ alls á golfi vill &
\ind þíns of fręista frama, &
hvé ǫ́ hęitir, \hld\ sú's dęilir með jǫtna sonum &
\ind grund ok með goðum.\eva

\bvb Webthrithner quoth: “Say this, Gainred, since on the floor I will to try thy fame; How the river is called, which divides the ground between the sons of ettins and the gods?"\evb
\evg


\bvg {\small (Gagnráðr kvað:)}
\bva Ífing hęitir ǫ́, \hld\ es dęilir með jǫtna sonum &
\ind grund ok með goðum; &
opin rinna \hld\ hón skal um aldrdaga; &
\ind verðr-at íss á ǫ́.\eva

\bvb Gainred quoth: “\inx{Iving} the river is called, which divides the ground between the sons of ettins and the gods. Throughout [her] life-days she shall flow open; ice forms not on the river."\evb
\evg


\bvg {\small (Vafþrúðnir kvað:)}
\bva Sęg þat, Gagnráðr, \hld\ alls á golfi vill &
\ind þíns of fręista frama, &
hvé vǫllr hęitir, \hld\ es finnask vigi at &
\ind Surtr ok hin svǫ́su goð.\eva

\bvb Webthrithner quoth: “Say this, Gainred, since on the floor I will to try thy fame: How that valley is called, where \inx{Surt} and the excellent gods find each other at war?"\evb
\evg


\bvg {\small (Gagnráðr kvað:)}
\bva Vígríðr hęitir vǫllr, \hld\ es finnask vígi at &
\ind Surtr ok hin svǫ́su goð; &
hundrað rasta \hld\ hann's á hvęrjan veg; &
\ind sá's þęim vǫllr vitaðr.\eva

\bvb Gainred quoth: “\inx{Battle-rider} is the valley called, where Surt and the cheerful gods find each other at war. A hundred rests\footnotemark[30], he stretches in each direction; that valley is known for them.\footnotemark[31]"\evb
\footnotetext[30]{An old unit of length, from its name prob. the length a horse could travel before resting.}
\footnotetext[31]{That is, known for its great size.}\evg


\bvg {\small (Vafþrúðnir kvað:)}
\bva Fróðr est nú gęstr, \hld\ far á bękk jǫtuns, &
\ind mælumk í sessi saman; &
hǫfði vęðja \hld\ vit skulum hǫllu í &
\ind gęstr, of gęðspęki.\eva

\bvb Webthrithner quoth: “Knowing art thou now, guest, sit down on the ettin's bench; let us speak while sitting together. In the hall we shall wager a head, guest, over [our] mind-wisdom."\evb
\evg


\bvg {\small (Gagnráðr kvað:)}
\bva Sęg þat hit ęina, \hld\ ef þitt ǿði\footnotemark[10] dugir &
\ind ok þú Vafþrúðnir vitir, &
hvaðan jǫrð of kom \hld\ eða upphiminn &
\ind fyrst, hinn fróði jǫtunn.\eva
\footnotetext[10]{Starting with \emph{ǿði}, the poem is also preserved in 748.}

\bvb Gainred quoth: “Say the first\footnotemark[32], if thy wisdom suffices, and thou, Webthrithner, might know: Whence, O knowing ettin, the earth first came, or \inx{up-heaven}?"\evb
\footnotemark[32]{Lit. ‘one'.}\evg


\bvg {\small (Vafþrúðnir kvað:)}
\bva Ór Ymis holdi \hld\ vas jǫrð of skǫpuð, &
\ind ęn ór bęinum bjǫrg, &
himinn ór hausi \hld\ hins hrimkalda jǫtuns, &
\ind ęn ór svęita sær.\eva

\bvb Webthrithner quoth: “Out of \inx{Yime's} hull\footnotemark[35], the earth was created, but the mountains out of his bones. Heaven out of the skull of the frost-cold ettin, but the sea out of his sweat.\footnotemark[36]"\evb
\footnotetext[35]{His body.}
\footnotetext[36]{\emph{svęiti} ‘sweat' is a common kenning for blood. — This v. closely resembles \emph{Grím} 40.}\evg


\bvg {\small (Gagnráðr kvað:)}
\bva Sęg þat annat, \hld\ ef þitt ǿði dugir &
\ind ok þú Vafþrúðnir vitir, &
hvaðan máni of kom, \hld\ svát fęrr menn yfir, &
\ind eða sól hit sama.\eva

\bvb Gainred quoth: “Say the second, if thy wisdom suffices, and thou, Webthrithner, might know: Whence the moon came, so that it travels over men, or likewise the sun?"\evb
\evg


\bvg {\small (Vafþrúðnir kvað:)}
\bva Mundilfari hęitir, \hld\ hann's Mána faðir &
\ind ok svá Solar hit sama; &
himin hverfa \hld\ þau skulu hvęrjan dag &
\ind ǫldum at ártali.\eva

\bvb Webthrithner quoth: “\inx{Moundelfare} [that one] is called, he is the father of the Moon, and likewise of the Sun. They shall circle in the heavens every day, for men to reckon time\footnotemark[40]."\evb
\footnotetext[40]{Lit. “for men to year-reckoning".}\evg


\bvg {\small (Gagnráðr kvað:)}
\bva Sęg þat þriðja, \hld\ alls þik svinnan kveða &
\ind ok þú Vafþrúðnir vitir, &
hvaðan dagr of kom, \hld\ sá's fęrr drótt yfir, &
\ind eða nótt með niðum.\eva

\bvb Gainred quoth: “Say the third, since [they] call thee wise, and thou, Webthrithner, might know: Whence the day came, the one that travels over the rettinue, or night with the moon-phases?"\evb
\evg


\bvg {\small (Vafþrúðnir kvað:)}
\bva Dęllingr hęitir, \hld\ hann's Dags faðir, &
\ind ęn Nótt vas Nǫrvi borin; &
ný ok nið \hld\ skópu nýt ręgin &
\ind ǫldum at ártali.\eva

\bvb Webthrithner quoth: “\inx{Delling} [that one] is called, he is the father of \inx{Day}, but \inx{Night} was born to \inx{Nare}. The waxing and waning [of the moon], the useful Reins created, for men to reckon time."\evb
\evg


\bvg {\small (Gagnráðr kvað:)}
\bva Sęg þat fjórða, \hld\ alls þik fróðan kveða, &
\ind ok þú Vafþrúðnir vitir, &
hvaðan vetr of kom \hld\ eða varmt sumar &
\ind fyrst með fróð ręgin.\eva

\bvb Gainred quoth: “Say the fourth, since [they] call thee knowing, and thou, Webthrithner, might know: Whence winter first came, or the warm summer, among the knowing Reins?"\evb
\evg


\bvg {\small (Vafþrúðnir kvað:)}
\bva Vindsvalr hęitir, \hld\ hann's Vetrar faðir, &
\ind ęn Svǫ́suðr Sumars.\footnotemark[15]\eva
\footnotetext[15]{Second half of the v. seems missing.}

\bvb Webthrithner quoth: “\inx{Wind-cool} [that one] is called, he is the father of \inx{Winter}, but \inx{Delightful} of \inx{Summer}."\evb
\evg


\bvg {\small (Gagnráðr kvað:)}
\bva Sęg þat fimta, \hld\ alls þik fróðan kveða, &
\ind ok þú Vafþrúðnir vitir, &
hvęrr ása ęlztr \hld\ eða Ymis niðja &
\ind yrði í árdaga.\eva

\bvb Gainred quoth: “Say the fifth, since [they] call thee knowing, and thou, Webthrithner, might know: Who, in days of yore became the eldest of the \inx{Anses}, or of the descendants of Yime?"\evb
\evg


\bvg {\small (Vafþrúðnir kvað:)}
\bva Ørófi vetra \hld\ áðr væri jǫrð skǫpuð, &
\ind þá vas Bergęlmir borinn, &
Þrúðgęlmir \hld\ vas þess faðir, &
\ind ęn Aurgęlmir afi.\eva

\bvb Webthrithner quoth: “Uncountable winters before the earth would be created, then \inx{Bear-yeller} was born. \inx{Strength-yeller} was \emph{that one's} father, and \inx{Mud-yeller} the grandfather."\evb
\evg


\bvg {\small (Gagnráðr kvað:)}
\bva Sęg þat sétta, \hld\ alls þik svinnan kveða, &
\ind ok þú Vafþrúðnir vitir, &
hvaðan Aurgęlmir kom \hld\ með jǫtna sonum &
\ind fyrst, hinn fróði jǫtunn.\eva

\bvb Gainred quoth: “Say the sixth, since [they] call thee wise, and thou, Webthrithner, might know: Whence, O knowing ettin, Mud-yeller first came among the sons of ettins?"\evb
\evg


\bvg {\small (Vafþrúðnir kvað:)}
\bva Ór Élivǫ́gum \hld\ stukku ęitrdropar, &
\ind svá óx unz ór varð jǫtunn; &
órar ættir \hld\ kómu þar allar saman; &
\ind því's þat æ alt til atalt.\footnotemark[20]\eva
\footnotetext[20]{Lines 3–4 missing in R and 748, but quoted in \emph{Gylf}.}

\bvb Webthrithner quoth: “From the \inx{Ell-waves}, poison-drops splashed; thus [it] grew until an ettin emerged. \emph{Our} family lines all together originated there, therefore our race\footnotemark[45] is forever fierce against all.\footnotemark[46]"\evb
\footnotetext[45]{Lit. ‘it' or ‘that'.}
\footnotetext[46]{Somewhat strange phrasing, but the line does not appear damaged. It is clearly an explanation of the fierce and maleficent nature of the ettins, as their first ancestors were created from poison.}\evg


\bvg {\small (Gagnráðr kvað:)}
\bva Sęg þat sjaunda, \hld\ alls þik svinnan kveða, &
\ind ok þú Vafþrúðnir vitir, &
hvé sá bǫrn gat \hld\ hinn baldni\footnotemark[25] jǫtunn, &
\ind es hann hafði-t gýgjar gaman.\eva
\footnotetext[25]{R has \emph{aldni}, ‘aged, old'. This breaks alliteration; \emph{baldni} ‘bold, defiant' has been substituted from 748.}

\bvb Gainred quoth: “Say the seventh, since [they] call thee wise, and thou, Webthrithner, might know: How did that one, the defiant ettin, beget children, when he did not enjoy the [carnal] pleasure of a troll-woman?"\evb
\evg


\bvg {\small (Vafþrúðnir kvað:)}
\bva Und hęndi vaxa \hld\ kvǫ́ðu hrímþursi &
\ind męy ok mǫg saman; &
fótr við fǿti \hld\ gat hins fróða jǫtuns &
\ind sexhǫfðaðan son.\eva

\bvb Webthrithner quoth: “Neath the hand\footnotemark[50] on the \inx{frost-thurse}, [they] said that a maiden and lad grew together. A foot against a foot begot, for the knowing ettin, a six-headed son."\evb
\footnotetext[50]{\emph{hęndi} (dative of \emph{hǫnd}) means ‘hand', but might here be a poetic circumlocution for ‘arm'.}\evg


\bvg {\small (Gagnráðr kvað:)}
\bva Sęg þat áttunda, \hld\ alls þik fróðan kveða, &
\ind ok þú Vafþrúðnir vitir, &
hvat fyrst of mant \hld\ eða fręmst of vęizt, &
\ind þú est alsviðr jǫtunn.\eva

\bvb Gainred quoth: “Say the eigth, since [they] call thee knowing, and thou, Webthrithner, might know: What dost thou first remember, or earliest know?\footnotemark[55] Thou art all-wise, ettin."\evb
\footnotetext[55]{Cf. Vsp 1.}\evg


\bvg {\small (Vafþrúðnir kvað:)}
\bva Ørófi vetra \hld\ áðr væri jǫrð of skǫpuð, &
\ind þá vas Bergęlmir borinn; &
þat fyrst um man'k, \hld\ es hinn fróði jǫtunn &
\ind á vas lúðr of lagiðr.\footnotemark[30]\eva
\footnotetext[30]{This verse is quoted in \emph{Gylf}.}

\bvb Webthrithner quoth: “Uncountable winters before the earth would be created, then Bear-yeller was born. \emph{That} I first remember, when the knowing ettin\footnotemark[60] was laid down on the funeral-bed\footnotemark[61]."\evb
\footnotetext[60]{That is, Bear-yeller. Cf. v. 29.}
\footnotetext[61]{\emph{lúðr}, a tricky word.}\evg


\bvg {\small (Gagnráðr kvað:)}
\bva Sęg þat níunda, \hld\ alls þik svinnan kveða, &
\ind ok þú Vafþrúðnir vitir, &
hvaðan vindr of kømr \hld\ svát fęrr vág yfir, &
\ind æ męnn hann sjalfan of séa.\eva

\bvb Gainred quoth: “Say the ninth, since [they] call thee wise, and thou, Webthrithner, might know: Whence the wind comes, so that he travels over the wave; forever men see him himself.\footnotemark[65]"\evb
\footnotetext[65]{Perhaps a negation has been lost here; the wind is never seen by men.}\evg


\bvg {\small (Vafþrúðnir kvað:)}
\bva Hræsvęlgr hęitir, \hld\ es sitr á himins ęnda, &
\ind jǫtunn í arnar ham; &
af hans vængjum \hld\ kveða vind koma &
\ind alla męnn yfir.\eva

\bvb Webthrithner quoth: “\inx{Corpse-swallower} [that one] is called, which sits at the end of the heavens, an ettin in the shape of an eagle. From his wings, they say [that] the wind comes over all men."\evb
\evg


\bvg {\small (Gagnráðr kvað:)}
\bva Sęg þat tíunda, \hld\ alls þú tíva rǫk &
\ind ǫll Vafþrúðnir vitir, &
hvaðan Njǫrðr of kom \hld\ með niðjum ása. &
Hófum ok hǫrgum \hld\ hann ræðr hundmǫrgum &
\ind ok varð-at hann ǫ́sum alinn.\eva

\bvb Gainred quoth: “Say the \emph{tenth}, since thou, Webthrithner, of all the fates of the \inx{Tues} might know: Whence \inx{Nearth} came into the company of the kinsmen of the \inx{Anses}? He rules an immense number\footnotemark[68] of \inx{hoves} and \inx{heargs}, and he was not begotten among the Anses."\evb
\footnotetext[68]{Lit. “he rules hound-many".}\evg


\bvg {\small (Vafþrúðnir kvað:)}
\bva Í Vanahęimi \hld\ skópu hann vís ręgin &
\ind ok sęldu at gíslingu goðum, &
í aldar rǫk \hld\ hann mun aptr koma &
\ind hęim með vísum vǫnum.\eva

\bvb Webthrithner quoth: “In \inx{Wane-Home}, the wise \inx{Reins}\footnotemark[69] created him, and sold him as a hostage to the gods. In the fate of the age, he will come back, home among the wise \inx{Wanes}."\evb
\footnotetext[69]{Though \emph{ręgin} usually serves as a direct synonym of \emph{goð} 'god(s)', it here seems to refer specifically to the Wanes, in contrast with the \inx{Eses} or gods.}\evg


\bvg {\small (Gagnráðr kvað:)}
\bva Sęg þat ęllipta, \hld\ hvar ýtar túnum í &
\ind hǫggvask hvęrjan dag; &
val þęir kjósa \hld\ ok ríða vígi frá, &
\ind sitja męir of sáttir saman.\footnotemark[35]\eva
\footnotetext[35]{This and the next v. are damaged in both R and 748; R has only this verse, but splits it in two (the 2nd starting with \emph{val}), while 748 has 40:1 (Ms.: \emph{S. þ. e. XI}) and then jumps to the answer v. 41. They have here been reconstructed, but it is possible some lines are still missing.}

\bvb Gainred quoth: “Say the eleventh, Where men in yards, hew away at each other every day? They choose those destined to die in war, and ride [away] from battle; [then] they sit more content together."\evb
\evg


\bvg {\small (Vafþrúðnir kvað:)}
\bva Allir ęinhęrjar \hld\ Óðins túnum í &
\ind hǫggvask hvęrjan dag, &
val þeir kjósa \hld\ ok ríða vígi frá, &
\ind sitja męir of sáttir saman.\eva

\bvb Webthrithner quoth: “In Weden's yards, all the \inx{Lone Warriors} hew away at each other every day. They choose those destined to die in war, and ride [away] from battle; [then] they sit more content together."\evb
\evg


\bvg {\small (Gagnráðr kvað:)}
\bva Sęg þat tolpta, \hld\ hví þú tíva rǫk &
\ind ǫll Vafþrúðnir vitir, &
frá jǫtna rúnum \hld\ ok allra goða &
\ind þú hit sannasta sęgir, &
\ind hinn alsvinni jǫtunn.\eva

\bvb Gainred quoth: “Say the twelfth, Why thou, Webthrithner, shouldst know all the fates of the \inx{Tues}\footnotemark[73]? From the \inx{runes} of the ettins and of all the gods, thou, the all-wise ettin, speakest most truly."\evb
\footnotetext[73]{The gods. Formation identical to \emph{ragna rǫk} ‘the fates of the Reins'.}\evg


\bvg {\small (Vafþrúðnir kvað:)}
\bva Frá jǫtna rúnum \hld\ ok allra goða &
\ind ek kann sęgja satt, &
þvíat hvęrn hęf'k \hld\ heim of komit, &
níu kom'k hęima \hld\ fyr niflhęl neðan; &
\ind hinig dęyja ór hęlju halir.\eva

\bvb Webthrithner quoth: “From the runes of the ettins and of all the gods I can speak truly, for I have been about each \inx{Home}. I was about nine Homes beneath Nivelhell; this way men die out of Hell\footnotemark[1]."\evb
\footnotetext[1]{A difficult verse. Finnur considers \emph{ór hęlju} “out of Hell” a later interpolation.}\evg


\bvg {\small (Gagnráðr kvað:)}
\bva Fjǫlð ek fór, \hld\ fjǫlð fręistaða'k, &
\ind fjǫlð ek ręynda ręgin; &
hvat lifir manna, \hld\ þá's hinn mæra líðr &
\ind fimbulvetr með firum?\eva

\bvb Gainred quoth: “Much I travelled, much I tried, much I tested the \inx{Reins}.\footnotemark[80] What remains\footnotemark[79] of men, when the famous \inx{fimbol-winter} passes among firs\footnotemark[81]?”\evb
\footnotetext[79]{Lit. “lives".}
\footnotetext[80]{Here begins the repetition of the same “mantra" used in v. 3, which continues until the final question (v. 54).}
\footnotetext[81]{Among men.}\evg


\bvg {\small (Vafþrúðnir kvað:)}
\bva Líf ok Lífþrasir, \hld\ ęn þau lęynask munu &
\ind í holti Hoddmímis; &
morgindǫggvar \hld\ þau sér at mat hafa; &
\ind þaðan af aldir alask.\eva

\bvb Webthrithner quoth: “Life and Lifethrasher, and they will hide themselves in the wood of Hoard-Mime\footnotemark[85]. Morning-dew they [will] have as food; thereof generations [will] be bred.”\evb
\footnotetext[85]{Prob. the same as Uggdrassle.}}\evg


\bvg {\small (Gagnráðr kvað:)}
\bva Fjǫlð ek fór, \hld\ fjǫlð fręistaða'k, &
\ind fjǫlð ek ręynda ręgin; &
hvaðan kømr sól \hld\ á hinn slétta himin, &
\ind es þessa hęfr Fęnrir farit?\eva

\bvb Gainred quoth: “Much I travelled, much I tried, much I tested the Reins. Whence comes sun onto the smooth heaven, when \inx{Fenner} has killed this one\footnotemark[1]?"\evb
\footnotetext[1]{That is, the current incarnation of the sun, as explained in the next v.}\evg


\bvg {\small (Vafþrúðnir kvað:)}
\bva Ęina dóttur \hld\ berr alfrǫðull, &
\ind áðr hana Fęnrir fari; &
sú skal ríða, \hld\ þá's ręgin dęyja, &
\ind móður brautir mær.\eva

\bvb Webthrithner quoth: “One daughter the elf-wheel <= Sun> bears, before \inx{Fenner} might kill her. When the Reins die, that one, the maiden, shall ride the paths of the mother.”\evb
\evg


\bvg {\small (Gagnráðr kvað:)}
\bva Fjǫlð ek fór, \hld\ fjǫlð fręistaða'k, &
\ind fjǫlð ek ręynda ręgin; &
hvęrjar 'ro męyjar, \hld\ es líða mar yfir, &
\ind fróðgęðjaðar fara.\eva

\bvb Weden quoth: “Much I travelled, much I tried, much I tested the Reins. Which are the maidens that pass over the ocean; wise-minded they go?”\evb
\evg


\bvg {\small (Vafþrúðnir kvað:)}
\bva Þríar þjóðár \hld\ falla þorp yfir &
\ind męyja Mǫgþrasis; &
hamingjur ęinar \hld\ þær’s í hęimi eru, &
\ind þó þær með jǫtnum alask.\eva

\bvb Webthrithner quoth: “Three great rivers fall over the settlement of the maidens of Maythrasher; the only Hamings that are in the Home,\footnotemark[1] though they are raised among the ettins\footnotemark[2]."\evb
\footnotetext[1]{Either in Ettinhome, or in the entire world.}
\footnotetext[2]{See index entry Maythrasher.}\evg


\bvg {\small (Gagnráðr kvað:)}
\bva Fjǫlð ek fór, \hld\ fjǫlð fręistaða'k, &
\ind fjǫlð ek ręynda ręgin; &
hvęrir ráða æsir \hld\ ęignum goða, &
\ind þá's sloknar Surtalogi?\eva

\bvb Gainred quoth: “Much I travelled, much I tried, much I tested the Reins. Which properties of the gods will the \inx{Anses} [still] rule\footnotemark[105], when the flame of \inx{Surt} burns out?"\evb
\footnotetext[105]{Or ‘control’.}
\evg


\bvg {\small (Vafþrúðnir kvað:)}
\bva Víðarr ok Váli \hld\ byggva vé goða, &
\ind þá's sloknar Surtalogi; &
Móði ok Magni \hld\ skulu Mjǫlni hafa &
\ind Vingnis at vígþroti.\eva

\bvb Webthrithner quoth: “\inx{Wider} and \inx{Weel} [will] inhabit the sanctuaries of the gods, when the \inx{flame of Surt} burns out. \inx{Mood} and \inx{Main} will own \inx{Meldner}, when \inx{Wingner} can no longer fight\footnotemark[110]."\evb
\footnotetext[110]{Lit. “at Wingner's fight-exhaustion", referring to his death.}
\evg


\bvg {\small (Gagnráðr kvað:)}
\bva Fjǫlð ek fór, \hld\ fjǫlð fręistaða'k, &
\ind fjǫlð ek ręynda ręgin; &
hvat verðr Óðni \hld\ at aldrlagi, &
\ind þá's rjúfask ręgin?\eva

\bvb Gainred quoth: “Much I travelled, much I tried, much I tested the Reins. What brings Weden’s life to an end, when the Reins are broken?"\evb
\evg


\bvg {\small (Vafþrúðnir kvað:)}
\bva Ulfr glęypa \hld\ mun Aldafǫðr, &
\ind þess mun Víðarr vreka; &
kalda kjapta \hld\ hann klyfja mun &
\ind vitnis vígi at.\eva

\bvb Webthrithner quoth: “The wolf will swallow \inx{Eldfather}; Wider will avenge that. He will cleave the cold jaws of the wolf at the battle."\evb
\evg


\bvg {\small (Gagnráðr kvað:) &
\bva Fjǫlð ek fór, \hld\ fjǫlð fręistaða'k, &
\ind fjǫlð ek ręynda ręgin; &
hvat mælti Óðinn, \hld\ áðr á bál stigi, &
\ind sjalfr í ęyra syni?\eva

\bvb Gainred quoth: “Much I travelled, much I tried, much I tested the Reins. What spoke Weden himself, before [he]\footnotemark[115] would step onto the funeral pyre, into the ear of the son?"\evb
\footnotetext[115]{Prob. Weden's son, that is \inx{Balder}.}\evg


\bvg {\small (Vafþrúðnir kvað:)}
\bva Ęy \edtext{manngi}{\Afootnote{manni \Regius\AM\ \emph{is impossible; we need a nominative here.}}} vęit, \hld\ hvat þú í árdaga &
\ind sagðir í ęyra syni; &
fęigum munni \hld\ mælta’k mína forna stafi &
\ind ok of ragna rǫk. &
Nú við Óðin \hld\ dęilda’k mína orðspęki; &
\ind þú est æ vísastr vera.\eva

\bvb Webthrithner quoth: “Ever no man knows, what thou in days of yore saidst in the ear of the son. With a death-doomed\footnoteB{Webthrithner here realizes that he was bound to die from the moment (v. 19) he proposed the wager, as no being can outwit Weden.} mouth I spoke my ancient utterings, and of the \inx{Rakes of the Reins}. Now with Weden I shared my word-wisdom\footnoteB{The same word-wisdom Weden in v. 5 set out to try.}; thou art ever wisest of beings\footnoteB{Word used is \emph{verr} ‘husband, man’. Perhaps in the broader sense of ‘male being’.}."\evb
\evg
% — Weden
	\bookStart{The Speeches of the High One}[Hávamǫ́l]

Introduction.
{\small The \textbf{Speeches of the High One} is the second poem of \Regius, which is also the only place where it is attested.} %TODO


\bvg Advice to wanderers.
\bva \alst{G}áttir allar \hld\ áðr \alst{g}angi framm &
\ind \edtext{of \alst{sk}oðask \alst{sk}yli,}{\lemma{of skoðask skyli}\Bfootnote{\emph{om.} \GylfMS}} &
\ind of \alst{sk}yggnask \alst{sk}yli; &
því’t ó\alst{v}íst’s at \alst{v}ita, \hld\ hvar ó\alst{v}inir &
\ind sitja á \alst{f}lęti \alst{f}yrir.\eva

\bvb All doorways—before one might go forth—should be watched, should be spied at; for uncertain it is to know where enemies sit on the benches inside.\evb
\evg


\bvg
\bva \alst{G}efęndr hęilir, \hld\ \alst{g}ęstr’s inn kominn, &
\ind hvar skal \alst{s}itja \alst{s}já? &
mjǫk es \alst{b}ráðr \hld\ sá’s á \alst{b}rǫndum skal &
\ind síns of \alst{f}ręista \alst{f}rama.\eva

\bvb Hail the givers\footnoteB{The hosts.}! A guest is come in; where shall this one sit? Greatly hurried is he who on the fires\footnoteB{According to \Finnur\ referring a Norwegian folk custom, wherein a guest would sit down on the wood-pile, waiting until being called in. See further TODO.} shall try his fame.\evb
\evg


\bvg
\bva \alst{Ę}lds es þǫrf \hld\ þęim’s \alst{i}nn es kominn &
\ind ok á \alst{k}néi \alst{k}alinn, &
\alst{m}atar ok váða \hld\ es \alst{m}anni þǫrf, &
\ind þęim’s hęfr umb \alst{f}jall \alst{f}arit.\eva

\bvb Of fire is there need for the one who is come in and cold about the knees; of food and clothing is there need for the man who over the fell has fared.\evb
\evg


\bvg
\bva \alst{V}ats es þǫrf \hld\ þęim’s til \alst{v}erðar kømr, &
\ind \alst{þ}ęrru ok \alst{þ}jóðlaðar, &
\alst{g}óðs of ǿðis, \hld\ —ef sér \alst{g}eta mǽtti— &
\ind \alst{o}rðs ok \alst{ę}ndrþǫgu.\eva

\bvb Of water\footnoteB{i.e. for washing oneself.} is there need for the one who comes for a meal, a towel and a good welcome; a kind reception—if he might get one—speech, and silence in return.\evb
\evg


\bvg
\bva \alst{V}its es þǫrf \hld\ þęim’s \alst{v}íða ratar; &
\ind dǽlt es \alst{h}ęima \alst{h}vat; &
at \alst{au}gabragði \hld\ verðr sá’s \alst{ę}kki kann &
\ind ok með \alst{s}notrum \alst{s}itr.\eva

\bvb Of wits is there need for the one who widely roams; all is familiar at home. A laughing-stock\footnoteB{An idiom, \emph{augabragð} lit. ‘twinkling of an eye, moment’.} becomes he who nothing knows, and among the clever sits.\evb
\evg


\bvg
\bva At \alst{h}yggjandi sinni \hld\ skyli-t maðr \alst{h}rǿsinn vesa, &
\ind hęldr \alst{g}ǽtinn at \alst{g}ęði, &
þá’s \alst{h}orskr ok þǫgull \hld\ kømr \alst{h}ęimisgarða til, &
\ind sjaldan verðr \alst{v}íti \alst{v}ǫrum. &
\edtext{því’t \alst{ó}brigðra vin \hld\ fǽr þú \alst{a}ldrigi, &
\ind an \alst{m}anvit \alst{m}ikit.}{\lemma{því ... mikit}\Bfootnote{The shift in person from third to second, along with the abnormal verse length (six lines instead of four), indicates that this is an insertion.}}\eva

\bvb Of his thinking should man not be boastful; rather guarding of his senses, when sharp and silent he comes to a homestead; sudden injury seldom strikes the wary, (for thou gettest never an unfickler friend, than much \inx{manwit}[C].)\evb
\evg


\bvg
\bva Hinn \alst{v}ari gęstr, \hld\ es til \alst{v}erðar kømr, &
\ind \alst{þ}unnu hljóði \alst{þ}ęgir; &
\alst{ęy}rum hlýðir, \hld\ ęn \alst{au}gum skoðar, &
\ind svá nýsisk \alst{f}róðra hvęrr \alst{f}yrir.\eva

\bvb The wary guest, when he comes for a meal, with thin heed is silent.\footnoteB{i.e. “is in attentive silence”.} With ears he heeds, but with eyes observes; so pries each learned man about.\evb
\evg


\bvg
\bva Hinn es \alst{s}ǽll, \hld\ es \alst{s}ér of getr &
\ind \alst{l}of ok \alst{l}íknstafi; &
\alst{ó}dǽlla es við þat, \hld\ es \alst{ęi}ga skal &
\ind \alst{a}nnars brjóstum \alst{í}.\eva

\bvb The one is fortunate, who for himself gets praise and staves of grace. Uneasy is it regarding that which one must own, in another man’s breast.\evb
\evg


\bvg
\bva \alst{S}á es \alst{s}ǽll, \hld\ es \alst{s}jalfr of á &
\ind \alst{l}of ok vit meðan \alst{l}ifir; &
því’t \alst{i}ll rǫ́ð \hld\ hęfr maðr \alst{o}pt þęgit &
\ind \alst{a}nnars brjóstum \alst{ó}r.\eva

\bvb That one is fortunate, who himself owns praise and wits while he lives; for ill counsels has man oft taken, out of another man’s breast.\evb
\evg


\bvg
\bva \alst{B}yrði \alst{b}ętri \hld\ berr-at maðr \alst{b}rautu at, &
\ind an sé \alst{m}anvit \alst{m}ikit; &
\alst{au}ði bętra \hld\ þykkir þat í \alst{ó}kunnum stað; &
\ind slíkt es \alst{v}álaðs \alst{v}era.\eva

\bvb A better burden bears man not on the road than much manwit. In an unknown place it seems better than wealth; such is the refuge of the wretched.\evb
\evg


\bvg
\bva \alst{B}yrði \alst{b}ętri \hld\ berr-at maðr \alst{b}rautu at, &
\ind an sé \alst{m}anvit \alst{m}ikit; &
\alst{v}egnest \alst{v}erra \hld\ \alst{v}egr-a \alst{v}ęlli at, &
\ind an sé \alst{o}fdrykkja \alst{ǫ}ls.\eva

\bvb A better burden bears man not on the road than much manwit. Worse provision is not dragged along on the plain\footnoteB{\emph{vǫllr} ‘plain, (uncultivated) field’ is repeated in vv. 38 and 49. It is easily seen that the heaths and plains of Iron Age Norway were particularly unsafe places, where a traveller needed to keep his wits with him lest he fall victim to robbers or murderers.} than a too great drink of ale.\evb
\evg


\bvg
\bva Es-a svá \alst{g}ótt, \hld\ sęm \alst{g}ótt kveða, &
\ind \alst{ǫ}l \alst{a}lda sonum; &
því’t \alst{f}ǽra vęit, \hld\ es \alst{f}lęira drekkr, &
\ind síns til \alst{g}ęðs \alst{g}umi.\eva

\bvb It is not as good, as good they sing, ale for the sons of men; for the less he knows, as the more he drinks, man of his own senses.\evb
\evg


\bvg
\bva \alst{Ó}minnishegri hęitir, \hld\ sá’s yfir \alst{ǫ}lðrum þrumir, &
\ind hann stelr \alst{g}ęði \alst{g}uma; &
þess \alst{f}ogls \alst{f}jǫðrum \hld\ ek \alst{f}jǫtraðr vas’k &
\ind í \alst{g}arði \alst{G}unnlaðar.\eva

\bvb The heron of forgetfulness is called he who above ale-feasts hovers;\footnoteB{Here drunkenness is personified as a bird, a “heron of forgetfulness”.} he robs men of their senses. With that bird’s feathers I was fettered, in the yards of Guthlathe.\evb
\evg


\bvg
\bva \alst{Ǫ}lr ek varð, \hld\ varð \alst{o}frǫlvi, &
\ind at hins \alst{f}róða \alst{F}jalars; &
því es \alst{ǫ}lðr bazt, \hld\ at \alst{a}ptr of hęimtir &
\ind hvęrr sitt \alst{g}ęð \alst{g}umi.\eva

\bvb Drunk I became—I became the drunkest by far—at the learned Fealer’s [abode]. Thus is an ale-feast best, as each man recovers his senses.\evb
\evg


\bvg
\bva \alst{Þ}agalt ok hugalt \hld\ skyli \alst{þ}jóðans barn &
\ind ok \alst{v}ígdjarft \alst{v}esa; &
\alst{g}laðr ok ręifr \hld\ skyli \alst{g}umna hvęrr, &
\ind unz sinn \alst{b}íðr \alst{b}ana.\eva

\bvb Silent and thoughtful should the ruler’s child be, and bold in battle. Glad and cheerful should each man be, until he suffer his bane.\evb
\evg


\bvg
\bva \alst{Ó}snjallr maðr \hld\ hyggsk munu \alst{ę}y lifa, &
\ind ef við \alst{v}íg \alst{v}arask; &
ęn \alst{ę}lli gefr hǫ́num \hld\ \alst{ę}ngi frið, &
\ind þótt hǫ́num \alst{g}ęirar \alst{g}efi.\eva

\bvb The unvalorous man thinks he will ever live, if he of war is wary; but old age gives him no peace, although spears would.\evb
\evg


\bvg
\bva \alst{K}ópir afglapi, \hld\ es til \alst{k}ynnis \alst{k}ømr, &
\ind \alst{þ}ylsk hann umb eða \alst{þ}rumir; &
alt es \alst{s}ęnn, \hld\ ef \alst{s}ylg of getr, &
\ind uppi es þá \alst{g}ęð \alst{g}uma.\eva

\bvb Gapes the oaf when to visit he comes; he mumbles about or loiters. All at once—if a sip he gets—are the senses of the man exposed.\evb
\evg


\bvg
\bva Sá ęinn \alst{v}ęit, \hld\ es \alst{v}íða ratar &
\ind ok hęfr \alst{f}jǫlð of \alst{f}arit, &
hvęrju \alst{g}ęði \hld\ stýrir \alst{g}umna hvęrr, &
\ind sá es \alst{v}itandi’s \alst{v}its.\eva

\bvb He alone knows, who widely roams, and has travelled much: his own senses does each man control, who is aware of his wits.\evb
\evg


\bvg
\bva \alst{H}aldi-t maðr á kęri, \hld\ drekki þó at \alst{h}ófi mjǫð, &
\ind mǽli \alst{þ}arft eða \alst{þ}ęgi; &
\alst{ó}kynnis þess \hld\ váar þik \alst{ę}ngi maðr, &
\ind at gangir \alst{s}nimma at \alst{s}ofa.\eva

\bvb Man ought not to hold onto the cask, yet drink a fitting serving of mead; he ought to speak the needful or be silent.\footnoteB{Identical to a certain verse in \Vafthrudnismal\ TODO: which one} For that uncouthness will no man blame thee, that thou go early to sleep.\evb
\evg


\bvg
\bva \alst{G}rǫ́ðugr halr, \hld\ nema \alst{g}ęðs viti, &
\ind \alst{e}tr sér \alst{a}ldrtrega; &
opt fǽr \alst{h}lǿgis, \hld\ es með \alst{h}orskum kømr, &
\ind \alst{m}anni hęimskum \alst{m}agi.\eva

\bvb The gluttonous man—unless he know his senses—eats himself a life-sorrow. Oft the belly—when among the sharp he comes—brings a foolish man ridicule.\evb
\evg


\bvg
\bva \alst{H}jarðir þat vitu, \hld\ nǽr \alst{h}ęim skulu, &
\ind ok \alst{g}anga þá af \alst{g}rasi; &
ęn \alst{ó}sviðr maðr \hld\ kann \alst{ǽ}vagi &
\ind síns of \alst{m}ál \alst{m}aga.\eva

\bvb Herds know when home they must [turn], and then part from the grass; but an unwise man never knows the measure of his own belly.\evb
\evg


\bvg
\bva \alst{V}esall maðr \hld\ ok \alst{i}lla skapi &
\ind \alst{h}lǽr at \alst{h}vívetna; &
hitki hann \alst{v}ęit, \hld\ es \alst{v}ita þyrpti, &
\ind at hann es-a \alst{v}amma \alst{v}anr.\eva

\bvb The wretched man, and the ill-spirited, laughs at whatever. He knows not that which he might need to know: he is not free of blemishes.\evb
\evg


\bvg
\bva \alst{Ó}sviðr maðr \hld\ vakir umb \alst{a}llar nǽtr &
\ind ok \alst{h}yggr at \alst{h}vívetna; &
þá es \alst{m}óðr, \hld\ es at \alst{m}orni kømr; &
\ind alt es \alst{v}íl sęm \alst{v}as.\eva

\bvb The unwise man is awake for all nights, and thinks of whatever. Then he is weary when the morning comes; his trouble is all as it was.\evb
\evg


\bvg
\bva \alst{Ó}snotr maðr \hld\ hyggr sér \alst{a}lla vesa &
\ind \alst{v}iðhlǽjęndr \alst{v}ini; &
hitki hann \alst{f}iðr, \hld\ þótt þęir of hann \alst{f}ár lesi, &
\ind ef með \alst{s}notrum \alst{s}itr.\eva

\bvb The unclever man thinks all who laugh with him\footnoteB{lit. ‘with-laughers, mutal laughers’.} his friends. He finds it not, though they speak poorly of him, if among the clever he sits.\evb
\evg


\bvg
\bva \alst{Ó}snotr maðr \hld\ hyggr sér \alst{a}lla vesa &
\ind \alst{v}iðhlǽjęndr \alst{v}ini; &
\alst{þ}á þat fiðr \hld\ es at \alst{þ}ingi kømr, &
\ind at á \alst{f}ormǽlęndr \alst{f}áa.\eva

\bvb The unclever man thinks all who laugh with him his friends. Then he finds, when to the \inx{Thing}[C] he comes, that he has spokesmen\footnoteB{Men ready to take his side.} few.\evb
\evg


\bvg
\bva \alst{Ó}snotr maðr \hld\ þykkisk \alst{a}lt vita, &
\ind ef á sér i \alst{v}ǫ́ \alst{v}eru; &
hitki hann \alst{v}ęit, \hld\ hvat hann skal \alst{v}ið kveða, &
\ind ef hans \alst{f}ręista \alst{f}irar.\eva

\bvb The unclever man seems to know everything, if he takes refuge in a nook. He knows it not, what he shall say in return if men test him.\evb
\evg


\bvg
\bva \alst{Ó}snotr maðr, \hld\ es með \alst{a}ldir kømr, &
\ind \alst{þ}at’s bazt at hann \alst{þ}ęgi; &
\alst{ę}ngi þat vęit, \hld\ at hann \alst{ę}kki kann, &
\ind nema hann \alst{m}ǽli til \alst{m}art. &
\alst{v}ęit-a maðr, \hld\ hinn’s \alst{v}ǽtki vęit, &
\ind þótt hann \alst{m}ǽli til \alst{m}art.\eva

\bvb The unclever man, when among people he comes—it is best that he is silent. None knows that he nothing knows, unless he speak too much. (Man knows not, who nothing knows, although he speak too much.\footnoteB{That is, mindless speech will not make him any wiser.})\evb
\evg


\bvg
\bva \alst{F}róðr sá þykkisk, \hld\ es \alst{f}regna kann, &
\ind ok \alst{s}ęgja hit \alst{s}ama, &
\alst{ęy}vitu lęyna \hld\ męgu \alst{ý}ta synir &
\ind því es \alst{g}ęngr umb \alst{g}uma.\eva

\bvb Learned seems he, who can ask and answer the same. Naught may the sons of men conceal, of that\footnoteB{Rumours and gossip.} which goes about a man.\evb
\evg


\bvg
\bva \alst{Ǿ}rna mǽlir, \hld\ sá’s \alst{ǽ}va þęgir, &
\ind \alst{st}aðlausu \alst{st}afi; &
\alst{h}raðmǽlt tunga, \hld\ nema \alst{h}aldęndr ęigi, &
\ind opt sér ó\alst{g}ótt of \alst{g}ęlr.\eva

\bvb Quite enough speaks he, who is never silent, utterings of absurdity. A quick-spoken tongue—unless it be held in place\footnoteB{lit. ‘unless holders own it’ or ‘unless it own holders’.}—oft sings evil [into being] for itself.\evb
\evg


\bvg
\bva At \alst{au}gabragði \hld\ skal-a maðr \alst{a}nnan hafa, &
\ind \edtext{þótt}{\lemma{þótt “although”}\Bfootnote{Perhaps an error? \emph{es} ‘when’ would surely work better in context.}} til \alst{k}ynnis \alst{k}omi; &
margr \alst{f}róðr þykkisk, \hld\ ef hann \alst{f}reginn es-at &
\ind ok nái hann \alst{þ}urrfjallr \alst{þ}ruma.\eva

\bvb As a laughing-stock shall man not have another, although he come to visit. Many a man seems learned if he is not asked, and manages to loiter about dry-skinned.\footnoteB{This sense of \emph{fjall} is apparently almost non-existent in Old Norse literature, but compare Swedish \emph{fjäll} ‘scale (on fish and reptiles)’. The meaning is in any case figurative, equivalent to the English “get one’s feet wet”.}\evb
\evg


\bvg
\bva \alst{F}róðr þykkisk \hld\ sá’s \alst{f}lótta tękr &
\ind \alst{g}ęstr at \alst{g}ęst hǽðinn; &
\alst{v}ęit-a gǫrla \hld\ sá’s of \alst{v}erði glissir, &
\ind þótt með \alst{g}rǫmum \alst{g}lami.\eva

\bvb Learned seems he who takes to flight,\footnoteB{Probably not literally, rather “pulls back, does not take part”.} when a guest at a guest is scoffing. He knows not clearly, who grins above the food, that he with fiends be prattling.\evb
\evg


\bvg
\bva \alst{G}umnar margir \hld\ erusk \alst{g}agnhollir, &
\ind ęn at \alst{v}irði \alst{v}rekask; &
\alst{a}ldar róg \hld\ þat mun \alst{ǽ} vesa; &
\ind órir \alst{g}ęstr við \alst{g}ęst.\eva

\bvb Many men are loyal to each other, but over a meal drive each other away. The strife of mankind will that ever be; guest raves against guest.\evb
\evg


\bvg
\bva \alst{Á}rliga verðar \hld\ skyli maðr \alst{o}pt fáa, &
\ind nema til \alst{k}ynnis \alst{k}omi; &
\alst{s}itr ok \alst{s}nópir, \hld\ lǽtr sęm \alst{s}olginn sé, &
\ind ok kann \alst{f}regna at \alst{f}ǫ́u.\eva

\bvb An early meal should man oft get, unless he come to visit; he sits and idles haplessly, makes as if starved, and can ask about little.\evb
\evg


\bvg
\bva \alst{A}fhvarf mikit \hld\ es til \alst{i}lls vinar, &
\ind þótt á \alst{b}rautu \alst{b}úi, &
ęn til \alst{g}óðs vinar \hld\ liggja \alst{g}agnvegir, &
\ind þótt hann sé \alst{f}irr \alst{f}arinn.\eva

\bvb A great detour is it to a wicked friend, though he on the highway live; but to a good friend lie the shortest ways, though he far gone be.\evb
\evg


\bvg
\bva \alst{G}anga skal, \hld\ skal-a \alst{g}ęstr vesa &
\ind \alst{ęy} í \alst{ęi}num stað; &
\alst{l}júfr verðr \alst{l}ęiðr, \hld\ ef \alst{l}ęngi sitr &
\ind \alst{a}nnars flętjum \alst{á}.\eva

\bvb Go one shall; shall not be a guest forever in one place. The beloved becomes loathed if long he sits, on another man’s benches.\evb
\evg


\bvg
\bva \alst{B}ú es \alst{b}ętra, \hld\ þótt lítit sé, &
\ind \alst{h}alr es \alst{h}ęima \alst{h}vęrr; &
þótt \alst{t}vǽr gęitr ęigi \hld\ ok \alst{t}augręptan sal, &
\ind þat es þó \alst{b}ętra an \alst{b}ǿn.\eva

\bvb A dwelling is better, though small it be: each is a man at home. Though two goats he own, and a cord-roofed hall, that is yet better than begging.\evb
\evg


\bvg
\bva \alst{B}ú es \alst{b}ętra, \hld\ þótt lítit sé, &
\ind \alst{h}alr es \alst{h}ęima \alst{h}vęrr; &
\alst{b}lóðugt es hjarta \hld\ þęim’s \alst{b}iðja skal &
\ind sér í \alst{m}ál hvęrt \alst{m}atar.\eva

\bvb A dwelling is better, though small it be: each is a man at home. Bloody is the heart of the one who must beg for himself each meal of food.\evb
\evg


\bvg
\bva \alst{V}ǫ́pnum sínum \hld\ skal-a maðr \alst{v}ęlli á &
\ind \alst{f}eti ganga \alst{f}ramar; &
því’t ó\alst{v}íst’s at \alst{v}ita, \hld\ nǽr verðr á \alst{v}egum úti &
\ind \alst{g}ęirs of þǫrf \alst{g}uma.\eva

\bvb From his weapons shall man on the plain not take a footstep further; for uncertain it is to know, when on the ways outside, man comes in need of a spear.\evb
\evg


\bvg
\bva Fann-k-a \alst{m}ildan mann \hld\ eða svá \alst{m}atar góðan, &
\ind at vǽri-t \alst{þ}iggja \alst{þ}ęgit; &
eða \alst{s}íns féar \hld\ \edtext{\alst{s}vági [...]}{\lemma{svági [...]}\Bfootnote{It is doubtless that a word has been lost here; the meter and sense require it. \Finnur\ inserts \emph{gløggvan} ‘miserly, stingy’ and this may very well be correct.}}, &
\ind at \alst{l}ęið sé \alst{l}aun, ef þegi.\eva

\bvb I found not a generous man, nor one so good of meat,\footnoteB{\emph{matar góðr} ‘good of meat, food’ is an old expression appearing in several Runic inscriptions, such as Sm 39: \emph{mildan orða ok mataʀ góðan} “mild of words and good of meat”, U 805: \emph{bónda góðan matar} “a farmer good of meat”, U 703: \emph{mandr matar góðr auk málsrisinn} “a man good of meat and gallant in speech”. Compare also U 739: \emph{hann vaʀ mildr mataʀ auk málsrisinn} “he was mild (i.e. generous) of meat and bold in speech.”} that a gift was not received; nor one of his wealth so [...], that the reward was loathed, if he received it.\evb
\evg


\bvg
\bva \alst{F}éar síns, \hld\ es \alst{f}ęngit hęfr, &
\ind skyli-t maðr \alst{þ}ǫrf \alst{þ}ola; &
opt sparir \alst{l}ęiðum \hld\ þat’s hęfr \alst{l}júfum hugat; &
\ind mart gęngr \alst{v}err an \alst{v}arir.\eva

\bvb Of his own \inx{fee}[C], which he has earned, should man not suffer need. Oft one saves for the loathed what was meant for the loved; much goes worse than one expects.\evb
\evg


\bvg
\bva \alst{V}ǫ́pnum ok \alst{v}ǫ́ðum \hld\ skulu \alst{v}inir glęðjask; &
\ind þat’s á \alst{s}jǫlfum \alst{s}ýnst; &
\alst{v}iðrgefęndr \hld\ erusk \alst{v}inir lęngst, &
\ind ef þat bíðr at \alst{v}erða \alst{v}el.\eva

\bvb With weapons and garments shall friends gladden each other; that is most seen on oneself.\footnoteB{In one’s own experience.} Mutual givers are friends for the longest, if it\footnoteB{The friendship.} comes to last long.\evb
\evg


\bvg
\bva \alst{V}in sínum \hld\ skal maðr \alst{v}inr \alst{v}esa, &
\ind ok \alst{g}jalda \alst{g}jǫf við \alst{g}jǫf; &
\alst{h}látr við \alst{h}látri \hld\ skyli \alst{h}ǫlðar taka, &
\ind ęn \alst{l}ausung við \alst{l}ygi.\eva

\bvb With his friend shall man be a friend, and reward gift with gift; laughter with laughter should men take, but a lie with duplicity.\evb
\evg


\bvg
\bva \alst{V}in sínum \hld\ skal maðr \alst{v}inr vesa, &
\ind \alst{þ}ęim ok \alst{þ}ess vin; &
ęn \alst{ó}vinar síns \hld\ skyli \alst{ę}ngi maðr &
\ind \alst{v}inar \alst{v}inr \alst{v}esa.\eva

\bvb With his friend shall man be a friend, with him and his friend; but with his enemy’s, should no man, friend’s friend be.\evb
\evg


\bvg
\bva \alst{V}ęizt, ef þú \alst{v}in átt, \hld\ þann’s þú \alst{v}el trúir &
\ind ok vilt af hǫ́num \alst{g}ótt \alst{g}eta, &
\alst{g}ęði skalt við þann \hld\ ok \alst{g}jǫfum skipta, &
\ind \alst{f}ara at \alst{f}inna opt.\eva

\bvb Know: if thou hast a friend, whom thou trustest well and wilt receive good from: thoughts shalt thou exchange with him, and gifts; travel to see him oft.\evb
\evg


\bvg
\bva Ef þú \alst{á}tt \alst{a}nnan, \hld\ þann’s þú \alst{i}lla trúir, &
\ind vild-u af hǫ́num þó \alst{g}ótt \alst{g}eta, &
\alst{f}agrt skalt mǽla, \hld\ ęn \alst{f}látt hyggja &
\ind ok gjalda \alst{l}ausung við \alst{l}ygi.\eva

\bvb If thou have another, whom thou trust little, and wilt yet receive good from: fairly shalt thou speak, but falsely think, and reward lie with duplicity.\evb
\evg


\bvg
\bva Þat’s \alst{ę}nn umb þann, \hld\ es þú \alst{i}lla trúir &
\ind ok þér es \alst{g}runr at \alst{g}ęði, &
\alst{h}lǽja skalt við þęim \hld\ ok of \alst{h}ug mǽla; &
\ind \alst{g}lík skulu \alst{g}jǫld \alst{g}jǫfum.\eva

\bvb It is yet regarding that one, whom thou poorly trustest, and causest thy senses doubt\footnoteB{lit. “and for thee is doubt in senses”.}: laugh shalt thou with him, and speak with care; rewards shall be equal to gifts.\footnoteB{Equivalent to the last line of the previous v. (“reward a lie with duplicity”).}\evb
\evg


\bvg
\bva Ungr vas’k \alst{f}orðum, \hld\ \alst{f}ór’k ęinn saman, &
\ind þá varð’k \alst{v}illr \alst{v}ega; &
\alst{au}ðigr þóttumk, \hld\ es \alst{a}nnan fann’k, &
\ind \alst{m}aðr es \alst{m}anns gaman.\eva

\bvb Young was I once; I travelled alone; then I got lost about the ways. Wealthy I thought myself when another I found; man is the joy of man.\evb
\evg


\bvg
\bva \alst{M}ildir frǿknir \hld\ \alst{m}ęnn bazt lifa, &
\ind \alst{s}jaldan \alst{s}út ala; &
\alst{ó}snjallr maðr \hld\ \alst{u}ggir hvatvetna, &
\ind sýtir ǽ \alst{g}løggr við \alst{g}jǫfum.\eva

\bvb Generous, bold men live the best; seldom they nourish sorrow. The unvalorous man is frightened by whatever; ever the stingy man laments at gifts.\footnoteB{Refer back to v. 39; after receiving a gift, one was culturally obliged to give something back.}\evb
\evg


\bvg
\bva \alst{V}áðir mínar \hld\ gaf’k \alst{v}ęlli at &
\ind \alst{t}vęim \alst{t}rémǫnnum; &
\alst{r}ekkar þat þóttusk, \hld\ es \alst{r}ipt hǫfðu; &
\ind \alst{n}ęiss es \alst{n}ǫkkviðr halr.\eva

\bvb My garments I gave at the plain, to two tree-men.\footnoteB{TODO: Note on their identity. Aniconic wooden statues? Scarecrows? What do previous authors write?} Champions they seemed when cloaks they had; shameful is the naked man.\evb
\evg


\bvg
\bva Hrørnar \alst{þ}ǫll, \hld\ sú’s stęndr \alst{þ}orpi á, &
\ind hlýrat hęnni \alst{b}ǫrkr né \alst{b}arr; &
svá es \alst{m}aðr, \hld\ sá’s \alst{m}anngi ann; &
\ind hvat skal hann \alst{l}ęngi \alst{l}ifa?\eva

\bvb Wilters the pine that stands on the yard; shields her not bark nor needle. So is the man who loves none; why shall he live long?\evb
\evg


\bvg
\bva \alst{Ę}ldi hęitari \hld\ brinnr með \alst{i}llum vinum &
\ind \alst{f}riðr \alst{f}imm daga, &
ęn þá \alst{sl}oknar, \hld\ es hinn \alst{s}étti kømr, &
\ind ok \alst{v}ersnar allr \alst{v}inskapr.\eva

\bvb Hotter than fire burns with wicked friends, the peace for five days;\footnoteB{As \Finnur\ points out, a reference to the five-day week; the number is symbolic.} but then goes out when the sixth one comes, and all the friendship worsens.\evb
\evg


\bvg
\bva \alst{M}ikit ęitt \hld\ skal-a \alst{m}anni gefa; &
\ind opt kaupir sér í \alst{l}ítlu \alst{l}of, &
með \alst{h}ǫlfum \alst{h}lęif \hld\ ok með \alst{h}ǫllu kęri &
\ind \alst{f}ekk ek mér \alst{f}élaga.\eva

\bvb Much at once shall one not give a man; oft one buys praise for little. With half a loaf and an awry cask, I got me a companion.\evb
\evg


\bvg
\bva \alst{L}ítilla sanda, \hld\ \alst{l}ítilla sǽva, &
\ind lítil eru \alst{g}ęð \alst{g}uma; &
því’t \alst{a}llir męnn \hld\ \alst{u}rðu-t jafnspakir; &
\ind \alst{h}ǫlf es ǫld \alst{h}var.\eva

\bvb Of small sands, of small seas; small are the senses of man. For all have not become evenly wise; half is each man.\footnoteB{Where shores are small, seas are small. Compared to the power of the natural forces man is but a grain of sand in the desert, a drop of water in the sea. His wisdom will always be incomplete.}\evb
\evg


\bvg
\bva \alst{M}eðalsnotr \hld\ skyli \alst{m}anna hvęrr, &
\ind ǽva til \alst{s}notr \alst{s}é; &
þęim es \alst{f}yrða \hld\ \alst{f}ęgrst at lifa, &
\ind es \alst{v}el mart \alst{v}itu.\eva

\bvb Middle-clever should each man be; never too clever. For those men is it fairest to live, who know well enough.\evb
\evg


\bvg
\bva \alst{M}eðalsnotr \hld\ skyli \alst{m}anna hvęrr, &
\ind ǽva til \alst{s}notr \alst{s}é; &
\alst{s}notrs manns hjarta \hld\ verðr \alst{s}jaldan glatt, &
\ind ef sá’s \alst{a}lsnotr es \alst{á}.\eva

\bvb Middle-clever should each man be; never too clever. The clever man’s heart turns seldom glad, if he is all-clever that owns it.\evb
\evg


\bvg
\bva \alst{M}eðalsnotr \hld\ skyli \alst{m}anna hvęrr, &
\ind ǽva til \alst{s}notr \alst{s}é; &
\alst{ø}rlǫg sín \hld\ viti \alst{ę}ngi fyrir; &
\ind þęim es \alst{s}orgalausastr \alst{s}efi.\eva

\bvb Middle-clever should each man be; never too clever. May no man know his \inx{orlay}[C] ahead; his is the most sorrowless mind.\footnoteB{Who knows not his fate. One may contrast Weden who has knowledge of his own inevitable doom.}\evb
\evg


\bvg
\bva \alst{B}randr af \alst{b}randi \hld\ \alst{b}rinnr unz \alst{b}runninn es, &
\ind \alst{f}uni kvęykisk af \alst{f}una; &
\alst{m}aðr af \alst{m}anni \hld\ verðr at \alst{m}áli kuðr; &
\ind ęn til \alst{d}ǿlskr af \alst{d}ul.\eva

\bvb Fire by fire burns until it burnt is; flame is kindled by flame. Man by man becomes known from speech, but the too dull by his conceit.\evb
\evg


\bvg
\bva \alst{Á}r skal rísa, \hld\ sá’s \alst{a}nnars vill &
\ind \alst{f}é eða \alst{f}jǫr hafa; &
sjaldan \alst{l}iggjandi ulfr \hld\ \alst{l}ǽr of getr, &
\ind né \alst{s}ofandi maðr \alst{s}igr.\eva

\bvb Early shall he rise, who another man’s \inx{fee}[C] or life will have. Seldom does the lying wolf get a thigh, or the sleeping man victory.\evb
\evg


\bvg
\bva \alst{Á}r skal rísa, \hld\ sá’s á \alst{y}rkjęndr fáa, &
\ind ok ganga síns \alst{v}erka á \alst{v}it; &
\alst{m}art of dvęlr \hld\ þann’s umb \alst{m}orgin sefr, &
\ind \alst{h}alfr es auðr und \alst{h}vǫtum.\eva

\bvb Early shall he rise, who owns workers few, and go his work to meet. Much is kept back from him who in the morning sleeps; half the wealth is due to the brisk.\footnoteB{Half of a man’s wealth is due to his briskness.}\evb
\evg


\bvg
\bva \alst{Þ}urra skíða \hld\ ok \alst{þ}akinna nǽfra, &
\ind þess kann \alst{m}aðr \alst{m}jǫt, &
ok þess \alst{v}iðar, \hld\ es \alst{v}innask męgi &
\ind \alst{m}ál ok \alst{m}issęri.\eva

\bvb Of dry planks and thatching birch bark: of that man knows the measure—and of that firewood, which may be used for a season and half-year.\footnoteB{Over the winter.}\evb
\evg


\bvg
\bva \alst{Þ}vęginn ok męttr \hld\ ríði maðr \alst{þ}ingi at, &
\ind þótt hann sé-t \alst{v}ǽddr til \alst{v}el; &
\alst{sk}úa ok bróka \hld\ \alst{sk}ammisk ęngi maðr &
\ind né \alst{h}ęsts in \alst{h}ęldr,
\ind \edtext{þótt hann \alst{h}afi’t góðan}{\lemma{þótt ... góðan “Although ... good one”}\Afootnote{As \Finnur\ points out, surely a later insertion. The insertor seems to have attempted a \Fornyrdislag\ B-verse, but this cannot work.}}.\eva

\bvb Washed and filled ought man to ride to the Thing, though he be not dressed too well; of his shoes and breeches ought no man to be ashamed, nor indeed of his horse, (although he has not a good one.)\evb
\evg


\bvg
\bva \alst{S}napir ok gnapir, \hld\ es til \alst{s}ǽvar kømr, &
\ind \alst{ǫ}rn á \alst{a}ldinn mar; &
svá es \alst{m}aðr, \hld\ es með \alst{m}ǫrgum kømr &
\ind ok á \alst{f}ormǽlęndr \alst{f}áa.\eva

\bvb Shuffles and stoops, when to the sea it comes, the eagle on the aged ocean. So is the man, who among the many comes, and has spokesmen few.\evb
\evg


\bvg
\bva \alst{F}regna ok sęgja \hld\ skal \alst{f}róðra hvęrr, &
\ind sá’s vill \alst{h}ęitinn \alst{h}orskr; &
\alst{ęi}nn vita \hld\ né \alst{a}nnarr skal, &
\ind \alst{þ}jóð vęit ef \alst{þ}rír ’ró.\eva

\bvb Ask and speak shall each learned man, who wishes to be called sharp; one shall know, but another not: thirty\footnoteB{\emph{þjóð} lit. ‘people, nation’; cf. \Skaldskaparmal\ (\GudniEdda\ p. 241): \emph{þjóð eru þrír tigir} “thirty are a \emph{people}”.} know if there are three.\evb
\evg


\bvg
\bva \alst{R}íki sitt \hld\ skyli \alst{r}áðsnotra &
\ind hvęrr í \alst{h}ófi \alst{h}afa; &
þá hann þat \alst{f}innr, \hld\ es með \alst{f}rǿknum kømr, &
\ind at \alst{ę}ngi es \alst{ęi}nna hvatastr.\eva

\bvb His power should each counsel-clever man use in moderation; then he finds it—when among the bold he comes—that none is the briskest of all.\footnoteB{i.e., every man has his match. For the expression compare particularly \Volsungasaga\ TODO \emph{þviat hverr sa, er med maurgum kemr, ma þat finna eitthvert sinn, at einge er einna hvataztr} “for each man, who comes among the many, must at some time find that none is the briskest of all.”}\evb
\evg


\bvg
\bva \alst{O}rða þęira, \hld\ es maðr \alst{ǫ}ðrum sęgir, &
\ind opt hann \alst{g}jǫld of \alst{g}etr.\eva

\bvb For those words that man to another says, he oft gets paid back.\evb
\evg


\bvg
\bva \alst{M}ikilsti snimma \hld\ kom’k í \alst{m}arga staði, &
\ind ęn til \alst{s}íð í \alst{s}uma; &
\alst{ǫ}l vas drukkit, \hld\ sumt vas \alst{ó}lagat; &
\ind sjaldan hittir \alst{l}ęiðr í \alst{l}ið.\eva

\bvb Much too early I came to many places, and too late to some. The ale was drunk, at other times yet unbrewed;\footnoteB{lit. “some [of it] was unbrewed”} seldom finds the loathsome man his place.\evb
\evg


\bvg
\bva \alst{H}ér ok \alst{h}var \hld\ myndi mér \alst{h}ęim of boðit, &
\ind ef þyrpta’k at \alst{m}ǫ́lungi \alst{m}at, &
eða \alst{t}vau lǽr hęngi \hld\ at hins \alst{t}ryggva vinar, &
\ind þar’s ek hafða \alst{ęi}tt \alst{e}tit.\eva

\bvb Here and there would I to homes be invited, if at no meal-time I needed food; or [if] two hams would hang at the trusty friend’s, where I had eaten one.\evb
\evg


\bvg
\bva \alst{Ę}ldr es baztr \hld\ með \alst{ý}ta sonum &
\ind ok \alst{s}ólar \alst{s}ýn, &
\alst{h}ęilyndi sitt, \hld\ ef \alst{h}afa náir, &
\ind án við \alst{l}ǫst at \alst{l}ifa.\eva

\bvb Fire is best among the sons of men, and the sight of the sun; one’s good health—if he manage to keep it—and living without vice.\evb
\evg


\bvg
\bva Es-at maðr \alst{a}lls vesall, \hld\ þótt sé \alst{i}lla hęill, &
\ind \alst{s}umr es af \alst{s}onum \alst{s}ǽll, &
\alst{s}umr af frǽndum, \hld\ \alst{s}umr af fé ǿrnu, &
\ind sumr af \alst{v}erkum \alst{v}el.\eva

\bvb Man is not all wretched, though he of poor health be: someone finds joy in sons, someone in friends, someone in ample \inx{fee}[C], someone in works done well.\evb
\evg


\bvg
\bva Bętra es \alst{l}ifðum, \hld\ ok sǽl\alst{l}ifðum, &
\ind ęy getr \alst{k}vikr \alst{k}ú; &
\alst{ę}ld sá’k \alst{u}pp brinna \hld\ \alst{au}ðgum manni fyr, &
\ind ęn úti vas \alst{d}auðr fyr \alst{d}urum.\eva

\bvb It is better with the living, and the joyfully living: ever gets the quick\footnoteB{i.e. the living.} a cow.\footnoteB{A reference to the cattle-based economy (see also v. 76), the cow being used as a metonym. The meaning is that new opportunities always present themselves.} A fire\footnoteB{His funeral-pyre.} I saw burn on high for a wealthy man, but outside he was dead before the door.\evb
\evg


\bvg
\bva \alst{H}altr ríðr \alst{h}rossi, \hld\ \alst{h}jǫrð rekr \alst{h}andarvanr, &
\ind daufr \alst{v}egr ok \alst{d}ugir; &
\alst{b}lindr es \alst{b}ętri, \hld\ an \alst{b}ręndr séi; &
\ind \alst{n}ýtr manngi \alst{n}ás.\eva

\bvb A halt man rides a horse, a handless drives a herd, a deaf fights and avails. Blind is better than burnèd be: no man has use for a corpse.\evb
\evg


\bvg
\bva \alst{S}onr es bętri, \hld\ þótt sé \alst{s}íð of alinn &
\ind ęptir \alst{g}inginn \alst{g}uma; &
sjaldan \alst{b}autarstęinar \hld\ standa \alst{b}rautu nǽr, &
\ind nema ręisi \alst{n}iðr at \alst{n}ið.\eva

\bvb A son is better, although he late be born after a passed-on man\footnoteB{i.e. after the father is dead.}: seldom beat-stones\footnoteB{Large menhirs raised as memorial stones, later and especially in Upland decorated with Runic inscriptions.} near the highway stand, unless by kinsman for kinsman raised.\evb
\evg


\bvg
\bva \edtext{\alst{T}vęir ’ro ęins hęrjar, \hld\ \alst{t}unga es hǫfuðs bani; &
\ind mér’s í \alst{h}eðin \alst{h}vęrn \hld\ \alst{h}andar vǽni.}{\lemma{Tvęir ... vǽni}\Bfootnote{Whole v. undoubtedly a later insertion, the divergent meter is proof enough.}}\eva

\bvb Two are of one host; the tongue is the head’s bane;\footnoteB{The tongue and the head are part of the same body and need each other, yet the former often leads to the demise of the latter. — For this phrase cf. especially the Old Swedish Heathen Law (Läffler 1879): \emph{Faldr þan orð havr giuit · Glöpr orða værstr · Tunga houuðbani · Liggi i vgildum acri} “Falls the one who has given the word—wickedness is the worst of words; the tongue the head’s bane-man—may he lie in an unpaid field (i.e. no weregild will be paid for him).”} in every cloak I expect a hand.\evb
\evg


\bvg
\bva \alst{N}ótt verðr fęginn, \hld\ sá’s \alst{n}esti trúir, &
\ind \alst{sk}ammar ’ro \alst{sk}ips ráar, &
\ind \alst{h}verf es \alst{h}austgríma; &
\alst{f}jǫlð of viðrir \hld\ á \alst{f}imm dǫgum, &
\ind ęn \alst{m}ęir á \alst{m}ánaði.\eva

\bvb At night he rejoices, who can rely on his provisions; short are the ship’s sailyards;\footnoteB{TODO: Write about the varying interpretations (Finnur, Cleasby, Skp) of this line.} fickle is the autumn night. The weather shifts much in five days\footnoteB{i.e. a week; see note to v. 51.} but more in a month.\evb
\evg


\bvg
\bva \alst{V}ęit-a hinn, \hld\ es \alst{v}ǽtki \alst{v}ęit, &
\ind margr verðr \edtext{af \alst{au}rum}{\lemma{af aurum}\Afootnote{‘aflꜹðrom’ \emph{ms.}}} \alst{a}pi; &
maðr es \alst{au}ðigr, \hld\ annarr \alst{ó}auðigr, &
\ind skyli-t þann \alst{v}ítka \alst{v}áar.\eva

\bvb The one knows not, who nothing knows: treasures make many a man a fool. A man is wealthy; another not wealthy; one oughtn’t to curse him for his woe.\evb
\evg


\bvg
\bva \alst{D}ęyr fé, \hld\ \alst{d}ęyja frǽndr, &
\ind dęyr \alst{s}jalfr hit \alst{s}ama; &
ęn \alst{o}rðstírr \hld\ dęyr \alst{a}ldrigi &
\ind hvęim’s sér \alst{g}óðan \alst{g}etr.\eva

\bvb \inx{Fee}[C] dies, kinsmen die, oneself dies the same;\footnoteB{The power of this succinct expression may be less clear to the modern reader. In Germanic Iron Age society a man’s wealth was reckoned by how many heads of cattle he owned, and his social power by the number of able male relatives ready to side with him in conflict. The meaning is thus: all earthly power passes away, and so will you.} but a word-glory never dies, for whomever gets himself a good one.\evb
\evg


\bvg
\bva \alst{D}ęyr fé, \hld\ \alst{d}ęyja frǽndr, &
\ind dęyr \alst{s}jalfr hit \alst{s}ama; &
\alst{e}k vęit \alst{ęi}nn \hld\ at \alst{a}ldri dęyr: &
\ind \alst{d}ómr umb \alst{d}auðan hvęrn.\eva

\bvb Fee dies, kinsmen die, oneself dies the same; I know but one that never dies: the \inx{Doom}[C] over each man dead.\evb
\evg


\bvg
\bva \alst{F}ullar grindr \hld\ sá’k fyr \alst{F}itjungs sonum, &
\ind nú bera þęir \alst{v}ánar \alst{v}ǫl; &
svá es \alst{au}ðr \hld\ sęm \alst{au}gabragð, &
\ind hann es \alst{v}altastr \alst{v}ina.\eva

\bvb Full pens I saw by the sons of Fitting; now they bear a beggar’s staff.\footnoteB{lit. “the staff of hope”.} Thus is wealth like the twinkling of an eye; it is the ficklest of friends.\evb
\evg


\bvg
\bva \alst{Ó}snotr maðr, \hld\ es \alst{ęi}gnask getr &
\ind \alst{f}é eða \alst{f}ljóðs munuð; &
\alst{m}etnaðr hǫ́num þróask, \hld\ ęn \alst{m}anvit aldrigi; &
\ind framm gęngr hann \alst{d}rjúgt í \alst{d}ul.\eva

\bvb 78\evb
\evg


\bvg
\bva Þat es þá \alst{r}ęynt, \hld\ es þú at \alst{r}únum spyrr \hld\ hinum \alst{r}ęginkunnum, &
\ind þęim’s \alst{g}ęrðu \alst{g}innręgin &
\ind ok \alst{f}áði \alst{f}imbulþulr; &
\ind þá hęfr hann bazt, ef þęgir.\eva

\bvb Then that is proven of which thou inquires the runes, the ones born of the Powers, those which the yin-Powers made, and the Fimble-thyle \ken{Weden}[1] painted. (Then he has it best, if he shuts up.)\evb
\evg


\bvg
\bva At \alst{k}veldi skal dag lęyfa, \hld\ \alst{k}onu es bręnd es, &
\alst{m}ǽki es ręyndr es, \hld\ \alst{m}ęy es gefin es, &
\alst{í}s es \alst{y}fir kømr, \hld\ \alst{ǫ}l es drukkit es.\eva

\bvb At evening shall one praise day, a woman when she is burned, a sword when it is tried, a maiden when she is given,\footnoteB{i.e. in marriage.} ice when one crosses over, ale when it is drunk.\evb
\evg


\bvg
\bva Í \alst{v}indi skal \alst{v}ið hǫggva, \hld\ \alst{v}eðri á sǽ róa, &
\alst{m}yrkri við \alst{m}an spjalla, \hld\ \alst{m}ǫrg eru dags augu, &
á \alst{sk}ip skal \alst{sk}riðar orka, \hld\ ęn á \alst{sk}jǫld til hlífar, &
\alst{m}ǽki til hǫggs, \hld\ ęn \alst{m}ęy til kossa.\eva

\bvb In wind shall one cut wood, in storm row on the sea, in darkness meet with a maiden; many are the eyes of day. A ship shall one have for its speed, a shield for shelter, a sword for striking, but a maiden for her kisses.\evb
\evg


\bvg
\bva Við \alst{e}ld skal \alst{ǫ}l drekka, \hld\ ęn á \alst{í}si skríða, &
\alst{m}agran \alst{m}ar kaupa, \hld\ ęn \alst{m}ǽki saurgan, &
\alst{h}ęima \alst{h}ęst fęita, \hld\ ęn \alst{h}und á búi. \eva

\bvb By fire shall one drink ale, and on the ice skate; buy a meager stallion, and a rusty sword; fatten the horse at home, and the hound in the household.\evb
\evg


\bvg Regarding the love of women, and Woden’s failed love-adventures.
\bva \alst{M}ęyjar orðum \hld\ skyli \alst{m}anngi trúa, &
\ind né því’s \alst{k}veðr \alst{k}ona; &
\edtext{\edtext{þvít}{\Afootnote{\emph{om.} \Fostrbroedhra}} á \alst{h}verfanda \alst{h}véli \hld\ \edtext{vǫ́ru}{\Afootnote{er \Fostrbroedhra}} þęim \edtext{\alst{h}jǫrtu skǫpuð}{\lemma{hjǫrtu skǫpuð}\Afootnote{hjarta skapat \Fostrbroedhra}}, &
\ind \edtext{\alst{b}rigð}{\lemma{brigð}\Afootnote{ok brigð \Fostrbroedhra}} í \alst{b}rjóst of \edtext{lagið}{\Afootnote{‘laginn’ \Fostrbroedhra}}.}{\lemma{þvít ... lagið}\Bfootnote{Quoted in slightly divergent form in \Fostrbroedhra\ (Thott 1768 4°\textsuperscript{x}, fol. 210r): \emph{“And then he remembered the ditty which had been composed about loose women: [...]”}}}\eva

\bvb The words of a maiden should no man believe, nor that which a woman sings. For on a spinning wheel were their hearts shaped; fickleness in their breasts was laid.\evb
\evg


\bvg
\bva \alst{B}restanda \alst{b}oga, \hld\ \alst{b}rinnanda loga, &
\alst{g}ínanda ulfi, \hld\ \alst{g}alandi krǫ́ku, &
\alst{r}ýtanda svíni, \hld\ \alst{r}ótlausum viði, &
\alst{v}axanda \alst{v}ági, \hld\ \alst{v}ellanda katli,\eva

\bvb The bursting bow, the burning flame, the gaping wolf, the crowing crow, the roaring swine, the rootless tree, the waxing wave, the swelling kettle,\evb
\evg


\bvg
\bva \alst{f}ljúganda \alst{f}lęini, \hld\ \alst{f}allandi bǫ́ru, &
\alst{í}si \alst{ęi}nnǽttum, \hld\ \alst{o}rmi hringlęgnum, &
\alst{b}rúðar \alst{b}ęðmǫ́lum \hld\ eða \alst{b}rotnu sverði, &
\alst{b}jarnar lęiki \hld\ eða \alst{b}arni konungs,
\alst{s}júkum kalfi, \hld\ \alst{s}jalfráða þrǽli, &
\alst{v}ǫlu \alst{v}ilmǽli, \hld\ \alst{v}al nýfęldum.\eva

\bvb the flying spear, the falling billow, the one-night old ice, the coiled-up serpent, the bed-speaking of a bride, or the broken sword, the play of a bear, or the child of a king, the sick calf, the freed slave, the kind word of a wallow, newly felled corpses.\evb
\evg


\bvg
\bva \alst{A}kri \alst{á}rsǫ́num \hld\ trúi \alst{ę}ngi maðr, &
\ind né til \alst{s}nimma \alst{s}yni; &
\alst{v}eðr rǽðr akri, \hld\ ęn \alst{v}it syni; &
\ind \alst{h}ǽtt es þęira \alst{h}várt.\eva

\bvb An early sown field ought no man to trust, nor too early\footnoteB{i.e. in life.} a son. The weather rules the field, but the wits the son; there is risk to both of them.\evb
\evg


\bvg
\bva \alst{B}róðurbana sínum \hld\ þótt á \alst{b}rautu mǿti, &
\alst{h}úsi \alst{h}alfbrunnu, \hld\ \alst{h}ęsti alskjótum, &
þá’s \alst{jó}r \alst{ó}nýtr, \hld\ ef \alst{ęi}nn fótr brotnar; &
verðr-it maðr svá \alst{t}ryggr \hld\ at þessu \alst{t}rúi ǫllu.\eva

\bvb His brother’s bane-man—though on the highway they meet,—a half-burned house, an all-fleet horse; then is the steed of no use if one foot breaks. There is no man so trusting, that he trust all of these.\evb
\evg


\bvg
\bva Svá’s \alst{f}riðr kvinna \hld\ þęira’s \alst{f}látt hyggja, &
sęm aki \alst{jó} \alst{ó}bryddum \hld\ á \alst{í}si hǫ́lum &
\alst{t}ęitum, \alst{t}vévetrum \hld\ ok sé \alst{t}amr illa, &
eða í \alst{b}yr óðum \hld\ \alst{b}ęiti stjórnlausu, &
eða skyli \alst{h}altr \alst{h}ęnda \hld\ \alst{h}ręin í þáfjalli.\eva

\bvb So is the peace of women—those who falsely think—like riding an unshod horse on slippery ice—a joyous, two winters old, and poorly tamed one—or in a mad gust tacking without a rudder;\footnoteB{lit. “tacking a rudderless [ship]”.} or as if a halt man would catch a reindeer on a thawing hill.\evb
\evg


\bvg
\bva \alst{B}ert nú mǽli’k, \hld\ því-at \alst{b}ǽði vęit’k, &
\ind brigðr es \alst{k}arla hugr \alst{k}onum, &
þá \alst{f}ęgrst mǽlum, \hld\ es \alst{f}lást hyggjum; &
\ind þat tǽlir \alst{h}orska \alst{h}ugi.\eva

\bvb Plainly I now speak, for I know both: fickle are men’s hearts towards women. We then speak the most fairly, when the most falsely we think; that entices sharp minds.\evb
\evg


\bvg
\bva \alst{F}agrt skal mǽla \hld\ ok \alst{f}é bjóða, &
\ind sá’s vill \alst{f}ljóðs ǫ́st \alst{f}áa, &
\alst{l}íki \alst{l}ęyfa \hld\ hins \alst{l}jósa mans, &
\ind sá \alst{f}ǽr, es \alst{f}ríar.\eva

\bvb 90\evb
\evg


\bvg
\bva \alst{Á}star firna \hld\ skyli \alst{ę}ngi maðr &
\ind \alst{a}nnan \alst{a}ldrigi; &
opt fáa á \alst{h}orskan, \hld\ es á \alst{h}ęimskan né fáa, &
\ind \alst{l}ostfagrir \alst{l}itir.\eva

\bvb 91\evb
\evg


\bvg
\bva \alst{Ęy}vitar firna, \hld\ es maðr \alst{a}nnan skal, &
\ind þess’s of margan \alst{g}ęngr \alst{g}uma; &
\alst{h}ęimska ór \alst{h}orskum \hld\ gęrir \alst{h}ǫlða sonu &
\ind sá hinn \alst{m}átki \alst{m}unr.\eva

\bvb 92\evb
\evg


\bvg
\bva \alst{H}ugr ęinn þat vęit, \hld\ es býr \alst{h}jarta nǽr, &
\ind ęinn es hann \alst{s}ér of \alst{s}efa; &
øng es \alst{s}ótt verri \hld\ hvęim \alst{s}notrum manni &
\ind an sér \alst{ø}ngu at \alst{u}na.\eva

\bvb The mind alone knows what lives close to the heart; each one’s mind is his own. No worse ailment is there for each clever man, than to be content with nothing.\evb
\evg


\bvg
\bva Þat þá \alst{r}ęyndak, \hld\ es í \alst{r}ęyri sat’k, &
\ind ok vǽtta’k \alst{m}íns \alst{m}unar, &
\alst{h}old ok \alst{h}jarta \hld\ vas mér hin \alst{h}orska mǽr, &
\ind þęygi hana at \alst{h}ęldr \alst{h}ęf’k.\eva

\bvb 94\evb
\evg


\bvg
\bva \alst{B}illings męy \hld\ ek fann \alst{b}ęðjum á &
\ind \alst{s}ólhvíta \alst{s}ofa; &
\alst{ja}rls \alst{y}nði \hld\ þótti mér \alst{ę}kki vesa &
\ind nema við þat \alst{l}ík at \alst{l}ifa.\eva

\bvb 95\evb
\evg


\bvg
\bva “\alst{Au}k nǽr \alst{a}ptni \hld\ skalt-u \alst{Ó}ðinn koma, &
\ind ef vilt þér \alst{m}ǽla \alst{m}an, &
\alst{a}lt eru \alst{ó}skǫp, \hld\ nema \alst{ęi}n vitim &
\ind \alst{s}likan lǫst \alst{s}aman.”\eva

\bvb 96\evb
\evg


\bvg
\bva \alst{A}ptr ek hvarf \hld\ ok \alst{u}nna þóttumk &
\ind \alst{v}ísum \alst{v}ilja frá; &
\alst{h}itt ek \alst{h}ugða, \hld\ at \alst{h}afa mynda’k &
\ind \alst{g}ęð hęnnar alt ok \alst{g}aman.\eva

\bvb 97\evb
\evg


\bvg
\bva Svá kom’k \alst{n}ǽst, \hld\ at hin \alst{n}ýta vas &
\ind \alst{v}ígdrótt ǫll of \alst{v}akin; &
með \alst{b}rinnǫndum ljósum \hld\ ok \alst{b}ornum viði, &
\ind svá vas mér \alst{v}ílstígr of \alst{v}itaðr.\eva

\bvb 98\evb
\evg


\bvg
\bva \alst{Au}k nǽr morni, \hld\ es vas’k \alst{ę}nn of kominn, &
\ind þá vas \alst{s}aldrótt of \alst{s}ofin; &
\alst{g}ręy ęitt þá fan’k \hld\ hinnar \alst{g}óðu konu &
\ind \alst{b}undit \alst{b}ęðjum á.\eva

\bvb 99\evb
\evg


\bvg
\bva Mǫrg es \alst{g}óð mǽr, \hld\ ef \alst{g}ǫrva kannar, &
\ind \alst{h}ugbrigð við \alst{h}ali; &
þá þat \alst{r}ęynda’k, \hld\ es hit \alst{r}áðspaka &
\ind tęygða’k á \alst{f}lǽrðir \alst{f}ljóð. &
\alst{h}ǫ́ðungar \alst{h}vęrrar \hld\ lęitaði mér hit \alst{h}orska man &
\ind ok hafða’k þess \alst{v}ǽtki \alst{v}ífs.\eva

\bvb 100\evb
\evg


\bvg Side-composition to the previous poem, starting with a general maxim.
\bva Hęima \alst{g}laðr \hld\ ok við \alst{g}ęsti ręifr, &
\ind \alst{s}viðr skal of \alst{s}ik vesa; &
\alst{m}innigr ok \alst{m}ǫ́lugr, \hld\ ef vill \alst{m}argfróðr vesa; &
\ind opt skal \alst{g}óðs \alst{g}eta; &
\alst{f}imbul\alst{f}ambi hęitir, \hld\ sás \alst{f}átt kann sęgja; &
\ind þat es \alst{ó}snotrs \alst{a}ðal.\eva

\bvb 101\evb
\evg


\bvg
\bva Hinn \alst{a}ldna \alst{jǫ}tun sóttak, \hld\ nú em’k \alst{a}ptr of kominn; &
\ind fátt gat’k \alst{þ}ęgjandi \alst{þ}ar; &
\alst{m}ǫrgum orðum \hld\ \alst{m}ǽlta’k í minn frama &
\ind í \alst{S}uttungs \alst{s}ǫlum.\eva

\bvb The old ettin I sought, now am I come back; I got little silence there. Many words I spoke to my fame, in the halls of Sutting.\evb
\evg


\bvg
\bva \alst{G}unnlǫð mér of \alst{g}af \hld\ \alst{g}ollnum stóli á &
\ind \alst{d}rykk hins \alst{d}ýra mjaðar; &
\alst{i}ll \alst{i}ðgjǫld \hld\ lét’k hana \alst{ę}ptir hafa &
\ind síns hins \alst{h}ęila \alst{h}ugar. &
\ind (síns hins \alst{s}vára \alst{s}efa).\eva

\bvb 103\evb
\evg


\bvg
\bva \alst{R}ata munn \hld\ létumk \alst{r}úms of fáa &
\ind ok of \alst{g}rjót \alst{g}naga; &
\alst{y}fir ok \alst{u}ndir \hld\ stóðumk \alst{jǫ}tna vegir, &
\ind svá \alst{h}ǽttak \alst{h}ǫfði til.\eva

\bvb 104\evb
\evg


\bvg
\bva \alst{V}el kęypts hlutar \hld\ hęf’k \alst{v}el notit; &
\ind \alst{f}ás es \alst{f}róðum vant; &
\alst{Ó}ðrerir \hld\ nú \alst{u}pp’s kominn &
\ind á \alst{a}lda vé \alst{ja}ðars.\eva

\bvb 105\evb
\evg


\bvg
\bva \alst{I}fi es mér á, \hld\ at vǽra’k \alst{ę}nn kominn &
\ind \alst{jǫ}tna gǫrðum \alst{ó}r, &
ef \alst{G}unnlaðar né nyta’k, \hld\ hinnar \alst{g}óðu konu, &
\ind es lǫgðumk \alst{a}rm \alst{y}fir.\eva

\bvb I have doubt, of whether I were yet come out of the yards of the Ettins, if Guthlathe I had not used, that good woman, whom I laid my arm over.\evb
\evg


\bvg
\bva Hins \alst{h}indra dags \hld\ gingu \alst{h}rímþursar &
\ind (\alst{H}áva ráðs at fregna,) &
\ind \alst{H}áva \alst{h}ǫllu í, &
at \alst{B}ǫlverki spurðu, \hld\ ef vǽri með \alst{b}ǫndum kominn &
\ind eða hęfði hǫ́num \alst{S}uttungr of \alst{s}óit.\eva

\bvb 107\evb
\evg


\bvg
\bva Baugęið \alst{Ó}ðinn \hld\ hygg at \alst{u}nnit hafi, &
\ind hvat skal hans \alst{t}ryggðum \alst{t}rúa? &
\alst{S}uttung \alst{s}vikvinn \hld\ hann lét \alst{s}umbli frá &
\ind ok \alst{g}rǿtta \alst{G}unnlǫðu.\eva

\bvb A \inx{bigh-oath}[C] I ween that Weden has sworn; how shall one trust his truces? He let Sutting walk betrayed from the feast, and Guthlathe made to weep.\evb
\evg


Advice of the Fimble-Thyle, given to Loddfathomer.


\bvg
\bva Mál’s at \alst{þ}ylja \hld\ \alst{þ}ular stóli á; &
\ind \alst{U}rðar brunni \alst{a}t &
\alst{s}á’k ok þagða’k, \hld\ \alst{s}á’k ok hugða’k, &
\ind hlýdda’k á \alst{m}anna \alst{m}ál; &
of \alst{r}únar hęyrða’k dǿma, \hld\ né umb \alst{r}ǫ́ðum þǫgðu &
\ind \alst{H}áva \alst{h}ǫllu at, &
\ind \alst{H}áva \alst{h}ǫllu í &
\ind hęyrða’k \alst{s}ęgja \alst{s}vá:\eva

\bvb It is time to \inx{thilly}[C], upon the chair of the \inx{thyle}[C]. At the well of Weird, I saw and I was silent: I saw and I pondered: I heeded the matters of men. Of runes I heard them speak, nor about counsels were they silent, at the hall of the High One, in the hall of the High One, I heard them say thus:\evb
\evg


\bvg
\bva \alst{R}ǫ́ðumk þér Loddfáfnir, \hld\ at þú \alst{r}ǫ́ð nemir, &
\ind \alst{n}jǫta munt ef \alst{n}emr, &
\ind þér munu \alst{g}óð ef \alst{g}etr: &
\alst{n}ótt þú rís-at, \hld\ nema á \alst{n}jósn séir, &
\ind eða lęitir þér \alst{i}nnan \alst{ú}t staðar.\eva

\bvb I counsel thee Loddfathomer, that thou take the counsels; thou wilt benefit if thou take; they will be good for thee if thou get: At night thou rise not, unless at scouting thou be, or TODO\evb
\evg


\bvg
\bva \alst{R}ǫ́ðumk þér Loddfáfnir, \hld\ at þú \alst{r}ǫ́ð nemir, &
\ind \alst{n}jóta munt ef \alst{n}emr, &
\ind þér munu \alst{g}óð ef \alst{g}etr: &
\alst{f}jǫlkunnigri konu \hld\ skal-at-tu í \alst{f}aðmi sofa, &
\ind svá at hon \alst{l}yki þik \alst{l}iðum. &
Hón svá \alst{g}ęrir \hld\ at þú \alst{g}áir ęigi &
\ind \alst{þ}ings né \alst{þ}jóðans máls; &
\alst{m}at þú vill-at \hld\ né \alst{m}anskis gaman &
\ind fęrr þú \alst{s}orgafullr at \alst{s}ofa.\eva

\bvb 111\evb
\evg


\bvg
\bva \alst{R}ǫ́ðumk þér Loddfáfnir, \hld\ at þú \alst{r}ǫ́ð nemir, &
\ind \alst{n}jóta munt ef \alst{n}emr, &
\ind þér munu \alst{g}óð ef \alst{g}etr: &
\alst{a}nnars konu \hld\ tęyg þér \alst{a}ldrigi &
\ind \alst{ęy}rarúnu \alst{a}t.\eva

\bvb 112\evb
\evg


\bvg
\bva \alst{R}ǫ́ðumk þér Loddfáfnir, \hld\ at þú \alst{r}ǫ́ð nemir, &
\ind \alst{n}jóta munt ef \alst{n}emr, &
\ind þér munu \alst{g}óð ef \alst{g}etr: &
á \alst{f}jalli eða \alst{f}irði, \hld\ ef þik \alst{f}ara tíðir, &
\ind fásk-tu at \alst{v}irði \alst{v}el.\eva

\bvb 113\evb
\evg


\bvg
\bva \alst{R}ǫ́ðumk þér Loddfáfnir, \hld\ at þú \alst{r}ǫ́ð nemir, &
\ind \alst{n}jóta munt ef \alst{n}emr, &
\ind þér munu \alst{g}óð ef \alst{g}etr: &
\alst{i}llan mann \hld\ lát \alst{a}ldrigi &
\ind óhǫpp at þér vita. &
af \alst{i}llum manni \hld\ fǽr þú \alst{a}ldrigi &
\ind \alst{g}jǫld hins \alst{g}óða hugar.\eva

\bvb 114\evb
\evg


\bvg
\bva \alst{O}farla bíta \hld\ sá’k \alst{ęi}num hal &
\ind \alst{o}rð \alst{i}llrar konu, &
\alst{f}lárǫ́ð tunga \hld\ varð hǫ́num at \alst{f}jǫrlagi &
\ind ok þęygi of \alst{s}anna \alst{s}ǫk.\eva

\bvb 115\evb
\evg


\bvg
\bva \alst{R}ǫ́ðumk þér Loddfáfnir, \hld\ at þú \alst{r}ǫ́ð nemir, &
\ind \alst{n}jóta munt ef \alst{n}emr, &
\ind þér munu \alst{g}óð ef \alst{g}etr: &
\alst{v}ęizt ef \alst{v}in átt, \hld\ þann’s \alst{v}el trúir, &
\ind \alst{f}ar þú at \alst{f}inna opt. &
því’t \alst{h}rísi vęx \hld\ ok \alst{h}óu grasi &
\ind \alst{v}egr, es \alst{v}ǽtki trøðr,\eva

\bvb 116\evb
\evg


\bvg
\bva \alst{R}ǫ́ðumk þér Loddfáfnir, \hld\ at þú \alst{r}ǫ́ð nemir, &
\ind \alst{n}jóta munt ef \alst{n}emr, &
\ind þér munu \alst{g}óð ef \alst{g}etr: &
\alst{v}in þínum \hld\ \alst{v}es þú aldrigi &
\ind \alst{f}yrri at \alst{f}laumslitum. &
\alst{s}org etr hjarta, \hld\ ef þú \alst{s}ęgja né náir &
\ind \alst{ęi}nhvęrjum \alst{a}llan hug.\eva

\bvb 117\evb
\evg


\bvg
\bva \alst{R}ǫ́ðumk þér Loddfáfnir, \hld\ at þú \alst{r}ǫ́ð nemir, &
\ind \alst{n}jóta munt ef \alst{n}emr, &
\ind þér munu \alst{g}óð ef \alst{g}etr: &
\alst{g}óðan mann \hld\ tęyg þér at \alst{g}amanrúnum &
\ind ok nem \alst{l}íknargaldr meðan \alst{l}ifir.\eva

\bvb 118\evb
\evg


\bvg
\bva \alst{R}ǫ́ðumk þér Loddfáfnir, \hld\ at þú \alst{r}ǫ́ð nemir, &
\ind \alst{n}jóta munt ef \alst{n}emr, &
\ind þér munu \alst{g}óð ef \alst{g}etr: &
orðum \alst{sk}ipta \hld\ þú \alst{sk}alt aldrigi &
\ind við \alst{ó}svinna \alst{a}pa.\eva

\bvb 119\evb
\evg


\bvg
\bva Af illum \alst{m}anni \hld\ \alst{m}undu aldrigi &
\ind \alst{g}óðs laun of \alst{g}eta, &
ęn \alst{g}óðr maðr \hld\ mun þik \alst{g}ęrva męga &
\ind \alst{l}íknfastan at \alst{l}ofi.\eva

\bvb 120\evb
\evg


\bvg
\bva \alst{S}ifjum es þá blandit \hld\ hvęrr es \alst{s}ęgja rǽðr &
\ind \alst{ęi}num \alst{a}llan hug; &
alt es \alst{b}ętra \hld\ an sé \alst{b}rigðum at vesa: &
\ind es-a sá \alst{v}inr es \alst{v}ilt ęitt sęgir.\eva

\bvb 121\evb
\evg


\bvg
\bva \alst{R}ǫ́ðumk þér Loddfáfnir, \hld\ at þú \alst{r}ǫ́ð nemir, &
\ind \alst{n}jóta munt ef \alst{n}emr, &
\ind þér munu \alst{g}óð ef \alst{g}etr: &
þrimr orðum sęnna \hld\ skal-at-tu þér við verra mann, &
\ind opt hinn \alst{b}ętri \alst{b}ilar. &
\ind þás hinn \alst{v}erri \alst{v}egr.\eva

\bvb 122\evb
\evg


\bvg
\bva \alst{R}ǫ́ðumk þér Loddfáfnir, \hld\ at þú \alst{r}ǫ́ð nemir, &
\ind \alst{n}jóta munt ef \alst{n}emr, &
\ind þér munu \alst{g}óð ef \alst{g}etr: &
\alst{sk}ósmiðr þú verir\hld\ né \alst{sk}ęptismiðr, &
\ind nema \alst{s}jǫlfum þér \alst{s}éir. &
\alst{Sk}ór’s \alst{sk}apaðr illa\hld\ eða \alst{sk}apt sé rangt, &
\ind þá’s þér \alst{b}ǫls \alst{b}eðit.\eva

\bvb 123\evb
\evg


\bvg
\bva \alst{R}ǫ́ðumk þér Loddfáfnir, \hld\ at þú \alst{r}ǫ́ð nemir, &
\ind \alst{n}jóta munt ef \alst{n}emr, &
\ind þér munu \alst{g}óð ef \alst{g}etr: &
hvars þú \alst{b}ǫl kant, \hld\ kveð þér \alst{b}ǫlvi at &
\ind ok gefat þínum \alst{f}jǫ́ndum \alst{f}rið.\eva

\bvb 124\evb
\evg


\bvg
\bva \alst{R}ǫ́ðumk þér Loddfáfnir, \hld\ at þú \alst{r}ǫ́ð nemir, &
\ind \alst{n}jóta munt ef \alst{n}emr, &
\ind þér munu \alst{g}óð ef \alst{g}etr: &
\alst{i}llu fęginn \hld\ ves þú \alst{a}ldrigi, &
\ind ęn lát þér at \alst{g}óðu \alst{g}etit.\eva

\bvb 125\evb
\evg


\bvg
\bva \alst{R}ǫ́ðumk þér Loddfáfnir, \hld\ at þú \alst{r}ǫ́ð nemir, &
\ind \alst{n}jóta munt ef \alst{n}emr, &
\ind þér munu \alst{g}óð ef \alst{g}etr: &
\alst{u}pp líta \hld\ skal-at-tu í \alst{o}rrostu &
\alst{g}jalti \alst{g}líkir \hld\ verða \alst{g}umna synir &
\ind síðr þitt of \alst{h}ęilli \alst{h}alir.\eva

\bvb 126\evb
\evg


\bvg
\bva \alst{R}ǫ́ðumk þér Loddfáfnir, \hld\ at þú \alst{r}ǫ́ð nemir, &
\ind \alst{n}jóta munt ef \alst{n}emr, &
\ind þér munu \alst{g}óð ef \alst{g}etr: &
Ef vilt þér góða \alst{k}onu \hld\ \alst{k}vęðja at gamanrúnum &
\ind ok \alst{f}á \alst{f}ǫgnuð af, &
\alst{f}ǫgru skaldu heita \hld\ ok láta \alst{f}ast vesa; &
\ind lęiðisk manngi \alst{g}ótt ef \alst{g}etr.\eva

\bvb 127\evb
\evg


\bvg
\bva \alst{R}ǫ́ðumk þér Loddfáfnir, \hld\ at þú \alst{r}ǫ́ð nemir, &
\ind \alst{n}jóta munt ef \alst{n}emr, &
\ind þér munu \alst{g}óð ef \alst{g}etr: &
\ind \alst{v}aran bið’k þik \alst{v}esa &
\ind ok \alst{ęi}gi \alst{o}fvaran, &
ves þú við \alst{ǫ}l varastr. \hld\ ok við \alst{a}nnars konu &
ok við \alst{þ}at hit \alst{þ}riðja, \hld\ at \alst{þ}jófar né lęiki.\eva

\bvb 128\evb
\evg


\bvg
\bva \alst{R}ǫ́ðumk þér Loddfáfnir, \hld\ at þú \alst{r}ǫ́ð nemir, &
\ind \alst{n}jóta munt ef \alst{n}emr, &
\ind þér munu \alst{g}óð ef \alst{g}etr: &
at \alst{h}áði né \alst{h}látri \hld\ \alst{h}af þú aldrigi &
\ind \alst{g}ęst né \alst{g}anganda.\eva

\bvb 129\evb
\evg


\bvg
\bva \alst{O}pt vitu \alst{ó}gǫrla, \hld\ þęir’s sitja \alst{i}nni fyrir, &
\ind hvęrs þęir ’ro \alst{k}yns es \alst{k}oma; &
es-at maðr svá \alst{g}óðr \hld\ at \alst{g}alli né fylgi, &
\ind né svá \alst{i}llr at \alst{ęi}nugi dugi.\eva

\bvb 130\evb
\evg


\bvg
\bva \alst{R}ǫ́ðumk þér Loddfáfnir, \hld\ at þú \alst{r}ǫ́ð nemir, &
\ind \alst{n}jóta munt ef \alst{n}emr, &
\ind þér munu \alst{g}óð ef \alst{g}etr: &
at \alst{h}ǫ́rum þul \hld\ \alst{h}lǽ þú aldrigi, &
\ind opt es \alst{g}ótt þats \alst{g}amlir kveða, &
opt ór \alst{sk}ǫrpum bęlg \hld\ \alst{sk}ilin orð koma &
\ind þęims \alst{h}angir með \alst{h}ǫ́m &
\ind ok \alst{sk}ollir með \alst{sk}rǫ́m, &
\ind ok \alst{v}áfir með \alst{v}ilmǫgum.\eva

\bvb 131\evb
\evg


\bvg
\bva \alst{R}ǫ́ðumk þér Loddfáfnir, \hld\ at þú \alst{r}ǫ́ð nemir, &
\ind \alst{n}jóta munt ef \alst{n}emr, &
\ind þér munu \alst{g}óð ef \alst{g}etr: &
\alst{g}ęst né \alst{g}ęyja \hld\ né á \alst{g}rind hrękir; &
\ind get þú \alst{v}ǫ́luðum \alst{v}el.\eva

\bvb 132\evb
\evg


\bvg
\bva \alst{R}amt es þat tré, \hld\ es \alst{r}íða skal &
\ind \alst{ǫ}llum at \alst{u}pploki; &
\alst{b}aug þú gef \hld\ eða þat \alst{b}iðja mun &
\ind þér \alst{l}ǽs hvęrs á \alst{l}iðu.\eva

\bvb 133\evb
\evg


\bvg
\bva \alst{R}ǫ́ðumk þér Loddfáfnir, \hld\ at þú \alst{r}ǫ́ð nemir, &
\ind \alst{n}jóta munt ef \alst{n}emr, &
\ind þér munu \alst{g}óð ef \alst{g}etr: &
hvars \alst{ǫ}l drekkir \hld\ kjós þér \alst{ja}rðar męgin, &
því’t \alst{jǫ}rð tękr við \alst{ǫ}ldri, \hld\ ęn \alst{ę}ldr við sóttum, &
\alst{ęi}k við \alst{a}bbindi, \hld\ \alst{a}x við fjǫlkyngi, &
\alst{h}ǫll við \alst{h}ýrógi; \hld\ \alst{h}ęiptum skal mána kvęðja, &
\alst{b}ęiti við \alst{b}itsóttum, \hld\ ęn við \alst{b}ǫlvi rúnar; &
\ind \alst{f}old skal við \alst{f}lóði taka.\eva

\bvb For earth takes against drunkenness, but fire against sickness; oak against dysentery, the ear [of corn] against sorcery, bearded rye against hernia, in conflicts shall one invoke the moon. TODO
\evb
\evg


Of Woden’s taking of the runes.
It is clear that these verses have very little to do with the rest of the poem, but instead are separate. It is for this reason that they are labelled as \emph{Rúnatals þáttr} (The strand of the Runecount) in younger Eddic paper manuscripts. Many give an archaic, pagan impression. It is as if they were drawn from the lips of an Odinic priest.


\bvg
\bva\alst{V}ęit’k at ek hekk \hld\ \alst{v}indga męiði á &
\ind \alst{n}ǽtr allar \alst{n}íu, &
\alst{g}ęiri undaðr \hld\ ok \alst{g}efinn Óðni, &
\ind \alst{s}jalfr \alst{s}jǫlfum mér, &
á þęim \alst{m}ęiði, \hld\ es \alst{m}anngi vęit, &
\ind hvęrs af \alst{r}ótum \alst{r}innr.\eva

\bvb I know that I hung on a windy tree, for all of nine nights; wounded by spear and given to Weden—myself to myself—on that tree, which no man knows, of whose roots it runs.\evb
\evg


\bvg
\bva Við \alst{h}lęifi mik sǽldu-t \hld\ né við \alst{h}ornigi; &
\alst{n}ýsta’k \alst{n}iðr, \hld\ \alst{n}am’k upp rúnar, &
\alst{ǿ}pandi nam, \hld\ fell’k \alst{a}ptr þaðan.\eva

\bvb With loaf they gladdened me not, nor with horn’s drink. I peered down, I took up the runes, screaming I took; then I fell back thence.\evb
\evg


\bvg
\bva \alst{F}imbulljóð níu \hld\ nam’k af hinum \alst{f}rǽgja syni &
\ind \alst{B}ǫlþorns, \alst{B}ęstlu fǫður, &
ok ek \alst{d}rykk of gat \hld\ hins \alst{d}ýra mjaðar &
\ind \alst{au}sinn \alst{Ó}ðreri.\eva
  \footnotetext[1]{It has been noted (FJ) that this verse fits better in the next section of the poem. It is awkwardly placed here, since it mentions \emph{ljóð} ‘(magical) songs, incantations’, rather than runes.}

\bvb Nine fimble-songs, I got from the famous son of \textbf{Balethorn}, the father of \textbf{Bestle}—and a drink I got, of that expensive mead, poured to \textbf{Woderearer}.\evb
\evg


\bvg
\bva Þá nam’k \alst{f}rǽvask \hld\ ok \alst{f}róðr vesa &
\ind ok \alst{v}axa ok \alst{v}el hafask; &
\alst{o}rð mér af \alst{o}rði \hld\ \alst{o}rðs lęitaði &
\ind \alst{v}erk mér af \alst{v}erki \alst{v}erks.\eva

\bvb Then I began to thrive, and be learned, and grow and have it well. A word for me of a word a word sought out; a work for me of a work a work.\footnoteB{Each good word and deed was followed by another.}\evb
\evg


\bvg
\bva \alst{R}únar munt finna \hld\ ok \alst{r}áðna stafi, &
\ind mjǫk \alst{st}óra \alst{st}afi, &
\ind mjǫk \alst{st}inna \alst{st}afi, &
\ind es \alst{f}áði \alst{f}imbulþulr &
\ind ok \alst{g}ęrðu \alst{g}innręgin &
\ind ok \alst{r}ęist Hroptr \alst{r}agna\footnotemark[5].\eva
\footnotetext[5]{Corrected from \emph{rǫgna}. Cf. Eskál \emph{Vell} 31/2 in SkP I, p. 322.}

\bvb \textbf{Runes} wilt thou find, and interpreted staves: much large staves, much stiff staves, as painted the \textbf{Fimble-thyle}, and made the \textbf{yin-Powers}, and carved \textbf{Roft} of the Powers.\evb
\evg


\bvg
\bva \alst{Ó}ðinn með \alst{ǫ́}sum, \hld\ ęn fyr \alst{ǫ}lfum Dáinn, &
\ind \alst{D}valinn \alst{d}vergum fyrir, &
\ind \alst{Á}sviðr \alst{jǫ}tnum fyrir, &
ek ręist \alst{s}jalfr \alst{s}umar.\eva

\bvb \textbf{Weden} among the \textbf{Ease}, but before the \textbf{Elves} \textbf{Dowen}, \textbf{Dwollen} before the \textbf{Dwarfs}, \textbf{Osswith} before the Ettins; I myself\footnoteB{Weden?} carved some.\evb
\evg


\bvg
\bva Vęiz-tu, hvé \alst{r}ísta skal? \hld\ vęiz-tu, hvé \alst{r}áða skal? &
vęiz-tu, hvé \alst{f}áa skal? \hld\ vęiz-tu, hvé \alst{f}ręista skal? &
vęiz-tu, hvé \alst{b}iðja skal? \hld\ vęiz-tu, hvé \alst{b}lóta skal? &
vęiz-tu, hvé \alst{s}ęnda skal? \hld\ vęiz-tu, hvé \alst{s}óa skal?\eva

\bvb Knowest thou how one shall carve? Knowest thou how one shall read? Knowest thou how one shall paint? Knowest thou how one shall tempt? Knowest thou how one shall bid? Knowest thou how one shall \inx{bloot}[C]? Knowest thou one shall send? Knowest thou how one shall \inx{soo}[C]?\footnoteB{This v. bears strong resemblance with Vg 216 (Högstena golder). TODO: Elaborate.}\evb
\evg


\bvg
\bva \alst{B}ętra’s ó\alst{b}eðit \hld\ an sé of\alst{b}lótit, &
\ind ęy sér til \alst{g}ildis \alst{g}jǫf; &
bętra’s ó\alst{s}ęnt \hld\ an sé of\alst{s}óit.\footnotemark[6]\eva
\footnotetext[6]{A final line is likely missing here. — Identical word-pairing (\emph{biðja} – \emph{blóta}, \emph{senda} – \emph{sóa}) may reveal this v.’s relation with the previous one.}

\bvb Better is unbid than be excessively blooted; a gift always looks to a tribute. Better is unsent than be excessively sooed.\evb
\evg


\bvg
\bva Svá \alst{Þ}undr of ręist \hld\ fyr \alst{þ}jóða rǫk &
þar’s \alst{u}pp of ręis, \hld\ es \alst{a}ptr of kom.\eva

\bvb Thus \inx{Thound}[P] did carve for the fate of the nations, where up [he] rose, when back he came.\footnoteB{A most cryptic v.}\evb
\evg


\bvg Weden’s recounting of his Songs.
\bva Ljóð \alst{þ}au kann’k, \hld\ es kann-at \alst{þ}jóðans kona &
\ind ok \alst{m}anskis \alst{m}ǫgr. &
\alst{H}jǫlp hęitir ęitt, \hld\ þat þér \alst{h}jalpa mun &
\ind við \alst{s}orgum ok \alst{s}ǫkum, \hld\ ok \alst{s}útum gǫrvǫllum.\eva

\bvb Those \inx{leeds}[C] I know, as knows not the ruler’s woman, and no man’s lad. Help is called one, it will help thee against sorrows and sakes,\footnoteB{Legal proceedings.} and all kinds of misfortunes.\footnoteB{TODO: elaborate on translatioon}\evb
\evg


\bvg
\bva Þat kann’k \alst{a}nnat, \hld\ es þurfu \alst{ý}ta synir,\footnoteB{(TODO NUMBERING) Identical wording to 163/2.} &
\ind þęir’s vilja \alst{l}ǽknar \alst{l}ifa.\eva

\bvb I know another, which the sons of men need; they who wish to live as healers.\evb
\evg


\bvg
\bva \alst{Þ}at kann’k \alst{þ}riðja, \hld\ ef mér verðr \alst{þ}ǫrf mikil &
\ind \alst{h}apts við mina \alst{h}ęiptmǫgu, &
\alst{ę}ggjar dęyfi’k \hld\ minna \alst{a}ndskota, &
\ind bítat þęim \alst{v}ǫ́pn né \alst{v}élir.\eva

\bvb I know the third,\evb
\evg


\bvg
\bva Þat kann’k \alst{f}jórða, \hld\ ef mér \alst{f}yrðar bera &
\ind \alst{b}ǫnd at \alst{b}oglimum, &
svá ek \alst{g}ęl, \hld\ at \alst{g}anga má’k, &
\ind sprettr mér af \alst{f}ótum \alst{f}jǫturr. &
\ind ęn af \alst{h}ǫndum \alst{h}apt.\eva

\bvb 147\evb
\evg


\bvg
\bva Þat kann’k \alst{f}imta, \hld\ ef sé’k af \alst{f}ári skotinn &
\ind \alst{f}lęin í \alst{f}olki vaða, &
flýgr-a svá \alst{st}int, \hld\ at \alst{st}ǫðvigak, &
\ind ef hann \alst{s}jónum of \alst{s}é’k.\eva

\bvb 148\evb
\evg


\bvg
\bva Þat kann’k \alst{s}étta, \hld\ ef mik \alst{s}ǽrir þegn &
\ind á \alst{v}rótum hrás \alst{v}iðar. &
þann \alst{h}al, \hld\ es mik \alst{h}ęipta kvęðr, &
\ind þann eta \alst{m}ęin hęldr an \alst{m}ik.\eva

\bvb 149\evb
\evg


\bvg
\bva Þat kann’k \alst{s}jaunda, \hld\ ef \alst{s}é’k hóvan loga &
\ind \alst{s}al of \alst{s}essmǫgum, &
\alst{b}rinnrat svá \alst{b}ręitt, \hld\ at hǫ́num \alst{b}jargigak; &
\ind þann kann’k \alst{g}aldr at \alst{g}ala.\eva

\bvb 150\evb
\evg


\bvg
\bva Þat kann’k \alst{á}tta, \hld\ es \alst{ǫ}llum es &
\ind \alst{n}ytsamligt at \alst{n}ema, &
\alst{h}var’s \alst{h}atr vęx \hld\ með \alst{h}ildings sonum, &
\ind þat má’k \alst{b}ǿta \alst{b}rátt.\eva

\bvb 151\evb
\evg


\bvg
\bva Þat kann’k \alst{n}íunda, \hld\ ef mik \alst{n}auðr of stęndr &
\ind at bjarga \alst{f}ari á \alst{f}loti, &
\alst{v}ind ek kyrri \hld\ \alst{v}ági á &
\ind ok \alst{s}vǽfi’k allan \alst{s}ǽ.\eva

\bvb 152\evb
\evg


\bvg
\bva Þat kann’k \alst{t}íunda, \hld\ ef sé’k \alst{t}únriður &
\ind \alst{l}ęika \alst{l}opti á, &
ek svá \alst{v}in’k, \hld\ at þǽr \alst{v}illar fara &
\ind sinna \alst{h}ęim-\alst{h}ama &
\ind sinna \alst{h}ęim-\alst{h}uga.\eva

\bvb 153\evb
\evg


\bvg
\bva Þat kann’k \alst{ę}llipta, \hld\ ef skal’k til \alst{o}rrostu &
\ind \alst{l}ęiða \alst{l}angvini, &
und \alst{r}andir gęlk, \hld\ ęn þęir með \alst{r}íki fara, &
\ind \alst{h}ęilir \alst{h}ildar til, &
\ind \alst{h}ęilir \alst{h}ildi frá, &
\ind koma þęir \alst{h}ęilir \alst{h}vaðan.\eva

\bvb 154\evb
\evg


\bvg
\bva Þat kann’k \alst{t}olpta, \hld\ ef sé’k á \alst{t}ré uppi &
\ind \alst{v}áfa \alst{v}irgilná, &
svá ek \alst{r}íst \hld\ ok í \alst{r}únum fá’k, &
\ind at sá \alst{g}ęngr \alst{g}umi. &
\ind ok \alst{m}ǽlir við \alst{m}ik.\eva

\bvb 155\evb
\evg


\bvg
\bva \alst{Þ}at kann’k \alst{þ}rettánda \hld\ ef skal’k \alst{þ}egn ungan &
\ind \alst{v}erpa \alst{v}atni á, &
munat hann \alst{f}alla, \hld\ þótt í \alst{f}olk komi, &
\ind \alst{h}nígr-a sá \alst{h}alr fyr \alst{h}jǫrum.\eva

\bvb 156\footnoteB{Describing the pagan ritual of pouring water on a newborn child. Cf. \Rigsthula 7, 21, 34.}\evb
\evg


\bvg
\bva Þat kann’k \alst{f}jogurtánda, \hld\ ef skal’k \alst{f}yrða liði &
\ind \alst{t}ęlja \alst{t}íva fyrir, &
\alst{á}sa ok \alst{a}lfa \hld\ ek kann \alst{a}llra skil, &
\ind fár kann ó\alst{s}notr \alst{s}vá.\eva

\bvb 157\evb
\evg


\bvg
\bva \alst{Þ}at kann’k fimtánda, \hld\ es gól {Þ}jóðrørir &
\ind \alst{d}vergr fyr \alst{D}ęllings \alst{d}urum, &
\alst{a}fl gól \alst{ǫ́}sum, \hld\ ęn \alst{ǫ}lfum frama, &
\ind \alst{h}yggju \alst{H}roptatý.\eva

\bvb 158\evb
\evg


\bvg
\bva Þat kann’k sextánda, \hld\ ef vil’k hins svinna mans &
\ind hafa gęð alt ok gaman, &
\alst{h}ugi \alst{h}vęrfi’k \hld\ \alst{h}vitarmri konu &
\ind ok \alst{s}ný’k hęnnar ǫllum \alst{s}efa.\eva

\bvb 159\evb
\evg


\bvg
\bva Þat kann’k \alst{s}jautjánda \hld\ at mik \alst{s}ęint mun firrask &
\ind hit \alst{m}anunga \alst{m}an.\eva

\bvb 160\evb
\evg


\bvg
\bva Þat kann’k \alst{á}tjánda, \hld\ es \alst{ǽ}va kęnni’k &
\ind \alst{m}ęy né \alst{m}anns konu, &
\alst{a}lt es bętra \hld\ es \alst{ęi}nn of kann, &
\ind þat fylgir \alst{l}jóða \alst{l}okum, &
nema þęiri \alst{ęi}nni, \hld\ es mik \alst{a}rmi vęrr, &
\ind eða mín \alst{s}ystir \alst{s}é.\eva

\bvb 161\evb
\evg


\bvg
\bva Nú eru \alst{H}áva mál kveðin \hld\ \alst{H}áva\alst{h}ǫllu í &
\ind \alst{a}llþǫrf \alst{ý}ta sonum, &
\ind \alst{ó}þǫrf \edtext{\alst{jǫ}tna}{\lemma{jǫtna}\Afootnote{ýta \emph{corrected in margin} \Regius}} sonum; &
hęill sá’s \alst{k}vað, \hld\ hęill sá’s \alst{k}ann, &
\ind \alst{n}jóti sá’s \alst{n}am, &
\ind \alst{h}ęilir þęir’s \alst{h}lýddu.\eva

\bvb Now are the speeches of the High One sung, in the hall of the High One, of great need for the sons of men, of harm for the sons of ettins! Hail he who sang, hail he who knows! May he benefit who took, hail they who heeded!\evb
\evg
% — Weden
%	\book{From the Sons of king Reeding. (Frá sonum Hrauðungs konungs)}\bookStart

\bva Hrauðungr konungr átti tvá sonu. Hét annarr Agnarr, enn annarr Geirrøðr.
\bva Agnarr var tíu vetra enn Geirrøðr átta vetra. Þeir reru tveir á báti með dorgar sínar at smáfiski.
\bva Vindr rak þá í haf út. Í náttmyrkri brutu þeir við land ok gingu upp; fundu kotbónda einn.
\bva Þar vǫ́ru þeir um vetrinn. Kerling fostraði Agnar enn karl Geirrøð.
\bva At vári fekk karl þeim skip. Enn er þau kerling leiddu þá til strandar, þá mælti karl einmæli við Geirrøð.
\bva Þeir fengu byr ok kvǫ́mu til stǫðva fǫður síns. Geirrøðr var fram í skipi.
\bva Hann hljóp upp á land enn hratt út skipinu, ok mælti: ”Far þú þar er smyl hafi þik.”
\bva Skipit rak út. Enn Geirrøðr gekk út til bǿjar; hánum var vel fagnat;
\bva þá var faðir hans andaðr. Var þá Geirrøðr til konungs tekinn, ok varð maðr ágætr.

\bvb King Reeding owned two sons. One was called Eynhere, and the other Garred.
\bvb Eynhere was ten winters old, and Garred eight winters. The two were rowing in a boat with their trolling-lines for small fishing.
\bvb Wind then drove them out into the sea. In the darkness of night they crashed into land and walked up; they found a lone cottage-farmer.
\bvb There they were about the winter. The wife fostered Eynhere, but the husband Garred.
\bvb At spring the man got them ships. But when the woman led them to the shore, the husband spoke privately with Garred.
\bvb They got favourable wind, and came to their father's harbour. Garred was in the front of the ship.
\bvb He leapt up onto land and pushed out the ship, and spoke: ”Go thou where the \textbf{smil} might have thee.”
\bvb The ship drove out. But Garred walked towards the farm; he was welcomed well;
\bvb his father had by then drawn his final breath. Then was Garred taken as king, and became an excellent man.

\bva Óðinn ok Frigg sátu í Hliðskjǫlfu ok sá um heima alla.
\bva Óðinn mælti: Sér þú Agnar fóstra þinn, hvar hann elr bǫrn við gýgi í hellinum? 
\bva En Geirrøðr, fóstri minn, er konungr ok sitr nú at landi.
\bva Frigg segir: Hann er matníðingr sá at hann kvelr gesti sína ef hánum þykkja ofmargir koma.
\bva Óðinn segir at þat er in mesta lygi. Þau veðja um þetta mál.
\bva Frigg sendi eskismey sína, Fullu, til Geirrøðar. Hon bað konung varask at eigi fyrgerði hánum fjǫlkunnigr maðr sá er þar var kominn í land ok sagði þat mark á at engi hundr var svá olmr at á hann myndi hlaupa.
\bva En þat var inn mesti hégómi at Geirrøðr væri eigi matgóðr ok þó lætr hann handtaka þann mann er eigi vildu hundar á ráða.
\bva Sá var í feldi blám ok nefndisk Grímnir ok sagði ekki fleira frá sér þótt hann væri atspurðr.
\bva Konungr lét hann pína til sagna ok setja milli elda tveggja ok sat hann þar átta nætr.
\bva Geirrøðr konungr átti son tíu vetra gamlan ok hét Agnarr eftir bróður hans.
\bva Agnarr gekk at Grímni ok gaf hánum horn fullt at drekka, sagði að konungr gerði illa er hann lét pína hann saklausan.
\bva Grímnir drakk af. Þá var eldrinn svá kominn at feldrinn brann af Grímni. Hann kvað: 

\bvb Weden and Frie sat in Litheshelf and looked about all the Homes.
\bvb Weden spoke: Seest thou Eynhere thy foster-son, where he begets children with the troll-woman in the cave?
\bvb But Garred, my foster-son, is king and now sits at land.
\bvb Frie says: He is such a meat-nithing that he tortures his guests if he thinks there are too many of them.
\bvb Weden says that this is the greatest lie; they make a bet about this matter.
\bvb Frie sent her handmaid Full to Garred's. She asked the king to be wary, that he might not be ended by that fealcunning man who was come in the land, and said that his mark was that no hound were so fierce that he would leap onto him.
\bvb But that was the greatest vainglory that Garred would not be meat-good, and yet he has that man seized, whom the dogs would not touch.
\bvb He was clad in a blue cloak, and called himself Grimen, and did not tell any more about himself, even though he was interrogated.
\bvb The king had him tortured so that he would speak, and set him between two fires, and he remained there for eight nights.
\bvb King Garred had a son ten winters old, and he was named Eynhere after his brother.
\bvb Eynhere walked up to Grimen, and gave him a full horn to drink, saying that the king did ill as he had him tortured without cause.
\bvb Grimen drank from it; then the fire had come such that the cloak burned on Grimen. He quoth:
% — Weden
%	\book{\emph{Grímnismǫ́l} — The Speeches of Grimner.}\bookStart

\bvg
\bva Hęitr est hripuðr \hld ok hęldr til mikill, &
\ind gǫngumk firr funi. &
Loði sviðnar, \hld þótt á lopt bera'k; &
\ind brinnumk feldr fyrir.\eva

\bvb Hot art thou, rippeth <= fire>, and rather too large; go far from me, fire! The woolen cape is singed, though I hold it aloft; the cloak burns before me.\evb
\evg


\bvg
\bva Átta nætr satk \hld milli ęlda hér, &
\ind svát mér mangi \hld mat né bauð &
nema ęinn Agnarr, \hld es ęinn skal ráða, &
\ind Gęirrøðar sonr, \hld Gotna landi.\eva

\bvb For eight nights sat I between the fires here, while no man offered me food, but for lone Eyner, who lone shall rule, that son of Garred, the land of the Gots!\evb
\evg


\bvg
\bva Hęill skalt, Agnarr, \hld alls hęilan biðr &
\ind þik Veratýr vesa; &
ęins drykkjar \hld þú skalt aldrigi &
\ind bętri gjǫld geta.\eva

\bvb Whole shalt thou [be], Eyner, as whole thee Weretue <= Weden> bids to be; for one drink shalt thou never a better yield get.\evb
\evg


\bvg
\bva Land es hęilagt, \hld es liggja sé’k &
\ind ǫ́sum ok ǫlfum nær; &
ęn í Þrúðhęimi \hld skal Þórr vesa &
\ind unz of rjúfask ręgin.\eva

\bvb The land is holy, which I see lie close to the Ease and elves; but in Thritham shall Thunder be, until the Powers are rent.\footnoteB{As \Finnur\ points out, this disagrees with the later numbering. It then seems likely that this half-v. is out of place.}\evb
\evg


\bvg
\bva Ýdalir hęita, \hld þar’s Ullr of hęfr &
\ind sér of gǫrva sali; &
Alfhęim Fręy \hld gǫ́fu í árdaga &
\ind tívar at tannféi.\eva

\bvb Yewdales are called where Woulder has made himself a hall; Elfham to Free gave in days of yore the Tues as a tooth-gift\footnoteB{Agreeing with \Finnur, a gift that the child receives when he gets his first tooth.}.
\evg


\bvg
\bva Bǿr ’s hinn þriði, \hld es blíð ręgin &
\ind silfri þǫkðu sali; &
Valaskjǫlf hęitir, \hld es vélti sér &
\ind ǫ́ss í árdaga.\eva

\bvb Bower is the third, where the blithe Powers with silver thatched a hall; Waleshelf is called, where tricked himself, the os in days of yore.\evb
\evg


\bvg
\bva Søkkvabękkr hęitir hinn fjórði, \hld ęn þar svalar knegu &
\ind unnir glymja yfir; &
þar þau Óðinn ok Sága \hld drekka umb alla daga &
\ind glǫð ór gollnum kęrum.\eva

\bvb Sinkbench is called the fourth, but there cool waves do clash above; there Weden and Sey drink all days, gladly, out of golden vats.\evb
\evg


\bvg
\bva Glaðshęimr hęitir hinn fimti \hld þar’s hin gollbjarta &
\ind Valhǫll víð of þrumir; &
ęn þar Hroptr \hld kýss hvęrjan dag &
\ind vápndauða vera.\eva

\bvb Gladsham is called the fifth, where the gold-bright Walhall, wide, stands fast; but there Roft <= Weden> chooses every day weapon-dead men.\evb
\evg


\bvg
\bva Mjǫk ’s auðkęnt \hld þęim’s til Óðins koma &
\ind salkynni at séa, &
skǫptum ’s rann rępt, \hld skjǫldum ’s salr þakiðr, &
\ind brynjum of bękki stráat.\eva

\bvb Greatly easily recognized, for those who to Weden come, is the hall to see: With shafts is the house roofed; with shields is the hall thatched; with byrnies the benches strewn.\evb
\evg


\bvg
\bva Mjǫk ’s auðkęnt \hld þęim’s til Óðins koma &
\ind salkynni at séa, &
vargr hangir \hld fyr vestan dyrr &
\ind ok drúpir ǫrn yfir.\eva

\bvb Greatly easily recognized, for those who to Weden come, is the hall to see: A wolf hangs for the western door, and an eagle droops over.\evb
\evg


\bvg
\bva Þrymhęimr hęitir hinn sétti, \hld es Þjazi bjó, &
\ind sá hinn ámátki jǫtunn; &
ęn nú Skaði byggvir, \hld skír brúðr goða, &
\ind fornar toptir fǫður.\eva

\bvb Thrimham is called the sixth, where Thedse dwelled, that terrifying ettin; but now Scathe bedwells — the pure bride of the gods — the ancient plots of her father.\evb
\evg


\bvg
\bva Bręiðablik eru hin sjaundu, \hld ęn þar Baldr hęfir &
\ind sér of gǫrva sali, &
á því landi \hld es liggja vęit’k &
\ind fæsta fęiknstafi.\eva

\bvb Broadblicks are the seventh, and there Bolder has made himself a hall; on that land, where I know lie the fewest staves of treachery\footnoteB{Evil deeds.}.\evb
\evg


\bvg
\bva Himinbjǫrg eru in áttu \hld ęn þar Hęimdall &
\ind kveða valda véum. &
þar vǫrðr goða \hld drękkr í væru ranni 6
\ind glaðr góða mjǫð.\eva

\bvb Heavenbarrows are the eighth, and there Homedall, they say, wields over wighs. There in the tranquil house the ward of the gods drinks glad the good mead.\evb
\evg


\bvg
\bva Folkvangr es inn níundi \hld en þar Fręyja ræðr &
\ind sessa kostum í sal; &
halfan val \hld hon kýss hvęrjan dag &
\ind ęn halfan Óðinn á.\eva

\bvb Folkwong is the ninth, and there Frow wields the choice of seats in the hall; half of the slain she chooses each day, but half Weden owns.\evb
\evg


\bvg
\bva Glitnir es inn tíundi; \hld hann es gulli studdr &
\ind ok silfri þakðr it sama; &
ęn þar Forseti \hld byggir flęstan dag &
\ind ok svæfir allar sakir.\eva

\bvb Glitner is the tenth, it is studded by gold, and thatched by silver the same; but there Forset dwells most of the day, and resolves\footnoteB{Puts to sleep,} all [legal] matters.\evb
\evg


\bvg
\bva Nóatún eru in ęlliftu \hld ęn þar Njǫrðr hęfir
\ind sér um gǫrva sali,
manna þęngill \hld inn męinsvani
\ind hátimbruðum hǫrgi ræðr.\eva

\bvb Nowetowns are the tenth, and there Nearth has made himself a hall. The prince of men, the guileless one, rules the high-timbered \inx{harrow}.\evb



\bvg
\bva Hrísi vęx \hld ok há grasi
\ind Víðars land, viði,
ęn þar mǫgr of læzk \hld af mars baki
\ind frǿkn at hęfna fǫður.\eva

\bvb TO-DO.\evb
\evg


\bvg
\bva Andhrímnir \hld lætr í Ęldhrímni
\ind Sæhrímni soðinn,
flęska bęzt, \hld ęn þat fáir vitu
\ind við hvat ęinhęrjar alask.\eva

\bvb Andrimen lets in Eldrimen Sowrimen be boiled. Of pork the best, but few know that; by what the Lone Warriors are nourished.\footnotemark[1]\evb
\footnotetext[1]{The cook Andrimen 'face-sooty' has the boar Sowrimen 'sow-sooty' boiled in the cauldron Eldrimen 'fire-sooty'; by this meat are the Lone Warriors nouished. See the index.}
\evg


\bvg
\bva Gera ok Freka \hld sęðr gunntamiðr,
\ind hróðigr Hęrjafǫðr,
ęn við vín ęitt \hld vápngǫfugr
\ind Óðinn æ lifir.\eva

\bvb Gerr and Freck satiates the battle-accustomed, glorious Father of Hosts; yet by wine alone, the weapon-worshipful Weden ever lives.\evb
\evg


\bvg
\bva Huginn ok Muninn \hld fljúga hvęrjan dag
\ind jǫrmungrund yfir;
óumk of Hugin, \hld at aptr né komit;
\ind þó séumk męir of Munin.\eva

\bvb Highen and Minden fly every day, over the ermin-ground. I fear for Highen, that he not come back, yet I worry more for Minden.\evb
\evg


\bvg
\bva Þýtr Þund, \hld unir Þjóðvitnis
\ind fiskr flóði í;
áarstraumr \hld þykkir ofmikill
\ind valglaumi at vaða.\eva

\bvb Thound howls, the fish of Thedwitner is content in the flood; the river-stream seems far too great for the cheery slain host to wade.\evb
\evg

...


\bvg
\bva Askr Yggdrasils, \hld hann es ǿztr viða
\ind ęn Skíðblaðnir skipa,
Óðinn ása \hld ęn jóa Slęipnir,
Bilrǫst brúa \hld ęn Bragi skalda,
Hábrók hauka \hld ęn hunda Garmr.\eva

\bvb The ash of Ugdrassle, that is the noblest of trees, but Shidebladner of ships; Weden of the Ease, but of horses Slopner; Bilrest of bridges, but Bray of scolds; Highbrook of hawks, but of hounds Garm.\evb
\evg

% — Weden
%	Svá sęgja menn í fornum sǫgum, at ęinnhvęrr af ǫ́sum, sá es Heimdallr hét, fór fęrðar sinnar ok framm með sjóvarstrǫndu nǫkkurri, kom at ęinum húsabǿ ok nefndisk Rígr; ęptir þęiri sǫgu es kvæði þetta.

Thus men say in ancient saws†, that one of the Ease†, he who was called Homedall, went on his journey and forth along some lake-shore, arrived to one  homestead and called himself Righ. According to that that saw is this poem:

Ár kvǫ́ðu ganga \hld grǿnar brautir
ǫflgan ok aldinn \hld ǫ́s kunnigan,
ramman ok rǫskvan \hld Ríg stíganda. 

Yore they said, did walk the green paths a powerful and aged os†, cunning; the strong and quick Righ, striding.

Gekk hann męir at þat \hld miðrar brautar,
kom hann at húsi, \hld hurð var á gætti;
inn nam at ganga, \hld eldr var á golfi,
hjón sǫ́tu þar \hld hǫ́r at arni,
Ái ok Edda \hld aldinfalda. 

Went he further at that, on the middle of the road; came he to a house, the door was wide open; he began to walk inside, fire was on the floor; a couple sat there, hoary by the hearth: Great Grandfather and Great Grandmother, old-fashioned.
% — Weden
%	\bookStart{The Leed of Hoarbeard}[Hárbarðsljóð]

In my opinion the poem can be seen as an allegory on class relations, namely between the self-owning Norwegian and later Icelandic farmers, and the warlike Norwegian earls.

Of all Eddic poems this one is probably the strangest in terms of form. Verse length varies greatly, and many of the lines (see especially the final verse) are of an obscene length reminiscent of late continental Germanic poems like the Heliand; some simply have no metrical qualities at all. The young clitic definite is (uniquely) employed frequently throughout the poem. These criteria would seem to point towards a late origin for the poem (though not later than the late 13th century, when \Regius\ was written).

Against this late origin speaks the presence of rare words (e.g. \emph{ǫgurr} v. 13) and a thorough understanding of the personalities of the two gods which would seem unlikely to stem from several centuries after the conversion of Iceland. The model devised by Sapp gives the poem a 57.8\% likelihood of being from the early 11th century, and a 37.7\% likelihood of being from the late 11th. These scores are most similar to those obtained by \Gripisspa, a poem that on the surface seems much more archaic.

What could we then be dealing with? It may of course be that the poet is heavily corrupt, but there is really no good evidence for this (apart from the above-mentioned irregularities). Most lines are readily understandable and fit well within their respective context and the poem as a whole. I think a better solution to this problem is that the poem has been acted out as a sort of carnivalesque theatre, with two masked actors, each playing one of the gods. This would explain the variations in meter and line length, and the prose; some lines were simply shouted out, and the lack of alliteration in these still gives a powerful, discordant effect when read aloud.

This is shown also by uses of the word ‘here’ in vv. 9 and 14. TODO: mention concept of "double scene" by Lars Lönnroth?


\sectionline


\bpg
\bpa Þórr fór ór austrvegi ok kom at sundi einu. Ǫðrum megum sundsins var ferjukarlinn með skipit. Þórr kallaði:\epa

\bpb Thunder journeyed out of the eastern ways and came to a sound. At the other side of the sound was the ferryman with the ship. Thunder called out:\epb
\epg


\bvg
\bva „Hvęrr ’s sá svęinn svęina \hld\ es stęndr fyr sundit handan?“\eva

\bvb “Who is that swain of swains, that stands across the sound?”\evb
\evg


\bvg
\bva Hann svaraði:
„Hvęrr ’s sá karl karla \hld\ es kallar of váginn?“\eva

\bvb He answered:
“Who is that churl of churls, that calls out over the wave?”\evb
\evg


\bvg
\bva „Fęr þú mik of sundit, \hld\ fǿði’k þik á morgun; &
męis hęfi’k á baki, \hld\ verðr-a matrinn bętri. &
Át ek í hvíld \hld\ áðr ek hęiman fór, &
síldr ok hafra; \hld\ saðr em’k ęnn þęss.“\eva

\bvb [Thunder:]
“Ferry me over the sound, I feed thee in the morning! A basket I have on my back; the food does not get better.\footnoteB{i.e. ‘you will not get better food than that.’} I ate for a while before I journeyed from home, herring and hegoats; I am still full from that.”\evb
\evg


\bvg
\bva „Árligum verkum \hld\ hrósar þú vęrðinum; &
\ind vęizt-at-tu fyr gǫrla, &
dǫpr ’ru þín hęimkynni, \hld\ dauð hygg’k at þín móðir sé.“\eva

\bvb “[In place of] early works boastest thou of thy eating! Thou knowest not the future clearly; dismal is the state of thy home, dead I think thy mother might be.”\evb
\evg


\bvg
\bva „Þat sęgir þú nú \hld\ es hvęrjum þikkir &
męst at vita— \hld\ at mín móðir dauð sé.“\eva

\bvb “Thou now sayest that which to each man seems most important to know: that my mother might be dead!”\evb
\evg


\bvg
\bva „Þęygi ’s sem þú \hld\ þrjú bú ęigir góð; &
bęrbęinn þú stęndr \hld\ ok hęfir brautinga gørvi, &
\ind þat-ki at þú hafir brę́kr þínar.“\eva

\bvb “’Tis hardly as if thou might own three good homesteads; bare-legged thou standest, and hast the gear of a tramp; ’tis not even as if thou have thy own breeches!”\evb
\evg


\bvg
\bva „Stýrðu hingat ęikjunni, \hld\ ek mun þér stǫðna kęnna &
eða hvęrr á skipit \hld\ es þú hęldr við landit?“\eva

\bvb “Steer hither the boat! I will show thee to the harbour—or who owns the ship which thou holdest by the shore?”\evb
\evg


\bvg
\bva „Hildólfr sá hęitir \hld\ es mik halda bað, &
rekkr inn ráðsvinni \hld\ es býr í Ráðsęyjarsundi; &
bað-at hann hlęnnimęnn flytja \hld\ eða hrossaþjófa, &
góða ęina \hld\ ok þá’s ek gørva kunna; &
sęg-ðu til nafns þíns \hld\ ef þú vill of sundit fara.“\eva

\bvb “Hildolf is called he who asked me to hold it, the counsel-wise man who lives in Redeseysound. He did not bid me to carry thief-men, nor horse-thiefs; good men only, and those whom I know well—state thy name if thou wilt fare o’er the sound!”\evb
\evg


\bvg
\bva „Sęgja mun’k til nafns míns \hld\ þótt ek sękr sjá’k &
ok til alls øðlis: \hld\ Ek em Óðins sonr, &
Męila bróðir \hld\ ęn Magna faðir, &
þrúðvaldr goða \hld\ við Þór knátt-u hér dǿma!
Hins vil’k nú spyrja \hld\ hvat þú hęitir?“\eva

\bvb “I will state my name—[and would] even if I were outlawed—and all my origin: I am Weden’s son, Male’s brother and Main’s father, the strength-wielder of the Gods; with Thunder thou here speakest! This I will now ask, what thou art called?”\evb
\evg


\bvg
\bva „Hárbarðr ek hęiti, \hld\ hyl’k of nafn sjaldan.“\eva

\bvb “Hoarbeard I am called, seldom I conceal my name.”\evb
\evg


\bvg
\bva „Hvat skalt-u of nafn hylja \hld\ nema þú sakar ęigir?“\eva

\bvb “Why shalt thou conceal thy name, unless thou be guilty of crime?”\evb
\evg


\bvg
\bva „En þótt ek sakar ęiga \hld\ fyr slíkum sem þú est &
þá mun’k forða fjǫrvi mínu \hld\ nema ek fęigr sé.“\eva

\bvb “Even though I were guilty of crime, for such a one as thou art I would still protect by life, unless I be \inx[C]{fey}.”\evb
\evg


\bvg
\bva „Harm ljótan mér þikkir í því &
at vaða of váginn til þín \hld\ ok vę́ta \edtrans{ǫgur}{burden}{\Bfootnote{The sense of this word is not clear, though it is probably the same as the first element of the compound \emph{ǫgurstund} ‘burdensome hour’, found in \Volundarkvida\ 42. Some authors have read it as a crude euphemism for ‘penis’, which would not be out of character for this poem. I however consider the best interpretation to be that of an author whose name I've forgotten (TODO!), namely that Thunder is referring to the food he carries on his back (cf. v. 3).}} minn; &
skylda’k launa kǫgursveini þínum kanginyrði \hld\ ef ek komumk yfir sundit.“\eva

\bvb “An ugly harm it seems to me to wade o’er the wave to thee, and wet my burden. I would repay thee, swaddle-swain, for thy mocking words if myself I could bring over the sound.”\evb
\evg


\bvg
\bva „Hér mun ek standa \hld\ ok þín heðan bíða; &
fannt-a-tu mann inn harðara \hld\ at Hrungni dauðan.“\eva

\bvb “Here I will stand, and hence await thee; thou foundest not a harder man since the death of \inx[P]{Rungner}!\footnoteB{Rungner was an ettin slain by Thunder, TODO. Hoarbeard’s mentioning of him sets off a long interchange, wherein the two boast of their deeds, and ask what the other one was doing meanwhile.}”\evb
\evg


\bvg
\bva „Hins vilt-u nú geta \hld\ es vit Hrungnir dęildum, &
sá inn stórúðgi jǫtunn, \hld\ es ór stęini vas hǫfuðit á, &
þó lét’k hann falla \hld\ ok fyr hníga; &
\ind hvat vannt-u þá meðan, Hárbarðr?“\eva

\bvb “This wilt thou now mention, of when I and Rungner dealt with each other; that great-minded ettin on which the head was made of stone. Yet I let him fall, and sink down before [me]—what didst thou then meanwhile, Hoarbeard?”\evb
\evg


\bvg
\bva „Vas’k með Fjǫlvari \hld\ fimm vetr alla &
í ęy þeiri \hld\ er Algrǿn hęitir; &
vega vér þar knǫ́ttum \hld\ ok val fęlla, &
margs at fręista, \hld\ mans at kosta.“\eva

\bvb “I was with Felwar for five winters all in that island which Allgreen is called. There we knew to fight, and fell corpses; many to tempt, a girl to win.\footnoteB{I read \emph{margs} ‘many a’ as modifying \emph{mans} ‘girl’, thus giving ‘(we knew) to tempt and to win many a girl’.}”\evb
\evg


\bvg
\bva „Hversu snúnuðu yðr konur yðrar?“\eva

\bvb “How did your women pleasure (TODO!!!) you?.\footnoteB{Seemingly a prose line; see Introduction.}”\evb
\evg


\bvg
\bva „Sparkar ǫ́ttum vér konur \hld\ ef oss at spǫkum yrði; &
horskar ǫ́ttum vér konur \hld\ ef oss hollar vę́ri, &
þę́r ór sandi \hld\ síma undu &
\ind ok ór dali djúpum &
\ind grund of grófu; &
varð’k þęim ęinn ǫllum \hld\ øfri at rǫ́ðum; &
\ind hvílda’k hjá systrum sjau &
\ind ok hafða’k gęð þęira allt ok gaman;
\ind hvat vannt-u þá meðan, Þórr?“\eva

\bvb “We \ken*{I} owned frisky women, if they were pleasing towards us \ken*{me}; we \ken*{I} owned wise women, if they were \inx[C]{hold} towards us \ken*{me}; out of the sand a rope they wound, and out of a deep dale dug up the ground; I alone became superior to all of them in counsels; I rested by those sisters seven, and had their senses all, and pleasure—what didst thou then meanwhile, Thunder?”\evb
\evg


\bvg
\bva „Ek drap Þjaza, \hld\ hinn þrúðmóðga jǫtun, &
upp ek varp augum \hld\ Allvalda sonar &
\ind á þann hinn hęiða himin; &
þau ’ru męrki męst \hld\ minna verka, &
\ind þau’s allir męnn síðan of sé; &
\ind hvat vannt-u þá meðan, Hárbarðr?“\eva

\bvb “I slew \inx[C]{Thedse}, the strength-minded ettin; up I threw the eyes of the son of Allwald \ken*{= Thedse} onto that clear heaven; those are the greatest marks of my works, those that all men since do see\footnoteB{We here have a rare example of native Germanic star-lore. Is the exact constellation identifiable? TODO.}—what didst thou then meanwhile, Hoarbeard?”\evb
\evg


\bvg
\bva „Miklar manvélar \hld\ hafða’k við myrkriður &
\ind þá’s ek vélta þę́r frá verum; &
harðan jǫtun \hld\ hugða’k Hlébarð vesa; &
\ind gaf hann mér gambantęin &
\ind en ek vélta hann ór viti.“\eva

\bvb “Great girl-tricks I used against \inx[C]{murkriders}, when I tricked them away from their husbands.\footnoteB{Alternatiely ‘away from men’. The \emph{riður} ‘(female) riders’ were witches thought to torment people and cause disease and suffering. See \Havamal\ 154 for a more detailed explanation.} A hard ettin I judged Leebeard to be; he gave me a \inx[C]{gombentoe}, but I tricked him out of his wits.”\evb
\evg


\bvg
\bva „Illum huga launaðir þú þá góðar gjafar.“\eva

\bvb “With an evil mind rewardedst thou that good gift.”\evb
\evg


\bvg
\bva „Þat hęfir ęik \hld\ es af annarri skęfr; &
\ind umb sik es hvęrr í slíku; &
\ind hvat vannt-u þá meðan, Þórr?“\eva

\bvb “An oak has that which it scrapes from another; each is for himself in such [a matter]—what didst thou then meanwhile, Thunder?”\evb
\evg


\bvg
\bva „Ek vas austr \hld\ ok jǫtna barða’k &
brúðir bǫlvísar \hld\ es til bjargs gengu; &
mikil myndi ę́tt jǫtna \hld\ ef allir lifði, &
vę́tr myndi manna \hld\ undir Miðgarði; &
hvat vannt-u þá meðan, Hárbarðr?\eva

\bvb “I was in the east, and ettins I fought; bale-wise brides who walked to the mountain. Great would the lineage of ettins be if all lived; naught would remain of men within Middenyard—what didst thou then meanwhile, Hoarbeard?”\evb
\evg


\bvg
\bva „Vas’k á Vallandi \hld\ ok vígum fylgða’k, &
atta ek jǫfrum \hld\ en aldrigi sę́tta’k; &
Óðinn á jarla \hld\ þá’s í val falla &
\ind en Þórr á þrę́la kyn.“\eva

\bvb “I was in \inx[L]{Walland} and followed conflicts; I incited princes, and never reconciled them. Weden owns the earls which fall among the slain, but Thunder owns the kin of thralls.\footnoteB{We see here a sort of aristocratic, Odinic disregard for lower life and life as a good in itself; where Thunder boasts of saving men, Weden sarcastically responds that he caused the deaths of men so that he could have them for himself.}”\evb
\evg


\bvg
\bva „Ójafnt skipta \hld\ es þú myndir með ǫ́sum liði &
\ind ef þú ę́ttir vilgi mikils vald.“\eva

\bvb “Translation.”\evb%TODO: There’s something very weird going on here.
\evg


\bvg
\bva „Þórr á afl ǿrit \hld\ en ękki hjarta; &
af hrę́ðslu ok hugblęyði \hld\ þér vas í hanzka troðit &
\ind ok þóttisk-a þú þá Þórr vesa; &
hvárki þú þá þorðir \hld\ fyr hrę́ðslu þinni &
hnjósa né físa \hld\ svá’t Fjalarr hęyrði.“\eva

\bvb “Thunder owns ample strength, but no heart; out of fear and mind-softness didst thou tread into a glove, and then seemedest thou not to be Thunder. Thou daredest neither—for thy fear—to sneeze nor to fart so that Feller might hear [it].\footnoteB{This story is also referenced in \Lokasenna\ 60. It is elaborated heavily on in \Gylfaginning\ 45: Thunder, Lock, and the siblings Thelve and Wrash had travelled east for a long time when they discovered a large hall, with an opening on one end, as wide as the building. They took rest inside, but in the middle of the night there was a great earthquake and the ground beneath them trembled. Thunder rose and led the party to a side-room to the right in the middle of the hall. He sat closest to the opening with his hammer ready, while the others sat terrified further inside. At daybreak they left the hall and found a huge ettin named \emph{Skrymir} (\inx[P]{Shrimer}) sleeping next to them. His snoring had caused the earth-quakes, and the hall was his mitten; the side-room was the thumb-part.}”\evb
\evg


\bvg
\bva „Hárbarðr hinn ragi, \hld\ munda’k þik í Hęl drepa &
\ind ef mę́tta’k sęilask of sund.“\eva

\bvb “Hoarbeard the \inx[C]{degenerate}, I would strike thee into \inx[L]{Hell}, if I might sail o’er the sound!”\evb
\evg


\bvg
\bva „Hvat skyldir of sund sęilask \hld\ es sakir ’ru allz øngar? &
\ind hvat vannt-u þá meðan, Þórr? “\eva

\bvb “Why should thou sail o’er the sound when there are no offenses?—what didst thou then meanwhile, Thunder?”\evb
\evg


\bvg
\bva „Ek vas austr \hld\ ok ána varða’k &
þá’s mik sóttu \hld\ þęir Svárangs synir; &
grjóti mik bǫrðu, \hld\ gagni urðu þó lítt fęgnir, &
þó urðu mik fyrri \hld\ friðar at biðja. &
\ind hvat vannt-u þá meðan, Hárbarðr?“\eva

\bvb “I was in the east, and warded the river, when the sons of Sweering attacked me. With rocks they fought me, yet they rejoiced little in victory; yet they earlier had to beg me for peace—what didst thou then meanwhile, Hoarbeard?”\evb
\evg


\bvg
\bva „Ek var austr \hld\ ok við ęinhvęrja dǿmða’k, &
lék’k við ina lindhvítu \hld\ ok lǫng þing háða’k, &
gladda’k ina gullbjǫrtu, \hld\ gamni mę́r unði.“\eva

\bvb “I was in the east, and with a certain woman conversed; I played with the linen-white one, and held long \inx[C]{Thing}[Things]; I gladdened the gold-bright one; the maiden enjoyed pleasure.”\evb
\evg


\bvg
\bva „Góð ǫ́ttu þęir mankynni þar þá.“\eva

\bvb “Then they had good girl-visits there.”\evb
\evg


\bvg
\bva „Liðs þíns vę́ra’k þá þurfi, Þórr, \hld\ at hęlda’k þęiri inni línhvítu męy.“\eva

\bvb “Of thy help I might have been in need then, Thunder, that I might hold that linen-white maiden.”\evb
\evg


\bvg
\bva „Ek mynda þér þá þat vęita \hld\ ef ek viðr of kę́misk.“\eva

\bvb “I would then have granted thee that, if I were able.”\evb
\evg


\bvg
\bva „Ek mynda þér þá trúa, \hld\ nema mik í tryggð véltir.“\eva

\bvb “I would then have trusted thee, unless thou betrayed my trust.”\evb
\evg


\bvg
\bva „Em’k-at ek sá hę́lbítr \hld\ sem húðskór forn á vár.“\eva

\bvb “I am not such a heel-biter as an old hide-shoe in spring.\footnoteB{Proverbial (a heel-biter being someone who betrays his companions); the leather of a shoe would become very stiff and chafing over the winter.}”\evb
\evg


\bvg
\bva „Brúðir bersęrkja \hld\ barða’k í Hléseyju; &
þę́r hǫfðu vęrst unnit, \hld vélta þjóð alla.“\eva

\bvb “The brides of bearserks I fought in Leesie; they had done the worst: deceived a whole people.”\evb
\evg


\bvg
\bva „Klę́ki vannt-u þá, Þórr, \hld\ es þú á konum barðir.“\eva

\bvb “A great disgrace didst thou then, Thunder, when thou foughtst women.”\evb
\evg


\bvg
\bva „Vargynjur vǫ́ru þę́r \hld\ en varla konur, &
skęlldu skip mitt \hld\ es ek skorðat hafða’k, &
ǿgðu mér járnlurki \hld\ en ęltu Þjálfa. &
hvat vannt-u þá meðan, Hárbarðr?“\eva

\bvb “She-wolves were they, but hardly women; they knocked my ship which I had propped; frightened me with an iron-cudgel, but chased Thelve around—what didst thou then meanwhile, Hoarbeard?”\evb
\evg


\bvg
\bva „Ek vas’k í hęrnum \hld\ es hingat gjǫrðisk &
gnę́fa gunnfana, \hld\ gęir at rjóða.“\eva

\bvb “I was in the army, as hence it made ready to raise the war-standard; to redden the spear.”\evb
\evg


\bvg
\bva „Þess vilt-u nú geta, es þú fórt oss \edtext{óljúfan}{\footnoteB{oliyfan \AM; †olubann† \Regius}} at bjóða.“\eva

\bvb “This wilt thou now mention, as thou wentest to bid us \ken*{= the Ease} hatred!”\evb
\evg


\bvg
\bva „Bǿta skal þér þat þá \hld\ munda baugi &
sem jafnęndr unnu \hld\ þęir’s okkr vilja sę́tta.“\eva

\bvb “I will then restore thee for that with a hand-bigh, like the settlers [have] considered, those who wish to reconcile us.”\evb
\evg


\bvg
\bva „Hvar namt þęssi \hld\ in hnǿfiligu orð &
es ek hęyrða aldrigi \hld\ hnǿfiligri?“\eva

\bvb “Where learnedest thou these sarcastic words, as I never heard more sarcastic ones?”\evb
\evg


\bvg
\bva „Nam’k at mǫnnum þęim inum aldrǿnum es búa í hęimisskógum.“\eva

\bvb “I learned them from the old men who dwell in the home-forests.”\evb
\evg


\bvg
\bva „Þó gefr þú gótt nafn dysjum, es þú kallar þat hęimisskóga.“\eva

\bvb “Yet thou givest a good name to poor cairns,\footnoteB{cf. his waking the dead in various poems TODO.} as thou callest them home-forests.”\evb
\evg


\bvg
\bva „Svá dǿmi’k of slíkt far.“\eva

\bvb “So I speak about such things.”\evb
\evg


\bvg
\bva „Orðkringi þín \hld\ mun þér illa koma &
\ind ef ek rę́ð á vág at vaða; &
ulfi hę́ra \hld\ hygg’k at ǿpa mynir &
ef hlýtr af hamri hǫgg.“\eva

\bvb “Thy word-glibness will bring thee evil, if I resolve to wade on the wave; higher than a wolf I think that thou wilt scream, if thou suffer a strike from the hammer.”\evb
\evg


\bvg
\bva „Sif á hó hęima, \hld\ hans munt fund vilja, &
þann munt þręk drýgja, \hld\ þat ’s þér skyldara.“\eva

\bvb “Sib has a whoremonger at home, him wilt thou wish to meet; then shalt thou use thy strength, that is thee more befitting!”\evb
\evg


\bvg
\bva „Mę́lir þú at munns ráði \hld\ svá’t mér skyldi vęrst þikkja, &
halr inn hugblauði, \hld\ hygg’k at þú ljúgir.“\eva

\bvb “Thou speakest to the counsel of thy mouth that which would seem me the worst; heart-soft man, I think that thou liest!”\evb
\evg


\bvg
\bva „Satt hygg’k mik sęgja, \hld\ sęinn est at fǫr þinni, &
langt myndir nú kominn, Þórr, \hld\ ef þú \edtrans{litum fǿrir}{brought thy colours}{\Bfootnote{Very unclear expression. \emph{fǿra litum} TODO.}}.“\eva

\bvb “I think myself to speak truly: late art thou in thy journey; far would thou now be come, Thunder, if thou had brought thy colours.”\evb
\evg


\bvg
\bva „Hárbarðr inn ragi, \hld\ hęldr hęfir nú mik dvalðan!“\eva

\bvb “Hoarbeard the degenerate; thou hast now delayed me greatly!”\evb
\evg


\bvg
\bva „Ása-Þórs \hld\ hugða’k aldrigi myndu &
\ind glępja féhirði farar.“\eva

\bvb “The journey of Thunder of the Ease I never thought that a shepherd \ken*{= I} would divert.”\evb
\evg


\bvg
\bva „Ráð mun’k þér nú ráða: \hld\ Ró þú hingat bátinum, &
hę́ttum hǿtingi, \hld\ hitt fǫður Magna!“\eva

\bvb “I will now counsel thee a counsel: Row hither the boat; seize with the taunting; come to the father of Main \ken*{= Thunder = me}!”\evb
\evg


\bvg
\bva „Far þú firr sundi, \hld\ þér skal fars synja!“\eva

\bvb “Go far from the sound; the ferry shall be denied thee!”\evb
\evg


\bvg
\bva „Vísa þú mér nú lęiðina \hld\ allz þú vill mik ęigi of váginn fęrja!“\eva

\bvb “Show me now the path, as thou wilt not ferry me o’er the wave!”\evb
\evg


\bvg
\bva „Lítit ’s at synja, \hld\ langt ’s at fara; &
stund ’s til stokksins, \hld\ ǫnnur til stęinsins, &
halt svá til vinstra vegsins \hld\ unz þú hittir Verland; &
þar mun Fjǫrgyn \hld\ hitta Þór, son sinn,
ok mun hǫ́n kęnna hǫ́num ǫ́ttunga brautir \hld\ til Óðins landa.“\eva

\bvb “’Tis little to deny, ’tis long to journey: an hour to the log, another to the stone; hold thus to the left road, until thou findest Wereland; there will Firgyn find Thunder, her son, and she will teach him the highways of her ancestors, to Weden’s lands \ken*{= Osyard}.”\evb
\evg


\bvg
\bva „Mun’k taka þangat í dag?“\eva

\bvb “Will I come thither today?”\evb
\evg


\bvg
\bva „Taka við víl ok ęrfiði \hld\ at uppverandi sólu &
es ek get þána.“\eva

\bvb “[Thou wilt] come with toil and hardship at the rising of the sun, as I think it might thaw.”\evb
\evg


\bvg
\bva „Skammt mun nú mál okkat vesa, \hld\ allz þú mér skǿtingu ęinni svarar; &
launa mun ek þér farsynjun \hld\ ef vit finnumk í sinn annat. &
Far þú nú þar’s þik hafi allan gramir!“\eva

\bvb “Short will now our speech be, as thou answerest me with scoffing alone; I will reward thee for this ferry-denial if we meet another time. Now go, whither the fiends may have all of thee!”\evb
\evg
% — Weden, Thunder
%	\bookStart{\emph{Hymiskviða} — The Lay of Hymer.}

% Introduction.
Attested in two manuscripts, \Regius\ and \AM. The two are surprisingly consistent.

Þórr dró Miðgarðsorm. % TO-DO: Format as header.

Thunder pulled up the Middenyardsworm.\footnotetext{This is the only title the poem has in \Regius. \AM\ has the proper title \emph{Hymiskviða} instead.}


\bvg
\bva Ár valtívar \hld vęiðar nǫ́mu &
ok sumblsamir \hld áðr saðir yrði, &
hristu tęina \hld ok á hlaut sǫ́u, &
fundu þęir at Ægis \hld ørkost hvera.\eva

\bvb Of yore the slaughter-Tues had caught game\footnoteB{Lit. ‘took game’}, and banqueting before they might eat\footnoteB{Lit. ‘might become sated’}, they shook the twigs and looked at the \inx{leat}; they found at Eyer’s a great choice of cauldrons.\footnoteB{The gods sprinkled the leat (sacrificial blood) of the beasts and interpreted the pattern; they found it most auspicious to feast at Eyer’s.}\evb
\evg


\bvg
\bva Sat bergbúi \hld barntęitr fyr, &
mjǫk glíkr męgi \hld Miskorblinda, &
lęit í augu \hld Yggs barn í þrá: &
„þú skalt ǫ́sum \hld opt sumbl \edtext{gęra}{\lemma{gęra “host”}\Afootnote{gefa “give” \AM}}!“\eva

\bvb — Sat the mountain-dweller \ken{Eyer}[1] there, joyous like a child, much like the lad of Misherblind\footnoteB{A reference to a lost myth? Unless Misherblind is an alternative name for Firneet, Eyer’s father.}; into his eyes looked the child of Ug <= Weden> \ken{Thunder}[1] in defiance: “Thou shalt for the Ease oft’ host banquets!”\footnoteB{Having seen that Eyer has a great store of cauldrons, Thunder orders him to host future banquets for the Ease.}\evb
\evg


\bvg
\bva Ǫnn fekk jǫtni \hld orðbæginn halr, &
hugði at hefndum \hld hann næst við goð, &
bað hann Sifjar ver \hld sér fǿra hver, &
„þann’s ek ǫllum ǫl \hld yðr of hęita.“\eva

\bvb Great toil for the ettin the word-peevish man \ken{Thunder}[1] caused; thought he of revenge, soon, against the god: asked he Sib’s husband to bring him a cauldron, “that one with which I for you all ale might brew.”\footnoteB{Eyer asks Thunder to find a single cauldron which can hold enough ale to supply all the Ease.}
\evg


\bvg
\bva Né þat mǫ́ttu \hld mærir tívar &
ok ginnręgin \hld of geta hvęrgi, &
unz af tryggðum \hld Týr Hlórriða &
ástráð mikit \hld ęinum sagði:\eva

\bvb But that might the renowned Tues and the \inx{Gin-Reins} nowhere get ahold of, until out of loyalty, a great word of loving advice Tue to Loride <= Thunder> alone did say:\evb
\evg


\bvg
\bva „Býr fyr austan \hld Élivága &
hundvíss Hymir \hld at himins ęnda, &
á minn faðir \hld móðugr kętil, &
\edtext{rúmbrugðinn}{\Afootnote{‘rumbrygðan’ \AM}} hver \hld rastar djúpan.“\eva

\bvb “Lives to the east of the Ilewaves the houndwise Hymer, at the end of heaven. Owns my father\footnoteB{Hymer being Tue’s father.}, fierce, a kettle; a size-renowned cauldron one \inx{rest} deep.”\evb
\evg


\bvg
\bva „Veiztu, ef þiggjum \hld þann lǫgvelli?“ &
„Ef, vinr, vélar \hld vit gørvum til!“\eva

\bvb “Knowest thou if we will receive that ale-boiler?” — “If, friend, we two make use of wiles!”\footnoteB{The speakers are not indicated, but it is most sensible that Thunder asks and Tue answers.}\evb
\evg

\bvg
\bva Fóru drjúgum \hld \edtext{dag þann framan}{\lemma{dag þann framan “from the beginning of the day”}\Afootnote{\emph{Emendation from Finnur 1932}; dag þann fram “on that day forth” \Regius; dag fráliga “swiftly at day” \AM}} &
Ásgarði frá \hld unz til \edtext{Ęgils}{\lemma{Ęgils “Agle’s”}\Afootnote{\emph{thus} \Regius; Ægis “Eyer’s” \AM; — \AM\ \emph{reading possibly from confusion with Eyer described earlier in the poem, but or the shepherd did share his name.}}} kvǫ́mu. &
Hirði hann hafra \hld horngǫfgasta; &
hurfu at hǫllu \hld es Hymir átti.\eva

\bvb — They travelled with great strides from the beginning of the day, from Osyard, until to Agle’s they came—he herded bucks with the noblest of horns—they turned to the hall which Hymer owned.\evb
\evg


\bvg
\bva Mǫgr fann ǫmmu, \hld mjǫk lęiða sér, &
hafði hǫfða \hld hundruð níu. &
ęn ǫnnur gekk \hld algollin framm &
brúnhvít bera \hld bjórvęig syni.\eva

\bvb The lad found his grandmother greatly loathsome; she had of heads nine hundred. But another woman, all-golden, stepped forth: white-browed, she carried a beer-draught for the son \ken{Tue}[1].\evb
\evg


\bvg
\bva „Áttniðr jǫtna \hld ek vilja’k ykr &
hugfulla tvá \hld und hvera sętja; &
es mínn \edtext{fríi}{\lemma{fríi “lover”}\Afootnote{\emph{thus} \Regius; faðir “father” \AM}} \hld mǫrgu sinni &
gløggr við gęsti \hld gǫrr ills hugar.“\eva

\bvb “Kinsman of ettins! I would wish to set you high-mettled two under the cauldrons; my lover has many a time been stingy against guests, quick to ill temper.”\footnoteB{Tue’s mother (the all-golden woman in previous v.) wishes to hide him and Thunder, lest her husband (Hymer) find them.}\evb
\evg


\bvg
\bva Ęn váskapaðr \hld varð \edtext{síðbúinn}{\Afootnote{\emph{om.} \AM}}, &
harðráðr Hymir, \hld hęim af vęiðum; &
gekk inn í sal, \hld glumðu jǫklar, &
vas karls, es kom, \hld kinnskógr frørinn.\eva

\bvb But the misshapen one was come late—the hard-minded Hymer—home from the hunt. He entered the hall—icicles clattered—frozen was the cheek-forest \ken{beard} of the churl who came.\evb
\evg


\bvg
\bva „Ves þú hęill, Hymir, \hld í hugum góðum! &
Nú ’s sonr kominn \hld til sala þinna, &
sá’s vit vættum \hld af vęgi lǫngum; &
fylgir hǫ́num \hld Hróðrs andskoti, &
vinr verliða; \hld Véurr hęitir sá.\eva

\bvb “Be thou hale, Hymer, in good spirits!\footnoteB{Formula identically mirrored in runic inscription N B380: \emph{Heill sé þú / ok í hugum góðum. / Þórr þik þiggi, / Óðinn þik eigi.} “May thou be hale, and in good spirits! May Thunder receive thee, may Weden own thee.” Cf. also \Beowulf\ l. 407: \emph{Wæs þú Hróðgár hál!} “Be thou, Rothgar, hale!”} Now the son is come to thy halls, the one whom we two have been expecting, from a long way off. Follows him the opponent of Rooder <ettin> \ken{Thunder}[1], the friend of manly retinues \ken{Thunder}[1]; Wighward he is called.\evb
\evg


\bvg
\bva Sé þú hvar sitja \hld und salar gafli, &
svá \edtext{forða sér}{\Afootnote{forðask \AM}}, \hld stęndr \edtext{súl}{\Afootnote{‘sol’ \AM}} fyrir.“ &
Sundr stǫkk súla \hld fyr sjón jǫtuns, &
ęn \edtext{allr}{\Afootnote{áðr \Regius\AM TODO: elaborate, mention Finnur}} í tvau \hld áss brotnaði.\eva

\bvb See where they sit, ’neath the hall’s gable: thus they hide themselves—a pillar stands before them!” The pillars sprang asunder before the sight of the ettin, but all in two the beam was broken.\evb
\evg


\bvg
\bva Stukku átta, \hld ęn ęinn af þęim &
hverr harðslęginn \hld hęill af þolli; &
framm gingu þęir, \hld ęn forn jǫtunn &
sjónum lęiddi \hld sinn andskota.\eva

\bvb Eight\footnoteB{Eight kettles.} sprung apart, but one of them, a hard-forged kettle, [came] whole off its peg\footnoteB{Presumably the one in which Tue and Thunder were hiding.}. Forth went they, but the ancient ettin with his sight beheld\footnoteB{Literally “led with his sight”.} his opponent.\evb
\evg


\bvg
\bva Sagðit hǫ́num \hld hugr vęl þá’s sá &
gýgjar \edtext{grǿti}{\lemma{grǿti “distresser”}\Afootnote{gæti “keeper, warder” \AM}} \hld á golf kominn, &
þar vǫ́ru þjórar \hld þrír of tęknir, &
bað \edtext{sęnn}{\Afootnote{‘sun’ \AM}} jǫtunn \hld sjóða ganga.\eva

\bvb His heart was not pleased then, when he saw the distresser of Gows <ettin-women> \ken{Thunder}[1] come on the floor. There were three bulls taken: the ettin at once bade them be cooked.\evb
\evg


\bvg
\bva Hvęrn létu þęir \hld hǫfði skęmra &
ok á sęyði \hld síðan bǫ́ru, &
át Sifjar verr \hld áðr sofa gingi, &
ęinn með ǫllu \hld øxn tvá Hymis.\eva

\bvb Each one they let shorten by a head, and onto the fire-pit then carried: ate the husband of Sib \ken{Thunder}[1], before he might go to sleep, alone all together two of Hymer’s oxen.\evb
\evg


\bvg
\bva Þótti hǫ́rum \hld Hrungnis spjalla &
verðr Hlórriða \hld vęl fullmikill, &
„munum at aptni \hld ǫðrum verða &
við vęiðimat \hld vér þrír lifa.“\eva

\bvb To the hoary friend of Rungner \ken{Hymer}[1] seemed Loride’s eating far too large; “we must this evening eat something else: by hunted game we three shall live.\Bfootnote{A clear example of inhospitality, illustrating the otherness of the Ettins. See introduction to the poem.}”\evb
\evg
% — Thunder, Tue
%	\bookStart{The Lay of Thrim}[Þrymskviða]

% Introduction

Compare \Haustlong, \Hymiskvida, other poems and refer to the SkP intro to one of the big Thunder poems. TODO.

\bvg
\bva \edtext{\alst{V}ręiðr}{\lemma{Vręiðr}\Afootnote{TODO: Note about ambiguity of alliteration.}} vas þá Ving-Þórr \hld\ es hann vaknaði &
ok síns hamars \hld\ of saknaði, &
skegg nam at hrista, \hld\ skǫr nam at dýja, &
réð Jarðar burr \hld\ umb at þreifask.\eva

\bvb Wroth was then Wing-Thunder when he woke, and of his hammer was bereaved. His beard he took to shake, his locks he took to pull; resolved the son of Earth to look about.\evb
\evg


\bvg
\bva Ok hann þat orða \hld\ allz fyrst of kvað: &
“Hęyrðu nú, Loki, \hld\ hvat ek nú mę́li &
es ęigi vęit \hld\ jarðar hvęrgi &
né upphimins: \hld\ áss es stolinn hamri!”\eva

\bvb And he that word first of all did speak: “Hear thou now, Lock, what I now speak, which nowhere is known, not on earth nor \inx[L]{Up-heaven}:\footnoteB{A common Germanic poetic formula, see Index: \inx[L]{Earth and Up-heaven}.} the \inx[G]{Ease}[os] \ken*{= Thunder = I} has been robbed of his hammer!”\evb
\evg


\bvg
\bva Gengu þęir fagra \hld\ Fręyju túna &
ok hann þat orða \hld\ allz fyrst of kvað: &
“Muntu mér, Fręyja, \hld\ fjaðrhams ljá &
ef ek mínn hamar \hld\ mę́tta’k hitta?”\eva

\bvb Went they to the fair yards of \inx[P]{Frow}, and he that word, first of all did speak: “Wilt thou me, Frow, the \inx[P]{feather-hame} lend, if I my hammer might find?”\evb
\evg


\bvg {\small [Frow quoth:]}
\bva “Þó mynda’k gefa þér \hld\ þótt ór gulli vę́ri &
ok þó sęlja \hld\ at vę́ri ór silfri.”\eva

\bvb “I would yet give it to thee, though it were out of gold, and yet offer\footnoteB{\emph{sęlja} ‘sell’ here has its earlier meaning, cf. Gothic \emph{saljan} ‘\emph{opfern}; θύειν’ (Streitberg 1910:116).} it to thee, as it were out of silver.”\footnoteB{Regaining the hammer is of such importance to the gods (cf. v. 17; without it the Ease stand powerless against the \inx[G]{Ettins}), that Frow would lend the feather-hame to the greedy and untrusty Lock, even if it were made out of solid gold or silver.}\evb
\evg

\bvg
\bva Fló þá Loki, \hld\ fjaðrhamr dunði, &
unz fyr útan kom \hld\ ása garða &
ok fyr innan kom \hld\ jǫtna hęima.\eva

\bvb Flew then Lock\footnoteB{Though Thunder is the one asking for the hame (“if I \emph{my} hammer might find”), Lock is the one that takes off flying.}—the feather-hame rustled—until outside he came of the \inx[L]{Osyard}[yards of the Ease], and inside he came of the \inx[L]{Ettinham}[homes of the Ettins].\evb
\evg


\bvg
\bva Þrymr sat á haugi, \hld\ þursa dróttinn, &
gręyjum sínum \hld\ gullbǫnd snøri &
ok mǫrum sínum \hld\ mǫn jafnaði.\eva

\bvb Thrim sat on the howe, the lord of \inx[G]{Thurses}: on his greyhounds the golden leashes he twirled, and on his mares the manes he cut even.\evb
\evg


\bvg
\bva „Hvat es með ǫ́sum? \hld\ Hvat es með ǫlfum? &
Hví estu ęinn kominn \hld\ í jǫtunhęima?“ &
„Illt es með ǫ́sum, \hld\ \edtext{illt es með ǫlfum!}{\Bfootnote{Inserted in analogy with the first pair, regardless it is needed for metrical reasons.}} &
Hęfir þú Hlórriða \hld\ hamar of folginn?“\eva

\bvb [Thrim quoth:] “What is with the Ease? What is with the elves? Why art thou alone come into the \inx[L]{Ettinham}[Ettin-homes]?” — [Lock quoth:] “’Tis ill with the Ease, ’tis ill with the elves! Hast thou the hammer of Loride \name{= Thunder} hidden?”\evb
\evg


\bvg {\small [Thrim quoth:]}
\bva „Ek hęfi Hlórriða \hld\ hamar of folginn &
átta rǫstum \hld\ fyr jǫrð neðan; &
hann ęngi maðr \hld\ aptr of hęimtir &
nęma fǿri mér \hld\ Fręyju at kvę́n.“\eva

\bvb “I have the hammer of Loride hidden, eight \inx[C]{rest}[rests] beneath the earth; it no man will fetch again, unless he bring me Frow as wife.”\evb
\evg


\bvg
\bva Fló þá Loki, \hld\ fjaðrhamr dunði, &
unz fyr útan kom \hld\ jǫtna hęima &
ok fyr innan kom \hld\ ása garða; &
mǿtti hann Þór \hld\ miðra garða &
ok þat hann orða \hld\ allz fyrst of kvað:\eva

\bvb Flew then Lock—the feather-hame rustled—until outside he came of the homes of the Ettins, and inside he came of the yards of the Ease. He met Thunder in the middle of the yards, and he \ken*{= Thunder} that word first of all did say:\evb
\evg


\bvg {\small [Thunder quoth:]}
\bva „Hęfir þú ørendi \hld\ sem ęrfiði? &
Segðu á lopti \hld\ lǫng tíðendi! &
Opt sitjanda \hld\ sǫgur of fallask &
ok liggjandi \hld\ lygi of bęllir.“\eva

\bvb “Hast thou an errand of hardship?\footnoteB{lit. “Hast thou an errand, as hardship?” Thunder asks Lock if he has bad news.} Say thou aloft, the long tidings! Often sitting, tales fail each other, and lying down, lies are dealt.”\footnoteB{Proverbial. If one sits down and thinks too much over bad news, details will be left out, excuses thought up. Thus it is best that Lock immediately tell Thunder what he has learned.}\evb
\evg


\bvg {\small [Lock quoth:]}
\bva „Hefi ek ørindi \hld\ erfiði ok: &
Þrymr hęfir þinn hamar, \hld\ þursa dróttinn; &
hann ęngi maðr \hld\ aptr of hęimtir &
nęma hǫ́num fǿri \hld\ Fręyju at kvę́n.“\eva

\bvb “I have an errand, hardship also: Thrim has thy hammer, the lord of Thurses; it no man will fetch again, unless he bring him Frow as wife.”\evb
\evg


\bvg
\bva Ganga þęir fagra \hld\ Fręyju at hitta &
ok hann þat orða \hld\ allz fyrst of kvað: &
„Bittu þik, Fręyja, \hld\ brúðar líni! &
Vit skulum aka tvau \hld\ í jǫtunhęima.“\eva

\bvb Go they the fair Frow to find, and he\footnoteB{Unclear. Possibly Lock, since he was the speaker of the last verse.} that word, first of all did say: “Bind thee, Frow, with a bride’s linen\footnoteB{A linen band tied around the bride’s head. TODO: Reference this note.}! We two shall drive into the Ettin-homes.”\evb
\evg


\bvg
\bva Vręið varð þá Fręyja \hld\ ok fnasaði, &
allr ása salr \hld\ undir bifðisk, &
stǫkk þat it mikla \hld\ męn Brísinga: &
„Mik vęizt verða \hld\ vergjarnasta &
ef ek ęk með þér \hld\ í jǫtunhęima.“\eva

\bvb Wroth became then Frow, and snorted—the whole hall of the Ease trembled below—threw she off the great necklace of the Brisings: “Thou knowest that I will become the most man-eager,\footnoteB{Either Frow is speaking out of self-awareness of her own lust, or the sense is that she will be accused of being lustful by the other gods, but there is no verb here corresponding to ‘accuse’.} if I drive with thee into the Ettin-homes.”\evb
\evg


\bvg
\bva Sęnn vǫ́ru ę́sir \hld\ allir á þingi &
ok ǫ́synjur \hld\ allar á máli, &
ok of þat réðu \hld\ ríkir tívar: &
hvé þęir Hlórriða \hld\ hamar of sǿtti.\eva

\bvb Soon were the \inx[G]{Ease} all at the \inx[C]{Thing}, and the \inx[C]{Ossens} all at speech, and of this counseled the mighty \inx[G]{Tues}:\footnoteB{Identical to \Baldrsdraumar\ 1.} how they the hammer of Loride would seek out.\evb
\evg


\bvg
\bva Þá kvað þat Heimdallr, \hld\ hvítastr ása, &
vissi hann vel framm \hld\ sęm vanir aðrir: &
„Bindu vér Þór þá \hld\ brúðar líni; &
hafi hann it mikla \hld\ męn Brísinga!\eva

\bvb Then quoth that \inx[P]{Homedall}, the whitest of the Ease; he knew well forth,\footnoteB{\emph{vita framm} ‘to know forward’ i.e. to know the future. Compare \emph{framvíss} ’forth-wise; prescient.’} like the other \inx[G]{Wanes}: “Let us bind Thunder with the bride’s linen; may he have the great \inx[P]{necklace of the Brisings}.\evb
\evg


\bvg
\bva Lǫ́tum und hǫ́num \hld\ hrynja lukla &
ok kvenváðir \hld\ umb kné falla &
en á brjósti \hld\ bręiða stęina &
ok hagliga \hld\ umb hǫfuð typpum!“\eva

\bvb Let us place by his side keys to jingle, and women’s garments to fall down about his knees, and on the breast broad stones, and skillfully let us tip his head!\footnoteB{This verse contains an interesting description of Viking age bridal dress: As the everyday manager of the household, keys were the mark of a respectable married woman. The “broad stones” on the breast are probably tortoise brooches, while the tipping of the head refers to some sort of bridal hat (TODO: Literature). Breast-brooches are also mentioned in \Volundarkvida\ 25, 36.}”\evb
\evg


\bvg
\bva Þá kvað þat Þórr, \hld\ þrúðugr áss: &
„Mik munu ę́sir \hld\ argan kalla &
ef ek bindask lę́t \hld\ brúðar líni!“\eva

\bvb Then quoth that Thunder, the mighty os: “Me would the Ease call \inx[C]{degenerate}, if I let myself be bound with bride’s linen!”\evb
\evg


\bvg
\bva Þá kvað þat Loki \hld\ Laufęyjar sonr: &
„Þęgi þú, Þórr, \hld\ þęira orða! &
Þegar munu jǫtnar \hld\ Ásgarð búa &
nęma þú þinn hamar \hld\ þér of hęimtir.“\eva

\bvb Then quoth that Lock, the son of Leafie: “Shut thou, Thunder, those words up! Shortly the Ettins will settle Osyard, unless thou thy hammer for thyself dost fetch!”\evb
\evg


\bvg
\bva Bundu þęir Þór þá \hld\ brúðar líni &
ok inu mikla \hld\ męni Brísinga, &
létu und hǫ́num \hld\ hrynja lukla &
ok kvenváðir \hld\ umb kné falla &
ęn á brjósti \hld\ bręiða stęina &
ok hagliga \hld\ of hǫfuð typpðu.\eva

\bvb Bound they Thunder then, with bride’s linen, and with the great necklace of the Brisings. They placed by his side keys to jingle, and women’s garments to fall down about his knees, and on the breast broad stones, and skillfully they tipped his head.\evb
\evg


\bvg
\bva Þá kvað þat Loki \hld\ Laufęyjar sonr: &
„Mun ek ok með þér \hld\ ambǫ́tt vesa, &
vit skulum aka tvau \hld\ í jǫtunhęima.“\eva

\bvb Then quoth that Lock, the son of Leafie: “I will also with thee be a handmaid; we two\footnoteB{The form used, \emph{tvau}, is the neuter plural, ie. one of the pair is female and the other male. This is either an error due to mindless copying of v. 11, or a backhanded insult against Thunder.} shall drive into the Ettin-homes.”\evb
\evg


\bvg
\bva Sęnn vǫ́ru hafrar \hld\ hęim of vreknir, &
skyndir at skǫklum, \hld\ skyldu vel renna; &
bjǫrg brotnuðu, \hld\ brann jǫrð loga; &
ók Óðins sonr \hld\ í jǫtunhęima.\eva

\bvb Soon \inx[C]{he-goats}\footnoteB{Thunder’s cart was driven by he-goats, and he is likewise called “the lord of he-goats” in \Hymiskvida\ 20, 31. See Index.} were driven home, hasted onto the cart-poles; they were to run well. Crags burst, the earth burned with flame; the son of Weden \ken*{= Thunder} drove into the Ettin-homes.\footnoteB{A very similar but more detailed description of Thunder driving is found in Thedwolf’s \Haustlong\ 14–16. In both poems his wagon is drawn by he-goats, causing great cosmic disturbance: crags (\emph{bjǫrg} in both) are rent asunder and fires rage before him. See also \Baldrsdraumar\ 3 for a related description of Weden riding.}\evb
\evg


\bvg
\bva Þá kvað þat Þrymr, \hld\ þursa dróttinn: &
„Standið upp, jǫtnar, \hld\ ok stráið bękki! &
Nú fǿrið mér \hld\ Fręyju at kván, &
Njarðar dóttur \hld\ ór Nóatúnum.\eva

\bvb Then quoth that Thrim, the lord of Thurses: “Stand ye up, ettins, and strew the benches! Now bring me Frow as wife; the daughter of \inx[P]{Nearth} of the \inx[L]{Nowetowns}.\evb
\evg


\bvg
\bva Ganga hér at garði \hld\ gullhyrnðar kýr, &
øxn alsvartir, \hld\ jǫtni at gamni, &
fjǫlð á’k męiðma, \hld\ fjǫlð á’k męnja; &
ęinnar mér Fręyju \hld\ ávant þykkir.“\eva

\bvb Here march to the estate golden-horned cows, all-black oxen, to the enjoyment of the ettin \ken*{= me}. A great deal I own of treasures, a great deal I own of necklaces; of Frow alone methinks is missing.”\evb
\evg


\bvg
\bva Vas þar at kveldi \hld\ of komit snimma &
ok fyr jǫtna \hld\ ǫl framm borit. &
Ęinn át oxa, \hld\ átta laxa, &
krásir allar, \hld\ þę́r’s konur skyldu, &
drakk Sifjar verr \hld\ sáld þrjú mjaðar.\eva

\bvb There was the evening come quickly, and before the ettins ale brought forth. Ate he \ken*{= Thunder} one ox, eight salmons, and all the delicacies which were meant for the women; drank the husband of Sib \ken*{= Thunder} three sieves of mead.\footnoteB{Compare \Hymiskvida\ 15 for a similar description of Thunder’s great eating.}\evb
\evg


\bvg
\bva Þá kvað þat Þrymr, \hld\ þursa dróttinn: &
„Hvar sáttu brúðir \hld\ bíta hvassara? &
Sá’k-a brúðir \hld\ bíta ęnn bręiðara &
né ęnn męira mjǫð \hld\ męy of drekka!“\eva

\bvb Then quoth that Thrim, the lord of Thurses: “Where sawest thou brides bite sharper? Saw I never brides bite yet broader, nor yet more mead a maiden drink.”\evb
\evg


\bvg
\bva Sat in alsnotra \hld\ ambǫ́tt fyr &
es orð of fann \hld\ við jǫtuns máli: &
„Át vę́tr Fręyja \hld\ átta nóttum, &
svá vas hón óðfús \hld\ í jǫtunhęima.“\eva

\bvb Sat the allclever maid-servant \ken*{= Lock} in front, when she a word did find against the speech of the ettin: “Ate Frow naught, for eight nights; so madly was she longing for the Ettin-homes.”\evb
\evg


\bvg
\bva Laut und línu, \hld\ lysti at kyssa, &
ęn hann útan stǫkk \hld\ ęndlangan sal: &
„Hví eru ǫndótt \hld\ augu Fręyju? &
Þykki mér ór \hld\ augum brenna!“\eva

\bvb He looked ’neath the linen, he lusted for a kiss, but he from the outside leapt back, across the length of the hall: “Why are the eyes of Frow fiery? Methinks there is flame coming out of the eyes!\footnoteB{Lit. “Methinks out of the eyes burn.”}”\evb
\evg


\bvg
\bva Sat in alsnotra \hld\ ambǫ́tt \edtext{fyrir}{\Afootnote{‘ſ.’ \emph{add.} \Regius \emph{possibly a lost word}}} &
es orð of fann \hld\ við jǫtuns máli: &
„Svaf vę́tr Fręyja \hld\ átta nóttum, &
svá vas hón óðfús \hld\ í jǫtunhęima.“\eva

\bvb Sat the allclever maid-servant \ken*{= Lock} in front, when she a word did find against the speech of the ettin: “Slept Frow naught, for eight nights; so madly was she longing for the Ettin-homes.”\evb
\evg


\bvg
\bva Inn kom in arma \hld\ jǫtna systir, &
hin es brúðfjár \hld\ biðja þorði: &
„Láttu þér af hǫndum \hld\ hringa rauða &
ef þú ǫðlask vill \hld\ ástir mínar, &
ástir mínar, \hld\ alla hylli!“\eva

\bvb In came the wretched sister of the ettins, the one who for the bride-price had dared ask: “Take off from thy hands the red rings, if thou wilt win my loves; my loves, [and] all [my] \inx[C]{holdness}.”\footnoteB{The sister, who already asked for the hammer, now has the audacity to ask Thunder (still disguised as Frow) to give her the very rings on his hands.}\evb
\evg


\bvg
\bva Þá kvað þat Þrymr, \hld\ þursa dróttinn: &
„Berið inn hamar \hld\ brúði at vígja, &
leggið Mjǫllni \hld\ í męyjar kné, &
vígið okkr saman \hld\ Várar hęndi!“\eva

\bvb Then quoth that Thrim, the lord of Thurses: “Bear ye in the hammer, the bride to bless; lay Millner in the maiden’s knee, bless us two together by the hand of \inx[P]{Ware}!\footnoteB{A minor goddess presumably presiding over marriage. See Index.}”\evb
\evg


\bvg
\bva Hló Hlórriða \hld\ hugr í brjósti &
es harðhugaðr \hld\ hamar of þękkði; &
Þrym drap hann fyrstan, \hld\ þursa dróttin, &
ok ę́tt jǫtuns \hld\ alla lamði.\eva

\bvb The heart of Loride laughed in his breast, when, hard-hearted, he recognized the hammer. Thrim he slew first, the lord of Thurses, and all the lineage of the ettin he thrashed.\evb
\evg


\bvg
\bva Drap hann ina ǫldnu \hld\ jǫtna systur, &
hin es brúðfjár \hld\ of beðit hafði; &
hón skell of hlaut \hld\ fyr skillinga &
en hǫgg hamars \hld\ fyr hringa fjǫlð.\eva

\bvb He slew the old sister of the ettins, the one who for the bride-price had asked; she received a smiting before shillings, and a strike of the hammer before a multitude of rings.\evb
\evg


\bvg
\bva Svá kom Óðins sonr \hld\ ęndr at hamri.\eva

\bvb Thus Weden’s son regained his hammer.\evb
\evg
% — Thunder
%	\bookStart{The Leed of Hindle}[Hyndluljóð]

\bvg
\bva „Vaki mę́r męyja, \hld\ vaki mín vina, &
Hyndla systir, \hld\ es í hęlli býr; &
nú ’s røkr røkra, \hld\ ríða vit skulum &
til Valhallar \hld\ ok til vés hęilags.\eva

\bvb Frow quoth:
“Wake maiden of maidens, wake my friend, sister Hindle, who lives in the rock-face. Now is the twilight of twilights, we two shall ride to Walhall, and to the holy wigh†!\evb
\evg


\bvg
\bva Biðjum Hęrjafǫðr \hld\ í hugum sitja, &
hann geldr ok gefr \hld\ gull \edtext{verðugum}{\Afootnote{verðungu ‘to the retinue’ \emph{emend.} \FinnurEdda\ \GudniEdda}}, &
gaf hann Hęrmóði \hld\ hjalm ok brynju, &
ęn Sigmundi \hld\ sverð at þiggja.\eva

\bvb Let us bid the Father of Hosts \ken{Weden}[1] to be in his favour; he rewards and gives gold to the worthy. Gave he to Heremood helmet and byrnie, but Sighmund a sword to receive.\evb
\evg


\bvg
\bva Gefr hann sigr sumum\footnotetext[1], \hld\ ęn sumum\footnotetext[2] aura, &
mę́lsku mǫrgum \hld\ ok manvit firum, &
byri gefr brǫgnum, \hld\ ęn brag skǫldum, &
gefr hann mannsęmi \hld\ mǫrgum rekki. &
\footnotetext[1] ms. \emph{sonum}
\footnotetext[2] ms. \emph{suinnum}\eva

\bvb He gives victory to some, but to some silver\footnotemark[1]; speech to many, and manwit to men. Fair wind he gives to noble ones, and poetry to scolds†; he gives valour to many a champion.
\footnotemark[1] Lit. "ounces".\evb
\evg


\bvg
\bva Þór munk blóta, \hld\ þess munk biðja, &
at hann ę́ við þik \hld\ einart láti; &
þó ’s hǫ́num ótítt \hld\ við jǫtuns brúðir.\eva

\bvb To Thunder I will bloot†, of this I will bid, that he always show friendliness to thee, though he is prejudiced against the brides of the ettins\footnotemark[1].
\footnotetext[1] Lit. “though [it] is to him infrequent with ettin's brides”.\evb
\evg


\bvg
\bva Nú taktu ulf þinn \hld\ ęinn af stalli, &
lát hann rinna \hld\ með runa mínum.“ &
[Hyndla kvað:] „Sęinn es gǫltr þinn \hld\ goðveg troða, &
vilkat mar minn \hld\ mę́tan hlǿða.\eva

\bvb Now take thy single wolf from the stable; let him run with my boar.” [Hindle quoth:] “Slow is thy boar, to tread the Godways; I wish not lade my dear steed.”\evb
\evg


\bvg
\bva Flǫ́ est Fręyja, \hld\ es fręistar mín, &
visar þú augum \hld\ á oss þannig, &
es hafir ver þinn \hld\ í valsinni &
Óttar unga \hld\ Innsteins bur.“\eva

\bvb Deicitful art thou, Frow, as thou temptest me; thou showest thy eyes on us this way, as thou hast thy man on the Walways: the young Oughthere, Instone's offspring.”\evb
\evg


\bvg
\bva Fręyja kvað:
„Dulið est Hyndla, \hld\ draums ę́tlak þér, &
es kveðr ver minn \hld\ í valsinni.\eva

\bvb Frow quoth:
Thou art foolish, Hindle, I think thee dreamy, who sayest that my man is on the Walways.\evb
\evg


\bvg
\bva Þar’s gǫltr glóar \hld\ Gullinbursti, &
Hildisvíni, \hld\ es mér hagir gęrðu, &
dvergar tvęir \hld\ Dáinn ok Nabbi.\eva

\bvb Where the boar glows, Goldenbristle; the hildswine\footnotemark[1], which the skillful for me made: the two dwarves Dowen and Nab.
\footnotemark[1] \emph{Hildisvíni} 'battle-swine', in this case probably an alternative name for Goldenbristle.\evb
\evg


\bvg
\bva Sęnn í sǫðlum \hld\ sitja vit skulum &
ok of jǫfra \hld\ ę́ttir dǿma, &
gumna þęira, \hld\ es frá goðum kómu.\eva

\bvb Soon in the saddles we two shall sit, and judge about the aughts† of princes, of those men who came from the gods.\evb
\evg


\bvg
\bva Þęir hafa vęðjat \hld\ Vala malmi &
Óttarr ungi \hld\ ok Angantýr; &
skylt ’s at vęita, \hld\ svá’t skati hinn ungi & &
fǫðurlęifð hafi \hld\ ępt frę́ndr sína.\eva

\bvb They have wagered the Welsh ore [GOLD], young Oughter and Ongenthew; it is required to grant, so that the young prince might have the fatherly inheritance left behind by his kinsmen.\footnotemark[1]
\footnotemark[1] Lit. 'the father-remains after his kinsmen'. — Happening seems to be that Oughthere and Ongenthew each lay claim the inheritance. In order to settle the matter (in Oughthere's favour) Hindle must (\emph{skylt es} “it is required, obligated”) divulge (\emph{vęita} ‘to grant, to give away’) what she knows about his lineage.\evb
\evg


\bvg
\bva Hǫrg hann mér gęrði \hld\ hlaðinn stęinum; &
nú ’s grjót þat \hld\ at glęri orðit; &
rauð hann í nýju \hld\ nauta blóði; &
ę́ trúði Óttarr \hld\ á ǫ́synjur.\footnotemark[1]
\footnotemark[1] Frow argues yet further in favour of Oughthere, bringing up his piety shown towards the godesses.\eva

\bvb A harrow† he made for me, loaded with stones; now that stone-pile is become into glass. He reddened [it] in fresh blood of oxen; Oughthere ever trusted on the osennies†.\evb
\evg


\bvg
\bva Nú lát-tu forna \hld\ niðja talða &
ok uppbornar \hld\ ę́ttir manna &
hvat ’s Skjǫldunga, \hld\ hvat ’s Skilfinga, &
hvat ’s Ǫðlinga \hld\ hvat ’s Ylfinga & &
hvat ’s hǫldborit, \hld\ hvat ’s hęrsborit &
męst manna val \hld\ und Miðgarði?“\eva

\bvb Now let be recounted the ancient lines of kinsmen, and the upborn\footnote[1] aughts† of men: What is of the Shieldings? What is of the Shilvings? What is of the Athlings? What is of the Wolvings? What is born of hero? What is born of chief, the mightiest choice of men in Midyard?”
\footnote[1] Noble.\evb
\evg


\bvg
\bva „Þú est Óttarr \hld\ borinn Innstęini, &
ęn Innstęinn vas \hld\ Alfi inum gamla, &
Alfr vas Ulfi, \hld\ Ulfr Sę́fara, &
ęn Sę́fari \hld\ Svan inum rauða.\eva

\bvb Hindle quoth:
“Thou\footnote[1] art, Oughthere, born to Instone, but Instone was born to Elf the old, Elf to Wolf, Wolf to Seafare, but Seafare to Swan the red.
\footnote[1] Hindle, apparently in a trance-like state, speaks straight to Oughthere.\evb
\evg


\bvg
\bva Móður átti faðir þinn \hld\ męnjum gǫfga, &
hygg at héti \hld\ Hlédís gyðja, &
Fróði vas faðir þęirar, \hld\ ęn Fríund\footnotemark[1] móðir; &
ǫll þótti ę́tt sú \hld\ með yfirmǫnnum.\eva
\footnotemark[1] Emended from the meaningless ms. reading \emph{friaut}.

\bvb Thy father had thy mother, beautiful with neck-rings, I think that she was called Leedise yidde†. Frood was her father, but Friend her mother; all her aught seemed to be among overmen.\evb
\evg


\bvg
\bva Auði vas áðr \hld\ ǫflgastr manna, &
Halfdanr fyrri \hld\ hę́str Skjǫldunga, &
frę́g vǫ́ru folkvíg, \hld\ þaus framir gęrðu, &
hvarfla þóttu verk \hld\ með himins skautum.\eva

\bvb Ed was before [that] the most powerful of men, Halfdane earlier the highest of Shieldings. Renowned were the troop-battles which the famous ones performed; his <= Halfdane's> works seemed to travel around the corners of heaven.\evb
\evg


\bvg
\bva Ęflðisk við Ęymund \hld\ ǿztan manna &
ęn vá Sigtrygg \hld\ með svǫlum ęggjum, &
ęiga gekk Almvęig, \hld\ ǿzta kvinna, &
ólu þau ok ǫ́ttu \hld\ átján sonu.\eva

\bvb He <= Halfdane> became the in-law of Iemund\footnotemark[1], the noblest of men, but he slew Sightrue with cool edges. He went on to have Elmwey, the noblest of women; they begot and had eighteen sons.
\footnotemark[1] Lit. "[he] was strengthened by". Parallelism of "noblest of men/women" makes the meaning yet clearer. Elmwey was Iemund's daughter or sister.\evb
\evg


\bvg
\bva Þaðan eru Skjǫldungar, \hld\ þaðan eru Skilfingar, &
þaðan eru Ǫðlingar, \hld\ þaðan eru Ynglingar, &
þaðan es hǫldborit, \hld\ þaðan es hęrsborit, &
mest mannaval \hld\ und Miðgarði; &
alt ’s þat ę́tt þín, \hld\ Óttarr heimski.\eva

\bvb Thereof are the Shieldings! Thereof are the Shilvings! Thereof are the Inglings!\footnotemark[1] Thereof is born of hero! Thereof is born of chief, the mightiest choice of men in Midyard! That is all thy aught†, foolish Oughthere!”
\footnotemark[1] Note the contradiction with v. 12. Since the Inglings have already been mentioned (under the name Shilvings, of the difference between the two see the index), it seems likely that Wolvings is the original reading.\evb
\evg


\bvg
\bva Vas Hildigunnr \hld\ hęnnar móðir, &
Svǫ́fu barn \hld\ ok sę́konungs; &
alt ’s þat ę́tt þín, \hld\ Óttarr hęimski. &
varðar\footnote[1] at viti svá, \hld\ viltu ęnn lęngra?\eva
\footnote[1] Emended from ms. \emph{varði}.

\bvb Hildguth was her mother, the child of Swabe and Seaking; that is all thy aught†, foolish Oughthere!—It is meaningful that one might know thus; wilt thou [go] yet further?\evb
\evg


\bvg
\bva Dagr átti Þóru \hld\ dręngjamóður, &
ólusk í ę́tt þar \hld\ ǿztir kappar, &
Fraðmarr ok Gyrðr \hld\ ok Frekar báðir, &
Ámr ok Jǫsurmarr, \hld\ Alfr hinn gamli. &
varðar at viti svá, \hld\ viltu ęnn lęngra?\eva

\bvb Day had Thure, the mother of valiant men; in that aught were begotten the noblest champions: Fradmer and Yird, and both Frecks; Ame and Essirmer; Elf the old.—It is meaningful that one might know thus; wilt thou [go] yet further?\evb
\evg


\bvg
\bva Kętill hét vinr þęira \hld\ Klypps arfþęgi, &
vas hann móðurfaðir \hld\ móður þinnar; &
þar vas Fróði \hld\ fyrr ęnn Kári, &
ęn Hildi vas \hld\ Hóalfr of getinn.\eva

\bvb Kettle, the inheritor of Clip, was their friend; he was the father of thy mother's mother. There was Frood, yet earlier Keer, but Highelf was by Hild begotten.\evb
\evg

... %TODO More dialogue
% — Frow

% Heroic poems, in order of the Codex Regius
% \book{The Lay of Wayland. (\emph{Vǫlundarkviða})}\bookStart

\bva Níðuðr hét konungr í Svíþjóð.
\bva Hann átti tvá sonu ok eina dóttur. Hon hét Böðvildr.
\bva Bræðr váru þrír, synir Finnakonungs.
\bva Hét einn Slagfiðr, annarr Egill, þriði Völundr.
\bva Þeir skriðu ok veiddu dýr. Þeir kómu í Úlfdali ok gerðu sér þar hús.
\bva Þar er vatn, er heitir Úlfsjár.
\bva Snemma of morgin fundu þeir á vatnsströndu konur þrjár, ok spunnu lín.
\bva Þar váru hjá þeim álftarhamir þeira. Þat váru valkyrjur.
\bva Þar váru tvær dætr Hlöðvés konungs, Hlaðguðr svanhvít ok Hervör alvitr, in þriðja var Ölrún Kjársdóttir af Vallandi.
\bva Þeir höfðu þær heim til skála með sér. Fekk Egill Ölrúnar, en Slagfiðr Svanhvítrar, en Völundr Alvitrar.
\bva Þau bjuggu sjau vetr. Þá flugu þær at vitja víga ok kómu eigi aftr.
\bva Þá skreið Egill at leita Ölrúnar, en Slagfiðr leitaði Svanhvítrar, en Völundr sat í Úlfdölum.
\bva Hann var hagastr maðr, svá at menn viti, í fornum sögum.
\bva Níðuðr konungr lét hann höndum taka, svá sem hér er um kveðit: \\%E

\bvb Nithad was named a king in Sweden.
\bvb He owned two sons and one daughter, she was called Beadhild.
\bvb There were three brothers, the sons of a Finnish king.
\bvb The first was called Beatfinn, the second Egil, the third Wayland.
\bvb They travelled on skis and hunted wild animals. They came into Wolfdale and made for themselves houses there.
\bvb There is a water there, called Wolfsea.
\bvb Early in the morning they found on the lake-shore three women, and they were spinning linen.
\bvb By them were their swan-\textbf{hames}; those were \textbf{Walkyrries}.
\bvb Two of them were the daughters of King Latheway, Loadguth Swanwhite and Hereware Allwit; the third was Alerune Kear's daughter, from \textbf{Walland}.
\bvb The [brothers] brought the [women] with them to their halls. Egil got Alerune, but Beatfinn Swanwhite, but Wayland Allwit.
\bvb They lived there for seven winters, then they left to attend battles, and did not return.
\bvb Then Egil left on skis to seek out Alerune, but Beatfinn sought out Swanwhite, but Wayland stayed in Wolfdale.
\bvb He was the most skillful man, which men have known in ancient tales.
\bvb King Nithad had him taken, about which this has been sung: \\

\chapterStart

\begin{verse}
\bva Męyjar flugu sunnan \hld Myrkvið í gǫgnum
alvitr ungar, \hld ørlǫg drýgja; \\%E
\end{verse}

\bvb Maidens flew from the south through \textbf{Mirkwood}, young allwits†, to fulfill orlay†. \\

\begin{verse}
\bva þær á sævarstrǫnd \hld sęttusk at hvílask
drósir suðrǿnar, \hld dýrt lín spunnu. \\%E
\end{verse}

\bvb They on the sea-shore set down to rest, the southern ladies, expensive linen they span. \\

\begin{verse}
\bva Ęin nam þęira \hld Ęgil at vęrja
fǫgr mær fira \hld faðmi ljósum.
Ǫnnur vas Svanhvít, \hld svanfjaðrar dró,
ęn hin þriðja \hld þęira systir
varði hvítan \hld hals Vǫlundar. \\%E
\end{verse}

\bvb One of them took to ward Egil, the wise maiden of men by the light bosom; the second was Swanwhite, her swan-feathers she pulled, but the third of the sisters warded the white neck of Wayland. \\

\begin{verse}
\bva Sǫ́tu síðan \hld sjau vetr at þat,
ęn hinn átta \hld allan þrǫ́ðu,
ęn hinn níunda \hld nauðr of skilði,
męyjar fýstusk \hld á myrkvan við,
alvitr ungar \hld ørlǫg drýgja. \\%E
\end{verse}

\bvb Then they remained for seven winters after that, — and all the eighth, they yearned, — and on the ninth, need divorced them: the maidens longed for the mirky wood; the young allwits, to fulfill orlay.\footnotemark[1] \\
\footnotetext[1]{The swan-maidens long to return to Mirkwood (the ravaged lands of the Gots and Huns) to judge battles for Weden, which as walkirries is their orlay (fate and duty).}

\begin{verse}
\bva Kom þar af vęiði \hld veðręygr skyti
Vǫlundr líðandi \hld of langan veg,
Slagfiðr ok Ęgill, \hld sali fundu auða,
gingu út ok inn \hld ok umb sǫ́usk. \\%E
\end{verse}

\bvb Came there from the hunt, the weather-eyed shooter, Wayland passing from a long journey. Slayfinn and Egil found the halls deserted, they walked out and in, and about them looked. \\

\begin{verse}
\bva Austr skręið Ęgill \hld at Ǫlrúnu,
ęn suðr Slagfiðr \hld at Svanhvítu,
ęn ęinn Vǫlundr \hld sat í Ulfdǫlum. \\%E
\end{verse}

\bvb East travelled Egil for Alerune, but southwards Slayfinn for Swanwhite, but alone Wayland remained, in the Wolfdales. \\

\begin{verse}
\bva Hann sló goll rautt \hld við gim fastan,
lukði hann alla \hld lindbaugum vel;
svá bęið hann \hld sinnar ljóssar
kvánar, ef hǫ́num \hld of koma gęrði. \\%E
\end{verse}

\bvb He struck the red gold by fastened gemstone, enclosed he all the armrings well; thus awaited he his bright wife, if to him she might come. \\

\begin{verse}
\bva Þat spyrr Níðuðr, \hld Níara dróttinn,
at ęinn Vǫlundr \hld sat í Ulfdǫlum;
nóttum fóru sęggir, \hld nęglðar vǫ́ru brynjur,
skildir bliku þęira \hld við hinn skarða mána. \\%E
\end{verse}

\bvb Nithad learns, Lord of the Nears, that alone Wayland remained in Wolfdales; at nights travelled warriors — nailed were their byrnies — their shields gleamed by the waning moon. \\

\begin{verse}
\bva Stigu ór sǫðlum \hld at salar gafli,
gingu inn þaðan \hld ęndlangan sal,
sǫ́u þęir á bast \hld bauga dręgna,
sjau hundruð allra, \hld es sá sęggr átti. \\%E
\end{verse}

\bvb They stepped out of the saddles, towards the hall’s gables, they walked inside thence across the length of the hall. They saw on a bast-rope, armrings drawn, seven hundred in all, which that man owned. \\

\begin{verse}
\bva Ok þęir af tóku \hld ok þęir á létu
fyr ęinn útan, \hld es af létu;
kom þar af vęiði \hld veðręygr skyti
Vǫlundr líðandi \hld of langan veg. \\%E
\end{verse}

\bvb And they took off, and they put back on, but for one, which away they put. — Came there from the hunt, the weather-eyed shooter, Wayland passing from a long journey. \\

\begin{verse}
\bva Gekk brúnni \hld beru hold stęikja,
ár brann hrísi \hld allþurru fura,
viðr hinn vindþurri, \hld fyr Vǫlundi. \\%E
\end{verse}

\bvb He went, the brown she-bear’s hull to roast; early burned the brushwood, the all-dry pine, the wind-dry wood, before Wayland.

\begin{verse}
\bva Sat á berfjalli, \hld bauga talði,
alfa ljóði \hld ęins saknaði.
hugði at hęfði \hld Hlǫðvés dóttir,
Alvitr unga, \hld væri aptr komin.
 
\bvb Sat he on the bare mountain, his rings counted — the prince of elves was missing one! He thought that the daughter of Ladwigh might have it, that the young Allwit might be come again.

\begin{verse}
\bva Sat hann svá lęngi, \hld at hann sofnaði,
ok hann vaknaði \hld viljalauss;
vissi sér á hǫndum \hld hǫfgar nauðir,
ęn á fótum \hld fjǫtur of spęntan. \\%E
\end{verse}

\bvb He sat so long, that asleep he fell, and he awoke, powerless; he knew on his hands tortuous restraints, and on his feet fetters tightened.

\begin{verse}
(Vǫlundr kvað) \\
\bva Hvęrir ’ró jǫfrar \hld þęir’s á lǫgðu
bęstisíma \hld ok bundu mik? \\%E
\end{verse}

\bvb Wayland quoth: \\
“Who are those princes, that laid on thick bast-ropes, and bound me?”

\begin{verse}
\bva (Kallaði Níðuðr, \hld Níara dróttinn):
“Hvar gazt Vǫlundr, \hld vísi alfa,
óra aura, \hld í Ulfdǫlum?” \\%E

\end{verse}

\bvb Nithad called, Lord of the Nears: “Where got thou Wayland, leader of Elves, our ounces\footnotemark[1], in Wolfdales?”
\footnotetext[1]{Of gold.}

\begin{verse}
(Vǫlundr kvað) \\
\bva Goll vas þar ęigi \hld á Grana lęiðu,
fjarri hugða’k várt land \hld fjǫllum Rínar.
Man’k at męiri \hld mæti ǫ́ttum,
es vér hęil hjú \hld hęima vǫ́rum. \\%E
Hlaðguðr ok Hervǫr \hld borin vas Hlǫðvé,
kunn vas Ǫlrún \hld Kíars dóttir.  \\%E
\end{verse}

\bvb Wayland quoth: \\
\bvb “Gold was there not on Grane’s path, far I judge our land from the mountains of the Rhine. I remembered that we owned a more precious thing, when we a healthy household were at home. Ladguth and Harware were born to Ladwigh, known was Alerune, Keer’s daughter.”

\begin{verse}
\bva Úti stóð kunnig \hld kvǫ́n Níðaðar,
hón inn of gekk \hld ęndlangan sal,
stóð á golfi, \hld stilti rǫddu:
es-a sá nú hýrr, \hld es ór holti fęrr. \\%E
\end{verse}

\bvb Outside stood the cunning wife of Nithad; she walked inside across the length of the hall; she stood on the floor, steered her voice: “He is not happy, who now comes out of the wood.\footnotemark[1]”
\footnotetext[1]{The abducted Wayland.}

\begin{verse}
\bva Tęnn hǫ́num tęygjask \hld es hǫ́num’s tét sverð
ok hann Bǫðvildar \hld baug of þękkir.
Ǫ́mun eru augu \hld ormi hinum frána,
tęnn hǫ́num tęygjask, \hld es tét es sverð
ok hann Bǫðvildar \hld baug of þękkir,
sníðið hann sina \hld sinna magni,
sętið hann síðan \hld í Sævarstǫð. \\%E
\end{verse}

\bvb The teeth on him are bared, when he is shown the sword, and he recognizes the Beadhild’s arm-ring; like are the eyes to those of the bright snake. — Cut ye from him the might of his sinews, and place him then on Seastead!

\begin{verse}
\bva Svá var gǫrt, at skornar váru sinar í knésfótum ok settr í holm einn, er þar var fyrir landi, er hét Sævarstaðr. Þar smíðaði hann konungi allskyns gǫrsimar; engi maðr þorði at fara til hans, nema konungr einn. Vǫlundr kvað: \\%E
\end{verse}

\bvb Thus was done, that the sinews in his houghs were cut, and he was placed on an alone islet, which there lay by the land, and was called Seastead. There he smithed for the king all manner of jewels; no man dared to travel to him, but the king alone. Wayland quoth: \\

\begin{verse}
\bva Sé’k Níðaði \hld sverð á linda,
þat’s ek hvęsta \hld sęm hagast kunna’k
ok ek hęrða’k \hld sęm hǿgst þótti;
sá ’s mér fránn mækir \hld æ fjarri borinn.
sé’kk-a þann Vǫlundi \hld til smiðju borinn. \\%E
\end{verse}

\bvb I see a sword on Nithad’s belt, the one I sharped as I most handily knew, and I hardened as to me most easily seemed; now that gleaming sword is ever carried far from me, I see it not carried to Wayland’s smithy.

\begin{verse}
\bva Nú berr Bǫðvildr \hld brúðar minnar,
bíð’k-a þess bót, \hld bauga rauða. \\%E
\end{verse}

\bvb Now Beadhild bears the red armrings—I get no recompense for that—of my bride.

\begin{verse}
\bva Sat hann né svaf ávalt \hld ok sló hamri;
vél gęrði hęldr \hld hvatt Níðaðí;
drifu ungir tvęir \hld á dýr séa
synir Níðaðar \hld í Sævarstǫð. \\%E
\end{verse}

\bvb He sat nor slept always, and struck the hammer, rather he keenly planned treachery for Nithad; two young ones were hurrying to look at the precious things, the sons of Nithad, towards Seastead.

\begin{verse}
\bva Kvǫ́mu til kistu, \hld krǫfðu lukla,
opin vas illúð, \hld es í sǫ́u,
fjǫlð vas þar męina, \hld es mǫgum sýndisk
at væri goll rautt \hld ok gǫrsimar. \\%E
\end{verse}

\bvb They came to the chest, demanded the keys, open was the evil, when inside they looked; in there was a multitude of harm, which to the lads seemed like it were red gold and jewels.

\begin{verse}
\bva Komið ęinir tvęir, \hld komið annars dags;
ykkr læt’k þat goll \hld of gefit verða;
sęgið-a męyjum \hld né salþjóðum,
manni ęngum, \hld at mik fyndið. \\%E
\end{verse}

\bvb Come alone ye two, come another day! To you I will have that gold be given; tell not maidens, nor the people of the hall, not any man, that ye met me.

\begin{verse}
\bva Snimma kallaði \hld sęggr á annan,
bróðir á bróður: \hld gǫngum baug séa.
Kómu til kistu, \hld krǫfðu lukla,
opin vas illúð \hld es í litu. \\%E
\end{verse}

\bvb Early called one man to another, brother to brother: “Let us go see the rings!”. They came to the chest, demanded the keys, open was the evil, when inside they looked.

\begin{verse}
\bva Snęið af hǫfuð \hld húna þęira
ok und fęn fjǫturs \hld fǿtr of lagði,
ęn þær skálar, \hld es und skǫrum vǫ́ru,
svęip útan silfri, \hld sęldi Níðaði. \\%E
\end{verse}

\bvb He sliced off the heads of those bear-cubs\footnotemark[1], and under the fetter’s fen their feet did lay, — but the bowls, which under their locks were\footnotemark[2], he coated with silver, and gave to Nithad.
\footnotetext[1]{The sons of Nithad.}
\footnotetext[2]{Their skulls.}

\begin{verse}
\bva Ęn ór augum \hld jarknastęina
sęndi kunnigri \hld kvǫ́n Níðaðar;
ęn ór tǫnnum \hld tvęggja þęira
sló brjóstkringlur, \hld sęndi Bǫðvildi. \\%E
\end{verse}

\bvb But out of the eyes earkenstones, he sent to the cunning wife of Nithad, but out of the teeth of the two, he struck breast-brooches, [which he] sent to Beadhild.

\begin{verse}
\bva Þá nam Bǫðvildr \hld baugi at hrósa
[...] \hld es brotit hafði,
“þori’k-a’k sęgja, \hld nema þér ęinum.”  \\%E
\end{verse}

\bvb Then Beadhild began to praise the ring, [...] which she had broken, “I dare not tell anyone, but thee alone.”\footnotemark[1]
\footnotetext[1]{Clearly the verse is incomplete. Beadhild breaks a ring she has been given, but does not dare tell anybody but Wayland.}

\begin{verse}
\bva “Ek bǿti svá \hld brest á golli,
at fęðr þínum \hld fęgri þykkir,
ok mǿðr þinni \hld miklu bętri,
ok sjalfri þér \hld at sama hófi.”  \\%E
\end{verse}

\bvb “I will mend so the crack on the gold, that to thy father it will seem fairer, and to thy mother much better, and to thyself just the same.”

\begin{verse}
\bva Bar hann hána bjóri, \hld þvíat hann bętr kunni,
svát hón í sessi \hld of sofnaði.
»Nú hęfk hęfnt \hld harma minna
allra nema ęinna \hld íviðgjǫrnum.” \\%E
\end{verse}

\bvb He overcame her with beer — for he was more cunning — so that she in the seat asleep did fall. “Now I have avenged my injustices, — all but one, — on the insidious ones\footnotemark[1].”
\footnotetext[1]{King Nithad and his wife.}

\begin{verse}
\bva “Vęl ek, kvað Vǫlundr, \hld verða’k á fitjum,
þęim’s mik Níðaðar \hld nǫ́mu rekkar.«”
Hlæjandi Vǫlundr \hld hófsk at lopti,
grátandi Bǫðvildr \hld gekk ór ęyju.
tregði fǫr friðils \hld ok fǫður vreiði. \\%E
\end{verse}

\bvb “Well I”, quoth Wayland, “fall on my paddles, those which Nithad’s men bereaved me of!”\footnotemark[1] Laughing Wayland threw himself in the air; weeping Beadhild went from the island, grieved the flight of the lover, and the fury of the father.
\footnotetext[1]{\emph{C-V}: \emph{fit} ‘the webbed foot of water-birds’, the reader may picture for himself. Wayland has crafted wings in stead of his feet, of which use Nithad’s men deprived him.}

\begin{verse}
\bva Úti stóð kunnig \hld kvǫ́n Níðaðar,
ok hón inn of gekk \hld ęndlangan sal,
— ęn hann á salgarð \hld sęttisk at hvílask —,
»Vakir þú Níðuðr, \hld Níara dróttinn?«  \\%E
\end{verse}

\bvb Outside stood the cunning wife of Nithad; she walked inside across the length of the hall, — but he, on the courtyard, set down to rest —, “Art thou awake, Nithad, Lord of the Nears?”

\begin{verse}
\bva “Vaki’k ávalt \hld viljalauss,
sofna’k minst, \hld síz sonu dauða,
kęll mik í hǫfuð, \hld kǫld erumk rǫ́ð þín,
vilnumk þess nú, \hld at við Vǫlund dǿma’k.” \\%E
\end{verse}

\bvb “I wake always, powerless; I sleep the least, since the death of my sons. My head freezes, cold are thy counsels; I wish now but that, to speak with Wayland.”

\begin{verse}
\bva “Sęg mér þat Vǫlundr, \hld vísi alfa,
af hęilum hvat varð \hld húnum mínum?” \\%E
\end{verse}

\bvb “Say that to me, Wayland, leader of Elves, what became of my healthy bear-cubs?”

\begin{verse}
\bva Ęiða skalt mér áðr \hld alla vinna,
at skips borði \hld ok at skjaldar rǫnd,
at mars bǿgi \hld ok at mækis ęgg
at þú kvęlj-at \hld kvǫ́n Vǫlundar,
né brúði minni \hld at bana verðir,
þótt kvǫ́n ęigim, \hld þá’s ér kunnið,
eða jóð ęigim \hld innan hallar. \\%E
\end{verse}

\bvb “Before that shalt thou swear me all oaths, — by the deck of the ship and the rim of the shield, by the bough of the steed and the edge of the sword, — that thou wilt not torment the wife of Wayland, nor of my bride become the bane, though we might own a wife, which ye might know, or own a baby inside the hall.\footnotemark[1]
\footnotetext[1]{Wayland makes Nithad swear an oath that he will not harm Beadhild (“a wife, which ye might know”), nor their (yet unborn) child.}

\begin{verse}
\bva Gakk til smiðju, \hld es gęrðir þú,
þar fiðr þú bęlgi \hld blóði stokna,
snęið’k af hǫfuð \hld húna þinna
ok und fęn fjǫturs \hld fǿtr of lagða’k. \\%E
\end{verse}

\bvb Go to the smithy, which thou made; there thou wilt find bellows, with blood sprinkled. I sliced off the heads of those bear-cubs, and under the fetter’s fen their feet did I lay.

\begin{verse}
\bva Ęn þær skálar, \hld es und skǫrum vǫ́ru,
svęip’k útan silfri, \hld sęlda’k Níðaði,
ęn ór augum \hld jarknastęina,
sęnda'k kunnigri \hld kvǫ́n Níðaðar. \\%E
\end{verse}

\bvb But the bowls, which under their locks were, I coated with silver, and gave to Nithad. But out of the eyes arkenstones, I sent to the cunning wife of Nithad.

\begin{verse}
\bva Ęn ór tǫnnum \hld tvęggja þęira
sló’k brjóstkringlur, \hld sęnda’k Bǫðvildi;
nú gęngr Bǫðvildr \hld barni aukin,
ęingadóttir \hld ykkur bęggja.” \\%E
\end{verse}

\bvb But out of the teeth of the two, I struck breast-brooches, [which] I sent to Beadhild; now goes Beadhild pregnant with child, the only daughter of you both.”

\begin{verse}
\bva “Mæltir-a þú þat mál, \hld es mik męir tregi,
né þik vilja’k Vǫlundr \hld verr of níta;
es-at svá maðr hǫ́r, \hld at þik af hęsti taki,
né svá ǫflugr, \hld at þik neðan skjóti.
þar’s þú skollir \hld við ský uppi.” \\%E
\end{verse}

\bvb “Thou could not have spoken that speech, which might grieve me more, nor could I wish worse, Wayland, to deny thee. There is no man so high, that he might from a horse take thee, nor so mighty, that he might shoot thee down, there where thou jeers, high against the clouds.”

\begin{verse}
\bva Hlæjandi Vǫlundr \hld hófsk at lopti,
ęn ókátr Níðuðr \hld þá ęptir sat. \\%E
\end{verse}

\bvb Laughing Wayland threw himself in the air, but bleak Nithad then afterwards remained.

\begin{verse}
\bva “Upp rís Þakkráðr, \hld þræll minn bazti,
bið Bǫðvildi, \hld męy hina bráhvítu,
gangi fagrvarið \hld við fǫður rǿða.” \\%E
\end{verse}

\bvb “Rise up Thankred, my best thrall; ask Beadhild, — the brow-white maiden, — to go fair-clothed, with her father to counsel.”

\begin{verse}
\bva “Es þat satt Bǫðvildr, \hld es sǫgðu mér,
sǫ́tuð it Vǫlundr \hld saman í holmi?” \\%E
\end{verse}

\bvb “Is it true, Beadhild, what they said to me; sat thou and Wayland, together on the island?” \\

\begin{verse}
\bva Satt ’s þat Níðuðr \hld es sagði þér.
Sǫ́tum vit Vǫlundr \hld saman í holmi
ęina ǫgurstund, \hld æva skyldi;
ek vætr hǫ́num \hld vinna kunna’k,
ek vætr hǫ́num \hld vinna mátta’k.
\end{verse}

\bvb “It is true, Nithad, what \emph{he} said\footnotemark[1] to thee; I and Wayland sat together on the island, for one grave moment — it should never have been. I \emph{knew} nought struggle against him, I \emph{could} nought struggle against him.\footnotemark[2]”
\footnotetext[1]{Beadhild, knowing that the only one who is aware of what happened is Wayland, changes the person from her father’s general plural, to the specific singular.}
\footnotetext[2]{She was both mentally (\emph{C-V}: \emph{kunna} ‘know, understand’) and physically (\emph{C-V}: \emph{mega} ‘to have strength to do, avail’) incapable of struggling against him. As Finnur comments, an unsurpassed ending.}
% — Heroic/mythological, transition
%	\bookStart{First Lay of Hallow Hundingsbane}[Helgakviða Hundingsbana fyrsta]

\bvg
\bva Ár vas alda \hld\ þat’s arar gullu &
hnigu hęilǫg vǫtn \hld\ af Himinfjǫllum; &
þá hafði Hęlga \hld\ inn hugumstóra &
Borghildr borit \hld\ í Brálundi.\eva

\bvb It was the beginning of \inx[C]{eld}[elds], as eagles shrieked; holy waters poured down from the Heavenfells; then Burhild in Browlund gave birth to Hallow the Great-hearted.\evb
\evg


\bvg
\bva Nótt varð í bǿ, \hld\ nornir kvǫ́mu, &
þę́r’s ǫðlingi \hld\ aldr of skópu; &
þann bǫ́ðu fylki \hld\ frę́gstan verða &
ok buðlunga \hld\ bęztan þykkja.\eva

\bvb Night came in the settlement; norns came, those who did shape the prince’s life; that marshaller <= Hallow> they declared would become most renowned, and of kings seem the foremost.\evb
\evg


\bvg
\bva Sneru þę́r af afli \hld\ ørlǫgþǫ́ttu &
þá’s borgir braut \hld\ í Brálundi; &
þę́r um gręiddu \hld\ gullinsímu &
ok und mána sal \hld\ miðjan fęstu.\eva

\bvb They turned with their might the strands of \inx[C]{orlay}, as he broke cities in Browlund; they arranged golden bands, and under the moon's hall fastened [them in] the middle.\evb
\evg

%	Helgi fekk Sigrúnar ok áttu þau sonu; var Helgi eigi gamall. Dagr Hǫgna sonr blótaði Óðin til fǫðurhefnda. Óðinn léði Dag geirs síns. Dagr fann Helga, mág sinn, þar sem heitir at Fjǫturlundi. Hann lagði í gǫgnum Helga með geirnum. Þar fell Helgi en Dagr reið til fjalla ok sagði Sigrúnu tíðindi: 

Hallow got Sighrun, and they owned sons; Hallow was not old. Day, son of Hain, blooted† to Weden to take revenge for his father. Weden lent Day his spear. Day found Hallow, his brother-in-law, at a place called Fetterlund; he laid the spear through Hallow. There fell Hallow, but Day rode to the fells and told Sighrun the news:

“Trauðr em ek, systir, \hld trega þér at sęgja
þvíat ek hęfi nauðigr \hld nipti grætta:
Fell í morgun \hld und Fjǫturlundi
buðlungr sá’s vas \hld bęztr í hęimi
ok hildingum \hld á halsi stóð.”

“Regretful am I, sister, to grieve thee by saying — for forced must I cause my kinswoman to cry: This morning fell, ’neath Fetterlund, that prince who was in the world the best, and on the throats of rulers stood.”

...

Fyrr vil’k kyssa \hld konung ólifðan
an þú blóðugri \hld brynju kastir;
hár es þitt, Helgi, \hld hélu þrungit,
allr es vísi \hld valdǫgg slęginn,
hęndr úrsvalar \hld Hǫgna mági;
hvé skal’k þér, buðlungr, \hld þess bót of vinna? 

“Sooner would I kiss the unliving king, than thou the bloody byrnie mightst cast away. Thy hair is, Hallow, with hoarfrost thick: the prince is all with corpse-dew whipped: the hands wet-cold on the kinsman of Hain. How shall I for thee, lord, remedy that?”

Ęin vęldr þú, Sigrún \hld frá Sefafjǫllum,
es Hęlgi es \hld harmdǫgg slęginn:
Grætr þú, gullvarit, \hld grimmum tǫ́rum,
sólbjǫrt suðrǿn, \hld áðr þú sofa gangir,
hvęrt fęllr blóðugt \hld á brjóst grami,
úrsvalt, innfjalgt \hld ękka þrungit.

“Thou alone causest, Sighrun from the Sevefells, that Hallow be by harm-dew whipped; thou criest, gold-covered, bitter tears, sun-bright southern lady, before thou to sleep mightst go. Each one falls bloody on the breast of the ruler, wet-cold and stifled, pressed forth by grief.”

%	\bookStart{The Lay of Hallow Harwardson}[Hęlgakviða Hjǫrvarðssonar]

Frá Hjǫrvarði ok Sigrlinn

Hjǫrvarðr hét konungr. Hann átti fjórar konur. Ein hét Alfhildr; sonr þeira hét Heðinn. Ǫnnur hét Sę́reiþr; þeira sonr hét Humlungr. In þriðja hét Sinrjóð; þeira sonr hét Hymlingr. Hjǫrvarðr konungr hafði þess heit strengt at eiga þá konu er hann vissi vę́nsta. Hann spurði at Sváfnir konungr átti dóttur allra\footnote[‘vęnallra’ \emph{corr.} \Regius] fegrsta; sú hét Sigrlinn. Iðmundr hét jarl hans; Atli var hans sonr er fór at biðja Sigrlinnar til handa konungi. Hann dvalðisk vetrlangt með Sváfni konungi. Fránmarr hét þar jarl, fóstri Sigrlinnar; dóttir hans hét Álǫf. Jarlinn réð, at meyjar var synjat, ok fór jarlinn heim. Atli jarls sonr stóð einn dag við lund nǫkkurn, en fugl sat í limunum uppi yfir hánum ok hafði heyrt til, at hans menn kǫlluðu vę́nstar konur þę́r, er Hjǫrvarðr konungr átti. Fuglinn kvakaði, en Atli hlýddi, hvat hann sagði. Hann kvað:

Regarding Harward and Sighlind


\bvg
\bva Sáttu Sigrlinn, \hld\ Sváfnis dóttur, &
męyna fęgrstu \hld\ î munarhęimi? &
Þó hagligar \hld\ Hjǫrvarðs konur &
gumnum þykkja \hld\ at Glasislundi.\eva

\bvb 1\evb
\evg


\bvg
\bva „Mundu við Atla \hld\ Iðmundar son &
fugl fróðhugaðr \hld\ flęira mę́la?“ &
„Mun’k ef mik buðlungr \hld\ blóta vildi &
ok kýs’k þat’s ek vil \hld\ ór konungs garði.“\eva

\bvb 2\evb
\evg


\bvg
\bva 3\eva

\bvb 3\evb
\evg


\bvg
\bva 4\eva

\bvb 4\evb
\evg


\bvg
\bva 5\eva

\bvb 5\evb
\evg


\bvg
\bva 6\eva

\bvb 6\evb
\evg


\bvg
\bva 7\eva

\bvb 7\evb
\evg


\bvg
\bva Sverð vęit’k liggja \hld\ î Sigarsholmi, &
fjórum fę́ra \hld\ enn fimm tǫgu; &
ęitt es þęira \hld\ ǫllum bętra &
vígnesta bǫl \hld\ ok varið golli.\eva

\bvb Swords I know lying, in Sigharsholm, four less than fifty. One of them is better than all—the bale of war-needles\footnoteB{The kenning \emph{vígnest} also appears in} \ken{spears?}—and inlaid with gold.\evb
\evg


\bvg
\bva Hringr ’s î hjalti, \hld\ hugr ’s î miðju, &
ógn ’s î oddi, \hld\ þęim’s ęiga getr; &
liggr með ęggju \hld\ ormr dręyrfáiðr &
en ȧ valbǫstu \hld\ verpr naðr hala.\eva

\bvb A ring is in the hilt; courage is in the middle; fear is in the point, for the one who gets to own it; along the blade lies a serpent painted in blood, but on the walbast\footnoteB{An unclear part of the sword-hilt; see \Sigrdrifumal\ 7.} an adder chases its tail.\evb
\evg

%	\bookStart{The Lay of Attle}[Atlakviða]

BPG %TODO prose formatting
Dauði Atla.

Guðrún Gjúkadóttir hefndi brǿðra sinna, svá sem frę́gt er orðit. Hon drap fyrst sonu Atla, en eptir drap hon Atla ok brendi hǫllina ok hirðina alla; um þetta er sjá kviða ort.

The Death of Attle

Guthrun Yivicksdaughter avenged her brothers, as has become famous. She first killed the sons of Attle, and after that she killed Attle, and burned the hall and the whole hird. Regarding that this lay is wrought.

\bvg
\bva Atli sęndi \hld\ ár til Gunnars &
kunnan sęgg at ríða, \hld\ Knéfrøðr vas sá hęitinn; &
at gǫrðum kom hann Gjúka \hld\ ok at Gunnars hǫllu, &
bękkjum aringręypum \hld\ ok at bjóri svǫ́sum.\eva

\bvb Attle sent early to Guther a well-known messenger to ride; Kneefred that one was called. To the estates of Yivick he came, and to the hall of Guther; to the hearth-surrounding benches, and to the lovely beer.\evb
\evg


\bvg
\bva Drukku þar dróttmęgir \hld\ —ęn \edtext{dyljęndr}{\lemma{dyljęndr ‘concealed ones’}\Bfootnote{\textcite{FinnurEdda} reasonably interprets this as referring to Attle’s spies at Guther’s court.}} þǫgðu— &
vín í \edtext{valhǫllu}{\lemma{valhǫllu ‘the walhall’}\Bfootnote{The interpretation of this compound is difficult in context. The first element \emph{val-} could be (1) \emph{valr} ‘falcon’, referring to the aristocratic hunting practice; (2) \emph{valr} ‘\inx[G]{Wales}[Wale]’, cognate with ‘Welsh’ but in ON referring to the French or Romans, stressing the southern location or appearance of the hall; or (3) \emph{valr} ‘(collective) the battle-slain’, foreshadowing the inevitable death (\inx[C]{feyness}) of the \inx[G]{Yivickings}. In this case it is linguistically identical to \inx[L]{Walhall}, Weden’s hall, whither the battle-slain go.}}, \hld\ vręiði sǫ́usk þęir Húna; &
kallaði þá Knéfrøðr \hld\ kaldri rǫddu, &
sęggr inn suðrǿni \hld\ sat hann á bękk hǫ́m:\eva

\bvb There the dright-lads drank—but the concealed ones were silent—wine in the walhall; wary were they of the wrath of the Huns. Then Kneefred, the southern man, called with cold voice; he sat on a high bench:\evb
\evg


\bvg
\bva “Atli mik hingat sęndi \hld\ ríða øręndi, &
mar inum mélgręypa, \hld\ Myrkvið inn ókunna &
at biðja yðr, Gunnarr, \hld\ at it á bękk kǿmið &
með hjǫlmum aringręypum \hld\ at sǿkja hęim Atla.\eva

\bvb “Attle me hither sent to ride an errand, with the bit-champing horse through the uncharted Mirkwood, to ask you, Guther, that ye two on the bench might come, with hearth-surrounding helmets, to seek the home of Attle.\evb
\evg


\bvg
\bva Skjǫldu kneguð þar vęlja \hld\ ok skafna aska, &
hjalma gullroðna \hld\ ok Húna męngi, &
silfrgyllt sǫðulklę́ði, \hld\ sęrki valrauða, &
dafar, darraða, \hld\ drǫsla mélgręypa.\eva

\bvb There ye might choose shields, and smooth ash-spears, helmets gold-reddened, and the multitude of the Huns, silver-gilt saddle-cloth, walred serks, dafs, standards, bit-champing steeds.\evb
\evg


\bvg
\bva Vǫll lézk ykkr ok myndu gefa \hld\ víðrar Gnitahęiðar &
af gęiri gjallanda \hld\ ok af gylltum stǫfnum, &
stórar męiðmar \hld\ ok staði Danpar, &
hrís þat it mę́ra \hld\ es meðr Myrkvið kalla.\eva

\bvb GAGAGA\evb
\evg


\bvg
\bva Hǫfði vatt þá Gunnarr \hld\ ok Hǫgna til sagði: &
Hvat rę́ðr þú okkr, sęggr inn ǿri, \hld\ allz vit slíkt hęyrum? &
Gull vissa ek ekki \hld\ á Gnitahęiði, &
þat es vit ę́ttim-a \hld\ annat slíkt.\eva

\bvb His head turned Guther then, and to Hain said: “What counselest thou we two do, younger man, as we such things hear? I knew of no gold on the Gnitheath, that we did not own as much of.\evb
\evg


\bvg
\bva Sjau ęigu vit salhús \hld\ sverða full, &
hvęrju eru þęira \hld\ hjǫlt ór gulli; &
mínn vęit ek mar bęztan \hld\ ęn mę́ki hvassastan, &
boga bękksǿma \hld\ ęn brynjur ór gulli.\eva

\bvb We own seven hallhouses, filled with swords—on each of them is a golden hilt; I know my horse to be the best, and my sword the sharpest; my bow bench-fit, and my byrnies of gold.\evb
\evg


\bvg
\bva Hjalm ok skjǫld hvítastan, \hld\ kominn ór hǫll Kjárs; &
ęinn es mínn bętri \hld\ ęn sé allra Húna.\eva

\bvb A helmet and the whitest shield, taken out of the hall of Chear; alone is mine better, than that of all of the Huns.”\evb
\evg


\bvg
\bva Hvat hyggr þú brúði bęndu \hld\ þá es hón okkr baug sęndi, &
varinn váðum hęiðingja? \hld\ Hykk at hón vǫrnuð byði! &
Hár fann ek hęiðingja \hld\ riðit í hring rauðum; &
ylfskr es vegr okkarr \hld\ at ríða øręndi.\eva

\bvb “What does thou think the bride meant, when she us two an armlet sent, wrapped with the cloth of a heath-dweller \ken{wolf}? I think that she bid us a warning! I found the hair of a heath-dweller wrapped round the red ring; wolven is our way, to ride that errand.”\evb
\evg


\bvg
\bva Niðjar-gi hvǫttu Gunnar \hld\ né náungr annarr, &
rýnęndr né ráðęndr, \hld\ né þęir es ríkir vǫ́ru; &
kvaddi þá Gunnarr \hld\ sęm konungr skyldi, &
mę́rr í mjǫðranni \hld\ af móði stórum:\eva

\bvb No kinsmen urged Guther, nor any other close one, nor counselors nor advisors, nor those who mighty were. Guther then announced—as a king should, renowned in the mead-house—out of great courage:\evb
\evg


\bvg
\bva Rís-tu nú, Fjǫrnir, \hld\ lát-tu á flęt vaða &
gręppa gullskálir \hld\ með gumna hǫndum!\eva

\bvb “Rise now, Ferner; let on the floorboards wade forth the golden bowls of warriors, along the hands of men!\evb
\evg


\bvg
\bva Ulfr mun ráða \hld\ arfi Niflunga, &
gamlir granvarðir, \hld\ ef Gunnars missir, &
birnir blakkfjallir \hld\ bíta þreftǫnnum, &
gamna gręystóði, \hld\ ef Gunnarr né kømr-at.\eva

\bvb The wolf will rule the inheritance of the Niflings: the old grey guardians, if Guther is missing. Bears black-furred bite with wrangling teeth, amusing the pack of bitches, if Guther comes not.”\evb
\evg


\bvg
\bva Lęiddu landrǫgni \hld\ lýðar ónęisir, &
grátęndr, gunnhvatan, \hld\ ór garði Húna; &
þá kvað þat inn ǿri \hld\ ęrfivǫrðr Hǫgna: &
Hęilir farið nú ok horskir \hld\ hvar’s ykkr hugr tęygir!\eva

\bvb GAGAGA\evb
\evg


\bvg
\bva Fetum létu frǿknir \hld\ um fjǫll at þyrja &
marina mélgręypu, \hld\ Myrkvið inn ókunna; &
hristisk ǫll Húnmǫrk \hld\ þar es harðmóðgir fóru, &
vrǫ́ku þęir vannstyggva \hld\ vǫllu algrǿna.\eva

\bvb GAGAGA\evb
\evg


\bvg
\bva Land sǫ́u þęir Atla \hld\ ok liðskjalfar djúpar &
Bikka greppar standa \hld\ á borg inni há &
sal of suðrþjóðum, \hld\ slęginn sessmęiðum, &
bundnum rǫndum, \hld\ blęikum skjǫldum,\eva

\bvb The land of Attle saw they, TODO\evb
\evg


\bvg
\bva dafar, darraða; \hld\ ęn þar drakk Atli &
vín í valhǫllu; \hld\ vęrðir sǫ́tu úti &
at varða þęim Gunnari \hld\ ef þęir hér vitja kǿmi &
með gęiri gjallanda \hld\ at vękja gram hildi.\eva

\bvb but there drank Attle wine in the wale-hall\footnoteB{TODO: this is not Weden’s hall, rather ‘the Roman hall’.} ... \evb
\evg


\bvg
\bva Systir fann þęira snemmst \hld\ at þęir í sal kvǫ́mu, &
brǿðr hęnnar báðir, \hld\ bjóri var hón lítt drukkin: &
Ráðinn ert-u nú, Gunnarr, \hld\ hvat munt-u, ríkr, vinna &
við Húna harmbrǫgðum? \hld\ Hǫll gakk þú ór snemma!\eva

\bvb Their sister found earliest they they had come into the hall, both of her brothers—on beer was she lightly drunk—“Betrayed art thou now, Guther; why wilt thou, mighty one, struggle against Hunnish harm-tricks? Go early out of the hall!\footnoteB{Before anything evil might happen.}”\evb
\evg


\bvg
\bva Bętr hęfðir þú, bróðir, \hld\ at þú í brynju fǿrir, &
sęm hjǫlmum aringręypum \hld\ at séa hęim Atla; &
sę́tir þú í sǫðlum \hld\ sólhęiða daga, &
nái nauðfǫlva \hld\ létir nornir gráta.\eva

\bvb Better hadst thou, brother, if thou in byrnie travelled, and with hearth-surrounding helmets, to see the home of Attle.\evb
\evg


\bvg
\bva Húna skjaldmęyjar \hld\ hęrfi kanna &
ęn Atla sjalfan \hld\ létir þú í ormgarð koma; &
nú es sá ormgarðr \hld\ ykkr of folginn.\eva

\bvb GAGAGA\evb
\evg


\bvg
\bva Sęinað es nú, systir, \hld\ at samna Niflungum, &
langt es at lęita \hld\ lýða sinnis til, &
of rosmufjǫll Rínar, \hld\ rekka ónęissa.\eva

\bvb GAGAGA\evb
\evg


\bvg
\bva Fengu þęir Gunnar \hld\ ok í fjǫtur sęttu, &
vinir Borgunda, \hld\ ok bundu fastla; &
sjau hjó Hǫgni \hld\ sverði hvǫssu &
ęn inum átta hratt hann \hld\ í ęld hęitan.\eva

\bvb Caught they Guther, and in fetters set him—the friends of the Burgends—and bound them tightly. Seven Hain hewed down with sharp sword, and the eighth one threw he into the hot fire.\evb
\evg


\bvg
\bva \edtext{Svá skal frǿkn \hld\ fjándum vęrjask;}{\lemma{Svá ... vęrjask}\Bfootnote{Line moved from the last verse to this one since it seems to connect semantically with the immediately following line, and also creates a regular line distribution of 4-4 instead of 5-3.}} &
Hǫgni varði \hld\ hęndr Gunnars. &
frǫ́gu frǿknan \hld\ ef fjǫr vildi &
Gotna þjóðann \hld\ gulli kaupa.\eva

\bvb Thus shall the bold against fiends ward himself; Hain warded the hands of Guther. They asked the bold one if to buy he wished—the ruler of the Gots—his life with gold.\footnoteB{The Huns ask Guther (it is clear that “ruler of the Gots” refers to him, cf. 1, 3, 10) if he wishes to ransom Hain. He instead responds with the following:}\evb
\evg


\bvg {\small [Guther quoth:]}
\bva “Hjarta skal mér Hǫgna \hld\ í hęndi liggja &
blóðugt, ór brjósti \hld\ skorit baldriða, &
saxi slíðrbęitu, \hld\ syni þjóðans.”\eva

\bvb “The heart of Hain shall lie me in the hands: bloody from the breast—cut from the bold rider with a slide-biting sax\footnoteB{i.e. a short-sword with a blade so sharp that it draws blood when one slides the finger across it.}—of the son of the sovereign.”\evb
\evg


\bvg
\bva Skǫ́ru þęir hjarta \hld\ Hjalla ór brjósti &
blóðugt ok á bjóð lǫgðu \hld\ ok bǫ́ru þat fyr Gunnar.\eva

\bvb They cut the heart of Helle out of the breast; bloody on a platter they laid it, and carried it before Guther.\evb
\evg


\bvg
\bva Þá kvað þat Gunnarr, \hld\ gumna dróttinn: &
Hér hęfi ek hjarta \hld\ Hjalla ins blauða, &
ólíkt hjarta \hld\ Hǫgna ins frǿkna, &
es mjǫk bifask \hld\ es á bjóði liggr; &
bifðisk hǫlfu męirr \hld\ es í brjósti lá!\eva

\bvb Then quoth that Guther, the lord of men: “Here have I the heart of Helle the soft—unlike the heart of Hain the bold!—which much trembles, when on the platter it lies; it trembled twice as much, when in the breast it lay.”\evb
\evg


\bvg
\bva Hló þá Hǫgni \hld\ es til hjarta skǫ́ru &
kvikvan kumblasmið \hld\ kløkkva hann sízt hugði; &
blóðugt þat á bjóð lǫgðu \hld\ ok bǫ́ru fyr Gunnar.\eva

\bvb Hain laughed then, when to the heart they cut on the living wound-smith \ken{warrior}; he thought least of sobbing. Bloody on a platter they laid it, and carried it before Guther.\evb
\evg


\bvg
\bva Mę́rr kvað þat Gunnarr, \hld\ Gęir-Niflungr: &
Hér hęfi ek hjarta \hld\ Hǫgna ins frǿkna, &
ólíkt hjarta \hld\ Hjalla ins blauða, &
es lítt bifask \hld\ es á bjóði liggr; &
bifðisk svági mjǫk \hld\ þá’s í brjósti lá!\eva

\bvb Renowned quoth that Guther, the Gore-Nifling: “Here have I the heart of Hain the bold—unlike the heart of Helle the soft!—which little trembles, when on the platter it lies; it trembled not as much, when in the breast it lay.\evb
\evg


\bvg
\bva Svá skaltu, Atli, \hld\ augum fjarri &
sęm munt \hld\ męnjum verða; &
es und ęinum mér \hld\ ǫll of folgin &
hodd Niflunga: \hld\ Lifir-a nú Hǫgni!\eva

\bvb Thus shalt thou, Attle, be as far from the eyes, as thou wilt from the neck-rings. ’Tis by me alone all concealed, the hoard of the Niflings—now Hain lives not!\evb
\evg


\bvg
\bva Ęy vas mér týja \hld\ meðan vit tvęir lifðum, &
nú es mér ęngi \hld\ es ęinn lifi’k; &
Rín skal ráða \hld\ rógmalmi skatna, &
svinn, ǫ́skunna \hld\ arfi Niflunga.\eva

\bvb I was ever in doubt when we two lived; now I am not when alone I live. The Rhine shall rule the strife-ore of princes \ken{gold}, swift, the os-born inheritance of the Niflings.\evb
\evg


\bvg
\bva Í veltanda vatni \hld\ lýsask valbaugar &
hęldr an á hǫndum gull \hld\ skíni Húna bǫrnum.\eva

\bvb In tumbling water the Welsh bighs gleam, rather than gold might shine on the hands of the children of Huns.”\evb
\evg

...

\bvg
\bva Ęldi gaf hón alla \hld\ es inni vǫ́ru &
ok frá morði þęira Gunnars \hld\ komnir vǫ́ru ór Myrkhęimi; &
forn timbr fellu, \hld\ fjarghús ruku, &
bǿr Buðlunga, \hld\ brunnu ok skjaldmęyjar, &
inni aldrstamar, \hld\ hnigu í ęld hęitan.\eva

\bvb To the fire she gave all those who were inside, who from their murder of Guther were come out of Mirkham. Ancient timbers fell, great houses smoked—the settlement of the Buthlungs—burned the shield–maidens likewise; inside aged trunks bowed into hot fire.\evb
\evg


\bvg
\bva Fullrǿtt’s umb þetta; \hld\ fęrr ęngi svá síðan &
brúðr í brynju \hld\ brǿðra at hęfna; &
hón hęfir þriggja \hld\ þjóðkonunga &
banorð borið, \hld\ bjǫrt, áðr sylti.\eva

\bvb ’Tis fully told of this; none hence fares so, a bride in byrnie, her brothers to avenge. She has of three great kings borne the bane-word,\footnoteB{i.e. ‘She has slain three great kings.’ This expression and its Germanic and Indo-European relatives is discussed in detail in \textcite{Watkins1995}[417--422].} bright woman, before she may die.\evb
\evg


\bvg
\bva Enn segir gleggra í Atlamálum inum grǿnlenskum.\eva

\bvb Yet this is told more clearly in the Greenlendish Speeches of Attle.\evb
\evg

%	\bookStart{The Third Lay of Guthrun}[Guðrúnarkviða þriðja]

BPG
BPA Herkja hét ambǫ́tt Atla; hón hafði verit frilla hans. Hón sagði Atla at hón hefði sét Þjóðrek ok Guðrúnu bę́ði saman. Atli var þá allókátr. Þá kvað Guðrún: EPA

BPB Hark was named the female thrall of Attle; she had been his concubine. She told Attle that she had seen Thederick and Guthrun both together. Attle was then wholly displeased. Then Guthrun quoth: EPB
EPG


\bvg
\bva “Hvat es þér, Atli? \hld ę́, Buðla sonr, &
es þér hryggt í hug; \hld hví hlę́r þú ę́va? &
Hitt myndi ǿðra \hld jǫrlum þykkja &
at við menn mę́ltir \hld ok mik sę́ir.”\eva

\bvb What is with thee, Attle? Always, son of Bodle, art thou sad at heart; why laughest thou never? TO-DO\evb
\evg


\bvg
\bva “Tregr mik þat, Guðrún, \hld Gjúka dóttir, &
mér í hǫllu \hld Hęrkja sagði &
at þit Þjóðrekr \hld undir þaki svę́fið &
ok léttliga \hld líni vęrðið.”\eva

\bvb It troubles me, Guthrun, Yivick’s daughter, which in the hall Hark has said me: that thou and Thederick beneath thatched roof slept, and ye lightly warded the linen.\footnoteB{i.e., they threw off their clothes and slept together.}\evb
\evg


\bvg
\bva “Þér mun’k alls þęss \hld ęiða vinna &
at inum hvíta \hld helga stęini. &
at ek við Þjóðmar \hld þat-ki átta’k &
es vǫrðr né verr \hld vinna knátti.\eva

\bvb GAGAGGAGAG\evb
\evg


\bvg
\bva Nema ek halsaða \hld hęrja stilli, &
jǫfur óneisinn, \hld ęinu sinni; &
aðrar vǫ́ru \hld okkrar spękjur &
es við hǫrmug tvau \hld hnigum at rúnum.\eva

\bvb TESTETET STET T\evb
\evg


\bvg
\bva Hér kom Þjóðrekr \hld með þrjá tǫgu, &
lifa þęir né ęinir, \hld þriggja tega manna; &
hrinktu mik at brǿðrum \hld ok at brynjuðum, &
hrinktu mik at ǫllum \hld á hǫfuðniðjum.\eva

\bvb TESTE TEST EST TES\evb
\evg


\bvg
\bva Sęntu at Saxa, \hld sunnmanna gram; &
hann kann hęlga \hld hver vellanda;” &
sjau hundruð manna \hld í sal gengu &
áðr kvę́n konungs \hld í kętil tǿki.\eva

\bvb Send for Saxe, the prince of southmen; he knows how to hallow a swelling cauldron!” — Seven hundred men went into the hall, before the wife of the king might touch the kettle.\evb
\evg


\bvg
\bva “Kęmr-a nú Gunnarr, \hld kalli’k-a Hǫgna,
sé’k-a síðan \hld svása brǿðr;
sverði myndi Hǫgni \hld slíks harms reka,
nú verð’k sjǫlf fyr mik \hld synja lýta.”\eva

\bvb “Now Guthhere comes not, I call not on Hain; I see not hence [my] sweet brothers. With sword would Hain drive away such an affront; now I will for myself disprove the slanders.”\evb
\evg


\bvg
\bva Brá hón til botns \hld bjǫrtum lófa &
ok hón upp of tók \hld jarknastęina: &
Sé nú sęggir \hld sykn em ek orðin &
hęilagliga— \hld hvé sjá hverr velli.\eva

\bvb Brought she the bright palms to the bottom, and she up did take the earkenstones: “See now, men—I am proven innocent, through holy means—how this cauldron boils!”\evb
\evg


\bvg
\bva Hló þá Atla \hld hugr í brjósti &
es hann hęilar sá \hld hęndr Guðrúnar: &
Nú skal Hęrkja \hld til hvers ganga, &
sú er Guðrúnu \hld grandi vę́nti. \eva

\bvb Then the heart of Attle laughed in his breast, when he saw the hands of Guthrun unscathed: “Now shall Hark go to the cauldron, she who to Guthrun hoped to cause harm.”\evb
\evg


\bvg
\bva Sá-at maðr armligt, \hld hvęrr es þat sá at, &
hvé þar á Hęrkju \hld hęndr sviðnuðu; &
lęiddu þá męy \hld í mýri fúla, &
svá þá Guðrún \hld sinna harma.\eva

\bvb Each man saw not something so pitiful, who saw that: how there on Hark the hands were scorched. Led they the maiden into the foul bog; thus was Guðrún reconstituted for her affronts.\evb
\evg

%	\bookStart{The Speeches of Sighdrive}[Sigrdrífumǫ́l]

\begin{flushright}%
Dating \parencite{Sapp2022}: C10th (0.961)

Meter: \Ljodahattr%
\end{flushright}

% Introduction

Many of the verses are quoted in \VolsungaSaga, but notably the two prayer-verses are missing; possibly an instance of Christian censorship. TODO

\sectionline

\bvg {\small [Sighdrive quoth:]}
\bva „Lęngi ek svaf, \hld\ lęngi ek sofnuð vas, &
\ind lǫng eru lýða lę́; &
Óðinn því vęldr \hld\ es ęigi mátta’k &
\ind bregða blundstǫfum.“\eva

\bvb “Long I slept, long was I asleep, long are the deceits”\evb
\evg

\bpg
\bpa Sigurðr sęttisk niðr ok spyrr hana nafns. Hón tók þá horn fullt mjaðar ok gaf hǫ́num minnisvęig.\epa

\bpb Siward set himself down, asking for her name. Then she took a horn full of mead, and gave him a mind-draught:\epb
\epg


\bvg
\bva Hęill \alst{D}agr, \hld\ hęilir \alst{D}ags synir, &
\ind hęil \alst{N}ǫ́tt ok \alst{n}ipt! &
\alst{Ó}ręiðum \alst{au}gum \hld\ lítið \alst{o}kkr þinig &
\ind ok gefið \alst{s}itjǫndum \alst{s}igr!\eva

\bvb “Hail \inx[P]{Day}! Hail the sons of Day!\footnoteB{TODO. Who?} Hail Night and [her] kinswoman \ken*{= Earth}!\footnoteB{According to \Gylfaginning\ 10 Earth is the daughter of Night and \inx[P]{Aner}.} With unwrathful eyes look ye upon us two, and give the sitting ones \ken*{= us} victory.\evb
\evg


\bvg
\bva Hęilir \alst{ę́}sir, \hld\ hęilar \alst{ǫ́}synjur, &
\ind hęil sjá in \alst{f}jǫlnýta \alst{f}old! &
\alst{M}ál ok \alst{m}anvit \hld\ gefið okkr \alst{m}ę́rum tvęim &
\ind ok \alst{l}ę́knishęndr meðan \alst{l}ifum!\eva

\bvb Hail the \inx[G]{Ease}! Hail the \inx[G]{Ossens}! Hail this bountiful fold \ken{earth}! Speech and \inx[C]{manwit} give ye us renowned two, and \inx[C]{healing-hands}\footnoteB{Hands with the power to heal (perhaps supernaturally). The singular form \emph{lę́knishǫnd} occurs in the semi-Christianized prayer on a c. 1300 stick from Ribe, Denmark (signum DR EM85;493).} while we live.”\evb
\evg


BPG
BPA Hon nefndisk Sigrdrífa ok var valkyrja. Hon sagði, at tveir konvngar bǫrðusk. Hét annarr Hjalmgunnarr; hann var þá gamall ok inn mesti hermaðr, ok hafði Óðinn hánum sigri heitit.
En \alst{a}nnarr hét \alst{A}gnarr, \hld\ \alst{Au}ðu bróðir // er \alst{v}ę́tr engi \hld\ \alst{v}ildi þiggja.
Sigrdrífa felldi Hjalmgunnar í orrostunni. En Óðinn stakk hana svefnþorni í hefnd þess ok kvað hana aldri skyldu síðan sigr vega í orrostu, ok kvað hana giftask skyldu, „en sagða’k hánum at strengða’k heit þar í mót, at giptask øngom þeim manni er hrę́ðask kynni.“ Hann segir ok biðr hana kenna sér speki ef hon\footnoteA{\emph{hánom} ms.} vissi tíðendi ór ǫllum heimum. Sigrdrífa kvað:EPA

BPB She called herself Sighdrive and was a walkirrie. She said that two kings fought. One of them was called Helmguther; he was then old and the greatest harrier, and Weden had promised him victory.
But another one was called Eyner, Eade’s brother, who in no way wished to accept.\footnoteB{i.e. ‘wished to lose’ TODO}
Sighdrive felled Helmguther in the battle, but Weden pierced her with the sleeping-thorn as revenge for that, and said that she would never thenceforth win victory in battle, and said that she must marry, “but I told him that I made a vow against that, to marry no man who could be frightened.” He [= Siward] speaks and asks her to teach him wisdom, if she knew any tidings out of all the \inx[C]{Home}[Homes]. Sighdrive quoth: EPB
EPG


\bvg
\bva „Bjór fǿri’k þér, \hld\ brynþings apaldr, &
magni blandinn \hld\ ok męgintíri, &
fullr ’s hann ljóða \hld\ ok líknstafa, &
góðra galdra \hld\ ok gamanrúna.\eva

\bvb Beer I bring thee—apple-tree of the byrnie-\inx[C]{Thing} \ken{battle > warrior}!—mixed with might, and might-glory; it is full of \inx[C]{leed}[leeds], and grace-staves, of good \inx[C]{galder}[galders], and pleasure-\inx[C]{rune}[runes].\evb
\evg


\bvg
\bva Sigrúnar skalt kunna, \hld\ ef vilt sigr hafa, &
\ind ok rísta á hjalti hjǫrs, &
sumar á véttrimum, \hld\ sumar á valbǫstum, &
\ind ok nęfna tysvar Tý.\eva

\bvb Victory-runes shalt thou know, if thou wilt have victory, and carve on the hilt of the sword; some on weight-rims;\footnoteB{Unclear.} some on walbasts\footnoteB{Possibly the sword-pommel, the word also occurs in \HelgakvidaHjorvardssonar\ 9.}, and name \inx[P]{Tue} twice.\evb
\evg


\bvg
\bva Ǫlrúnar skalt kunna \hld\ ef þu vilt annars kvę́n &
\ind vęli t þik i trygd ef þú trúir. &
á horni skal þér rísta \hld\ ok á handar baki &
\ind ok merkia a nagli nꜹþ.\eva

\bvb Ale-runes shalt thou know, if TODO\evb
\evg


\bvg
\bva Full skal signa \hld\ ok við fári séa &
\ind ok verpa lauki í lǫg; &
\edtext{þá þat vęitk, \hld\ at þér verðr aldri &
męini blandinn mjǫðr.}{\lemma{þá \dots\ mjǫðr}\Bfootnote{\emph{thus} \VolsungaSaga, \emph{om.} \Regius}}\eva

\bvb TODO\evb
\evg

...


\bvg
\bva Þá mę́lti \hld\ Míms hǫfuð &
\ind fróðligt it fyrsta orð, &
\ind ok sagði sanna stafi.\eva

\bvb Then spoke the head of Mime learnedly the first word, and said true staves:\evb
\evg


\bvg
\bva Á skildi kvað ristnar \hld\ þęim’s stęndr fyr skínanda goði, &
á ęyra Árvakrs, \hld\ ok á Alsvinns hófi, &
á því hvéli es snýz \hld\ undir ręið Hrungnis, &
á Slęipnis tǫnnum \hld\ ok á slęða fjǫtrum, &
á bjarnar hrammi \hld\ ok á Braga tungu, &
á ulfs klóm \hld\ ok á arnar nęfi, &
á blóðgum vę́ngjum \hld\ ok á brúar sporði, &
á lausnar lófa \hld\ ok á líknar spori, &
á glęri ok á gulli \hld\ ok á gumna hęillum, &
í víni ok virtri \hld\ ok vilisessi. &
Á Gungnis oddi \hld\ ok á Grana brjósti, &
á nornar nagli \hld\ ok á nęfi uglu;\eva

\bvb On a shield it said were carved [runes]—[the shield] that stands before the shining god\footnoteB{According to \Grimnismal\ 39 the sun is covered by a shield, protecting the earth from its heat. Without it, the whole world would burn up.} \ken{sun}—[also] on the ear of Yorewaker, on the hoof of Allswith,\footnoteB{The two horses that pull the sun across the heavens; cf. \Grimnismal\ 38.} on that wheel which turns beneath the chariot of Rungner, on the teeth of Slopner, and on the fetters of sleds, on the paw of the bear, and on the tongue of Bray, on the claws of the wolf, and on the beak of the eagle, on bloody wings, and on the supports of the bridge, on the palm of release, and the track of grace, on glass and on gold, and on the good healths of men, in wine and beerwort, and on the comfortable seat, on the point of Gungner, and on the breast of Grane, on the nail of a norn, and on the beak of an owl.\evb
\evg


\bvg
\bva Allar vǫ́ru af skafnar, \hld\ þę́r es vǫ́ru á ristnar, &
\ind ok hvęrfðar við inn hęlga mjǫð &
\ind ok sęndar á víða vega.\eva

\bvb All were shaven off—those that were carved on—and thrown into the holy mead, and sent on wide ways:\evb
\evg


\bvg
\bva Þę́r ’ru með ǫ́sum, \hld\ þę́r ’ru með ǫlfum, &
\ind sumar með vísum vǫnum, &
\ind sumar hafa męnskir męnn.\eva

\bvb They are among the Ease, they are among the Elves; some among wise Wanes; some manly men have.\evb
\evg

...


\bvg {\small [Sighdrive quoth:]}
\bva ...\eva

\bvb “Now shalt thou choose, as the choice is offered to thee, maple-tree of sharp weapons \ken{warrior}! Speech or silence have thou in thy own heart; all the harms are measured [by the Norns].”\evb
\evg


\bvg {\small [Siwrd quoth:]}
\bva ...\eva

\bvb “I shall not flee, although thou know me to be fey; I am not born with softness.\footnoteB{Note about this common heroic expression.} Thy loving counsels all will I have, for as long as I live.”\evb
\evg


\bvg {\small [Sighdrive quoth:]}
\bva ...\eva

\bvb “That I counsel thee first: that thou against thy kinsmen defend thyself faultlessly. Late ought thou to take revenge, although they incur charges; that they say befits the dead.\evb
\evg


\bvg
\bva Þat rę́ð’k þér annat, \hld\ at ęið né svęrir, &
\ind nema þann ’s saðr séi, &
grimmar simar \hld\ ganga at tryggðrofi; &
\ind armr es vára vargr.\eva

\bvb That I counsel thee second: that thou not swear an oath, save for that one which is true. Grim strands befall the troth-breaker; wretched is the outlaw of vows.\evb
\evg


\bvg
\bva ...\eva

\bvb That I counsel thee third: that thou on the Thing bandy not with foolish men; for an unwise man often lets be spoken worse words than he ought to know.\evb
\evg


\bvg
\bva ...\eva

\bvb All is missing if thou shut up towards it; then thou seemest born with softness, or truthfully accused. Risky is the verdict of neighbours, unless one gets himself a good one.\evb
\evg


\bvg
\bva ...\eva

\bvb At another day make his breath go away, and thus repay the people for the lie.\evb
\evg


% Additional heroic poems
%	For the text of original poem, I do not present the manuscript text, but rather a standardized text of my own. I have however aimed to generally follow the dialect of the manuscript, rather than present a standardized Old High German or Old Saxon. The rules of normalization have been as follows:
Vowels:
> Ms. \emph{ae}, \emph{ei} and \emph{e}, where etymologically from \emph{ai}, have been normalized as \emph{ei}.
> Ms. \emph{o} and \emph{ao}, where etymologically from \emph{au}, have been normalized as \emph{ao}. This may be somewhat controversial.
> \emph{ostar}, \emph{Otachre} > \emph{aostar}, \emph{Aotachre}).
> Ms. \emph{uo} and \emph{o}, where etymologically from long \emph{ō}, have been normalized as \emph{ō}.
Consonants:
> Ms. \emph{r} and \emph{w}, where etymologically from \emph{hw} and \emph{hr}, have been thus normalized. That this was the case in the original poem is obvious; such words never alliterate with \emph{w} or \emph{r}, but only with \emph{r}, as can be most definitively seen in lines 56 (ms.: \alst{h}eremo ... \alst{h}rusti) and 66 (ms.: \alst{h}ewun \alst{h}armlicco \alst{h}uitte). If this were not enough, the retention in the ms. of the \emph{h} at previously given places is yet further support.
> Ms. \emph{tt}, where etymologically from \emph{t}, has been thus normalized.
> Ms. \emph{ƿ} (wynn), \emph{u} and \emph{uu}, where representing \emph{w}, have been thus normalized.

The pronoun which exclusively appears in the ms. as \emph{her} ‘he’ has been so kept, rather than normalized to the standard OHG \emph{er}.
The punctuation of the original (entirely consisting of interpuncts) has not been retained.

----

\bva Ik gihōrta dat seggen &
dat sih urhettun \hld einon mōtīn &
Hiltibrant enti Hadubrant \hld untar heriun tweim &
sunufatarungo \hld iro saro rihtun &
garutun \edtext{sie}{se \HildMS} iro gūdhamun \hld gurtun sih iro swert ana &
helidos ubar \edtext{hringa}{ringa \HildMS} \hld dō sie to dero hiltiu ritun\eva

\bvb I heard it said, that two contenders alone did meet: Hildbrand and Hathbrand, under two hosts. Son and father ordered their armour, readied their war-cloth, girded their swords on, the heroes over the mail, when to that battle they rode.\evb

\bva Hiltibrant gimahalta Heribrantes sunu \hld her was hērōro man &
ferahes frōtōro \hld her frāgēn gistōnt &
fōhēm wortum \hld \edtext{hwer}{wer \HildMS} sin fater wāri &
fireo in folche \hld [...] &
[...] \hld eddo \edtext{hwelīhhes}{welihhes \HildMS} cnōsles dū sīs &
ibu dū mī ēnan sagēs \hld ik mī de odre wēt &
chind in \edtext{chunincrīche}{chunnincriche \HildMS} \hld chūd ist mīn al irmindeot\eva

\bvb Hildbrand spoke, Harbrand's son — he was the hoarier man, more learned in life, — he began to ask with few words, who his father might be, of men in the folk, [...] “or of which lineage thou be; if thou me one say, I the others will know; child, in the kingdom, known to me are all great men.”\evb

\bva Hadubrant gimahalta \hld Hiltibrantes sunu &
dat sagetun mī ūsere liuti &
alte enti frōte \hld dea ērhina wārun &
dat Hiltibrant hēti min fater \hld ih heitu hadubrant &
forn her aostar giweit \hld  flauh her Aotachres nīd &
hina miti Deotrihhe \hld enti sīnero degano filu &
her furlēt in lante \hld luttila sitten &
brūt in būre \hld barn unwahsan &
arbeolaosa \hld her reit aostar hina &
des sid Deotrihhe \hld darba gistōntum &
\edtext{fateres}{fatereres \HildMS} mīnes \hld dat was sō friuntlaos man &
her was Aotachre \hld ummet tirri &
degano dechisto \hld unti \edtext{Deotrihhe}{\emph{add.} darba gistontun \HildMS} &
her was eo folches at ente \hld imo was eo \edtext{fehta}{peheta \HildMS} ti leob &
chūd was her \hld chōnēm mannum &
ni wāniu ih iu līb habbe\eva

\bvb Hathbrand spoke, Hildbrand's son: “It told me our people, the old and learned, those who earlier lived, that Hildbrand was called my father — I am called Hathbrand, — he previously hurried east; he fled Edwaker's hate, thither with Thedrich, and his multitude of thanes. He left in the land a little one to stay, a bride in the bower, a bairn ungrown, without inheritance; he rode east thither, as Thedrich was in great need of my father — that was such a friendless man. He was to Edwaker exceptionally hostile, the dearest of thanes under Thedrich. He was ever at the front of the troop; ever did the fight gladden him; known was he among keen men. — I ween not that he have life.”\evb

\bva weitu irmingot {\small [quad hiltibrant]} \hld obana ab hebane &
dat dū neo dana halt mit sus sibban man &
dinc ni gileitōs &
want her dō ar arme \hld wuntane baoga &
cheisuringu gitān \hld so imo sie der chuning gab &
huneo truhtin \hld dat ih dir it nū bī huldī gibu\eva

\bvb “I call on God as witness, [quoth Hildbrand], above in heaven, that thou never with such a close man once more lead dispute.” Unwound he then from his arm some twisted bighs, made from imperial coins, which the king once gave him, the lord of the Huns: — “This I now give thee as pledge.”\evb

\bva Hadubrant gimahalta \hld Hiltibrantes sunu &
mit gēru scal man \hld geba infāhan &
ort widar orte \hld [...] &
dū bist dir altēr hun \hld ummet spāhēr &
spenis mih mit dīnem wortum \hld wili mih dinu speru werpan &
bist alsō gialtēt man \hld sō dū ēwīn inwit fōrtōs &
dat sagetun mi \hld sēolīdante &
westar ubar wentilsēo \hld dat man wīc furnam &
tōt ist Hiltibrant \hld Heribrantes sunu\eva

\bvb Hathbrand spoke, Hildbrand's son: “With spear shall one earn gifts, point against point! Thou art, old hun, exceptionally clever; thou lurest me with thy words, wilt thou at me hurl thy spear! Thou art thus old, though thou ever deceit hast worked. — It told me seafarers, heading west o’er the Wendle-sea <= Mediterranean>, that war took that man: — dead is Hildbrand, Harbrand's son!”\evb

\bva Hiltibrant gimahalta \hld Heribrantes sunu &
wela gisihu ih in dīnēm hrustim &
dat dū habēs heime \hld hērron gōten &
dat dū noh bī desemo rīche \hld reccheo ni wurti\eva

\bvb Hildbrand spoke, Harbrand's son: “I see well on thy equipment, that thou hast a good lord at home, that thou yet in his reign art not become an exile.\evb

\bva welaga nu waltant got {\small [quad hiltibrant]} \hld weiwurt skihit &
ih wallōta sumaro enti wintro \hld sehstic ur lante &
dar man mih eo scerita \hld in folc sceotantero &
sō man mir at burc einīgeru \hld banun ni gifasta &
nu scal mih swāsat chind \hld swertu haowan &
bretōn mit sīnu billiu \hld eddo ih imo ti banin werdan &
doh maht dū nū aodlīhho \hld ibu dir dīn ellen taoc &
in sus hēremo man \hld hrusti giwinnan &
raoba birahanen \hld ibu du dar einīg reht habēs\eva

\bvb Well now, wielding God, [quoth Hildbrand], woeful Weird passes. I wallowed for summers and winters sixty, out of the land, where one ever placed me in the troop of shooters; thus one at no fortress my bane did inflict. Now shall my own child hew at me with sword; beat down with blade, or I become his bane; — yet canst thou now easily, if thy courage avail thee, from such a hoary man win the equipment, bear away the booty, if thou thereto have any right.\evb

\eva der sī doh nu argōsto {\small [quad hiltibrant]} aostarliuto &
der dir nū wīges warne \hld nū dih et sō wel lustit &
gudea gimeinun \hld niuse der mōti &
hwedar sih \edtext{hiutu dēro}{dēro hiuti \HildMS} hregilo \hld hrōmen mōti &
eddo desero brunnōno \hld beidero waltan\eva

\bvb Yet now he may be the weakest, [quoth Hildbrand], of the eastern peoples, who would refuse thee the fight, when thou so greatly cravest to struggle together. Try he who might, which one today of his arms may boast, or both of these byrnies wield!”\evb

\bva dō lietun sie aerist \hld askim scrītan &
scarpēn scūrim \hld dat in dem sciltim stōnt &
dō stōptun tosamane \hld staimbort hlūdun &
hewun harmlīcco \hld hwīte scilti &
unti imo iro lintūn \hld luttilo wurtun &
giwigan miti wābnum \hld [...]\eva

\bvb Then they first let their ash-spears glide, in a harsh torrent, that they stuck in the shields. Then charged they into each other — the war-boards [SHIELDS] resounded — struck they bitterly the white shields, until their linden-planks [SHIELDS] became little, worn down by the weapons, [...]\evb

%	\include{books/Fight at Finnsbury.tex}
%	\include{books/Beewolf.tex} — Probably not happening.
\end{document}
