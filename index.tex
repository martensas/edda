\section{Cultural and religious terms}
\begin{itemize}

\inxitem{aught} (ON. \emph{ætt}, OE. \emph{ǣht})
  The Nordic (paternal) clan or family line.

\inxitem{fimble-} (ON. \emph{fimbul-})
  The ultimate, final, greatest. See \inx{Fimble-thyle}, \inx{Fimble-winter}.

\inxitem{hame} (ON. \emph{hamr})
  A skin, shape. Individuals can through magic “shift hames” (ON. \emph{skipta hǫmum}), and leave their human \emph{hames} behind, instead entering into the shapes of wolves, bears, birds. During this process the original hame would be sleeping in a vulnerable state, as described in the Saw of the Walsings, chap. TODO: . See also \inx{feather-hame}, \inx{town-riders}, \inx{evening-riders}.

\inxitem{harrow} (ON. \emph{hǫrgr}, OE. \emph{hearg}, PNWGmc. \emph{*harugaʀ})
  A cairn constructed for ritual purposes. \emph{Hind} 10 describes one: “A \inx{harrow} he made for me, loaded with stones; now that stone-pile is become into glass. He reddened [it] in fresh blood of oxen; Oughthere ever trusted on the osennies.” See also \inx{wigh}.

\inxitem{leed} (ON. \emph{ljóð}, OE. \emph{lēod})
  A song or chant with magical qualities.

\inxitem{thyle} (ON. \emph{þulr}, OE. \emph{þyle}, PNWGmc. \emph{*þuliʀ})
  A sage who through rote learning has acquired a large amount of mythological lore (cf. \emph{þula} 'a list in poetic form; a meaningless poem' and \emph{þylja} 'to recite, to chant'). Thus \inx{Weden} is the \inx{Fimble-thyle}, being the unbeaten master of lore, as can be seen in his wisdom contests (see \emph{Alw}, \emph{Web}). Runic inscription DR 248 (Snoldelev) suggests the thyle was somehow bound to a specific place, and in Beowulf it seems to have been a court position, with \inx{Unferth} being described as the "thyle of Rothgar".

\inxitem{wigh} (ON. \emph{vé}, OE. \emph{wēoh}, \emph{wīh}, PNWGmc. \emph{*wīhą})
  A holy shrine or sanctuary. It seems that where the \inx{harrow} was a pile of stones or cairn used for carrying out rituals, the \emph{wigh} was an enclosed space. The earliest Norse attestation is the runic inscription Ög N288 (Oklunda), which reads: “Guthhere <= Gunnarr> painted these runes, and he fled, guilty. Sought this wigh, and he fled into this clearing. And he bound. [...]” The implication seems to be that the wigh was considered so sacred that Guthhere could not be apprehended or punished for his crime while in it. — In Old English the word means ‘pagan idol’. It is not immediately clear which meaning is the original one, but in this edition the Norse sense has been adopted, since the Anglo-Saxon sources are all of a Christian nature. The \emph{Beewolf} name \emph{Wighstone} (\emph{Wīh-} or \emph{Wēohstān}) in any case suggests it is the Norse meaning, since ‘idol-stone’ makes little sense.

\inxitem{wode} (ON. \emph{óðr}, OE. \emph{wōd}, PNWGmc. \emph{*wōþuʀ})
  \inx{Hean}'s gift to men, though the name would suggest it be from \inx{Weden}. The word has several related meanings: ‘poetic inspiration’, ‘madness’, ‘rage’.

\end{itemize}


\section{Personal names and objects}
\begin{itemize}

\inxitem{Attle} (\emph{Attila}, ON. \emph{Atli}, OE. \emph{Ætla}, MHG. \emph{Etzel}, PNWGmc. \emph{*Attilō})
  The ruler of the \inx{Huns} (historically from 434–453). Husband of \inx{Guthrun}, and with her father of \inx{Earp and Oatle}. and murderer of
  I HHb 54, SiL 11, I Gr 23, ShS 28, 29, 33, 37, 54, 56, 57, II Gr 26, 38, 45, III Gr 1, 9, BnOr 0, OdW A, 2, 22, 23, 25, 26, 30, 31, AtD 0, AtL 1, 3, 15, 17, 18, 27, 31, 32, 34, 36, 37, 38, 41, 43, B, AtS 2, 4, 21, 22, 44, 52, 60, 64, 71, 73, 77, 80, 86, 87, 97, 98, 108, 113, 117, FGr 0, GrB 12, Ham 6.

\inxitem{Earp and Oatle} (ON. \emph{Erpr ok Eitill})
  The sons of \inx{Attle} and \inx{Guthrun}.

\inxitem{Feather-hame} (ON. \emph{fjaðrhamr})
  A \inx{hame} owned by the Ease that lets the wearer fly like a bird, more specifically a falcon.

\inxitem{Guthrun} (ON. \emph{Guðrún})
  Daughter of king \inx{Yivick}, sister of \inx{Guthhere} and \inx{Hain}. The wife of \inx{Attle}.

\inxitem{Hain} 1 (ON. \emph{Hǫgni}, OE. \emph{Haguna}, \emph{Hagena}, OHG. \emph{Hagano}, Ger. \emph{Hagen}, PNWGmc. \emph{*Hagunō})
  A \inx{Nifling} and \inx{Yifking}, son of king \inx{Yivick}, brother of \inx{Guthhere} and \inx{Guthrun}. In \emph{AtL} he defeats seven warriors before being captured by \inx{Attle}, who has his heart cut out at the request of Guthhere.

\inxitem{Hain} 2 A petty king of \inx{East Geatland}, contemporary with \inx{Granmer}, the king of \inx{Southmanland} and Ingeld Illrede, the \inx{Ingling} king of \inx{Upland}.

\inxitem{Hindle} (ON. \emph{Hyndla}) A witch awoken by Frow in \emph{Hind}.

\inxitem{Millner} (ON. \emph{Mjǫllnir}, OE. \emph{*Meldne}, PNWGmc. \emph{*Meldunjaʀ})
  Powerful hammer owned by Thunder.

\inxitem{Oughter} (ON. \emph{Óttarr}, OE. \emph{Ōhthere}, PNWGmc. \emph{*Ōhtaharjaʀ})
  TODO

\inxitem{Rotholf} (ON. \emph{Hrólfr kraki}, OE. \emph{Hrōþulf}, PNWGmc. \emph{*Hrōþiwulfaʀ})
  A king of the \inx{Shieldings} (see family tree). As foreshadowed in \emph{Beewolf} (1017–9, 1180–90), he betrays the sons of \inx{Rothgar}, his cousins \inx{Rethrich and Rothmund}, in order to take the throne for himself. In the later Icelandic tradition this has been forgotten, and he is consistently portrayed as a heroic king.

\inxitem{Rothgar} (ON. \emph{Hróarr}, OE. \emph{Hrōþgār}, PNWGmc. \emph{*Hrōþigaiʀaʀ})
  A king of the \inx{Shieldings} (see family tree), one of the main characters in \emph{Beewolf}.

\inxitem{Weden} (rhymes with \emph{leaden}; ON. \emph{Óðinn}, OE. \emph{Wōden}, \emph{Wēden}, OHG. \emph{Wuotan}, PNWGmc. \emph{*Wōdanaʀ})
  Chief of the \inx{Ease}, his name is clearly related to \inx{wode}, referring to his role as the patron of \inx{scolds} and \inx{bearserks}. For the meaning of his other names see \inx{Fimblethyle}, \inx{Harn} TODO. Husband of \inx{Frie}, and by her father of \inx{Bolder}. Also father of \inx{Thunder} by \inx{Earth}. Brother of \inx{Hean} and \inx{Lother}.

\inxitem{Yivick} (ON. \emph{Gjúki}, OE. \emph{Gifica}, OHG. \emph{Gibicho}, MHG. \emph{Gibeche})
  King of the \inx{Burgends} (historically from late 300s–407) of the Nifling dynasty, founder of the \inx{Yifking} aught†. Father of \inx{Guthrun}, \inx{Guthhere} and \inx{Hain}.

\end{itemize}


\section{Groups and place names}
\begin{itemize}

TODO: Map of rough tribal areas..

\inxitem{Danes} (ON. \emph{danir}, OE. \emph{Dene})
  A tribe in eastern modern-day Denmark and southern Sweden. They probably originated in Scania in southern Sweden, before moving westwards into the Danish isles and eventually Jutland, driving out the \inx{Earls} and \inx{Jutes}.
  Noted members: TODO
  Attestations: TODO

\inxitem{Ease} (rhymes with \emph{geese}; ON. \emph{æsir}, OE. \emph{ēse}, PNWGmc. \emph{*ansiwiʀ})
  A group of Gods, though the word can also refer to all the Gods. Singular \inx{os}. See \inx{Gods}, \inx{Tues}, \inx{Wanes}, \inx{Powers}.
  Noted members: \inx{Weden}, \inx{Thunder}, \inx{Frie}, \inx{Hath} and \inx{Bolder}
  Attestations: TODO

\inxitem{Ettins} (ON. \emph{jǫtnar}, OE. \emph{eotenas}, PNWGmc. \emph{*etunōʀ})
  The fundamental enemies of the Gods, the agents of chaos and disorder. See \inx{Rises}, \inx{Thurses}.
  Noted members: \inx{Thrym}
  Attestations: TODO

\inxitem{Geats} (ON. \emph{gautar}, OE. \emph{Gēatas}, PNWGmc. \emph{*gautōʀ})
  A tribe in what is today southern-central Sweden. See also \inx{Geatland}.
  Noted members: TODO
  Attestations: TODO

\inxitem{Gods} (ON. \emph{goð}, OE. \emph{godu}, OHG. \emph{gota})
  TODO.
  Noted members: TODO
  Attestations: TODO

\inxitem{Huns} (ON. \emph{húnir}, OE. \emph{Hūne}, OHG. \emph{Hūni}, \emph{Hunni}, PNWGmc. \emph{*hūnīʀ})
  TODO.
  Noted members: TODO
  Attestations: TODO

\inxitem{os} (ON. \emph{áss}, OE. \emph{ōs}, PNWGmc. \emph{*ansuʀ})
  A member of the Ease, or a god in general. See \inx{Ease}, \inx{Gods}.
  Noted members: TODO
  Attestations: TODO

\inxitem{Saxes} (ON. \emph{saxar}, OE. \emph{Seaxan}, \emph{Seaxe})
  TODO.
  Noted members: TODO
  Attestations: TODO

\inxitem{Shieldings} (ON. \emph{Skjǫldungar}, OE. \emph{Scyldingas})
  The descendants of \inx{Shield}, the legendary ruling dynasty of the \inx{Danes}. With \inx{Harward}'s death after his slaying of \inx{Rotholf} their rule ended. TODO
  Noted members: TODO
  Attestations: TODO

\inxitem{Shelfings} (ON. \emph{Skilfingar}, OE. \emph{Scilfingas})
  The exact difference between Shelfings and \inx{Inglings} is unclear. According to the Saw of Geatrich
  Noted members: TODO
  Attestations: \emph{Hind} 15, 20

\inxitem{Swedes} (ON. \emph{svíar}, OE. \emph{Swēon})
  TODO.
  Noted members: TODO
  Attestations: TODO

\inxitem{Thurses} (ON. \emph{þursar}, OE. \emph{þyrs}, OS. \emph{thuris}, OHG. \emph{duris}, PNWGmc. \emph{*þurisaʀ})
  Possibly a poetic synonym for \inx{Ettins}. See also \inx{Rime-Thurse}
  Noted members: TODO
  Attestations: Wal 8, Shr 31, 35, 36, Hyme 17, Thr 5, 10, 21, 24, 29, 30, Alw 2, I HHb 40, HHw 27.

\inxitem{Tues} (ON. \emph{tívar})
  A poetic synonym for \inx{Gods}.
  Noted members: —
  Attestations: TODO

\inxitem{Yifkings} (ON. \emph{Gjúkungar})
  The descendants of \inx{Yivick}, including \inx{Guthhere}, \inx{Guthrun} and \inx{Hain}.
  Noted members: TODO
  Attestations: TODO

\end{itemize}
