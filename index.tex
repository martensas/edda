\bookStart{Index}

\section{Cultural and religious terms (C)}
\begin{itemize}

\inxitem{ape}[C] (ON. \emph{api}, OE. \emph{apa}, OS. \emph{apo}, OHG. \emph{affo}, PNWGmc. \emph{*apó})
  In the Old Norse the word seems to mean ‘fool, buffoon’, in the other old languages apparently ‘monkey’, though this sense should be a later development of the former; why would the early Germanic tribes have a word for an animal that they had never encountered?

\inxitem{aught}[C] (ON. \emph{ę́tt}, OE. \emph{ǽht} ‘possession, property’)
  The Nordic (paternal) clan or family line.

\inxitem{begale}[C] (OHG. \emph{bigalan})
  To affect something using \inx{galder}[C][galders]. See also \inx{gale}[C].

\inxitem{bigh}[C] (ON. \emph{baugr}, OE. \emph{béag}, OHG. \emph{boug})
  A torc or armlet, in the migration period used as currency or tokens of loyalty (see particularly \Hildebrandslied). often referenced in ruler-kennings.

\inxitem{bloot}[C] (ON. \emph{blót}, OE. \emph{blót}, OHG. \emph{bluóz})
  Sacrifice or a sacrificial feast.

\inxitem{Doom}[C] (ON. \emph{dómr}, OE. \emph{dóm})
  Commonly ‘judgement’ (whence Doomsday, ‘day of judgement’), but also specifically referring to one’s fame or good reputation (that is, how other men judge one’s character and deeds). Thus \Havamal\ 77: “I know one that never dies: the \textbf{Doom} over each man dead.”; this is further illuminated by passages in \Beowulf\ such as 884b–887a: \\ \emph{... · Sigemunde gesprong \\ æfter déaðdæge · dóm unlýtel \\ syþðan wíges heard · wyrm ácwealde \\ hordes hyrde · ...} \\ “For Sighmund sprang up after his day of death unlittle \textbf{Doom}, since hard in conflict he defeated the \inx{Worm}, the herder of the hoard.”; \\ 953b–955a: \\ \emph{... · þú þé self hafast \\ dę́dum gefremed · þæt þín dóm lyfað \\ áwa tó aldre · ...} \\ “Thou hast for thyself by deeds accomplished that thy \textbf{Doom} lives for ever and ever.”

\inxitem{fee}[C] (ON. \emph{fé}, OE. \emph{féoh})
  Originally ‘cattle’, however also used in a broader sense to refer to one’s mobile wealth. For this cf. particularly \Havamal TODO.

\inxitem{feelcunning}[C] (ON. \emph{fjǫlkunnigr})
  Skilled with sorcery.

\inxitem{fimble-}[C] (ON. \emph{fimbul-})
  The ultimate, final, greatest. See \inx{Fimble-thyle}, \inx{Fimble-winter}.

\inxitem{five days}[C] (ON. \emph{fimm dagar})
  That the old Scandinavian week was \textbf{five days} long is well attested. According to the \Gulatingslog\ there were six weeks in a month, and the expression \textbf{five days} is used as the equivalent of \emph{week} in \Havamal 51 and 74, in the second of which it is contrasted with \emph{month}. Related to this is the legal term \emph{fifth} (ON. \emph{fimmt}, OSw. \emph{fæmt}), a meeting or gathering set to be held at a five-day notice. See \emph{fimt} in \CV, \LMNL\ for further discussion.

\inxitem{galder}[C] (ON. \emph{galdr}, OE. \emph{gealdor}, OHG. \emph{galdar})
  A magical spell or song. See the Merseburg charms (TODO?) for examples. See also \inx{gale}[C].

\inxitem{gand}[C] (ON. \emph{gandr}, Latin \emph{gandus})
  A witch’s familiar, a spirit sent out to do her bidding. See PCRN HS I:17, p. 361 and II:26, p. 656. TODO

\inxitem{gin-}[C] (ON. \emph{ginn-})
  A rare prefix, maybe referring to sacrosanctity. TODO.

\inxitem{hame}[C] (ON. \emph{hamr})
  A skin, shape. Individuals can through magic “shift hames” (ON. \emph{skipta hǫmum}), and leave their human \emph{hames} behind, instead entering into the shapes of wolves, bears, birds. During this process the original hame would be sleeping in a vulnerable state, as described in the Saw of the Walsings, chap. TODO: . See also \inx{feather-hame}, \inx{town-riders}, \inx{evening-riders}.

\inxitem{harrow}[C] (ON. \emph{hǫrgr}, OE. \emph{hearg}, PNWGmc. \emph{*harugaʀ})
  A cairn constructed for ritual purposes. \emph{Hind} 10 describes one: “A \inx{harrow} he made for me, loaded with stones; now that stone-pile is become into glass. He reddened [it] in fresh blood of oxen; Oughthere ever trusted on the osennies.” See also \inx{wigh}.

\inxitem{leat}[C] (ON. \emph{hlaut})
  Sacrificial blood (that is, taken from the animal), especially when used for auguries.

\inxitem{leat-twig}[C] (ON. \emph{hlauttęinn})
  A twig used to sprinkle the \inx{leat}[C] in auguries (presumably the pattern of the blood would then be inspected).

\inxitem{leed}[C] (ON. \emph{ljóð}, OE. \emph{léod})
  A magical chant or incantation. See also \inx{galder}[C], \inx{gale}[C], \inx{begale}[C].

\inxitem{manwit}[C] (ON. \emph{manvit})
  Practical sense and wisdom, situational awareness, ‘common sense’.

\inxitem{orlay}[C] (ON. \emph{ørlǫg}, OE. \emph{orlæg})
  One’s predetermined fate, destiny, purpose as decreed by the \inx{Norns}.

\inxitem{rest}[C] (ON. \emph{rǫst})
  The distance between two rest-stops, a geographical mile (about 1850 metres). See especially \CV.

\inxitem{rune}[C] (ON. \emph{rún}, OE. \emph{rún}, OS. \emph{rúna}, OHG. \emph{rúna}, Got. \emph{rúna}, PNWGmc. \emph{rūnu})
  An (esoteric) secret message or formula. That this—rather than ‘letter (of a Runic alphabet)’—is the original and proper sense is apparent from among others the Finnish borrowing \emph{runo} ‘poem; poetry; a division of a poem (specifically of the \emph{Kalevala})’, and its use in the singular in the earliest Runic inscriptions such as Noleby Vg 63 (which contains the linguistically indecipherable string of letters {ᚢᚾᚨᚦᛟᚢᛊᚢᚺᚢᚱᚨᚺᛊᚢᛊᛁᚺ[--]ᚨᛁ\rotatebox[origin=c]{180}{ᛏ}ᛁᚾ}, a \emph{rune} in the proper sense) or the recently discovered Svingerud fragment. Thus, Weden’s taking of the \emph{runes} should not be interpreted as merely a myth for the invention of profane writing, but rather the origin of esoteric incantations, not at all unlike Indian \emph{mantras}.
  The word for letter was instead \inx{stave}[C], see also there.

\inxitem{soo}[C] (ON. \emph{sóa})
  To ritually waste, the slaying in the animal sacrifice.

\inxitem{thill}[C] (ON. \emph{þylja})
  To chant poetry or lists (so called \inx{thules}[C][thule]) acquired by rote memorization. See also {thyle}[C].

\inxitem{Thing}[C] (ON., OE. \emph{þing}, OS. \emph{thing}, OHG. \emph{ding})
  The legal assembly and gathering place where matters would be settled and the law recited.

\inxitem{thyle}[C] (ON. \emph{þulr}, OE. \emph{þyle}, PNWGmc. \emph{*þuliʀ})
  A sage who through rote learning has acquired a large amount of mythological lore (cf. \emph{þula} 'a list in poetic form; a meaningless poem' and \emph{þylja} 'to recite, to chant'). Thus \inx{Weden} is the \inx{Fimble-thyle}, being the unbeaten master of lore, as can be seen in his wisdom contests (see \Allvismal, \Vafthrudnismal). Runic inscription DR 248 (Snoldelev) suggests the thyle may have tied to a specific place, and in Beowulf it seems to have been a court position, with \inx{Unferth} being described as the "thyle of Rothgar".

\inxitem{wale}[C] (ON. \emph{vǫlr})
  The staff or sceptre, especially of a wallow. TODO: archeological finds, mention Sutton Hoo.

\inxitem{wallow}[C] (ON. \emph{vǫlva}, OE. \emph{*wealwe} (cf. ON. \emph{svǫlva}, OE. \emph{swealwe} ‘swallow’))
  A sibyl, seeress, oracle. The word derives from the \inx{wale}[C], a staff or sceptre probably used for ritual purposes.

\inxitem{wigh}[C] (ON. \emph{vé}, OE. \emph{wéoh}, \emph{wíh}, PNWGmc. \emph{*wīhą})
  A holy shrine or sanctuary. It seems that where the \inx{harrow} was a pile of stones or cairn used for carrying out rituals, the \emph{wigh} was an enclosed space. The earliest Norse attestation is the runic inscription Ög N288 (Oklunda), which reads: “Guthhere <= Gunnarr> painted these runes, and he fled, guilty. Sought this wigh, and he fled into this clearing. And he bound. [...]” The implication seems to be that the wigh was considered so sacred that Guthhere could not be apprehended or punished for his crime while in it. — In Old English the word means ‘pagan idol’. It is not immediately clear which meaning is the original one, but in this edition the Norse sense has been adopted, since the Anglo-Saxon sources are all of a Christian nature. The \emph{Beewolf} name \emph{Wighstone} (\emph{Wīh-} or \emph{Wēohstān}) in any case suggests it is the Norse meaning, since ‘idol-stone’ makes little sense.

\inxitem{wode}[C] (ON. \emph{óðr}, OE. \emph{wód}, PNWGmc. \emph{*wōþuʀ})
  \inx{Hean}'s gift to men, though the name would suggest it be from \inx{Weden}. The word has several related meanings: ‘poetic inspiration’, ‘madness’, ‘rage’.

\end{itemize}


\section{Personal names, objects and events (P)}

\begin{itemize}

\inxitem{Attle}[P] (\emph{Attila}, ON. \emph{Atli}, OE. \emph{Ætla}, MHG. \emph{Etzel}, PNWGmc. \emph{*Attilō})
  The ruler of the \inx{Huns} (historically from 434–453). Husband of \inx{Guthrun}, and with her father of \inx{Earp and Oatle}. and murderer of
  I HHb 54, SiL 11, I Gr 23, ShS 28, 29, 33, 37, 54, 56, 57, II Gr 26, 38, 45, III Gr 1, 9, BnOr 0, OdW A, 2, 22, 23, 25, 26, 30, 31, AtD 0, AtL 1, 3, 15, 17, 18, 27, 31, 32, 34, 36, 37, 38, 41, 43, B, AtS 2, 4, 21, 22, 44, 52, 60, 64, 71, 73, 77, 80, 86, 87, 97, 98, 108, 113, 117, FGr 0, GrB 12, Ham 6.

\inxitem{Earp and Oatle}[P] (ON. \emph{Erpr ok Eitill})
  The sons of \inx{Attle} and \inx{Guthrun}.

\inxitem{Feather-hame}[P] (ON. \emph{fjaðrhamr})
  A \inx{hame} owned by the Ease that lets the wearer fly like a bird, more specifically a falcon.

\inxitem{Guthrun}[P] (ON. \emph{Guðrún})
  Daughter of king \inx{Yivick}, sister of \inx{Guthhere} and \inx{Hain}. The wife of \inx{Attle}.

\inxitem{Hain}[P][Hain 1] (ON. \emph{Hǫgni}, OE. \emph{Haguna}, \emph{Hagena}, OHG. \emph{Hagano}, Ger. \emph{Hagen}, PNWGmc. \emph{*Hagunō})
  A \inx{Nifling} and \inx{Yifking}, son of king \inx{Yivick}, brother of \inx{Guthhere} and \inx{Guthrun}. In \emph{AtL} he defeats seven warriors before being captured by \inx{Attle}, who has his heart cut out at the request of Guthhere.

\inxitem{2}[P] A petty king of \inx{East Geatland}, contemporary with \inx{Granmer}, the king of \inx{Southmanland} and Ingeld Illrede, the \inx{Ingling} king of \inx{Upland}.

\inxitem{Hindle}[P] (ON. \emph{Hyndla}) A witch awoken by Frow in \emph{Hind}.

\inxitem{Millner}[P] (ON. \emph{Mjǫllnir}, OE. \emph{*Meldne}, PNWGmc. \emph{*Meldunjaʀ})
  Powerful hammer owned by Thunder.

\inxitem{Oughter}[P] (ON. \emph{Óttarr}, OE. \emph{Óhthere}, PNWGmc. \emph{*Ōhtaharjaʀ})
  TODO

\inxitem{Rakes of the Reins}[P] (ON. \emph{ragna rǫk})
  The ‘sequence of events of the \inx{Reins}[G]’, euphemistic for the destruction of the Home.

\inxitem{Rakes of the Tues}[P] (ON. \emph{tíva rǫk})
  See the \inx{Rakes of the Reins}[P].

\inxitem{Rotholf}[P] (ON. \emph{Hrólfr kraki}, OE. \emph{Hróþulf}, PNWGmc. \emph{*Hrōþiwulfaʀ})
  A king of the \inx{Shieldings} (see family tree). As foreshadowed in \emph{Beewolf} (1017–9, 1180–90), he betrays the sons of \inx{Rothgar}, his cousins \inx{Rethrich and Rothmund}, in order to take the throne for himself. In the later Icelandic tradition this has been forgotten, and he is consistently portrayed as a heroic king.

\inxitem{Rothgar}[P] (ON. \emph{Hróarr}, OE. \emph{Hróþgár}, PNWGmc. \emph{*Hrōþigaiʀaʀ})
  A king of the \inx{Shieldings} (see family tree), one of the main characters in \emph{Beewolf}.

\inxitem{Weden}[P] (rhymes with \emph{leaden}; ON. \emph{Óðinn}, OE. \emph{Wóden}, \emph{Wéden}, OHG. \emph{Wuotan}, PNWGmc. \emph{*Wōdanaʀ})
  Chief of the \inx{Ease}, his name is clearly related to \inx{wode}, referring to his role as the patron of \inx{scolds} and \inx{bearserks}. For the meaning of his other names see \inx{Fimblethyle}, \inx{Harn} TODO. Husband of \inx{Frie}, and by her father of \inx{Bolder}. Also father of \inx{Thunder} by \inx{Earth}. Brother of \inx{Hean} and \inx{Lother}.

\inxitem{Yivick}[P] (ON. \emph{Gjúki}, OE. \emph{Gifica}, OHG. \emph{Gibicho}, MHG. \emph{Gibeche})
  King of the \inx{Burgends} (historically from late 300s–407) of the Nifling dynasty, founder of the \inx{Yifking} aught†. Father of \inx{Guthrun}, \inx{Guthhere} and \inx{Hain}.

\end{itemize}


\section{Groups and place names (G)}

TODO: Map of rough tribal areas. Geneaologies.

\begin{itemize}

\inxitem{Danes}[G] (ON. \emph{danir}, OE. \emph{dene}, PNWGmc. \emph{*daníʀ})
  A tribe in eastern modern-day Denmark and southern Sweden. They probably originated in Scania in southern Sweden, before moving westwards into the Danish isles and eventually Jutland, driving out the \inx{Earls} and \inx{Jutes}.
  Noted members: TODO
  Attestations: TODO

\inxitem{Dwarfs}[G] (ON. \emph{dvergar}, OE. \emph{dweorgas}, OHG. \emph{twerca}, PNWGmc. \emph{*dwergóʀ})
  Earthly (chthonic) supernatural beings, often referred to as living in rocks and mountains.
  Noted members: TODO
  Attestations: TODO

\inxitem{Ease}[G] (rhyming with \emph{geese}; ON. \emph{ę́sir}, OE. \emph{ése}, PNWGmc. \emph{*ansiwiʀ}; sg. \emph{os}, ON. \emph{áss}, OE. \emph{ós}, PNWGmc. \emph{*ansuʀ})
  A group of Gods, though the word can also refer to all the Gods. See \inx{Gods}, \inx{Tues}, \inx{Wanes}, \inx{Powers}.
  Noted members: \inx{Weden}, \inx{Thunder}, \inx{Frie}, \inx{Hath} and \inx{Bolder}
  Attestations: TODO

\inxitem{Ease and Elves}[G] (ON. \emph{ę́sir ok alfar}, OE. \emph{ése ende ielfe}, PNWGmc. \emph{*alβíʀ jah ansiwiʀ})
  A merism; both heavenly and earthly spiritual beings. Notably the words always occur in this order.

\inxitem{Elves}[G] (ON. \emph{alfar}, OE. \emph{ielfe}, PNWGmc. \emph{*alβíʀ})
  Earthly (chthonic) supernatural beings. Possibly ancestral spirits?
  Noted members: TODO
  Attestations: TODO

\inxitem{Ettins}[G] (ON. \emph{jǫtnar}, OE. \emph{eotenas}, PNWGmc. \emph{*etunóʀ})
  The fundamental enemies of the Gods, the agents of chaos and disorder. See \inx{Rises}, \inx{Thurses}.
  Noted members: \inx{Thrym}
  Attestations: TODO

\inxitem{Geats}[G] (ON. \emph{gautar}, OE. \emph{géatas}, PNWGmc. \emph{*gautóʀ} from \emph{*geut-} ‘to pour’, perhaps ‘the libators’)
  A tribe in what is today southern-central Sweden. See also \inx{Geatland}.
  Noted members: TODO
  Attestations: TODO

\inxitem{gin-Reins}[G] (ON. \emph{ginnręgin})
  \inx{gin-}[C] + \inx{Reins}[G]. The sacrosanct, highest divine powers.

\inxitem{Gods}[G] (ON. \emph{goð}, OE. \emph{godu}, OHG. \emph{gota}, PNWGmc. \emph{*godu})
  TODO.
  Noted members: TODO
  Attestations: TODO

\inxitem{Huns}[G] (ON. \emph{húnir}, OE. \emph{Húne}, OHG. \emph{Húni}, \emph{Hunni}, PNWGmc. \emph{*húníʀ})
  An invading Asiatic tribe in the Migration Period. In the legendary material their cultural and ethnic foreignness is not seen.
  Noted members: TODO
  Attestations: TODO

\inxitem{Nears}[G] (ON. \emph{níarar} ~ \emph{njárar})
  A Swedish tribe, only mentioned in \Volundarkvida, where it is ruled by king \inx{Nithad}[P]. The name and location may allow us to connect them with the Swedish province of Närke, cf. Old Swedish: \emph{Nærikiar} ‘inhabitants of Närke’, \emph{Nærisker} ‘belonging to Närke; Nearish’. The Old Swedish stem \emph{nær-} would then be a reduced form of \emph{níar-}, \emph{njár-}.

\inxitem{Reins}[G] (ON. \emph{rǫgn}, \emph{ręgin})
  The divine powers. Based on \Vafthrudnismal\ (TODO) the term may be more closely associated with the \inx{Wanes}[G] than the \inx{Ease}[G].

\inxitem{Saxons}[G] (ON. \emph{saxar}, OE. \emph{Seaxan}, \emph{Seaxe})
  TODO.
  Noted members: TODO
  Attestations: TODO

\inxitem{Shieldings}[G] (ON. \emph{skjǫldungar}, OE. \emph{Scyldingas}, PNWGmc. \emph{*skeldungóʀ})
  The descendants of \inx{Shield}[P], the legendary ruling dynasty of the \inx{Danes}. With \inx{Harward}'s death after his slaying of \inx{Rotholf} their rule ended. TODO
  Noted members: TODO
  Attestations: TODO

\inxitem{Shelfings}[G] (ON. \emph{skilfingar}, OE. \emph{Scilfingas}, PNWGmc. \emph{*skilβingóʀ})
  The descendants of \inx{Shelf}[P]. The exact difference between Shelfings and \inx{Inglings} is unclear. According to the Saw of Geatrich TODO
  Noted members: TODO
  Attestations: \Hyndluljod 15, 20

\inxitem{Swedes}[G] (ON. \emph{svíar}, OE. \emph{Swéon}, PNWGmc. \emph{*swihaníʀ})
  TODO.
  Noted members: TODO
  Attestations: TODO

\inxitem{Thurses}[G] (sg. Thurse; ON. \emph{þurs}, OE. \emph{þyrs}, OS. \emph{thuris}, OHG. \emph{duris}, PNWGmc. \emph{*þurisaʀ})
  Possibly a poetic synonym for \inx{Ettins}. See also \inx{Rime-Thurse}
  Noted members: TODO
  Attestations: Wal 8, Shr 31, 35, 36, Hyme 17, Thr 5, 10, 21, 24, 29, 30, Alw 2, I HHb 40, HHw 27.

\inxitem{Tues}[G] (ON. \emph{tívar}, PNWGmc. \emph{*tíwóʀ})
  A poetic synonym for \inx{Gods}.
  Noted members: —
  Attestations: TODO

\inxitem{Wanes}[G] (ON. \emph{vanir}, OE. \emph{wan-?})
  A tribe of the gods, associated with fertility, harvests and fishing.
  Noted members: TODO
  Attestations: TODO

\inxitem{Yifkings}[G] (ON. \emph{gjúkungar})
  The descendants of \inx{Yivick}, including \inx{Guthhere}, \inx{Guthrun} and \inx{Hain}.
  Noted members: TODO
  Attestations: TODO

\end{itemize}
