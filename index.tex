\bookStart{Index}

NOTE: This index or rather dictionary is both incomplete and inconsistently formatted. New entries will be added, and old ones be corrected in the future.

\section{Cultural and religious expressions (C)}
\begin{itemize}

\inxitem[C]{ape} (ON. \emph{api}, OE. \emph{apa}, OS. \emph{apo}, OHG. \emph{affo}, PNWGmc. \emph{*apó})
  In the Old Norse the word seems to mean ‘fool, buffoon’, in the other old languages apparently ‘monkey’, though this sense should be a later development of the former; why would the early Germanic tribes have a word for an animal that they had never encountered?

\inxitem[C]{aught} (ON. \emph{ę́tt}, OE. \emph{ǽht} ‘possession, property’)
  The Nordic (paternal) clan or family line.

\inxitem[C]{begale} (OHG. \emph{bigalan})
  To affect something using \inx[C]{galder}[galders]. See also \inx[C]{gale}.

\inxitem[C]{bigh} (ON. \emph{baugr}, OE. \emph{béag}, OHG. \emph{boug})
  A torc or armlet, in the migration period used as currency or tokens of loyalty (see particularly \Hildebrandslied). often referenced in ruler-kennings.

\inxitem[C]{bloot} (ON. \emph{blót}, OE. \emph{blót}, OHG. \emph{bluoz})
  Sacrifice or a sacrificial feast.

\inxitem[C]{bloot-kettle}
  The large pots used for cooking the bloot-stew.

\inxitem[C]{Doom} (ON. \emph{dómr}, OE. \emph{dóm})
  Commonly ‘judgement’ (whence Doomsday, ‘Judgement Day’), but in the Norse and English poetry also specifically referring to one’s fame or good reputation (that is, how others will judge one’s character and deeds). Thus \Havamal\ 77: “I know one that never dies: the \textbf{Doom} over each man dead.” is illuminated by passages in \Beowulf\ like 884b–887a: \\ \emph{... · Sigemunde gesprong \\ æfter déaðdæge · \textbf{dóm} unlýtel \\ syþðan wíges heard · wyrm ácwealde \\ hordes hyrde · ...} \\ “For Sighmund sprang up after his day of death unlittle \textbf{Doom}, since hard in conflict he defeated the \inx[C]{Worm}, the herder of the hoard.”; \\ 953b–955a: \\ \emph{... · þú þé self hafast \\ dę́dum gefremed · þæt þín \textbf{dóm} lyfað \\ áwa tó aldre · ...} \\ “Thou hast for thyself by deeds accomplished that thy \textbf{Doom} lives for ever and ever.”

\inxitem[C]{fee} (ON. \emph{fé}, OE. \emph{féoh})
  Originally ‘cattle’, however also used in a broader sense to refer to one’s mobile wealth. For this cf. particularly \Havamal TODO.

\inxitem[C]{feelcunning} (ON. \emph{fjǫlkunnigr})
  Literally ‘much-cunning, cunning in many ways’. Skilled with sorcery.

\inxitem[C]{fey} (ON. \emph{fęigr}, OE. \emph{fǽge}, OHG. \emph{feigi} ‘cowardly’)
  One doomed or fated to die, with a sense of predestination and inevitability. Its earliest use is on the Rök stone: \textbf{aft uamuþ stąnta runaʀ þaʀ ᛭ n uarin faþi faþiʀ aft} faikiąn \textbf{sunu} “After Woemood (\emph{Vámóðr}) stand these \inx[C]{runes}, but Warren (\emph{Varinn}) painted, the father after the \textbf{fey} son.” It was believed that one’s See PCRN HS II:35, p. 928 ff.

\inxitem[C]{fimble-} (ON. \emph{fimbul-})
  The ultimate, final, greatest. See \inx[P]{Fimblethyle}, \inx[L]{Fimblewinter}.

\inxitem[C]{five days} (ON. \emph{fimm dagar})
  That the old Scandinavian week was \textbf{five days} long is well attested. According to the \Gulatingslog\ there were six weeks in a month, and the expression \textbf{five days} is used as the equivalent of \emph{week} in \Havamal 51 and 74, in the second of which it is contrasted with \emph{month}. Related to this is the legal term \emph{fifth} (ON. \emph{fimmt}, OSw. \emph{fæmt}), a meeting or gathering set to be held at a five-day notice. See \emph{fimt} in \CV, \LMNL\ for further discussion.

\inxitem[C]{galder} (ON. \emph{galdr}, OE. \emph{gealdor}, OHG. \emph{galdar})
  A magical spell or song. See the Merseburg charms (TODO?) for examples. See also \inx[C]{gale}.

\inxitem[C]{gand} (ON. \emph{gandr}, Latin \emph{gandus})
  A witch’s familiar, a spirit sent out to do her bidding. See PCRN HS I:17, p. 361 and II:26, p. 656. TODO

\inxitem[C]{gin-} (ON. \emph{ginn-})
  A rare augmentative prefix. TODO.

\inxitem[C]{good of meat} (ON. \emph{matar gó})
   An old expression, appearing not just in \Havamal\ 39 (I found not a generous man, or so \textbf{good of meat}, that a gift were not accepted;”) but also several Viking Age Runic inscriptions, such as Sm 39: \emph{mildan orða · ok mataʀ góðan} “mild of words and \textbf{good of meat}”, U 805: \emph{bónda góðan matar} “a farmer \textbf{good of meat}”, U 703: \emph{mandr matar góðr · auk máls risinn} “a man \textbf{good of meat} and proud in speech”; compare also U 739: \emph{hann vaʀ mildr mataʀ · auk máls risinn} “he was \textbf{mild of meat} and proud in speech”. — See \inx[C]{meat-nithing} for its opposite.

\inxitem[C]{hame} (ON. \emph{hamr})
  A skin, shape. Individuals can through magic “shift hames” (ON. \emph{skipta hǫmum}), and leave their human \emph{hames} behind, instead entering into the shapes of wolves, bears, birds. During this process the original hame would be sleeping in a vulnerable state, as described in the Saw of the Walsings, chap. TODO: . See also \inx[P]{feather-hame}, \inx[C]{town-riders}, \inx[C]{evening-riders}.

\inxitem[C]{harrow} (ON. \emph{hǫrgr}, OE. \emph{hearg}, PNWGmc. \emph{*harugaʀ})
  A cairn constructed for ritual purposes. \emph{Hind} 10 describes one: “A \inx[C]{harrow} he made for me, loaded with stones; now that stone-pile is become into glass. He reddened [it] in fresh blood of oxen; \inx[P]{Oughthere} ever trusted on the \inx[G]{Ossens}.” See also \inx[C]{wigh}.

\inxitem[C]{Home} (ON. \emph{hęimr}, OE. \emph{hám}, PNWGmc. \emph{*haimaʀ})
  .In the Norse often referring to a realm in the cosmology (\Voluspa 2: “I remember nine \textbf{Homes}”, \Vafthrudnismal\ TODO: “From the runes of the \inx[G]{Ettins} and of all the gods I can speak truly, for I have come into each \textbf{Home}”). Thus \inx[L]{Ettinham} is the ‘\textbf{Home}/realm of the ettins,’ and when used alone the term simply means ‘the world (that we inhabit)’ See also \inx[L]{nine Homes}, \inx[L]{Thrithham}.

\inxitem[C]{leat} (ON. \emph{hlaut})
  Sacrificial blood (that is, taken from the animal), especially when used for auguries.

\inxitem[C]{leat-twig} (ON. \emph{hlauttęinn})
  A twig used to sprinkle the \inx[C]{leat} in auguries (presumably the pattern of the blood would then be inspected).

\inxitem[C]{leed} (ON. \emph{ljóð}, OE. \emph{léod})
  A magical chant or incantation. See also \inx[C]{galder}, \inx[C]{gale}, \inx[C]{begale}.

\inxitem[C]{manwit} (ON. \emph{manvit})
  Practical sense and wisdom, situational awareness, ‘common sense’.

\inxitem[C]{orlay} (ON. \emph{ørlǫg}, OE. \emph{orlæg})
  One’s predetermined fate, destiny, purpose as decreed by the \inx[G]{Norns}.

\inxitem[C]{rest} (ON. \emph{rǫst})
  The distance between two rest-stops, a geographical mile (about 1850 metres). See especially \CV.

\inxitem[C]{rune} (ON. \emph{rún}, OE. \emph{rún}, OS. \emph{rúna}, OHG. \emph{rúna}, Got. \emph{rúna}, PNWGmc. \emph{rūnu})
  An (esoteric) secret message or formula. That this—rather than ‘letter (of a Runic alphabet)’—is the original and proper sense is apparent from among others the Finnish borrowing \emph{runo} ‘poem; poetry; a division of a poem (specifically of the \emph{Kalevala})’, and its use in the singular in the earliest Runic inscriptions such as Noleby Vg 63 (which contains the linguistically indecipherable string of letters {ᚢᚾᚨᚦᛟᚢᛊᚢᚺᚢᚱᚨᚺᛊᚢᛊᛁᚺ[--]ᚨᛁ\rotatebox[origin=c]{180}{ᛏ}ᛁᚾ}, a \emph{rune} in the proper sense) or the recently discovered Svingerud fragment. Thus, Weden’s taking of the \emph{runes} should not be interpreted as merely a myth for the invention of profane writing, but rather the origin of esoteric incantations, not at all unlike Indian \emph{mantras}.
  The word for letter was instead \inx[C]{stave}, see also there.

\inxitem[C]{soo} (ON. \emph{sóa})
  To ritually waste, the slaying in the animal sacrifice.

\inxitem[C]{thill} (ON. \emph{þylja})
  To chant poetry or lists (so called \inx[C]{thule}[thules]) acquired by rote memorization. See also \inx[C]{thyle}.

\inxitem[C]{Thing} (ON., OE. \emph{þing}, OS. \emph{thing}, OHG. \emph{ding})
  The legal assembly and gathering place where matters would be settled and the law recited.

\inxitem[C]{thule} (ON. \emph{þula})
  A poetic list, typically of various items of a category (e.g. gods, legendary horses) or poetic synonyms (e.g. for swords, men, Weden). Degoratively also a ditty, poorly composed poem.

\inxitem[C]{thyle} (ON. \emph{þulr}, OE. \emph{þyle}, PNWGmc. \emph{*þuliʀ})
  A sage who through rote learning has acquired a large amount of mythological lore (cf. \emph{þula} 'a list in poetic form; a meaningless poem' and \emph{þylja} 'to recite, to chant'). Thus \inx[P]{Weden} is the \inx[P]{Fimblethyle}, being the unbeaten master of lore, as can be seen in his wisdom contests (like \Vafthrudnismal). Runic inscription DR 248 (Snoldelev) suggests the thyle may have tied to a specific place, and in \Beowulf\ it seems to have been a court position, with the poet Unferth being described as the "thyle of Rothgar".

\inxitem[C]{wale} (ON. \emph{vǫlr})
  The staff or sceptre, especially of a wallow. TODO: archeological finds, mention Sutton Hoo.

\inxitem[C]{wallow} (ON. \emph{vǫlva}, OE. \emph{*wealwe} (cf. ON. \emph{svǫlva}, OE. \emph{swealwe} ‘swallow’))
  A sibyl, seeress, oracle. The word derives from the \inx[C]{wale}, a staff or sceptre probably used for ritual purposes.

\inxitem[C]{wigh} (ON. \emph{vé}, OE. \emph{wéoh}, \emph{wíh}, PNWGmc. \emph{*wīhą})
  A holy shrine or sanctuary. It seems that where the \inx[C]{harrow} was a pile of stones or cairn used for carrying out rituals, the \textbf{wigh} was an enclosed space. The earliest Norse attestation is the runic inscription Ög N288 (Oklunda), which reads: “Guthhere <= Gunnarr> painted these runes, and he fled, guilty. Sought this wigh, and he fled into this clearing. And he bound. [...]” The implication seems to be that the wigh was considered so sacred that Guthhere could not be apprehended or punished for his crime while in it. — In Old English the word means ‘pagan idol’. It is not immediately clear which meaning is the original one, but in this edition the Norse sense has been adopted, since the Anglo-Saxon sources are all of a Christian nature. The \Beowulf\ name \emph{Wighstone} (\emph{Wīh-} or \emph{Wēohstān}) in any case suggests it is the Norse meaning, since ‘idol-stone’ makes little sense.

\inxitem[C]{wode} (ON. \emph{óðr}, OE. \emph{wód}, PNWGmc. \emph{*wōþuʀ})
  \inx[P]{Heener}'s gift to men, though the name would suggest it be from \inx[P]{Weden}. The word has several related meanings: ‘poetic inspiration’, ‘madness’, ‘rage’.

\end{itemize}


\section{People and objects (P)}

\begin{itemize}

\inxitem[P]{Attle} (\emph{Attila}, ON. \emph{Atli}, OE. \emph{Ætla}, MHG. \emph{Etzel}, PNWGmc. \emph{*Attilō})
  The ruler of the \inx[G]{Huns} (historically from 434–453). Husband of \inx[P]{Guthrun}, and with her father of \inx[P]{Earp and Oatle}. and murderer of
  I HHb 54, SiL 11, I Gr 23, ShS 28, 29, 33, 37, 54, 56, 57, II Gr 26, 38, 45, III Gr 1, 9, BnOr 0, OdW A, 2, 22, 23, 25, 26, 30, 31, AtD 0, AtL 1, 3, 15, 17, 18, 27, 31, 32, 34, 36, 37, 38, 41, 43, B, AtS 2, 4, 21, 22, 44, 52, 60, 64, 71, 73, 77, 80, 86, 87, 97, 98, 108, 113, 117, FGr 0, GrB 12, Ham 6.

\inxitem[P]{Balder} (ON. \emph{Baldr}, OE. \emph{Bældæg} (not directly cognate), OHG. \emph{Balter}, PWGmc. \emph{Baldraʀ})
  The beautiful son of \inx[P]{Weden}, slayed by his brother \inx[P]{Hath}, avenged by his other brother \inx[P]{Wonnel}.

\inxitem[P]{Earp and Oatle} (ON. \emph{Erpr ok Eitill})
  The sons of \inx[P]{Attle} and \inx[P]{Guthrun}.

\inxitem[P]{Feather-hame} (ON. \emph{fjaðrhamr})
  A \inx[C]{hame} owned by the Ease that lets the wearer fly like a bird, more specifically a falcon.

\inxitem[P]{Guthrun} (ON. \emph{Guðrún})
  Daughter of king \inx[P]{Yivick}, sister of \inx[P]{Guthhere} and \inx[P]{Hain}. The wife of \inx[P]{Attle}.

\inxitem[P]{Hain}[Hain 1] (ON. \emph{Hǫgni}, OE. \emph{Haguna}, \emph{Hagena}, OHG. \emph{Hagano}, Ger. \emph{Hagen}, PNWGmc. \emph{*Hagunō})
  A \inx[G]{Niflings}[Nifling] and \inx[G]{Yivickings}[Yivicking], son of king \inx[P]{Yivick}, brother of \inx[P]{Guthhere} and \inx[P]{Guthrun}. In \emph{AtL} he defeats seven warriors before being captured by \inx[P]{Attle}, who has his heart cut out at the request of Guthhere.

\inxitem[P]{2}
  A petty king of \inx[L]{East Geatland}, contemporary with \inx[P]{Granmer}, the king of \inx[L]{Southmanland} and Ingeld Illred, the \inx[G]{Inglings}[Ingling] king of \inx[L]{Upland}.

\inxitem[P]{Hath} (ON. \emph{Hǫðr})
  The blind son of \inx[P]{Weden}, the slayer of his brother \inx[P]{Balder}.

\inxitem[P]{Hindle} (ON. \emph{Hyndla}) A witch awoken by Frow in \emph{Hind}.

\inxitem[P]{Millner} (ON. \emph{Mjǫllnir}, OE. \emph{*Meldne}, PNWGmc. \emph{*Meldunjaʀ})
  Powerful hammer owned by Thunder.

\inxitem[P]{Oughter} (ON. \emph{Óttarr}, OE. \emph{Óhthere}, PNWGmc. \emph{*Ōhtaharjaʀ})
  TODO

\inxitem[P]{Rotholf} (ON. \emph{Hrólfr kraki}, OE. \emph{Hróþulf}, PNWGmc. \emph{*Hrōþiwulfaʀ})
  A king of the \inx[G]{Shieldings} (see family tree). As foreshadowed in \Beowulf\ 1017–9, 1180–90, he betrays the sons of \inx[P]{Rothgar}, his cousins \inx[P]{Rethrich and Rothmund}, in order to take the throne for himself. In the later Icelandic tradition this has been forgotten, and he is consistently portrayed as a heroic king.

\inxitem[P]{Rothgar} (ON. \emph{Hróarr}, OE. \emph{Hróþgár}, PNWGmc. \emph{*Hrōþigaiʀaʀ})
  A king of the \inx[G]{Shieldings} (see family tree), one of the main characters in \Beowulf.

\inxitem[P]{Weden} (rhymes with \emph{leaden}; ON. \emph{Óðinn}, OE. \emph{Wóden}, \emph{Wéden}, OHG. \emph{Wuotan}, PNWGmc. \emph{*Wōdanaʀ})
  Chief of the \inx[G]{Ease}, his name is clearly related to \inx[C]{wode}, referring to his role as the patron of \inx[C]{scold}[scolds] and \inx[C]{bearserk}[bearserks]. Husband of \inx[P]{Frie}, and by her father of \inx[P]{Balder}. Also father of \inx[P]{Thunder} by \inx[P]{Earth}. Brother of \inx[P]{Heener} and \inx[P]{Lother}.

\inxitem[P]{Wider} (ON. \emph{Víðarr})
  A son of \inx[P]{Weden}, who avenges him at the \inx[L]{Rakes of the Reins}.

\inxitem[P]{Wonnel} (ON. \emph{Váli}, PWgmc. \emph{Wanila} ‘the little \inx[P]{Wanes}[Wane] (uncertain)’)
  The son of \inx[P]{Weden}, who one-night old avenged his brother \inx[P]{Balder} through slaying another brother, \inx[P]{Hath}.

\inxitem[P]{Woulder} (ON. \emph{Ullr})
  A rather obscure god. He is mentioned in connection with oath-rings (TODO) and the setting of ritual fires (\Grimnismal\ TODO). These obscure references are likely related to the interesting finds at Lilla Ullevi (‘the small \inx[C]{wigh} of Woulder’) in Upland, Sweden, consisting of several dozen fire striker-shaped iron amulet rings dating to 660–780 (for a detailed description see af Edholm 2009).

\inxitem[P]{Yimer} (ON. \emph{Ymir}, OE. \emph{*Yime})
  The first ettin, probably equivalent to \inx[P]{Earyelmer}.

\inxitem[P]{Yivick} (ON. \emph{Gjúki}, OE. \emph{Gifica}, OHG. \emph{Gibicho}, MHG. \emph{Gibeche})
  King of the \inx[G]{Burgends} (historically from late 300s–407) of the Nifling dynasty, ancestor of the \inx[G]{Yivickings}. Father of \inx[P]{Guthrun}, \inx[P]{Guthhere} and \inx[P]{Hain}.

\end{itemize}


\section{Groups and tribes (G)}

TODO: Map of rough tribal areas. Geneaologies.

\begin{itemize}

\inxitem[G]{Danes} (ON. \emph{danir}, OE. \emph{dene}, PNWGmc. \emph{*daníʀ})
  A tribe in eastern modern-day Denmark and southern Sweden. They probably originated in Scania in southern Sweden, before moving westwards into the Danish isles and eventually Jutland, driving out the \inx[G]{Earls} and \inx[G]{Jutes}.
  Noted members: TODO
  Attestations: TODO

\inxitem[G]{Dwarfs} (ON. \emph{dvergar}, OE. \emph{dweorgas}, OHG. \emph{twerca}, PNWGmc. \emph{*dwergóʀ})
  Earthly (chthonic) supernatural beings, often referred to as living in rocks and mountains.
  Noted members: TODO
  Attestations: TODO

\inxitem[G]{Ease} (rhyming with \emph{geese}; ON. \emph{ę́sir}, OE. \emph{ése}, PNWGmc. \emph{*ansiwiʀ}; sg. \emph{os}, ON. \emph{áss}, OE. \emph{ós}, PNWGmc. \emph{*ansuʀ})
  A group of Gods, though the word can also refer to all the Gods. See \inx[G]{Gods}, \inx[G]{Tues}, \inx[G]{Wanes}, \inx[G]{Reins}.
  Noted members: \inx[P]{Weden}, \inx[P]{Thunder}, \inx[P]{Frie}, \inx[P]{Hath} and \inx[P]{Balder}
  Attestations: TODO

\inxitem[G]{Ease and Elves} (ON. \emph{ę́sir ok alfar}, OE. \emph{ése ende ielfe}, PNWGmc. \emph{*alβíʀ jah ansiwiʀ})
  A merism; both heavenly and earthly spiritual beings. Notably the two words always occur in this order (never ‘Elves and Ease’). This applies even to the Old English.

\inxitem[G]{Elves} (ON. \emph{alfar}, OE. \emph{ielfe}, PNWGmc. \emph{*alβíʀ})
  Earthly (chthonic) supernatural beings. Possibly ancestral spirits?
  Noted members: TODO
  Attestations: TODO

\inxitem[G]{Ettins} (ON. \emph{jǫtnar}, OE. \emph{eotenas}, PNWGmc. \emph{*etunóʀ})
  The fundamental enemies of the Gods, the agents of chaos and disorder. See \inx[G]{Rises}, \inx[G]{Thurses}.
  Noted members: \inx[P]{Hymer}, \inx[P]{Thrim}, \inx[P]{Webthrithner}, \inx[P]{Yimer}
  Attestations: TODO

\inxitem[G]{Geats} (ON. \emph{gautar}, OE. \emph{géatas}, PNWGmc. \emph{*gautóʀ} from \emph{*geut-} ‘to pour’, perhaps ‘the libators’)
  A tribe in what is today southern-central Sweden. See also \inx[L]{Geatland}, \inx[G]{Swedes}.
  Noted members: TODO
  Attestations: TODO

\inxitem[G]{gin-Reins} (ON. \emph{ginnręgin})
  \inx[C]{gin-} + \inx[G]{Reins}. The sacrosanct, highest divine powers.

\inxitem[G]{Gods} (ON. \emph{goð}, OE. \emph{godu}, OHG. \emph{gota}, PNWGmc. \emph{*godu})
  TODO.
  Noted members: TODO
  Attestations: TODO

\inxitem[G]{Huns} (ON. \emph{húnir}, OE. \emph{Húne}, OHG. \emph{Húni}, \emph{Hunni}, PNWGmc. \emph{*húníʀ})
  An invading Asiatic tribe in the Migration Period. In the legendary material their cultural and ethnic foreignness is not seen.
  Noted members: TODO
  Attestations: TODO

\inxitem[G]{Nears} (ON. \emph{níarar} ~ \emph{njárar})
  A Swedish tribe, only mentioned in \Volundarkvida, where it is ruled by king \inx[P]{Nithad}. The name and location may allow us to connect them with the Swedish province of Närke, cf. Old Swedish: \emph{Nærikiar} ‘inhabitants of Närke’, \emph{Nærisker} ‘belonging to Närke; Nearish’. The Old Swedish stem \emph{nær-} would then be a reduced form of \emph{níar-}, \emph{njár-}.

\inxitem[G]{Ossens} (ON. \emph{ǫ́synjur})
  The women of the \inx[G]{Ease}, see there.

\inxitem[G]{Reins} (ON. \emph{rǫgn}, \emph{ręgin})
  The divine powers. Based on \Vafthrudnismal\ (TODO) the term may be more closely associated with the \inx[G]{Wanes} than the \inx[G]{Ease}.

\inxitem[G]{Saxons} (ON. \emph{saxar}, OE. \emph{Seaxan}, \emph{Seaxe})
  TODO.
  Noted members: TODO
  Attestations: TODO

\inxitem[G]{Shieldings} (ON. \emph{skjǫldungar}, OE. \emph{Scyldingas}, PNWGmc. \emph{*skeldungóʀ})
  The descendants of \inx[P]{Shield}; the legendary \inx[G]{Danes}[Danish] royal dynasty. With \inx[P]{Harward}'s death after his slaying of \inx[P]{Rotholf} their rule ended. TODO
  Noted members: TODO
  Attestations: TODO

\inxitem[G]{Shelvings} (ON. \emph{skilfingar}, OE. \emph{scilfingas}, PNWGmc. \emph{*skilβingóʀ})
  The descendants of \inx[P]{Shelf}; the legendary \inx[G]{Swedes}[Swedish] royal dynasty. The exact difference between the terms Shelvings and \inx[G]{Inglings} is unclear, but the first may have referred to the old royal family in Sweden, while the latter to the Norwegian branch which claimed descent from the former. TODO
  Noted members: TODO
  Attestations: \Hyndluljod\ 15, 20

\inxitem[G]{Swedes} (ON. \emph{svíar}, OE. \emph{swéon}, PNWGmc. \emph{*swihaníʀ})
  The tribe around the Mälar valley in eastern Sweden.
  Noted members: TODO
  Attestations: TODO

\inxitem[G]{Thurses} (sg. Thurse; ON. \emph{þurs}, OE. \emph{þyrs}, OS. \emph{thuris}, OHG. \emph{duris}, PNWGmc. \emph{*þurisaʀ})
  Possibly a poetic synonym for \inx[G]{Ettins}. See also \inx[G]{Rime-Thurses}.
  Noted members: TODO
  Attestations: Wal 8, Shr 31, 35, 36, Hyme 17, Thr 5, 10, 21, 24, 29, 30, Alw 2, I HHb 40, HHw 27.

\inxitem[G]{Tues} (ON. \emph{tívar}, PNWGmc. \emph{*tíwóʀ})
  A poetic synonym for \inx[G]{Gods}.
  Attestations: TODO

\inxitem[G]{Wanes} (ON. \emph{vanir}, OE. \emph{wan-?})
  A subgroup or tribe of the gods, associated with fertility, harvests and fishing.
  Noted members: \inx[P]{Nearth}, \inx[P]{Ing}, \inx[P]{Frow}
  Attestations: TODO

\inxitem[G]{Yivickings} (ON. \emph{gjúkungar})
  The descendants of \inx[P]{Yivick}, including \inx[P]{Guthhere}, \inx[P]{Guthrun} and \inx[P]{Hain}.
  Attestations: TODO

\end{itemize}


\section{Place names, locations and events (L)}

\begin{itemize}

\inxitem[L]{Ettinham} (ON. \emph{Jǫtunhęimr}, \emph{Jǫtnahęimr})
  The ‘\inx[G]{Ettin}-\inx[C]{Home}’ or ‘home of the Ettins’; the eastern realm of chaotic and inhospitable beings. See also \inx[L]{Eastway}, \inx[L]{Outyard}.

\inxitem[L]{Hell} (ON. \emph{hęl}, PNWGmc. \emph{*halju}, Got. \emph{halja})
  The underworld, personfied as and formally identical with \inx[P]{Hell}. After Christianity the word came to refer to the Christian hell (= Gehenna), as is the case in all attested languages apart from the Old Norse. See also \inx[L]{Nivelhell}.

\inxitem[L]{Middenyard} (ON. \emph{Miðgarðr}, OE. \emph{Middangeard}, OS. \emph{Middilgard}, OHG. \emph{Mittilgart}, Got. \emph{midjungards})
  The ‘middle enclosure’; the realm of men. See also \inx[L]{Osyard}, \inx[L]{Outyard}.

\inxitem[L]{Nivelhell} (ON. \emph{niflhęl})
  ‘Mist-Hell’, from the poetic evidence it seems like it may originally have been a synonym for \inx[L]{Hell}. In poetry it is attested in \Vafthrudnismal\ TODO: \emph{níu kom’k hęima |hld\ fyr Niflhel neðan, \\ hinig deyja ór helju halir. } ‘into nine homes I came, beneath Nivelhell; thither die men out of Hell’, the second by \Baldrsdraumar\ 2: \emph{ręið niðr þaðan |hld\ niflhęljar til; \\ mǿtti hvelpi, |hld\ þęim’s ór hęlju kom.} ‘[Weden] rode down thence to Nivel-hell; met the whelp that out of Hell came.’ Possibly the distinction was held by the first poet but not the second.

\inxitem[L]{Osyard} (ON. \emph{Ásgarðr})
  The ‘enclosure of the \inx[G]{Ease}’; the heavenly realm. See also \inx[L]{Middenyard}, \inx[L]{Outyard}.

\inxitem[L]{Outyards} (ON. \emph{Útgarðar})
  Not eddic. The ‘outer enclosures’, described in \Gylfaginning. See also \inx[L]{Ettinham}, \inx[L]{Middenyard}, \inx[L]{Osyard}.

\inxitem[L]{Rakes of the Reins} (ON. \emph{ragna rǫk})
  The ‘fates of the \inx[G]{Reins}’, euphemism for the destruction of the world.

\inxitem[L]{Rakes of the Tues} (ON. \emph{tíva rǫk})
  The \inx[L]{Rakes of the Reins}.

\inxitem[L]{Up-heaven} (ON. \emph{Upphiminn}, OE. \emph{Upheofon}, OS. \emph{Upphimil}, OHG. \emph{úfhimil})
  Highest heaven. See also \inx[L]{Earth and Up-heaven}.

\end{itemize}
