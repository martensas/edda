\part{Encyclopedia (INCOMPLETE!)}

NOTE: This encyclopedia is both incomplete and inconsistently formatted. New entries will be added, and old ones be corrected and expanded in the future.

\section{Cultural and religious expressions (C)}
\begin{itemize}

\inxitem[C]{ape} (ON \emph{api}, OE \emph{apa}, OS \emph{apo}, OHG \emph{affo}, PNWGmc. \emph{*apó})
  In the Old Norse the word seems to mean ‘fool, buffoon’, in the other old languages apparently ‘monkey’, though this sense should be a later development of the former; why would the early Germanic tribes have a word for an animal that they had never encountered?

\inxitem[C]{aught} (ON \emph{ę́tt}, OE \emph{ǽht} ‘possession, property’)
  The Nordic (paternal) clan or family line.

\inxitem[C]{begale} (OHG \emph{bigalan})
  To affect, bewitch something using \inx[C]{galder}[galders]. See also \inx[C]{gale}.

\inxitem[C]{bigh} (ON \emph{baugr}, OE \emph{béag}, OHG \emph{boug})
  Armlets used as currency during the Migration Period. — The giving of rings and armlets in exchange for loyalty (\inx[C]{holdness} being the word used for a warrior’s loyalty towards his lord, and of a lord’s grace towards his servants) was common across all of Germanic Europe, as seen in the many poetic ruler-kennings of the type “breaker of rings” (e.g. \emph{béaga brytta} ‘the breaker of bighs’ in \Beowulf\ ll. 35, 352, 1487). An illustrative example of this is \Hildebrandslied\ 33–35.
  This is also connected with the oath-ring, and the famous ring-swords. TODO? reference some literature on this.

\inxitem[C]{bloot} (ON \emph{blót}, OE \emph{blót}, OHG \emph{bluoz})
  A sacrifice or a sacrificial feast, one of the best attested Germanic pagan practices. The animals would be sacrificed by the host, cooked in large kettles and eaten communally.

\inxitem[C]{bloot-kettle}
  The large pots used for cooking the bloot-stew.

\inxitem[C]{Doom} (ON \emph{dómr}, OE \emph{dóm})
  Commonly ‘judgement, verdict’ (whence Doomsday, ‘Judgement Day’), in the Norse and Anglo-Saxon poetry often specifically referring to one’s fame or good reputation (that is, how others will judge one’s character and deeds), especially after death. It is clear that this verdict was of utmost importance to the ancient Germanic people. The clearest examples are \Havamal\ 77 (see there): \emph{I know one that never dies: the \textbf{Doom} o’er each man dead.} and \Beowulf\ 1384-1389, where Beewolf consols king Rothgar after Grendle’s mother has slain his trusted advisor Asher (\emph{Æschere}):
  \bvg\bva[] \emph{Ne sorga, snotor guma! \hld\ Sélre bið ǽghwǽm, //
  þæt hé his fréond wrece, \hld\ þonne hé fela murne. //
  Úre ǽghwylc sceal \hld\ ende gebídan //
  worolde lífes; \hld\ wyrce sé þe móte //
  \textbf{dómes} ǽr déaþe; \hld\ þæt bið drihtguman //
  unlifgendum \hld\ æfter sélest.}\eva
  \bvb ‘Sorrow not, wise man! ’Tis better for each one that he avenge his friend, than that he mourn much. Each one of us shall suffer the end of worldly life—win he who might \textbf{Doom} before death: that is for the warrior, unliving, afterwards the best.’\evb\evg
  Other illustrative examples in \Beowulf\ include 884b–887a: \emph{[...] Sigemunde gesprong // æfter déaðdæge \hld\ \textbf{dóm} unlýtel // syþðan wíges heard \hld\ wyrm ácwealde // hordes hyrde [...]} ‘For \inx[P]{Sighmund} sprang up after his day of death an unlittle \ken*{= great} \textbf{Doom}, since hard in conflict he defeated the \inx[C]{Worm}, the herder of the hoard.’
  and 953b–955a: \emph{[...] þú þé self hafast // dę́dum gefremed \hld\ þæt þín \textbf{dóm} lyfað // áwa tó aldre [...]} ‘Thou hast for thyself by deeds accomplished that thy \textbf{Doom} lives for ever and ever.’

\inxitem[C]{fee} (ON \emph{fé}, OE \emph{féoh})
  Originally ‘cattle’, however also used in a broader sense to refer to one’s mobile wealth. For this cf. particularly \Havamal\ TODO.

\inxitem[C]{many-cunning} (ON \emph{fjǫlkunnigr})
  Literally ‘much-cunning, cunning in many ways’. Skilled with sorcery.

\inxitem[C]{fey} (ON \emph{fęigr}, OE \emph{fǽge}, OHG \emph{feigi} ‘cowardly’)
  Being doomed or fated to die, with a sense of predestination and inevitability. Its earliest use is on the Rök stone: \textbf{aft uamuþ stąnta runaʀ þaʀ ᛭ n uarin faþi faþiʀ aft} faikiąn \textbf{sunu} \emph{Apt Vámóð standa rúnaʀ þáʀ, en Varinn fáði, faðir aft \textbf{fęigjan} sonu} ‘After Woemood (\emph{Vámóðr}) stand these \inx[C]{rune}[runes], but Warren (\emph{Varinn}) painted, the father after the \textbf{fey} son.’ It was believed that one’s TODO. See \textciteshorttitle{PCRN-HS} II:35, p. 928 ff. (TODO)

\inxitem[C]{feyness} (ON \emph{fęigð})
  The state of being \inx[C]{fey}.

\inxitem[C]{fimble-} (ON \emph{fimbul-})
  The ultimate, final, greatest. See \inx[P]{Fimblethyle}, \inx[L]{Fimble-winter}.

\inxitem[C]{five days} (ON \emph{fimm dagar})
  That the old Scandinavian week was \textbf{five days} long is well attested. According to the \Gulatingslog\ there were six weeks in a month, and the expression \textbf{five days} is used as the equivalent of \emph{week} in \Havamal\ 51 and 74, in the second of which it is contrasted with \emph{month}. Related to this is the legal term \emph{fifth} (ON \emph{fimmt}, OSw. \emph{fæmt}), a meeting or gathering set to be held at a five-day notice. See \emph{fimt} in \CV, \textcite{LMNL} for further discussion.

\inxitem[C]{galder} (ON \emph{galdr}, OE \emph{gealdor}, OHG \emph{galdar})
  A magical spell or song. See the Merseburg charms (TODO?) for examples. See also \inx[C]{gale}.

\inxitem[C]{gale} (ON \emph{gala}, OE \emph{galan}, OHG \emph{galan})
  To sing \inx[C]{galder}[galders].

\inxitem[C]{gand} (ON \emph{gandr}, Latin \emph{gandus})
  A witch’s familiar, a spirit sent out to do her bidding. See \textciteshorttitle{PCRN-HS} I:17, p. 361 and II:26, p. 656. TODO

\inxitem[C]{gid} (ON \emph{goði}, OE \emph{Gydda} masc. nom. prop.)
  A heathen priest or master of ceremonies.

\inxitem[C]{gidden} (ON \emph{gyðja}, OE \emph{gyden} ‘goddess’)
  The feminine equivalent of \inx[C]{gid}.

\inxitem[C]{gin-} (ON \emph{ginn-})
  A rare augmentative prefix. TODO.

\inxitem[C]{gin-holy} (ON \emph{ginnhęilagr})
  Sacrosanct, highest holy.

\inxitem[C]{good of meat} (ON \emph{matar góðr})
   An old expression, appearing not just in \Havamal\ 39 (“I found not a generous man, or so \textbf{good of meat}, that a gift were not accepted;”) but also several Viking Age Runic inscriptions, such as Sm 39: \emph{mildan orða \hld\ ok mataʀ góðan} ‘mild of words and \textbf{good of meat}’, U 805: \emph{bónda góðan matar} ‘a farmer \textbf{good of meat}’, U 703: \emph{mandr matar góðr \hld\ auk máls risinn} ‘a man \textbf{good of meat} and proud in speech™; compare also U 739: \emph{hann vaʀ mildr mataʀ \hld\ auk máls risinn} ‘he was \textbf{mild of meat} and proud in speech’. — See \inx[C]{meat-nithing} for its opposite.

\inxitem[C]{hame} (ON \emph{hamr})
  A skin, shape. Individuals can through magic “shift hames” (ON \emph{skipta hǫmum}), and leave their human \emph{hames} behind, instead entering into the shapes of wolves, bears, birds. During this process the original hame would be sleeping in a vulnerable state, as described in the Saw of the Walsings, chap. TODO: . See also \inx[P]{feather-hame}, \inx[C]{town-riders}, \inx[C]{evening-riders}.

\inxitem[C]{harrow} (ON \emph{hǫrgr}, OE \emph{hearg}, PNWGmc. \emph{*harugaʀ})
  A cairn constructed for ritual purposes. \Hyndluljod\ 10 describes one: “A \inx[C]{harrow} he made for me, loaded with stones; now that stone-pile is become into glass. He reddened [it] in fresh blood of oxen; \inx[P]{Oughter} ever trusted on the \inx[G]{Ossens}.” See also \inx[C]{wigh}.

\inxitem[C]{hold} (ON \emph{hollr}, OE \emph{hold}, OS \emph{hold}, OHG \emph{hold})
  %TODO Mention: unhold wights, Old Saxon baptismal formula.
  ‘Favourable, loyal, gracious’, often of a ruler towards his subject (in the sense of ‘gracious, benevolent’) or the reverse (in the sense of ‘loyal, devoted’). Mirroring these earthly relations, it is likewise often used to refer to divine grace, both of the Christian God—thus in the \emph{Ecclesiastical Laws of King Cnut} \textciteshorttitle[372]{ALIE1}: \emph{Ðam byþ witodlíce God hold þe bið his hláforde rihtlíce hold} ‘Indeed God is \textbf{hold} towards him who is rightly \textbf{hold} towards his lord’—but in the oldest Scandinavian material likewise of the Heathen gods.
  Thus \Lokasenna\ 4: \emph{holl ręgin} ‘\textbf{hold} \inx[G]{Reins}’, and \Oddrunargratr\ 10 (TODO: Numbering is very uncertain): \\ \emph{Svá hjalpi þér \hld\ hollar véttir, \\ Frigg ok Fręyja \hld\ ok flęiri goð} \\ ‘So help thee \textbf{hold} \inx[C]{wights}; \inx[P]{Frie} and \inx[P]{Frow}, and more gods [...]’.

  The word is also used in this way several medieval oath-formulæ, for instance in the Elder West-Geatish Law: \emph{Svá sé mér goð holl} ‘So may the gods(!) be \textbf{hold} towards me,’ in medieval Norwegian laws (\textciteshorttitle{NGL2}[197,397]) and Grey-Goose (TODO: cite): \emph{Guð sé mér hollr ef ek satt segi, gramr ef ek lýg} ‘God be \textbf{hold} towards me if I speak truly, wroth if I lie,’ in Grey-Goose (TODO) also: \emph{Sé guð hollr þeim er heldr griðum, en gramr þeim er grið rýfr} ‘God be \textbf{hold} towards him who keeps the truce, but wroth against him who breaks the truce’. I refer to \textcite{Läffler1895} for further discussion on these formulæ.

  \inxitem[C]{holdness} Closely connected to this is of course the abstract noun \textbf{holdness} (ON \emph{hylli}, OE \emph{hyldu}, OHG \emph{huldí}) ‘favour, loyalty, grace,’ with the same semantics as the adjective.

  Notably, this word appears three times in connection with the grace of gods in the poetry, namely in \Grimnismal\ 43, where (according to my interpretation) the preparer of food at the bloot is said to earn the “\textbf{holdness} of \inx[P]{Woulder} and of all the gods;” and \Grimnismal\ 53 where the disgraced king Garfrith is said to have been bereft of “my [= Weden’s] support; of all the Ownharriers (see note to the v.), and of Weden’s \textbf{holdness}”. Weden’s holdness (\emph{Óðins hylli}; the phrase is identical in both occurences) is also mentioned in a stanza by Hallfred (edited as Hfr Lv 7 by Diana Whaley in \Skp\ V) where the scold states that: ‘The whole race of man has wrought songs to win the \textbf{holdness} of Weden; I recall the fully rewarded works of our kinsmen/ancestors.’

  From all these citations the Germanic view on divine favour is clear: the gods are \textbf{hold} towards those who do good works, which in the aforementioned instances include swearing true oaths, faithfully observing truces, partaking in the bloot, following rules of hospitality and composing poetry—and \inx[C]{gram} ‘wroth’ towards those who do the opposite.

\inxitem[C]{Home} (ON \emph{hęimr}, OE \emph{hám}, PNWGmc. \emph{*haimaʀ})
  In the Norse often referring to a realm in the cosmology (\Voluspa\ 2: “I remember nine \textbf{Homes}”, \Vafthrudnismal\ TODO: “From the runes of the \inx[G]{Ettins} and of all the gods I can speak truly, for I have come into each \textbf{Home}”). Thus \inx[L]{Ettinham} is the ‘\textbf{Home}/realm of the ettins’. When used alone the term simply means ‘the world (that we inhabit)’. See also \inx[L]{nine Homes}, \inx[L]{Thrithham}.

\inxitem[C]{leat} (ON \emph{hlaut})
  Sacrificial blood (that is, taken from the animal), especially when used for auguries.

\inxitem[C]{leat-twig} (ON \emph{hlauttęinn})
  A twig used to sprinkle the \inx[C]{leat} in auguries (presumably the pattern of the blood would then be inspected).

\inxitem[C]{leed} (ON \emph{ljóð}, OE \emph{léod})
  A magical chant or incantation. See also \inx[C]{galder}, \inx[C]{gale}, \inx[C]{begale}.

\inxitem[C]{manwit} (ON \emph{manvit})
  Practical/common sense and wisdom, situational awareness.

\inxitem[C]{nithe} (ON \emph{níð}, OE \emph{níþ}, OHG \emph{níd})
  Originally probably ‘hatred, emnity’, in the Norse a sort of ritual libel that brought great dishonor.

\inxitem[C]{orlay} (ON \emph{ørlǫg}, OE \emph{orlæg})
  One’s predetermined fate, destiny, purpose as decreed by the \inx[G]{Norns}.

\inxitem[C]{rest} (ON \emph{rǫst})
  The distance between two rest-stops, a geographical mile (about 1850 metres). See especially \CV.

\inxitem[C]{rune} (ON \emph{rún}, OE \emph{rún}, OS \emph{rúna}, OHG \emph{rúna}, Got. \emph{rúna}, PNWGmc. \emph{rūnu})
  An (esoteric) secret message or formula. That this—rather than ‘letter (of a Runic alphabet)’—is the original and proper sense is apparent from among others the Finnish borrowing \emph{runo} ‘poem; poetry; a division of a poem (specifically of the \emph{Kalevala})’, and its use in the singular in the earliest Runic inscriptions (e.g. Noleby Vg 63, which contains the linguistically indecipherable string of letters {ᚢᚾᚨᚦᛟᚢᛊᚢᚺᚢᚱᚨᚺᛊᚢᛊᛁᚺ[--]ᚨᛁ\rotatebox[origin=c]{180}{ᛏ}ᛁᚾ}, a \emph{rune} in the proper sense or the recently discovered Svingerud fragment.) Thus, Weden’s taking of the \emph{runes} should not be interpreted as merely a myth for the invention of profane writing, but rather the origin of esoteric incantations, not at all unlike Indian \emph{mantras}.
  The word for letter was instead \inx[C]{stave}, see also there.

\inxitem[C]{scold} (ON \emph{skald})
  A Scandinavian poet. The name probably comes from their ability to slander with words.

\inxitem[C]{simble} (ON \emph{sumbl}, OE \emph{symbol})
  A banquet.

\inxitem[C]{soo} (ON \emph{sóa})
  To ritually waste, to slay (especially in a sacrificial context).

\inxitem[C]{thill} (ON \emph{þylja})
  To chant poetry or lists (so called \inx[C]{thule}[thules]) acquired by rote memorization. See \inx[C]{thyle}.

\inxitem[C]{Thing} (ON, OE \emph{þing}, OS \emph{thing}, OHG \emph{ding})
  The legal assembly and gathering place where matters would be settled and the law recited.

\inxitem[C]{thule} (ON \emph{þula})
  A poetic list, typically of various items of a category (e.g. gods, legendary horses) or poetic synonyms (e.g. for swords, men, Weden). Degoratively also a ditty, poorly composed poem. See \inx[C]{thyle}.

\inxitem[C]{thyle} (ON \emph{þulr}, OE \emph{þyle}, PNWGmc. \emph{*þuliʀ})
  A sage who through rote learning has acquired a large amount of mythological lore (cf. \inx[C]{thule} ‘a list in poetic form; a ditty, bad poem’ and \inx[C]{thill} ‘to recite, to chant’). Thus \inx[P]{Weden} is the \inx[P]{Fimblethyle}, being the unbeaten master of lore, as can be seen in his wisdom contests (like \Vafthrudnismal). Runic inscription DR 248 (Snoldelev) suggests the thyle may have tied to a specific place, and in \Beowulf\ it seems to have been a court position, with the poet Unferth being described (l. 1456) as the “thyle of Rothgar”.

\inxitem[C]{wale} (ON \emph{vǫlr})
  The staff or sceptre, especially of a wallow. TODO: archeological finds, mention Sutton Hoo.

\inxitem[C]{wallow} (ON \emph{vǫlva}, OE \emph{*wealwe} (cf. ON \emph{svǫlva}, OE \emph{swealwe} ‘swallow’))
  A sibyl, seeress, oracle. The word derives from the \inx[C]{wale}, a staff or sceptre probably used for ritual purposes.

\inxitem[C]{wigh} (ON \emph{vé}, OE \emph{wéoh}, \emph{wíh}, PNWGmc. \emph{*wīhą})
  A holy shrine or sanctuary. It seems that where the \inx[C]{harrow} was a pile of stones or cairn used for carrying out rituals, the \textbf{wigh} was an enclosed space. The earliest Norse attestation is the runic inscription Ög N288 (Oklunda), which reads: “Guther <= Gunnarr> painted these runes, and he fled, guilty. Sought this wigh, and he fled into this clearing. And he bound. [...]” The implication seems to be that the wigh was considered so sacred that Guther could not be apprehended or punished for his crime while in it. — In OE the word means ‘pagan idol’. It is not immediately clear which meaning is the original one, but in the present edition the Norse sense has been adopted, since the Anglo-Saxon sources are all of a Christian nature. The \Beowulf\ name \emph{Wighstone} (\emph{Wīh-} or \emph{Wēohstān}) in any case suggests it is the Norse meaning, since ‘idol-stone’ makes little sense.

\inxitem[C]{wode} (ON \emph{óðr}, OE \emph{wód}, PNWGmc. \emph{*wóþuʀ})
  \inx[P]{Heener}’s gift to men, though the name would suggest it be from \inx[P]{Weden}. The word has several related meanings: ‘poetic inspiration, madness, rage’.
\end{itemize}

\section{Persons and objects (P)}
\begin{itemize}

\inxitem[P]{Attle} (\emph{Attila}, ON \emph{Atli}, OE \emph{Ætla}, MHG. \emph{Etzel}, PNWGmc. \emph{*Attiló})
  The ruler of the \inx[G]{Huns} (historically from 434–453). Husband of \inx[P]{Guthrun}, and with her father of \inx[P]{Earp and Oatle}. and murderer of
  I HHb 54, SiL 11, I Gr 23, ShS 28, 29, 33, 37, 54, 56, 57, II Gr 26, 38, 45, III Gr 1, 9, BnOr 0, OdW A, 2, 22, 23, 25, 26, 30, 31, AtD 0, AtL 1, 3, 15, 17, 18, 27, 31, 32, 34, 36, 37, 38, 41, 43, B, AtS 2, 4, 21, 22, 44, 52, 60, 64, 71, 73, 77, 80, 86, 87, 97, 98, 108, 113, 117, FGr 0, GrB 12, Ham 6.

\inxitem[P]{Balder} (ON \emph{Baldr}, OE \emph{Bældæg} (not directly cognate), OHG \emph{Balter}, PWGmc. \emph{*Baldraʀ})
  The beautiful son of \inx[P]{Weden}, slayed by his brother \inx[P]{Hath}, avenged by his other brother \inx[P]{Wonnel}.

\inxitem[P]{Earp and Oatle} (ON \emph{Erpr ok Ęitill})
  The sons of \inx[P]{Attle} and \inx[P]{Guthrun}.

\inxitem[P]{Earth} (ON \emph{jǫrð}, OE \emph{eorþe}, OHG \emph{erda}, PNWGmc. \emph{*erþu}, PGmc. \emph{*erþó})
  The personified Earth. Through \inx[P]{Weden} the mother of \inx[P]{Thunder}.

\inxitem[P]{feather-hame} (ON \emph{fjaðrhamr})
  A \inx[C]{hame} owned by the Ease, by which it wearer flies like a bird, more specifically a falcon, between the \inx[C]{Home}[Homes].

\inxitem[P]{Free} (ON \emph{Fręyr}, OE \emph{fréa} ‘lord’, PNWGmc. \emph{*Frawjaʀ})
  Son of \inx[P]{Nearth}, brother of \inx[P]{Frow}. See also \inx[P]{Ing}.

\inxitem[P]{Frie} (ON \emph{Frigg}, OE \emph{*Frige}, OHG \emph{Frija}, PNWGmc. \emph{*Frijju})
  Wife of \inx[P]{Weden}, mother of \inx[P]{Balder}. Related to \inx[P]{Full}, who is either her sister (Second Merseburg Charm, though this may be metaphorical, as in \Hyndluljod\ 1) or her maid-servant (the Norse sources).

\inxitem[P]{Frow} (ON \emph{Fręyja})
  Cat-goddess, daughter of \inx[P]{Nearth}, sister of \inx[P]{Free}, wife of \inx[P]{Wode}. Promised to the Ettin. Possibly = Easter?

\inxitem[P]{Full} (ON \emph{Fulla}, OHG \emph{Folla})
  Maid-servant (or sister?) of \inx[P]{Frie}; see there.

\inxitem[P]{Guthrun} (ON \emph{Guðrún})
  Daughter of king \inx[P]{Yivick}, sister of \inx[P]{Guther} and \inx[P]{Hain}. The wife of \inx[P]{Attle}.

\inxitem[P]{Hain}[Hain 1] (ON \emph{Hǫgni}, OE \emph{Haguna}, \emph{Hagena}, OHG \emph{Hagano}, Ger. \emph{Hagen}, PNWGmc. \emph{*Hagunó})
  A \inx[G]{Nivlings}[Nifling] and \inx[G]{Yivickings}[Yivicking], son of king \inx[P]{Yivick}, brother of \inx[P]{Guther} and \inx[P]{Guthrun}. In \emph{AtL} he defeats seven warriors before being captured by \inx[P]{Attle}, who has his heart cut out at the request of Guther.

\inxitem[P]{Hain 2}[2]
  A petty king of \inx[L]{East Geatland}, contemporary with \inx[P]{Granmer}, the king of \inx[L]{Southmanland} and Ingeld Illred, the \inx[G]{Inglings}[Ingling] king of \inx[L]{Upland}.

\inxitem[P]{Hath} (ON \emph{Hǫðr})
  The blind son of \inx[P]{Weden}, the slayer of his brother \inx[P]{Balder}.

\inxitem[P]{Heener} (ON \emph{Hǿnir}, PNWGmc. \emph{Hónijaʀ} ‘the little swan(?)’)
  An obscure god. \textcite{Rydberg1886}[552] has convincingly argued that he is connected with the stork, connecting his name with the Greek \textgreek{κύκνος} ‘swan’ and Sanskrit \emph{śakuna} ‘bird of omen’, and noting that his epithets \emph{langi fótr} ‘long foot’ and \emph{aurkonungr} ‘mud-king’ (both found in \Skaldskaparmal\ 22) accurately describe the stork. He gives \inx[C]{wode} TODO.

\inxitem[P]{Hindle} (ON \emph{Hyndla})
  A witch awoken by \inx[P]{Frow} in \Hyndluljod.

\inxitem[P]{Homedall} (ON \emph{Hęimdallr}, OE \emph{*Hámdall})
  Ward of the gods, whitest of the \inx[G]{Ease}.

\inxitem[P]{Hymer} (ON \emph{Hymir})
  \inx[P]{Tew}’s father according to \Hymiskvida.

\inxitem[P]{Ing} (ON \emph{Yngvi}, OE \emph{Ing})
  Probably an older name of \inx[P]{Free}. The legendary ancestor of the \inx[G]{Inglings}. Cf. the Old English Rune Poem.

\inxitem[P]{Lother} (ON \emph{Lóðurr}, OS \emph{Logaþore}, PNWGmc. \emph{*Logaþorjaʀ} ‘Flame-darer(?)’)
  Gives three gifts to man. The Old-Saxon attestation is a bit uncertain.

\inxitem[P]{Millner} (ON \emph{Mjǫllnir}, OE \emph{*Meldne}, PNWGmc. \emph{*Meldunjaʀ})
  Powerful hammer owned by Thunder.

\inxitem[P]{Nearth} (ON \emph{Njǫrðr})
  The father of \inx[P]{Free} and \inx[P]{Frow} by \inx[P]{Shede}.

\inxitem[P]{Nithad} (ON \emph{Níðuðr}, OE \emph{*Hámdall})
  The Swedish king that imprisons \inx[P]{Wayland} in \Volundarkvida. Father of \inx[P]{Beadhild}.

\inxitem[P]{Oughter} (ON \emph{Óttarr}, OE \emph{Óhthere}, PNWGmc. \emph{*Óhtaharjaʀ})
  Legendary Swedish king.

\inxitem[P]{Rotholf} (ON \emph{Hrólfr kraki}, OE \emph{Hróþulf}, PNWGmc. \emph{*Hróþiwulfaʀ})
  A king of the \inx[G]{Shieldings} (see family tree). As foreshadowed in \Beowulf\ 1017–9, 1180–90, he betrays the sons of \inx[P]{Rothgar}, his cousins \inx[P]{Rethrich and Rothmund}, in order to take the throne for himself. In the later Icelandic tradition this has been forgotten, and he is consistently portrayed as a heroic king.

\inxitem[P]{Rothgar} (ON \emph{Hróarr}, OE \emph{Hróþgár}, PNWGmc. \emph{*Hróþigaiʀaʀ})
  A king of the \inx[G]{Shieldings} (see family tree), one of the main characters in \Beowulf.

\inxitem[P]{Shield} (ON \emph{Skjǫldr}, OE \emph{Scyld})
  Legendary Danish king, founder of the \inx[G]{Shieldings}.

\inxitem[P]{Sighmund} (ON \emph{Sigmundr}, OE \emph{Sigemund}, MHG. \emph{Siegmund})
  A hero of the \inx[G]{Walsings}, in \Beowulf\ attested as the slayer of the dragon along with his nephew \inx[P]{Sinfittle}. In the Norse tradition however, it is his half-brother \inx[P]{Siward} that slays the dragon instead.

\inxitem[P]{Sithguth} (OHG \emph{Sinthgunt}, PNWGmc. \emph{*Sinþagunþiz})
  Only known from \MerseburgTwo\ as the sister of \inx[C]{Sun}.

\inxitem[P]{Sun} (ON \emph{Sól}, OHG \emph{Sunna})
  The personified sun (see also \inx[P]{Moon}). In \MerseburgTwo, described as the sister of \inx[C]{Sithguth}.

\inxitem[P]{Thrim} (ON \emph{Þrymr})
  The ettin responsible for stealing Thunder’s hammer in \Thrymskvida.

\inxitem[P]{Thunder} (ON \emph{Þórr}, OE \emph{Þunor}, OHG \emph{Donar}, PNWGmc. \emph{*Þonaraʀ})
  Son of \inx[P]{Weden} and \inx[P]{Earth}.

\inxitem[P]{Tew} (ON \emph{Týr}, OE \emph{Tíw})
  Son of \inx[P]{Hymer}. One-handed god. TODO.

\inxitem[P]{Webthrithner} (ON \emph{Vafþrúðnir})
  The ettin defeated by Weden in the wisdom contest in \Vafthrudnismal.

\inxitem[P]{Weden} (rhymes with \emph{leaden}; ON \emph{Óðinn}, OE \emph{Wóden}, \emph{Wéden}, OHG \emph{Wuotan}, PNWGmc. \emph{*Wódanaʀ})
  Chief of the \inx[G]{Ease}, his name is clearly related to \inx[C]{wode}, referring to his role as the patron of \inx[C]{scold}[scolds] and \inx[C]{bearserk}[bearserks]. Husband of \inx[P]{Frie}, and by her father of \inx[P]{Balder}. Also father of \inx[P]{Thunder} by \inx[P]{Earth}. Brother of \inx[P]{Heener} and \inx[P]{Lother}.

\inxitem[P]{Wider} (ON \emph{Víðarr}, OE \emph{*Wídhere})
  A son of \inx[P]{Weden}, who avenges him at the \inx[L]{Rakes of the Reins}.

\inxitem[P]{Wode} (ON \emph{Óðr}, OE \emph{Wód})
  Husband of \inx[P]{Frow}. His name looks to be the same word as \inx[C]{wode}.

\inxitem[P]{Wonnel} (ON \emph{Váli}, OE \emph{*Wonela}, PNWGmc. \emph{*Wanilô} ‘the little \inx[G]{Wanes}[Wane]?’)
  The son of \inx[P]{Weden}, who one-night old avenged his brother \inx[P]{Balder} through slaying \inx[P]{Hath}, his half-brother.

\inxitem[P]{Woulder} (ON \emph{Ullr}, \emph{*Wuldor}, PNWGmc. \emph{*Wulþuz})
  A rather obscure god. He is mentioned in connection with oath-rings (TODO) and the setting of ritual fires (\Grimnismal\ TODO). These obscure references are likely related to the interesting finds at Lilla Ullevi (‘the small \inx[C]{wigh} of Woulder’) in Upland, Sweden, consisting of several dozen fire striker-shaped iron amulet rings dating to 660–780 (for a detailed description see \parencite{afEdholm2009}).

\inxitem[P]{Yimer} (ON \emph{Ymir}, OE \emph{*Yime})
  The first ettin, probably equivalent to \inx[P]{Earyelmer}.

\inxitem[P]{Yivick} (ON \emph{Gjúki}, OE \emph{Gifica}, OHG \emph{Gibicho}, MHG. \emph{Gibeche})
  King of the \inx[G]{Burgends} (historically from late 300s–407) of the Nifling dynasty, ancestor of the \inx[G]{Yivickings}. Father of \inx[P]{Guthrun}, \inx[P]{Guther} and \inx[P]{Hain}.
\end{itemize}

\section{Groups and tribes (G)}
TODO: Map of rough tribal areas. Geneaologies.

\begin{itemize}

\inxitem[G]{Danes} (ON \emph{danir}, OE \emph{dene}, PNWGmc. \emph{*daníʀ})
  A tribe in eastern modern-day Denmark and southern Sweden. They probably originated in Scania in southern Sweden, before moving westwards into the Danish isles and eventually Jutland, driving out the \inx[G]{Earls} and \inx[G]{Jutes}.
  Noted members: TODO
  Attestations: TODO

\inxitem[G]{Dwarfs} (ON \emph{dvergar}, OE \emph{dweorgas}, OHG \emph{twerca}, PNWGmc. \emph{*dwergóʀ})
  Earthly (chthonic) supernatural beings, often referred to as living in rocks and mountains.
  Noted members: TODO
  Attestations: TODO

\inxitem[G]{Ease} (rhyming with \emph{geese}; ON \emph{ę́sir}, OE \emph{ése}, PNWGmc. \emph{*ansiwiʀ}; sg. \emph{os}, ON \emph{áss}, OE \emph{ós}, PNWGmc. \emph{*ansuʀ})
  A group of Gods, though the word can also refer to all the Gods. See \inx[G]{Gods}, \inx[G]{Tews}, \inx[G]{Wanes}, \inx[G]{Reins}.
  Noted members: \inx[P]{Weden}, \inx[P]{Thunder}, \inx[P]{Frie}, \inx[P]{Hath} and \inx[P]{Balder}
  Attestations: TODO

\inxitem[G]{Elves} (ON \emph{alfar}, OE \emph{ielfe}, PNWGmc. \emph{*alβíʀ})
  Earthly (chthonic) supernatural beings. Possibly ancestral spirits?
  Noted members: TODO
  Attestations: TODO

\inxitem[G]{Ettins} (ON \emph{jǫtnar}, OE \emph{eotenas}, PNWGmc. \emph{*etunóʀ})
  The fundamental enemies of the Gods, the agents of chaos and disorder. See \inx[G]{Rises}, \inx[G]{Thurses}.
  Noted members: \inx[P]{Hymer}, \inx[P]{Thrim}, \inx[P]{Webthrithner}, \inx[P]{Yimer}
  Attestations: TODO

\inxitem[G]{Geats} (ON \emph{gautar}, OE \emph{géatas}, PNWGmc. \emph{*gautóʀ} from \emph{*geut-} ‘to pour’, perhaps ‘the libators’)
  A tribe in what is today southern-central Sweden. See also \inx[L]{Geatland}, \inx[G]{Swedes}.
  Noted members: TODO
  Attestations: TODO

\inxitem[G]{gin-Reins} (ON \emph{ginnręgin})
  \inx[C]{gin-} + \inx[G]{Reins}. The sacrosanct, highest divine powers.

\inxitem[G]{Gods} (ON \emph{goð}, OE \emph{godu}, OHG \emph{gota}, PNWGmc. \emph{*godu})
  TODO.
  Noted members: TODO
  Attestations: TODO

\inxitem[G]{Huns} (ON \emph{húnir}, OE \emph{Húne}, OHG \emph{Húni}, \emph{Hunni}, PNWGmc. \emph{*húníʀ})
  An invading Asiatic tribe in the Migration Period. In the legendary material their cultural and ethnic foreignness is not seen.
  Noted members: TODO
  Attestations: TODO

\inxitem[G]{Inglings} (ON \emph{ynglingar}, PNWGmc. \emph{*ingwalingóʀ} ‘the descendants of \inx[P]{Ing}’)
  Difference between this term and \inx[G]{Shelvings} is a bit unclear. They seem to be used synonymously in the Norse sources, whereas the English only use the later.

\inxitem[G]{Nears} (ON \emph{níarar} \char`~ \emph{njárar})
  A Swedish tribe, only mentioned in \Volundarkvida, where it is ruled by king \inx[P]{Nithad}. The name and location may allow us to connect them with the Swedish province of Närke, cf. Old Swedish: \emph{Nærikiar} ‘inhabitants of Närke’, \emph{Nærisker} ‘belonging to Närke; Nearish’, in which case the Old Swedish stem \emph{nær-} (with unclear vowel length, though it is probably long) would be a reduced form of \emph{níar-}, \emph{njár-}.

\inxitem[G]{Norns} (ON \emph{nornir})
  A group of supernatural women responsible for declaring the fates of men.

\inxitem[G]{Ossens} (ON \emph{ǫ́synjur})
  The women of the \inx[G]{Ease}, see there.

\inxitem[G]{Ownharriers} (ON \emph{ęinhęrjar}, OE \emph{*ánhergas})
  Earthly (chthonic) supernatural beings, often referred to as living in rocks and mountains.
  Noted members: TODO
  Attestations: TODO

\inxitem[G]{Reins} (ON \emph{rǫgn}, \emph{ręgin})
  The divine powers. Based on \Vafthrudnismal\ (TODO) the term may be more closely associated with the \inx[G]{Wanes} than the \inx[G]{Ease}.

\inxitem[G]{Saxons} (ON \emph{saxar}, OE \emph{Seaxan}, \emph{Seaxe})
  TODO.
  Noted members: TODO
  Attestations: TODO

\inxitem[G]{Shieldings} (ON \emph{skjǫldungar}, OE \emph{Scyldingas}, PNWGmc. \emph{*skeldungóʀ})
  The descendants of \inx[P]{Shield}; the legendary \inx[G]{Danes}[Danish] royal dynasty. With \inx[P]{Harward}’s death after his slaying of \inx[P]{Rotholf} their rule ended. TODO
  Noted members: TODO
  Attestations: TODO

\inxitem[G]{Shelvings} (ON \emph{skilfingar}, OE \emph{scilfingas}, PNWGmc. \emph{*skilβingóʀ})
  The descendants of \inx[P]{Shelf}; the legendary \inx[G]{Swedes}[Swedish] royal dynasty. The exact difference between the terms Shelvings and \inx[G]{Inglings} is unclear, but the first may have referred to the old royal family in Sweden, while the latter to the Norwegian branch which claimed descent from the former. TODO
  Noted members: TODO
  Attestations: \Hyndluljod\ 15, 20

\inxitem[G]{Swedes} (ON \emph{svíar}, OE \emph{swéon}, PNWGmc. \emph{*swihaníʀ})
  The tribe around the Mälar valley in eastern Sweden.
  Noted members: TODO
  Attestations: TODO

\inxitem[G]{Thurses} (sg. Thurse; ON \emph{þurs}, OE \emph{þyrs}, OS \emph{thuris}, OHG \emph{duris}, PNWGmc. \emph{*þurisaʀ})
  Possibly a poetic synonym for \inx[G]{Ettins}. See also \inx[G]{Rime-Thurses}.
  Noted members: TODO
  Attestations: Wal 8, Shr 31, 35, 36, Hyme 17, Thr 5, 10, 21, 24, 29, 30, Alw 2, I HHb 40, HHw 27.

\inxitem[G]{Tews} (ON \emph{tívar}, PNWGmc. \emph{*tíwóʀ})
  A poetic synonym for \inx[G]{Gods}.
  Attestations: TODO

\inxitem[G]{Wanes} (ON \emph{vanir}, OE \emph{wan-?})
  A subgroup or tribe of the gods, associated with fertility, harvests and fishing.
  Noted members: \inx[P]{Nearth}, \inx[P]{Ing}, \inx[P]{Frow}
  Attestations: TODO

\inxitem[G]{Yivickings} (ON \emph{gjúkungar})
  The descendants of \inx[P]{Yivick}, including \inx[P]{Guther}, \inx[P]{Guthrun} and \inx[P]{Hain}.
  Attestations: TODO
\end{itemize}

\section{Place names, locations and events (L)}
\begin{itemize}

\inxitem[L]{Eastern Way} (ON \emph{Austrvegr})
  The eastern lands of the \inx[G]{Ettins} (probably identical in meaning to \inx[L]{Ettinham}), whither \inx[P]{Thunder} goes to fight.

\inxitem[L]{Ettinham} (ON \emph{Jǫtunhęimr}, \emph{Jǫtnahęimr})
  The ‘\inx[G]{Ettins}[Ettin]-\inx[C]{Home}’ or ‘home of the Ettins’; the eastern realm of chaotic and inhospitable beings. See also \inx[L]{Eastern Way}, \inx[L]{Outyards}.

\inxitem[L]{Fimble-winter} (ON \emph{fimbulvetr})
  The great winter, which kills all humans apart from \inx[P]{Life and Lifethrasher}.

\inxitem[L]{Hell} (ON \emph{hęl}, PNWGmc. \emph{*halju}, Got. \emph{halja})
  The underworld, personfied as and formally identical with \inx[P]{Hell}. After Christianity the word came to refer to the Christian hell (= Gehenna), as is the case in all attested languages apart from the Old Norse. See also \inx[L]{Nivelhell}.

\inxitem[L]{Middenyard} (ON \emph{Miðgarðr}, OE \emph{Middangeard}, OS \emph{Middilgard}, OHG \emph{Mittilgart}, Got. \emph{midjungards})
  The ‘middle enclosure’; the realm of men. See also \inx[L]{Osyard}, \inx[L]{Outyards}.

\inxitem[L]{Nivelhell} (ON \emph{niflhęl})
  ‘Mist-Hell’, from the poetic evidence it seems like it may originally have been a synonym for \inx[L]{Hell}. In poetry it is attested in \Vafthrudnismal\ TODO: \emph{níu kom’k hęima |hld\ fyr Niflhel neðan, \\ hinig deyja ór helju halir. } ‘into nine homes I came, beneath Nivelhell; thither die men out of Hell’, the second by \Baldrsdraumar\ 2: \emph{ręið niðr þaðan |hld\ niflhęljar til; \\ mǿtti hvelpi, |hld\ þęim’s ór hęlju kom.} ‘[Weden] rode down thence to Nivel-hell; met the whelp that out of Hell came.’ Possibly the distinction was held by the first poet but not the second.

\inxitem[L]{Osyard} (ON \emph{Ásgarðr})
  The ‘enclosure of the \inx[G]{Ease}’; the heavenly realm. See also \inx[L]{Middenyard}, \inx[L]{Outyards}.

\inxitem[L]{Outyards} (ON \emph{Útgarðar})
  Not eddic. The ‘outer enclosures’, described in \Gylfaginning. See also \inx[L]{Ettinham}, \inx[L]{Middenyard}, \inx[L]{Osyard}.

\inxitem[L]{Rakes of the Reins} (ON \emph{ragna rǫk})
  The ‘fates of the \inx[G]{Reins}’, euphemism for the destruction of the world.

\inxitem[L]{Rakes of the Tews} (ON \emph{tíva rǫk})
  The \inx[L]{Rakes of the Reins}.

\inxitem[L]{Up-heaven} (ON \emph{Upphiminn}, OE \emph{Upheofon}, OS \emph{Upphimil}, OHG \emph{úfhimil})
  Highest heaven. See also \inx[F]{Earth and Up-heaven}.

\inxitem[L]{Walhall} (ON \emph{Valhǫll}, OE \emph{Wælheall})
  The hall of the slain, held by \inx[P]{Weden} and inhabited by the \inx[G]{Ownharriers}.
\end{itemize}

\section{Poetic formulæ (F)}
All formulæ are given in English translation, their attested forms and a Proto-Germanic rendition. For those consisting of two words bound together by a conjunction, \& is written in its place.

\begin{itemize}
\inxitem[F]{Earth and Up-heaven} (ON \emph{jǫrð \& upphiminn}, OE \emph{eorþe \& upheofon}, PGmc. \emph{*erþō \& uphiminaz})
  ON: Ribe charm \Voluspa\ 3, \Vafthrudnismal\ 20, \Thrymskvida\ 2, \Oddrunargratr\ 17, OE: Acreboot

\inxitem[F]{Ease and Elves} (ON \emph{ę́sir \& alfar}, OE \emph{ése \& ielfe}, PNWGmc. \emph{*alβíʀ \& ansiwiʀ})
  A merism; both heavenly and earthly spiritual beings. Notably the two words always occur in this order (never ‘Elves and Ease’), even in OE.

\inxitem[F]{words and works} (ON \emph{orð \& verk}, OE \emph{word \& weorc}, PGmc. \emph{*wurdó \& werkó})
  \Beowulf\ 289, 1100, 1833

\end{itemize}
