\part{Index (INCOMPLETE!)}

NOTE: This encyclopedia is both incomplete and inconsistently formatted. New entries will be added, and old ones be corrected and expanded in the future.

\section{Cultural and religious terms and expressions (C)}
\begin{itemize}

\inxitem[C]{All Gods} (ON \emph{ǫll goð})
  Occurs especially in ritual or ritual-adjacent use (\Grimnismal\ 43, \Lokasenna\ 11; cf. \Hakonarmal\ 18, where the piety of the dead king Hathkin is shown by his being greeted by \emph{rǫ́ð ǫll ok ręgin} ‘all the Redes and \inx[G]{Reins}’, and the prayer in \Sigrdrifumal\ 3–4, which collectively invokes the \inx[G]{Eese} and \inx[G]{Ossens}).  This suggests a native Germanic conception of Godly Oneness; see also the \inx[L]{Thing of the Gods}, where the Gods gather to steer the fates of the world.

  Similar expressions are found in other old Indo-European religions, e.g. the Vedic \emph{víşve devā́ḥ} ‘All Gods’, to Whom are dedicated numerous hymns of \Rigveda, and the Greek \textgreek{Πάν·θειον}, that is, a temple dedicated to All Gods.

  The idea of Godly Oneness may have been disputed; about this \textcite{Saxo} 1.7.2 gives an interesting anecdote.  At one point Weden departed, and during his absence was usurped by the obscure \emph{Mithothin} (perhaps “With-Weden”), who reformed the cult:
  \begin{quote}
    {\small \emph{Cuius secessu Mithothyn quidam prestigiis celeber, perinde ac celesti beneficio vegetatus, occasionem et ipse fingende divinitatis arripuit barbarasque mentes novis erroris tenebris circumfusas prestigiarum fama ad cerimonias suo nomini persolvendas adduxit. Hic deorum iram aut numinum violationem confusis permixtisque sacrificiis expiari negabat ideoque eis vota communiter nuncupari prohibebat, discreta superum cuique libamenta constituens. Qui cum Othino redeunte relicta prestigiarum ope latendi gratia Pheoniam accessisset, concursu incolarum occiditur.}

    ‘A certain Mithodin, a famous illusionist, was animated at his departure as if by a kindness from heaven and snatched the chance to pretend divinity himself; his reputation for magicianship clouded the barbarians’ minds with the murk of a new superstition and led them to perform holy rites to his name. He asserted that the gods’ wrath and the profanation of their divine authority could not be expiated by confused and mingled sacrifices; so he arranged that they must not be prayed to as a group, but separate offerings (\emph{libamenta}) be made to each deity. When Odin returned, the other no longer resorted to his conjuring but went off to hide in Funen, where he was rushed upon and killed by the inhabitants.’}
  \end{quote}

  This obviously mythologised retelling may perhaps reflect an actual historical theological conflict or attempted religious reform, but if that is the case it does not appear to have been successful.

\inxitem[C]{ape} (ON \emph{api}, OE \emph{apa}, OS \emph{apo}, OHG \emph{affo}, PNWGmc. \emph{*apó})
  In the Old Norse the word seems to mean ‘fool, buffoon’, in the other old languages apparently ‘monkey’, though this sense should be a later development of the former; why would the early Germanic tribes have a word for an animal that they had never encountered?

\inxitem[C]{aught} (ON \emph{ę́tt}, OE \emph{ǽht} ‘possession, property’)
  The Nordic (paternal) clan or family line.

\inxitem[C]{begale} (OHG \emph{bi·galan})
  To enchant, bewitch something or someone by singing a \inx[C]{galder}. Transitive of \inx[C]{gale}.

\inxitem[C]{bigh} (ON \emph{baugr}, OE \emph{béag}, OHG \emph{boug})
  Armlets used as currency during the Migration Period. — The giving of rings and armlets in exchange for loyalty (\inx[C]{holdness} being the word used for a warrior’s loyalty towards his lord, and of a lord’s grace towards his servants) was common across all of Germanic Europe, as seen in the many poetic ruler-kennings of the type “breaker of rings” (e.g. \emph{béaga brytta} ‘the breaker of bighs’ in \Beowulf\ ll. 35, 352, 1487). An illustrative example of this is \Hildebrandslied\ 33–35.
  This is also connected with the oath-ring, and the famous ring-swords. TODO? reference some literature on this.

\inxitem[C]{bloot} (ON \emph{blót}, OE \emph{blót}, OHG \emph{bluoz})
  A sacrifice or a sacrificial feast, one of the best attested Germanic pagan practices. The animals would be sacrificed by the host, cooked in large kettles and eaten communally. See also \inx[C]{bloot-house}.

\inxitem[C]{bloot-house} (ON \emph{blót-hús}, OHG \emph{bluoz-hús})
  A heathen temple. Glosses Latin \emph{fānum} in OHG. See also \inx[C]{harrow}, \inx[C]{hove}, \inx[C]{wigh}.

\inxitem[C]{Doom} (ON \emph{dómr}, OE \emph{dóm})
  Base meaning ‘judgment, verdict’ (whence Doomsday, ‘judgment Day’), but in the Norse and Anglo-Saxon poetry often specifically referring to one’s fame or good reputation (that is, how others will judge one’s character and deeds), especially after death. It is clear that this verdict was of utmost importance to the ancient Germanic people. The clearest examples are \Havamal\ 77 (see there): \emph{I know one that never dies: the \textbf{Doom} o’er each man dead.} and \Beowulf\ 1384-1389, where Beewolf consols king Rothgar after Grendle’s mother has slain his trusted advisor Asher (\emph{Æschere}):
  \emph{Ne sorga, snotor guma! \hld\ Sélre bið ǽg-hwǽm, /
  þæt hé his fréond wrece, \hld\ þonne hé fela murne. /
  Úre ǽghwylc sceal \hld\ ende ge·bídan /
  worolde lífes; \hld\ wyrce sé þe móte /
  \textbf{dómes} ǽr déaþe; \hld\ þæt bið driht-guman /
  un·lifgendum \hld\ æfter sélest.}

  ‘Grieve not, wise man! ’Tis better for each one / that he avenge his friend than that he mourn much. / Each one of us shall suffer the end / of worldly life—win he who might / \textbf{Doom} before death: that is for the warrior, / unliving, afterwards the best.’
  Other illustrative examples in \Beowulf\ include 884b–887a:
  \emph{[...] Sige-munde ge·sprǫng /
  æfter déað-dæge \hld\ \textbf{dóm} un-lýtel /
  syþðan wíges heard \hld\ wyrm á·cwealde /
  hordes hyrde [...]}
  ‘For \inx[P]{Syemund} sprang up / after his death-day an unlittle \ken*{great} \textbf{Doom}, / since hard in conflict he defeated the \inx[C]{wyrm}, / the hoard’s herder.’
  and 953b–955a: \emph{[...] þú þé self hafast /
  dę́dum ge·fręmed \hld\ þæt þín \textbf{dóm} lyfað /
  áwa tó aldre [...]}
  ‘Thou hast for thyself / by deeds accomplished that thy \textbf{Doom} lives / for ever and ever.’

\inxitem[C]{feather-hame} (ON \emph{fjaðr-hamr}, OE \emph{feðer-hama}, OS \emph{feðar-}, \emph{feðer-hamo})
  A plumage which when donned by the wearer lets him fly like, or become a bird.  One is owned by Frow and used by Lock to fly between the homes in \Thrymskvida.  In the Christian \Heliand\ feather-hames are donned by angels who fly from heaven to earth.  See also \inx[C]{hame}.

\inxitem[C]{fee} (ON \emph{fé}, OE \emph{féoh})
  Originally ‘cattle, kine’, however also used in a broader sense to refer to one’s mobile wealth; for that cf. particularly \Havamal.

\inxitem[C]{fey} (ON \emph{fęigr}, OE \emph{fǽge}, OHG \emph{feigi} ‘cowardly’)
  Being doomed or fated to die, with a sense of predestination and inevitability.  Its earliest documented Scandinavian use is on the Rök stone: \textbf{aft uamuþ stąnta runaʀ þaʀ ᛭ n uarin faþi faþiʀ aft \emph{faikiąn} sunu} \emph{Apt Vámóð standa rúnaʀ þáʀ, en Varinn fáði, faðir aft \emph{fęigjan} sonu} ‘After Woemood (\emph{Vámóðr}) stand these \inx[C]{rune}[runes], but Warren (\emph{Varinn}) painted, the father after the \textbf{fey} son.’  See \textciteshorttitle{PCRN-HS} II:35, p. 928 ff. (TODO)

\inxitem[C]{feyness} (ON \emph{fęigð})
  The state of being \inx[C]{fey}.

\inxitem[C]{fimble-} (ON \emph{fimbul-})
  The ultimate, final, greatest.  See \inx[P]{Fimblethyle}, \inx[L]{Fimble-winter}.

\inxitem[C]{five days} (ON \emph{fimm dagar})
  The Old Scandinavian (and perhaps Germanic) week was originally five days long, the seven-day week being a later import, as seen by the names of the days, which are obviously calqued from the Latin (\emph{Dies Mercurii} = Weden’s day, et.c.).  According to the \Gulatingslog\ there were six weeks in a month, and “five days” is used as a generic period of time in \Havamal\ 51 and 74; in st. 74 it is contrasted with month.  Related to this is the legal term \emph{fifth} (ON \emph{fimmt}, OSw. \emph{fæmt}), a meeting or gathering set to be held at a five-day notice.  See \emph{fimt} in \CV, \textcite{LMNL} for further discussion.

\inxitem[C]{galder} (ON \emph{galdr}, OE \emph{gealdor}, OHG \emph{galdar})
  A magical song or incantation, probably synonymous with \inx[C]{leed}.  Verbal noun formed to \inx[C]{gale} ‘to sing, chant’.

\inxitem[C]{gale} (ON \emph{gala}, OE \emph{galan}, OHG \emph{galan})
  To sing, chant, especially of magical songs; verbal root of \inx[C]{galder} ‘something sung, chanted’.

\inxitem[C]{gand} (ON \emph{gandr}, Latin \emph{gandus})
  A witch’s familiar or foul spirit sent out to do her bidding. See \textciteshorttitle{PCRN-HS} I:17, p. 361 and II:26, p. 656. TODO

\inxitem[C]{gid} (ON \emph{goði}, OE \emph{Gydda} masc. given name)
  A heathen priest or master of ceremonies.

\inxitem[C]{gidden} (ON \emph{gyðja}, OE \emph{gyden} ‘goddess’)
  The womanly equivalent or wife of a \inx[C]{gid}.

\inxitem[C]{good of meat} (ON \emph{matar góðr}, \emph{góðr matar})
   An old formula appearing in \Havamal\ 39 and numerous Swedish Wiking Age Runic inscriptions Sm 39, Sm 44, Sö 130, U 703, and U 805.  Cf. U 739 which has the related \emph{mildr mataʀ} ‘mild of meat’.  Antonyms are \emph{matar illr} ‘evil of meat’ and \inx[C]{meat-nithing}.

\inxitem[C]{guest} (ON \emph{gęstr}, OE \emph{giest}, OS \emph{gast}, OHG \emph{gast}, Got. \emph{gasts}, PGmc. \emph{gastiz})
  Guests were often strangers, wanderers, who would come to beg for food and lodgings.  The Old Germanic peoples placed great value on hospitality. TODO.

\inxitem[C]{hame} (ON \emph{hamr})
  A skin, shape. People could “shift hames” (ON \emph{skipta hǫmum}), leaving their human hames behind and instead entering into the shapes of wolves, bears, birds.  During this process the original hame, that is, the human body, would be sleeping in a vulnerable state. A concise description of this is found in \YnglingaSaga\ 7: \emph{Óðinn \emph{skipti hǫmum}, lá þá búkr’inn sem sofinn eða dauðr, en hann var þá fugl eða dýr, fiskr eða ormr, ok fór á einni svipstund á fjarlæg lǫnd at sínum erendum eða annarra manna.} ‘Weden \emph{shifted hames}; then lay the trunk of his body as if sleeping or dead, but he was then a fowl or beast, a fish or serpent, and journeyed in a short while to foreign lands with his errands or those of other men.’.

  See also \inx[C]{feather-hame}, \inx[C]{town-rideresses}, \inx[C]{evening-rideresses}.

\inxitem[C]{harrow} (ON \emph{hǫrgr}, OE \emph{hearg}, PNWGmc. \emph{*harugaʀ})
  A hallowed cairn or stone-heap.  \Hyndluljod\ 10 describes the construction of one.  The Norwegian laws prescribe the “breaking of harrows and burning of hoves”.

  See also \inx[C]{hove}, \inx[C]{wigh}.

\inxitem[C]{hold} (ON \emph{hollr}, OE \emph{hold}, OS \emph{hold}, OHG \emph{hold})
  %TODO Mention: unhold wights, Old Saxon baptismal formula.
  ‘Favourable, loyal, gracious’, often of a ruler towards his subject (in the sense of ‘gracious, benevolent’) or vice-versa (in the sense of ‘loyal, devoted’).  Mirroring these earthly relationships, it is often used to refer to divine grace, both of the Christian God—thus in the \emph{Ecclesiastical Laws of King Cnut} \textciteshorttitle[372]{ALIE1}: \emph{Þam byþ witod-líce God hold, þe bið his hlâforde riht-líce hold} ‘Indeed God is \textbf{hold} to him who is rightly \textbf{hold} to his lord’—but in the oldest Scandinavian material likewise of the Heathen gods.
  So \Lokasenna\ 4 (e.): \emph{holl ręgin} ‘\textbf{hold} \inx[G]{Reins}’, and \Oddrunargratr\ 9/1: \emph{Svá hjalpi þér \hld\ hollar véttir} ‘So help thee \textbf{hold} \inx[C]{wights}’. %TODO: Numbering of Oddrunargratr is very uncertain.

  This word is common in old Scandinavian oath formulæ, e.g. in the elder redaction of the West-Geatish Law: \emph{Svá sé mér goð holl} ‘So may the Gods(!) be \textbf{hold} to me,’ in medieval Norwegian laws (\textciteshorttitle{NGL2}[197,397]) and Grey-Goose (TODO: cite): \emph{Guð sé mér hollr ef ek satt segi, gramr ef ek lýg} ‘God be \textbf{hold} to me if I speak truly, wroth if I lie,’ in Grey-Goose (TODO) also: \emph{Sé guð hollr þeim er heldr griðum, en gramr þeim er grið rýfr} ‘God be \textbf{hold} to him who keeps the truce, but wroth against him who breaks the truce’. I refer to \textcite{Läffler1895} for further discussion on these formulæ.

\inxitem[C]{holdness} (ON \emph{hylli}, OE \emph{hyldu}, OHG \emph{huldí})
  Abstract noun formed to \inx[C]{hold}, meaning ‘favour, loyalty, grace,’ with the same semantics as the adjective.

  Notably, this word appears three times in connection with the grace of gods in the poetry, namely in \Grimnismal\ 43, where (according to my interpretation) the preparer of food at the bloot is said to earn the “\textbf{holdness} of \inx[P]{Woulder} and of all the gods;” and \Grimnismal\ 53 where the disgraced king Garfrith is said to have been bereft of the support \emph{gęngi} of Weden and all the Oneharriers, and of “Weden’s \textbf{holdness}” (\emph{Óðins hylli}).  “Weden’s holdness” is also mentioned in a stanza by Hallfred (edited as Hfr Lv 7 by Diana Whaley in \Skp\ V), who laments that: “The whole race of man has wrought songs to win the \textbf{holdness} of Weden; I recall the fully rewarded works of our kinsmen/ancestors.”

  From the semantics of this word the Germanic view on heavenly grace is clear: the Gods are \textbf{hold} towards those who do good works, which include swearing true oaths, faithfully observing truces, partaking in the bloot, following rules of hospitality, and composing poetry—and \inx[C]{gram} ‘wroth’ towards those who do the opposite.

\inxitem[C]{Home} (ON \emph{hęimr}, OE \emph{hám}, PNWGmc. \emph{*haimaʀ})
  In the Norse often referring to a realm in the cosmology (\Voluspa\ 2: “I remember nine \textbf{Homes}”, \Vafthrudnismal\ TODO: “From the runes of the \inx[G]{Ettins} and of all the gods I can speak truly, for I have come into each \textbf{Home}”). Thus \inx[L]{Ettinham} is the ‘\textbf{Home}/realm of the ettins’. When used on its own it means ‘the world (that we inhabit)’. See also \inx[L]{Nine Homes}, \inx[L]{Thrithham}.

\inxitem[C]{leat} (ON \emph{hlaut})
  In some saws explained as the blood drained from the offered animal; the verbal root is \emph{hljóta} ‘to get by lot’ and this word certainly refers to the use of the blood for auguries.

\inxitem[C]{leat-twig} (ON \emph{hlaut-tęinn})
  A twig used to sprinkle the \inx[C]{leat}.  The pattern of the blood would presumably be inspected for the augury; cf. \Hymiskvida\ 1.

\inxitem[C]{leek} (ON \emph{laukr}, OE \emph{léac}, PNWGmc \emph{laukaʀ})
  The leek was a plant of great cultural importance.  It was seen as the noblest plant, so \GudrunTwo\ 2, where \inx[P]{Siward}’s superiority to the \inx[P]{Yivickings} is compared to a stag among wild beasts, gold among silver, and a green leek in grass; and \Voluspa\ 4, where the earth of the \inx[L]{Golden Age} was grown with green leek.

  The leek was highly valued in folk magic, as seen already on gold bracteates from the C5th and C6th, where it often appears as a charm word in the old form {ᛚᚨᚢᚲᚨᛉ} \emph{laukaʀ}; in one inscription also paired with {ᛚᛁᚾᚨ} \emph{lína} ‘linen’.  Classical Norse attestations of magic use include \Sigrdrifumal\ 8, where the leek is thrown into mead against poison; and the \Volsathattr, where a horse penis is said to be \emph{\alst{l}íni gǿddr \hld\ en \alst{l}aukum studdr} ‘endowed with linen and supported by leeks’ in a poetic line.  The leek was particularly associated with women and domestic life, as seen by its pairing with “linen” and its frequent use as the determinant in women-kennings \parencite[418]{Meissner1921}).  Anon \emph{Sveinfl} 1 (\Skp\ I) sarcastically states that a battle was not \emph{sem manni \hld\ mę́r lauk eða ǫl bę́ri} ‘as if a maiden brought a man leek or ale’.%TODO: bibliography Meissner, 96. Frau, h, \textgreek{ω}

\inxitem[C]{leed} (ON \emph{ljóð}, OE \emph{léod})
  A magical chant or incantation, as seen by \Havamal\ 153 near-synonymous with \inx[C]{galder}.  See also \inx[C]{gale}, \inx[C]{begale}.

\inxitem[C]{manwit} (ON \emph{man-vit})
  Common sense and wits.

\inxitem[C]{many-cunning} (ON \emph{fjǫl-kunnigr})
  Skilled with sorcery or the dark arts.

\inxitem[C]{meat-nithing} (ON \emph{mat-níðingr})
  One who is a \inx[C]{nithing} with food, i.e. one who does not properly furnish his \inx[C]{guest}.

  See also \inx[C]{good of meat}.

\inxitem[C]{nithe} (ON \emph{níð}, OE \emph{níþ}, OHG \emph{níd})
  Originally ‘hatred, emnity’.  In the Norse the sense has developed in the direction of ‘shame’, not just as a social abstract, but almost a tangible thing.  So the curse ritual of Eyel, where the curser will “turn nithe” (\emph{snýja níð} against his enemy to cause him misfortune.  \inx[C]{scold}[Scolds] would “compose nithe” (\emph{yrkja níð}) through singing slanderous verses, which likewise had an adverse supernatural effect on their subject.  See also \inx[C]{nithing}.

\inxitem[C]{nithing} (ON \emph{níðingr}, OE \emph{níþing})
  One afflicted with \inx[C]{nithe}; a villain, criminal.  Among the Scandinavians a legal term; a nithing could not swear oaths or bear witness and was forbidden to marry.

\inxitem[C]{orlay} (ON \emph{ørlǫg}, OE \emph{orlæg})
  One’s predetermined fate, destiny, purpose as decreed by the \inx[G]{Norns}.

\inxitem[C]{queer} (ON \emph{argr, ragr} (with metathesis), OE \emph{earg}, OHG \emph{arg})
  This derogatory adjective refers to gendered sexual deviancy, typically promiscuity for women and effeminacy or cowardice for men.  This is the reason for the present English translation.  Unlike the English word, the Old Germanich \emph{arg} was always a severe insult, and this from an early period; so the Longbeardish Edict of Rothari, codified in 643 AD: \emph{Si quis alium \emph{arga} per furorem clamaverit et negare non potuerit et dixerit, quod per furorem dixisset, tunc iuratus dicat, quod eum \emph{arga} non cognovisset; postea conponat pro ipso iniurioso verbo solidos duodecim. Et si perseveraverit, convincat per pugnam, si potuerit, aut certe conponat, ut supra.} ‘If anyone calls another man \emph{queer} in anger, and cannot deny it, and says that it was said in anger, then in his oath he says that he does not know him as \emph{queer}; let him thereafter settle for the insulting word with twelve solidi.  But if he persists, let him prove it by fighting if he can, or otherwise settle it as above.’

\inxitem[C]{queerness} (ON \emph{ęrgi, ręgi})
  See \inx[C]{queer} above.

\inxitem[C]{rest} (ON \emph{rǫst})
  The distance between two rest-stops, a geographical mile (about 1850 metres).  See \CV: \emph{rǫst}.

\inxitem[C]{rune} (ON \emph{rún}, OE \emph{rún}, OS \emph{rúna}, OHG \emph{rúna}, Got. \emph{rúna}, PNWGmc. \emph{rūnu})
  An (esoteric) secret message or formula. That this—rather than ‘letter (of a Runic alphabet)’—is the original and proper sense is apparent from among others the Finnish borrowing \emph{runo} ‘poem; poetry; a division of a poem (specifically of the \emph{Kalevala})’, and its use in the singular in the earliest Runic inscriptions (e.g. Noleby Vg 63, which contains the linguistically indecipherable string of letters {ᚢᚾᚨᚦᛟᚢᛊᚢᚺᚢᚱᚨᚺᛊᚢᛊᛁᚺ[--]ᚨᛁ\rotatebox[origin=c]{180}{ᛏ}ᛁᚾ}, a \emph{rune} in the proper sense or the recently discovered Svingerud fragment.) Thus, Weden’s taking of the \emph{runes} should not be interpreted as merely a myth for the invention of profane writing, but rather the origin of esoteric incantations, not at all unlike Indian \emph{mántrās}.
  The word for letter was instead \inx[C]{stave}, see also there.

\inxitem[C]{scold} (ON \emph{skald})
  A Scandinavian court poet.  The name probably comes from their ability to slander with words.

\inxitem[C]{simble} (ON \emph{sumbl}, OE \emph{symbol})
  A banquet, symposium.

\inxitem[C]{soo} (ON \emph{sóa})
  To ritually waste, to slay in a sacrificial context.

\inxitem[C]{spae} (ON \emph{spǫ́})
  Prophecy, foresight.

\inxitem[C]{Tables} (ON \emph{tafl}, OE \emph{tæfl})
  Generic term for board games (e.g. chess).  In the \inx[C]{golden age} the \inx[G]{Eese} played such games (\Voluspa\ 8).  Pre-Christian Germanic burials commonly feature boards and bricks (TODO: reference, maybe to the Salme ship burials).

\inxitem[C]{thill} (ON \emph{þylja})
  To recite poetry learned by heart.  Cf. the so called \inx[C]{thule}[thules] (poetic lists) and the title \inx[C]{thyle}.

\inxitem[C]{Thing} (ON, OE \emph{þing}, OS \emph{thing}, OHG \emph{ding})
  The Old Germanic assembly, where cases were settled and the law determined.  In connection with the Thing certain rituals were in order, viz. the enclosing of the space wherein the judges sat by means of \inx[C]{wigh-bonds} or sacred ropes.  Cf. \Havamal\ 61 for an excerpt from \emph{Germania} ch. 22.  See also the \inx[L]{Thing of the Gods}.

\inxitem[C]{thule} (ON \emph{þula})
  A poetic list, typically of various items of a category (e.g. gods, legendary horses) or poetic synonyms (e.g. for swords, men, Weden). Degoratively also a ditty, poorly composed poem. See \inx[C]{thyle}.

\inxitem[C]{thyle} (ON \emph{þulr}, OE \emph{þyle}, PNWGmc. \emph{*þuliʀ})
  A sage who through rote learning has acquired a large amount of mythological lore (cf. \inx[C]{thule} ‘a list in poetic form; a ditty, bad poem’ and \inx[C]{thill} ‘to recite, to chant’).  Thus \inx[P]{Weden} is the \inx[P]{Fimblethyle}, being the unbeaten master of lore, as can be seen in his wisdom contests (like \Vafthrudnismal).  Runic inscription DR 248 (Snoldelev) suggests that the thyle may have tied to a specific place, and in \Beowulf\ it seems to have been a court position, with the poet Unferth being described (l. 1456) as the “Rothgar’s thyle”.

\inxitem[C]{wale} (ON \emph{vǫlr})
  The staff or sceptre of a \inx[C]{wallow}.  TODO: archeological finds, mention Sutton Hoo.

\inxitem[C]{wallow} (ON \emph{vǫlva}, OE \emph{*wealwe} (cf. ON \emph{svǫlva}, OE \emph{swealwe} ‘swallow’))
  A sibyl, seeress, oracle.  The word derives from the \inx[C]{wale}, a staff or sceptre probably used for ritual purposes.

\inxitem[C]{wigh} (ON \emph{vé}, OE \emph{wéoh}, \emph{wíh}, PNWGmc. \emph{*wīhą})
  A holy place or sanctuary. It seems that where the \inx[C]{harrow} was a pile of stones or cairn used for carrying out rituals, the \textbf{wigh} was an enclosed space. The earliest Norse attestation is the runic inscription Ög N288 (Oklunda), which reads: “Guther <= Gunnarr> painted these runes, and he fled, charged (with a crime, sought out this wigh, and he fled into this clearing. [...]” The implication seems to be that the wigh was considered so sacred that Guther could not be apprehended or punished for his crime while in it.

  In OE the word means ‘pagan idol’. It is not immediately clear which meaning is the original one, but in the present edition the Norse sense has been adopted, since the Anglo-Saxon sources are all of a Christian nature.  The name \emph{Wighstone} (\emph{Wíh-} or \emph{Wéohstān}) as found in \Beowulf\ in any case suggests it is the Norse meaning, since ‘idol-stone’ makes little sense.

\inxitem[C]{wode} (ON \emph{óðr}, OE \emph{wód}, PNWGmc. \emph{*wóduʀ})
  \inx[P]{Heener}’s gift to men, though the name may suggest it be from \inx[P]{Weden}. The word has several related meanings: ‘mind, (poetic) inspiration, rage’. See also \inx[P]{Woderearer}.

\inxitem[C]{wyrm} (ON \emph{ormr}, OE \emph{wyrm}, PNWGmc. \emph{*wurmiʀ})
  A dragon, serpent.  The distinction between “wyrm” and “worm; snake” is purely editorial and not made in the original languages.

\inxitem[C]{yin-} (ON \emph{ginn-})
  A rare augmentative prefix. TODO.

\inxitem[C]{yin-holy} (ON \emph{ginn-hęilagr})
  High holy, sacrosanct.  Used of the Gods in the formula \emph{ginn-hęilǫg goð} ‘yin-holy Gods’.

\end{itemize}

\section{Persons and objects (P)}
\begin{itemize}

\inxitem[P]{Attle} (\emph{Attila}, ON \emph{Atli}, OE \emph{Ætla}, MHG. \emph{Etzel}, PNWGmc. \emph{*Attiló})
  The ruler of the \inx[G]{Huns} (historically from 434–453). Husband of \inx[P]{Guthrun}, and with her father of \inx[P]{Earp and Oatle}.

\inxitem[P]{Balder} (ON \emph{Baldr}, OE \emph{Bældæg} (not directly cognate), OHG \emph{Balter}, PWGmc. \emph{*Baldraʀ})
  The beautiful son of \inx[P]{Weden}, slayed by his brother \inx[P]{Hath}, avenged by his other brother \inx[P]{Wonnel}.  Husband of \inx[P]{Nan}.

\inxitem[P]{Beadhild} (ON \emph{Bǫðvildr}, OE \emph{Beadohild})
  The daughter of the tyrannical king \inx[P]{Nithad}.  She is raped by her father’s prisoner, \inx[P]{Wayland}.

\inxitem[P]{Bellower} (ON \emph{Bęli})
  A being fought by \inx[P]{Free}, who killed him with an antler, having lost his sword after the events of \Skirnismal.  The myth is very obscure and never told in full.  It is shortly mentioned in \Gylfaginning\ 37 and informs the kenning \emph{bani Bęlja} ‘bane of Bellower \ken*{= Free}’ in \Voluspa\ 51/3, along with two Scaldic kennings of the same type.

\inxitem[P]{Bicke} (ON \emph{Bikki})
  A servant or general of \inx[P]{Attle}.

\inxitem[P]{Earp and Oatle} (ON \emph{Erpr ok Ęitill})
  The sons of \inx[P]{Attle} and \inx[P]{Guthrun}.

\inxitem[P]{Earth} (ON \emph{jǫrð}, OE \emph{eorþe}, OHG \emph{erda}, PNWGmc. \emph{*erþu}, PGmc. \emph{*erþó})
  The personified Earth.  By \inx[P]{Weden} the mother of \inx[P]{Thunder}.

\inxitem[P]{Erminric} (ON \emph{Jǫrmunrekr}, OE \emph{Eormanríc}, MHG \emph{Ermenrîch})
  Legendary king of the eastern \inx[G]{Gots}, based on the historical \emph{Ermanaric} (dead 376).  TODO: Jordanes.

\inxitem[P]{Fathomer} (ON \emph{Fáfnir})
  The son of \inx[P]{Rethmar}, brother of \inx[P]{Otter} and \inx[P]{Rein}.  He turns into a great \inx[C]{wyrm} and is eventually slain by \inx[P]{Siward}, who takes his treasure.

\inxitem[P]{Fimblethyle} (ON \emph{Fimbulþulr})
  The ‘ultimate \inx[C]{thyle}’ or sage; name for \inx[P]{Weden}.

\inxitem[P]{Fold} (ON \emph{Fold}, OE \emph{Folde})
  A poetic or ritual name of \inx[P]{Earth}, especially in her role as Mother Earth.  In Germanic poetry the word \emph{fold} is typically used to simply refer to ‘land’, however.  It is cognate with Sanskrit TODO.

\inxitem[P]{Foresitter} (ON \emph{Forseti})
  An obscure god associated with legal proceedings. TODO.

\inxitem[P]{Free} (ON \emph{Fręyr}, OE \emph{fréa} ‘lord’, PNWGmc. \emph{*Frawjaʀ})
  Son of \inx[P]{Nearth}, brother of \inx[P]{Frow}. See also \inx[P]{Ing}.

\inxitem[P]{Frie} (ON \emph{Frigg}, OE \emph{*Frige}, OHG \emph{Frija}, PNWGmc. \emph{*Frijju})
  Wife of \inx[P]{Weden}, mother of \inx[P]{Balder}. Related to \inx[P]{Full}.

\inxitem[P]{Frow} (ON \emph{Fręyja})
  Cat-goddess, daughter of \inx[P]{Nearth}, sister of \inx[P]{Free}, wife of \inx[P]{Wode}. Promised to the Ettin. Possibly = Easter?

\inxitem[P]{Full} (ON \emph{Fulla}, OHG \emph{Folla})
  In the Norse sources the maid-servant of \inx[P]{Frie}.  \MerseburgTwo\ has her as Frie’s sister, though this need not be literal (cf. \Hyndluljod\ 1).

\inxitem[P]{Guther} (ON \emph{Gunnarr}, MHG \emph{Gunther})
  The lord of the \inx[G]{Gots}.  In the Norse sources the brother of \inx[P]{Hain}.  Historically he is based on king \emph{Gundaharius} (\emph{*Gunþiharjaz}) of the Burgundians.

\inxitem[P]{Guthlathe} (ON \emph{Gunnlǫð})
  Daughter of the ettin \inx[P]{Sutting}; she guarded the \inx[C]{Mead of Poetry} in the mountain, but gave it to \inx[P]{Weden} after he seduced her.

  See \Havamal\ 103–110.

\inxitem[P]{Guthrun} (ON \emph{Guðrún})
  Daughter of king \inx[P]{Yivick}, sister of \inx[P]{Guther} and \inx[P]{Hain}. The wife of \inx[P]{Attle}.

\inxitem[P]{Hain}[Hain 1] (ON \emph{Hǫgni}, OE \emph{Haguna}, \emph{Hagena}, OHG \emph{Hagano}, Ger. \emph{Hagen}, PNWGmc. \emph{*Hagunó})
  A \inx[G]{Nivlings}[Nivling] and \inx[G]{Yivickings}[Yivicking], son of king \inx[P]{Yivick}, brother of \inx[P]{Guther} and \inx[P]{Guthrun}. In \Atlakvida\ he defeats seven warriors before being captured by \inx[P]{Attle}, who has his heart cut out at the request of Guther.

\inxitem[P]{Hain 2}[2]
  A petty king of \inx[L]{East Geatland}, contemporary with \inx[P]{Granmer}, the king of \inx[L]{Southmanland} and Ingeld Illred, the \inx[G]{Inglings}[Ingling] king of \inx[L]{Upland}.

\inxitem[P]{Hath} (ON \emph{Hǫðr})
  The blind son of \inx[P]{Weden}, the slayer of his brother \inx[P]{Balder}.

\inxitem[P]{Heener} (ON \emph{Hǿnir}, PNWGmc. \emph{Hónijaʀ} ‘the little swan(?)’)
  An obscure god. \textcite{Rydberg1886}[552] has convincingly argued that he is connected with the stork, connecting his name with the Greek \textgreek{κύκνος} ‘swan’ and Sanskrit \emph{şakuná} ‘bird of omen’, and noting that his epithets \emph{langi fótr} ‘long foot’ and \emph{aurkonungr} ‘mud-king’ (both found in \Skaldskaparmal\ 22) accurately describe the stork. He gives \inx[C]{wode} TODO.

\inxitem[P]{Hell} (ON \emph{Hęl})
  Owneress of \inx[L]{Hell}.

\inxitem[P]{Hindle} (ON \emph{Hyndla})
  A witch awoken by \inx[P]{Frow} in \Hyndluljod.

\inxitem[P]{Homedal} (ON \emph{Hęimdal(l)r}, OE \emph{*Hâmdeall})
  The Watchman of the Gods (\emph{vǫrðr goða} \Grimnismal\ 13, \Lokasenna\ 48), whose home is the \inx[L]{Heavenbarrows} (\Grimnismal\ 13).  According to \Rigsthula\ he fathered the three castes of men, which may also be referenced in \Voluspa\ 1/2b.  He is the whitest of the \inx[G]{Eese} (\Thrymskvida\ 15).  Homedal was the subject of the lost poem “Homedal’s galder” (\emph{Hęimdallargaldr}), of which only two lines survive; see Eddic Fragment 3 under Mythic Poetry.

\inxitem[P]{Hymer} (ON \emph{Hymir})
  An ettin, \inx[P]{Tew}’s father according to \Hymiskvida.

\inxitem[P]{Ing} (ON \emph{Yngvi}, OE \emph{Ing})
  Probably an older name of \inx[P]{Free}. The legendary ancestor of the \inx[G]{Inglings}. Cf. the Old English Rune Poem.

\inxitem[P]{Life and Lifethrasher} (ON \emph{Líf ok Líf-þrasir})
  The only surviving humans after the \inx[L]{Rakes of the Reins}.

\inxitem[P]{Lock} (ON \emph{Loki})
  The bound Os. TODO.

\inxitem[P]{Loride} (ON \emph{Hlórriði})
  “Loud/Roaring Rider”, poetic name of \inx[P]{Thunder}.

\inxitem[P]{Lother} (ON \emph{Lóðurr}, OS \emph{Logaþore}, PNWGmc. \emph{*Logaþorjaʀ} ‘Flame-darer(?)’)
  Gives three gifts to man.  The Old Saxon attestation is uncertain.

\inxitem[P]{Millner} (ON \emph{Mjǫllnir}, OE \emph{*Meldne}, PNWGmc. \emph{*Meldunjaʀ})
  The hammer of \inx[P]{Thunder}.

\inxitem[P]{Moon} (ON \emph{Máni})
  The personfied moon.  Son of \inx[P]{Mundlefare} and brother of the \inx[P]{Sun} (\Vafthrudnismal\ 23).  For ritual invocations of the Moon see Note to \Havamal\ TODO (\emph{hęiptum kveða}).

\inxitem[P]{Mundlefarer} (ON \emph{Mundilfǿri} or \emph{Mundilfari})
  The father of \inx[P]{Sun} and \inx[P]{Moon} (\Vafthrudnismal\ 23).  Perhaps ‘Millhandle-turner’, if the first element = ON \emph{mǫndull} ‘handle of a mill’.

\inxitem[P]{Nearth} (ON \emph{Njǫrðr})
  One of the \inx[G]{Wanes}. Father of \inx[P]{Free} and \inx[P]{Frow}.

\inxitem[P]{Nithad} (ON \emph{Níðuðr}, OE \emph{Níþhad}, PNWGmc. \emph{*Níþa-haduz})
  The king that imprisoned \inx[P]{Wayland}, father of \inx[P]{Beadhild} and two unnamed sons (\Volundarkvida, \Deor).

\inxitem[P]{Oughter} (ON \emph{Óttarr}, OE \emph{Óhthere}, PNWGmc. \emph{*Óhta-harjaʀ})
  Legendary Swedish king.

\inxitem[P]{Reading} (ON \emph{Hrauðungr})
  A king in the prologue to \Grimnismal.

\inxitem[P]{Rotholf} (ON \emph{Hrólfr kraki}, OE \emph{Hróþulf}, PNWGmc. \emph{*Hróþi-wulfaʀ})
  A king of the \inx[G]{Shieldings} (see family tree). As foreshadowed in \Beowulf\ 1017–9, 1180–90, he betrays the sons of \inx[P]{Rothgar}, his cousins \inx[P]{Rethrich and Rothmund}, in order to take the throne for himself. In the later Icelandic tradition this has been forgotten, and he is consistently portrayed as a heroic king.

\inxitem[P]{Rothgar} (ON \emph{Hróarr}, OE \emph{Hróþgár}, PNWGmc. \emph{*Hróþi-gaiʀaʀ})
  A king of the \inx[G]{Shieldings} (see family tree), one of the main characters in \Beowulf.

\inxitem[P]{Rungner} (ON \emph{Hrungnir})
  Famous ettin fought by Thunder.  The full story is told in \Haustlong\ 14–20 and \Skaldskaparmal\ 24–25, which cites the former.

\inxitem[P]{Shede} (ON \emph{Skaði}, OE \emph{Scede}(?), PGmc. \emph{*Skadī})
  A female figure, possibly the namesake of \inx[L]{Shedeny} and the \inx[L]{Shedelands}, in which case she was in an early period closely associated with, and perhaps thought to guard, the Scandinavian (or properly \emph{Scadinavian}, see Shedeny) peninsula.
  In the Norse tradition the daughter of \inx[P]{Thedse}, and later wife of \inx[P]{Nearth}.  Their marriage is the subject of \Gylfaginning\ which preserves.

\inxitem[P]{Shield} (ON \emph{Skjǫldr}, OE \emph{Scyld}, PNWGmc. \emph{*Skelduz})
  Legendary Danish king, founder of the \inx[G]{Shieldings}.

\inxitem[P]{Syemund} (ON \emph{Sig-mundr}, OE \emph{Sige-mund}, MHG. \emph{Sieg-mund}, PNWGmc. \emph{*Sigi-munduʀ})
  In the Norse tradition the son of king \inx[P]{Walsing}.  He begets \inx[P]{Siward}, the slayer of the wyrm \inx[P]{Fathomer}.  In \Beowulf\ it is Syemund himself who slays an unnamed wyrm.  Connected with his nephew \inx[P]{Sinfittle}.

\inxitem[P]{Sithguth} (OHG \emph{Sinthgunt}, PNWGmc. \emph{*Sinþa-gunþiʀ}(?))
  Only known from \MerseburgTwo\ as the sister of \inx[P]{Sun}.

\inxitem[P]{Siward} (ON \emph{Sigurðr})
  A hero of the \inx[G]{Walsings}, slayer of the \inx[C]{wyrm} \inx[P]{Fathomer}.

\inxitem[P]{Sun} (ON \emph{Sól}, OHG \emph{Sunna})
  The personified Sun, who in the Germanic mythology is a woman.  In \Vafthrudnismal\ 22 the daughter of \inx[P]{Mundlefare} and sister of \inx[P]{Moon}.  In \MerseburgTwo\ the sister of \inx[P]{Sithguth}.

\inxitem[P]{Thedse} (ON \emph{Þjatsi})
  An ettin slain by the Gods; his myth is told at length in \Haustlong.  Father of \inx[P]{Shede}.

\inxitem[P]{Thrim} (ON \emph{Þrymr})
  Ettin who steals Thunder’s hammer in \Thrymskvida\ and is later killed.

\inxitem[P]{Thunder} (ON \emph{Þórr}, OE \emph{Þunor}, OHG \emph{Donar}, PNWGmc. \emph{*Þonaraʀ})
  Son of \inx[P]{Weden} and \inx[P]{Earth}.  Friend of men, guarding of Middenyard.

\inxitem[P]{Tew} (ON \emph{Týr}, OE \emph{Tíw})
  Son of \inx[P]{Hymer} or \inx[P]{Weden}, one-handed god.  His name is not identical to Sanskrit \emph{Dyāús}, Greek \emph{Zeus}, Latin \emph{Iuppiter}, but rather is the singular of \inx[G]{Tews} and simply means ‘god’, cognate with Sanskrit \emph{devá}, Latin \emph{deus}.

\inxitem[P]{Walfather} (ON \emph{Val-fǫðr})
  ‘Father of the Slain’; name for \inx[P]{Weden}.

  \Voluspa\ 1/3a, 26/4a, 28/4a, \Grimnismal\ 49/2a

\inxitem[P]{Wayland} (ON \emph{Vǫlundr}, OE \emph{Wéland, Wélund})
  A legendary smith captured by the tyrannical king \inx[P]{Nithad}.  In both the Norse \Volundarkvida\ and English \Deor\ he takes his revenge by first killing Nithad’s unnamed sons and then raping his daughter \inx[P]{Beadhild}.  In the Norse version he is married to \inx[P]{Harware Elwight}.

\inxitem[P]{Webthrithner} (ON \emph{Vaf-þrúðnir})
  An Ettin defeated by Weden in the wisdom contest in \Vafthrudnismal.

\inxitem[P]{Weden} (rhymes with \emph{leaden}; ON \emph{Óðinn}, OE \emph{Wóden}, \emph{Wéden}, OHG \emph{Wuotan}, PNWGmc. \emph{*Wódanaʀ} ‘Lord of \inx[C]{wode} (poetry, intelligence)’)
  Chief of the \inx[G]{Eese}, God of Wisdom, Galder, Poetry, War.  Husband of \inx[P]{Frie}, and by her father of \inx[P]{Balder}.  Father of \inx[P]{Thunder} by \inx[P]{Earth}.  Brother of \inx[P]{Heener} and \inx[P]{Lother} or \inx[P]{Will} and \inx[P]{Wigh}.

\inxitem[P]{Wider} (ON \emph{Víð-arr}, OE \emph{*Wíd-here}, PNWGmc. \emph{*Wída-harjaʀ})
  Son of \inx[P]{Weden}, who avenges him at the \inx[L]{Rakes of the Reins}.

\inxitem[P]{Wigh} (ON \emph{Véi}, PNWGmc. \emph{*Wíhá} ‘hallower, (heathen) priest’)
  Brother of \inx[P]{Weden} and \inx[P]{Will}.

\inxitem[P]{Wighward} (ON \emph{Véurr} < PNWGmc. \emph{*Wíha-warjaʀ})
  “Wigh-Guardian, Sanctuary-Defender”, poetic name of \inx[P]{Thunder}.  Sometimes extended to \emph{Miðgarðs Véurr} ‘Middenyard’s Wighward’.  See \inx[C]{wigh}.

\inxitem[P]{Will} (ON \emph{Vili}, PNWGmc. \emph{*Wiljá})
  Brother of \inx[P]{Weden} and \inx[P]{Wigh}.

\inxitem[P]{Wing-Thunder} (ON \emph{Ving-Þórr})
  Rare poetic name of Thunder.  The first element is not \emph{vę́ngr} ‘wing (of a bird)’.  It may mean ‘swinging’ (cf. Swedish \emph{vingla}), referring to the swinging of his hammer, or ‘victorious’, representing a n-infixed extension of the verb \emph{vega} ‘to strike, smite, fight’ (cf. Latin \emph{vincere} ‘to win, vanquish’); cf. the related name \inx[P]{Wingner}.

  Occurs in \Thrymskvida\ 1, \Allvismal\ 6.

\inxitem[P]{Wode} (ON \emph{Óðr}, OE \emph{Wód})
  Husband of \inx[P]{Frow} of whom very little is known.  His name seems to be the same word as \inx[C]{wode}.

\inxitem[P]{Wonnel} (ON \emph{Váli}, OE \emph{*Wonela}, PNWGmc. \emph{*Wanilô} ‘the little \inx[G]{Wanes}[Wane]?’)
  Son of \inx[P]{Weden}, who just one night old avenges his brother \inx[P]{Balder} through slaying \inx[P]{Hath}, his half-brother.

\inxitem[P]{Woulder} (ON \emph{Ullr}, \emph{*Wuldor}, PNWGmc. \emph{*Wulþuz})
  Obscure god mentioned in connection with oath-rings (TODO) and the setting of ritual fires (\Grimnismal\ 43). These details may be related to the interesting finds at Lilla Ullevi (‘the small \inx[C]{wigh} of Woulder’) in Upland, Sweden, consisting of several dozen fire striker-shaped iron amulet rings dating to 660–780 \textcite{afEdholm2009}.

\inxitem[P]{Yimer} (ON \emph{Ymir}, OE \emph{*Yime})
  The primeval ancestor of the \inx[G]{Ettins}, probably equivalent to \inx[P]{Earyelmer}.  The first Gods sacrificed Yimer and created the world from his corpse (\Vafthrudnismal\ 21, \Grimnismal\ 41–42).

\inxitem[P]{Yivick} (ON \emph{Gjúki}, OE \emph{Gifica}, OHG \emph{Gibicho}, MHG. \emph{Gibeche})
  King of the \inx[G]{Burgends} (historically from late 300s–407) of the Nivling dynasty, ancestor of the \inx[G]{Yivickings}. Father of \inx[P]{Guthrun}, \inx[P]{Guther} and \inx[P]{Hain}.
\end{itemize}

\section{Groups and tribes (G)}
TODO: Map of rough tribal areas. Geneaologies.

\begin{itemize}

\inxitem[G]{Danes} (ON \emph{danir}, OE \emph{dene}, PNWGmc. \emph{*daníʀ})
  A tribe in eastern modern-day Denmark and southern Sweden. They probably originated in Scania in southern Sweden, before moving westwards into the Danish isles and eventually Jutland, driving out the \inx[G]{Earls} and \inx[G]{Jutes}.
  Noted members: TODO
  Attestations: TODO

\inxitem[G]{Dwarfs} (ON \emph{dvergar}, OE \emph{dweorgas}, OHG \emph{twerca}, PNWGmc. \emph{*dwergóʀ})
  Earthly (chthonic) supernatural beings, often referred to as living in rocks and mountains.
  Noted members: TODO
  Attestations: TODO

\inxitem[G]{Eese} (rhyming with \emph{geese}; ON \emph{ę́sir}, OE \emph{ése}, PNWGmc. \emph{*ansiwiʀ}; sg. \emph{os}, ON \emph{ǫ́ss}, OE \emph{ós}, PNWGmc. \emph{*ansuʀ})
  The (male) gods. Snorre has them as a separate tribe from the \inx[G]{Wanes}. See also \inx[G]{Gods}, \inx[G]{Tews}, \inx[G]{Reins}.
  Noted members: \inx[P]{Weden}, \inx[P]{Thunder}, \inx[P]{Frie}, \inx[P]{Hath} and \inx[P]{Balder}
  Attestations: TODO

\inxitem[G]{Elves} (ON \emph{alfar}, OE \emph{ielfe}, PNWGmc. \emph{*alβíʀ})
  Earthly (chthonic) minor deities. Possibly ancestral spirits?
  Noted members: TODO
  Attestations: TODO

\inxitem[G]{Ettins} (ON \emph{jǫtnar}, OE \emph{eotenas}, PNWGmc. \emph{*etunóʀ})
  The fundamental enemies of the Gods, the agents of chaos and disorder. See \inx[G]{Rises}, \inx[G]{Thurses}.
  Noted members: \inx[P]{Hymer}, \inx[P]{Thrim}, \inx[P]{Webthrithner}, \inx[P]{Yimer}
  Attestations: TODO

\inxitem[G]{Geats} (ON \emph{gautar}, OE \emph{géatas}, PNWGmc. \emph{*gautóʀ} from \emph{*geut-} ‘to pour’, perhaps ‘the libators’)
  A tribe in what is today southern-central Sweden. See also \inx[L]{Geatland}, \inx[G]{Swedes}.
  Noted members: TODO
  Attestations: TODO

\inxitem[G]{yin-Reins} (ON \emph{ginn-ręgin})
  \inx[C]{yin-} + \inx[G]{Reins}. The sacrosanct, highest Divine Powers.

\inxitem[G]{Gods} (ON \emph{goð}, OE \emph{godu}, OHG \emph{gota}, PNWGmc. \emph{*godu})
  TODO.
  Noted members: TODO
  Attestations: TODO

\inxitem[G]{Huns} (ON \emph{húnir}, OE \emph{Húne}, OHG \emph{Húni}, \emph{Hunni}, PNWGmc. \emph{*húníʀ})
  An invading Asiatic tribe in the Migration Period. In the Scandinavian legends they have been assimilated into the Germanic framework, and are not presented as racially or culturally distinct.
  Noted members: \inx[P]{Attle}, TODO
  Attestations: TODO

\inxitem[G]{Inglings} (ON \emph{ynglingar}, PNWGmc. \emph{*ingwalingóʀ} ‘the descendants of \inx[P]{Ing}’)
  The oldest known Swedish kingly lineage. The difference between this term and \inx[G]{Shelvings} is a bit unclear; \Beowulf\ knows them only by the latter term, while they seem to be used synonymously in the Norse sources.

\inxitem[G]{Nears} (ON \emph{njárar} \char`~\ \emph{níarar})
  An old Swedish tribe mentioned in \Volundarkvida, where it is ruled by king \inx[P]{Nithad}.  The location may allow us to connect them with the Swedish province of Närke, cf. Old Swedish \emph{Næríkjar} ‘inhabitants of Närke’, \emph{Nærisker} ‘belonging to Närke’.  The Old Swedish stem \emph{nær-} (with unclear vowel length, though it is probably long) would then be a reduced form of \emph{níar-}, \emph{njár-}.

\inxitem[G]{Norns} (ON \emph{nornir})
  Supernatural women responsible for the fates (\inx[C]{orlay}s) of men.  Probably synonymous with \inx[G]{Dises}, \inx[G]{Mothers}.

\inxitem[G]{Ossens} (ON \emph{ǫ́synjur})
  The wives of the \inx[G]{Eese}, the goddesses.

\inxitem[G]{Oneharriers} (ON \emph{ęin-hęrjar}, OE \emph{*án-hęrgas})
  Weden’s chosen warriors, probably corresponding to the Vedic \emph{Marútas}.  The Oneharriers have some agency (\Grimnismal\ 53/3) and were likely also invoked in rituals.
  Attestations: TODO

\inxitem[G]{Reins} (ON \emph{rǫgn}, \emph{ręgin})
  The heavenly powers.  Judging from \Vafthrudnismal\ TODO the term may be more closely associated with the \inx[G]{Wanes} than the \inx[G]{Eese}.

\inxitem[G]{Saxons} (ON \emph{saxar}, OE \emph{Seaxan}, \emph{Seaxe})
  TODO.
  Noted members: TODO
  Attestations: TODO

\inxitem[G]{Shieldings} (ON \emph{skjǫldungar}, OE \emph{Scyldingas}, PNWGmc. \emph{*skeldungóʀ})
  The descendants of \inx[P]{Shield}; the legendary \inx[G]{Danes}[Danish] royal dynasty. With \inx[P]{Harward}’s death after his slaying of \inx[P]{Rotholf} their rule ended. TODO
  Noted members: TODO
  Attestations: TODO

\inxitem[G]{Shelvings} (ON \emph{skilfingar}, OE \emph{scilfingas}, PNWGmc. \emph{*skilβingóʀ})
  The descendants of \inx[P]{Shelf}; the legendary \inx[G]{Swedes}[Swedish] royal dynasty. The exact difference between the terms Shelvings and \inx[G]{Inglings} is unclear, but the first may have referred to the old royal family in Sweden, while the latter to the Norwegian branch which claimed descent from the former. TODO
  Noted members: TODO
  Attestations: \Hyndluljod\ 15, 20

\inxitem[G]{Swedes} (ON \emph{svíar}, OE \emph{swéon}, PNWGmc. \emph{*swihaníʀ})
  The tribe around the Mälar valley in eastern Sweden.
  Noted members: TODO
  Attestations: TODO

\inxitem[G]{Thurses} (sg. Thurse; ON \emph{þurs}, OE \emph{þyrs}, OS \emph{thuris}, OHG \emph{duris}, PNWGmc. \emph{*þurisaʀ})
  Possibly a poetic synonym for \inx[G]{Ettins}. See also \inx[G]{Rime-Thurses}.
  Noted members: TODO
  Attestations: TODO

\inxitem[G]{Tews} (ON \emph{tívar}, PNWGmc. \emph{*tíwóʀ})
  A poetic synonym for \inx[G]{Gods}.  The word derives from the PIE \emph{*deywós} and is thus cognate with Sanskrit \emph{devá} ‘god’, Latin \emph{deus} ‘id.’
  Attestations: TODO

\inxitem[G]{Walsings} (ON \emph{vǫlsungar})
  The descendants of king \inx[P]{Walsing}.

\inxitem[G]{Wanes} (ON \emph{vanir}, OE \emph{wan-?})
  A subgroup or tribe of the gods, associated with fertility, harvests and the sea.
  Noted members: \inx[P]{Nearth}, \inx[P]{Free}, \inx[P]{Frow}
  Attestations: TODO

\inxitem[G]{Yivickings} (ON \emph{gjúkungar})
  The descendants of \inx[P]{Yivick}, including \inx[P]{Guther}, \inx[P]{Guthrun} and \inx[P]{Hain}.
  Attestations: TODO
\end{itemize}

\section{Places and events (L)}
\begin{itemize}

\inxitem[L]{Eastern Way} (ON \emph{Austr-vegr})
  In the mythology the eastern lands of the \inx[G]{Ettins}, to which \inx[P]{Thunder} goes to fight the Ettins and protect the realms of Gods and Men; see also \inx[L]{Ettinham}.  In human geography referring to Eastern Europe and Asia.

\inxitem[L]{Ettinham} (ON \emph{Jǫtun-hęimr}, \emph{Jǫtna-hęimar})
  The ‘\inx[G]{Ettins}[Ettin]-\inx[C]{Home}’ or ‘home of the Ettins’; the eastern realm of chaotic and inhospitable beings.  See also \inx[L]{Eastern Way}, \inx[L]{Outyards}.

\inxitem[L]{Fimble-winter} (ON \emph{fimbulvetr})
  The great winter, which kills all humans apart from \inx[P]{Life and Lifethrasher}.

\inxitem[L]{Gap of Ginnings} (ON \emph{Ginnunga-gap})
  The ‘gap of hawks’ (\emph{ginnungr} ‘ginning’ being a poetic name for the hawk); a kenning for the air, which in the old Germanic cosmology was the midspace between \inx[L]{Earth} and \inx[L]{Upheaven}; not synonymous with the latter.

  In the Eddic corpus only occurring once, viz. in \Voluspa\ 3.

\inxitem[L]{Geatland} (ON \emph{Gaut-land, Gauta-land})
  The land of the \inx[G]{Geats}.

\inxitem[L]{Hell} (ON \emph{hęl}, PNWGmc. \emph{*halju}, Got. \emph{halja})
  The Underworld, personfied as and formally identical to \inx[P]{Hell}.  After the arrival of Christianity the word came to refer to the Christian hell-fire (= \emph{Gehenna}), which is the case in all attested languages apart from the Old Norse.  See also \inx[L]{Nivelhell}.

\inxitem[L]{Idewolds} (ON \emph{Iða-vęllir})
  The ‘Plains of Industry’, where the Gods settled and built Osyard.  Mentioned in \Voluspa.

\inxitem[L]{Lithshelf} (ON \emph{Hlið-skjǫlf})
  The ‘Cliffside Shelf’; the lookout post of the gods from which they can see the whole world (\Grimnismal, \Skirnismal).

\inxitem[L]{Middenyard} (ON \emph{Mið-garðr}, OE \emph{Middan-geard}, OS \emph{Middil-gard}, OHG \emph{Mittil-gart}, Got. \emph{midjun-gards})
  The ‘Middle Enclosure’, which the Gods made as a home for men.  The enclosing poles were the hair-strands of \inx[P]{Yimer}’s eyebrows (\Grimnismal\ 42); Middenyard is defended by \inx[P]{Thunder} (\Harbardsljod\ TODO, \Voluspa\ 53).  See also \inx[L]{Osyard}, \inx[L]{Outyards}.
  \textbf{Occurrences:} \Voluspa\ 4, 53, \Grimnismal\ 42, \Harbardsljod\ TODO.

\inxitem[L]{Nivelhell} (ON \emph{nifl-hęl})
  ‘Mist-Hell’. From the poetic evidence it seems like it may originally have been a synonym for \inx[L]{Hell}.

\inxitem[L]{Osyard} (ON \emph{Ǫ́s-garðr})
  The ‘Enclosure of the \inx[G]{Eese}’; the heavenly realm.  See also \inx[L]{Middenyard}, \inx[L]{Outyards}.

\inxitem[L]{Outyards} (ON \emph{Út-garðar})
  Not Eddic.  The ‘Outer Enclosures’, described in \Gylfaginning. See also \inx[L]{Ettinham}, \inx[L]{Middenyard}, \inx[L]{Osyard}.

\inxitem[L]{Rakes of the Reins} (ON \emph{ragna rǫk})
  The ‘judgments, fated events of the \inx[G]{Reins}’, namely the destruction of the world as narrated most completely in \Voluspa.

\inxitem[L]{Rakes of the Tews} (ON \emph{tíva rǫk})
  See \inx[L]{Rakes of the Reins}.

\inxitem[L]{Thing of the Gods} (ON \emph{þing goða})
  The Divine Council or Assembly, where the Gods convene and make decisions; a conception well known from Near Eastern literature.  Like the historical Germanic assemblies, the Thing is only attended by the male \inx[G]{Eese}, whereas the \inx[G]{Ossens} are \emph{á máli} ‘at speech’ (\Baldrsdraumar\ 1, \Thrymskvida\ 14).  The Thing is held every day at \inx[L]{Ugdrassle’s Ash}; Thunder wades to it, and the other Eese ride to it (\Grimnismal\ 29–30).  Thirteen Gods were present at the Thing: \inx[P]{Weden}, \inx[P]{Thunder}, \inx[P]{Nearth}, \inx[P]{Free}, \inx[P]{Tew}, \inx[P]{Homedal}, \inx[P]{Bray}, \inx[P]{Wider}, \inx[P]{Wonnel}, \inx[P]{Woulder}, \inx[P]{Heener}, \inx[P]{Foresitter}, \inx[P]{Lock}) (\Gylfaginning\ TODO).  With Lock excluded this makes twelve, which corresponds to the Old Germanic jury of twelve men.

  The Germanic Thing of the Gods has Near Eastern equivalents, including in the Hebrew Bible.  TODO.

  Occurrences: \Voluspa\ 6, 9, et c.; \Baldrsdraumar 1; \Grimnismal\ 29–30; \Thrymskvida\ 14; \Hymiskvida\ 39.

\inxitem[L]{Thrithham} (ON \emph{Þrúð-hęimr})
  \inx[P]{Thunder}’s \inx[C]{home}.  See \inx[C]{thrith}.

\inxitem[L]{Ugdrassle’s Ash} (ON \emph{askr Yggdrasils})
  The noblest tree; the site of the \inx[L]{Thing of the Gods}.

\inxitem[L]{Up-heaven} (ON \emph{upp-himinn}, OE \emph{up-heofon}, OS \emph{upp-himil}, OHG \emph{úf-himil})
  Highest Heaven; used in \inx[F]{Earth and Up-heaven}.

\inxitem[L]{Walhall} (ON \emph{Valhǫll}, OE \emph{*Wælheall})
  The ‘Hall of the Slain’, owned by \inx[P]{Weden} and inhabited by the \inx[G]{Oneharriers}.

  \Voluspa\ 33/4a, \Grimnismal\ 8/2, 24/2, \Hyndluljod\ 1/4a, \HelgakvidaTwo\ P2, \Atlakvida\ 2/2a(?), Icelandic Rune Poem 4/2, Eddic Fragment 7/1.
\end{itemize}

\section{Poetic formulæ (F)}
All formulæ are given in English translation, their attested forms and a Proto-Germanic rendition. For those consisting of two words bound together by a conjunction, \& is written in its place.

\begin{itemize}
\inxitem[F]{Earth and Up-heaven} (ON \emph{jǫrð \& upphiminn}, OE \emph{eorþe \& upheofon}, OS \emph{erþa \& uphimil}, OHG \emph{erdo \& úfhimil}, PGmc. \emph{*erþō \& uphiminaz})
  An old merism; earth and heaven and everything in between, i.e. the whole universe. It has a particular connection to the creation and destruction of the world, and in prayers.
  ON: \Voluspa\ 3/3, \Vafthrudnismal\ 20, \Thrymskvida\ 2, \Oddrunargratr\ 17, DR EM85;493 (under Galders), Sö 154 (under Runic Poetry);
  OE: \emph{Acreboot};
  OS: \Heliand\ 2886;
  OHG: \emph{Wessobrunn} 2.

\inxitem[F]{Eese and Elves} (ON \emph{ę́sir \& alfar}, OE \emph{ése \& ielfe}, PNWGmc. \emph{*alβíʀ \& ansiwiʀ})
  A merism; both heavenly and earthly spiritual beings.  Notably the two words always occur in this order (never ‘Elves and Eese’), even in OE.

\inxitem[F]{words and works} (ON \emph{orð \& verk}, OE \emph{word \& weorc}, PGmc. \emph{*wurdó \& werkó})
  \Beowulf\ 289, 1100, 1833

\end{itemize}
